%---------------------------------
%Author: Vincent ISOZ
%Last Update: 2017-06-11
%Contact: info@sciences.ch
%Compiled on: TeXMaker 4.5/MiKTeX 2.9/Win10 CreatorUpdate x64/PC 16GB RAM (recommanded otherwise the document could not compiled correctly or not at all without deactivating the LOF and LOT!)
%Remarks: 

%R1. The tkz-tab and etex packages must be installed manually from the MikTeX Package Manager!!!

%R2. When you install first MiKTeX and after TeXMaker don't forget to reboot computer. Compile the document (probably will fail first time...). Do an update of packages using the MiKTeX package updater. Try compile again and use once again de MiKTeX package update. And it should work! If it still don't work i have notice that sometimes corrupted images cause the compilation to not be possible.

%R3. Try using a less as possible styles and math operators to avoid confusion (thanks).

%R4. Since 2015-08-29 the ctable package makes conflicts and therefore you can't compile the document with it anymore. We have commented it.

%R5. Above 2,600 pages the MikTeX compilation reach TeX memory limit. You will get the message:
%TeX capacity exceeded, sorry [main memory size=3000000]
%To solve this open a command Window or the PowerShell and enter:
%initexmf --edit-config-file=pdflatex
%Add the following line in the new document that appears on the screen:
%pool_size=5000000
%main_memory=6000000
%extra_mem_bot=2000000
%font_mem_size=2000000
%and save it (ctrl+s) and quit the editor. After rebuild the format with:
%initexmf --dump=pdflatex
%that's it!!!!

%R6. To remove all equation in Notepad++ with a RegEx, use:
% (?s)\\begin\{gather\}.*?\\end\{gather\}
%don't forget to activate the field ". match newlines"
%---------------------------------
\documentclass[12pt,a4paper,twoside,openright]{report}
%solve problem of figures position
\usepackage{float}
%text styles and some special operators
\newcommand{\NewTerm}[1]{\textcolor{red}{#1}}
\newcommand{\SeeChapter}[1]{\textcolor{gray}{#1}}
\newcommand{\Ima}{\text{Im}}
\newcommand{\mi}{\mathrm{i}}
\newcommand{\card}[1]{\ensuremath{\left\|#1\right\|}}
%package to write TexMaker logo
\usepackage{hologo}
%package to write the reycling logo
\usepackage{recycle}
%package for tree logo (avoid to print)
\usepackage{fontawesome}
%package for beautiful quote font
\usepackage{calligra}
%package to control margine and paper size
\usepackage[paperheight=297mm,paperwidth=210mm,top=2.5cm,bottom=2.5cm,left=2cm,right=2cm,bindingoffset=1cm]{geometry}
%packages pour l'internationalisation de la police et son rendu
\usepackage[utf8]{inputenc}
\usepackage[T1]{fontenc}
\usepackage{emerald} %package for nice font for book title
\renewcommand{\rmdefault}{ptm}%to darken the text
\usepackage[french,english]{babel}
%package for specials chars (wingdings like)
\usepackage{pifont}
%package for per-thousand symbol
\usepackage{textcomp}
%package for version history
\usepackage{vhistory}
%packages for headers and footers
\usepackage{fancyhdr}
%package to force page notes to have no idents and to be at bottom of page
\usepackage[hang,flushmargin,bottom]{footmisc} 
%package to have more deep levels than by default
\usepackage{enumitem}
%package to have enumerations on multiple columsn
\usepackage{multicol}
%package for colors in tables, title page and anywhere else
\usepackage[usenames,dvipsnames]{xcolor}
\definecolor{BrickRed}{rgb}{0.72,0,0}
%change color of sections
\usepackage{sectsty}
\allsectionsfont{\color{black!50}}

%package to make captions small, bold and second package to make multiple captions
\definecolor{ocre}{RGB}{243,102,25}
\usepackage[font={scriptsize,bf},labelfont={color=ocre,bf},skip=0pt,justification=centering]{caption}

%\renewcommand{\thefigure}{\textcolor{ocre}{\bfseries\itshape\thechapter.\arabic{figure}}}
%\renewcommand{\figurename}{\textcolor{ocre}{\bfseries\itshape Fig.}}
%\renewcommand{\thetable}{\textcolor{ocre}{\bfseries\itshape\arabic{table}}}
%\renewcommand{\tablename}{\textcolor{ocre}{\bfseries\itshape Table}}

\usepackage{subcaption}
%definition of progress circles for sections
\newlength\charwidth
\newlength\chwidth
\newcommand*\circled[1]%
  \settototalheight\chwidth{#1\,\%}%
  \ifdim\chwidth>\charwidth\let\charwidth\chwidth\fi
  \addtolength\charwidth{15pt}% twice inner sep plus half line width
  \tikz[baseline=(char.base)];
    \draw [line width=2pt, color=basecol] (char.north) arc (90:90-#1*3.6:.5\charwidth) coordinate (a);
    \draw [line width=2pt, color=othercol]  (a) arc (90-#1*3.6:-270:.5\charwidth);
  }%
}
\colorlet{basecol}{purple}
\colorlet{othercol}{purple!25}
%some colors for the title page
\definecolor{titlepagecolor}{cmyk}{0.3,0.3,0.3,0.3}
\definecolor{namecolor}{cmyk}{1,.50,0,.10}
%toc/lof/lot stuff
\usepackage{minitoc}
\setcounter{minitocdepth}{2}
\usepackage{tocloft} %this line and above two is to minimize vertical space in LOF and to increase number alignement in LOF
\addtolength{\cftfignumwidth}{1em}
\addtolength{\cfttabnumwidth}{1em}
%various packages for tables complications
\usepackage{hhline} %to draw correctyl double vertical/horizotnal lines
\usepackage{dcolumn} %to align decimals
\usepackage{multirow} %to merge cells
\usepackage{colortbl} %for background colors
\usepackage{array} %to center text horizontally in cells
\usepackage{variations} %to build variation tables
\usepackage{spreadtab} %for formulas in tables
\usepackage{booktabs} %for special borders
\usepackage{slashbox} %for diagonals in cells
\usepackage{fancybox} %for rows/columns with bck colors
\usepackage{hhline} %for double horizontal lines
\usepackage{diagbox}
%\usepackage{diagbox}
\usepackage{rotating} %rotation of text
\usepackage{makecell}
\setlength{\textfloatsep}{0.1cm} %space below table (between table and text)
%macro to vertically center images in cells
\newcommand\cincludegraphics[2][]{\raisebox{-0.6\height}{\includegraphics[#1]{#2}}}
%\usepackage{ctable} %for long tables
\usepackage{longtable} %to repeat title row on multiple pages
\usepackage{pbox} %for carriage return in cells
%package for charts and images
\usepackage{graphicx}
\usepackage{bclogo} %four countries flags
%package for videos (flash animations)
\usepackage{media9}
%package for lettrine
\usepackage{lettrine,oldgerm,yfonts}
%packages to include eps into final pdf (must be declared after graphicx package
\usepackage{epstopdf}
\epstopdfsetup{update} % only regenerate pdf files when eps file is newer
\usepackage{pdfpages}
%packages for *.eps images and others
\usepackage{graphicx}
\usepackage{epsfig}
\usepackage{transparent}
\usepackage{eso-pic}
%for background images
\newcommand\BackImage[2][scale=1]{%
\BgThispage
\backgroundsetup{
  contents={\includegraphics[#1]{#2}}
  }
}
%load packages for text canvas (borders)
\usepackage{bclogo,environ,wrapfig}
\usepackage[many]{tcolorbox}
\usepackage{pst-all}
%page to use footnotes in tables
\usepackage{footnote}
%to force big footnotes to stay on the same page
\interfootnotelinepenalty=10000

%numbering Definitions
\newcounter{def}
\newcommand{\mydef}{%
        \stepcounter{def}%
        \thedef}
        
%numbering lines
\usepackage[modulo,right]{lineno}
%put personal paragraphs number or simply margin notes
\usepackage{marginnote}
\newcounter{importantparagraph}[chapter]
\newcommand{\myparagraph}{%
        \stepcounter{importantparagraph}%
        \theimportantparagraph}

%style for equations boxes
\newcommand*{\boxcolor}{orange}
\makeatletter
\renewcommand{\boxed}[1]{\textcolor{\boxcolor}{%
\tikz[baseline={([yshift=-1ex]current bounding box.center)}] \node [rectangle, minimum width=1ex,rounded corners,draw] {\normalcolor\m@th$\displaystyle#1$};}}
 \makeatother


%package for flowchart/workflows and definitions of some items
\usepackage{tikz}
\usetikzlibrary{backgrounds,arrows.meta,calc,arrows,fadings,mindmap}
\usetikzlibrary{decorations.pathreplacing,decorations.markings,patterns}
\usetikzlibrary{shapes.geometric,shapes.gates.logic.US,trees,positioning,arrows}
\usepackage{tkz-tab} %to draw tikz tables
%for circle numbers (in section Automata Theory)
\newcommand*\circledtext[1]{\tikz[baseline=(char.base)]{
            \node[shape=circle,draw,inner sep=2pt] (char) {#1};}}
%package for karnaugh map
\usepackage{karnaughmap}
%\usepackage{karnaugh-map}
%for smileys
\usepackage{tikzsymbols}
%for bohr atoms
\usepackage{bohr}
%package for the font size of the morphisms venn diagram
\usepackage{scalefnt}
%to build matrix multiplication scheme
\usepackage{amsbsy}
\newcommand*{\clap}[1]{\hbox to 0pt{\hss#1\hss}}
\newcommand*{\mat}[1]{\boldsymbol{\mathrm{#1}}}
\newcommand*{\subdims}[3]{\clap{\raisebox{#1}[0pt][0pt]{$\scriptstyle(#2 \times #3)$}}}
\fboxrule=1pt
%to build star vote system
\usetikzlibrary{arrows,shapes.geometric,positioning,matrix}
\newcommand\score[2]{
\pgfmathsetmacro\pgfxa{#1+1}
\tikzstyle{scorestars}=[star, star points=5, star point ratio=2.25, draw,inner sep=1.3pt,anchor=outer point 3]
  \begin{tikzpicture}[baseline]
    \foreach \i in {1,...,#2} {
    \pgfmathparse{(\i<=#1?"yellow":"gray")}
    \edef\starcolor{\pgfmathresult}
    \draw (\i*1.75ex,0) node[name=star\i,scorestars,fill=\starcolor]  {};
   }
  \end{tikzpicture}
}
\usetikzlibrary{matrix}
\usetikzlibrary{shapes,arrows}
\usetikzlibrary{shapes.geometric, arrows}
\tikzstyle{startstop}=[rectangle,rounded corners,minimum width=3cm,minimum height=1cm,text centered,draw=black,fill=red!30]
\tikzstyle{io}=[trapezium,trapezium left angle=70,trapezium right angle=110,minimum width=3cm,minimum height=1cm, text centered,draw=black,fill=blue!30]
\tikzstyle{process}=[rectangle,minimum width=3cm,minimum height=1cm,text centered,text width=3cm,draw=black,fill=orange!30]
\tikzstyle{decision}=[diamond,minimum width=3cm,minimum height=1cm,text centered,draw=black,fill=green!30]
\tikzstyle{arrow}=[thick,->,>=stealth]
\tikzstyle{decision}=[diamond, draw,fill=blue!20,text width=4.5em,text badly centered,node distance=3cm, inner sep=0pt]
\tikzstyle{block}=[rectangle,draw,fill=blue!20,text width=5em,text centered,rounded corners,minimum height=4em]
\tikzstyle{line}=[draw,-latex']
\tikzstyle{cloud}=[draw,ellipse,fill=red!20,node distance=3cm,minimum height=2em]
%paragraphe identation by default
\setlength\parindent{0pt}
%package of report visual style
\usepackage{background}
\backgroundsetup{ contents= {\begin{tikzpicture}[remember picture, overlay] \draw [line width=0.3pt,color=gray,step=0.5cm] (current page.south west) grid (current page.north east); \end{tikzpicture} } scale=1, angle=0} 
\usepackage{atbegshi}% http://ctan.org/pkg/atbegshi
%for codes
\usepackage{listings,lstautogobble,xcolor}
%color definition for matlab script
\definecolor{mygreen}{RGB}{28,172,0} % color values Red, Green, Blue
\definecolor{mylilas}{RGB}{170,55,241}
\lstset{language=Matlab,%
    %basicstyle=\color{red},
    breaklines=true,%
    morekeywords={matlab2tikz},
    keywordstyle=\color{blue},%
    morekeywords=[2]{1}, keywordstyle=[2]{\color{black}},
    identifierstyle=\color{black},%
    stringstyle=\color{mylilas},
    commentstyle=\color{mygreen},%
    showstringspaces=false,%without this there will be a symbol in the places where there is a space
    numbers=left,%
    numberstyle={\tiny \color{black}},% size of the numbers
    numbersep=9pt, % this defines how far the numbers are from the text
    emph=[1]{for,end,break},emphstyle=[1]\color{red}, %some words to emphasise
    %emph=[2]{word1,word2}, emphstyle=[2]{style}, 
    autogobble=true   
}
%for algorithms
\usepackage[linesnumbered,ruled]{algorithm2e}
%load americal mathematical society packages for equations
\usepackage{amsmath,amsthm,amssymb,amsfonts}
\DeclareMathOperator{\sgn}{sgn} %must be after package ams!
\usepackage{xfrac} %package for small fractions like spin 1/2
\usepackage{gensymb} %for degree symbole
\usepackage{empheq} %for multiple equations box
\AtBeginDocument{\AtBeginShipoutNext{\AtBeginShipoutDiscard}}
%\package to automatically resize equation to page width
\usepackage{resizegather}
\usepackage{mathtools} %for text above arrows
%package for linear system of equations
\usepackage{systeme}
%packages for special integral
\usepackage{wasysym} %closed inetegrals
\usepackage{esint} %intégrales de contour orientées (analyse complexe)
\usepackage{bigints} %write huge integral
\usepackage{yhmath} %write arc symbol above multiple letters
%package for unitary matrix
\usepackage{dsfont}
%definition for theorems
\theoremstyle{definition}
\newtheorem{theorem}{Theorem}[chapter]
\newtheorem{dem}{Proof}[theorem]
\newtheorem{corollary}{Corollary}[theorem]
\newtheorem{lemma}{Lemma}[theorem]
\newtheoremstyle{itexmp}
 {\topsep}
 {\topsep}
 {\normalfont}
 {0pt}
 {\itshape\bfseries}{.}
 { }
 {#1 \textit{#2}}
\theoremstyle{itexmp}
\newtheorem{exmp}{Example}[chapter]
\newcommand{\statedefn}[2]{
 \definecolor{shadethmcolor}{cmyk}{0.1,0.05,0,0}
 \definecolor{shaderulecolor}{cmyk}{0.73,0.19,0,0}
 \begin{defn}\label{#1}{#2}\end{defn}
}
%for annuities symboles
\usepackage{lifecon}
%for euro symbol
\usepackage{eurosym}
%for tensor calculus AND annuities notations
\usepackage{tensor}
%for quantum bra-ket symbols
\usepackage{braket}
%numbering of equations restart at each section
\numberwithin{equation}{section} 
%package for terms simplifications in equations
\usepackage[makeroom]{cancel}
%package for watermark
\usepackage{draftwatermark}
\SetWatermarkLightness{0.85}
\SetWatermarkAngle{25}
\SetWatermarkScale{2}
\SetWatermarkFontSize{1.5cm}
\SetWatermarkText{4th Edition Draft}
%packages for chapters color margin markers and definitions
\usepackage{background}
\usepackage{xifthen}
\usepackage{totcount}
\regtotcounter{chapter}

\backgroundsetup
{   contents={
        \begin{tikzpicture}[overlay]
            \pgfmathtruncatemacro{\mytotalchapters}{\totvalue{chapter} > 0 ? \totvalue{chapter} : 20}
            \pgfmathsetmacro{\mypaperheight}{\paperheight/28.453}
            \pgfmathsetmacro{\mytop}{-(\thechapter-1)/\mytotalchapters*\mypaperheight}
            \pgfmathsetmacro{\mybottom}{-\thechapter/\mytotalchapters*\mypaperheight}
            \ifcase\thechapter
                \xdef\mycolor{white}
                \or \xdef\mycolor{red}
                \or \xdef\mycolor{orange}
                \or \xdef\mycolor{yellow}
                \or \xdef\mycolor{green}
                \or \xdef\mycolor{blue}
                \or \xdef\mycolor{violet}
                \or \xdef\mycolor{Apricot}
                \or \xdef\mycolor{Magenta}
                \or \xdef\mycolor{GreenYellow}
                \or \xdef\mycolor{Sepia}
                \or \xdef\mycolor{SkyBlue}
                \or \xdef\mycolor{Aquamarine}
                \else \xdef\mycolor{black}
            \fi
            \ifthenelse{\isodd{\value{page}}}
            {\fill[\mycolor] ($(current page.north east)+(0,\mytop)$) rectangle ($(current page.north east)+(-0.5,\mybottom)$);}
            {\fill[\mycolor] ($(current page.north west)+(0,\mytop)$) rectangle ($(current page.north west)+(0.5,\mybottom)$);}
        \end{tikzpicture}
    },
    scale=1,
    angle=0
}
%package for image positions
\usepackage{wrapfig}
\usepackage{picins}
%package lipsum and blindtext for various tests
\usepackage{lipsum,blindtext}
%definition for numbering sections
\setcounter{secnumdepth}{5}
\setcounter{tocdepth}{5}
%definition for main TOC depth
%\setcounter{\tocdepth}{5}
%package to get lastpage number
\usepackage{lastpage}
\AtBeginShipout{
\ifnum\value{page}=\number\numexpr\getpagerefnumber{LastPage}-0\relax
\phantomsection\label{preLastPage}
\fi}
%package for chapter styles formatting (gray box) and keept it here otherwise it will make a render bug
\usepackage[Bjornstrup]{fncychap}
\ChNameVar{\Huge}
%for electronic circuit
\usepackage[compatibility]{circuitikz}
\input{electronic_symbols_macros.tex}
%package for small TOC
\usepackage{shorttoc}
%was supposed to be use for Creative Common logo but doesn't work since August 2015 update
%\usepackage[type={CC},modifier={by},version={3.0},]{doclicense}
%part for bibtex
\usepackage[backend=bibtex]{biblatex}
\bibliography{biblio}
%package to make index
\usepackage{makeidx,imakeidx}
%to make work the options below first do an index with the options commented and after compile again with the option uncommented!
\makeindex[options=-s my_index_style.ist -c]

%decoration for title page (if \titlepacagedecoration uncommented further below
\newcommand\titlepagedecoration{
\begin{tikzpicture}[remember picture,overlay,shorten >= -10pt]

\coordinate (aux1) at ([yshift=-15pt]current page.north east);
\coordinate (aux2) at ([yshift=-410pt]current page.north east);
\coordinate (aux3) at ([xshift=-4.5cm]current page.north east);
\coordinate (aux4) at ([yshift=-150pt]current page.north east);

\begin{scope}[black!40,line width=12pt,rounded corners=12pt]
\draw
  (aux1) -- coordinate (a)
  ++(225:5) --
  ++(-45:5.1) coordinate (b);
\draw[shorten <= -10pt]
  (aux3) --
  (a) --
  (aux1);
\draw[opacity=0.6,black,shorten <= -10pt]
  (b) --
  ++(225:2.2) --
  ++(-45:2.2);
\end{scope}
\draw[black,line width=8pt,rounded corners=8pt,shorten <= -10pt]
  (aux4) --
  ++(225:0.8) --
  ++(-45:0.8);
\begin{scope}[black!70,line width=6pt,rounded corners=8pt]
\draw[shorten <= -10pt]
  (aux2) --
  ++(225:3) coordinate[pos=0.45] (c) --
  ++(-45:3.1);
\draw
  (aux2) --
  (c) --
  ++(135:2.5) --
  ++(45:2.5) --
  ++(-45:2.5) coordinate[pos=0.3] (d);   
\draw 
  (d) -- +(45:1);
\end{scope}
\end{tikzpicture}
}

%dedication page
\newenvironment{dedication}
  {\clearpage           % we want a new page
   \thispagestyle{empty}% no header and footer
   \vspace*{\stretch{1}}% some space at the top 
   \itshape             % the text is in italics
   \raggedleft          % flush to the right margin
  }
  {\par % end the paragraph
   \vspace{\stretch{3}} % space at bottom is three times that at the top
   \clearpage           % finish off the page
  }

%for start page
\usepackage{authblk}
%package for international date format
\usepackage[yyyymmdd]{datetime}

%package hyperref for links and pdf metadata, Must be the last package!
\usepackage[unicode,pdfencoding=auto,hidelinks,pdfinfo={pdfauthor={ISOZ Vincent},pdftitle={Elementary Applied Mathematics - Sciences.ch},pdfsubject={Elementary Applied Mathematics},pdfkeywords={Arithmetics;Statistics;Geometry;Trigonometry;Quantitative Finance;Engineering;Theoretical Computing},pdfproducer={Scientifc Evolution Sàrl}, pdfcreator={ISOZ Vincent}}]{hyperref}
\definecolor{wine-stain}{rgb}{0.5,0,0}
\hypersetup{colorlinks,linkcolor=wine-stain,linktoc=all}

\renewcommand{\dateseparator}{--}
%package to control globally space between items
\usepackage{enumitem}
\setlist[1]{itemsep=6pt}
%three packages for old-style background
\usepackage{eso-pic}
\usepackage{changepage}
\strictpagecheck

%*****************************************fancy quotes starts here
\definecolor{quotemark}{gray}{0.7}
\makeatletter
\def\fquote{%
    \@ifnextchar[{\fquote@i}{\fquote@i[]}%]
           }%
\def\fquote@i[#1]{%
    \def\tempa{#1}%
    \@ifnextchar[{\fquote@ii}{\fquote@ii[]}%]
                 }%
\def\fquote@ii[#1]{%
    \def\tempb{#1}%
    \@ifnextchar[{\fquote@iii}{\fquote@iii[]}%]
                      }%
\def\fquote@iii[#1]{%
    \def\tempc{#1}%
    \vspace{1em}%
    \noindent%
    \begin{list}{}{%
         \setlength{\leftmargin}{0.1\textwidth}%
         \setlength{\rightmargin}{0.1\textwidth}%
                  }%
         \item[]%
         \begin{picture}(0,0)%
         \put(-15,-5){\makebox(0,0){\scalebox{3}{\textcolor{quotemark}{''}}}}%
         \end{picture}%
         \begingroup\itshape}%
 \def\endfquote{%
 \endgroup\par%
 \makebox[0pt][l]{%
 \hspace{0.8\textwidth}%
 \begin{picture}(0,0)(0,0)%
 \put(15,15){\makebox(0,0){%
 \scalebox{3}{\color{quotemark}''}}}%
 \end{picture}}%
 \ifx\tempa\empty%
 \else%
    \ifx\tempc\empty%
       \hfill\rule{100pt}{0.5pt}\\\mbox{}\hfill\tempa\ \emph{\tempb}%
   \else%
       \hfill\rule{100pt}{0.5pt}\\\mbox{}\hfill\tempa\ \emph{\tempb}\ \tempc%
   \fi\fi\par%
   \vspace{0.5em}%
 \end{list}%
 }%
 \makeatother
 %*****************************************fancy quotes ends here
%package for some advanced plots
\usepackage{pgfplots}
\usetikzlibrary{patterns}

\usepackage{titlesec}
%this is a test to make chapter title bigger
%\titleformat{\chapter}{\bf\huge}{\thechapter}{1em}{}

%to make beautiful section titles
\newcommand\titlebar{%
\tikz[baseline,trim left=3.1cm,trim right=3cm] {
    \fill [BrickRed!25] (4cm,-1ex) rectangle (\textwidth+3.1cm,2.5ex);
    \node [
        fill=BrickRed,
        anchor= base east,
        rounded rectangle,
        minimum height=3.5ex] at (4.5cm,0) {
        \textbf{\arabic{chapter}.\thesection.}
    };
}%
}
\titleformat{\section}{\LARGE}{\titlebar}{2cm}{} %distance from section text to number
%to have chapter number in front of sections and subsections
%\renewcommand*{\thesection}{\arabic{section}}

%push memory limits of LaTeX
%\usepackage{morewrites}
%if the above "morewrites" does not work...
\usepackage{scrwfile}


\begin{document}
	\sloppy %to force texts and equations to respects pages margins
	\pageref{LastPage}
	\raggedbottom
	\pagenumbering{roman}
	\newgeometry{margin=2.5cm}
%page de couverture
	\begin{titlepage}
	\newgeometry{left=2cm,bottom=2cm,top=2cm} %defines the geometry for the titlepage
	\BackImage[scale=1]{img/quantum.jpg}
	%\pagecolor{titlepagecolor}
	\noindent
	\color{gray}
	{\normalsize The free 6,000 pages super-quick, super-painless undergraduate transportable Book}\\
	{\normalsize on Elementary Applied Mathematics for Engineers (EAME)}
	\color{white}
	\makebox[0pt][l]{\rule{1.3\textwidth}{1pt}}\\
	\par
	\noindent
	\scalebox{2.2}{\fontsize{32pt}{0pt}\selectfont \textbf{Opera Magistris}} \\ \textcolor{namecolor}	{\textsf{Original distribution}}
	\vfill
	\noindent
	{\huge \textsf{3rd Edition}}
	\vskip\baselineskip
	\noindent
	\textsf{\{ISBN 978-2-8399-0932-7\} October 2014}
	\vskip\baselineskip
	Compiled with {\huge \LaTeXe} on \TeX maker
	\vskip\baselineskip
	{\small
EAME-3 work is licensed under a Creative Commons "Attribution
3.0 Unported" license.\\ paternity; no commercial usage; sharing of conditions identical\\
\url{https://creativecommons.org/licenses/by-nc-sa/4.0/}}\\[2pt]
\includegraphics[width=1cm]{img/icons/share2.eps}\includegraphics[width=1cm]{img/icons/remix2.eps}
\includegraphics[width=1cm]{img/icons/by2.eps}
\includegraphics[width=1cm]{img/icons/nc-eu2.eps}
\includegraphics[width=1cm]{img/icons/sa2.eps}\\
2001-2017: \textit{One book to rule them (almost) all!} \href{https://www.facebook.com/sharer/sharer.php?u=http://www.sciences.ch/htmlen/latex/LaTeX_SciencesCh.pdf}{\faThumbsOUp{}}
\\\\
	
	%\titlepagedecoration
	\end{titlepage}
	\restoregeometry %restores the geometry
	\nopagecolor %use this to restore the color pages to white (or other color depending on your choice with \pagecolor{yellow!20} for example)
%end of title page
	\newpage\null\thispagestyle{empty}\newpage
	\restoregeometry
	%old style background
	%\AddToShipoutPicture{\checkoddpage
	%\ifoddpage
    %\put(0,0){\includegraphics[width=\paperwidth,height=\paperheight]{img/odd_page.jpg}}
	%\else
	%     \put(0,0){\includegraphics[width=\paperwidth,height=\paperheight]{img/even_page.jpg}}
	%\fi
 	%}r
	\begin{titlepage}

\newcommand{\HRule}{\rule{\linewidth}{0.5mm}} % Defines a new command for the horizontal lines, change thickness here

\center % Center everything on the page
 
%----------------------------------------------------------------------------------------
%	HEADING SECTIONS
%----------------------------------------------------------------------------------------

\textsc{\Large Original Distribution}\\[1.5cm] % Name of your university/college
\textsc{\Huge \textbf{Opera Magistris}}\\[0.5cm] % Major heading such as course name
\textsc{\large 3rd Edition}\\[0.5cm] % Minor heading such as course title

%----------------------------------------------------------------------------------------
%	TITLE SECTION
%----------------------------------------------------------------------------------------

\HRule \\[0.4cm]
{ \Large \bfseries Compendium on \\ Elementary Applied Mathematics for Engineers }\\[0.4cm] % Title of your document
\HRule \\[1.5cm]
 
%----------------------------------------------------------------------------------------
%	AUTHOR SECTION
%----------------------------------------------------------------------------------------

%\begin{minipage}{0.4\textwidth}
%\begin{flushleft} \large
%\emph{Co-editors:}\\
%Léon \textsc{HARMEL}\\
%Vincent \textsc{ISOZ}\\
%\end{flushleft}
%\end{minipage}
%~
%\begin{minipage}{0.4\textwidth}
%\begin{flushright} \large
%\emph{Supervisors:} \\
%F.D.C. Tigrou % Supervisor's Name Felis domesticus catus Tigrou
%\end{flushright}
%\end{minipage}\\[2cm]

% If you don't want a supervisor, uncomment the two lines below and remove the section above
%\Large \emph{Author:}\\
%John \textsc{Smith}\\[3cm] % Your name

%----------------------------------------------------------------------------------------
%	DATE SECTION
%----------------------------------------------------------------------------------------

{\large \today}\\[2cm] % Date, change the \today to a set date if you want to be precise

%----------------------------------------------------------------------------------------
%	LOGO SECTION
%----------------------------------------------------------------------------------------

%\BackImage[scale=1.75]{img/god.jpg}
%----------------------------------------------------------------------------------------

\vfill % Fill the rest of the page with whitespace

\end{titlepage}
	\begin{versionhistory}
		\vhEntry{3.8}{2018-01-01}{VI}{Some minor updates and corrections (see change log)}
		\vhEntry{3.7}{2017-06-01}{VI}{French 3rd Edition translation into English finished!}
		\vhEntry{3.6}{2017-05-01}{VI}{French 3rd Edition translation into English. Translation progress $\sim 99\%$}
		\vhEntry{3.5}{2017-04-01}{VI}{French 3rd Edition translation into English. Translation progress $\sim 96\%$}
		\vhEntry{3.4}{2017-02-11}{VI}{French 3rd Edition translation into English. Translation progress $\sim 96\%$}
		\vhEntry{3.3}{2016-01-27}{VI}{French 3rd Edition translation into English. Translation progress $\sim 96\%$}
		\vhEntry{3.0}{2014-01-22}{VI}{French 3rd Edition published on the Internet as PDF ($4,888$ pages).}
		\vhEntry{2.0}{2005-03-10}{VI}{French 2nd Edition published on the Internet as PDF ($2,001$ pages).}
		\vhEntry{1.0}{2002-05-01}{VI}{French 1st Edition published on the Internet as HTML only.}
	\end{versionhistory}
	\newpage\null\thispagestyle{empty}\newpage %création d'une nouvelle page en forcant la disparition du numéro de page
	\dominitoc
	\shorttoc{Contents}{0} % Only chapters
	\pagebreak
	\renewcommand{\contentsname}{Table of Contents}
	\tableofcontents
	\newpage\null\thispagestyle{empty}\newpage
	\setlength{\parskip}{12pt}
	\pagenumbering{arabic}
	\clearpage %eliminate headers and footer of table of contents page
	\pagestyle{fancy} %du package fancyhdr!!!
	\renewcommand{\chaptermark}[1]{\markboth{\thechapter.\space#1}{}}
	\fancyhead[LE,RO]{\nouppercase\leftmark} %LE=Left Even,RO=Right Odd
	\fancyhead[LO,RE]{EAME v3.8-2013}
	\renewcommand{\footrulewidth}{1pt}
	\fancyfoot[LE,RO]{{\thepage}/\pageref{LastPage} (\Acrobatmenu{GoBack}{Go back})}
	\fancyfoot[LO,RE]{\href{mailto:info@sciences.ch}{info@sciences.ch}}
	\fancyfoot[C]{}
	\let\cleardoublepage\clearpage

	\begin{dedication}
	{\LARGE Dedicated to \textbf{Mother Nature}}
	\end{dedication}

	\chapter{Warnings}
	\minitoc
		%to make section start on odd page
	\newpage
	\thispagestyle{empty}
	\mbox{}
	\section{Impressum}	
	\subsection{Use of content}

	The contents of this book are elaborated by a development process by which volunteers reach a consensus. This process that brings together volunteers, research also the point of view of people interested in the topics of this book. The person in charge of this book administers the process and establishes rules to promote fairness in the consensus approach. It is also responsible for drafting the text, sometime for testing/evaluating or independently verifying the accuracy or completeness of the presented information.

	We decline no responsibility for any injury, damage or any other kind, special, incidental, consequential or compensatory, arising from the publication, application or reliance on the content of this book. We make no express or implied warranty on the accuracy or completeness of any information published in this book, and do not guarantee that the information contained in this book meet any specific need or goal of the reader. We do not guarantee the performance of products or services of one manufacturer or vendor solely by virtue of this book content.
	
	The technical descriptions, procedures, and computer programs in this book have been developed without care, therefore they are provide without warranty of any kind. We make also no warranties that the equations, programs, and procedures in this books or its associated software are free of error, or are consistent with any particular standard of merchantability, or will meet your requirements for any particular application. They should not be relied upon for solving a problem whose incorrect solution could result in injury to a person or loss of property. Any use of the content of this book as at the reader's own risk. The authors, redactors, and publisher disclaim all liability for direct, incidental, or consequent damages resulting form use of the content of this book or the associated software.

	By publishing texts, it is not the intention of this book to provide services on behalf of any person or entity or performing any task to be accomplished by any person or entity for the benefit of a third party. Anyone using this book should rely on its own independent judgment or, where that is appropriate, seek the advice of a qualified expert to determine how to exercise reasonable care under all circumstances. The information and standards on the topics covered by this book may be available from other sources that the reader may wish to visit in search of points of view or additional information not covered by the contents of this book.

	We have no power in order to enforce compliance with the contents of this book, and we do not undertake to monitor or enforce such compliance. We have no certification, testing or inspection activity of products, designs or installations for safety or health of persons and property. Any certification or other statement of compliance regarding information relating to health or safety of persons and property, mentioned in this book, cannot possibly be attributed to the content of this book and remains under the responsibility of the certification center or the concerned reporter.

	\pagebreak	
	\subsection{How to use this book}

	At the university level, this book can be used for a Ph.D., graduate level or advanced undergraduate level seminar in many exact and pure sciences fields. The seminars where we use this material is part of \textbf{Scientific Evolution Sàrl} program, where the trainees typically already have taken undergraduate or graduate courses in their respective specialization. In reality this books also aims to cover the full Kindergarten to PhD curriculum.

	Because the methods of Applied Mathematics are learned by practice and experience, we view a seminar on Applied Mathematics as a learning-by-doing (project oriented) seminar. We structure our mathematical modelling seminars around a set of problems that require the trainee to construct models that help with planning and decision making. The imperative is that the models should be consistent with the theory and back-tested. To fulfill this imperative, it is necessary for the trainee to combine mathematical theory with modeling. The result is that the trainee learns the theory, and more importantly, learns how that theory is applied and combined in the real world. The ability to criticize and identify limitations of dangerous mathematical tools is the most valuable feature of our seminars.

	The problems with solutions in this book provide the opportunity to apply the text material to a comprehensive set of fairly realistic situations. By the end of the seminars the trainees will have enhanced their skills and knowledge of the most important theoretical and computing tools. These are valuable skills that are in demand by the businesses at the highest levels.

	It is very difficult to cover all the material in this book in a semester. It takes a lot of time to explain the concepts to the trainees. The reader is encouraged to pick and choose which topics will be covered during the term. It is not necessary strictly necessary to cover them in sequence but it can help in a significant way?

	In a nutshell, this book offers you a wide variety of topics that are amenable to modeling. All are practical.
	
	\subsubsection{Ancilliaries}
	We offer an array of ancilliaries for students, instructors and practitioners.
	
	First there are some free companion eBooks and tools in French and English written by Vincent ISOZ \& Daname KOLANI for the people that want to put in practice the theory presented in this book.
	
	Here is the list:
	\begin{itemize}
		\item MATLAB™ in English (1,361 pages):\\ \href{http://www.sciences.ch/htmlfr/php/cliccount/click.php?id=319}{http://www.sciences.ch/dwnldbl/divers/Matlab.pdf}
		
		\item Maple in French (99 pages):\\ \href{http://www.sciences.ch/dwnldbl/divers/Maple.pdf}{http://www.sciences.ch/dwnldbl/divers/Maple.pdf}
		
		\item \textsf{R} in French (1,800 pages):\\ \href{http://www.sciences.ch/htmlfr/php/cliccount/click.php?id=313}{http://www.sciences.ch/dwnldbl/divers/R.pdf}
		
		\item Minitab in French (1,118 pages):\\ \href{http://www.sciences.ch/htmlfr/php/cliccount/click.php?id=282}{http://www.sciences.ch/dwnldbl/divers/Minitab.pdf}
		
		\item Scientific Linux installation \& Configuration  (211 pages):\\ \href{http://www.sciences.ch/dwnldbl/divers/ScientificLinux.pdf}{http://www.sciences.ch/dwnldbl/divers/ScientificLinux.pdf}
	\end{itemize}
	\begin{center}
		\includegraphics[scale=0.75]{img/books/matlab.jpg}
		\includegraphics[scale=0.75]{img/books/maple.jpg}
		\includegraphics[scale=0.75]{img/books/r.jpg}
		\includegraphics[scale=0.75]{img/books/minitab.jpg}
		\includegraphics[scale=0.75]{img/books/scientificlinux.jpg} 
	\end{center}	
		
	 In second we offer a few Quizzes and Flashcards in French and English to challenge your students or just yourself with the rest of the world:
	 \begin{itemize}
		\item MATLAB™ Basics L1 Challenge level in French (100 questions)\\ \url{http://www.scientific-evolution.com/qcm/start_session/a73647cf3b/}
		
		\item Astronomy/Astrophysics H1 Challenge level in English (100 questions):\\ \url{http://www.scientific-evolution.com/qcm/start_session/ffd0810fa0/}
		
		\item Greek Letter Flashcards (48 cards):\\
		\url{http://www.scientific-evolution.com/qcm/fr/start_session/6d9f1fef90/}
		
		\item Common Derivatives Flashcards (29 cards):\\
		\url{http://www.scientific-evolution.com/qcm/fr/start_session/c15a40f2c4/}
		
		\item Common Primitives Flashcards (60 cards):\\
		\url{http://www.scientific-evolution.com/qcm/fr/start_session/ccfc20fdef/}
		
		\item Common Trigonometric Identities Flashcards (68 cards):\\
		\url{http://www.scientific-evolution.com/qcm/fr/start_session/882f9696cd/}
		
		\item \LaTeX{} L3 Challenge level in French (100 questions):\\ \url{http://www.scientific-evolution.com/qcm/fr/start_session/ff1e1d1b91/}
		
		\item R Software 3.1.2 L3 Challenge level in French (100 questions):\\ \url{http://www.scientific-evolution.com/qcm/fr/start_session/2a6fca7473/}
		
		\item C++ L3 Challenge level in French (100 questions):\\
		\url{http://www.scientific-evolution.com/qcm/fr/start_session/e031ce4b43/}
	\end{itemize}
	
	And as any technical book should have a forum, the reader ca go through this link for any discussions about the content of this book: \url{https://www.physicsforums.com}.
	
	For those who prefer social networks we have also a dedicated  Facebook page:
	\begin{center}
		\faFacebook{} \href{https://www.facebook.com/operamagistris/}{https://www.facebook.com/operamagistris/}
	\end{center}
	or for more fun (science pics, quotes, jokes, videos, etc.) there is also an associated Instagram account:
	\begin{center}
		\faInstagram{} \href{https://www.instagram.com/opera.magistris/}{https://www.instagram.com/opera.magistris/}
	\end{center}
	And a collection of a selection of what we consider a interesting scientific videos on our YouTube channel (see at the end of this book a section of interesting science YouTube channels):
	\begin{center}
		\faYoutube{} \href{https://www.youtube.com/user/AdminSciences}{https://www.youtube.com/user/AdminSciences}
	\end{center}

	As for this book, the companion books above are only samples of the complete one. The full version with \underline{perpetual free updates} are available for the price of \$ 299.- each and for \$ 499.- you get the exercise files and \LaTeX{} sources (for information on purchase you can simply send me an {\href{mailto:isoz@sciences.ch}{{\color{blue}email}}}).
	
	For people who want to help translate this book in 66 languages, here is the link with the \LaTeX{} sources to the original texts on GitHub:
	\begin{center}
		\faGithubSquare{} \href{https://github.com/vincentisoz}{https://github.com/vincentisoz}
	\end{center}
	Because this book mainly focus on mathematical aspect of physical phenomena we can only strongly recommend to the reader an another free book that is in our point of view actually the best one that focus on the popular science aspect of the subjects that we will cover:
	\begin{center}
	Motion Mountain by Dr. Christoph Schiller: \url{http://www.motionmountain.net}
	\end{center}
	\begin{center}
		\includegraphics[scale=1]{img/books/motion_mountain.jpg}
	\end{center}

	\pagebreak
	\subsection{Data Protection}
	When looking at information on the Internet companion site (Sciences.ch), some data are automatically saved. We try to save as less as possible data and as brief as possible. Wherever we can, we ave only anonymous data. We undertake to process the data you send us personally with the utmost diligence.

	However, your IP address and the source page that takes you on Sciences.ch and the associated keywords, are freely available to everybody \href{here}{http://www.sciences.ch/htmlen/php/cedstat/references.php} for the current month. After which detailed data are destroyed. You can object at any time in the publication of your data by contacting us.

	\subsection{Use of data}

	Your data are only used for sending the Sciences.ch newsletter. Communication of personal data (except the e-mail address, title and name) is optional. When registering for the newsletter, you can of course specify an alternate address and/or a fictitious name.

	\subsection{Data transmission}

	We will never sell or commercialize the data of our customers or interested parties and will never affects the rights of the person. In addition, we will not rent mailing lists and will not send you advertising from third parties or on our behalf.

	\subsection{Agreement}

	When you provide us personal information, you authorize us to save them and use them within the meaning of the Swiss Federal Law on Data Protection. If you ask us not to send you emails, we are obliged, in your interest, save your e-mail in an internal negative list.
	
	\subsection{Errata}
	Altought we have taken every care to ensure the accuracy of our content, mistakes do happen. If you find a mistake in this book - maybe a mistake in the text, scripts or illustrations - we would be grateful if you would report this to us. By doing so, you can save other readers from frustration and help us improve subsequent versions of this book. Our e-mail is given on the footer every page of this book. Once your errata are verified, your submissions will be accepted and the error will be visible on the change log of update versions.

	%to make section start on odd page
	\newpage
	\thispagestyle{empty}
	\mbox{}
	\section{License}
	The entire contents of this book is subject to the GNU Free Documentation License, which means:
	\begin{itemize}
			\item[$\bullet$] that everyone has the right to freely use the texts for non-commercial usage (Google Ads or any equivalent being considered as a commercial usage!)
			\item[$\bullet$] that any person is authorized to broadcast items for non-commercial usage (Google Ads or any equivalent being considered as a commercial usage!)
			\item[$\bullet$] that anyone can freely edit the texts for non-commmercial usage (Google Ads or any equivalent being considered as a commercial usage!)
	\end{itemize}
	
	and bla bla bla...

	in accordance with the license described below: 

	\begin{center}
	Version 1.1, March 2000
		
	Copyright (C) 2000 Free Software Foundation, Inc. 59 Temple Place, Suite 330, Boston, MA 02111-1307 USA Everyone is permitted to copy and distribute verbatim copies of this license document, but changing it is not allowed. 
	\end{center}

	\subsection{Preamble} 

	The purpose of this License is to make a manual, textbook, or other written document "free" in the sense of freedom: to assure everyone the effective freedom to copy and redistribute it, with or without modifying it only a non-commercial purpose\footnote{The immediate benefit of this is that you can generate the book, with brand new  problem sets, and distribute it to your students simply as a PDF (in an email, for instance). But more generally, if you are not keen on the way we explained (or failed to explain) something, then you are free to rewrite it. If you would like to cover more (or less) material, then you are free to add (or delete) whatever Chapters/Sections/Paragraphs
that you wish. And since you have the source code, you do not need to recreate the wheel.} (sources available on \href{https://github.com/vincentisoz}{Github}). Secondarily, this License preserves for the author and publisher a way to get credit for their work, while not being considered responsible for modifications made by others.

	This License is a kind of "copyleft", which means that derivative works of the document must themselves be free in the same sense. It complements the GNU General Public License, which is a copyleft license designed for free software. 

	We have designed this License in order to use it for manuals for free software, because free software needs free documentation: a free program should come with manuals providing the same freedoms that the software does. But this License is not limited to software manuals; it can be used for any textual work, regardless of subject matter or whether it is published as a printed book. We recommend this License principally for works whose purpose is instruction or reference. 

	\subsection{Applicability and Definitions}
	This License applies to any manual or other work that contains a notice placed by the copyright holder saying it can be distributed under the terms of this License. The "Document", below, refers to any such manual or work. Any member of the public is a licensee, and is addressed as "you". 

	A "Modified Version" of the Document means any work containing the Document or a portion of it, either copied verbatim, or with modifications and/or translated into another language. 

	A "Secondary Section" is a named appendix or a front-matter section of the Document that deals exclusively with the relationship of the publishers or authors of the Document to the Document's overall subject (or to related matters) and contains nothing that could fall directly within that overall subject. (For example, if the Document is in part a textbook of mathematics, a Secondary Section may not explain any mathematics.) The relationship could be a matter of historical connection with the subject or with related matters, or of legal, commercial, philosophical, ethical or political position regarding them. 

	The "Invariant Sections" are certain Secondary Sections whose titles are designated, as being those of Invariant Sections, in the notice that says that the Document is released under this License. 

	The "Cover Texts" are certain short passages of text that are listed, as Front-Cover Texts or Back-Cover Texts, in the notice that says that the Document is released under this License. 

	A "Transparent" copy of the Document means a machine-readable copy, represented in a format whose specification is available to the general public, whose contents can be viewed and edited directly and straightforwardly with generic text editors or (for images composed of pixels) generic paint programs or (for drawings) some widely available drawing editor, and that is suitable for input to text formatters or for automatic translation to a variety of formats suitable for input to text formatters. A copy made in an otherwise Transparent file format whose markup has been designed to thwart or discourage subsequent modification by readers is not Transparent. A copy that is not "Transparent" is named "Opaque". 

	Examples of suitable formats for Transparent copies include plain ASCII without markup, Texinfo input format, LaTeX input format, SGML or XML using a publicly available DTD, and standard-conforming simple HTML designed for human modification. Opaque formats include PostScript, PDF, proprietary formats that can be read and edited only by proprietary word processors, SGML or XML for which the DTD and/or processing tools are not generally available, and the machine-generated HTML produced by some word processors for output purposes only. 

	The "Title Page" means, for a printed book, the title page itself, plus such following pages as are needed to hold, legibly, the material this License requires to appear in the title page. For works in formats which do not have any title page as such, "Title Page" means the text near the most prominent appearance of the work's title, preceding the beginning of the body of the text. 

	\subsection{Verbatim Copying} 
	You may copy and distribute the Document in any medium, noncommercially, provided that this License, the copyright notices, and the license notice saying this License applies to the Document are reproduced in all copies, and that you add no other conditions whatsoever to those of this License. You may not use technical measures to obstruct or control the reading or further copying of the copies you make or distribute. However, you may accept compensation in exchange for copies. If you distribute a large enough number of copies you must also follow the conditions in section 3. 

	You may also lend copies, under the same conditions stated above, and you may publicly display copies. 

	\subsection{Copying in Quantity}

	If you publish printed copies of the Document numbering more than 100, and the Document's license notice requires Cover Texts, you must enclose the copies in covers that carry, clearly and legibly, all these Cover Texts: Front-Cover Texts on the front cover, and Back-Cover Texts on the back cover. Both covers must also clearly and legibly identify you as the publisher of these copies. The front cover must present the full title with all words of the title equally prominent and visible. You may add other material on the covers in addition. Copying with changes limited to the covers, as long as they preserve the title of the Document and satisfy these conditions, can be treated as verbatim copying in other respects. 

	If the required texts for either cover are too voluminous to fit legibly, you should put the first ones listed (as many as fit reasonably) on the actual cover, and continue the rest onto adjacent pages. 

	If you publish or distribute Opaque copies of the Document numbering more than 100, you must either include a machine-readable Transparent copy along with each Opaque copy, or state in or with each Opaque copy a publicly-accessible computer-network location containing a complete Transparent copy of the Document, free of added material, which the general network-using public has access to download anonymously at no charge using public-standard network protocols. If you use the latter option, you must take reasonably prudent steps, when you begin distribution of Opaque copies in quantity, to ensure that this Transparent copy will remain thus accessible at the stated location until at least one year after the last time you distribute an Opaque copy (directly or through your agents or retailers) of that edition to the public. 

	It is requested, but not required, that you contact the authors of the Document well before redistributing any large number of copies, to give them a chance to provide you with an updated version of the Document. 

	\subsection{Modifications}

	You may copy and distribute a Modified Version of the Document under the conditions of sections 2 and 3 above, provided that you release the Modified Version under precisely this License, with the Modified Version filling the role of the Document, thus licensing distribution and modification of the Modified Version to whoever possesses a copy of it. In addition, you must do these things in the Modified Version: 
	
	\begin{itemize}
		\item Use in the Title Page (and on the covers, if any) a title distinct from that of the Document, and from those of previous versions (which should, if there were any, be listed in the History section of the Document). You may use the same title as a previous version if the original publisher of that version gives permission. 

		\item List on the Title Page, as authors, one or more persons or entities responsible for authorship of the modifications in the Modified Version, together with at least five of the principal authors of the Document (all of its principal authors, if it has less than five). 

		\item State on the Title page the name of the publisher of the Modified Version, as the publisher. 

		\item Preserve all the copyright notices of the Document. 

		\item Add an appropriate copyright notice for your modifications adjacent to the other copyright notices. 

		\item Include, immediately after the copyright notices, a license notice giving the public permission to use the Modified Version under the terms of this License, in the form shown in the Addendum below. 

		\item Preserve in that license notice the full lists of Invariant Sections and required Cover Texts given in the Document's license notice. 

		\item Include an unaltered copy of this License. 

		\item Preserve the section entitled "Revision History", and its title, and add to it an item stating at least the title, year, new authors, and publisher of the Modified Version as given on the Title Page. If there is no section entitled "History" in the Document, create one stating the title, year, authors, and publisher of the Document as given on its Title Page, then add an item describing the Modified Version as stated in the previous sentence. 

		\item Preserve the network location, if any, given in the Document for public access to a Transparent copy of the Document, and likewise the network locations given in the Document for previous versions it was based on. These may be placed in the "History" section. You may omit a network location for a work that was published at least four years before the Document itself, or if the original publisher of the version it refers to gives permission. 

		\item In any section entitled "Acknowledgements" or "Dedications", preserve the section's title, and preserve in the section all the substance and tone of each of the contributor acknowledgements and/or dedications given therein. 

		\item Preserve all the Invariant Sections of the Document, unaltered in their text and in their titles. Section numbers or the equivalent are not considered part of the section titles. 

		\item Delete any section entitled "Endorsements". Such a section may not be included in the Modified Version. 

		\item Do not retitle any existing section as "Endorsements" or to conflict in title with any Invariant Section.

		\item If the Modified Version includes new front-matter sections or appendices that qualify as Secondary Sections and contain no material copied from the Document, you may at your option designate some or all of these sections as invariant. To do this, add their titles to the list of Invariant Sections in the Modified Version's license notice. These titles must be distinct from any other section titles. 

		\item You may add a section entitled "Endorsements", provided it contains nothing but endorsements of your Modified Version by various parties--for example, statements of peer review or that the text has been approved by an organization as the authoritative definition of a standard. 

		\item You may add a passage of up to five words as a Front-Cover Text, and a passage of up to 25 words as a Back-Cover Text, to the end of the list of Cover Texts in the Modified Version. Only one passage of Front-Cover Text and one of Back-Cover Text may be added by (or through arrangements made by) any one entity. If the Document already includes a cover text for the same cover, previously added by you or by arrangement made by the same entity you are acting on behalf of, you may not add another; but you may replace the old one, on explicit permission from the previous publisher that added the old one. 

		\item The author(s) and publisher(s) of the Document do not by this License give permission to use their names for publicity for or to assert or imply endorsement of any Modified Version.
	\end{itemize} 

	\subsection{Combining Documents}
	You may combine the Document with other documents released under this License, under the terms defined in section 4 above for modified versions, provided that you include in the combination all of the Invariant Sections of all of the original documents, unmodified, and list them all as Invariant Sections of your combined work in its license notice. 

	The combined work need only contain one copy of this License, and multiple identical Invariant Sections may be replaced with a single copy. If there are multiple Invariant Sections with the same name but different contents, make the title of each such section unique by adding at the end of it, in parentheses, the name of the original author or publisher of that section if known, or else a unique number. Make the same adjustment to the section titles in the list of Invariant Sections in the license notice of the combined work. 

	In the combination, you must combine any sections entitled "History" in the various original documents, forming one section entitled "History"; likewise combine any sections entitled "Acknowledgements", and any sections entitled "Dedications". You must delete all sections entitled "Endorsements." 

	\subsection{Collections of Documents}
	You may make a collection consisting of the Document and other documents released under this License, and replace the individual copies of this License in the various documents with a single copy that is included in the collection, provided that you follow the rules of this License for verbatim copying of each of the documents in all other respects. 

	You may extract a single document from such a collection, and distribute it individually under this License, provided you insert a copy of this License into the extracted document, and follow this License in all other respects regarding verbatim copying of that document. 

	\subsection{Aggregation with independent Works} 
	A compilation of the Document or its derivatives with other separate and independent documents or works, in or on a volume of a storage or distribution medium, does not as a whole count as a Modified Version of the Document, provided no compilation copyright is claimed for the compilation. Such a compilation is named an "aggregate", and this License does not apply to the other self-contained works thus compiled with the Document, on account of their being thus compiled, if they are not themselves derivative works of the Document. 

	If the Cover Text requirement of section 3 is applicable to these copies of the Document, then if the Document is less than one quarter of the entire aggregate, the Document's Cover Texts may be placed on covers that surround only the Document within the aggregate. Otherwise they must appear on covers around the whole aggregate. 
	
	\subsection{Generating this document}
	You will need three three things to generate this document for yourself:
	\begin{enumerate}
		\item Install \href{https://miktex.org/}{MiKTeX}

		\item Install \href{http://www.xm1math.net/texmaker/index_fr.html}{TeXMaker}

		\item An internet connection

		\item Configure MikTeX as indicated in the remarks at the beginning of the \textit{LaTeX\_SciencesCh\_EN.tex} file
	\end{enumerate}

	\subsection{Translation}
	Translation is considered a kind of modification, so you may distribute translations of the Document under the terms of the corresponding section about transformation. Replacing Invariant Sections with translations requires special permission from their copyright holders, but you may include translations of some or all Invariant Sections in addition to the original versions of these Invariant Sections. You may include a translation of this License provided that you also include the original English version of this License. In case of a disagreement between the translation and the original English version of this License, the original English version will prevail. 

	\subsection{Termination}
	You may not copy, modify, sublicense, or distribute the Document except as expressly provided for under this License. Any other attempt to copy, modify, sublicense or distribute the Document is void, and will automatically terminate your rights under this License. However, parties who have received copies, or rights, from you under this License will not have their licenses terminated so long as such parties remain in full compliance. 

	\subsection{Future revisions of this License}
	The Free Software Foundation may publish new, revised versions of the GNU Free Documentation License from time to time. Such new versions will be similar in spirit to the present version, but may differ in detail to address new problems or concerns. See \href{http://www.gnu.org/copyleft/}{{\color{blue} http://www.gnu.org/copyleft/}}. 

	Each version of the License is given a distinguishing version number. If the Document specifies that a particular numbered version of this License "or any later version" applies to it, you have the option of following the terms and conditions either of that specified version or of any later version that has been published (not as a draft) by the Free Software Foundation. If the Document does not specify a version number of this License, you may choose any version ever published (not as a draft) by the Free Software Foundation. 
	
	\begin{center}
	\color{ForestGreen}{{\Large \faTree} \textbf{Please consider the environment before printing}}
	\end{center}
	
	%to make section start on odd page
	\newpage
	\thispagestyle{empty}
	\mbox{}
	\section{Roadmap}
	This book has a simple progression rule that is: $1$ new A4 page per day since May 2001 on subjects that interest the supervisor of the \textit{Original} distribution of the book \textit{Opera Magistris}. The following subjects below are already planned for a near or far future still with the same level of details and pedagogical approach in the proofs as the rest of the book (all the subjects below should take almost $1,000$ pages):
	\begin{itemize}
		\item Introduction:
			\begin{itemize}
				\item Add acknowledgments to all scientific people that contributed to the theorems and models presented in this book (including thereof book/articles authors)
				\item Add acknowledgments to all people that created MikTeX/LaTeX distro and packages used for this book
			\end{itemize}
		\item Probabilities:
			\begin{itemize}
				\item Baysian conjugation for Normal and Binomial law
				\item Hidden Markov Chains
				\item Log-loss
			\end{itemize}
		\item Statistics: 
			\begin{itemize}
				\item Mode and Median of statistical laws				
				\item Partial and semi-partial correlation
				\item M-Estimators for localization and for dispersion
				\item Likelihood of censored data
				\item Normal Law Entropy
				\item Propension score
				\item Equivalence test
				\item Factorial Analysis
				\item Hotelling T-Test
				\item Standardized Person residuals
				\item Welch Test with Welch-Satterhwaite equation
				\item ANCOVA
				\item Wald-Wolfowitz Test (binary sequence)
				\item Levene-Wolfwitz Test\footnote{also named "turning point test" or "trend test"} (continuous up/down sequence)
				\item Risk Ratio and its confidence interval
				\item Ellipse of control
				\item Entropy-based measures of contingency tables
				\item Armitage trend test
				\item Ansari-Bradley test
				\item Regular Dickey-Fuller test
				\item Poisson Model for the average (2D) spatial distance
				\item Canonical Correlation
				\item G-test of periodicity
				\item Gaussian and Student copula
				\item Introduction to MANOVA
				\item Extreme Values Theorem
				\item Survey Theory
				\item Generalized Linear Models (Gauss, Poissson, Negative Binomial, Gamma)
				\item PLS Regression (partial least squares)
				\item Two Stage Least Squares (2SLS)
				\item Logic regression
				\item Adjusted Chi-squared
				\item Probability of generating functions
			\end{itemize}
		\item Sequences and Series:
			\begin{itemize}
				\item Properties of Fourier transforms	
				\item Discrete Fourier transform			
				\item Laplace Transform
				\item Z transform (common Z transforms, inverse common Z transforms)
			\end{itemize}
		\item Geometry:
			\begin{itemize}
				\item Hypersphere volume
			\end{itemize}
		\item Differential Calculus
			\begin{itemize}
				\item Lebesgue Integral with numerical application in MATLAB™
				\item Integral representation of Bessel functions
				\item Linearization of power trigonometric functions
				\item Elliptic integrals and Elliptic functions
			\end{itemize}
		\item Complex Analysis: 
			\begin{itemize}
				\item Residue Theorem for polynomial ratios
				\item Gauss' mean value theorem
			\end{itemize}
		\item Topology: 
			\begin{itemize}
				\item Mahalanobis Distance
			\end{itemize}			
		\item Differential Geometry: 
			\begin{itemize}
				\item Normal coordinates
				\item Gauss curvature
				\item Circumference on a plane, sphere and hyperbolic surface
				\item Isoperimetric plane theorem
			\end{itemize}
		\item Mechanics: 
			\begin{itemize}
				\item Magnus effect
				\item Lawson Plasma criteria
				\item Flow through submerged orifice
				\item Flow over notches and weirs
				\item Force due to the flow around a pipe bend
				\item Force on a pipe nozzle
				\item Impact of a jet on plane
				\item Pelton wheel blade
				\item Force due to a jet hitting an inclined plane
				\item Fluid Hammer Effect
				\item Saint-Venant equations
				\item Kutta-Joukowski lift
			\end{itemize}	
		\item Electrodynamics:
			\begin{itemize}		
				\item Lorentz oscillator model
				\item Electromagnetic Fields of a rotating shell of charge
				\item Sellmeier equation proof
				\item Antenna radiation
				\item Brehmstrahlung (started but not finished)
				\item Michelson interferometer principle
				\item Electromagnet rotor torque
				\item Rayleigh scattering
			\end{itemize}
		\item Electrokinetics:
			\begin{itemize}		
				\item Photoelectric converter
				\item PN junctions
			\end{itemize}
		\item Optical Wave:
			\begin{itemize}		
				\item Fresnel Diffraction
				\item Fraunhoffer Diffraction
				\item Optical Fiber basics
			\end{itemize}
		\item Astronomy:
			\begin{itemize}	
				\item MacCullagh's formula
				\item Body flatness indirect calculation
				\item Synchronous locking of tidally evolving satellites	
				\item Flateness spheroid
				\item Sphere of influence
				\item Planetary departure
			\end{itemize}		
		\item General Relativity:
			\begin{itemize}
				\item Real volume of an object in General Relativity
				\item Einstein radius derivation
				\item General Birkhoff's theorem
				\item Global Positioning System (GPS)		
			\end{itemize}
		\item Cosmology:
			\begin{itemize}
				\item Finish text about Friedmann Universe models based on General Relativity (started but not finished)
				\item Cosmic microwave background theoretical temperature	
				\item Kaluza-Klein theory
			\end{itemize}
		\item Nuclear Physics:
			\begin{itemize}
				\item Rayleigh diffusion (Rayleigh scattering)
				\item Neutron transport
			\end{itemize}
		\item Wave Quantum Physics:
			\begin{itemize}
				\item Kennard inequality of Heisenberg incertitude
				\item Quantum tunelling time
				\item Bell inequalities
				\item Lamb shift detailed calculation
				\item Davisson-Germer Experiment
				\item EPR paradox formalism
			\end{itemize}
		\item Relativistic Quantum Physics:
			\begin{itemize}
				\item Photon "spin" (helicity) and polarity relation
				\item Parity, charge conjugation and time reversal (CPT)
			\end{itemize}
		\item Quantum Chemistry:
			\begin{itemize}
				\item Valence shell electron pair repulsion theory
				\item Sackur–Tetrode equation derivation
			\end{itemize}
		\item Numerical Methods: 
			\begin{itemize}
				\item Univariate optimization problem with substitution method					
				\item Acceptation/Rejection Sampling
				\item Gibbs Sampling				
				\item Outliers vs Influential values
				\item Cronbach coherence indicator
				\item Linear discriminant Analysis (LDS)
				\item Quadratic discriminant Analysis (QDA)
				\item Multidimensional scaling (MDS)
				\item Linear Mixture Model (LMM)
				\item Mean Shift
				\item Factorial Analysis (FA)
				\item Correspondence Factorial Analysis
				\item GRG Generalized Reduced Gradient (GRG) optimization method
				\item Normalized mutual information
				\item Support Vector Machine (SVM)
				\item Chi-squared Automatic Interaction Detector (CHAID)
				\item Latent semantic analysis (LSA)
				\item Importance sampling
				\item Stratified sampling
				\item Monte Carlo control variable
				\item Correlation based feature selection
				\item ID3, PRISM, AQ, CN2 and C4.5 algorithms
				\item Procrustes analysis
				\item independent component analysis (ICA)
			\end{itemize}
		\item Quantum Computing: 
			\begin{itemize}
				\item No-cloning theorem
			\end{itemize}
		\item Cryptography: 
			\begin{itemize}
				\item Elliptic curves
			\end{itemize}	
		\item Engineering:
			\begin{itemize}
				\item Box Domains
				\item Definitive Screening Design
				\item Split-plots designs				
				\item Central Composite Design
				\item Center Face Cube Design
				\item Cox Survival Model (Cox Proportional Hazard Model)
				\item Modelization by Structural Equations				
				\item Accelerated life testing
				\item Microelectronics (npn/pnp jonctions, diodes, amplifiers)
				\item Telegraph equation
				\item Kutta-Joukowski lift theorem
			\end{itemize} 
		\item Game Theory: 
			\begin{itemize}
				\item Coalition
				\item Shapley value
				\item Kelly criterion 
			\end{itemize}
		\item Economy: 
			\begin{itemize}
				\item Continuous Yield rate
				\item Zero-Coupon curve rates
				\item Equivalence of an obligation rate for a treasure bond
				\item Spot rate and Forward rate
				\item Adjusting the beta of a portfolio with Futures
				\item Cox-Ingersoll Future/Forward price equality 
				\item Solution of Black \& Scholes ODE
				\item Macaulay Duration 
				\item Modified Duration
				\item Modified Internal Rate of Return (MIRR)
				\item Options Portfolio hedging
				\begin{itemize}
					\item Protective Put/Call
					\item Bull Spread/Call
					\item Bear Spread/Call
					\item Butterfly
					\item Straddle
					\item Strangle
					\item Collar
					\item Box spreads
					\item Calendar spreads
					\item Portfolio allocation methods
					\begin{itemize}
						\item Optimal weighted portfolio for balanced risk 
						\item Optimal weighted portfolio for error tracking
						\item Optimal weighted Sharp's portfolio
						\item Optimal weighted portfolio with maximum diversification
						\item Optimal market-bench weighted Treynor-Black Portfolio
					\end{itemize}
				\end{itemize}
				\item Greeks for binomial trees
				\item Swaps
				\item Margrabe's formula
				\item Kirk's (approximation) formula
				\item Default Credit Risk (based on Standard \& Poor rating)
				\item CreditRisk+
				\item VaR Equity Coverage
				\item Condition VaR loss (CVaR)
				\item KMV-Merton approach for measuring probabilities of default
				\item Distance to default
				\item Fokker-Planck equation
				\item ARCH-GARCH stochastic process
				\item Vector autoregressive models for multivariate time series
				\item Granger Test for times series causation
				\item Karman Filter
				\item Heston option pricing model
				\item Spread option pricing
			\end{itemize}	
		\item Quantitative Management: 
			\begin{itemize}
				\item Gale-Shapley Algorithm
				\item Newsvendor problem
				\item Bullwhip Effect
				\item Condorcet paradox	
				\item Computerized Relative Allocation of Facilities Technique (CRAFT)
				\item Real options
				\item Differed Capital in living case (life assurance)
				\item CBOE's Volatility Index (started but not finished)
				\item Death differed temporary (life assurance)
			\end{itemize}
	\end{itemize}
	Remember that the \LaTeX{} sources of this book can be obtained actually depending on your donation on Patreon, Paypal, Tipee or if you participate to the translation of the book in another language..
	
	As every robust product has a lifecycle. The lifecycle begins when a product is released and ends when it's no longer supported. Knowing key dates in this lifecycle helps you make informed decisions about when to upgrade. This book has the following lifecycle: a new major or minor version is published every time a given threshold of donations is reached and can be downloaded by clicking on the following button ($410$ Megabytes PDF...):
	\begin{center}
		\href{http://www.sciences.ch/htmlfr/php/cliccount/click.php?id=317}{\includegraphics[scale=0.6]{img/books/download.jpg}}
	\end{center}
	or if this link would not work, a copy of the PDF is available on the Internet Archive:
	\begin{center}
		\includegraphics[scale=0.1]{img/internet_archive.jpg}
	\end{center}
	\begin{center}
	\href{https://archive.org/details/OperaMagistris}{https://archive.org/details/OperaMagistris}
	\end{center}
	
	To quote this book:
	\begin{quote}
	\noindent @book\{OperaMagistris2014v3, \\
		  author =       \{Vincent Isoz and Léon Harmel\}, \\
		  title =        \{Opera Magistris - Elements of Applied Mathematics for Engineers\}, \\
		  year =         \{2014\}, \\
	      publisher=     \{Sciences.ch\}, \\
		  keywords =     \{science, physics, maths, engineering, finance, management\}, \\
		  isbn =          \{978239909327\},\\
	\}
	\end{quote}
	
	
	\chapter{Acknowledgements}

	The ideas in this book have been developed and reinforced by many people. I have greatly benefited from my regular interactions with  students (university, engineering schools) and executives from all backgrounds, including CEOs, CFOs, PMs of many companies around the world, teaching sessions, developing company-specific programs, consulting, and even informal conversations. I am grateful to them for sharing their wisdom with me and inspiring many of the ideas in the book.

	This book and its companion website would not have been possible without the valuable support of the people mentioned below. They find here the expression of my gratitude (and for sure if some errors remains in this book this is obviously their fault...):
	
	\begin{itemize}		
		\item \textbf{Harmel Léon }(†2012), graduate electrical engineer with a specialization in electronics and automation, responsible in the physical research laboratory at ACEC in Charleroi (BEL), for the provision of documentation that was used in the sections of Corpuscular Quantum Physics, Wave Quantum Physics,  Quantum Field Theory, Spinor calculus and General Relativity.
		
		\item \textbf{Legrand Mathias}, Ph.D. École Centrale de Nantes (FRA) for his help on the redaction of the first $550$ pages of the \LaTeX{} Book version of the website.
		
		\item \textbf{Ricchiuto Ruben}, engineer degree in Physics HES (B.Sc.) from the Engineering School of Geneva (CHE) and mathematician from the University of Geneva for his valuable help in plasma physics, electromagnetism, quantum physics, statistics, topology, quantum chemistry, fractals theory, analysis and many other areas affecting pure mathematics and computing.
		
		\item Regulars users of \href{http://www.les-mathematiques.net}{{\color{blue} Les Mathematiques.net}} and \href{http://www.futura-sciences.com}{{\color{blue} Futura-Sciences.com}} forum, for their valuable assistance in many areas of mathematics and physics. The debates and discussions that took place on the forums helps to constantly improve the educational aspect of this book.
		
		\item The \href{http://www.wikipedia.com}{{\color{blue} Wikipedia}} and \href{http://www.planetmath.com}{{\color{blue} PlanetMath}} websites to whom I am indebted to many borrow almost word by word (and this is mutual...).
		
		\item The professors Gabriel Nagy of Michigan State University , Liam Revell of Massachusetts Boston University, Jorge Cham (of PhD Comics) and the OpenStax team that have received \underline{and} read my multiple e-mails inquiries about images credits and texts credits but that never answered to me (not quite good examples of what should be the "scientific collaboration and spirit"...)
	\end{itemize}
	
	And thanks to all readers, webmasters and teachers for their websites and quality documents available for free and anonymously on the Internet and regular forum stakeholders. I sometimes verbatim recovered their explanations that do not require additions or corrections. It's probably needless to say that you should not assume that these people are in total agreement with the scientific purposes views expressed in this book; and are not responsible for any errors or obscurities that you might accidentally find in it.
	
	Thanks also to the few colleagues and customers who were willing to give me their comments to improve the content of this book. However, it is certain that it can still be improved on many points.
	
	I would like finally to thank especially all of my family for their continued support and my friends for their patience as I was almost completely absent, but I would like to send a special thanks to my Dad and Mom, for all of her incredibly help and support over the last months of translation of this book! I would like also to apologize to some of my customers and colleagues because as I answered very slowly to their e-mails and phones during thirteen months to better focus on the translation of this book. Thanks also to my girlfriend for always being there to take care of me when I forget to take care of myself...
	
	For any public feedback or comment you can use the guestbook associated to this PDF (for questions please use the forum!):
	\begin{center}
	\url{http://www.sciences.ch/htmlen/guestbook.php}
	\end{center}
	or if you want to do a private feedback or comment you can contact me by {\href{mailto:info@sciences.ch}{{\color{blue}email}}}.
	
\chapter{Introduction}
	%To add line numbers on all paragraphes each 5 lines if necessary	
	%\linenumbers
	
	\minitoc
	\pagebreak
		%to make section start on odd page
	\newpage
	\thispagestyle{empty}
	\mbox{}
	\parpic[l][t]{%
	  \begin{minipage}{30mm}
	    \fbox{\includegraphics[width=80px,height=100px]{img/einstein.eps}}
	  \end{minipage}
	}		
	This book who first Edition has been published in 2001 is designed so that the knowledge required to read it is as basic as possible. It is not necessary to have a Ph.D. to consult it, you just have to know reasoning, to think critically, to observe and have time...
	\begin{flushright}
	\textit{"Simplicity is the seal of truth and it radiates beauty"} \\
	 Albert EINSTEIN
	\end{flushright}
	
	\section{Forewords}
	No human endeavor has had more impact than Science\footnote{From Latin \textit{scientia} "knowledge, a knowing, expertness". Itself from \textit{sciens} (genitive scientis) that means "intelligent, skilled", present participle of \textit{scire} that means "to know" probably originally comes from "to separate one thing from another, to distinguish" related to \textit{scindere} "to cut, divide".} on our lifes and our conception of the world and ourselves. Its theories, conquests and results are all around us.

	Omnipresent in the industry (aerospace, imaging, cryptography, transportation, chemistry, algorithmic, etc.) or in the services (banking, fintech, insurance, human resources, projects, logistics, architecture, communications, etc.), Applied Mathematics also appears in many other areas: surveys, risk modeling, data protection, politics, etc.  Applied Mathematics (also sometimes named "Mathematics Machinery") influence our lives (telecommunications, transport, medicine, meteorology, music, project management) and contribute to the resolution of current issues: energy, health, environment, climate, optimization, sustainable development, etc. much more than any soft skill techniques or methodology! They great success are their fabulous dispersion in the real world and their increasing integration in all human and artificial intelligence activities. We are going therefore to a situation where mathematicians and engineers will no longer have the monopoly of mathematics, but where almost any graduate job position will have to do advanced mathematics.

	As a former student in the field of engineering I have often regretted the absence of a single book fairly comprehensive, detailed (without going to the extreme...) and educational if possible free (!) and portable (being personally a fan of eBooks...) containing at least a non exhaustive idea of the overall program of Applied Mathematics in engineering schools with an overview of what is used for real in companies with more intuitive than rigorous proofs but with enough details to avoid unnecessary effort to the reader. Also a book that does not require the reader to adopt each time a new notation or terminology specific to the author when it is not outright to change to a foreign language... and where anyone can suggest improvements or additions (through the forum, guest-books or by e-mail).

	I was also frustrated during my studies to have quite often have to swallow "formulas" or "laws" supposedly (and wrongly) non-provable or too complicated as my teachers says or even disappointed by renowned authors books (where developments which are left to the reader or as exercise and no real applications are even mention...). In this book predominates the will to never confuse the reader with empty sentences like "it is evident that...", "it is easy to prove that...", "we leave it to the reader as an exercise...", since all developments are presented in detail. But I'm not a purist of maths! I have only one ambition: to explain the easiest way possible.

	Although I have to admit that prove some mathematical relations presented within the engineering schools curriculum can not be done because of a lack of time in the official program or size limit in a book, I can not accept that a teacher or author tells his students (respectively, his readers) that certain laws are non-provable (because most of the time this is not true!) or that such or such proof is too complicated without giving a reference (where the student can find the information necessary to satisfy his curiosity) or at least a simplified but satisfactory proof.

	Moreover, I think that it is totally archaic today that some teachers continue to ask to their students to take a massive quantity of notes during classes. It would be much more favorable and optimal to distribute a course handout containing all the details in order to be able to concentrate on the essentials points with students, that is to say the oral explanations, interpretations, understanding, reasoning and practice rather than excessive blackboard copy... Obviously by giving a complete course handout some students will be brilliant by their absence but ... it is the better! Thus, those who are passionate can deepen subjects at home or at the university library, the weak do what they have to do and the rest (struggling students but workers) will follow the course given by the teacher to profit to ask questions rather than mindlessly copying a blackboard.

	Inspired on a learning model of an American scholar, whose I forgot the name (...), this book proposes and imposes the following properties to the reader: discover, memorize, cite, integrate, explain, restate, infer, select, use, decompose, compare, interpret, judge, argue, model, develop, create, search, reasoning, develop in a clear progressive teaching way to develop the analytic skills and openness.

	So, in my mind, this non-exhaustive book (and its associated companion PDFs) must be a substitute, free of charge for all students and employees around the World, to many references and gaps of the scholar system, allowing any curious student not to be frustrated for many years during his academic curriculum. Otherwise, the science of the engineer could have the aspect of a frozen science, apart from the scientific and technical developments, a heteroclit accumulation of knowledge and especially of formulas which made he considered as a tasteless subproduct of mathematics and that brings companies and governments to many false results and bad decisions...
	
	This book has also been designed to meet the needs of executives, both finance as well as non-finance managers. Any executive who wants to probe further and grasp the fundamentals of strategic finance, strategic marketing or project management engineering and supply chain issues will benefit from its lecture. 
	
	This book has also for purpose to describes and explains how our Universe and our World (also other "worlds" in our Universe) works in a much more accurate, more complete and detailed way than any Holy book. It gives models and quantification methods for the origin of species, of galaxies, of planets, of quantum phenomenon, of physics movements, of stellar physics, of extreme observable events and also extreme rare events and explains social strategies and modern technologies in a mathematical and provable way that everyone can check by himself and by exposing every-time the assumptions that any reasonable entity should take care of!
	
	Obviously Applied Mathematics is such an abundant topic that a book of this scale can only accommodate the basis. Readers are certainly encourage to go beyond this (see the bibliography at the end of the book).

	Now, those who see Applied Mathematics only as a tool (what it also is), or as the enemy of religious beliefs, or as a boring school field school, are legion. However, it is perhaps useful to recall that, as Galileo said, "\textit{the book of nature is written in the language of mathematics}" (without wishing to do scientism!). When you go to China, you learn chinese. When you want go to the Universe you learn maths! Because maths are the language of the Universe. This is why maths are so fundamental and their are amazing as they apply to the whole Universe and across time. It is in this spirit that this book discusses Applied Mathematics for students in the Natural, Earth and Life sciences, as well as for all those who have an occupation related to the various subjects including philosophy or for anyone curious to learn about the involvement of science in everyday life.

	The choice to study engineering in this book as a branch of Applied Mathematics comes from the fact that the differences between all areas of physics (formerly known as "natural philosophy") and mathematics are so hardly notable that Fields medal (the highest award today in the field of mathematics) was awarded in 1990 to physicist Edward Witten, who used physical ideas to prove a mathematical theorem. This trend is certainly not fortuitous, because we can observe that all science, since it seeks to achieve a more detailed understanding of the subject it studies, always finish its trials in the pure mathematics (the absolute path by excellence ...). Thus, we can predict in a far future, the convergence of all the sciences (pure, exact or social) to the mathematics for the modelisation techniques (see for example the French PDF "\textit{L'explosion des mathématiques}" available in the download page of the companion website).

	It can sometimes seem to us difficult (due to irrational as obscure and unjustified fear of pure sciences in a large fraction of our contemporaries) to transmit the feeling of the mathematical beauty of nature, its deepest harmony and the well-oiled mechanics of the Universe, to those who know only the basics of algebra. The physicist Richard Feynman spoke a day of "two cultures": people who have and those who do not have sufficient understanding of mathematics to appreciate the scientific structure of nature. It is a pity that mathematics are necessary to deeply understand nature and that they also have a bad reputation. For the record, it is claimed that a King who asked Euclid to teach him geometry complained about its difficulty. Euclid replied, "There is no royal road". Physicists and mathematicians can not convert themselves to a different language. If you want to learn about nature, to appreciate its true value, you must understand its language. The nature is revealed only in this form and we can not be pretentious to the point of asking him to change this fact.

	In the same way, no intellectual discussion will allow you to communicate with a deaf person what you really feel while listening music. Similarly, all discussion of the world remain powerless to transmit an intimate understanding of the nature of those of the "other culture". Philosophers and theologians may try to give you qualitative ideas about the Universe. The fact that the scientific method (in the full sense of the term) can not convince the world of its truth and purity, is perhaps the fact of the limited horizon of some people who imagine that the human or another intuitive concept, sentimental or arbitrarily is the center of the Universe (anthropocentric principle).

	Of course, in order to share this mathematical knowledge, it may seem paradoxical to increase, with our work, the long list of books already available in libraries, in commerce and on the Internet. Nevertheless, I must be able to present arguments that justifies the creation of such a book (and its associated website) as compared to books such as Feynman, Landau or Bourbaki and Wikipedia/Wolfram themselves or Khan Academy or OpenStax. So what do I think I can add to such a wealth of material? 
	\begin{enumerate}
		\item The great pleasure that we take to write this book ("keep the hand" and improve our skills) and have a detailed high quality compendium of tools for our customers and our students (and also all those around the World) for free.

		\item The passion for sharing knowledge for free (battle again "copyright madness" (RIP Aaron Swartz!)) and without frontiers with a tool of quality as \LaTeX{} (at the opposite of Wikipedia that mixes \LaTeX{} and normal text and the awful and shameful content of Khan Academy\footnote{OpenStax has good undergraduate PDF - especially the example in their books - but there are between 40-60\% of missing proofs and the table of contents of their PDF and also the Index are not interactive... and major issue...: the content is limited only to undergraduate subjects}).
		
		\item Support free scientific education, critical thinking, and evidence-based understanding of the natural world. Furthermore it is clear that there is an undeserved appetite for people to understand and this book has been written for this purpose.
		
		\item Write a modern 3rd millennial version of the "Almagest" hence the name "Opera Magistris" that means in English "Major Work".
		
		\item Because we can't wait as there are places in the world where the absence of teaching modern science and its methodology takes peoples to have believes that bring them to some dangerous and obscure paths.
		
		\item We want to offer Applied Mathematics in an enjoyable and easy-to-learn manner ("keep it simple and stupid" at the opposite of the $9$ Landau's graduate level books), because we believe that Applied Mathematics change the way we understand the Universe.
		
		\item This book was first written in French before (in year 2001) that the French version of Wikipedia had good mathematical content and long before Khan Academy or OpenStax did even exist.

		\item The quick updates/corrections opportunities (at the opposite of Khan Academy) and collaborations of a free e-book (with associated effective search tools) without having topics that disappears (at the opposite of Wikipedia).

		\item The content depending on readers requests/comments and on our interests (at the opposite of Khan Academy, OpenStax or Landau books)!
		
		\item At the opposite of Scientific publications (PRL or other similar) that sucks because don't give detailed proofs and sometimes turn in an infinite loop of references.
		
		\item The access to \LaTeX{} sources to everybody so nobody need to recreate the wheel and loose hundred or thousand of hours on redaction instead of innovation (at the opposite of Landau books)!

		\item Rigorous presentation with simplified detailed proofs of all presented concepts (at the opposite of Wikipedia, Khan Academy and OpenStax that focus only of the mathematical proofs of undergraduate concepts).

		\item The presentation of many advanced and detailed mathematical tools used in business and R\&D keeping in mind that the mathematical language seems eternal and to be one of the only common denominator between all countries in the World.

		\item The opportunity for students and teachers to reuse content by copy/paste (at the opposite of Khan Academy or Landau Books).

		\item Constant and fixed notation (at the opposite of Wikipedia, Khan Academy and OpenStax) throughout the book, for mathematical operators, a clear language on all topics (3.C. criterion: clear, complete and concise) and focus on the basics to make an important pedagogical work on the subjects (at the opposite of Landau's books).

		\item Gather as much information about pure and exact sciences in one electronic (portable), homogeneous and rigorous book (but that don't go as far as Landau's books).

		\item Release from all pseudo-truths, only truths that can be proven.

		\item Benefit from the development of teaching methods that use the Internet to search for the solution of mathematical problems.

		\item The dramatic improvement of automatic translation software and computing power that will make of this book, at least we hope, a reference in the fields of sciences.
		
		\item A PDF is better than a website as first all people that use the Internet since 1990 know that the huge majority of website disappear after ten years and secondly it is well know that some countries block Wikipedia and other knowledge website to keep their population in the ignorance (and block a PDF that can be shared in a e-mail is much more difficult).
		
		\item and... because Applied Mathematics are beautiful and especially when written in \LaTeX{} and illustrated (at the opposite of Landau books whose illustrations are quite old and poor).
\end{enumerate}

	And also ... I believe that the results of individual research are the property of humanity and should be available to all those who explore anywhere the phenomena of nature. In this way the work of each benefit to all, and that is for all humanity that our knowledge cumulates and this is the trend that allows Internet.

	I do not hide that my contribution is limited largely to this day to that of a collector who gleans his information in the works of masters or publications or from anonymous web pages and who completes and argues developments and improved them when this is possible. Therefore some of the material in this book is original, and some comes from primary literature. However the vast majority of what we wrote is a rephrasing of results presented in the existings vast library of some (rare) fantastic books. For those who would accuse me of plagiarism, they should think on the fact that the theorems presented in most non-free books and commercially available have been discovered and written by their predecessors and their own personal contribution was also made, like mine, to put all this information in a clear and modern form a few hundred years later. In addition, it can be seen as doubtful that we ask to pay for access to a culture that is certainly the only truly valid and fair one in this world and where there is no patent or intellectual property rights.

	This book also reflects my own intellectual limitations. Although I try to study as much science and math fields as possible, it is impossible to master them all. This book shows clearly only my own interests and experiences as consultant, but also my strengths and my weaknesses. I am responsible for the selection of inputs and, of course, of possible errors and imperfections.

	After attempting a strict (linear) order of presentation of the subject, I decided to arrange this book in a more pedagogical (thematic) way and always with practical examples o applications. It is in my opinion very difficult to speak of so vast subject in a purely mathematical order in only one human life, that is to say, when the concepts are introduced one by one, from those already known (where each theory, operator, tools, etc.. would not appear before its definition). Such a plan would require cutting the book, in pieces that are not more thematic. So I decided to present things in a logical order and not in order of need. Thus the reader will encounter, as the editor himself, to the extreme complexity of the subject.

	The consequences of this choice are the following:
	\begin{enumerate}
		\item Sometimes it will necessary to admit certain concepts, even to understand later.
	
		\item It will probably be necessary for the reader to go at least twice throughout the book. At the first reading, we apprehend the essential and at the second reading, we understand the details (I congratulate this who understand all the subtleties the first time).
	
		\item You must accept the fact that some topics are repeated and that there are many cross-references and complementary remarks.
	\end{enumerate}
	
	Some know that for every theorem and mathematical model, there are almost always several methods of proofs. I've always tried to choose the one that seemed the most simple (e.g. in relativity and quantum physics there is the algebraic and matrix formalism). The objective is to arrive at the same result anyway.
	
	This book being in its draft version, it necessarily has lacks on convergence controls, on continuity, grammar and others... (which will horrify some readers and mathematicians ...)! However, I have avoided (or, otherwise, I indicate it) the usual approximations of physics and the use of dimensional analysis, by using it as little as possible. I also try to avoid as much as possible subjects with mathematical tools that have not previously been presented and demonstrated rigorously.
	
	Finally, this presentation, that can still be improved, is not an absolute reference and contains errors. Any comment is welcome. I shall endeavour, as far as possible, to correct the weaknesses and make the necessary changes as soon as possible.
	
	However, while mathematics is accurate and indisputable, theoretical physics (its models), is still interpreted in the common vocabulary (but not in the mathematical vocabulary) and its conclusions all relative. I can only advise, when you read this book, to read by for yourself and not to be subjected to outside influences. You must have a very (very) critical mind, take nothing for granted and question everything without hesitation. In addition, the keyword of good scientist should be: "Doubt, doubt, doubt ... doubt still, and always checks.". We also recall that "nothing that we can see, hear, smell, touch or taste, is what it seems to be", therefore do not rely on your daily experience to draw hasty conclusions, be critical, Cartesian, rational and rigorous in your development, reasoning and conclusions!
	
	I want to say to those who would try to find themselves the results of some developments of this book, do not worry if they do not success or if they doubt about their competences because of the time spent solving an equation or problem: some theories that seem obvious or easy today, have sometimes needed several weeks, months, even years, to be developed by mathematicians or leading physicists in the past!
	
	I also tried to ensure that this book is pleasing to the eye and to read through.
	
	Finally, I have chosen to write this work in the first person plural form: "we". Indeed, the mathematical physics is not a science that has been made or has evolve through individual work but with intensive collaboration between people connected by the same passion and desire of knowledge. Thus, by making use of "we", I would like pay tribute to the dead and missing scientists, to contemporary and future researchers for the work they will perform in order to approach the truth and wisdom.
	
	\begin{center}
	\includegraphics[scale=0.7]{img/humour/pure_math_vs_applied_math.jpg}
	\end{center}

	%to make section start on odd page
	\newpage
	\thispagestyle{empty}
	\mbox{}
	\section{Methods}	
	Science is the set of all systematic efforts (scrupulous observations and plausible assumptions until the evidence of the contrary) to acquire knowledge about our environment, to organize and synthesize them into testable laws and theories, whose main purpose is to explain the "how" of things (and NOT the why!) often by a five-step approach:
	\begin{itemize}
		\item[$-$] What do we have?
		\item[$-$] Where will we go?
		\item[$-$] What is our goal?	
		\item[$-$] Does it fit the data?
	\end{itemize}
	Scientists have to submit their ideas and results to independent verification and replication of their peers ("\NewTerm{peer-review}\index{peer-review}"). They must abandon or modify their conclusions when confronted with more complete or different evidences. The credibility of Science is based therefore on this self-correcting mechanism and this is what still makes in the 21st century that Science is not the best tool (as we do not know what will exist in the future...) but is has been proven as being the best investigation method for truth in comparison for all other actual existing methods or beliefs. The history of science shows that this system works very long and very well compared to all the others. In each area, progress has been spectacular. However, the system sometimes failed and has also to be corrected before small drifts accumulate.

	The downside is that scientists are humans. They have the imperfections of all humans, and especially, vanity, pride, anger and conceit. Nowadays, it happens that many people working on the same topic for a given time develop a common faith and believe they hold the truth. The leader of the faith is the Pope and distills his opinion. The Pope that plays the game, takes his miter and his pilgrim's staff to evangelize his fellow heretics. Until then, this makes smile. But, as in real religions, they are sometimes annoying to want to expand their opinion to those who do not believe. Some of these "churches" do not hesitate to behave like the Inquisition. Those who dare to express a different opinion are burned at every opportunity, during conferences, or at their place of work. Some young researchers, uninspired, prefer to convert to the dominant religion, to become clerics faster rather than innovative researchers or even iconoclasts. The great Pope write his Bible to disseminate his ideas, imposes it to read to students and newcomers. He formats then the thought of younger generations and ensures his throne. This is a medieval attitude that can block progress. Some Popes go so far that they believe be the pope in their specialization field automatically gives them the same throne in all other areas...

	This warning, and the reminders that will follow, must serve the scientific or any reader to ask himself by making good use of what we consider today as the good working/reasoning practices (we will discuss the principles of the Descartes method more below) to solve problems or develop theoretical models.

\pagebreak
For this purpose, here is a summary table that provides the steps that should be followed by a scientific who works in mathematics or theoretical physics (for definitions, see just below):

	\begin{table}[!ht]
	\begin{center}
		\definecolor{gris}{gray}{0.85}
			\begin{tabular}{|p{7.5cm}|p{7.5cm}|}
				\hline
				\multicolumn{1}{c}{\cellcolor{black!30}\textbf{Mathematics}} & 
  \multicolumn{1}{c}{\cellcolor{black!30}\textbf{Physics}} \\ \hline
				\textbf{1.} Expose formally or in common language the "hypothesis", the "conjecture" the "property" to prove (hypothesis are denoted H1., H2., etc. the conjectures CJ1., CJ2., etc. and the properties P1., P2., etc.). & \textbf{1.} Expose correctly in a formally or common language all the details of the "problems" to solve (problems are denoted P1., P2., etc.). \\ \hline
				\textbf{2.} Define the "axioms" (non-demonstrable, independent and non-contradictory) that will give the starting points and establish restrictions on development (the axioms are denoted A1., A2, etc.)\footnotemark. \newline\newline
In the same vein, the mathematicians defines the specialized vocabulary related to mathematical operators which will be denoted by D1., D2., etc. & \textbf{2.} Define (or state) the "postulates" or "principles" or the "hypothesis" and "assumptions" (supposedly unprovable...) that will give the starting point and establish restrictions on the developments (typically, assumptions and principles are denoted P1., P2., etc. and assumptions H1., H2., etc. trying to avoid the notation confusion between postulates and principles)\footnotemark. \\ \hline
				\textbf{3.} Once the Axioms laid, pull directly "lemmas" or "properties" whose validity follows directly and prepare the development of theorem supposed to validate departure hypothesis or conjectures (Lemmas being denoted L1., L2., etc. and properties P1., P2., etc.). & \textbf{3.} Once the "theoretical model" developed, check equations units for possible errors in the developments (such checks being marked VA1., VA2., etc.).\\ \hline
				\textbf{4.} Once the "theorems" (noted T1., T2., etc.) prooved conclude on "consequences" (denoted C1., C2., etc.) and even properties (noted P1., P2., etc.). & \textbf{4.} Search for borderline cases (including "singularities") of the model to verify the validity intuitively (these borderline controls are denoted CL1., CL2., etc.).\\ \hline
				\textbf{5.} Test the strength (robustness) or usefulness of the conjectures or hypothesis by proving the reciprocal of the theorem or by comparing them with other examples of mathematical well-know theories to see if form together a coherent structure (examples being denoted E1., E2., etc.). & \textbf{5.} Experimentally test the theoretical model obtained and submit work to compare with other independent research teams. The new model should provide experimental results and never observed (predictions to falsify). If the model is validated then it is the official status of "theory".\\ \hline
				\textbf{6.} Possible remarks may be shown in a hierarchically structured order and noted R1., R2., etc. & \textbf{6.} Possible remarks may be shown in a hierarchically structured order and noted R1., R2., etc.			
				\\ \hline
		\end{tabular}
	\end{center}
	\caption{Methodology for Maths \& Physics Developments}
	\end{table}	
	\footnotetext[1]{Sometimes "properties", "conditions" and "axioms" are confused while the concept of axiom is much more accurate and profound.}
	\footnotetext[2]{You should not forget, however, that the validity of a model is not dependent on the realism of its assumptions but on the conformity of its implications with reality.}	
	
Proceed as in the above table is a possible workflow basis for people active in the field in mathematics or physics. Obviously, proceed cleanly and traditionally as above takes a little more time than doing things no matter how (this is why most teachers do not follow these rules, they don't have enough time to cover the entire course program) and this is one of the reasons why science takes a waste majority of people outside the comfort zone (as most people are look to fix problems and interrogation in less than $2$ minutes).

\begin{center}
\includegraphics[scale=0.75]{img/intro/hypothesis_definitions.eps}
\end{center}
I must be also be known to the reader that we insist on the fact that real scientific should no have emotions behind the subjects they study or speak about. They have to only use evidence (facts based on data, peer-review, reproducible experiences, consensus of scientific community) rather than emotional, biased, subjective educational individual analysis that are not data driven.

Notice also a fun shape of scientific $8$ commandments:
\begin{enumerate}
\item The phenomenas you will observe\\
And never measures you will falsify\\
(attention to the confirmation error: study only phenomena that validate your belief)

\item Hypothesis you will proposed\\
That with experiment you will test

\item The experiment precisely you will describe\\
Because your colleague will reproduce it\\
(attention to the narrative discipline trap: the facts will be fitted to the desired results)

\item With your results\\
A theory you will build

\item Parsimony you will use\\
And the simplest hypothesis you will retain

\item Ultimate truth will never be (epistemic humility)\\
And always you will search for the truth

\item From a non-refutable thesis you will refrain\\
Because outside of the science it will remain

\item All failures will be like a success\\
Because science can confirm but also invalidate
\end{enumerate}

	\begin{tcolorbox}[title=Remarks,colframe=black,arc=10pt]
\textbf{R1.} Caution! It is very easy to make new physical theories by just aligning words. This is named "\NewTerm{philosophy}\index{philosophy}" and the Greeks thought of the atoms in this method. This can lead with a lot of luck to a true theory. Against it is much more difficult to make a "\NewTerm{predictive theory}"\index{predictive theory}, that is to say with equations that predict the outcome of an experiment.\\

\textbf{R2.} What separates mathematics and physics is that in mathematics, the hypothesis is always true. Mathematical discourse is not a proof of an external seeking truth, but a target of consistency. What should be correct is just the reasoning. 
	\end{tcolorbox}

When these rules are not respected, we speak of "\NewTerm{scientific fraud}"\index{scientific fraud} (which often leads to being fired from his job but unfortunately we still not retired the diplomas when it happens). In general, scientific fraud itself comes in three main forms: plagiarism, fabrication of data and alteration of results unfavourable to the hypothesis, the omission of clear working hypotheses and recolted datas. To these frauds we can also add behaviors that pose problems regarding to the quality of work or more specifically to ethics, such as those aimed at increasing appearance in the production (and through the famous of the scientist) by submitting for example several times the same publication with only a few modifications, the omission of conflict of interest, the dangerous experiments, the non-conservation of primary data, etc.
	\begin{figure}[H]
		\centering
		\includegraphics[scale=0.73]{img/intro/peer_review.jpg}
		\caption[]{Source: \url{http://cartoonsbyjosh.co.uk}}
	\end{figure}	

	\subsection{Descartes' Method}
	Now we present the four principles of the Descartes' method which, as remind, is considered as the first scientific in history by his method of analysis:
	\begin{itemize}
	\item[P1.] Never accept anything as true that I obviously knew her to be such. That is to say, carefully avoid precipitation and to understand nothing more in my judgments than what would appear so clearly and distinctly to my mind, that I had no occasion to doubt.
	
	\item[P2.] Divide each of the difficulties I have to examine into as many parts as possible (scrupulous observations and plausible hypothesis until evidence of the opposite), and that would be required to resolve them in the best way.
	
	\item[P3.] Driving my thoughts in order, beginning with the simplest objects and easiest to know, to go up gradually by degrees to the knowledge of the most compounds, and even assuming the order between those who not naturally precede each other.
	
	\item[P4.] Make everywhere so complete enumerations and so general reviews, that I'm sure not to omit anything.
	\end{itemize}	

	\subsubsection{Blind studies}
	Scientific experiments\footnote{This text is a copy/past of an article written by Manuel Gnida at \url{http://www.symmetrymagazine.org/article/the-facts-and-nothing-but-the-facts}} are designed to determine facts about our world using either "\NewTerm{retrospective studies}\index{retrospective studies}" based on the search of correlations by exploiting existing databases or "\NewTerm{prospective studies}\index{prospective studies}" based on the search of causalities using controlled/randomized/double-blinde experiments. But in complicated analyses, there's a risk that researchers will unintentionally skew results to match what they were expecting to find. To reduce or eliminate this potential bias, scientists apply a method known as "\NewTerm{blind analysis}\index{blind analysis}".
	
	Blind studies are probably best known from their use in clinical drug trials (the term "triple-blinding" sometimes refers to this), in which patients are kept in the dark about - or blind to - whether they're receiving an actual drug or a placebo. This approach helps researchers judge whether their results stem from the treatment itself or from the patients' belief that they are receiving it. But the method is also use in Gastronomy tasting or in forensic laboratories as well.
	
	Particle physicists and astrophysicists do blind studies, too. The approach is particularly valuable when scientists search for extremely small effects hidden among background noise that point to the existence of something new, not accounted for in the current model. Examples include the much-publicized discoveries of the Higgs boson by experiments at CERN's Large Hadron Collider and of gravitational waves by the Advanced LIGO detector.
	\begin{figure}[H]
		\centering
		\includegraphics[scale=0.8]{img/intro/scientific_evidence.jpg}
		\caption{Scientific evidence hierarchy}
	\end{figure}
	"\textit{Scientific analyses are iterative processes, in which we make a series of small adjustments to theoretical models until the models accurately describe the experimental data}" says Elisabeth Krause, a postdoc at the Kavli Institute for Particle Astrophysics and Cosmology, which is jointly operated by Stanford University and the Department of Energy's SLAC National Accelerator Laboratory. "\textit{At each step of an analysis, there is the danger that prior knowledge guides the way we make adjustments. Blind analyses help us make independent and better decisions}".
	
	Return on experience (REX) shows as expected that blind analyses need to be designed individually for each experiment. The way the blinding is done needs to leave researchers with enough information to allow a meaningful analysis, and it depends on the type of data coming out of a specific experiment.

	A common approach is to base the analysis on only some of the data, excluding the part in which an anomaly is thought to be hiding. The excluded data is said to be in a "black box" or "hidden signal box".

	Take the search for the Higgs boson. Using data collected with the Large Hadron Collider until the end of 2011, researchers saw hints of a bump as a potential sign of a new particle with a mass of about $125$ gigaelectronvolts. So when they looked at new data, they deliberately quarantined the mass range around this bump and focused on the remaining data instead.

	They used that data to make sure they were working with a sufficiently accurate model. Then they "opened the box" and applied that same model to the untouched region. The bump turned out to be the long-sought Higgs particle.

	That worked well for the Higgs researchers. However, as scientists involved with the Large Underground Xenon (LUX) experiment reported at the workshop, the "black box" method of blind analysis can cause problems if the data you're expressly not looking at contains rare events crucial to figuring out your model in the first place.
	
	LUX has recently completed one of the world’s most sensitive searches for WIMPs - hypothetical particles of dark matter, an invisible form of matter that is five times more prevalent than regular matter. LUX scientists have done a lot of work to guard LUX against background particles—building the detector in a cleanroom, filling it with thoroughly purified liquid, surrounding it with shielding and installing it under a mile of rock. But a few stray particles make it through nonetheless, and the scientists need to look at all of their data to find and eliminate them.

	For that reason, LUX researchers chose a different blinding approach for their analyses. Instead of using a "black box", they use a process called "salting".

	LUX scientists not involved in the most recent LUX analysis added fake events to the data—simulated signals that just look like real ones. Just like the patients in a blind drug trial, the LUX scientists didn't know whether they were analyzing real or placebo data. Once they completed their analysis, the scientists that did the "salting" revealed which events were false.

	A similar technique was used by LIGO scientists, who eventually made the first detection of extremely tiny ripples in space-time called gravitational waves.

	Not everyone in the scientific community is convinced that blinding is necessary. Blind analyses are more complicated to design than non-blind analyses and take more time to complete. Some scientists participating in blind analyses inevitably spend time looking at fake data, which can feel like a waste.
	
	\pagebreak
	\subsection{Archimedean Oath}
	Inspired by the Hippocratic Oath, a group of students of the Ecole Polytechnique Fédérale de Lausanne in 1990 developed an oath of Archimedes expressing the responsibilities and duties of the engineer and technician. It was taken in various versions by other European engineering schools and could serve as basic inspiration oath for scientific researchers (even if there are some important points missing).

	"Considering the life of Archimedes of Syracuse which illustrated as of Antiquity the ambivalent potential of the technique, considering the responsibility increasing for the engineers and scientists with regard to the men and nature, considering the importance of the ethical problems that the technique and its applications raise, today, I pledge following and will endeavour to tend towards the ideal which they represent:
	\begin{enumerate}
		\item I will practice my profession for the good of the people, in the respect of the Human Rights and of the Environment.

		\item I will recognize, being as well as possible informed to me, the responsibility for my acts and will not discharge me to in no case on others.

		\item I will endeavor to perfect my professional competences.

		\item In the choice and the realization of my projects, I will remain attentive with their context and their consequences, in particular from the point of view technical, economic, social, ecological... I will pay a detailed attention to the projects being able to have fine soldiers.

		\item I will contribute, in the measurement of my means, to promote equitable relationships between humans and to support the development of the countries lower-income group.

		\item I will transmit, with rigor and honesty, with interlocutors chosen with understanding, any information important, if it represents an asset for the company or if its retention constitutes a danger to others. In the latter case, I will take care that information leads to concrete provisions.

		\item I will not let myself dominate by the defense of my interests or those of my profession.

		\item I will make an effort, in the measurement of my means, to lead my company to take into account the concerns of this Oath.

		\item I will practice my profession in all intellectual honesty, with conscience and dignity.

		\item I promise it solemnly, freely and on my honor."
\end{enumerate}
Sadly this oath should be completed with the "\NewTerm{Münich Declaration of the Duties and Rights of Journalists (1971)}\index{Münich declaration of the duties and rights of journalists}". That is, the essential duties of the scientist in gathering, reporting on and commenting on data consist in:
\begin{itemize}
	\item Respecting the truth no matter what consequences it may bring abut to him, and this is because the right of the public is to know the truth.

	\item Defending the freedom of information, of commentaries and of criticism.

	\item Publishing only such pieces of information the origin of which is known or – in the opposite case – accompanying them with due reservations; not suppressing essential information and not altering texts and documents.

	\item Not making use of disloyal methods to get information, photographs and documents.

	\item Feeling obliged to respect the private life of people.

	\item Correcting any published information which has proved to be inaccurate.

	\item Observing the professional secrecy and not divulging the source of information obtained confidentially.

	\item Abstaining from plagiarism, slander, defamation and unfounded accusations as well as from receiving any advantage owing to the publication or suppression of information.

	\item Never confusing the profession of journalist with that of advertiser or propagandist and not accepting any consideration, direct or not, from advertisers.

	\item Refusing any pressure and accepting editorial directives only from the leading persons in charge in the editorial office. Every journalist worthy of this name feels honoured to observe the above-mentioned principles; while recognising the law in force in each country, he does accept only the jurisdiction of his colleagues in professional matters, free from governmental or other interventions.
\end{itemize}

	\pagebreak
	\subsection{Scientific Publication Rules (SPR)}
	It is impossible to have a constructive debate or analysis if the basis material is unusable. Sadly still in the 21st century it is easy to found Nobel Price publication that were peer-reviewed and that are scientifically unusable. This is why we recall here the basic scientific publication rules for a publication be accepted by a real scientific peer-review committee:
	\begin{enumerate}
		\item Use of LaTeX for the writing of the publication
		\item All redaction files and raw data files must have ISO compliant names
		\item The publication should have a GUID
		\item Put the publishing date
		\item Put the major and minor version of the publication (eg: v3.6 r58)
		\item Put the experiment (development) period date (ISO date format)
		\item Write an abstract
		\item Write an introduction
		\item All measurement units must follow ISO standards
		\item Use the "principle of precaution" (use of conditional)
		\item Use "reactive responses", that is to say the make the confrontations between hypotheses / data, hypotheses / facts, hypotheses / observations 
		\item Use, when available, "leverage factors" to give substance and credit to the work by making reference to other corresponding publication on the same subject\footnote{This also the very important step of "personal review", that is to say a personal analysis of several tens / hundreds of scientific publications and that you have made one critical analysis that you use to build your own argument.}
		\item Material and Methods should be described in details. For theoretical papers, they should provide a link (URL) or reference where the full detailed proof can be found (if detailed proof is omitted in the original publication!)
		\item Put high resolution print-screens of charts or photos
		\item Write the results and for experimental data always provide a statistical analysis to show if the effect seems significant or not (sample size effect also or fluctuation interval)
		\item Calculate the propagation of errors of measurement instruments
		\item Write the precautional conclusion
		\item Give access to the raw data in a non-proprietary format to the scientific community
		\item Give access to the scripts/code used for data analysis to the scientific community
		\item Give access to the LaTeX sources of the publication to the scientific community
		\item Provide exact version (with minor release) of the softwares used to publish the paper
		\item Put the bibliography with the references
		\item Cite equivalent studies for meta-analysis\footnote{If there are no equivalent studies, then no meta-analysis are possible, then the results and conclusions don't reach any scientific consensus for recall!}
		\item Put the \% financial support of each sponsor (competing interests, funding sources)
		\item Submit the paper to the peer-review committee (in single or double blind way\footnote{"single blind" is that the peer-reviews doesn't know the name of the authors, "double blind" is that neither the authors nor the reviewers know each others' identities.})
		\item List all actors (with position, grade, e-mail) and peer-reviewers (only name for that latter) of the paper
	\end{enumerate}
	Any publication that doesn't respect at least one of this rule cannot be considered as a "scientific" publication!
	\begin{tcolorbox}[title=Remark,colframe=black,arc=10pt]
	Even if is there is a consensus between scientists, a unique oriented study (which can be very important) can be used to influence the opinion of mainstream media, governments and people. This is why a study must always be repeated, peer-reviewed and meta-analyzed by independent teams and laboratories.
	\end{tcolorbox}
	Caution! Many people think that a "\NewTerm{scientific consensus}" refers to a large group of scientists who all agree that something is true. In reality, a scientific consensus is a large body of scientific studies that all agree with and support each other ("conensus of data"). The agreement among the scientists themselves is simply a by-product of the consistent evidence.
	
	An well known example of non-existing consensus are religions. Indeed, if someone argue that as the statistics don't lie the Christian God must exist as it is the most followed religion in the world with $2$ billion Christians and that $2$ billion people can't be wrong, you can recall this same person that as there is $7$ billion people in the World, the $5$ other billion that not believe in the Christian God cannot be wrong as... statistic don't lie... Same if you merge Muslims and Christians together, then only $55\%$ of the people in the World believe in a unique God and $55\%$ is statistically not enough to reach the scientific consensus that is at a threshold level of $95\%$...
	\begin{center}
		\includegraphics[scale=0.4]{img/intro/scientific_papers.jpg}
	\end{center}
	It is then easy to understand why Internet Web Pages and YouTube video (or any other similar platform) are not a reliable scientific sources according to the above protocol:
	\begin{enumerate}
	   \item The peer-reviewers names are the huge majority of time not indicated
	   \item Contributor/Editors are anonymous are can therefore not be identified (typically an issue of Wikipedia)
	   \item The mathematical details are not provided (or even worst, there is not equations given at all!) so it is hard or even impossible to check by yourself if the reasoning is accurate
	   \item The experiment exact protocol is not given so it is impossible to know if the results are fake or real.
	   \item No sources or cross-references are given.
	   \item The content is in a not reliable format (a video or a web page are not perennial and protected\footnote{In the 21st century a PDF for example should be protect against edition and electronically signed} sources)
	   \item The new presented theoretical models predict indeed what the previous one do, but doesn't predict anything new and is therefore not falsifiable
	   \item The speaker on the video makes assumption that are not falsifiable (reference to God or to theories who mathematical details are not provided)
	   \item etc.
	\end{enumerate}
	\begin{center}
		\includegraphics[scale=0.5]{img/intro/fake_science.jpg}
	\end{center}

	\pagebreak
	\subsection{Scientific Mainstream Media communication}
	The reader of mainstream media or also social networks must never trust a scientific study if the reference and peer-reviewed paper is not given as link (and that latter must respect the scientific publication rules that we have introduced earlier below!). The study must also not be taken as absolute by reader if there is a consensus of the scientific community but only on... ONE... study. The only way to be almost sure is to read the study itself if it respects the above protocol.
	
	A typical bad example is a news that was taken by many international mainstream media on the Lyme-Borreliose disease as following:
	\begin{figure}[H]
		\centering
		\includegraphics[scale=0.28]{img/intro/lyme_borreliose.jpg}
		\caption[Swiss TV publication about Lyme-Borreliose treatment]{Swiss TV publication about Lyme-Borreliose treatment the 2017-01-08 (source: RTS App)}
	\end{figure}
	In summary what the "scientific journalist" (humm humm... I think it must be a new intern in fact...), of one of the main National Swiss Television (so a TV that has enough money to investigate correctly any news... at least in theory... in a country that assess to be number one in almost everything...), has published is a very bad (catastrophic) interpretation of the real article. The above article report that: "\textit{...a treatment applied during $3$ days not later than $72$ hour after after the bite of the tick has revealed and efficiency of $100\%$...}.
	
	In reality (if medias did have read the publication until the end...) the study was stopped after $8$ weeks and it has been shown that the treatment has no better effect than a placebo...

	%to make section start on odd page
	\newpage
	\thispagestyle{empty}
	\mbox{}
	\section{Vocabulary}
	Physics and mathematics, like any field of specialization, has its own vocabulary. So that the reader is not lost in the understanding of certain texts he can read in this PDF, we have chosen to present here a few fundamentals words, abbreviations and definitions to know.
	
	Thus, the mathematician like to finish his proofs (when he thinks they are correct) by the abbreviation "Q.E.D." which means "Quod Erat Demonstrandum" (this is Latin).
	
	And during definitions (they are many in math and physics ...) scientist often use the following terminology:
	
	\begin{itemize}
	\item ... it is sufficient that ...
	
	\item ... if and only if ...
	
	\item ... necessary and sufficient ...
	
	\item ... means ...
	
	\item ... prove it ...
	\end{itemize}
	These four are not equivalent (identical in the strict sense). Because "it is sufficient that" correspond to a sufficient condition, but not to a necessary condition. Also it must be notice that these four are place in the context of data analysis, data accuracy, reproduction and peer-review and not on any personal or common belief or also emotional aspect of a group of people (even if this group of people is more than a few billion individuals...)!
	\begin{center}
		\includegraphics[scale=0.55]{img/intro/an_old_age_argument.jpg}
	\end{center}

	\subsection{On Sciences}	
	It is important that we define rigorously the different types of sciences to which humans often refers. Indeed, it seems that in the 21st century a misnomer is established and that it became impossible for people to distinguish the "intrinsic quality" between a "science" and another one.

	\begin{tcolorbox}[title=Remark,colframe=black,arc=10pt]
Etymologically, the word "science" comes from the Latin "Scienta" (knowledge) whose root is the verb "scire" which means "to know".
	\end{tcolorbox}

This abuse of language is probably the fact that pure and accurate sciences lose their illusions of universality and objectivity, in the sense that they are self-correcting. This has for effect that some sciences are relegated to the background and try to borrow these methods, principles and origins to create confusion. We must therefore be very careful about the claims of scientificity in the human sciences, and this is also (or especially) true for the dominant trends in economics, sociology and psychology. Quite simply, the issues addressed by the human sciences are extremely complex, poorly reproducible, and empirical arguments supporting their theories are often quite low.

	\marginnote{\textcolor{NavyBlue}{{\footnotesize \textbf{~\thechapter:\myparagraph}}}}By itself, however, science does not produce absolute truth. By principle, a scientific theory is valid as long as it can predict measurable and reproducible results. But the problems of interpretation of these results are part of natural philosophy.
	
	\begin{center}
		\NewTerm{\textbf{No scientific theory is proven or provable. It is simply not falsified as long as an experiment has not come to say otherwise.}}
	\end{center}
	However, the scientific methodology is reliable enough so that Justice is not legitimate to take position on any scientific truths.

	Given the diversity of phenomena to be studied, over the centuries there has been a growing number of disciplines such as chemistry, biology, thermodynamics, etc. All these disciplines that are a priori heterogeneous have common foundation physics, for language mathematics and for elementary principle the scientific method.

	Thus, a small memory refresh seems useful:

\textbf{Definitions (\#\mydef):}

\begin{itemize}
	\item[D1.] We define as "\NewTerm{pure science}"\index{pure science} any set of knowledge based on rigorous reasoning valid whatever the (arbitrary) elementary factor selected (when we say then "independent of sensible reality") and restricted to the minimum necessary. Only mathematics (often named the "queen of sciences") can be classified in this category. 

	\item[D2.] We define as "\NewTerm{exact science}"\index{exact science} or "\NewTerm{hard science}"\index{hard science}, any set of knowledge based on the study of an observation, observation that has been transcribed in symbolic form  and that can be reproduced and refuted (theoretical physics for example... sometimes...). Primarily, the purpose of exact sciences is not to explain the "why" but the "how". 
	
	And never forget... Science (especially physics) doesn't have to "make sense" it just has to make all the right, testable predictions (instrumentalism)! According to the philosopher Karl Popper, a theory is scientifically acceptable if, as presented, it can be "\NewTerm{falsifiable}\index{falsiable}" (synonyms are "\NewTerm{refutable}\index{refutable}" or "\NewTerm{testable}\index{testable}"), i.e. subjected to experimental tests (or  if it is possible to conceive of an observation or an argument which negates the statement in question). The "scientific knowledge" is then by definition the set of theories that have resisted to falsification. Science is by nature subject to continuous questioning. 

	Caution! There is no doubt that the exact sciences have yet an enormous prestige, even among their opponents because of their theoretical and practical success. It is certain that some scientists sometimes abuse of this prestige by showing a sense of superiority that is not necessarily justified. Moreover, it often happens that this same scientists exposed in the popular literature, very speculative ideas as if they were very approved, and extrapolate their results outside the context in which they were tested (and ... under the hypotheses they were checked once...). 

	\begin{tcolorbox}[title=Remark,colframe=black,arc=10pt]
The two previous definitions are often included in the definition of "\NewTerm{deductive sciences}"\index{deductive science} or even "\NewTerm{phenomenological science}"\index{phenomenological science}.
	\end{tcolorbox}
	
	\item[D3.] We define as "\NewTerm{engineering science}"\index{engineering science} any set of knowledge or practices applied to the needs of human society such as electronics, chemistry, computer science, telecommunications, robotics, aerospace, biotechnology... 

	\item[D4.] We define as "\NewTerm{science}"\index{science} any body of knowledge based on studies or observations of events whose interpretation has not yet been transcribed and verified with mathematical rigour, characteristic of previous sciences, but using comparative statistics. We include in this definition: medicine (we should however be careful because some parts of medicine are studying phenomena using mathematical descriptions such as neural networks or other phenomena associated with known physical causes), sociology, psychology, history, biology, etc.
	
	Some teachers like to play with the word "science" as the acronym of (that's not stupid for college students): \textbf{S}olve, \textbf{C}reate, \textbf{I}nvestigate, \textbf{E}valuate, \textbf{N}otice, \textbf{C}lassify, \textbf{E}xperiment.

	\item[D5.] We define as "\NewTerm{soft science}"\index{soft science}, "\NewTerm{para-science}"\index{para-science} or "\NewTerm{pseudo-science}"\index{pseudo-science} any set of knowledge or practices that are currently based on non-verifiable and non-refutable facts (not scientifically reproducible) by experience or by mathematics. We include in this definition typically: astrology, theology, paranormal (which was demolished by zetetic science), graphology, justice\footnote{Indeed, for example in Switzerland, it is common that the cantonal Judge and the Federal Judge don't give the same judgment as that latter is non-scientific but rather subjectively based on the Judge experience of life}, etc. 
	
	As some scientists say: «\textit{It looks like science, it use the vocabulary of science... but that's not science at all.}»
	
	Especially pseudo-sciences are characterized by:
	\begin{itemize}
		\item They start with a conclusion (believe), then works backwards to confirm.

		\item They are hostile to criticism

		\item They use circular reasoning/arguments

		\item They use vague jargon to confuse and evade

		\item They use subtle strategies to change influence people minds (especially children)

		\item They do cherry picking only on favorable evidence

		\item They use non-reproducible/non-refutable methods with unrepeatable results

		\item They use bullshit-random language to impress the audience

		\item They use inconsistent and invalid logic

		\item People working in the field are dogmatic and unyielding
	\end{itemize}

	\item[D6.] We define as "\NewTerm{phenomenological science}" or "\NewTerm{natural sciences}"\index{natural science}, any science which is not included in the above definitions (history, sociology, psychology, zoology, biology, ...) 

	\item[D7.] "\NewTerm{Scientism}"\index{scientism} is an ideology that considers experimental science is the only valid mode of knowledge, or, at least, superior to all other forms of interpretation in the world. In this perspective, there is no philosophical, religious or moral truths superior of scientific theories. Only account what is scientifically proven. 

	\item[D8.] "\NewTerm{Positivism}"\index{positivism} is a set of ideas that considers that only the analysis and understanding of facts verified by experience can explain the phenomena of the sensible world. Certainty is provided solely by the scientific experiment. He rejects introspection, intuition and metaphysical approach to explain any knowledge of the phenomena. \\\\
	What is interesting about this doctrine is that it is certainly one of the few that requires people to have to think for themselves and to understand the environment around them by continually questioning everything and by never accepting anything as granted (...). In addition, the real sciences have this extraordinary property that they give the possibility to understand things beyond what we can see. 
\end{itemize}

But, science is science, and nothing more: a certain ordering, not too bad success, things that no longer leads to the metaphysics as the time of Aristotle, but that does not pretend to give us the whole story on reality or even the bottom of visible things.

	\pagebreak
	\subsection{Terminology}

The table of methods we presented above contains terms that may perhaps seem unknown or barbarians for you. This is why it seems important to provide definitions of these and some other equally important that can avoid important confusion. 

\textbf{Definitions (\#\mydef):}

\begin{itemize}
	\item[D1.] Beyond its negative sense, the idea of "\NewTerm{problem}"\index{problem} refers to the first step of the scientific method. Formulate a problem is also essential for its resolution and allows to properly understand what is the problem and see what needs to be resolved. \\\\
	The concept of "problem" is intimately connected to the concept of "assumption" which will see the definition below. 

	\item[D2.] A "\NewTerm{hypothesis}\index{hypothesis}" is always, in the context of a theory already established or underlying, a supposition awaiting confirmation or refutation that attempts to explain a group of facts or predict the onset of new facts.\\\\
	Thus, a hypothesis can be at the origin of a theoretical problem that has to be resolved formally. 

	\item[D3.] The "\NewTerm{postulate}\index{postulate}" or  "\NewTerm{assumption}\index{assumption}" in physics corresponds frequently to a principle (see definition below) which admission is required to establish a proof (we mean that this is a non-provable proposition).\\\\
	The mathematical equivalent (but in a more rigorous version) of the assumption is the "axiom" for which we will see the definition below. 

	\item[D4.] A "\NewTerm{principle}"\index{principle} (close parent of "postulate") is a proposal accepted as a basis for reasoning or a general theoretical guide line for reasoning that needs to be performed. In physics, it is also a general law governing a set of phenomena and verified by the accuracy of its consequences. \\\\
The word "principle" is used with abuse in small classes or engineering schools by teachers not knowing (which is very rare), or unwilling (rather common), or that can't because lack of time (almost exclusively ) prove a relation.\\\\
The equivalent of the postulate or principle in mathematics is the "axiom" which we define as follows: 

	\item[D5.] An "\NewTerm{axiom}"\index{axiom} is a self-evident proposition or truth by itself which admission is necessary to establish a proof. 
\end{itemize}

	\begin{tcolorbox}[title=Remarks,colframe=black,arc=10pt]
	\textbf{R1.} We could say that this is something we define as the truth for the speech that we argue, like a rule of the game, and that it does not necessarily a universal truth value in the sensitive world around us.\\

	\textbf{R2.} Axioms must always be independent (one should not be able to be proved from the other) and non-contradictory (sometimes we also say that they must be "consistent"). 
	\end{tcolorbox}	
	
\begin{itemize}
	\item[D6.] The "\NewTerm{corollary}"\index{corollary} is a term unfortunately almost nonexistent in physics (wrongly!) and that is in fact a proposal resulting from a truth already demonstrated. We can also say that a corollary is and obvious and necessary consequence of a theorem (or sometimes of a postulate in physics). 

	\item[D7.] A "\NewTerm{lemma}"\index{lemma} is a proposal deduce from one or more assumptions or axioms and that for which the proof prepares this of a theorem.
\end{itemize}

	\begin{tcolorbox}[title=Remark,colframe=black,arc=10pt]
The concept of "lemma" is also (and this is unfortunate) almost used only in the field of mathematics. 
	\end{tcolorbox}	

\begin{itemize}
	\item[D8.] A "\NewTerm{conjecture}"\index{conjecture} is a supposition or opinion based on the likelihood of a mathematical result.\\\\
	Many conjectures have as as little similar to lemmas, as they are checkpoints to obtain significant results.
	
	\item[D9.] Beyond its weak conjecture sense, a "\NewTerm{theory}"\index{theory} or "\NewTerm{theorem}"\index{theorem} is a set articulated around a hypothesis and supported by a set of facts or developments that give it a positive content and make the hypothesis well-founded (or at least plausible in the case of theoretical physics). 

	\item[D10.]  A "\NewTerm{singularity}"\index{singularity} is an indeterminacy in a calculation That takes the appearance of a division by zero. This term is both used in mathematics and in physics. 

	\item[D11.] A "\NewTerm{proof}"\index{proof} is a set of mathematical procedures to follow to prove the result already known or not of a theorem. 

	\item[D12.] If the word "\NewTerm{paradox}"\index{paradox} etymologically means: contrary to common opinion, it is not by pure taste for provocation, but rather for solid reasons. A "\NewTerm{sophism}"\index{sophism} meanwhile, is a deliberately provocative statement, a false proposition based on an apparently valid reasoning. Thus we speak about the "Zeno's paradox" when in reality it is only a sophism. The paradox is not limited to falsity, but implies the coexistence of truth and falsity, so that one can no longer distinguish true and the false. The paradox appears as an unsolvable problem an "\NewTerm{aporia}"\index{aporia}. 
	
\end{itemize}

	\begin{tcolorbox}[title=Remark,colframe=black,arc=10pt]
It should be added that the well-knows paradoxes, by the questions they raised, have permitted significant advances to science and led to major conceptual revolutions in mathematics as in theoretical physics (the paradoxes on sets and on infinity in mathematical, and those at the base of relativity and quantum physics).
	\end{tcolorbox}	

	%to make section start on odd page
	\newpage
	\thispagestyle{empty}
	\mbox{}
	\section{Science and Faith}
	We will see that in Science, a theory is usually incomplete because it can not fully describe the complexity of the real world or because it does not predict what we don't know (excepted for Quantum Physics or General Relativity). It is thus for theories like the Big Bang (\SeeChapter{see section Astrophysics page \pageref{astrophysics}}) or the Evolution of species (\SeeChapter{see sections Populations Dynamics page \pageref{population dynamics} or Decision and Games Theory page \pageref{game and decision theory}}) because they are not reproducible in laboratories under identical conditions.  But some other theories are so accurate to predict physical phenomena that some people \underline{believe} that mathematics is the nearest language with God (at least for those that believe in a divinity...) even if we know, as we have already mention id, that science is (should) be driven only by data and peer-review.
	\begin{center}
		\includegraphics[scale=0.9]{img/intro/science_we_trust.jpg}
	\end{center}	

	We should distinguish between different scientific currents: 
	\begin{itemize}
		\item "\NewTerm{Realism}"\index{realism} is a doctrine where physical theories have the aim to describe reality as it is in itself, in its unobservable components. 
	
		\item "\NewTerm{Instrumentalism}"\index{instrumentalism} is a doctrine where theories are only tools to predict observations but do not describe reality itself. 
	
		\item "\NewTerm{Fictionalism}"\index{fictionalism} is the doctrine where the content repository (principles and postulates) of theories is just an illusion, useful only to ensure the linguistic articulation of the fundamental equations. 
	\end{itemize}

	\pagebreak
	Even if today the scientific theories are sponsored by many specialists, alternative theories have valid arguments and we can not totally dismiss them. However, the creation of the world in seven days as described in the Bible is difficult to accept, and many believers recognize that a literal reading of the Bible is not compatible with the current state of our knowledge and that it is more prudent to interpret it as a parable. If science never provides definitive answer, it is no longer possible to ignore it. 

	Faith (whether religious, superstitious, pseudo-scientific or other not data driven) on the contrary is intended to provide absolute truths of a different nature as it is a personal unverifiable belief (for example, science requires proof to be believed/support, religions requires believes to be proved). This is why many people say that \textit{Science adjusts views based on what's observed when Faith is the denial of observation so that belief can be preserved}.... In fact, one of the functions of religion is to give meaning to the phenomena that can not be explained rationally with actual knowledge\footnote{This was the case with the rain, the thunder, diseases, stars, comets, earthquakes, volcanic eruptions, etc. a few hundred years ago and is often designated by scientists under the name of "argument of ignorance"\index{argument of ignorance}}. Progress of knowledge trough science therefore cause sometimes (...) questioning the religious dogma. 

	Conversely, except try to impose his own faith (which is nothing but a subjective and intimate personal conviction) to others, we must defy the natural temptation to characterize scientifically proven fact extrapolations of scientific models beyond their scope.

	The word "science" is, as we have already mentioned above, increasingly used to argue that there is a scientific evidence where there is only a belief (some web pages like this proliferate always more and more and especially to get followers and a lot of clicks on the Internet). According to its detractors it is, for example, the case of the movement of Scientology (but there are many others). According to them, we should rather speak about "\NewTerm{occult sciences}"\index{occult science}.

	The occult sciences and traditional sciences exist since antiquity; they consist on a series of mysterious knowledge and practices designed to penetrate and dominate the secrets of nature. Over the past centuries, they have been progressively excluded from science. The philosopher Karl Popper has longly questioned himself about the nature of the demarcation between science and pseudoscience. After noticing that it is possible to find observations to confirm almost any theory, he proposes a methodology based on falsifiability. A theory must according to him, to deserve the adjective "scientific", guarantee the impossibility of some events. It becomes therefore refutable, so (and only then) capable of integrating science. It would suffice to observe any of these events to invalidate the theory, and therefore take the way to improving it.
	
	And also let us notice that major difference between science books and religion books is that if you destroyed that latter, in a thousand year's time that wouldn't come back just as it was. Whereas if we took every science book and every fact and destroyed them all, in a thousand years they'd all be back. Because all the same tests would be the same results.
	
	\pagebreak
	\subsubsection{Baloney detection kit}
	Through their training, scientists are equipped with what Carl Sagan name the "\NewTerm{baloney detection kit}\index{baloney detection kit}" or "\NewTerm{bullshit detection kit}\index{bullshit detection kit}" that is a set of cognitive tools and techniques that fortify the mind against penetration by falsehoods and to draw boundaries between science and pseudoscience. It isn't merely a tool of science, it contains invaluable tools of healthy skepticism that apply just as elegantly, and just as necessarily, to everyday life. By adopting the kit, we can all shield ourselves against clueless guile and deliberate manipulation. 

	There are many version of these detection tool but here is an quite complete one (but still incomplete by construction) a proposed by Michael Shermer (founding publisher of \href{http://www.skeptic.com}{<Skeptic Magazine} and author of \textit{The Borderlands of Science}):
	
	\begin{enumerate}
		\item \textit{\textbf{How reliable is the source of the claim?}}

		Pseudoscientists often appear quite reliable, but when examined closely, the facts and figures they cite are distorted, taken out of context or occasionally even fabricated. Of course, everyone makes some mistakes. And as historian of science Daniel Kevles showed so effectively in his book The Baltimore Affair, it can be hard to detect a fraudulent signal within the background noise of sloppiness that is a normal part of the scientific process. The question is, Do the data and interpretations show signs of intentional distortion? When an independent committee established to investigate potential fraud scrutinized a set of research notes in Nobel laureate David Baltimore's laboratory, it revealed a surprising number of mistakes. Baltimore was exonerated because his lab's mistakes were random and nondirectional... So in science, there are no authorities. At most, there are experts!

		\item \textit{\textbf{Does this source often make similar claims?}}

		Pseudoscientists have a habit of going well beyond the facts. Flood geologists (creationists who believe that Noah's flood can account for many of the earth's geologic formations) consistently make outrageous claims that bear no relation to geological science. Of course, some great thinkers do frequently go beyond the data in their creative speculations. Thomas Gold of Cornell University is notorious for his radical ideas, but he has been right often enough that other scientists listen to what he has to say. Gold proposes, for example, that oil is not a fossil fuel at all but the by-product of a deep, hot biosphere (microorganisms living at unexpected depths within the crust). Hardly any earth scientists with whom I have spoken think Gold is right, yet they do not consider him a crank. Watch out for a pattern of fringe thinking that consistently ignores or distorts data.

		\item \textit{\textbf{Have the claims been verified by another source?}}

		Typically pseudoscientists make statements that are unverified or verified only by a source within their own belief circle. We must ask, Who is checking the claims, and even who is checking the checkers? The biggest problem with the cold fusion debacle, for instance, was not that Stanley Pons and Martin Fleischman were wrong. It was that they announced their  spectacular discovery at a press conference before other laboratories verified it. Worse, when cold fusion was not replicated, they continued to cling to their claim. Outside verification is crucial to good science.

		\item \textit{\textbf{How does the claim fit with what we know about how the world works?}}

		An extraordinary claim must be placed into a larger context to see how it fits. When people claim that the Egyptian pyramids and the Sphinx were built more than 10,000 years ago by an unknown, advanced race, they are not presenting any context for that earlier civilization. Where are the rest of the artifacts of those people? Where are their works of art, their weapons, their clothing, their tools, their trash? Archaeology simply does not operate this way.

		\item \textit{\textbf{Has anyone gone out of the way to disprove the claim, or has only supportive evidence been sought?}}

		This is the "confirmation bias" (we will come back on cognitive bias in the section of Decision Theory), or the tendency to seek confirmatory evidence and to reject or ignore disconfirmatory evidence. The confirmation bias is powerful, pervasive and almost impossible for any of us to avoid. It is why the methods of science that emphasize checking and rechecking, verification and replication, and especially attempts to falsify a claim, are so critical. 

		\item \textit{\textbf{Does the preponderance of evidence point to the claimant's conclusion or to a  different one?}}

		The theory of evolution, for example, is "proved" through a convergence of evidence from a number of independent lines of inquiry. No one fossil, no one piece of biological or paleontological evidence has "evolution" written on it; instead tens of thousands of evidentiary bits add up to a story of the evolution of life. Creationists conveniently ignore this confluence, focusing instead on trivial anomalies or currently unexplained phenomena in the history of life.

		\item \textit{\textbf{Is the claimant employing the accepted rules of reason and tools of research, or have these been abandoned in favor of others that lead to the desired conclusion?}} 

		A clear distinction can be made between SETI (Search for Extraterrestrial Intelligence) scientists and UFOlogists. SETI scientists begin with the null hypothesis that ETIs do not exist and that they must provide concrete evidence before making the extraordinary claim that we are not alone in the universe. UFOlogists begin with the positive hypothesis that ETIs exist and have visited us, then employ questionable research techniques to support that belief, such as hypnotic regression (revelations of abduction experiences), anecdotal reasoning (countless stories of UFO sightings), conspiratorial thinking (governmental cover-ups of alien encounters), low-quality visual evidence (blurry photographs and grainy videos), and anomalistic thinking (atmospheric anomalies and visual misperceptions by eyewitnesses).

		\item \textit{\textbf{Is the claimant providing an explanation for the observed phenomena or merely 
             denying the existing explanation?}}
	
		This is a classic debate strategy-criticize your opponent and never affirm what you believe to avoid criticism. It is next to impossible to get creationists to offer an explanation for life (other than "God did it"). Intelligent Design (ID) creationists have done no better, picking away at weaknesses in scientific explanations for difficult problems and offering in their stead. "ID did it." This stratagem is unacceptable in science.

		\item \textit{\textbf{If the claimant proffers a new explanation, does it account for as many phenomena as the old explanation did?}}
	
		Many HIV/AIDS skeptics argue that lifestyle causes AIDS. Yet their alternative theory does not explain nearly as much of the data as the HIV theory does. To make their argument, they must ignore the diverse evidence in support of HIV as the causal vector in AIDS while ignoring the significant correlation between the rise in AIDS among hemophiliacs shortly after HIV was inadvertently introduced into the blood supply.

		\item \textit{\textbf{Do the claimant's personal beliefs and biases drive the conclusions, or vice versa?}}

		All scientists hold social, political and ideological beliefs that could potentially slant their interpretations of the data (this is a "confirmation bias" also named "cherry picking" that is also by non-scientists the main cause of rejecting science results and tools), but how do those biases and beliefs affect their research in practice? Usually during the peer-review system, such biases and beliefs are rooted out, or the paper or book is rejected.  
	\end{enumerate}
	
	By fine tuning we can go more far about reasoning fallacies. Here is a most exhaustive list:
	\begin{enumerate}
		\item Ad hominem: An ad hominem argument attacks the messenger, not the message itself.

		\item Argument from authority: Argument that relies on the identity of an authority rather than the components of the argument itself.

		\item Argument from adverse consequences: Saying that because the implications of a statement being true would create negative results, it must not be true.

		\item Appeal to ignorance: If something is not known to be false, it must be true.

		\item Special pleading: Stating a universal principle, then insisting that it doesn't apply to your assertions for some reason.

		\item Begging the question/ assuming the answer: This occurs when a statement has an unproven premise. It is also named "circular reasoning" or "circular logic".

		\item Observational selection: Looking at only positive evidence while ignoring the negative and vice versa.

		\item Statistics of small numbers: Using small numbers in order to report large percentage increases.

		\item Misunderstanding of the nature of statistics: 	
Ignorance about central statistical assumptions and the definition of metrics (the confusion of correlation and causation, the sample size and hate of maths bias are well known example).

		\item Post hoc, ergo propter hoc: Basing an effect on a cause only on the basis of chronology.

		\item Excluded middle, or false dichotomy: Portraying an issue or argument as having only two options and no spectrum in between.

		\item Short-term vs. long-term: Assuming a current trend has remained constant throughout its history and will continue to do so in the future, even though no evidence suggests such an extrapolation is justified.

		\item Slippery slope, related to excluded middle: Saying something is wrong because it is next to or loosely related to something wrong.

		\item Suppressed evidence and half-truths: Drawing an unwarranted conclusion from premises that are at least in part correct.

		\item Weasel words: The usage of vague, non-specific references.
	\end{enumerate}
	
	In addition to teaching us what to do when evaluating a claim to knowledge, any good baloney detection kit must also teach us what not to do. It helps us recognize the most common and perilous fallacies of logic and rhetoric. Many good examples can be found in religion and politics, because their practitioners are so often obliged to justify two contradictory propositions.


	Finally, we would like to quote Lavoisier: «The physicist may also, in the silence of his laboratory and his cabinet, perform patriotic functions; he can thanks to his works reduce the mass of evils which afflict happiness and, had he not, contributed by the new roads that he opened to himself, only to delay of a few years, of a few days, the average life of humans, he could also aspire to the glorious title of benefactor of humanity.»
	
	\pagebreak
	\section{Scientific communication backfire}
	Another point that is important to highlight about science communication: Scientists, stop thinking explaining science will fix things and avoid people bias especially if you find yourself in a state of disbelief or evidence probably drives you crazy as there are nowadays many conspiracy about flat Earth, vaccines, climate change, etc. as mainstream media don't know how to communicate scientific papers.

	The reasons are the following and applied outside the case where people come listen to you or to other scientists in the context of a conference or seminar:
	\begin{enumerate}
		\item Most people don't want to listen about scientific method especially when they never asked you "is it true?", "is it the best method?", "is this not a bias?". If you use "your science" just to point out they are wrong about what they are saying or arguing you will just take them out of their comfort zone and make them even more hate science.
		
		\item Most humans are full of bias and they don't like to admit is it true as they assume the human is the top specy of evolution and therefore cannot have such biases. So when you explain them they have biases, you just point out that they are not reliable. Speak about bias only if people ask you to do so.
		
		\item The huge majority of people believe than their personal experience is more robust than the hundred of years of peer-review, tests, checks of the "scientific method" that has seems so far, if not THE best, at least the best one we know nowadays.
	\end{enumerate}
	Now let us quote some paragraphs of an excellent \href{http://www.slate.com/articles/health_and_science/science/2017/04/explaining_science_won_t_fix_information_illiteracy.html}{{\color{blue} article}} of Tim Requarth as it is almost perfect:
	

	«The theory many scientists seem to swear by is technically known as the deficit model, which states that people's opinions differ from scientific consensus because they lack scientific knowledge. In 2010, Dan Kahan, a Yale psychologist, essentially proved this theory wrong. He \href{http://www.nature.com/nclimate/journal/v2/n10/full/nclimate1547.html}{{\color{blue} surveyed }} over 1,500 Americans, classifying each person's "cultural worldview" on a scale that roughly correlates with politically liberal or conservative. He then assessed each person's scientific literacy with questions such as "True or False: Electrons are smaller than atoms". Finally, he asked them about climate change. If the deficit model were correct, Kahan reasoned, then people with increased scientific literacy, regardless of worldview, should agree with scientists that climate change poses a serious risk to humanity.
  
	That's not what he found. Instead, Kahan found that increased scientific literacy actually had a small negative effect: The conservative-leaning respondents who knew the most about science thought climate change posed the least risk. Scientific literacy, it seemed, increased polarization. In a later study, Kahan added a twist: He asked respondents what climate scientists believed. Respondents who knew more about science generally, regardless of political leaning, were better able to identify the scientific consensus-in other words, the polarization disappeared. Yet, when the same people were asked for their own opinions about climate change, the polarization returned. It showed that even when people understand the scientific consensus, they may not accept it.

	The takeaway is clear: Increasing science literacy alone won't change minds. In fact, well-meaning attempts by scientists to inform the public might even backfire. Presenting facts that conflict with an individual's worldview, it turns out, can cause people to dig in further. Psychologists, aptly, dubbed this the "backfire effect".
	\begin{figure}[H]
		\centering
		\includegraphics[scale=0.45]{img/intro/explain_science.jpg}
		\caption[]{Source: Dr. Jones, https://www.ratbotcomics.com}
	\end{figure}
	If scientists simply want to explain science to a curious audience, disseminate their research more broadly, or write for fun, this doesn't matter much. But if scientists are motivated to change minds-and many enrolled in science communication workshops do seem to have this goal-they will be sorely disappointed.

	That's not to say scientists should return to the bench and keep their mouths shut. They should just realize that closing the "information gap" isn't the goal. And instead, they need to learn how to communicate science strategically.

	There are obvious reasons why science communication is a necessary and worthwhile endeavor, but a huge one is that there's a politically motivated push to destabilize scientific authority. At a Heartland Institute conference last month, Lamar Smith, the Republican chairman of the House science committee, told attendees he would now refer to "climate science" as "politically correct science", to loud cheers. This lumps scientists in with the nebulous "left" and, as Daniel Engber pointed out here in Slate about the upcoming March for Science, rebrands scientific authority as just another form of elitism.

	Is it any surprise, then, that lectures from scientists built on the premise that they simply know more (even if it's true) fail to convince this audience? Rather than fill the information deficit by building an arsenal of facts, scientists should instead consider how they deploy their knowledge. They may have more luck communicating if, in addition to presenting facts and figures, they appeal to emotions. This could mean not simply explaining the science of how something works but spending time on why it matters to the author and why it ought to matter to the reader. Research also shows that science communicators can be more effective after they've gained the audience's trust. With that in mind, it may be more worthwhile to figure out how to talk about science with people they already know, through, say, local and community interactions, than it is to try to publish explainers on national news sites. And they might consider writing op-eds for their local papers, focusing on why science matters to their particular communities.

	Scientists can also learn to avoid certain pitfalls. I spoke with Gretchen Goldman, research director of the Union of Concerned Scientists' Center for Science and Democracy, which offers communication and advocacy workshops. A counterintuitive lesson she's learned is that refuting stories that deny climate change by addressing each claim and explaining why it's wrong is not that productive. In fact, it could be counterproductive: "If you repeat the myth, that's the part people remember even if you immediately debunk it", she says. A better approach, she suggests, is to reframe the issue. Don't just keep explaining why climate change is real, explain how climate change will hurt public health or the local economy. Communication that appeals to values, not just intellect, research shows, can be far more effective.

	[...] But the obstacles faced by science communicators are not epistemological but cultural. The skills required are not those of a university lecturer but a rhetorician.

	So it's an admirable goal to communicate about science, but almost certainly destined to fail. This is because the way most scientists think about science communication - that just explaining the real science better will help - is quite wrong. In fact, it's so wrong that it many times the opposite effect of what they're trying to achieve.[...]»
	
	\begin{flushright}
	Section quality score: \score{4}{5} 151 votes, 75.23\%
	\end{flushright}
	

\chapter{Arithmetic}

	\textit{\textbf{Mathematics is the ultimate form of forced art.}} (unknown)
	\minitoc
	%to make section start on odd page
	\newpage
	\thispagestyle{empty}
	\mbox{}
	\section{Proof Theory}
	
	\lettrine[lines=4]{\color{BrickRed}W}e have chosed to begin the study of Applied Mathematics by the theory that seems to us the most fundamental and important in the field of pure and exact sciences: Proof Theory. The proof theory and of propositional calculus (logic) has three objectives through this book:

	\begin{enumerate}
		\item Teach to the reader how to reason and demonstrate (prove), and this independently of the specialization field.
		\item Show that the process of a demonstration (proof) is independent of the language used.
		\item Prepare the reader to the Logic Theory (\SeeChapter{see section Logic Systems}).
		\item Prepare the path to Gödel's incompleteness theorem (main goal of this section!).
		\item Prepare the reader to the Automata Theory (\SeeChapter{see section Automata Theory}).
	\end{enumerate}
Gödel's theorem is probably the most exciting point because if we define religion as a system of thought that contains unprovable statements, then it contains elements of faith, and Gödel tells us that mathematics is not only a religion, but that then it is the only religion that can prove it is one!

	\begin{tcolorbox}[title=Remarks,colframe=black,arc=10pt]
	\textbf{R1.} It is (very) strongly advised to read this section in parallel with those on Automata Theory and Logical Systems (including Boolean Algebra) available in Theoretical Computing chapter of this book.\\

	\textbf{R2.} We must approach Proof Theory as a sympathetic curiosity but which basically brings nothing much except working/reasoning methods. Moreover, its purpose is not to show that everything is demonstrable but that any proof can be done on a common language starting from a finite number of rules.
	\end{tcolorbox}

Often when a student arrives in a graduate class he learned how to calculate or use algorithms but almost only a little or even not at all to reason. For all the reasoning the visual media is a powerful tool (a picture is worth a thousand words) and people who do not see that in tracing a given curve or straight line the solution appears or who do not see in space are really penalized.

During high school we already manipulate unknown objects but especially to make calculations and when we reason about objects represented by letters, we can replace them visually by a real number, a vector, etc. At a given level we ask people to reason on more abstract structures and therefore to work on unknown objects which are elements of a set itself unknown, for example elements of any group (see section Set Theory). This visual support thus doesn't exist anymore.

We ask so often to students to reason, to demonstrate the properties, but almost no one has ever taught them to reason properly, writing proofs, control proofs. If we ask a graduate student what is a proof, it most likely he will have some difficulty to answer. He can say that it is a text in which there are keywords like "therefore", "because", "if", "if and only if", "take a x such that", "assume", "lemma", "theorem", "let us look for a contradiction", etc. But he will probably be unable to provide the grammar of these texts nor their basics, and besides, its teachers, if they have not taken a course in Proof Theory, would probably be unable too.

To understand this situation, remember that to speak a child does not need to know the grammar. He imitates his surroundings and it works very well: most of time a six year old child know to use complicated sentences without ever having done grammar. Most teachers also do not know the grammar of reasoning but, for them, the imitation process has work well and thus they reason correctly. The experience of the majority of university teachers shows that this process of imitation works well for very good students, and then it is enough, but it works much less, if not at all, for many others.

As the complexity level is low (especially during an "equational" type reasoning), grammar is almost useless but when the level increase or when we do not understand why something is wrong, it becomes necessary to do some grammar to progress. Teachers and students are familiar with the following situation: in a school assignment the corrector barred whole page of a large red line and write "false" in the margin. When the student asks what is wrong, the corrector can only say things like "this has no relation with the requested proof", "nothing is right", ..., which help obviously not the student to understand. This is partly because the text written by the student uses the appropriate words but in a more or less random way and can not give meaning to the assembly of these words. In addition, the teacher does not have the tools to explain what is wrong. We must therefore give them to him!

These tools exist but are fairly recent. The proof theory is a branch of mathematical logic whose origin is the crisis of the foundations: there was a doubt about what we had the "right" to do in a mathematical reasoning (see the "foundations crisis" further below). Paradoxes appeared, and it was then necessary to clarify the rules of proof and to verify that these rules are not contradictory. This theory appeared in the early 20th century, which is very new since most of the mathematics taught in the first half of the university is known since the 16th-17th century.

	\subsubsection{Foundations Crisis}
	For the Greeks philosophers geometry was considered the highest form of knowledge, a powerful key to the metaphysical mysteries of the universe. It was rather a mystical belief and the link between mysticism and religion was made explicit in cults like those of the Pythagoreans. No culture has been deified a man for discovering a geometrical theorem! Later, mathematics was regarded as the model of a priori knowledge in the Aristotelian tradition of rationalism.

	No culture has since challenged a man for having discovered a geometrical theorem! Later, mathematics was regarded as the model of a priori knowledge in the Aristotelian tradition of rationalism.

	The astonishment of the Greeks philosophers for mathematics has not left us, we find it in the traditional metaphor of mathematics as "Queen of Science". It was strengthened by the spectacular success of mathematical models in science, success that the Greeks (even ignoring the simple algebra) had not anticipated. Since the discovery by Isaac Newton's of integral calculus and the inverse square law of gravity in the late 1600s the phenomenal sciences and higher mathematics remained in close symbiosis - to the point that a predictive mathematical formalism was became the hallmark of a "hard science".

	After Newton, during the next two centuries, science aspired to that kind of rigour and purity that seemed inherent in mathematics. The metaphysical question seemed simple: mathematics seemed to have a perfect a priori knowledge, and among all sciences, those that were able to mathematize most perfectly were the most effective for predicting phenomena. The perfect knowledge therefore, was in a mathematical formalism that, once reached by science and embracing all aspects of reality, could found a posteriori empirical knowledge on an a priori rational logic. It was in this spirit that Marie Jean-Antoine Nicolas de Caritat, Marquis de Condorcet (French philosopher and mathematician), undertook to imagine describing the entire Universe as a set of partial differential equations being solved one after the other.

	The first break in this inspiring picture appeared in the second half of the 19th century, when Riemann and Lobachevsky separately proved that Euclid's parallel axiom could be replaced by other geometries that produced "consistent" (we will come back more on this word further below). Riemannian geometry was modelled on a sphere, these of Lobatschewsky, on rotation of a hyperboloid.

	The impact of this discovery was later obscured by great upheaval, but at the time it made a thunderclap in the intellectual world. The existence of mutually inconsistent axiomatic systems, each of which could be a model for the phenomenal Universe, relied entirely into question the relation between mathematics and theoretical physics.

	When we knew only Euclid, there was only one possible geometry. One could believe that the Euclid's axiom (see section of Euclidien Geometry) were a kind of knowledge a priori perfect on the geometry in the phenomenal world. But suddenly we had three geometries, embarrassing for metaphysical subtleties.

	Why would we choose between the axioms of plane geometry, spherical and hyperbolic geometry as real descriptions? Because all three are consistent, we can not choose any a priori as a foundation - the choice must be empirical, based on their predictive power in a given situation.

	Of course, the theoretical physicists have long been accustomed to choose a formalism to study a scientific problem. But it was already accepted widely, if not unconsciously, that the need to do so was based on human ignorance, and with logic or good enough mathematics, one could infer the right choice from principles first, and produce a priori descriptions of reality that had to be confirmed afterwards by empirical verification.

	However, Euclidean geometry, seen for hundreds of years as the model of axiomatic perfection of mathematics, had been dethroned. If we could not know a priori something as basic as the geometry in space, what hope was there for a pure rational theory that would encompass all of nature? Psychologically, Riemann and Lobachevsky had struck at the heart the mathematical enterprise as it had been designed before.

	Moreover, Riemann and Lobachevsky have pushed the nature of mathematical intuition into question. It was easy to believe implicitly that mathematical intuition was a form of perception - a way to glimpse the Platonic world behind reality. But with two other geometries pushing the Euclid one in it's limit, no one could never be sure to know what the world really looks like.

	Mathematicians responded to this dual problem with excessive rigour, trying to apply the axiomatic method in all mathematics. In the pre-axiomatic period, the proofs were often based on commonly accepted intuitions of the "reality" of mathematics, which could not automatically be regarded as valid.

	The new way of thinking about mathematics led to a series of spectacular success. Yet this had also a price. The axiomatic method made the connection between mathematics and the phenomenal  reality increasingly close. Meanwhile, discoveries suggested that mathematical axioms that appeared to be consistent with phenomenal experience could lead to dizzying contradictions with this experience.

	Most mathematicians quickly became "formalist" arguing that pure mathematics could only be regarded as a kind of elaborate philosophy game that was played with symbols on paper (that's the theory that is behind the mathematical prophetic qualification "zero content system" by Robert Heinlein). The "Platonic" belief in the reality of mathematical objects, in the old-fashioned way, seemed good for the trash, despite the fact that mathematicians still feel like platoniciens during the process of discovery of mathematics.

	Philosophically, then, the axiomatic method led most mathematicians to abandon previous beliefs in the metaphysical specificity of mathematics. It also produced the contemporary rupture between pure and Applied Mathematics. Most of the great mathematicians of the early modern period - Newton, Leibniz, Fourier, Gauss and others - also occupied phenomenal science. The axiomatic method had hatched the modern idea of the pure mathematician as a great aesthete, heedless of physics. Ironically, formalism gave the pure mathematicians a bad addiction to the Platonic attitude. The researchers in Applied Mathematics ceased to meet physicists and learned to put themselves in their behind.

	This takes us to the early 20th century. For the beleaguered minority of Platonists, the worst was yet to come. Cantor, Frege, Russell and Whitehead showed that all pure mathematics could be built on the simple foundation of the Set Theory axiomatic. This suited well the formalists: the mathematics were reunifying, at least in principle, from a small set of rules detached of a big one. Platoniciens also were satisfied, if a great structure appeared, consistent keystone for the whole mathematics, the metaphysical specificity of mathematics could still be saved.

	In a negative way, though, a Platonist had the last word. Kurt Gödel put his grain of sand in the program of axiomatization formalism when he proved that any sufficiently powerful axiom system to include integers numbers had to be either inconsistent (contain contradictions) or incomplete (too weak to decide the rightness or the falsity of some statements of the system). And that's more or less where things stand today. Mathematicians know that many attempts to advance mathematics as a priori knowledge of the Universe must face numerous paradoxes and unable to decide which axiom system describes the real mathematics. They have been reduced to hope that standards axiomatizations are not inconsistent but just incomplete, and wondering anxiously what contradictions or unprovable theorems are waiting to be discovered elsewhere.

	However, on the front of empiricism, mathematics was always a spectacular success as a theoretical construction tool. The great success of physics in the 20th century (General Relativity and Quantum physics) pushed so far out of the realm of physical intuition, they could only be understood by meditating deeply on their mathematical formalism, and extending their logical conclusions, even when those findings seemed wildly bizarre. What irony! Just as the mathematical perception were to appear always less reliable in pure mathematics, it became more and more indispensable in phenomenal science.

	In contrast to this background, the applicability of mathematics to phenomenal science poses a more difficult problem than at first appears. The relation between the mathematical models and prediction of phenomena is complex, not only in practice but also in principle. Even more complex, as we now know, there are ways to axiomatize mathematics that mutually exclude themselves!

	But why is there only one good choice of mathematical model? That is, why is there  a mathematical formalism, for example for quantum physics, so productive that it predicts the discovery of new observable particles?

	To answer this question we will can observe that, as well, works as a kind of definition. For many phenomenal systems, such exact predictive formalism has not been found, and none seem plausible. You can easily find such examples: climate or the behavior of a superior economy to that of a town - systems so chaotically interdependent that exact prediction is actually impossible (not only in practice but in principle).

	\pagebreak
	\subsection{Paradoxes}
	Since ancient times, some logicians had noticed the presence of many paradoxes within rationality. In fact, we can say that despite their number, these paradoxes are merely illustrations of a few paradoxical structures. Let us look to for general culture to the most famous which constitute the class of "\NewTerm{undecidable propositions}"\index{undecidable propositions}.

	\begin{tcolorbox}[colframe=black,colback=white,sharp corners]
	\textbf{{\Large \ding{45}}Example:}\\\\
The paradox of the class of classes (Russell)\\\\
There are two types of classes: those that contain themselves (or reflexive classes: the class of non-empty sets, the class of classes, ...) and those who do not contains themselves (or non-reflexives classes: the class of work to be returned, the class of blood oranges, ...). The question is the following: is the class of non-reflexives classes itself reflexive or non-reflexive? If it is reflexive, it contains itself and is thus in the class of non-reflexives classes that it represents, which is contradictory. If it is non-reflexive, it must be included in the class of non-reflexives classes and becomes ipso facto reflexive, we are facing again a contradiction.\\\\
This Russell's paradox is often known mainly under the two following variants:
	\begin{itemize}
		\item Does the set of all sets that do not contain themselves contain himself?\\\\
		The answer is: If "Yes", then "No" and if "No" then "Yes"...
		\item Those who do not shave themselves are shaved by the barber but not those who shave themselves. So who shaves the barber?\\\\
		The answer is: If the barber shave himself he enters in the category of people that shave themselves so he does not shave himself because he is the barber... But if he does not shave himself he enters in the category of people that are shaved by the barber... The answer is also undecidable...
	\end{itemize}
	\end{tcolorbox}

	Russell's paradox challenges the notion of a set as a collection defined by common ownership! In one shot it destroys the logic (undecidable proposition) and set theory... because the overall concept of all sets is an impossibility!!! The self-reference is the center of this logical problem!

	This paradox also returns to the question whether a math question correctly formulated (logical) necessarily admits an answer? Said in another way: is any mathematical statement provable... and it is Gödel that many years after the statement of Russell's paradox proved mathematically that the answer is No !!!!!! In other words, there will always be questions unanswered because any system (living language or mathematical tool) based itself is necessarily incomplete! This is the famous impact of Gödel's incompleteness theorem that is technically written as following (and that we will have to prove):
	
	that means that given a proposition $P$, whatever the set of axioms, there exist proposition that we will be never able to prove as true or false.
	
	Let us see another application of the Russell's Paradox:
	\begin{tcolorbox}[colframe=black,colback=white,sharp corners]
	\textbf{{\Large \ding{45}}Example:}\\\\
	In a library, there are two types of catalogues: Those who mention themselves and those who does not mention themselves. A librarian must draw up a catalogue of all catalogues that do not mention themselves. Having completed its work, our librarian asks whether or not to mention the catalogue that is precisely drafting. At this point, he is struck perplexity. If he does not mention this catalogue it will be a catalogue that is not mentioned and which should therefore be included in the list of catalogues that does not mention themselves. On the other hand, if he mentions the catalogue, this catalogue will become a catalogue that is mentioned and must therefore not be included in this catalogue, since it is the catalogue of catalogues which does not mention themselves.
	\end{tcolorbox}
	
	A variations of the previous paradox is the well-known liar paradox:

	\begin{tcolorbox}[colframe=black,colback=white,sharp corners]
	\textbf{{\Large \ding{45}}Example:}\\\\
	Let us provisionally define lying as the work of making a false proposition. The Cretan poet Epimenides said: "All Cretans are liars", this is the proposal $P$. How to decide the truthfulness of $P$? If $P$ is true, as Epimenides is Cretan, $P$ must be false. $P$ must therefore be false to be true, which is contradictory.
	\end{tcolorbox}

As would have made understand the logician Ludwig Wittgenstein, these paradoxes ultimately show that mathematics is a pretty good tool to show the logic but not to talk about it. Give with mathematics an independent existence to this algebraic entities is madness and it is this that produces monsters like the set of all the sets... The logic is empty and can not tell the reality, it restrict to be just a picture of it.

	\pagebreak
	\subsubsection{Hypothetical-Deductive Reasoning}
	The hypothetical-deductive reasoning is, we know (see the Introduction of the book), the ability of the learner to deduce conclusions from pure hypotheses and not only of a real observation. It is a thought process that seeks to identify a causal explanation of any phenomenon (we will come back on this during our first steps in physics). The learner who uses this type of reasoning begins with a hypothesis and then tries to prove or disprove his hypothesis following the block diagram below:

	\begin{figure}[H]
		\begin{center}
			\includegraphics[scale=0.75]{img/intro/hypothesis_definitions.eps}
		\end{center}	
		\caption{Hypothetical-Deductive Reasoning block diagram}
	\end{figure}
	The deductive procedure is to hold as true, provisionally, this first proposal that we name, in logic a "predicate" (see further below for more details) and to draw all the consequences logically necessary, that is to say to look for its implications.

	\begin{tcolorbox}[colframe=black,colback=white,sharp corners]
	\textbf{{\Large \ding{45}}Example:}\\\\
	Consider the proposal $P$: "X is a man", it implies the following proposition $Q$: "X is mortal".\\
	
	The expression $P \Rightarrow Q$ (if it is a human it is necessarily mortal) is a predicative implication (hence the term "predicate"). There is no case in this example where we can state $P$ without $Q$. This example is that of strict implication, as we find in the "syllogism" (logical reasoning figure).
	\end{tcolorbox}

	\begin{tcolorbox}[title=Remark,colframe=black,arc=10pt]
Experts have shown that the hypothetical-deductive reasoning develops gradually by children from six to seven years old and that this kind of reasoning is used systematically starting with a strict propositional function until the age of eleven-twelve.
	\end{tcolorbox}
	
	\subsection{Propositional Calculus}
	
The "\NewTerm{propositional calculus}"\index{propositional calculus} (or "propositional logic"\index{propositional logic}) is an absolutely indispensable preliminary to tackle a background in science, philosophy, law, politics, economics, etc. This type of calculation allows for decisions or testing procedures. These help to determine when a logical expression (proposition)  is true and especially if it is always true.

\textbf{Definitions (\#\mydef):}

	\begin{enumerate}
		\item[D1.] An expression that is always true whatever the content language of the variables that compose it is named a "\NewTerm{valid expression}"\index{valid expression}, a "\NewTerm{tautology}"\index{tautology} or a "\NewTerm{law of propositional logic}"\index{law of propositional logic}.
		\item[D2.] An expression that is always false is named a "\NewTerm{contradiction}"\index{contradiction} or "{antilogy}"\index{antilogy}.
		\item[D3.] An expression that is sometimes true, sometimes false is named a "\NewTerm{contingent expression}"\index{contingent expression}.
		\item[D4.] We name "\NewTerm{assertion}"\index{assertion} an expression that we can say unambiguously whether it is true or false.
		\item[D5.] The "\NewTerm{object language}"\index{object language} is the language used to write logical expressions.
		\item[D6.] The "\NewTerm{meta-language}"\index{meta language} is the language used to talk about the object language in everyday language.
	\end{enumerate}

	\begin{tcolorbox}[title=Remarks,colframe=black,arc=10pt]
\textbf{R1.} There are expressions that are actually not assertions. For example, the statement "this statement is false" is a paradox that can be neither true nor false.\\\\
\textbf{R2.} Consider a logical expression $A$. If it is a tautology, we frequently note it $\models A$ and the $A \models$ if it is a contradiction.\\\\
\textbf{R3.} In mathematics we can try to prove in a general way that an assertion is true, but not that it is false (if this is the case we give just one example).
	\end{tcolorbox}

	\pagebreak
	\subsubsection{Propositions (premises)}

	\textbf{Definition (\#\mydef):} In logic, a  "\NewTerm{proposition}"\index{proposition} is a statement that has meaning. That means we can say unambiguously whether this statement is true ($T$) or false ($F$). This is what we name the "\NewTerm{Law of excluded middle}"\index{law of excluded middle}.

	\begin{tcolorbox}[colframe=black,colback=white,sharp corners]
	\textbf{{\Large \ding{45}}Examples:}\\\\
	E1. "I lie" is not a proposition (premise). If we assume that this statement is true, it is an affirmation of his own disability, so we should conclude that it is false. But if we assume that it is false, then the author of this statement does not lie, so he told the truth, thus the proposal would be true...\\
	
	E2. Another funny example is:
	\begin{itemize}
		\item Everything has a creator
		\item God is that creator
		\item God does not have creator
	\end{itemize}
	It's a solution that fails since it violates its own premise...
	\end{tcolorbox}

\textbf{Definition (\#\mydef):} A proposition in binary logic (where the proposals are either true or false) is therefore never true and false at the same time. This is what we call the "\NewTerm{principle of non-contradiction}"\index{principle of non-contradiction}.

Thus, a property on the set of propositions $E$ is an application $P$ from $E$ to the set of "\NewTerm{truth values True, False}\index{truth values (True, False)}" $\left\lbrace T,F\right\rbrace$:
	
We speak about "\NewTerm{associated subset}"\index{associated subset} , when the proposition only generates a portion $E'$ of $E$ and vice versa.

	\begin{tcolorbox}[colframe=black,colback=white,sharp corners]
	\textbf{{\Large \ding{45}}Example:}\\\\
	In $E=\mathbb{N}$, if $P(x)$ states "$x$ is even", then $P=\{0,2,4,\ldots,2k,\ldots\}$ which is indeed only an associated subset of $E$ but of same Cardinal (\SeeChapter{see section Sets Theory}).	
	\end{tcolorbox}
	
	In $E=\mathbb{N}$, if the proposition $P(x)$ is "$x$ is even", then $\left\lbrace 0,2,4,...,2k,...\right\rbrace$ which is effectively an associated subset of $E$  but with same Cardinal (\SeeChapter{section Set Theory}).
	
	\pagebreak
	\textbf{Definition (\#\mydef):} Let $P$ be a property of the set $E$. A property $Q$ on $E$ is a "\NewTerm{negation}"\index{negation} of $P$ if and only if, for any $ x \in E$:
	\begin{itemize}
		\item $Q(x)$ is $F$ (false) if $P(x)$ is $T$ (true)
		\item $Q(x)$ is $T$ (true) if $P(x)$ is $F$ (false)
	\end{itemize}
We can gather these conditions in a table named "\NewTerm{truth table}"\index{truth table}:
		
Table that we can also find or also write in the most explicit following form:

	

or in binary form:

		

In other words, $P$ and $Q$ always have opposite truth values. We denote this kind of statement "$Q$ is a negation of $P$":
	
where the symbol $\neg$ is the "\NewTerm{negation connector}"\index{negation connector}.

	\begin{tcolorbox}[title=Remark,colframe=black,arc=10pt]
	The expressions must be well-formed expressions (often abbreviated "WFE"). By definition, any variable is a well-formed expression, thus $\neg P$ is a well-formed expression. If $P, Q$ are well-formed formulas, then $P \Rightarrow Q$ is a well-formed expression (the expression "I am lying" is not well-formed because it contradicts itself).
	\end{tcolorbox}

	\pagebreak
	\subsubsection{Connectors}
There are other types of logical connectors:

\textbf{Definition (\#\mydef):} Let $P$ and $Q$ two properties set defined on the same set $E$. $P\vee Q $ (read "$P$ \texttt{OR} $Q$") is a property on $E$ defined by:
	\begin{itemize}
		\item $P\vee Q$ is true if at least $P$ or $Q$ are true
		\item $P\vee Q$ is false otherwise
	\end{itemize}

We can create the truth table of the "\NewTerm{\texttt{OR} connector}"\index{OR connector} or "\NewTerm{disjunction connector}"\index{disjonction connector} $\vee$:
		
It should be easy to convince yourself that if the parts $P, Q$ of $E$ à are respectively associated with the properties $P, Q$ thus $P \cup Q$ (\SeeChapter{see section Set Theory}) is associated to $P \vee Q$:
	
The connector $\vee$ is associative (no doubt about the fact that it is commutative!). For proof, just do a truth table where you can check that:
	
\textbf{Definition (\#\mydef):} There is also the "\NewTerm{\texttt{AND} connector}"\index{AND connector} or also named "\NewTerm{conjunction connector}"\index{conjuction connector} $\wedge$ for whatever are $P, Q$ two properties defined on $E$, $P \wedge Q$ is a property defined on $E$ by:
	\begin{itemize}
		\item $P \wedge Q$ is true if both properties $P, Q$ are true (the famous syllogism: \textit{All men are mortal, Socrates is a man, therefore Socrates is mortal} is a famous example).
		\item $P \wedge Q$ if false otherwise
	\end{itemize}
We can create the truth table of the "\NewTerm{\texttt{AND} connector}"\index{AND connector} or "\NewTerm{disjunction connector}"\index{disjunction connector} $\vee$:
	
It should be also almost easy to convince yourself that if the parts $P, Q$ of $E$ à are respectively associated with the properties $P, Q$ thus $P \cap Q$ (\SeeChapter{see section Set Theory}) is associated to $P \wedge Q$.
	
The connector $\wedge$ is associative (no doubt about the fact that it is commutative!). For proof, just do a truth table where you can check that:
	
The  connectors $\vee,\wedge$ are distributive one on the other. Using a simple truth table, we can show that (ask me if you want a put the truth table):
	
as well as:
	
	
\textbf{Definition (\#\mydef):} The "\NewTerm{negation}"\index{negation}  operator $\lnot$ transform a True value into a False value such that:
	
So in logic, negation, also named "\NewTerm{logical complement}"\index{legal complement}, is an operation that takes a proposition $P$ to another proposition "not $P$", written $\lnot P$ or sometimes $\bar{P}$, which is interpreted intuitively as being True when $P$ is false and False when $P$ is True. Negation is thus a unary (single-argument) logical connective.

As we will prove it in detail in the section of Logic System (using a simple truth table) the "\NewTerm{De Morgan's laws}"\index{De Morgan's laws} provide a way of distributing negation over disjunction and conjunction:
	
	\begin{tcolorbox}[title=Remark,colframe=black,arc=10pt]
To see the details of all logical operators, the reader should read the section of Logical Systems (\SeeChapter{see chapter Theoretical Computing}) where the identity, the double negative, the idempotence, associativity, the distributive properties, the De Morgan relations are presented more formally and with full details.
	\end{tcolorbox}

Let us now come back on the "\NewTerm{logical implication connector}" \index{logical implication connector} sometimes also named just the  "\NewTerm{conditional}"\index{conditional}  denoted by the symbol $\Rightarrow$.

	\begin{tcolorbox}[title=Remark,colframe=black,arc=10pt]
In some books on propositional calculus, this connector is denoted by the symbol $\supset$ and as part of the proof theory we often prefer the symbol $\rightarrow$.
	\end{tcolorbox}

Let $P, Q$ two properties given on $E$. $P \Rightarrow Q$ is a property on $E$ defined by:
	\begin{enumerate}
		\item[P1.] $P \Rightarrow Q$ is False if $P$ is True and $Q$ is False.
		\item[P2.] $P \Rightarrow Q$ is True otherwise.
	\end{enumerate}
In other words, the fact that $P$ logically implies $Q$ means that $Q$ is True for any assessment for which $P$ is True. The implication is therefore the famous "if... then ...".

If we write the truth table of the implication (caution with the before last line!!!):

	
In other terms, a False proposition implies that any conclusion will always be True. If the proposition is True the implication can be True only if the result is True.

	\begin{tcolorbox}[colframe=black,colback=white,sharp corners]
\textbf{{\Large \ding{45}}Example:}\\\\
Consider the proposition: "If you get your diploma, I buy you a computer".\\

Of all cases, only one corresponds to a broken promise: the one where the child graduates, and still has no computer (second line in the table above).\\

What means exactly this promise, that we will write as following:
\begin{center}
You have your degree $\Rightarrow$ I buy you a computer"?
\end{center}

Exactly this:
	\begin{itemize}
		\item If you have get graduate, for sure, I will buy you a computer (I can not not buy it).
		\item If you do not get graduate, I said nothing.
	\end{itemize}
	\end{tcolorbox}
The implication gives us that from any false proposition we can deduce any proposal (last two lines).
	\begin{tcolorbox}[colframe=black,colback=white,sharp corners]
\textbf{{\Large \ding{45}}Example:}\\\\
In a course teached by Russell on the subject from a false proposition, any proposal can be inferred, a student asked him the following question (anecdote or legend?):
	\begin{itemize}
		\item "Are you saying that from the proposition $2 + 2 = 5$, it follows that you are the Pope?".
		\item "Yes", answered Russell.
		\item "And could you prove it!", asked the student skeptical...
		\item "Certainly", answered Russell, who immediately offered the following proof:
			\begin{enumerate}
				\item Suppose that $2 + 2 = 5$.
				\item Subtract $3$ from each member of the equality, we thus get $1 = 2$.
				\item By symmetry $2=1$.
				\item The Pope and I are $2$. Since $2 = 1$, Pope and I are $1$. It follow I'm the Pope.
			\end{enumerate}
	\end{itemize}
	\end{tcolorbox}
The implication connector is essential in mathematics, philosophy, etc. It is a backbone of any proof, evidence or deduction. It has the following useful properties (normally easy to check with a small truth table):
	
And we have from the last property (again verifiable by a truth table):
	
The "\NewTerm{logical equivalence connector}"\index{logical equivalence connector}  or "\NewTerm{biconditional connector}"\index{biconditional connector} denoted most of times by "$\Leftrightarrow$" or sometimes by "$\leftrightarrow$" meaning by definition:
	
in other words, the first expression has the same value for all evaluation of the second. It is the same with the following relation that is more "atomic" as the logical equivalence is reduced only to the use of $\wedge,\vee$ and negation $\lnot$ (combination of what we have seen above):
	
When we prove such equivalence of two expressions we can therefore say that: "we prove that the equivalence is a tautology".

The truth table of the equivalence is logically given by:

	

$P \Leftrightarrow Q$ means (when its true!) that "$P$ and $Q$ always have the same truth value" or "$P$ and $Q$ are equivalent." This is True if $P$ and $Q$ have the same value, False otherwise.

Of course (it is a tautology):
	
	The relation $P \Leftrightarrow Q$ is equivalent to that $P$ is a necessary and sufficient condition for $Q$ and that $Q$ is a necessary and sufficient condition for $P$.

	The conclusion is that the conditions of the types: "necessary", "sufficient", "necessary and sufficient" can be reformulated with the terms "only if", "if", "if and only if".

	Therefore:
	\begin{enumerate}
		\item $P \Rightarrow Q$ reflects the fact that $Q$ is a \NewTerm{necessary}\index{condition: necessary} condition for $P$ or in other words, $P$ is True \NewTerm{only if}\index{condition: only if} $Q$ is True (in the truth table, when $P \Rightarrow Q$ is equal to $1$ we see that $P$ is $1$ \NewTerm{only if} $Q$ is also $1$). We also say that \NewTerm{if} $P$ is true \NewTerm{then} $Q$ is true.
		\item $P \Longleftarrow Q$ or what is the same $Q \Rightarrow P$ reflects the fact that $Q$ is a \NewTerm{sufficient}\index{condition: sufficient} condition for $P$ or in other words, $P$ is True \NewTerm{if} $Q$ is True (in the truth table, when $Q \Rightarrow P$ takes the value $1$ we see that $P$ is $1$ \NewTerm{if} $Q$ is $1$ too).
		\item $P \Leftrightarrow Q$ reflects the fact that $Q$ is a \NewTerm{necessary and sufficient}\index{necessary and sufficient}\index{condition: necessary and sufficient}  condition for $P$ or in other words, $P$ is True \NewTerm{if and only if}\index{condition: if and only if}  $Q$ is True (in the truth table, when $P \Leftrightarrow Q$ takes the value $1$ we see that $P$ is $1$ \NewTerm{if} $Q$ is $1$ and \NewTerm{if and only if} $Q$ is equal to $1$).
	\end{enumerate}

	\begin{tcolorbox}[title=Remark,colframe=black,arc=10pt]
	The expression "if and only if" therefore corresponds to a logical equivalence and can only be used to describe a bi-implication !!
	\end{tcolorbox}
	
	The first stage of propositional calculus is the formalization of natural language statements. To make this work, the propositional calculus finally provides three types of tools:
	
	\begin{enumerate}
		 \item The "\NewTerm{propositional variables}\index{propositional variables}" ($P, Q, R, ...$) symbolize any simple proposals. If the same variable occurs multiple times, each time it symbolizes the same proposal.
		 
		 \item The five logical operators: $\neg , \wedge, \vee, \Leftrightarrow, \Rightarrow$.
		 
		 \item Punctuation are reduced to only opening and closing parentheses that organize reading so as to avoid ambiguity.
	\end{enumerate}
	
		
	The reader should find sometimes by some authors that like to use at minimum the natural language in their books the symbol the sign "$\therefore$" that is sometimes placed before a logical consequence, such as the conclusion of a syllogism.  The symbol consists of three dots placed in an upright triangle and is read \textit{therefore}. We can also make use of the symbol "$\because$" and is read \textit{because}. 
	
	\begin{tcolorbox}[colframe=black,colback=white,sharp corners]
	\textbf{{\Large \ding{45}}Example:}\\\\
	$\because$ All men are mortal.\\
	$\because$  Socrates is a man.\\
	$\therefore$ Socrates is mortal
	\end{tcolorbox}
	In this book we will avoid using this notation as the engineers don't make use of this a lot. 
	
	It is possible to establish equivalences between these operators. We have already seen how the biconditional could be defined as a product of reciprocal conditional, let us see now other equivalences:
	
	\begin{tcolorbox}[title=Remark,colframe=black,arc=10pt]
	The classical operators $\wedge, \vee, \Leftrightarrow$ can therefore be defined using the canonical operators $\neg, \Rightarrow$ through equivalence laws between them.
	\end{tcolorbox}
	Also notice the two relations of De Morgan (see sectionof Boolean Algebra for the proof):
	
	They allow to transform the disjunction into conjunction and vice versa:
	
	
	\subsubsection{Decision procedures}
	We have previously introduced the basic elements allowing us to operate on expressions from properties (propositional variables) without saying much about the handling of such expressions. So now you need to know that in propositional calculus there are two ways to establish that a proposition is a law of propositional logic. We can either:
	
	\begin{enumerate}
		\item Use non-automated procedures
		
		\item Use axiomatic and demonstrative procedures
	\end{enumerate}
	\begin{tcolorbox}[title=Remark,colframe=black,arc=10pt]
	In many books these procedures are presented before the structure of the propositional language. We chose here to do the opposite thinking that the approach would be easier.
	\end{tcolorbox}
	
	\paragraph{Non-axiomatic procedural decisions}\mbox{}\\\\
	Several of these methods exist, but we will limit ourselves here to the simplest of them, that of the matrix calculation, often referred to as "\NewTerm{methods of truth tables}\index{method of truth tables}".
	
	The construction procedure is as we have already seen quite simple. Indeed, the truth value of a complex expression is a function of the truth value of the simple statements that compose it, and finally based on the truth value of propositional variables that makes it. Considering all possible combinations of truth values of propositional variables, we can determine the truth values of the complex expression.
	
	Truth tables, as we have seen it, give us the possibility to decide, about any proposition, if this latter is a tautology (always true), a contradiction (always false) or a contingent expression (sometimes true, sometimes false).
	
	We can distinguish at least four ways to combine propositional variables, brackets and connectors:
	
	The method of truth tables helps to determine the type of expression that are well-formed to which we have to face. It requires, in principle, no invention, it is "only" a mechanical procedure. Axiomatized procedures, however, are not entirely mechanical. Inventing a proof as part of an axiomatized system requires sometimes hability, practice or luck. Regarding to truth tables, here is the protocol to follow:
	
	When facing a well-formed expression, or function of truth, we first determine how many distinct propositional variables we are dealing with. We then examine the various arguments that form this expression. We then construct a table with $2^n$ columns ($n$ being the number of variables and without forgotting that they are binary variables!) and a number of columns equal to the number of arguments plus columns for the expression itself and its other components (see previous examples). Then we assign to the variables the various combinations of True ($1$) and False ($0$) values that may be conferred upon them. Each row corresponds to a possible outcome and all of the rows is the set of all possible outcomes. There is, for example, a possible outcome wherein $P$ is a true statement while $Q$ is false.
	
	\paragraph{Axiomatic procedural decisions}\mbox{}\\\\
	The axiomatization of a theory implies, besides its formalization, that we start form a finite number of axioms and that through the controlled transformation of these, we can get all the theorems of this theory. So we start from a few axioms whose truth is a statement (not proven). We determine afterwards deduction rules for manipulating the axioms or expression obtained from these. The sequence of these deductions is a proof that leads to a theorem, a law or lemma.
	
	We will now briefly present two axiomatic systems, each consisting of axioms using two specific rules named "\NewTerm{inference rules}\index{inference rules}" (intuitive rules):
	
	\begin{enumerate}
		\item The "\NewTerm{modus ponens}\index{modus ponens}": If we prove $A$ and $A\Rightarrow B$, then we can deduce $B$. $A$ is named the "\NewTerm{minor premise}\index{minor premise}" and $B$ the "\NewTerm{major premise}\index{major premise}" of the modus ponens rule.
		\begin{tcolorbox}[colframe=black,colback=white,sharp corners]
		\textbf{{\Large \ding{45}}Example:}\\\\
		From:
		
		and:
		
		we can deduce that:
		
		\end{tcolorbox}
		\begin{tcolorbox}[title=Remark,colframe=black,arc=10pt]
		Humans typically communicate in a way that resists shallow logical analysis. In a real conversation, people use words rather than terms, make utterances rather than sentences, and employ a wider variety of inference methods than modus ponens. A great deal of what is communicated and inferred in a conversation depends on context, the speakers and audience, their history, their shared knowledge and confidences, the feelers they lay out to establish mutual trust and rapport.
		\end{tcolorbox}
		
		\item The "\NewTerm{substitution}\index{substitution}": we can in a schema of axioms replace a letter by any formula, at the condition that all identical letters are replaced by identical formulas!
		
		Let us give as an example, two axiomatic systems: the axiomatic system of Whitehead and Russell, the axiomatic system of Lukasiewicz.
		\begin{enumerate}
			\item The axiomatic system of Russell and Whitehead adopts $\neg, \vee$ as primitive symbols and define $\Rightarrow, \wedge ,\Leftrightarrow$ from these latter as follows (easily verifiable relations with truth tables):
			
			This system includes $5$ axioms, somewhat quite obvious plus two rules of inference. Axioms are given here using non-primitive symbols, as did Whitehead and Russell:
			\begin{enumerate}
				\item[A1.] $(A \vee A)\Rightarrow A$
				\item[A2.] $B \Rightarrow (A\vee B)$
				\item[A3.] $(A\vee B) \Rightarrow (B\vee A)$
				\item[A4.] $(A\vee (B\vee C)) \Rightarrow (B\vee (A\vee C))$
				\item[A5.] $(B\Rightarrow C)\Rightarrow (A\vee C)$
			\end{enumerate}
			we have already presented above some of these.
			\begin{tcolorbox}[title=Remark,colframe=black,arc=10pt]
			These five axioms are not independent of each other. The fourth can be obtained from the other four.
			\end{tcolorbox}
			For example, to justify that $\neg A\vee A$ has a sense, we can proceed as following:
			
			
			\item The axiomatic system of Lukasiewicz includes three axioms, plus the two rules of inference (modus ponens and substitution):
			\begin{enumerate}
				\item[A1.] $(A\Rightarrow B) ((B\Rightarrow C) \Rightarrow (A\Rightarrow C))$
				\item[A2.] $A\Rightarrow  (\neg A\Rightarrow B)$
				\item[A3.] $(\neg A \Rightarrow A)\Rightarrow  A$
			\end{enumerate}
			Here is the proof of the first two axioms in the system of Russell and Whitehead. These are the formulas (6) and (16) of the following derivation:
			
		\end{enumerate}
	\end{enumerate}
	These axiomatizations let us found as theorems all tautologies or laws of the propositional logic. From everything that has been said so far, we can try to define what is a proof!!!!
	
	\textbf{Definition (\#\mydef):} A finite sequence of formulas $B_1,B_2,\ldots,B_m$ is name a "\NewTerm{proof}\index{proof}" from the assumptions/hypothesis $A_1,A_2,\ldots,A_n$ if for each $i$:
	\begin{itemize}
		\item $B_i$ is one of the assumptions/hypothesis $A_1,A_2,\ldots,A_n$
		
		\item or $B_i$ is a variant of an axiom
		
		\item or $B_i$ is inferred (by the application of the modus ponens rule) from the major premise $B_j$ and minor premise $B_k$ where $j,k<i$
		
		\item or $B_i$ is inferred (by the application of the substitution rule) from an anterior premise $B_j$, the replaced variable not appearing in $A_1,A_2,\ldots,A_n$
	\end{itemize}
	Such a sequence of formulas, $B_m$ being the final formula of the sequence, is more explicitly named "\NewTerm{proof of $B_m$}" from the assumptions/hypothesis (or axioms) $A_1,A_2,\ldots,A_n$, what we also write:
	
	 More explicitly a proof is a deductive argument for a mathematical statement. In the argument, other previously established statements, such as theorems, can be used. In principle, a proof can be traced back to self-evident or assumed statements, known as axioms.

	Proofs employ logic but usually include some amount of natural language which usually admits some ambiguity. In fact, the vast majority of proofs in written mathematics can be considered as applications of rigorous informal logic. Purely formal proofs, written in symbolic language instead of natural language, are considered in proof theory. 
	\begin{tcolorbox}[title=Remark,colframe=black,arc=10pt]
	Note that when we try to prove a result from a number of assumptions, we do not try to prove the assumptions themselves!
	\end{tcolorbox}
	
	\pagebreak	
	\subsubsection{Quantifiers}
	We have to complete the use of the connectors of propositional calculus by what we name "\NewTerm{quantifiers}\index{quantifiers}" if we wish to solve some problems. Indeed, the propositional calculus does not allow us to state general things about the elements of a set, for example. In that sense, propositional logic is only part of the reasoning. The calculus of predicates on the contrary allows to formally handle statements such as "there exists an $x$ such that [$x$ has an American car]" or "for all $x$ [if $x$ is a dachshund, then $x$ is small]". In short, we extend the composed formulas in order to assert existential quantifiers ( "there...") and universal quantifiers ( "for every..."). The examples we just gave involve a bit special proposals like "$x$ has an American car." This is proposition with a variable. These proposals are in fact the application of a function to $x$. This function, is this that associates "$x$ has an American car" with $x$. We will denote this function by "... has an American car" and we say that is a propositional function because it is a function whose value is a proposal. Or a "predicate" as we already know.
	
	The existential and universal quantifiers go hand in hand with the use of propositional functions. The predicate calculus is however limited in the existential and universal formulas. Thus, we prohibit ourselves to use sentences like "there is an affirmation of $x$ such that ...". In fact, we allow ourselves to quantify only "individuals". This is why predicate logic is named "\NewTerm{first-order logic}\index{first order logic}" because it uses variables as basic mathematical objects (while in the second-order logic they can also be sets).
	\begin{center}
	\includegraphics[scale=0.9]{img/arithmetics/first_vs_second_order_logic.jpg}
	\end{center}
	
	Before turning to the study of the predicate calculus we define:
	\begin{enumerate}
		\item[D1.] The "\NewTerm{universal quantifier}\index{universal quantifier}": $\forall$  (for all)

		\item[D2.]  The "\NewTerm{existential quantifier}\index{existential quantifier}": $\exists$  (exists)
	\end{enumerate}
	\begin{tcolorbox}[colframe=black,colback=white,sharp corners]
	\textbf{{\Large \ding{45}}Example:}\\\\
	If any complex number is the product of a non-negative number and a number of modulus $1$ we will write:
	
	\end{tcolorbox}
	
	The order of quantifiers is critical to meaning, as is illustrated by the following two propositions:
	\begin{center}
	For every natural number n, there exists a natural number $s$ such that $s$ = $n^2$.
	\end{center}

	This is clearly true! It just asserts that every natural number has a square. The meaning of the assertion in which the quantifiers are turned around is different:

	\begin{center}
	There exists a natural number $s$ such that for every natural number $n$, $s = n^2$.
	\end{center}
	This is clearly false! It asserts that there is a single natural number s that is at the same time the square of every natural number. 
		
	A frequent question in physics and mathematics is to know if the universal quantifier has to be before of after the predicates they refer to. In fact, strictly in terms of formal logic, quantifiers are always at the beginning of any formula. However, almost no one gives a proof that is written in the formal language. Even simple proofs would be very long and unreadable. But anyone, regardless of what natural language they speak, will interpret a sentence in the formal language in the same way. The price for this clarity of course is readability. Natural languages, because of their inherent ambiguity, are subject to many more limitations.
		
	Obviously the proper usage of a formal notation or of a more informal one depends particularly on the context of presentation. It is essential to whom we communicate an idea and this should guide us to use a suitable level of formal notation.
	
	We use the sometimes the symbol $\exists !$ to say briefly: "there is one and only one". 
	\begin{tcolorbox}[colframe=black,colback=white,sharp corners]
	\textbf{{\Large \ding{45}}Example:}\\\\
	A famous example is the way to explicit that the logarithm is a bijective function:
	
	\end{tcolorbox}
	We will see now that the Proof Theory and Set Theory is the exact transcription of the principles and results of Logic (the one with a capitalized "L" )
	
	\pagebreak
	\subsection{Predicate Calculus}
	In mathematics courses (algebra, analysis, geometry, ...), we prove the properties of different object types (integer, real, matrices, sequences, continuous functions, curves, triangles, ...) . To prove these properties, we need of course that the objects on which we work are clearly defined (what is a set, what is a real, what is point, ...?).
	
	In first-order logic and, in particular, in proof theory, the objects we study are the formulas and proofs. We must therefore give a precise definition of what are these objects. The terms and formulas are the grammar of a language, oversimplified and calculated exactly to say what we want without ambiguity and without unnecessary detour.
	
	\subsubsection{Grammar}
	\textbf{Definitions (\#\mydef):}
	\begin{enumerate}
		\item[D1.] The "\NewTerm{terms}\index{terms}" designate items for which we want to prove some properties (we will discussed the latter much more in details further below):
		\begin{itemize}
			\item In algebra, the terms refer to the elements of a set (group, ring, field, vector space, etc.). We also manipulate sets of objects (subgroups, subrings, subfields, etc.). The terms which will designate the objects are named "\NewTerm{second-order terms}\index{second-order terms}".

			\item In analysis, the terms refer most of time to real numbers (for example, if we put ourselves in functional spaces) or functions.
		\end{itemize}
		
		\item[D2.] The "\NewTerm{Formulas}\index{formulas}", are the properties of objects we study (we will discussed the latter also much more in details further below):
		\begin{itemize}
			\item In algebra, we can write formulas to express that two elements commute, that a subspace is of dimension $3$, etc.

			\item In analysis, we will write formulas to express the continuity of a function, the convergence of a sequence, etc.

			\item In set theory, formulas can express inclusion of two sets, membership of an element in a set, etc.
		\end{itemize}
		
		\item[D3.] The "\NewTerm{proof}\index{proof}", enable to check if a formula is true. The precise meaning of this word will also need to be defined. More precisely, they are deductions under assumptions, they allow to "lead from truth to truth", the question of the truth of the conclusion is then returned to that of the hypothesis, which does not look at the logic but is based on the knowledge we have on things we talk about.
	\end{enumerate}
	
	\pagebreak
	\subsubsection{Language}
	In mathematics we use, depending on the area, different languages that are distinguished by the symbols used. The definition below simply expresses that it is sufficient to only have to give the list of symbols to specify the language.
	
	\textbf{Definition (\#\mydef):} A "\NewTerm{language}\index{language}" is the content of a family (not necessarily finite) of symbols. We distinguish three kinds of languages: symbols, terms and formulas.
	
	\begin{tcolorbox}[title=Remarks,colframe=black,arc=10pt]
	\textbf{R1.} We use sometimes the word "vocabulary" or "signature" instead of the word "language".\\
	
	\textbf{R2.} We already know that the word "predicate" is used instead of the word "relation". We speak then of "predicate calculus" instead of "first-order logic").
	\end{tcolorbox}
	
	\paragraph{Symbols}\mbox{}\\\\
	There are different types of symbols we will try to define:
	\begin{enumerate}
		\item[D1.] The "\NewTerm{constant symbols}\index{constant symbols}" (see note below)
		\begin{tcolorbox}[colframe=black,colback=white,sharp corners]
		\textbf{{\Large \ding{45}}Example:}\\\\
		The neutral element $n$ in Set Theory (\SeeChapter{see section Set Theory})
		\end{tcolorbox}
	
		\item[D2.] The "\NewTerm{function symbols}\index{functions symbols}" or "\NewTerm{functors}\index{functors}". To each function symbol is assigned a strictly positive integer that we name her "\NewTerm{ary}": this is the number of arguments of the function arguments. If the arity is $1$ (resp. $2, \ldots, n$), we say then that the function is unary (resp binary., ..., $n$-ary).
		\begin{tcolorbox}[colframe=black,colback=white,sharp corners]
		\textbf{{\Large \ding{45}}Example:}\\\\
		The binary functor of multiplicaton $\times$ or $\cdot$ in group theory (\SeeChapter{see section Set Theory})
		\end{tcolorbox}
	
		\item[D3.] The "\NewTerm{relation symbol}\index{relation symbol}". Similarly to the previous definition, every relation symbol is associated with a positive or null integer (its arity) that corresponds to its number of arguments and we talk then of unary, binary, $n$-ary relation.
		\begin{tcolorbox}[colframe=black,colback=white,sharp corners]
		\textbf{{\Large \ding{45}}Example:}\\\\
		The relation $=$ is a binary relation (\SeeChapter{see section Set Theory})
		\end{tcolorbox}
	
		\item[D4.] The "\NewTerm{individual variables}\index{individual variables}". In what will follow we will give us an infinite set $\mathcal{V}$ of variables. The variables will be recorded as it is traditional by the latin lowercase letters: $x$, $y$, $z$ (possibly indexed: $x_1,x_2,x_3$).
	
		\item[D5.] To this we should add the connectors $\neg, \Rightarrow, \vee,\wedge$ and quantifiers $\forall,\exists,\exists!$ that we extensively discussed above, on which it is now useless to return.
	\end{enumerate}
	\begin{tcolorbox}[title=Remarks,colframe=black,arc=10pt]
	\textbf{R1.} A constant symbol can be seen as a function symbol with $0$ argument (arity zero).\\
	
	\textbf{R2.} We consider (unless otherwise stated) that each language contains the binary relation symbol $=$ (read "equal") and the relation symbol with zero argument denoted $\perp$ (read "bottom" or "absurd") representing as we already know the value FALSE. In the description of a language, we will often omit to mention them. The symbol $\perp$ is often redundant. We can indeed, without using it, write a formula that is always false. However, it can represent the FALSE in a canonical way and therefore to write general proofs rules. \\
	
	\textbf{R3.} The role of functions and relations is very different. As we will see, the function symbols are used to construct the terms (of language objects) and the relation symbols to build formulas (the properties of these objects).
	\end{tcolorbox}
	
	\paragraph{Terms}\mbox{}\\\\
	The terms, we also say the "first order terms", are the objects associated with the language. 
	
	\begin{enumerate}
		\item[D1.] Given $\mathcal{L}$ a language, the set $\mathcal{T}$ of terms on $\mathcal{L}$ is the smallest set containing the variables, the constants and stable (we do not go out of the set) by applying function symbols of   $\mathcal{L}$ to the terms.

		\item[D2.] A "\NewTerm{closed term}\index{closed term}" is a term that does not contain variables (by extension, only constants).

		\item[D3.] For a more formal definition, we can write:
		
		where $t$ is a variable or constant symbol and, for any $k\in \mathbb{N}$:
		
		where $f$ is obviously a function of arity $n$ (let us recall that the arity is the number of function arguments). Thus, for each arity, there is a degree of set of terms. We have finally:
		

		\item[D5]. We name "\NewTerm{height}\index{height of a term}" of a term $t$ the smallest $k$ such that $t\in \mathcal{T}_k$.
		
		This definition means that variables and constants are terms and that if $f$ is a $n$-ary function symbol and $t_1,\ldots,t_n$ are terms then $f(t_1,\ldots,t_n)$ is also a term itself. The set $\mathcal{T}$ of terms is defined by the grammar:
		
		This expression must be read as follows: a element of the set $\mathcal{T}$ we are defining is either an element of $V$ (variable) or an element of $S_c$ (the set of symbols of constant), or the application of a $n$-ary function symbol $f\in S_f$ (constants or variable) of $\mathcal{T}$.
		
		Caution: The fact that $f$ is of the good arity is only implicit in this notation. Moreover, writing $S_f(\mathcal{T},\ldots,\mathcal{T})$ does not mean that all function arguments are identical, but simply that these arguments are elements of $\mathcal{T}$.
		\begin{tcolorbox}[title=Remark,colframe=black,arc=10pt]
		It is often convenient to see a term (expression) as a tree, where each node is labeled with a function symbol (operator or function) and each sheet by a variable or constant.
		\end{tcolorbox}
	\end{enumerate}
	In what follows, we will almost always define concepts (or prove results) "by recurrence" on the structure or the size of a term.

	\textbf{Definitions (\#\mydef):}
	\begin{enumerate}
		\item[D1.] To prove a property $P$ on the terms, it suffices to prove $P$ for the variables and constants and to prove $P(f(t_1,\ldots,t_n))$ from $P(t_1),\ldots,P(t_n)$. We do then here a "\NewTerm{proof by induction on the height of a term}\index{proof by induction on the height of a term}". It is a technique that we will find in the following chapters.

		Mathematical induction as an inference rule can be formalized as a second-order axiom. The axiom of induction is, in logical symbols:
		
		In words, the basis $P(0)$ and the inductive step (namely, that the inductive hypothesis $P(k)$ implies $P(k + 1)$) together imply that $P(n)$ for any natural number $n$. The axiom of induction asserts that the validity of inferring that $P(n)$ holds for any natural number $n$ from the basis and the inductive step.
		
		Induction can be compared to falling dominoes: whenever one domino falls, the next one also falls. The first step, proving that $P(1)$ is true, starts the infinite chain reaction.
	\begin{center}
	\includegraphics[scale=0.3]{img/arithmetics/dominoes.jpg}
	\end{center}

		\item[D2.] To define a function $\Phi$ based on the terms, it is enough to define it on the variables and constants and tell how we get $\Phi(f(t_1,\ldots,t_n))$ from $\Phi(t_1),\ldots,\Phi(t_n)$. We do here again a "\NewTerm{definition by induction on the height of a term}\index{definition by induction on the height of a term}".
	\end{enumerate}
	\begin{tcolorbox}[colframe=black,colback=white,sharp corners]
	\textbf{{\Large \ding{45}}Example:}\\\\
	The size (we also say the "length") of a term $t$ (size denoted $\tau(t)$) is the number of function symbols occurring in $t$. formally:
	\begin{itemize}
		\item $\tau(x)=\tau(c)=0$ if $x$ is a variable and $c$ is a constant.

		\item  $\tau(f(t_1,\ldots,t_n))=1+\sum_{i\leq i\leq n}\tau(t_i)$
	\end{itemize}
	where the $1$ in the last relation represents the term $f$ itself.
	\end{tcolorbox}
	
	\paragraph{Formulas}\mbox{}\\\\
	\textbf{Definition (\#\mydef):} A "\NewTerm{well-formed formula WEF}\index{well-formed formula}", often simply "\NewTerm{formula}" is a word (i.e. a finite sequence of symbols from a given alphabet) that is part of a formal language. A formal language can be considered to be identical to the set containing all and only its formulas.

	The formulas of propositional calculus, also named "\NewTerm{propositional formulas}\index{propositional formulas}", are expressions such as $(A \land (B \lor C))$.

	An atomic formula is a formula that contains no logical connectives nor quantifiers, or equivalently a formula that has no strict subformulas. The precise form of atomic formulas depends on the formal system under consideration; for propositional logic, for example, the atomic formulas are the propositional variables. For predicate logic, the atoms are predicate symbols together with their arguments, each argument being a term.
	\begin{figure}[H]
		\centering
		\includegraphics{img/arithmetics/theorems.jpg}
	\end{figure}
	\pagebreak
	\textbf{Definition (\#\mydef):} Formulas are built from "\NewTerm{atomic formulas}\index{atomic formulas}" using connectors and quantifiers. We will use the following connectors and quantifiers (which we already known):
	\begin{itemize}
		\item Unary negation connector: $\neg$
		
		\item Binary connectors of conjunction and disjunction and implication: $\wedge, \vee, \rightarrow$
		
		\item Quantifiers: $\exists$ which must be read "it exists" and $\forall$ that must be read "for all"
	\end{itemize}
	This notation of the connectors is almost (it should at least). It is used to avoid confusion between the formulas and the current language (metalanguage).
	
	\textbf{Definitions (\#\mydef):}
	\begin{enumerate}
		\item[D1.] Given $\mathcal{L}$ a language, the "\NewTerm{atomic formulas}" of $\mathcal{L}$are the formulas of the form $R(t_1,\ldots,t_n$ where $R$ is an $n$-ary relation symbol of $\mathcal{L}$ of and $t_1,\ldots,t_n$ are terms of  $\mathcal{L}$. We denote by "Atom" all atomic formulas. If we denote by $S_r$ the set of relation symbols, we can write all terms related between them by the expression:
		
		The set $\mathcal{F}$ of formulas of the first order logic of $\mathcal{L}$ is thus defined by the grammar (where $x$ is a variable):
		
		where should be read: the set of formulas is the smallest set containing formulas and such that if $F_1$ and $F_2$ are formulas then $F_1\vee F_2$, etc. are formulas and can be related to each other. 
		
		The reader must be careful to not to confuse terms and formulas. $\sin(x)$ is a term (function), $x=3$ is a formula. But $\sin(x)\wedge (x=3)$ is nothing: we can not, in fact, put a connector between a term and a formula (meaningless).
		\begin{tcolorbox}[title=Remarks,colframe=black,arc=10pt]
		\textbf{R1.} To define a function $\Phi$ based on formulas, we simply need to define $\Phi$ on atomic formulas.\\

		\textbf{R2.} To prove a property $P$ on formulas, it suffices to prove $P$ for the atomic formulas.\\
		
		\textbf{R3.} To prove a property $P$ on the formulas, it is enough assume that the property holds for all formulas of size $p<n$ and to prove the property for formulas of size $n$.\\
		\end{tcolorbox}
		
		\item[D2.] A "\NewTerm{sub-formula}\index{sub-formula}" of a formula (or expression) $F$ is one of its components, verbatim a formula from which $F$ is built. Formally, we define the set $SF(F)$ of the sub-formulas $F$ by:
		\begin{itemize}
			\item If $F$ is atomic: 
			
			\item If $F=F_1\oplus F_2$ (that is to say a composition) with $\oplus\in\{\vee,\wedge,\rightarrow\}$:
			
			\item If $F=\neg F$ or $\text{Q}\in F_1$ with $\text{Q}\in \{\forall,\exists\}$:
			
		\end{itemize}

		\item[D3.] A formula $F$ of $\mathcal{L}$ uses only a finite number of symbols of $\mathcal{L}$. This subset is named the "\NewTerm{language of the formula}\index{language of the formula}" and is denoted by $\mathcal{L}(F)$.

		\item[D4.] The "\NewTerm{size (or length) of a formula $F$}\index{size or length of a formula}", denoted by $\tau(F)$ is the number of connectors or quantifiers occurring in $F$. Formally:
		\begin{itemize}
			\item $\tau(F)=0$ if $F$ is an atomic formula

			\item $\tau(F_1\oplus F_2)=1+\tau(F_1)+\tau(F_2)$ where once again  $\oplus\in\{\vee,\wedge,\rightarrow\}$

			\item $\tau(\neg F_1)=\tau(\text{Q}xF_1)=1\tau(F_1)$ with once again $\text{Q}\in \{\forall,\exists\}$
		\end{itemize}

		\item[D5.] The "\NewTerm{main operator}\index{main operator}" (we also say the "main connector") of a formula is defined as:
		\begin{itemize}
			\item If $A$ is atomic, so it has then it has no main operator
	
			\item If $A=\neg B$, the $\neg$ is the main operator of $A$
	
			\item If $A=B\oplus C$ where once again  $\oplus\in\{\vee,\wedge,\rightarrow\}$ then $\oplus$ is the main operator of $A$
	
			\item If $A=\text{Q}xB$ where once again $\text{Q}\in \{\forall,\exists\}$, then $\text{Q}$ is the main operator of $A$
		\end{itemize}

		\item[D6.] Given $F$ a formula. The set $VL(F)$ of free variables of $F$ and the set $VM(F)$ of dummies variables (or "linked variables") of $F$ are defined by induction on $\tau(F)$.
		
		An occurrence of a given variable is named "\NewTerm{linked variable}\index{linked variable}" or "\NewTerm{dummy variable}\index{dummy variable}" in a formula $F$ in a formula $F$, if in this formula a quantifier refers to it. Otherwise, we say we have a "\NewTerm{free variable}\index{free variable}".
		
		\begin{tcolorbox}[title=Remark,colframe=black,arc=10pt]
		An occurrence of a variable $x$ in a formula $F$ is a position of the variable in the formula $F$. Do not confuse with the object that is the variable itself!
		\end{tcolorbox}
		To clarify the possible free variables of a formula $F$, we write $F[x_1,\ldots,x_n]$. This means that free variables of $F$ are among the variables $x_1,\ldots,x_n$, verbatim if $y$ is free in $F$, then is one of the $x_i$ but the $x_i$ do not necessarily appear in $F$.
		
		We can define the dummy or free variables more formally:
		\begin{enumerate}
			\item If $F=R(t_1,\ldots,t_n)$ is atomic then $VL(F)$ is the set of free variables appearing in the $t_i$ and we have then for dummy variables $VM(F)=\varnothing$.
	
			\item If $F=F_1\oplus F_2$ where $\oplus \in\{\vee,\wedge,\rightarrow\}:VL(F)=VL(F_1)\cup VL(F_2)$ then $VM(F)=VM(F_1)\cup VM(F_2)$.
	
			\item If $F=\neg F_1$ then $VL(F)=VL(F_1)$ and $VM(F)=VM(F_1)$.
	
			\item If $F=\text{Q}xF_1$ with $\text{Q}\in\{\forall,\exists\}:VL(F)=VL(F_1)-\{x\}$ and $VM(F)=VM(F_1)\cup\{x\}$.
		\end{enumerate}
		\begin{tcolorbox}[colframe=black,colback=white,sharp corners]
		\textbf{{\Large \ding{45}}Examples:}\\\\
		E1. Given $F$: $\forall x\;(x\cdot y=y\cdot x)$ then $VL(F)=\{y\}$ and $VM(F)=\{x\}$\\
		
		E2. Given $G$: $\{\forall x\exists y(x\cdot z=z\cdot y)\}\wedge \{x=z\cdot z\}$ then $VL(G)=\{x,z\}$ and $VM(G)=\{x,y\}$.
		\end{tcolorbox}
		
		\item[D7.] We say that formulas $F$ and $G$ are "\NewTerm{$\alpha$-equivalent}\index{$\alpha$-equivalent formulas}" if they are (syntactically) identical only after the renaming of their related variables.

		\item[D8.] A "\NewTerm{closed formula}\index{closed formulas}" is a formula without free variables.

		\item[D9.] Given $F$ a formula, $x$ a variable and $t$ a term. $F[x:=t]$ is the formula obtained by replacing in $F$ all free occurrences of $x$ by $t$, after possible renaming of linked occurrences in $F$ which are free in $t$.

	\end{enumerate}
	\begin{tcolorbox}[title=Remarks,colframe=black,arc=10pt]
	\textbf{R1.} We will notice in the examples seen previously that a variable can have both free occurrences and linked occurrences. So we do not always have $VL(F)\cap VM(F)=\varnothing$.\\
	
	\textbf{R2.} We can not rename $y$ into $x$ in the formula $\forall y\;(x\cdot y=y\cdot x)$ of the previous example and get the formula $\forall x\;(x\cdot x=x\cdot x)$: the variable $x$ would be then "captured". We therefore can not rename variables without precautions: we must avoid capture of free occurrences.
	\end{tcolorbox}
	
	\subsection{Proofs}
	The proofs that we found in mathematical books or theoretical physics books are assemblies of mathematical symbols and sentences containing keywords such as: "So", "because", "if", "if and only if", "it is necessary", "just", "take an $x$ such that", "therefore", "assume", "seek a contradiction", etc. These words are assumed to be understood by all in the same way, which is in fact, not always the case.
	
	In any work, the purpose of a proof is to convince the reader of the truth of a statement by show him the intellectual path that gives him the possibility to control himself the truth and rigour of the statement. Depending on the level of the reader, this proof will be more or less detailed or formal: something that can be considered obvious in a graduate course may not be in a undergraduate level course.
	
	In a homework, the corrector know that the result given by the student is (normally) true and he knows the proof of it. The student must prove (correctly) the required result. The level of detail that the student must give will depends sometimes on the confidence possessed by the corrector: in a good copy, a "proof by an evident recurrence" will be accepted, while a copy where there previously had an "obvious" which was ... obviously false, will not pass!
	
	To properly manage the level of details, we should know what is a complete proof. This work of formalization has been done at the beginning of the 20th century only!!

	Several things may seem surprising:
	\begin{enumerate}
		\item There is only a finite number of rules: two for each of the connectors (and the equality) more three general rules. It was not at all clear before that a piori a finite number of rules we engough to prove all that is true. We will show this result (this is essentially the Completeness Theorem). The proof is not trivial at all.
	
		\item These are the same rules for all the mathematics and physics: algebra, analysis, geometry, etc. This means that we have managed to isolate what is general in reasoning!!! We will see later that a proof is a assembly of pairs $(\Gamma, A)$, where $\Gamma$ is a set of formulas (the assumptions) and $A$ a formula (the conclusion). When we do the arithmetic, geometry or real analysis, we use, in addition to rules, assumptions that are named as we know "axioms". These express the particular properties of objects that we manipulate (for details on the the concept of axioms Introduction section of the book).
	\end{enumerate}
	We prove therefore, in general, formulas using a set of assumptions, and this set can vary during the proof: when we say "suppose $F$ and let us prove $G$", $F$ is then a new hypothesis that we can use to prove $G$. to formalize this, we introduce the concept of "sequent":
	\begin{enumerate}
		\item[D1.] A "\NewTerm{sequent}\index{sequent}" is a pair $\Gamma\vdash F$ where:
		\begin{enumerate}
			\item $\Gamma$ is a finite set of formulas that represents the assumptions that we can use. This set is also named the "\NewTerm{context of the sequent}\index{context of the sequent}".
	
			\item $F$ is a formula. This is the formula we want to prove. We say that this formula is the "\NewTerm{conclusion of the sequent}\index{conclusion of a sequent}".
		\end{enumerate}
		The sign "$\vdash$" must be read "\NewTerm{thesis}\index{symbol: thesis}" or "\NewTerm{prove that}\index{symbol: prove that}".
		\begin{tcolorbox}[title=Remarks,colframe=black,arc=10pt]
		\textbf{R1.} If $\Gamma=\{A_1,\ldots,A_n$ we can then write $A_1,\ldots,A_n \vdash F$ instead of $\Gamma \vdash F$. \\
		
		\textbf{R2.} We write $\vdash F$ a sequent whose set of assumptions is empty and $\Gamma_1,\ldots,\Gamma_n \vdash F$ a sequent whose set of assumptions is $\bigcup_i \Gamma_i$.\\
		
		\textbf{R3.} We write $\Gamma \not\vdash F$ to say that "$F$ is not provable".
		\end{tcolorbox}
		
		\item[D2.] A sequent $\Gamma\vdash Fhree$ is said to be "\NewTerm{provable}\index{provable}" (or \NewTerm{demonstrable}) if it can be obtained by a finite application of rules. A formula $F$ is provable if the sequent $\vdash F$ is provable.
	\end{enumerate}

	\subsubsection{Rules of Proofs}
	Proofs rules are the bricks used to build demonstration steps. A formal demonstration is a finite (and correct!)  assembly of rules. This assembly is not linear (not a suite) but a "tree." Indeed, we are often forced to make connections.

	We will present a choice of rules. We could have introduced other (instead of or in addition) that would give the same notion of provability. Those that have been chosen are "natural" and correspond to the arguments that we usually made in mathematics. In the common practice we use, in addition to the rules below, many other rules, but these can be deduced from previous ones. We name them "\NewTerm{derived rules}\index{derived rules}".
	
	It is traditional to write the root of the tree (the sequent conclusion) at the bottom, the leaves at the top: the nature is done as this... As it is also tradition to write on a sheet of paper, from up to down, it would not be unreasonable to write the root at the top and the leaves at the bottom . We must make a choice!
	
	A rule consists of:
	\begin{itemize}
		\item a set of "premises" each is a sequent. There may be zero, one or more of them.

		\item the conclusion sequent of the rule

		\item a horizontal bar separating the premises (top) from the conclusion (bottom). On the right of the bar, we will indicate the name of the rule.
	\end{itemize}
	\begin{tcolorbox}[colframe=black,colback=white,sharp corners]
	\textbf{{\Large \ding{45}}Example:}\\\\
	
	This rule has two premises ($\Gamma\vdash A \rightarrow$ and $\Gamma\vdash A$) and a conclusion ($\Gamma B$) and is denoted in an abbreviated under the form $\rightarrow_e$. It can be read in two ways:
	\begin{itemize}
		\item from bottom to top: if we want to prove the conclusion, it suffices by using the rule to prove the premises. This is what we do when we seek a proof. This corresponds to the "analysis".

		\item from top to bottom: if we proved the premises, so we also proved the conclusion. This is what we do when we write a proof. This corresponds to the "synthesis".
	\end{itemize}
	\end{tcolorbox}
	
	\pagebreak
	For the proofs there is a finite number of 17 rules in number that we will define below:
	\begin{enumerate}
		\item Axiom:
		
		From bottom to top: if the conclusion of the sequent is one of the hypothesis, then the sequent is provable.
		
		\item Weakening:
		
		Explanations:
		\begin{itemize}
		\item From top to bottom: if we prove $A$ under the assumptions $\Gamma$, adding other hypotheses can still prove $A$.

		\item From bottom to top: there are assumptions that may not serve
		\end{itemize}

		\item Introduction of implication:
		
		From bottom to top: to prove that $A \rightarrow B$ we assume $A$ (that is to say, we add it to the assumptions) and we prove $B$.	
		
		\item Elimination of implication:
		
		From bottom to top: to prove $B$, if we know a theorem of the form $A\rightarrow B$ and if we can prove the lemma $A\rightarrow B$, it suffices to prove $A$.
		
		\item Introduction to the conjunction:
		
		From bottom to top: to prove $A\wedge B$, it suffices to prove $A$ and prove $B$.
		
		\item Elimination of the conjonction:
		
		From top to bottom: from $A\wedge B$, we can deduce $A$ (left elimination) and $B$ (right elimination).
		
		\item Introduction to the disjunction:
		
		From bottom to top: to prove $A\vee B$, it suffices to prove $A$ (left disjunction) or prove $B$ (right disjunction).
		
		\item Elimination of  the disjunction:
		
		From bottom to top: if we want to prove $C$ and we know we have $A\vdash B$, it is enough to prove in  first time by assuming $A$, and in a second time by assuming $B$. This is a case-based reasoning.
		
		\item Introduction of the negation:
		
		From bottom to top: to show $\neg A$, we assume $A$ and we prove the absurd ($\perp$).
		
		\item Eliminiation of the negation:
		
		From top to bottom: if we proved $\neg A$ and $A$, then we proved by absurdity ($\perp$).

		\item Classic absurdity (reductio ad absurdum):
		
		From bottom to top: to prove $A$, it suffices to prove the absurdity by assuming $\neg A$.
		
		This rule is equivalent to say: $A$ is true if and only if it is false that $A$ is false. This rule is not obvious: it is necessary to prove some results (there are results we can not prove if we do not have this rule). Contrary to many others, this rule may also be applied at any time. We can, in fact, always say: To prove $A$, I suppose $\neg A$ and I will seek for a contradiction.
		
		\item Introduction of the universal quantifier:
		
		From bottom to top: to prove $\forall x\; A$, it suffices to show $A$ doing no assumption about $x$.
		\begin{tcolorbox}[title=Remark,colframe=black,arc=10pt]
		For proofs this check (no assumption on $x$) is often a source of error.
		\end{tcolorbox}
		
		\item Elimination of the universal quantifier:
		
		From top to bottom: from $\forall x\; A$, we can deduce $A[x:0t]$ for any term $t$. What we can also say under the form: if we proved $A$ for all $x$, then we can use $A$ with any object $t$ (!!).
		
		\item Introduction of the existential quantifier:
		
		From bottom to top: to prove $\exists x\; A$, it suffices to found an object (verbatim a term $t$) for which we know how to prove $A[x:=t]$.
		
		\item Elimination of the existential quantifier:
		
		From bottom to top: we prove that there is indeed a set of assumptions such that $\exists x \; A$ and hence this result as new hypothesis, we prove $C$. This formula $C$ inherits then from the formula $\exists x \; A$  and therefore $x$ is not free in $C$ because it already was not in $\exists x \; A$.
		
		\item Introduction of equality:
		
		From bottom to top: we can always prove $t=t$. This rule means that equality is reflexive (\SeeChapter{see section Operators}).
		
		\item Elimination of equality:
		
		From top to bottom: if we prove that $\Gamma\vdash A[x:=t]$ and $t = u$, then we have prove $A[x:=u]$. This rule expresses that equal objects have the same properties. We notice however that the formulas (or relations) $t =u$ and  $u = t$ are not, formally, identical. We will have to prove that equality is symmetric (we will benefit also to prove on the way that equality is transitive).
	\end{enumerate}
	Let us see now three examples by introducing them in the form of theorems as it should be in proof theory!
	
	\pagebreak
	\begin{theorem}
	The equality is symmetric (a little bit not trivial but quite good to begin with the subject):
	\end{theorem}
	\begin{dem}
	
	From top to bottom: we introduce the equality $=_i$ and prove from the assumption $x_1=x_2$ the formula $x_1=x_1$. At the same time, we define the axiom as what $x_1=x_2$. Then from these premises, we eliminate the equality $=_e$ by substituting the terms so that from the assumption $x_1=x_2$ (from the axiom) we get $x_2=x_1$. Then, the elimination of equality automatically implies without assumption that $x_1=x_2\rightarrow x_2=x-1$. Therefore, we simply insert the universal quantifier for each variable (ie twice) without any assumption to achieve that equality is symmetric.
	\begin{flushright}
		$\square$  Q.E.D.
	\end{flushright}
	\end{dem}
	
	\begin{theorem}
	The equality is transitive (that is to say if $x_1=x_2$ and $x_2=x_3$ then $x_1=x_3$). By denoting $F$ the formula $(x_1=x_2)\wedge (x_2=x_3)$:
	\end{theorem}

	\begin{dem}
	
	What do we do here? We first introduce the formula $F$ twice as axiom to "dissect" it latter left and right (we do not introduce the equality supposed already introduced as a rule). Once done, we eliminate on the left and on the right the conjunction on the formula to work on left and right terms only and we introduce the equality of the two terms which makes that from the formula we have the transitive equality. It follows without any assumption that automatically implies that equality is transitive and finally we say that this is valid for any value of the different variables (if the formula is true, then equality is transitive).
	\begin{flushright}
		$\square$  Q.E.D.
	\end{flushright}
	\end{dem}
	And now last big example always in the form of a theorem:
	\begin{theorem}
	Any involution is a bijection (\SeeChapter{see section Set Theory}).
	\end{theorem}

	\begin{dem}
	Let $f$ be a unary function symbol (with one variable), we write (for the details see the section of Set Theory):
	\begin{itemize}
		\item $\text{Inj}[f]$ the formula:
		
		which means that $f$ is injective.
		
		\item $\text{Surj}[f]$ the formula:
		
		
		\item  $\text{Bij}[f]$ the formula:
		
		
		\item $\text{Inv}[f]$ the formula:
		
	\end{itemize}
	which means that$ $f is an involution (we also writh this $f\circ f=\text{Id}$ that this is to say that the composition of $f$ is the identity).
	
	We would like to know if:\\
	
	
	We will present (trying this to be done as easy as possible) this proof in four (!!!) different ways: traditional (informal), classic (pseudo-formal), formal in tree and formal in-line.
	
	\begin{itemize}
		\item \textbf{Traditional method (informal):}
		
		We must prove that if $f$ is involutive then it is bijective. So we have two things to prove (and both must be satisfied simultaneously): the function is injective and surjective.
		
		\begin{enumerate}
			\item So we prove first that involution is injective. 
			
			We assume for this, because $f$ is an involution it therefore injective, such that:
			
			implies that:
			
			However, this assumption automatically comes from the definition of involution that:
			
			and to the application of $f$ to the relation:
			
			(thus three equalities so far) such that:
			
			we therefore have:
			
			
			\item Let us prove that involution is surjective: if it is surjective, then we must have:
			
			But, let us define the variable $x$ by definition of the involution itself:
			
			as $y=f(x)$..., and a change of variable after we get:
			
			and therefore surjectivity is ensured.
			\end{enumerate}
			
			\item \textbf{Pseudo-formal method:}
			
			We take again the same and we inject in it the rules of proof theory:
			
			We must show that $f$ is involutive and therefore bijective. So we have two things to show ($\wedge_i$) (and both must be satisfied simultaneously): that the function injective and surjective:
			
			\begin{enumerate}
			\item Let us first prove that involution is injective. We assume for this, since $f$ is involutive and therefore injective, that:
			
			implies:
			
			However, this assumption automatically comes from the definition of involution that:
			
			and from the application of $f$ to the relation:
			
			(therefore three equalities $=_e \times 2$) such that:
			
			We therefore have:
			
			
			\item Let us prove that involution is surjective. If it is surjective, then we must have:
			
			Now, we define the variable $x$ by definition of the involution itself:
			
			since $y=f(x)$...., after a change of variables we get:
			
			and therefore:
			
			surjectivity is assured.
		\end{enumerate}
		\item \textbf{Formal tree method:}
			
		Let us do this now with the graphical method that we have presented above.
		\begin{enumerate}
			\item Let us prove that involution is injective:
			
			For this we must prove first that:
			
			Therefore:
			
			
			That bring us to write:
			
			\begin{tcolorbox}[title=Remark,colframe=black,arc=10pt]
			The latter relation is abbreviated $=_c$ and named (as other existing) "derived rule" because it is an argument that is often made during proofs and a little time consuming to develop each time ...
			\end{tcolorbox}
			Therefore:
			
			
			\item Let us prove now that involution is surjective:
			
			It follows:
			
		\end{enumerate}
		
		\item \textbf{Formal in-line method:}
		
		We can do the same thing in a slightly less... wide form ... and more ... tabbed (it is no less indigestible):
		\begin{subequations}
		\begin{align}
			&\text{Inv}[f]\vdash \text{Bij}[f]&& \vee i\\
			&\quad 1:=\text{Inv}[f]\vdash \text{Inj}[f]&&\nonumber\\
			&\quad 2:=\text{Inv}[f]\vdash \text{Surj}[f]&&\nonumber\\[5pt]
			&(1)\text{Inv}[f]\vdash \text{Inj}[f]&& \forall i\\
			&\quad\text{Inv}[f]\vdash f(x)=f(y)\rightarrow x=y &&\rightarrow_i\nonumber\\
			&\quad\text{Inv}[f], f(x)=f(y)\vdash x=y&&=_e\times 2 (i)(ii)(iii)\nonumber\\
			&\quad(i)\quad\text{Inv}[f]\vdash f(f(x))=x &&\forall_e\nonumber\\
			&\quad\quad\quad\text{Inv}[f]\vdash \text{Inv}[f] &&ax\nonumber\\
			&\quad(ii)\quad\text{Inv}[f]\vdash f(f(y))=y &&\forall_e\nonumber\\
			&\quad\quad\quad\text{Inv}[f]\vdash \text{Inv}[f] &&ax\nonumber\\
			&\quad(iii)\quad f(x)=f(y)\vdash f(f(x))=f(f(y))&&=_c\nonumber\\
			&\quad\quad(1')f(x)=f(y)\vdash f(x)=f(y) &&ax\nonumber\\[5pt]
			&(2)\text{Inv}[f]\vdash \text{Surj}[f]&& \forall i\\
			&\quad\text{Inv}[f]\vdash \exists x \{f(x)=y\}&&\exists_i\nonumber\\
			&\quad\text{Inv}[f]\vdash \{f(x)=y\}[x:=f(y)]&&\forall_e\nonumber\\
			&\quad\text{Inv}[f]\vdash \forall x \{f(f(x))=x\} &&ax\nonumber
			\end{align}
		\end{subequations}
	\end{itemize}

	\begin{flushright}
		$\square$  Q.E.D.
	\end{flushright}
	\end{dem}
	Caution! However, all this highly technical formalism don't make it always obvious to found where is the error in the following "pseudo-proof":
	Assume:
	
	Let us multiply both side by $a$:
	
	Let us subtract by $b^2$:
	
	We factorize:
	
	We divide by $(a-b)$:
	
	Since $a=b$, we have:
	
	Therefore:
	
	If you don't see the error, here is the analysis:
	
	
	\begin{flushright}
	\begin{tabular}{l c}
	\circled{80} & \pbox{20cm}{\score{4}{5} \\ {\tiny 64 votes, 80.94\%}} 
	\end{tabular} 
	\end{flushright}
	\begin{flushright}
	{\setlength{\parskip}{0pt}{\tiny Version: 3.1 Revision 5 | Last update: 2015-09-06 16:33}}
	\end{flushright}
	
	%to make section start on odd page
	\newpage
	\thispagestyle{empty}
	\mbox{}
	\section{Numbers}

\lettrine[lines=4]{\color{BrickRed}T}he basis of mathematics, apart the reasoning (see section Proof Theory), is undoubtedly to ordinary people: arithmetic. It is therefore mandatory that we make a stop on it to study its origin, some of its properties and consequences.\\\\

The numbers, like geometric figures are the basis of Geometry, are the basis of Arithmetic. These are also the historical basis because mathematics probably started with the study of these objects, but also the educational foundation, because it is by learning to count that we enter in the world of mathematics.

The history of numbers, also sometimes named "\NewTerm{scalar}"\index{scalar} is far too long to be told here, but we can only advise you one of the best book on the subject: The Universal History of Numbers (~2,000 pages in three volumes) Georges Ifrah, ISBN: 2221057791.

But here's a little flange of this latter which seems fundamental to us:

Our current decimal system, on base 10, uses the digits $0$ to $9$, named "Arabic numbers", but the fact of Indian origin (Hindus). The first numbers seems to have been created in the third century BC in India by Brahmagupta, an Indian mathematician. he created the figures in Devanagari.

Indeed, the Arabic numbers (of Indian origin...) in the table below are the first line and we see that they are significantly different from the "Indian numbers" of the second line:

\begin{figure}[H]
\centering
\includegraphics{img/arithmetics/numbers.eps}
\caption{Indo-Arabic numbers}
\end{figure}

You have to read this table as following from left to right: 0 "\NewTerm{zero}", 1 "\NewTerm{one}", 2 "\NewTerm{two}", 3 "\NewTerm{three}", 4 "\NewTerm{four}", 5 "\NewTerm{five}", 6 "\NewTerm{six}", 7 "\NewTerm{seven}", 8 "\NewTerm{eight}, 9 "\NewTerm{nine}". This system is much more efficient than the Roman numerals (try doing a calculation with Roman notation system you will see...).

It is commonly accepted that these numbers were introduced in Europe only about the year 1,000. Used in India, they were transmitted by Arabs to the Western world by the Pope Gerbert of Aurillac during his stay in Andalusia at the end of the 9th century.

	\begin{tcolorbox}[title=Remark,colframe=black,arc=10pt]
The French word "chiffre" (number) is a corruption of the Arabic word "sifr" meaning "zero". In Italian, "zero" is "zero", and seems to be a contraction of "zefiro", we again see here an Arabic root but the "zero" could also be of Indian origin... So the words "chiffre" and "zero" have the same origin.
	\end{tcolorbox}

The early use of a numerical symbol for the "nothing" in the sense of "no amount", i.e. our "\NewTerm{zero}"\index{zero} is because the Indians used a system named "\NewTerm{positional system}"\index{positional system}. In such a system, the position of a digit in the writing of a number expresses the power of 10 and the number of times it occurs ... and the absence of a position in this system arise from huge proofreading  problems and could lead to large errors in calculations. The revolutionary and simple introduction of the concept of "nothing" allowed a proofreading without error of numbers.

The absence of a power is denoted by a small circle...: the zero. Our current system is thus the "\NewTerm{decimal and positional system}"\index{decimal and positional system}.

	\begin{tcolorbox}[colframe=black,colback=white,sharp corners]
\textbf{{\Large \ding{45}}Example:}\\\\
\begin{figure}[H]
\centering
\includegraphics{img/arithmetics/decimal_system.eps}
\caption{Description of decimal and positional system}
\end{figure}

The number $324$ is written from left to right as three hundred: $3$ times $100$, two tens: $2$ times $10$ and four units: $4$ times $1$.
	\end{tcolorbox}

Thus a "\NewTerm{decimal number}\index{decimal number}" is thus a number that has a finite writing in base $10$.

	We sometimes see (and this is recommended) a thousands separator represented by a coma in United States (put all three numbers from the first from the right for the whole numbers). Thus, we write $1,034$ instead of $1034$ or $1,344,567,569$ instead of $1344567569$. Thousand separators permits to quickly quantify the magnitude of the read numbers.

	So: 
	\begin{itemize}
		\item If we see only one coma we know that the number is about thousands
		\item If we see two apostrophes we know that the number is about millions
		\item If we see three apostrophes we know that the number is about billions
		\item etc.
	\end{itemize}
and so on... also with decimals this gives:
\begin{figure}[H]
\centering
\includegraphics{img/arithmetics/numbers_scale.eps}
\caption{Scale representation of the positional system}
\end{figure}

	In fact, any integer other than the unit can be taken as the basis of a numbering system. We have for example the binary, ternary, quaternary, ..., decimal, duodecimal numbering systems which correspond respectively to the bases two, three, four, ..., ten, twelve.

A generalization of what has been seen above, can be written as follows:

Any positive integer $N$ can be represented in a base $b$ as a sum, where each coefficient $a_i$ are multiplied by their respective weight $b^i$. Such as:
	
More elegantly written:
	
with $a_i \in \left[0,b-1\right]$ and $b_i \in \left[1,b^{n-1}\right]$

	\begin{tcolorbox}[title=Remarks,colframe=black,arc=10pt]
	\textbf{R1.} As frequently in mathematics, we will replace numbers with Latin or Greek letters in order to generalize their representation. So when we speak of a base b, the value of $b$ can take any positive integer value 1, 2, 3,...\\\\
	\textbf{R2.} When we take the value $2$ for $b$, the maximum value of $N$ will be $2^n-1$. The numbers that are written in this form are named "\NewTerm{Mersenne numbers}\index{Mersenne number}". These numbers can be prime numbers (see further below what a prime number is) if and only if $n$ is also a prime number.\\\\
	Indeed, if we take (for example) $b=10$ and $n=3$ the largest value we can get will be:
	
	\textbf{R3.} When a number is the same read from left to right or right to left, we name it a "\NewTerm{palindrome}\index{palindrome}".
	\end{tcolorbox}

\subsection{Digital Bases}

To write a number in base $b$ system, we must first adopt $b$ characters for representing the $b$ first numbers for example in the decimal system: $\left\lbrace 0, 1, 2, 3, 4, 5, 6, 7, 8, 9\right\rbrace $. These characters are as we already defined them, the "\NewTerm{digits}\index{digits}" that we pronounce as usual $\left\lbrace \text{zero}, \text{one}, \text{two}, \text{three}, \text{four}, \text{five}, \text{six}, \text{seven}, \text{eigth}, \text{nine}\right\rbrace $.

For the written numbers, we make this convention that a digit, placed to the left of another represents the order units immediately above, or $b$ times larger. To take the place of units that may be lacking in certain orders, we use the zero "$0$" and consequently, the number of digits may vary.

\textbf{Definition (\#\mydef):} For the spoken numbers, we agree to name "single unit", "ten", "hundred", "thousand", etc., units of the first, second, third, fourth order, etc. Thus the numbers $10, 11, ..., 19$ will be readen in the same way in all numbering systems. The numbers $1a, 1b, a0, b0, ...$ will be readen ten-$a$, ten-$b$, $a$-ten, $b$-ten, etc. Thus, the number 5b6a71c will be readen:
\begin{center}
five million be-hundred sixty-$a$ thousand seven hundred ten-$c$
\end{center}
This small example is relevant because it shows the general expression of the spoken language we use daily is intuitively in base ten (fault of our education).

	\begin{tcolorbox}[title=Remarks,colframe=black,arc=10pt]
\textbf{R1.} The rules of mathematical operations defined for numbers written in the decimal system are the same for numbers written in any numbering system.\\

\textbf{R2.} To quickly operate in any numbering system, it is useful to know by heart all sums and products of two numbers of a single digit.\\

\textbf{R3.} The decimal seems has its origin in the fact that humans being have ten fingers.
	\end{tcolorbox}
Let's see how we convert a numbering system in another one:

	\begin{tcolorbox}[colframe=black,colback=white,sharp corners]
\textbf{{\Large \ding{45}}Examples:}\\\\
E1. In base ten we have seen above that $142,713$ will be written as:
	
E2. The number $0110$ that is in base two (binary base) would be written in base $10$:
	
	\end{tcolorbox}
and so on...
The reverse operation is often a little trickier (for example the case of the binary base):
	\begin{tcolorbox}[colframe=black,colback=white,sharp corners]
\textbf{{\Large \ding{45}}Examples:}\\\\
E1. Converting the decimal number $1,492$ in binary base is done by successive divisions by $2$ of residues and gives (the principle is about the same for all other bases):
\begin{figure}[H]
\centering
\includegraphics[scale=0.75]{img/arithmetics/decimal_to_binary.eps}
\caption{Decimal to binary conversion}
\end{figure}
E2. To convert the number $142,713$ (decimal base) in duodecimal base (base twelve) we have (notation: $q$ is the "quotient", and $r$ is the "residue"):
	
Thus we have the residues 6, 10, 7, 0, 9 which leads us to write:
	
where we have chosen for this particular example the symbolism that we have previously defined ($a$-ten) to avoid any confusion.
	\end{tcolorbox}

	\pagebreak
	\subsection{Type of Numbers}
	
Now that we know that number is a mathematical object used to count, measure and label it must be know that it exists in mathematics a wide variety of numbers (natural, rational, real, irrational, complex, p-adic quaternions, transfinite, algebraic, constructibles, etc.) since any mathematician may at leisure create its own numbers just by defining axioms (rules) for the manipulating them (\SeeChapter{see section Set Theory}).

However, there are a few of them that we find much more often than others through this book and some that serve as basic construction for others and which that should be defined sufficiently rigorously (without going to the extremes) in order to know what we will talk about when we will use them.

	\subsubsection{Natural Integer Numbers}

The idea of "\NewTerm{integer}\index{integer}" (the numbers for which there are no decimals) is the fundamental concept of mathematics and comes at the view of a group of objects of the same types (a sheep, another sheep, yet another sheep, etc.).

When the amount of objects in a group is different from that of another group when the speak about a group that is numerically higher or lower regardless of the type of objects in these groups. When the amount of objects of one or multiples groups is equivalent, then we speak about "\NewTerm{equality}\index{equality}". 

To each single object the number "\NewTerm{one}" or "{unit}\index{unit}" denoted by "$1$" in the decimal system will be used.

To form groups of objects, we can operate as follows: to an object, add another object, then another, and so on... each of of the clusters, from the point of view of its community, is characterized by a number. It follows from this that a number can be regarded as representing a group of units (single items) such that each unit corresponds to one single object of the collection.

\textbf{Definition (\#\mydef):} Two numbers are said to be "\NewTerm{equal}\index{equal}" if each of the units of one we can match a unique unit of the other and vice versa (in a bijective way as seen in the section of Set Theory). If this does not hold true when we talk about "\NewTerm{inequality}\index{inequality}".

Let us take an object, then another, then to the formed group add again an object and so on. The groups thus formed are characterized by numbers which, taken in the same order as the groups successively obtained, are the "\NewTerm{natural sequence $\mathbb{N}$}\index{natural sequence}", also sometimes named "\NewTerm{whole numbers}\index{integers}", and denoted by:
	
To be unambiguous about whether $0$ is included or not, sometimes an index (or superscript) is added in the former case:
	
	\begin{tcolorbox}[title=Remark,colframe=black,arc=10pt]
The presence of the $0$ (zero) in our definition of $\mathbb{N}$ is debatable since it is neither positive nor negative. That is why in some books you will find a definition of $\mathbb{N}$ without the $0$.
	\end{tcolorbox}

The components of this natural set can be defined by (we own this definition to the mathematician Frege Gottlob) the following the properties (having read first the section on Set Theory is strongly recommended...):
	\begin{enumerate}
		\item[P1.] $0$ (read "zero") is the number of elements (defined as an equivalence relation) of all sets equivalent to (in bijection with) the empty set.
		\item[P2.] $1$ (read "one") is the number of elements of all sets equivalent to the set whose only element is 1.
		\item[P3.] $2$ (read "two") is the number of elements of all sets equivalent to the set whose only element are $1$ and $2$.
		\item[P4.] In general, an integer is the number of elements of all sets equivalent to the set of integers preceding it!
	\end{enumerate}
The construction of the set of natural numbers is made of the most natural and consistent manner. Natural numbers get their name from what they were, in the beginnings of their existence, to count quantities and things of nature or intervened in human life. The originality of this set lies in the empirical way he has  been built since it does not actually the result of a mathematical definition, but more by awareness of the human by the concept of countable quantity, of number and operations that reflect the relations between them.

The question about the origin of $\mathbb{N}$ is therefore the question of the origin of mathematics. And since thousands of years debates confronting the thoughts of the greatest philosophical minds have attempted to elucidate this deep mystery as to whether mathematics is a pure creation of the human mind or whether the man has only rediscovered a science that already existed in nature. Besides the many philosophical questions that the set of Natural numbers can generate, it is nonetheless interesting from an exclusively mathematical point of view. Because of its structure, it has remarkable properties that can be very useful when we practice some given reasoning or calculations.

The sequence of natural numbers is unlimited (\SeeChapter{see section Theory Of Numbers}) but countable (we will this property in details below), because in a group of objects that is represented by a number $n$, it will be enough to add an object to get another group that will be defined by the integer $n + 1$.

\textbf{Definition (\#\mydef):} Two integers that differs from a single positive unit are said to be "\NewTerm{consecutive}\index{consecutive numbers}".

	\pagebreak
	\paragraph{Peano axioms}\mbox{}\\\\
During the crisis of foundations of mathematics, mathematicians have obviously sought to axiomatize the set $\mathbb{N}$ and we own the actual axiomatisation to Peano and Dedekind.

The axioms of this system include the symbols $<$ and $=$ to represent the relations "smaller than" and "equal to" (\SeeChapter{see section Operators}). They include also the symbols "$0$" for the number zero and $s$ to represent the "successor" number. In this system, $1$ is denoted by:
	
named "successor to zero" and $2$ is denoted by:
	
The Peano axioms that builds $\mathbb{N}$ are (see section of Proof Theory for details on some of the symbols use below):
	\begin{enumerate}
		\item[A1.] $0$ is a natural number (this permits $\mathbb{0}$ to be not empty).
		\item[A2.] Every natural number $n$ has a successor, denoted by $s(n)$.
		Therefore $s$ is an injective application (\SeeChapter{see section Set Theory}), that is to say:
		
		That is to say that if two successors are equal, they are the successors of the same number.
		\item[A3.] The successor of a natural number is never zero (therefore $\mathbb{N}$ has a first element):
		
		\item[A4.] If we prove a property $\varphi$ that is true for $x$ and its successor $s(x)$, then this property is true for any $x$ (\NewTerm{axiom of recurrence}\index{axiom of recurrence}"):
		
		So the set of all the numbers satisfying the four above axioms is denoted by:
	\end{enumerate}
	So the set of all the numbers satisfying the four above axioms is denoted by:
	
	\begin{tcolorbox}[title=Remark,colframe=black,arc=10pt]
	The Peano axioms allow to build very rigorously the two basic operations of arithmetic in $\mathbb{N}$ that are addition and multiplication (see section on Operators) and so all the other sets that we will see later (subtraction in $\mathbb{N}$ can not be applied because it can give negative numbers).
	\end{tcolorbox}
	
	\pagebreak
	\paragraph{Odd, Even and Perfect Numbers}\mbox{}\\\\
	In arithmetic, study the parity of an integer, its determiner if this integer is or not a multiple of $2$. An integer multiple of $2$ is an even integer, the others are odd integers.
	
	\textbf{Definitions (\#\mydef):}
	\begin{enumerate}
		\item[D1.] The numbers obtained by counting by step of $2$ from zero (i.e.. 0, 2, 4, 6, 8, ...) in the set of natural integer numbers $\mathbb{N}$ are named "\NewTerm{even numbers}\index{even numbers}".
		
		The $n^{\text{th}}$ even number is obviously given by the relation:
		
		\item[D2.] The numbers we get by counting by step of $2$ starting from $1$ (i.e.. 1, 3, 5, 7, ...) in the set of natural integer numbers $\mathbb{N}$ are named "\NewTerm{odd numbers}\index{odd numbers}".
		
		The $(n+1)^{\text{th}}$ even number is almost obviously given by the relation:
		
	\end{enumerate}
	
	\begin{tcolorbox}[title=Remarks,colframe=black,arc=10pt]
	We name "\NewTerm{perfect numbers}\index{perfect numbers}", numbers equal to the sum of their integer divisors strictly smaller than themselves (concept we will see in detail later) such as: $6 = 1 + 2 + 3$ and $28 + 1 = 2 + 4 + 7 + 14$.
	\end{tcolorbox}
	
	\paragraph{Prime Numbers}\mbox{}\\\\
	\textbf{Definition (\#\mydef):} A "\NewTerm{prime number}\index{prime number}" is an integer with exactly two positive divisors (these divisors are both: "1" and the number itself). In the case where there are more than two dividers it is named a "\NewTerm{composite number}\index{composite number}". The property of being prime (or not) is named "\NewTerm{primality}\index{primality}".
	
	The study of prime numbers is a huge subject in mathematics (see for a small example the section of Number Theory or of Cryptography). There are books of thousands of pages on the subject and probable hundreds of research article per month even nowadays. Most theorems are largely out of the study of the site book (and out of the interest of its main author...)!
	
	Here is the set of prime numbers less than $1000$:
	
	2, 3, 5, 7, 11, 13, 17, 19, 23, 29, 31, 37, 41, 43, 47, 53, 59, 61,
	 67, 71, 73, 79, 83, 89, 97, 101, 103, 107, 109, 113, 127, 131, 137, 
	 139, 149, 151, 157, 163, 167, 173, 179, 181, 191, 193, 197, 199, 211, 
	 223, 227, 229, 233, 239, 241, 251, 257, 263, 269, 271, 277, 281, 283, 
	 293, 307, 311, 313, 317, 331, 337, 347, 349, 353, 359, 367, 373, 379, 
	 383, 389, 397, 401, 409, 419, 421, 431, 433, 439, 443, 449, 457, 461, 
	 463, 467, 479, 487, 491, 499, 503, 509, 521, 523, 541, 547, 557, 563, 
	 569, 571, 577, 587, 593, 599, 601, 607, 613, 617, 619, 631, 641, 643, 
	 647, 653, 659, 661, 673, 677, 683, 691, 701, 709, 719, 727, 733, 739, 
	 743, 751, 757, 761, 769, 773, 787, 797, 809, 811, 821, 823, 827, 829, 
	 839, 853, 857, 859, 863, 877, 881, 883, 887, 907, 911, 919, 929, 937, 
	 941, 947, 953, 967, 971, 977, 983, 991, 997
	
	The whole set of prime numbers is sometimes denoted by $\mathbb{P}$.
	
	\begin{tcolorbox}[title=Remark,colframe=black,arc=10pt]
	Note that the primes numbers set does not include the number "1" because it has a only a single divider (himself) and not two as is the definition.
	\end{tcolorbox}
	
	We can ask ourselves if there are infinitely many prime numbers? The answer is YES and here is a proof (among others) by contradiction.
	
	\begin{dem}
	Suppose that there is a finite number of prime numbers that would be denoted by:
	
	We create a new number from the product of this prime number to which we add "1":
	
	According to our initial hypothesis and the fundamental theorem of arithmetic (\SeeChapter{see section Number Theory}) the new number $N$ should be divisible by one of the existing prime $p_i$ such that we can write:
	
	where $q$ is an integer. We can make the division:
	
	The first term is simplified as $p_i$ is in the product. Let us note the resulting integer $E$:
	
	But, $q$ and $E$ are integers, so $1/p_i$ should be an integer. But $p_i$ is by definition greater than $1$. So $1/p_i$ is not an integer and so is also $q$.
	
	Then there is contradiction, and we can conclude that the prime numbers are not finite but are infinite.
	\begin{flushright}
		$\square$  Q.E.D.
	\end{flushright}
	
	\end{dem}
	\begin{tcolorbox}[title=Remarks,colframe=black,arc=10pt]
	\textbf{R1.} The product $p_n=p_1p_2...p_n$ of the indexed prime numbers $\leq n$ is named the "\NewTerm{$n$-th primorial}\index{$n$-th primorial}".\\
	
	\textbf{R2.} We send the reader to the section Cryptography of the chapter on Theoretical Computing (or Number Theory section of the chapter Arithmetic) for the study of some remarkable properties of prime numbers including the famous Euler $\phi$ function (also named "indicator function") and a 20th-21th century industrial application of prime numbers.
	\end{tcolorbox}
	
	\subsubsection{Relative Integer Numbers}
	The set of natural integers $\mathbb{N}$ has a few issues that we did not set out earlier. For example, subtracting two numbers into $\mathbb{N}$ does not always have a result in $\mathbb{N}$ (negative numbers not existing in this set). Other issue... dividing two numbers in $\mathbb{N}$ also does not always have a result in $\mathbb{N}$ (fractional numbers - rational or irrational -  not existing in this set). We then say in the language of set theory that: the substraction and division is not an internal operation of $\mathbb{N}$.
	
	We can first resolve the problem of subtraction by adding to the set of natural numbers $\mathbb{N}$, negative integers (revolutionary concept for those who where behind this concept at their time!) to get the set of "\NewTerm{relative integers}\index{relative integers}" denoted by $\mathbb{Z}$ (for "Zahl" from German, meaning "Number"):
	
	
	The set of natural integers is therefore included in the set of relative integers. This is what we denote by (\SeeChapter{see section Set Theory}):
	
	and we have by definition (it is a notations to be learned!!!):
	
	This set was originally created to make the natural numbers an object that we name a "\NewTerm{group}" (\SeeChapter{see section Set Theory}) relatively to the addition.
	
	\textbf{Definition (\#\mydef):} We say that a set $A$ is a "\NewTerm{countable set}\index{countable set}", if it is equipotent to $\mathbb{N}$. That is to say if there is a bijection (\SeeChapter{see section Set Theory}) of $S$ on $\mathbb{N}$. Thus, roughly said, two equipotent sets have the same number of elements in the meaning of their cardinal (\SeeChapter{see section Set Theory}), or at least the same infinity.
	
	The purpose of this concept is to understand that the sets $\mathbb{N}$ and $\mathbb{Z}$ are countable.
	
	\begin{dem}
		Let us show that $\mathbb{Z}$ is countable by writing:
		
		for any integer $k\geq 0$. This gives the following ordered list:
		
		of all relative integers from natural integers only!
		\begin{flushright}
			$\square$  Q.E.D.
		\end{flushright}
	\end{dem}
	
	\pagebreak
	\subsubsection{Rational Numbers}
	The set of relative integers $\mathbb{Z}$ also still has an issue. Dividing two numbers in $\mathbb{Z}$ also does not always have a result in $\mathbb{Z}$ (fraction numbers - rationnal or irrational - not existing in this set). We then say in the language of set theory that: the division is not an internal operation of $\mathbb{Z}$.
	
	We can thus define a new set that contains all the numbers which can be written as a "fraction" that is to say the ratio of a dividend (numerator) and a divider (denominator). When a number can be written in this form, we say that it is a "\NewTerm{fractional number}\index{fractional number}":
	
	A fraction can be used to express a part or fraction of something (of an object, of a distance, of a land, of an amount of money, of a cake...).
	
	To better understand rational number (fractions) let us consider two individuals: Andy and Bobby that bot love pizza. On Monday night, they share a pizza equally. How much of the pizza does each one get? Are you thinking that each boy gets half of the pizza? That's right. There is one whole pizza, evenly divided into two parts, so each boy gets one of the two equal parts. In math, we write  $\dfrac{1}{2}$ to mean one out of two parts:
	\begin{figure}[H]
		\centering
		\includegraphics[scale=0.75]{img/arithmetics/pizza_fraction_example_01.jpg}
		\caption{A $1/2$ fraction example (source: OpenStax)}
	\end{figure}
	On Tuesday, Andy and Bobby share a pizza with their parents, Fred and Christy, with each person getting an equal amount of the whole pizza. How much of the pizza does each person get? There is one whole pizza, divided evenly into four equal parts. Each person has one of the four equal parts, so each has $\dfrac{1}{4}$ of the pizza:
	\begin{figure}[H]
		\centering
		\includegraphics[scale=0.75]{img/arithmetics/pizza_fraction_example_02.jpg}
		\caption{A $1/4$ fraction example (source: OpenStax)}
	\end{figure}
	On Wednesday, the family invites some friends over for a pizza dinner. There are a total of $12$ people. If they share the pizza equally, each person would get $\dfrac{1}{12}$ of the pizza:
	\begin{figure}[H]
		\centering
		\includegraphics[scale=0.75]{img/arithmetics/pizza_fraction_example_03.jpg}
		\caption{A $1/12$ fraction example (source: OpenStax)}
	\end{figure}
	
	By definition, the "\NewTerm{set of rational numbers}\index{rational numbers}" is given by:
	
	
	In other words, any rational number is any number that can be expressed as the quotient or fraction $p/q$ of two integers, $p$ and $q$, with the denominator $q$ not equal to zero. Since $q$ may be equal to $1$, every integer is a rational number.
	
	We also assume as obvious that:
	
	
	The logic of the creation of the set of rational numbers $\mathbb{Q}$ is similar to that of relative integers $\mathbb{Z}$. Indeed, mathematicians wanted to make of the set relative numbers $\mathbb{Z}$ a "group" with respect to the law of multiplication and division (\SeeChapter{see section Set Theory}).
	
	Moreover, contrary to the intuition of most people, the set of natural integers $\mathbb{N}$ and rational numbers $\mathbb{Q}$ are equipotent. We can convince ourselves of this equipotence by ranking ,as Cantor did, rational numbers in a first time as follows:
	
	\begin{figure}[H]
	\centering
	\begin{tikzpicture}
\matrix(m)[matrix of math nodes,column sep=0.5cm,row sep=0.5cm]{
    1/1 & 2/1 & 3/1 & 4/1 & 5/1 & 6/1 & 7/1 & \cdots \\
    1/2 & 2/2 & 3/2 & 4/2 & 5/2 & 6/2 & 7/2 & \cdots \\
    1/3 & 2/3 & 3/3 & 4/3 & 5/3 & 6/3 & 7/3 & \cdots \\
    1/4 & 2/4 & 3/4 & 4/4 & 5/4 & 6/4 & 7/4 & \cdots \\
    1/5 & 2/5 & 3/5 & 4/5 & 5/5 & 6/5 & 7/5 & \cdots \\
    1/6 & 2/6 & 3/6 & 4/6 & 5/6 & 6/6 & 7/6 & \cdots \\
    1/7 & 2/7 & 3/7 & 4/7 & 5/7 & 6/7 & 7/7 & \cdots \\
    \vdots & \vdots & \vdots & \vdots & \vdots & \vdots & \vdots & \cdots \\
};

\draw[->]
         (m-1-1)edge(m-1-2)
         (m-1-2)edge(m-2-1)
         (m-2-1)edge(m-3-1)
         (m-3-1)edge(m-2-2)
         (m-2-2)edge(m-1-3)
         (m-1-3)edge(m-1-4)
         (m-1-4)edge(m-2-3)
         (m-2-3)edge(m-3-2)
         (m-3-2)edge(m-4-1)
         (m-1-5)edge(m-1-6)
         (m-1-7)edge(m-1-8)
         
         (m-2-4)edge(m-1-5)
         (m-3-3)edge(m-2-4)
         (m-4-2)edge(m-3-3)
         (m-5-1)edge(m-4-2)
         
         (m-5-2)edge(m-6-1)
         (m-4-3)edge(m-5-2)
         (m-3-4)edge(m-4-3)
         (m-2-5)edge(m-3-4)
         (m-1-6)edge(m-2-5)
         
         (m-2-6)edge(m-1-7)
         (m-3-5)edge(m-2-6)
         (m-4-4)edge(m-3-5)
         (m-5-3)edge(m-4-4)
         (m-6-2)edge(m-5-3)
         (m-7-1)edge(m-6-2)
         
         (m-1-8)edge(m-2-7)
         (m-2-7)edge(m-3-6)
         (m-3-6)edge(m-4-5)
         (m-4-5)edge(m-5-4)
         (m-5-4)edge(m-6-3)
         (m-6-3)edge(m-7-2)
         (m-7-2)edge(m-8-1)
         
         (m-3-7)edge(m-2-8)
         (m-4-6)edge(m-3-7)
         (m-5-5)edge(m-4-6)
         (m-6-4)edge(m-5-5)
         (m-7-3)edge(m-6-4)
         (m-8-2)edge(m-7-3)
         
         (m-3-8)edge(m-4-7)
         (m-4-7)edge(m-5-6)
         (m-5-6)edge(m-6-5)
         (m-6-5)edge(m-7-4)
         (m-7-4)edge(m-8-3)
         
         (m-5-7)edge(m-4-8)
         (m-6-6)edge(m-5-7)
         (m-7-5)edge(m-6-6)
         (m-8-4)edge(m-7-5)
         
         (m-5-8)edge(m-6-7)
         (m-6-7)edge(m-7-6)
         (m-7-6)edge(m-8-5)
         
         (m-7-7)edge(m-6-8)
         (m-8-6)edge(m-7-7)
         
         (m-7-8)edge(m-8-7)
         
         (m-4-1)edge(m-5-1)
         (m-6-1)edge(m-7-1);
\end{tikzpicture}
	\caption{Cantor diagonal method}
	\end{figure}
	This table is constructed so that each rational number appears only once (in the sense of its decimal value) by diagonal hence the name of the method: "\NewTerm{Cantor diagonal}\index{Cantor diagonal}".
	
	If we eliminate from each diagonal the rational numbers that appear more than one time (the "\NewTerm{equivalent fractions}\index{equivalent fractions}") in order to keep only those who are irreducible (i.e. those with the greatest common divisor of the numerator and denominator is equal to $1$), then we can with this distinction define an application $f:\mathbb{N} \Rightarrow \mathbb{Q}$ that is injective (two distinct rational numbers have distinct ranks) and surjective (at any place will be written a rational number).
	
	The application $f$ is therefore bijective: $\mathbb{N}$ and $\mathbb{Q}$ are then effectively equipotent!
	
	The definition a little bit more rigorous (and therefore less funny) of $\mathbb{Q}$ from $\mathbb{Z}$ is as follows (it is interesting to see the notation used):
	
	On the set $\mathbb{Z}\times \mathbb{Z} \ \lbrace 0 \rbrace$, which should be read as the set constructed from two relative integer whose zero is excluded from the second one, we consider the relation $R$ between two relative pairs of integers defined by:
	
	We then easily verify that $R$ is an equivalence relation (\SeeChapter{see section Operators}) on $\mathbb{Z}\times \mathbb{Z} \setminus \lbrace 0 \rbrace$.
	
	The set of equivalence classes for this relation $R$ denoted then $\left(\mathbb{Z}\times \mathbb{Z} \setminus \lbrace 0 \rbrace\right) / R$ is by definition $\mathbb{Q}$. That is to say that we write therefore more rigorously:
	
	The equivalence class $(a,b) \in \mathbb{Z}\times\mathbb{Z}\setminus \{0\}$ is explicitly denoted by:
	
	in accordance with the notation that everyone is accustomed to use.

	We easily check the addition and multiplication operations that were operations defined on $\mathbb{Z}$ pass without problems to $\mathbb{Q}$ by writing:
	
	Moreover, these operations provide $\mathbb{Q}$ with the structure of a body (\SeeChapter{see section Set Theory}) with $\dfrac{0}{1}$ as neutral element for the addition and $\dfrac{1}{1}$ as neutral element for the multiplication. Thus, any non-zero element of $\mathbb{Q}$ is reversible, in fact:
	
	what is written also more technically:
	
	\begin{tcolorbox}[title=Remark,colframe=black,arc=10pt]
	Even if we want to define $\mathbb{Q}$ as the being the set $\mathbb{Z}\times \mathbb{Z}\setminus{0}$ where $\mathbb{Z}$ represents the numerators and $\mathbb{Z}\setminus{0}$ the denominators of the rationals, this is not possible because otherwise we would for example $(1,2)\neq(2,4)$ while we expect for an equality.
	
	Hence the need to introduce an equivalence relation which enables us to identify, to return to the previous example, with $(1,2)$ and $(2,4)$. The relation $R$ that we have defined does not fall from heaven, indeed the reader who handled the rational so far without ever having seen their formal definition knows that:
	
	It is therefore almost natural to define the relation $R$ as we have done. In particular, regarding the above example, $\dfrac{1}{2}=\dfrac{2}{4}$ because $(1,2) R (2,4)$ and the problem is solved.
	\end{tcolorbox}
	In addition to the historical circumstances of its establishment, this new entity (set) is distinguished from relative numbers because it induces the original and paradoxical concept of partial quantities. This notion that a priori does not make sense, find its place in the mind of man thanks to the geometry where the idea of fraction of length, of proportion are illustrated more intuitively.
	\pagebreak
	\subsubsection{Irrational Numbers}
	It can seem obvious to present irrational numbers before real number (see further below) but this can be explained by the fact of this is the order of the discovering in the human history and therefore is seems more pedagogical to us to present them in this order.
	
	So the set of rational $\mathbb{Q}$ is limited and sadly not sufficient too. Indeed, we may think that all mathematical computation with commonly known operations are reduced to this set but it is not the case!
	
	\begin{tcolorbox}[colframe=black,colback=white,sharp corners]
	\textbf{{\Large \ding{45}}Examples:}\\\\
	E1. Let us calculate the square root of two which we denote $\sqrt{2}$ (thing to Pythagorean theorem with a triangle of side $1$ and $1$ then the third one is of size $\sqrt{2}$). Suppose it is a rational root. So if this is truly a rational, we should be able to express it as $a / b$, where by the definition of a rational $a$ and $b$ are integers with no common factors. For this reason, $a$ and $b$ can not both be even numbers. There are three remaining possibilities:
	\begin{enumerate}
		\item $a$ is odd (then  $b$ is even)
		\item $a$ is even (then  $b$ is odd)
		\item $a$ is odd (then  $b$ is odd)
	\end{enumerate}
	By squaring, we have:
	
	That can be written:
	
	Since the square of an odd number is odd and the square of an even number is even, the case (1) is not possible because $a^2$ would be odd and $2b^2$ would be even.\\
	
	If case (2) is also impossible, because then we could write $a=2c$, where $c$ is any integer, and so if we take the square then we have $a^2=4c^2$ that is to say an even number on both sides of equality. Substituting in $2b^2=a^2$ we obtain after simplification that $b^2=2c^2$. Then $b^2$ would be odd while $2c^2$ would even.\\
	
	The case (3) is also impossible because $a^2$ is then odd and $2b^2$ is even (that $b$ is even or odd!).\\
	
	There is the no solutions! That is to say that the start assumption is false and there does not two integers $a$ and $b$ such that $\sqrt{2}=a/b$.
	\end{tcolorbox}

	\pagebreak
	\begin{tcolorbox}[colframe=black,colback=white,sharp corners]
	E2. Let us prove by contradiction, that the famous Euler number $e$ is irrational. To do this, remember that $e$ (\SeeChapter{see section Functional Analysis}) can also be defined by the Taylor series (\SeeChapter{see section Sequences and Series}):
		
	Then if $e$ is rational, it could be written in the form $p / q$ (with $q>1$, because we know that is not an integer). Let us multiply both sides of the equality by $q!$:	
	
	The first member $q!e$ would then be an integer, because by definition of the factorial:
	
	is an integer.\\
	
	The first terms of the second member of the previous prior-previous relation, until the term $q! /q! = 1$ are also integer because $q! /m!$ is simplified if $q> m$. So by subtraction we find:
	
	when the right sequences should be an integer!\\
	
	After simplification, the second member of the equality becomes:
	
	the first term in this sum is strictly less than $1/2$, the second strictly less than $1/4$ second, the third strictly less than $1/8$, etc.\\
	
	So, since each term is strictly less than the following harmonic series which converges to $1$:
	
	then therefore the sequence is not an integer as being strictly less than $1$. This is a contradiction!
	\end{tcolorbox}
	Thus, the rational numbers do not satisfy the numerical expression of $\sqrt{2}$ and $e$ (to cite only these two particular examples).
	
	They must therefore be complemented by the set of all numbers that can not be written as a fraction (the ratio of an integer dividend and an integer divisor without common factors) and that we name "irrational numbers".
	Finally we can say that:
	
	\textbf{Definition (\#\mydef):} In mathematics, an "\NewTerm{irrational number}\index{irrational number}" is any real number that cannot be expressed as a ratio of integers. Irrational numbers cannot be represented as terminating or repeating decimals.

	\subsubsection{Real Numbers}
	\textbf{Definition (\#\mydef):} The union of rational and irrational numbers gives the set of "\NewTerm{real numbers}\index{real numbers}" that we denote by:
	
	\begin{tcolorbox}[title=Remark,colframe=black,arc=10pt]
	Mathematicians in their usual rigour have different techniques to define real numbers. They use the properties of topology (among others) and especially Cauchy sequences but that's another story that goes beyond the formal scope of this section. For a "set point of view definition of $\mathbb{R}$ the reader should report to the section on Set Theory.
	\end{tcolorbox}
	\begin{figure}[H]
		\centering
		\includegraphics{img/arithmetics/real_numbers.jpg}
		\caption{Simple number sets summary}
	\end{figure}
	Obviously we are led to ask ourselves whether $\mathbb{R}$ is countable or not. The proof is quite simple.
	
	\begin{dem}
	By definition, we have seen above that there must be a bijective correspondence between $\mathbb{Q}$ and $\mathbb{R}$ to that $\mathbb{R}$ is countable.
	
	For simplicity, we will show that the interval $[0,1[$ is then not countable. This will involve of course by extension that $\mathbb{R}$ is not countable!
	
	The elements of this interval are represented by infinite sequences between $0$ and $9$ (in the decimal system):
	\begin{itemize}
		\item Some of these suites are zero from starting from a given rank, some not.
		
		\item So we can identify $[0,1[$ to the set of all sequences (finite or infinite) of integers between $0$ and $9$.
		
		If this set was countable, we could classify these sequences (with a first, second, etc.). Thus, the sequence $x_{11}x_{12}x_{13}x_{14}...x_{1p}...$ would be classified first and so on ... as proposed in the above table.
		
		We could then edit this infinite matrix as follows: to each element of the diagonal, we add $1$, according to the rule: $0 + 1 = 1, 1 + 1 = 2, 8 + 1 = 9$ and $9 + 1 = 0$:

		
		Then let us consider the sequence on the diagonal:
		\begin{itemize}
			\item It cannot be equal to the first sequence of the first row in the prior-previous table since it is distinguished at least by the first element.
			
			\item It cannot be equal to the second sequence of the second row of the prior-previous table since is distinguished at least by the second element.
			
			\item It cannot be equal to the third sequence of the second row of the prior-previous table since is distinguished at least by the third element.
		\end{itemize}
		and so on ... It the cannot be equal to any of the sequences in this table!
	\end{itemize}
	So whatever the chosen classification of infinite sequences of $0 ... 9$, there is always one who escapes this classification! So it is that it is impossible to number them ... simply because they do not form a countable set!
	\begin{flushright}
		$\square$  Q.E.D.
	\end{flushright}	
	\end{dem}
	The technique that has allowed us to achieve this result is known as the "\NewTerm{Cantor diagonal process}\index{Cantor diagonal process}" (because similar to that used for equipotence between the natural and rational set) and the set of real numbers is said to have the  "\NewTerm{power of continuum}\index{power of continuum}" by the fact that it is uncountable.
	\begin{tcolorbox}[title=Remark,colframe=black,arc=10pt]
	We assume that it is intuitive for the reader intuitive that any real number can be approximated infinitely close by a rational number (for irrational numbers we simply stop at a given number of decimals and find the corresponding rational). Mathematicians say therefore that $\mathbb{Q}$ is "\NewTerm{dense}\index{dense set}" in $\mathbb{R}$ and denote this by:
	
	\end{tcolorbox}
	
	In business it is of usage with real numbers to communicate in percentages or per-thousand.
	
	\textbf{Definitions (\#\mydef):}
	\begin{itemize}
		\item[D1.] Given a scalar $x \in \mathbb{R}$ then expressed in percentage it will denoted by:
			
		\item[D2.] Given a scalar $x \in \mathbb{R}$ then expressed in per-thousand it will denoted by:
			
	\end{itemize}
		
	\subsubsection{Transfinite Numbers}
	We now are with an infinity of real numbers which is different from that of natural numbers. Cantor then dared what no one had dared since Aristotle: the positive integers sequence is also infinite, the set $\mathbb{N}$, is then a set that has a countable infinity of elements, then he said that the cardinal (\SeeChapter{see section Set Theory}) of this set was a number that existed as such without we use the tote symbol $\infty$, he denote it:
	
	This symbol is as we know (\SeeChapter{see section Set Theory}) the first letter of the Hebrew alphabet, pronounced "aleph zero". Cantor was going to name this strange number, a "\NewTerm{transfinite number}\index{transfinite number}".
	
	The decisive act is to assert that there is, after the finite, a transfinite, that is to say an unlimited scale of determined modes which by nature are infinite, and yet can be specified, as for the finite, by specific numbers, well defined and distinguishable from each other!! This tool was necessary as a set cardinal can be equal to one of its parts as we will see just below!
	
	After this first stroke going against most ideas for over two thousand years, Cantor would continue its path and build the calculation rules, paradoxical at first glance, of the transfinite numbers. These rules were based, as we said earlier, on the fact that two infinite sets are equivalent if there exists a bijection between the two sets.
	
	Thus, we can easily show that the infinity of even numbers is equivalent to the infinity of integers: for this, it suffices to show that for every integer, we can associate an even number, his double, and vice versa. Therefore the cardinal of integers is equal to those of even numbers (the cardinal of a set can be equal to one of its parts!).
	
	Thus, although if even numbers are included in the set of integers, there is an infinity $\alpha_0$ of them, the two sets are equipotent. By stating that a set can be equal to one of its parts, Cantor goes against what seemed obvious to Aristotle and Euclid: the set of all sets is infinite! This will shake the whole of mathematics and will bring the axiomatic Zermelo-Fraenkel we will see in the section of Set Theory.
	
	From the above, Cantor define the following calculations rules on the Cardinals:
	
	At first glance these rules seem non-intuitive, but in fact they are! Indeed, Cantor defined the addition of two transfinite numbers as cardinal of the disjoint union of the corresponding sets.
	
	\begin{tcolorbox}[colframe=black,colback=white,sharp corners]
	\textbf{{\Large \ding{45}}Examples:}\\\\
	E1. By noting $\aleph_0$ the cardinal of $\mathbb{N}$ we have $\aleph_0+\aleph_0$ which is equivalent to saying that we summ the cardinal of $\mathbb{N}$ disjoint union $\mathbb{N}$. But as $\mathbb{N}$ disjoint union $\mathbb{N}$ is equipotent to $\mathbb{N}$  then $\aleph_0+\aleph_0=\aleph_0$ (it is enough to be convinced to take the set of odd and even integers which are both countable and which disjoint union is also countable).\\
	
	E2. Other trivial example: $\aleph_0+1$ corresponds to the cardinality of $\mathbb{N}$ union a point. This set is still equipotent to $\mathbb{N}$ therefore $\aleph_0+1=\aleph_0$.
	\end{tcolorbox}
	We will also during our study of the section Set Theory that the concept of Cartesian product of two countable sets is such that we have:
	
	and therefore:
	
	Similarly (\SeeChapter{see section Set Theory}) since $\mathbb{Z}=\mathbb{Z}^{+}\cup\mathbb{Z}^{-}$  we have:
	
	and identifying $\mathbb{Q}$ to $\mathbb{Z}\times\mathbb{Z}$ (ratio of a numerator over denominator) we have immediately:
	
	We can also prove an interesting statement: if we consider the cardinality of the set of all the cardinals, it is necessarily greater than all the cardinals, including itself (it is better to have read previously the section of Set Theory)! In other words: the cardinality of the set of all sets of $A$ is greater than the cardinal of $A$ itself.
	
	This implies that there is no set containing all sets since there is always a bigger one (it is an equivalent form of the famous old Cantor's paradox)!!!
	
	In technical language it means considering a non-empty set $A$ and then to state that:
	
	where $\mathcal{P}(A)$ is the set of subsets of $A$ (see the section Set Theory for the general calculation of the cardinal of the set of all parts of a countable set).
	
	That is to say, by definition of the order relation $<$ (strictly less), it suffices to prove that there is no surjective application $f:A\mapsto \mathcal{P}(A)$, in other words that to each element of the set of parts of $A$ it does not match at least one pre-image in $A$.
	
	\begin{tcolorbox}[title=Remark,colframe=black,arc=10pt]
	The set $\mathcal{P}(\mathbb{N})$ for example consists of the set of even numbers, odd numbers, natural numbers, as well as the empty set itself, etc. $\mathcal{P}(\mathbb{N})$ is therefore the set of all "potatoes" (to borrow the vocabulary of high school ...) that make $\mathbb{N}$.
	\end{tcolorbox}
	\begin{dem}
	Suppose that we can number each potatoe of $\mathcal{P}(A)$ with at least one element of $A$ (imagine that with $\mathbb{N}$ or see the example in the section of Set Theory). In other words it is equivalent to suppose that $f:A\mapsto \mathcal{P}(A)$ is surjective and let us consider a subset $E$ of $A$ such that:
	
	that is to say the set of elements $x$ of $A$ that do not belong to the set numbered by $x$ (the element $x$ does not belong to the "potato" that it numbers in other terms...).
	
	Or, if $f$ is surjective it must also be a $y \in A$ for this subset $E$ such that:
	
	since $E$ is also a subset of $A$.
	
	Suppose that $y$ belongs to $E$. In this case, by definition of $E$, $y \notin f(y)=E$ (by definition of $E$ that applies for every $x$ and $x$ can also be obviously $y$ or $z$ or don't matter what). By consequence, $y\not in E$, but in this second case, always by definition of $E$, $y\in f(y)=E$ (as $y$ is not in $E$). We see therefore that the element $y$ cannot exists and therefore $f$ cannot be surjective. 
	
	We strongly recommend the reader to read the previous sentence more than on time if necessary.
	\begin{flushright}
		$\square$  Q.E.D.
	\end{flushright}
	\end{dem}
	
	\pagebreak
	\subsubsection{Complex Numbers}
	Invented in the 16th century among others by Girolamo Cardano and Rafaello Bombelli, "\NewTerm{complex numbers}\index{complex numbers}" (also named "\NewTerm{imaginary numbers}\index{imaginary numbers}") are used to solve problems with no solutions in $\mathbb{R}$ and also used to mathematically formalize certain transformations in the plan such as rotation, similarity, translation, etc. and also to generalized some theorem restricted to $\mathbb{R}$ and therefore hiding some interesting results for practical engineering. For physicists, complex numbers above are also a very convenient way to simplify notations. It is thus very difficult to study wave phenomena, General Relativity or quantum mechanics without using complex numbers and expressions.
	
	There are several ways to construct complex numbers. The first is typical of the construction way that mathematicians used as part of Set Theory. They define a couple of real numbers and define the operations between these couples to finally arrive at a meaning of the complex number concept. The second one is less rigorous but its approach is simpler and consist to define the pure unit imaginary number $\mathrm{i}$ and then build arithmetic operations from its definition. We will opt for the second method in the texts that will follow!
	
	\textbf{Definitions (\#\mydef):}
	\begin{enumerate}
		\item We define the "unit pure imaginary number" that we denote by $\mathrm{i}$ by the following property:
		
			
		\item A "\NewTerm{complex number}\index{complex number}" is a pair of a real number $a$ and an imaginary number $\mathrm{i}b$ and generally written in the following form:
		
		where $a$ and $b$ are numbers belonging to  $\mathbb{R}$.
		
		\item We note the set of complex numbers by $\mathbb{C}$ and therefore we have by construction:
		
	\end{enumerate}
	\begin{tcolorbox}[title=Remark,colframe=black,arc=10pt]
	The set $\mathbb{C}$ is identified to the oriented Euclidean plane $E$ (\SeeChapter{see section Vector Calculus}) thanks to the choice of a direct orthonormal basis (we therefore get "\NewTerm{Argand-Cauchy plane}\index{Argand-Cauchy plane}", also named "\NewTerm{Gauss-Argand plane}\index{Gauss-Argand plane}" or more commonly "\NewTerm{Gauss plane}\index{Gauss plane}" that we will see a little further below and that seems have be defined for the first time in 1806).
	\end{tcolorbox}
	The set of complex numbers that constitutes a field (\SeeChapter{see section Set Theory}) and denoted by $\mathbb{C}$, is defined (in a simple way to start) in the notation of set theory by:
	
	In other words we say that the field $\mathbb{C}$ is the field $\mathbb{R}$ to which we have added the imaginary number $\mathrm{i}$. Which is formally denoted by:
	
	The addition and multiplication of complex numbers are internal operations to the set (field) of complex numbers (we will come back much more in detail on certain properties of complex numbers in the section of Set Theory) and defined by:
	
	The "\NewTerm{real part}\index{complex number: real part}" of $z$ is traditionally denoted by:
	
	The "\NewTerm{imaginary part}\index{complex number: imaginary part}" of $z$ is traditionally denoted by:
	
	The "\NewTerm{conjugate}\index{conjugate}" of $z$ is defined by:
	
	and is sometimes also denoted $z^{*}$ (particularly in quantum physics in some books!).
	
	From a complex and its conjugate, it is possible to find its real and imaginary parts. These are the following obvious relations:
	
	The "\NewTerm{module}\index{module}" of $z$ (or "\NewTerm{norm}\index{norm}") is the length from the center of the Gaussian plane (see further below a figure of the Gaussian plane) and is simply calculated using the Pythagorean theorem :
		
	and is always a positive number or or equal to zero.
	
	We consider as obvious that is satisfy all the properties of a distance (see section of Topology and Vector Calculus).
	
	\begin{tcolorbox}[title=Remark,colframe=black,arc=10pt]
	The notation $|z|$ for the module is not innocent since $|z|$ coincides with the absolute value of $z$ when $z$ is real.
	\end{tcolorbox}
	The division between two complex number s is calculated as (the denominator is obviously not zero):
	
	The opposite of a complex number is calculated similarly:
	
	We can therefore list $8$ important properties of the module and the complex conjugate:
	\begin{itemize}
		\item[P1.] We affirm that:
		
		\begin{dem}
		By definition of the module $|z|=\sqrt{x^2+y^2}$  so that the sum $x^2+y^2$ is zero, the necessary condition is that as $(x,y)\in \mathbb{R}$:
		
		\begin{flushright}
			$\square$  Q.E.D.
		\end{flushright}
		\end{dem}
		
		\item[P2.] We affirm that:
		
		\begin{dem}
		This is immediate by:
		
		\begin{flushright}
			$\square$  Q.E.D.
		\end{flushright}
		\end{dem}
		
		\item[P3.] We affirm that:
		
		\begin{dem}
		The two above inequalities can be written:
		
		thus equivalent respectively to:
		
		which are trivial. The rest of the proof is therefore trivial!
		\begin{flushright}
			$\square$  Q.E.D.
		\end{flushright}
		\end{dem}
		
		\item[P4.] We have:
		
		and if $z_2\neq 0$:
		
		\begin{dem}
		First:
		
		(we will prove a little further below that generally $\overline{z_1z_2}=\bar{z}_1\bar{z}_2$) and for $z_2\neq 0$:
		
		and taking root square this finish the proof.
		\begin{flushright}
			$\square$  Q.E.D.
		\end{flushright}
		\end{dem}
		
		\item[P5.] We affirm that:
		
		\begin{dem}
		This is immediate:
		
		\begin{flushright}
			$\square$  Q.E.D.
		\end{flushright}
		\end{dem}
		
		\item[P6.] We affirm that:
		
		\begin{dem}
		The first one is immediate:
		
		and:
		
		and:
		
		\begin{flushright}
			$\square$  Q.E.D.
		\end{flushright}
		\end{dem}
		\begin{tcolorbox}[title=Remark,colframe=black,arc=10pt]
		\textbf{R1.} In mathematical terms, the first proof helps to show that complex conjugation is what is named an "\NewTerm{involution}\index{involution}" (in the sense that it is changing anything ...).\\
		
		\textbf{R2.} Also in mathematical terms (it is only the vocabulary!), the second proof shows that the combination of the sum of two complex numbers is what we name a "group automorphism $(\mathbb{C},+)$" (\SeeChapter{see section Set Theory}).\\
		
		 \textbf{R3.} Again, for vocabulary ... the third proof show that the combination of the product of two complex numbers is what we name a "field automorphism $(\mathbb{C},+,\times)$" (\SeeChapter{see section Set Theory}).
		\end{tcolorbox}
		
		\item[P7.] We affirm that for $z$ different from zero:
		
		\begin{dem}
		We will restrict ourselves to the proof of the second relation that is a general case of the first (for $z=1$).
		
		\begin{flushright}
			$\square$  Q.E.D.
		\end{flushright}
		\end{dem}
		
		\item[P8.] We have:
		
		for any complex number $z_1,z_2$ (strictly speaking non-zero complex numbers, otherwise the concept of argument of the complex number that we will see further below is undetermined). Furthermore the equality holds if and only if $z_1$ and $z_2$ are collinear (the vectors are "on the same straight line") and of the same direction, in other words.... if it exist $\lambda \in \mathbb{R}$ such as $\lambda z_1=z_2$.
		\begin{dem}
		Directly we have:
		
	This inequality may not be obvious to everyone, therefore let us develop it a bit and let us assume it true:
		
		After simplification:
		
		and again after simplification:
		
		So as the square brackets is necessarily positive or zero it follows:
		
		This last relation thus shows that inequality is true.
		\begin{flushright}
			$\square$  Q.E.D.
		\end{flushright}
		\end{dem}
		\begin{tcolorbox}[title=Remark,colframe=black,arc=10pt]
		In fact there is a more general form of this inequality named "\NewTerm{Minkowski inequality}\index{Minkowvski inequality}", proved in the section of Vector Calculus (complex numbers can indeed be written in the form of vectors as we will see later).
		\end{tcolorbox}
	\end{itemize}
	
	\paragraph{Geometric Interpretation of Complex Numbers}\mbox{}\\\\
	We can also represent any complex number $a+\mathrm{i}b$ or $a-\mathrm{i}b$ in a plane defined by two axes (two dimensions) of infinite length and orthogonal between them. The vertical axis represents the imaginary part of a complex number and the horizontal axis the real part (see figure below).
	
	So there is correspondence between the set of complex numbers and the set of vectors of the Gaussian plane (notion of affix as we will see more deeply in the section of Vector Calculus).
	
	We sometimes named this type of representation "\NewTerm{Gauss plane}\index{Gauss plane}" or "\NewTerm{Gauss map}\index{Gauss plane}":
	
	\begin{figure}[H]
		\centering
		\includegraphics{img/arithmetics/gauss_plane.jpg}
		\caption{Complex Gauss plane}
	\end{figure}
	and then we write:
	
	We see on this diagram that a complex number has thus a vector interpretation (\SeeChapter{see section Vector Calculus}) given by:
	
	where the canonical basis is defined such as:
	
	with:
	
	Thus, $\begin{pmatrix}1\\0\end{pmatrix}$ is the unitary basis vector of the carried by the horizontal axis $\mathbb{R}$ and $\begin{pmatrix}0\\1\end{pmatrix}$ is the unitary basis vector carried by the vertical imaginary axis $\mathbb{R}_{\text{i}}$ and $r$ is the module (the norm) that is positive or zero.
	
	This has to be compared with the vectors of $\mathbb{R}^2$ (\SeeChapter{see section Vector Calculus}):
	
	with:
	
	so that we can identify the complex plane with the Euclidean plane.
	Thanks to the geometric interpretation of the Gaussian plane, the equality below is immediate for example and avoids making some developments:
	
	In addition, the definitions of the cosine and sine (\SeeChapter{see sectionTrigonometry}) give us:
	
	Finally:
	\begin{equation}
	  \addtolength{\fboxsep}{5pt}
	   \boxed{
	   \begin{gathered}
	   	\begin{aligned}
		&r=\sqrt{a^2+b^2}\\
		&\varphi^{-1}=\cos^{-1}\left(\dfrac{a}{r}\right)=\sin^{-1}\left(\dfrac{b}{r}\right)=\tan^{-1}\left(\dfrac{b}{a}\right)
	 	\end{aligned}
	   \end{gathered}
	   }
	\end{equation}
	Therefore:
	
	complex number which is always equal to itself modulo $2\pi$ by the properties of trigonometric functions:
	
	with $k\in \mathbb{N}$ and where $\varphi$ is named the "\NewTerm{argument of $z$}\index{argument}" and is traditionally denoted by:
	
	The properties of the cosine and sine (\SeeChapter{see section Trigonometry}) lead us directly to write for the argument:
	
	We also prove among other things with the Taylor series (\SeeChapter{see section Sequences and  and Series}) that:
	
	and:
	
	which sum is similar to:
	
	but instead perfectly identical to the Taylor expansion of $e^{\mathrm{i}x}$:
	
	So finally, we can write:
	
	relation named "\NewTerm{Euler's formula}\index{Euler's formula}".
	
	Using the properties of trigonometric functions:
	
	
	Depending on we sum or subtract the this gives us the "\NewTerm{Euler formulas}" or "\NewTerm{Moivre and Euler formulas}\index{Moivre and Euler formulas}":
	\begin{equation}
	  \addtolength{\fboxsep}{5pt}
	   \boxed{
	   \begin{gathered}
	   	\begin{aligned}
		\cos(\varphi)&=\dfrac{e^{\mathrm{i}\varphi}+e^{-\mathrm{i}\varphi}}{2}\\
		\sin(\varphi)&=\dfrac{e^{\mathrm{i}\varphi}-e^{-\mathrm{i}\varphi}}{2\mathrm{i}}\\
	 	\end{aligned}
	   \end{gathered}
	   }
	\end{equation}
	Note that the angle can be a purely a complex number! This is to say that in all  generality trigonometric functions can be considered as functions that go from $\mathbb{C}$ to $\mathbb{C}$.
	
	Thanks to the exponential form of a complex number, very commonly used in many fields of physics and engineering, we can easily draw relations such that starting from (remember that $\text{cis}$ is an old notation that stands for the $\cos(\varphi)+\mathrm{i}\sin(\varphi)$ being in the parenthesis):
	
	and assuming known the basic trigonometric identities (\SeeChapter{see section Trigonometry}) we have the following relations for the multiplication of two complex numbers:
	
	therefore:
	
	and therefore if $n$ is a positive integer:
	
	For the module (norm) of the multiplication:
	
	Therefore:
	
	For the division of two complex numbers:
	
	The module of their division then comes immediately:
	
	therefore we have for the argument:
	
	and it comes immediately:
	
	For the power of a complex number (or root):
	
	which gives us immediate a already proved previously:
	
	and for the argument:
	
	In case we have a unit module (norm equal to $1$) as $z=\cos(\varphi)+\mathrm{i}\sin(\varphi)$ we then have the relation:
	
	named "\NewTerm{De Moivre formula}\index{De Moivre formula}".

	For the natural logarithm of a complex number, we trivially have the following relation which is discussed in the section of Analysis Analysis:
		
	where $\ln(z)$ is often in the complex case written $\mathrm{Log}(z)$ with an uppercase "L".
	
	All previous relations could of course be obtained with the trigonometric form of complex numbers but then require some additional lines of mathematical developments.
	
	\subparagraph{Fresnel Vectors (phasors)}\mbox{}\\\\
	A sinusoidal variation $f(t)=r\sin(\omega t)$ can be represented as the projection (\SeeChapter{see section Trigonometry}) on the vertical $y$-axis (imaginary axis the set $\mathbb{C}$) of a rotating vector $\vec{r}$ at angular velocity $\omega$ around the origin in the plane $x\text{O}y$:
	\begin{figure}[H]
		\centering
		\includegraphics{img/arithmetics/fresnel_representation.jpg}
		\caption{Fresnel Representation}
	\end{figure}
	Such a rotating vector is named "\NewTerm{Fresnel vector}\index{Fresnel Vector}" and can be well interpreted as the imaginary part of a complex number given by:
	
	That is to say:
	\begin{figure}[H]
		\centering
		\includegraphics{img/arithmetics/fresnel_rotating_vector.jpg}
		\caption{Fresnel rotating vector}
	\end{figure}
	We will see the phasor again explicitly in our study of wave mechanics and geometrical optics (as part of diffraction) in the sections with the corresponding names.
	
	\paragraph{Transformation in the plane}\mbox{}\\\\	
	It is customary to represent real numbers as points on a graduated line. The algebraic operations have their geometric interpretation on it: the addition is a translation, a multiplication a centered scaling.
	
	In particular we can talk about the "square root of a transformation." A translation of amplitude $T$ may be obtained as the iteration of a translation of amplitude $T / 2$. Similarly, a scaling of amplitu $S$ can be achieved as iterated scaling of faction $\sqrt{S}$. In particular an homothety (scaling) of a factor $9$ can be composed of two homotheties (scaling) of respectively $3$ (or $-3$).
	
	Then we can say that the square root takes on a geometric sense. But what about the square root of negative numbers? In particular of the square root of $-1$???
	
	A scaling of factor $-1$ can be seen as a symmetry with respect to the origin. But if we see this transformation in a continuous manner. Therefore a $-1$ scaling factor also be seen as rotation of $\pi$ rotation around the origin.
	
	So, the problem of negative square root is simplified. Indeed, it is not difficult to break down a rotation of $\pi$ radians inot two transformations: we can repeat either a rotation of $\pi/2$ or of $-\pi/2$. The image of $1$ is the square root of $-1$ and $\mathrm{i}$ is situated on a perpendicular to the origin at a distance $1$ either up or down.
	
	Having successfully positioned the number $\mathrm{i}$ it not difficult anymore to put other complex numbers in the Gauss plane. We can therefore associate to $2\mathrm{i}$ the product of the scaling of a factor $2$ (\SeeChapter{see section Euclidean Geometry}) by the rotation of center O with angle of $\pi/2$, that is to say a similitude centered at the origin. This is what we will endeavor to prove now.
	
	Given:
	
	We have the following geometric transformations properties for complex numbers (see the section Trigonometry for the properties of sine and cosine) that we can happily combine at our discretion:
	\begin{enumerate}
		\item[P1.] The multiplication of $z_1$ by a real number $\lambda$ in the Gauss plane corresponds trivially to a homothety of center O (the intersection of real and imaginary axis for recall...) and of ration $\lambda$.
		
		Indeed:
		
		
		\item[P2.] Multiplying of $z_1$ by a complex number of unit module corresponds a rotation of center O and of angle corresponding to the argument of $z_1$. Indeed:
		
		\begin{tcolorbox}[title=Remark,colframe=black,arc=10pt]
		Then we see immediately, for example, that multiplying a complex number by $\mathrm{i}$ (that is to say a complex number with $ \sin(\omega)=1,\cos(\omega)=0$) corresponds a rotation of $\pi/2$
		\end{tcolorbox}
		\begin{theorem}
		It is interesting to notice that in vector form the rotation of center O of $z_1$ by $z_0$ can be written using the following matrix:
		
		\end{theorem}
		\begin{dem}
		We have just seen before that $z_0z_1$  is a rotation of center O of and angle $\omega$. We just need to write it first in the old style:
		
		giving in vector form:
		
		thus the linear equivalent application is:
		
		or as well (we fall back on the rotation matrix in the plane we that we will see in the section of Euclidean Geometry which is a remarkable result!) using:
		
		and in the particular and arbitrary case where $r$ is unitary (in order to have a pure rotation!):
		
		we have immediately (we took again the same notations for the angle as the one we we have in the chapter Geometry):
		
		Note that the rotation matrix can also be written as:
		
		as well:
		
		\begin{flushright}
			$\square$  Q.E.D.
		\end{flushright}
		\end{dem}
		Thus we see that the rotation matrices are not only applications but also are complex numbers (well it was obvious from the start but we had to show it in an aesthetic and simple way).
		
		So, we have for usage to put that:
		
		or with another common notation in linear algebra:
		
		The field of complex numbers is isomorphic to the field of real square matrices of dimension $2$ of the type:
		
		It is a result that we use many times in various section of this book for specific studies in algebra, geometry and relativistic quantum physics.
		
		\item[P3.] The multiplication of two complex corresponds to a homothety added to a rotation. In other words, a "\NewTerm{direct similarity}\index{direct similarity}".
		\begin{dem}
		
		so this is indeed a similarity of ratio $b$ and angle $\beta$.
		
		At the opposite, the following operation:
		
		will be named a "\NewTerm{retrograde linear similarity}\index{retrograde linear similarity}".
		
		Otherwise, it returns trivially an already known following relation:
		
		\begin{tcolorbox}[title=Remarks,colframe=black,arc=10pt]
		\textbf{R1.} As the sum of two complex numbers $z_1+z_2$ can not have a special simplified mathematical notation in any form whatsoever, then we say that the resulting quantity is equivalent to an "\NewTerm{amplitude translation}\index{amplitude translation}".\\
		
		\textbf{R2.} The combination of a direct linear similarity (multiplication of two complex numbers) and an amplitude translation (sum by a third complex number) is what we name a "\NewTerm{direct linear similarity}\index{direct linear similarity}".
		\end{tcolorbox}
		\begin{flushright}
			$\square$  Q.E.D.
		\end{flushright}
		\end{dem}
		
		\item[P4.] The conjugate of a complex number is geometrically symmetrical with respect to the axis such that:
		
		without forgetting that (basis of trigonometry):
		
		This gives us a known result:
		
		From which we get the following property:
		
		Hence:
		
		
		\item[P5.] The negation of the conjugate of a complex number is geometrically its symmetrical with respect to the imaginary axis such that:
		
		\begin{tcolorbox}[title=Remarks,colframe=black,arc=10pt]
		\textbf{R1.} The combination of the properties P4, P5 is named a "\NewTerm{retrograde similarity}\index{retrograde similarity}".\\
		
		\textbf{R2.} The geometric operation that consist to take the inverse of the conjugate of a complex number (that is to say $\bar{z}^{-1}$) is named a "\NewTerm{pole inversion}\index{pole inversion}".
		\end{tcolorbox}
		
		\item[P6.] The rotation of coordinate cente $c$ and angle $\varphi$ is given and denoted by:
		
		Some explanations could be useful for some readers:
		
		The complex $c$ gives a point in the Gaussian plane, which will be the center of rotation. The difference $z_1-c$ gives the chosen radius $r$. The multiplication by $e^{\mathrm{i}\varphi}$ is the counterclockwise rotation of the radius from the origin of the Gaussian plane. Finally, the addition by $c$ is the necessary translation to take back the rotated radius $r$ at its original place before the rotation (center $c$). Which gives schematically:
		\begin{figure}[H]
			\centering
			\includegraphics{img/arithmetics/complex_rotation.jpg}
			\caption{Representation of the complex rotation}
		\end{figure}
	
		\item[P7.] On the same idea, we get and denote an homothety of center $c$ and ratio $\lambda$ by:
		
		Some explanations could be useful for some readers:
		
		The difference $z_1-c$ always gives the radius $r$ and $c$ a central point in the Gauss plane. The expression $\lambda(z_1-c)$ gives the homothety of the radius from the origin of the Gaussian plane, and finally by adding $c$ gives the necessary translation for the homothety to be see as being made from center $c$.
	\end{enumerate}
	
	\subsubsection{Quaternion Numbers}
	Also named "\NewTerm{hypercomplex}\index{hypercomplex}" quaternions numbers were invented in 1843 by William Rowan Hamilton to generalize complex numbers.
	
	\textbf{Definition (\#\mydef):} A "\NewTerm{quaternion}\index{quaternion}" is an element $(a,b,c,d)\in \mathbb{R}^4$ and for which we denote by $\mathbb{H}$ the set that contains it and what we name the "\NewTerm{set of quaternions}\index{set of quaternions}".
	
	A quaternion can also be represented in a row or column such as:
	
	We define the sum of two quaternions $(a, b, c, d)$ and $(a ', b', c ', d')$ by:
	
	\begin{tcolorbox}[title=Remark,colframe=black,arc=10pt]
	It is the natural addition in $\mathbb{R}^4$ seen as a $\mathbb{R}$-vector space (\SeeChapter{see section Set Theory}).
	\end{tcolorbox}
	The associativity is verified by applying the corresponding properties of the operations on $\mathbb{R}$.
	
	We also define the multiplication:
	
	of two quaternions $(a, b, c, d)$ and $(a', b', c', d')$ by the expression:
	
	It may be hard to accept but we will be a little further below that there is a family resemblance with the complex numbers.
	
	We can notice that the law of multiplication is not commutative. Indeed, taking the definition of the multiplication above, we have:
	
	But we can also notice that:
	
	\begin{tcolorbox}[title=Remark,colframe=black,arc=10pt]
	It is the natural addition in $\mathbb{R}^4$ seen as a $\mathbb{R}$-vector space (\SeeChapter{see section Set Theory}).
	\end{tcolorbox}
	The law of multiplication is distributive with the addition law but it is an excellent example where we must still be careful to prove the left and right distributivity, since the product is not commutative!
	The multiplication is neutral element:
	
	Indeed:
	
	Any element:
	
	is inversible.
	
	Indeed, if $(a,b,c,d)$ is a non-null quaternion, we then have necessarily:
	
	otherwise the four numbers $a, b, c, d$ are of square null, so all zero. Given then the quaternion $(a_1,b_1,c_1,d_1)$ defined by:
	
	then by applying mechanically  the definition of the multiplication of quaternions, we check that:
	
	this latter quaternion is therefore the inverse for the multiplication!
	
	Let us prove (for general knowledge) that the field of complex numbers $(\mathbb{C},+,\times)$ is a subfield of $(\mathbb{H},+,\times)$.
	\begin{tcolorbox}[title=Remark,colframe=black,arc=10pt]
	We could also have put this proof in the section of  Set Theory because we will make use of a lot of concepts that are have seen there but it seemed to us a little more relevant to put instead the proof here. We expect the reader to tolerate this choice.
	\end{tcolorbox}
	Given $\mathbb{H}'$ set set of quaternions of the form $(a, b, 0,0)$. If $\mathbb{H}'$ is not empty, and if $(a, b, 0,0)$, $(a ', b', 0.0)$ are elements $\mathbb{H}'$ the $(\mathbb{H}',+\times)$ is a field. Indeed:
	\begin{enumerate}
		\item[P1.] For subtraction (and therefore the addition):
		

		\item[P2.] The multiplication:
			

		\item[P3.] The neutral element:
		

		\item[P4.] And finally the inverse:
		
		of $(a,b,0,0)$ is still in.
	\end{enumerate}
	Therefore $(\mathbb{H}',+,\times)$ is a subfield of $\mathbb{H}$. Given then the application:
	
	$f$ is bijective, and we easily check that for any complex $z_1,z_2$, we have:
	
	Therefore $f$ is an isomorphism of $(\mathbb{C},+,\times)$ on $(\mathbb{H}',+,\times)$.
	
	This isomorphism has for interest (caused) to identify $\mathbb{C}$ to $\mathbb{H}'$ and to write $\mathbb{C} \subset\mathbb{H}$, the laws of addition and subtraction on $\mathbb{H}$ extending the already known operations of $\mathbb{C}$.
	
	Thus, by convention, we will write any element of $(a, b, 0,0)$ of $\mathbb{H}'$ in the complex form $a + \mathrm{i}b$. Particularly $0$ is the element $(0,0,0,0)$, $1$ is the element $(1,0,0,0)$ and $\mathrm{i}$ and the element $(0,1,0,0)$.
	
	We denote by analogy and by extension $j$ the element $(0,0,1,0)$ and $k$ the element $(0,0,0,1)$. The family $\{1, i, j, k\}$ form a basis of all quaternions seen as a vector space on $\mathbb{R}$, and we will write:
	
	the quaternion $(a, b, c, d)$.

	The notation of quaternions as defined above is perfectly suited to the multiplication operation. For the product of two quaternions we get by developing the expression:
	
	$16$ terms that we have to identify to the original definition of multiplication of quaternions to get the following relations:
	
	Which can be summarized in a table:
	
	We can see that the expression of the multiplication of two quaternions looks partly much like a vector product (denoted $\times$ in this book) and dot product (denoted $\circ$ in this book):
	
	If this is not evident (which would be quite understandable), let make a concrete example:
	\begin{tcolorbox}[colframe=black,colback=white,sharp corners]
	\textbf{{\Large \ding{45}}Example:}\\\\
	Given two quaternions without real part:
	
	and $\vec{u},\vec{v}$  the vectors of $\mathbb{R}^3$ of respective components $(x,y,z)$ and $(x',y',z')$. Then the product:
	
	is equal to:
	
	We can also for curiosity interest us to the general case ... Given for this two quaternions:
	
	Then we have:
	
	\end{tcolorbox}
	\textbf{Definition (\#\mydef):} The center of the non-commutative field $(\mathbb{H},+,\times)$ is the set of elements of $\mathbb{H}$ commuting for to the law of multiplication with all the elements of  $\mathbb{H}$.

	\begin{theorem}
	The center of $(\mathbb{H},+,\times)$ is the set of real numbers!
	\end{theorem}
	\begin{dem}
Give $\mathbb{H}_1$ is the center of $(\mathbb{H},+,\times)$, and $(x, y, z, t)$ a quaternion. We must have the following conditions are met:

	Given $(x,y,z,t)\in \mathbb{H}_1 $ then for any $(a,b,c,d)\in \mathbb{H}$ we seek:
	
	which give by developing:
	
	after simplification (the first line of the previous system is equal to zero on both sides of equality):
	
	the resolution of this system gives us:
	So that the quaternion $(x, y, z, t)$ is the center of  $\mathbb{H}$ it must be real (not imaginary parts)!
	\begin{flushright}
		$\square$  Q.E.D.
	\end{flushright}
	\end{dem}
	Just as for complex numbers, we can define a conjugate of quaternions:
	
	\textbf{Definition (\#\mydef):} The conjugate of a quaternion $Z=(a,b,c,d)$ is the quaternion $\bar{Z}=(a,-b,-c,-d)$.

	Just as for the complex number, we notice that:
	\begin{enumerate}
		\item First clearly that if $Z=\bar{Z}$ then it means that $Z\in \mathbb{R}$

		\item That $Z+\bar{Z}\in \mathbb{R}$

		\item That by developping the product $Z\bar{Z}$ we have:
		
		that we will adopt, by analogy with complex numbers, as a definition of the norm (or module) of quaternions such as:
		
		Therefore we also have immediately (relation which will be useful later):
		
	\end{enumerate}
	As for complex numbers (see below), it is easy to show that the conjugation is an automorphism of the group $(\mathbb{H},+)$.
	
	It is also easy to prove that it is involutive. Indeed:
	
	But the conjugation is not a multiplicative automorphism of the field $(\mathbb{H},+,\times)$. Indeed, if we consider the multiplication of $Z$, $\bar{Z'}$ and take the conjugate:
	
	we see immediately (at least for the second row) that we have:
	
	Let us now back to our norm (or module) .... For this, let us calculate the square of the norm $|ZZ'|$:
	
	We know (by definition) that:
	
	Let us denote this product in such a way that:
	
	Then we have:
	
	substituting it comes:
	
	after an elementary algebraic development (frankly boring) we find:
	
	Therefore:
	
	\begin{tcolorbox}[title=Remark,colframe=black,arc=10pt]
	The norm is therefore a homomorphism of $(\mathbb{H},\times)$ in $(\mathbb{R},\times)$. Subsequently, we will denote by $\mathbb{G}$ all the quaternions of unit norm.
	\end{tcolorbox}
	
	\paragraph{Matrix Interpretation of Quaternions}\mbox{}\\\\
	Given $q$ and $p$ two quaternions and given the application:
	
	The  (left) multiplication can be made with a linear application (\SeeChapter{see section Linear Algebra}) on $\mathbb{H}$.
	
	If $q$ is written:
	
	this application has for matrix in the basis $1,i,j,k$:
	
	What we check well:
	
	In fact, we can then define the quaternions as the set of matrices with the visible structure above if we wanted to. This will then reduce them to a sub vector space of $M_4(\mathbb{R})$.
	
	Especially, the matrix of $1$ (the real part of the quaternion $q$) is then nothing other than the identity matrix:
	
	as well:
	
	
	\paragraph{Rotations with Quaternions}\mbox{}\\\\
	We will see now that conjugation by an element of the group $\mathbb{G}$ of the quaternions of unit norm can be interpreted as a pure rotation in space!
	
	\textbf{Definition (\#\mydef):} The "\NewTerm{conjugation}\index{conjugation}" by a non-nul quaternion $q$ of unit norm is the application $S_q$ defined on $\mathbb{H}$ by:
	
	and we affirm that this application is a rotation.
	
	\begin{tcolorbox}[title=Remarks,colframe=black,arc=10pt]
	\textbf{R1.} As $q$ is of unit norm $1$, we have obviously $|q|=q\bar{q}=1$ therefore $q^{-1}=\bar{q}$. This quaternion can be seen as the proper value (of unit norm) to the application (matrix) $p$ on the vector $\bar{q}$ (we are in a similar situation as the orthogonal rotation matrices seen in the in section Linear Algebra).
	
	\textbf{R2.} $S_q$ is a linear application (so if it is rotation, the rotation can be decomposed into several rotations). Indeed, let consider two quaternions $p_1,p_2$ and two real number $\lambda_1,\lambda_2$, then we have:
	
	\end{tcolorbox}
	Let us now check that the application is indeed a pure rotation. As we saw in our study of Linear Algebra and in particular of orthogonal matrices (\SeeChapter{see section Linear Algebra}), a first obvious condition is that the application conserves the norm.
	
	Let us check this:
	
	Moreover, we can check that a rotation of a purely complex quaternion (such that then we restrict ourselves to $\mathbb{R}^3$) and the same summed reverse rotation is zero (the vector sum up to its opposite cancel):
	
	we trivially check that if we have two quaternions $q, p$ then $\overline{p\cdot q}=\bar{q}\bar{p}$ since then:
	
	for this operation to be zero, we immediately see that we need to restrict ourselves to the purely complex quaternions $p$. Since then:
	
	We conclude then that $p$ must be purely complex so the for the application $S_q$ is a rotation and that $S_q(p)$ is a pure quaternion. In other words, this application is stable (in other words: a pure quaternion by this application remains a pure quaternion).
	
	The application $S_q$ restricted to all purely complex quaternions is thus a vectorial isometry, that is to say a symmetry or a rotation.
	
	We have also seen during our study of the rotation matrices in the section of Linear Algebra and Euclidean Geometry  that such matrices should have a determinant equal to $1$ so that we have a rotation. Let's see if this is the case of $S_q$:
	
	For this, we explicitly calculate in function of:
	
	the matrix (in the canonical basis $(i,j,k)$) of $S_q$ and we calculate its determinant. Thus we obtain the coefficients of the columns of this application by remembering that:
	
	and then by calculating:
	
	
	
	
	
	We must then calculate the determinant of the following matrix (pfff ...):
	
	remembering that (which also simplifies the expression of the terms of the diagonal as we can see in some books):
	
	we find that the determinant is indeed equal to $1$. Otherwise, we can check this with Maple 4.00b:
	
	\texttt{>with(linalg):\\
	>A:=linalg[matrix](3,3,[a\string^2+b\string^2-c\string^2-d\string^2,2*(a*d+b*c),\\
	2*(b*d-a*c),2*(b*c-a*d),a\string^2-b\string^2+c\string^2-d\string^2,2*(a*b+c*d),\\
	2*(a*c+b*d),2*(c*d-a*b),a\string^2-b\string^2-c\string^2+d\string^2]);\\
	>factor(det(A));}
	
	Let us now show that this rotation is a half axis turn (the example that may seem particular is in fact general!):
	
	First, if:
	
	we have:
	
	which means that the axis of rotation $(x, y, z)$ is fixed by the application $S_q$ itself!
	
	On the other hand, we have seen that if $q$ is a purely complex quaternion of norm $1$ then:
	
	Which gives us the relation:
	
	This result leads us to calculate the rotation of a rotation:
	
	Conclusion: Since the rotation of a rotation is a full turn, then $S_q$ is necessarily a half-turn:
	
	relatively (!) to the axis $(x, y, z)$.
	
	At this stage, we can say that any rotation of the space can be represented by $S_q$ (the conjugation by a quaternion $q$ of norm $1$). Indeed, the half turns generates the group of rotations, that is to say that any rotation can be expressed as the product of a finite number of half-turns, and therefore by conjugation of a product of quaternions unitary norm (product which is itself a quaternion of unitary norm...).
	
	We will still give an explicit form connecting a rotation and the quaternion that represents it, just as we did for complex numbers.
	\begin{theorem}
	Given $\vec{u}(x,y,z)$ a unit vector and $\theta \in [0,2\pi]$ angle. The we affirm that the rotation of axis $\vec{u}$ and angle $\theta$ corresponds to the application $S_q$, where $q$ is the quaternion:
	
	For this assertion is verified, we know we need that: 
	\begin{itemize}
		\item The norm of $q$ is equal to $1$

		\item The determinant of the application $S_q$ is equal to $1$

		\item The application $S_q$ conserves the norm
	
		\item The application $S_q$ returns all collinear vector to the axis of rotation on the axis of rotation itself.
	\end{itemize}
	\end{theorem}
	\begin{dem}
	Ok let us check every point:
	\begin{enumerate}
		\item The norm of the quaternion previously proposed is indeed equal to $1$:
		
		and as $\vec{u}(x,y,z)$ is of unit norm, we have:
		
		Therefore:
		
		
		\item The fact that $q$ is a quaternion of unit norm  immediately leads to the fact that the determinant of the application $S_q$ is also equal to $1$. We have already proved it above in the general case of any quaternion of norm $1$ (necessary and sufficient condition).
		
		\item It is the same for the conservation of the norm. We have already proved earlier above that this was the case anyway when the quaternion $q$ of norm $1$ (necessary and sufficient condition).

		\item Let us now prove that all collinear vector to the axis of rotation is projected onto the axis of rotation itself. Let us denote by $q'$ the purely imaginary unitary quaternion $xi+yj+zk$. Then we have:
		
		
		Then:
		
		but as $q'$ is the restriction of $q$ to the pure elements that constitute it, this is equivalent as to write:
		
		Let us now show why we choose the writing $\theta/2$. If $\vec{v}=(x_1,y_1,z_1)$ denotes a unit vector orthogonal to $\vec{u}$ (therefore perpendicular to the axis of rotation), and $p$ the quaternion $xi+yj+zk$ then we have:	
		
		We have shown during the definition of multiplication of two quaternions that:
		
		therefore we get:
		
		We have also prove earlier above that:
		
		Therefore:
		
		(the half turn of axis $(x, y, z)$). So:
		
		\begin{tcolorbox}[title=Remark,colframe=black,arc=10pt]
		We are beginning to see here already the usefulness of having chose from the beginning $\theta/2$ for the angle!
		\end{tcolorbox}
	\end{enumerate}
	\begin{flushright}
		$\square$  Q.E.D.
	\end{flushright}
	\end{dem}
	We know that $p$ is the pure quaternion likened to a unit vector $\vec{v}$ orthogonal to the axis of rotation $\vec{u}$ itself equated withthe purely imaginary part of $q'$. We notice then immediately that  the imaginary part of the product (defined!) of the quaternion $q'p$ is equal to the cross product $\vec{u}\times\vec{v}=\vec{w}$. This vector product therefore generates a vector perpendicular to $\vec{u},\vec{v}$.
	
	The pair $(\vec{v},\vec{w})$ thus form a plane perpendicular to the axis of rotation $\vec{u}$ (that's as for the simple complex numbers $\mathbb{C}$ in which we have the Gaussian plane and perpendicular to it the axis of rotation!).
	
	Then finally:
	
	We fall back with on rotation based on a plane (but therefore be in space!) identical to that shown earlier above with the standard complex numbers $\mathbb{C}$ in the Gaussian plane. For more details the reader can refer the  section of Spinor Calculus.
	
	So we know how to do any kind of rotation in space in a single mathematical operation and with a bonus: with the free choice of the axis!

	We can now better understand why the algebra of quaternions is not commutative. Indeed, the vector rotations of the plan are commutative but those of space are not like show us the example below:
	
	Given the initial configuration:
	\begin{figure}[H]
		\centering
		\includegraphics{img/arithmetics/quaternion_initial_configuration.jpg}
		\caption[]{Starting situation for quaternion rotations}
	\end{figure}
	Then a rotation about the $X$-axis followed by a rotation around the $Y$ axis:
	\begin{figure}[H]
		\centering
		\includegraphics{img/arithmetics/quaternion_x_y_rotation.jpg}
		\caption[]{Example quaternion $X-Y$ rotation}
	\end{figure}
	is not equal to a rotation around the $Y$-axis followed by a rotation about the axis $X$:
	\begin{figure}[H]
		\centering
		\includegraphics{img/arithmetics/quaternion_y_x_rotation.jpg}
		\caption[]{Example of non-equivalence for quaternion rotation }
	\end{figure}
	The results will be fundamental for our understanding of spinors (\SeeChapter{see section Spinor Calculus})!
	
	\subsubsection{Algebraic and Transcendental Numbers}
	\textbf{Definitions (\#\mydef):}
	\begin{enumerate}
		\item[D1.] We name "\NewTerm{algebraic integer of degree $n$}\index{algebraic integer of defree $n$}", any complex number that is a solution of an univariate algebraic equation of degree $n$, ie a polynomial of degree $n$ (concept that we will discuss in the chapter of Algebra) whose coefficients are integers and whose dominant coefficient is equal to $1$.

		\item[D2.] We name "\NewTerm{algebraic number of degree $n$}\index{algebraic number of degree $n$}", any complex number that is a solution of an univariate algebraic equation of degree $n$, ie a polynomial of degree $n$ whose coefficients are rational.
	\end{enumerate}
	
	The set of algebraic number is sometimes denoted by $\overline{\mathbb{Q}}$ or $\mathbb{A}$.
	
	\pagebreak
	\begin{theorem}
	A first interesting result and particularly in this area of study (mathematical curiosity ...) is that a rational number is an "algebraic integer of degree $n$" if and only if it's an integer (read several times need...). In scientific terms, we the say that the ring $\mathbb{Z}$ is "\NewTerm{fully closed}\index{fully closed ring}".
	\end{theorem}
	\begin{dem}
	We will assume that the number $p/q$ , where $p$ and $q$ are two prime integers (that is to say that their ratio does not give an integer or more rigorously ... that the greatest common divisor of $p,q$ is equal to $1$! , is a root of the following polynomial (\SeeChapter{see section Calculus}) with relativer integer coefficients ($\in\mathbb{Z}$) and whose dominant coefficient is equal to $1$:
	
	where the equality with zero of the polynomial is implicit.
	
	In this case:
	
	Since the coefficients are by definition all integers and their multiple in the parenthesis also, then the parenthesis has necessarily a value in $\mathbb{Z}$.
	
	Therefore, $q$ (at the right of the parenthesis) divides a power of $p$ (at the left of the equality), which is possible, in the set $\mathbb{Z}$ (because our bracket has a value in this same set for recall...), only if $q$ is equal to $\pm 1$ (as they were prime together).
	
	So among all rational numbers the only that are solutions of polynomial equations with relative integer coefficients $(\in \mathbb{Z}$) for which the dominant coefficient is equal to $1$ are relative integers!
	\begin{flushright}
		$\square$  Q.E.D.
	\end{flushright}
	\end{dem}
	To take another interesting and particular case, it is easy to show that any rational number is an algebraic number. Indeed, if we take the simplest following univariate polynomial:
	
	where $p$ and $q$ are relatively prime and where $q$ is different from $1$. So as this is a simple polynomial with rational coefficients ($\in\mathbb{Q}$), after remaniment we have:
	
	So since $p$ and $q$ are relatively prime and $q$ is different from $1$, we have indeed that every rational number is an "algebraic number of degree $1$".
	
	We also have the real (and irrational) number $\sqrt{2}$ which is an "algebraic integer of degree $2$" because it is the root of:
	
	and the complex number $\mathrm{i}$ is also an "algebraic integer of degree $2$" because it is the root of the equation:
	
	etc.
	
	\textbf{Definition (\#\mydef):} A "\NewTerm{transcendental number}\index{transcendant number}" is a real or complex number that is not algebraic. That is, it is not a root of a non-zero polynomial equation with rational coefficients.
	
	\begin{theorem}
	The set of all transcendental numbers is uncountable. The proof is simple and requires no difficult mathematical development.
	\end{theorem}
	\begin{dem}
	Indeed, since the polynomial with integer coefficients are countable, and since each of these polynomials has a finite number of roots (see the Factorization Theorem in the section Calculus), the set of algebraic numbers is countable! But the argument of Cantor's diagonal (\SeeChapter{see section Set Theory}) states that real numbers (and therefore also the complex numbers) are uncountable, so the set of all transcendental numbers must be uncountable.
	
	In other words, there is much more transcendental numbers than algebraic numbers...
	\begin{flushright}
		$\square$  Q.E.D.
	\end{flushright}
	\end{dem} 
	The best known transcendent numbers are $\pi$ and $e$. We are still looking to provide you a proof more nice and intuitive than that of Hilbert or  Lindemann–Weierstrass.
	
	Here is a small summary of all the stuff see until now:
	\begin{figure}[H]
		\centering
		\includegraphics{img/arithmetics/numbers_type.jpg}
		\caption{Numbers Type $\mathbb{N},\mathbb{Z},\mathbb{Q},\mathbb{R},\mathbb{C},$...}
	\end{figure}
	
	\pagebreak
	\subsubsection{Universe Numbers (normal numbers)}
	\textbf{Definition (\#\mydef):} A "\NewTerm{Universe number}\index{Universe number}" also named "\NewTerm{normal number}\index{normal number}" is a real number whose infinite sequence of digits in every base $b$ is distributed uniformly in the sense that each of the $b$ digit values has the same natural density $1/b$. Intuitively this means that no digit, or (finite) combination of digits, occurs more frequently than any other. The set of Universe numbers is sometimes denoted $\mathbb{U}$.

	While a general proof can be given that almost all purely real numbers are Universe numbers \cite{filip2010elementary} this proof is not constructive and only very few specific numbers have been shown to be Universe numbers. It is widely believed that the (computable) numbers $\sqrt{2}$, $\pi$, and $e$ are Universe numbers, but a proof remains elusive still in this year 2016. All of them however are strongly conjectured to be because of some empirical evidence. It is not even known whether all digits occur infinitely often in the decimal expansions of those constants. In particular, the popular claim "every string of numbers eventually occurs in $\pi$" or "the whole Holy book is contained in $\pi$ is not known to be true. It has been conjectured that every irrational algebraic number is a Universe number, while no counterexamples are known, there also exists no algebraic number that has been proven to be a Universe number in any base.
	
	More formally, let $\sum$ be a finite alphabet of $b$ digits, and $\sum^\infty$ the set of all sequences that may be drawn from that alphabet. Let $S\in\sum^\infty$ be such a sequence. For each $a$ in $\sum$ let $N_S(a, n)$ denote the number of times the letter a appears in the first $n$ digits of the sequence $S$. We say that $S$ is a "\NewTerm{simple Universe number}\index{simple Universe number}" if the limit:
	
	for each $a$. 

	Now let $w$ be any finite string in $\sum^{*}$ and let $N_S(w, n)$ to be the number of times the string $w$ appears as a substring in the first $n$ digits of the sequence $S$ (for instance, if $S = 01010101$..., then $N_S(010, 8) = 3$). Then $S$ is a "\NewTerm{Universe number}\index{Universe number}" if, for all finite strings $w\in \sum^{*}$:
	
	$S$ is therefore a Universe number if all strings of equal length occur with equal asymptotic frequency. A given infinite sequence is either a Universe number or not, whereas a pure real number, having a different base-$b$ expansion for each integer $b\geq 2$, may be a Universe number in one base but not in another. A "\NewTerm{disjunctive sequence}\index{disjunctive sequence}" is a sequence in which every finite string appears. A Universe number sequence is a "\NewTerm{disjunctive sequence}\index{disjunctive sequence}" but a disjunctive sequence need obviously not be a Universe number.
	
	It is possible to prove (yet we don't wish not present this proof in a book on applied mathematics) with the "Universe number theorem" that almost all pure real numbers are Universe number. The set of non-Universe numbers, though "small" in the sense of being a null set, is "large" in the sense of being uncountable (for example o rational number is normal to any base, since the digit sequences of rational numbers are eventually periodic!). For instance, there are uncountable many numbers whose decimal expansion does not contain the digit $5$, and none of these are Universe number.
	
	\pagebreak
	\subsubsection{Abstract Numbers (variables)}
	\textbf{Definitions (\#\mydef):} A number may be considered as doing abstraction from the nature of the objects that constitute the group that it characterizes as well as how to codify it (Indian notation, Roman notation, etc.). We then say that the number is an "\NewTerm{abstract number}\index{abstract number}".  In other words, an abstract number, is a number that does not designate the quantity of any particular kind of thing.
	\begin{tcolorbox}[title=Remark,colframe=black,arc=10pt]
	Arbitrarily, the human being has adopted a numerical system mainly used in the World and represented by the symbols $0, 1, 2, 3, 4, 5, 7, 8, 9$ of the decimal system that will be supposedly known both in writing thant orally by the reader (language learning).
	\end{tcolorbox}
	For mathematicians, it is not advantageous to work with these symbols because they represent only specific cases. What seek theoretical physicists and mathematicians are "\NewTerm{literal relations}\index{literal relations}" applicable in a general case and that engineers can according to their needs change these abstract numbers by numeric values that correspond to the problem they need resolve.

These abstract numbers today commonly named "\NewTerm{variable}\index{variable}" or "\NewTerm{unknown}", used in the context of "\NewTerm{literal calculation}\index{literal calculation}" are very often represented since the $16$th century by:
	\begin{enumerate}
		\item The Latin alphabet:
		\begin{gather*}
			a,b,c,d,e,\ldots,x,y,z;A,B,C,D,E,\ldots, X,Y,Z
		\end{gather*}
		where the first lower case letters of the latin alphabet ($a, b, c, d, e ...$) are often used to represent an abstract constant, while the lowercase letters of the end of the latin alphabet ($...,x, y, z$) are used to represent entities (variables or unknowns) we seek the value.
		
		\item The Greek alphabet:
		
		which is particularly used to represent more or less complex mathematical operators (such as the index sum $\Sigma$, the indexed product $\Pi$, the variational $\delta$, the infinitesimal element $\varepsilon$, partial differential $\partial$, etc.) or variables in the field of physics (as $\omega$ for the pulsation, $\nu$ for the frequency, $\rho$ for the density, etc.).
		
		\item The modernized Hebrew alphabet (with less intensity...)
		
		As we have seen, a transfinite cardinal for example is denoted by the letter "aleph": $\mathcal{N}_0$.
	\end{enumerate}
	Although these symbols can represent any number there are some who can represent physical constants also named "\NewTerm{Universal constant}\index{universal physical constant}" as the speed of light $c$, the gravitational constant $G$, the Planck constant $h$, the number $\pi$, etc.
	
	We use very often still other symbols that we will introduce and define when reading this book.
	\begin{tcolorbox}[title=Remark,colframe=black,arc=10pt]
	The letters to represent numbers had been used for the first time by Vieta in the $16$th century.
	\end{tcolorbox}
	
	\paragraph{Domain of a Variable}\mbox{}\\\\
	A variable is therefore likely to take different numerical values. All these values can vary according to the character of the problem considered. 
	
	Given two numbers $a$ and $b$ such that $a<b$, then:
	\textbf{Definitions (\#\mydef):}
	\begin{enumerate}
		\item[D1.] We name "\NewTerm{domain of definition}\index{domain of definition}" of a variable, all numerical values it is likely to take between two specified limits (endpoints) or on a set (like $\mathbb{N}, \mathbb{R},\mathbb{R}^+,$ etc.).
		
		\item[D2.] We name "\NewTerm{closed interval with endpoints $a$ and $b$}\index{closed interval}", the set of all numbers $x$ between these two values and we denote as example as follows:
		
		The left notation is named obviously "\NewTerm{interval notation}\index{interval notation}", the right one is named "\NewTerm{setbuilder notation}\index{setbuilder notation}".
		
		\item[D3.] We name "\NewTerm{open interval with endpoints $a$ and $b$}\index{open interval}", the set of all numbers $x$ between these two values not included and we denote it as example as follows:
		
		
		\item[D4.] We name "\NewTerm{interval closed, left open right}\index{semi-interval}" or "\NewTerm{semi-closed left}" the following relation as example:
		
		
		\item[D5.] We name "\NewTerm{interval open left, closed right}\index{semi-interval}" or "\NewTerm{semi-closed right}" the following relation as example:
		
	\end{enumerate}
	Or in a summary and imaged  form and as often denoted in Switzerland:
	
	and according to the international norm ISO 80000-2: 2009 (since Switzerland has the art not respecting international norms and standards):
	

	\begin{tcolorbox}[title=Remarks,colframe=black,arc=10pt]
	\textbf{R1.} The notation $\{x\text{ such that} a<x<b\}$ denotes the set of real numbers $x$ strictly greater than $x$ and strictly less than $b$.\\
	
	\textbf{R2.} To fact that an interval is for example opened on $b$ means that the real number $b$ is not part thereof. By cons, if it had been closed then $b$ would be part of it.\\
	
	\textbf{R3.} If the variable $x$ can take all possible negative and positive values we write therefore: $\left] -\infty,+\infty \right[$ where the symbol "$\infty$" means "infinite". Obviously there can be combinations of open infinite right intervals with left endpoint and vice versa.\\
	
	\textbf{R4.} We will recall some of these concepts with a different approach when studying Algebra (literal calculation).
	\end{tcolorbox}	

	We say that the variable $x$ is an "\NewTerm{ordered variable}\index{ordered variable}" if by representing its domain of definition by a horizontal axis where each point on the axis represents a value of $x$, then for each pair of values, we can say that that there is an "\NewTerm{antecedent}\index{antecedent}" and one that is a "\NewTerm{subsequent}\index{subsequent}". Here the notion of antecedent and subsequent is not related to the concept of time it expresses just how the values of the variable are ordered.
	
	\textbf{Definitions (\#\mydef):}
	\begin{enumerate}
		\item[D1.] A variable is said to be "\NewTerm{increasing}\index{increasing variable}" if each subsequent value is greater than each antecedent value.

		\item[D2.] A variable is said to be "\NewTerm{decreasing}\index{decreasing variable}" if each subsequent value is smaller than each antecedent value.

		\item[D3.] The increasing and decreasing variables are named "\NewTerm{variables with monotonic variations}\index{monotonic variable}" or simply "\NewTerm{monotonic variables}".
	\end{enumerate}

	
	\begin{flushright}
	\begin{tabular}{l c}
	\circled{90} & \pbox{20cm}{\score{4}{5} \\ {\tiny 31 votes, 69.68\%}} 
	\end{tabular} 
	\end{flushright}
	
	%to make section start on odd page
	\newpage
	\thispagestyle{empty}
	\mbox{}
	\section{Arithmetic Operators}
	Talking about numbers like we did in the previous section naturally leads us to consider the operations of calculus. It is therefore logic that we make a non-exhaustive description of the operations that may exist between the numbers. This will be the goal of this section.
	
	We will consider in this book that there are two types of key tools in Arithmetics (we do not speak of Algebra but Arithmetic!):
	
	\begin{itemize}
		\item Arithmetic operators:
		
		There are two basic operators (addition "$+$" and subtraction "$-$") from which we can build other operators: the "multiplication "  (whose contemporary symbol $\times$ was introduced in 1574 by William Oughtred) and the "division" (whose old symbol was "$\div$" but since the end of the 20th century we use simple the slash $/$ symbol).
		
		These four operators ($+$, $-$, $\times$, $/$) are commonly named "\NewTerm{rational operators}\index{rational operators}". We will see them more in details after setting the binary relations.
		
		\begin{tcolorbox}[title=Remark,colframe=black,arc=10pt]
		Rigorously addition could be enough if we consider the common set of real number $\mathbb{R}$ because therefore the subtraction is only the addition of a negative number.
		\end{tcolorbox}
	
		\item Binary operators (relations):
		
		There are six basic binary relations (equal $=$, different $\neq$, greater than $>$, less than $<$, greater or equal $\geq$, less than or equal $\leq$) that compare the order of amplitude of elements that are on the left and on the right of these relations (thus at the number of two, hence the name "binary") in order to draw some conclusions. The majority of binary relations symbols were introduced by Vieta and Harriot in the 16th century..
	\end{itemize}

	It is obviously essential to know as best a possible these tools and their properties before going through into more strenuous calculations.
	\begin{figure}[H]
		\centering
		\includegraphics[scale=0.4]{img/arithmetics/operators.jpg}
	\end{figure}
	
	\subsection{Binary Relations}
	
	\textbf{Definitions (\#\mydef):}
	\begin{enumerate}
		\item[D1.] Consider two non-empty sets $E$ and $F$ (\SeeChapter{see section Set Theory}) not necessarily identical. If to some given elements $x$ of $A$ we can associate with a precise mathematical rule $R$ (unambiguous) one element $y$ of $F$, we define therefore a "\NewTerm{functional relation}\index{functional relation}" that maps $E$ to $F$ and that we write:
		
		Thus, more generally, a functional relation $R$ can be defined as a mathematical rule that associates to given components $x$ of $E$, some given elements $y$ of $F$.
		
		So, in this more general context, if $xRy$, we say that there $y$ is an "\NewTerm{image}\index{image}" of $x$ through $R$ and that $x$ is a "\NewTerm{precedent}" or "\NewTerm{preimage}\index{preimage}" of $y$.
		
		The set of pairs $(x, y)$ such that $xRy$ is a true statement generates a "graph" or "representation" of the relation $R$. We can represent these couples in a proper chosen way to make a graphical representation of the relation $R$.
		
		This is a type of relation on which we will come back in the section Functional Analysis under the form: $R:f(x)=y$of and that does not interest us directly in this section.
		
		\item[D2.] Consider a non-empty set $E$, if we associate with this set (and only to this one!) tools to compare its items between them when we talk about a "\NewTerm{binary relation}\index{binary relation}" or "\NewTerm{comparison relation}\index{comparison relation}" and that we write for any element $x$ and $y$ of $A$:
		
		These relations can also most of time be presented graphically. In the case of conventional binary operators comparison where $A$ is the set of natural numbers $\mathbb{N}$, relative $\mathbb{Z}$, rationals $\mathbb{Q}$ or real $\mathbb{R}$, that is graphically represented by a horizontal line (typically...); in the case of congruence (\SeeChapter{see section Number Theory}) it is represented by lines in the plane whose points are given by the constraint of congruence.
	\end{enumerate}
	
	\subsubsection{Equalities}
	It is  difficult to define the term "equality" in a general case applicable to any situation. For our part, we will allow ourselves for this definition to take the inspiration of the extensionality theorem of Set Theory (discussed later in another section).
	
	\textbf{Definitions (\#\mydef):}
	\begin{enumerate}
		\item[D1.] Two elements are "\NewTerm{equal}\index{equal}" if and only if they have the same values. The strict equality is described by the symbol $=$ that therefore means "equal to" (this symbol was introduced in 1557 by Robert Rocorde).
		
		If we have $a=b$ and $c$ is any given number (or vector/matrix) and $\star$ any operation (such as addition, subtraction, multiplication or division) then:
		
		This property is used to solve or simplify any type of equations. In practice, the abbreviation "LHS" is informal shorthand for the left-hand side of an equality. Similarly, "RHS" is the right-hand side abbreviation of that latter . 
		
		Obviously we have (property of reflexivity):
		
		And also (property of transitivity):
		\begin{gather*}
			\begin{rcases*}
			a=c \\
			b=c
			\end{rcases*} a=b
		\end{gather*}
		
		
		We will not enumerate the other properties of the equaliy in the section (for more details see the section Set Theory).
		
		\item[D2.] If two elements are not strictly equal, that is to say "\NewTerm{inequal}\index{inequal}"..., we are connecting them by the symbol $\neq$ and we say they are "not equal".
		
		If we have $a>b$ or $a<b$ then:
		
	\end{enumerate}
	There are still other equality symbols, which are an extension of two we have defined previously. Unfortunately, they are often misused (we could say rather that they are used in the wrong places) in most of the books available on the market (and this book is not an exception):
	\begin{enumerate}
		\item $\cong$: Should be used for congruence but in fact is mostly used to indicate an approxmation.
		
		\item $\approx$: Should be used for approximations but in fact $\cong$ is used instead.
		
		\item $\equiv$: Should be used to say that two elements are equivalent but in practice most people use $=$.
		
		\item $:=$: Is used to say that one element is by definition equal to another one.
		
		\item $\doteq$: Should be used to say "equal by definition to" but in fact most people use instead $:=$.
		
		\item $\sim$: Is used most of time in Statistics to say "follows the law..." but some practitioners use instead $=$ or to say "asymptotically equal".
	\end{enumerate}
	
	\subsubsection{Comparators}
	The comparators are tools that allow us to compare and order any pair of numbers (and also Sets!).
	
	The possibility of order numbers is fundamental in mathematics. Otherwise (if it was not possible to order), there would be a lot of things that would shock our habits, for example (some of the concepts presented in the following sentence have not yet been presented but we would still make reference to them): no more monotonic functions (especially sequences) and linked to it the derivation would therefore indicate nothing more about the "variation direction", no more approach of roots of polynomial by dichotomy (classical  research algorithm in an ordered set that split in two at each iteration), no more segments in geometry, no more than half space, no more convexity, we can not oriented space anymore, etc. It is therefore important to be able to order things as you can see...!
	
	Thus, for any $a,b,c\in \mathbb{R}$ we write when $a$ is greater than or equal to $b$:
	
	and when $a$ is less than or equal to $b$:
	
	\begin{tcolorbox}[title=Remark,colframe=black,arc=10pt]
	It is useful to recall that the set of real numbers $\mathbb{R}$ is a totally ordered group (\SeeChapter{see section Set Theory}), otherwise we could not establish order relations among its elements (which is not the case for complex numbers $\mathbb{C}$ that we can not order!).
	\end{tcolorbox}
	
	\textbf{Definition (\#\mydef):} The symbol $\leq$ is an "\NewTerm{relation order}\index{relation order}" (see the rigorous definition further below!) which means "\NewTerm{less than or equal to}" and conversely the symbol $\geq$ is also an order relation that means "\NewTerm{greater than or equal to}\index{greater than or equal to}".
	
	We also have relatively to the strict comparison the following properties that are relatively intuitive:
	
	and:
	
	if:
	
	if:
	
	and vice versa:
	
	We also have:
	
	and vice versa:
	
	We can obviously multiply, divide, add or subtract a term from each side of the relation as it is always true. Notice, however, that if you multiply both sides by a negative number it will obviously change as the comparator such that:
	
	and vice versa:
	
	We also have:
	
	Consider now that $b<a<0$ and $p\in \mathbb{N}^{*}$. Then if $p$ is an even integer:
	
	else if $p$ is odd:
	
	This result simply comes from the multiplication of signs rule since the power when not fractional is only a multiplication.

	Finally:
	
	The relations
	
	thus correspond respectively to: (strictly) greater than, (strictly) smaller than, smaller or equal, greater or equal, much bigger than, much smaller than.
	
	These relations can be defined in a little more subtle and rigorous way and apply not only to comparators (see for example the congruence relation in the section of Set Theory)!
	
	Let us see this (the vocabulary that follows is also defined in the section of Set Theory):
	
	\textbf{Definition (\#\mydef):} Given a binary relation $R$ of a set $A$ to itself, a relation $R$ on $A$ is a subset of the cartesian product $R\subseteq A\times A$ (that is to say, the binary relation generates a subset by the constraints it imposes on the elements of $A$ satisfying the relation) with the property of being:
	\begin{enumerate}
		\item[P1.] A "\NewTerm{reflexive relation}\index{reflexive relation}" if $\forall x \in A$:
		
		
		\item[P2.] A "\NewTerm{symmetrical relation}\index{symmetrical relation}" if $\forall x,y \in A$:
		
		
		\item[P3.] An "\NewTerm{anti-symmetrical relation}\index{anti-symmetrical relation}" if $\forall x,y \in A$:
		
		
		\item[P4.] A "\NewTerm{transitive relation}\index{transitive relation}" if $\forall x,y,z \in A$:
		
		
		\item[P5.] An "\NewTerm{connex relation}\index{connex relation}" if $\forall x,y \in A$:
		
	\end{enumerate}
	Mathematicians have given special names to the families of relations satisfying some of these properties.
	
	\textbf{Definitions (\#\mydef):}
	\begin{enumerate}
		\item[D1.] A relation is named "\NewTerm{strict order relation}\index{strict order relation}" if and only if it is only transitive (some specify then that it is necessarily antireflexive but this last fact is then obvious...).
		
		\item[D2.] A relation is named a "\NewTerm{preorder}\index{preorder}" if and only if it is reflexive and transitive.
		
		\item[D3.] A relation is named an "\NewTerm{equivalence relation}\index{equivalence relation}" if and only if it is reflexive, symmetric, and transitive.
		
		\item[D4.] A relation is named "\NewTerm{order relation}\index{order relation}" if and only if it is reflexive, transitive and antisymmetric (thus the relations $>, <$ are not order relations because obviously not reflexive relations).
		
		\item[D5.] A relation is named "\NewTerm{total order relation}\index{total order relation}" if and only if it is reflexive, transitive, connex and antisymmetric.
	\end{enumerate}
	For the other combinations it seems (as far as we know) that there are no special name among the mathematicians ...

	\begin{tcolorbox}[title=Remark,colframe=black,arc=10pt]
	The binary relations have all similar properties in natural sets $\mathbb{N}$, relative $\mathbb{Z}$,rational $\mathbb{Q}$ and real $\mathbb{R}$ (there is no natural order relation on the set of complex numbers $\mathbb{C}$).
	\end{tcolorbox}
	If we summarize:
	
	Thus we see that the binary relations $\leq, \geq$ form with the previously mentioned sets, total order relations and it is very easy to see which binary relations are partial, total or equivalence order relations.
	
	\textbf{Definition (\#\mydef):} If $R$ is an equivalence relation on $A$. For $\forall x\in A$, the "\NewTerm{equivalence class}\index{equivalence class}" of $x$ is by definition the set:
	
	$[x]$ is therefore a subset of $A$ ($x \subseteq A$) which we denote also thereafter ... $R$ (so be careful not to confuse in what follows the equivalence relation and the subset itself...).
	
	We thus have a new set that is named the "\NewTerm{set of equivalence classes}\index{set of equivalence classes}" or "\NewTerm{quotient set}\index{quotient set}" denoted in this book by $A / R$. So:
	
	You should know that in $A/R$ we do not look anymore at $[x]$ as a subset of $A$, but as an element!
	
	An relation of equivalence, presented in a popularized manner... thus serves to stick one unique label to items that satisfy the same property, and to confuse them with the said label (knowing what we do with this label).
	
	\begin{tcolorbox}[colframe=black,colback=white,sharp corners]
	\textbf{{\Large \ding{45}}Example:}\\\\
	In the set of integers $\mathbb{Z}$, if we study the remains of the division of number by $2$, we have that the result is always $0$ or $1$.\\
	
	The zero equivalence class is then named the "set of even integers numbers", the one equivalence class is therefore named the "set of odd integers". So we have two classes of equivalence for two partitions of $\mathbb{Z}$ (always keep in mind this simple example for theoretical elements that follow it helps a lot!).\\
	
	If we name the first $0$ and the second $1$, we fall back on the operation rules between odd and even numbers:
	
	which respectively means that the sum of two even integers is even, that the sum of an even and an odd integer is odd and that the sum of two odd integer is even.\\
	
	And for the multiplication:
	
	which respectively means that the two product of two even integer is even, the product of an even and an odd integer is even and that the product of two odd integer is odd.\\
	
	Now, to verify that we are dealing with an equivalence relation, we should still check that it is reflexive ($xRx$), symmetrical (if $xRy$ then $yRx$) and transitive (if $xRy$ and $yRz$ then $xRz$). We will see how to check it a few paragraphs further below because this example is a very special case of congruence relation.
	\end{tcolorbox}
	
	\textbf{Definition (\#\mydef):} The application $f:A\mapsto A/R$ defined by $x\mapsto [x]$ is named "\NewTerm{canonical projection}\index{canonical proejction}". Any element $z\in [x]$ is therefore named  "\NewTerm{class representative}\index{class representative}" of $[x]$.
	\begin{theorem}
	Now consider a set $E$. Then we propose to proved that there is correspondence between the set of equivalence relations on $E$ and all partitions of $E$. In other words, this theorem says that an equivalence relation on $E$ is nothing more but a partition on $E$ (this is intuitive).
	\end{theorem}
	\begin{dem}
	Let $R$ be an equivalence relation on $E$. We choose $I=E/R$ as set partition indexing and all we ask for any $[x]\in E/R$, $E_{[x]}=[x]$.
	
	We just have to check the following two properties of the definition of partitions to show that the family $\left(E_{[x]}\right)_I$ is a partition of $E$:
	
	\begin{enumerate}
		\item[P1.] Given $[x],[y]\in E/R$ such that $[x]\neq [y]$ then (obvious) $E_{[x]}\cap E_{[y]}=\varnothing$.
		
		\item[P2.] $E=\displaystyle\bigcup_{[x]\in E/R}$ is obvious because if $x\in E$ then $x\in [x]=E_{[x]}$.
	\end{enumerate}
	\begin{flushright}
		$\square$  Q.E.D.
	\end{flushright}
	\end{dem}
	Again, it should by easy to check with the practical example of the division by $2$ given previously that the partition of even and odd numbers satisfies these two properties (if not reader can contact us we will add this as an example).
	
	We have therefore associated to the equivalence relation $R$ a partition $E$. Conversely, if $(E_i)_I$ is a partition of $E$ then we almost easily verify that the relation $R$ is defined by $xRy$ if and only if there exists $i \in I$ such as $x,y \in E_i$ is an equivalence relation! Both applications are thus bijective and the inverses of each other.
	
	\begin{tcolorbox}[colframe=black,colback=white,sharp corners]
	\textbf{{\Large \ding{45}}Example:}\\\\
	We will now apply an example a little less trivial than the last we have seen to the construction of rings $\mathbb{Z}/\mathbb{Z}$ after a few reminders equation (for the concept of ring see the section Set Theory).\\
	
	Reminders:
	\begin{enumerate}
		\item Given two numbers $n,m\in \mathbb{Z}$. We say that "\NewTerm{$n$ divides $m$}\index{divide}" and we write $n|m$ if and only if there exists an integer $k\in \mathbb{Z}$ such as $m=kn$ (\SeeChapter{see section Numbers Theory}).
		
		\item Given $d\geq 1$ is an integer. We define the relation $R$ by $nRm$ if and only if $d|(n-m)$ or in other words $nRm$ if and only if there exists $d\in\mathbb{Z}$ such that $n=m+kd$. Usually we write this $n\equiv m\; (\text{modulo } d)$ instead of $nRm$  and we say that "\NewTerm{$n$ is congruent to $m$ modulo $d$}\index{congruent}". Remember also that $n\equiv 0\; (\text{modulo } d)$ if and only if $d$ divides $n$ (\SeeChapter{see section Numbers Theory}).
	\end{enumerate}
	We will now introduce an equivalence relation on $\mathbb{Z}$. Let us prove that for any integer $d\geq 1$, the congruence modulo $d$ is an equivalence relation on $\mathbb{Z}$ (we have already proved this in the section of Number Theory in our study of congruence but let us redo this work for the fun...).\\
	
	To prove this we simply have to control the three properties of the equivalence relation:
	\begin{enumerate}
		\item[P1.] Reflexivity: $n\equiv n$ since $n=n+0d$.
		
		\item[P2.] Symmetry: If $n \equiv m$ then $n=m+kd$ and therefore $m=n+(-k)d$ that is to say $m\equiv n$.
		
		\item[P3.] Transitivity: If $n\equiv m$ and then $mj$ then $n=m+kd$ and $m=j+k'd$ therefore $n=j+(k+k')d$ that is to say $n\equiv j$.
	\end{enumerate}
	In the above situation, we denote by $\mathbb{Z}/d\mathbb{Z}$ the set of equivalence classes and we will deonte by $[n]_d$ the equivalence class of congruence of a given integer $n$ given by:
	
	(each difference of two values in the braces is divisible by $d$ and this is therefore an equivalence class), thus:
	
	In particular (trivial since we obtain thus the all $\mathbb{Z}$):
	
	\end{tcolorbox}
	
	\begin{tcolorbox}[title=Remark,colframe=black,arc=10pt]
	The operations of addition and multiplication on $\mathbb{Z}$ define also the operation of addition and multiplication on $\mathbb{Z}/d\mathbb{Z}$. Then we say that these operations are compatible with the equivalence relation and then form a ring (\SeeChapter{see section Set Theory}).
	\end{tcolorbox}
	
	\pagebreak
	\subsection{Fundamental Arithmetic Laws}
	As we have said before, there is a fundamental operator (addition) from which we can define multiplication, subtraction (provided that the chosen Numbers Set is adapted to it....) and division (provided that the chosen Numbers Set is also adapted to it....) and around which we can build the entire Analytical Mathematics.
	
	Obviously there are some subtleties to be considered when the level of rigour increase. The reader can then refer to the section of Set Theory where fundamental laws are redefined more accurately than what will follow.
	
	\begin{figure}[H]
		\centering
		\includegraphics{img/arithmetics/fundamentals_delucq.jpg}
	\end{figure}
	
	\subsubsection{Addition}
	\textbf{Definition (\#\mydef):} The addition of integers is an operation denoted "$+$" which has for only purpose to bring together in one number all the units contained in several others. The result of the operation is named the "\NewTerm{sum"}\index{sum}, the "\NewTerm{total}" or "\NewTerm{cumul}". The numbers to be added are named therefore "\NewTerm{terms of the addition}\index{terms of the addition}".
	
	\begin{tcolorbox}[title=Remark,colframe=black,arc=10pt]
	The signs of addition "$+$" and subtraction "$-$" are due to the German mathematician Johannes Widmann (1489).
	\end{tcolorbox}
	Thus, $A + B + C ...$ are the terms of the addition and the result is the sum of the terms of the addition.
	
	Or in schematic form of a special case:
	\begin{gather*}
		0+4+3=4+3=7
	\end{gather*}
	\begin{figure}[H]
		\centering
		\includegraphics{img/arithmetics/addition.jpg}
		\caption{One possible schema for addition}
	\end{figure}
	Here is a list of some intuitive properties that we assume without proofs (as in fact they are axioms) of the operation of addition:
	\begin{enumerate}
		\item[P1.] The sum of several numbers do not depend on the order of terms. Then we say that the addition is a "\NewTerm{commutative operation}\index{commutative operation}". This means concretely for any two numbers:
		
		
		\item[P2.] The sum of several numbers does not change if we replace two or more of them by their intermediate result. Then we say that the addition is an "\NewTerm{associative operation}\index{associative operation}":
		
		
		\item[P3.] The Zero is the neutral element of addition because any number added to zero gives that number:
		
		
		\item[P4.] Depending on the set in which we work ($\mathbb{Z},\mathbb{Q},\mathbb{R},...$), the addition may include a term in such a way that a sum is zero. Then we say that there exists an "\NewTerm{opposite}\index{opposite}" to the sum such as:
		
	\end{enumerate}
	We have define more rigorously the addition using the Peano axioms in the particular case of all natural numbers $\mathbb{N}$ as we have already see in the section Numbers. So, with these axioms it is possible to prove that there exists one and only one application (uniqueness), denoted "$+$" of $\mathbb{N}\times \mathbb{N}$ in $\mathbb{N}$ satisfying:
	
	where $S$ means "successor".
	\begin{tcolorbox}[title=Remark,colframe=black,arc=10pt]
	As this book has not be written for mathematicians, we will pass the proof (relatively long and of little interest in the case of business) and we will assume that the application "$+$" exists and is unique ... and that it follows from the above properties.
	\end{tcolorbox}
	Let $x_1,x_2,...,x_n$ be any numbers then we can write the sum as following:
	
	by defining upper and lower bound to the indexed sum (below and above the upercase greek symbol Sigma).
	
	Here are some properties relatively to this condenses notation that should be obvious (if not the reader can send us a request we will add the details):
	
	where $k$ is a constant.
	
	Let us see now some concrete examples of additions  of various simple number in the purpose to practice the basis:
	\begin{tcolorbox}[colframe=black,colback=white,sharp corners]
	\textbf{{\Large \ding{45}}Examples:}\\\\
	The addition of two numbers relatively small is quite easy since we have learn by heart to count to a number resulting of the operation. Therefore (examples taken on decimal basis):
	
	and:
	
	and:
	
	For more bigger number we can adopt another method that human must also learn by heart. For example:
	
	The algorithm (process) is therefore the following: We add the columns ($4$ columns in this example) from right to left. For the first column we have therefore $4+5=9$ this gives:
	\end{tcolorbox}
	
	\begin{tcolorbox}[colframe=black,colback=white,sharp corners]
	
	and we continue like this for the second column where we have $4+7=11$ at the difference that now we have a number $>10$, then we report the first left digit on the next (left) column for the addition. Therefore:
	
	The third column we be calculated therefore as $1+2+4=7$ which give us:
	
	For the last column we have $9+3=12$ and once again we report the first digit from the left on the next column of the addition. Therefore:
	
	Finally:
	
	\end{tcolorbox}
	This example show how we can proceed for the addition of any real numbers: we do an addition column by column from the right to the left and if the result of one addition is greater than $10$, we report the left digit on the next (left) column.
	
	This algorithm (process or methodology) of addition is quite simple to understand and to execute. We will not go further on this subject add this day.
	
	\pagebreak
	\subsubsection{Subtraction}
	\textbf{Definition (\#\mydef):}  Subtraction is a mathematical operation that represents the operation of removing objects from a collection. More formally the subtraction of the number $A$ by the number $B$ denoted by the symbol "$-$" consist in founding the number $C$ such that added to $B$ gives $A$.

	\begin{tcolorbox}[title=Remark,colframe=black,arc=10pt]
	As we saw it in the section of Set Theory the subtraction in the set $\mathbb{N}$ could be possible only if $A>B$.
	\end{tcolorbox}
	
	Formally we write an inline literal subtraction in the form:
	
	That must satisfies:
	
	Or in schematic form of a special case:
	\begin{gather*}
		10-3-4=7-4=3
	\end{gather*}
	\begin{figure}[H]
		\centering
		\includegraphics{img/arithmetics/subtraction.jpg}
		\caption{One possible schema for subtraction}
	\end{figure}
	Here are some intuitive properties that we assume without proof for the subtraction operation (as it can be deduce from the addition...):
	\begin{enumerate}
		\item[P1.] The subtraction of several numbers depends on the order of the terms. We say when than subtraction is a "\NewTerm{non-commutative operation}\index{non-commutative operation}". Indeed:
		
		
		\item[P2.] The subtraction of several numbers change if we replace two or more of them by their intermediate result. We say when the subtraction is a "\NewTerm{non-associative operation}\index{non-associative operation}". Indeed:
		
		
		\item[P3.]The zero is not the neutral element of subtraction. Indeed, any number to which we subtract zero gives the same number, so zero is neutral on the right ... but not left because any number we subtract to zero does not give zero! We then say that the zero is only "\NewTerm{neutral on the right}\index{neutral on the right}" in the case of subtraction. Indeed:
		
	\end{enumerate}
	
	In most complicated cases we have a special vocabulary:
	
	The "\NewTerm{minuend}\index{minuend}" is $704$, the "\NewTerm{subtrahend}\index{subtrahend}" is $512$. The minuend digits are $m_3= 7$, $m_2 = 0$ and $m_1 = 4$. The subtrahend digits are $s_3 = 5$, $s_2 = 1$ and $s_1 = 2$. Beginning at the one's place, $4$ is not less than $2$ so the difference $2$ is written down in the result's one place. In the ten's place, $0$ is less than $1$, so the $0$ is increased by $10$, and the difference with $1$, which is $9$, is written down in the ten's place. The American method corrects for the increase of ten by reducing the digit in the minuend's hundreds place by one. That is, the $7$ is struck through and replaced by a $6$. The subtraction then proceeds in the hundreds place, where $6$ is not less than $5$, so the difference is written down in the result's hundred's place. We are now done, the result is $192$.
	
	Let us see now some concrete examples of additions  of various simple number in the purpose to practice the basis:
	\begin{tcolorbox}[colframe=black,colback=white,sharp corners]
	\textbf{{\Large \ding{45}}Example:}\\\\
	The subtraction of two relatively small numbers is pretty easy once we memorized to count to at least the number resulting from this operation. So:
	
	and:
	
	and:
	
	For larger numbers another possible method must be learned by heart (as well as for the addition). For example:
	
	\end{tcolorbox}
	
	\pagebreak
	\begin{tcolorbox}[colframe=black,colback=white,sharp corners]
	we subtract the columns ($4$ columns in this example) from right to left. In the first column we have $4-5=-1<0$ so we report $-1$ to the next column (second one) and we write $10-1=9$ below the horizontal line of the first column:
	
	and we continue as well for the second column $7-8=-1<0$ so that we report $-1$ on the next column (third one) and as $-1-1=-2$ we report $10-2=8$ below the horizontal bar of the second column:
	
	The third column is calculated as $5-7=-2<0$ and we report $-1$ on the next column (fourth one) and as $-1-2=-3$ we report $10-3=7$ below the line of the third column bar:
	
	In the last column we have $4-3=1>0$ therefore we report the nothing on the next column and as $1-1=0$ we report $0$ below the line of the fourth column bar:
	
	\end{tcolorbox}
	That's how we therefore we proceed to subtracting any numbers. We make a subtraction by column from the right to the left and if the result is a subtraction is less than zero we report $-1$ to the next column and the addition of the latest report on the subtraction obtained below the line.
	
	We have when we mix the addition and subtraction the following resulting relation that should be obvious for most readers:
	
	The methodology used for subtraction being based on exactly the same rules that for addition we will expand the subject more as this seems actually useless in our point of view. This method is very simple and of course requires some habits to work with  numbers to be fully understood and mastered.
	
	\subsubsection{Multiplication}
	\textbf{Definition (\#\mydef):} The multiplication of numbers is an operation that has for purpose, given two numbers, one named "\NewTerm{multiplier}\index{multiplier}" $m$, and the other "\NewTerm{multiplicand}\index{multiplicand}" $M$, to find a third number named "\NewTerm{product}\index{product}" $P$ that is the sum (multiplication is only a successive number of sums!) as many equal numbers to the multiplicand as there are units multiplier:
	
	The multiplicand and multiplier are named "\NewTerm{product factors}\index{product factors}".

	The multiplication is indicated in kindergarten by the symbol "$\times $" of of the elevated dot symbol in higher classes "$\cdot $" or even when there is no possible confusion... without anything:
	
	We can define the multiplication using the Peano axioms in the special case of natural numbers $\mathbb{N}$ as we have already mentioned in the sectionNumbers. Thus, with these axioms it is possible to prove that there is (exists) one and only one (unique) application, denoted "$\times$" or more often "$\cdot$" of of $\mathbb{N}^2$ to $\mathbb{N}$ satisfying:
	
	\begin{tcolorbox}[title=Remark,colframe=black,arc=10pt]
	As this book has not be written for mathematicians, we will pass the proof (relatively long and of little interest in the case of business) and we will assume that the application "$\times $" exists and is unique ... and that it follows from the above properties.
	\end{tcolorbox}
	The power is a specific notations of a special case of the multiplication. When to multiplicand(s) and the multiplier(s) are typically identical in numerical values, we denote therefore the multiplication by (for example):
	
	This is what we name the "\NewTerm{power notation}\index{power notation}" or "\NewTerm{exponentations}\index{exponentation}". The number in superscript is what we the name the "\NewTerm{power}\index{power}" or the "\NewTerm{exponant}\index{exponant}" of the number. The notation with exponants is said to be see for the first time in a book of Chuquet in 1484.
	
	You can check by yourself that is properties are the following (for example):
	
	and also:
	
	Here are some obvious properties about the multiplication that we will admit without proof (this is a Set properties point of view listing):
	\begin{enumerate}
		\item[P1.] The multiplication of several numbers does not depend on the order of terms. Then we say that multiplication is a "\NewTerm{commutative operation}\index{commutative operation}".
		
		\item[P2.] The multiplication of several numbers does not change if we replace two or more of them by their intermediate result. We then say that the multiplication is an "\NewTerm{associative operation}\index{associative operation}".
		
		\item[P3.] The unit is the neutral element of the multiplication as any multiplicand multiplied by the multiplier $1$ is equal to the multiplicand itself.
		
		\item[P4.] The multiplication may have a term such that the product is equal to unity (the neutral element). Then we say that there exists a "\NewTerm{multiplicative inverse}\index{multiplicative inverse}" (but this depends strictly speaking in what set of numbers we work as in some the concept of decimal number does not exist!).
		
		\item[P5.] Multiplication is a "\NewTerm{distributive operation}\index{distributive operation}", that is to say:	
		
		the reverse being named a "\NewTerm{factorization operation}\index{factorization operation}".
	\end{enumerate}
			Let us also introduce some special notations for the multiplication:
	\begin{enumerate}
		\item Given any numbers $x_1,x_2,...,x_n$ (not necessarily equal) then we can write the product as following:
		
		by defining upper and lower bounds to the indexed product (above and below the uppercase Greek letter "Pi").
		
		We trivially have respectively to the latter notation (on request we can detail more...):
		
		for any number $k$ such that:
		
		We also have for example:
		
		
		\item We define the "\NewTerm{factorial}\index{factorial}" simply ("simply"... because it exists also a more complex way of defining it through the Euler Gamma function as it is done in the section of Integral and Differential Calculus) by:
		
		with the special fact that (only the complex definition mentioned before can make this fact obvious...):
		
	\end{enumerate}
	Let us see some simple examples of basic multiplications:
	\begin{tcolorbox}[colframe=black,colback=white,sharp corners]
	\textbf{{\Large \ding{45}}Example:}\\\\
	E1. The multiplication of two relatively small numbers is fairly easy once we have memorized count to at least the number resulting from this operation. So:
	
	E2. For much larger numbers we must adopt another method that has to be memorized. \\
	
	For example:
	
	This methodology is very logical if you understand how we build a a number in base ten. Thus we have (we'll assume that the distributive property is mastered):
	
	To avoid overloading the notations in the multiplication by the "vertical" method, we do not represent the zeros that would overload unnecessarily the calculations (and even more if the multiplier and / or the multiplicand are very large numbers).
	\end{tcolorbox}
	
	\pagebreak
	\subsubsection{Division}
	\textbf{Definition (\#\mydef):} The division of integers (to start with the simplest case ...) is an operation, which aims, given two integers, one named "\NewTerm{dividend}\index{dividend}" $D$, the other named "\NewTerm{divider}\index{divider}" $d$ , to find a third number named "\NewTerm{quotient}\index{quotient}" $Q$ which is the largest number whose product by the divisor can be subtracted (so the division result of the subtraction!) the dividend (the difference being named the "\NewTerm{rest}\index{rest}" $R$ or sometimes the "\NewTerm{congruence}\index{congruence}").
	\begin{tcolorbox}[title=Remark,colframe=black,arc=10pt]
	In the case of real numbers there are never any rest at the end of the division operation (because the quotient multiplied by the divisor gives always exactly the dividend)!
	\end{tcolorbox}
	Generally in the context of integers (or algebraic equation division), if we denote by $D$ the dividend and by $d$ the divisor, the quotient $Q$ and the remainder $R$ we have the relation:
	
	knowing that the division was initially written as follows:
	
	We indicate the operation of division by placing between the two numbers, the dividend and the divider, a symbol "$:$" or a slash "$/$" or even in kindergarten with the symbol $\div$.
	
	We refer also often by the term "\NewTerm{fraction}\index{fraction}" (instead of "quotient"), the ratio of two numbers or in other words, the division of the first by the second.
	\begin{tcolorbox}[title=Remark,colframe=black,arc=10pt]
	The sign of division "$:$" is said to be due to Gottfried Wilhelm Leibniz. The slash symbol could have been see for the first time in the works of Leonardo Fibonacci  (1202) and is probably due to the Hindus.
	\end{tcolorbox}
	If we divide two numbers and we want an integer as quotient and as remainder (if there is one...), then we speak of "\NewTerm{euclidiean division}\index{euclidean division}".
	
	For example, dividing a cake, is not a Euclidean division because the quotient is not an integer, except if one takes the four quarters ...:
	\begin{figure}[H]
		\centering
		\includegraphics{img/arithmetics/division_cake.jpg}
		\caption{Schematic example of a division (fractions)}
	\end{figure}
	If we have:
	
	we name $i_D$ the inverse of the dividend. At any number is associated an inverse that satisfies this condition.
	From this definition it comes the notation (with $x$ being any number other than zero)
	
	In the case of two fractional numbers, we say they are "\NewTerm{inverse}\index{inverse}" or "\NewTerm{reciprocal}\index{reciprocal}", when their product is equal to unity (as the previous relations).
	\begin{tcolorbox}[title=Remarks,colframe=black,arc=10pt]
	\textbf{R1.} A division by zero is what we name a "\NewTerm{singularity}\index{singularity}". That is to say the result of the division is: undetermined!!\\
	
	\textbf{R2.} When we multiply the dividend and the divisor of a division (fraction) by a same number, the quotient does not change (this is an: "\NewTerm{equivalent fraction}\index{equivalent fraction}"), but the remainder is multiplied by that number.\\
	
	\textbf{R3.} Divide a number by a product made of several factors is equivalent to divide this number successively by each of the factors of the product and vice versa.\\
	
	\textbf{R4.} Fractions that are greater than $0$ but less than $1$ are named "\NewTerm{proper fractions}\index{proper fractions}". In proper fractions, the numerator is less than the denominator. When a fraction has a numerator that is greater than or equal to the denominator, the fraction is an "\NewTerm{improper fraction}\index{improper fraction}". An improper fraction is always $1$ or greater than $1$. And, finally, a mixed number is a combination of a whole number and a proper fraction.
	\end{tcolorbox}
	The properties of the divisions with the condensed power notations (exponentiation) are typically as example (we will leave to the reader the fact to check this up to with numerical values):
	
	or obviously another example:
	
	We therefore deduce that:
	
	Let us recall that a prime number (relative integer $\mathbb{Z}$) is a number greater than $1$ that has for divisors only itself and unity (remember that $2$ is prime for example). Therefore any number that is not prime has at least one prime number as a divisor (except $1$ by definition!). The smallest divisors of an integer is a prime number (we will detail the properties of prime numbers relatively to the operation of division in the section Numbers Theory).
	
	Let us see some properties of the division (some of us are already known because they arise from logical reasoning of the multiplication properties):
	
	where: 
	
	is what we we name a "\NewTerm{terms amplification}\index{terms amplification}" and: 
	
	is an operation consisting by putting everything with a "\NewTerm{common denominator}\index{common denominator}".
	
	We also have the following properties:
	\begin{enumerate}
		\item[P1.] The division of several numbers depend on the order of terms. We then say that the division is a "\NewTerm{non-commutative operation}". This means we have when $a$ that is different from $b$ and that both are different from zero:
		
		
		\item[P2.] The result of the division of several numbers change if we replace two or more of them by their intermediate result. We then say that the division is a "\NewTerm{non-associative operation}":
		
		
		\item[P3.] The unit is the neutral element has that we multiply the divident or the divider by $1$ the result of the division remains the same.
		
		
		\item[P4.] The division may include a divider in such a way that the division is equal to unity (neutral element $1$). We then say that there exist a "\NewTerm{symmetrical to the division}\index{division symmetrical}" that is obviously equal to the numerator (dividend) itself.
		
		\item[P5.] The incrementation of numerator and denominator by a constant value is not equal to the initial ratio in the general case where $a\neq b$:
		
	\end{enumerate}
	Now that we know the multiplication (and therefore power notation) and division, if we consider $a$ and $b$ are two positive real numbers, different from zero we have:
	
	and (named sometimes the "\NewTerm{zero exponent rule of exponents}\index{zero exponent rule of exponents}"):
	
	and:
	
	We have also obviously:
	
	Also:
	
	
	\pagebreak
	\paragraph{$n$-root}\mbox{}\\\\
	Now that we have introduce in a simple and not too much formal way the operations of multiplication (and power notation) and division we can introduction the concept of $n$-root.
	
	As we know for example that:
	\begin{gather*}
		2^32^2=2^{3+2}
	\end{gather*}
	we can by reverse inference for example also write:
	
	and therefore it means that fractional power exist! This is what we name $n$-root (in the above example we speak of $2$-root).
	
	We can now define the principal $n$-root of any number!
	
	\textbf{Definition (\#\mydef):} In mathematics, the nth root of a number $a$, where $n$ is a positive integer, is a number $r$ which, when raised to the power $n$ yields $x$. That is to say such that: $r^n=x$, where $n$ is the degree of the root. By convention we write:
	
	Roots are usually written using the "\NewTerm{radical}\index{radical}" symbol $\sqrt[n]{\ldots}$ or also named the "\NewTerm{radix}\index{radix}". The number $n\in\mathbb{N}$ is named the "\NewTerm{radicand}\index{radicand}" and sometimes the "\NewTerm{index}\index{index}".
	From what has been said for the powers, we can easily conclude that the $n$-th root of a product of several factors is the product of $n$-th roots of each factor:
	
	as (seen previously):
	
	And therefore:
	
	Obviously it comes:
	
	We also have if $a<0$:
	
	if $n\in \mathbb{N}^{*}$ is odd and:
	
	if $n\in \mathbb{N}^{*}$ is even.

	If $x<0$ and $n\in \mathbb{N}^{*}$ is odd then:
	
	is the number $y$ such that:
	
	If $n\in \mathbb{N}^{*}$ is even then obviously, as we already have seen it earlier, the root belong to $\mathbb{C}$ (\SeeChapter{see section Numbers}).
	
	If the denominator of a fraction contains a factor of the form $\sqrt[n]{a^k}$ with $a\neq 0$, by multiplying the numerator and denominator by $\sqrt[n]{a^{n-}}$, we will remove the root of the denominator, since:
	
	\begin{tcolorbox}[colframe=black,colback=white,sharp corners]
	\textbf{{\Large \ding{45}}Example:}\\\\
	Let us see a world famous example of the application of the root about the origin of the ISO paper formats: A6, A5, A4, A3, A2, A1, A0, etc.\\
	
	This format of paper has in fact the property (there is a goal at the origin!) to keep the proportions when we bend or cut the sheet in half in its largest dimension. Thus, if we denote by $L$ the length and $W$ the width of the sheet, we have:
	
	Hence we have:
	
	As the A0 format by definition has an area of $1\;[\text{m}^2]$. For this format we have then:
	
	Therefore we deduce that:
	
	and therefore:
	
	from whence we derive:
	
	\end{tcolorbox}
	
	\pagebreak
	\subsection{Arithmetic Polynomials}
	\textbf{Definition (\#\mydef):} An "\NewTerm{arithmetic polynomial}\index{arithmetic polynomial}" (not to be confused with "algebraic polynomial" that will be studied later in the section Algebra) is a set of numbers separated from each other by the operators of addition or subtraction ($+$ or $-$) including therefore the multiplication...
	
	The components enclosed in the polynomial are known as "\NewTerm{terms}" of the polynomial. When the polynomial contains a single term, then we speak of "\NewTerm{monomial}\index{monomial}", if there are two terms we speak of "\NewTerm{binomial}\index{binomial}", and so on...
	
	\begin{theorem}
	The value of an arithmetic polynomial is equal to the excess of the sum of the terms preceded by the $+$ sign on the sum of the terms preceded by the sign $-$.
	\end{theorem}
	\begin{dem}
	
	whatever the values of the terms.
	\begin{flushright}
		$\square$  Q.E.D.
	\end{flushright}
	\end{dem}
	Highlight the negative unit $-1$ is what we name, as we already know, a "\NewTerm{factorization}". The reverse operation is named as we also already know a "\NewTerm{distribution}" or "\NewTerm{development}".
	
	The product of several polynomials can always be replaced by a single polynomial that we name the... "\NewTerm{resulting product}\index{resulting product}". We usually operate as follows: we multiply successively all the terms of the first polynomial, starting from the left, with the first, the second, ..., the last by the second polynomial. We obtain a first partial product. We do, if necessary, a reduction (simplification) of similar terms. We then multiply each of the terms of the partial product successively by the first, the second, ..., the last term of the third polynomial starting from the left and so on.
	
	\begin{tcolorbox}[colframe=black,colback=white,sharp corners]
	\textbf{{\Large \ding{45}}Example:}\\\\
	
	\end{tcolorbox}
	The product of the polynomials $P_1$, $P_2$, $P_3$, ..., $P_k$, ... is the sum of all products of $n_i$ factors formed with a term of $P_i$, of a term of $P_2$, ..., and a term of $P_k$ and so. if there is no reduction, the number of terms is equal to the product of the numbers of terms of each polynomial such that the final number of therms is equals to:
	

	\subsection{Absolute Value}
	\textbf{Definition (\#\mydef):} In mathematics, the "\NewTerm{absolute value}\index{absolute value}"  $|x|$ of a real number $x$ is the non-negative value of $x$ without regard to its sign. Namely, $|x| = x$ for a positive $x$, $|x| = -x$ for a negative $x$ (in which case $-x$ is positive), and $|0| = 0$. For example, the absolute value of $3$ is $3$, and the absolute value of $-3$ is also $3$. The absolute value of a number may be thought of as its distance from zero.
	
	\begin{tcolorbox}[title=Remarks,colframe=black,arc=10pt]
	\textbf{R1.}  The term absolute value has been used in this sense from at least 1806. The notation $|x|$, with a vertical bar on each side, was introduced by Karl Weierstrass in 1841.\\
	
	\textbf{R2.} For plots about the absolute value the reader is referred to the Functional Analysis section of this book.
	\end{tcolorbox}
	For any real number $x$, the "absolute value" $x$, is formally given by:	
	
	At the origin the absolute value was defined as:
	
	
	We notice that also the following possible notation:
	
	And the equivalent expressions:
	
	and also:
	
	the latter being often used in the context of solving inequalities.
	
	\begin{tcolorbox}[colframe=black,colback=white,sharp corners]
	\textbf{{\Large \ding{45}}Example:}\\\\
	Solving an inequality such that: 
	
	is then solved simply by using the intuitive concept of distance. The solution is the set of real numbers whose distance from the real number $3$ is less than or equal to $9$. This is the range of center $3$ and radius $9$ or formally:
	
	\end{tcolorbox}
	
	Let us indicate that it is also useful to interpret the term: 
	
	as the (euclidean!) distance between the two numbers $x$ and $y$ on the real line. Thus, by providing the set of real numbers of the absolute value distance , it becomes a metric space (see the section of Topology to have a robust introduction to what is a distance)!!!
	
	The absolute value has some trivial properties that we will give without proof (excepted on reader request) as they seem to us quite intuitive:
	
	The absolute value has the following four fundamental properties:
	\begin{enumerate}
		\item[P1.] Non-negativity:
			
		
		\item[P2.] Positive-definiteness:
			
			
		\item[P3.] Multiplicativeness:
			

		\item[P4.] Subadditivity ("first" triangle inequality):
			
	\end{enumerate}
	
	Other important properties of the absolute value include:
	\begin{enumerate}
		\item[P5.] 	Idempotence (the absolute value of the absolute value is the absolute value):
			
		
		\item[P6.] 	Evenness (reflection symmetry of the graph):
			
			
		\item[P7.] Preservation of division (equivalent to multiplicativeness) if $y\neq 0$:
			

		\item[P8.] 	Reverse ("second") triangle inequality (equivalent to subadditivity):
			
	\end{enumerate}

	\pagebreak
	\subsection{Calculation Rules (operators priorities)}
	Frequently in computing (in development in particular), we speak of "\NewTerm{operators precedence}\index{operators precedence}". In mathematics we speak of "\NewTerm{priority of the sets of operations and rules of signs}\index{rules of signs}". What is this exactly?
	
	We have already seen what are the properties of addition, subtraction, multiplication, division and power. We therefore insist that the reader distinguishes the concept of "property" of this of "priority" (that we will immediately see) which are (obviously) two completely different things!
	
	In mathematics, in particular, we first define the priorities of the symbols $\left\lbrace\left[\left(\right)\right]\right\rbrace$:
	\begin{enumerate}
		\item Operations that are in brackets $()$ should be performed first in the polynomial.
	
		\item Operations that are in brackets $[\,]$ should be made afterwards from the results of operations that were in brackets $()$.
	
		\item Finally, from the intermediate results of operations that were in $()$ and brackets $[\,]$, we calculate the operations that are between the braces $\left\lbrace \right\rbrace$.
	\end{enumerate}	

	Let us do an example, this will be more telling.
	\begin{tcolorbox}[colframe=black,colback=white,sharp corners]
	\textbf{{\Large \ding{45}}Example:}\\\\
	Consider the calculation of the polynomial:
	
	According to the rules we defined earlier, we first calculate all the elements that are in parenthesis $()$, that is to say:
	
	Which give us:
	
	Always according to the rules we defined earlier, now we calculate all the elements between brackets by always starting to calculate the terms that are in brackets $[\,]$ at the lowest level of the other brackets $[\,]$. Thus, we first calculate the expression $[4+14\cdot 2]$  that is in the top-level bracket: $[5\cdot 10+3\cdot ...]$.\\
	
	This give us $[4+14\cdot 2]=32$ and therefore:
	
	\end{tcolorbox}
	
	\pagebreak
	\begin{tcolorbox}[colframe=black,colback=white,sharp corners]
	It remains to us to calculate now $[5\cdot 10+3\cdot 32]=146$ and therefore:
	
	We now calculate the single term in braces, which gives us:
	
	Finally it remains:
	
	Obviously this is a special case ... But the idea remains the same in general.
	\end{tcolorbox}
	The priority of arithmetic operators is a problem mainly related to computer languages (as we have already mentioned) because we can only write mathematical relation on a single line and this is many times as source of confusion for people not having technical skill.
	
	will be written (pretty much on most computer languages):
	
	A non initiated could read this in many ways:
	
	Thus it has logically be defined an order of prioritization of operators such that the operations are carried out in the following order:
	\begin{enumerate}
		\item $-$ Negation
		
		\item $\string^$ Power (exponentiation)

		\item $*$ Multiplication and $/$ division\footnote{Physical Review journals stat that in the submission instruction that multiplication is of higher precedence that division.}

		\item $\backslash$ Integer division (specific to computer science)
		\item $\mathrm{mod}$ Module (\SeeChapter{see section Number Theory})
		\item $+,-$ Addition and subtraction
	\end{enumerate}
	Obviously the rules of parentheses $()$, brackets $[\,]$, and braces $\left\lbrace \right\rbrace$ that were defined in mathematics apply also to computing.
	
	Thus we get in the order (we replace every transaction made with a symbol):
	
	First the terms in parentheses:
	
	
	\begin{enumerate}
		\item First the negation (rule 1):
		
		
		\item The power (rule 2):
		
		
		\item We apply the multiplication (rule 3):
		
		
		\item And we apply division (rule 3 again):
		
		The rules (4) and (5) does not apply to this particular example.
		\item And Finally (rule 6):
		
	\end{enumerate}
	Thus, following these rules, neither a computer nor a human can (should) be wrong in interpreting an equation written on a single line to avoid such issues (hence the important to have ISO standards in all field of the industry and administration):
	\begin{figure}[H]
		\centering
		\includegraphics{img/arithmetics/operators_priorities_calculators.jpg}
	\end{figure}
	\begin{tcolorbox}[title=Remark,colframe=black,arc=10pt]
	Mnemonics are often used to help students in highs-school to remember the most basic rules, but the rules taught by the use of acronyms can be misleading. In the United States, the acronym "\NewTerm{PEMDAS}\index{PEMDAS}" is common. It stands for Parentheses, Exponents, Multiplication, Division, Addition, Subtraction. PEMDAS is often expanded to the mnemonic "Please Excuse My Dear Aunt Sally". Canada and New Zealand use BEDMAS, standing for Brackets, Exponents, Division, Multiplication, Addition, Subtraction. Most common in the UK, India and Australia[11] are BODMAS meaning "B"rackets, "O"f or "O"rder, "D"ivision, "M"ultiplication, "A"ddition and "S"ubtraction. Nigeria and some other West African countries use BIDMAS.
	\end{tcolorbox}
	
	
	In computer code, however, there are several operators that we do not always find in pure mathematics and which order property frequently change depending from a computer language to another. We will not dwell too much on that stuff as it is almost without end, however, we have below a small description:
	\begin{itemize}
		\item The concatenation operator "\&" is evaluated before comparisons operators.
		
		\item Comparison operators ($=, <,>, ...$) all have equal priority.
	\end{itemize}
	However, the leftmost operator in an expression, hold a higher priority.
	
	The logical operators are evaluated in the following order of priority in most computing languages:
	\begin{enumerate}
		\item Not ($\neg$)
		\item And ($\wedge$)
		\item Or ($\vee$)
		\item Xor ($\oplus$)
		\item Eqv ($\Leftrightarrow$)
		\item Imp ($\Rightarrow$)
	\end{enumerate}
	Now that we have seen the operator priorities, what are the rules about signs applicable in mathematics and computing science?

	First, you must know that these latter rules only apply in the case of multiplication and division. Given two positive numbers $(+x),(+y)$. We have:
	
	In other words, the multiplication of two positive numbers is a positive number and this can be generalized to the multiplication of $n$ positive numbers.
	
	We have:
	
	In other words, the multiplication of a positive number to a negative number is negative. Which can be generalized: to a positive result of a multiplication if there is an even number of negative numbers, and a negative result if there is an odd number of negative numbers on all $n$ numbers included in the multiplication.

	We have:
	
	In other words, multiplying two negative numbers is positive. What can be generalized: to a positive result of the multiplication if there is an even number of negative numbers and a negative result if there is an odd number of negative numbers.

	About divisions, the reasoning is the same:
	
	In other words, if the numerator and denominator are positive, then the result of the division will be positive.

	We have:
	
	In other words, if either the numerator or denominator is negative, then the result of the division will be necessarily negative.
	
	We have:
	
	In other words, if the numerator and denominator are positive, then the result of the division, will necessarily be positive.
	
	Obviously if we have a subtraction of terms, it is possible to rewrite it in the form:
	 
	
	\begin{flushright}
	\begin{tabular}{l c}
	\circled{80} & \pbox{20cm}{\score{4}{5} \\ {\tiny 11 votes, 76.36\%}} 
	\end{tabular} 
	\end{flushright}
	
	
	%to make section start on odd page
	\newpage
	\thispagestyle{empty}
	\mbox{}
	\section{Number Theory}
	\lettrine[lines=4]{\color{BrickRed}T}raditionally, number theory is a branch of mathematics that deals with properties of integers, whether natural or whole integers. More generally, the field of study of this theory concerns a broad class of problems that naturally come from the study of integers. Number theory can be divided into several branches of study (algebraic number theory, computational number theory, etc.) depending on the methods used and the issues addressed.
	
	\begin{tcolorbox}[title=Remark,colframe=black,arc=10pt]
	The sign of the cross "$\times$" for multiplication is said to be for the first time in the book of Oughtred (1631), about the halfway point (modern notation for multiplication), we ought it to Leibniz. From 1544, Stiefel, in one of his books did not employ any sign and designated the product of two numbers by placing them next to each other.
	\end{tcolorbox}
	We chose to introduce in this section only the subjects that are essential to the study of mathematics and theoretical physics of this book as well as those to be absolutely part of the general culture of the engineer (some results have application in Biostatistics!).
		
	\subsection{Principle of good order}
	We will take for granted the principle that says that every nonempty set $S \subset \mathbb{N}$ contains a smaller element.
	
	We can use this theorem to prove an important property of numbers named " \NewTerm{Archimedean property}\index{Archimedean property}" or "\NewTerm{Archimedes' axiom}\index{Archimedes' axiom}" which states:
	
	For $\forall a,b \in \mathbb{N}$ where $a$ is non-zero, there is at least one positive integer $n$ such that:
	
	In other words, for two unequal values, there is always an integer multiple of the smallest, bigger than the larger one. We name "\NewTerm{Archimedean}\index{Archimedean integer}" structures whose elements satisfy a comparison property (\SeeChapter{see section Set Theory}).
	
	While this is trivial to understand in the case of integers let us prove it because it allows us to see the type of approaches used by mathematicians when they must prove trivial items like this...
	
	\begin{dem}
	Let us suppose the opposite by saying that for $\forall n \in \mathbb{N}$ we have:
	
	If we can prove that it is absurd for any $n$ then we will have prove the Archimedean property (and also if $a, b$ are real).
	
	Let us consider then the set:
	
	Using the principle of good order, we deduce that there exist $s_0  \in S$ such as $s_0 \leq s$ for all $s \in S$. Let us write that this smaller element is:
	
	and therefore we also have:
	
	As by hypothesis $na\leq b$ then we must have:
	
	and if we reorganize and simplify:
	
	and that we simplify the negative sign we had to get...:
	
	 an obvious contradiction!
	
	This contradiction leads that the initial assumption as $na < b$ for all $n$ then is false and therefore the Archimedean property is proved by the absurd.
	\begin{flushright}
		$\square$  Q.E.D.
	\end{flushright}
	\end{dem}
	
	\subsection{Induction Principle}
	Let $S$ be a set of natural numbers that has the following two properties:
	\begin{enumerate}
		\item[P1.] $1\in S$

		\item[P2.] If $k\in S$, then $k+1\in S$
	\end{enumerate}
	then:
	
	We are build like this the set of natural numbers (refer to the section Set Theory to see the rigorous construction of the set of naturla number with the Zermelo-Fraenkel axioms).
	
	\begin{theorem}
	Given now:
	
	the symbol "$\setminus$" meaning for recall "excluding". We want to prove that:
	
	\end{theorem}
	Again, even if it is trivial to understand, let us do the proof because it allows us to see the type of approaches used by mathematicians when they must prove trivial stuff like this...
	\begin{dem}
	Let us suppose the opposite, that is to say:
	
	By the principle of good order, since $B\subset \mathbb{N}$, $B$ must have a smallest element which we will denote by $b_0$.

	But since $1\in S$ by the property (P1), we have that $b_0>1$ and of course also that $1\in B$, that is to say also $b_0-1\in S$. By using the property (P2), we finally have that $b_0\in S$, that is to say that $b_0\not\in B$, therefore we get a contradiction.
	\begin{flushright}
		$\square$  Q.E.D.
	\end{flushright}
	\end{dem}
	\begin{tcolorbox}[colframe=black,colback=white,sharp corners]
	\textbf{{\Large \ding{45}}Example:}\\\\
	We want to show thanks to the induction principle, that the sum of the first $n$ square equals $n(n+1)(2n+1)/6$, that is to say for $n\geq 1$, we would have to (\SeeChapter{see section Sequences and Series}):
	
	First the above relation is easily verified for $n=1$ we will show that $n=k+1$ also verifies that relation. Under the induction hypothesis:
	
	although we fall back on the assumption of the validity of the first relation but with $n=k+1$, hence the result.
	\end{tcolorbox}
	This prove process is therefore of great importance in the study of arithmetic. Often observation and induction have led to a suspicion of laws it would have been more difficult to find by a priori. We realize the accuracy of formulas by the previous method that gave birth to modern algebra by Fermat and Pascal studies on the Pascal's triangle (\SeeChapter{see section Calculus}).
	
	\pagebreak
	\subsection{Divisibility}
	\textbf{Definition (\#\mydef):} Given $A,B\in \mathbb{Z}$ with $A\neq 0$. We say that "\NewTerm{$A$ divides $B$ (without rest)}" if there is an integer $q$ (the quotient) such that:
	
	in which case we write to differentiate of the class division:
	
	Otherwise, we write
	
	and we say that "\NewTerm{$A$ does not divide $B$}".
	\begin{tcolorbox}[title=Remarks,colframe=black,arc=10pt]
	\textbf{R1.}  Remember that $|$ is a relation when the symbol $/$ is an operation!\\
	
	\textbf{R2.} Do not confuse the expression "$A$ divides $B$" which means that rest is necessarily zero and the expression "$A$ is the divisor of the division by $B$" indicating that the rest is not necessarily zero!
	\end{tcolorbox}
	Moreover, if $A|B$, we also say that "\NewTerm{$B$ can be divided by $A$}" or "\NewTerm{$B$ is a multiple of $A$}".
	
	In case where $A|B$ and that $1\geq A <B$, we will say that $A$ is a "\NewTerm{proper divisor}\index{proper divisor}" of $B$.
	
	Moreover, it is clear that $A|0$ regardless of $A\in \mathbb{Z}\setminus \{0\}$ otherwise what we have a singularity.

	Here are some basic theorems relating to the division:
	\begin{theorem}
	If $A|B$, then $A|BC$ whatever $C\in \mathbb{Z}$. Or more formally:
	
	\end{theorem}
	\begin{dem}
	If $A|B$, the it exists an integer $q$ such that:
	
	Then:
	
	and therefore:
	
	\begin{flushright}
		$\square$  Q.E.D.
	\end{flushright}
	\end{dem}
	\begin{theorem}
	If $A|B$ and $B|C$, then $A|C$ or more formally:
	
	\end{theorem}
	\begin{dem}
	If $A|B$ and $B|C$ then, there exists two integers $q$ and $r$ such that $B=Aq$ and $C=Br$. More formally:
	
	Therefore:
	
	and hence:
	
	\begin{flushright}
		$\square$  Q.E.D.
	\end{flushright}
	\end{dem}
	\begin{theorem}
	If $A|B$ and $A|C$ then:
	
	\end{theorem}
	\begin{dem}
	If $A|B$ and $A|C$ then, there exists two integers $q$ and $r$ such that $B=Aq$ and $C=Ar$. It follows:
	
	and therefore:
	
	\begin{flushright}
		$\square$  Q.E.D.
	\end{flushright}
	\end{dem}
	\begin{theorem}
	If $A|B$ and $B|A$ then:
	
	\end{theorem}
	\begin{dem}
	If $A|B$ and $B|A$ then, there exists two integers $q$ and $r$ such that $B=Aq$ and $A=Br$.

	We then have:
	 
	and thus $qr=1$. This is why we can have $q=\pm 1$ if $r=\pm 1$ and thus:
	
	\begin{flushright}
		$\square$  Q.E.D.
	\end{flushright}
	\end{dem}
	\begin{theorem}
	If $A|B$ and $B\neq =$ the:
	
	\end{theorem}
	\begin{dem}
	If $A|B$ then there exist an integer $q\neq 0$ such that $B=Aq$. But then:
	
	as $|q|\geq 1$.
	\begin{flushright}
		$\square$  Q.E.D.
	\end{flushright}
	\end{dem}
	
	\subsubsection{Euclidean Division}
	The Euclidean division is an operation that, to two integers named respectively the "\NewTerm{dividend}\index{dividend} and "\NewTerm{divisor}\index{divisor}" combines two other integers named the "\NewTerm{quotient}\index{quotient}" and "\NewTerm{remainder}\index{remainder}". Initially define only for nonzero integers, it can be generalized to relative integers and polynomials, for example.
	
	\textbf{Definition (\#\mydef):} We name "\NewTerm{euclidean division}\index{euclidean division}" or "\NewTerm{integer division}\index{integer division}" of two numbers $A$ and $B$ the operation of dividing $B$ by $A$, stopping when the rest is strictly less than $A$.
	
	Let us recall (\SeeChapter{see section Numbers}) that any number which admits exactly two euclidean divisors (such that division gives no remainder) that are the $1$ and itself is named a "\NewTerm{prime number}\index{prime number}" (which excludes the number $1$ of the list of primes) and that any pair of numbers which have only $1$ as common Euclidean divider are say to be "\NewTerm{relatively prime}\index{relatively prime numbers}", "\NewTerm{mutually prime}\index{mutually prime numbers}", or "\NewTerm{coprime}\index{coprime numbers}".
	\begin{theorem}
	Given $A,B\in \mathbb{Z}$ with $A>0$. The "\NewTerm{theorem of the Euclidean division}\index{theorem of the Euclidean division}" state that there are unique integers $q$ (quotient) and $r$ (remainder) such as:
	
	where $0\geq r <A$. Furthermore, if $A\nmid B$, then $0<r<A$.
	\end{theorem}
	\begin{tcolorbox}[colframe=black,colback=white,sharp corners]
	\textbf{{\Large \ding{45}}Example:}\\\\
	One cake with $9$ parts ($B$), we then have to divide it between $4$ people ($A$) with one part remaining ($r$=1) such that $q$=2
	\begin{figure}[H]
		\centering
		\includegraphics{img/arithmetics/euclidean_division.jpg}
		\caption[]{The pie has $9$ slices, so each of the $4$ people receive $2$ slices and $1$ is left over.}
	\end{figure}
	and therefore:
	
	\end{tcolorbox}
		\begin{dem}
	Let us consider the set:
	
	It is relatively easy to see that $S\subset \mathbb{N}^{*} \{0\}$ and that $S\neq \varnothing$, hence, according to the principle of good order, we conclude that $S$ contains a smaller element $r\geq 0$.
	Given $q$ the integer satisfying thus:
	
	We want to first show that $r<A$ assuming the opposite (proof ad absurdum), that is to say that $r\neq A$. So, in this case, we have:
	
	which is equivalent to:
	
	but $B-(q+1)A\in S$ and:
	
	This contradicts the fact that:
	
	is the smallest element of $S$. So $r<A$. Finally, it is clear that if $r=0$, we have $A|B$, hence the second statement of the theorem.
	\begin{flushright}
		$\square$  Q.E.D.
	\end{flushright}
	\end{dem}
	\begin{tcolorbox}[title=Remark,colframe=black,arc=10pt]
	In the statement of the Euclidean division, we assumed that $A>0$. What do we get when $A<0$? In this situation, $-A$ is obviously positive, and then we can apply the Euclidean division to $B$ and $-A$. Therefore, there are integers $q$ and $r$ integers such that:
	
	where $0\geq r <|A|$. But this relation can be written:
	
	where obviously, $-q$ is an integer. The conclusion is that the Euclidean division can be stated in a more general form.

	Given $A,B\in \mathbb{Z}$, there exist two integers $q$ and $r$ such that:
	
	where $0\geq r <|A|$. Furthermore, if $A\nmid B$, then $0<r<|A|$
	\end{tcolorbox}
	The integers $q$ and $r$ are unique in the Euclidean division. Indeed, if there are two other integers $q'$ and $r'$ such as:
	
	always with $0\leq r'<A$, then:
	
	and therefore:
	
	Following theorem 4.13 we have if $r-r'\neq 0$ that $|r-r'|\geq A$.

	But, this last inequality is impossible as by construction $-A<r-r'$. Therefore $r=r'$ and, as $A\neq 0$, then $q'=q$ hence the unicity.
	
	\paragraph{Greatest common divisor}\mbox{}\\\\
	The greatest common divisor (gcd) (also known as  greatest common factor (gcf), highest common factor (hcf), greatest common measure (gcm), or highest common divisor) of two or more integers, when at least one of them is not zero, is the largest positive integer that divides the numbers without a remainder. 
	
	\textbf{Definition (\#\mydef):} Given $a,b\in\mathbb{Z}$ such as $ab\neq 0$. The "greatest common divisor" (gc) of $a$ and $b$, denoted:

	
	is the positive integer $n$ that satisfies the following two properties:
	\begin{enumerate}
		\item[P1.] $d|a$ and $d|b$ (so without remainder $r$ in the division!)

		\item[P2.] If $c|a$ and $c|b$ the $c\leq d$ and $c|d$ (by division!)
	\end{enumerate}
	Note that $1$ is always a common divisor of two arbitrary integers.
	\begin{tcolorbox}[colframe=black,colback=white,sharp corners]
	\textbf{{\Large \ding{45}}Example:}\\\\
	Let us consider the positive integers $36$ and $54$. A common divisor of $36$ and $54$ is a positive integer that divides $36$, and also $54$. For example, $1$ and $2$ are common divisors $36$ and $54$.

	
	We have the intersection represented by the following Venn diagram:
	\begin{figure}[H]
		\centering
		\includegraphics{img/arithmetics/gcd_set.jpg}
		\caption{Venn diagram of common divisors}
	\end{figure}
	with the following set of common divisors:
	\end{tcolorbox}
	
	However it is not necessarily obvious that the greatest common divisor other than $1$ (that is to say different of $1$) of two integers $a$ and $b$ that are not relatively prime always exists. This is proved by the following theorem (however, if the gcd exists, it is by definition unique!) named "\NewTerm{Bézout theorem}\index{Bézout theorem}" that can also gives the opportunity to prove other interesting properties of two numbers as we shall see later .
	\begin{theorem}
	Given $a,b\in \mathbb{Z} $such that $ab\neq 0$. If $d$ divides $a$ and $d$ divides $b$ (for both without remainder $r$ !) then there must two integers $x$ and $y$ such that:
	
	This relation is named the "\NewTerm{Bézout identity}\index{Bézout identity}" and it is a linear Diophantine equation (\SeeChapter{see section Calculus}).
	\end{theorem}
	\begin{dem}
	Obviously, if $a$ and $b$ are relatively prime we know that $d$ is then $1$.
	
	To prove the Bézout identity let first consider the set:
	
	As $S\subset \mathbb{N}$ and $S\neq \varnothing$, we can use the principle of good order and conclude that $S$ has a smaller element $d$. We can then write:
	
	for some given choice $x_0,y_0\in\mathbb{Z}$. So it is sufficient to prove that $d=(a,b)$ to prove the Bézout identity!
	
	Let us proceed with a proof by contradiction by assuming $d\nmid a$. Then if this is the case, following the Euclidean division, there exist $q,r\in\mathbb{Z}$ such as $a=qd+r$, where $0<r<d$. But then:
	
	Thus we have that $r\in S$ and $r<d$, which contradicts the fact that $d$ is the smallest possible element of $S$. Thus we have proven not only that $d | a$, but also that $d$ always exists and, in the same way we prove that $d | b$.
	\begin{flushright}
		$\square$  Q.E.D.
	\end{flushright}
	\end{dem}
	\begin{corollary}
	As important corollary let us now prove that if $a,b\in\mathbb{Z}$ such that $ab\neq 0$, then:
	
	is the set of all multiples of $d(a,b)$.
	\end{corollary}
	\begin{dem}
	As $d | a$ and $d | b$, then we have necessarily $dax+by|$ for any $x,y\in \mathbb{Z}$. Either $M=\{nd|n\in\mathbb{Z}\}$. Our problem is then reduced to prove the fact that $S=M$.
	
	Given first $s\in S$ which means that $d|s$ and involves $s\in M$.

	Given a $m\in M$, this would mean that $m=nd$ for a certain $n\in\mathbb{Z}$.
	
	As $d=ax_0+by_0$ for any choice of integers $x_0,y_0\in\mathbb{Z}$, then:
	
	\begin{flushright}
		$\square$  Q.E.D.
	\end{flushright}
	\end{dem}
	The assumptions may seem complicated but put your attention a given time on the last equality. You will quickly understand!
	\begin{tcolorbox}[title=Remark,colframe=black,arc=10pt]
	If instead of defining the greatest common divisor of two non-zero integers, we allow one of them to be equal to $0$, say: $a\neq b$, $b=0$. In this case, we have $a|b$ and, according to our definition of the GCD, it is clear that $(a,0)=|a|$.
	\end{tcolorbox}
	Given $d=(a,b)$ and $m\in\mathbb{Z}$, then we have the following properties of the GCD (without proof but if a reader request them we will give the details):
	\begin{enumerate}
		\item[P1.] $(a,b+ma)=(a,b)=(a,-b)$
		\item[P2.] $(am,bm)=|m|(a,b)$ where $m\neq 0$
		\item[P3.] $\left(\dfrac{a}{d},\dfrac{b}{d}\right)=1$
		\item[P4.] If $g\in\mathbb{Z}\setminus \{0\}$ such that $g|a$ and $g|b$ then $\left(\dfrac{a}{g},\dfrac{b}{g}\right)=\dfrac{1}{|g|}(a,b)$
	\end{enumerate}
	In some books, these four properties are proved using intrinsically the property itself. Personally we abstain make usage of this approach because doing this is more ridiculous than anything else as the statement of the property is a proof in itself.
	
	Let us now develop a method (algorithm) that will be very useful to us to calculate (determine) the greatest common divisor of two integers (sometimes useful computing science).
	
	\subsubsection{Euclidean Algorithm}
	The Euclidean algorithm is an algorithm for determining the greatest common divisor of two integers (we have hesitate to put this subject in the section of Theoretical Computing...).
	
	To address this method intuitively, you must know that that you need to see that an integer as a length, a pair of integers as a rectangle (sides) and their GCD is the size of the largest square for tile (paving) their rectangle by definition (yes if you think for a moment it's quite logical!).

	The algorithm decomposes the original rectangle into squares, always smaller and smaller, by successive Euclidean division of the length by the width, then the width by the remainder until a zero remainder. We must understand this geometric approach to then understand the algorithm.
	\begin{tcolorbox}[colframe=black,colback=white,sharp corners]
	\textbf{{\Large \ding{45}}Example:}\\\\
	Let us consider that we seek the GCD of $(a, b)$ where $b$ is equal $21$ and $a$ is equal $15$ and keep in mind that the GCD, besides the fact that it divides $a$ and $b$, must leave a zero remainder! In other words it must divide the remainder of the division of $b$ by $a$ also!\\
	
	So we have the following rectangle of $21$ by $15$:
	\begin{figure}[H]
		\centering
		\includegraphics[scale=0.7]{img/arithmetics/euclids_algorithm_step1.jpg}
		\caption[]{First step of the GCD algorithm}
	\end{figure}
	\end{tcolorbox}
	
	\begin{tcolorbox}[colframe=black,colback=white,sharp corners]
	First we see if $15$ is the GCD (it always starts with the smallest). We then divide $21$ by $15$, which is equivalent geometrically to:
	\begin{figure}[H]
		\centering
		\includegraphics{img/arithmetics/euclids_algorithm_step2.jpg}
		\caption[]{Second step of the GCD algorithm}
	\end{figure}
	$15$ is therefore not the GCD (we suspected it...). We immediately see that we can not pave the rectangle with a square of $15$ by $15$.\\
	
	So we have a remainder of $6$ (left rectangle). The GCD as we know must, if it exists, by definition divide that remains and leave a zero remainder.\\

	So we have a rectangle of $15$ by $6$. So we are looking now to pave this new rectangle because we know that the greatest common divisor is by construction less than or equal to $6$. Then we have:
	\begin{figure}[H]
		\centering
		\includegraphics{img/arithmetics/euclids_algorithm_step3.jpg}
		\caption[]{Third step of the GCD algorithm}
	\end{figure}
	So we divide $15$ by remainder $6$ (this result will be less than $6$ and immediately permits to tests whether the reamainder will be the GCD). We are getting:
	\begin{figure}[H]
		\centering
		\includegraphics{img/arithmetics/euclids_algorithm_step4.jpg}
		\caption[]{Fourth step of the GCD algorithm}
	\end{figure}
	Again, we can not pave the rectangle only with squares. In other words, we have a non-zero remainder which is $3$. Given now a rectangle of $6$ by $3$. So we are looking now to pave the new rectangle because we know that the greatest common divisor is by construction less than or equal to $3$ and that it will leave a remainder equal to zero, if it exists. We then have geometrically:
	\end{tcolorbox}
	
	\begin{tcolorbox}[colframe=black,colback=white,sharp corners]
	\begin{figure}[H]
		\centering
		\includegraphics{img/arithmetics/euclids_algorithm_step5.jpg}
		\caption[]{Fifth step of the GCD algorithm}
	\end{figure}
	We divide $6$ by $3$ (which will be less than $3$ and permits us to test immediately whether the rest will be the GCD):
	\begin{figure}[H]
		\centering
		\includegraphics{img/arithmetics/euclids_algorithm_step6.jpg}
		\caption[]{Sixth step of the GCD algorithm}
	\end{figure}
	and it's all good! We then have $3$ that leave us with a remainder equal to zero and divides the remainder $6$ so this is the GCD. So we have in the end:
	\begin{figure}[H]
		\centering
		\includegraphics{img/arithmetics/euclids_algorithm_step7.jpg}
		\caption{Summary of the GCD algorithm}
	\end{figure}
	\end{tcolorbox}
	Now let us see the equivalent formal approach.
	
	Given $a,b\in\mathbb{Z}$, where $a>0$. Applying successively the Euclidean division (with $b> a$), we get the following sequence of equations:
	
	if $d=(a,b)$, then $d=r_j$.
	with the corresponding pseudo-code algorithm:
	
	\begin{algorithm}[H]
	 \KwData{$a$,$ b$}
	 \KwResult{$b$}
	 initialization\;
	$r=a\mod b$\;
	 \While{$r\neq 0$}{
	  $a:=b$\;
	  $b:=r$\;
	   $f(b):=f(a)$\;
	   $a:=x_1$\;
	   $f(a):=f(x_1)$\;
	   }
	  Display $x_1$\;
	 \caption{GCD pseudo-code algorithm}
	\end{algorithm}
	Otherwise even more formally:
	\begin{dem}
	We want first prove that $r_j=(a,b)$. But, following the property P1:
	
	we have:
	
	To prove the second property of the Euclide's algorithm, we write the prior-previous equation of the system under the form:
	
	Now, using the previous equation this prior-previous equation of the system, we have:
	
	Continuing this process, we can express $r_j$ as a linear combination of $a$ and $b$.
	\begin{flushright}
		$\square$  Q.E.D.
	\end{flushright}
	\end{dem}
	\begin{tcolorbox}[colframe=black,colback=white,sharp corners]
	\textbf{{\Large \ding{45}}Example:}\\\\
	Let us calculate the greatest common divisor of  $(429,966)$ and express this number as a linear combination of $429$ and $966$.
	
	We therefore conclude that:
	
	and, in addition, that:
	
	Thus the GCD is indeed expressed as a linear combination of $a$ and $b$ and constitutes as such the GCD.
	\end{tcolorbox}
	\textbf{Definition (\#\mydef):} We say that the integers $a_1,a_2,\ldots,a_n$ are for recall "\NewTerm{relatively prime}" if:
	
	
	\subsubsection{Least Common Multiple}
	The least common multiple (also named the "\NewTerm{lowest common multiple}\index{lowest common multiple}" or "\NewTerm{smallest common multiple}\index{smallest common multiple}") of two integers $a$ and $b$, usually denoted by $\text{LCM}(a, b)$, is the smallest positive integer that is divisible by both $a$ and $b$. Since division of integers by zero is undefined, this definition has meaning only if a and b are both different from zero.
	
	The LCM is familiar from grade-school arithmetic as the "\NewTerm{lowest common denominator LCD} \index{lowest common denominator}" (also named "\NewTerm{smallest common denominator}\index{smallest common denominator}") that must be determined before fractions can be added, subtracted or compared. The LCM of more than two integers is also well-defined: it is the smallest positive integer that is divisible by each of them.
	
	\textbf{Definitions (\#\mydef):}
	\begin{enumerate}
		\item[D1.] Given $a_1,a_2,\ldots,a_n\in \mathbb{Z}\setminus \{0\}$, we say that $m$ is a "\NewTerm{common multiple}\index{common multiple}" of $a_1,a_2,\ldots,a_n$ if $a_i|m$ for $i=1,2,\ldots,n$

		\item[D2.] Given $a_1,a_2,\ldots,a_n\in \mathbb{Z}\setminus \{0\}$, we name "\NewTerm{lowest common multiple LCM}\index{lowest common multiple}" of $a_1,a_2,\ldots,a_n$ if $a_i|m$ for $i=1,2,\ldots,n$ denoted:
		
		the lowest integer positive common multiple to all common multiples of $a_1,a_2,\ldots,a_n$.
	\end{enumerate}
	\begin{tcolorbox}[colframe=black,colback=white,sharp corners]
	\textbf{{\Large \ding{45}}Examples:}\\\\
	E1. Let us consider the positive integers $3$ and $5$. A common multiple of $3$ and $5$ is a positive integer which is both a multiple of $3$, and a multiple of $5$. In other words, which is divisible by $3$ and $5$. We have therefore:
	
	We then have the intersection represented by the following Venn diagram:
	\begin{figure}[H]
		\centering
		\includegraphics{img/arithmetics/lcm_binary_venn_diagram.jpg}
	\end{figure}
	with then have following set of common multiples:
	
	and therefore the LCM is given by:
	
	Or if it can help here is another possible visualization of the concept:
	\begin{figure}[H]
		\centering
		\includegraphics{img/arithmetics/lcm_binary_multiples_line.jpg}
	\end{figure}
	We see obviously that all the common multiples of $3$ and $5$ is the set of multiples of $15$.
	\end{tcolorbox}
	
	\begin{tcolorbox}[colframe=black,colback=white,sharp corners]
	E2. Wikipedia gives also a nice visual example of a Venn Diagram when we seek for a LCM of $5$ multiples using a visual tool:
	\begin{figure}[H]
		\centering
		\includegraphics[scale=0.5]{img/arithmetics/lcm_5ary_venn_diagramm.jpg}
	\end{figure}
	\end{tcolorbox}
	\begin{tcolorbox}[title=Remark,colframe=black,arc=10pt]
	Given $a_1,a_2,\ldots,a_n\in\mathbb{Z}\setminus \{0\}$. Then the least common multiple exists. Indeed, consider the set $E$ of natural integers $m$ that for all $i$ divide $a_i$. What we will write:
	
	Since we have necessarily $|a_1a_2\ldots a_n|\in E$, then the set is not empty and, according to the axiom of good order, the set $E$ contains a smaller positive element.
	\end{tcolorbox}
	Let us now see some theorems related to the LCM:
	\begin{theorem}
	If $m$ is any common multiple of $a_1,a_2,\ldots,a_n$ then $[a_1,a_2,\ldots,a_n]|m$ that is to say that $m$ divides each of the $a_i$.
	\end{theorem}
	\begin{dem}
	Given $M=[a_1,a_2,\ldots,a_n]$. Then, by the Euclidean division, there are integers $q$ and $r$ such that:
	
	It suffices to show that $r=0$. Let us suppose that $r\neq 0$ (reductio ad absurdum). Since $a_i|m$ and $a_i|M$, the we have $a_i|r$ and this for $i=1,2,\ldots,n$. So $r$ is common multiple of $a_1,a_2,\ldots,a_n$of the smallest than the LCM. We just obtained a contradiction, which proves the theorem.
	\begin{flushright}
		$\square$  Q.E.D.
	\end{flushright}
	\end{dem}
	\begin{theorem}
	If $k>$, then $[ka_1,ka_2,\ldots,ka_n]=k[a_1,a_2,\ldots,a_n]$

	The proof will be assumed obvious (if not as always contact us and will add the details!)
	\end{theorem}
	\begin{theorem}
	$[a,b]\cdot(a,b)=|ab|$
	\end{theorem}
	\begin{dem}
	\begin{lemma}
	For this proof, we will use the "\NewTerm{Euclid's lemma}\index{Euclid's lemma}" that says that if $a|bc$ and $(a,b)=1$ then $a|c$.
	
	In other words Euclid's lemma captures a fundamental property of prime numbers, namely: If a prime divides the product of two numbers, it must divide at least one of those numbers. It is also named "\NewTerm{Euclid's first theorem}\index{Euclid's first theorem}". This lemma is the key of the proof of the fundamental theorem of arithmetic that we will see just further below. 
	
	Indeed, this can be easily verified because we have seen that there exists $x,y\in\mathbb{Z}$ such as $1=ax+by$ and then $c=acx+bcy$. But $a|ac$ and $a|bc$ imply that $a|(acx+bcy)$, that is to say also that $a|c$.
	\end{lemma}
	
	Ok let us now return to our theorem:
	
	Since $(a,b)=(a,-b)$ and $[a,b]=[a,-b]$, it suffices to prove the result for positive integers $a$ and $b$. 

First of all, let consider the case where $(a,b)=1$. The integer $[a, b]$ being a multiple of $a$, we can write $[a,b]=ma$. Thus, we have $b|ma$ and since $(a,b)=01$, it follows, by Euclid's lemma, that $b | m$. Therefore, $b\leq m$ and then $ab\leq am$. But $ab$ is a common multiple of $a$ and $b$ that can not be smaller than the LCM. therefore $ab=ma=[a,b]$.

	For the general case, that is to say $(a,b)=d>1$, we have, according to the property:
	
	and with the result obtained previously that:
	
	When we multiply both sides of the equation by $d^2$, the result follows and the proof is done.
	\begin{flushright}
		$\square$  Q.E.D.
	\end{flushright}
	\end{dem}
	
	\pagebreak
	\subsubsection{Fundamental Theorem of Arithmetic}
	The fundamental theorem of arithmetic says that every natural number $n>1$ can be written as a product of primes, and this representation is unique, except for the order in which the prime factors are arranged.
	
	The theorem establishes the importance of prime numbers. Essentially, they are the building blocks of building positive integers, each positive integer containing primes in a unique way.
	\begin{tcolorbox}[title=Remark,colframe=black,arc=10pt]
	This theorem is sometimes named "factorization theorem" (wrongly ... because some other theorems have the same name ...).
	\end{tcolorbox}
	So let's go:
	\begin{theorem}
	Every integer greater than $1$ either is prime itself or is the product of prime numbers, and that this product is unique, up to the order of the factors.
	\begin{tcolorbox}[title=Remark,colframe=black,arc=10pt]
	This theorem is one of the main reasons why $1$ is not considered a prime number: if $1$ were prime, the factorization would not be unique.
	\end{tcolorbox}
	\end{theorem}
	\begin{dem}
	The proof uses Euclid's lemma: if a prime $p$ divides the product of two natural numbers $a$ and $b$, then either $p$ divides $a$ or $p$ divides $b$ (or both).
	
	If $n$ is prime, and therefore product of a unique prime integer, namely itself, the result is true and the proof is complete (say that a prime number is product of itself is obviously a misnomer! ). Suppose that $n$ is not prime and therefore strictly greater than $1$ and consider the set:
	
	So, $D\subset \mathbb{N}$ and since $n$ is composite, we have that $D\neq \varnothing$. According to the principle of good order, $D$ has a smaller element $p_1$ that is prime, otherwise the minimum choice of $p_1$ is contradicted. We can the write $n=p_1n_1$. If $n_1$ is prime, then the proof is complete. If $n_1$ is also composite, then we repeat the same argument as before and we deduce the existence of a prime number $p_2$ and of an integer $n_2<n_1$, such as $n=p_1p_2n_2$. By continuing we come inevitably to the conclusion that $n_k$ will be prime.
	
	So finally we well show that any number can be decomposed into prime numbers factors with the principle of good order.
	\begin{flushright}
		$\square$  Q.E.D.
	\end{flushright}
	\end{dem}
	We do not know to this day a simple law that allows to calculate the $n$-th prime factor $p_n$. Thus, to know if an integer $m$ is a prime, it is almost easier at this date to verify its presence in a table of prime numbers.

	In fact, we use nowadays the following method:

	Given an integer $m$, if we want to determine whether it is prime or not, we calculate if it is divisible by the primes number $p_n$ belonging to the set:
	

	\begin{tcolorbox}[colframe=black,colback=white,sharp corners]
	\textbf{{\Large \ding{45}}Examples:}\\\\
	The integer $223$ is neither divisible by $2$ or by $3$ or by $5$ or by $7$, or by $11$, or by $13$. It is useless to continue with the next prime number, because $17^2=289>223$. We conclude therefore that the number $223$ is prime.
	\end{tcolorbox}

	\subsubsection{Congruences (modular arithmetic)}
	Modular arithmetic is a system of arithmetic for integers, where numbers "wrap around" upon reaching a certain value—the modulus (plural moduli).
	
	A familiar use of modular arithmetic is in the $12$-hour clock (and also the calendar), in which the day is divided into two $12$-hour periods. If the time is 7:00 now, then $8$ hours later it will be 3:00. Usual addition would suggest that the later time should be $7 + 8 = 15$, but this is not the answer because clock time "wraps around" every $12$ hours; in $12$-hour time, there is no "$15$ o'clock". Likewise, if the clock starts at 12:00 (noon) and $21$ hours elapse, then the time will be $9:00$ the next day, rather than 33:00. Because the hour number starts over after it reaches $12$, this is arithmetic modulo $12$. According to the definition below, $12$ is congruent not only to $12$ itself, but also to $0$, so the time "12:00" could also be written "0:00", since $12$ is congruent to $0$ modulo $12$.
	\begin{figure}[H]
		\centering
		\includegraphics{img/arithmetics/modular_arithmetics_clock.jpg}
		\caption{Time-keeping on this clock uses arithmetic modulo $12$ (source: Wikipedia)}
	\end{figure}
	\textbf{Definition (\#\mydef):} Let $m\in\mathbb{Z}\setminus 0$. If $a$ and $b$ have the same remainder when divided by $m$ in the Euclidean division then we say "$a$ is congruent to $b$ modulo $m$", and we write:
	
	or equivalently there are (at least) on relative integer $k$ such that:
	
	We also name the number $b$ "\NewTerm{residue}\index{residue}". Thus, a residue is an integer congruent to another, modulo a given integer $m$. The reader can verify that this requires that:
	
	\begin{tcolorbox}[title=Remarks,colframe=black,arc=10pt]
	\textbf{R1.} The reader must well understand that congruence implies a null remainder for the division!\\
	
	\textbf{R2.} We exclude in addition to the $0$ also the $1$ and $-1$for the possible values of $m$ in the definition of congruence in some books.\\
	
	\textbf{R3.} Behind the term congruence are hidden similar concepts of different levels of abstraction:
	\begin{itemize}
		\item In modular arithmetic, so we say that "two integers $a$ and $b$ are congruent modulo $m$ if they have the same remaining in the Euclidean division by $m$". We can also say that they are congruent modulo $m$ if their difference is a multiple of $m$.

		\item In the study of oriented angles, we say that "two measurements are congruent modulo $2\pi$ [rad] if and only if their difference is a multiple of $2\pi$ [rad]". This characterize two measures of the same angle (\SeeChapter{see section Trigonometry}).

		\item In algebra, we speak of congruence modulo $I$ in a commutative ring (\SeeChapter{see section Set Theory}) for which $I$ is an ideal: "$x$ is congruent to $y$ modulo $I$ if and only if their difference belongs to $I$". This congruence is an equivalence relation compatible with the operations of addition and multiplication and gives the possibility to define a quotient ring of the parent set with its ideal $I$.

		\item We sometimes see in the study of geometry (\SeeChapter{see section Euclidean Geometry}) the term "congruence" used in place of "similar". It is then a simple equivalence relation on the set of plane figures.
	\end{itemize}
	\end{tcolorbox}
	The relation of congruence $\equiv$ is an equivalence relation (\SeeChapter{see section Operators}), in other words, given $a,b,c,m\in\mathbb{Z},m>1$ then the congruence relation is:
	\begin{enumerate}
		\item[P1.] Reflexive:
		
		\item[P2.] Symmetric:
		
		\item[P3.] Transitive:
		
	\end{enumerate}
	The properties $P1$ and $P2$ are obvious (if this is not the case please let us know we will develop!). We will prove only P3.
	\begin{dem}
	The assumptions imply that
	
	 But then:
	
	This prove that $a$ and $c$ are congruent modulo $m$.
	\begin{flushright}
		$\square$  Q.E.D.
	\end{flushright}
	\end{dem}
	The relation of congruence $\equiv$ is compatible with the sum and the product (remember that power is ultimately an extension of the product!).
	
	Indeed, given $(a,b,a',b',m)\in\mathbb{Z},m>1$ such that $a\equiv \mod(m)$ and $a'\equiv b'$ then:
	\begin{enumerate}
		\item[P1.] $a+a'\equiv b+b'\mod(m)$

		\item[P2.] $aa'\equiv bb'\mod(m)$
	\end{enumerate}
	\begin{dem}
	We have:
	
	by hypothesis. But then:
	
	which proves P1. We also have:
	
	which proves P2.
	\begin{flushright}
		$\square$  Q.E.D.
	\end{flushright}
	\end{dem}
	\begin{tcolorbox}[title=Remark,colframe=black,arc=10pt]
	The congruence relation behaves in many point like the relation of equality. However a property of the relation of equality $=$ is not true for that of congruence  $\equiv$, namely the simplification: If $ab\equiv \mod(m)$, we do not have  necessarily $b\equiv c \mod(m)$.
	\end{tcolorbox}
	\begin{tcolorbox}[colframe=black,colback=white,sharp corners]
	\textbf{{\Large \ding{45}}Examples:}\\
	
	\end{tcolorbox}
	So far we have seen the properties of congruences involving a single modulus. We will now study the behavior of the congruence relation on a change of modulus.
	\begin{enumerate}
		\item[P1.] If $a\equiv b \mod(m)$ and $d|m$, then $a\equiv b \mod(d)$
		\item[P2.] If $a\equiv b$ and $a\equiv b \mod(s)$ then $a$ and $b$ are congruate modulus $[r,s]$
	\end{enumerate}
	We think this two properties are obvious. We do not need to go into details for P1. For P2, since $b-a$ is a multiple of $r$ and $s$ since by hypothesis:
	
	$b-a$ is then a multiple of the LCM of $r$ and $s$, which proves P2.
	
	From these properties it follows that if we denote by $f(x)$ a polynomial with integer coefficients (positive or negative):
	
	The congruence $a \equiv b \mod(m)$ will also give $f(a)\equiv f(b) \mod(m)$.

	If we replace $x$ successively by all integers in a polynomial $f(x)$ with integer coefficients, and if we take the remaining modulus $m$, these remaining are reproduced from $m$ to $m$ (in the sense where the congruence is satisfied), since we have, regardless of the number $m$ and $x$:
	
	We then deduce then the impossibility of solving the following congruence:
	
	with integer numbers, if $r$ is anyone of the "non-remaining" (a residue that does not satisfy the congruence).
	
	\paragraph{Congruence Class}\mbox{}\\\\
	\textbf{Definition (\#\mydef):} We name "\NewTerm{modulo $m$ congruence class}\index{module congruence class}", the subset of the set $\mathbb{Z}$ defined by the property that two elements $a$ and $b$ of $\mathbb{Z}$are in the same class if and only if $a\equiv b \mod(m)$ or that a set of elements are congruent by this same modulo.
	\begin{tcolorbox}[title=Remark,colframe=black,arc=10pt]
	We saw in the section Operators that this is in fact an equivalence class as the congruence modulo $m$ is, as we have proved above, a relation of equivalence!!!\\
	\end{tcolorbox}
	
	\begin{tcolorbox}[colframe=black,colback=white,sharp corners]
	\textbf{{\Large \ding{45}}Example:}\\\\
	Given $m=3$. We divide the set of integers into congruence classes of modulo $3$. Here are for example three sets whose elements are congruent with one another without rest (see well what gives the set of all these classes together!):
	
	Thus we see that for each pair of elements of a congruence class, the congruence modulo $3$ exists. However, we see that we can not take that $-9\equiv -8 \mod(3)$ where $-9$ is in the first class and $-8$ in the second.\\
	
	The smallest non-negative number of the first class is $0$, this of the second is $1$ and the last is $2$. Thus, we will denote these three classes respectively $[0]_3,[1]_3,[2]_3$, the number $3$ in the index indicating the modulus.\\
	
	It is interesting to notice that if we take any number of the first class and any number of the second class, then their sum is always in the second class. This can be generalized and allows to define a sum of classes modulo $3$ by writing:
	
	and also:
	
	\end{tcolorbox}
	Thus, for any $m>1$, the congruence class:
	
	is the set of integers congruent to a modulo $m$ (and congruent modulo $m$ between them)!!! This class is denoted by:
	
	\begin{tcolorbox}[title=Remark,colframe=black,arc=10pt]
	Having bracketed the "and congruent modulo $m$ between them" is due to the fact that the congruence, being an equivalence relation we have as we have proved above that $b\equiv a \mod(m)$, $c\equiv \mod(m)$, then $b\equiv a\mod(m)$.
	\end{tcolorbox}
	
	\textbf{Definition (\#\mydef):} The set of congruence classes $[a]_m$ (that forms by the fact that congruence is an equivalence relation: "equivalence classes"), for a fixed $m$ gives what we name a "\NewTerm{quotient set}\index{quotient set}" (\SeeChapter{see section Operators}). More rigorously, we speak of the "quotient set of $\mathbb{Z}$ by the congruence relation" whose elements are the congruence classes (or: equivalence classes) and then form the ring $\mathbb{Z}/m\mathbb{Z}$.
	We deduce from the definition the following two trivial properties:
	\begin{enumerate}
		\item The number $b$ is in the class $[a]_m$ if and only if $a\equiv b\mod(m)$
		\item The classes $[a]_m$ and $[b]_m$ are equal if and only if $a\equiv b\mod(m)$
	\end{enumerate}
	\begin{theorem}
	There are exactly $m$ different congruence classes of modulo $m$, ie $[0]_m,[1]_m,\ldots,[m-1]_m$.
	\end{theorem}
	\begin{dem}
	Given $m>1$, than any integer $a$ is congruent modulo $m$ to one and only one integer $r$ of the set $\{0,1,2,\ldots,m-1\}$ (notice well, it is important, that we restrict ourselves to the positive integers without taking into account the negative one!) . In addition, this integer $r$ is exactly the remaining of the division of $a$ by $m$. In other words, if $0\leq r<m$, then:
	
	if and only if $a=qm+r$ where $x$ is the quotientof $a$ by $m$ and $r$ is the remainder. The proof is an immediate consequence of the definition of the congruence and of the Euclidean division.
	\begin{flushright}
		$\square$  Q.E.D.
	\end{flushright}
	\end{dem}
	\textbf{Definition (\#\mydef):} An integer $b$ in a congruence class modulo $m$ is named a "\NewTerm{representative of this class}\index{representative of a class}" (it is clear that by the equivalence relation that two representative of the same class are congruent modulo $m$ to each other).
	
	We can now be able to build an addition and a multiplication on the congruence classes. To define the sum of two classes $[a]_m,[b]_m$, it suffices to take one representative from each class, to their sum and take the congruence class of the result. Thus (see examples above ):
	
	And same for the multiplication:
	
	By construction of the addition and multiplication, we see that 0 (zero) is the neutral element for addition:
	
	and the class of the integer $1$ is the neutral element for multiplication:
	
	\textbf{Definition (\#\mydef):} An element $[a]_m$ of $\mathbb{Z}/m\mathbb{Z}$ is "\NewTerm{one unit}" if there is an element $[b]_m\in \mathbb{Z}/m\mathbb{Z}$ such that $[a]_m\cdot[b]_m$.
	The following theorem helps to characterize classes modulo $m$ which are units in $\mathbb{Z}/m/\mathbb{Z}$:
	\begin{theorem}
		Given $[a]$ un element of $\mathbb{Z}/m/\mathbb{Z}$. Then $[a]$ is a unit if and only if $(a,m)=1$.
	\end{theorem}
	\begin{dem}
	Suppose first that $(a,m)=1$. Then by Bezout theorem, we have its identity:
	
	In other words, $as$ is congruent to $1$ modulo $m$. But this is equivalent to write by definition that $[a][s]=1$ showing that $[a]$ is a unit. Conversely, if $[a]$ is a unit, this implies that there exists a class $[s]$ such that $[a][s]=1$.
	
	Thus, we have just proved that $\mathbb{Z}/\mathbb{Z}$ is indeed a ring since it has an addition, a multiplication, a neutral element and an inverse!!
	\begin{flushright}
		$\square$  Q.E.D.
	\end{flushright}
	\end{dem}
	
	\paragraph{Complete set of residues}\mbox{}\\\\
	\textbf{Definition (\#\mydef):} A set of numbers $a_0, a_1, ..., a_(m-1) \mod (m)$ form a "\NewTerm{complete set of residues}\index{complete set of residue}", also named a "\NewTerm{covering system}\index{covering system}", if they satisfy $a_i=i \mod (m) $ for $i=0, 1, ..., m-1$. 

	This type of systems will help us to introduce in the section of Cryptography to introduce an important function used in secured communication devices at the end of the 20th century and beginning of the 21st century.

	To introduce this concept, consider the following finite system of congruences modulo $6$:
	
	where as the reader will have probably noticed it: no residue is repeated in the list and no residue taken in pairs are congruent between them modulo $m$ (is this last point that oblige to stop at $5$ in our example). We then say that the residues are "\NewTerm{mutually incongruent}\index{mutually incongruent}".

	If these conditions are met, then we say that the ordered set $\{6, 13, 2, -3, 22, 11\}$ is a "complete system of residues modulo $m$" as already defined. Such a set is not unique for a given module. Thus, the set $\{0, 1, 2, 3, 4, 5\}$ is also a complete  (trivial) system of residues modulo $6$.

	If we eliminate from this entire system all numbers that are not prime to $m$, then we have a "\NewTerm{system of reduced residue modulo $m$}\index{system of reduced residue}". So in the above example, the reduced residue system modulo $6$ will be $\{13, 11\}$.
	
	Reduces systems will be useful tou us in the section Cryptography to prove an important result in the asymmetric public key systems.

	We will see also in the section Cryptography, the "\NewTerm{Euler indicator function}\index{Euler indicator}" when $m$ is prime (which is not the case in the previous example) gives the cardinal of the reduced system modulus $m$ as being equal to:
	
	So under the assumption condition that $m$ is prime, the reduced system of residue is obviously written:
	
	
\paragraph{Chinese remainder theorem}\mbox{}\\\
	In its basic form, the Chinese remainder theorem will determine a number $n$ that, when divided by some given divisors, leaves given remainders. In Sun Tzu's example (stated in modern terminology), what is the smallest number $n$ that when divided by $3$ leaves a remainder of $2$, when divided by $5$ leaves a remainder of $3$, and when divided by $7$ leaves a remainder of $2$?
	
	The Chinese remainder theorem can therefore be seen as solving a linear system but in a modular system. For many students and future engineers, this theorem will never be used in practice, but some will see it again it in the field of cryptography (in the context of decryption especially).
	
	There are several possible proofs as always but we opted for the one that, as always for us, seemed the most educational.

	Given $M$ and $n$ both prime integers between them. Then special case of a system of two congruences (see further below for an example of resolution of a system of three congruences):
	
	has a unique solution.
	\begin{dem}
	As $m$ and $n$ are assumed as prime between them, there exists then $u$ and $v$ two integers such as (application of the Bézout identity proved earlier above):
	
	Therefore we have:
	
	That is to say:
	
	Then we have also by extension:
	
	That is to say:
	
	So to be clear, we have so far:
	
	We then have for recall:
	
	But we can also writhe with $k\in\mathbb{Z}$:
	
	Therefore:
	
	Then we also have:
	
	But we can also writhe with $k\in\mathbb{Z}$:
	
	Therefore:
	
	So to be alway clear, we have so far:
	
	So finally we get that:
	
	is a particular solution of the system. But we also have $\forall i,j\in\mathbb{Z}$ by  the definition of the congruence:
	
	So that x is always solution of the system, we must have:
	
	and therefore:
	
	Theref ore a little bit more general solution is
	
	But by extension, we have the general solution:
	
	with $z\in\mathbb{Z}$. We then say sometimes that the solution is "$x$ modulo $nm$".
	\begin{flushright}
		$\square$  Q.E.D.
	\end{flushright}
	\end{dem}
	\begin{tcolorbox}[colframe=black,colback=white,sharp corners]
	\textbf{{\Large \ding{45}}Examples:}\\\\
	As an example, consider the problem of finding an integer $x$ such that:
	
	A brute-force approach converts these congruences into sets and writes the elements out to the product of $3\cdot 4\cdot 5 = 60$ (the solutions modulo $60$ for each congruence):
	
	To find an $x$ that satisfies all three congruences, intersect the three sets to get:
	
	This solution is modulo 60, hence all solutions are expressed as:
	
	Another way to find a solution is with basic algebra, modular arithmetic, and stepwise substitution.\\
	\end{tcolorbox}
	
	\begin{tcolorbox}[colframe=black,colback=white,sharp corners]
	We start by translating these congruences into equations for some $t$, $s$, and $u$:
	
	Start by substituting the $x$ from the first equation into the second congruence:
	
	That is to say:
	
	Hence:
	
	meaning that $t = 3 + 4s$ for some integer $s$. Substitute now $t$ into the first equation:
	
	Substitute this $x$ into the third congruence:
	
	That is to say:
	
	Hence:
	
	meaning that $s = 0 + 5u$ for some integer $u$. Finally:
	
	So, we have solution $\{11,71,131,191,\ldots\}$.
	\end{tcolorbox}
	
	\pagebreak
	\subsubsection{Continued fraction}
	A continued fraction is an expression obtained through an iterative process of representing a number as the sum of its integer part and the reciprocal of another number, then writing this other number as the sum of its integer part and another reciprocal, and so on. In a finite continued fraction (or terminated continued fraction), the iteration/recursion is terminated after finitely many steps by using an integer in lieu of another continued fraction. In contrast, an infinite continued fraction is an infinite expression. In either case, all integers in the sequence, other than the first, must be positive. The integers $a_i$ are named the "coefficients" or "terms" of the continued fraction.
	
	The notion of continued fraction come back from the time of Fermat and culminated with the work of Lagrange and Legendre in the late 18th century. These fractions are important in physics because we find them back in our study of acoustic and also in the thought process that led Galois to create his group theory and also in the studies  gear ratios of (for watch complications as discussed in the section of Mechanical Engineering).

	To understand the motivation of continued fraction let us introduce a basic example.
	
	Consider a typical rational number:
	
	which is around 4.4624.

	As first approximation, stat with 4, which is the integer part:
	
	Note that the fractional part is the reciprocal of $93/43$ which is about $2.1628$. Use the integer part, $2$, as an approximation for the reciprocal, to get a second approximation of:
	
	So we have so far:
	
	Note that the fractional part is the reciprocal of $43/7$ which is about $6.1429$. Use the integer part, $6$, as an approximation for the reciprocal, to get a second approximation of:
	
	Therefore:
	
	Note that the fractional part $1/7$ is the reciprocal of $7$ which is about... $7$ Use the integer part, $7$, as an approximation for the reciprocal, to get a second approximation of:
	
	Therefore we get:
	
	This expression is named as we know the "continued fraction representation of the number".
	
	Dropping some of the less essential parts of the expression:
	
	gives the abbreviated notation:
	 
	Note that it is customary to replace only the first comma by a semicolon.

	 As generalization of the previous example let us consider in a first time the rational number $a / b$ with $(a,b)=1$ with $b>0$ and $a>b$. We know that all the quotients $q_i$ and the remaining $r_i$ are within the scope of the Euclidean division positive integers.

	Let us recall that the Euclidean algorithm already seen earlier (but written in a slightly different way):
	
	By successive substitutions, we get:
	
	What is also sometimes written:
	
	So any positive rational number can be expressed as a finite continued fraction where $q_n\in\mathbb{N}$.
	
	Taking our introducing example:
	
	we notice indeed that $q_n\in\mathbb{N}$ and that we have by construction:
	
	where the brackets represent the integer part and that we also have:
	 
	The development of the number $a / b$ is named the "\NewTerm{development of the number $a / b$ in finite continued fraction}\index{infinite fraction}" and is condensed in the following notation:
	
	Let us see now another example:
	\begin{tcolorbox}[colframe=black,colback=white,sharp corners]
	\textbf{{\Large \ding{45}}Example:}\\\\
	Let us see how the extract the square root of a number $A$ (for example $A=2$ such that we want to extract $\sqrt{2}$) by the continued fraction method.

	Given $a$ the largest integer whose square $a^2$ is smaller than $A$. We subtract it to $A$. So there is a remaining of (for $A=2$, we have $a=1$):
	
	where we have used a remarkable identities that we will prove in the section Calculus later. Hence dividing both members by the second parenthesis, we have:
	
	Therefore:
	
	In the denominator, we replace $\sqrt{A}$ by:
	
	That gives:
	
	etc... we thus see that the system is simple to determine the expression of a root square in terms of continued fraction.
	\end{tcolorbox}
	We consider now as intuitive that every rational number can be expressed as finite continued fraction and conversely that any finite continued fraction represents a rational number. By extension, an irrational number is represented by an infinite continued fraction!

	Now consider $[q_1;q_2,q_3,\ldots,q_n]$ a finite continued fraction. The continued fraction:
	
	where $k=1,2,\ldots,n$ is named the "\NewTerm{$k$-th reduced}\index{$k$-reduced}" or "\NewTerm{$k$-th convergent}\index{$k$-convergent}" or the "\NewTerm{$k$-th partial quotient}\index{$k$-th partial quotient}".

	With this notation, we have:
	
	To simplify the expressions above, we introduce the the sequences $\{n_i\},\{d_i\}$ ($n$ is for \textbf{n}umerator and $d$ for \textbf{d}enominator) defined by:
	
	thanks to this construction, we have an interesting immediate little inequality that will be useful to us further below:
	
	With the above definition, we find that:
	
	Either by generalizing:
	
	Now let us show for later use that for $i\geq 1$, we have:
	
	The result is immediate for $i=1$. Assuming that the result is true for $i$ let us show that it is true for $i+1$. Since:
	
	then using the induction hypothesis, we get the result!
	
	We can now establish a vital relation for what will follow.
	\begin{theorem}
	Let us prove that if $C_k$ is the $k$-th reduced to of the simple finite continued fraction $[q_1;q_2,\ldots,q_n]$ then:
	
	\end{theorem}
	\begin{dem}
	
	as:
	
	therefore:
	
	indicating to us that the sign of $C_{k+1}-C_k$ is the same as $(-1)^{k+1}$.

	It follows that $C_{k+2}>C_k$ for an odd $k$, and $C_{k+2}<C_k$ for $k$ even. Then:
	
	and after as:
	
	So for $k$ even, we have $C_k>C_{k-1}$, we therefore deduce that:
	
	\begin{flushright}
		$\square$  Q.E.D.
	\end{flushright}
	\end{dem}
	Let us show now that every infinite continued fraction can be any irrational number.
	
	In formal terms, if $\{q_n\}$ is a sequence of positive integers and that we consider $C_n=[q_1;q_2,\ldots,q_n]$ then it necessarily converges to a real number if $n\rightarrow +\infty$.

	Actually it is not difficult to observe (it's quite intuitive) with a practical example that we have:
	
	when $k\rightarrow +\infty$.
	
	Now, let us denote by $x$ any real number and $q_1=[x]$ the integer part of this real number. Then we saw at the beginning of our study of continued fractions that:
	
	Therefore it comes:
	
	
	Let's look for the needs of the section on Acoustics on the calculation of a continued fraction of a logarithm using the previous relation!

	First let us recall that:
	
	That is (relation proved in the section of Functional Analysis):
	
	with $1<a<u$ and $(a,u)=1$.
	
	Given $y_n$ defined by:
	
	Therefore let us prove that:
	
	Indeed for $n=1$ we have:
	
	for $n=2$ we use first the fact that:
	
	
	Therefore:
	
	and as we had proved that:
	
	etc... by induction demonstrating our right to use this notation changes.
	
	\begin{tcolorbox}[colframe=black,colback=white,sharp corners]
	\textbf{{\Large \ding{45}}Example:}\\\\
	Let us look for the expression of the continuous fraction of:
	
	We know by playing with the definition of the logarithm that:
	
	therefore:
	
	therefore $q_1=1$. Then we have:
	
	and as:
	
	it comes:
	
	So we have the first partial quotient:
	
	Verbatim we have already:
	
	Let us simplify:
	
	So the first partial quotient can be written:
	\end{tcolorbox}
	
	\begin{tcolorbox}[colframe=black,colback=white,sharp corners]
	
	and let us go to the second partial quotient. We already know for this that:
	
	So it is immediate that $q_2=1$ and then as:
	
	We have:
	
	Finally we get:
	
	etc... etc.
	\end{tcolorbox}
	
	\begin{flushright}
	\begin{tabular}{l c}
	\circled{90} & \pbox{20cm}{\score{4}{5} \\ {\tiny 21 votes, 68.57\%}} 
	\end{tabular} 
	\end{flushright}
	
	
	%to make section start on odd page
	\newpage
	\thispagestyle{empty}
	\mbox{}
	\section{Set Theory}
	\lettrine[lines=4]{\color{BrickRed}D}uring our study of numbers, operators, and number theory (in the chapters of the same name), we often used the terms "groups", "rings", "body", "homomorphism", etc. and thereafter we will continue to do it again many times. Besides the fact that these concepts are of utmost importance, to give demonstrations or build mathematical concepts essential to the study of contemporary theoretical physics (quantum field physics, string theory, standard model, ... ), they allow us to understand the components and the basic properties of mathematics and its operators by storing them in separate categories. So, choose to present the Set Theory as the 5th chapter of this book is a very questionable choice as rigorously that it is where almost everything begins... However, we still needed to expose the Proof Theory  for the notations and methods that will be used here.
	
Moreover, when teaching modern mathematics in the secondary or primary (in the years 1970), the language of sets and the preliminary study of binary relations to a more rigorous approach to the notion of functions and applications of mathematics in general was introduced (see definition below).

\textbf{Definition (\#\mydef):} We talk about "\NewTerm{arrow diagram}\index{arrow diagram}" ( or "\NewTerm{sagittal diagram}\index{sagittal diagram}" from latin "sagitta" = arrow) to all diagram showing a correspondence between the two sets of components connected wholly or partially by a set of arrows.

For example, the graphical representation of a defined function of the set $E=\left\lbrace -3,-2,-1,0,1,2,3\right\rbrace $ to the set $F=\left\lbrace 0,1,2,4,9\right\rbrace $ lead to the sagittal diagram below:

\begin{figure}[H]
\centering
\includegraphics{img/arithmetics/figure1.eps}
\caption{Sagittal Diagram example from a definition set to a destination set}
\end{figure}

A relation from $E$ to $E$ provide an arrow diagram of the type:

\begin{figure}[H]
\centering
\includegraphics{img/arithmetics/figure2.eps}
\caption{Function returning in its own set of definitions}
\end{figure}

The closure of each element showing a "\NewTerm{reflexive relation}\index{reflexive relation}" and the systematic presence of a back arrow indicating a "\NewTerm{symmetrical relation}\index{symmetrical relation}".

\textbf{Definition (\#\mydef):} If the target set is identical to the original set, we say that we have a "\NewTerm{binary relation}\index{binary relation}".

However choosing to introduce the Set Theory in school classrooms has also some other reason. In fact, for the sake of internal rigor (i.e. not related to reality), a very large part of mathematics was rebuilt within a single axiomatic framework, so named "\NewTerm{Set Theory}\index{set theory}", in the sense that each mathematical concept (previously independent of the other) is returned to a definition where all the logical components come from this same framework: it is regarded as fundamental! Thus, the rigor of reasoning carried out within Set Theory is guaranteed by the fact that the frame is "non-contradictory" or "consistent". Let us see now the definitions that build this framework.

\textbf{Definitions (\#\mydef):}

\begin{itemize}
	\item[D1.] We name "\NewTerm{set}\index{set}" any list, collection or gathering of well-defined objects, explicitly or implicitly.
	
	\item[D2.] A "\NewTerm{Universe}\index{Universe (mathematics)}" $U$ is an object whose constituents are sets .\\\\
	Note that what mathematicians name "Universe" is not a set! In fact it is a model that satisfies to the axioms of sets.\\\\
	Indeed, we will see that we can not talk about the set of all sets (because this is not a set) to designate the object that consists of all the sets and that's why we talk about "Universe".

	\item[D3.] We name "\NewTerm{elements}\index{elements (mathematics)}" or "\NewTerm{members of the set}\index{members of a set}" objects belonging to the set and we write:
	
	if $p$ is an element of the set $A$ and in the contrary case:
	
	If $B$ is a "\NewTerm{part}\index{part of a set}" of $A$, or "\NewTerm{subset}\index{subset}" of $A$, we write this:
	
	Thus:
	
	
	\begin{tcolorbox}[colframe=black,colback=white,sharp corners]
	\textbf{{\Large \ding{45}}Examples:}\\\\
	E1. $A=\lbrace 1,2,3 \rbrace$\\\\
	E2. $X=\lbrace X \mid x\:\text{is a positive integer} \rbrace$
	\end{tcolorbox}
	
		\item[D4.] We can provide sets with a number of relations that compare (useful sometimes...) their elements or to compare some of their properties. These relations are named "\NewTerm{comparison relations}\index{comparison relations}" or "\NewTerm{order relations}\index{order relations}" (\SeeChapter{see section Operators}).

\end{itemize}

	\begin{tcolorbox}[title=Remarks,colframe=black,arc=10pt]
	\textbf{R1.} The structure of ordered set has original been set up in the framework of the Numbers Theory by Cantor and Dedekind .\\
	
	\textbf{R2.} As we have proved in the chapter on Operators, $\mathbb{N},\mathbb{Z},\mathbb{Q},\mathbb{R}$ are totally ordered by the usual relations $\leq,\geq$. The relation $<$, often named "\NewTerm{strict order}\index{stric order}" is not an order relation because not reflexive and not antisymmetric (\SeeChapter{see section Operators}). For example, in $\mathbb{N}$ the relation "$a$ divides $b$" , often denoted by the symbol "|" is a partial order.\\
	
	\textbf{R3.} If $R$ is an ordering on $E$ and $F$ is a subset of $E$, the restriction to $F$ of the relation $R$ is an order on $F$, named "\NewTerm{order induced by $R$ in $F$}".\\
	
	\textbf{R4.} If $R$ is an order on $E$, the relation $R '$ defined by:
	\begin{gather*}
		xR'y \Leftrightarrow yRx
	\end{gather*}
	
	is an order on $E$, named "\NewTerm{reciprocal order}\index{reciprocal order}" of $R$. The reciprocal order $\leq$ of the usual order is the order noted $\geq$ and reciprocal order to the order "$a$ divides $b$" in $\mathbb{N}$ is the order "$b$ is a multiple of $a$".
	\end{tcolorbox}

The set is the basic mathematical entity whose existence is defined: it is not defined as itself but by its properties, given by the axioms. It uses a human process: a kind of categorization feature, which allows thought to distinguish several independent qualified elements.

\begin{theorem}
We can demonstrate from these concepts, that the number of subsets of a set of cardinal $n$ is $2^n$.
\end{theorem}

\begin{dem}
First there is the empty set $\varnothing$, that is 0 items Chosen from $n$, i.e. $C_{0}^{n}$ (notation of binomial coefficient non-conform with ISO 31-11!) as we have seen in chapter Probabilities:


and so on...

	The number of subsets (cardinal) of $E$ corresponds to the summation of all binomial coefficients:
	

	But, we have (\SeeChapter{see section Algebraic Calculation}):
	

	therefore:
	
	\begin{flushright}
		$\square$  Q.E.D.
	\end{flushright}
	\end{dem}

	\begin{tcolorbox}[colframe=black,colback=white,sharp corners]
	\textbf{{\Large \ding{45}}Example:}\\\\
	Consider the set $S=\left\lbrace x_1,x_2,x_3\right\rbrace $, we have the set of all parts of  $P(S)$ consisting of:
	
	\begin{itemize}
		\item[$-$] The empty set: $\left\lbrace \right\rbrace =\varnothing$
		\item[$-$] The singletons: ${x_1},{x_2},{x_3}$
		\item[$-$] The duets: ${x_1,x_2},{x_1,x_3},{x_2,x_3}$	
		\item[$-$] Itself: $\left\lbrace x_1,x_2,x_3\right\rbrace $
	\end{itemize}
	
	Such that:
	
	
	What makes effectively 8 elements!
	\end{tcolorbox}

	\begin{tcolorbox}[title=Remark,colframe=black,arc=10pt]
The order in which the elements are differentiated does not come into account when counting parts of the original set.
	\end{tcolorbox}

In Applied Mathematics, we work almost exclusively with sets of numbers. Therefore, we will limit our study of definitions and properties of these.

Now let us formalize the basic concepts for working with the most common sets we encounter in the basic school curriculum.

	\subsection{Zermelo-Fraenkel Axiomatic}	

	The Zermelo-Fraenkel axiomatic, abbreviated sometimes "\NewTerm{ZF-C axioms}\index{Zermelo-Fraenkel axiomatic}" shown below was formulated by Ernst Zermelo and Abraham Adolf Fraenkel specified by the early 20th century and completed by the axiom of choice (hence the capital C in ZF-C). It is considered as the most natural axiomatic structure in the context of set theory.

	\begin{tcolorbox}[title=Remark,colframe=black,arc=10pt]
	There are many other axiomatic structures, based on the more general concept of "class", as developed by von Neumann, Bernays and Gödel (for the notations, see section Proof Theory).
	\end{tcolorbox}
	
	Strictly technically speaking..., the ZF axioms are statements of calculation for first order predicate (\SeeChapter{see section Proof Theory}) egalitarian in a language with only one primitive symbol for membership (binary relation). The following should therefore only be seen as an attempt (...) to express in English the expected significance of these axioms.

	\begin{itemize}
		\item[A1.] Axiom of extensionality:
		
		Two sets are equal if and only if they have the same elements. This is what we note :
		
		So $A$ and $B$ are equal if every element $x$ of $A$ is also in $B$ and every element $x$ of $B$ also belongs to $A$.
		
		\item[A2.] Axiom of empty set:
		
		The empty set exists, we note it:
		
		and it has no element, its cardinality is therefore 0.
		In fact this axiom can be deduced from another axiom that we will see a little further but it is convenient to introduce it by convenience for teaching in high-school classes.
		
		\item[A3.] Axiom of pairing:
		
		If $A$ and $B$ are two sets, then, there exist a set $C$ containing $A$ and $B$ alone and as components. This set $C$ is then noted ${A,B}$.
		From the perspective of the sets considered elements that gives: 
		
		This axiom also shows the existence of the "\NewTerm{singleton}\index{singleton}" a set noted:
		
		which is a set whose only element is $X$ (and therefore with unitary cardinal). We simply need to apply the axiom asking equality between $A$ and $B$.
		
		\item[A4.] Axiom of the sum (also named "axiom of union"):
		
		This axiom allows us to build the union (merge) of sets. Said in a most common way: the union (merge) of any family of a set, is... a set.
		The union of any family of sets is often noted:
		
		or if we take some of its elements:
		
		
		\item[A5.] Axiom of subsets:
		
		He expressed that for any set $A$, the set of all its parts $P(A)$ exists (do not confuse with the "$P$" of probability!).
		So for any set $A$, we can associate a set $B$ which contains exactly the parts C (verbatim the subsets) of the first:
			
		
		\item[A6.] Axiom of infinity:
		
		This axiom express the fact that there exist an infinite set. To formalize it, we say that there exist a set, named "\NewTerm{autosuccessor set}\index{autosuccessor set}" $A$ containing $\varnothing$ (the empty set) such that if $x$ belongs to $A$, then also $x \cup \lbrace x\rbrace $ belongs to $A$:
		
		This axiom expresses for example that the set of integers exists. Indeed, $\mathbb{N}$ is so the smallest autosuccessor set in the sense of inclusion $\mathbb{N}=\lbrace \varnothing ,\lbrace \varnothing , \lbrace \varnothing , ...\rbrace \rbrace \rbrace$ and by convention we note (where we build the Natural Set):
		
			
		\item[A7.] Axiom of regularity (also named "foundation axiom"):
		
		The main purpose of this axiom is just to eliminate the possibility of having $A$ as part of itself.
		
		Thus, for any non-empty set $A$, there exists a set $B$ which is an element of $A$ such that no element of $A$ is an element of $B$ (you  must distinguish the level of the language used, a set and its elements have not the same status!) that we note:
		
		and thus result we expected to have:
		
		\begin{dem}
		Indeed, let $A$ be a set such that $A \in A$. Consider the singleton $\lbrace A \rbrace$, set whose only element is $A$. According to the axiom of foundation, we must have an element of this singleton that has no element in common with him. But the only possible element is $A$ itself , that is to say that we must have:
		
		But by hypothesis $A \in A$ and by construction $A \in \lbrace A \rbrace$. So:
		
		which contradicts the previous assertion. Therefore:
		
		\end{dem}
		\begin{flushright}
			$\square$  Q.E.D.
		\end{flushright}
		
		\item[A8.] Axiom of replacement (also named "Axiom schema of replacement"):
		
		This axiom expresses the fact that if a formula $f$ is a functional then for any set $A$, there is a set $B$ consisting precisely of the images of $A$ by this function.
	
		So, in a little more formally way, the set $A$ of elements $a$ and a binary relation $f$ (which is quite generally a functional), there exist a set $B$ consisting of elements $b$ such that $f(a, b)$ is true. If $f$ is a function where $b$ is not free then it means that:
		
		In a technical way we write this axiom as following:
		
		So for every set $A$ and any item it contains, there is one and only one $b$ defined by the functional $f$ such that there exists a set $B$ for which any element a belonging to the set $A$ there is a $b$ belonging to set $B$ defined by the functional $f$.
	
		Let's see an example with the following binary predicate that for the value of any $a$ from $A$ determines the value of any $b$ of $B$:
		
		Therefore from the knowledge that $a$ is equal 1 we derive that $b$ is equal 2 and similarly (i.e. by replacement) when $a$ is equal 3, we derive that $b$ is equal 4.
	
		We see well through this small example the strong relation that exists considerating the predicate $P$ as a naive function! Moreover, as there an infinity of possible functions f, the replacement scheme is considered as an infinite number of axioms.
	
		\item[A9.] Axiom of selection (also named "Axiom comprehension schema"):
		
		This axiom simply expresses that for any set $A$ and any property $P$ expressible in the language of set theory, the set of all elements of $A$ satisfying the property $P$ exist.
	
		So more formally, to any set $A$ and any condition or proposition $P(x)$, there is a set $B$ whose elements are exactly the elements $x$ of $A$ for which $P(x)$ is true. This is what we write:
		
		In a more comprehensive and rigorous way we have in fat for any functional $f$ that does not include $a$ as free variable:
		
		It is typically the axiom that we use to construct the set of even numbers:
		
		or to prove the existence of the empty set (which invalidates the axiom of the empty set) because you just have to ask that there exist a set that satisfies the property:
		
		and regardless of the set $A$. And only the empty set satisfies this property by the selection axiom.
		
		The compliance with the strict conditions of this axiom eliminates the paradoxes of the "\NewTerm{naive set theory}\index{naive set theory}", as Russell's paradox or Cantor's paradox who invalidated the naive set theory.
		
		For example, consider the Russell set $R$ of all sets that do not contain themselves (note that we give a property of $R$ without specifying what is this set):
		
		The problem is to know whether or not $R$ contains itself or not. If $R \in R$, then, $R$ is self-contained, and, by definition $R \not\in R$, and vice versa. Each possibility is contradictory.
	
		If we now denote by $C$ the set of all sets (Cantor Universe) we have in particular:
		
	
		which is impossible (i.e. with the power of the continuum of real numbers), according to Cantor's theorem (\SeeChapter{see section Numbers}).
	
		These "paradoxes" (or "syntactic antinomies") come from a non-compliance with the conditions of application of the selection axiom: to define $E$ (in the example of Russell), there must be a proposition $P$ which bears on the set $R$, which should be explicated. The proposal defining the set of Russell or that of Cantor does not indicate what is the set $E$. It is therefore invalid!
	
		A very nice and well known example (this is why we present it) helps to better understand (this is the "Russel paradox" which we have already spoken about int length in the section on Proof Theory):
	
		A young student went one day to his barber. He entered into conversation and asked him if he had many competitors in his pretty city. Seemingly innocent way, the barber replied, «I have no concurrence. Because of all the men of the city, I obviously do not shave those who shave themselves, but I am fortunate to shave. all those who do not shave themselves».
	
		What then in such a so simple statement could take to the fault the logic of our young smart student?
	
		The answer is in fact innocent, until we decide to apply to the case of the barber: Does he shaves himself, Yes or No?
	
		Suppose he shaves himself: he then belongs to the category of those who shave themselves, those who the barber said he did of course not shave.... So he does not shave himself..........
	
		Finally, this unfortunate barber is in a strange position: if he shaves himself, he does not shave himself, and if he does not shave himself, he shaves himself. This logic is self-destructive, contradictory stupidly, rationally irrational.
	
		Then comes the selection axiom: We exclude the barber of all persons to which the declaration applies. Because in reality, the problem is that the barber is a member of the set of all the men of the city. So what applies to all men does not apply to the individual case of the barber.
	
		\item[A10.] Axiom of choice:
		
		Given a set $A$ of non-empty mutually disjoint sets, there exist a set $B$ (the set of choices for $A$) containing exactly one element for each member $A$.
		
		However let us indicate that the issue of the axiomatization and therefore of the foundations found himself still shaken by two questions at the time of their construction: what valid axioms must be chosen and in a system of axioms are the mathematics coherent (do we not have a risk of seeing a contradiction)?
		
		The first issue was first raised by the continuum hypothesis: if we can put two sets of numbers in correspondence term to term, they have the same number of elements (cardinal). We can thus map all integer numbers with rational numbers as we have shown in the section on Numbers, so they have the same cardinality, be we can not map integer numbers with all the real numbers. The question then is whether there is a set whose number of elements would be located between the two or not? This question is important to build the classical theory of analysis and mathematicians usually choose to say there is none, but we can also say the opposite.
	
		In fact the continuum hypothesis is linked in a more profound way we could thing to the axiom of choice which can also be formulated as follows: if $C$ is a collection of non-empty sets then we can select any element of each set of the collection. If $C$ has a finite number of elements or a countable number of elements, the axiom seems pretty trivial: we can sort and number the sets of $C$ and the selection of an element in each set is simple. Where it begin to get complicated is when the set $C$ has the power of the continuum: how to choose the elements if it is not possible to number them?
	
		Finally in 1938 Kurt Gödel shows that set theory is consistent without the axiom of choice and without the continuum hypothesis as well as with! And to end it all Paul Cohen in 1963 shows that the axiom of choice and the continuum hypothesis are not related.
	\end{itemize}
	
	Ok to make a pedagogical summary of all this stuff consider the following figure (excluding the axiom of choice):
	\begin{figure}[H]
		\centering
		\includegraphics{img/arithmetics/zf_axioms.jpg}	
		\caption{Zermelo-Frankel axioms visual summary (source:?)}
	\end{figure}

\subsubsection{Cardinals}

\textbf{Definition (\#\mydef):} Sets are said to be "\NewTerm{equipotent}\index{equipotent}" if there exists a bijection (one-one correspondence) between these sets. We thus say they have same "\NewTerm{cardinal}\index{cardinal}" that the norm ISO 3111 advocated to write $card(S)$ but in this book we will also use the notation $\mathrm{Card}(S)$ (many U.S. books use non-official notation that looks exactly like the absolute value $\mid S \mid$ or $\# S$).

Thus, more rigorously, a cardinal (which quantifies the number of items in the set) is an equivalence class (\SeeChapter{see section Operators}) for the relation of equipotence.

	\begin{tcolorbox}[title=Remark,colframe=black,arc=10pt]
Cantor is the main creator of set theory, in a form that we name today "naive set theory". But, apart to elementary considerations, his theory was also consisting of higher abstraction levels. The real novelty of the Cantor theory is that it lets talk about infinity. For example, an important idea Cantor was precisely to define the "equipotence".
	\end{tcolorbox}
	
If we write $c_1=c_2$ as equality of cardinals, we mean by that there are two equipotent sets $A$ and $B$ such that:
	
	
Cardinals can be compared. The order thus defined is a total ordering (\SeeChapter{see section Operators}) between the Cardinals (the proof that the order relation is complete uses the axiom of choice and the proof that it is antisymmetric is known under the name of Cantor-Bernstein's theorem that we will demonstrate later below).

	Say that $c_1<c_2$ means in simple language that $A$ is equipotent to a proper part of $B$, but $B$ is not equipotent to any own part of $A$. Mathematicians would say that $\mathrm{Card}(A)$ is smaller or equal to the $\mathrm{Card}(B)$ if there is an injection of $A$ into $B$.
	
	We saw during our study of numbers (\SeeChapter{see section Numbers}), especially of transfinite numbers, that an equipotent set (or bijection) to $\mathbb{N}$ was told to "\NewTerm{countable set}\index{countable set}".
	
	Let us now see this notion a little more in detail:
	
	Let $A$ be a set, if there is an integer $n$ such that there is at least for each element of $A$  a corresponding item int the set $\left\lbrace 1,2, ...,n\right\rbrace $ (in fact this is rigorously a bijection... concept that we will define later) then we say that the cardinal of $A$, denoted $\mathrm{Card}(A)$ or $\mathrm{Card}(A)$ is a "\NewTerm{finite cardinal}\index{finite cardinal}" and its value is $n$.
	
	Otherwise, we say that the set $A$ has an "\NewTerm{infinite cardinal}\index{infinite cardinal}" and we write:
	

A set $A$ is "\NewTerm{countable}\index{countable set}" if there is a bijection between $A$ and $\mathbb(N)$. A set of numbers $A$ is "\NewTerm{countable}" if there is a bijection between $A$ and part of $\mathbb(N)$. A set at maximum countable is thus of finite cardinal, or countable.

We can therefore check the following proposals: 
\begin{itemize}
	\item[P1.] A part of a countable set is at most countable.

	\item[P2.] A set containing a non-countable set is also not countable.

	\item[P3.] The product of two countable sets is countable.
\end{itemize}

	\begin{tcolorbox}[title=Remark,colframe=black,arc=10pt]
We can restrict a set of numbers relatively with the null element and the negative or positive elements in it and therefore we write (example for the real set):

These concepts are similar for $\mathbb{N},\mathbb{Z},\mathbb{Q}$ (the set of complex numbers $\mathbb{C}$ being not ordered, the second and third line does not apply to).
	\end{tcolorbox}
So any infinite subset of $\mathbb{N}$ is equipotent to $\mathbb{N}$ itself, what may seem counter-intuitive at first...!

In particular, there are as many even integers as any natural integer numbers (use the bijection $f(n)=2n$) from $\mathbb{N}$ to $P$, where $P$ is the set of even natural numbers. As many relative numbers as  integers, as many integers as rational numbers (see the section on Numbers for the proofs).

Thus we can write:
	
and more generally, any infinite part of $\mathbb{Q}$ is countable.

Thus we have an important result: any infinite set therefore has an infinite countable part.

Since we have proved in the section on Numbers that the set of real numbers has the "\NewTerm{power of the continuum}\index{power of the continuum}" and that the set of natural numbers has transfinite cardinal $\aleph_0$, Cantor raised the question whether there was a cardinal between the transfinite cardinal $\aleph_0$ and the cardinal of $\mathbb{R}$? In other words, we have an infinite amount of integers, and an even greater amount of real numbers. So does it exist an infinite greater  than the infinite of integers  and smaller than that of the real numbers?

The problem arose by writing $\aleph_0$ the cardinal  of $\mathbb{N}$ and $\aleph_1$ (new) the cardinal of $\mathbb{R}$ and offering to demonstrate or contradict that:
	
according to the combinatorial law that gives the number of elements that we can get from from all subsets of a set (as we have proved it before).

The rest of his life, Cantor tried, in vain, to prove this result that we name the "\NewTerm{continuum hypothesis}\index{continuum hypothesis}". He did not succeed and descended into madness. In 1900, the International Congress of Mathematicians, Hilbert considered that this was one of the 23 major issues that should be resolved in the 20th century.

This problem is solved in a rather surprising way. First, in 1938, one of the greatest logicians of the 20th century, Kurt Gödel showed that the hypothesis of Cantor was not rebuttable, that is to say, we could never prove that it was false. Then in 1963, the mathematician Paul Cohen closed the debate. He demonstrated that we could never prove that it was true!!! We can conclude rightly that Cantor had become mad to try to demonstrate a problem that could not be proved.

\subsubsection{Cartesian Product}

If $E$ and $F$ are two sets, we name "\NewTerm{Cartesian product of $E$ by $F$}\index{cartesian product}" the set noted $E \times F$ (not to be confused with the vector product notation) consisting of all possible pairs $(e,f)$ where $e$ is an element of $E$ and $f$ an element of $F$.

More formally:
	
We note the Cartesian product of $E$ by itself:
	
and then we say that $E^2$ is the "\NewTerm{set of pairs of elements of $E$}".

We can perform the Cartesian product of a sequence $E_1 \times E_2 \times ... \times E_n$ of sets and get all $n$-tuples $(e_1,e_2,...,e_n)$ where $e_1 \in E_1,e_2 \in E_2,...,e_n \in E_n$.

In the case where all sets $E_i$ are identical to $E$, the Cartesian product is obviously noted $E^n$. We then say that $e^n$ is the "\NewTerm{set of all $n$-tuples of elements of $E$}".

If $E$ and $F$ are finite then the Cartesian product $E \times F$ is finished. Moreover:
	
From here we see that if the sets $E_1, E_2, ..., E_n$ are finished then the Cartesian product $E_1 \times E_2 \times ... \times E_n$ is finished and we have:
	
In particular:
	
if $E$ is a finite set.

	\begin{tcolorbox}[colframe=black,colback=white,sharp corners]
	\textbf{{\Large \ding{45}}Examples:}\\\\
	E1. If $\mathbb{R}$ is the set of real numbers, then  $\mathbb{R}^2$ is the set of all couples of real numbers. In the plane reported to a referential, any point $M$ has the coordinates that are an element of $\mathbb{R}^2$.\\
	
	E2. When we run two dice whose faces are numbered 1 through 6, each die can be symbolized by by the set $E=\left\lbrace 1,2,3,4,5,6\right\rbrace $. The outcome of a roll of dies is then an element of $E^2=E \times E$. The Cardinal of $E \times E$ is then 36. There is therefore 36 possible results when we launch two dices whose faces are numbered 1 to 6.
	\end{tcolorbox}

	\begin{tcolorbox}[title=Remark,colframe=black,arc=10pt]
Set theory and the concept of cardinal is the theoretical basis of relational database softwares.
	\end{tcolorbox}
	
\subsubsection{Intervals}

	Let $M$ be a set of any numbers so that $M \subset \mathbb{R}$ (particular but frequent example). We have for definitions.
	\begin{itemize}
		\item[D1.] $x \in \mathbb{R}$ is named "\NewTerm{upper bound}\index{upper bound}" of the set $M$, if $x \geq m$ for $\forall m \in M$. Conversely, we speak about "\NewTerm{lower bound}\index{lower bound}" (so do not confuse the concept of terminal with the concept of interval!).
		\item[D2.] Either $M \subset \mathbb{R},M \neq \varnothing$. $x \in \mathbb{R}$ is named the "\NewTerm{smallest upper bound}\index{smallest bound}" noted:
		
	of $M$ if $x$ is an upper bound of $M$ and if for any upper bound $y \in \mathbb{R}$ we have $x \leq y$.  Conversely, we speak about the "\NewTerm{smaller lower bound}\index{smaller lower bound}" that we note:
				
	\end{itemize}
	The definitions are equivalent in the context of functional analysis (see section of the same name) as the functions are defined on sets.

	Indeed, let $f$ be a function whose domain of definition $I$ swept all $\mathbb{R}$. We note that:
		
and let $x_0 \in \mathbb{R}$.

\textbf{Definitions (\#\mydef):}
	\begin{itemize}
		\item[D1.] We say that $f$ has a "\NewTerm{global maximum}\index{global maximum}" on $x_0$ if:
		
		
		\item[D2.] We say that $f$ has a "\NewTerm{global minimum}\index{global minimum}" on $x_0$ if:
		
		In each of these cases, we say that $f$ has an "\NewTerm{global extremum}\index{global extremum}" on (it is a concept that we often use in the sections of Analytical Mechanics and Numerical Methods!).
		\begin{figure}[H]
			\centering
			\includegraphics{img/arithmetics/global_local_maximum_minimum.jpg}	
			\caption{Global/Local Maximum and Minimum example (source: Wikipedia)}
		\end{figure}
		
		\item[D3.] $f$ is "\NewTerm{upper bounded}" if there is a real number $M$ such as $\forall x \in I, f(x) \leq M$. In this case, the function has an upper bound of $f$ on its domain of definition $I$ traditionally denoted:
		
		
		\item[D4.] $f$ is "\NewTerm{lower bounded}" if there is a real $M$ such that $\forall x \in I, f(x) \geq M$. In this case, the function has a lower bound of $f$ on its domain of definition $I$ traditionally denoted:
		
		
		\item[D5.] We say that $f$ is "\NewTerm{bounded}\index{bounded function}" if it is both lower bounded and upper bounded (typically the case of trigonometric functions).
	\end{itemize}

	\subsection{Set Operations}	
	
	We can build from at least three sets $A, B, C$ all sets operations (which notations are due to Dedekind) existing in set theory (very useful in the study of probability and statistics).

	\begin{tcolorbox}[title=Remark,colframe=black,arc=10pt]
Some of the notations below will be frequently use later in relatively complex theorems, so it is necessary to understand them deeply!
	\end{tcolorbox}	

Thus, we can construct the following set operations:

\subsubsection{Inclusion}
In the simplest case, we define the "\NewTerm{inclusion}\index{inclusion}" as:
	
	In a non-specialized language here's what to you have to read: $A$ is "included" (is a "part", or is a "subset") in $B$ then for all $x$ belonging to each of these $x$ also belongs to $B$:
	\begin{figure}[H]
		\begin{center}
			\includegraphics{img/arithmetics/inclusion.eps}
		\end{center}	
		\caption{Visual example (Euler Diagram) of the inclusion}
	\end{figure}
where the $U$ in the lower right corner of the figure represents the Cantor Universe.

From this it follows the following properties:
\begin{itemize}
	\item[P1.] If $A \in B$ and $B \in A$ then it implies  $A=B$ and vice versa.
	\item[P2.] If $A \in B$ and $B \in C$ then implies  $A \in C$.
\end{itemize}

\subsubsection{Intersection}

In the simplest case, we define the "\NewTerm{intersection}\index{intersection}" as:

In a non-specialized language here's what you have to read: the "intersection" of sets $A$ and $B$ consists of all the elements that are both in $A$ and in $B$:
	\begin{figure}[H]
		\begin{center}
			\includegraphics{img/arithmetics/intersection.eps}
		\end{center}	
		\caption{Visual example (Euler Diagram) of the intersection}
	\end{figure}
More generally, if $(A_i)$ is a family of sets indexed by $i \in I$, the intersection of the $(A_i),i \in I$ is denoted:

This intersection is explicitly defined by:

That is to say the intersection of the family of indexed sets includes all $x$ that are located in each set of all sets of the family.

Given two sets $A$ and $B$, we say they are "\NewTerm{disjoint}\index{disjoint}" if and only if:


Furthermore, if:

Mathematicians note that:

and name it "\NewTerm{disjoint union}\index{disjoint union}".

We sometimes joke that knowledge is built on the disjunction... (those who understand will appreciate...).

\textbf{Definition (\#\mydef):} An collection $S={S_i}$ of non-empty sets form a "\NewTerm{partition}\index{partition}" of a set $A$ if the following properties hold:
\begin{itemize}
	\item[P1.] $\forall S_i,S_j \in S$ and $i \neq j \Rightarrow S_i \cap S_j = \varnothing$
	\item[P2.] $A= \displaystyle\bigcup_{S_i \in S} S_i$
\end{itemize}

	\begin{tcolorbox}[colframe=black,colback=white,sharp corners]
\textbf{{\Large \ding{45}}Example:}\\\\
The set of even numbers and the set of odd numbers are a partitions of $\mathbb{Z}$.
	\end{tcolorbox}

The intersection law is trivially a commutative law (see further below the definition of the concept of "law") as:


\subsubsection{Union}

In the simplest case, we define the "\NewTerm{union}\index{union}" (also sometimes named "merge") as:

In a non-specialized language here's what you have to read: the "union" (or "merge") of the sets $A$ and $B$ is the set of elements that are in $A$ plus those that are in $B$.
	\begin{figure}[H]
		\begin{center}
			\includegraphics{img/arithmetics/union.eps}
		\end{center}	
		\caption{Visual example (Euler Diagram) of the union}
	\end{figure}
More generally, if $(A_i)$ is a family of sets indexed by $i \in I$, the union of the $(A_i),i \in I$ is denoted:

This union is explicitly defined by:

That is to say that the union of the family of indexed sets includes all $x$ for which there is a set indexed by $i$ such that $x$ is included in on of the set $A_i$.

We have the following distributive properties:

The law of union $\cup$  is a commutative law (see further below the definition of the concept of "law") as:

We also name "\NewTerm{idempotences laws}\index{idempotence laws}" the relations (note that for the general culture):

and "\NewTerm{absorptions laws}\index{absorptions laws}" the relations:

The laws of intersection and union are associative, such that:

and distributive such that:


If we recall the concept of "cardinal" (see above) we have with the previously defined operations, the following relation:


\subsubsection{Difference}

In the simplest case, we define the "\NewTerm{difference}\index{difference}" as:

In a non-specialized language here's what you have to read: The "difference" of the sets $A$ and $B$ consists of all the elements found only in $A$ (and thus excluding those of $B$):
	\begin{figure}[H]
		\begin{center}
			\includegraphics{img/arithmetics/difference.eps}
		\end{center}	
		\caption{Visual example (Euler Diagram) of the difference}
	\end{figure}
	
\subsubsection{Symmetric Difference}

Let $U$ be a set. For any equation we define the "\NewTerm{symmetric difference}\index{symmetric difference}" $A\delta B$ between $A$ and $B$ by:

In a non-specialized language here's what you have to read: The "symmetric difference" of the sets $A$ and $B$ consists of all items that are only in $A$ and those found only in $B$ (we pass aside elements that are common):
	\begin{figure}[H]
		\begin{center}
			\includegraphics{img/arithmetics/symetric_difference.eps}
		\end{center}	
		\caption{Visual example (Euler Diagram) of the symetric difference}
	\end{figure}
So as we can see we have:

Some trivial properties are given below:
\begin{itemize}
	\item[P1.] Commutativity: $A \bigtriangleup B= B \bigtriangleup A$
	\item[P2.] Complementarity (see definition below): $A^c \bigtriangleup B^c = A \bigtriangleup B$
\end{itemize}

\subsubsection{Product}

In the simplest case, we define the "\NewTerm{set product}\index{set product}" or "\NewTerm{cartesian product}\index{cartesian product}" as:


In a non-specialized language here's what to you have to read: "product" (not to be confused with the multiplication or cross product of vectors) of two sets $A$ and $B$ is the set of pairs such as each element of each set is combined with all elements of the other set.

The product set of real numbers for example generates the plane where each element is defined by $X$ and $Y$ axis. 

We often find products sets in mathematics and physics when we work with functions. For example, a function of two real variables which gives real output will be written:
	
or more simply:
	
	
\subsubsection{Complementarity}

In the simplest case, we define the "\NewTerm{complementarity}\index{complementary}" as:

In a non-specialized language here's what you have to read: The "complementary" is defined as taking a set $U$ and a subset $A$ of $U$ then the complement of $A$ in $U$ is the set of elements that are in $U$ but not in $A$:
	\begin{figure}[H]
		\begin{center}
			\includegraphics{img/arithmetics/complementarity.eps}
		\end{center}	
		\caption{Visual example (Euler diagram) of the difference}
	\end{figure}
Other notations of complementarity that is sometimes found in the literature and the following book are (depending on the context to avoid confusion with other stuff):

or in the particular example above, we could also just write $B \setminus A$.

We have for properties for all $A_i$ included in any $B$:

Here are some trivial properties regarding to complementarity:

There are other very important relations that also applied to Boolean logic (\SeeChapter{see section Logic Systems}). If we consider three sets $A, B, C$ as shown below we have:


and the famous "\NewTerm{De Morgan's laws}\index{De Morgan's laws}" in set form (\SeeChapter{see section Logic Systems}), which are given by the relations:
\begin{equation}
  \addtolength{\fboxsep}{5pt}
   \boxed{
   \begin{gathered}
     \overline{A \cap B} = \bar{A} \cup \bar{B} \\
	\overline{A \cup B} = \bar{A} \cap \bar{B}
   \end{gathered}
   }
\end{equation}

We would like indicate before moving on to another topic, that a significant number of adults in employment (mostly managers) having forgotten the previous defined concepts after leaving high school must study them again when they learn the SQL language (Structured Query Language) which is the most common worldwide language to query corporate databases servers in the 20th and 21st century. Most of them learn in training centers the following scheme to build queries with joins:
	\begin{figure}[H]
		\begin{center}
			\includegraphics[scale=0.7]{img/arithmetics/sql_joins.eps}
		\end{center}	
		\caption{Common SQL query expressions with joins}
	\end{figure}


	\subsection{Functions and Applications}	

\textbf{Definition (\#\mydef):} In mathematics, an "\NewTerm{application}\index{application}" (or "\NewTerm{function}\index{function}") denoted typically $f$ - in analysis - or $A$ - in linear algebra - is the information of two sets, the departure set $E$ and arrival set $F$ (or "image of $E$"), and a relation associating each element $x$ of the departure set one and only one element of the arrival set, which we call "\NewTerm{image of $x$ by $f$}" in the analysis field we note that $f(x)$ or $f(E)$ to explicit the departure set. We name "\NewTerm{images}\index{images}" the elements of $f(E)$ and the elements of $E$ are named the "\NewTerm{antecedents}\index{antecedent}".

Then we say that $f$ is an application from $E$ to $F$ denoted:

(remember the first arrow/sagittal diagram presented at the beginning of this section), or we also say that this is an application of arguments in $E$ and values in $F$.

	\begin{tcolorbox}[title=Remark,colframe=black,arc=10pt]
Note: The term "function" is often used for applications with scalar numeric values, real or complex, that is to say when the arrival set is $\mathbb{R}$ or $\mathbb{C}$. We speak then of "real function" or "complex function". In the case of vector we prefer to use the word "application" as we already mention it in the definition.
	\end{tcolorbox}	
	
\textbf{Definitions (\#\mydef):}

\begin{enumerate}
	\item[D1.] The "\NewTerm{graph}\index{graph}" or "\NewTerm{plot}\index{plot}" (or also named "graphic" or "representative") of an application or function $f:E \mapsto F$ is the subset of the cartesian product $E \times F$ consisting of pairs $(x, f (x))$ for $x$ varying in $E$. The data of the graph $f$ determines its starting set (by projection on the first argument often denoted $x$) and image (projection on the second argument often denoted $y$).
	
	\item[D2.] If the triplet $f(E,F,\Gamma)$ is a function where $E$ and $F$ are two sets and $\Gamma \subset (E \times F)$ is a graph, $E$ and $F$ are the source and purpose of $f$ respectively. The "\NewTerm{definition domain}\index{definition domain}" or "\NewTerm{departure set}\index{departure set}" of $f$ is:
	
	
	\item[D3.] Given three non empty sets $E, F, G$, any function of $E \times F$ to $G$ is named a "\NewTerm{composition law}\index{composition law}" of  $E \times F$ with values in $G$.
	
	\item[D4.] An "\NewTerm{internal composition law}\index{internal composition law}" (or simply "\NewTerm{internal law}\index{internal law}") in $E$ is a composition law of $E \times E$ with values in $E$ (that is to say this is the case $E = F = G$).
	\begin{tcolorbox}[title=Remark,colframe=black,arc=10pt]
The subtraction in $\mathbb{N}$ is not an internal composition law although it is part of the four basic high-school arithmetic operators. But the addition in $\mathbb{N}$ is such an internal law.
	\end{tcolorbox}
		
	\item[D5.] An "\NewTerm{external composition law}\index{external composition law}" (or simply "\NewTerm{external law}"\index{external law}) in $E$ is a composition law of $F \times E$ with values in $E$, where $F$ is a separate set of $E$. In general, $F$ is a set, named "\NewTerm{scalar set}\index{scalar set}".

	\begin{tcolorbox}[colframe=black,colback=white,sharp corners]
\textbf{{\Large \ding{45}}Example:}\\\\
In the case of a vector space (see definition much lower) the multiplication of a vector (whose components are based on a given set) by a real scalar is an example of external composition law.
	\end{tcolorbox}

	\begin{tcolorbox}[title=Remark,colframe=black,arc=10pt]
An external composition law with values in $E$ is also named "\NewTerm{action of $F$ on $E$}". The set $F$ is then the field operators. They also say that $F$ operates on $E$ (keep in mind the example of the vectors mentioned above).
	\end{tcolorbox}	
	
	\item[D5.] We name "\NewTerm{image of $f$}", and note $Im(f)$, the subset defined by:
	
	Thus, "the image" of a function $f:E \mapsto F$  is the collection of $f(x)$ for $x$ browsing $E$. It is a subset of $F$.
	
	And we name "\NewTerm{kerr of $f$}\index{kerr}", and we note $\ker (f)$, the very important subset in mathematics defined by:
	
According to the figure (you must deeply understand this concept because we will reuse the ker many times to prove theorems that have important practical applications later in various chapters):
	\begin{figure}[H]
		\begin{center}
			\includegraphics[scale=0.75]{img/arithmetics/ker.eps}
		\end{center}	
		\caption{ker concept of a function}
	\end{figure}
	\begin{tcolorbox}[title=Remark,colframe=black,arc=10pt]
	\textbf{R1.} $\ker (f)$ is derived from the German "Kern", simply meaning "kernel".\\
	
	\textbf{R2.} Normally the notations $\Ima$ and $\ker$ are reserved for group homomorphisms, rings, fields and to linear applications between vector spaces and modules, etc. (see further below). We do not usually use them for any applications between any sets. But ... it does not really matter for the moment at this level of the book.
	\end{tcolorbox}	
\end{enumerate}	

Applications and functions can have a phenomenal amount of properties. Below you can found some easy one that are part of the general knowledge of the physicist (for more information about what a function is, see the section on Functional Analysis).

Let $f$ be an application or function of a set $E$ to a set $F$ then we have the following properties:

\begin{enumerate}
	\item[P1.] An application or function is said to be "\NewTerm{surjective}\index{surjective application}" if:\\\\
	Any element $y$ of $F$ is the image by $f$ of at least (we emphasize on the "at least") an element of $E$. We thus say that it is a "surjection" from $E$ to $F$. It follows from this definition, that an application or function $f:E\rightarrow F$ is surjective and denoted:
	
	if and only if $F= \Ima f$. In other words, we also write this definition as following:
	
	\begin{figure}[H]
		\begin{center}
			\includegraphics[scale=0.75]{img/arithmetics/surjective.eps}
		\end{center}	
		\caption{Schematic representation of a surjective application or function}
	\end{figure}
	
	\item[P2.] An application or function is said to be "\NewTerm{injective}\index{injective application}" if:\\\\
	Any element $y$ of $F$ is the image by $f$ of at most (we emphasize the "at most") a single element of $E$. We thus say that $f$ is an injection of $E$ to $F$. It follows from this definition, that an application or function $f:E\rightarrow F$ is injective and denoted:
		
	if and only if the relations $x_1,x_2 \in E$ and $f(x_1)=f(x_2)$ involve. In other words: an application or function for which two separate elements have distinct images is say to be "injective". Or an application or function is injective at least if one of the following equivalent properties holds: 
	\begin{enumerate}
		\item[P2.1] $\forall x,y\in E^2:\;f(x)=f(y)\Rightarrow x=y$
		\item[P2.2] $\forall x,y:\;  x\neq y \Rightarrow f(x) \neq f(y)$
		\item[P2.3] $\forall y \in F$ the equation in $x$, $y=f(x)$ has at least one solution in $E$
	\end{enumerate}
	All this can be resumed by:
	\begin{figure}[H]
		\begin{center}
			\includegraphics[scale=0.75]{img/arithmetics/injective.eps}
		\end{center}	
		\caption{Schematic representation of an injective application or function}
	\end{figure}
	\item[P3.] An application or function is said to be "\NewTerm{bijective}\index{bijective application}" or "\NewTerm{total application/function}" if:\\\\
	An application or function $f$ from $E$ to $F$ is both injective and surjective. In this case, we have that for any element $y$ of $F$, the equation $y=f(x)$ admits in $E$ a single (not "at least" or not "at most") pre-image $x$ and denoted:
	 
	In other words, we also write this definition as following:
	
	This is illustrated by:
	\begin{figure}[H]
		\begin{center}
			\includegraphics[scale=0.75]{img/arithmetics/bijective.eps}
		\end{center}	
		\caption{Schematic representation of a bijective application or function}
	\end{figure}
We are thus naturally led to define a new application from $F$ to $E$, named "\NewTerm{inverse function}\index{inverse application}" or "\NewTerm{reciprocal function}\index{reciprocal application}" of $f$ and noted $f^{-1}$ that to every element of $F$ matches the unique pre-image element $x$ of $E$ (also named sometimes "solution") of the equation $y=f(x)$. In other words:
	
The existence of an inverse (reciprocal) function or application implies that the graph of a bijective function or application (in the set of real numbers...) and that of its inverse (reciprocal) are symmetric with respect to the right of equation $y=x$.

Indeed, we notice that if $y=f(x)$ is equivalent to $x=f^{-1}(y)$, then these equations imply that the point $(x, y)$ is on the graph of $f$ if and only if the point $(y, x)$ is the graph of equation $f^{-1}$.

As you can see for example in the figure below with the sinus function (\SeeChapter{see section Trigonometry}):

	\begin{figure}[!ht]
		\begin{center}
			\includegraphics[scale=0.75]{img/arithmetics/bijective_example.eps}
		\end{center}	
		\caption{Bijective function example}
	\end{figure}

	\begin{tcolorbox}[colframe=black,colback=white,sharp corners]
\textbf{{\Large \ding{45}}Example:}\\\\
Take the case of a holiday station where a group of tourists must be housed in a hotel. Each way to allocate these tourists in hotel rooms may be represented by an application of all tourists to all the rooms (to each tourist is assigned a room).
	\begin{itemize}
		\item Tourists want the application to be injective, that is to say, each of them has a single room. This is only possible if the number of tourists does not exceed the number of rooms.
		
		\item The hotel manager hopes that the application is surjective, that is to say, each room is occupied. This is only possible if there are at least as many tourists than rooms.
		
		\item If it is possible to spread the tourists so that there is only one per room, and all the rooms are occupied: the application will then be both injective and surjective that is to say bijective.
	\end{itemize}
	\end{tcolorbox}
	
	\begin{tcolorbox}[title=Remarks,colframe=black,arc=10pt]
	\textbf{R1.} It comes from the definitions above that $f$ is bijective in the set of real numbers if and only if any horizontal line intersects the graph of the function at a single point. This leads us to the second following remark:\\

	\textbf{R2.} An application that satisfies the test of the horizontal line is continuously increasing or decreasing at any point in its domain.
	\end{tcolorbox}
	
	\item[P4.] An application or function is named "\NewTerm{composite application}\index{composite application}" or "\NewTerm{composite function}" if:
\end{enumerate}
Let $\varphi$ be an application or function from $E$ to $F$ and $\psi$ an application or function of $F$ in $G$. The application or function that associates to each element $x$ of the set $E$ an element $\psi(\varphi(x))$ of $G$  is named "\NewTerm{composed application}\index{composed application}" of $\varphi$ and $\psi$ and is denoted by:
	
	where the symbol "$\circ$" is named "\NewTerm{round}\index{round}" (do not confused with the scalar product we will see later in the section of Vectorial Calculus). Thus, the above relation is written "psi round phi" but has to be read "phi round psi" (...). So:
	
	Let, moreover, $\chi$ be an application (not a function!) of $G$ in $H$. We check immediately that the composition operation is associative for applications (for more details see the section of Linear Algebra):
	
	This allows us to omit parentheses and write more simply:
	
	In the particular case where $\varphi$ would be an application or function from $E$ to $E$, we note $\varphi^k$ the composed application $\varphi \circ \varphi \circ ... \circ \varphi$ ($k$ times).
	
	What's important in what we have seen until now in this section is that all defined properties listed above are applicable to Numbers' Sets.
	
	Let us see a concrete and very powerful example:


	\subsubsection{Cantor-Bernstein Theorem}
	Warning! This theorem, for which the result may seem trivial, is not necessarily easy to approach (its mathematical formalism is not very aesthetic...). We advise you to read the proof slowly and imagine the sagittal diagrams in your head during the development.
	
	Here is the hypothesis to prove: 
	\begin{theorem}
	Let $X$ and $Y$ be two sets. If there is an injection (remember the definition of an injective function or application above) from $X$ to $Y$ and another from $Y$ to $X$, then both sets are in bijection (remember the definition of an bijective function or application above). It is therefore an antisymmetric relation.

	This is illustrated by:
	\begin{figure}[!ht]
		\begin{center}
			\includegraphics[scale=0.75]{img/arithmetics/cantor_bernstein.jpg}
		\end{center}	
		\caption{Representation of a antisymmetric relation}
	\end{figure}
\end{theorem}
	Formally this theorem is sometimes written:
	
	or more technically:
	
	For the proof we need rigorously to demonstrate beforehand a lemma (intuitively obvious... but not formally) who's statement is as follows:

	\begin{lemma}
	Let $X, Y, Z$ three sets such that $X \subseteq Z \subseteq Y$. If $X$ and $Y$ are in bijection through a function $f$, then $X$ and $Z$ are in bijection through a function $g$.

	Technically, we write this:
	
	
	An example of application of this lemma is the set of natural numbers and rational numbers which are in bijection (see the section of Number Theory for the proof). Therefore, all the rational numbers are in bijection with the set of natural numbers since $\mathbb{N} \subseteq \mathbb{Z} \subseteq \mathbb{Q}$.
\end{lemma}

	\begin{dem}
	First, formally, we create a function $f$ from $Y$ to $X$ such that it is bijective:
	
	To continue we need a set $A$ that will be defined by the union of the images of the functions of the functions $f$ (of the kind $f(f(f ...)))$... and build such a tool is the trick of the proof!) of the pre-images of the set $Z$ (remember that $Z \subseteq Y$) from which we exclude the elements of $X$ (that we will be  noted for this proof: $Z-X$):
	\begin{figure}[H]
		\centering
		\includegraphics[scale=1]{img/arithmetics/cantor_bernstein_construction_lemma.jpg}
	\end{figure}
In other words (if the first form is not clear...) we define the set $A$ as the union of images of $(Z-X)$ by the applications $f \circ f \circ ... \circ f$. What we write in a condensed a pretty way as following:
	
Because $f:Y \mapsto X$ and that $(Z-X)\subseteq Y$ we have by construction $A \subseteq X$ and thus:
	
 	Notice that we also have:
	
and by reindexing:
	
	We then have (make a pattern in your head of the arrow diagrams can help at that level of the proof...) whatever $A$:
	
	We can elegantly prove this last equality as this is one of the main result (!):
	
	Since $Z$ can be partitioned (nothing avoid us to do this!) in two disjoint subsets (we can draw a figure on request):
	
	 and without forgetting that $X \subseteq Z \subseteq Y$ and $A \subseteq X$, now we introduce as definition the function $g$ (we don't give more information about it yet) such that:
	
and for every pre-image $a$ of $g$ of the partition $((Z-X)\cup A) \subseteq Z$ we have:
	
	This means that because $((Z-X)\cup A) \subseteq Z$ and $Z \subseteq Y$ we can thus apply (associate) the bijective function $f$ (remember that $f:Y \rightarrow X$) as equivalent of the function $g$ to any element of $((Z-X)\cup A)$.

	We also have also for every pre-image $a$ of $g$ of the partition $(X-A)$ (remember that $A \subseteq X$):
	
	this means that we just apply the identity function we could also associate it to $g$.
	
	So to sum up, we have can build an $g$ that then bijective because its restrictions to the $((Z-X)\cup A)$ is $f$ and that to $(X-A)$ is the identity which are both bijective, the first by definition, the second by construction!

	Finally there exists, by construction, a bijection between $X$ and $Z$ and we have proved indeed, the lemma that:
	
	\begin{flushright}
		$\square$  Q.E.D.
	\end{flushright}
\end{dem}

	Now that we have proved the Lemma let us recall the assumptions of the Cantor-Bernstein theorem using the result of the Lemma:
	
	Consider $\varphi$ an injection from $X$ to $Y$  and $\psi$ an injection for $Y$ to $X$ with $X \subseteq Y$. 
	
	We thus have:
	
	Hence:
	
	Then far we have:
	\begin{itemize}
		\item As $\varphi$ is injective, then $X$ and $\varphi(X)$ are by definition bijective (yes! indeed, an injective function is by definition bijective when we reduce its image set to the pre-image set\footnote{but it must be notice that this work only for ininite sets!})
		
		\item As $\psi$ is injective, $\psi(\varphi(X))$ and $\varphi(X)$ are bijective.
	\end{itemize}

	Therefore from the previous proved lemma, $X$ and $\psi(\varphi(X))$ are also in bijection!
	
	By using also the lemma on $\psi(\varphi(X))$, $\psi(Y)$ and $X$, it follows that $\psi(\varphi(X))$ is the also in bijection with $\psi(Y)$ which gives us with what have just seen, that since $\psi(\varphi(X))$ and $\varphi(X)$ are in bijection, that that $\psi(Y)$ is in bijection with $\varphi(X)$, then $X$ and $Y$ are indeed related by an injective application (phew!!! it is a beautiful reasoning but it is as vicious as simple...) and we have:
	
	This theorem can obviously be interpreted in the following way: If we can count a part of a set with all the elements of another set, and vice versa, then they have the same number of elements. This is quite obvious for finished sets. The theorem above then generalizes this notion for infinite sets and this is its strength!

	From there, this theorem represents one of the basic bricks to generalize the notion of set sizes to infinite sets.
	
	\pagebreak
	\subsection{Structures}
	The so-named "\NewTerm{modern algebra}\index{modern algebra}" or "\NewTerm{abstract algebra}" begins with the theory of algebraic structures due in part to Carl F. Gauss and especially to Évariste Galois. These structures exist in a very large number but only the fundamentals will interest us here (this book being mainly dedicated, for recall, to engineers). Before describing them, here is a synoptic diagram of these main structures and their hierarchy:
	\begin{figure}[H]
		\centering
		\includegraphics[scale=1]{img/arithmetics/structures_common.jpg}
		\caption{Synoptic Diagram of Common Algebraic Structures}
	\end{figure}
	\begin{tcolorbox}[title=Remark,colframe=black,arc=10pt]
	At the top of the diagram, the structure has the minimum number of constraints (exactly one), at the bottom, a maximum (to seven). The more we descend, the more specialized the structure is.
	\end{tcolorbox}	
	To simplify the notations, let us assume that $\star$ and $\circ$ represents a composition\footnote{This generalized notation is sometimes named "\NewTerm{stellar notation}\index{stellar notation}".} like (like the addition, subtraction, multiplication or division, ...)... then:
	
	\textbf{Definitions (\#\mydef):} Given $\star$ and $\circ $ the symbols of internal laws (this could be addition and multiplication to take the best known case) to a given set $E$ then:
	\begin{enumerate}
		\item[D1.] $\star$ is a "\NewTerm{commutative law}\index{commutative law}" if: 
		
		
		\item[D2.] $\star$ is an "\NewTerm{associative law}\index{associative law}" if:
		
		
		\item[D3.] $n$ is the "\NewTerm{neutral element}\index{neutral element}" for $\star$ if:
		
		We will also admit without proof (it is quite intuitive) that if there is a neutral element, then it is unique.
		
		\item[D4.] $a'$ is the "\NewTerm{symmetrical element}\index{symmetrical element}" (in the general sense of the "\NewTerm{opposite}\index{opposite}", for example for addition and "\NewTerm{inverse}\index{inverse}" for multiplication) of $a$ for $\star$ if:
		
		We shall also admit, and without proof, that the symmetric of any element is unique.		
		
		\item[D5.] $\circ$ is a "\NewTerm{distributive law}\index{distributive law}" with respect to $\star$ if:
		
	
		\item[D6.] $b$ is the "\NewTerm{absorbing element}\index{absorbing element}" if for all $a$ and a law $\star$ we have:
		
	\end{enumerate}
	\begin{tcolorbox}[title=Remarks,colframe=black,arc=10pt]
	\textbf{R1.} If $a$ is its own symmetric with respect to the law $\star$, mathematicians say that $a$ is "\NewTerm{involutive}\index{involutive}".\\

	\textbf{R2.} If an element $b$ of $E$ satisfies:
	
	then $b$ is say to be an "\NewTerm{absorbing element}\index{absorbing element}" for the law $\star$.\\

	\textbf{R3.} It must always be checked that the neutral and the symmetrical elements are such on the "left" \underline{and} on the "right". Thus, for example, in $(\mathbb{Z},-)$, the element $0$ is neutral to the right because $x-0=x$ but $0-x=-x$.
	\end{tcolorbox}	
	
	\subsubsection{Magma}
	\textbf{Definition (\#\mydef):} We denote a set by the term "\NewTerm{magma $M$}\index{magma}", if the components constituting it are operable with respect to an internal law $\star$:
	
	It is therefore important to remember that if we designate an algebraic structure by the term "magma", it does not mean in any case that the internal law is commutative, associative or even that it possesses a neutral element!
	
	\begin{tcolorbox}[title=Remarks,colframe=black,arc=10pt]
	\textbf{R1.} If, moreover, the internal law $\star$ is commutative, we speak of "\NewTerm{commutative magma}\index{commutative magma}".\\

	\textbf{R2.} If, moreover, the internal law $\star$ is commutative, we speak of "\NewTerm{associative magma}\index{associative magma}".\\

	\textbf{R3.} If, moreover, the internal law $\star$ possesses a neutral element $n$, we speak sometimes of "unitary associative magma" or respectively "unitary commutative magma" but we will see just below that the the both have in practice other official names.
	\end{tcolorbox}	
	
	\textbf{Definition (\#\mydef):} In a magma $(M,\star)$, an element $x$ is named "\NewTerm{regular element}\index{regular element}" (or "\NewTerm{simplifiable element}\index{simplifiable element}") to the left if for any pair $(a,b)\in M$ we have:
	
	\begin{tcolorbox}[title=Remark,colframe=black,arc=10pt]
	We also define in the same way a regular element to the right.
	\end{tcolorbox}	
	Thus, an element is say to be "\NewTerm{regular}" if it is regular to the right and to the left. If $\star$ is commutative (which is the case for a commutative magma), the notions of regular element to the left or to the right coincide.
	
	A magma $(M,\star)$ is therefore an elementary algebraic structure. There are more subtle structures (monoids, groups, rings, body, vector space, etc.) in which a set is provided with several laws and different properties. We will see them right now and use them throughout this book explicitly or implicitly.
	
	\pagebreak
	\subsubsection{Monoid}
	\textbf{Definition (\#\mydef):} If the law $\star$ is associative and has a neutral element $n$, then we say that the "unitary associative magma" is a "\NewTerm{monoid}\index{monoid}":
	
	\begin{tcolorbox}[title=Remarks,colframe=black,arc=10pt]
	\textbf{R1.} If the internal law $\star$ is commutative then we say that the structure forms an "\NewTerm{Abelian monoid}\index{Abelian monoid}" (or simply "\NewTerm{commutative monoid}\index{commutative monoid}").\\

	\textbf{R2.} In some textbooks we also find that the monoid is a "\NewTerm{semi-group}" (with an associative $\star$ law!) that has a neutral element $n$.
	\end{tcolorbox}
	Let us show immediately that the set of natural integers $\mathbb{N}$ is a totally ordered Abelian monoid (as we have partially seen it in the section Operators) with respect to the laws of addition and multiplication:
	
	The addition law "+" is an internal operation such that $\forall a,b \in\mathbb{N}$ we have:
	
	We can prove that this is indeed the case by knowing that $1$ belongs to $\mathbb{N}$ such that:
	
	Therefore $c\in\mathbb{N}$ and the addition is indeed an internal law (we also say that the set $\mathbb{N}$ is "\NewTerm{stable}\index{stable}" with respect to addition) and at the same time associative since $1$ can be added to itself by definition in any order without the result being altered. If you remember that multiplication is a law that is built on addition (\SeeChapter{see section Operators}), then the multiplication law $\times$ is also an internal and associative law!
	
	We will assume starting from here that it is trivial that the addition law $+$ is also commutative and that the zero "$0$" is the neutral element $n$. Thus, the multiplication law $\times$ is also commutative and it is trivial that "$1$" is the neutral element $n$.

	On the other hand, to speak about something that is not directly related to the monoids... but that will be useful to us a little further below, does it exist in line with the previous example for the  addition law $+$ a symmetric $\exists c$ such that $\forall a,b\in\mathbb{N}$ we have:
	
	with $c\in\mathbb{N}$?
	
	It is quite trivial that for this equality to be satisfied we must have:
	
	thus:
	
	but negative numbers do not exist in $\mathbb{N}$. This also leads us to the conclusion that the addition law $+$ has no symmetric element in $\mathbb{N}$ and that the subtraction law $-$ does not exist in $\mathbb{N}$ (the subtraction being rigorously the addition of a number negative!).
	
	Similarly, since it will also be useful to us a little further below, is there a symmetric a for the multiplication law $\times$ such that $\forall a\in\mathbb{N}$ we have:
	
	with $a'\in\mathbb{N}$?
	
	First it is obvious that:
	
	But excepted for $a=1$, the quotient $1/a$ does not exist in $\mathbb{N}$. Therefore we must conclude that there do not exist for any element of $\mathbb{N}$ for the multiplication law $\times$ and therefore that the division law $/$ does not exist in $\mathbb{N}$ and that the multiplication law does not form a monoid in this set .

	Synthesis:
	
	We have, for example, the following properties with respect to the set of natural integers $\mathbb{N}$ and the concept of monoid:
	\begin{enumerate}
		\item[P1.] $(\mathbb{N},\le,\ge)$ is completely ordered (noticed that this notation is somewhat abusive, it suffices that there is just one of the two order relations $R$ so that the set is totally ordered).
		
		\item[P2.] $(\mathbb{N},+)$ and $(\mathbb{N},\times)$ are abelian monoids.
		
		\item[P3.] The element zero "$0$" is the absorbing element for the monoid $(\mathbb{N},\times)$.
		
		\item[P4.] The laws of subtraction $-$ and division $/$ do not exist in the set $\mathbb{N}$.
		
		\item[P5.] $\mathbb{N}$ is an abelian monoid totally ordered with respect to the laws of addition $+$ and multiplication $\times$ (caution! the following notation is abusive because the monoid is composed of only one internal law and a unique order relation $R$ which would give in total $4$ unique monoids):
		
	\end{enumerate}
	\begin{tcolorbox}[title=Remarks,colframe=black,arc=10pt]
	\textbf{R1.} It is rare to use monoids in the practice of the engineer or physicist. Indeed, this is because often when we find ourselves faced with a structure too poor to really discuss about something or develop a physical model. Then we extend it to something richer, like a group, or a field (see below) such as the set of relative integers $\mathbb{Q}$ or of real number $\mathbb{R}$ (at least...).\\

	\textbf{R2.} Say that an algebraic structure is totally ordered with respect to some laws means that given a law $\star$, and $R$ an order relation and $a$, $b$, $c$, $d$ four elements of the structure concerned, then if $a\;R\;b$ and $c\;R\;d$ imply $(a\star c)R(b\star d)$, we denote this structure $(S,\star, R)$ or simply $(S, R)$ and indicating the concerned law later in the text.
	\end{tcolorbox}	
	
	\subsubsection{Groups}
	\textbf{Definition (\#\mydef):}  We designate a set by the term "\NewTerm{group}\index{group}", if the constituent components satisfy the three conditions of what we name the "\NewTerm{internal group law}", defined below:
	
	In this case, the law of internal composition $\star$ will often (but not exclusively!) be denoted "$+$" and named "\NewTerm{addition}\index{addition}", the neutral element is denoted $e$ and its value is equal to "$0$" and the symmetric of $x$ is denoted "$-x$".

	Let us emphasize that group structure is probably one of the most important in the practice of modern engineering and physics in general. This is why it is necessary to pay special attention to it (\SeeChapter{see section Set Algebra})!

	If, moreover, the internal law $\star$ is also commutative, then we say that the group is an "\NewTerm{abelian group}\index{abelian group}" or simply a "\NewTerm{commutative group}\index{commutative group}".

	If there exists in $G$ at least one element $a$ such that every element of $G$ is a power of$ $a or of the symmetric $a'$ of $a$, we say that $(G,\star)$ is a "\NewTerm{cyclic group of generator $a$}\index{cyclic group}" if it is finite, otherwise say that it is "\NewTerm{monogenic}" (we will come back in details on cyclic groups in the section of Set Algebra).

	More generally, a group $(G,\star)$ of neutral element $e$, not reduced only to $\{e\}$ will be monogeneous, if there exists an element $a$ of $G$ distinct from $e$ such that $G=\left\lbrace e,a^1,a^2,a^3,\ldots,a^n,\ldots\right\rbrace$. Such a group will be cyclic, if there exists a non-zero integer $n$ for which $a^n=e$. The smallest non-zero integer satisfying this equality is then the "\NewTerm{order of the group}\index{order of a group}". 
	
	Let us show immediately that the set of relative integers $\mathbb{Z}$ this a totally ordered Abelian group (as we have seen in the section Operators) with respect to the laws of addition $+$ and multiplication $\times$.

	First, to shorten the developments, it is useful to recall that the set $\mathbb{Z}$ is an "extension" of $\mathbb{N}$ by the fact that we have added to it all the symmetric negative sign numbers ($\mathbb{N}\subset \mathbb{Z}$).
	
	Thus, always with the abuse of notations (because normally a group has only one law $\star$ and one order relation $R$ thjat is sufficient to order it):
	
	forms a completely ordered Abelian group ($4$ groups by the way!) And:
	
	a completely ordered abelian monoid (two monoids in facts!).

	Let us also notice that the law of division does not exist for any element of the set $\mathbb{Z}$! So in general we say that the division does not exist in it!

	Synthesis:
	
	We therefore have the following properties:
	\begin{enumerate}
		\item[P1.] $(\mathbb{Z},\le,\ge)$ is completely ordered (caution! again this notation is a little abusive! It is enough that there is just one of the two order relation $R$ so that the set is totally ordered).

		\item[P2.] $(\mathbb{Z},+)$ is a commutative group whose zero "$0$" is the neutral element.

		\item[P3.] The division law does not exist in the set $\mathbb{Z}$.

		\item[P4.] The set $\mathbb{Z}$ is an abelian group totally ordered with respect to the law of addition $+$ (caution! the following notation is again abusive because the group is composed only of one order relation $R$ which would give a total of two groups):
		
		The set $\mathbb{Z}$ set is not a commutative group totally ordered with respect to the law of multiplication:
		
	\end{enumerate}
	We see then immediately that $\mathbb{Z}$ has too restricted properties, that is why it is interesting to extend it by the set of rational $\mathbb{Q}$ defined in a very simplistic way ... by (\SeeChapter{see section Numbers}):
	
	This means for recall the set of rationals $\mathbb{Q}$ is defined by the set of quotients $p$ and $q$ belonging each to $\mathbb{Z}$ of which we exclude to $q$ to take the value zero (the notation $/q$ signifying for recall the "exclusion").
	
	And we obviously have:
	
	It is therefore obvious (without proof and always using the abusive notation already commented many times earlier above...) that $(\mathbb{Q},\le,\ge)$ is also totally ordered and also that $\mathbb{Q}$ is an abelian group totally ordered with respect to the law of addition $+$ only:
	
	What becomes interesting with $\mathbb{Q}$ is that the law of multiplication becomes an internal law and forms a commutative abelian group named "multiplicative group" with respect to $\mathbb{Q}^{*}$.
	\begin{dem}
	Let us prove then that the symmetric exists for the law of multiplication $\times$ such that:
	
	Since in $\mathbb{Q}^{*}$ any number can be put in the form:
	
	with $p\in\mathbb{Z}$, $q\in\mathbb{Z}^{*}$.
	
	So since:
	
	There is therefore a symmetric to every rational in $\mathbb{Q}^{*}$ for the law of multiplication.
	\begin{flushright}
		$\square$  Q.E.D.
	\end{flushright}
	\end{dem}
	By definition, or by construction, the division exists in $\mathbb{Q}^{*}$ and is an internal operation. But is it associative such that $\forall (p,q,r)\in\mathbb{Q}^{*}$ we have:
	
	In fact, the verification is fairly trivial if we remember that division is defined from the law of multiplication of the inverse and that the latter law is... associative! So it comes:
	
	We can also ask ourselves the division law ($/$) is commutative such that the relation:
	
	with $\forall (a,b)\in\mathbb{Q}^{*}$.
	
	We see very well that this is not the case since we can write this last relation in the form:
	
	To sum up:
	
	We therefore have the following properties:
	\begin{enumerate}
		\item[P1.] $(\mathbb{Q},\le,\ge)$ is totally ordered
	
		\item[P2.] $(\mathbb{Q},+),(\mathbb{Q},\times)$ are independently totally ordered abelian groups
	
		\item[P3.] Zero "$0$" is the absorbent element with respect to the group $(\mathbb{Q}^{*},\times)$
	
		\item[P4.] The set $\mathbb{Q}$ is an abelian group totally ordered with respect to the laws of addition and multiplication that we denote:
		
	\end{enumerate}
	The same properties are applicable to $\mathbb{R}$ and $\mathbb{C}$ but with the difference that the latter are not orderable.

	However, it may be understandable that for $\mathbb{C}$ the reader is skeptical. Let us develop all this:
	
	We must make sure that the sum "$+$", the difference "$-$", the product "$\times$" and the quotient "$/$" of two numbers of the type $x+\mathrm{i}x$ gives something of the same type again.

	Let us add the numbers $a+\mathrm{i}b$ and $c+\mathrm{i}d$ where $a$, $b$, $c$ and $d$ are real numbers:
	
	Therefore the addition is indeed a commutative and associative internal law for which there exists a neutral and symmetric element in the set of complexes $\mathbb{C}$.
	
	Let us subtracts the numbers $a+\mathrm{i}b$ and $c+\mathrm{i}d$ where $a$, $b$, $c$ and $d$ are here again, real numbers:
	
	Therefore the subtraction is an internal law, but it is not commutative, neither associative it has no neutral element to the left and neither symmetric elements.
	
	Let us now multiply the numbers $a+\mathrm{i}b$ and $c+\mathrm{i}b$ where $a$, $b$, $c$ and $d$ are still real numbers. To achieve our work, we use the the distributivity of multiplication with respect to addition:
	
	Thus the law of multiplication is indeed a commutative, associative and distributive (!) internal operation for which there exists a neutral and symmetric element in $\mathbb{C}^{*}$ (see below).
	
	A division is before all a multiplication by the inverse. Prove that there exists an inverse proves that there exists a symmetric for multiplication. Let us therefore inverse the number $x+\mathrm{i}y$ where $x$ and $y$ are still real numbers (different from zero!):
	
	So the inverse of a complex number is indeed an internal non-associative and non-commutative operation for which there exists a neutral element, and it is symmetric. The same is true for division, which corresponds to the product by the inverse of a complex number.
	
	Let us now consider quickly an example of a cyclic group (we will in-deep the subject in the section of Set Algebra): In $\mathbb{C}$, let us consider $G = \{1, \mathrm{i}, -1, -\mathrm{i}\}$ equipped with the usual multiplication of complex numbers. Then $(G,\times)$ is obviously an abelian group. Such a group is also monogenic because it is generated by the powers of one of its elements: $\mathrm{i}$ (or $-\mathrm{i}$). This monogenic group being finite, it is then a named a "cyclic group\index{cyclic group}".
	
	\subsubsection{Ring}
	The ring is the heart of the commutative algebra which is the algebraic structure corresponding to the high-school concepts of addition, subtraction, and multiplication.
	
	\textbf{Definition (\#\mydef):}  A commutative group (or "abelian group") $A$ is a "\NewTerm{ring}\index{ring}" if it is provided with a second internal law of composition satisfying the following properties:
	
	As we already know, the neutral element of the first internal composition law "$+$" is denoted "$0$" and named "zero" of the ring. The second internal law $\times$ is often denoted by a mid-height point "$\cdot$" and named "multiplication".
	
	\begin{tcolorbox}[title=Remarks,colframe=black,arc=10pt]
	\textbf{R1.} If, moreover, the second internal law of composition "$\times$" is also commutative, the ring is say to be a "\NewTerm{commutative ring}\index{commutative ring}". We also encounter non-commutative rings in which the commutativity relation is not imposed or just not necessary and then we sometimes have to impose it, then we must reinforce the property of the neutral element of this second law by imposing "$1$" to be a neutral element both to right and to left such that: $1a=a1=a$ (an example of a non-commutative ring is provided by the set $n\times n$ of matrices with coefficients in a ring $A$, for example $M_n(\mathbb{R})$- see section of Linear Algebra).\\

	\textbf{R2.} If, in addition, there exists in $A$ a neutral element for the second internal composition law "$\times$", and that this neutral element is the unit "$1$" then we say that the ring is a "\NewTerm{unitary ring}\index{unitary ring}" and "$1$" is named "\NewTerm{unit of the ring}". If the ring is commutative and has a neutral element for the second internal composition law then we speak of "\NewTerm{commutative unitary ring}\index{commutative unitary ring}".\\
	
	\textbf{R3.} If $a\times b=0\Rightarrow (a=0\text{ or } b=0)$, regardless of the elements $a$, $b$ of $A$, the ring is named an "\NewTerm{integral ring}\index{integral ring}" or "\NewTerm{ring without zero divisors}" (if not, it is obviously named "\NewTerm{non-integral ring}").\\
	
	\textbf{R4.} A "\NewTerm{factorial ring}\index{factorial ring}" is a unitary and integral commutative ring in which the fundamental theorem of arithmetic (\SeeChapter{see section Number Theory}) is verified.
	\end{tcolorbox}	
	
	\pagebreak
	\textbf{Definitions (\#\mydef):}
	\begin{enumerate}
		\item[D1.] An element $a$ of a ring $A$ is a "unitary element" if there is a $b\in A$ such that $ab=ba=1$. If such a $b$ exists it is unique (we saw such an example in our study of congruence classes in the section of Number Theory).

		\item[D2.] An element $a$ of a ring $A$ is named a "\NewTerm{left zero divisor}\index{left zero divisor}" if there exists $x \neq 0$ such that $ax = 0$. Similarly, an element a of a ring is named a "\NewTerm{right zero divisor}\index{right zero divisor}" if there exists $y\neq 0$ such that $ya = 0$. An element $a$ that is both a left AND a right zero divisor is named a "\NewTerm{two-sided zero divisor}\footnote{If the ring is commutative, then the left and right zero divisors are obviously the same.}\index{two-sided zero divisor}" 
	\end{enumerate}
	\begin{tcolorbox}[colframe=black,colback=white,sharp corners]
	\textbf{{\Large \ding{45}}Examples:}\\\\
	E1. The only zero divisor of the ring $\mathbb {Z}$  of integers is $0$.\\
	
	E2. In the ring of $2\times 2$ matrices (over any nonzero ring) we can found and $a$ and a non zero $x$ as following:
	
	In the latter example it is difficult to choose which of the matrix is the "zero divisor" this is why we can found in some textbooks that $a$ AND $x$ are considered as "zero divisors".
	\end{tcolorbox} 
	\begin{tcolorbox}[title=Remarks,colframe=black,arc=10pt]
	\textbf{R1.} It should be obvious from the preceeding relations that clear that a ring is integral if and only if it has no zero divisors (excepted the $0$ as already mentioned earlier!).\\

	\textbf{R2.} The concepts of "unit$2 and $zero divisor" are incompatible but one element of a ring can be neither. This is the case, for example, of all integers different from $\{0,-1,1\}$ in $\mathbb{Z}$. These are neither units nor divisors from zero. We name them "\NewTerm{regular elements}\index{regular elements}".
	\end{tcolorbox}	
	We will see an important example of ring in the context of our study of polynomials (\SeeChapter{see section Calculus}), but we have already seen some very important ones in our study of congruence classes in the section of Number Theory.
	Let's see some examples of rings! During our study of groups we have seen that the structures:
	
	are all four abelian groups and the first three are in addition totally ordered.
	
	Since the law of division is in no way associative, we can restrict ourselves to studying for each of the above groups the pair of laws: "$+$" and "$\times$".

	So it comes very quickly that:
	
	constitute unitary and integral commutative rings.
	
	\begin{tcolorbox}[title=Remark,colframe=black,arc=10pt]
	We will consider as obvious, at this level of our discourse, that the reader will have noticed that $\mathbb{Z}$ is a "sub-ring" of $\mathbb{Q}$ in the sense that the operations defined are internal to each set and that the neutral and identity elements  are identical and that there exists for each element of these sets an opposite which is in the same set. We will deepen the concept of "sub-ring" a little further.
	\end{tcolorbox}	
	Let $A$ be a ring. We have the following properties:
	\begin{enumerate}
		\item[P1.] $a+b=a+c\Rightarrow (b=c)\qquad \forall a,b,c\in A$
		\begin{dem}
		This derives from the definition D4 seen at the beginning of the part concerning the algebraic structures (every element has an opposite / symmetric). Indeed, we can add to:
		
		the element $-a$. We then get:
		
		 by the existence of the opposite this gives:
		
		hence:
		
		\begin{flushright}
			$\square$  Q.E.D.
		\end{flushright}
		\end{dem}

		\item[P2.] $0\cdot a=0\qquad \forall a\in A$
		\begin{dem}
		This property derives from the definitions D3 (existence of the neutral element), and D4 (existence of the opposite / symmetric), and D5 (distributivity with respect to the other law) and finally of property P1 above. Indeed, we have:
		
		We therefore have:
		
	 	The property P1 above allows us to conclude that:
		
		(we could discuss the relevance of this kind of proof...).
		\begin{flushright}
		$\square$  Q.E.D.
		\end{flushright}
		\end{dem}

		\item[P3.] $(-1)\cdot a=-a\qquad \forall a\in A$
		\begin{dem}
		This property is proved  using P2. We have:
		
		by adding $-a$ to this last equality, we get:
		
			\begin{flushright}
			$\square$  Q.E.D.
		\end{flushright}
		\end{dem}
	\end{enumerate}
	
	\paragraph{Sub-ring}\mbox{}\\\\
	\textbf{Definitions (\#\mydef):} Given $A$ a ring and $S\subset A$ a subset of $A$. We say that $S$ is a "\NewTerm{sub-ring}\index{sub-ring}" of $A$ if:
	\begin{enumerate}
		\item[P1.] $n\in S$ (the neutral element of $A$ is also that of $S$)

		\item[P2.] $a\in S \Rightarrow -a\in S$

		\item[P3.] $(a,b)\in S\Rightarrow a+b\in S$

		\item[P4.] $(a,b)\in S\Rightarrow a\cdot b\in S$
	\end{enumerate}
	\begin{tcolorbox}[colframe=black,colback=white,sharp corners]
	\textbf{{\Large \ding{45}}Example:}\\\\
	The ring $\mathbb{Z}$ is a sub-ring of $\mathbb{Q}$ and $\mathbb{R}$.
	\end{tcolorbox}
	
	\subsubsection{Field}
	\textbf{Definition (\#\mydef):} We denote a set of numbers by the term "\NewTerm{field}\index{field (set)}" if:
	
	Hence a field is a non-zero ring in which any non-zero element is invertible or in other words: a ring of which all non-zero elements are unit elements is a body.
	\begin{tcolorbox}[title=Remarks,colframe=black,arc=10pt]
	\textbf{R1.} If the internal law $\times$ is also commutative, the field is obviously named a "\NewTerm{commutative field}\index{commutative field}".\\

	\textbf{R2.} The quaternions (\SeeChapter{see section Numbers}) form a "\NewTerm{non-commutative field}\index{non-commutative field}" for addition and multiplication.
	\end{tcolorbox}
	Let us see examples of field among the following unitary rings:
	
	We must first determine which ones do not constitute groups with respect to the internal law of multiplication "$\times$".

	As we have already seen in our study of the groups earlier above, it is evident that we must eliminate $(\mathbb{Z},+,\times)$ because of the existence of inverses which is not assured in this set.

	Thus, the fundamental fields of arithmetic are:
	
	and since the law of multiplication "$\times$" is commutative in these sets, we can affirm that these fields are also commutative fields.

	We often have in the small classes the following diagram for the most important field used by students, engineers and managers.
	
	Thus, we named "Field" a set $F$ of real or complex numbers $a$ such that the sum, difference, product and quotient of any two of these numbers $a$ belong to the same system $F$.

	We also state this property in the following way: the numbers of a field reproduce by the rational operations (addition, subtraction, multiplication, division). Thus it is evident that the number zero can never form the denominator of a quotient and the set of integers can not form a field because the division in the set of integers does not necessarily give a numerical value that exist in this same set.
	
	\pagebreak
	\subsubsection{Vector Spaces}
	When we define a "\NewTerm{vector}" (\SeeChapter{see section Vector Calculus}), we usually refer in high-school to an "Euclidean space" (see also the section Vector Calculus) of $n$ dimensions denoted $\mathbb{R}^n$. However, the notion of vector space is much more larger as we will see by reading the other sections and chapters of this book than the latter which represents only one particular case.
	
	\textbf{Definition (\#\mydef):} A "\NewTerm{vector space (EV)}\index{vector space}" or "\NewTerm{$K$-vector space}" (abbreviated sometimes: $K$-ev) on the field $K$ (we will frequently take and use for this field $\mathbb{R}$ or $\mathbb{C}$) is a set $(E,+,\cdot)$ have the following  properties:
	
	We thus have two composition laws  (taking the traditional notations of the vectors which will perhaps be more intuitive and useful for the reader and for what will also follow...):
	\begin{enumerate}
		\item An internal law of composition: the addition denoted "$+$" which satisfies:
		\begin{enumerate}[label*=\arabic*.]
			\item Associativity:
			
			
			\item Commutativity: 
			
			
			\item Neutral element:
			
			
			\item Opposite element: 
			
		\end{enumerate}
		\item An external composition law: the multiplication by a scalar, denoted "$\cdot$" (to avoid the confusion with the cross product "$\times$" that we will introduce in the section of Vector Calculus), which satisfies:
		\begin{enumerate}[label*=\arabic*.]
			\item Associativity:
			
			
			\item Distributivity on the right with respect to the field $K$: 
			
			
			\item Left distributivity with respect to $E$:
				
	
			\item Neutral element (from $K$ to $E$):
			
		\end{enumerate}
	\end{enumerate}
	Hence $10$ properties at the total (yes indeed... the fact of having an internal or external law is a property on itself!).
	
	We then say that the vector space has a "\NewTerm{vectorial algebraic structure}" and that these elements are "\NewTerm{vectors}" (at least in the most common case...), the elements of $K$ are the "scalars" (\SeeChapter{see section Numbers}).
	\begin{tcolorbox}[title=Remarks,colframe=black,arc=10pt]
	\textbf{R1.} The respective internal laws that are frequently used as addition and multiplication that we already know very well on $\mathbb{R}$, which is very convenient for our habits...\\
	
	\textbf{R2.} From now on, to distinguish the elements of the body $K$ from the set $E$, we shall denote those of $K$ by Greek letters and those of $E$ by Latin capital letters.
	\end{tcolorbox}
	It is should not be necessary to prove that these properties are satisfied for $\mathbb{R}^n$ and consequently for $\mathbb{R}^2$. We can, however, ask ourselves some subsets of $\mathbb{R}^n$?
	\begin{tcolorbox}[colframe=black,colback=white,sharp corners]
	\textbf{{\Large \ding{45}}Examples:}\\\\
	E1. Let us consider the rectangular region of $\mathbb{R}^3$ shown in figure ($a$) and in perspective in the ($c$) below:
	\begin{figure}[H]
		\begin{center}
			\includegraphics[scale=0.9]{img/arithmetics/vector_space_concept.jpg}
		\end{center}	
		\caption{Example of a concept of vector space}
	\end{figure}
	This subset (the rectangle) of $\mathbb{R}^2$ is not a vector space because, among other things, the internal operation property (closure) of the abelian group is not satisfied. Indeed, if we take two vectors $\vec{v}$,$\vec{w}$ inside the rectangle and add them, the result $\vec{v}+\vec{w}$ may come out of the rectangle. On the other hand, it is easy to see that the (infinite) line $\mathbb{R}^1$ illustrated in figure ($b$) follows all the properties enumerated above and, consequently, defines a vector space. Let us notice, however, that this line must pass through the origin, otherwise the property of the neutral element of the abelian group would not be respected (the neutral element then would no longer exists).
	\end{tcolorbox}
	
	\begin{tcolorbox}[colframe=black,colback=white,sharp corners]
	E2. Another example of a vector space is the "\NewTerm{polynomial vector space}\index{polynomial vector space}" with real coefficients of degree two or less denoted $\mathbb{P}^2$ (\SeeChapter{see section Calculus}). For example, two randomly chosen elements of this space are:
	
	This polynomial set respects the properties of a vector space. Indeed, if we add two polynomials of degree two or less, we get another polynomial of degree two or less. We can also multiply a polynomial by a scalar without changing the order (or degree) of that latter, and so on.... We can therefore represent a polynomial by vectors whose terms are the coefficients of the polynomial.
	\end{tcolorbox}
	Let us notice that we can also form vector spaces with sets of functions more general than polynomials. It is only important to respect the $10$ fundamental properties of a vector space!

	So defined, a vector space $E$ on $K$ is an action of $(K,\times)$ on $(E,+)$ which is compatible with the group law (by extension an "automorphism" - see definition below - on $(E,+)$).
	
	\textbf{Definition (\#\mydef):} Let $E$ be a vector space, we name "\NewTerm{vectorial subspace}\index{vectorial subspace}" $F$ of $E$ a subset of $E$ if and only if (as written by nice mathematicians...):
	
	or by using another notation (the one used more by physicists and engineers):
	
	
	\subsubsection{$C$-algebra $A$}
	\textbf{Definition (\#\mydef):} A "\NewTerm{$C$-algebra $A$}\index{$C$-algebra $A$}" where $C$ is a commutative field (also named sometimes "\NewTerm{$K$-algebra $A$}" where the $K$ state for "Körper" in German)) is a set $A$ with two internal composition laws: the addition "$+$" and the cross product "$\times$" and one external law (multiplication) "$\cdot$" with operator domain $C$ (multiplication by a scalar) if and only if:
	
	\begin{tcolorbox}[colframe=black,colback=white,sharp corners]
	\textbf{{\Large \ding{45}}Examples:}\\\\
	E1. To take an example in the line of the one on vector space examples, the Euclidean space $\mathbb{R}^3$ with the addition "$+$", the multiplication "$\cdot$" and the cross product "$\times$" is a non-associative and non-commutative $\mathbb{R}$-algebra denoted $(\mathbb{R}^3,\mathbb{R},+,\cdot,\times)$.\\
	
	E2. The set $\mathbb{C}$ is an $\mathbb{R}$-algebra (as complex number can be seen as a vector with two components according to what we saw in the section Numbers).
	\end{tcolorbox}
	
	
	\pagebreak
	\subsubsection{Summary}
	Ok... so far a lot of definitions and concepts. To sum up the most important algebraic structures seen so far we get the authorization of Kevin Binz to reproduce very nice visual figures that could help the reader brain to have an overview of most important concepts seen so far.
	
	So first let us introduce the different set structures in the point of view where we build them by adding each time a new property:
	
	Of course, there is no particularly strong reason to "knife" axioms in this order. More esoteric options are available (in the below, red axioms are removed, green are re-introduced) introducing at the same time some set structure that we did note introduce before:
	\begin{figure}[H]
		\begin{center}
			\includegraphics[scale=0.84]{img/arithmetics/abelian_other_group_types.eps}
		\end{center}	
		\caption{Summary of most common algebraic structures (source: \url{https://kevinbinz.com/tag/identity-element/}, author: Kevin Binz)}
	\end{figure}
	
	\begin{figure}[H]
		\begin{center}
			\includegraphics[scale=0.9]{img/arithmetics/group_addition_summary.eps}
		\end{center}	
		\caption{Group Theory addition summary (source: \url{https://kevinbinz.com/tag/identity-element/}, author: Kevin Binz)}
	\end{figure}
	If we think of the above as a function, three inputs are relevant to us: the target set, the operator, and the identity element. Thus, we can condense the above into $(S, +, 0)$.
	
	\begin{figure}[H]
		\begin{center}
			\includegraphics[scale=0.9]{img/arithmetics/group_multiplication_summary.eps}
		\end{center}	
		\caption{Group Theory multiplication summary (source: \url{https://kevinbinz.com/tag/identity-element/}, author: Kevin Binz)}
	\end{figure}
	If we think of the above as a function, three inputs are relevant to us: the target set, the operator, and the identity element. Thus, we can condense the above into $(S, \times, 0)$.
	
	Did the above two sections feel painfully similar? Yes it seems!

	One lesson they teach you in computer science is: if you notice yourself copy-pasting code, you should try to consolidate your software into one function.

	Analogously, we can generalize addition and multiplication, like this:
	\begin{figure}[H]
		\begin{center}
			\includegraphics[scale=0.4]{img/arithmetics/group_addition_multiplication_merge.eps}
		\end{center}	
		\caption{Merging of addition and multiplication (source: \url{https://kevinbinz.com/tag/identity-element/}, author: Kevin Binz)}
	\end{figure}
	Here, we generalize our three inputs defined above:
	\begin{itemize}
		\item $S$ becomes a specific instance of an input set
		\item $+$ and $\times$ become specific instances of a general class of operator.
		\item $0$ and $1$ become specific instances of a general class of identity elements.
	\end{itemize}
	We have seen earlier that this particular set of five axioms is an "Abelian group" (also known as a "commutative group") that can be summarized as:
	\begin{figure}[H]
		\begin{center}
			\includegraphics[scale=1]{img/arithmetics/abelian_group_summary_structure.eps}
		\end{center}	
		\caption{Abelian group structure (source: \url{https://kevinbinz.com/tag/identity-element/}, author: Kevin Binz)}
	\end{figure}
	Let us marry addition-groups and multiplication-groups together, into a field:
	\begin{figure}[H]
		\begin{center}
			\includegraphics[scale=0.85]{img/arithmetics/field.eps}
		\end{center}	
		\caption{Field set structure (source: \url{https://kevinbinz.com/tag/identity-element/}, author: Kevin Binz)}
	\end{figure}
	
	\pagebreak
	\subsection{Morphisms}
	In many fields of mathematics, morphism refers to a structure-preserving map from one mathematical structure to another. The notion of morphism recurs in much of contemporary mathematics. In set theory, morphisms are functions; in linear algebra, linear transformations; in group theory, group homomorphisms; in topology, continuous functions, and so on.
	
	The concept of homomorphisms (from the Greek homoios = similar and morphê = form) was defined by mathematicians because it makes it possible to highlight remarkable properties of functions in particular with their structures, their nucleus, and what we name the "ideals "(see further below). They will allow us to identify one algebraic structure of another.
	
	\textbf{Definitions (\#\mydef):} 
	\begin{enumerate}
		\item[D1.] If $(A,\star)$ and $(B,\circ)$ are two magmas (regardless of the notation used for internal laws), an application $f$ of $A$ into $B$ is a named "\NewTerm{homomorphism of magma}\index{homomorphism of magma}" or "\NewTerm{morphism of magma}\index{morphism of magma}" (by abuse of language we sometimes write just "homomorphism") if:
		
		In other words: if the image of a composition in $A$ is the composition of the images in $B$.
		
		\item[D2.] If $(A,\star)$ and $(B,\circ)$ are two monoids, an application $f$ of $A$ into $B$ is a "\NewTerm{homomorphism of monoid}\index{homomorphism of monoid}" if:
		
		where $1_A$, $1_B$ are the respective neutral elements of the monoids $A$, $B$.
		
		\item[D3.] If $A$, $B$ are two rings, a "\NewTerm{homomorphism of rings}\index{homomorphism of rings}" (very important for the Cryptography section of this book!) of $A$ in $B$ is a function $f:A\mapsto B$ such that for any $a,a'\in A$ we have:
		
		where $1_A$, $1_B$ are the neutral elements of the rings $A$, $B$ with respect to the multiplication "$\cdot$".
		
		Given $f:A\mapsto B$ a homomorphism of rings. Then:
		\begin{enumerate}
			\item[P1.] $f(0)=0$
			\begin{dem}
			By:
			
			we have
			
			By adding $-f(a)$ on both sides of the equality, we get:
			
			\begin{flushright}
				$\square$  Q.E.D.
			\end{flushright}
			\end{dem}
	
			\item[P2.] $f(-a)=-f(a)$
			\begin{dem}
			This property also follows from
			
			and of the property P1. Indeed, we have:
			
			 By adding $-f(a)$ to the two sides of the last equality, we get:
			
			\begin{flushright}
				$\square$  Q.E.D.
			\end{flushright}
			\end{dem}
			
			\item[P3.] If $a$ is a unit of $A$, then $f(a)$ is a unit of $B$ and $f(a)^{-1}=f(a^{-1})$.
			\begin{dem}
			Given $a,b\in A$ such that:
			 
			Then by $f(a\cdot b)=f(a)\cdot f(b)$ and $f(1_A)=1_B$, we have:
			
			and also:
			
			which shows that $f(b)$ is the inverse of $f(a)$ if $b$ is the inverse of $a$.
			\begin{flushright}
				$\square$  Q.E.D.
			\end{flushright}
			\end{dem}
		\end{enumerate}
		Let us now introduce a quite powerful theorem:
		\begin{theorem}
		Let us now show that a homomorphism of rings $f:A\mapsto B$ is injective if and only if the element $0$ is the only pre-image of $0$ (and therefore reciprocally), which is technically written:
		
		That is to say: the kernel is "trivial".
		\end{theorem}
		\begin{dem}
		The condition is clearly necessary. Let us show that it is sufficient:

		We therefore assume that $\ker(f)=0$. Given $a,a'\in A$ such that $f(a)=f(a')$. Then as we have a homomorphism of ring we can write:
		
		Which implies that $a-a'=0$ and therefore that $a=a'$.
	
		This shows that $f$ is injective if it is a homomorphism and that $\ker (f)=0$ is indeed a sufficient condition.
		\begin{flushright}
			$\square$  Q.E.D.
		\end{flushright}
		\end{dem}
	
		\item[D4.]  Given $(A,+)$ and $(B,\star)$, two groups and $f$ an application $f:A\mapsto B$. We say that $f$ is a "\NewTerm{homomorphism of group}\index{homomorphism of group}" if (we could just as well put the W$\star$W instead of the W$+$W in the first group and the "$+$" instead of the "$\star$" in the second group, the definition would remain the same by simply replacing the respective operators !):
		
		where $1_A$, $1_B$ are the respective neutral elements of the groups $A$, $B$. We notice that the only difference between a homomorphism of rigns and a homomorphism of groups is that it has two laws instead of one and that we add the concept of inverse.

		That said, the third proposition above is in fact a consequence of the definition composed only of the first two lines. Indeed, consider a homomorphism $f$ between the groups $(A,+)$ and $(B,\star)$ with $1_A$ and $1_B$ respectively the neutral elements of $A$ and $B$, then we have:
		
		Hence:
		
		Therefore:
		
		\begin{tcolorbox}[colframe=black,colback=white,sharp corners]
		\textbf{{\Large \ding{45}}Example:}\\\\
		The exponential function $e$ is a morphism of the group $(\mathbb{R}^+, +)$ on the group $(\mathbb{R}^+, \cdot)$.
		\end{tcolorbox}
	
		\item[D5.] Let $f$ be an application $f:A\mapsto B$ from one field to another. We say that $f$ is a "homomorphism of field" if $f$ is a homomorphism of ring...
		
		Indeed, the fact that the homomorphism of field is the same as that of a ring is due to the fact that the difference between the two structures is that the elements of the field are all invertible (no law or law property differs between the both according to their definition of the homomorphism).
		
		Let us show now that every homomorphism of field is injective ("injective homomorphism") by remembering that earlier above we have proved that every homomorphism of rings was injective!
		\begin{dem}
		If $a$ is different from $0$ and $b=a^{-1}$ (we use here the property that the elements of a field are invertible!) then:
		
		So when $a$ is different from zero, $f(a)$ is also different from $0$ which proves that $\ker (f)=\{0\}$ and therefore that $f$ is injective.
		\begin{flushright}
			$\square$  Q.E.D.
		\end{flushright}
		\end{dem}
	
		\item[D6.] Given $A$ and $B$ be two $K$-vector spaces and $f:A\mapsto B$ an application of $A$ into $B$. We say that $f$ is a "\NewTerm{linear mapping}\index{linear mapping}" or "\NewTerm{linear application}\index{linear application}"  or "\NewTerm{homomorphism of vector spaces}\index{homomorphism of vector spaces}" (it is implicit that this is relative to the indicated laws and for the chosen application) if:
		
		and we denote by $L(A, B)$ the set of linear applications.
		\begin{tcolorbox}[title=Remarks,colframe=black,arc=10pt]
		\textbf{R1.} We have already defined the concept of linear appliction but did not specify that the two sets $A$ and $B$ were $K$-vector spaces.\\
	
		\textbf{R2.} A linear application is named a "\NewTerm{linear form}\index{linear form}" if and only if $B=K$.
		\end{tcolorbox}
	
		\item[D7.] If the homomorphism is bijective we will say that $f$ is an "\NewTerm{isomorphism}\index{isomorphism}". If there is an isomorphism between $A$ and $B$, we say that $A$ and $B$ are "\NewTerm{isomorphic}" and we will denote this $A\simeq B$.
		\begin{tcolorbox}[title=Remark,colframe=black,arc=10pt]
		The isomorphism allows the identification of two sets with an identical algebraic structure (group, ring, etc.) but whose elements are named in a different way.
		\end{tcolorbox}
		
		\item[D8.] If the homomorphism $f$ is a purely internal mapping, then we will say that $f$ is an "\NewTerm{endomorphism}\index{endomorphism}" (in other words, we have an endomorphism if in the definition of the homomorphism we have $A = B$).
		\begin{tcolorbox}[title=Remark,colframe=black,arc=10pt]
		If we have an endomorphism $f$ of $E$, then $f$ is then restricted to $\Im(f)$. So the term "endomorphism" just means that the application $f$ arrives in $E$ and not that it touches all the elements of $E$ itself. We have then $f(E)\subset e$ and not necessarily $f(E)=E$ because in the latter case we say that $f$ is surjective as we We have already seen.
		\end{tcolorbox}
		
		\item[D9.] If the endomorphism $f$ is furthermore bijective (hence in other words if the homomorphism is an endomorphism \underline{and} an isomorphism), then we will say that $f$ is an "\NewTerm{automorphism}\index{automorphism}".
	\end{enumerate}
	\begin{figure}[H]
		\centering
	    \begin{tikzpicture}
		\def\homomorphism{(0:0cm) circle (5.0cm)}
	  	\def\isomorphism{(180:2.5cm) ellipse (2.0cm and 3.0cm)}
	  	\def\endomorphism{(90:2.0cm) ellipse (3.0cm and 1.5cm)}
	      \begin{scope}[fill opacity=0.1]
	        \fill[magenta] \homomorphism;
	      \end{scope}
	
	      \begin{scope}[fill opacity=0.5]
	        \fill[cyan] \endomorphism;
	      \end{scope}
	
	      \begin{scope}[fill opacity=0.5]
	        \fill[orange] \isomorphism;
	      \end{scope}
	
	      \draw \homomorphism;
	      \draw \isomorphism;
	      \draw \endomorphism;
	
	      {
	        \scalefont{2.0}
	        \node[text=black] at ( 1.8,-2) {Homomorphism};
	      }
	
	      {
	        \scalefont{1.6}
	        \node[text=black] at (-2.5, 0) {Isomorphism};
	      }
	
	      {
	        \scalefont{1.4}
	        \node[text=black] at (   1, 2) {Endomorphism};
	      }
	      \node[align=left] at (  -2, 2) {Auto-\\morphism};
	    \end{tikzpicture}
		\caption{Venn diagram of different types of homomorphisms (source: Wikipedia, author: Martin Thoma}
	\end{figure}
	
	
	\subsubsection{Ideal}
	\textbf{Definition (\#\mydef):} 
	Let $A$ be a commutative ring (like $(\mathbb{R},+,\cdot)$ for example). A subset $S\subset A$ is an "\NewTerm{ideal}\index{idela}" if:
	\begin{enumerate}
		\item[P1.] For all $a,a'\in S$:
		
	
		\item[P2.] For all $a\in S$ and all $r\in A$:
		
	\end{enumerate}
	In other words, an ideal is a closed subset for addition and stable for the multiplication by any element of $A$.
	\begin{tcolorbox}[colframe=black,colback=white,sharp corners]
	\textbf{{\Large \ding{45}}Example:}\\\\
	The set of even numbers $\mathbb{Z}_{2k}$ is an example of ideal of the set of relative integers $\mathbb{Z}$.
	\end{tcolorbox}
	\begin{tcolorbox}[title=Remark,colframe=black,arc=10pt]
	The ideals $S=\{0\}$ and $S=A$ are named the "\NewTerm{trivial ideals}".
	\end{tcolorbox}
	\begin{dem}
	This results from the property P2 of the definition of an ideal:
	
	For any $r\in A$, we have $r=r\cdot 1\in 1$ because $1\in I$.
	\begin{flushright}
		$\square$  Q.E.D.
	\end{flushright}
	\end{dem}
	Another quite general example of ideal is given by the kernel of a homomorphism of rings. Indeed, let us prove that the kernel of a homomorphism $F:R\mapsto S$ is an ideal of $R$.
	\begin{dem}
	Given $a,a'\in \ker (f)$. Then:
	
	which shows that $a+a'\in \ker(f)$. Given $r\in R$, then:
	
	Which shows that $r\cdot a\in \ker(f)$.
	\begin{flushright}
		$\square$  Q.E.D.
	\end{flushright}
	\end{dem}
	Proposition: Let $A$ be a ring and let $a\in A$. The subset:
	
	Denoted $(a)$ or $aA$, is an ideal (we will see a concrete example after the next small set of definitions).
	
	\textbf{Definitions (\#\mydef):}
	\begin{enumerate}
		\item[D1.] An ideal $I\neq A$ of a ring $A$ is named a "\NewTerm{principal ideal}" if there exists $a\in A$ such as $I=(a)$.
	
		\item[D2.] A ring of which all the ideals are principal is named "\NewTerm{principal ring}".
	\end{enumerate}
	\begin{tcolorbox}[colframe=black,colback=white,sharp corners]
	\textbf{{\Large \ding{45}}Example:}\\\\
	Let us now show that the ring $\mathbb{Z}$ is principal (because all its ideals are principal).\\
	
	Let $I$ be an ideal of $\mathbb{Z}$ (it is easy to choose one: for example, all multiples of $2$ or $3$, etc.). Let $r\in I$ be the smallest non-zero positive integer of $I$. We will show that $I=r\mathbb{Z}=(r)$.\\
	
	Let $a$ be any element of $I$. The Euclidean division allows us to write:
	
	with $0\ge r' <r$ (as we have already prove it).\\
	
	But as $r'=a-qr$ and that $a,r\in I$, by the definition of an ideal, we have $r'\in I$ (the sum or difference of the elements of an ideal belonging to the ideal). By the choice of $r$ ($r'$ being less than $r$) this implies that $r'=0$ and therefore that $a=rq$.\\
	
	Thus every element of $I$ is a multiple $r$ of $q$:
	
	So for the set of even number (denoted as we we know $2\mathbb{Z}$ or $\mathbb{Z}_{2k}$) we have:
	
	\end{tcolorbox}
	The example above uses only the Euclidean division on $\mathbb{Z}$. We can then generalize this result to the rings which possess a Euclidean division. Thus, for example, the ring $K[X]$ of the polynomials (\SeeChapter{see section Calculus}) with coefficients in a field $K$ is a principal ring because it has an Euclidean division.

	\begin{theorem}
	The ring $K[X]$ of the polynomials with coefficients in a field $K$ is a principal ring (all its ideal are principal)
	\end{theorem}
	\begin{dem}
	Let $I$ be an ideal of $K[X]$. Let us denote by $d$ the smallest degree that can have a non-zero polynomial of $I$. If $d=0$ then $1\in I$ and therefore $I=1\cdot K[X]=K[X]$. Otherwise, given $a(X)$ a polynomial of degree $d$. If $u(X)\in I$ then one can divide $u(X)$ by $a(X)$. There exists therefore $q(X),r(X)\in K[X]$  such as $\deg(r)<\deg(a)=d$ and:
	
	Therefore $r(X)\in I$ which lead to $r=0$ (otherwise contradiction with the minimality of $d$). Therefore,
	
	 We have just shown that:
	
	\begin{flushright}
		$\square$  Q.E.D.
	\end{flushright}
	\end{dem}
	To come back on $\mathbb{Z}$... we have then proved that the only ideals are those of the form $r\mathbb{Z}$. Moreover if we have $d$ and $r$ which are integers $>1$. Then $r\mathbb{Z}\subset d\mathbb{Z}$ if and only if $d | r$.
	\begin{dem}
	If $d | r$ then there exists $n$ with $r=d\cdot n$. Let $m\cdot a$ be an element of $r\mathbb{Z}$. Therefore:
	
	which shows that $r\mathbb{Z}\subset d\mathbb{Z}$.

	Conversely, if $r\in d\mathbb{Z}$ this implies that $r$ is of the form $d\cdot n$ and this proves that $d$ divides $r$.
	\begin{flushright}
		$\square$  Q.E.D.
	\end{flushright}
	\end{dem}
	\begin{theorem}
	Let us also prove that a ring $R$ is a field if and only if it possesses only the trivial ideals:
	
	\end{theorem}
	\begin{dem}
	Let us show that the condition is necessary: Let $I$ be a non-zero ideal of $R$ (that is so say different of $\{0\}$) and $r\in I$ a non-zero element. By hypothesis (that it is a field), it is invertible, that is to say that there exists $t\in R$ such that:
	
	This implies that $1\in I$ and therefore, by a result obtained earlier above $I=R$.
	
	Conversely, suppose that every ideal $I\neq R$ is the null ideal (that is to say $\{0\}$). Then if $r\in R$ is a non-zero element of $R$, the principal ideal $(r)$ must be equal to $R$. But this implies that $1\in (r)$ and therefore that there exists $x\in R$ with $r\cdot x=1$ which shows that $r$ is invertible. The ring $R$ is therefore a field.
	\begin{flushright}
		$\square$  Q.E.D.
	\end{flushright}
	\end{dem}
	This characterization will enable us to easily demonstrate to quite easily prove that:
	\begin{theorem}
	Any homomorphism starting from a field is injective. That is, if $f:R\mapsto S$ is a homomorphism where R is a field, then $f$ is injective.
	\end{theorem}
	\begin{dem}
	We put together what has been seen so far:
	\begin{itemize}
		\item We have proved earlier above that the kernel $\ker(f)$ of a homomorphism of ring is an ideal. 
	
		\item We have also proved just earlier above that we a set was a field if $\{0\}$ and $R$ where the only trivial ideals.
	\end{itemize}
	Therefore for the both point above we have either $\ker(f)=0$ or $\ker(f)=R$ for the field (as it encompass the concept of ring!).
	
	But since $f(1)=1\neq 0$ (by definition of a homomorphism!) it follows that it remains to us only the choice $\ker(f)=\{0\}$. This implies by a previous theorem (where we have proved long time before that if $\ker(f)=\{0\}$ the homomorphism is injective) that ... $f$ is injective.
	\begin{flushright}
		$\square$  Q.E.D.
	\end{flushright}
	\end{dem}
	Let us now study the homomorphisms whose starting ring is $\mathbb{Z}$. Let $A$ be a ring and $f:\mathbb{Z}\mapsto A$ a homomorphism. By definition of a homomorphism and by its properties, it is necessary that as we have defined long time before that $f(0)=0$ and $f(1)=1$ (among others). But we also still need that:
	
	for any $k\in\mathbb{Z}$. Thus $f$ is completely determined by the information of $f(1)$ and is therefore unique.
	
	Conversely, we show that the application $f:\mathbb{Z}\mapsto A$ defined by:
	
	is a homomorphism of rings. In summary, there exists one and only one homomorphism of $\mathbb{Z}$ in any ring $A$.
	
	\textbf{Definition (\#\mydef):} Given $R$ a ring and $f:\mathbb{Z}\mapsto R$ the unique homomorphism defined just previously. If $f$ is injective, we say that $A$ is of "\NewTerm{zero characteristic}\index{zero characteristic}" and we denote it:
	
	Otherwise, $\ker(f)$ is a non trivial ideal of $\mathbb{Z}$ and as $\mathbb{Z}$ is therefore principal (as we have demonstrated above) it is of the form $k\mathbb{Z}$ with $k>0$. The integer $k$ is named the "\NewTerm{characteristic of $A$}\index{characteristic}" and we have therefore:
	
	The definition above may not be very clear (at least for me it is not!). So let us see another approach to introduce the characteristic much more detailed...:
	
	Given $R$ a ring and any of its elements $r$, let us denote integers with boldface type. So $\pmb{1}$ is the integer number one, for instance, $\pmb{0}$ is the integer zero, whereas $0_R$ and $1_R$ are the neutral elements in $R$.

	Since $R$ under addition is an abelian group, we can as usual define:
	
	By definition of a homomorphism, we have if $f:\mathbb{Z}\mapsto R$ that:
	
	then for recall:
	\begin{itemize}
		\item $f(\pmb{m}+\pmb{n})=f(\pmb{m})+f(\pmb{n})$
		\item $f(\pmb{mn})=f(\pmb{m})f(\pmb{n})$
		\item $f(\pmb{1})=1_R$
	\end{itemize}
	so $f$ is a homomorphism of rings. Like all ring homomorphisms, $f:\mathbb{Z}\mapsto R$ has a kernel which is an ideal as we have proved just earlier, so we have $\ker(f)=\pmb{k}\mathbb{Z}$ for a unique $\pmb{k}\ge \pmb{0}$.
	
	If $k=0$, then as we have proved much earlier above, then $f$ is injective. Otherwise, we have:
	
	by definition of the kernel, which means that:
	
	for every $r\in R$. This integer $\pmb{k}$ is the "characteristic" of $S$.
	
	That is, $\text{char}(R)$ is the smallest positive number $\pmb{k}$ such that:
	
	if such a number $\pmb{k}\in\mathbb{N}$ exists, and $0$ otherwise.
	
	\begin{tcolorbox}[colframe=black,colback=white,sharp corners]
	\textbf{{\Large \ding{45}}Example:}\\\\
	E1. The only ring that has $\text{char}(R)=1$ such that:
	
	is the trivial ring $R=\{0_R\}$.\\
	
	E2. The ring $\mathbb{Z}$ is of zero characteristic because the unique homomorphism $f:\mathbb{Z}\mapsto \mathbb{Z}$ is the identity. It is therefore injective.\\
	
	E3. The injections $\mathbb{Z}\mapsto \mathbb{Q}$ and $\mathbb{Z}\mapsto \mathbb{R}$ show that $\mathbb{Q}$ and $\mathbb{R}$ (and also $\mathbb{C}$) are fields of zero characteristic.
	\end{tcolorbox}
	\begin{theorem}
	We now propose to prove that the characteristic of an integral ring (and in particular of a field) is equal to $0$ or to a prime number $p$.
	\end{theorem}
	\begin{dem}
	We show the contraposition. Let $R$ be a ring of characteristic $m\neq$ with $m$ not prime.
	
	There are then natural $n,r\in\mathbb{N}$ constraint by $n,r<m$ such that $m=n\cdot r$.

	Given $f:\mathbb{Z}\mapsto A$ the unique homomorphism (defined earlier above). By definition of $m$ we have then $f(m)=0_A$ but... (!) $f(r)\neq 0 \neq f(n)$. But then (and this it the trick of the proof!):
	
	which shows that $A$ is not integral.
	\begin{flushright}
		$\square$  Q.E.D.
	\end{flushright}
	\end{dem}
	\begin{tcolorbox}[title=Remark,colframe=black,arc=10pt]
	The reciprocal of the theorem is not true as shown in the example of the ring $\mathbb{R}\times\mathbb{R}$ where addition and multiplication are made component by component. It is a ring of zero characteristic but with divisors of zero:
	
	\end{tcolorbox}	
	
	
	\begin{flushright}
	\begin{tabular}{l c}
	\circled{95} & \pbox{20cm}{\score{4}{5} \\ {\tiny 16 votes, 62.5\%}} 
	\end{tabular} 
	\end{flushright}

	%to make section start on odd page
	\newpage
	\thispagestyle{empty}
	\mbox{}
	\section{Probabilities}

\lettrine[lines=4]{\color{BrickRed}P}robability is the measure of the likelihood that an event will occur and therefore the calculation of probabilities handles random phenomena (known more aesthetically as "\NewTerm{stochastic processes}\index{stochastic process}" when their are time-dependent), that is to say, phenomena that do not always lead to the same outcome and that can be studied using numbers their implications and occurrences. However, even if these phenomena have variable outcomes, depending on chance, we observe a certain statistical regularity.

Probability is quantified as a number between $0$ and $1$ (where $0$ indicates impossibility and 1$ $indicates certainty). The higher the probability of an event, the more certain we are that the event will occur.

The concepts related to probabilities have been given an axiomatic mathematical formalization in probability theory (see further below), which is used widely in such areas of study as mathematics, statistics, finance, gambling, science (in particular physics), artificial intelligence/machine learning, computer science, game theory, and philosophy to, for example, draw inferences about the expected frequency of events. Probability theory is also used to describe the underlying mechanics and regularities of complex systems.

\textbf{Definitions (\#\mydef):} There are several ways to define a probability. Mainly we are talking about:
\begin{itemize}
	\item[D1.] "\NewTerm{Experimental or inductive probability}\index{inductive probability}" which is the probability derived from the whole population.

	\item[D2.] "\NewTerm{Theoretical or deductive probability}\index{deductive probability}" which is the known probability through the study of the underlying phenomenon without experimentation. It is therefore an "a priori" knowledge as opposed to the previous definition that was rather referring to a notion of "a posteriori" probability.
\end{itemize}
As it is not always possible to determine a priori probabilities, we are often asked to perform experiments. We must then be able to pass from the first to the second solution. This passage is supposed to be possible in terms of limit (with a population sample whose size approaches the size of the whole population).

The formal modeling of the probability calculus was invented by A.N: Kolmogorov in a book published in 1933. This model is based on the probability space (U, A, P) that we will define a little further and that we can relate to the theory of measurement (\SeeChapter{see section Measure Theory}). However, the probabilities were studied in the scientific point of view by Fermat and Pascal in the mid 17th century.

	\begin{tcolorbox}[title=Remark,colframe=black,arc=10pt]
If you have a teacher or trainer who dare to teach statistics and probabilities with examples based on gambling (cards, dice, match, toss, etc.) dispose it to whom it may concern because it would mean that he has no experience in the field and he will teach you anything and no matter how (examples could normally be based on industry, economy or R\&D, in short: areas daily used in companies but especially not on gambling ...!).
	\end{tcolorbox}	
	
\subsection{Event Universe}

\textbf{Definitions (\#\mydef):}

\begin{itemize}
	\item[D1.] The "\NewTerm{universe of events}\index{universe of events}", or "\NewTerm{universe of observables}\index{universe of observables}", $U$ is the set of all possible outcomes (results), named "elementary events" that occur during a random determined test. The universe can be finite (countable) if the elementary events are finite or continuous (uncountable) if they are infinite.

	\item[D2.] Any "\NewTerm{event}\index{event}" $A$ is a set of elementary events and is part of the universe of possible $U$. It is possible that an event is composed of only a single elementary event.
\end{itemize}

	\begin{tcolorbox}[colframe=black,colback=white,sharp corners]
	\textbf{{\Large \ding{45}}Example:}\\\\
	In $(\mathbb{N},+)$ every element is regular and $(\mathbb{N},\times)$ any non-zero element is regular.
	\end{tcolorbox}
	
\begin{itemize}
	\item[D3.] Let $U$ be a universe and $A$ an event, we say that the event $A$ "occurs" (or "is realized") if during the run of the trial the issue $i\:\left( i \in U \right)$ occurs and that $i \in A$ . Otherwise, we say that $A$ "was not realised".

	\item[D4.] The empty subset $\varnothing$ of $U$ is named "\NewTerm{impossible event}\index{impossible event}". Indeed, if during a trial where the event $i$ occurs, we always have $i \in \varnothing$ and the event $\varnothing$ then never occurred.\\\\
If $U$ is finite, or countably infinite, any subset of $U$ is an event, that is no longer true if $U$ is uncountable (we will see in the chapter Statistics why).

	\item[D5.] The set $U$ is also named "\NewTerm{certain event}\index{certain event}". Indeed, if at the end of the trial the event $i$ occurs, we have always (since $U$ is the universe of events). The event $U$ then always occurred.

	\item[D6.] Let $A$ and $B$ be two subsets of $U$. We know that the events $A \cup B$ and $A \cap B$ are both subsets of $U$ then, events that are respectively "\NewTerm{joint events}\index{joint events}" and "\NewTerm{disjoint events}\index{disjoint events}".
	
	\item[D7.] If two events $A$ and $B$ are such that:
	
	the two events may not be feasible during the same trial, then we say that they are "\NewTerm{mutually exclusive events}\index{mutually exclusive events}".

	\item[D8.] If two events $A$ and $B$ are such that:
	
	the two events may be feasible during the same trial (the possibility to see a black cat when we pass under a ladder, for example), we say conversely that they are "\NewTerm{independent events}\index{independent events}".
	
	\item[D9.] "\NewTerm{Randomness}\index{random}" is the lack of pattern or predictability in events. A random sequence of events, symbols or steps has no order and does not follow an intelligible pattern or combination. Individual random events are by definition unpredictable, but in many cases the frequency of different outcomes over a large number of events (or "trials") is predictable.
	
	\begin{tcolorbox}[title=Remark,colframe=black,arc=10pt]
	If we randomly (i.e. uniformly) choose a real number in the interval $[0,1]$ then for every number there is a zero probability that we will pick this number. This does not mean that we did not pick any number at all!\\

	Similarly with the rationals, while infinite, and dense and all that, they are very very sparse in the aspect of measure and probability. It is perfectly possible that if we throw countably many darts at the real line we will hit exactly all the rationals and every rational exactly once. This scenario is highly unlikely, because the rational numbers is a measure zero set.\\

	Probability deals with \textit{what are the odds of that happening}? \underline{a priori}, not a posteriori!! So we are interested in measuring a certain structure a set has, in modern aspects of probability and measure, the rationals have size zero and this means zero probability.\\
	
	More formally, as we will see in the section Statistics, the probabilities are obtained by integrating a probability density function $f(x)$ over an interval. The function is non negative and it has the property:
	
	The probability of selecting a real in an interval $[a,b]\subset \mathbb{R}$ is then given by:
	
	The reader can see now that, given a real number $x_0\in\mathbb{R}$, we have:
		
	\end{tcolorbox}	
\end{itemize}

	\pagebreak
	\subsubsection{Infinite monkey theorem}
	The infinite monkey theorem states that a monkey hitting keys at random on a typewriter keyboard for an infinite amount of time will almost surely type a given text, such as the complete works of William Shakespeare. In fact the monkey would almost surely type every possible finite text an infinite number of times. However, the probability of a universe full of monkeys typing a complete work such as Shakespeare's Hamlet is so tiny that the chance of it occurring during a period of time hundreds of thousands of orders of magnitude longer than the age of the universe is extremely low (but technically not zero).
	
	In this context, "almost surely" is a mathematical term with a precise meaning, and the "monkey" is not an actual monkey, but a metaphor for an abstract device that produces an endless random sequence of letters and symbols.
	\begin{theorem}
	If we have an infinite number of monkeys each hitting keys at random on typewriter keyboards then, with probability $1$, one of them will type the complete works of William Shakespeare (something like a randomness machine, which is typing keys randomly at a typewriter, would eventually write any book that's ever existed, given enough time).
	\end{theorem}

	\begin{dem}
	Let $A_n$ be the event that the $n$th monkey types the complete works of Shakespeare. Then if there are $m$ characters on the keyboard and $N$ characters in the complete works of Shakespeare:
	
	for each $n$. Furthermore the $A_n$ are mutually independent (disjoint event). Hence:
	
	Infinitely many of the events $A_n$ occur i.e. infinitely many monkeys will type the complete works of Shakespeare.
	\begin{flushright}
		$\square$  Q.E.D.
	\end{flushright}
	\end{dem}
	 Therefore if we accept infinity in an argument, then we can end up also accepting that "given infinity, anything can happen".

	\pagebreak
	\subsection{Kolmogorov's Axioms}
	The probability of an event is somehow responding to the notion of frequency of a random phenomena, in other words, at each event we will attach a real number in the interval $[0,1]$, which measure the probability (chance) of realization. The properties of frequencies we can highlight during various trials allow us to determine the properties of probabilities.
	
	Let $U$ be a universe. We say that we define a probability on the events of $U$ if any event $A$ of $U$ we associate a number or measure $P(A)$, named "\NewTerm{a priori probability of event $A$}\index{a priori probability}" or "\NewTerm{marginal probability of $A$}\index{marginal probability}".
	
	Here are the "\NewTerm{Kolmogorov's axioms}\index{Kolmogorov's axioms}":
	\begin{enumerate}
		\item[A1.] For any event $A$:
		
		Thus, the probability of any event is a real number between 0 and 1 inclusive (this is common human sense...).

		\item[A2.] The probability of the certain event or of the set (sum) of possible events is equal to 1:
		

		\item[A3.] If $A \cap B = \varnothing $ two events are incompatible (disjoint), then:
		
		the probability of the merge ("or") of two mutually incompatibles events (or mutually exclusive) is therefore equal to the sum of their probabilities (law of addition). We then speak of "\NewTerm{disjoint probability}\index{disjoint probability}".
	
		We understand better that the third axiom requires  $A \cap B = \varnothing$ otherwise the  sum of all probabilities could be greater than 1 (imagine again the set diagram of the two events in your head!).
	\end{enumerate}

	\begin{tcolorbox}[colframe=black,colback=white,sharp corners]
	\textbf{{\Large \ding{45}}Example:}\\\\
	Suppose the typewriter has $50$ keys, and the word to be typed is banana. If the keys are pressed randomly and independently, it means that each key has an equal chance of being pressed. Then, the chance that the first letter typed is 'b' is $1/50$, and the chance that the second letter typed is a is also $1/50$, and so on. Therefore, the chance of the first six letters spelling banana is:
	
	less than one in $15$ billion, but not zero, hence a possible outcome.\\
	
	From the above, the chance of not typing banana in a given block of $6$ letters is $1-(1/50)^6$. Because each block is typed independently, the chance $X_n$ of not typing banana in any of the first $n$ blocks of $6$ letters is:
	
	As $n$ grows, $X_n$ gets smaller. For an $n$ of a million, $X_n$ is roughly $0.9999$, but for an $n$ of $10$ billion $X_n$ is roughly $0.53$ and for an $n$ of $100$ billion it is roughly $0.0017$. As $n$ approaches infinity, the probability $X_n$ approaches zero; that is, by making $n$ large enough, $X_n$ can be made as small as is desired, and the chance of typing banana approaches $100\%$.
	\end{tcolorbox}

We will find an example of this kind of disjoint probability in the chapter of Industrial Engineering when studying F.M.E.A. (Failure Modes and Effects Analysis) for fault analysis systems with a complex structure.

In other words in a more general form if $\left( A_{i} \right)_{i \in \mathbb{N}}$ is a sequence of pairwise disjoint events ($A_{i}$ and $A_{j}$ can not occur at the same time though $i \neq j$) then:	


We then speak of "\NewTerm{$\sigma$-additivity}\index{$\sigma$-additivity}" because if we look more closely at the three axioms above the measure $P$ forms a $\sigma$-algebra (\SeeChapter{see section Measure Theory}).

At the opposite, if the events are not incompatibles (they can overlap or in other words: they have a joint probability), we then have for probability that at most one of the two takes place:


This means that the probability that at most one of the events $A$ or $B$ occurs is equal to the sum of the probabilities for the realization of $A$ or $B$ occurred, minus the probability that $A$ and $B$ occurred simultaneously (we will show later that this is simply equal to the probability that the two do not occur at the same time!).

	\begin{tcolorbox}[colframe=black,colback=white,sharp corners]
\textbf{{\Large \ding{45}}Example:}\\\\
Consider that in a given area, over 50 years, the probability of a major earthquake is 5\% and on the same period the probability a major flood is 10\%... We would like to know what is the probability that a nuclear plant meets at most one of two events during the same period if the are not incompatibles. We then calculate the probability that from the above equation that gives 14.5\%...
	\end{tcolorbox}
	
And thus if they were incompatibles we would have then $A \cap B = \varnothing $ we find again the disjoint probability:


An immediate consequence of the axioms (A2) and (A3) is the relations between the probability of an event $A$ and its complement, noted $\bar{A}$ (or more rarely in accordance with the notation used in the chapter of Proof Theory the complementary may be noted $\neg A$):


Let $U$ be a universe with a finite number of $n$ possible outcomes:


where the events:

are named "\NewTerm{elementary events}\index{elementary events}". When these events have the same probability, we say they are "\NewTerm{equiprobables\index{equiprobables}}". In this case, it is very easy to calculate the probability. Indeed, these events being by definition incompatible with each other at this level of our discussion, we have under the third axiom (A3) of probabilities:


but since:


and that the probability of the right hand are by hypothesis equiprobable, we have:


\textbf{Definition (\#\mydef):}
If $A$ and $B$ are not mutually exclusive but independent, we know that by their compatibility $A \cap B=\varnothing$, that (very important in statistics!):


the probability of the intersection ("and" operator) of two independent events is equal to the product of their probabilities (law of multiplication). We name it "\NewTerm{joint probability}\index{joint probability}" (this is the most common case).

\begin{tcolorbox}[colframe=black,colback=white,sharp corners]
\textbf{{\Large \ding{45}}Example:}\\\\
Consider that in a given area, over 50 years, the probability of a major earthquake is 5\% and on the same period the probability a major flood is 10\%... Assume that these two events are not mutually exclusive. In other words that they are compatible. We will be interested to their independence. Thus, we would like to know what is the probability that a nuclear power plants meets the two events at the same time, at any time, during this same period. We then calculate the probability from the above equation that gives 0.05\%...
	\end{tcolorbox}
	
Under a more general form, the events $A_1,A_2,...,A_n$ are independent if the probability of the intersection is the product of the probabilities:


	\begin{tcolorbox}[title=Remark,colframe=black,arc=10pt]
	Be careful to not confuse "independent" and "incompatible"!
	\end{tcolorbox}

	So far to summarize a bit we have:

		

Thanks the above definition, we can show that the probability that either $A$ or $B$ is to take place (e.g. at least one of the two but not both at the same time), is simply equal to... the probability that the two do not does not occur at the same time:


We can also use this definition to determine the probability that only one of two events occurs:


\begin{tcolorbox}[colframe=black,colback=white,sharp corners]
\textbf{{\Large \ding{45}}Example:}\\\\
Consider that in a given area, over 50 years, the probability of a major earthquake is 5\% and on the same period the probability a major flood is 10\%.... We would like to know what is the probability that a nuclear power plant exactly meets one of the both events during the same period, assuming they can not occur at the same time. We then calculate the probability from the above equation and that gives 14\%...
	\end{tcolorbox}

There is a common and important area in the industry where the four following relations are frequently used:



This is the "\NewTerm{tree analysis error}\index{tree analysis error}" or "\NewTerm{probabilistic tree analysis}\index{probabilistic tree analysis}" which is used to analyse the possible reasons for failure of a system of any kind (industrial, administrative or other).

To close this part of the chapter consider the following figure displaying Venn diagrams (\SeeChapter{see section Set Theory}) for all 16 events (including the impossible event) that can be described in terms of two given events $A$ and $B$. In each case, the event is represented by the red area:

	\begin{figure}[!ht]
		\begin{center}
			\includegraphics{img/arithmetics/venn.eps}
		\end{center}	
		\caption{Possible Venn diagrams for two events}
	\end{figure}

Consider the situation where $A$ represents and earthquake and $B$ represents a major flood and $U$ the universe of all dramatics events for a nuclear power plant. We consider that the two events are independents. Then each of the 16 events can be described as follows, either mathematically or verbally.
\begin{enumerate}
	\item An earthquake can occur or a flood or nothing or the both together or any other event (to resume: any event can occur).
	
	\item $A \cup B$: Any event with an earthquake a flood or the both event together can occur.
	
	\item $A \cup B^c$: Any event with earthquake can occur with or without a flood excepted events with a flood not associated to an earthquake.
	
	\item $A^c \cup B$: Any event with earthquake can occur with or without a flood excepted events with a flood not associated to an earthquake.
	
	\item $A^c \cup B^c$: Any event can occur excepted those associated with an earthquake together with a flood.
	
	\item $A$: Any event with an earthquake can occur (this include the events associating an earthquake and a flood).
	
	\item $B$: Any event with a flood can occur (this include the events associating a flood and an earthquake).
	
	\item $(A \cap B) \cup (A^c \cap B^c)$: Any event can occur excepted those including an earthquake without a flood or those including a flood without an earthquake.
	
	\item $(A \cap B^c) \cup (A^c \cup B)$: Any event including an earthquake without a flood or a flood without an earthquake can occur.
	
	\item $B^c$: Any event excepted those including a flood can occur.
	
	\item $A^c$: Any event excepted those including an earthquake can occur.
			
	\item $A \cap B$: Any events associating an earthquake and a flood together can occur.
	
	\item $A \cap B^c$: Any event with an earthquake and without a flood can occur.
	
	\item $A^c \cap B$: Any event with a flood and without an earthquake can occur.
	
	\item $A^c \cap B^c$: Any event can occur excepted those including an earthquake and/or a flood.
	
	\item $A \cap A^c$ or $B \cap B^c$: Impossible Event.
			
\end{enumerate}

\subsection{Conditional Probabilities}

What can we infer about the probability of an event $B$ knowing that an event $A$ has occurred, aware that there is a link between $A$ and $B$? In other words, if there is a link between $A$ and $B$, the completion of $A$ has to change our understanding of $B$ and we want to know if it is possible to define the conditional probability of an event (relatively) to another event.

This type of probability is named "\NewTerm{conditional probability}\index{conditional probability}" or "\NewTerm{a posterior probability}\index{a posteriori probability}" of $B$ knowing $A$, and is denoted in the context of the study of conditional probabilities:
	
and often in practice to avoid confusion with a possible division:
	
and we sometimes find in U.S. books the notation:
	
or also:
	
We also have the case:
	
	which is named "\NewTerm{likelihood function of $A$}\index{likelihood function}" or "\NewTerm{a priori probability of $A$ given $B$}\index{a priori probability}".

Historically, the first mathematician to have used the correct notion of conditional probability was Thomas Bayes (1702-1761). This is why we often say "\NewTerm{Bayes}\index{Bayes probabilities}" or "\NewTerm{Bayesian}\index{Bayesian probabilities}" probabilities as soon as conditional probabilities are involved: "\NewTerm{Bayes formula}\index{Bayes formula}", "\NewTerm{Bayesian statistics}\index{Bayesian statistics}",etc.

The notion of conditional probability that we will introduce is much less simple than it first appears and the conditionals problems are an inexhaustible source of errors of any kind (there are famous paradoxes on the subject and even expert requires peer review to minimize mistakes).

Let's start with a simple example: Suppose we have two dice. Now imagine that we only launched the first die. We want to know what is the probability that by throwing the second dice, the sum of the two numbers is equal to a given minimum value. Thus, the probability of obtaining the minimum value given the value of the first die is totally different from the probability of obtaining the same minimum value in throwing two dice at the same time. How to calculate this new probability?

Let us now formalize the process! After the launch of the first dice, we have:
	
Under the hypothesis that $B \subset A$ , we feel that $P(B/A)$ must be proportional to $P(B)$, the proportionality constant being determined by the normalization:
	
Now let $B \subset A^c$ ($B$ is included in the complement of $A$ so that the events are mutually exclusive). It is then relatively intuitive .... that under the previous hypothesis of incompatibility we have the conditional probability:
	
This leads us to the following definitions of respectively a posteriori and a priori probabilities:
	
Thus, the fact to know that $A$ has occurred reduces all possible outcomes of the universe $U$ of $B$. From there, only the events of type $A \cap B$ are important. The probability of $A$ given $B$ or vice versa (by symmetry) must be proportional to $P(A \cap B)$!

The coefficient of proportionality is the denominator and it ensures the certain event. Indeed, if two events $A$ and $B$ are independent (think the black cat and the scale for example), then we have:
	
and then we see that $P(B/A)$ is equal to $P(B)$ and therefore the event $A$ adds no information to $B$ and vice versa! So in other words, if $A$ and $B$ are independent, we have:
	
Another fairly intuitive way to see things is to represent the probability measure $P$ as a measure of subsets areas (surface) of $\mathbb{R}^2$.

Indeed, if $A$ and $B$ are two subsets of respective areas $P(A)$ and $P(B)$ then the question of what is the probability that a point in the plane belongs to $B$ knowing that it belongs to $A$ it is quite obvious that the probability is given by answer:
	
We would like to indicate that the definition of conditional probability is often used in the following way:
	
call "\NewTerm{formula of compound probabilities}\index{formula of compound probabilities}". Thus, the a posteriori probability of $B$ knowing $A$ can also be written as:
	
	The way that tis formula gives an update of the probability hypothesis, $B$, in light of some body of data, $A$, is named the "\NewTerm{diachronic interpretation}\index{diachronic interpretation}". "Diachronic" meaning that something is hapenning over time, in this case the probability of the hypothesis changes, over time, as we see new data.

	In this interpretation the different terms have a name:
	\begin{itemize}
		\item $p(B)$ is the probability of the hypothesis before we see the data, name as we already know, the "prior probability", or just "prior".

		\item $p(B/A)$ is what we want to compute, the probability of the hypothesis after we see the data, named as we already know, the "posterior".

		\item $p(A/B)$ is the probability of the data under the hypothesis, named the "\NewTerm{likelihood}\index{likelihood}".

		\item $p(A)$ is the probability of the data under my hypothesis, name the "\NewTerm{normalizing constant}".
	\end{itemize}
	\begin{tcolorbox}[colframe=black,colback=white,sharp corners]
\textbf{{\Large \ding{45}}Example:}\\\\
Suppose a disease like meningitis. The probability of having the meningitis will be denoted by $P(M)=0.001$ (arbitrary value for the example) and a sign of this disease like headache will be noted $P(S)=0.1$. We assume known that the a posteriori probability of having a headache if we have meningitis is:
	
The Bayes' theorem then gives the a priori probability of having meningitis if we have a headache!:
	
	\end{tcolorbox}
We also note that:
	
	So we can know the probability of the event $A$ knowing the elementary probabilities $P(B_i)$ of its causes and the conditional probabilities of $A$ for each $B_i$:
	
	which is named the "\NewTerm{formula of total probabilities}\index{formula of total probabilities}" or "\NewTerm{total probabilities theorem}\index{total probabilities theorem}". But also, for any $j$, we have the following corollary using the previous results that gives us following an event $A$, the probability that it is the cause $B_i$ that produced it:
	
	which is the general form of the "\NewTerm{Bayes formula}\index{Bayes formula}" or "\NewTerm{Bayes' theorem}\index{Bayes' theorem}" that we will us a little in the Statistical Mechanics chapter and through the study of the theory of queues (\SeeChapter{see section Quantitative Management}). You should know that the implications of this theorem are, however, considerable in daily life, in medicine, in industry and in the field of Data Mining.

We often find in the literature many examples of applications of the previous relation with only two possible outcomes $B$ with respect to the event $A$. Therefore we find the Bayes formula written in the following form for each issue:
	
and note that in this particular case (binary outcomes):
	
is an intuitive result.

For binary events, we also have (returning to the theorem of total probabilities seen above):
	

	\begin{tcolorbox}[colframe=black,colback=white,sharp corners]
\textbf{{\Large \ding{45}}Examples:}\\\\
E1. A disease affects 10 people on 10'000 (0.1\% = 0.001). A test has been developed which has a 5\% false positives (people not having the disease but for which the test says they are affected) but still always detects the disease if a person has it. What is the probability that a random person for which the test gives a positive result really has this disease?\\\\There is therefore 10,000 people, 500 of which are false positives, and we know a posteriori that 10 people have really the disease. Then the probability that somebody who has a positive test result is really sick is:
	
This is often a shocking and counter-intuitive result. It also highlights why diagnostic tests must be extremely reliable!\\\\
E2. Two machines $M1$ and $M2$ produce respectively 100 and 200 pieces. $M1$ produces 5\% defective pieces and $M2$ produces 6\% (posterior probabilities). What is the a priori probability that a defective piece was manufactured by the machine $M1$?
We then have:
	
E3. From a batch of 10 pieces with 30\% defective, we take a sample of size 3 without replacement. What is the probability that the second piece is correct (whatever the first is)?\\\\
We have:
	
	\end{tcolorbox}
	\begin{tcolorbox}[colframe=black,colback=white,sharp corners]
E4. We conclude with an important example for companies where employees have more time in their career to pass exams or assessments in the form of multiple choice questions (MCQs). If an employee responds to a question there are two issues: either he knows the answer or he try to guess it. Let $p$ be the probability that the employee know the answer and therefore $1-p$ that he guess it. We admit that the employee who guesses will correctly answer with a probability of $1/m$ where $m$ is the number of proposed answers. What is the a prori probability that an employee (really) knows the answer to a question with 5 choices if he answered correctly?\\\\

Let $B$ and $A$ be respectively the events "the employee knows the answer" and "the employee correctly answers the question". Then the a priori probability that an employee knows (really) the answer to a question that he answered correctly is:
	
	\end{tcolorbox}
Bayesian analysis provides also a powerful tool to formalize reasoning under uncertainty and the examples we have shown above illustrate how this tool can be difficult to use.	

\subsubsection{Conditional Expectation}

Now, we will see to the continuous version of the conditional probability by introducing the subject directly with a particular example (the general theory being indigestible) infinitely important in the field of social statistics and quantitative finance. However, this choice (the study of a particular case) implies that the reader has read the first chapter of Statistics to study the functions of continuous distributions and especially that of the Pareto law.

So here's the scenario: Often, in social sciences or economics, we find in the literature dealing with the laws of Pareto statements like the following (but almost never with a detailed proof): whatever your income, the average income of those who have an income above yours is in a constant ratio, greater than 1, to your income if it follows a Pareto random variable. Then we say that the law is isomorphic to any truncated part itself.

Let $X$ be a random variable equal to the income and following a Pareto with the density (\SeeChapter{see section Statistics}):

Let's see what it is exactly:

with $k>1,x_m>0,x \geq x_m$ and that has for distribution function (see also the Statistics chapter for the detailed proof):

The sentence begins with "whatever your income ...", then select any income $x_0 \geq x_m$.

Now we need to compute "the average income of those with income higher than $x_0$". It is therefore asked to calculate the expected (average income) of a new random variable $Y$ that is equal to $X$, but restricted to the population of people with an income above $x_0$:

The distribution function of $Y$ is given by:

This expression is of course equal to zero if $x \leq x_0$.
Well, so far we have only do vocabulary. First recall the following conditional probability relation already seen before:

for $x \geq x_0 $ we have for the conditional law:

Before going further, you should be aware that the numerator and denominator are independent but that the whole must be considered, however, as the realization of a single random variable which we denote $Y$. Furthermore, only the numerator is a dependent variable. The denominator can it be considered as a normalization constant.

So we see that the density of $Y$ is given by the function:
	\begin{gather*}
	f_Y(y) = \begin{cases}
	\qquad 0 & y<x_0 \\
	\dfrac{f(y)}{P(X \geq x_0)} & y \geq x_0 \\
	\end{cases}\\
	\end{gather*}
Now we can calculate the expectation of $Y$:
	
Knowing that (\SeeChapter{see section Statistics}):
	
We finally have:
	
$E(Y)$ represents also the average income of those with an income above $x_0$ and as can be seen from the above equality it is in a constant ratio, greater than 1, to your income $x_0$.

We can check this result by doing a Monte Carlo simulation in a spreadsheet software (it is interesting to mention it to generalize to situations not computable by hand). You just need to simulate the inverse of the distribution function:
	
in Microsoft Excel 11.8346:

\begin{center}
 \texttt{=(\$B\$7\^{}\$B\$6}/(1-RANDBETWEEN(1,10000)/10000))\^{}(1/\$B\$6)
\end{center}

and then take the average of the values obtained above or equal to a given $X$ (which corresponds to $x_0$) and ensure that we get the good results as proved above!

Obviously, we could also calculate the conditional variance (verbatim the conditional standard deviation). It may come one day...

\subsubsection{Bayesian Networks}

Bayesian networks are simply a graphical representation of a problem of conditional probabilities to better visualize the interaction between the different variables when they begin to be in large numbers.

This is a technique increasingly used in decision aided software (Data Mining), artificial intelligence (A.I.) and also in the analysis and risk management (ISO 31010 norm).

Bayesian networks are by definition directed acyclic graphs (\SeeChapter{see section  Graphs Theory}), so that an event can not (even indirectly) influence its own probability, with quantitative description of dependencies between events.

These graphs are used for both knowledge representation models and calculating conditional probabilities machines. They are mainly used for diagnosis (medical and industrial), risk analysis (diagnostics failures, faults or accidents), spam detection (Bayesian filter), voice text and image opinions analysis, fraud detection or bad payers as well as data mining (M.K.M.: Mining and Knowledge Management) in general.

	\begin{tcolorbox}[title=Remark,colframe=black,arc=10pt]
Many systems and software based on drawings or on information in existing databases exists to build and analyse Bayesian networks. Paid solutions: SQL Server, Oracle, Hugin. Free solutions (at this date): Tanagra, Microsoft Belief Network MSBNX 1.4.2, RapidMiner. Personally I prefer the simplicity of the small software MSBNX from Microsoft. For information, in 15 years of professional experience as a consultant I have met so far only one company on more than 800 multinationals which used Bayesian networks... (in transportation).
	\end{tcolorbox}

Use a Bayesian network is assimilated to do "\NewTerm{Bayesian inference}\index{Bayesian inference}". Based on observed information, we calculate the probability of possible known data but not observed.

For a given domain (e.g. medical), we describe the causal relations between variables of interest by a graph (we do not need again to specify that it is acyclic). In this graph, the causal relations between variables are not deterministic, but probabilistic. Thus, the observation of a cause or multiple causes does not always implies the effect or effects that depend on it, but only changes the probability of observing them.

The particular interest of Bayesian networks is to consider simultaneously a priori knowledge of experts (in the graph) and experience contained in the data.

Example of 5 variables with relations (directed acyclic graph) and numbering of states/variables:
\begin{figure}[H]
\centering
\includegraphics{img/arithmetics/bayes_network_example.eps}
\caption{Example of Bayesian network (acyclic oriented) with 5 states}
\end{figure}

Obviously, the construction of the causal graph is based primarily on return of experience (REX) and sometimes results on standards or reports of expert committees. In computing science, the causal graph automatically change depending on the content of databases (think at the Amazon book store in real time target advertisements based on your past purchases or at the Genius Apple service). But we can rarely think to all possibilities and there will sometimes hidden states between two known states that have been forgotten and that would have allowed to better modelize the situations.

Suppose in the example above that with the help of a corporate database, we know that in about $100,000$ man-days, we hat in the company $1,000$ accidents (i.e.. $1\%$ of total) and $100$ machines failures (i.e.. $0.01\% $of total). Then we represent it in the traditional form as follows:
\begin{figure}[H]
\centering
\includegraphics{img/arithmetics/bayes_network_departure.eps}
\caption{Directed acyclic Bayesian network with probability of departure}
\end{figure}

where we have the subset $S2, S4, S5$ which is what experts name a "\NewTerm{serial or linear connection}\index{serial or linear connection}", the triplet $S3, S2, S4$ is a named a "\NewTerm{divergent relation}\index{divergent relation}" (if the arrows were reversed for the triplet, we would have a "\NewTerm{converging relation}\index{converging relation}").

Before going further with our example we will make some observations in relation to these three types of relations:

For clarity, we distinguish first "\NewTerm{conditional independence}\index{conditional independence}" and "\NewTerm{conditional dependence}\index{condition dependence}".

We say that events $A$ and $C$ are "\NewTerm{conditionally independent}\index{conditionally independent}" if given an event $B$ the following equality holds:
	
So the term "conditional" implies the presence of $B$ and the fact that $C$ does not influence the probability of the event $A$.

About "\NewTerm{conditional dependence}", this time we can distinguish three types of relations.
\begin{enumerate}
	\item The conditional dependence of the following type is named a "\NewTerm{serial or linear connection}\index{serial or linear connection}" (already mentioned above):
\begin{figure}[H]
\centering
\includegraphics[scale=0.8]{img/arithmetics/linearconnection.eps}
\caption{Serial/linear Bayesian network}
\end{figure}
where $A$, $B$ and $C$ are dependent (in this particular example there are 3 dependent nodes $A$, $B$ and $C$, but in general this dependence relates to all nodes if there were more than 3)

In addition, $A$ and $C$ are conditionally dependent to $B$. But if the variable $B$ is known, $A$ no longer provides any useful information about $C$ (the path of uncertainty is somehow broken) and therefore $A$ and $C$ become conditionally independent. We then have the conditional probability that simplify as follows:
	
	\item The conditional dependence of the following type is named a "\NewTerm{divergent connection}\index{divergent connection}" (as already mentioned above):
\begin{figure}[H]
\centering
\includegraphics{img/arithmetics/divergentlink.eps}
\caption{Divergent Bayesian network}
\end{figure}
In addition, $B$ and $C$ are conditionally dependent on $A$. But if $A$ is known, $B$ does not provide any more information on $C$ (again the path of uncertainty is somehow broken) and therefore $B$ and $C$ become independent. We then have for example if $A$ is known:
	
	\item The conditional dependence of the following type is named a "\NewTerm{convergence connection}\index{convergence connection}" or "\NewTerm{$V$-Structure}\index{$V$-structure}" (as already mentioned above):
\begin{figure}[H]
\centering
\includegraphics{img/arithmetics/vstructure.eps}
\caption{Convergent Bayesian network}
\end{figure}
where this time the parents are independent.
So $B$ and $C$ are independent but become conditionally dependent on $A$. If $A$ is known, then we have:
	
The dependence between parents therefore requires the observation of their common child.
\end{enumerate}

Now to make a concrete example, suppose our database gives us (thanks to quality managers who always inputs the quality issues) that when a machine failure occurred, 99 times out of 100 (99\%) there has been a total production stop (i.e. 1\% of time there was no production stop) and on all stop production 1\% was not due to a machine failure. What we traditionally represent as follows:
\begin{figure}[H]
	\centering
	\includegraphics[scale=0.8]{img/arithmetics/first_level_bayesian_network.jpg}
	\caption{1st level Bayesian network}
\end{figure}
So the "\NewTerm{implicit probability}\index{implicit probability}" that there is a production stop is given by:
	
	This value represents the implicit proportion of productions stop from the 100,000 man-days (so we can give a proportion of rows in the database that represents a production stop regardless of the cause and even without knowing the details of the database)..
	
	It then follows immediately that the implicit probability that there is no production stop is given by:
		
	This is consistent with what gives us the freeware MSBNX 1.4.2:
	\begin{figure}[H]
		\centering
		\includegraphics[scale=0.75]{img/arithmetics/msbnx_beginning.eps}
		\caption{Beginning of the Bayesian network with MSBNX 1.4.2}
	\end{figure}
	Now suppose we observed a production stop. What is the a posteriori probability that it is due to a machine failure? We then have:
		
	We can also check this with the software MSBNX 1.4.2:
	\begin{figure}[H]
		\centering
		\includegraphics[scale=0.75]{img/arithmetics/msbnx_machine_failure_aposteriori_probability.eps}
		\caption{A Posteriori probability of a production stop due to a machine failure in MSBNX 1.4.2}
	\end{figure}
	Now, imagine that our database gives us (always thanks to quality managers who ensured to input quality issues) that 99 times out of 100 (99\%) when there was a production stop, there was an evacuation. However 5\% of evacuations were identified as having nothing to do with a production stop (i.e. 95\% of evacuations are due to fire exercises OR other events):
	\begin{figure}[H]
		\centering
		\includegraphics[scale=0.8]{img/arithmetics/second_level_bayesian_network.jpg}
		\caption{2nd level Bayesian network}
	\end{figure}
	Now to calculate the implicit probability retrospectively (a posteriori) of evacuations compared to machines failures, we saw that when we had a conditional dependence serie, the conditional probability depends only on the direct parent. Thus, we get:
		
	We can also check this with the software MSBNX 1.4.2:
	\begin{figure}[H]
	\centering
	\includegraphics[scale=0.75]{img/arithmetics/msbnx_implicit_probability_evacuation.eps}
	\caption{Implicit probability of an evacuation in MSBNX 1.4.2}
	\end{figure}
	So the implicit probability of evacuation does not actually depend on machine failures.

	Now suppose we have observed an evacuation. We want to know what is the a posteriori probability that it is due to a machine failure! Then we have:
	
	We can also check this with the software MSBNX 1.4.2:
	\begin{figure}[H]
		\centering
		\includegraphics[scale=0.75]{img/arithmetics/msbnx_a_posteriori_probability_evacuation.eps}
		\caption{A Posteriori probability of an evacuation due to a machine failure in MSBNX 1.4.2}
	\end{figure}
	
	Now we study the case with the alarm and again a database allows us to build a table with different probabilities:

	\begin{figure}[H]
		\centering
		\includegraphics[scale=0.75]{img/arithmetics/second_level_bayesian_network_second_branch.jpg}
		\caption{2nd level Bayesian network with second branch}
	\end{figure}

	Now to calculate the implicit probability that there is an alarm, we will have to consider the four possible situations. We then use the theorem of total probability:
	
What a little more rigorously should be written:
	
The numerical application therefore provides for the implied probability of an alarm:
	
	What can be built and check as follows with MSBNX 1.4.2:
\begin{figure}[H]
	\centering
	\includegraphics[scale=0.75]{img/arithmetics/msbnx_implicit_probability.eps}
	\caption{Implicit probability in MSBNX 1.4.2}
\end{figure}

	It may be useful to the reader to know that he can sometimes found in the literature the following notation:
	
	
	\begin{tcolorbox}[title=Remark,colframe=black,arc=10pt]
	In the particular example studied above all event have only two states. But in practice they can have 3, 4 and more states. Therefore probabilities cross-tables quickly become enormous.
	\end{tcolorbox}

	As in the previous case, suppose we know that there was a working accident. We wish then calculate the a priori probability of an alarm. We then have (observe that the probability depends actually only to the state $S2$ state since the state $S1$ is completely known!):
	
	We can also check this with the software MSBNX 1.4.2:
	\begin{figure}[H]
		\centering
		\includegraphics[scale=0.75]{img/arithmetics/msbnx_implicit_probability_alarm.eps}
		\caption{Implicit probability of an alarm in MSBNX 1.4.2}
	\end{figure}
	So, knowing that there was a work accident increases the probability that there is an alarm (we start from a probability of $10.089\%$ to go to a probability of $10.65\%$).

	To complete this example, we would calculate the a posteriori probabilities $P(S2/S3)$ and $P(S1/S3)$. To do this, we must first calculate the a priori probabilities  $P(S3/S2)$ and $P(S3/S1)$ (this last one has been calculated just before).

	We have for the missing value (which can be easily checked as before with MSNBX 1.4.2 software):
	
We then have:
	
We now have everything we need to calculate the a priori probability of $P(S2/S3)$ and $P(S1/S3)$:
	
	So the a priori probability that there is a machine breakdown when we know that there is an alarm is 0.979\% (i.e. 0.021\% that the trigger of the alarm is not a priori due to a machine failure). Respectively there is, a priori, 0.998\% probability that there is a work accident when we know there is an alarm (and then 0.002\% that it is not a priori due to a work accident).

	From the critical point of view, when there is finally an alarm we can not say a lot of things.... This is because, in this case, to the fact that the events of significant interest both have low probability to occur (work accident and machine failure) and that the employees respond quite well at the start of the alarm (otherwise if the a priori probabilities were high it would mean that the behavior of the employees is not good because we can guess - with exasperation - in advance which problem occurs with a good confidence).

	\begin{tcolorbox}[title=Remark,colframe=black,arc=10pt]
	We did not find how to check these last calculations with MSNBX 1.4.2. If someone finds how to do, it would be great to give us the details of the process.
	\end{tcolorbox}	
	
	To conclude, the reader will have noticed that the calculations can quickly become annoying when the graph becomes complex and this explains the use of computer software. Furthermore, in the banking sector that uses for example Bayesian networks for credit risk, the a priori probability can be more complex. For example we might want to know the a priori probability that there is a machine failure knowing that we have an alarm and an accident:
	

\subsection{Martingales}

A martingale in probabilities (there is another one in stochastic processes) is a technique to increase the chances of winning in gambling while respecting the rules of the game. The principle is completely dependent on the type of game we are focusing, but this term is accompanied by an aura of mystery that some players would know efficient secret techniques to cheat with chance. For example, many players (or candidates to play) search THE martingale that will beat the bank in the most common games in casinos (institutions whose profitability relies almost entirely on the difference - however small - between the chances of winning and losing).

Many martingales are the dream of their author, some are actually inapplicable, some could actually give the possibility to cheat a little. Gambling in general are unfair: whatever the shot played, the probability of winning of the casino (or of the State in the case of a lottery) is greater than this of the player. In this type of game, it is not possible to inverse the chances, just to minimize the probability of gambler's ruin.

The most common example is the roulette wheels martingale. It consists to play a single chance to the roulette wheels (red or black, odd or even) to win, for example, a unit in a series of moves by doubling his bet if we lose, and that until we earn. Example: the player bets 1 unit on red, if red comes out, it stops playing and won 1 unit (2 units less gain setting unit), if black comes out, he doubles his betting by 2 units on red and so on until he wins.

\begin{figure}[H]
\centering
\includegraphics{img/arithmetics/casino_roulette.eps}
\caption[]{Casino roulette wheel}
\end{figure}

Having a chance on two to win, he may think he will eventually win, and when he wins, he is necessarily paid for everything he has played more one unit of his initial bet.

This martingale appears to be safe in practice. Note that in theory, to be sure of winning, we should have the opportunity to play an unlimited number of times.... This has major drawbacks:

This martingale is in fact limited by the bets that the player can do because you have to double the bet every time you lose: 2 times the initial bet, then 4, 8, 16 .... if he loses 10 times, he must be able to bet 1024 times its initial investment for the 11th party! Therefore a lot of money for little gain!

The roulette wheels also have a "0" which is neither red nor black. The risk of losing at every shot is is then larger than 1/2...

In addition, to paralyze this strategy, casinos offer table games per set: from 1 to 100.-, from 2 to 200.-, from 5 to 500.-, ... Therefore it is impossible to use this method on a large number of shots, which increases the risk of losing it all.

Blackjack is a game that has winning strategies: several playing techniques, which usually require to memorize the cards, can overturn the chances in favour of the player. The mathematician Edward Thorp has published in 1962 a book that was at the time a real best-seller. But all these methods require long training weeks and are easily detectable by the croupier (sudden changes in the amounts of bets are typical). The casino has then the opportunity to banish from its establishment the players using this playing martingale.

It should be noted that there are enough advanced methods. One of them is based on the less played combinations. In games where the gain depends on the number of winning players (Lotto...), playing the least played combinations maximize gains. This is how some people sell combinations that would be statistically very rarely used by other players.

Based on this reasoning, we can still conclude that a player who would have been able to determine statistically the least played combinations, to maximize its expected payoff, will in fact certainly not be the only player to have achieved this by the analysis of these famous combinations! This means that in theory the numbers the least played are actually overplayed combinations, the best might be to achieve a mix of played numbers and overplayed numbers to play for the ideal combinations Another conclusion to all this is maybe that the best is to play random combinations which ultimately are less likely to be chosen by the players who incorporate a human and harmonious factor in the choice of their numbers.

\subsection{Combinatorial Analysis}

"\NewTerm{Combinatorial analysis}\index{combinatorial analysis}" (counting techniques) is the field of mathematics that deals with the study of all the issues, events or facts (distinguishable or indistinguishable) with their arrangements (combinations) ordered or not according to some given constraints.

\textbf{Definitions (\#\mydef):}
\begin{enumerate}
	\item[D1.] A sequence of objects (events, issues, objects, ...) is said "\NewTerm{ordered}\index{ordered sequence}" if each suite with a particular order of objects is recognized as a particular configuration.
	
	\item[D2.] A sequel is "\NewTerm{unordered}\index{unordered sequence}" if and only if we are interested in the frequency of appearance of objects regardless of their order.
	
	\item[D3.] The objects (of a sequence) are said "\NewTerm{distincts}\index{distincts objects}" if their characteristics can not to be confused with the other objects.
\end{enumerate}

	\begin{tcolorbox}[title=Remark,colframe=black,arc=10pt]
We chose to put combinatorial analysis in this chapter because when we calculate probabilities, we also often need to know what is the probability of finding a combination or arrangement of given events under certain constraints.
	\end{tcolorbox}

Students often have difficulty remembering the difference between a permutation, an arrangement and a combination. Here is a little summary of what we'll see:
	\begin{itemize}
	 	\item \NewTerm{Permutation}\index{permutation}: We take all the objects.

	 	\item \NewTerm{Arrangement}\index{arrangement}: We choose objects from the original set and the order intervenes.

	 	\item \NewTerm{Combination}\index{combinatorial}: Same as for the arrangement, but the order does not interfere.	
	\end{itemize}

You must not forget that for each result, the reverse will give the probability of falling respectively on a given permutation/arrangement/combination!

We will present and demonstrate below the 6 most common cases from which we can find (usually) all others:

\subsubsection{Simple Arrangements with Repetition}

\textbf{Definition (\#\mydef):} A "\NewTerm{simple arrangement with repetition}\index{simple arrangement with repetition}" or equivalently a "\NewTerm{random sampling with replacement (RSWR)}\index{random sampling with replacement (RSWR)} is an ordered sequence of length $m$ of $n$ distinct objects not necessarily all different in the sequence (either: with possible repetitions!).

Let $A$ and $B$ be two finite sets of respective cardinal $m, n$ such that there is trivially $m$ ways to choose an object in $A$ (of type $a$) and $n$ ways to choose an object in $B$ (of type $b$).

We saw in the section Set Theory that if $A$ and $B$ are disjoint, that:
	

We therefore deduce the following properties:
	\begin{enumerate}
		\item[P1.] If an object can not be at the same time of type $a$ and type $b$ and if there is $m$ ways to select an object of type $a$ and $n$ ways to choose an object of type $b$, then the union of objects gives $m+n$ selections (this is typically the result of the SQL UNION queries without filters in corporate Relational Databases Management System).
		\item[P2.] If we can choose an object type of type $a$ in $m$ ways then an object of type $b$ in $n$ ways, then there is according to the Cartesian product of two sets (\SeeChapter{see section Set Theory})
		
ways to choose a single object of type $a$ then an object of type $b$.
	\end{enumerate}
With the same notation for $m$ and $n$, we can choose for each element of $A$, its single image among the $n$ elements of $B$. So there are $n$ ways to choose the image of the first element of $A$, then also $n$ ways to choose the image of the second element of $A$, ..., and $n$ ways to choose the image of the $m$-th element of $A$ . The total number of consecutive possible applications from $A$ to $B$ is thus equal to the $m$ product of $n$ (thus $m$ times the cartesian product of the cardinality of the set $B$ with itself!). It is usual to write it under the following way (we have indicated the different ways to write his result as it can be found in various textbooks):
	
where $B^A$ is the set of applications from $A$ to $B$. The increase in the number of possibilities is geometric (not "exponential" as it is often wrongly said!).

This result is mathematically similar to the ordered result (an arrangement where the order of elements in the sequence is taken into account) of $m$ trials in a bag containing $n$ different balls with replacement after each trial. In France this result is traditionally named a "\NewTerm{$p$-list}\index{$p$-list}".

	\begin{tcolorbox}[colframe=black,colback=white,sharp corners]
\textbf{{\Large \ding{45}}Examples:}\\\\
E1. How many (ordered) "words" of 7 letters can we form from a separate alphabet of 24 letters (very useful to know the number of trials to find a password for example)? The solution is:
	
E2. How many groups of people will we have in a referendum on 5 subjects and where each can be either accepted or rejected? The solution is (widely used in some Swiss companies):
	
	\end{tcolorbox}
A simple generalization of this result can consist of the following problem statement:

If we have $m$ such objects $k_1,k_2,...,k_m$ as $k_i$ may take $n_i$ different values then the number of possible combinations is:
	
And if $n_1=n_2=...=n_m$ we have equation then we fall back on:
	
	
	\subsubsection{Simple Permutations without Repetitions}
	\textbf{Definition (\#\mydef):} A "\NewTerm{simple permutation without repetition}\index{simple permutation without repetition}" (formerly named "substitution") of $n$ distinct objects is an ordered (different) sequence of these $n$ objects all different by definition in the sequence (without repetition).

	\begin{tcolorbox}[title=Remark,colframe=black,arc=10pt]
	Be careful not to confuse the concept of permutation ($n$ elements between them) and this of arrangement (of $n$ elements among $m$)!
	\end{tcolorbox}	

	The number of permutations of $n$ items can be calculated by induction: there are $n$ places for a first element, $n-1$ for a second element, ..., and there will be only one place for the last remaining element.

	It is therefore trivial that we the number of permutations is given by:
	
	Recall that the product:
	
is named the "\NewTerm{factorial of $n$}\index{factorial}" and we note it $n!$ for $n \in \mathbb{N}$.

There is therefore for $n$ distinguishable elements:
	
as possible permutations. This type of calculation can be useful for example in project management (calculation of the number of different ways to get in a production line $n$ different parts ordered from external suppliers).	

\begin{tcolorbox}[colframe=black,colback=white,sharp corners]
\textbf{{\Large \ding{45}}Example:}\\\\
How many (ordered) "words" of 7 different letters without repetition can we create?
	
This result leads us to assimilate it to the ordered results (an arrangement $A_n$ equation in which the order of elements in the sequence is taken into account) of the trial of balls that are all different from a bag containing $n$ distinguishable balls without replacement.
	\end{tcolorbox}

\subsubsection{Simple Permutations with Repetitions}

\textbf{Definition (\#\mydef):} A "\NewTerm{simple permutation with repetition}\index{simple permutation with repetition}" is when we consider the number of ordered permutations (different) of a sequence of $n$ distinct objects not necessarily all different in a given quantity.

	\begin{tcolorbox}[title=Remark,colframe=black,arc=10pt]
Do not confuse this definition with the "simple arrangement with repetition" seen before!
	\end{tcolorbox}	
	
When some elements are not all distinguishable in a sequence of objects (they are repeated in the sequence), then the number of permutations that we can be do are then trivially reduced to a smaller number then if all the elements were all distinguishable.

Consider $n_i$ as the number of objects of the type $i$, with:	
	
then we write:
	
the number of possible permutations (yet unknown) with repetition (one or more elements in a sequence of repetitive elements are not distinguishable by permutation).

If each of the $n_i$ positions occupied by identical elements were occupied by different elements, the number of permutations could then have to be multiplied by each of the $n_i!$ (previous case).
	
then we deduce:
	
If the $n$ objects are all different in the sequence, we then have:
	
and we fall back again on a simple permutation (without repetition) as:
	

	\begin{tcolorbox}[colframe=black,colback=white,sharp corners]
\textbf{{\Large \ding{45}}Example:}\\\\
How many (ordered) "words" can we create with the letters of the word "Mississippi":
	
	\end{tcolorbox}
This result leads us to assimilate it to an ordered result (a permutation $\bar{A}_n$  where the order of elements in the sequence is not taken into account) of the trial of balls that are not all different from a bag containing $k \geq n$ balls with limited replacement for each ball.

\subsubsection{Simple Arrangements with Repetitions}

\textbf{Definition (\#\mydef):} A "\NewTerm{simple arrangement without repetition}\index{simple arrangement without repetition}" or equivalently a "\NewTerm{random sampling without replacement (RSWOR)}\index{random sampling without replacement (RSWOR)} is an ordered sequence of $p$ objects all distinct taken from $n$ distinct objects with $n \geq p$.

We now propose to enumerate the possible arrangements of $n$ objects among $p$ without repetition. We denote $A_n^p$ the number of these arrangements.

It is easy to calculate that $A_n^1=n$ and to check that $A_n^2=n(n-1)$. Indeed, there are $n$ ways to choose the first object and $(n-1)$ ways to choose the second when we already have the first.

To determine a nice expression for $A_n^p$ , we reason by induction. We assume equation known and we deduce that:
	
It comes: 
	
then:
	
whence:
	
This result leads us to assimilate it to the ordered results (an arrangement $A_n^p$ in which the order of elements in the sequence is taken into account) of the trial of $p$ distinct balls from a bag containing n different balls without replacement.


	\begin{tcolorbox}[colframe=black,colback=white,sharp corners]
	\textbf{{\Large \ding{45}}Example:}\\\\
	Consider the 24 letters of the alphabet, how many (ordered) "words" of 7 distinct letters can we create?
	
	\end{tcolorbox}
The reader may have noticed that if $p=n$ we end up with:
	
So we can say that a simple permutation of $n$ elements without repetition is like a simple arrangement without repetition when $n=p$.

\subsubsection{Simple Combinations without Repetitions}

\textbf{Definition (\#\mydef):} A "\NewTerm{simple combination without repetitions}\index{simple combination without repetitions}" or "\NewTerm{choice function}\index{choice function}" is an non-ordered sequence (where the order doesn't interest us!) of $p$ elements all different (not necessarily in the visual sense of the word!) selected from $n$ distinct objects and is by definition denoted $C_p^n$ in this book and named the "\NewTerm{binomial}\index{binomial}" or "\NewTerm{binomial coefficient}\index{binomial coefficient}".

If we permute the elements of each simple arrangement of elements $p$ of $n$, we get all simple permutations and we know that there are in a number of $p!$, using the notation convention of this book we then have (contrary to that recommended one by ISO 31-11!):
	
It is a relation often used in gambling but also in the industry trough the hypergeometric distribution (\SeeChapter{see section Quantitative Management} as well as quite high level statistics like order statistics (\SeeChapter{see chapter Statistics}).

	A simple way to remember this function is the following trick: Consider we must select $p$ among $n$ independently of the ordery what are the number of possibilities? 
	
	We know that we have $6\cdot 5\cdot 4=120$ to select them all taking into account the order! The calculation we just made is obviously equal to $n!/p!=6!/3!= 6\cdot 5\cdot 4$. But as the order must not be taken into account we must divide the $120$ by the number of ways we can arrange the $3$ people in the group. So we divide $120$ by $3!$ or more generally and logically by $(n-p)!$. Hence the relation above!
	\begin{tcolorbox}[title=Remark,colframe=black,arc=10pt]
	\textbf{R1.} We have necessarily by construction $C_p^n \leq A_n^p$.\\

	\textbf{R2.} Depending on the authors we inverse the index and suffix of $C$ then you must be careful!
	\end{tcolorbox}
This result leads us to assimilate it to the unordered result (an arrangement $C_n^p$ in which the order of elements of the sequence is not taken into account) of the trial of $p$ balls of a bag containing $n$ different balls without replacement.
	\begin{tcolorbox}[colframe=black,colback=white,sharp corners]
	\textbf{{\Large \ding{45}}Example:}\\\\
	E1. Consider the 24 letters of the alphabet, how many choices do we have to take 7 letters in the 24 without taking into account the trial order?
		
	The same value can be obtained with the function $\texttt{COMBIN( )}$ of Microsoft Excel 11.8346 (English version).\\
	
	E2. In a Design of Experiment (\SeeChapter{see section Industrial Engineering}) we have $2$ factors of $L=3$ levels each and therefore we need $N=9$ runs to completely determine all the interactions. If we consider that we can take a subset of $S=3$ runs, how many combinations of $3$ among the $9$ can we choose if repititions are vorbidden?
	
	We understand therefore why in Design of Experiments it is important to found a trick to choose the best subset (D-optimum designs)
	\end{tcolorbox}
	There is, in relation to the binomial coefficient, another relation very often used in many case studies and also more globally in physics or functional analysis. This is the "\NewTerm{Pascal's Formula}\index{Pascal's Formula}":
	\begin{dem}
		
We also have $p!=p(p-1)!$, then:
	
and because $(n-p)(n-p-1)!=(n-p)!$:
	
Then:
	
	\begin{flushright}
		$\square$  Q.E.D.
	\end{flushright}
	\end{dem}
	
\subsubsection{Simple Combinations with Repetitions}
	
\textbf{Definition (\#\mydef):} A "\NewTerm{simple combination with repetition}\index{simple combination with repetition}" of $p$ elements of $n$ is a collection of $p$ non-ordered elements, not necessarily distinct.

Simple combinations with repetition are very important for the Wald-Wolfowitz statistical test used in economics and biology and that we will study in the Statistics section.

We will Introduce this kind of combination directly with an example an ingenious approach that we have thanks to the physicist and 1938 Nobel Prize in Physics: Enrico Fermi.

Consider $\left\lbrace a, b, c, d, e, f\right\rbrace $ a set having a number $n$ of elements equal to 6 and where we draw a number of elements $p$ equal to 8. We would like to calculate the number of combinations with repetition of elements in a starting set of cardinal 6 in a destination set of cardinal 8.

Consider, for example, the following three combinations:
	
where as the order of elements does not occur, we have grouped the elements to facilitate the reading. Now represent all the above elements by the same symbol "0" and separate the groups consisting of a single element by bars (this is the trick Enrico Fermi). Thus, when one or more elements are not included in a combination, we still denote the separation bars (corresponding to the number of missing elements + the separation of group). Thus, the three combinations above can be written as:
	
We see above that in each case, there are eight "0" (logic. ..) but also that there are also always five "$\mid$". The number of combinations with repetitions of six elements of a starting set to the final one of 8 elements is equal to the number of permutations with repetitions of $8+5=13$ elements, so:
	
We also see that in the general case the number of combinations without consideration of repetitions order can also be written:
	
It is traditional to write this:
	
We also see that:
	
Then:
	
That we also sometimes write:
	
	To resume:
	
	
	\begin{figure}[H]
	\centering
	\includegraphics[scale=0.75]{img/arithmetics/gaming.eps}
	\end{figure}


\subsection{Markov Chains}

Markov chains are simple but powerful probabilistic and statistical tools but for which the choice of the mathematical presentation can sometimes be a nightmare... We will try here to simplify a maximum the writings to introduce this great tool widely used in businesses to manage supply chain, in queuing theore for call centers or cash desk, in failure theory for preventive maintenance, statistical physics and biological engineering and also in time series analysis and forecasting (and the list goes on and for more details the reader should refer to the relevant chapters available in this book...).

\textbf{Definitions (\#\mydef):}
	\begin{enumerate}
		\item[D1.] We note by $\left\lbrace X(t) \right\rbrace_{t \in T} $  a probabilistic process function of time whose value at any time depends on the outcome of a random experiment. Thus, at each time $t, X(t)$ is a random variable that we name "\NewTerm{stochastic process}\index{stochastic process}" (for more details on financial applications see the chapter Economy).

		\item[D2.] If we consider a discrete time, we then note  "\NewTerm{discrete time stochastic process}\index{discrete time stochastic process}" as $\left\lbrace X_n \right\rbrace_{n \in \mathbb{N}} $.

		\item[D3.] If we further assume that the random variables $X_n$ can take only a discrete set of values we then speak of "\NewTerm{process in discrete time and discrete space}\index{process in discrete time and discrete space}".
	\end{enumerate}
	\begin{tcolorbox}[title=Remark,colframe=black,arc=10pt]
It is quite possible as in the study of communications flows (\SeeChapter{see section Quantitative Management}) of having a process in continuous time and discrete state space.
	\end{tcolorbox}
\textbf{Definition (\#\mydef):} $\left\lbrace X_n \right\rbrace_{n \in \mathbb{N}} $ is a "\NewTerm{Markov chain}\index{Markov chain}" if and only if:
	
	
in other words (it is very easy!) the probability that the chain is in a certain state on the $n$-th step of the process depends only on the state of the process at step $n-1$ and not on any previous steps!

	\begin{tcolorbox}[title=Remark,colframe=black,arc=10pt]
Also in probabilities a stochastic process verifies the Markov property if and only if the conditional probability distribution of future states, given the present moment, depends only on the present state and not even past states as the relation above. A process with this property is also named a "\NewTerm{Markov process}\index{Markov process}".
	\end{tcolorbox}
\textbf{Definition (\#\mydef):} A "\NewTerm{homogeneous Markov chain}\index{homogenous Markove chain}" is a chain such that the probability that it has to go in a certain state at the $n$-th stage is independent of time. In other words, the probability distribution characterizing the next step does not depend on time (of the previous step), at all times the probability distribution of the chain is always the same for characterizing the transition to the current step.

We can then define (reduce) the law of "\NewTerm{probability transition}\index{probability transition}" of a state $i$ to state $j$ by:
	
It is then natural to define the "\NewTerm{transition matrix}\index{transition matrix}" or "\NewTerm{stochastic matrix}\index{stochastic matrix}":
	
as the matrix that contains all possible transition probabilities between states in an oriented graph.

Markov chains can be represented graphically as an oriented graph $G$ (\SeeChapter{see section Graph Theory}) sometimes named "\NewTerm{automate}\index{automate}" having for the top points (states) $i$ and for the edges the oriented couples $(i, j)$. We then associate to each component an oriented arc and a transition probability.

	\begin{tcolorbox}[colframe=black,colback=white,sharp corners]
	\textbf{{\Large \ding{45}}Example:}\\\\
	\begin{figure}[H]
	\centering
	\includegraphics{img/arithmetics/markov_chain.eps}
	\caption{Generic example of a Markov chain}
	\end{figure}
	Thus, in the example of the oriented graph above, the only allowed transitions from the above $4$ states ($4 \times 4$ matrix) are those indicated by the arrows. So that the transition matrix is then simplified to:
	
	\end{tcolorbox}

The reader has seen in the previous example that we have the trivial property (by construction!) that the sum of the terms (probabilities) of a row of the matrix $P$ is always unitary (and therefore the sum of the terms of a column of the transpose of the matrix unit is still equal to the unit too):
	
and that the matrix is positive (meaning that all its terms are non-negative).

	\begin{tcolorbox}[title=Remark,colframe=black,arc=10pt]
Remember that the sum of the probabilities of the columns is always equal to $1$ for the transpose of the stochastic matrix!
	\end{tcolorbox}	

The analysis of transient state (or: random walk) of a Markov chain consist to determine (or to impose!) to the column-matrix (vector) $p(n)$ to be in a given state at $n$-th step of the walk:
	
with the sum of the components that is always equal $1$ (since the sum of the probabilities of being in any of the vertices of the graph at given a time/step must be equal to $100\%$).

We frequently name this column matrix "\NewTerm{stochastic vector}\index{stochastic vector}" or "\NewTerm{probability measure on the vertex $i$}\index{probability measure}".

\begin{theorem}
We want to prove that the total probability of this stochastic vector is always equal to $1$.
\end{theorem}

\begin{dem}
If $p(n)$ is a stochastic vector, then its image:
	
is also a stochastic vector. Effectively, $p_i(n+1) \geq 0$ because:
	
is a sum of positive or zero values. Furthermore, we find:
	
		\begin{flushright}
			$\square$  Q.E.D.
		\end{flushright}
\end{dem}
This probability vector whose components are positive or zero, depends (it's pretty intuitive) on the transition matrix $P$ and the vector of initial probabilities $p(0)$.

Although it is provable (Perron-Frobenius theorem), the reader may verify by a practical case (computerized or not!) that if we choose any vector state $p(n)$ then there exists for any stochastic matrix $P$ a unique probability vector traditionally noted $\pi$ as:
	
Such a probability measure $\pi$ satisfying the above equation is named an "\NewTerm{invariant measure}\index{invariant measure}" or "\NewTerm{stationary measure}\index{stationary measure}" or "\NewTerm{balance measure}\index{balance measure}" which represents the equilibrium state of the system. In terms of linear algebra (see section of the same name) for the eigenvalue $1$, is an eigenvector of $P$ (\SeeChapter{see section Linear Algebra}).

We will see a trivial example in the Graph Theory section which will be redeveloped in detailed as in the section of Game and Decision Theory in the context of pharmaco-economics and in the section of Software Engineering when we will study the fundamentals of the Google Page Rank algorithm. But also note that the Markov chains are used for example in meteorology (or in the case of computer passwords hacks):
\begin{figure}[H]
\centering
\includegraphics{img/arithmetics/markov_chain_meteo.eps}
\caption{Concrete example of a very simple Markov chain}
\end{figure}
or in medicine, finance, transportation, marketing, etc.

In the field of language analysis, from the frequency analysis of a sequence of words, computers are able to also build Markov chains and therefore propose a more correct semantic during grammatical computerized corrections or in written transcription of oral presentations.

\textbf{Definitions (\#\mydef):}
	\begin{enumerate}
		\item[D1.] A Markov chain is said to be an "\NewTerm{irreducible Markov chain}\index{irreducible Markov chain}" if all states are bound to others (it's the case of the example in the figure above).

		\item[D2.] A Markov chain is said to be an "\NewTerm{absorbing Markov chain}\index{absorbing Markov chain}" if one of the states of the chain absorbs the transitions (so nothing comes out just to say things in a more simple way!).
	\end{enumerate}
	
	\begin{flushright}
	\begin{tabular}{l c}
	\circled{90} & \pbox{20cm}{\score{4}{5} \\ {\tiny 27 votes, 51.11\%}} 
	\end{tabular} 
	\end{flushright}
	
	%to make section start on odd page
	\newpage
	\thispagestyle{empty}
	\mbox{}
	\section{Statistics}

\lettrine[lines=4]{\color{BrickRed}S}tatistics is a science that concerns the systematic grouping of facts or recurring events that lend themselves to a numerical or qualitative assessment over time according to a given law. In the industry and the economy in general, statistics is a science that helps in an uncertain environment to make valid inferences.

You should know that among all areas of mathematics, the one that is used the most widely in business and research centres is Statistics and especially since softwares greatly facilitates the calculations! This is why this chapter is one of the biggest in this book even if only the basic concepts are presented!

Note also that Statistics have a very bad reputation at the university because the notations are often confusing and vary greatly from one teacher to another, from one book to another, from one practitioner to another. Strictly speaking, it should comply with the vocabulary and notation of the ISO 3534-1:2006 norm and unfortunately this chapter was written before the publication of this standard ... a certain period of adaptation will be necessary to obtain the full compliance.

It is perhaps useless to precise that Statistic is widely used in engineering, theoretical physics, fundamental physics, econometrics, project management and in the industry of process, in the fields of life and non-life insurance, in the actuarial science or in the database analysis (with Microsoft Excel very often ... unfortunately ....) and the list goes on. We will also meet quite often the tools presented here in the chapters of Fluid Mechanics, Thermodynamics, Technical Management, Industrial Engineering and Economy (especially in the last two). The reader can then refer to them to have some concrete practical applications of the most important theoretical elements that will be seen here.

Note also that in addition to a few simple examples on these pages, many other application examples are given on the exercise server of the companion website in the categories Probability and Statistics, Industrial Engineering, Econometrics and Management Techniques.

\textbf{Definition (\#\mydef):} The main purpose of Statistics is to determine the characteristics of a given population from the study of a part of the population, named "\NewTerm{sample}\index{sample}" or "\NewTerm{representative sample}\index{representative sample}". The determination of these characteristics should enable statistics to be a tool for the decision help!

	\begin{tcolorbox}[title=Remark,colframe=black,arc=10pt]
The data processing concerns the "\NewTerm{descriptive statistics}\index{descriptive statistics}". The interpretation of data from estimators is named "\NewTerm{statistical inference}" (or "\NewTerm{inferential statistics}\index{inferential statistics}"), and mass data analysis "\NewTerm{statistical frequency}\index{statistical family}" as opposed to Bayesian inference (\SeeChapter{see section Probabilities}).
	\end{tcolorbox}	

When we observe an event taking into account some given factors, there can happen that a second observation takes place in conditions that seem identical. By repeating these steps several times on different supposedly similar objects, we find that the observed results are statistically distributed around a mean value that is ultimately the most likely possible outcome. In practice, however, we sometimes perform a single measurement and then the goal is determine the value of the error we make by adopting it as average measure. This determination requires knowledge of the type of statistical distribution we are dealing with and that is on what we will focus (among others) to study here (at least the basics!). However, there are several common methodological approaches whe we face the hazard (less common are not mentioned yet):
	\begin{enumerate}
		\item A first is to simply ignore the random elements, for the simple reason that we do not know how to integrate them. We then use the "scenarios method" also named "deterministic simulation". This is typically the tool used by financial managers and non-graduates managers with tools like Microsoft Excel (which includes a scenarios management tool) or MS Project (which includes a tool to manage the deterministic optimistic, pessimistic and expected scenarios).
		\item A second possible approach, when we do not know how to associate probabilities to specific future random events, is game theory (\SeeChapter{see section Game and Decision Theory}) where semi-empirical criteria of selection are used as the criterion of maximax, minimax, Laplace, Savage, etc.
		\item Finally, when we can link probabilities to random events, whether these probabilities derived from calculations or measurements, whether they are based on experience from previous similar situations as the current situation, we can use descriptive and inferential statistics (contents of this chapter) to obtain usable and relevant information from this mass of acquired datas.
		\item A last approach when we know the relative probabilities from intervening events in response to strategic choices is the use of decision theory (\SeeChapter{see section Game and Decision Theory}).
	\end{enumerate}
	\begin{tcolorbox}[title=Remarks,colframe=black,arc=10pt]
\textbf{R1.} Without mathematical statistics, a calculation on datas (e.g. an average), is a "\NewTerm{punctual indicator}\index{punctual indicator}". This is mathematical statistics which gives it the status of estimator whose bias, uncertainty and other statistical characteristics are controlled. We generally seek to ensure that the estimator is unbiased, convergent and efficient (we will see during our study of estimators further what is exactly all that stuff).\\\\
\textbf{R2.} When we communicate a statistic it should be an obligation to specify the confidence interval, the $p$-value and the size of the studied sample (absolute statistics) and its detailed characteristics and make available the sources and data protocol otherwise it has almost no scientific value (we will see all these concepts in detail further below). A common mistake is to communicate in relative values. For example, on a test group of 1,000 women, 5 women will die from breast cancer without screening check, with screening check 4 women will die. Some will say a little to quickly (typically physicians....) that screening checks saves 20\% of women (relative value as one the of five could have been saved...). In fact this is wrong since the absolute benefit of screening is insignificant!\\\\
\textbf{R3.} If you have a teacher or trainer who dares to teach you statistics and probabilities only with examples based on gambling (cards, dice, matches, coins, etc.) dispose or denounce him. Normally examples should be based on the industry, economy or R\&D, i.e. in areas used in daily by businesses!).
	\end{tcolorbox}	
	
	\subsection{Samples}

During the statistical analysis of sets of information, the way to select the sample is as important as how to analyze it.The sample must be representative of the population (we do not necessarily make reference to human populations!). For this, the random sampling is the best way to achieve it.

\textbf{Definitions (\#\mydef):}
\begin{enumerate}
	\item[D1.] The statistician always starts from the observation of a finite number of elements, which we name the "\NewTerm{population}\index{population}". The observed elements, in quantity $n$, are all of the same nature, but this nature can be very different from one population to another.
	\item[D2.] We are in the presence of a "\NewTerm{quantitative character}\index{quantitative character}" when each observed element is explicitly subject to the same measure. To a given quantitative character, we associate a "\NewTerm{quantitative variable}\index{quantitative variable}" continuous or discrete, which summarizes all the possible values that the measure can take (such information being represented by functions like the Gauss-Laplace distribution, the beta distribution, the Poisson distribution, etc.).
	\begin{tcolorbox}[title=Remark,colframe=black,arc=10pt]
We will come back on the concept of "variable" and "distribution" a little further...
	\end{tcolorbox}	
	\item[D3.] We are in the presence of a "\NewTerm{qualitative character}\index{qualitative variable}" when each observed element is explicitly subject to a single connection to a "\NewTerm{modality}\index{modality}" from a set of exclusive modalities (e.g.: man $\mid$ woman) that permits to classify all studied elements in a given certain point of view (such information being represented by bar charts, sector charts, bubble charts, etc.). All modalities of a character can be established a priori before the survey (a list, a nomenclature, a code) or after the survey. A study population can be represented by a mixed character, or set of modalities such as gender, wage range, age, number of children, marital status for example for an individual.
	\item[D4.] A "\NewTerm{random sample}\index{random sample}" is by default (without more precision) a sample in which all individuals in a population have the same chance, or "\NewTerm{equally likely probability}\index{equally likely probability}" (and we emphasize that this probability must be equal), to end up in the sample.
	\item[D5.]  In the opposite in a sample whose elements were not chosen randomly, then we talk about a "\NewTerm{biased sample}\index{biased sample}" (in the opposite case we talk about a "\NewTerm{non-biased sample}\index{non-biases sample}").
	\begin{tcolorbox}[title=Remark,colframe=black,arc=10pt]
A small representative sample is by far preferable to a large biased sample. But when the sample sizes are small, the hazard can a result worst than the biased one...
	\end{tcolorbox}	
\end{enumerate}

	\subsection{Averages}
	
The concept of "\NewTerm{average}\index{average}" or "\NewTerm{central tendency}\index{central tendency}" (financial analysts call it a "measure of location"...) is with the notion of "variable" at the basis of statistics.

This notion seems very familiar to us and we talk a lot about it without asking too many questions. But there are various qualifiers (we emphasize that this are only qualifiers!) to distinguish the way of the resolution of a problem of calculating the average.

Thus, you must be very very careful about the calculations of averages because there is a tendency in business to rush and to systematically use the arithmetic mean without thinking, which can lead to serious errors! A nice example (for an analogy) is that a considerable number of laws require only moderate levels of pollution per year, while for example, smoking one cigarette per day during 365 days does not have the same impact as smoking 365 cigarettes in one day during one year when both have the same average taken over a year ... This is a clear evidence of statistical incompetence of the legislature.

Here is a small sample of common mistakes:

\begin{itemize}
	\item Consider that the arithmetic mean is the value that divides the population into two equal parts (although it is the median that does this).

	\item Consider that the average of the ratios of the type goals/realisations is equal to the ratio of the average of the goals and of the average of the realizations (although it is not the same thing!).

	\item Consider that the average salaries of different subsidiaries, is equal to the global average (while this is true if and only if there is the same number of employees in each subsidiary of the company).

	\item Consider that the average of the average of the rows in a table is always equal to the average of the columns of the same table (although this is true if and only if the cell contents are not empty).

	\item Calculate the arithmetic average growth of the revenue in \% (as the geometric mean must be used).

	\item etc.
\end{itemize}

We will see below different average with examples relative to arithmetic, to the enumeration, to physics, to econometrics, to geometry and sociology. The reader will find other practical examples by browsing the entire book.

\pagebreak
\textbf{Definitions (\#\mydef):} As given $x_i$ real numbers, then we have:
	\begin{enumerate}
		\item[D1.] The "\NewTerm{arithmetic average}\index{arithmetic average}" or "\NewTerm{sample average}\index{sample average}" (the most commonly known) is defined as the quotient of the sum of $n$ observed $x_i$ values by the total size $n$ of the sample:
			
	and is very often written $\bar{x}$ or $\widehat{\mu}$  and is for any discrete or continuous statistical distribution an unbiased estimator of the mean.

	The arithmetic average represents a statistical measure expressing the magnitude that would have each member of a set of measures if the total must be equal to the product of the arithmetic average by the number of members.

	If some values repeats more than once in the measurements, the arithmetic mean is then often formally noted as following:
	
	and is named "\NewTerm{weighted average}\index{weighted average}". Finally, we could indicate that under this approach, the actual weighted average will be named "\NewTerm{mathematical mean}\index{mathematical mean}" or just "\NewTerm{mean}\index{mean}" in the field of study of probabilities.

	We may as well use the frequencies of occurrence of the observed values named "\NewTerm{classes frequencies}\index{class frequencies}":
	
	So that we get another equivalent definition named the "\NewTerm{weighted average by the classes frequencies}\index{weighted average by the classes frequencies}":
	
	Before continuing, it's important to know that in the field of statistics it is useful and often necessary to combine measurements/data in class intervals of a given width (see examples below). We often have to make several tries to choose the intervals even if there are semi-empirical formulas for choosing the number of classes when we have $n$ available values. One of these semi-empirical rules used by many practitioners is to retain the smallest integer $k$ of classes such as:
	
	the width of the class interval is then obtained by dividing the range (difference between the maximum and minimum measured value) by $k$. By convention and rigorously... (so rarely respected in the notations), a class interval is closed on the left and open on the right (\SeeChapter{see section Functional Analysis}):
	
	This empirical rule is named the "\NewTerm{Sturges rule}\index{Sturges rule}" and is based on the following reasoning:

	We assume that the values of the binomial coefficient $C_k^i$ gives the number of individuals in an ideal histogram (we let the reader check this simply with a spreadsheet software like Microsoft Excel 11.8346 and the $\texttt{COMBIN(k,i)}$ function) of $k$ intervals for the $i$-th interval. As as $k$ becomes large the histogram looks more and more like a continuous curve named the "Normal curve" or "bell curve" as as we will see later.

	Therefore, based on the binomial theorem (\SeeChapter{see section Calculus}), we have:
	
Then, for each interval $i$ the practitioner will traditionally take the average between the lower and upper limit for the calculation and multiply it by the corresponding class frequency $f_i$. Therefore, the grouping of class frequencies implies that:
	\begin{enumerate}
		\item The weighted average by the frequencies differs from the arithmetic average.
		\item As the approximation seen above it will be a worst indicator compare to the arithmetic average...
		\item It is very sensitive to the choice of the number of classes, than very bad at this level.
	\end{enumerate}
There are many other empirical rules for the discretization of random variables. For example, the software XLStat offers not less than 10 rules (constant amplitude, Fisher algorithm, $K$-means, 20/80, etc.).

Later, we will see two very important properties of the arithmetic average and of the mean that you will have to understand absolutely (the weighted average of deviations from the average and the average deviations from the average).

	\begin{tcolorbox}[title=Remark,colframe=black,arc=10pt]
The "\NewTerm{mode}\index{mode}", noted $Mod$ or simply $M_0$, is defined as the value that appears most often in a set of data. In Microsoft Excel 11.8346, it is important to know that the $\texttt{MODE( )}$ function returns the first value in the order of values having the largest number of occurrences therefore assuming a unimodal distribution.
	\end{tcolorbox}	
		\item[D2.] The "\NewTerm{median}\index{median}" or "\NewTerm{middle value}\index{middle value}", noted $M_e$ is the value that cut the population values into two equal parts. In the case of a continuous statistical distribution $f(x)$ of a random variable $X$, it is the value that represents the value that has 50\% of cumulative probability to occur (we will see further in details the concept of statistical distribution):
		
In the case of a series of ordered values $x_1,x_2,...,x_i,...x_n$, the median is therefore by definition the value such that we have the same number of values that are greater than or equal to it than the number of values that are less or equal to it.
	\begin{tcolorbox}[title=Remarks,colframe=black,arc=10pt]
R1. The median is mainly used for skewed distributions, because it represent them better than the arithmetic average.\\\\
R2. The median is in practice often not a single value (at least in the case where $n$ is even). Indeed, between the values corresponding to ranges $\dfrac{n}{2}$ and $\dfrac{n}{2}+1$ there is an infinite number of values to choose which cut the population in half.
	\end{tcolorbox}	
More rigorously:
	\begin{itemize}
		\item If the number of terms is odd, i.e. of the form $2n + 1$, the median of the series is the term of order $n + 1$ (that the terms are all distinct or not!).
		\item If the number of terms is even, i.e. of the form $2n$, the median of the series is half-sum (arithmetic average) of the values of the terms of rank $n$ and $n + 1$ (that the terms are all distinct or not!).	
	\end{itemize}

In any case, by this definition, it follows that there are at least 50\% of the terms of the serie that are smaller than or equal to the median, and at least 50\% of the terms of the serie that are greater than or equal to the median.

For example, consider the table of wages below:

		

There is in the table an odd number $2n + 1$ of values. So the median of the series is the term of rank $n + 1$. This is $1,600.-$ (result that give any spreadsheet software). The arithmetic average is in this case about $2,020.-$.

In direct relation with the median it is important to define the following concept to understand the underlying mechanism:

\textbf{Definition (\#\mydef):} Be given a statistical series $x_1,x_2,...,x_i,...,x_n$, we name "\NewTerm{dispersion of absolute differences}\index{disperson of absolute differences}" around $x$ the number $\varepsilon '(x)$ defined by:
	
$\varepsilon '(x)$ is minimum for a value of $x$ closest to a given value $x_i$ in the sense of the absolute error value. The median is the value that achieves this minimum (extremum)! The idea will then be to study the variations of the function to find the position of this extremum.

Indeed, we can write:
	
Then by definition of the $x$ value:
	
What allows us to skip the absolute values is simply the choice of the index $r$ that is taken so that the serie of values in practice can always be split into two parts: all that is less than the element indexed by $r$ and all that is superior to it (i.e.. the median by anticipation...).

$\varepsilon '(x)$ is also a piecewise (discrete) affine function (similar to the equation of a line for fixed fixed values of $r$ and $n$) where we see that by analogy the factor:
	
is the slope of the function and:
	
the $Y$-intercept (ordinates at the origin).

The function is decreasing (negative slope) until $r$ is less than $\dfrac{n}{2}$ and increasing when r is greater than $\dfrac{n}{2}$ (it passes trough an extremum!). Specifically, we distinguish two particularly cases of interest since $n$ is an integer:

	\begin{itemize}
		\item If $n$ is even, we can say that $n=2n'$, then the slope can be written $2(r-n')$ and it is equal to zero if $r=n$ and then, as the result is valid by construction only for $\forall x \in \left[ x_r,x_{r+1}\right] $ then $\varepsilon '(x)$ is constant on $\left[ x_{n'},x_{n'+1}\right]$ and we have an extremum necessarily in the middle of this range (arithmetic average of the two terms).
		\item If $n$ is odd, we can say that $n=2n'+1$ (we cut the series into two equal parts), then the slope can be written $(2r-2n'-1)$ and it is zero if $r=n'+\dfrac{1}{2}$, as the result is only valid for $\forall x \in \left[ x_r,x_{r+1}\right] $ then it is immediate that the middle value is the median $x_{n'+1}$.
	\end{itemize}
We find out the median in both cases. We will also see later how the median is defined for a continuous random variable (the underlying idea is the same).

There is another practical case where the statistician has at its disposal only the values grouped in intervals of statistical classes. The procedure for determining the median is then different:

When we have at our disposal only a values grouped in intervals of statistical classes, the abscissa of the point of the median is generally within a class. To then get a more accurate value of the median, we perform a linear interpolation. This is what we name the "\NewTerm{linear interpolation method of the median}\index{linear interpolation method of the median}".

The median value can be read from the graph or calculated analytically. Indeed, consider the graph of the cumulative probability $F(x)$ in class intervals as below where the bounds of the intervals were connected by straight lines: 
\begin{figure}[H]
\centering
\includegraphics{img/arithmetics/median_linear_interpolation.eps}
\caption{Graphical representation of the estimation of the median by linear interpolation}
\end{figure}
The value of the median $M_e$ is obviously located at the crossroads between the cumulated probability of 50\% (0.5) and the abscissa. Thus, by applying the basics of functional analysis, we have (just by observing that the slope in the interval containing the median is equal in the half-interval to the left and to right adjacent to the median):
	
What we frequently write:
	
Thus the value of the median:
		
Consider the following table that we will see again much later in this chapter:

	
	We see that the "\NewTerm{median class}\index{median class}" is in the range $[150,200]$ because the cumulative value of $0.5$ is there (column at the right of the table) but the median has, using the previously established relation, the precise value of (it is trivial in the particular example of the table above, but we still do the calculation...):
	
and of course we can do the same with any other percentile!

We can also give a definition to determine the modal value if we are only in possession of the frequencies of class intervals. To see that we start with diagram below named "\NewTerm{grouped distribution}\index{grouped distribution}" in frequencies bar:
\begin{figure}[H]
\centering
\includegraphics{img/arithmetics/modal_value_class_interval.eps}
\caption{Graphical representation of the estimation of the modal value with classes intervals}
\end{figure}
Using Thales relations (\SeeChapter{see section Euclidean Geometry}), it comes immediately, noting $M$ the modal value:
	
As in a proportion, we do not change the value of the ratio by adding the numerators and adding the denominators, we get:
	
We then have:
	
With the previous example this gives then:
	
The question that then arises is to the appropriateness of the choice of the mean, mode or median in terms of communication ... (normally we communicate them all three in corporate reports!).

	A good example is that of the labor market where in general, while the average wage and the median wage are quite different, the institutions of state statistics calculate the median than many traditional media then explicitly equate to he concept of "arithmetic average" in their news...

	\begin{tcolorbox}[title=Remark,colframe=black,arc=10pt]
To avoid getting an arithmetic average having little sense, we often calculate a "\NewTerm{trimmed average}\index{trimmed average}", i.e. an arithmetic average calculated after removing outliers in the series (using Grubbs or Dixon Tests).
	\end{tcolorbox}
	The "\NewTerm{quantile}\index{quantile}" generalize the concept of "median" by cutting the distribution in sets of equal parts (of the same cardinality we might say ...) or in other words in regular intervals. We define the "quartiles," the "decile" and "percentile" on the population, ordered in ascending order, that we divide by 4, 10 or 100 parts of the same size.	

	So we talk about the 90th percentile to indicate the value separating the first $90\%$ of the population and the $10\%$ remaining.

	Note that in Microsoft Excel 11.8346 the functions \texttt{QUARTILE( )}, \texttt{PERCENTILE( )}, \texttt{MEDIAN( )}, \texttt{PERCENTRANK( )} are available and it can be useful that we specify that there are several variants of calculating these percentiles that explains possible variation between the results of different spreadsheet softwares.

	This concept is very important in the context of confidence intervals that we will see much further in this section and very useful in the field of quality with the use of "\NewTerm{box plots}\index{box plots}" (also named "\NewTerm{Box \& Whiskers plots"}\index{Box \& Whiskers plots}) to compare ("discriminate" as experts say) quickly two populations of data or more and especially to eliminate outliers (taking as reference the median will just make more sense!).
	
	\begin{figure}[H]
		\centering
		\includegraphics{img/arithmetics/box_plots.eps}
		\caption{Box \& Whiskers Plot painfully made with Microsoft Excel 11.8346}
	\end{figure}
	Or more explicitly as it should be in any good statiscal software:
	\begin{figure}[H]
		\centering
		\includegraphics[scale=0.8]{img/arithmetics/box_plot_template.jpg}
		\caption{Box \& Whiskers Plot ideal chart type}
	\end{figure}
	Another very important mental representation of box plots is the following (it can get an idea of the asymmetry of the distribution as is able to do the R software):
	\begin{figure}[H]
		\centering
		\includegraphics{img/arithmetics/median_mode_quartiles_symetric.eps}
	\end{figure}
	\begin{figure}[H]
		\centering
		\includegraphics[scale=0.75]{img/arithmetics/median_mode_quartiles_asymetric.eps}
		\caption{Graphical representation of the mode, median and 1st + 3rd quartile compared to a distribution}
	\end{figure}
The concepts of median, outliers and confidence intervals that have yet been proved and/or just defined are so significant that there exists international standards for their proper use. First let us cite the norm ISO 16269-7:2001 \textit{Median - Estimation and confidence intervals} and also the norm ISO 16269-4:2010 \textit{Detection and treatment of outliers}.

		A beautiful demonstration on how governments can make statistics lie (hopefully now most of them has to give the data to the population for free) is for example the USA Federal Reserve that pointed out the fact a growth in average income since the 2010. But when you have access to the raw data:
		\begin{center}
			\url{https://www.federalreserve.gov/econres/scfindex.htm}
		\end{center}
		and plot the median versus the average with for example Minitab 17 you get:
		\begin{figure}[H]
		\centering
		\includegraphics[scale=0.8]{img/arithmetics/median_vs_average.jpg}
	\end{figure}
		and after some analysis you can point out that the increase in the average is largely due to increases in the highest $10\%$ of incomes. The decline in the median at the same time suggests that typical Americans are in reality not doing as well... Q.E.D!
		
		\item[D3.] By analogy with the median, we define the "\NewTerm{medial}\index{medial}" as the value (in ascending order of values) that shares the (cumulative) sum values into two equal masses (i.e. the total sum divided by two).

In the case of our wages example, while the median gives the $50\%$ of the salaries being below and above the medial gives how many employees share (and therefore the sharing wage) the first half and how many employees share the second half of the total of the wages costs.

		

The sum of all wages is equal to $34,340$ and therefore the medial is $17,170$ then the medial is between the employees No. 11 and 12, while the median was $1,600$. We see then that the medial corresponds to $50\%$ of the aggregate. This is a very useful indicator in Pareto or Lorenz analysis (\SeeChapter{see section Quantitative Management}).
		\item[D4.] The "\NewTerm{root mean square}\index{root mean square}" sometimes denoted simply $Q$ which comes from the general relation:
	
but where we take $m=2$.
	\begin{tcolorbox}[colframe=black,colback=white,sharp corners]
\textbf{{\Large \ding{45}}Example:}\\\\
Consider a square of side $a$, and another square of side $b$. The average area of two squares equals one square of side:
	
	\end{tcolorbox}
In Microsoft Excel 11.8346 you can combine the functions \texttt{SUMSQ( )}, \texttt{COUNT( )} and to quickly calculate the root mean square as following:
\begin{center}
\texttt{=(SUMSQ(...)/COUNT(...))\string^(1/COUNT(...))}
\end{center}
	\item[D5.]  The "\NewTerm{harmonic mean}\index{harmonic mean}" sometimes simply denoted $H$ is defined by:
		
It is little known but is often the result of simple and relevant arguments (typically the equivalent resistance of an electrical circuit with several resistors in parallel). There is a \texttt{HARMEAN( )} function in Microsoft Excel 11.8346 to calculate it.
	\begin{tcolorbox}[colframe=black,colback=white,sharp corners]
\textbf{{\Large \ding{45}}Example:}\\\\
Consider a distance $d$ travelled in one direction at the speed $v_1$ and in the other  direction (or not) at the speed $v_2$. The arithmetic average speed will be obtained by dividing the total distance $2d$ by the time of the travel:
	
If we calculate the time it takes when travel $d$ with a speed $v_i$ that is simply the quotient:
	
The total time is therefore:
	
If the distance is not the same for the both velocities anyway each velocity remains the same this is why $d$ disappears!
	\end{tcolorbox}
In other words: We use the harmonic mean when are given to us ratios.
		\item[D6.] The "\NewTerm{geometric mean}\index{geometric mean}" sometimes simply denoted $G$ is defined in the general case by:
		
		This average is often forgotten by undergraduate employees but famous it is famous in the field of finance (\SeeChapter{see section Economy}) this is also why there is an \texttt{GEOMEAN( )} function in Microsoft Excel 11.8346 to calculate it.
		
		A geometric mean is often used when comparing different items – finding a single "figure of merit" for these items – when each item has multiple properties that have different numeric ranges. For example, the geometric mean can give a meaningful "average" to compare two companies which are each rated at $0$ to $5$ for their environmental sustainability, and are rated at $0$ to $100$ for their financial viability. If an arithmetic mean were used instead of a geometric mean, the financial viability is given more weight because its numeric range is larger so a small percentage change in the financial rating (e.g. going from $80$ to $90$) makes a much larger difference in the arithmetic mean than a large percentage change in environmental sustainability (e.g. going from $2$ to $5$). The use of a geometric mean "normalizes" the ranges being averaged, so that no range dominates the weighting, and a given percentage change in any of the properties has the same effect on the geometric mean. So, a $20\%$ change in environmental sustainability from $4$ to $4.8$ has the same effect on the geometric mean as a $20\%$ change in financial viability from $60$ to $72$.
		
		\begin{tcolorbox}[title=Remark,colframe=black,arc=10pt]
		In 2010, the geometric mean was introduced to compute the Human Development Index by United Nations Development Programme. Poor performance in any dimension is directly reflected in the geometric mean. That is to say, a low achievement in one dimension is not anymore linearly compensated for by high achievement in another dimension. The geometric mean reduces the level of substitutability between dimensions and at the same time ensures that a $1\%$  decline in index of, say, life expectancy has the same impact on the HDI as a $1\%$ decline in education or income index. Thus, as a basis for comparisons of achievements, this method is also more respectful of the intrinsic differences across the dimensions than a simple average.
		\end{tcolorbox}
		
		Like for the number $0$, it is impossible to calculate the geometric mean with negative numbers. However, there are several work-arounds for this problem, all of which require that the negative values be converted or transformed to a meaningful positive equivalent value. Most often this problem arises when it is desired to calculate the geometric mean of a percent change in a population or a financial return, which includes negative numbers.

		For example, to calculate the geometric mean of the values $+12\%, -8\%, 0\%$ and +$2\%$, instead calculate the geometric mean of their decimal multiplier equivalents of $1.12, 0.92, 1$ and $1.02$, to compute a geometric mean of $1.0125$. Subtracting $1$ from this value gives the geometric mean of $+1.25\%$ as a net rate of growth (or in financial circles is named the "\NewTerm{Compound Annual Growth Rate C.A.G.R.}\index{compound annual growth rate}").
		\begin{tcolorbox}[colframe=black,colback=white,sharp corners]
		\textbf{{\Large \ding{45}}Example:}\\\\
		Suppose that a bank offers an investment opportunity and plans for the first year an interest (this is absurd, but this is an example) with a rate $(X-Y)\%$ but for the second year with an interest rate $(X+Y)\%$. At the same time another bank provides a constant interest rate for two years: $X\%$. We will say a little bit to fast that this is the same... In fact the two investments do not have the same profitability!\\
	
		In the first bank, a capital $C_0$ will give after the first year of interest:
		
		and the second year:
		
		In the other bank we will have after one year:
		
		and after the second year:
		
		and so on...\\
		As you can probably see it the placement will not be identical if $Y\neq 0$! $X\%$ is the not the arithmetic average of $(X-Y)\%$ and $(X+Y)\%$.

		Now if we write:
		
		What is in reality the average value of the global interest rate $r$?
	
		After $2$ years (for example), the capital is multiplied by $r_1 \cdot r_2$. If an average exists it will be denoted by $r$ and the capital will thus be multiplied by $r^2$. Then we have the relation:
		
		This is an example where we therefore see the geometric mean. Forgetting to use of the geometric mean a common mistake in companies where some employees calculate the arithmetic average rate of increase of a reference value.
		\end{tcolorbox}
		\item[D7.] The "\NewTerm{moving average}\index{moving average}" of order $n$ is defined as:
		
	The moving average is used particularly in economics, where it represents a trend of a series of values, where the number of points is equal to the total number of points of the serie of values less the number that you specify for the period.

	A moving average in finance is calculated from the average of a stock price over a given period: each point of a moving average of 100 sessions is the average of 100 last current values. This curve, displayed simultaneously with the evolution of the curve of the values, smooths the daily changes in the value and gives the possibility to better see the trends.

	The moving averages can be calculated for different time periods, which can generate short-term trends MMC (20 sessions according to the habits of the domain), medium (50-100 sessions) or long-term MML (over 100 sessions):

	\begin{figure}[H]
		\centering
		\includegraphics[scale=0.75]{img/analysis/time_serie.eps}
		\caption{Graphical representation of a few moving averages}
	\end{figure}

	The crosses of the moving averages with the price curve (cutted with a certain granularity) of the value generate purchase or sale (basic) signals depending on the case:
	\begin{itemize}
		\item Buy signal: when the price curve crosses the MM up.
		\item Sell signal: when the price curve crosses the MM down.
	\end{itemize}
	In addition to the moving average, note that there are a lot of other artificial indicators often used in finance (the R software has a package dedicated only to such indicators). As for example the "\NewTerm{upside/downside ratio}\index{upside/downside ratio}".

The idea is the following: If you have a financial product (\SeeChapter{see section Economy}) whose current price is $P_c$ for which you have a goal of high gain corresponding to a high price, which we will denote by $P_h$ (high price) and conversely, the potential loss that you feel is at a price $P_l$ (low price).
	

	\begin{tcolorbox}[colframe=black,colback=white,sharp corners]
	\textbf{{\Large \ding{45}}Examples:}\\\\
	E1. For example, a financial product of $10.-$ with a low price of $5.-$ and a high price of $15.-$ has a ratio of $\text{UD}_R=1$ and therefore an identical speculative factor to allow a gain or loss of $5.-$.\\
	
	E2. A financial product of $10.-$ with a low price of $5.-$ and a high price of $20.-$ has a ratio of $\text{UD}_R=2$ and therefore twice the speculative potential gain compared to the loss.
	\end{tcolorbox}
	\begin{tcolorbox}[title=Remark,colframe=black,arc=10pt]
	Some financial institutions recommend to refuse equation below $3$. Investors also tend to reject too high equation that can be a sign of artificial inflation.
	\end{tcolorbox}	
		\item[D8.] The "\NewTerm{weighted average}\index{weighted average}" (the moving average and arithmetic average are just a special cases of the weighted average with $w_i=1$) is defined by:
		
Is used for example in geometry to locate the centroid of a polygon, in physics to determine the center of gravity or in statistics to calculate the mean and other advanced regression techniques and in project management for estimating task durations forecast.

In the general case the weights $w_i$ represents the weighted influence or arbitrary/empirical influence of the element $x_i$ relatively to the others one.
		\item[D9.] The "\NewTerm{functional mean}\index{functional mean}" or "\NewTerm{integral average}\index{integral average}" is defined as:
		
where $\mu_f$ depends of a function $f$ of a real integrated variable (\SeeChapter{see Differential and Integral Calculus}) on a range $[a, b]$. It is often used in signal theory (electronics, electrotechnichs).
	\end{enumerate}
	
	\pagebreak
	\subsubsection{Laplace Smoothing}

To come back to our class frequencies seen above and before proceeding with the study of some mathematical properties of the averages... you should know that when we work with discrete probabilities distributions it happens very (very) often that we meet a typical problem whose source is the size of the population.

Consider as an example the case where we have $12$ documents and that we would like estimate the probability of occurrence of the word "Viagra". We have on a sample the following values:

	

Table that we can represent in another way:

	
	
And here we have a common phenomenon. There is no record with $5$ occurrences of the word of interest. The idea (very common in the field of Data Mining) is then to add artificially and empirically using a count using a technique named "\NewTerm{Laplace smoothing}\index{Laplace smoothing}" which involves adding k units at each occurrence. Therefore the table becomes:

	
	
Obviously this type of technique is debatable and beyond the scientific framework ... We even hesitated to introduce this technique in the chapter of Numerical Methods (with the rest of all the empirical numerical techniques) rather than here...

	\subsubsection{Means and Averages properties}
	
	Now we will see some relevant properties that connect some of these means and averages or are specific to a particular mean/average.

The first properties are important so beware to understand them:

	\begin{enumerate}
		\item[P1.] The calculation of the arithmetic, root mean square and harmonic average/mean can be generalized using the following expression:
			
		where we see:
			\begin{enumerate}
				\item For $m=1$, we get the arithmetic average
				\item For $m=2$, we get the root mean square
				\item For $m=-1$ we get the harmonic mean
			\end{enumerate}
		\item[P2.] The arithmetic average has the property of linearity, that is to say (without proof because it is simple to check):
			
			This is the statistical version of the property of the mean in the field of probabilities that we will see further.
			\item[P3.] The weighted sum of the deviations from the arithmetic average is zero.
				\begin{dem}
					First, by definition, we know that:
						
					then we have:
						\thickmuskip=0mu
						\medmuskip=0mu
						
						\thickmuskip=3mu
						\medmuskip=3mu
					Thus, this tool can not be used as a measure of dispersion!

					By extension, the arithmetic average of the weighted deviations from the average is also equal to zero:
					
					\begin{flushright}
						$\square$  Q.E.D.
					\end{flushright}
				\end{dem}
				This result is quite important because it will further be useful for a better understanding of the concept of standard deviation and variance.
			\item[P4.] Now we would like to prove that:
				
					\begin{tcolorbox}[title=Remark,colframe=black,arc=10pt]
The comparisons between the above means/averages and the median or the weighted or moving averages does not make sense this is why we won't compare them.
	\end{tcolorbox}
				\begin{dem}
					First, we consider two nonzero real numbers $x_1$ and $x_2$ as $x_2>x_1>0$ and then we write:
					\begin{enumerate}
						\item The arithmetic average:
							
						\item The geometric mean:
							
						\item The harmonic mean:
							
						\item The root mean square:
										
					\end{enumerate}
					We will start to prove that $\mu_g>\mu_h$ by contradiction by putting $\mu_g-\mu_h<0$:
					
					By convenience we will now put:
						
					and we know that $y>1$. We therefore have:
						
						and remember we search if it is possible that:
						
						 We can now easily check this statement from the following equivalences:
						 
						There is also a contradiction, and this validates our initial hypothesis:
							
						Let see if $\mu_g>\mu_a$.
						
						Under the hypothesis $x_2>x_1>0$. We search now to prove that:
							
						Now we have the following equivalences:
							
						and the last expression is obviously correct because the square of a (real) number is always positive which verifies our initial hypothesis:
							
						We will prove now that $\mu_q > mu_a$ by contradiction by putting $\mu_q-\mu_a<0$:
						
						But the square of a (real) number is always positive which verifies our initial hypothesis:
						
						We then have:
						
						\begin{flushright}
							$\square$  Q.E.D.
						\end{flushright}
				\end{dem}
				Once these inequalities proved, we can then move on to a figure that we attribute to Archimedes to place three of these averages. The interest of this example is to show that there are some remarkable relations between statistics and geometry (coincidence??).
				
				\begin{figure}[H]
					\centering
					\includegraphics[scale=0.75]{img/arithmetics/averages.eps}
					\caption{Starting point for the geometric representation for the various averages}
				\end{figure}
				We will first write $a=\overline{AB},b=\overline{BC}$ and $\text{O}$ is the midpoint of $\overline{AC}$. Thus, the circle is drawn with center $\text{O}$ and radius $\overline{\text{O}A}$. $D$ is the intersection of the perpendicular segment $\overline{AC}$ through $B$ and of the circle $\Omega$ (we can choose the intersection we want). $H$ is itself the orthogonal projection of $B$ on $\overline{OD}$.
				
				Archimedes says that $\overline{\text{O}A}$ is the arithmetic average of $a$ and $b$ and that $\overline{BD}$ is the geometric mean of $a$ and $b$, and $\overline{DH}$ is the harmonic mean of $a$ and $b$.
				
				We then prove that (could be trivial):
					
Therefore $\overline{\text{O}A}$ is the arithmetic average $\mu_a$ of $a$ and $b$.

We have in the right-angled triangle $ADB$:
	
Then we have in the right-angled triangle $BDC$:
	
	We then add these two relations, and we get:
	
	We know that $D$ is on a circle of diameter $\overline{AC}$, so $ADC$ is rectangle on $D$. Therefore:
	
	And then we replace $\overline{BA}$ and $\overline{BC}$ by $a$ and $b$:
		
	So finally:
		
	And therefore, $\overline{DB}$  is the geometric mean $\mu_g$ of $a$ and $b$.
	We have now prove that $\overline{DH}$ is the harmonic mean of $a$ and $b$. We have in a first time using the orthogonal projection as study in the section of Vector Calculus:
		
		Then we also have (also orthogonal projection)
		
		Therefore we have:
		
		and since ${DB}=\sqrt{ab}$, we have then:
			
		$\overline{DH}$ is therefore the harmonic mean of $a$ and $b.$ Archimedes was not wrong!
	\end{enumerate} 
	
	\paragraph{Jensen Inequality}\mbox{}\\\\
	The "\NewTerm{Jensen's inequality}\index{Jensen's inequality}" is a particularly important relation (or property) in finance and insurance because it shows by the proof why an options seller will take the expected mean of of pay-off rather than pay-off of the expect mean or why ultimately the insured will always pay more or equal than the differential of the expected mean of accident costs, and also in time series analysis why GARCH processes are naturally leptokurtic.

	\begin{theorem}
	Jensen's inequality will allow us to prove that for a convex function:
	
	with $p_i>0$, $\sum_i p_i=1$. Either by using notations specific to the domain of statistics:
	
	and to the valuation of options (\SeeChapter{see section Economy}):
	
	\end{theorem}
	Before proceeding further, let us recall that a function $f$ is concave if $-f$ is convex and vice versa. It is then immediate that:
	
	Either by using notations specific to the domain of statistics:
	
	Let's go to the proof and then we will make a small practical simplified example with derivatives in finance.
	\begin{dem}
	Let us see if: 
	
	is true by proceeding by induction. First for $m=2$, we fall back on the classical definition (\SeeChapter{see section Functional Analysis}) of a convex function:
	
	Now, we assume the previous relation true for $m=n$, and we show that it is also true for $m=n+1$:
	
	Since is $f$ is convex, we can write:
	
	Let us put:
	
	for $i=1\ldots n$. We have $q_i>0$ for all $i=1\ldots n$ and:
	
	because for recall:
	
	and consequently:
	
	By hypothesis of recurrence we then have:
	
	And therefore:
	
	by simplifying the expression on the right we get indeed:
	
	\begin{flushright}
		$\square$  Q.E.D.
	\end{flushright}
	\end{dem}
	\begin{tcolorbox}[colframe=black,colback=white,sharp corners]
	\textbf{{\Large \ding{45}}Example:}\\\\
	We know that the pay-offs of a Call and a Put are respectively at maturity (from the point of view of a buyer and therefore by symmetry from the point of view of a seller also in absolute value):
	
	Let us consider a price at maturity of $102$.- and that the possible strikes at maturity are $\{100,110,150\}$, we have then have:
	
	and:
	
	and so we have well for a Call from the point of view of a buyer or seller in absolute values:
	
	\end{tcolorbox}

	\pagebreak
	\subsection{Type of variables}
In talking about variables quantitative or qualitative variables, sometimes you hear variables being described as categorical (or sometimes nominal), or ordinal, or interval.  Below we will define these terms and explain why they are important.

\textbf{Definitions (\#\mydef):}
	\begin{enumerate}
		\item[D1.] The "\NewTerm{discrete variables}\index{discrete variables}" (by counting) that belongs to $\mathbb{Z}$: Are analyzed with statistical laws based on a countable definition domain always strictly positive (the Poisson or Hypergeoemetric distribution are such typical case in the industry). Are almost always represented graphically by histograms.

	\item[D2.] The "\NewTerm{continuous variables}\index{continuous variable}" (by measure) that belong to $\mathbb{R}$: Are analyzed with statistical laws based on an uncountable domain of definition strictly positive or may take any positive or negative value (typically the Normal distribution in the industry). Are almost always represented graphically by histograms with class intervals.

	\item[D3.] The "\NewTerm{attribute variables}\index{attribute variable}" (by classification): They are not digital data (only when they are coded with digits!) but qualitative data type {Yes, No}, {Passed, Failed}, {On time, Late}, {red, green blue, black}, etc. The binary data type attribute follow a Bernoulli while higher order qualitative variables have no average or standard deviation (effectively... try to calculate the mean and standard deviation between the qualitative variables Red, Green and Pink...).
	
	In attribute variable we mainly distinct two subtypes of variables:
		\begin{enumerate}
			\item A "\NewTerm{categorical variable}\index{categorical variable}" (sometimes named a \NewTerm{nominal variable}\index{nominal variable}) is one that has two or more categories, but there is no intrinsic ordering to the categories.  For example, gender is a categorical variable having two categories (male and female) and there is no intrinsic ordering to the categories.  Hair color is also a categorical variable having a number of categories (blonde, brown, brunette, red, etc.) and again, there is no agreed way to order these from highest to lowest.  A purely categorical variable is one that simply allows you to assign categories but you cannot clearly order the variables.  If the variable has a clear ordering, then that variable would be an ordinal variable, as described below.
			\item An "\NewTerm{ordinal variable}\index{ordinal variable}" is similar to a categorical variable.  The difference between the two is that there is a clear ordering of the variables.  For example, suppose you have a variable, economic status, with three categories (low, medium and high).  In addition to being able to classify people into these three categories, you can order the categories as low, medium and high. Now consider a variable like educational experience (with values such as elementary school graduate, high school graduate, some college and college graduate). These also can be ordered as elementary school, high school, some college, and college graduate.
		\end{enumerate}
	\end{enumerate}
	
	Understanding the different types of data is an important discipline for the engineers because it has important implications for the type of analysis tools and techniques that will be used.

A common question regarding the collection of data is what is the amount that should be collected. In fact it depends on the desired level of accuracy. We will see much further in this section (with proof!) how to mathematically determine the amount of data to collect.

Now that we are relatively familiar with the concept of average (mean), we can discuss on more formal calculations and that will make sense.

	\subsubsection{Discrete Variables and Moments}
	
	Consider $X$ is an independent variable (an individual of a sample, whose property is independent of other individuals) that can take discrete random values (realizations of the vector $(X_1,X_2,...,X_n)$) with respective probabilities $(p_1,p_2,...,p_n)$ where, by the axioms of probabilities (\SeeChapter{see section Probabilities}):
		
	
	\textbf{Definitions (\#\mydef):}
	\begin{enumerate}
		\item[D1.] Let $X$ be a numeric (quantitative) random variable (r.v.). It will be fully described in practice most of time by the value of the probability (for discrete variables) for a realization of this variable or by the cumulative probability (for discrete AND continuous variables) to be typically less than or equal $X$ for all realizations $x$. This cumulative probability is denoted by:
		
		with:
		
		where $F(x)$ is named the "\NewTerm{repartition function}\index{repartition function}" of the random variable $X$. It is the theoretical proportion of the population whose value is less than or equal to $x$. It follows for example:
		
		More generally, for any two numbers $a$ and $b$ with $a<b$, we have:
		
		
		\item[D2.] The "\NewTerm{empirical repartition function}\index{empirical repartition function}" is naturally defined by (we have indicated the different notations that you can found in the literature):
		
		associated with the sample of independent and identically distributed variables which as we know is named a "random vector" denoted by $(x_1,x_2,...,x_n)$.
		
		It is simply the cumulative frequencies of appearance normalized to unity below a certain fixed value (approach that the majority of human beings are naturally using when seeking the repartition function).

		So if we take again the example of wages already used above, then we have for example for x fixed to 1,800:
		
		And then:
		
		The repartition function is clearly a monotonically increasing function (or more precisely "non-decreasing") whose values range from $0$ to $1$.
	\end{enumerate}

\paragraph{Mean and Deviation of Discrete Random Variables}\mbox{}\\\\
\textbf{Definition (\#\mydef):} We define the "\NewTerm{expectation}\index{expectation}" or "\NewTerm{mean}\index{mean}", also named "\NewTerm{moment of order $1$}\index{moment of order $1$}", of the random variable $X$ by the relation (with various notations):
	
also sometimes named "\NewTerm{parts rule}\index{parts rule}".

	In other words, we know that for every event in the sample space is associated with a probability that we also associate with a value (given by the random variable). The question then is know what value we can get at long term? The expected value (the mean...) is then the weighted average, by the probability, of all values of the events of sample space.

	If the probability is given by a discrete distribution function $f(x_i)$ (see the definitions of distribution functions later below in the text) of the random variable, we then have: 
	

	\begin{tcolorbox}[title=Remark,colframe=black,arc=10pt]
\textbf{R1.} The mean $\mu_X$ can also be written just simply $\mu$ if there are no possible confusion on the random variable.\\\\
\textbf{R2.} If we consider each realization of the random variables $(x_1,x_2,...,x_n)$ as the components of a vector $\vec{x}$ and each associated probability (or ponderation) $(p_1,p_2,...,p_n)$ as the components of a vector $\vec{p}$ we can write the mean in a technical way using the scalar product (\SeeChapter{Vector Calculus}) often written:
	
	\end{tcolorbox}	
Here are the most important mathematical properties of the mean for any random variable (whatever the distribution law!) and that we will often use throughout this section (and many other involving statistics):
	\begin{enumerate}
		\item[P1.] Multiplication by a constant (homogeneous):
			
		\item[P2.] Sum of two random variables (independant or not!):
			
			Where we used in the 4th line, the property view in the section of Probabilities:
			
			We deduce that for $n$ random variables $X_i$ following any probability distribution:
			
			\item[P3.] Then mean of a constant $a$ is equal to the constant itself:
				
			\item[P4.] Mean of a product of two random variables:
				
				And if the two random variables are independent, then the probability is equal to the joint probability (\SeeChapter{see section Probabilities}). Therefore we have:
				
	\end{enumerate}
So the mean of the product of independent random variables is always equal to the product of their means.

We will assume as obvious that these four properties extend to the continuous case!

\textbf{Definition (\#\mydef):} After having translated the trend by the mean it is interesting to have and indicator that reflects the dispersion or "\NewTerm{standard deviation}\index{standard deviation}" around the mean by a value named "\NewTerm{variance of $X$}\index{variance}" or "\NewTerm{second-order centered moment}\index{second-order centered moment}" or "\NewTerm{mean square error (MSE)}\index{mean square error}, written $V(X)$ or $\sigma_X^2$ (read "sigma square") and given in its discrete form by:
	
The variance is however not directly comparable to the mean because of the fact that the units of the variance are the square of the unit of the random variable, which follows directly from its definition. To have an indicator of dispersion that can be compared to the parameters of central tendency (mean, median and ... mode), it then suffices to take the square root of the variance.

For convenience, we define the "\NewTerm{standard deviation}" of $X$ by:
	

	\begin{tcolorbox}[title=Remark,colframe=black,arc=10pt]
\textbf{R1.} The standard deviation $\sigma_X$ of the random variable $X$ can be written simple $\sigma$ if there is no possible confusion.\\\\
\textbf{R2.} The standard deviation and variance are, in the literature, often named "\NewTerm{dispersion parameters}\index{dispersion parameters}" as opposed to the mean, mode and median that are named "\NewTerm{positional parameters}\index{propositional parameters}". 
	\end{tcolorbox}	

\textbf{Definition (\#\mydef):} The ratio (expressed in \%):
	
is often used in business to compare the mean and the standard deviation and is named the "\NewTerm{coefficient of variation C.V.}\index{coefficient of variation}" because it has no units (which is it's main advantage!) and because many industrial statistical methods consider that a good C.V should ideally be just about a few \% only.

	More generally for any statistics estimator $\hat{\theta}$ (sum, average, median, etc.) we can build a coefficient of variation such that:
	

Thus, in practice we consider that:

		

Why do we find a square (respectively a square root) in the definition of the variance? The intuitive reason is simple (the rigorous much less ...). Remember, that we have shown above that the sum of the deviations from the actual weighted average is always zero:
	
If we assimilate the size of each sample by the probability by normalizing the sample size with respect to $n$, we come upon a relation that is the same as the variance with the difference that the term in brackets is not squared. And then we immediately see the problem... the dispersion measure is always zero, hence the need to bring this to the square.

We could, however, imagine to use the absolute value of deviations from the mean, but for a number of reasons that we will see later during our study of estimators, the choice of squaring is quite natural.

Note, however, the common use in the industry of two common other indicators of dispersion:
	\begin{enumerate}
		\item "The \NewTerm{mean absolute deviation}\index{mean absolute deviation}" (mean of the absolute values of deviations from the mean):
			
			Which is a very elementary indicator used when we do not want to make statistical inference on a series of measures. This deviation can be easily calculated in the English version of Microsoft Excel 11.8346 using the \texttt{AVEDEV( )} function.
		\item The "\NewTerm{median absolute deviation}\index{median absolute deviation}" denoted MAD (median of absolute values of deviations from the median):
			
			which is considered as a more robust measure of dispersion than those given by the mean absolute deviation or the standard deviation (unfortunately this indicator is not natively integrated in spreadsheets softwares).
	\end{enumerate}

	\begin{tcolorbox}[colframe=black,colback=white,sharp corners]
\textbf{{\Large \ding{45}}Example:}\\\\
Consider the following measure of a random variable $X$:
	
	and where the median value is given as we know by:
	
	The absolute deviations from the median are then:
	
	Placed in ascending order, we then have:
	
	where we easily identify the absolute deviation from the median, which is:
	
	\end{tcolorbox}
	
	In the case where we have at disposition a series of measures, we can estimate the experimental value of the mean (expectation) and of the variance with the following estimators (it is simply the of average and standard deviation of a sample when the events are equally likely) with the specific notation:
		
		\begin{dem}
			First for the mean:
			
			And for the variance:
			
			\begin{flushright}
				$\square$  Q.E.D.
			\end{flushright}
		\end{dem}
		\begin{theorem}
			Let us prove now a very nice little property as the arithmetic average is an optimum for the sum of squared errors.
		\end{theorem}
		\begin{dem}
			
			And if we search for $\alpha$ as the derivative of the above expression is equal to zero:
			
			then $\alpha$ is an optimum. We have therefore:
			
			or after rearrangement and an elementary simplification we get:
						
			\begin{flushright}
				$\square$  Q.E.D.
			\end{flushright}		
		\end{dem}
		It is effectively the arithmetic average! Now to see if it is an maximum extrema or minimum extrema we just need calculate the second derivative (\SeeChapter{see section Differential and Integral Calculus}) and see if it gives a positive constant (i.e. the first derivative increases when $\alpha$ increase). Therefore we immediately see that it is effectively a minimum extrema!!!
		
	\begin{tcolorbox}[title=Remark,colframe=black,arc=10pt]
The term of the sum that we see in the expression of the variance (standard deviation) is named the "\NewTerm{sum of squared deviations from the mean}\index{sum of squared deviations from the mean}" or "\NewTerm{sum of squared errors from the mean}\index{sum of squared errors from the mean}". We also name it the "\NewTerm{total sum of squares}\index{total sum of squares}", or "\NewTerm{total variation}\index{total variation}", or "\NewTerm{sum of square errors}\index{sum of square errors}" in the context of the study of the ANOVA (see the further below)
	\end{tcolorbox}	
	Before that we continue, let us recall the concept of geometric mean seen above (widely used for returns in finance or growth analyzes in \% of sales):
	
	It's fine but employees in financial departments also need to calculate the standard deviation of this average. The idea is then to take the logarithm to reduce it to a simple arithmetic mean (it is still obviously an estimator!):
	
	Therefore, since taking the logarithm of the values we have the arithmetic mean of the log values, then the logarithm of the geometric standard deviation (with physicist reasoning like...) will be:
	
	Then we just take the exponential of the standard deviation of the logarithms of the values to have the "\NewTerm{geometric standard deviation}\index{geometric standard deviation}":
	
	The variance can also be written in the very important way named the "\NewTerm{Huygens relation}\index{Huygens relation}" or "\NewTerm{König-Huygens theorem}\index{König-Huygens theorem}" or "\NewTerm{Steiner translation theorem}\index{Steiner translation theorem}" that we will reuse several times thereafter. Let's see what it is:
	
	Let us now do a relatively small hook to a common scenario generator of errors in business when several statistical series are handled (very common case in the industry as well as in insurance or finance)!
	
	Consider two data series on the same character:
	\begin{itemize}
		\item $(x_1,n_1),(x_2,n_2),...,(x_p,n_p)$ sample of total size $n$, arithmetic average $\bar{x}$, standard deviation $\sigma_x$.
		\item $(y_1,m_1),(y_2,m_2),...,(y_p,m_q)$ sample of total size $m$, arithmetic average $\bar{y}$, standard deviation $\sigma_y$.
	\end{itemize}
	We then have:
		
		So the average of the averages is not equal to the overall average (first common mistake in business) except if the two data series have the same sample size ($n=m$)!!!
		
		Let have a look at the standard deviation always with the same situation. First remember that we have:
		
		
		
		To continue, recall that we have previously proved the Huygens theorem and therefore:
		
		Therefore we have:
		
		And we continue on the next page...:
		\pagebreak
		
So we see that the overall standard deviation is not equal to the sum of the deviations (second common mistake in business) unless the sample sizes and arithmetic averages are the same in both series (that is to say $n=m$ and $\bar{x}=\bar{y}$)!!!

Consider now $X$ being a random variable of mean $\mu$ (constant and determined value) and variance $\sigma^2$ (constant and determined value), we define the "\NewTerm{reduced centered variable}" by the relation:
	
	\begin{theorem}
		We prove in a very simple way by using the property of linearity of the mean and property of scalar multiplication of the variance that:
		
	\end{theorem}
	\begin{dem}
		For the proof we just use the definitions of the expected mean and variance (using Huygens theorem for this latter). So let us begin with the mean:
		
		And now with the variance using the Huygens theorem:
		
		\begin{flushright}
			$\square$  Q.E.D.
		\end{flushright}
	\end{dem}
	
	Thus, any statistical distribution defined by a mean and standard deviation can be transformed into another distribution often easier to analyze statistical. Therefore making this transformation, we obtain a random variable for which the parameters of the distribution low are now useless to know. When we do that with other laws, and in the general case, when we speak of "\NewTerm{pivotal variables}\index{pivot variables}".

Here are some important mathematical properties of the variance:
	\begin{enumerate}
		\item[P1.] Multiplication by a constant:
		
		\item[P2.] Sum of two random variables:
		
		Where we meet for the first time the concept of "\NewTerm{covariance}\index{covariance}" denoted by $\text{cov}()$.
		\item[P4.] Product of two random variables (using the Huyghens theorem):
		
		And if the two random variables are independent, we get:
		
		What we can rewrite using once again the Huygens theorem:
		
	\end{enumerate}
	We will assume as obvious that these four properties extend to the continuous case!
	
	\pagebreak
	\paragraph{Discrete Covariance}\mbox{}\\\\
	We have seen in on of the last equations the concept of "\NewTerm{covariance}" for which we will determine a more convenient expression later:
	
		
	We introduce now a more general and very important expression of the covariance in many application fields:
	
	Now we change the notation to simplify even more:
	
	Therefore in the general case:
	
	Or using standard deviation:
	
	Using the properties of the mean (especially $\text{E}(X)=c^{te}$ and $\text{E}(c^{te})=c^{te}$) we can write the covariance in a much simpler way for calculation purposes:
	
	and we obtain the relation widely used in statistics and finance in the practice named the "\NewTerm{covariance formula}"...:
	
	which is however best known when written as:
	
	
	If $X=Y$ (equivalent to a univariate covariance) we fall back again on the Huyghens theorem:
	

	\begin{tcolorbox}[title=Remark,colframe=black,arc=10pt]
	Statistics can be partitioned according to the number of random variables we study. Thus, when a single random variable is studied, we speak of "\NewTerm{univariate statistics}\index{univariate statistics}", for two random variables of "\NewTerm{bivariate statistics}\index{bivariate statistics}" and in general, of "\NewTerm{multivariate statistics}\index{multivariate statistics}". 
	\end{tcolorbox}	
	
	If and only if the variables are equally likely, we find the covariance in the literature in the following form, sometimes named "\NewTerm{Pearson covariance}\index{Pearson covariance}", which derives from calculations that we have done previously with the mean:
	
	Covariance is a measure of the simultaneous variation of $X$ and $Y$. Indeed, if $X$ and $Y$ generally grow simultaneously, the products $(y_i-\mu_Y)(x_i-\mu_X)$ will be positive (positively correlated), whereas if $Y$ decreases as $X$ increases, these same products will be negative (negative correlation).
	Note that if we distribute the terms of the last equation, we have:
	
	and we have already shown that the sum of the deviations from the mean is zero. Hence we get another common way to write the covariance:
	
	and by symmetry:
	
	So in the end, in the equiprobable case, we finally have the equivalent three important relations used in various sections of this book:
	\begin{equation}
  	\addtolength{\fboxsep}{5pt}
   \boxed{
   \begin{gathered}
		\begin{aligned}
			\text{c}_{X,Y}&=\text{cov}(X,Y)=\dfrac{1}{n}\sum_{i=1}^n(y_i-\bar{y})(x_i-\bar{x})\\
			\text{c}_{X,Y}&=\text{cov}(X,Y)=\dfrac{1}{n}\sum_{i=1}^nx_iy_i-\bar{x}\bar{y}\\
			\text{c}_{X,Y}&=\text{cov}(X,Y)=\dfrac{1}{n}\sum_{i=1}^n x_i(y_i-\bar{y})=\dfrac{1}{n}\sum_{i=1}^n y_i(x_i-\bar{x})
		\end{aligned}
   \end{gathered}
   }
	\end{equation}
	In the section Theoretical Computing for the study of linear regression and factor analysis we will need the explicit expression of the bilinearity property of the variance. To see what it is exactly, consider three random variables $X, Y$ and $Z$ and two constants $a$ and $b$. Then using the third relation given above, we have:
	
	The last relation is also important and will be used in several sections of this book (Economy, Numerical Methods). It also allows us to directly obtain the covariance for the sums of various random variables.
	\begin{tcolorbox}[colframe=black,colback=white,sharp corners]
	\textbf{{\Large \ding{45}}Example:}\\\\
If $X, Y, Z, T$ are four random variables defined on the same population, we want to compute the following covariance:
	
	We will develop that in two phases (this is also why we call that "bilinearity property"). First with respect to the second argument (random choice!):
	
	And then with respect to the first:
	
	So in the end:
	
	\end{tcolorbox}
	Now, consider a set of random vectors $\vec{X}_i:=X_i$ of components $(x_1,x_2,...,x_n)_i$. The calculation of the covariance of the components by pairs gives what is named the "\NewTerm{covariance matrix}\index{covariance matrix}" (a tool widely used in finance, management and statistical numerical methods!).

	Indeed, we define the component $(m,n)$ of the covariance matrix by:
	
	We can therefore write a symmetric matrix (usually in practice it must be a square matrix...) in the form:
	
	where $\Sigma$ is the usual tradition letter to denote the covariance matrix.
	By symmetry and because it is a square $n$ by $n$ matrix only the number $\dfrac{n(n+1)}{2}$ of components is useful for us to determine the whole matrix (trivial but important information for when we will study the structural equation modeling in the Numerical Methods section).
	
	This matrix has the remarkable property that if we take the set of all random vectors and we calculate the covariance matrix, then the diagonal will give us obviously the variances of each pair of vectors (see examples in the chapters Economics, Numerical Methods or Industrial Engineering) because we have for recall:
	
	 This is why this matrix is often named "variance-covariance matrices" and finds itself sometimes also written as follows:
	
	And this is a little bit abusively sometimes written as:
	
	This matrix has the advantage of quickly showing what pairs of random variables have a negative covariance and there... for which random variable the variance of the sum is smaller than the sum of the variances!
	
	\begin{tcolorbox}[title=Remark,colframe=black,arc=10pt]
	As we already mention it, this matrix is very important and we will often see it again in the section Economy during our study of modern portfolio theory and also for data mining techniques in the section of Theoretical Computing (principal compoments analysis for example but not only!) and also in Industrial Engineering during our study of bivariate control charts.
	\end{tcolorbox}	
	
	Recall now that we have an axiom in probability (\SeeChapter{see section Probabilities}) which stated that two events $A$ and $B$ are independent if and only if:
	
	Similarly, by extension, we define the independence of discrete random variables.
	
\textbf{Definition (\#\mydef):} Let $X, Y$ be two discrete random variables. We say that $X, Y$ are independent if and only if:
	
	More generally, the discrete variables $X_1,X_2,...,X_n$ are independent (in block) if:
	
	\begin{theorem}
		The independence of two random variables implies that their covariance is zero (the opposite is false!). 
	\end{theorem}
	\begin{dem}
		We will prove this in the case where the random variables take only a finite number of values $\left\lbrace x_i \right\rbrace_I$ and $\left\lbrace y_j \right\rbrace_J$ , respectively, with $I, J$ finite sets.
		
		For the proof let us recall that:
		
		and therefore:
		
	\begin{tcolorbox}[title=Remark,colframe=black,arc=10pt]
So small is the covariance (near to zero), more the series are independent. Conversely, the greater the covariance (in absolute value) higher the series are dependant.
	\end{tcolorbox}	
	Given that:
	
	and the fact that if $X$ and $Y$ are independent we have $c_{X,Y}=0$. Then:
	
	More generally if $X_1,...,X_n$ are independent (in block) then for any discrete or continuous statistical distribution law (!) we have using the two most common notations:
	
	Or using the standard deviation:
	
	\begin{flushright}
		$\square$  Q.E.D.
	\end{flushright}
	\end{dem}
	
	\pagebreak
	\subparagraph{Anscombe's famous quartet}\mbox{}\\\\
	Anscombe's quartet comprises four datasets that have nearly identical elementary statistical properties, yet appear very different when graphed or analyzed with undergraduate statistics rather than high-school one. Each dataset consists of eleven $(x,y)$ points. They were constructed in 1973 by the statistician Francis Anscombe to demonstrate both the importance of graphing data before analyzing it and the effect of outliers on statistical properties. This quartet is also used to test if an analytical tool can be accepted a "statistics-compliant" (as the six corresponding used statistics should be the minimum provided by any high-school level analytical tool!).
	
	The datasets are as follows. The $x$ values are the same for the first three datasets:
	
	The quartet is still often used to illustrate the importance of looking at a set of data graphically before starting to analyze according to a particular type of relation, and the inadequacy of basic statistic properties for describing realistic datasets.
	
	With Microsoft Excel 14.0.7166 we get:
	\begin{figure}[H]
		\begin{center}
			\includegraphics[scale=0.85]{img/arithmetics/anscombe_quartet.jpg}
		\end{center}	
		\caption{Anscombe's quartet Statistics Summary}
	\end{figure}
	As we can see with elementary statistical indicators it is almost impossible to guess a difference between the four data sets. But if we use the skewness or the kurtosis this change everything!
	
	Looking to the corresponding charts we get the same conclusion:
	\begin{figure}[H]
		\begin{center}
			\includegraphics[scale=0.8]{img/arithmetics/anscombe_quartet_chart.jpg}
		\end{center}	
		\caption{Anscombe's quartet Graphs Summary}
	\end{figure}

	
	\paragraph{Mean and Variance of the Average}\mbox{}\\\\
	Often in statistics, it is (verrrrry!) useful to determine the standard deviation of the sample mean and to work with it to get important analytical results in management and manufacturing. Let's see what it is!
	
	Given the average of a series of terms, each determined by the measurement of several values (it is in fact its estimator in a particular case as we will see later):
	
	then using the properties of the mean:
	
	and if all the random variables are independent and identically distributed then we have:
	
	\begin{tcolorbox}[title=Remark,colframe=black,arc=10pt]
We will prove much further below that if all the random variables are independent and identically distributed with finite variance, then the mean follows asymptotically what we name a "Normal distribution".
	\end{tcolorbox}
	For the variance, the same reasoning applies:
	
	And if the random variables are independent and identically distributed (we will study further the very important case current in practice where the last condition is not satisfied):
	
	Then we get the "\NewTerm{standard deviation of the mean}\index{standard deviation of the mean}" also named "\NewTerm{standard error}\index{standard error}" or "\NewTerm{non-systematic variation}\index{non-systematic variation}":
	
	and this is strictly the standard deviation of the estimator of mean!
	
	The more intuitive form to express the Standard Error in terms of percent for non-analytical workers, managers and chief executives is named "\NewTerm{Relative Standard Error (RSE)}\index{relative standard error}" which is the expression of the Standard Error as percent, that is:
	
	The latter is quite useful when we have to deal with many variables with different units!!
	
	The value of $\sigma_{\bar{X}}$ is available in many softwares including Microsoft Excel charts (but there is no built-in function in Microsoft Excel) and is written with the standard deviation (as above) or with the notation of the variance (then we only have to take the square root...).

	Note that the last relation can be used even if the average of n random variables is not the same! The main condition is just that the standard deviations are all equal and this is the case in the industry (production).

	We then have:
	\begin{equation}
  	\addtolength{\fboxsep}{5pt}
   	\boxed{
   	\begin{gathered}
   		\begin{aligned}
		\text{E}(S_n)&=n\mu & \text{E}(M_n)&=\mu\\
		\text{V}(S_n)&=\sigma_{S_n}^2=n\sigma^2 & \text{V}(M_n)&=\sigma_{M_n}^2=\dfrac{\sigma^2}{n}
   		\end{aligned}
   	\end{gathered}
   	}
	\end{equation}
	where $S_n$ is the sum of $n$ independent identically distributed random variables and $M_n$ their estimated average.
	
	The reduced centered variable that we introduced earlier:
	
	can then be written in several very useful ways:
	
	Furthermore, assuming that the reader already knows what is a Normal distribution $\mathcal{N}(\mu,\sigma)$, we will show later in detail because it is extremely important (!) that the probability of the random variable $\bar{X}$, average of $n$ identically distributed and linearly independent random variables, has for law (obviously):
	
	
	\pagebreak
	\paragraph{Coefficient of Correlation}\mbox{}\\\\
	Now consider $X$ and $Y$ two random variables having for covariance:
	
	\begin{theorem}
	We have:
	
	We will prove this relation immediately because the use of the covariance alone for data analysis is not always great because it is not strictly limited and easy to use (at interpretation). We will construct an indicator easier to use in business.
	\end{theorem}
	\begin{dem}
		We choose any constant $a$ and we calculate the variance of:
		
		We can then immediately write using the properties of the variance and the of the mean:
		
		The right quantity is positive or null for any $a$ by construction of the variance (left). So the discriminant of the expression, seen as a polynomial in $a$ is of the type:
		
		Because $P(a)$ is positive for any $a$ we have as only possibility that:
		
		Therefore after simplification:
		
		\begin{flushright}
			$\square$  Q.E.D.
		\end{flushright}
	\end{dem}
	This gives us also:
	
	Finally we get some a statistical inequality named "\NewTerm{Cauchy-Schwarz inequality}\index{Cauchy-Schwarz inequality}":
	
	If the variances of $X$ and $Y$ are non-zero, the correlation between $X$ and $Y$ is defined by the "\NewTerm{linear correlation coefficient}\index{linear correlation coefficient}" (it is a standardized covariance so that its amplitude does not depend on the chosen unit measure) and written:
	
	Which can also be written in an expanded form (using Huyghens theorem):
	
	or more condensed:
	
	\begin{tcolorbox}[title=Remark,colframe=black,arc=10pt]
Note that normally, the letter $R$ is reserved to say that this is an estimator of the correlation coefficient but the definition above is not an estimator (the variances doesn't have the small hat...) and that, strictly speaking, we should then write $\rho_{X,Y}$ according to the traditions of use.
	\end{tcolorbox}	
	
	Whatever the units and the orders of magnitude, the correlation coefficient is a number between $-1$ and $1$ without units (so its value does not depend on the unit of measure, which is by far not the case for all statistical indicators!). It reflects more or less the linear dependence of $X$ and $Y$ or geometrically more or less the flattness magnitude. We can therefore say that a coefficient of correlation of zero or close to $0$ correlation means that there is no linear relation between the characters. But it does not involve any notion of more general independence.	
	
	When the correlation coefficient is near $1$ or $-1$, the characters are said to be strongly correlated. We must be careful with the frequent confusion between correlation and causality. Thus, two phenomena that are correlated does not imply in any way that one is the cause of the other (this fallacy is also known as "\NewTerm{spurious correlation}" or "\NewTerm{cum hoc ergo propter hoc}", latin for "with this, therefore because of this," and "false cause")!!!!
	
	Indeed, for any two correlated events, $A$ and $B$, the different possible relationships include:
	\begin{itemize}
		\item $A$ causes $B$ (direct causation);
		\item $B$ causes $A$ (reverse causation);
		\item $A$ and $B$ are consequences of a common cause, but do not cause each other;
		\item $A$ causes $B$ and $B$ causes $A$ (bidirectional or cyclic causation);
		\item $A$ causes $C$ which causes $B$ (indirect causation);
		\item There is no connection between $A$ and $B$;
		\item The correlation is a coincidence.
	\end{itemize}

	Coming back to the mathematical aspect of the correlation:
	\begin{itemize}
		\item If $R_{X,Y}=-1$ we are dealing with a "\NewTerm{pure negative correlation}\index{pure negative correlation}" (in the case of a linear relation all measurement points are located on a straight line with a negative slope).
		
		\item If $-1<R_{X,Y}<1$ we are dealing with a negative or positive correlation named "\NewTerm{imperfect correlation}\index{imperfect correlation}" (in the case of a linear relation all measurement points are located on a straight positive or negative slope respectively).
		
		 \item If $R_{X,Y}=0$ the correlation is zero... (in the case of a linear relation all the measurement points are located on a straight line of slope zero).
		 
		 \item If $R_{X,Y}=1$ we are dealing with a "\NewTerm{pure positive correlation}\index{pure positive correlation}" (in the case of a linear relation all measurement points are located on a straight positive slope).
	\end{itemize}
	
	The analysis of the correlation coefficient has the objective of determining the degree of association between variables: it is often expressed as the coefficient of determination, which is the square of the correlation coefficient. The coefficient of determination thus measures the contribution of a variable to the explanation of the second. However, asking \textit{how high should  $R$ (or $R^2$) be?} doesn't make sense. A low $R$ doesn't negate a signficiant predictor or change the meaning of its coefficient. $R$ is simply whatever value it is, and it doesn't need to be any particular value to allow for a valid interpretation. 

Using the expressions of mean and standard deviation of equiprobable variables as demonstrated above (thus the idea of computing the correlation of two random variables is a good idea if they are jointly gaussian), we start:
	
	To obtain the estimator of the coefficient of correlation
	
	where we see that the covariance becomes the average of the products minus the product of averages.
	
	Thus after simplification we get a famous expression:
	
	The correlation coefficient can be calculated in the English version of Microsoft Excel 11.8346 and others with the integrated \texttt{CORREL( )} function.
	
	We will see in the section Theoretical Computing a more general expression of the correlation coefficient.
	
		\begin{tcolorbox}[title=Remarks,colframe=black,arc=10pt]
\textbf{R1.} In the literature, the experimental correlation coefficient is often named "\NewTerm{sampling Pearson coefficient}\index{sampling Pearson coefficient}" (in the equiprobable case) and when we carry it to the square, then we name it the "\NewTerm{coefficient of determination}\index{coefficient of determination}".\\

\textbf{R2.} Often the square of the coefficient is somewhat improperly interpreted as the \% of variation explained in the response variable $Y$ by the explanatory variable $X$. 
	\end{tcolorbox}
	
	Finally, note that we have the following relation which is used a lot in practice (see the section Economics for famous detailed examples!):
	
	or the version with the standard deviation even more famous:
	
	It is a relation that we can often see in finance in the calculation of the VaR (Value at Risk) according to RiskMetrics methodology proposed by JP Morgan (see section Economy).
	
	Let us see a small application example of the correlation but that has nothing to do with VaR (at least for the moment...).
	
	\begin{tcolorbox}[colframe=black,colback=white,sharp corners]
\textbf{{\Large \ding{45}}Example:}\\\\
An airline company has $120$ seats available that she reserves for connecting passengers from two flights arrived earlier in the journey and that have to go to Frankfurt. The first flight arrived from Manila and the number of passengers on board follows a Normal distribution with mean 50 and variance $169$. The second flight arrives in Taipei and the number of passengers on board follows a Normal distribution with mean $45$ and variance $196$.\\

The linear correlation coefficient between the number of passengers of both flights was measured as:
	
	The law that follows the number of passengers for Frankfurt if we assume that the law of the couple also follows a Normal distribution (according to statement!) is:
	
	with:
	
	
	The law that follows the number of passengers for Frankfurt if we assume that the law of the couple also follows a Normal distribution (according to statement!) is:
	
	with:
	
	This is a bad start for customer satisfaction in the long term...
	\end{tcolorbox}
	
	\pagebreak
	\subsubsection{Continuous Variables and Moments}
	
	\textbf{Definitions (\#\mydef):}
	\begin{enumerate}
		\item We say that $X$ is a continuous variable if its "\NewTerm{cumulative distribution function C.D.F.}\index{cumultative distribution function}" is continuous (already defined above). The distribution function of $X$ is defined by for $x \in \mathbb{R}$ or a truncated subset of $\mathbb{R}$:
		
		that is the cumulative probability that the random variable $X$ is smaller than or equal to the set value $x$. We also have of course:
		
		
		\item We denote by:
		
		the "\NewTerm{survival function}\index{survival function}" or "\NewTerm{tail function}\index{tail function}".
		
		\item If furthermore the distribution function $F$ of $X$ is continuously differentiable of derivative $f$ (or sometimes denoted by $\rho$) named "\NewTerm{density function}\index{density function}" or "\NewTerm{mass function}\index{mass function}" or just simply "\NewTerm{distribution function}\index{distribution function}" then we say that $X$ is absolutely continuous and in this case we have:
		
		with the normalization condition:
		
		Any probability distribution function must satisfy the integral of normalization in its domain of definition!
	\end{enumerate}
	\begin{tcolorbox}[title=Remark,colframe=black,arc=10pt]
It is interesting to note that the definition implies that the probability that a completely continuous random variable takes a given value tends to zero! So it is not because an event has almost a zero probability that it can not happen!!!
	\end{tcolorbox}	
	The average being defined by a sum weighted by probabilities for a discrete variable, it becomes an integral for a continuous variable:
	
	and therefore the variance is written as:
	
	Then we have also the median that is logically redefined in the case of a continuous random variable by:
	
	and it rarely coincides with the average!
	
	And the modal value is given by the value of $x$ where:
	
	
	Statisticians often use the following notations for the expected mean of a continuous variable:
	
	and for the variance:
	
	That is the same as for the moment of discrete variable.

	Thereafter, we will calculate these different moments indicators with detailed proofs only for the most used cases.
	
	\subsection{Fundamental postulate of statistics}
	
	One of the ultimate goals of statistics is, starting from a sample, to find the analytical distribution function that gave birth to the sample. This goal will be presented on this web site as a postulate (although this assumption is very difficult to apply in practice).
	
	Postulate: For any empirical distribution function $\hat{F}_n(x)$ of the $n$-th measurement of the $x$ random variable we can associate a theoretical distribution function $F(x)$ to which it converges when the sample size is large enough if:
	
	is the random variable defined as the largest difference (in absolute value) between $\hat{F}_n(x)$ and $F(x)$ (observed for all values of $x$ for a given sample), then $X_n$ converges to $0$ almost surely.
	
	\begin{tcolorbox}[title=Remark,colframe=black,arc=10pt]
Mathematicians of Statistics prove this postulate rigorously as a theorem named the "\NewTerm{fundamental theorem of statistics}\index{fundamental theorem of statistics}" or the "\NewTerm{Glivenko-Cantelli theorem}\index{Glivenko-Cantelli theorem}" regarding continuous functions. Personally, even if we offends the experts, we think that this proof is not one because because it is very far away from the practical reality (yes this is our physicist side that emerges...) and this theoretical result leads many practitioners do their utmost (excluding data, transformations and other abominations) to find a known distribution law that they can adjust to their measured data.
	\end{tcolorbox}	
	
	\subsection{Diversity Index}
	
	It happens in the field of biology or business that you it is asked to a statistician or analyst to measure the diversity of a number of predefined elements. For example, imagine a multinational with a range of well-defined products and some of the stores (customers) in the world can choose a subset of this range for their business sales. The request is then to make a ranking of stores that sell the widest range of branded products and that by taking also into account the quantity.

	For example, we have a list of a total 4 products in our catalog. By hazard, three of our customers sell our 4 products but we would like to know which customers sells the greatest diversity and this by taking into account the quantities.

	We have the following sales data by product for the customer $1$:
	
	
	For the customer $2$:
	
	and for the customer $3$:
	
	A measure of information (diversity of states) that is well suited to this purpose is the Shannon formula introduced in the section of Statistical Mechanics whose mean:
	
	Arbitrarily, we will take and the logarithm in base $10$ (so, if we have $10$ equiprobable variables, entropy is unitary for example...).
	
	Therefore we have:
	
	
	We will rewrite this more adequately for the application in business. Thus, if $n$ is the number of products and $p_i$ the proportion (or "relative frequency") of sales of product $i$ from all sales $N$ then:
	
	Then we have:
	
	This gives for the customer 1 (we stay in base $10$ for the logarithm):
	
	which is the maximum possible value (each state is equally likely). And for customer 2 we have:
	
	And finally for customer 3:
	
	Thus, the customer that has the greatest diversity is the first one. We also see an interesting property of the Shannon formula with customer 2 and 3 and this is that the quantity does not affect diversity (since the only difference between the two customers is that the quantity is multiplied by a factor of 2 and not diversity)!
	
	Notice that if we put all the product in only one customer (set), we get:
	
	Therefore we guess that (without general proof):
	
	Hence the fact that it is better to have various product in one place than in multiple place (result used for example in the field of document management where the purpose is to avoid to have the same type of documents saved in different share drive or libraries).
	
	\pagebreak
	\subsection{Distribution Functions (probabilities laws)}
	When we observe probabilistic phenomena, and we take note of the values taken by them and that we report them graphically, we can observe that the individual measurements follow a typical characteristic which is sometimes adjustable theoretically with a good level of quality.

In the field of probabilities and statistics, we call these characteristics "\NewTerm{distribution functions}" because they indicate the frequency with which the random variable appears for given values.

	\begin{tcolorbox}[title=Remark,colframe=black,arc=10pt]
We sometimes simply use the term "function" or "law" to describe these characteristics.
	\end{tcolorbox}	
	
	These functions are in practice bounded by what we name the "\NewTerm{range of the distribution}\index{range of a distribution}" which is the difference between the maximum value (on the right) and the minimum value (on the left) of the observed values:
	
	In theory they are not necessarily bounded and then we talk (\SeeChapter{see section Functional Analysis}) about a "\NewTerm{domain of definition}" or more simply about the "\NewTerm{support}\index{support}" of the function.

If the observed values are distributed in a certain way then there is a probability (or "cumulative probability" in the case of continuous distribution functions) to have a certain value of the distribution function.

In industrial practice (\SeeChapter{see section Industrial Engineering}), the range of statistical values is important (as well as the standard deviation) because it gives an indication of the variation of a process (variability).

If $L$ denote any possible univariate distribution function the range of the function is simply denoted by $L$ if its domain of definition is $\mathbb{R}$  otherwise if it is bounded you will typically see something like $L_{]a,b]}$.

\textbf{Definitions (\#\mydef):}
	\begin{enumerate}
		\item[D1.] The mathematical relation that gives the probability of a given value of the distribution function a random variable is named the "\NewTerm{density function}\index{density function}" (or "\NewTerm{probability density function}\index{probability density function}"), "\NewTerm{mass function}\index{mass function}" or "\NewTerm{marginal function}\index{marginal function}".
		
		\item[D2.] The mathematical relation that gives the cumulative probability that a random variable to be lower than or equal to a certain value of the distribution function is referred to as the "\NewTerm{repartition function}\index{repartition function}" or "\NewTerm{cumulative function}\index{cumulative function}" or "\NewTerm{cumulative distribution function}\index{cumulative distribution function}".
		
		\item[D3.] Random variables are "\NewTerm{independent and identically distributed (i.i.d.)}\index{idependent and identically distributed variables}" if they all follow the same distribution function, with the same parameters values and that they are independent.
	\end{enumerate}
	Such functions are very numerous, we offer then here to the reader a detailed study of the most known only.

Before going any further it could be useful to know that if $X$ is a continuous or discrete random variable, then are several tradition of notation in the literature to indicate that it follows a given probability distribution $L$. Here are the most common:
	
	In this section and throughout the book in general, we will use the last notation!

Here is the list of the distribution functions that we will see here as well as distribution functions commonly used in the industry and located in other chapters/section and those whose proof has yet still to be written:

	\begin{itemize}[noitemsep,nolistsep]
		\item Discrete Uniform Distribution $\mathcal{U}(a,b)$ (see below)
		\item Bernoulli Distribution $\text{B}(1,p)$ (see below)
		\item Geometric Distribution $\mathcal{G}(N)$ (see below)
		\item Binomial Distribution $\mathcal{B}(N,k)$ (see below)
		\item Binomial Negative Distribution $\text{NB}(N,k,p)$ (see below)
		\item Hypergeometric Distribution $\mathcal{H}(n,p,m,k)$ (see below)
		\item Multinomial Distribution (see below)
		\item Poisson Distribution $\mathcal{P}(\mu,k)$ (see below)
		\item Gauss-Laplace/Normal Distribution $\mathcal{N}(\mu,\sigma)$ (see below)
		\item Log-Normal Distribution $\mathcal{LN}(\mu,\sigma)$ (see below)
		\item Continuous Uniform Distribution (see below)
		\item Triangular Distribution (see below)
		\item Pareto Distribution (see below)
		\item Exponential Distribution (see below)
		\item Weibull Distribution (see section Industrial Engineering) 
		\item Generalized Exponential Distribution (see section Theoretical Computing)
		\item Erlang/Erlang-B/Erlang-C Distributions (see section Quantitative Management)
		\item Cauchy Distribution (see below)
		\item Beta Distribution (below and section Quantitative Management)
		\item Gamma Distribution (see below)
		\item Chi-2 Distribution (see below)
		\item Student Distribution (see below)
		\item Fisher-Snedecor Distribution (see below)
		\item Benford Distribution (see below)
		\item Logistic Distribution (see section Theoretical Computing)	
		\item Square Gauss distribution (still must be written)
		\item Extreme value distribution (still must be written)
	\end{itemize}

	\begin{tcolorbox}[title=Remark,colframe=black,arc=10pt]
The reader will find the mathematical developments of the Weibull distribution function in the section on Industrial Engineering (Engineering chapter), and the logistic distribution function in the section of Theoretical Computing.
	\end{tcolorbox}	
	
	\subsubsection{Discrete Uniform Distribution}
	
	If we accept that it is possible to associate a probability to an event, we can conceive of situations where we can assume a priori that all elementary events are equally likely (that is to say, they have the same probability to occur). We then use the ratio between the number of favorable cases and the number of possible cases to calculate the probability of all events in the Universe of events $U$. More generally, if $U$ is a finite set of equally likely events and $A$ is part of $U$, then we have using set theory notation (\SeeChapter{see section Set Theory}):
	 
	 More commonly, if $e$ is an event that may have $N$ equally likely possible outcomes. Then the probability of observing the outcome of this given event follows a "\NewTerm{discrete uniform function}\index{discrete uniform function}" (or "\NewTerm{discrete uniform law}\index{discrete uniform law}") given by the relation:
	 
	 Whose mean (or average) is given by:
	  
	 If we put ourselves in the particular case where $x_i=1$ with $i=1...N$. We then have (\SeeChapter{see Sequences And Series}):
	 
	 If the random variable $e$ take all values between $[a,b]$ (another special case) such the distribution will be now denoted by $\mathcal{U}(a,b)$ then it should be obvious that we have for the expected mean:
	 
	 
	 For the variance we have (always using the results of the section on Sequences and Series):
	 
	 
	 If the random variable $e$ take all values between $[a,b]$ (another special case) such the distribution will be now denoted by $\mathcal{U}(a,b)$ then it should be obvious that we have for the variance:
	 
	 
	 By symmetry of the distribution if all values of the domain of definition $[a,b]$ are taken by the random variable we have for the median:
	 
	 
	 Here is an plot example of the mass distribution function and cumulative distribution function respectively for discrete uniform law of parameters $\left\lbrace 1,5,8,11,12\right\rbrace$ (we see that each value is equally likely):
	 
	\begin{figure}[H]
		\begin{center}
			\includegraphics[scale=0.75]{img/arithmetics/law_uniform.jpg}
		\end{center}	
		\caption{Uniform law $\mathcal{U}$ (density and cumulative distribution function)}
	\end{figure}
	
	As we can see in the above diagram the cumulative distribution function can be written:
	
 
	 \begin{tcolorbox}[title=Remark,colframe=black,arc=10pt]
For sure the discrete uniform distribution has no specific modal value $M_0$!
	\end{tcolorbox}	
	

	\subsubsection{Bernoulli Distribution}
	
	If we are dealing with a binary observation then the probability of an event is constant from one observation to the other if there is no memory effect (in other words: a sum of Bernoulli variables, two by two independent).
	
	We name this kind of observations where the randoms variables takes the values $0$ (false) or $1$ (true), with probability $q=1-p$ respectively $p$, "\NewTerm{Bernoulli trials}\index{Bernoulli trials}" with "\NewTerm{contrary events with contrary probabilities}".
	
	Thus, a random variable $X$ follows a "\NewTerm{Bernoulli function}\index{Bernoulli function}" $\text{B}(1,p)$ (or "\NewTerm{Bernoulli law}\index{Bernoulli law}") if it can take only the values $0$ or $1$ , associated with probabilities $p$ and $q$ and so that $q+p=1$ and:
	
	The classic example of such a process is the game of piece face or sampling with replacement or be considered as such (this last case is very important in industrial practice). There certainly is no need for the reader to formally verify that the cumulative probability is unitary...
	
	\begin{tcolorbox}[title=Remark,colframe=black,arc=10pt]
The introduction above is perhaps not relevant for business, but we will see in the section of Quantitative Techniques that the Bernoulli function naturally appears at the beginning of our study of queuing theory.
	\end{tcolorbox}	
	
	Note that, by extension, if we consider $N$ events where we get in a particular order $k$ times one possible outcomes (success) and the other $N-k$ (fail) times, then the probability of such a series ($k$ successes and $N-k$ failures ordered in any particular way) is given by :
	
	with $N \in \mathbb{N}^{*}$ according to what we got during the study of combinatorics in the section of Probabilities!

Here is an example plot of the cumulative distribution function for $q=0.3$:

	\begin{figure}[H]
		\begin{center}
			\includegraphics[scale=0.75]{img/arithmetics/law_bernoulli.jpg}
		\end{center}	
		\caption{Bernoulli law $\text{B}$ (cumulative distribution function)}
	\end{figure}
	The Bernoulli function has therefore for expected mean (average) choosing $p$ as the probability of the event of interest:
	
	and for variance (we use the Huygens theorem proved above):
	
	
	The modal value $M_0$ of the Bernoulli law depends on the values of $p$ or $q$. So we have (it could be obvious for the reader):
	
	
	\begin{tcolorbox}[title=Remark,colframe=black,arc=10pt]
For sure the Bernoulli distribution has no specific median value $M_e$!
	\end{tcolorbox}

	
	\subsubsection{Geometric Distribution}
	The geometric law $\mathcal{G}(N)$ or "\NewTerm{Pascal's law}\index{Pascal's law}" consist in a Bernoulli trial, where the probability of success is $p$ and that of failure $q=1-p$ are constant, that we renew independently until the first success.
	
	Remember that during our presentation of the Bernoulli law we have deduce an extension to $N$ such that:
	
	Therefore the probability to get the first success $k=1$ after $N$ trials is:
	
	with $N \in \mathbb{N}^{*}$.
	
	As you can see, greater is $N$, smaller is the probability $\mathcal{G}(N)$. This can be seem non-logic but in fact it is! Indeed in the sentence "\textit{the probability to get the first success after $N$ trials}", you must not forget that it is written \underline{after} and not \underline{during}.
	
	Therefore for sure... the probability to have $N-1$ failures followed by $1$ success will be always be smaller when $N$ increase (have a look the figure a little bet further below for $p=0.5$ can help to understand).
	
	This law has for expected mean:
	
	However, the last relation can also be written:
	
	Indeed, we proved in the section of Sequences and Series during our study of geometric series that:
	
	Taking the limit $n\rightarrow +\infty$ when we get:
	
	because $0\leq q < 1$. 
	Then we just derivate both members of equality with respect to $q$ and we get:
	
	This done let us continue...
	
	We have then the average number of trials $X$ it takes to get the first success (or in other words, the expected rank - number of expected trials - to see the first success):
	
	
	Now we calculate the variance and reminding once again (Huygens theorem):
	
	So let's start by calculating $\text{E}(X^2)$:
	
	The last term of this expression is equivalent to the expected mean calculated previously. Thus:
	
	It remains to calculate:
	
	We have:
	
	But deriving the following equality:
	
	We get:
	
	Therefore:
	
	Thus:
	
	Finally when it comes to ranking the expected variance of the first success (i.e.: the variance expected number before the first successful trials):
	
	The modal value is easy to get because we need to find the value of $N$ that maximize the definition of the geometric law:
	
	and we hope that it is immediate to the reader that this is satisfy when $N=1$ therefore:
	
	Now let us determine the median $M_e$ to finish. For this, by definition we know we must have:
	
	But we can rewrite:
	
	Therefore (in base $10$):
	
	Finally base on our definition of the median we get:
	
	Now we determine the cumulative function of the geometrical law. We start from:
	
	Then we have by definition the cumulative probability of that the experience is successful in the first $N$ trials:
	
	with $N$ being for sure an integer of values $0,1,2,...$.
	We write: 
	
	We then have for the CDF:
	
	\begin{tcolorbox}[colframe=black,colback=white,sharp corners]
\textbf{{\Large \ding{45}}Example:}\\
	\begin{flushleft}
	You try late at night and in the dark, to open a lock with a bunch of five keys, without attention, because you are a little tired (or a little tipsy ...) you will try each key. Knowing that only one key will work, what is the probability of using the right key at the $N$-th test?
	
	The solution is:
	
	\end{flushleft}
	\end{tcolorbox}
	
	Plot of the mass function and cumulative distribution function for the Geometric distribution with parameter $p=0.5$:
	\begin{figure}[H]
		\begin{center}
			\includegraphics{img/arithmetics/law_geometric.jpg}
		\end{center}	
		\caption{Geometric law $\mathcal{G}$ (mass and cumulative distribution function)}
	\end{figure}
	
	\subsubsection{Binomial Distribution}
	
	We come back now to our Bernoulli experiment. More generally, any particular $N$-tuple consisting of $k$ successes and of $N-k$ failures will have for probability (within a sampling with replacement or without replacement if the population is large ... in a first approximation):
	
	
	
	to be drawn (or appear) whatever the order of appearance of successes and failures (the reader will have perhaps notice that this is a generalization of the geometric distribution, just write $k = 1$ to find the geometric distribution back).
	
	But we know that the combinatorial determines the number of $N$-tuples of this type (the number of ways to order the appearance of failures and successes). The number of possible arrangements is, as we proved it (\SeeChapter{see section Probabilities}), given by the binomial coefficient (we recall that the notation in this book does not comply with ISO standard 31-11):
	
	So as the probability of obtaining a given series of $k$ successes and $N-k$ failures is always the same (regardless of the order) then we have just to multiply the probability of a particular series by the binomial coefficient (this is equivalent to a sum ) such that:
	
	to get the total probability to obtain any of these possible series (since each is possible).
	
	\begin{tcolorbox}[title=Remark,colframe=black,arc=10pt]
This is equivalent to the study of a sampling with simple replacement (see Probabilities) with constraint on the order or to the study of a series of successes and failures. We will use this relation in the context of the queuing theory or reliability (\SeeChapter{see section Industrial Engineering}). Note that in the case of large populations, even if the sampling is not with replacement it can be considered as with...
	\end{tcolorbox}
		
	Written in another way this gives the "\NewTerm{binomial function}\index{Binomial function}" (or "\NewTerm{binomial law}\index{binomial law}"), also known as the following distribution function:
	
	and sometimes also denoted by $\beta(n,p)$ with a lowercase $n$ or uppercase $N$ (it does not really matter...) and can be calculated in the English version of Microsoft Excel 11.8346 using the \texttt{BINOMDIST( )} function.
	
	We sometimes say that the binomial law is not exhaustive as the size of the initial population is not apparent in the expression of the law.
	
	\begin{tcolorbox}[title=Remark,colframe=black,arc=10pt]
		The Binomial distribution is named "\NewTerm{Symmetric Binomial Distribution}\index{symmetric binomial distribution}" when $p=0.5$.
	\end{tcolorbox} 
	
	\begin{tcolorbox}[colframe=black,colback=white,sharp corners]
\textbf{{\Large \ding{45}}Example:}\\
We want to test the alternator of a generator. The probability of failure at solicitation of this material is estimated to be 1 failure per $1,000$ starts.\\

We decided to test $100$ starts. The probability of observing one failure in this test is:
	
	\end{tcolorbox}
	We obviously have for the cumulative distribution function (very useful in practice for suppliers batch control or reliability as we will see in the section of Industrial Engineering!):
	
	Indeed, we have proved in the section of Calculus the "\NewTerm{binomial theorem}\index{binomial theorem}":
	
	Therefore:
	
	Instead of calculating such cumulated probability rather than hand it is better to use Microsoft Excel 11.8346 (or any other widely known software) with the function \texttt{CRITBINOM()} to not bother to calculate these type of values.
	
	The expected mean (average) of $\mathcal{B}(N,k)$ is given by:
	
	
	But having:
	
	We finally get:
	
	that gives the average number of times that we will get the desired outcome of probability $p$ after $N$ trials.
	
	The mean of the binomial distribution is sometimes noted in the specialized literature with the following notation if $r$ is the potential number of possible expected outcomes in a population of size $n$:
	
	Before calculating the variance, we need to introduce the following equality:
	
	Indeed, let us proof this relation using the previous developments:
	
	We recognize in the last equality the cumulative distribution function that is equal to 1. Therefore:
	
	
	We start now the (long) calculation of the variance of the binomial distribution by using the previous results:
	
	Finally:
	
	The standard deviation of the binomial distribution is sometimes noted in the specialized literature in the following way if $r$ is the potential number of expected outcomes in a population of size $n$ and $s$ the not expected one:
	
	Here is a plot example of the binomial $\mathcal{B}(10,0.5)$ distribution and cumulative distribution function:
	\begin{figure}[H]
		\begin{center}
			\includegraphics{img/arithmetics/law_binomial.jpg}
		\end{center}	
		\caption{Binomial law $\mathcal{B}$ (mass and cumulative distribution function)}
	\end{figure}
	
	It could be useful to note that some employees in companies normalize the calculation of the mean and standard deviation to the unit of $N$. Then we have:
	
	
	\begin{tcolorbox}[colframe=black,colback=white,sharp corners]
\textbf{{\Large \ding{45}}Example:}\\
	In a sample of $100$ workers, $25\%$ are late at least once a week. The mean number and variance of late people is then:
	
	Normalized to the unit of $N$ this give us:
	
	\end{tcolorbox}
	Let us now calculate the mode. Because the function is discrete we can not use derivative. Then we will use a hint. We compute the ratio:
		
	 and we check that this ratio is $>1$ for every $k<k^{*}$ and $\leq 1$ for every $k\geq k^{*}$, for some integer $k^{*}$ that is the $k$ value corresponding to the modal value. 
	 
	 Let $a_k=P(X=k)$. We have:
	 
	 We calculate the ratio $\dfrac{a_{k+1}}{a_k}$. Note that:
	 
	 What is important now is to analyze:
	  
	 depending on the value of $k$. First we can see that this ratio is equal to 1 and therefore we have to modes if:
	 
	 That is to say if $k=np+p-1=p(n+1)-1$. This can be seen as the limit point of interest. But don't forget we are looking for the $k$ such that the ratio is less than 1. So we try two values: 
	 
	 Injecting this in our ratio we see that
	 
	 Is the value we were looking for. Finally there are two possible values for the modes. A unique modal value and a double modal value.
	
	As we know the median value, is the value of $X$ such that we have:
	
	But we did not yet found an easy proof to determine $M_e$ in the general case for the Binomial law.
	
	To conclude on the binomial law, we will develop now a result that we will need to build the McNemar paired test for a square contingency table (and as it is squared it is also dichotomous) that we willl study in the section of Theoretical Computing.

	We need for this test to calculate the covariance of two-paired binomial random variables (this is why the covariance is non-zero):
	
	As they are paired, this means that:
	
	And therefore:
	
	Now comes the difficulty that is to calculate $\text{E}(n_in_j)$. To calculate this term it does not exist to our knowledge other methods than looking for the law of the pair (sometimes we can get around such approach). In this case it is a multinomial distribution (more precisely: trinomial) that it is customary to write in the following way by construction:
	
	that we will write now temporarily as following to condense the expression:
	
	So we have a trinomial law as we are looking for the number of times we have the event $k$, the event $l$ and neither one nor the other (so the rest of the time). 
	
	We then get:
	
	If $k \geq 1$ and $l \geq 1$, we obtain:
	
	Now we use this relation in the joint mean:
	
	Consider now the special case where $n$ is equal $2$. We then have: 
	
	where the sum is reduced to only one term because if we take for example $k=2,l=1$ we get a negative factorial at the denominator.
	
	For $n$ equal $3$, the result will be also $1$, and so on (we will assume to simplify... that some numerical examples will suffice to convince the reader of the generality of this property because it is very boring to write with \LaTeX).
	
	Then we have:
	
	So in the end:
	
	And this is the major result we will need for the study of the McNemar test.

	\subsubsection{Negative Binomial Distribution}
	
	The negative binomial distribution is applied in the same situation as for binomial distribution, but it gives the probability to have $E$ failures before the $R$-th success when the probability of success is $p$ (or, at contrary, the probability to have $R$ success before $E$-th failure when the failure probability is $p$).

	We will introduce this important distribution with an example. Consider this for this purpose the following probabilities:
	
	Imagine that we have done $10$ trials and we wanted to stop at the third success and that the 10th trial is the third successful one! We will write this:
	
	Now we highlight what we will consider as the successes (R) and failures (E):
	

We have also $7$ failures and $3$ successes. In an experiment where the draws are independent (or can be considered as independent...), the probability that we get this particular result is:
	
	But the order of successes and failures in the bracketed part is irrelevant. So as we have $2$ success among the $9$ trial in brackets it follows that the probability of obtaining the same result regardless of the order is then using combinatorics:
	
	Which corresponds to the probability of having $7$ failures before the 3rd success (or otherwise seen: 3 successes after $10$ trials). This can be written with Microsoft Excel 14.0.6123 or later ($7+3 = 10$ trials, $7$ failures including $3$ successes):
	\begin{center}
		\texttt{=NEGBINOMDIST(7,3,0.2,0)=0.0604}
	\end{center}
	We now generalize the prior-previous notation by writing the number of failures $k, N$ the total number of trials and $p$ the probability of success:
	
	However, there are several possible notations because the previous relation is not very intuitive to practice as may have perhaps noticed the reader. Thus, if we denote $k$ as the number of successes and not the number of failures, then we have (the most common writing way from my point of view among a lot of others notations) the following probability of having $N-k$ success before having a number $k$ of failures with a probability $p$ (or of failures before having $k$ successes ... it's symmetrical!):
	
	therefore the comparison with the formulation of the binomial distribution proved above is then perhaps more obvious!
	
	However, it is more common to write the previous relation by removing $N$ because for the moment the notation is still perhaps not very clear. For this, we note $R$ the number of successes , $E$ the number of failures, $p$ the probability of success and then comes the probability of having $R$ success after $E$ failures (this is perhaps much more clear...):
	
	We sometimes find this last relation with another relation using explicitly the binomial coefficient:
	
	The cumulative probability that we have at least $R$ successes before the $E$-th failure is obviously given by:
	
	
	\begin{tcolorbox}[title=Remark,colframe=black,arc=10pt]
	The name of this law comes from the fact that some statisticians use a definition of the binomial coefficient with a negative value for the expression of the function. Since this is a rather a rare notation, we do not want lose time to prove the origin of the name. You should also know that this law is also known as the "\NewTerm{Pascal's law}\index{Pascal's law}" (as well as the geometric distribution ...) in honor of Blaise Pascal and also as "\NewTerm{Polya's law}\index{Polya's law}" in honor of George Pólya. 
	\end{tcolorbox}
	
	\begin{tcolorbox}[colframe=black,colback=white,sharp corners]
\textbf{{\Large \ding{45}}Examples:}\\\\
	E1. A long-term quality control has enabled us to compute the estimator of the proportion $p$ of nonconforming pieces as equal to $2\%$ at the output of a production line. We would like to know the cumulative probability to have $200$ pieces before the 3rd defective piece appears. With Microsoft Excel 14.0.6123 or later it comes using the negative binomial distribution:
	\begin{center}
		\texttt{=NEGBINOM.DIST(200,3,0.02,1)=77.35\%}
	\end{center}
	E2. To compare with the binomial distribution, we can ask ourselves what is the cumulative probability of drawing 198 non-defective parts from 201 using Microsoft Excel 14.0.6123 or later:
	\begin{center}
		\texttt{=BINOM.DIST(198,201,0.98,1)=76.77\%}
	\end{center}
	Therefore we see that the difference is small. In fact the difference between the two laws is in practice so small that we then use almost always the binomial law (but you should still be careful with this choice!).
	\end{tcolorbox}
	As usual, we will now determine the variance and mean of this law. Let's start with the mean of having $R$ successes when the $E$-th failure appears knowing that the probability of a failure is $p$. For this we will use a very simple and ingenious trick (all art was thinking about it...). If we return to our initial example:
	
	and we rewrite this example as follows:
	
	We then notice that the third success $R$ of the first notation can be decomposed into the sum of three geometric random variables such that:	
	
	With in the case of this particular example $n=3$ corresponding in fact to $E=3$. So quite generally the sum of $n$ random geometric variables always gives a negative binomial distribution if the probability $p$ is equal for each geometric variable!!! Anyway... as we have proved the expression of the mean and variance of the Geometric law as (thus giving us the mean rank of the first failure):
	
	since the random variables are independent and of same parameters then it comes for the negative binomial the mean of the rank of $E$-th failure using the property of the mean:
	
	And therefore for the variance of the negative binomial distribution:
	
	So the mean and variance of the rank (corresponding to the number of trials $N$ or from another point of view: the mean number of successes by the simple subtraction $X - E$) to have the $E$-th failure is then to summarize:
	
	Thus, putting $E = 1$, we fall back on the mean and variance of the geometric distribution!
	
	Now, let $Y$ be the random variable representing the number of trials \underline{before} the $R$-th success. We then have the following expressions for the variance and the mean that are very common in the literature (these expressions of mean and variance corresponds to what we can find for the negative binomial law in Wikipedia for example):
	
	\begin{tcolorbox}[colframe=black,colback=white,sharp corners]
\textbf{{\Large \ding{45}}Example:}\\\\
		What is the expected number (mean) of trials we can expect before we fall on the third non-conforming part, knowing that the probability of a non-conforming part is $2\%$?
		
		and for the standard deviation:
		
	\end{tcolorbox}
	Like always the reader will find below a plot example of the distribution and cumulative distribution function for the negative binomial law of parameters $\text{NB}(N,k,p)=P(N,3,0.6)$ based on the example of the begging, but where the only difference is the probability of success where we he have taken $60\%$ instead of $20\%$.
	
	Thus, there is $21.6\%$ of probability of having the third success after the third successive trial (i.e. $0$ trials more than the number of successes), $25.92\%$ of probability of having the third success after the fourth successive trial (i.e. one trial more than the number of successes), $20.7\%$ of probability of having the third success after the fifth successive trial (i.e. two trial more than the number of successes) and so on...:
	
	\begin{figure}[H]
		\begin{center}
			\includegraphics{img/arithmetics/law_binomial_negative.jpg}
		\end{center}	
		\caption{Negative Binomial law $\texttt{NB}$ (mass and cumulative distribution function)}
	\end{figure}
	
	The above distributions are truncated to $9$ (corresponding to $12$ trials) but they theoretically continue indefinitely. What particularly distinguishes the binomial and geometric distributions from the negative binomial are the tails of the distribution.

The binomial negative distribution has an important place in a special regression technique that we will see in the section of Theoretical Computing.

	\subsubsection{Hypergeometric Distribution}
	
	We consider to approach this function a simple example (but not very interesting in practice) that is this of an urn containing $n$ balls where $m$ are black and the other $m$ white (for several important examples used in the industry refer to the sections of Industrial Engineering or Numerical Methods). We take successively, and without replacement in the urn, $p$ balls. The question is to find the probability that among the $p$ balls, there is $k$ that are black (in this statement the order of the drawing does not interest us).
	
	We often talk about "exhaustive sampling" with the hypergeometric distribution because at the opposite of to the binomial distribution, the size of the lot which is the basis for the sampling will appear in the law.
	
	\begin{tcolorbox}[title=Remark,colframe=black,arc=10pt]
		This is also equivalent to an non-ordered sampling without replacement (\SeeChapter{see section Probability}) with constraint on the occurrences sometimes named "simultaneous sampling". We will often use the hypergeometric distribution in the field of quality and reliability where the black balls are associated to items with defects and the white one to items without defects.
	\end{tcolorbox}
	
	The $p$ balls can be chosen among $n$ balls in $C_p^n$ ways (thus representing the number of possible different outcomes) with as reminder (\SeeChapter{see section Probability}):
	
	
	The $k$ black balls can be chosen among the $m$ black in $C_k^m$ ways. The $p-k$ white balls can be chosen in $C_{p-k}^{n-m}$ ways. There is therefore $C_k^mC_{p-k}^{n-m}$ ways to have $k$ black balls and $p-k$ white balls.
	
	The searched probability is therefore given by (we will see an alternative notation in the section of Industrial Engineering):
	
	and is said to follow a "\NewTerm{Hypergeometric distribution}\index{Hypergeometric distribution}" (or "\NewTerm{Hypergeometric law"}\index{Hypergeometric law}) and can be obtained fortunately directly in Microsoft Excel 14.0.7153 with the function \texttt{HYPGEOM.DIST()}.
	\begin{tcolorbox}[colframe=black,colback=white,sharp corners]
\textbf{{\Large \ding{45}}Examples:}\\
	E1. We want to develop a small computer program of $10,000$ lines of code ($n$). The return on experience shows that the probability of failure is one bug per $1,000$ lines of code (or $0.1\%$ of $10,000$ lines) that corresponds to the value $m$.
	
		We test about $50\%$ of the functionality of the software randomly before sending it to the customer (corresponding to the equivalent of $5,000$ lines that is $p$). The probability of observing $5$ bugs ($k$) is then given with Microsoft Excel 14.0.715:
	\begin{center}
		\texttt{HYPGEOMDIST(k,p,m,n)=HYPGEOMDIST(5,5000,1\%*10000,10000)=24.62\%}
	\end{center}

	E2. In a small single production of a batch of $1,000$ pieces we know that $30\%$ on average are bad because of the complexity of the pieces and by return on experience from a previous similiar manufacturing. We know that a customer will randomly draw $20$ pieces to decide whether to accept or reject the lot. He will not reject the lot if he finds zero defective pieces on the $20$. What is the probability of having exactly $0$ defective?
	\begin{center}
		\texttt{=HYPGEOMDIST(0,20,300,1000)=0.073\%}
	\end{center}
	and as we require a null draw drawing result, the calculation of the hypergeometric distribution simplifies manually to: 
	
	\end{tcolorbox}
	It is not forbidden to make direct calculation of the mean and variance of the hypergeometric distribution, but the reader will without much trouble guess that this calculation will be ... relatively indigestible. Then we can use an indirect method that is much more interesting!

First, the reader will perhaps, even certainly, have noticed that experienced of the hypergeometric distribution is a series of Bernoulli trials (without replacement of course!).

So we will cheat by using initially the property of linearity of the mean. We define for this purpose a new variable corresponding implicitly in fact to the experience of the hypergeometric distribution (a sequence of $k$ Bernoulli trials!):
	
	where $X_i$ is the success of obtaining at the $i$-th drawing a black ball (either $0$ or $1$). But, we know that for all $i$ the random variable $X_i$ follows a Bernoulli function for which we have proved in our study of the Bernoulli distribution that $\text{E}(X_i)=p$.
	
	Therefore, by the property of linearity of the mean we have (caution! here $p$ is not the number of balls, but the probability associated with an expected event!):
	
	In the Bernoulli trial, $p$ is the probability of obtaining the desired item or event (for reminder...). In the hypergeometric distribution what interests us is the probability of a black ball (which are in quantity $m$, therefore with $m'$ white balls) compared to the total amount of $n$ balls. And the ratio obviously gives us this probability. Thus, we have:
	
	where $k$ is the number of trials (do not confuse with the notation of the initial statement where it was by the variable $p!$). The mean gives then the average number of black balls in a drawing of $k$ balls among $n$, where $m$ are known as being black. The reader will have noticed that the hope of the hypergeometric distribution is the same as the binomial distribution!
	
	To determine the variance, we use the variance of the Bernoulli distribution and the following relation proved during the introduction of the mean and covariance at the beginning of this chapter:
	
	Recalling that we have $X=\displaystyle\sum_{i=1}^k X_i$, we get:
	
	However, for the Bernoulli law, we have:
	
	Then we first already get:
	
	The calculation of $\text{E}(X_iX_j)$ requires a good understanding of probabilities (this will be a good refresh!).
	
	The mean $\text{E}(X_iX_j)$ is given (implicitly), as we know, by the weighted sum of the probabilities that two events occur at the same time. However, our events are binary: either it is a black ball ($1$) or it is a white ball ($0$). So all terms of the sum without two consecutive black balls consecutively will be null!
	
	The problem is then to calculate the probability of having two consecutive black balls and it is thus written:
	
	So we finally have:
	
	Therefore:
	
	Finally (using the result of Gauss series seen in the section of Sequences and Series):
	
	where we have used the fact that:
	
	is composed of:
	
	terms as correspond to the number of ways there are to choose a pair $(i, j)$ with $i<j$.
	Because:
	
	We can write:
	
	In the specialized literature, we often find the variance written in the following way by noting the expected event $r$ and the non-expected event $s$:
	
	so with $l = n - k$. This last notation will be very useful in the section of Theoretical Computing for our study of the Mantel-Haenszel test.
	Furthermore, we see that in:
	
	there is the standard deviation as the binomial distribution, at the difference of a factor that is noted:
	
	the we found often in statistics and is named "\NewTerm{finite population correction factor}\index{finite population correction factor}".

	Here is an example plot example of the distribution function and cumulative distribution for the Hypergeometric function of parameters $(n,p,m,k)=(10,6,5,k)$:
	\begin{figure}[H]
	\begin{center}
			\includegraphics{img/arithmetics/law_hypergeometric.jpg}
		\end{center}	
		\caption{Hypergeometric law $\mathcal{H}$ (mass and cumulative distribution function)}
	\end{figure}
	We will prove now that the Hypergeometric distribution tends to a binomial distribution since this property is used many times in different sections of this book (especially the section of Industrial Engineering).

	To do this, we decompose:
	
	We then get:
	
	For the second term:
	
	For $m\rightarrow +\infty$ (...) all the terms are of then of the order of $m$. Then we have:
	
	For the third term, a identical development to the previous one provides (for sure we need that also $n \rightarrow +\infty$ (...)):
	
	And for sure... we can discuss therefore about $n-m$ when both terms tends to infinity...
	Ditto for the fourth term:
	
	In conclusion we have:
	
	We change the notation by writing $p$ (the number of individuals drawn) as being $N$. We get then:
	
	We make another change of notation by writing $b$ the black balls and $w$ the white balls. We get then:
	
	Finally, we note $p$ the proportion of black balls and $q$ that of white balls in the lot. We then get:
	
	We find out the binomial distribution! In practice, it is common to approximate the hypergeometric distribution with a binomial distribution when the ratio of the number of individuals from the total number of individuals is less than 10% (that is to say when the sample is 10 times smaller than the whole population). It follows that the hypergeometric distribution tends also (as we will show later) to a Normal distribution when the population tends to infinity and that the sample is small.

	In practice, Monte Carlo simulations with testing adjustments (see late in this chapter) have shown that the hypergeometric distribution could be approximated by a Normal distribution (very important case in contingency statistical tests that we will study in the section of Theoretical computing) if the following three conditions are met simultaneously:
	
	Thus graphically and approximately...:
	\begin{figure}[H]
	\begin{center}
		\includegraphics{img/arithmetics/hypergeometrice_normal_approximation.jpg}
		\end{center}	
		\caption{Conditions of application of the approximation by a Normal distribution}
	\end{figure}
	Thus:
	
	
	\subsubsection{Multinomial Distribution}
	The multinomial distribution (so named because it involves several times the binomial coefficient) is a law applicable to $n$ distinguishable events, each with a given probability, which occur one or more times and it is not necessarily ordered. This is a frequent case in marketing research and that will be useful to build the statistical McNemar test that we will study much later (\SeeChapter{see section Theoretical Computing}). We also use this law in quantitative finance (\SeeChapter{see section Economy}). 
	
	More technically, consider the space of events $\Omega=\left\lbrace 1,2,...,m \right\rbrace$ with the probabilities $P(\left\lbrace i\right\rbrace)=p_i,i=1,2,...,n$. We take $n$ times with a given element of $\Omega$ with replacement (\SeeChapter{see chapter Probability}) with the probability $p_i,i=1,2,...,n$. We will search what is the probability of such a non-necessarily ordered the event $1$, $k_1$ times, event 2, $k_2$ times and this on a sequence of $n$ drawings.
	
	\begin{tcolorbox}[title=Remark,colframe=black,arc=10pt]
This is equivalent to the study of a sampling with replacement (\SeeChapter{see section Probability}) and constraints on the occurrences. So without constraints we will see with an example that we fall back on a sampling with simple replacement.
	\end{tcolorbox}	
	
	We saw in the section of Probabilities, that if we take a set of events with multiple outcomes, then different combinations of sequences we can get taking $p$ selected elements among $n$ is given by:
	
	We have therefore:
	
	different ways to get $k_1$ times a given event. Thus an associated probability of:
	
	Now comes the particularity of the multinomial distribution!: there are no failures in contrast to the binomial distribution. Each "pseudo-failure" can be considered as a subset draw of $k_2$ items from the $n-k_1$ remaining elements.
	Thus the term:
	
	will be written on the whole experience if we consider a particular case limited to two types of events:
	
	so with:
	
	which gives us the number of different times to get $k_2$ times a second event because in the whole sequence of $n$ elements $k_1$ of them have already been taken so have now only $n-k_1$ remaining on which we can get the $k_2$ desired.
	
	These relations then show us that this is a situation where each event probability is considered as a binomial (hence its name ...).

So we have in the case of two sets of $t$-uples:
	
	and because:
	
	we get:
	
	and we see that the construction of this distribution therefore requires that:
	
	Thus, by induction we have the probability $\mathcal{M}$ we were looking for and named the "\NewTerm{Multinomial function}\index{multinomial function}" and using previous relation given by:
	
	in the spreadsheet software Microsoft Excel 11.8346, the term:
	
	is named "\NewTerm{multinomial coefficient}\index{multinomial coefficient}" and is available under the name of the function \texttt{MULTINOMIAL( )}. In the literature we also find this term sometimes under the following respective notations:
	
	\begin{theorem}	
	We will show now that the multinomial distribution is effectively a probability distribution (because we could doubt ...). If this is the case, as we know it, the sum of the probabilities must be equal to $1$.
	\end{theorem}	
	\begin{dem}
		Recall that in the chapter of Calculus we proved that (binomial theorem):
		
		Now do a little bit of notation:
		
		and this time a change of variables:
		
		This last relation (which is a special case of the two terms "\NewTerm{multinomial theorem}\index{multinomial theorem}") will be useful to us to show that the multinomial distribution is effectively a probability distribution. We also take the special case with two groups of drawing:
		
		which can is also written by the construction of the multinomial distribution:
		
		and therefore, the sum must be equal to the unit such that:
		
		To check this we use the multinomial theorem shown above:
		
		However, by construction of the multinomial sum of probabilities is unitary, we have effectively:
		
		\begin{flushright}
			$\square$  Q.E.D.
		\end{flushright}
	\end{dem}
	\begin{tcolorbox}[colframe=black,colback=white,sharp corners]
	\textbf{{\Large \ding{45}}Examples:}\\\\
		E1. We launch an unbiased die $12$ times. What is the probability that all $6$ faces appear the same number of times (not necessarily consecutively!) that means twice for each:
		
		where we see well that $m$ is the number of success groups.\\
		
		E2. We launch an unbiased die $12$ times. What is the probability that a single unique face appears $12$ times (hence the "$1$" appears $12$ times, or the "$2$" or the "$3$", etc.):
		
		So we end up with this last example known a being a binomial distribution result.
	\end{tcolorbox}

	\pagebreak
	\subsubsection{Poisson Distribution}
	
	For some rare events, the probability $p$ is very small and tends to zero. However, the average value $np$ tends to a fixed value as $n$ tends to infinity.
	
	We start from a binomial distribution with mean $\mu=np$ that we will assume finite when $n$ tends to infinity.
	
	The probability of $k$ successes after $n$ trials is (Binomial distribution):
	
	
	By writing $p=\dfrac{m}{n}$ (where $m$ will be temporarily the new notation for the mean according to $\mu=np$), this expression can be rewritten as:
	
	
	By grouping the terms, we can put the value under the form:
	
	We recognize that when $n$ tends to infinity, the second factor of the product has for limit $e^{-\mu}$ (\SeeChapter{see Functional Analysis}). The third factor, since we are interested to the small values of $k$ (the probability of success is very small), its limit for $n$ tending to infinity is equal to $1$.
	
	This technique of passing to the limit is sometimes named in this context: the "\NewTerm{Poisson limit theorem}\index{Poisson limit theorem}".
	
	So we get the "\NewTerm{Poisson distribution}\index{Poisson distribution}" (or "\NewTerm{Poisson law}\index{Poisson law}"), also sometimes named the "\NewTerm{law of rare events}\index{law of rare events}" therefore given by:
	
	which can be obtained in Microsoft Excel 11.8346 with the function \texttt{POISSON( )} and in practice and the specialized literature is often indicated by the letter $u$.
	
	It is indeed a probability distribution since using the Taylor series (\SeeChapter{see chapter Sequences And Series}), we show that the sum of the cumulative probabilities is:
	
	\begin{tcolorbox}[title=Remark,colframe=black,arc=10pt]
We will frequently encounter this distribution in different sections of the book such as in the study of preventive maintenance in the section of Industrial Engineering or in the section of Quantitative Management for the study of queuing theory (the reader can refer to them for interesting and pragmatic examples) and finally in the field of life and non-life insurance.
	\end{tcolorbox}	
	Here is a plot example of the Poisson distribution and cumulative distribution function with parameter $\mu=3$:
	\begin{figure}[H]
	\begin{center}
		\includegraphics{img/arithmetics/law_poisson.jpg}
		\end{center}	
		\caption{Poisson law $\mathcal{P}$ (mass and cumulative distribution function)}
	\end{figure}
	This distribution is important because it describes many processes whose probability is small and constant. It is often used in the "queing theory" (waiting time), acceptability and reliability test and statistical quality control. Among other things, it applies to processes such as the emission of light quanta by excited atoms, the number of red blood cells seen under the microscope, the number of incoming calls to a call center. The Poisson distribution is valid for many observations in nuclear and particle physics.
	
	The mean (average) of the Poisson distributions is (we use the Taylor series of the exponential):
	
	and gives the average number of times that you get the desired outcome.
	This result may seem confusing .... the mean is expressed by the mean?? Yes must simply not forget that it is given since the beginning by:
	
	\begin{tcolorbox}[title=Remark,colframe=black,arc=10pt]
	For more details the reader may also refer to the subsection on "estimators" later in this section. 
	\end{tcolorbox}	
	The variance of the Poisson distribution function is itself given by (again we use the Taylor series):
	
	always with:
	
	The important fact for the Poisson distribution is that the variance that is equal to the mean is name the "\NewTerm{equidispersion property of the Poisson distribution}\index{equidispersion of the Poisson distribution}". This is a property often used in practice as an indicator to identify whether the data (with discrete support) are distributed according to a Poisson distribution.

	The theoretical laws of statistical distributions are determined assuming completion of an infinite number of measurements. It is obvious that we can only perform in practice a finite number $N$. Hence the utility to establish correspondence between the theoretical and experimental values. For the experimental values we obviously obtain only an approximation whose validity is, however, often accepted as sufficient.

	Now we will prove an important property of the Poisson distribution in the field of engineering and statistics that we name "stability by addition". The idea is as follows:
	
	
Let $X$ and $Y$ be two independent random variables with Poisson distribution of respective parameters $\lambda$ and $\mu$. We want to ensure that their sum is also a Poisson distribution:
	
	See this:
	
	because the events are independent. Then we have:
	
	However, by applying the binomial theorem (\SeeChapter{see section Calculus}):
	
	So in the end:
	
	and therefore the Poisson distribution is stable by addition. So any Poisson distribution where the parameter is verbatim indefinitely dividable into a finite or infinite sum of independent Poisson distributions.
	\subsubsection{Normal \& Gauss-Laplace Distribution}
	
	This characteristic is the most important function of distribution in the field of statistics following a famous theorem named the "central limit theorem", which as we will see later, permits to prove (among other things) that any sum of independent identically distributed random variables with a finite mean and variance converges to a Laplace-Gaussian function (Normal distribution).
	
	It is therefore very important to focus your attention on the developments that will be presented right now!

	Let start from a binomial function and make tender the number of trials $n$ to infinity. If $p$ is set from the beginning , the mean $\mu=np$ also tends to infinity, furthermore the standard deviation $\sigma=npq$ also tends to infinity.

	\begin{tcolorbox}[title=Remark,colframe=black,arc=10pt]
	The case where $p$ varies and tends to $0$ while keeping fixed the mean has already been studied during the study of the Poisson function.
	\end{tcolorbox}
	
	If we want to calculate the limit of the binomial function, it will then be necessary to make a change of origin, which stabilizes the mean, to $0$ for example, and a change of unit change that stabilizes the standard deviation to $1$ for example.
	
	Let us now denote by $P_n(k)$ the binomial probability of $k$ success and let's see first how $P_n(k)$ vary with $k$ and calculate the difference:
	
	We conclude that $P_n(k)$ is an increasing function of $k$, as $np-k-q$ is positive (for $n, p$ and $q$ fixed). Too see it, juste take a few values (of the right term of the equality) or to observe the graph of the binomial distribution function, remembering that:
	
	As $q<1$ it is therefore evident that the value of $k$ close to the mean $\mu=np$ of the binomial distribution is the maxima of $P_n(k)$.
	
	On the other hand the difference $P_n(k+1)-P_n(k)$ is the increase rate of the function $P_n(k)$. We can then write:
	
	as being the slope of the function.

	We now define a new random variable such that its average is zero (negligible variations) and its standard deviation equal to the unit (this will be a centered-reduced variable in other words). Then we have:
	
	Then we also have with this new random variable:
	
	Let us write denote $F(x)$ as being $P_n(k)$ calculated using the new random variable with zero mean and unit standard deviation which we seek the expression when n tends to infinity.
	
	Let go back to:
	
	
	To simplify the study of this relation when $n$ tends to infinity and $k$ to mean $\mu=np$, multiply both sides by $npq/\sqrt{npq}$:
	
	We rewrite now the right-hand side of this equality. Then we get:
	
	And now let us rewrite the left term of the prior-previous relation. We then get:
	
	After passing to the limit for $n$ tending to infinity we have in a first time for the denominator of the second term of the prior-previous relation:
	
	the following simplification:
	
	Thus:
	
	and in a second time, taking into account that the considered values of $k$ are then in the neighborhood of the mean $np$, we get:
	
	and:
	
	Thus:
	
	and as:
	
	where $F(x)$ represents (awkwardly) for the few lines that follow, the density function as $n$ tends to infinity.
	
	Finally we have:
	
	This relation can also be rewritten rearranging the terms:
	
	and by integrating both sides of this equality we obtain (\SeeChapter{see section Differential And Integral Calculus}):
	
	The following function is a solution of the above equation:
	
	Effectively:
	
	The constant is determined by the condition that:
	
	which represents the sum of all probabilities, that mus be equal to $1$. We prove for this that we need to have:
	
	\begin{dem}
		We have:
		
		So let us focus on the last term of this equality. Thus:
		
		since $e^{-x^2}$ is an even function (\SeeChapter{see section Functional Analysis}). Now we write the square of the integral as follows:
		
		and make a change of variable passing in polar coordinates, therefore we also use the Jacobian in these same coordinates (\SeeChapter{see section Differential And Integral Calculus}):
		
		Therefore:
		
		By extension for $e^{-\dfrac{x^2}{2}}$ we have:
		
		\begin{flushright}
			$\square$  Q.E.D.
		\end{flushright}
	\end{dem}
	We thus obtain the "\NewTerm{standard Normal distribution}\index{standard Normal distribution}" noted as probability density function (noted with the capital letter $F$ that can unfortunately lead to confusion in the present development with the notation of the cumulative distribution function... we apologize...):
		
		which can be calculated in Microsoft Excel 11.8346 with the function \texttt{NORMSDIST( )}.
		
		For information, a variable following a Normal centered reduced distribution is by tradition often noted $Z$ ("Zentriert" in German).
		
		Returning to non-normalized variables:
		
		so we get the "\NewTerm{Gauss-Laplace function}\index{Gauss-Laplace function}" (or "\NewTerm{Gauss-Laplace law}\index{Gauss-Laplace law}") or also named "\NewTerm{Normal distribution}\index{Normal distribution}"\index{Normal law} given in the form of probability density in this book by:
		
		The cumulative probability (distribution cumulative function) to have a certain value $k$ is obviously given by:
		
		Here is a plot example of the distribution and cumulative distribution function for the Normal law with the parameters example $(\mu,\sigma)=(0,1)$ that is therefore the standard centered reduced Normal distribution:
	\begin{figure}[H]
	\begin{center}
		\includegraphics{img/arithmetics/law_normal.jpg}
		\end{center}	
		\caption{Normal law $\mathcal{N}$ (mass and cumulative distribution function)}
	\end{figure}
	This law governs under very general conditions, and often encountered, in many random phenomena. It is also symmetrical with respect to the mean (this is important to remember!).
	
	We will now show that $\mu$ well represents the mean (or average) of $x$ (this is a bit silly but we can still check ...):
	
	We put:
	
	We then have:
	
	Let us now calculate the first integral:
	
	So we finally get:
	
	\begin{tcolorbox}[title=Remarks,colframe=black,arc=10pt]
	\textbf{R1.} The reader might find confusing at first that the parameter of a function is one of the results that we seek of this same function (as for the Poisson distribution). What bothers is to put in practice such a thing. In fact, everything will be more clear when we will discuss later in this chapter the concepts of "likelihood estimators".\\

	\textbf{R2.} It could be interesting to know for the reader that in practice (finance, quality assurance, etc.) it is common to have to calculate only mean only positive values of the random variable which is then naturally defined as "positive mean" and given by:
	
	We will see a practical example of this last relation in the section Economy during our study of the theoretical model of speculation of Louis Bachelier.
	\end{tcolorbox}
	Also we will prove now (...) that $\sigma$ is the standard deviation of $X$ (in other words to prove that $\text{V}(X)=\sigma^2$) and for this we recall that we had prove that (Huyghens relation):
	
	We already know that at the level of the notations we have:
	
	then we first calculate $\text{E}(X^2)$:
	
	Let $y=(x-\mu)/\sqrt{2}\sigma$ that therefore leads us to:
	
	And we know that (already proved above):
	
	It remains to calculate therefore only the first integral. To do this, we proceed by integration by parts (\SeeChapter{see Differential and Integral Calculus}):
	
	leads us to:
	
	Then we get:
	
	And finally:
	
	An additional signification of the standard deviation of Gauss-Laplace distribution is a measure of the width of the distribution as (this can be checked only with the aid of integration by using numerical methods) for any non-zero mean and standard deviation we have (thanks to John Cannin for the \LaTeX figure):
	\begin{figure}[H]
		\centering
		\pgfplotsset{compat=1.7}
		\pgfmathdeclarefunction{gauss}{2}{\pgfmathparse{1/(#2*sqrt(2*pi))*exp(-((x-#1)^2)/(2*#2^2))}%
		}
		\begin{tikzpicture}
		\begin{axis}[no markers, domain=0:10, samples=100,
		axis lines*=left, xlabel=Standard deviations, ylabel=Frequency,,
		height=6cm, width=10cm,
		xtick={-3, -2, -1, 0, 1, 2, 3}, ytick=\empty,
		enlargelimits=false, clip=false, axis on top,
		grid = major]
		\addplot [fill=cyan!20, draw=none, domain=-3:3] {gauss(0,1)} \closedcycle;
		\addplot [fill=orange!20, draw=none, domain=-3:-2] {gauss(0,1)} \closedcycle;
		\addplot [fill=orange!20, draw=none, domain=2:3] {gauss(0,1)} \closedcycle;
		\addplot [fill=blue!20, draw=none, domain=-2:-1] {gauss(0,1)} \closedcycle;
		\addplot [fill=blue!20, draw=none, domain=1:2] {gauss(0,1)} \closedcycle;
		\addplot[] coordinates {(-1,0.4) (1,0.4)};
		\addplot[] coordinates {(-2,0.3) (2,0.3)};
		\addplot[] coordinates {(-3,0.2) (3,0.2)};
		\node[coordinate, pin={68.2\%}] at (axis cs: 0, 0.4){};
		\node[coordinate, pin={95\%}] at (axis cs: 0, 0.3){};
		\node[coordinate, pin={99.7\%}] at (axis cs: 0, 0.2){};
		\node[coordinate, pin={34.1\%}] at (axis cs: -0.5, 0){};
		\node[coordinate, pin={34.1\%}] at (axis cs: 0.5, 0){};
		\node[coordinate, pin={13.6\%}] at (axis cs: 1.5, 0){};
		\node[coordinate, pin={13.6\%}] at (axis cs: -1.5, 0){};
		\node[coordinate, pin={2.1\%}] at (axis cs: 2.5, 0){};
		\node[coordinate, pin={2.1\%}] at (axis cs: -2.5, 0){};
		\end{axis}
		\end{tikzpicture}
		\caption{Sigma intervals for the Normal distribution}
	\end{figure}
	An additional signification of the standard deviation of Gauss-Laplace distribution is a measure of the width of the distribution as (this can be checked only with the aid of integration by using numerical methods) for any non-zero mean and standard deviation we have:
	
	The width of the interval has a great importance in the interpretation of uncertainties measurement. The presentation of a result like $\bar{N}\pm \sigma$ has for signification that the average value has about $68.3\%$ chance (probability) to lie between the limits of $\bar{N}- \sigma$ and $\bar{N}+ \sigma$ or has approximately $95.4\%$ to lie between the limits of $\bar{N}- 2\sigma$ and $\bar{N}+2\sigma$ etc.
	\begin{tcolorbox}[title=Remarks,colframe=black,arc=10pt]
		This concept is widely used in quality management in industrial business especially with the Six Sigma methodology (\SeeChapter{see Industrial Engineering}) which requires a mastery of $6$ around each side of the mean (!) of the manufacturing (or anything else whose deviation is measured).\\
	
	The second column of the table can easily be obtained with Maple 4.00b (or also with the spreadsheet software from Microsoft). For example for the first line:\\

	\texttt{>S:=evalf(int(1/sqrt(2*Pi)*exp(-x\string^ 2/2),x=-1..1));}\\

	and the first row of the third column:\\

	\texttt{>(1-S)*1E6;}\\

	If the Normal distribution was not centered, then we just would write for the second column:\\

	\texttt{>S:=evalf(int(1/sqrt(2*Pi)*exp(-(x-mu)\string^ 2/2),x=-1..1));}\\

	and so on for any deviation and mean we will then obtain exactly the same intervals!!!
	\end{tcolorbox}
	
	The Gauss-Laplace distribution is also not only a tool for data analysis but also for data generation. Indeed, this distribution is one of the largest used in the world of multinationals that use statistical tools for risk management, project management and simulation where a large number of random variables are to be controlled. The best examples of applications use the softwares CrystalBall or Palisade @Risk (this last one being my favorite...).

	In this context of application (project management), it is also very common to use the sum (task duration) or the product of random variables (customer uncertainty) following Gauss-Laplace distributions. We will see now how to to calculate this:

	\paragraph{Sum of two random Normal variables}\mbox{}\\\\
	Let $X, Y$ be two independent random variables. Suppose that $X$ follows the distribution $\mathcal{N}(\mu_1,\sigma_1)$ and that $Y$ follows the distribution $\mathcal{N}(\mu_2,\sigma_2)$. Then the random variable $Z=X+Y$ has a density equal to the convolution product of $f_x$ and $f_y$. That is to say:
	
	which is equivalent to the joint product (\SeeChapter{see Probabilities}) of the probabilities of occurrence of the two continuous variables (remember the same kind of calculation in discrete form!).
	
	To simplify the expression, make the change of variable $t=x-\mu_1$ and let us write $a=\mu_1+\mu_2-s,\sigma=\sqrt{\sigma_1^2+\sigma_2^2}$. 
	
	As:
	
	we get after a hard to guess rearrangement trick:
	
	We write:
	
	Then:
	
	Knowing that (proved above):
	
	and:
	
	our relation becomes:
	
	We recognize the expression of the Gauss-Laplace distribution (Normal law) of mean $\mu_1+\mu_2$ and standard deviation $\sigma=\sqrt{\sigma_1^2+\sigma_2^2}$.
	Therefore, $X+Y$ follows the distribution as written by the physicist (both argument have same units):
	
	and as noted by most mathematicians, statisticians:
	
	The fact that the sum of two Normal distributions always give also a Normal distribution is what we name in statistics "\NewTerm{stability of the sum}\index{stability of the sum (statistics)}" of the Gauss-Laplace distribution (Normal law). We will find such properties for other distribution that will be discussed later.

	So as well as for the Poisson distribution, any Normal distribution whose parameters are known is verbatim indefinitely divisible into a finite or infinite number of independent Normal distribution that are summed as:
	
	
	\begin{tcolorbox}[title=Remark,colframe=black,arc=10pt]
		The families of stable distribution by the sum is an important field of study in physics, finance and statistics named "\NewTerm{Levy alpha-stable distribution}\index{Levy alpha-stable}". If time permits, I will present the details of this extremely important study in this chapter.
	\end{tcolorbox}
	
	\paragraph{Product of two random Normal variables}\mbox{}\\\\
	Let $X, Y$ be two real independent random variables. We denote by $f_X$ and $f_Y$ the corresponding densities and we seek to determine the density of the variable $Z=XY$ (very important case, particularly in engineering).
	
	Let $F$ denote the density function of the pair $(X, Y)$. Since $X, Y$ are independent (\SeeChapter{see section Probabilities}):
	
	The distribution function of $Z$ is:
	
	where $D=\left\lbrace(x,y) \vert xy<z\right\rbrace$. 
	
	$D$ can be rewritten as a disjoint union (we do this for anticipating in the future change of variables a division by zero):
	
	with:
	
	We have:
	
	The last integral is equal to zero because $D_3$ is of measurement (thickness) zero for the integral along $x$.
	
	We then perform the following change of variable:
	
	The Jacobian of the transformation (\SeeChapter{see section Differential and Differential Calculus}) is:
	
	Thus:
	
	Let $f_Z$ be the density of the variable $Z$. By definition:
	
	On the other hand:
	
	as we have seen. Therefore:
	
	What is a bit sad is that in the case of a Gauss-Laplace distribution (Normal distribution), this integral can only be easily calculated numerically ... it is then necessary to use Monte Carlo integration type methods (\SeeChapter{see section Theoretical Computing}).

	However according to some research done on the Internet, but without certainty, this integral may be calculated and give a new distribution named "\NewTerm{Bessel distribution}\index{Bessel distribution}".
	
	\paragraph{Bivariate Normal Distribution}\mbox{}\\\\
	If two Normal distributed random variables are independent, we know that their joint probability is equal to the product of their probabilities. So we have:
	
	Now comes an approach that we will often find in the follow developments: to generalize simple algebra models, you have to think in a Linear Algebra way! Therefore we are left with two vectors involving a scalar product:
	
	But we can do even better because for the moment there is no added value to this notation! Effectively a subtle idea is to involve the determinant of a matrix (\SeeChapter{see section Linear Algebra}) and the inverse of this same matrix in the previous relation:
	
	We thus find a particular case of the variance-covariance matrix. In the field of the bivariate Normal distribution is it is customary to write this last relation in the following form:
	
	If we make a plot of this function we get:
	\begin{figure}[H]
		\centering
		\includegraphics{img/arithmetics/law_bivariate_normal_perspective.jpg}
		\caption{Plot of the bivariate Normal function in MATLAB™}
	\end{figure}
	
	or another one (not with the same values) with corresponding projections:
	\begin{figure}[H]
		\centering
		\includegraphics[scale=0.7]{img/arithmetics/normal_bivariate_projection.jpg}
		\caption{Plot of the bivariate Normal function with pgfplots}
	\end{figure}
	Now consider the important case in engineering, astronomy and quantum physics by returning to the following notation:
	
	and by focusing on to the iso-lines such that for any pair of values of the two random variables, we have:
	
	By doing some very basic algebraic manipulations, we get:
	
	Thus:
	
	and we get:
	
	We recognize here the analytical equation of an ellipse (\SeeChapter{see section Analytical Geometric})!
	
	A plot of iso-lines with $\mu=\begin{pmatrix}3\\2\end{pmatrix},\Sigma=\begin{bmatrix}25 & 0\\0 & 9\end{bmatrix}$ give us:
	\begin{figure}[H]
		\centering
		\includegraphics{img/arithmetics/law_bivariate_normal_isolines.jpg}
		\caption{Plot of the iso-lines of the bivariate Normal function (non-correlated case)}
	\end{figure}
	But now recall that when we got:
	
	the variance-covariance matrix was zero everywhere except on the diagonal, implying verbatim the independence of the two random variables. We can obviously guess that the generalization is that the variance-covariance matrix is non-zero in the diagonal and then the two random variables are correlated. Consequently, the iso-lines become with values such as $\mu=\begin{pmatrix}3\\2\end{pmatrix},\Sigma=\begin{bmatrix}10 & 5\\5 & 5\end{bmatrix}$:
	\begin{figure}[H]
		\centering
		\includegraphics{img/arithmetics/law_bivariate_normal_isolines_correlation.jpg}
		\caption{Plot of the iso-lines of the bivariate Normal function (correlated case)}
	\end{figure}
	So the correlation rotates the axis of the ellipses! Note that we have therefore:
	
	and thus verbatim:
	
	Recall that we saw during our study of the correlation coefficient that (well... normally... the $R$ notation for the correlation is used only if the variances are estimated but as it is the most common notation in practice we will still us it...):
	
	Thus:
	
	and the exponent of the exponential of the bivariate Normal takes a form that we can found very often in the literature:
	
	Note that if the random variables are centered reduced, then we have:
	
	and thus the exponent of the exponential of the bivariate Normal distribution becomes:
	
	Thus, the density function of the bivariate Normal centered reduced distribution will be written:
	Now consider the important case in engineering, astronomy and quantum physics by returning to the following notation:
	
	Thus, we can see that a bivariate Normal reduced centered distribution function normal can be constructed by the multiplication of two Normal centered and reduced distributions themselves multiplied by a term that depends mainly on the correlation parameter. The latter term includes the nature of the dependence of the two random variables and provides the link between the marginal distributions (both Normal centered and reduced) to obtain the joint bivariate Normal distribution.

	If necessary (this can be very useful in practice), here is the Maple 4.00b code to plot a bivariate Normal function (taking the last example) even if it is also very simple to do with a spreadsheet software like Microsoft Excel:

	\texttt{>f:=(x,y,rho,mu1,mu2,sigma1,sigma2)->(1/(2*Pi*sqrt(sigma1*sigma2*(1-rho\string^2))))}\\
	\texttt{*exp((-1/(2*(1-rho\string^2)))*(((x-mu1)/sqrt(sigma1))\string^2+((y-mu2)/sqrt(sigma2))\string^2}\\
	\texttt{-2*rho*((x-mu1)/sqrt(sigma1))*((y-mu2)/sqrt(sigma2))));}\\

	\texttt{>plot3d(f(x,y,5/sqrt(10*5),3,2,10,5),x=-4..10,y=-4..9,grid=[40,40]);}

	and for the plot with the iso-lines:

	\texttt{>with(plots):}\\
	\texttt{>contourplot(f(x,y,5/sqrt(10*5),3,2,10,5),x=-4..10,y=-4..9,grid=[40,40]);}

	and we can check that it is a probability density function by writing:

	\texttt{>int(int(f(x,y,5/sqrt(10*5),3,2,10,5),x=-infinity...+infinity)}\\
	\texttt{,y=-infinity...+infinity);}

	or calculate the cumulative probability between two intervals:

	\texttt{>evalf(int(int(f(x,y,5/sqrt(10*5),3,2,10,5),x=-3...+4),y=-5...+2));}

	\paragraph{Normal Reduced Centered Distribution}\mbox{}\\\\
	
	The Gauss-Laplace distribution is not tabulated as we must then have so many numerical tables as possible values for the mean $\mu$ and standard deviation $\sigma$ (which are the parameters of the function as we have seen it).

	Therefore, by a change of variable, the Normal distribution becomes the Normal reduced centered distribution more often named the "\NewTerm{standard Normal distribution}\index{Standard Normal distribution}" where:
	\begin{enumerate}
		\item "Centered" refers to subtracting the mean $\mu$ to the measures (thus the distribution function is symmetric to the vertical axis).
		
		\item "Reduced" refers to the division by the standard deviation $\sigma$ (thus the distribution function has a unit variance).
	\end{enumerate}
	By this change of variable, the variable k is replaced by the reduced centered random variable:
	
	If the variable $k$ has for mean $\mu$ and standard deviation $\sigma$ then the variable $k^{*}$ has a mean of $0$ and standard deviation of $1$ (this last variable is usually denoted by the letter $Z$).
	
	Thus the relation:
	
	is therefore written (trivially) more simply:
	
	
	which is just the explicit expression of the reduced centered Normal distribution ("standard Normal") often denoted $\mathcal{N}(0,1)$ which we will find very often in the sections of physics, finance, quantitative management and engineering!
	
	\begin{tcolorbox}[title=Remark,colframe=black,arc=10pt]
	Calculate the integral of the previous relation for an interval can not be done accurately formally speaking. One possible and simple idea is then to express the exponential in a Taylor series and then be integrated term by term of the series (making sure to take enough terms for convergence!).
	\end{tcolorbox}	
	
	\paragraph{Henry's Line}\mbox{}\\\\
	Often in business it is the Gauss-Laplace (Normal) distribution that is analyzed but common and easily accessible software like Microsoft Excel are unable to verify that the measured data follow a Normal distribution when we do the frequency analysis (there are no default integrated tool allowing users to check this assumption) and we do not have the original ungrouped data.

	The trick then is then to use the reduced centered variable that is build as we have see above with the following relation:
	
	The idea of the Henry's Line is then to use the linear relation between $k$ and $k^*$ given by the equation of the line:
	
	
	\begin{tcolorbox}[colframe=black,colback=white,sharp corners]
\textbf{{\Large \ding{45}}Example:}\\
	Suppose we have the following frequency analysis of 10,000 receipts in a supermarket:
	
	If we now plot this in Microsoft Excel 11.8346 we get:
	\begin{figure}[H]
		\centering
		\fbox{\includegraphics{img/arithmetics/distribution_example_henry_law.jpg}}
		\caption{Distribution of receipts amount}
	\end{figure}
	What looks terribly like a Normal distribution, thus the authorization, without too much risk to use in this example the technique of Henry's line.\\

	But what can we do now? Well... now that we know the cumulative frequency, it remains for us to calculate each $k^*$ using numerical tables or the \texttt{NORMSINV( )} function of Microsoft Excel 11.8346 (remember that formal integration of the Gaussian function is not easy...).\\

	This will give us the values of the standard Normal distribution $\mathcal{N}(0,1)$ of these respective cumulative frequencies (cumulative distribution function). So we get (we leave to the reader to take its statistic table or open its favorite software...):
	\end{tcolorbox}
	
	\pagebreak
	\begin{tcolorbox}[colframe=black,colback=white,sharp corners]
	
	Note that in the type of table above, in Microsoft Excel, the null and unit cumulative frequencies will generated some errors. You should then play a little bit...

	As we specified earlier, we have under discrete form:
	
	So graphically in Microsoft Excel 11.8346 we can thanks to our table plot the following chart (obviously we could do strictly a linear regression in the rules of art as seen in the chapter of Numerical Methods with confidence, prediction intervals and other stuffs...):
	\begin{figure}[H]
		\centering
		\fbox{\includegraphics{img/arithmetics/linearized_distribution_for_henry.jpg}}
		\caption{Linearized form of the distribution}
	\end{figure}
	So thanks to the linear regression given by Microsoft Excel 11.8346 (or calculated by you using the techniques of linear regressions seen in the chapter on Numerical Methods). It comes:
	
	\end{tcolorbox}
	
	\pagebreak
	\begin{tcolorbox}[colframe=black,colback=white,sharp corners]
	we immediately deduce that:
	
	This is thus a particular technique for a particular distribution! Similar techniques more or less simple (or complicated depending on the case...) exist for others distributions.

	See now another approximate approach to solve this problem. Let take us again our table for this example:
	
	The average is now calculated using the central value of the intervals and sample sizes according to the relation we have seen at the beginning of this section:
	
		
	The average that we have calculated yet is also quite close to the average obtained previously with the Henry's line.
	\end{tcolorbox}
	
	\pagebreak
	\begin{tcolorbox}[colframe=black,colback=white,sharp corners]

	
	The standard deviation will now be calculated using also the central value of the intervals and sample sizes according to the relation seen at the beginning of this chapter:
	
	
	The standard deviation that we have calculated yet is also quite close to the standard deviation obtained with the method of the Henry's line.
	\end{tcolorbox}
	
	\paragraph{Q-Q plot}\mbox{}\\\\
	Another way to judge of the quality of fit of experimental data with a theoretical distribution (whatever that is!) is the use of a "\NewTerm{quantile-quantile plot}\index{quantile-quantile plot}" or simply named "\NewTerm{q-q plot}\index{q-q plot}".

	The idea is pretty simple, it based on the comparison the experimental data relatively to the theoretical data that are supposed to follow a particular distribution. Thus, in the case of our example, if we take the values of the mean ($\sim 200$) and standard deviation ($\sim 100$) obtained with the Henry's line as theoretical parameters for the Normal distribution, we get:
		
	Plotted, this gives us the famous Q-Q plot:
	\begin{figure}[H]
		\centering
		\fbox{\includegraphics{img/arithmetics/q_q_plot.jpg}}
		\caption{Q-Q plot of the distribution}
	\end{figure}
	And of course we can compare the observed quantiles with the supposed theoretical distribution. More the points will be aligned on the line of unit slope and zero intercept origin, the better will be the fit! It's very visual, very simple and widely used by non-specialists in business statistics.
	
	\pagebreak
	\subsubsection{Log-Normal Distribution}
	We say that a positive random variable $X$ follows a "\NewTerm{log-normal function}\index{log-normal function}" (or "\NewTerm{log-normal distribution}\index{log-normal distribution}") if by writing:
	
	we see that $y$ follows a Normal distribution of mean $\mu$ and variance $\sigma^2$ (moments of the Normal distribution). 
	
	Verbatim by the properties of logarithms, a variable can be modeled by a log-normal distribution if it results of the multiplication of many small independent factors (property of the product in sum of the logarithms and stability of the Normal distribution by the addition).
	
	The density function of $X$ for $x \geq 0$ is then (\SeeChapter{see section Differential And Integral Calculus}):
	
	that can be calculated in Microsoft Excel 11.8346 with the \texttt{LOGNORMDIST( )} function or its inverse by \texttt{LOGINV( )}.

	This type of scenario is frequent in physics, in technical maintenance or financial markets in the options pricing model (see the respective sections of the book for various application examples). There is also an important remark with respect to the log-normal distribution further when we will develop the central limit theorem!

	Let us show that the cumulative probability function corresponds to a Normal distribution if we make the change of variables mentioned above:
	
	by writing:
	
	and (by definition):
	
	we then get:
	
	So we found again the Normal distribution!

The mean (average) of $X$ is then given by (the natural logarithm being not defined for $x<0$ we start the integral from zero):
	
	where we performed the change of variable:
	
	The expression:
	
	moreover being equal to:
	
	the last integral also becomes:
	
	and where we used the property that emerged during our study of the Normal distribution, that is to say that any integral of the form:	
	
	always has the same value!
	
	To calculate the variance, recall that for a random variable $X$, we have the Huygens theorem:
	
	
	Let us calculate $\text{E}(X^2)$ by performing similarly to previous developments:
	
	where once again we have the change of variable:
	
	and where we transformed the expression:
	
	as:
	
	Then:
	
	Here is a plot example of the distribution and cumulative distribution of the Log-Normal function of parameters $(\mu,\sigma)=(0,1)$:
	\begin{figure}[H]
		\centering
		\includegraphics{img/arithmetics/law_log_normal.jpg}
		\caption{Log-Normal law (mass and cumulative distribution function)}
	\end{figure}
	
	\subsubsection{Continuous Uniform Distribution}
	Let us choose $a<b$. We define the continuous uniform distribution function or "\NewTerm{uniform function}\index{uniform distribution}" by the relation:
	
	where $1_{[a,b]}$ means that outside the domain of definition $[a, b]$ the distribution function is zero. We will find this type of notation later in some other distribution functions.
	
	So we have for the cumulative distribution function:
	
	It is indeed a distribution function because it satisfies (simple integral):
	
	The continuous uniform function has for expected mean:
	
	and for the variance using the Huygens theorem:
	
	Here is a plot example of the distribution and cumulative distribution of the continuous uniform function of parameters $(a,b)=(0,1)$:
	\begin{figure}[H]
		\centering
		\includegraphics{img/arithmetics/law_uniform_continuous.jpg}
		\caption{Uniform continuous law (mass and cumulative distribution function)}
	\end{figure}
	\begin{tcolorbox}[title=Remark,colframe=black,arc=10pt]
	This function is often used in business simulation to indicate that the random variable has equal probabilities to have a value within a certain interval (typically in portfolio returns or in the estimation of project durations). The best example of application is again CrystalBall or @Risk software that integrate with Microsoft Project. 
	\end{tcolorbox}

	Let us see an interesting result of the continuous uniform distribution (and that applies also to the discrete one as well...).

	I often hear managers (who consider themselves at high level) that if we have a measure with an equal probability to occur in a closed given interval, then the sum of two such independent random variables have also the same equal probability in the same interval!

	Now we will prove here that this is not the case (if someone has a more elegant proof I'm interested)!
	\begin{dem}
	Consider two independent random variables $X$ and $Y$ that follow a uniform distribution in a closed interval $[0, a]$. We are searching the density of their sum will be written:
	
	Then we have:
	
	with the variable:
	
	To calculate the distribution of the sum, remember that we know that in discrete terms this is equivalent to the joint product of probabilities (\SeeChapter{see section Probabilities}) of the occurrence of two continuous variables (remember the same kind of calculation in the discreet form!).

	That is to say:
	
	As $f_Y(y)=1$ if $0\leq y \leq a$ and $0$ otherwise then the product of the previous convolution reduces to:
	
	The integrand is by definition $0$ except by construction in the interval $0\leq z-y \leq a$ it is then $1$.
	
	Let us focus on the limits of the integral that is in this case the only one that is interesting ....

	First we make a change of variables by writing:
	
	thus:
	
	The integral can be then written in this interval after the change of variable:
	
	Remembering that we have seen at the beginning that $0\leq z \leq 2a$, then we have immediately if $z<0$ and $z >2a$ that the integral is zero.
	
	We will consider two cases for the interval because the convolution of these two rectangular functions can be distinguished according to the situation where at first they cross (nest), that is to say where $0\leq z \leq a$, and then recede from each other, that is to say $a< z \leq 2a$.
	
	\begin{itemize}
		\item In the first case (nest) where $0\leq z \leq a$:
		
		where we changed the lower bound to $0$ because anyway $f_X(u)$ is zero for any negative value (and when $0\leq z \leq a$,$z-a$ is precisely zero or negative!).
		
		\item In the second case (dislocation) where $a< z \leq 2a$:
		
		where we changed the upper terminal $a$ because anyway $f_X(u)$ is zero for any higher value (and when $a \leq  z \leq 2a$, $z$ is just larger than $a$).
		
		So in the end, we have:
		
	\end{itemize}
	\begin{flushright}
		$\square$  Q.E.D.
	\end{flushright}
	\end{dem}
	This is a particular case, deliberately simplified, of the triangular distribution that we will discover just after...

	This result (which may seems perhaps not intuitive) can be check in a few seconds with a spreadsheet software like Microsoft Excel 11.8346 using the \texttt{RANDBETWEEN()} and the \texttt{FREQUENCY( )} functions.
	
	\pagebreak
	\subsubsection{Triangular Distribution}
	
	Let $a<c<b$.We define the "\NewTerm{triangular distribution}\index{triangular distribution}" (or "\NewTerm{triangular function}\index{triangular function}") by construction based on the following two distribution functions:

	
	
	where $a$ is often assimilated with the optimistic value, $c$ to the modal value and $b$ the pessimistic value.

	It is also the only way to write this distribution function if the reader keeps in mind that the base of a triangle of lenght $c-a$ must have a height $h$ equal to $2/(c-a)$ as its total area is equal to unity (we will soon prove it).

	Here is a plot example of the triangular distribution and cumulative distribution for the parameters $(a, c, b) = (0, 3, 5)$:
	\begin{figure}[H]
		\centering
		\includegraphics{img/arithmetics/law_triangular.jpg}
		\caption{Triangular law (mass and cumulative distribution function)}
	\end{figure}
	The slope of the first straight line (increasing from left) is obviously:
	
	and the slope of the second straight line (decreasing to the right):
	
	This function is a distribution function if it satisfies:
	
	It is in this case, simply the area of the triangle which we recall is simply the base multiplied by the height divided by $2$ (\SeeChapter{see section Geometric Shapes}):
	
	\begin{tcolorbox}[title=Remark,colframe=black,arc=10pt]
	This function is widely used in project management in the context of task duration estimations or in industrial simulations. Where $a$ corresponds to the optimistic value, $c$ to the expected value (mode) and the value $b$ to the pessimistic value. The best example of application is again the softwares CrystalBall or @Risk that are add-ins for Microsoft Project.
	\end{tcolorbox}
	The triangular function has also for mean (average):
	
	
	and for variance:
	
	We can replace $\mu$ by the result obtained before and we get after simplification (it is boring algebra...):
	
	We can show that the sum of two independent random variables, each uniformly distributed on $[a, b]$ (i.e. independent and identically distributed) follows a triangular distribution on $[2a, 2b]$ but if they do not have the same limits, then their sum gives something that has no name to my knowledge...
	
	\subsubsection{Pareto Distribution}
	The "\NewTerm{Pareto distribution}\index{Pareto distribution}" (or "\NewTerm{Pareto law}\index{Pareto law}"), also named "\NewTerm{power law}\index{power law (statistics)}" or "\NewTerm{scale law}\index{scale law (statistics)}" is the formalization of the $80-20$ principle. This decision tool helps determine the critical factors (about $20\%$) influencing the majority ($80\%$) of the goal.
	
	\begin{tcolorbox}[title=Remark,colframe=black,arc=10pt]
	This distribution is a fundamental and basic tool in quality management (see Industrial Engineering and Quantitative Management sections). It is also used in reinsurance. The theory of queues had also some interest in this distribution when some research in the 1990s showed that this distribution also seems ton explain well a number of variables observed in the Internet traffic (and more generally on all high speed data networks).
	\end{tcolorbox}
	
	A random variable is said by definition follow a Pareto distribution if its cumulative distribution function is given by:
	
	with $x$ that must be greater than or equal to $x_m$.
	\\
	The Pareto density function (distribution function) is then given by:
	
	with $k\in \mathbb{R}_+$ and $x \geq x_m \geq 0$ (then $x>0$).
	The Pareto distribution is defined by two parameters, $x_m$ and $k$ (Named  "\NewTerm{Pareto index}\index{Pareto index}"). This distribution is also said to be "\NewTerm{scale invariant}\index{scale invariant}" or "\NewTerm{fractal distribution}\index{fractal distribution}", because of the following property:
	
	The Pareto function is also well a distribution function as the cumulative distribution function known we have:
	
	The expected mean is given by:
	
	if $k>1$. If $k\leq 1$, the mean does not exist.
	To calculate the variance, using the Huygens theorem:
	
	we get:
	
	if $k>2$. If $k\leq 2$, $\text{E}(X^2)$ doesn't exists. So if $k>2$:
	
	If $k\leq 2$, the variance doesn't exists.
	Here is a plot example of the Pareto distribution and cumulative distribution for the parameters $(x,x_m,k)=(x,1,2)$:
	\begin{figure}[H]
		\centering
		\includegraphics{img/arithmetics/law_pareto.jpg}
		\caption{Pareto law (mass and cumulative distribution function)}
	\end{figure}
	\begin{tcolorbox}[title=Remark,colframe=black,arc=10pt]
	See that when $k\rightarrow +\infty$ the distribution approaches $\delta (x-x_m)$ where $\delta$ is the Dirac delta function. 
	\end{tcolorbox}
	There is another important way to deduce the family of Pareto distributions that allows us to understand many things about other distributions and that is often presented as follows:
	
	Let us write $x_0$ the threshold beyond which we calculate the mean of the considered quantity, and $\text{E}(Y)$ the mean beyond this threshold $x_0$ as it is proportional (linearly dependent) to the chosen threshold:
	
	This functional relation expresses the idea that the conditional mean beyond the threshold $x_0$ is a multiple of this threshold plus a constant, that is to say a linear function of the threshold.

	Thus, in project management, for example, we could say that once a certain threshold of time is exceeded, the expected duration is a multiple of this threshold plus a constant.

	If a linear relation of this type exists and is satisfied, then we talk about a probability distribution in the form of a generalized Pareto distribution.

	Consider the mean of the Bayesian conditional function given by (\SeeChapter{see section Probabilities}):
	
	If we write $F(y)$ the cumulative distribution function $f(y)$, then we have by definition:
	
	Thus:
	
	and if we define:
	
	what we can assimilate to the "tail of the distribution".
	
	We get:
	
	and therefore we seek the very special case where:
	
	this is to say:
	
	Differentiating with respect to $x$, we find:
	
	The derivative of the integral defined above will be the derivative of a constant (valorisation of the integral in $+\infty$) minus the derivative of the analytical expression of the integral for $x_0$. So we have:
	
	Thus:
	
	and as:
	
	it comes:
	
	After simplification and rearrangement we obtain:
	
	which is a differential equation in $\bar{F}(x)$. Its Resolution provides all forms of seek Pareto distributions, according to the values taken by the parameters $a$ and $b$.
	
	To solve this differential equation, consider the special case where $a>1,b=0$. Then we have:
	
	By writing:
	
	We then get:
	
	and therefore:
	
	It comes:
	
	and therefore:
	
	We have:
	
	Then it comes form the cumulative distribution function:
	
	If we seek for the distribution function, we derive by $x$ to get:
	
	This is the Pareto distribution we have used since the beginning and named "\NewTerm{Pareto distribution of type I}\index{Pareto distribution of type I}" (we won't see in this book those of type II).
	
	An interesting thing to observes is the case of the resolution of the following differential equation:
	
	when $a=1,b>0$. The differential equation is then reduced to:
	
	Thus:
	
	After integration:
	
	and therefore:
	
	If we make a small change in notation:
	
	and that we write the distribution function:
	
	and by derivating we get the distribution function of the exponential distribution:
	
	So the exponential distribution has a conditional mean threshold that is equal to:
	
	So the conditional mean threshold is equal to itself plus the standard deviation of the distribution.

	\pagebreak
	\subsubsection{Exponential Distribution}
	We define the "\NewTerm{exponential distribution}\index{exponential distribution}" (or "\NewTerm{exponential law}\index{exponential law}") by the following distribution function:	
	
	with $\lambda > 0$ that as we will immediately see is in the fact that the inverse of the mean and where $x$ is a random variable without memory. This law is also sometimes denoted $\mathcal{E}(\lambda)$.
	
	In fact the exponential distribution naturally appears from simple developments (see the Nuclear Physics chapter for example) under assumptions that impose a constance in the aging of phenomenon. In the section of Quantitative Management, we have also proved in detail in the section on the theory of queues, that this law was without memory. That is to say, that the cumulative probability of a phenomenon occurs between the time $t$ and $t + s$, if it is not realized before, is the same as that the cumulative probability of occurring between the time $0$ and $s$.
	
	\begin{tcolorbox}[title=Remarks,colframe=black,arc=10pt]
	\textbf{R1.} This function is occuring frequently in nuclear physics (see chapter of the same name) or quantum physics (see also chapter of the same name) as well in reliability (see Industrial Engineering) or in the theory of queues (\SeeChapter{see section Quantitative Management}).\\

	\textbf{R2.} We can get this distribution in Microsoft Excel 11.8346 with the \texttt{EXPONDIST( )} function.
	\end{tcolorbox}	
	
	It is also really a distribution function because it verifies:
	
	The exponential distribution has for expected mean using integration by parts:
	
	and for variance using once again the Huygens relation:
	
	it remains for us to only the to calculate:
	
	A variable change $y=\lambda$ leads us to:
	
	A double integration by parts gives us:
	
	Hence:
	
	we have therefore:
	
	So the standard deviation (square root of the variance for recall) and mean have exactly the same expression!

	Here is a plot example of the exponential distribution and cumulative distribution for the parameter $\lambda=1$:
	\begin{figure}[H]
		\centering
		\includegraphics{img/arithmetics/law_exponential.jpg}
		\caption{Exponential law (mass and cumulative distribution function)}
	\end{figure}
	Now let us determine the distribution function of the exponential law:
	
	\begin{tcolorbox}[title=Remark,colframe=black,arc=10pt]
		We will see later that the exponential distribution is a special case of a more general distribution which is the chi-square distribution, the chi-square is also a special case of a more general distribution that is the Gamma distribution. This is a very important property used in the "Poisson test" for rare events (see also below). 
	\end{tcolorbox}
	
	\subsubsection{Cauchy Distribution}
	Let $X, Y$ be two independent random variables following a Normal reduced centered distribution (with zero mean and unit variance). Thus the density function is given for each variable by:
	
	The random variable:
	
	(the absolute value will be useful in an integral during a change of variable) follows a characteristic appearance named "\NewTerm{Cauchy distribution}\index{Cauchy distribution}" (or "\NewTerm{Cauchy law}\index{Cauchy law}") or even "\NewTerm{Lorentz law}\index{Lorentz law}".
	
	Let us now determine its density function $f$. To do this, recall that $f$ is determined by the (general) relation:
	
	So (application of elementary differential calculus):
	
	in the case where $f$ is continuous.
	
	Since $X$ and $Y$ are independent, the density function of the random vector is given by one of the axioms of probabilities (\SeeChapter{see section  Probabilities}):
	
	therefore:
	
	where $D=\left\lbrace(x,y)\vert x<t\vert y\vert\right\rbrace$.
	This last integral becomes:
	
	Let us make the following change of variables $x=u\vert y \vert$ in the inner integral. We obtain:
	
	Therefore:
	
	Now the absolute value will be useful to write:
	
	For the first integral we have:
	
	It remains therefore only the second integral and making the change of variable $v=y^2$, we get:
	
	What we will denote thereafter (to respect the notations adopted so far):
	
	and that is just simply the so named "Cauchy distribution".
	It is also a effectively a distribution function because it verifies (\SeeChapter{see section Differential and Integral Calculus}):
	
	It is obvious that we get therefore for the cumulative distribution function:
	
	Here is plot example of the Cauchy distribution:
	\begin{figure}[H]
		\centering
		\includegraphics{img/arithmetics/law_cauchy.jpg}
		\caption{Cauchy law (mass function) }
	\end{figure}
	The Cauchy distribution has for expected mean:
	
	Caution!!!! The above calculations do not give zero in facts because the subtraction of infinite is not zero but indeterminate! The Cauchy distribution therefore and strictly speaking does not admits an expected mean!
	
	Thus, even if we can build a variance:
	
	this is absurd and does not exist strictly speaking as the mean doesn't exists...!
	
	The Cauchy distribution is used a lot in financial engineering as it is heavy tailed and therefore a very good candidate to be more accurate in predicting extreme values at the opposite to the Normal distribution that has the tails decreasing to quick. Further the Cauchy distribution is a heavy tailed law with a support on $\mathbb{R}$ when the Pareto distribution (also heavy tailed) is defined only on $\mathbb{R}^+$.
	
	The Cauchy distribution if one of the most famous distribution function that... we cannot found in the spreadhsheet softwares like Microsoft Excel. To be able to get the closed form of the inverse Cauchy CDF we start from the CDF proven previously:
	
	and therefore if we let:
	
	We immediately get the inverse CDF:
	
	That is useful in finance as we know (\SeeChapter{see section Theoretical Computing}) that to simulate a Cauchy variable when the use the inverse transforme sampling:
	
	
	\subsubsection{Beta Distribution}
	Let us first recall that the Euler Gamma function is defined by the relation (\SeeChapter{see section Differential And Integral Calculus}):
	
	We proved (\SeeChapter{see section Differential And Integral Calculus}) that a non-trivial property of this function is:
	
	Let us now write:
	
	where:
	
	By the change of variables:
	
	we get:
	
	For the internal integral we now use the substitution $v=ut, 0\leq t\leq 1$ and therefore we find:
	
	The function $B$ that appears in the expression above is named "\NewTerm{beta function}\index{beta function}" and therefore we have:
	
	Now that we have defined what we name the beta function, consider the two parameters $a>0,b>0$ and consider also the special relation below as the "\NewTerm{beta distribution}\index{beta distribution}" or "\NewTerm{beta law}\index{beta law}" (there are several formulations of the beta distribution and a very important one is studied in detail in the section of Quantitative  Management):
	
	where:
	
	We first check that $P_{a,b}(x)$ that is effectively a distribution function (without getting into too much details ...)
	
	Let us now calculate the expected mean:
	
	by using the relation:
	
	and its variance:
	
	As we know that:
	
	we find:
	
	and therefore:
	
	Examples of plots of the beta distribution function for $(a,b)=(0.1,0.5)$ in red, $(a,b)=(0.3,0.5)$ in green, $(a,b)=(0.5,0.5)$ in black, $(a,b)=(0.8,0.8)$ in blue, $(a,b)=(1,1)$ in magenta, $(a,b)=(1,1.5)$ in cyan, $(a,b)=(1,2)$ in gray, $(a,b)=(1.5,2)$ in turquoise, $(a,b)=(2,2)$ in yellow,$(a,b)=(3,3)$ in gold color:
	\begin{figure}[H]
		\centering
		\includegraphics{img/arithmetics/law_beta_samples.jpg}
		\caption{Some Beta law mass functions}
	\end{figure}
	Here is a plot example of the beta distribution and cumulative distribution for the parameters $(a,b)=(2,3)$:
	\begin{figure}[H]
		\centering
		\includegraphics{img/arithmetics/law_beta.jpg}
		\caption{Beta law (mass and cumulative distribution function) }
	\end{figure}
	
	\subsubsection{Gamma Distribution}
	The Euler Gamma function being known, consider two parameters $a>0,\lambda>0$ and let us define the "\NewTerm{Gamma distribution}\index{Gamma distribution}" (or "\NewTerm{Gamma law}\index{Gamma law}") as given by the relation (density function):
	
	
	By the change of variables $t=\lambda x$ we obtain:
	
	and we can then write the relation in a more conventional form that we find frequently in the literature:
	
	and it is under this notation that we find this distribution function in Microsoft Excel 11.8346 under the name \texttt{GAMMADIST( )} and its inverse by \texttt{GAMMAINV( )}.

Let us now see a simple property of the Gamma distribution that will be partially useful for the study of the Welch statistical test . First recall that we have shown above that:
	
	
	Let us write $Y=c^{te}X$, then we have immediately:
	
	So the multiplication by a constant of random variable that follows a Gamma distribution has only for effect of dividing the parameter $\lambda$ by the same constant. This is the reason why $\lambda$ is named "NewTerm{scale parameter}\index{scale parameter}".
	
	If $a \in \mathbb{N}$, the Gamma distribution at the denominator becomes (\SeeChapter{see section Differential And Integral Calculus}) the factorial $(a-1)!$. The Gamma function can then be written:
	
	
	This particular notation of the Gamma distribution is named the "\NewTerm{Erlang distribution}\index{Erlang distribution}" that we find naturally in the theory of queues and that is very important in practice!
	
	\begin{tcolorbox}[title=Remark,colframe=black,arc=10pt]
	If $a=1$ then $\Gamma(a)=1$ and $x^{a-1}=1$ and we fall back on the exponential distribution.
	\end{tcolorbox}	
	
	Then we check with a similar reasoning to this of the beta distribution that $P_{a,\lambda}(x)$ is a distribution function:
	
	
	
	
	Examples of plots of the beta distribution function for $(a,\lambda)=(0.5,1)$ in red, $(a,\lambda)=(1,1)$ in green, $(a,\lambda)=(2,1)$ in black, $(a,\lambda)=(4,2)$ in blue, $(a,\lambda)=(16,8)$ in magenta:
	
	\begin{figure}[H]
		\centering
		\includegraphics{img/arithmetics/law_gamma_samples.jpg}
		\caption{Some Gamma law mass functions}
	\end{figure}
	and a plot example of the Gamma distribution and cumulative distribution for the parameters $(a,\gamma)=(4,1)$:
	\begin{figure}[H]
		\centering
		\includegraphics{img/arithmetics/law_gamma.jpg}
		\caption{Gamma law (mass and cumulative distribution function) }
	\end{figure}
	
	The Gamma function has also for expected mean:
	
	
	and for variance:
	
	Let us now prove a property of the Gamma distribution that will permit us later in this chapter, during our study of the analysis of variance and confidence intervals based on small samples, another extremely important property of the Chi-square distribution.
	
	As we know, the distribution function of a random variable following a Gamma function of parameters $a,\lambda>0$ is:
		
	with (\SeeChapter{see section Differential And Integral Calculus}) the Euler Gamma function:
	
	Moreover, when a random variable follows a Gamma function we often notice it in the following way:
	
	Let $X, Y$ be two independent variables. We will prove that if $X=\gamma(p,\lambda)$ and $Y=gamma(q,\lambda)$, hence with the same scale parameter, then:
	
	We write $f$ the density function of the pair $X, Y,f(x)$ the density function of $X$ and $f_Y$ the density function of $Y$. Because $X$ and $Y$ are independent, we have:
	
	for all $x,y>0$.
	
	Let $Z=X+Y$. The distribution function of Z is therefore:
	
	where $D=\left\lbrace(x,y)\vert x+y\leq z \right\rbrace$.
	\begin{tcolorbox}[title=Remark,colframe=black,arc=10pt]
	As we already know we name such a calculation a "\NewTerm{convolution}\index{convolution}" and statisticians often have to handle such entities because they work on many random variables that they have to sum or even to multiply.
	\end{tcolorbox}
	Simplifying:
	
	We perform the following change of variable $x=x,y=s-x$. The Jacobian is therefore (\SeeChapter{see Differential And Integral Calculus}):
	
	Therefore with the new integration limits $s=x+y=x+(z-x)=z$ we have:
	
	If we denote by g the density function Z we have:
	
	Then it follows:
	
	$f_X$ and $f_Y$ being null when the argument is negative, we can change the limits of integration:
	
	Let us calculate $g$:
	
	After the change of variable $x=st$ we obtain:
	
	where $B$ is the beta function we saw earlier in our study of the beta distribution. But we have also proved the relation:
	
	Therefore:
	
	More explicitly:
	
	Which finally gives us:
	
	This shows that that if two random variables follow a Gamma distribution then their sum will also follow a Gamma distribution with parameters:
	
	So the Gamma distribution is stable by addition as are all distribution arising from the Gamma distribution that we will see below.
	
	\subsubsection{Generalized Gamma Distribution}
	The generalized gamma distribution is a continuous probability distribution with three parameters. It is a generalization of the two-parameter gamma distribution. Since many distributions commonly used for parametric models in survival analysis (such as the Exponential distribution, the Weibull distribution and the Gamma distribution, and lognormal) are special cases of the generalized gamma, it is sometimes used to determine which parametric model is appropriate for a given set of data.

	Therefore let us notice that if we write after trials and errors the following density function named "\NewTerm{generalized Gamma law}\index{generalized Gamma law}":
	
	with $x>0$, $\alpha>0$, $\eta>0$, $\kappa>0$.
 
	Then, for $\kappa=1$ we fall back on the density function of Weibull law (\SeeChapter{see section Industrial Engineering}) that is with our own notations of the corresponding section is given by:
	 
	For $\eta=1$, we fall back the Gamma density function just introduced before:
	
	For $\kappa=1$ and $\eta=1$, we fall back on the exponential distribution also seen previously:
	
	and finally for $\eta\rightarrow 0$, $\kappa\rightarrow +\infty$ we fall back on a log-normal distribution after developing the limits using the Stirling, Hospital and Taylor techniques (\SeeChapter{see section Functional Analysis, Sequences and Series, Differential and Integral Calculus}):
	
	As always, on request we can detail the developments!
	
	\subsubsection{Chi-Square (Pearson) Distribution}
	The "\NewTerm{chi-square distribution}\index{chi-square distribution}" (also named "\NewTerm{chi-square law}\index{chi-square law}" or "\NewTerm{Pearson law}\index{Pearson law}") has a very important place in the industrial practice for some common hypothesis tests (see far below...) and is by definition only a particular case of the Gamma distribution in the case where $a=k/2$ and $\lambda=1/2$, when $k$ is a positive integer:
	
	This relation that connects the chi-square distribution with the Gamma distribution is important in the in Microsoft Excel 11.8346 as the function \texttt{CHIDIST( )} returns the confidence level and not the distribution function. Then you must use the function \texttt{GAMMADIST()} with the parameters given above (except that you must take the inverse of $1/2$: also $2$ as parameter) to get the distribution and cumulative functions.
	
	The reader who wishes to check that the Chi-square distribution is only a special case of the Gamma distribution can write in Microsoft Excel 14.0.6123:
	
	\begin{center}
		\texttt{=CHISQ.DIST(2*x,2*k,TRUE)}\\
		\texttt{=GAMMA.DIST(x,k,1,TRUE)}
	\end{center}
	All calculations made previously still apply and we get immediately:
		
	Examples of plots of the chi-2 distribution function for $k=1$ in red, $k=3$ in green, in black, $k=4$ in blue:
	\begin{figure}[H]
		\centering
		\includegraphics{img/arithmetics/law_chi2_samples.jpg}
		\caption{Some Chi-2 $\chi^2$ law mass functions}
	\end{figure}
	and a plot example of the chi-2 distribution and cumulative distribution for the parameter $k=2$:
	\begin{figure}[H]
		\centering
		\includegraphics{img/arithmetics/law_chi2.jpg}
		\caption{Chi-2 $\chi^2$ law (mass and cumulative distribution function) }
	\end{figure}
	In the literature, it is traditional to write:
	
	to indicate that the distribution of the random variable X is a chi-square distribution. Furthermore it is common to name the parameter $k$ "\NewTerm{degree of freedom}\index{degree of freedom (statistics)}" and abbreviate it "$\text{df}$".
	
	The $\chi^2$ distribution is therefore a special case of the Gamma distribution and by taking $k=2$ we also find the exponential distribution (see above) for $\lambda=1/2$:
	
	Moreover, since (\SeeChapter{see section Differential And Integral Calculus}):
	
	the $\chi^2$ distribution with $k$ equal to unity can be written as:
	
	Finally, let us finish with a fairly large property in the field of statistical tests that we will investigate a little further and particularly for confidence intervals of rare events and the famous Fisher methode for multiple $p$-value test. Indeed, the reader can check in a spreadsheet software like Microsoft Excel 14.0.6123 that we have:
	\begin{tabbing}
		\= \\
		\>\texttt{=POISSON.DIST(} $x \in \mathbb{N},\mu,$\texttt{TRUE)}\\
		\>\texttt{=1-CHISQ.DIST(} $2\mu,2(x+1),$\texttt{TRUE)}\\
		\>\texttt{=1-GAMMA.DIST(} $2\mu,x+1,$\texttt{TRUE)}\\
		\>\texttt{=1-EXPON.DIST(} $x,0.5,$\texttt{TRUE)}
	\end{tabbing}
	So we need to prove this relation between law $\chi^2$ and Poisson distributions. See it starting from the Gamma distribution:
	
	If we write $\lambda=1/2$ and $a=k/2$ then we have the $\chi^2$ distribution with $k$ degrees of freedom:
	
	Now remember that we have seen in the section Sequences And Series, the following Taylor (Maclaurin) serie from order $n - 1$ around $0$ to $\lambda$ with integral rest:
	
	We multiply by $e^{-\lambda}$:
	
	And therefore:
	
	Now, let us focus on the term:
	
	and make a first change of variable:
	
	and a second change of variable (caution! the $k$ in the change of variable is not the same as this in the Poisson sum...):
	
	However, we have shown in the section of Differential And Integral Calculus that if $x$ is a positive integer:
	
	Then it comes:
	
	Finally we have:
	
	where we find out the chi-2 distribution under the integral! So at the end:
	
	This explains the formulas given above for the spreadsheet software.
	
	\subsubsection{Student Distribution}
	The "\NewTerm{Student distribution}\index{Student distribution}" (or "\NewTerm{Student's law}\index{Student's law}") of parameter $k$ is defined by the relation:
	
	with $k$ being the degree of freedom of the $\chi^2$ distribution underlying the construction of the Student function as we will see.

Let us indicate that this distribution can also be obtained in Microsoft Excel 11.8346 using the \texttt{TDIST( )} function and its inverse by \texttt{TINV()}.

It is indeed a distribution function because it also satisfies (remains to be proved directly, but as we will see it is the product of two distribution functions thus indirectly...):
	
	Let us see the easiest proof to justify the provenance of the Student distribution and that will also be very useful further in statistical inference and analysis of variance.
	
	\begin{enumerate}
		\item If $X$ and $Y $ are two independent random variables with respective densities $f_X,f_Y$, the distribution of the pair $(X, Y)$ has a density $f$ satisfying (axiom of probabilities!)
		
		\item The distribution $\mathcal{N}(0,1)$ is given by (see above):
		
		\item The distribution $\chi_n^2$ is given by (see above):
		
		for and $y\geq 0$ and $n \geq 1$.
		\item The function $\Gamma$ is defined for all $\alpha>0$ by (\SeeChapter{see section  Differential and Integral Calculus}):
		
		and satisfies (\SeeChapter{see section  Differential and Integral Calculus}):
		
		for $\alpha\geq 2$.
	\end{enumerate}
	These reminders made, now consider a random variable $X$ that follows the distribution $\mathcal{N}(0,1)$ and $Y$ a random variable following the distribution $\chi_n^2$.
	
	We assume $X$ and $Y$ being independent and we consider the random variable (this is at the origin the historical study of the Student distribution in the framework of statistical inference which led to define this variable for which we will deepen the origin later):	
	
	We will prove that $T$ follows a Student distribution of parameter $n$.
	\begin{dem}
	Let $F$ and $f$ and be respectively the repartition and density functions and $T$, $f_X,f_Y$ the density functions of $X, Y$ and $(X, Y)$ respectively. Then we have for all $t \in \mathbb{R}$:
	
	where: 
	
	the imposed positive and non-zero value and $y$ being due to the fact that it is under a root and furthermore at the denominator.
	Thus:
	
	where because $X$ follows a Normal $\mathcal{N}(0,1)$ distribution.
	
	is the Normal centered reduced cumulative distribution.
	Thus, we obtain the density distribution function of $T$ by deriving $F$:
	
	because (the derivative of a function is equal to its derivatives multiplied by its inner derivative):
	
	Therefore:
	
	By making the change of variable:
	
	we get:
	
	what is the Student distribution of parameter $n$.
	\begin{flushright}
		$\square$  Q.E.D.
	\end{flushright}
	\end{dem}
	Let us now prove what is the mean of the Student distribution:
		
	We have:
	
	But $\text{E}\left(\dfrac{1}{\sqrt{Y}}\right)$ exists if and only if $n\geq 2$. Effectively for $n=1$:
	
	and:
	
	Whereas for $n\geq 2$ we have:
	
	Thus, for $n=1$ the mean does not exist.
	
	So for $n\geq 2$:
	
	Now let us see the value of the variance. So we have:
	
	First we will discuss the existence of $\text{E}(T^2)$. We have trivially:
	
	$X$ follows a Normal centered reduced distribution thus:
	
	With regard to $\text{E}\left(\dfrac{1}{Y}\right)$ we have:
	
	where we made the change of variable $u=y/2$.
	But the integral defining $\Gamma\left(\dfrac{n}{2}-1\right)$ converges only if $n\geq 3$.
	Therefore $\text{E}(T^2)$ exists if and only if $n\geq 3$ so it's value is according to the properties of the Euler Gamma function demonstrated in the chapter of Differential And Integral Calculus:
	
	Therefore for $n\geq 3$:
	
	It is also important to note that this law is symmetrical about $0$!

	Plot example of the Student distribution and cumulative distribution for the parameter $k=3$:
	\begin{figure}[H]
		\centering
		\includegraphics{img/arithmetics/law_student.jpg}
		\caption{Student $T$ law (mass and cumulative distribution function) }
	\end{figure}
	
	\subsubsection{Fisher Distribution}
	The "\NewTerm{Fisher distribution}\index{Fisher distribution}" (or "\NewTerm{Fisher-Snedecor distribution}\index{Fisher-Snedecor distribution}") of parameters k and l is defined by the relation:
	
	if $x\geq 0$. The parameters $k$ and $l$ are positive integers and correspond to the two degrees of freedom of the underlying chi-square distributions. This distribution is often denoted by $F_{k,l}$ or by $F(k, l)$ and can be obtained in Microsoft Excel 11.8346 with \texttt{FDIST( )} distribution.
	It is indeed a distribution function because it satisfies the property:
	
	Let us see the easiest proof to justify the provenance of the Fisher distribution and that we will be us also very useful further in statistical inference and analysis of variance.

	For this proof, recall that:
	\begin{enumerate}
		\item The distribution $\chi_n^2$ is given by (see above):
		
		for $y\geq 0$ and $n\geq 1$.
		\item The Euler Gamma function $\Gamma$ is defined for all $\alpha>0$ by (\SeeChapter{see section Differential and Integral Calculus}):
		
		Let $X, Y$ be two independent random variables following respectively the distributions $\chi_n^2$ and $\chi_m^2$.
	\end{enumerate}
	We consider the random variable:
	
	We will prove that the distribution of $T$ is the Fisher-Snedecor distribution of parameters $n, m$.
	
	Let us note for this purpose $F$ and $f$ the distribution and cumulative distribution function of $T$ and $f_X,f_Y,f$ density functions of $X, Y$ and respectively $(X, Y)$. We have for all $t\in \mathbb{R}$:
	
	where:
	
	where the imposed positive values comes in fact that behind them there is a chi-square for $x$ and $y$.
	Therefore:
	
	We obtain the density function of $T$ by deriving $F$. First the inner derivative:
	
	Then explicitly because:
	
	we then have:
	
	By making the change of variable:
	
	we get:
	
	
	\subsubsection{General Folded Normal Distribution}
	The "\NewTerm{folded Normal distribution}\index{folded Normal distribution}" is the distribution of the absolute value of a random variable with a Normal distribution\footnote{The majority of the text below comes from \url{http://www.math.uah.edu/stat/}}. As we have mentioned before, the Normal distribution is perhaps the most important in probability and is used to model an incredible variety of random phenomena. Since one may only be interested in the magnitude of a Normally distributed variable, the folded Normal arises in a very natural way especially in Finance and Industrial Engineering (Design of Experiments). The name stems from the fact that the probability measure of the Normal distribution on $(-\infty, 0]$  is folded over to $[0, \infty)$.
	
	\textbf{Definitions (\#\mydef):} Suppose that $X$ has a Normal distribution with mean $\mu\in\mathbb{R}$ and standard deviation $\sigma\in (0,+\infty)$. Then $Y=|X|$ has the fold normal distribution with parameters $\mu$ and $\sigma$.
	
	Suppose that $Z$ follows the standard Normal distribution. Let us recall that then $Z$ has probability density function $\phi$ and distribution function $\Phi$ given by:
	
	with $z \in \mathbb{R}$.
	
	If $\mu\in\mathbb{R}$ and $\sigma\in [0,+\infty[$, then $X=\mu +\sigma Z$ has the Normal distribution with mean $\mu$ and standard deviation $\sigma$, and therefore it is obvious at this level that:
	 
	has the folded Normal distribution with parameters $\mu$ and $\sigma$.

	Now let us determine the cumulateded probability CDF function of such a variable! For $y\in[0,+\infty[$:
	 
	Since $\Phi(-Z)=1-\Phi(Z)$ we have:
	
	We cannot compute the quantile function $F^{-1}$ in closed form, but values of this function can be approximated. 
	
	It comes therefore immediately that $Y$  has probability density function  $f$ given by:
	 
	This follow from differentiating the CDF with respect to $y$ as we know!
	
	Now as always in this book we will focus only in what we need for the applications in the other chapters! So as we don't need the moments of the folder Normal distribution we will not calculate them. The only purpose of the above development were to build the tools to be able to introduce a special case of the folder Normal distribution.
	
	\paragraph{Half-normal distribution}\mbox{}\\\\
	In probability theory and statistics, the "\NewTerm{half-normal distribution}\index{half-normal distribution}" is a special case of the folded Normal distribution.

	Let $X$ follow an ordinary Normal distribution, $\mathcal{N}(0,\sigma ^{2})$, then $Y=|X|$ follows a half-Normal distribution. Thus, the half-normal distribution is a fold at the mean of an ordinary Normal distribution with mean $\mu=0$.
	
	Thus, let $Y=|\sigma Z|=\sigma |Z|$ where $Z$ has a standard Normal distribution and $\sigma\in [0,+\infty[$. Clearly $\sigma$ is a scale parameter, unlike the case for the general folded Normal distribution. The distribution of $Y$ when $\sigma=1$, $Y=|Z|$ has the "\NewTerm{standard half-normal distribution}\index{standard half-normal distribution}".
	
	As the half-Normal distribution is just a special case of the folded Normal distribution with $\mu=0$ it comes immediately:
	 
	with $y\in\mathbb{R}^+$ and sometimes denotes $\mathcal{HN}(0,\sigma^2)$.
	
	Now what interest us for the others chapters of this book (especially Design of Experiments) are the moments of that latter!
	
	To calculate them, first remember that:
	
	with $z\in\mathbb{R}$.	Therefore it is immediate that:
	
	Also remember that we have already proved that:
	
	
	We first need to determiner the moments for the Normal distribution! So for $n\in\mathbb{N}^+$:
	
	Now we integrate by parts (\SeeChapter{see section Differential and Integral Calculus}), with $u=z^n$ and $\mathrm{d}v=\phi'(z)\mathrm{d}z$ to get:
	
	Therefore for $n\in\mathbb{N}$, with $n> 1$:
	
	The moments of the standard normal distribution are now easy to compute. First we know that:
	\begin{itemize}
		\item $\text{E}(Z)=0$
		\item $\text{E}(Z^2)=1$
	\end{itemize}
	Therefore:
	\begin{itemize}
		\item $\text{E}(Z)=0$
		\item $\text{E}(Z^2)=1$
		\item $\text{E}(Z^3)=\text{E}(Z^{2+1})=2\cdot\text{E}(Z^{2-1})=2\cdot\text{E}(Z)=0$
		\item $\text{E}(Z^4)=\text{E}(Z^{3+1})=3\cdot\text{E}(Z^{3-1})=3\cdot\text{E}(Z^2)=3\cdot 1=1\cdot 3=3$
		\item $\text{E}(Z^5)=\text{E}(Z^{4+1})=4\cdot\text{E}(Z^{4-1})=4\cdot\text{E}(Z^3)=4\cdot 0=0$
		\item $\text{E}(Z^6)=\text{E}(Z^{5+1})=5\cdot\text{E}(Z^{5-1})=5\cdot\text{E}(Z^4)=5\cdot 3=1\cdot 3\cdot 5=15$
		\item $\text{E}(Z^7)=\text{E}(Z^{6+1})=6\cdot\text{E}(Z^{6-1})=6\cdot\text{E}(Z^5)=6\cdot 0=0$
		\item $\text{E}(Z^8)=\text{E}(Z^{7+1})=7\cdot\text{E}(Z^{7-1})=7\cdot\text{E}(Z^6)=7\cdot 15=1\cdot 3\cdot 5\cdot 7$
		\item $\ldots$
	\end{itemize}
	Therefore we see that for the odd powers, that is to say $Z^{2n+1}$ with $n\in\mathbb{N}$ then:
	
	and for even powers:
	
	It follows for $X=\mathcal{N}(0,\sigma)$ (just check with the special case of $n=0$ and $n=1$):
	
	The moments of the half-normal distribution can now be computed explicitly. 
	
	First it should be quite obvious by construction that the even order moments of $Y$ are the same as the even order moments of $\sigma Z$ (in both case the values are all positive and therefore equal). Hence:
	
	For the odd order moments we must use (see above):
	 
	with $x\in\mathbb{R}^+$. Therefore by definition:
	
	hence:
	
	Now we make the change of variable $u=y^2/(2\sigma^2)$, therefore first we have (this is obvious):
	
	and:
	
	Therefore we have so far:
	
	So finally:
	
	We recognize in this expression the Gamma Euler function integral (\SeeChapter{see section Differential and Integral Calculus})! Therefore is is immediate that:
	
	So as summary:
	
	So finally we get the result we need for some properties of the Brownian motion in finance:
	
	But still one property is missing and now for our needs in the section of Industrial Engineering (Design of Experiments): the value of the Median!
	
	So let $M_e$ denote the median of the half-normal distribution. Then by definition if follow:
	
	Substituting $y/(\sqrt{2}\sigma)=u$ we have
	
	We recognize here the Error function (\SeeChapter{see section Thermodynamics}). Therefore:
	
	Therefore:
	
	A spreadsheet software like Microsoft Excel give us for the complementary error function\index{error function}\index{complementary error function}:
	\begin{center}
	\texttt{=ERFC(0.5)=0.479500122}
	\end{center}
	Therefore:
	
	The technique that we will see in the section Industrial Engineering makes the approximation that therefore:
	
	indeed... it's engineering...
	
	\pagebreak
	\subsubsection{Benford Distribution}
	This distribution was discovered first in 1881 by Simon Newcomb, an American astronomer, after he saw that the wear (and so the use) of the preferred first pages of logarithms tables (at this time there we compiled into books). Frank Benford, around 1938 remarked at his turn this unequal wear, believing he was the first to formulate this law that unduly bears his name today and arrived at the same results after having listed tens of thousands of data (lengths of rivers , stock quotes, etc.).

	There is also one possible explanation: we need more often to extract the logarithm of numbers starting with $1$ that numbers starting with $9$, implying that the first are in "bigger quantity" than the second one.

	Although this idea may seem to him quite implausible, Benford began to test his hypothesis. Nothing more simple: he study tables of numerical values and calculates the percentage of occurrence of the left-most digit (first decimal). The results obtained confirm his intuition:
	
	From these data, Benford found experimentally that the cumulative probability of a number beginning with the digit $n$ (except $0$) is (we will prove this later) is given by the relation:
	
	named "\NewTerm{Benford distribution}\index{Benford distribution}" (or "\NewTerm{Benford law}\index{Benford law}").
	
	Here is a Maple plot of the previous function:
	\begin{figure}[H]
		\centering
		\includegraphics{img/arithmetics/benford.jpg}
		\caption{Plot of the Benford function (cumulative distribution function)}
	\end{figure}
	It should be noted that this distribution applies only to lists of values that are "natural", that is to say numbers with physical meaning. It obviously does not work on a list of numbers randomly drawn.

	The Benford distribution has been tested on all kinds of tables: length of the rivers of the world, country area, election results, price list of grocery store ... It is true almost every time.

	The distribution is said to to be independent of the selected unit. If we take for example a supermarket price list , it also works well with the costs expressed in dollars as with the same costs converted into Euros.

	This strange phenomenon remained unexplained and little studied until quite recently. Then a general proof was given in 1996, which uses the central limit theorem.

	As surprising as it may seem, this distribution has found application: it is said that the IRS use it to detect false statements. The principle is based on the restriction seen above: Benford's distributions applies only to values with physical meaning.

	Thus, if there is a universal probability distribution $P(n)$ on such numbers, they should be invariant under scaling such that:
	
	If:
	
	Then:
	
	and the normalization of the distribution gives:
	
	If we derivate $P(kn)=f(k)P(n)$ with respect to $k$ we obtain:
	
	choosing $k = 1$ we have:
	
	This differential equation has for solution:
	
	This function is not strictly speaking a distribution function (it diverges) and secondly, the physics and human laws impose limits.

	So we have to compare this distribution with respect to an arbitrary reference. Thus, if the decimal number studied contains power of $10$ ($10$ in total: $0,1,2,3,4,5,6,7,9$) the probability that the first nonzero digit (decimal) is $D$ is also given by:
	
	The limits of the integral are from $1$ to $10$ because the null value is prohibited.

	The integral in the denominator gives:
	
	The integral in the numerator gives:
	
	Finally:
	
	By the properties of logarithms (\SeeChapter{see section Functional Analysis}) we have:
	
	However, the Benford's distribution applies not only to non-scaling data but also to numbers from any sources. Explain this case involves a more rigorous investigation using the central limit theorem. This demonstration was conducted only in 1996 by T. Hill by an approach using the distribution of distributions.

	To summarize an important part of everything we've seen so far, the picture below is very useful because it summarizes the relation between 76 most common univariate distributions 76 (57 continuous and 19 discrete):
	\begin{figure}[H]
		\centering
		\includegraphics[scale=1.5]{img/arithmetics/distributions.jpg}
		\caption{Relations between distributions (Source: AMS Lawrence M. Leemis and Jacquelyn T. McQueston)}
	\end{figure}
	
	\subsection{Likelihood Estimators}
	What follows is of extreme importance in the field of statistics and is used widely in practice. It is important therefore to pay attention! Besides the fact that we will use this technique in this chapter, we shall find it in the chapter of Numerical Methods for advanced and generalized linear regression and also in the chapter of Industrial Engineering in the context of parametric estimation of reliability.
	
	We assume that we have observations $x_1,x_2,x_3,...,x_n$ which are realizations of unbiased independent random variables (in the sense that they are randomly selected from a batch) $X_1,X_2,X_3,...,X_n$ of a unknown probability distribution but having the same one.
	
	Suppose we proceed by trial and error to estimate the unknown probability distribution $P$. One way to proceed is to ask if the observations $x_1,x_2,x_3,...,x_n$ had a high probability to get out or not with this arbitrary probability distribution $P$.

	We need for this to calculate the joint probability that the observations $x_1,x_2,x_3,...,x_n$ had to get out with the probabilities $p_1,p_2,p_3,...,p_n$. This joint probability is equal to (\SeeChapter{see section Probabilities}):
	
	noting by the letter $P$ the assumed probability distribution associated to $p_1,p_2,p_3,...,p_n$. You must admit that it would be particularly awkward, at the intuition level of risk, to choose a probability distribution (with its parameters!) that minimizes this quantity...
	
	Instead, we will seek the probabilities $p_1,p_2,p_3,...,p_n$ (or the associated parameters of the probability distribution) that maximizes $\prod_{i=1}^n P(X_i=x_i)$, that is to say, that makes the observations $x_1,x_2,x_3,...,x_n$ the most likely possible.

	This leads us to seek the parameter(s) $\theta$ that maximizes the quantity:
	 
	and where the parameter $\theta$ is often in undergraduate school level problems a first order moment (mean) or second order moment (variance).

	The quantity $L$ is named "likelihood". It is a function of the parameter(s) $\theta$ and observations $\theta$.

	The value(s) of the parameter(s) $\theta$ that maximize the likelihood $L_n(\theta)$ are named "\NewTerm{maximum likelihood estimators}\index{maximum likelihoold estimators}" (MLE estimators).

	In the very special case but useful of the Normal distribution, one of the parameters $\theta$ will be the variance (see a little further concrete example) and can be considered intuitive to the physicist that to maximize the probability, the standard deviation should be as small as possible (so that the maximum numbers of events are in the same interval). Thus, when we calculate an MLE which is the smallest among several possible, then we are talking about a UMV estimator for "\NewTerm{Uniform Minimum Variance Unbiased}\index{Uniform minimum variance unbiased}" because their own variance should be as small as possible. This can be demonstrated (but the proof is not very elegant) using the definition of the Fisher Information and the Fréchet theorem (or Rao-Cramer) that makes use of the Cauchy-Schwartz inequality (\SeeChapter{see section Vector Calculus}) and the analogy between mean and scalar product ... This demonstration will not be in this book.

	Let us still do five small examples (very classic, useful and important in the industry) with in order of importance (i.e. not necessarily in order of ease...) the distribution function of Gauss-Laplace (Normal distribution), the Poisson distribution, the binomial distribution (and so the Geometric distribution), the Weibull distribution and finally the Gamma distribution.
	
	\begin{tcolorbox}[title=Remark,colframe=black,arc=10pt]
	These five examples are important as used in SPC (statistical process control) in various international companies around the world (\SeeChapter{see section Industrial Engineering}).
	\end{tcolorbox}	
	
	\subsubsection{Normal Distribution MLE}
	Let be $x_1,x_2,...,x_n$ an $n$-sample of identically distributed random variables assumed to follow a Gaussian-Laplace (Normal) distribution of parameters $\mu$ and $\sigma^2$.

	We are looking what are the values of the maximum likelihood estimators $\vec{\theta}$ that maximize the likelihood $L_n(\theta)$ of the Normal distribution?

	\begin{tcolorbox}[title=Remark,colframe=black,arc=10pt]
	It is trivial that the maximum likelihood estimators vector is here: 
	
	\end{tcolorbox}
	We have prove earlier above that the density of a Gaussian random variable was given by: 
	
	The likelihood is then given by:
	
	Maximize a function or maximize its logarithm is equivalent therefore the "\NewTerm{log-likelihood}\index{log-likelihood}" will be:
	
	To determine the two estimators of the Normal distribution, first let us fix the standard deviation. To do this, we derive $\ln(L(\mu,\sigma))$ over $\mu$ and look for what the average value of the function is equal to zero.
	
	It remains after simplification the following term that is equal to zero:
	
	Thus, the maximum likelihood estimator of the expected mean of the Normal distribution is after rearrangement:
	
	and we see that it is simply the arithmetic mean (or also named "\NewTerm{sample mean}\index{sample mean}").
	
	Let us now fix the mean. The cancellation $\ln(L(\mu,sigma))$ of the derivative over $\sigma$ leads us to:
	
	This allows us to write the maximum likelihood estimator for the standard deviation (the variance when the mean is known under the an assumed distribution also supposed known!):
	
	that some people also name "\NewTerm{Pearson standard deviation}\index{Pearson standard deviation}"...
	
	Even if it is a little bit redundant some people as us to show the proof of the estimator of the covariance matrix (and therefore the correlation matrix).
	
	Remember that we have prove earlier that for the bivariate case we have:
	
	In fact the relation is the same for the multivariate case with $T$!
	
	The log-likelihood is therefore immediate by analogy with the univariate case:
	
	That we can also write (as $\Sigma$ is diagonal and $\Sigma^{-1}$ also):
	
	Where by definition and using the estimator of the mean:
	
	Then we deduce that:
	
	and we get finally:
	
	However, we have not yet defined what is a good estimator! What we mean here is:
	\begin{itemize}
	\item If the mean of an estimator is equal to itself, we say that this estimator is "unbiased" and that's obviously what we want!
	
	\item If the mean of an estimator is not equal to itself, then we say that this estimator is "biased" and is necessarily less good...
	\end{itemize}
	In the previous example, the average is unbiased (this is trivial as the average of the arithmetic mean is equal to itself). But what about the variance (verbatim the standard deviation)?

	A simple little calculation by linearity of the mean (since the random variables are identically distributed) will give us the answer in the case where the theoretical average (mean) is approximated as in practice (industry) by the estimator of the mean (most common case).

	So we have for the calculation of the mean of the "\NewTerm{sample variance}\index{sample variance}":
	
	However, as the variables are supposed to be identically distributed:
	
	And as we have (Huyghens theorem):
	
	wherein the second relation can be written only because we use the maximum likelihood estimator of the average (empirical average). 
	Therefore combining the two above relations with the prior-previous 
 one we get:
	
	and as:
	
	Finally we have:
	
	so we have a bias of at least one standard error:
	
	then we say that this estimator has a negative bias (it underestimates the true value!).
	
	We also note that the estimator tends towards to an unbiased estimator of the variance (USV) when the number of items tends to infinity $n\rightarrow +\infty$. We say that we have a "\NewTerm{asymptotically unbiased}\index{asymptotically unbiased estimator}" or "\NewTerm{asymptotically unbiased estimator}".

	It is important to note that we have yet proved that the empirical variance tends towards the theoretical variance when $n$ tends to infinity and ... that the data follows or not a Normal distribution!
	
	\begin{tcolorbox}[title=Remark,colframe=black,arc=10pt]
	An estimator is named "\NewTerm{consistent estimator}\index{consistent estimator}" if it converges in probability, when $n\rightarrow +\infty$, towards the true parameter value.
	\end{tcolorbox}
	By the properties of the mean, we get:
	
	We have then:
	
	simply named the "\NewTerm{standard deviation}" ... (that must not be confused with the "standard error" as we shall see later).

So we finally summarize as following the two important previous results:

	\begin{enumerate}
	
		\item The "\NewTerm{biased maximum likelihood estimator}\index{biased maximum likelihood}" or also named "\NewTerm{empirical standard deviation}\index{empirical standard deviation}" or "\NewTerm{sample standard deviation}\index{sample standard deviation}" or "\NewTerm{Pearson standard deviation}\index{Pearson standard deviation}" ... is therefore given by:
		
		
		when $n\rightarrow +\infty$. We find this standard deviation depending on the context (by tradition) noted in five other ways that are:
		
		and sometimes (but this is very awkward because it often generates confusion with the unbiased estimator) $\sigma$ or $S$.
		
		\item The "\NewTerm{unbiased maximum likelihood estimator}\index{unbiased likelihood estimator}" or simply named "\NewTerm{standard deviation}\index{standard deviation}":
		
		which as we can see is a consistent estimator (when $n$ tends to infinity it tends to the biased maximum likelihood estimator).
		
		We find this standard deviation depending on the context (by tradition) noted in three other ways that are:
		
	
	\end{enumerate}

	We find these last two notations often in tables and in many softwares and we will use them later in the development of confidence intervals and hypothesis testing!

	For example, in the Microsoft Excel 11.8346 the unbiased estimator is given by the \texttt{STDEV( )} function and the non-biased by \texttt{STDEVP()}.

In total, this make us is three estimators for the same indicator! As in the overwhelming majority of cases of the industry the mean is not known, we usually use the last two relations bordered above. Now this is where comes the vicious part: when we calculate the bias of this two estimators, the first is biased, the second is not. So we tend to use only the latter. Nay! Because we could also talk about the variance and precision of an estimator, which are also important criteria for judging the quality of an estimator relative to another. If we were to calculate the variance of the two estimators, then the first, which is biased, is smaller than the second which is unbiased variance! All that to say that the criteria of bias is not (by far) the only one to be study to judge the quality of an estimator.

Finally, it is important to remember that the factor $-1$ in the denominator of the unbiased maximum likelihood estimator stems from the need to correct the mean of the biased estimator initially subtracted by one time the standard error!

	\subsubsection{Poisson Distribution MLE}

	Using the same method as for the Normal (Gauss-Laplace) distribution, we will seek the maximum likelihood estimators of the Poisson distribution which for recall is given by:
	
	Thus, the likelihood is given by
	
	Maximize a function or maximize its logarithm is equivalent therefore:
	
	We are now looking to maximize it:
	
	and thus we obtain the only maximum likelihood estimator that will be:
	
	It is quite normal to find in this example the sample mean because it is the best possible estimator for the parameter of the Poisson distribution (which also represents the mean of a Poisson distribution).

	Knowing that the standard deviation of this particular distribution (see above during the development of the Poisson distribution) is the square root of the mean, then we have for the standard deviation maximum likelihood :
	
	\begin{tcolorbox}[title=Remark,colframe=black,arc=10pt]
	We show in the same way identical results for the exponential distribution that is widely used in preventive maintenance and reliability!
	\end{tcolorbox}
	
	\subsubsection{Binomial (and Geometric) Distribution MLE}
	
	Using the same method as for the Normal distribution (Gauss-Laplace) and the Poisson distribution, we will seek the maximum likelihood estimator of the Binomial which we recall, is given by:
	
	Accordingly, the likelihood is given by:
	
	It should be remembered that the factor following the combinatorial term already expressed the successive variables according to what we saw during our study of the Bernoulli and Binomial distribution functions. Hence the disappearance of the product in the preceding equality.

	Maximize a function or maximize its logarithm is equivalent therefore:
	
	We are now looking to maximize it:
	
	The reader may have perhaps noticed that the binomial coefficient has disappeared. Therefore, we immediately deduce that the estimator of the binomial distribution is the same as the geometric distribution.
	
	

	Which gives:
	
	from which we derive the maximum likelihood estimator:
	
	This result is quite intuitive if we consider the classic example of a coin that has a chance on tow of dropping on one of its faces. The probability $p$ being the number of times $k$ a given face where was observed in the total number of tests (all sides combined).
	
	\begin{tcolorbox}[title=Remark,colframe=black,arc=10pt]
	In practice, it is not as easy to apply these estimators! We must carefully consider which are most suitable for a given experiment and ideally also calculate the mean squared error (standard error) of each of the estimators of the mean (as we have already done for the empirical mean earlier). In short... it is a long process of reflection.
	\end{tcolorbox}
	
	\subsubsection{Weibull Distribution MLE}
	
	We saw in the section of Industrial Engineering a very detailed study of the three-parameter Weibull distribution with its standard deviation and mean because as we mentioned it is quite used in the field of reliability engineering.

	Unfortunately, the three parameters of this distribution are unknown in practice. Using estimators however we can determine the expression of two of the three assuming $\gamma$ as zero. This gives us the following Weibull distribution named  "Weibull distribution with two parameters":
	
	and for recall with $\beta>0$ and $\eta>0$.
	
		
	Maximize a function or maximize its logarithm is equivalent therefore:	
	
	Now we seek to maximize this by remembering that (\SeeChapter{see section Differential And Integral Calculus}):
	
	then:
	
	And we get for the second parameter:
		
	then:
	
	Finally to resume with the correct notations (and in the resolution order in practice):
	
	Solving these equations involves heavy computations and we can a priori do nothing with that in conventional spreadsheets softwares such as Microsoft Excel or Open Office Calc without programming (at least as far as we know...).

	We then take a different approach by writing our Weibull distribution with two parameters as follows:	
	
	with for recall $\beta>0$ and $\theta>0$.

	Therefore the likelihood is given by:
	
	Maximize a function or maximize its logarithm is equivalent therefore:
	
	Now we seek to maximize this by remembering that (\SeeChapter{see section Differential And Integral Calculus}):
	
	then:
	
	And we have for the second parameter:
	
	It is then immediate that:
	
	injected into the equation:
	
	We get:
	
	simplifying:
	
	The resolution of the two equations (in order from top to bottom):
	\begin{equation}
	  \addtolength{\fboxsep}{5pt}
	   \boxed{
	   \begin{gathered}
	   	\begin{aligned}
	     &\dfrac{\displaystyle \sum_{i=1}^n x_i^\beta \ln(x_i)}{\dfrac{1}{n}\displaystyle \sum_{i=1}^n x_i^\beta}-\dfrac{1}{\beta}-\dfrac{1}{n}\sum_{i=1}^n \ln(x_i)=0\\
		&\dfrac{1}{n}\sum_{i=1}^n x_i^\beta-\bar{\theta}
		\end{aligned}
	   \end{gathered}
	   }
	\end{equation}
	
	can easily be calculated with the Target Tool of Microsoft Excel or Open Office Calc.
	
	\subsubsection{Gamma Distribution MLE}
	Here we will use a technique named "\NewTerm{method of moments}\index{method of moments}" to determine the estimators of the parameters of the Gamma distribution.

	Suppose that $X_1, ..., X_n$ are independent and identically distributed random variables according to the Gamma distribution with density:
	
	We seek to estimate $a,\lambda$. For this, we first determine some theoretical moments.
	The first moment is the expected mean that as we have proved before is given by:
	
	and the second moment, the mean of the square of the random variable, is as we have implicitly proved in the proof of the variance of the Gamma distribution given by:
	
	We then express the relation between the parameters and the theoretical moments:
	
	The resolution of this simple system gives:
	
	Once this system established, the method of moments consist to use the empirical moments, i.e. for our example the first two, $\hat{m}_1,\hat{m}_2$:
	
	that we define as equal to the true theoretical moments ... Therefore, it comes:
	\begin{equation}
	  \addtolength{\fboxsep}{5pt}
	   \boxed{
	   \begin{gathered}
	   	\begin{aligned}
	    	a&=\dfrac{\hat{m}_1^2}{\hat{m}_2-\hat{m}_1^2}\\
			\lambda&=\dfrac{\hat{m}_2-\hat{m}_1^2}{\hat{m}_1^2}
		\end{aligned}
	   \end{gathered}
	   }
	\end{equation}
	
	\subsection{Finite Population Correction Factor}
	Now we prove another result which we will be required in some statistical tests that we will see later.

Suppose we have a population of $N$ individuals that we we represent by the set $\left\lbrace1,2,...,N\right\rbrace$ and a random variable $X$ which is an application of $\left\lbrace1,2,...,N\right\rbrace$ in $\mathbb{R}$. We denote by $x_i=X(i)$. The mean of $X$ is thus given by:

	
	Remember the variance of $X$ is by definition:
	
	Now we consider the set $E$ of samples of size $n$ taken in $\left\lbrace1,2,...,N\right\rbrace$ with $0<n<N$. Each individual has a probability of being drawn equal to:
		
	We are interested in the random variable $\bar{X}$ defined on $E$ and that equal to the sample mean. More specifically:
	
	To calculate the variance $\text{V}(\bar{X})$, we will $\bar{X}$ express as a sum of random variables. Indeed, if we define the variables $X_k$ with $k=1...N$ by:
	
	We have naturally by the previous definition (see with caution the sum limits!):
	
	and thus we get:
	
	The random variables $X_k$ are not independent in pairs, in fact as we shall see, their covariances are not zero if $N$ is finite. Otherwise (zero covariance), we find a result already proved earlier:
	
	So we need to calculate the variances $\text{V}(X)$ and covariances $\text{cov}(X_i,X_j)$.
	
	For this purpose we will use the Huyghens relation and we will start by calculating the mean $\text{E}(X_k)$:
	
	But $P(X_k=x_k)$ is the probability that a sample contains $k$. This probability is obviously equal to $n/N$ and therefore:
	
	Similarly we obtain:
	
	We can therefore calculate the variance $\text{V}(X_k)$:
	
	To calculate the covariances we need now to calculate the means $\text{E}(X_iX_j)$:
	
	But $P(X_i=x_i,X_j=x_j)$ is the probability that a sample contains $i$ and $j$. This probability is obviously given by:
	
	and therefore:
	
	We can now compute the covariance:
	
	
	We are now able to simplify $\text{V}(\bar{X})$:
	
	Using Huyghens theorem we get:
	
	Using the result proved above and previous relation:
	
	Therefore:
	
	
	
	For the double sum $\displaystyle \sum_{i\neq j}^N x_ix_j$, we have:
	
	Therefore:
	
	Thus:
	
	The famous factor:
	
	that we have already encountered during our study the hypergeometric distribution is named "\NewTerm{finite correction factor (on finite population)}\index{finite correction factor}" and has the effect of reducing the standard error especially as $n$ is large. 
	
	In the science of survey, we can conclude the we have for "random sample without replacement (RSWOR)" and with "random sample with replacement (RSWR)" the average is always given by:
	
	but for the standard deviation:
	
	
	\pagebreak
	\subsection{Confidence Intervals}
	Until now we have always determined the likelihood estimators or simple estimators (variance, standard deviation) from theoretical statistics distributions or measured on an entire population of data. 

\textbf{Definition (\#\mydef):} A "\NewTerm{confidence interval}\index{confidence interval}" is a pair of numbers that defines (a posteriori) the range of possible values with a certain cumulative probability of an (punctual) estimator of a given statistical indicator form a sample of an experience (the range of the statistical indicator being usually calculated using real measured parameters). It is the most common statistical case.

	We now turn to the task that consists naturally to ask ourselves what must be the sample sizes of our measured data to have some validity (C.I.: confidence interval) for our estimators or even to which confidence interval correspond a given standard deviation or quantile in a Normal centered reduced distribution (for large samples), in a chi-square distribution, Student distribution or Fisher distribution (we will see the last two cases of small sample sizes in the section on analysis of variance or ANOVA) when the man or variance are known or unknown respectively on all or part of the given population.

	It is important to know that these confidence intervals often use the central limit theorem that will be proved late (to avoid any possible frustration) and the developments that we will do now are also useful in the field of (a posteriori) Hypotheses Tests that have a major role in statistics and therefore indirectly in all fields of science!!!

	Finally, it could be useful to indicate that a large numbers of organizations (private or institutional) make false statistics because the assumptions and conditions of use of these confidence intervals (verbatim hypotheses tests) are not rigorously verified or simply omitted or worse, the whole base (measurements) is not collected in the rules of art (reliability of the data collection and reproductibility protocols not validated by scientific peer).
	
	The reader must also know that we have put many other confidence interval techniques detailed proofs related for example for regression techniques in the section of Theoretical Computing.

	\begin{tcolorbox}[title=Remark,colframe=black,arc=10pt]
	The practitioner should be very careful about the calculation of confidence intervals and the use of hypothesis testing in practice. This is why, to avoid trivial usage error or interpretation, it is important to refer to the following international standards eg: ISO 2602:1980 \textit{•}, ISO 2854:1976 (Statistical interpretation of data - Techniques of estimation and tests relating to means and variances), ISO 3301:1975 (Statistical interpretation of data - Comparison of two means in the case of paired observations), ISO 3494:1976 (Statistical interpretation of data - Effectiveness of tests relating to means and variances ), ISO 5479:1997 (Statistical interpretation of data - Tests for departure from the normal distribution ), ISO 10725:2000 + ISO 11648-1:2003 + ISO 11648-2:2001 (Sampling plans and procedures for acceptance for control of bulk materials), ISO 11453:1996 (Statistical interpretation of data - Tests and confidence intervals relating to proportions), ISO 16269-4:2010 (Statistical interpretation of data - Detection and treatment of outliers), ISO 16269-6:2005 (Statistical interpretation of data - Determination of statistical tolerance intervals), ISO 16269-8:2004 (Statistical Interpretation of data - Determination of prediction intervals), ISO / TR 18532:2009 (Guidelines for the application of statistical quality and industrial standards ).
	\end{tcolorbox}
	
	\subsubsection{C.I. on the Mean with known Variance}
	Let's start with the simplest and most common case that is the determination of the number of individuals to have some confidence in the average of the measurements of a random variable assumed to follow a Normal distribution.

	First let us recall that we showed at the beginning of this chapter that the standard error (standard deviation of the mean) was under the assumptions of independent and identically distributed variables (i.i.d.):
		
	Now, before we go any further, consider $X$ as a random variable following a Normal distribution with mean $\mu$ and standard deviation $\sigma$. We would like that the random variable has for example $95\%$ cumulative probability of being in a given bounded symmetric interval. Which is therefore expressed as follows:
	
	\begin{tcolorbox}[title=Remark,colframe=black,arc=10pt]
	Therefore with a confidence interval of $95\%$ you will be right a posteriori $19$ times out of $20$, or any other level of confidence or risk level $\alpha$ ($1$-confidence level, $5\%$) that you will be set up in advance. On average, your conclusions will therefore be good, but we can never know whether a particular decision is good! If the risk level is very low but the event still occurs, specialists then speak about a "\NewTerm{large deviation}\index{large deviation}" or a "\NewTerm{black swan}\index{black swan}". Management of outliers is addressed in ISO 16269-4:2010 Detection and treatment of outliers that any engineer doing business statistics has to follow. 
	\end{tcolorbox}
	By centering and reducing the random variable:
	
	Let us now write Y the reduced centered variable:
	
	Since the Normal centered reduced distribution is symmetric:
	
	Therefore:
	
	From there reading statistical tables of the standard Normal distribution (or by using a simple spreadsheet software), we have to satisfy the equality that:
	
	Which can easily be obtained with Microsoft Excel 11.8346 by using the function: 
	\begin{center}
	\texttt{=-NORMSINV((1-0.95)/2)}
	\end{center}
	As noted in the traditional way in the general case other than the $95\%$ one ($Z$ is the random variable corresponding to the half quantile of the chosen threshold of the standard Normal distribution):
	
	Now, consider that the variable $X$ on which we wish to make statistical inference is the average (and we show later that it follows a Normal distribution centered reduced distribution). Therefore:
	
	Then we get:
	
	from which we obviously take (normally...) the upper integer value...
	The latter notation is usually written in the following way highlighting better the width of the confidence interval of an underlying threshold level:
	
	
	Relation named "\NewTerm{sample size estimation by Normal distribution}\index{sample size estimation by Normal distribution}".

	Thus, we now know the number of individuals we must have to ensure to get a given precision interval $\delta$ (margin of error) around the mean and that for a given percentage measures are in this range and assuming the theoretical standard deviation $\sigma$ is known (or imposed) in advance (typically used in quality engineering or surveys).

	In other words, we can calculate the number $n$ of individuals to measure to ensure a given confidence interval (associated to the quantile $Z$) of the measured average assuming the theoretical standard deviation known (or imposed) and wishing a precision $\delta$ in absolute value of the mean.

	However ... in reality, the variable $Z$ comes from the central limit theorem (see below) that gives for a large sample size (approximately):
	
	Rearranging we get the:
	
	and as Z can be negative or positive then it is more logic to write this as:
	
	Thus:
	
	That engineers sometimes write:
	
	where LCL is the lower confidence limit and UCL the upper confidence limit. This is the Six Sigma terminology (\SeeChapter{see section Industrial Engineering}).
	
	And we have seen earlier that for a confidence interval of $95\%$ we have $Z=1.96$. And since the Normal distribution is symmetric:
	
	Thus we finally write the "\NewTerm{one sample Z test}\index{one sample Z-test}":
	
	where we define for all tests having the same structure, the "\NewTerm{margin error}\index{margin error}" by:
	
	As we have already mentioned, and we will prove a little further, the arithmetic reduced centered mean of a series of independent and identically distributed random variables with finite variance asymptotically follows a standard Normal distribution, this is why the confidence interval above is very general! This is why we sometimes speak of "asymptotic confidence interval of the mean".

	These intervals obviously have for origin the fact that we work very often in statistics with samples and not the entire available population. The selected sampling thus affects the value of the punctual estimator. We then speak of "sampling fluctuation".
	
	In the particular case of an IC (confidence interval) at $95\%$, the last relation will be written:
	
	Sometimes we find the prior-previous inequality in the following equivalent notation:
	
	or more rarely with the following general notation (for all intervals):
	
	where ME stands for "\NewTerm{margin of error}\index{margin error}".
	We are thus now able to estimate population sizes needed to obtain a certain level of confidence $\alpha$ in an outcome or to estimate the confidence interval in which is the theoretical mean knowing the experimental (empirical) average and the estimator maximum likelihood of the standard deviation. We can of course therefore also determine the a posteriori probability that the mean is outside a given range ... (one as the other being widely used in the industry).

Finally, note that from the previous result, we deduce immediately the stability property of the Normal distribution (shown above) the following test that we find in many statistical softwares:
	
	named "\NewTerm{bilateral $Z$ test on the difference of two means}\index{bilateral $Z$ test on the difference of means}" or also sometimes named "\NewTerm{two sample $Z$ test}\index{two sample $Z$ test}" a with the corresponding confidence interval:
	
	And this is not because two averages are significantly different that their confidence intervals do not overlap!!!! As shows the graph below obtained with Minitab 16 software where the test-$Z$ of the difference is significant at $95\%$:
	
	\begin{figure}[H]
		\centering
		\includegraphics{img/arithmetics/confidence_interval_line_plot_overlap.jpg}
		\caption{Line plot illustration of the overlay of two confidence with $95\%$ confidence interval}
	\end{figure}
	
	while their mean is significantly different to a confidence level of $95\%$.
	
	\begin{tcolorbox}[title=Remark,colframe=black,arc=10pt]
	The size of the parent population for the relations developed above does not come into consideration in the calculations of confidence intervals or even not in the sample size, and because it is considered as infinite. So be careful not sometimes not to have sample sizes that are larger than the actual parent population... 
	\end{tcolorbox}
	
	\pagebreak
	\subsubsection{C.I. on the Variance with known Mean}
	
	Let's start by demonstrating a fundamental property of the Chi-square distribution:

If a random variable $X$ follows a Normal centered reduced distribution $X=\mathcal{N}(0,1)$ then its square follows a chi-square distribution of $1$ degree of freedom:	
	
	This result is sometimes named a "\NewTerm{Wald statistics}\index{Wald's statistics}" and any statistical test using it directly (we should better speak about a "test family") can be designer under the name "\NewTerm{Wald's test}\index{Wald's test}" (for a concrete example see Cochran-Mantel-Haenszel test in the section of Theoretical Computing).
	
	\begin{dem}
	To prove this property, it suffices to calculate the density of the random $X^2$ variable with $X=\mathcal{N}(0,1)$. However, if $X=\mathcal{N}(0,1)$ and if we set $Y=X^2$, then for all $y \geq 0$ we get:
	
	Since the standard Normal distribution is symmetric about 0 for the random variable X, we can write:
	
	Denoting by $\Phi$ the cumulative distribution function of the standard normal distribution, we have:
		
	and as:
	
	therefore:
	
	The cumulative distribution function of the random variable $Y=X^2$ is thus given by:
	
	if $y$ is greater than or equal to zero, null if $y$ is less than zero. We will note this cumulative distribution $f_Y(y)$ for the further calculations.

	Since the density distribution function is the derivative of the cumulative distribution function and $X$ follows a Normal centered reduced distribution so we reduced for the random variable $X$:
	
	and then it follows for the probability distribution of $Y$ (which is the square of $X$ for reminder!):

	
	this last expression corresponds is exactly the relation we obtained during our study of the chi-square distribution imposing a degree of freedom equal to the unit.

	The theorem is therefore proved, that is if $X$ follows a Normal centered reduced distribution while its square follows a Chi-square distribution of $1$ degree of freedom as:	
	
	\begin{flushright}
		$\square$  Q.E.D.
	\end{flushright}	
	\end{dem}
	
	This type of relation is used mainly in industrial processes and their control (\SeeChapter{see section Industrial Engineering}).
	
	Now let us open a parenthesis that is quite important in some linear regression software reports and especially in the curvature test for design of experiment (\SeeChapter{see section Industrial Engineering}). Let us recall that we have:
	
	And we have just prove above that:
	
	Therefore:
	
	And as we have also seen that:
	
	it follow that:
	
	or more commonly in practice:
	

	We will now use a result proved during our study of the Gamma distribution. We have effectively seen that the sum of two random variables following a Gamma distribution also follows a Gamma distribution where the two parameters are added:
	
	As the Chi-square distribution is a special case of the Gamma distribution, the same result applies.

	To be more precise, this is equivalent to say: If $X_1,...,X_k$are random independent and identically distributed (i.i.d.) variables $\mathcal{N}(0,1)$ then by extension of the above proof where we have shown that:	
	
	and by the property of linearity of the Gamma distribution, then sum of their squares follows a chi-square distribution of degree $k$ such that:
	
	Thus, the distribution of $\chi^2$ of $k$ degrees freedom is the probability distribution of the sum of squares of $k$ Normal centered reduced variables linearly independent of each other. It is in fact the linearity property of the chi-square distribution (implicitly the linearity of the Gamma distribution)!

	Now see another significant property of the chi-square distribution: If $X_1,...,X_n$ are independent and identically distributed $\mathcal{N}(\mu,\sigma)$ (thus the same mean and the same standard and following a Normal distribution) random variables and if we write the maximum likelihood estimator variance by:
	
	then, the ratio of the random variable $S_*^2$ on the standard deviation assumed to be known for the entire population ("the true standard deviation" or "theoretical standard deviation"!) multiplied by the number of individuals $n$ population follows a chi-square distribution of degree $n$ such that:
	
	This result is named the "\NewTerm{Cochran theorem}\index{Cochran theorem}" or "\NewTerm{Fisher-Cochran theorem}\index{Fisher-Cochran theorem}" (in the particular case of Gaussian samples) and thus gives us a distribution for the empirical standard deviations (whose parent law is a Normal distribution!).

	Using the value of the standard deviation proved during our study of chi-square distribution we have:
	
	But $n$ and $\sigma$ are imposed and are therefore considered as constants. We have therefore:
	
	And therefore we have an expression of the standard deviation of the empirical standard deviation if we know the standard deviation of the population:
	
	But we have prove during our study of estimators that:
	
	It follows:
	
	It follows therefore the sometimes important relation in the practice of the estimator of the standard deviation of ... the standard deviation:
	
	Recall that the parent population is said to be "infinite" if the sample selection with replacement or if the size $N$ of the parent population is much higher than this of the sample of size $n$.
	
	\begin{tcolorbox}[title=Remarks,colframe=black,arc=10pt]
	\textbf{R1.} In laboratories the $X_1,...,X_n$ can be seen as a class of individuals of the same product identically studied by different research teams with instruments of the same precision (standard deviation of the measure identically equal).\\
	
	\textbf{R2.} $S_{*^2}$ is the "\NewTerm{inter-class variance}\index{inter-class variance}" also named "\NewTerm{explained variance}\index{explained variance}". So it gives a measure of the variance occurring in different laboratories.
	\end{tcolorbox}
	
	What is interesting here is that from the calculation of the chi-square distribution and by knowing $n$ and the standard deviation $\sigma^2$ it is possible to estimate the interclass variance (and also interclass standard deviation).

	To see that this latter property is a generalization of the basic relation:
	
	it suffices to see that the random variable $nS_*^2/\sigma^2$ is a sum of $n$ squares of $\mathcal{N}(0,1)$ independent of each other. Indeed, recall that a centered reduced random variable (see our study of the Normal distribution) is given by:
	
	Therefore:
	
	However, since the random variables $X_1,...,X_n$ are independent and identically distributed according to a Normal distribution, then the random variables:
	
	are also independent and identically distributed according to a Normal distribution but a centered reduced one.
	
	Since:
	
	
	rearranging we get:
	
	
	So on the population of measurements, the true standard deviation follows the relation given above. It is therefore feasible to make statistical inference on the standard deviation When the theoretical mean is known (...).
	
	

Since the chi-square distributions is not symmetric, the only way to make this inference is to use numerical calculations and then we denote the confidence interval at the level of $95\%$ (for example ...) as follows:
	
	
	Either by writing $95\%=1-\alpha$:
	
	the denominator being obviously the quantile of the chi-2 distribution. This relation is rarely used in practice as the theoretical average (mean) is not known. In order to avoid confusion, the latter relation is often denoted as follows:
	
	Let's see the most common case:
	
	\subsubsection{C.I. on the Variance with empirical Mean}
	
	Let us now make statistical inference when the theoretical average of population (i.e. the mean) is not known. To do this, consider now the sum of:	
	
	where for recall is the empirical average (arithmetic mean) of the sample:
	
	Continuing the development we have:
	
	However, we have proved earlier in this chapter that the sum of the deviations from the mean was zero. So:
	
	and by taking back the unbiased estimator of the Normal distribution (we change notation to respect the traditions and differentiate the empirical average of the theoretical mean):
	
	Thus:
	
	or with another common notation:
	
	Since the second term (squared) follows a Normal centered reduced distribution too, so if we remove it we get by the proof made above about the chi-square distribution following property:
	
	These developments allow us this time to also make inferences about the variance of a $\mathcal{N}(0,1)$ distribution when the parameters $\mu$ and $\sigma$ of the parent population are both unknown. It is this result that gives us, for example, the confidence interval:
	
	when the theoretical average (mean) $\mu$ is unknown. And also to avoid any confusion, it is more usual to write:
	
	In the same way as above, we can calculate the standard deviation of the standard deviation that has a great importance in the practice of finance:
	
	
	\pagebreak
	\subsubsection{C.I. on the Mean with known unbiased Variance}
	
	We have proved much higher that the Student distribution came from the following relation:
	
	if $Z$ and $U$ are independent random variables and if $Z$ follows a Normal centered reduced distribution $\mathcal{N}(0,1)$ and $U$ a chi-square distribution $\chi^2(k)$ as:
	
	and remember that its density function is symmetrical!
	
	Here is a very important application of the above result:
	
	Suppose that $X_1,...,X_n$ is a random sample of size n from a distribution $\mathcal{N}(\mu,\sigma)$. So we can already write that following developments made above:
	
	And for $U$ that follows a $\chi^2(k)$ distribution, then if we ask that $k=n-1$ then according to the results above:
	
	
	We then get after some trivial simplifications:
	
	So since:
	
	follows a Student distribution with parameter $k$ then we get the "\NewTerm{independent one-sample $t$-test}\index{independent one-sample $t$-test}" or simply calld "\NewTerm{one-sample $t$-test}\index{one-sample $t$-test}":
	
	which also follows a Student distribution of parameter $n-1$ and is widely used in laboratories for calibration testing.

	This gives us also after rearrangement:
	
	This allows us to make inference about the mean $\mu$ of a Normal distribution with the theoretical standard deviation being unknown (meaning that there is not enough experimental values) but where the unbiased estimator of the standard deviation is known. It is this result that gives us the confidence interval:
	
	where we see the same factors as for the statistical inference on the average (mean) of a (theoretical) random with know standard deviation as the Student distribution is asymptotically equal to the Normal distribution for large values of $n$. Thus, the previous interval and the following interval:
	
	givers very similar values (to three decimal places) for values of $n$ at around $10,000$ (in practice we consider that for $100$ this is the same...).
	
	We immediately deduce by the stability property of the chi-square distribution (proved above in that this property arises from the Gamma distribution) the following test that we find in many statistical software:
	
	named "\NewTerm{bilateral $t$ (Student) test on the difference of two means}\index{bilateral $t$ (Student) test on the difference of two means}" or more simply "\NewTerm{two sample $t$-test}\index{two sample $t$-test}".

	We can of course therefore also determine the probability that the mean is inside or outside a certain range ... (the both case being widely used in industry).

	The reader can for fun control with Microsoft Excel 11.8346 that for a large number of measurements $n$, the Student distribution tends to the Normal centered reduced distribution by comparing the values of the two functions below:
	\begin{center}
	\texttt{=T.INV(5\%/2,n-1)}\\
	\texttt{=NORM.S.INV(5\%/2)}
	\end{center}
	
	\begin{tcolorbox}[title=Remark,colframe=black,arc=10pt]
	The previous result was obtained by William S. Gosset around 1910. Gosset who had studied mathematics and chemistry, worked as a statistician for the Guinness brewery in England. At that time, we knew that if $X_1,...,X_n$ are independent and identically distributed random variables then:
	
	However, in statistical applications we were rather obviously interested in the following quantity:
	
	We then merely assume that this amount followed almost a Normal centered reduced distribution, which was not a bad approximation as can show the image below ($\mathrm{d}f=n-1$):
	
	\begin{figure}[H]
		\centering
		\includegraphics{img/arithmetics/comparison_normal_student_distributions.jpg}
		\caption{Comparison between the Normal and the Student distribution functions}
	\end{figure}
	After numerous simulations, Gosset came to the conclusion that this approximation was valid only when $n$ is large enough (so that gave him the indication that there must be somewhere behind the central limit theorem). He decided to determine the origin of the distribution and after completing a course in statistics with Karl Pearson he obtained his famous result that he published under the pseudonym Student. Thus, is why we call Student distribution that law that should have been named the "Gosset distribution". 
	\end{tcolorbox}
	Finally, note that the Student's $T$-test is also used to identify whether changes (increasing or vice versa) in the average of two identical populations are statistically significant. That is to say, if the size of two dependent samples is the same then we can create the following test (we included all different types of writing that can be found in the literature and in many software implementing this test):
	
	
	With:
	
	The prior-previous relation is very useful for comparing the same sample twice in different measurement situations (sales before or after a discount on an article for example). This prior-previous relation is named "\NewTerm{$t$-test (Student) averages two paired samples (or dependent samples)}" or more simply "\NewTerm{paired sample $t$-test}\index{paired sample $t$-test}".

	\textbf{Definition (\#\mydef):} We speak of "\NewTerm{paired samples}\index{paired samples}" if the sample values are taken 2 times on the same individuals (i.e. the values of the pairs are not independent, unlike two samples taken independently).
	
	\subsubsection{Binomial exact Test}

	Often when measuring we want to compare two small samples taken randomly (without replacement!) from also a small population ... to know if they are statistically significantly different or not as when we were expecting a perfect equality!

	We are looking for a suitable test for the following cases:

	\begin{itemize}
		\item To know if the sample of a population prefers to use a given technical method of work rather than another when we expect that the population does not prefer one of the other

		\item To know if the sample of a population has a predominant characteristic among two possibilities when we expect that the population is well balanced

	\end{itemize}

	Before going further into details, let us remind that we must be extremely cautious about how to get the two samples. The experience must be unbiased, this is to say for reminder, that the sampling protocol must not favor one of the both characteristics of the population (if you study the balance between man/woman in a population by attracting people for the survey with a gift in the form of jewelry or just by calling during the workdays you will have a biased sample ... because you'll probably naturally have more women than men...).

	This said, this situation match with a binomial distribution for which we proved earlier in this chapter that the probability of $k$ successes in a population of size $N$ with a probability of success is p (probability of failure $q$ being therefore $1 - p$) was given by the relation:
		
	In the case before we are interesting we have $p=q=0.5$:
	
	while remembering that the distribution will not be symmetrical and especially if the population size $N$ is small.

	If we now denote by $x$ the number of successes (considered as the size of the first sample) and $y$ is the number of failures (considered as the size of the second sample), then we have:
	
	This being done, to build the test and by the asymmetry of the distribution, we will calculate the cumulative probability that $k$ is smaller than the $x$ obtained by the experience and sum it to the cumulative probability that $k$ is greater than the $y$ obtained by the experiment (which corresponds to a cumulative probability of respectively left and right tails of the distribution). So this sum corresponds to the probability:
	
	and this last relation is named "\NewTerm{binomial exact test (two-tailed)}\index{binomial exact test (two tailed)}".

	If the probability $P$ obtained by the sum is above a certain cumulative probability fixed in advance, then we say that the difference with a random sample in a perfectly balanced population is not statistically significant (bilaterally ...) and respectively if it is below, the difference will be statistically significant and therefore we reject the assumed equilibrium.

	Therefore, if:
	
	the difference with a balanced population will be considered not statistically significant. Often we will $\alpha$ to be at the maximum equal to $5\%$ (but rarely below) which corresponds to a confidence interval of $95\%$.

	Unfortunately from a statistical software to the other the required parameters or results will not necessarily be the same (spreadsheets softwares for example do not include a specific function for the binomial test, will often have to build a table or develop yourself a function). For example, some software automatically calculate and impose (which is quite logical in a sense...)
	
	\begin{tcolorbox}[colframe=black,colback=white,sharp corners]
\textbf{{\Large \ding{45}}Example:}\\
	From a small population having two particular characteristics $x$ and $y$ that interest us and which we expect to have a perfect balance but as $x=y$ we actually got $x=5$ and $y=7$. We would like do the calculation with Microsoft Excel 11.8346 to know whether this difference is statistically significant or not at a level of $5\%$?\\

	So, to answer this question, we will calculate the cumulative probability:
	
	which gives us:
	\begin{figure}[H]
		\centering
		\includegraphics[scale=0.55]{img/arithmetics/binomial_coefficient_calculated.jpg}
		\caption[]{Calculated values of the binomial coefficients in Microsoft Excel 11.8346}
	\end{figure}
	thus explicitly:
	\begin{figure}[H]
		\centering
		\includegraphics[scale=0.55]{img/arithmetics/binomial_coefficient_calculated_explicit.jpg}
		\caption[]{Formulas for calculating binomial coefficients in Microsoft Excel 11.8346}
	\end{figure}
	thus the cumulative probability being $0.774$ (i.e. $77.4\%$) the difference compared with balanced population will be considered as not statistically significant.
	\end{tcolorbox}
	\begin{tcolorbox}[title=Remark,colframe=black,arc=10pt]
	This test is also used by most statistical software (such as Minitab) to give a confidence interval of the conformity of opinions in relation to that of an expert. This is what we call an R\& R study (reproductability\& repeatability) by attributes (see my book on Minitab for an example).
	\end{tcolorbox}	
	
	\subsubsection{C.I. for a Proportion}
	
	For information some statisticians use the fact that the Normal distribution arises from the Poisson distribution which itself derives from the binomial distribution (we have proved it when $n$ tends to infinity and $p$ and $q$ are of the same order) to build a confidence interval in the context of the analysis of proportions (widely used in the analysis of the quality in the industry).

	To see this, we note $X_i$ the random variable defined by:
	
	where the attribute $A$ can be the property "defective" or "non-defective" for example, in an analysis of pieces. We note by $K$ the number of successes of the attribute $A$.

	The random variable $X=X_1+X_2+...+X_n$ we have proved it earlier in this chapter, follows a binomial distribution with parameters $n$ and $p$ with the moments:
	
	
	That said, we do not know the true value of $p$. We will use the estimator of the binomial distribution proved above:
	
	Based on the properties of the mean we have then:
	
	And by using the properties of the variance, we have following relation for the variance of the sample mean of the proportion:
	
	This then brings us to:
	
	Finally, remember that we have proved that the normal distribution resulted from the binomial distribution under certain conditions (practitioners admit that it is applicable as $n>50$ and $np \geq 5$). In other words, the random variable $X$ following a binomial distribution follows a Normal distribution under certain conditions. Obviously, if $X$ follows a Normal distribution then $X/n$ also (and so do $\hat{p}$...). Therefore we can center and reduce $\hat{p}$ so that it behaves as the reduced Normal centered random variable denoted by $Z$:
	
	\begin{tcolorbox}[colframe=black,colback=white,sharp corners]
\textbf{{\Large \ding{45}}Examples:}\\\\
	E1. If $5\%$ of the annual production of a business fails, what is the probability that by taking a sample of $75$ pieces of the production line only $2\%$ or less will be defective?

	We therefore have:
	
	The corresponding cumulative probability to that value can be easily obtained with Microsoft Excel 11.8346:

	\begin{center}
	\texttt{=NORMSDIST(-1.19)=11.66\%}
	\end{center}
	But note that we do not have $np\geq 5$ that is satisfied therefore we could exclude to use this result.\\
	
	E2. In its report from 1998, JP Morgan explained that during the year 1998 its losses went beyond the Value at Risk (\SeeChapter{see section Economy}) $20$ days on $252$ working days of the year based on a $95\%$ temporal VaR (thus $5\%$ of working days considered as loss). At the threshold of $95\%$ it is just bad luck or is that the VaR model used was bad?
	
	So it was just bad luck.
	\end{tcolorbox}
	We can now approximate the confidence interval for the proportion by using the fact that the binomial distribution has a Normal asymptotic behavior under the conditions demonstrated during our introduction of the Normal distribution such as we get the "\NewTerm{one-proportion $Z$ test}\index{one-proportion $Z$ test}" or also more commonly named "\NewTerm{one-proportion $p$ test}\index{one-proportion $p$ test}":
	
	Before proceeding to an example, it may be useful to clarify to the reader that this approximation by a Normal distribution is very common and that we'll meet it again numerous times in proofs that will follow. It is even so common that this approximation method has a name..: the "\NewTerm{Wald method}\index{Wald method}" (well actually there are several Wald methods but we will only use the most known one).
	
	\begin{tcolorbox}[colframe=black,colback=white,sharp corners]
\textbf{{\Large \ding{45}}Example:}\\\\
	We take $\alpha=5\%$, then we have:
	
	That is to say:
	
	On production of $300$ elements we found that $8$ were defect. What is the confidence interval?

	We check first with:
	
	that:
	
	So it is acceptable to use the confidence interval by the Normal distribution. We therefore have:
	
	Thats is to say:
	
	\end{tcolorbox}
	To conclude this subject, we can obviously be interested to the number of individuals (sample size) necessary to satisfy a certain (imposed) confidence interval accuracy when having an imposed standard deviation.

	We therefore have according to the above assumptions and in the acceptance of the approximation by a Normal distribution:
	
	And by proceeding in an identical manner to developments made above with the Normal distribution, we obtain:
	
	obviously we normally take the integer value in practice...

	A question that often comes up in practice is the fact to know whether you have to take a unilateral or bilateral test. In fact there is no precise answer, it depends on what we want to highlight.
	
	\begin{tcolorbox}[title=Remark,colframe=black,arc=10pt]
	The size of the parent population for the relations developed above does not come into consideration in the calculation of confidence intervals or in one of the sample size, and because it is considered infinite. So be careful to not have sometimes sample sizes that are larger than the possible real parent population... 
	\end{tcolorbox}
	\begin{tcolorbox}[colframe=black,colback=white,sharp corners]
\textbf{{\Large \ding{45}}Example:}\\\\
	We would like to know the number of individuals (sample size) to take in a production lot knowing that the proportion of defective units is imposed at $30\%$ with a tolerated error of about $5\%$ between the actual and empirical proportion and to obtain a confidence interval at a level of $95\%$ of the result:
	
	\end{tcolorbox}
	\begin{tcolorbox}[title=Remark,colframe=black,arc=10pt]
	The last relation is very often used in sampling theory (analysis for referendum with responses of type: Yes/No) where sometimes the sample size $n$ is imposed for costs reasons of the survey and for which we seek to calculate the uncertainty $\delta$ and sometimes the reciprocal (the uncertainty is imposed and therefore we seek to know the sample size). 
	\end{tcolorbox}
	
	\pagebreak
	\paragraph{Test of equality of two Proportions}\mbox{}\\\\
	Always in the same context as the previous approximation of the binomial distribution by a Normal distribution, the industry (especially biostatistics) likes to compare two proportions of two different populations to see if they are statistically equal or not (i.e. said statistically significantly different or not).

	Therefore, let us recall that we have proved the stability of the Normal distribution if two random variables are independent and identically distributed (according to a Normal distribution!):
	
	Under the above assumptions it is then approximately the same for the difference of two proportions:
	
	Therefore we know that this new reduced centered variable follows a Normal distribution as:
	
	and as we seek to know the cumulative probability that the mean of the difference is zero, the latter relation is reduced in this case to:
	
	Obviously we can build (as always...) a confidence interval from this relation.

	However, it seems that the latter approximate relation following the return on experience is more correct when taking a modified denominator:
	
	which $\hat{p}$ will be taken as the mixture of the two populations. That is to say:
	
	thus (by changing the notations of the indices of the experimental proportions):
	
	This test is named "\NewTerm{two proportions $Z$ test}\index{two proportions $Z$ test}" or more simply "\NewTerm{two proportions $p$ test}\index{two proportions $p$ test}". In medicine, it is named the "\NewTerm{test of differences in risk}\index{test of differences in risk}" (meaning implicitly that each proportion is a segment of the studied population in relation to an undesired event).
	\begin{tcolorbox}[colframe=black,colback=white,sharp corners]
\textbf{{\Large \ding{45}}Example:}\\\\
	In the context of a sampling plan (\SeeChapter{see section Industrial Engineering}) we have taken on a first batch of $50$ individuals, $48$ that are in perfect conditions. In a second batch of $30$ individuals, $26$ were in perfect condition.\\

	Thus we have:
	
	We would like to know if the difference is statistically significant with a $95\%$ confidence interval or simply due to chance. We then use:
	
	and:
	
	This corresponds to a cumulative probability using Microsoft Excel 11.8346:

	\begin{center}
		\texttt{=NORMSDIST(1.535)=93.77\%}
	\end{center}

	Therefore the difference is due to chance (that said it is almost in extremis...). In other words, it is not statistically significant under the set constraints.
	\end{tcolorbox}
	
	\pagebreak
	\subsubsection{Sign Test}
	We measure something on a sample and later, we measure the same thing on this same sample but with a different method (so it is therefore paired samples!). Both ordered rankings of measures are compared and too each case is assigned a sign ("+" for an increase in the rankings, "-" in case of descent). Those who remain at the same level are eliminated.

	According to the hypothesis to be tested, there are so many "+" as "-", that is to say, the median of the distribution has not changed (this statement may not seem obvious at first reading so be aware to take time to think about it).

	The idea is that for each pair of values, there are only two possible signs of change, we have a chance on two ($50\%$ probability) that the difference is positive or negative. This test is based only on the study of signs of the differences between the pairs of individuals, regardless of the values o these differences.

	We can wish to control two assumptions:

	\begin{itemize}

		\item The inequality of proportions of signs must be statistically significant. So one of the two signs must be in a small number compared to the other, which corresponds to a left-sided test (the cumulative probability of the small number of characters must be below a certain level $\alpha$).
		
		\item The proportion of the two signs must be low unbalanced $(P(+)=P(-)=0.5)$. It is therefore in this case a bilateral test (the most common case) with a given level $\alpha$.

	\end{itemize}

	To create such a test, we consider the appearance of the "+" and "-" as a binary random sampling system where the order of success is not taken into account (it is therefore based on a binomial or hypergeometric distribution) and with replacement (which immediately eliminates the hypergeometric distribution that is not symmetric and problematic to use in practice...). To consider a random sampling with replacement (with the fact that we do not reinject each individual in reality), it is necessary that the population $N$ is large. This is why the sign test considers that the paired values should be continuous (which allows verbatim to approach the hypergeometric distribution by the binomial distribution). However, some statistical software use the hypergeometric distribution for the sake of precision.
	
	\begin{tcolorbox}[title=Remark,colframe=black,arc=10pt]
	You should know that most statistical software, do implicitly the assumption in this test that the data are continuous and use therefore the binomial distribution.
	\end{tcolorbox}
	\begin{tcolorbox}[colframe=black,colback=white,sharp corners]
\textbf{{\Large \ding{45}}Example:}\\\\
	Consider two sets of measurements with two different methods. We would like to test the hypothesis with a confidence level $\alpha$ of $5\%$ if the difference between the two methods is statistically significant (thus we expect a balance of signs). This is therefore two samples sign test (knowing that it is also possible to do the same by comparing the values of a single sample to its median):
	
	Therefore we have the differences:
	
	With the signs:
	
	Well it is already clear that the result will be the rejection of the hypothesis as there is no difference. But we will still do the calculations. As the test is a two-sided at the level of $5\%$ , the cumulative probability of obtaining at least two signs "+" must not be less than $2.5\%$ and not more than $97.5\%$ if we want to accept (not reject) the assumption as that the difference is not statistically significant.
	\end{tcolorbox}
	\begin{tcolorbox}[colframe=black,colback=white,sharp corners]
	We then have:
	
	Either with Microsoft Excel 14.0.6123:

	\begin{center}
		\texttt{=BINOMDIST(2,12,0.5,1)=1.928\%}
	\end{center}
	
	or if we don't do the approximation by being more accurate with the hypergeometric distribution:
	
	\begin{center}
		\texttt{=HYPGEOM.DIST(2,24/2,12,24,TRUE)=0.17\%}
	\end{center}
	
	which is not really better...!\\
	
	So the cumulative probability is less than $2.5\%$ and is by far not more than $97.5\%$, therefore we reject the hypothesis as that the difference is not statistically significant.
	
	We could accept the hypothesis if we take for $\alpha$ the value:
	
	

	but this is not the case!\\

	To conclude on this sign test (median test), we have for information some statistical software propose a confidence interval of the median based on the calculation method described previously (confidence interval of the binomial distribution). However, we believe that it would be to use bootstrapping as we have seen in the chapter on Numerical Methods, so won't introduce this technique here. In addition it may be useful to know that some make an approximation using the Normal distribution (as with most tests but we won't study this approximation in this context).
	\end{tcolorbox}	

	\subsubsection{Mood's Median Test}
	Here we will introduce a test that has many names: "\NewTerm{median test}\index{median test}", "\NewTerm{Mood's median test}\index{Mood's median test}" or "\NewTerm{Westenberg-Mood's median test}\index{Westenberg-Mood's median test}" or "\NewTerm{Brown-Mood's median test}\index{Brown-Mood's median test}" ...

	We consider two independent samples $(X_1,...,X_{n_1})$ and $(Y_1,...,Y_{n_2})$. We assume that $(X_1,...,X_{n1})$ is an independent and identically distributed sample from a continuous distribution $F$ and $(Y_1,...,Y_{n_2})$ is an independent and identically distributed sample from a continuous distribution $G$.

	After the grouping of the $n_1+n_2$ values of the two samples, $k=n_1M_n$ is the number of observations $X_i$ of the first sample that are greater than the median $N=n_1+n_2$ of the observations (the notation is not great because it can give the impression that is is a multiplication...).

	Under the null hypothesis that $X$ and $Y$ variables follow the same continuous distribution (that is to say, $G = F$) hypothesis, the variable $k=n_1M_n$ can take the values $0,1,...,n_1$ according to the hypergeometric distribution:
	
	Therefore, we can calculate the unilateral cumulative probability of having $k$. Mood's test is also a purely unilateral test.
	\begin{tcolorbox}[colframe=black,colback=white,sharp corners]
\textbf{{\Large \ding{45}}Example:}\\\\
	Consider the two samples:
	
	The overall median calculated with Microsoft Excel 14.0.6123 is $26.10$. We have a total of:
	
	Then it comes with Microsoft Excel 14.0.6123:

	\begin{center}
		\texttt{=HYPGEOM.DIST(8,26/2,13,26,TRUE)=94.24\%}
	\end{center}

	So at a threshold of $5\%$, we do not reject the null hypothesis (but... being close to the limit this is a bit tedious to conclude that ...). If we do the same calculation using the binomial distribution we obtain:

	\begin{center}
		\texttt{=BINOM.DIST(8,26/2,0.5,1)=86.65\%}
	\end{center}

	But obviously here the approximation does not apply since a binomial approximation is acceptable in practice when the sample is about $10$ times smaller than the population.
	\end{tcolorbox}
	\begin{tcolorbox}[title=Remark,colframe=black,arc=10pt]
	Unfortunately, there are several versions of the Mood test. For example, a software such as Minitab compares thanks to a contingency table... the values above or below the median and made a simple chi-square test of independence (Pearson test) as seen in the Chapter of Numerical Methods. 
	\end{tcolorbox}
	
	\subsubsection{Poisson Test (1 sample)}
	We know that a number of rare events follow a Poisson distribution. We can then allow us as for any other distribution to calculate the cumulative probability in a given interval (bilateral or unilateral).

	So if we have a discrete random variable following a Poisson distribution:
	
	We then have to a certain right sided level of confidence $\alpha$, the closest value $n$ of $k$ satisfying the condition:
	
	

	So to find the value $n$ (strictly positive integer or null value) we should reverse the sum, which is not... something funny to do (this is why most spreadsheet software do not offer at this day the inverse of the Poisson distribution).

	Now recall that we have seen in the section on Sequences And Series, the following Taylor (Maclaurin) series with full integral rest to order $n - 1$ around $0$ to $\lambda$:
	
	
	Result we had also given in the form of functions for Microsoft Excel 14.0.6123 so that the reader can verify this equivalence:
	
	\begin{center}
		\texttt{=POISSON(}$x \in \mathbb{N},\mu,$\texttt{TRUE)}\\
		\texttt{=1-CHIDIST(}$2\mu,2(x+1),$\texttt{TRUE)}
	\end{center}
	Then it follows that in spreadsheets softwares, we can use the inverse chi-square distribution to calculate the inverse of the Poisson distribution with this time however a small nuance: the result will not necessarily be an integer.

	If for example we take (always with Microsoft Excel 14.0.6123)
	
	\begin{center}
		\texttt{=1-CHI.DIST(2*20,2*(15+1),TRUE)=15.6513135\%}
	\end{center}
	
		The question is then to find the notation for the opposite ... This is then given by (we divide by $2$ to fall back on the mean that is the value of interest):
		
	\begin{center}
		\texttt{=CHIINV(1-15.6513135\%,2*(15+1))/2=15.53194258}
	\end{center}
	
	Finally, the notation of the inverse is relatively natural. Thus, the "1 sample Poissons' test" at a given right sided $\alpha$ level can be written:
	\begin{center}
		$k\leq $\texttt{=CHIINV(1-alpha,2*(number of measures+1))/2}
	\end{center}
	
	Formally:
	
	Note however one thing! It seems that some statistical software approximate sometimes with abuse the Poisson distribution by a Normal distribution. Therefore, the unilateral interval is calculated with:
	

	But with the Poisson distribution, remember that we have:
	
	Therefore:
	
	\begin{tcolorbox}[colframe=black,colback=white,sharp corners]
\textbf{{\Large \ding{45}}Examples:}\\\\
	 A company manufactures televisions in a constant quantity and has measured the number of defective product produced each quarter for the past ten years (so $4$ times $10$ measures). The stakeholders determines the maximum acceptable number of defective units is $20$ per quarter and wants to determine if the production satisfies these requirements (under the assumption that the distribution of defective follow a Poisson distribution) at a confidence level of $5\%$.\\
	
	The $40$ measures give us an average of:
	
	Then we have with the rough approximation:
	
	Either in a spreadsheet software like Microsoft Excel 14.0.6123:
	\begin{center}
		\texttt{=NORM.S.INV(1-5\%)*SQRT(20/40)+17.825=18.988}
	\end{center}
	or:
	
	Either in a spreadsheet software like Microsoft Excel 11.8346:
	\begin{center}
		\texttt{=CHIINV(1-5\%,2*(20+1))/2=14.072 }
	\end{center}
	\end{tcolorbox}
	
	Obviously, in the bilateral case, we have:
	
	
	\begin{tcolorbox}[colframe=black,colback=white,sharp corners]
\textbf{{\Large \ding{45}}Example:}\\\\
	 An airline company had $2$ crash on $1,000,000$ flights (very rare event). What is two-sided the confidence at the level of $95\%$ knowing that globally the number of accidents per million is $0.4$.\\

	We have therefore:
	
	Either the upper bound with a spreadsheet software like Microsoft Excel 11.8346 is given by:
	\begin{center}
		\texttt{=CHIINV(1-5\%/2,2*(2+1))/2=7.224}
	\end{center}
	and for the lower bound:
	\begin{center}
		\texttt{=CHIINV(1-5\%/2,2*(2+1))/2=0.618 }
	\end{center}
	So statistically, the company is less secure than all companies.
	\end{tcolorbox}
	
	\subsubsection{Poisson Test (2 samples)}
	
	We have just seen that:
	
	
	However, following the same reasoning that led us to construct the following test of average comparison:
	
	or its equivalent with the Student distribution when the true standard deviation is not known and using the fact that we have shown that the Poisson distribution is stable by the addition (and hence by subtraction), that Gamma distribution was also stable by the addition (and therefore by subtraction) and also the chi-square distribution since it is only a special case of the gamma distribution, we tend to write perhaps a little bit to fast the extension of what we have seen before:
	
	And in the facts this is a trap as say some practitioners ... because the chi-square distribution has a support which is defined as being strictly positive and the confidence interval could naturally have a negative left terminal (... O\_o). One solution could be to use the test of the difference of two proportions that we have already discussed earlier:
	
	Of course, only in the case that the conditions for approaching the test with a Normal distribution are met (proportions have to be typically less than $0.1$ and no greater than $50$).
	
	Most softwares seem to have implemented this latter method (with which I do not necessarily agree).
	
	\begin{tcolorbox}[colframe=black,colback=white,sharp corners]
\textbf{{\Large \ding{45}}Example:}\\\\
	An airline company had $2$ airplanes crash was in $1,000,000$ flights (very rare event). Another company had $3$ crashes in $1,200,000$ flights. What is the two sided confidence interval at the level of $95\%$ assuming that the difference is zero.\\

	Therefore, the proportions are:
	
	We write:
	
	
	Then we have:
	
	which gives a confidence interval for the expected theoretical proportion difference.
	
	
	and therefore as $-0.0000005$ is in this interval, we accept the hypothesis as the difference of proportions are not statistically significant at the threshold of $5\%$.

	Or taking the non approximated expression, we have (with the same conclusion):
	\begin{gather*}
		\dfrac{\chi_{5\%/2}^2(2(3+1)-2(2+1))}{2}\leq \lambda_2-\lambda_1\leq \dfrac{\chi_{1-5\%/2}^2(2(3+1)-2(2+1))}{2}\\
		\Downarrow\\
		0.0253 \leq \mu \leq 3.6889
	\end{gather*}
	\end{tcolorbox}
	
	So to summarize some convergence of distributions in all these different tests and intervals that we have seen so far, we offer the reader the following diagram that we hope... will clarify perhaps more or less things:
	\begin{figure}[H]
		\centering
		\includegraphics{img/arithmetics/convergence_criteria_statistical_inference}
		\caption{Convergence of different customary distributions in elementary statistical inference}
	\end{figure}
	And also this table where all relations have been demonstrated in detail above, and some already used (others will be used later):
	
	
	\pagebreak
	\subsubsection{Confidence/Tolerance/Prediction Interval}
	Here we go and in order to avoid frequent confusion and before moving on to more complex subjects, we will compare the confidence interval, the tolerance interval (often named "fluctuation interval" in some school programs) and finally the prediction interval.	

	\textbf{Definitions (\#\mydef):}
	
	\begin{enumerate}
		\item[D1.] The "\NewTerm{tolerance interval}\index{tolerance interval}" (or "fluctuation interval") is an interval containing a certain percentage (usually $68.26, 95.44 \text{ or } 99.73\%$ in the case of a Normal distribution) of individuals in a population of measures.
		
		\item[D2.] The "\NewTerm{confidence interval}\index{confidence interval}" for a sample mean (or proportion $p$) contains the interval value to a given confidence level (usually $90, 95$ or $99\%$ in the case two sided case) for the expected mean (true average) or the proportion of the population.
		
		\item[D3.] The "\NewTerm{prediction interval}\index{prediction interval}" is used to determine an interval for a single value based on the knowledge of the sample mean and the standard deviation of the population.
	\end{enumerate}
	
	An example being more often better than a thousand words, consider the case where the mean and the standard deviation of prices are $49$ DVD are given by:
	
	Therefore we have:
	
	corresponding to the tolerance intervals according to a Normal distribution of $68.26, 95.44$ and $99.73\%$.

	But a a confidence interval of $95\%$ based on the relation proved above:
	
	gives:
	
	So $95\%$ cumulative probability that the true mean is between $31.32$ and $31.78$.
	\begin{figure}[H]
		\centering
		\includegraphics{img/arithmetics/tolerance_confidence_interval.jpg}
		\caption[]{Histogram of the prices of a sample of 49 DVD}
	\end{figure}
	Now lets us introduce a new concept that is rarely addressed in the statistics literature. The idea of the prediction interval is rather than look at the confidence interval of the mean based on an experimental average, to use this experimental average (sample mean) as a basis for predicting the interval of a single value (and no not of the average!).

	We'll look at the difference between the mean and a punctual value:
	
	that we will assume close to zero (it is better to have a reliable product and pass the tests to obtain the authorization of sales...). About the variance, what interests us is not just the standard deviation of the mean anymore, but the standard deviation of the difference... and as the sample is independent of the unique value we have:
	
	So we can write as a first approximation:
	
	And of course after what we saw:
	
	So we can build verbatim the prediction interval:
	
	
	\subsection{Weak Law of Large Numbers}
	We will now focus on a very interesting relation in statistics that can tell a lot of things while having little information and whatever the distribution (which is not bad!). This is a widely used property for example in statistical simulation in the context of the use of Monte Carlo technics.
	
	Given a random variable with values in $\mathbb{R}^+$. Then we will show the following relation named "\NewTerm{Markov inequality}\index{Markov inequality}":
	
	with in the particular context of probabilities $\text{E}(X)\leq \lambda$.

	In other words, we propose to prove that the probability that a random variable is greater than or equal to $\lambda$ is less then or equal to its mean divided by the value $\lambda$ considerated and regardless of the probability distribution of the random variable $X$!
	
	\begin{dem}
	Let us write the values of $X$ by $(x_1,...,x_n)$, where $0\leq x_1 < x_2 < ... <x_n$ (that is to say sorted in ascending order) and let us also write $x_0=0$. We note first that the inequality is trivial in case where $\lambda \geq x_n \geq 0$. Indeed, as $X$ can be included only between $0$ and $x_n$ by definition then the probability that it is greater to $x_n$ is equal to zero. In other words:
	
	
	
	and $X$ being positive, the $\text{E}(X)$ is also positive, thus the inequality in this special case in a first time. 
	
	Otherwise, we have $0<\lambda\leq x_n$ and then there exist one $k\in (1,...,n)$ such that $x_{k-1}<\lambda \leq x_k$. Thus:
	
	
	\begin{flushright}
		$\square$  Q.E.D.
	\end{flushright}
	\end{dem}
	
	\begin{tcolorbox}[colframe=black,colback=white,sharp corners]
\textbf{{\Large \ding{45}}Example:}\\\\
	We assume that the number of outgoing parts of a given factory during one week is a random variable of mean $50$. If we wish to estimate the cumulative probability that production exceeds $75$ parts we will simply apply:
	
	\end{tcolorbox}
	
	Consider now a kind of generalization of this inequality named "\NewTerm{Bienayme-Chebyshev inequality}\index{Bienayme-Chebyshev inequality}" (abbreviated "\NewTerm{BC inequality}") that will allow us to get a very very interesting and important result a little bit later.

Consider a real random variable $X$ (so we do not limit ourselves to the only cases where it is in $\mathbb{R}^+$). Then we will prove the following Bienayme-Chebyshev inequality:
	
	which expresses the fact that the smaller the standard deviation is, more the probability that the random variable $X$ moves away from its expectation is low.
	
	\begin{dem}
	We obtain this inequality by first writing:
	
	where the choice of the square will serve us for a future simplification.

	Then by applying Markov's inequality (as you see is something that can be useful ...) to the random variable $Y=\left[X-\text{E}(X)\right]^2$ with $\lambda=\epsilon^2$ it comes automatically:
	
	Then, using the definition of the variance:
	
	We get:
	
	\begin{flushright}
		$\square$  Q.E.D.
	\end{flushright}
	\end{dem}

	If we put:
	
	The equality will be written:
	
	and expresses the cumulative probability in order that $X$ moves away from the mean of more than $t$ times its standard deviation, is below $1/t^2$. There is, in particular, less than $1$ chance on $9$ that $X$ moves away from its mean by more than three times the standard deviation. This is also this theorem that is used by the Basel Committee to define the Value At Risk correction factor used in finance (\SeeChapter{see section Economy}).
	
	\begin{tcolorbox}[colframe=black,colback=white,sharp corners]
\textbf{{\Large \ding{45}}Example:}\\\\
	We take the example where the number of outgoing parts of a given factory during one week is a random variable of mean $50$. We assume in addition that the variance of the weekly production is $25$. We seek to calculate the probability that the production of next week is between $40$ and $60$ pieces.
	
	To calculate this we must first remember that the BC inequality is based in part on the term $\vert X-\text{E}(X) \vert$ thus we have:
	
	Then we just have to apply numerical values to the inequality:
	
	\end{tcolorbox}
	
	The last two inequalities obtained before the example will allow us to obtain a very important and powerful relation that we call "\NewTerm{weak law of large numbers WLLN}\index{weak law of large numbers}" or "\NewTerm{Khinchin theorem}\index{Khinchin theorem}".

	Consider a random variable $X$ having a variance and $(X_n)_{n \in \mathbb{N}^*}$ a sequence of independent random variables (i.e. uncorrelated by pairs ) of the same distribution as this of $X$ and all having the same mean $\mu$ and the same standard deviations $\sigma$.

	What we will show is that if we measure the same random quantity $X_n$ of the same distribution in the process of a series of independent experiments (so in this case, we say technically that the sequence $(X_n)_{n \in \mathbb{N}^*}$ of random variables is defined on the same probability space) , then the arithmetic average of the observed values will stabilize on the mean of $X$ when the number of measurements tends to infinity.
	
	when $n\rightarrow +\infty$ this is the very important result which we did mention above! The empirical estimator of the mean tends for any distribution to the true hope if $n$ is large! So by the this rule we assure that the sample average is a consistent estimator of mean! This result (quite intuitive) is sometimes named the "\NewTerm{fundamental theorem of Monte Carlo}\index{fundamental theorem of Monte Carlo}" because it is central to the simulations principle of the same name (see Numerical Methods), which are crucial in the study of advanced statistics.
	
	So in other words, the cumulative probability that the difference between the arithmetic average and the expected of the observed random variables to be in a given range around the average tends to zero as the number of measured random variables tends to infinity (that which is ultimately intuitive).
	
	This result allows us to estimate the expected mean value using the empirical mean (arithmetic average) calculated on a very large number of experiments.
	\begin{dem}
	We use the Chebyshev-Bienaymé inequality for the random variable (this relation is difficult to interpret but allows us to get the desired result):
	
	And we first calculate using the mathematical properties of the mean that we proved earlier:
	
	and in a second time using the mathematical properties of the variance as already proved above:
	
	and since we assumed uncorrelated variables then the covariance between them is zero therefore:
	
	So by injecting it into the BT inequality:
	
	that becomes:
	
	and the inequality tends effectively to zero as well $n$ at the denominator tends to infinity.
	\begin{flushright}
		$\square$  Q.E.D.
	\end{flushright}
	\end{dem}
	
	Note that the latter relation is often denoted in some works, and according to what we saw at the beginning of this section:
	
	or:
	
	Therefore for $\forall \varepsilon >0$:
	
	
	\subsection{Characteristic Function}
	In probability theory and statistics, the characteristic function of any real-valued random variable completely defines its probability distribution. If a random variable admits a probability density function, we will prove further below that then the characteristic function is the inverse Fourier transform of the probability density function. Thus it provides the basis of an alternative route to analytical results compared with working directly with probability density functions or cumulative distribution functions. 
	
	Before giving an engineer proof of the famous central limit theorem, first let us introduce the concept of "characteristic function", which is central in statistics.
	
	First, remember that the Fourier transform is given his in its physicist version (\SeeChapter{see section Sequences and Series}) by the relation:
	
	Let us recall that the Fourier transform is an analogue of the Fourier series theory for non-periodic functions, and allows to associate them a frequencies spectrum. To a given factor, it is a "\NewTerm{bilateral Laplace transform}\index{bilateral Laplace transform}" given by:
	
	where $p$ is the complex variable  given in the present case by (the real part is zero, because the Fourier transform is the particular case of a Laplace transform whose real part of the variable is zero: then make Fourier transform is like make a Laplace transform on the axis of imaginary numbers only):
	
	Now we want to prove that if:
	
	In other words, we are looking for a simplified expression of the Fourier transform of the derivative of $f(x)$.
	
	\begin{dem}
	We start from:
	
	An integration by parts gives (\SeeChapter{see section Differential and Integral Calculus}):
	
	By imposing that $f$ tends to zero at infinity, then we have:
	
	and:
	
	\begin{flushright}
		$\square$  Q.E.D.
	\end{flushright}
	This is the first result we needed.
	\end{dem}
	
	Now let us prove that if:
	
	
	\begin{dem}
	So we start from:
	
	\begin{flushright}
		$\square$  Q.E.D.
	\end{flushright}
	This is the second result we needed.
	\end{dem}
	
	Now let us perform the calculation of the Fourier transform of the Normal centered-reduced distribution (this choice is not innocent...):
	
	We know that this latter relation is trivially solution of the following differential equation (or it satisfies it in other words...):
	
	taking the Fourier transform of both sides of the equality, we use the previous two results:
	
	We have:
	
	Or:
	
	Then after integration:
	
	As:
		
	We therefore have:
	
	We proved during our study of the Normal distribution that:
	
	Therefore:
	
	We then have (very important result!):
	
	Let us now introduce the characteristic function as defined by statisticians:
	
	which is an important and powerful analytical tool for analyzing a sum of independent random variables. In addition, this function contains all the information characteristic of the random variable $X$.
	\begin{tcolorbox}[title=Remark,colframe=black,arc=10pt]
	The notation is not innocent since the $\text{E}[...]$ represents an expected mean of the density function with respect to the complex exponential.
	\end{tcolorbox}
	Therefore the characteristic function of the Normal reduced centered random variable of distribution law:
	
	becomes easy to determine because:
	
	This is why the characteristic function of the reduced centered Normal distribution is often assimilated to a simple Fourier transform (\SeeChapter{see section Sequences and Series}).
	
	And thanks to the previous result:
	
	Therefore:
	
	which is the result we will need for the central limit theorem that we will study just after. This characteristic function is equal, to a given constant, to the probability density of the law. Then we say that the characteristic function of a Gaussian function is Gaussian function....
	
	But before that, let us look a little closer this characteristic function:
	
	Using a Maclaurin development (\SeeChapter{see section Sequences and Series}) and changing some notations we get:
	
	and by inverting the sum and the integral, we have:
	
	This characteristic function contains therefore all the moments (general term used for the mean and variance) of $X$.
	
	
	
	\subsection{Central Limit Theorem}
	The central limit theorem is a set of results from the early 20th century on the weak convergence in probability of a sequence of random variables. Intuitively, from these results, any sum (implicitly: the average of these variable) of independent random variables identically distributed tends to some given random variable. The best known and most important result is simply named "\NewTerm{central limit theorem}\index{central limit theorem}" concerning a sum of independent random variables with existing variance whose number tends to infinity and it is this that we will prove heuristically here.
	
	In the simplest case, considered below for the proof of the theorem, these variables are continuous, independent and have the same mean and the same variance. To try to get a finite result, we should center this sum by subtracting its average and reduce it by dividing it by its standard deviation. Under fairly broad conditions, the probability distribution (of the average) converges to a Normal centered reduced distribution. The ubiquity of the Normal distribution is explained by the fact that many considered random phenomena are due to the superposition of an infinity of many small causes.
	
	This probability theorem has an interpretation in mathematical statistics. The latter associates a probability distribution to a population. Each element extracted from the population is considered as a random variable and by bringing together a number $n $ of these supposed independent variables, we get a sample. The sum of these random variables divided by $n$ gives a new variable named the "empirical mean". This, once reduced, tends to a Normal reduced variable as $n$ approaches infinity as we know.
	
	The central limit theorem tells us what we should expect in terms of sums of independent random variables. But what about the products? Well, the logarithm of a product (strictly positive factors) is the sum of logarithms of factors, so that the logarithm of a product of random variables (with strictly positive values) tends to a Normal distribution, resulting a log-normal distribution for the product itself (see our study of the Log-Normal distribution earlier above).
	
	In itself, the convergence to the Normal distribution ("\NewTerm{asymptotic normality}\index{asymptotic normality}") of many random variables when their number tends to infinity only interests the mathematician. For the practitioner, it is worth stopping shortly before the limit: the sum of a large number of these variables is nearly Gaussian, which often provides a more usable approximation than the exact law.
	
	Conversely, we can say that no concrete phenomenon is really Gaussian because it can not exceed certain limits, especially if its values are all positive (or all negative).
	
	\begin{dem}
	Given $\left\lbrace X_i \right\rbrace_{i=1...+\infty}$ a suite (sample) of continuous random variables (in our simplified proof...), independent (independent measures of physical or mechanical phenomena for example) and identically distributed, for which the average  $\mu_X$ and standard deviation $\sigma_X$ exist (this means that the central limit theorem works for finite variance phenomena only as far as we know!!!).
	
	We saw earlier in this section that:
	
	are the same expressions of centered reduced variable generated using a sequence of $n$ identically distributed random variables that by construction therefore has a zero mean and unit variance:
	
	Let us develop the first form of the prior-previous  equality (the both are anyway equal!):
	
	Now using the characteristic function of the Normal centered-reduced distribution  (we simplify at the same time the notations of the estimators of the average and standard deviation):
	
	As the random variables $X_i$ are independent and identically distributed, we get:
	
	A Taylor expansion (\SeeChapter{see section Sequences and Series}) the term between braces gives at the third order (Maclaurin series expansion of the exponential):
	
	Finally:
	\end{dem}
	
	
	Let us put:
	
	Then we have:
	
	So as $x$ approaches infinity (\SeeChapter{see section Functional Analysis}):
	
	We find back the characteristic function of the reduced centered Normal distribution!
	
	In two words, the Central Limit Theorem (CLT) said that for large samples, the centered and reduced sum of $n$ random variables independent ad identically distributed follow a Normal centered and reduced distribution. And so we have verbatim to the empirical mean:
	
	
	Now we will illustrate the central limit theorem in the case of a sequence $\left\lbrace X_i \right\rbrace$ of discrete random variables following a Bernoulli distribution with parameter equal to $1/2$.
	
	We can imagine that $XN$ represents the result obtained in the $n$-th launch of a coin (assigning the number $1$ for head and $0$ for tail). Let us write:
	
	the average. We obviously have for all $n$ in this special case:
	
	and therefore:
	
	After having centered and reduced $\bar{X}_n$ we get:
	
	Let us note $\Phi$ as always the cumulative Normal reduced centered distribution.
	
	The central limit theorem says that for any $t \in \mathbb{R}$:
	
	Using Maple 4.00b we have plot in blue some graphs of the function:
	
	for different values of $n$. We have Represented in red the function $\Phi$.
	
	For $n=1$:
	\begin{figure}[H]
		\centering
		\includegraphics{img/arithmetics/clt_bernoulli_n_1.jpg}
		\caption{First approach of the Bernoulli distribution by the Normal distribution according to the CLT}
	\end{figure}
	For $n=2$:
	\begin{figure}[H]
		\centering
		\includegraphics{img/arithmetics/clt_bernoulli_n_2.jpg}
		\caption{Second approach of the Bernoulli distribution by the Normal distribution according to the CLT}
	\end{figure}
	For $n=5$:
	\begin{figure}[H]
		\centering
		\includegraphics{img/arithmetics/clt_bernoulli_n_5.jpg}
		\caption{Fifth approach of the Bernoulli distribution by the Normal distribution according to the CLT}
	\end{figure}
	For $n=30$:
	\begin{figure}[H]
		\centering
		\includegraphics{img/arithmetics/clt_bernoulli_n_30.jpg}
		\caption{Thirteenth approach of the Bernoulli distribution by the Normal distribution according to the CLT}
	\end{figure}
	These graphs were obtained with Maple 4.00b using the following commands:
	
	\texttt{>with(stats):\\
	>with(plots):\\
	>e1:=plot(Heaviside(t+1)*statevalf[dcdf,binomiald[1,0.5]](trunc((t+1)/2)),t=-2..2\\
	,y=0..1,color=blue):\\
	>e2:=plot(Heaviside(t+sqrt(2))*statevalf[dcdf,binomiald[2,0.5]]\\
	(trunc((t*sqrt(2)+2)/2)),t=-sqrt(2)-1..sqrt(2)+1,y=0..1,color=blue):\\
	>e3:=plot(Heaviside(t+sqrt(5))*statevalf[dcdf,binomiald[5,0.5]]\\
	(trunc((t*sqrt(5)+5)/2)),t=-sqrt(5)-1..sqrt(5)+1,y=0..1,color=blue):\\
	>e4:=plot(statevalf[cdf,normald](t),t=-5..5):\\
	>e5:=plot(Heaviside(t+sqrt(30))*statevalf[dcdf,binomiald[30,0.5]]\\
	(trunc((t*sqrt(30)+30)/2)),t=-sqrt(30)-1..sqrt(30)+1,y=0..1,color=blue):\\
	>display({e1,e4});\\
	>display({e2,e4});\\
	>display({e4,e3});\\
	>display({e5,e4});}
	
	clearly show the convergence of $F_n$ to $\Phi$.
	
	In fact we see that convergence is downright uniform which is confirmed by the "\NewTerm{Moivre-Laplace central limit theorem}\index{Moivre-Laplace central limit theorem}":
	
	Given $X_n$ a sequence of independent random variables with the same Bernoulli parameter $p$, $0<p<1$. Therefore:
	
	tends uniformly to $\Phi(t)$ on $\mathbb{R}$ when $n \rightarrow +\infty$.
	
	\subsection{Univariate Hypothesis and Adequation tests}
	During our study of confidence intervals, remember that we get the below few relations (it is only a sample of the most important one proved above!):
	
	and:
	
	and:
	
	and finally:
	
	that allowed to do statistical inference based on the knowledge or not of the true mean or variance of the whole or on a sample of the population. In other words, under in what range stood a given moment (mean or variance) as a function of the chosen confidence level $\alpha$. We had seen that the second interval above can only be hardly used in practice (assumes known the theoretical average) and therefore we prefer the third.
	
	We will also prove in details later the following two intervals:
	
	and:
	
	The first interval above can be also used with difficulty in practice (assumes the theoretical average as know) and therefore we prefer the second.
	
	\textbf{Definition (\#\mydef):} When we want to know if we can trust the value of a statistic (mean, median, variance, correlation coefficient, etc.) with some certainty, we speak of "\NewTerm{hypothesis testing}\index{hypothesis testing}" and especially of "\NewTerm{conformity test}\index{conformity test}" (we speak of "\NewTerm{adequation test}\index{adequation test}" when the purpose is to check that measures will follow a particular distribution law).
	
	\begin{tcolorbox}[title=Remark,colframe=black,arc=10pt]
	The reader must also remember, as already said earlier in this section, that we have put many other confidence interval techniques detailed proofs related for example for regression techniques in the section of Theoretical Computing.
	\end{tcolorbox}	
	
	Hypothesis testing are intended to check whether the sample can be considered as extracted from a given population or be representative of this population vis-a-vis of parameter such as a mean, variance or the observed frequency. This implies that the theoretical distribution of the parameter is known at the population level. The hypothesis tests are not made to prove the null hypothesis (usually expressing equality or uniformity between different populations), but to eventually reject it (to be exact the rejection of the null hypothesis is more robust). In terms of the communication of statistical tests a number of experts recommend:
	
	\begin{enumerate}
		\item To always communicate the $p$-value with four decimal places (we will come back later on this concept).
		
		\item Never say that a low $p$-value shows a significant magnitude of the effect studied because it is not necessary true (to check you just take a phenomenon of very small amplitude on a large sample and then the $p$-value will become very small by construction). Once again will discuss this more deeply later.
		
		\item To always give the confidence interval of the test whether it is unilateral or bilateral.
		
		\item To be careful to not to set a rejection threshold to the test except if a standard (norm) or legislation requires it (in which case we will specify which one).
		
		\item To never say that the test is "proved" or "significant" or even "statistically significant". Just say that the result is "statistical" or that we have the "likelihood of the data knowing the null hypothesis" and that's it!
		
		 \item If the interest is to show the null hypothesis and that it is not rejected, since often its statistical power is low, you will need to repeat the experience to reinforce the conclusion.
		 
		 \item If the interest is to reject the null hypothesis and that this is true,  a good scientific practice is to look for additional studies that would faulty conclusion.
		 
		 \item If there is for example no statistical difference between two values, therefore this does not mean that there is presence of statistical equivalence. It is then necessary to proceed to "equivalence tests".
		 
		 \item The rejection of the null hypothesis does not mean that the mechanism of the phenomenon has been highlighted but just indicates, for refresh, an information about the size of the dataset a posteriori.
		 
		 \item We communicate the  posteriori power of the test.
	\end{enumerate}
	To resume, the studies must be communicated respecting the principle of truthfulness, after having been the subject of appropriate controls, and must be exposed, described and presented with impartially (some people say "\NewTerm{a-theoric}\index{a-theoric}"). We must not confuse between objectives and speculatives results. The conclusions should be the most faithful expression of the content of facts and datas.
	
	For example, if we want to know with some confidence whether a given mean of a population sample is realistic about the true unknown theoretical mean we will use the "\NewTerm{$Z$-test}\index{$Z$-test}" which is simply:
	
	Now remember that we have proved that if we have two random variables of law:
	
	then subtraction (differentiate) the averages gives:
	
	So for the difference of two average of random variables from two independent population samples we obtain directly:
	
	We can then adapt the $Z-test$ as:
	
	
	
	The relation that is very useful when for two samples of two populations, we want check if there is a statistically significant difference of the mean to a given fixed level of confidence level $\alpha$ and the associated probability for this difference:
	Therefore:
	
	
	
	We the speak of "\NewTerm{two-sample $Z$-test}\index{two-sample $Z$-test}" and it is used lost in the industry to ensure the equality of two averages of two measurements of populations when standard deviation is known.
	
	And if the theoretical standard deviation is not known, we use the "\NewTerm{Student $t$-test}\index{Student $t$-test}" (used a lot in pharmacoeconomics) proved above:
	
	In the same idea for standard deviation, we use the "Chi-square (variance) test" as already proved above:
	
	And when we want to test for the equality of variance of two populations we use "\NewTerm{Fisher $F$-test}\index{Fisher $F$-test}" (that will be proved below during our study of the analysis of variance):
	
	In practice we must be aware that the purpose of a test is very often to show that the effect is significant. It is then customary to say that the test is successful if the null hypothesis is rejected in favor of the alternative hypothesis! When the practitioner knows that the effect is significant and his test fails to reject the null hypothesis this is sometimes named the "\NewTerm{dilemma of not rejecting the null hypothesis}\index{dilemma of not rejecting the null hypothesis}". As we will see a little further, the idea is then to calculate an "\NewTerm{a posteriori test power}\index{a posteriori test power}" (it is then named by some software like SPSS, "\NewTerm{observed power}\index{observed power (statistics)}") and adapt the sample size accordingly to have an acceptable power according to tradition.
	
	\subsubsection{Direction of hypothesis test and $p$-values}
	The fact that we get all values satisfying a right bounded test AND (!) left bounded is what we name in the general case a "\NewTerm{two-tailed test}\index{two-tailed test}" as it includes the left sided and right sided unilateral tests. Thus, all the above tests are in a bilateral form, but we could make a unilateral use too! We use one-tailed test when the expected difference (or difference to highlight) can only go in one direction (typically in the case of clinical trials or during a corrective quality action in the industry for which we expect a improvement going in one single direction). Unilateral tests are sometimes named "\NewTerm{non-inferiority tests}\index{non-inferiority tests}" (left-sided) or "\NewTerm{non-superiority test}\index{non-superiority test}" (right-sided).
	
	Below we presented for example a right unilateral test (since the rejection region is on the right and therefore the cumulative probability is left-sided) and a two-sided test:
	\begin{figure}[H]
		\centering
		\includegraphics{img/arithmetics/unilateral_bilateral_test.jpg}
		\caption{Illustration of a test (or confidence level) unilateral right and bilateral}
	\end{figure}
	We can also summarize how to determine the $p$-value (which will be discussed more in detail further below) with the following diagram:
	\begin{figure}[H]
		\centering
		\includegraphics{img/arithmetics/p_values_construction.jpg}
		\caption{Resume figure to determine the $p$-value of parametrical test with symmetrical distributions.}
	\end{figure}
	
	\textbf{Definition (\#\mydef):} The "\NewTerm{$p$-value}\index{$p$-value}" is the probability of obtaining an effect at least as extreme as the one in your sample data, assuming the truth of the null hypothesis. In other words, if the null hypothesis is true, the $p$-value is the probability of obtaining your sample data. It answers the question, are your sample data unusual if the null hypothesis is true? If you are thinking that the $p$-value is the probability that the null hypothesis is true, the probability that you are making a mistake if you reject the null, or anything else along these lines, this is the most common misunderstanding of practitioners.
	
	The common misconception is what we'd really like to know. We would loooove to know the probability that a hypothesis is correct, or the probability that we're making a mistake. What we get instead is the probability of our observation, which just isn't as useful.
	
	It would be great if we could take evidence solely from a sample and determine the probability that the sample is wrong. Unfortunately, that's not possible and this for logical reasons when you think about it. Without outside information, a sample can't tell you whether it's representative of the population.
	
	The $p$-values are based exclusively on information contained within a sample. Consequently, $p$-values can't answer the question that we most want answered, but there seems to be an irresistible temptation towards interpreting it that way.
	
	Let us also note that the hypothesis tests on the standard deviation (variance), the mean or correlation are named "\NewTerm{parametric tests}\index{parametric tests}" in reverse of the non-parametric tests that we will see much further below.
	
	\begin{tcolorbox}[title=Remarks,colframe=black,arc=10pt]
	\textbf{R1.} There is also another definition of the concept of parametric and non-parametric tests that we will se later below (a little different because more precise).\\
	
	\textbf{R2.} Warning! Some authors or teachers sometimes talk about "left-sided" for a "right-sided test"... In fact it is simply a choice of vocabulary. If the reference of teaching is not the rejection area but the acceptance area, then it is clear that the right and left are concepts that are reversed...
	\end{tcolorbox}	
	Finally, many software calculates what we name therefore the "$p$-value" which is limit calculated risk (probability threshold) $\alpha$ that might have set the statistician to be at the limit between acceptance of the null hypothesis and its rejection (remember that a successful test does not prove anything and a rejection is better but still do not prove anything!). So the $p$-value is a fundamental value in statistics because it gives the possibility to quantify the likelihood of the null hypothesis  $H_0$ (acceptance or rejection).
	
	But strictly speaking the $p$-value is the conditional (Bayesian) probability, that our data satisfy the null hypothesis $P(data|H_0)$ and not the probability of the null hypothesis knowing the data! While the difference may be small as we saw in the section Probabilities, it is not zero! So in reality the $p$-value says nothing about the hypothesis itself, but it provides information on the experimental data.
	
	For hypothesis testing, for example, the $5\%$ risk threshold $\alpha$ is the risk to reject the null hypothesis even though it is true. If the risk imposed/chosen is $5\%$ and the calculated $p$-value is less (in most tests but be careful because this is not a generality!!!), the test fails (rejection of the null hypotheses) in favor of an alternative hypothesis denoted by $H_1$ or $H_a$. Never forget that reject the test is always better in term of power than accepting it.
	
	The alternative hypothesis has of course itself has its own risk that we denote by $\beta$ and its own $p$-value. So when the null hypothesis is not rejected, the risk associated with this decision is a "\NewTerm{risk of the second kind}\index{risk of the second kind}". To assess this, we should calculate the power of the test considered (see proofs later below).
	
	Perhaps, to better understand, here is an illustration of a particular case of an  bilateral hypothesis test of the average for a typical random variable following a Normal distribution (basically it's almost the same principle for all tests...):
	\begin{figure}[H]
		\centering
		\includegraphics{img/arithmetics/alpha_risk_beta_risk.jpg}
		\caption{Null and alternative hypothesis of a special case of two-sided test}
	\end{figure}
	Thus, in the case presented above, we see better why the null hypothesis can be accepted or rejected in favor of the alternative hypothesis (which is of the same law that the null hypothesis but just shifted) depending on the reference value measured that will be used for the test (in the special case of the above figure it is the arithmetic mean of measurements).
	
	We also note that the red area of the alternative hypothesis, corresponding to the cumulated probability $\beta$, is partially merged with the yellow part of the null hypothesis. This is why we can sometimes accept the null hypothesis wrongly. However, we see that smaller is $\beta$, more the alternative hypothesis would be far from the red boundary zone of the null hypothesis (this would correspond to a translation to the right in this case) and less the likelihood of a false conclusion is big. This is why we talk about "\NewTerm{risk $\beta$}\index{risk $beta$}" because smaller it is, the better. Verbatim more $1-\beta$ is big, then smaller is the risk of confusing the null and alternative hypothesis. This is why $1-\beta$ is named "\NewTerm{power of the test}\index{power of the test}" (see below the subsection that is devoted to this concept).
	
	We accept (sadly do not reject!) the null hypothesis if the $p$-value is greater than $5\%$ (0.05). In fact, bigger is the $p$-value is, the better it is because for people interested to the null hypothesis because the confidence interval is becoming smaller. If the confidence interval has to be huge (very close to $100\%$) for the $p$-value is very small so the analysis does not really make sense physically speaking in the point of view of the null hypothesis!
	
	Thus, if the $p$-value is low, that mean we should take a low risk $\alpha$ of error, therefore accept in almost all cases the tested hypothesis ($H_0$)...
	
	\begin{tcolorbox}[title=Remarks,colframe=black,arc=10pt]
	\textbf{R1.} We should never say that we "accept" a hypothesis or that it is "true" or "false" as these terms are too strong and might suggest a scientific proof. We should say if we "reject" or "does not reject" the null hypothesis and it is possibly "correct" or "incorrect".\\
	
	\textbf{R2.} For bilateral test hypotheses, we can for example say that we have (or not have) a significant difference between the measured reference value and the expected value. For one-sided tailed, we can say that the measured reference value is significantly bigger or smaller than the expected value.\\
	
	\textbf{R3.} Moreover if the reader has well understand the construction of hypothesis testing, the fact of wrongly reject a hypothesis ("\NewTerm{Type I error}\index{type I error}" or "\NewTerm{Error of the first kind}\index{Error of the first kind}") is more robust than accept it wrongly ("\NewTerm{Type II error}\index{type II error}" or "\NewTerm{Error of the second kind}\index{Error of the second kind}").\\
	
	\textbf{R4.} The reader will also have noticed with the help of the previous figure that a one sided test has a higher power than a one-sided test (for a same risk threshold of course!). Thus, a statistically insignificant difference in bilateral test, can be statistically significant in a one sided test.\\
	
	\textbf{R5.} If the $p$-value is close but not to the limit threshold (rejection) value we say that the "\NewTerm{effect is marginally significant}\index{marginally significant effect}".\\
	
	\textbf{R6.} A food company fills $18,000$ cereal boxes per shift, with a target weight of $360$ grams and a standard deviation of $2.5$ grams. An automated measuring system weighs every box at the end of the filling line. With that much data, the company can detect a difference of $0.06$ grams in the mean fill weight $90\%$ of the time. But that amounts to just one or two bits of cereal—not enough to notice or care about. The $0.06$-gram shift is statistically significant but not "\NewTerm{practically significant}\index{practically significant}". 
	\end{tcolorbox}	
	
	A big issue with hypothesis test is what we name "\NewTerm{$p$-hacking}\index{$p$-hacking}": the underlying idea consists of replicating a test dozens of times until it provides the conclusion that the experimenter like... or try all possible combinations of variables until we found something significant or even just take big samples (as we know that NHST tests fail systemically when samples are big).
	
	\textbf{Definitions (\#\mydef):}
	\begin{enumerate}
		\item[D1.] The probability $\alpha$ of Type I error (error of the first kind / false negative) is the probability of rejecting the null hypothesis when it is true.
		
		\item[D2.] The probability $\beta$ of Type II error (error of the second kind / false positive) is the probability of maintaining the null hypothesis when it is false. Since, by definition, power is equal to one $1-\beta$, the power of a test will get smaller as beta gets bigger.
	\end{enumerate}
	\begin{center}
	  \begin{tabular}{|l|c|c|c|}
	  \hline
	    \cellcolor{black!30}   & \multicolumn{2}{|c|}{\cellcolor{black!30}\textbf{State of Nature}} \\ \hline
	\cellcolor{black!30}\textbf{\parbox{3.5cm}{Decision resulting\\ from Data}} & $H_0$ \textbf{false} & $H_0$ \textbf{true} \\ \hline
	\textbf{Reject }$H_0$ & \cellcolor{green!30}\parbox{5.5cm}{Correctly reject null decision\\ \centering($1-\beta$: Power of the test)} & \cellcolor{red!30}\parbox{3cm}{Type I Error\\ \centering(Risk $\alpha$)} \\[3ex] \hline
	\textbf{Fail to reject }$H_0$ & \cellcolor{red!30}\parbox{3cm}{Type II Error\\ \centering(Risk $\beta$)} & \cellcolor{green!30}\parbox{3.5cm}{\centering Correct decision}  \\[3ex] \hline
	  \end{tabular}
	\end{center}
	Thus a traditional criteria selection is to use the following principle: among all the tests that have the same size of type I error, choose one that has the smallest size of the value of the Type II error.
	
	In general, the magnitude of the Type II error increases when that of the Type I error decreases. We can not, as far as we know, minimize the two errors at the same time. For this reason, we often take a fixed value of $\alpha$, size of the type I error, and we minimize $\beta$, magnitude of the Type II error (i.e. we increase the power $1-\beta$.
	
	To close this subject, here are the three types of situations testing hypotheses on the statistics that is the average in the framework of an underlying Normal distribution and whose mean is in this particular case zero and of unit variance (because we can often come back to this particular case by centering and reducing the underlying random variable or by doing special transformations):
	\begin{figure}[H]
		\centering
		\includegraphics{img/arithmetics/hypothesis_average_three_scenarios.jpg}
		\caption{The three possible scenarios for a hypothesis test on the average}
	\end{figure}
	Let us indicate that it makes no sense (as opposed to what we can sometimes read in some paper or electronic media) to have the following null hypotheses in the special case shown above:
	
	with the alternative hypothesis that automatically follows (we did not write it because it is useless and obvious). The reason is simple: how could you position your reduced centered Normal distribution if the mean is not fixed ... ??? This is why the null hypothesis in the context of the tests on the mean (and some other tests) are always written with an equality!
	
	To summarize, we can say that if we make a decision, we can make an error and it is best not to make mistakes often. Clearly, the probability of saying something stupid must be known and preferably small.
	
	\subsubsection{Fisher's method for multiple $p$-values}
	In statistics, "\NewTerm{Fisher's method}\index{Fisher's method}, also known as "\NewTerm{Fisher's combined probability test}\index{Fisher's combined probability test}", is a technique for data fusion or "meta-analysis" (analysis of analyses). It was developed by and named for Ronald Fisher. In its basic form, it is used to combine the results from several independent tests bearing upon the same overall hypothesis ($H_0$).

	Consider a set of $k$ independent tests, each of these to test a certain null hypothesis $H_{0|i}, i=\{1, 2, \ldots, k\}$. For each test, a significance level $p_{i}$, i.e., a $p$-value, is obtained. All these $p$-values can be combined into a joint test whether there is a global effect, i.e., if a global null hypothesis $H_0$ can be rejected.

	There are a number of ways to combine these independent, partial tests. The Fisher method is one of these, and is perhaps the most famous and most widely used. The test was presented in Fisher’s now classical book, \textit{Statistical Methods for Research Workers} (1932).

	The test is based on the fact that the probability of rejecting the global null hypothesis is related to intersection of the probabilities of each individual test, $\prod_i p_i$. However, $\prod_i p_i$ is not uniformly distributed (but each $p_i$ is assumed to be uniformly distributed!), even if the null is true for all partial tests, and cannot be used itself as the joint significance level for the global test. To remediate this fact, some interesting properties and relationships among distributions of random variables were exploited by Fisher and embodied in the succinct excerpt above. The proof is given below:
	\begin{dem}
	So we are looking for the distribution law of:
	
	but this seems quite difficult to do as we are dealing with a product. Then we take first logarithm (the natural one as we know in statistics that most of time in is the $\ln$ that appears:
	
	Therefore we have now a sum of the natural logarithm of uniform distribution $\mathcal{U}_{0,1}$. It's a bit better but now we have the problem of the $\ln\left(\mathcal{U}_{0,1}\right)$.

	But now let us see something nice! Remember that the cumulative distribution function of an expoential distribution is:
	
	The inverse is then given by:
	
	if $P$ is a random variable uniformly distributed in the interval $[0,1]$, so is $1-P$. As consequence the previous relation can be equivalently written as:
	
	where once again: $P=\mathcal{U}_{0,1}$. Therefore we have just found that the natural logarithm of a uniform random variable in the interval $[0,1]$ follows an exponential distribution with parameter $\lambda=1$!

	But now we have a small a problem. As we have a sum of exponential laws we have never proved if the exponential law is stable by addition...
	
	Let us now recall that the chi-squared distribution with $k$ degreen of freed is given by:
	
	If $k=2$ and we get:
	
	In other words, chi-squared distribution with two degrees of freedom is equal to a exponential distribution with $\lambda=1/2$!
	
	And remember now during our study of the Gamma distribution we have proved that it was stable by addition. And as we have proved that the chi-square distribution follows from a special case of the Gamma distribution, and just now that the exponential distribution follows from a special case of the chi-squared distribution, then that latter is also stable by addition with degrees of freedom!!!!
	
	Therefore multiplying the sum of logarithms by $-2$ to get rid of the $-$ coming from the exponential law and the $1/2$ coming from what bounds the chi-squared law to the exponential law, we get:
	
	from which a $p$-value for the global hypothesis can be easily obtained.
	\begin{flushright}
		$\square$  Q.E.D.
	\end{flushright}
	\end{dem}
	Under Fisher's method, two small $p$-values $p_1$ and $p_2$ combine to form a smaller $p$-value. The yellow-green boundary in the figure below defines the region where the meta-analysis $p$-value is below 0.05. For example, if both p-values are around $0.10$, or if one is around $0.04$ and one is around $0.25$, the meta-analysis p-value is around $0.05$:
	\begin{figure}[H]
		\centering
		\includegraphics[scale=0.7]{img/arithmetics/fisher_multiple_pvalues_test.jpg}
		\caption{Fisher's method for multiple $p$-values (source: Wikipedia)}
	\end{figure}
	\begin{tcolorbox}[title=Remark,colframe=black,arc=10pt]
	In the famous judgment case of Lucia de Berk the court made heavy use of statistical calculations to achieve its conviction. In a 2003 TV special of NOVA, Dutch professor of Criminal Law Theo de Roos stated: \textit{In the Lucia de Berk case statistical evidence has been of enormous importance. I do not see how one could have come to a conviction without it}. The law psychologist Henk Elffers, who was used by the courts as expert witness on statistics both in the original case and on appeal, was also interviewed on the programme and stated that the chance of a nurse working at the three hospitals being present at the scene of so many unexplained deaths and resuscitation is one in $342$ million. But sadly, this value was wrongly calculated (this is why any important statistical calculations should be peer-reviewed always at least by three other independent people) by making a simple calculations of the multiplication of the $p$-values... Furthermore, always in the context of Lucai de Berk the statisticians Richard D. Gill and Piet Groeneboom calculated a chance of one in twenty-five that a nurse could experience a sequence of events of the same type as Lucia de Berk.\\

	The use of probability arguments in the De Berk case was discussed in a 2007 Nature article by Mark Buchanan. He wrote \textit{The court needs to weigh up two different explanations: murder or coincidence. The argument that the deaths were unlikely to have occurred by chance (whether $1$ in $48$ or $1$ in $342$ million) is not that meaningful on its own - for instance, the probability that ten murders would occur in the same hospital might be even more unlikely. What matters is the relative likelihood of the two explanations! However, the court was given an estimate for only the first scenario.}\\
	
	So we see here that using simple multiplication of Fisher multiplication inference is not the point in such a situation!
	\end{tcolorbox}
	
	
	\paragraph{Simpson's Paradox (sophism)}\mbox{}\\\\
	\NewTerm{Simpson's paradox}\index{Simpson's paradox} (in fact it's a sophism and not a paradox because it is just a special case of omitted variable bias), is a paradox in probability and statistics, in which a trend appears in different groups of data but disappears or reverses when these groups are combined.
	
	This result is often encountered especially in social-science (HR; Marketing, Psychology), and medical-science statistics and is particularly confounding when frequency data are unduly given causal interpretations. Simpson's paradox disappears when causal relations are brought into consideration. 
	
	Let us see an famous real example for a real medical study:
	
	The table below shows the success rates and numbers of treatments for treatments involving both small and large kidney stones, where Treatment $A$ includes all open surgical procedures and Treatment $B$ is percutaneous nephrolithotomy (which involves only a small puncture). The numbers in parentheses indicate the number of success cases over the total size of the group (for example, $93\%$ equals $81$ divided by $87$):
	
	\begin{center}
		\definecolor{gris}{gray}{0.85}
			\begin{tabular}{|p{3cm}|p{3cm}|p{3cm}|}
				\hline
				 & \multicolumn{1}{c}{\cellcolor{black!30}\textbf{Treatment $A$}} & 
  \multicolumn{1}{c}{\cellcolor{black!30}\textbf{Treatment $B$}} \\ \hline
				\multicolumn{1}{c}{\cellcolor{black!30}\textbf{Small stones}}  & Group 1\newline $93\% (81/87)$  & Group 2\newline $87\% (234/270)$ \\ \hline
				\multicolumn{1}{c}{\cellcolor{black!30}\textbf{Large stones}}   & Group 3\newline $73\% (192/263)$ & Group 4\newline $69\% (55/88)$  \\ \hline
				\multicolumn{1}{c}{\cellcolor{black!30}\textbf{Both}} & \textbf{$78\% (273/350)$} & \textbf{$83\% (289/350)$} \\ \hline
		\end{tabular}
	\end{center}
	
	The paradoxical conclusion is that treatment$ A$ is more effective when used on small stones, and also when used on large stones, yet treatment $B$ is more effective when considering both sizes at the same time. In this example the "lurking" variable (or confounding variable) of the stone size was not previously known to be important until its effects were included.
	
	Formally, if we denote (we change the notations!!!!) $A$ the result, $B$ the treatment, and $C$ the stones, in the general case, with the Simpson paradox it is possible to have using Bayesian probabilities:
	

	Which treatment is considered better is determined by an inequality between two ratios (successes/total). The reversal of the inequality between the ratios, which creates Simpson's paradox, happens because two effects occur together:
	
	\begin{enumerate}
		\item The sizes of the groups, which are combined when the lurking variable is ignored, are very different. Doctors tend to give the severe cases (large stones) the better treatment $A$, and the milder cases (small stones) the inferior treatment $B$. Therefore, the totals are dominated by groups $3$ and $2$, and not by the two much smaller groups $1$ and $4$.
		
		\item The lurking variable has a large effect on the ratios, i.e. the success rate is more strongly influenced by the severity of the case than by the choice of treatment. Therefore, the group of patients with large stones using treatment $A$ (group $3$) does worse than the group with small stones, even if the latter used the inferior treatment $B$ (group $2$).
	\end{enumerate}

	Based on these effects, the paradoxical result is seen to arise by suppression of the causal effect of stone size on successful treatment. The paradoxical result can be rephrased more accurately as follows: When the less effective treatment (B) is applied more frequently to easier cases, it can appear to be a more effective treatment.
	
	Simpson's paradox usually fools us on tests of performance. In a famous example, researchers concluded that a newer treatment for kidney stones was more effective than traditional surgery, but it was later revealed that the newer treatment was more often being used on small kidney stones. More recently, on elementary school tests, minority students in Texas outperform their peers in Wisconsin, but Texas has so many minority students that Wisconsin beats it in state rankings. It would be a shame if Simpson's paradox led doctors to prescribe ineffective treatments or Texas schools to waste money copying Wisconsin.
	
	Consider another funny illustrated example in Marketing! Consider we have the following information:
	\begin{figure}[H]
		\centering
		\includegraphics{img/arithmetics/simpson_paradox_normal.jpg}
		\caption[]{Marketing Campaign performance}
	\end{figure}
	Obviously the campaign $B$ perform the best into coveting an e-mail to click!
	
	But now splitting in clicking high valuable links (costly products: high ticket conversion) or low valuable links (low ticket conversion) we get:
	\begin{figure}[H]
		\centering
		\includegraphics{img/arithmetics/simpson_paradox_ouch.jpg}
		\caption[]{Marketing Campaign performance in ticket conversion covariate}
	\end{figure}
	
	
	
	\subsubsection{Power of a test}
	When the effect is actually large, we can expect that we need less observations to demonstrate it that when the effect is small ... but how much exactly? Did we have the possibility to do this, in terms of number of measures, to "prove" what we seek? Should we go about it differently and change the device to do the  observation/experiment?
	
	To study in more detail the concept of "power of a test" that we have so far only mentioned, remember the following figure that we have already see just a little bit earlier:
	\begin{figure}[H]
		\centering
		\includegraphics{img/arithmetics/alpha_risk_beta_risk.jpg}
		\caption{Null and alternative hypothesis of a special case of two-sided test}
	\end{figure}
	In the particular example above, we will obviously reject the null hypothesis $H_0$ if  $\bar{X}>1.96$ or $\bar{X}<1.96$. Imagine that under the alternative hypothesis $H_a$, if we measured 2.5 for $\bar{X}$, we will have for power of the test:
	
	So the test is relatively strong/powerful (in practice, we consider a test to be powerful if its value is above of $80\%$ - that is to say the $\beta\leq 20\%$). Thus, we see that the (a posteriori!) power $1-\beta$ is even larger than the $p$-value is small (and respectively the power will be a posteriori even smaller than the $p$-value is great). Therefore the a posteriorii power is in decreasing correspondence with the $p$-value (in practice it is however a bit absurd to make these calculations a posteriori but some scientific paper require this information).
	
	\pagebreak
	\subsubsection{Power of the one sample Z-test}
	Very generally, in the case of a bilateral test, the above relation will be written:
	
	If the standard deviation of the mean (e.g. variance) is not unitary, we have:
	
	Therefore we get:
	
	written in another way:
	
	It is in this form that we find the power of a bilateral test of the average (power of a 1 sample Z-test):
	
	where $d_\sigma$ is sometimes named the "\NewTerm{size effect}\index{size effect}" and therefore defined by:
	
	and $\delta$ is named the "\NewTerm{difference}"!
	
	It goes without saying that if the true variance is not known, we must be replace the Normal law by the Student law as:
	
	with:
	
	\begin{tcolorbox}[title=Remark,colframe=black,arc=10pt]
	Caution wit a small recurring trap! The development above corresponds to a $\delta$ which is negative with respect to the first example! The relation is a little bit different in the case where $\delta$ is positive but it does not matter because the power of the test is the same in absolute value!
	\end{tcolorbox}	
	For the sample size it quite simple. We have by equating the quantiles:
	
	And therefore in bilateral:
	
	where we see that if the power of the test is imposed as being equal to $50\%$, with having therefore $Z_{1-\beta}$ that is $0$ then we fall back (!) on the relation of the sample size for the Normal distribution proved earlier above:
	
	Note also that we sometimes find in the literature the prior-previous relation as follows:
	
	Obviously we can set other parameters to determine the value of the remaining variable. We could also, for example, look for the value of the power of the test by fixing the standard deviation, the sample size and the risk threshold, etc.
	
	One reader offered us a very elegant way of finding the same result with much less developments... Indeed, you need only to see in the previous figure that we have:
	
	equality from which we draw immediately an equivalent relations to the two previous one (which obviously gives the same numerical result):
	
	\begin{tcolorbox}[title=Remark,colframe=black,arc=10pt]
	The attentive reader will have perhaps noticed that we assumed in the previous developments that the standard deviation of the mean of the null hypothesis and that of the alternative one is implicitly assumed to be the same... In practice this is almost all the time so, this is why almost all statistical software require only one standard deviation to calculate the power of the 1-Sample Z Test. However, for some rare academic softwares, they ask the standard deviation of the two means. But then the above mathematical developments are obviously different.
	\end{tcolorbox}
	
	A test power analysis may have several facets:
	\begin{enumerate}
		\item We know the level of the test, the sample size and effect size (implicitly the difference) and we seek to calculate its power. This allows us to see if our experimental device is properly calibrated.
		
		\item We know the desired test power, the risk threshold level and effect size to detect. We seek then to calculate the sample size needed to mount an effective experimental design.
		
		\item We know the desired test power, the risk threshold level and sample size and we seek to know the effect size we can hope to highlight.
	\end{enumerate}
	Without exception, we consider as unnecessary to show a test if the expected power is less than $80\%$. This power corresponds to a $80\%$ probability of not rejecting the null hypothesis wrongly or, what remains to the same: a $20\%$ of Type II error.
	
	Obviously, it is possible to make the same reasoning (analytically when possible, otherwise numerically) with absolutely ALL the hypothesis tests that we have seen until now. So as there is a little more than a hundred hypothesis testing in the field of statistics as we have already mentioned... it is obvious that we're not going to have.... fun ... to make the same developments to determine the sample size, effect size and power for all these tests but only for the classics one. As long as we have computers at our disposal with the algorithms integrated by computer scientists, we do not need to redo any developments that would not bring much. Moreover, the majority of statistical softwares can calculate usually the power of only 5 to 10 common tests.
	\begin{tcolorbox}[title=Remark,colframe=black,arc=10pt]
	We will not dealing with parametric statistical tests for the detection of outliers in this book as the Q-Dixon's test or Grubb's test and this just because they have too empirical origin and that they have no interest analytically speaking. By cons, if some readers insist, and that we have the time, we can put the details on these tests with detailed algorithms for calculating critical values using a simple spreadsheet and Monte Carlo technique with any distribution of their choice (but not only for data following a Normal law contrary to what is written in most books).
	\end{tcolorbox}
	
	\subsubsection{Power of the one and two samples P-test}
	Same as with the confidence interval of the Normal distribution with known theoretical standard deviation (that is to say the entire population), we can determine the number of individuals (sample size) for a fixed test power for the 1 sample proportion test studied in detail earlier. For this, we use the same technique as for the power of the one sample Z-test. We write so at first:
	
	Hence we deduce:
	
	So if the power is 50% we fall back on:
	
	For the power of the test of the difference of two proportions (two sample proportion test), with the objective of determining the sample size, we have to put $n=n_1=n_2$. Therefore the developments we get during our study of the test of the difference of two proportions can be written:
	
	with therefore:
	
	In the same way that we did for the $Z$-test and the one sample $p$-test, we have:
	
	Therefore:
	
	So what it takes us to assume that the real difference between the two proportions is the average (which is debatable...).
	
	But we also have (as the samples are independent and using the property of the variance):
	
	Therefore:
	
	That gives obviously:
	
	We have then after rearrangement:
	
	
	\pagebreak
	\subsubsection{Fieller's test (ratio of two means)}
	The purpose of the "\NewTerm{Fieller's test}" is the calculation of a confidence interval for the ratio of two means. Its is an especially trendy test in the field of webdesign when two design or userinterface of a same website (or the number of clicks) have to be compared. The Fieller's test is then most commonly know under the name "\NewTerm{A/B test}\index{A/B test}".
	
	Let us consider $X$ and $Y$ two random variables Normally distributed representing averages of samples of size $n$ and $m$ respectively and having same variance, such that:
	
 	So this is a homoscedastic case and then in practice we take the global standard deviation.

	In reality, however, we will have access only to the estimators:
	
 	What interests us is to build a confidence interval for the ratio:
	
 	And this is the purpose of Fieller's test obviously!

The idea is then to compute and to group the two random variables into one:
	
 	which by the null hypothesis will therefore follow a random variable of the type:
	
 	We then have in extenso:
	
 	If the standard deviation can only be estimated, we know that we then have:
	
 	It follows, of course, that:
	
	By symmetry of the Student's law we can therefore write that:
	
	After rearrangement, we get:
	
	There are now two interesting situations to consider depending on the sign of the coefficient $\theta^2$ of but let's put it aside for now concentrating only on the roots of the polynomial in the parenthesis. We then have (\SeeChapter{see section Calculus}):
	
	This is simplified immediately into:
	
 	The development in the root gives:
	 
	What can be rewritten in a form that will be useful to us:
	 
	Let us put:
	
 	Our roots can then be written:
	
 	or:
	
	or otherwise:
	
	We do a last simplification:
	
	Who are therefore for recall the roots of:
	
 	The coefficient of $\theta^2$ in the polynomial is $Y^2-bS^2T_{\alpha/2}^2(k)$. Therefore:
	\begin{itemize}
		\item If the latter is greater than zero, then the parabola is concave (seen from above) and the confidence interval is then between the two roots of the parabola (because the values there are negative according to the imposed inequality).

		\item If the coefficient of $\theta^2$ in the polynomial is negative then the parabola is convex (seen from above) and the confidence interval is then outside the two roots of the parabola because the values there are negative according to the imposed inequality).

		\item If the coefficient of $\theta^2$ has a near zero probability of being equal to zero we can ignore this scenario.
	\end{itemize}
	Obviously only the first case interests us because the second is difficult to interpret (the intervals would have the infinite bounds which is unrealistic and especially useless). So we must have:
	
	What can be written:
	
	Therefore:
	
	\begin{tcolorbox}[colframe=black,colback=white,sharp corners]
	\textbf{{\Large \ding{45}}Example:}\\
	Let us consider the case where we have:
	
	Therefore:
	
	We have:
	
	and:
	
	\end{tcolorbox}
	\begin{tcolorbox}[colframe=black,colback=white,sharp corners]
	As well as (we give this calculation because some statistical software - R in particular - given them in the output while normally it is only useful for Student's t test of two independent homoscedastic samples):
	
	We then have the following interval between the two roots of the parabola:
	
	\end{tcolorbox}
	We can go a little further by considering the case where:
	
	Either within the estimated framework:
	
 	This gives us the following distribution (assuming Normality) for the random variable grouping the first two according to $X-\theta Y$:
	
	We then  have:
	
 	And therefore, as before, we can write:
	
	What we can rewrite in the following form:
	
 	What developed gives:
	
	We group together the terms as earlier above:
	
 	We then have for roots after simplification numerical factors:
	
 	After the simplification method seen above (pfff...) we finally get:
	 
	
	\pagebreak
	\subsubsection{Analysis Of VAriance (ANOVA)}
	"\NewTerm{Analysis Of VAriance}\index{Analysis Of VAriance}" is a collection of statistical models used to analyze the differences among group means and their associated procedures (such as "variation" among and between groups), developed by statistician and evolutionary biologist Ronald Fisher. In the ANOVA setting, the observed variance in a particular variable is partitioned into components attributable to different sources of variation.
The main techniques of analysis of variances are the following one (some of them are detailed
below) in the order of most to less used one (and in increasing order of complexity):
	\begin{itemize}
		\item Two variable one way fixed factor ANOVA (Student T-test)
		\item One way fixed factor ANOVA (n independent fixed variables)
		\item Two way fixed factor ANOVA without repetitions
		\item Two way fixed factor ANOVA with repetitions
		\item Multifactor ANOVA with repeated measures
		\item Block (randomized)design (ANOVA in blocks)
		\item Latin square ANOVA
		\item Greaco-Latin square ANOVA
		\item Split-Plot ANOVA
		\item Strip-Plot ANOVA
		\item Split-Split-Plot ANOVA
		\item Analysis of Covariance (ANCOVA)
		\item Nested (Hierarchical) analysis of variance (HANOVA)
		\item Multivariate analysis of variance (MANOVA)
		\item Random effects ANOVA
		\item Mixed effects ANOVA
	\end{itemize}
	and some of the ANOVA above are said "balanced" or "unbalanced" if the number of measurement are equal or not equal for each variable.

	It must also be known that the terminology of ANOVA is largely from the statistical design of experiments (see section Industrial Engineering). The experimenter adjusts factors and measures responses in an attempt to determine an effect. Factors are assigned to experimental units by a combination of randomization and blocking to ensure the validity of the results. Blinding keeps the weighing impartial. Responses show a variability that is partially the result of the effect and is partially random error.	
	
	\pagebreak
	\paragraph{Analysis of Variance with one fixed factor}\mbox{}\\\\
	The objective of the analysis of variance (at the contrary to what its name might suggest...) is a statistical technique for comparing the means of two or more populations and that is very widely used in the pharmaceutical field or in the R\&D labs or benches test. This method, however, takes its name from that it use measures of variance to determine the statistical significance, or not, of the differences of measured averages on populations or samples.
	
	More precisely, the real meaning is to know if the fact that the sample averages are (slightly) different can be assigned at random sampling or due to the fact that a variability factor actually generates significantly different samples (if we have the population size, we have obviously not such question!). 
	\begin{tcolorbox}[title=Remark,colframe=black,arc=10pt]
	For more information about the vocabulary and application, the engineer and the researcher should refer to standard ISO 3534-3:1999.
	\end{tcolorbox}
	For the analysis of variance named also "\NewTerm{one factor ANOVA}\index{one factor ANOVA}" (Analysis Of VAriance\index{Analysis of Variance}) or "\NewTerm{one factor ANAVAR}\index{one factor ANAVAR}" (ANAlysis of VARiance), or "\NewTerm{one-way ANOVA}\index{one-way ANOVA}" or more rigorously "\NewTerm{one fixed factor ANOVA with repetitions}\index{one fixed factor ANOVA with repetitions}" or "\NewTerm{ANOVA with one fixed categorical variable with repetition}\index{ANOVA!One fixed categorical variable with repetition}", we first recall, as we proved, that the Fisher-Snedecor law is given by the ratio of two independent random variables that follow a Chi-square law and divided by the degree of freedom such as:
	
	and we will now see its importance.
	\begin{tcolorbox}[title=Remark,colframe=black,arc=10pt]
	When a factor can have a very large number of levels we consider having chosen some given level of the factor among a multitude of possible as a random selection. This is why we speak then in this latter case of "random factor" which is the subject specific ANOVA that we will study once those on fixed factors master (e.g. ANOVA mixing fixed and random factors factors are named "\NewTerm{mixed ANOVA}\index{mixed ANOVA}") .
	\end{tcolorbox}
	Let us consider a random sample of size $n$, say $X_1,X_2, ...,X_n$ sampled from the law $\mathcal{N}(\mu_X,\sigma_X)$ and a random sample of size $m$, say $Y_1,Y_2, ...,Y_m$ sampled from the law $\mathcal{N}(\mu_Y,\sigma_Y)$ .
	
	Consider now the maximum likelihood estimators of the standard deviation of the Normal law traditionally noted in the field of analysis of variance:
	
	The above statistics are those that we could use to estimate the variances if the average theoretical $\mu_X,\mu_Y$ were known. So we can use a result proved above in our study of confidence intervals:
	
	As the $X_i$ are independent of the $Y_j$ (hypothesis that suppose the covariance to be zero, the converse being for reminder not always!), the random variables:
	
	are independent from each other.
	\begin{tcolorbox}[title=Remark,colframe=black,arc=10pt]
	There is an existing type of ANOVA for the case where the variables are not independent (we then speak about "\NewTerm{covariate}\index{covariate}"). This is the "\NewTerm{ANCOVA}\index{ANCOVA}" which means "\NewTerm{ANalysis of COvariance and VAriance}\index{Analysis of Covariance an Variance}" that uses a mix of linear regression (\SeeChapter{see section Theoretical Computing}) and ANOVA. The purpose of the ANCOVA is to statistically eliminate the indirect effect of the covariate. We will see much later how it works in details.
	\end{tcolorbox}
	We can therefore apply the Fisher-Snedecor law with:
	
	Therefore we get:
	
	Finally:
	
	This theorem allows us to deduce the confidence interval of the ratio of two variances when the theoretical mean is known. Since Fisher function is not symmetrical, the only possibility to make the inference is to use numerical calculation and then we will note for a given confidence interval the test as follows:
	
	In the case where the averages $\mu_X,\mu_Y$ are unknow, we use the unbiased estimators of the variances noted traditionally in the field of analysis of variance:
	
	To estimate the theoretical variances, we use the result proved above:
	
	As the $X_i$ are independent of the $Y_j$ (assumption!), the variables:
	
	are independent from each other. We can therefore apply the Fisher-Snedecor distribution:
	
	Therefore we get:
	
	Finally:
	
	This theorem allows us to deduce the confidence interval of the ratio of two variances when the sample mean is known. Since Fisher function is not symmetrical, the only possibility to make the inference is to use numerical calculation and then we will note for a given confidence interval the "\NewTerm{Fisher $F$-test}\index{Fisher $F$-test}" as follows:
	
	keeping in mind that its use implicitly requires constraints of Normality of the studied variables.
	
	R.A. Fisher (1890-1962) is, as Karl Pearson, one of the main founders of the modern theory of statistics. Fisher studied at Cambridge, where he obtained in 1912 a degree in astronomy. It is by studying the theory of error in astronomical observations that Fisher became interested in statistics. Fisher is the inventor of the branch of statistics named "analysis of variance".
	
	In the early 20th century, R.A. Fischer therefore develops the very important methodology of experimental design (\SeeChapter{see section Industrial Engineering}). To confirm the usefulness of a factor, he developed a test to ensure that the different samples are of different natures. This test is based on the analysis of variance (of the samples), and named also ANOVA but for "\NewTerm{normalized analysis of variance}\index{normalized analysis of variance}".
	
	Let us take $k$ samples of $n$ random values each. Each of the values being considered as an observation or measurement of something or on the basis of something (a different place or a different object... in short: a single factor of variability between samples!). We will have a total of $N$ observations (measurements) given by:
	
	If each sample has an equal number of values $n$ (sample size) such that $n_1=n_2=...=n_k$ we then speak of "\NewTerm{balanced plan}\index{balanced plan}" or "\NewTerm{balanced design}\index{balanced design}" with $k$ levels (or $k$ modalities).
	\begin{tcolorbox}[title=Remark,colframe=black,arc=10pt]
	If we have more factors of variability (e.g. each place compares himself to different laboratories), then we will talk about "\NewTerm{multifactorial ANOVA}\index{multifactorial ANOVA}". Therefore, if there are only two sources of variability, we talking about "\NewTerm{two factor ANOVA}\index{two factor ANOVA}" (see below for details on the various of two-factor ANOVA).
	\end{tcolorbox}
	We will consider that each of the $k$ samples is sampled (follows) a random variable following Normally distributed:
	
	In terms of testing, we test whether the mean of the $k$ samples of size $n$ are equal under the assumption that their variances are equal. What we write as under hypothesis notation as follows:
	
	In other words, the samples are representative of the same population (i.e. of a same statistical law). That is to say, the variations between the values of different samples are essentially due to the hazard. For this we study the variability of results in the samples and between samples. It is exactly the same as asking (formulation found in some publications or books):
	
	So we will for the rest denote by $i$ the index number of the sample ($1$ to $k$) and $j$ the index of the observation (from $1$ to $n$). Therefore $x_{ij}$ will be the value of the $j$-th observation of the data sample number $i$ (we chose to reverse the usual notation so be careful not to mislead afterwards ... we're sorry ... it was bad choice me made when we started to write this book!).
	
	According to the above hypothesis, we have:	
	
	We will denote by $\bar{x}_i$ the empirical/estimated (arithmetic) average of the sample $i$ (often named "\NewTerm{marginal average}\index{marginal average}"):
	
	and $\bar{\bar{x}}$ the empirical/estimated (arithmetic) average of the $N$ value (therefore the average of the $\bar{x}_i$) and then given by:
	
	Using the properties of the mean and variance already proved above we know that:
	
	with $\mu$  which is the average of the true averages (expected means) $\mu_i$:
	
	Let us now introduce three important  variances:
	\begin{enumerate}
		\item The "\NewTerm{Total variance}\index{total variance}" as being intuitively the estimated unbiased variance unbiased considering the set of N observations as a single sample:
			
			where the numerator is named "\NewTerm{sum of the squares of total differences}\index{sum of the squares of total differences}".
			
			\item The "\NewTerm{variance between samples}\index{variance between sample}" (that is to say between the averages of the samples) is also intuitively the variance estimator of the averages of samples:
			
			where the numerator term is named  "\NewTerm{sum of the squared differences between samples}\index{sum of the squared differences between samples}".
			
			As we have proved that if all variables are identically distributed (same variance and same mean) and independent the variance of individuals is $n$ times that of the average:
			
			then the "\NewTerm{variance of observations}\index{variance of observations}" (random variables in a sample) is given by:
			
			We then have therefore above the hypothesis of equality of variances that is expressed in mathematical form for the developments that will follow.
			
			\item The "\NewTerm{residual variance}\index{residual variance}" is the effect of the "uncontrolled factors". It is by definition the average of the sample variances (its like: standard error):
			
			where the numerator term is named "\NewTerm{sum of squared residuals errors}\index{sum of squared residual errors}" or even more often "\NewTerm{residual error}\index{residual errors}".
	\end{enumerate}
	Finally, these indicators are sometimes summarized as follows:
	
	Note that if the samples do not have the same size (which is rare in practice), then we have:
	
	\begin{tcolorbox}[title=Remarks,colframe=black,arc=10pt]
	\textbf{R1.} The term $Q_T$ is often indicated in the industry by the acronym SST meaning in "\NewTerm{Sum of Squares Total}\index{total sum of squares}" or more rarely TSS for "\NewTerm{Total Sum of Squares}".\\
	
	\textbf{R2.} The term $Q_A$ is often indicated in the industry by the acronym SSB meaning in "\NewTerm{Sum of Squares Between (samples)}\index{sum of squares between samples}" or more rarely SSk for "\NewTerm{Sum of Squares Between treatments}\index{sum of squares between treatments}".\\
	
	\textbf{R3.} The term $Q_R$ is often indicated in the industry by the acronym SSW meaning in  "\NewTerm{Sum of Squares Within (samples)}\index{sum of squares within samples}" or more rarely SSE for "\NewTerm{Sum of Squares due to Errors}\index{sum of squares due to errors}".
	\end{tcolorbox}	
	Let us indicate that we often see in the literature (we will use a little further below this notation):
	
	so with the unbiased estimator of the variance of the observations:
	
	Before going further, let us stop on the residual variance. We then have for samples that are not of the same size:
	
	This writing is often named "\NewTerm{pooled variance}\index{pooled variance}\index{combined variance}\index{composite variance}\index{overall variance}". The square root of a pooled variance estimator is knows obviously as a "\NewTerm{pooled standard deviation}\index{pooled standard deviation}\index{combined standard deviation}\index{composite standard deviation}\index{overall standard deviation}". We also see that if we assume all the mean $\bar{x}_i$ to be equal and also all samples sizes $n_i$ we fall back on the standard error.
	
	Let us now open a small important parenthesis... Let's take the case of two samples only:
	
	Therefore by introducing the maximum likelihood estimator of the variance:
	
	We can also observe that in the specific case where $n_1=n_2=n$:
	
	So:
	
	Now suppose we want to compare with a confidence interval the mean of two populations with different variance to know whether they are different or not.

	We currently know  two tests to check the averages. The Z-test and T-test. As in the industry (practice) it is rare that we have time to take large samples, let's focus on the second we had proved above:
	
	And also recall that:
	
	Now let us recall that we have proved that if we have two random variables of distribution:
	
	then the subtraction of the averages gives (property of stability of the average):
	
	So for the difference average of two random variables coming from two population samples we obtain directly:
	
	And now the idea is to take the approximation (under the assumption that the variances are equal):
	
	This approximation is named "\NewTerm{homoscedastic hypothesis}\index{homoscedastic hypothesis}".
	
	We then have the following confidence interval (assuming that we know only an estimation of the variance) remembering that the subtraction or the sum of two independent random variables implies that their variances always add (so it is the same for degrees of freedom of the Student law as we have proved above due to the direct connection with the law of the Chi-2):
	
	with:
	
	As the idea in practice is often to test the equality of expected means (and therefore that their difference is zero) from the known estimators then:
	
	In most software on the market, the result is given only from the fact that quantile $T$ that we get is included in the $T_{\alpha/2}$ corresponding to the given confidence interval given by (for reminder):
	
	in the case of the homoscedastic hypothesis (equality of variances homogeneity of variances).
	\begin{tcolorbox}[title=Remarks,colframe=black,arc=10pt]
	The latter relation is named "\NewTerm{independent two-sample $t$-test}\index{independent two-sample $t$-test}", or "\NewTerm{homoscedastic $t$-test}\index{homoscedastic $t$-test}" or "\NewTerm{$t$-test of equality of $2$ observations expectations with equal variances}" or more simply but somewhat abusively "\NewTerm{$2$-sample $t$-Test}" with samples of different sizes and equal variances. Often in the literature, both theoretical means are equal when comparing. It follows that we then have:
	
	\end{tcolorbox}
	Otherwise, in the more general case of the assumption of heteroscedasticity (not equality of variances), we explicitly write (we'll come back on this later during our study of the Welch test...):
	
	Therefore:
	
	\begin{tcolorbox}[title=Remarks,colframe=black,arc=10pt]
	The ante-previous relation is named "\NewTerm{independent two-sample $t$-test}", or "\NewTerm{heteroskedastic $t$-test}\index{heteroskedastic $t$-test}" or even "\NewTerm{test of equality of expected means: two observations with different variances}". If the sizes of the samples are equal and that the variances are also equal and that we assume the two expected mean equal during the comparison, it follows that we then have:
	
	\end{tcolorbox}
	This finish, we close this parenthesis and return to our main topic... We were therefore at the following table if you remember:
	
	Where we have in the case of samples of the same size:
	
	and also the total error that is the sum of the errors of the averages (between classes) and of the residual error (intra-class) and this that the samples are of the same size or not:
	
	Indeed:
	
	But we have:
	
	Because:
	
	Therefore
	
	Now, under the strong assumption (which we will be absolutly required a little further below) that the true variances are linked such as:
	
	and therefore that respective estimators are asymptotically equal ... which in practice is approximately true only when certain conditions are met (which is why it is absolutely necessary before making an ANOVA to run an a priori calculation of the power and the sample size of the ANOVA test) we have:		
	
	that follows immediately from the proof that we made during our study of statistical inference with the Chi-square law where we got (for refresh!):
	
	To determine the number of degrees of freedom of the Chi-square law of:
	
	We will use the fact that (by the same reasoning as for the ante-previous relation):
	
	and since that $Q_T=Q_A+Q_R$, then we must have:
	
	It follows by the linearity property of the Chi-Square law:
	
	So to summarize we have:
	
	Now comes the Fisher law in the assumption that the variances are equal (and measurement Normally distributed)! Because:	
	
	What we want to do is to see if there is a difference between the variance of the averages (between classes) and the residual variance (intra-class). To compare two variances when the true averages are unknown we saw that the best was to use the Fisher test. But, we proved in our study of the Fisher law a little above that:
	
	where in our case study:
	
	Since there are dozens of different types of ANOVA we have to understand very good this choice of the simplest ANOVA we are studying right now! Thus, if the averages are the same, the null hypothesis is then that variance ratio is equal to unity (under the above assumptions already mentioned far above). If $F$ is too large at a given threshold, then we reject the null hypothesis of equality of means (because verbatim variances will be strongly different). So it seems here logic to compare the variances between groups (numerator) with that in the groups (denominator) but as we shall see this is not always the choice to be made (especially in hierarchical ANOVA)!
	
	Given the assumption of the first equality of the above relation (the one that precedes the implication), we understand at the same time much better the great sensitivity of the ANOVA results to the non equality of true variances!
	
	Let us also indicate that the previous relation:
	
	is often given in the literature as follows:
	
	where MSk is named "\NewTerm{Mean Square for treatments}\index{mean square for treatments}" and MSE "\NewTerm{Mean Square for Error}\index{mean square for error}". This ratio will therefore give us the value of the random variable $F$ (whose support is for reminder bounded at zero). Regarding the choice of the test (right/left one-sided or bilateral), note that if the averages are really equal, then for all $i$:
	
	So in this case:
	
	Which brings us of course to immediately adopt a right-sided test!
	
	Otherwise, in general, the interpretation of this fraction is basically as following: This is the ratio (normalized to the number of degrees of freedom) of the sum of the error of the averages (between classes) and that of the residual error (intra-class) or in other words the ratio of the interclass variance by the residual variance. The ratio thus follows a Fisher distribution with two parameters given by the degrees of freedom of the respective classes.
	
	\begin{tcolorbox}[title=Remark,colframe=black,arc=10pt]
	If there are only two populations (samples), we must understand that then that use of the Student T-test is more than enough and is perfectly equivalent! In fact, the ANOVA is an indirect comparison of means, the Student T-test a direct comparison... so it is obvious to guess which one is better in this particular situation!
	\end{tcolorbox}	
	
	All calculations we have made until now are very often represented in softwares in a standard table form as represented below (this is for example how Microsoft Excel 11.8346 or Minitab 15.1.1 give the results):
	
	The null hypothesis will not be rejected if the value of:
	
	is smaller or equal to the quantile of the distribution $F$ that corresponds to the cumulative probability to $1$ subtracted from confidence level $\alpha$ and denoted by $F_c$.
	
	\begin{tcolorbox}[title=Remarks,colframe=black,arc=10pt]
	\textbf{R1.} Notice that in practice, inter-class variance is often named "\NewTerm{inter-laboratory variance}\index{inter-laboratory variance}" and the intra-class variance is verbatim often named "\NewTerm{intra-laboratory variance}\index{intra-laboratory}".\\

	\textbf{R2.} There are, in this early 21st century more than $50$ existing tests or procedures comparison for the variance. The opinion varies among practitioners relatively to their relevance and the effectiveness of homogeneity of variance tests (\NewTerm{HVT}). Some argue that they are absolutely necessary to be applied before doing any ANOVA, others say that these tests are anyway of poor performance, the ANOVA being anyway more robust to homoscedasticity than the differences than can be detected by HVT, particularly in case of non-Normality. In fact, all these issues are related to the "\NewTerm{Behrens-Fisher problem}\index{Behrens-Fisher problem}", which is that the comparison of means without assuming the equivariance. However among the fifty existing tests, several comparative studies have highlighted the following efficient tests that we will present further below: Bartlett's test, Levene and Brown-Forsythe.\\

	\textbf{R3.} When certain levels of a factor are combined into one to be compared to a reference level statisticians then talk about creating "\NewTerm{contrasts}\index{contrast}" (see just below).
	\end{tcolorbox}
	
	\subparagraph{Contrasts}\mbox{}\\\\
	Many multiple comparison methods use the idea of what we name a "\NewTerm{contrast}". What is this? To introduce this concept let us consider a one-way fixed factor ANOVA where the null hypothesis $H_0$ was rejected but where actually we don't know what level of the variable cause the difference that brig us to reject the the null hypothesis.
	
	Consider for example that we suspect the level among the levels $i$, $j$, $k$, ... we suspect the levels $i$ and $k$ to differ. Therefore we would like to test the hypothesis:
	
	or equivalently:
	
	If we had suspected at the start of the experiment that the average of the two first level (considering an imaginary experiment with a factor having $5$ fixed levels) did not differ from the average of the two highest levels, then the hypothesis would have been:
	
	or:
	
	In general a "\NewTerm{contrast}\index{contrast}" is a linear combination of parameters of the form:
	
	Both of the hypothesis above can the be expressed in terms of contrasts:
	
	Therefore in the case of our previous example:
	
	we have $c_1=+1$, $c_2=+1$, $c_3=-1$, $c_4=-1$, $c_5=0$.
	
	Testing hypotheses involving contrasts can be done in three ways. We will introduce here only the one that gives the practitioner the possibility to build a confidence interval. This method uses a Student $T$-test. Indeed, let us write the contrast in terms of estimates:
	
	as under the null hypothesis we should have (don't) forget it!):
	
	The variance of the sum can be written:
	
	If the equality of variance assumption is satisfied and the design is equilibrated we have:
	
	But with estimated we know that this is written:
	
	and therefore:
	
	Hence:
	
	should follow a centered Normal law under the above assumptions:
	
	But as we work with estimators we know that we have instead:
	
	That is sometimes denoted in the field of analysis of variance:
	
	Therefore the $100\cdot (1-\alpha)$ percent confidence interval on the contrast $\sum_{i=1}^k c_i\mu_i$ is:
	
	Clearly if this confidence interval includes zero, we would be unable to reject the null hypothesis.
	
	\pagebreak
	\paragraph{Analysis of Variance with two fixed factors without repetitions}\mbox{}\\\\
	We will now see the concept of interaction that is essential to understand what is behind the fixed two-factor ANOVA (or "\NewTerm{ANOVA with two fixed categorical variables}\index{ANOVA!Two fixed categorical variables}") without and especially later with repetitions. Indeed, it is only with two-factor ANOVA with repetitions - by mathematical construction - that can be statistically (under given assumptions) consider objectively if two or more factors interact significantly together.
	
	We must therefore, before moving to the pure mathematical part, introduce some concepts!
	
	\textbf{Definitions (\#\mydef):}
	\begin{enumerate}
		\item[D1.] We say that there is "\NewTerm{no interaction}\index{ANOVA! Noninteracting factors}" when the average of responses by a factor based on its level varies by the same amount and with the same sign as the average of the responses of another factor depending on its levels. Then we say that: the response interaction curves in the diagram are parallel.
		
		\begin{tcolorbox}[title=Remark,colframe=black,arc=10pt]
		The parallelism of response curves is normal in the situation  of no interaction because it means that whatever the level of one or the other factors, the variation (if there is any) of the response will always be the same amplitude. What is characteristic of independence (at least locally).
		\end{tcolorbox}	
		
		\item[D2.] We say that two factors are "\NewTerm{in interaction}\index{ANOVA!Interacting factors}" when the average responses by a factor depending on its levels do not change by the same magnitude and / or not with the same sign as the average of the responses of another factor according to their levels. Then we say that: the response curves in the interactions diagram are not parallel.
		
		\begin{tcolorbox}[title=Remark,colframe=black,arc=10pt]
		The absence of interaction is a very strong assumption and a rarely observed. Often we have interactions or strong interactions.
		\end{tcolorbox}
	\end{enumerate}
	To understand the concept, we will use small examples have no repeated measures to such we will have a simple qualitative idea of the phenomenon, but by no means a scientific approach of the concept of interaction (that will be study later in details).
	
	For every small example below we visualize the situations by means of two types of representations: a graph illustrating the main effects on the one hand and a pattern of the interactions on the other hand.
	
	\pagebreak
	Consider the following small table with two factors at two levels ("explanatory variables") including four cells ("variable of interest"):
	
	We will get as result with software such as Minitab:
	\begin{figure}[H]
		\centering
		\includegraphics{img/arithmetics/anova_principal_effects_no_interactions.jpg}
		\caption{Main effects plot without interactions in Minitab 15}
	\end{figure}
	We see well that no factor has a major effect on anything. What is relatively obvious given the content of the previous table...
	
	The diagram of interactions (often named "\NewTerm{profiler}\index{profiler diagram}" in the industry) gives:
	\begin{figure}[H]
		\centering
		\includegraphics{img/arithmetics/anova_interactions_effects_no_interactions.jpg}
		\caption{Interactions plot without interactions... in Minitab 15}
	\end{figure}
	where we can see that the factors do not interact between them (or neutralize themselves it depends...). Then we say that there is "\NewTerm{(a priori) no effect or no interaction (locally)}\index{ANOVA!Non-locally interaction}". In fact in some experiments the absence of interaction is a very strong assumption and therefore very rare. This is why we must pay attention to the chosen words when interpreting interaction plots (because do not to go through pure calculations is dangerous for this step or simply not scientific!).
	
	Now consider the following table:
	
	It seems obvious that the factor one through considered with its levels seems to have an influence on the answer. But let us see the different representations:
	\begin{figure}[H]
		\centering
		\includegraphics{img/arithmetics/anova_principal_effects_with_interactions.jpg}
	\end{figure}
	\begin{figure}[H]
		\centering
		\includegraphics{img/arithmetics/anova_interactions_effects_with_interactions.jpg}
		\caption{Main effects and interactions plot with interactions in Minitab 15}
	\end{figure}
	Let us comment the first graph as proposed by a reader:
	
	This includes two parts: the left one analyzes the effects of Factor $1$ through its two levels; that of the right do the same for the Factor $2$.
	
	Have a closer look to the left part:
	
	We see two points connected by a line segment. Here the first point, the one for the level $1$1, is located in the ordinate $2$ while the second point, that for level $2$, is located in the ordinate $4$. Now remember that each point represents an average. And the ordinate of the first point is located at the average of $(2 + 2) / 2 = 2$.
	
	Having said that and hoping this helped to a better understanding, let us continue...
	
	It is quite clear in the chart above that only the level of Factor $1$ influences the answer, while Factor $2$ has no effect on the answer. Then we say that: there is a main effect (locally) of the Factor $1$.

	
	On the diagram of interactions, we have the same information but in a different form. We see that regardless of the level of Factor $2$, the answers are horizontal and thus that it does not influence the results. We are then in a situation where "\NewTerm{(a priori) the main effect is (locally) Factor $1$ and there is an absence of interactions between factors}".
	Now consider the following table:
	
	This time we can observe that the Factor $2$ has an influence but not the Factor $1$. But let us also see this with our two types of representations:
	\begin{figure}[H]
		\centering
		\includegraphics{img/arithmetics/anova_principal_effects_with_interactions_second_example.jpg}
	\end{figure}
	\begin{figure}[H]
		\centering
		\includegraphics{img/arithmetics/anova_interactions_effects_with_interactions_second_example.jpg}
		\caption{Main effects and interactions plot with interactions in Minitab 15}
	\end{figure}
	We see well on the top diagram that the Factor $1$ has no influence. On the bottom diagram it is less obvious but the superposition of the two lines indicates that the Factor $1$ has no influence. Then we say that there is: "\NewTerm{(a priori) main effect (locally) of Factor $2$ and absence of interactions between factors}".
	
	Now consider the following table:
	
	We see that both factors have an influence on the answer. We can clearly see it on the both diagrams below:
	\begin{figure}[H]
		\centering
		\includegraphics{img/arithmetics/anova_principal_effects_with_interactions_third_example.jpg}
	\end{figure}
	\begin{figure}[H]
		\centering
		\includegraphics{img/arithmetics/anova_interactions_effects_with_interactions_third_example.jpg}
		\caption{Main effects and interactions plot with interactions in Minitab 15}
	\end{figure}
	We can see very well on the top diagram that the Factor $1$ has an influence on the answer and that we have the same conclusion for the Factor $2$ (and furthermore to in the same magnitude regardless of the direction!). On the bottom diagram it is less obvious, but the same conclusion is valid. We then say that: "\NewTerm{(a priori) both factors are (locally) significant and without interactions}\index{ANOVA!Both factors locally significant}".
	
	\pagebreak
	Now consider the following table:
	
	which in this form is not always obvious to interpret. But with the diagrams we immediately have a more relevant information:
	\begin{figure}[H]
		\centering
		\includegraphics{img/arithmetics/anova_principal_effects_with_interactions_fourth_example.jpg}
	\end{figure}
	\begin{figure}[H]
		\centering
		\includegraphics{img/arithmetics/anova_interactions_effects_with_interactions_fourth_example.jpg}
		\caption{Main effects and interactions plot with interactions in Minitab 15}
	\end{figure}
	We see well on the top diagram that none of the factors has an effect in average on the answer a priori (same diagram as at the very beginning of this series of examples with the same mean!). The bottom diagram gives us additional information by cons (!!!): The factors have a cross-influence and as this cross-influence is of the same magnitude, the effects cancel out. Then we say that: "\NewTerm{(a priori) the two factors are (locally) in interaction in F$1$ * F$2$}\index{ANOVA!Both factors locally in interaction}".
	
	Now consider the following table:
	
	which in this form is not also always obvious to interpret. But with the diagrams we immediately have a more relevant information:
	\begin{figure}[H]
		\centering
		\includegraphics{img/arithmetics/anova_principal_effects_with_interactions_fifth_example.jpg}
	\end{figure}
	\begin{figure}[H]
		\centering
		\includegraphics{img/arithmetics/anova_interactions_effects_with_interactions_fifth_example.jpg}
		\caption{Main effects and interactions plot with interactions in Minitab 15}
	\end{figure}
	We observe well on the top diagram that the Factor $1$ appears to have an influence and the factor $2$ no (in average!). The bottom interactions diagram gives us, too, once again, additional information (!!!): It is that the factors are interacting. We say then that we have "\NewTerm{(a priori) two factors (locally) in interaction F$1$ * F$2$ where the influence of Factor $1$ is significant}".
	
	Now consider the following table:
	
	We see that both factors influence the answer. We see that well on the both diagrams below:
	\begin{figure}[H]
		\centering
		\includegraphics{img/arithmetics/anova_principal_effects_with_interactions_sixth_example.jpg}
	\end{figure}
	\begin{figure}[H]
		\centering
		\includegraphics{img/arithmetics/anova_interactions_effects_with_interactions_sixth_example.jpg}
		\caption{Main effects and interactions plot with interactions in Minitab 15}
	\end{figure}
	Here we say we have: "\NewTerm{(a priori) the two factors (locally) interacting F$1$ * F$2$ where the influence of Factor $2$ is significant}".
	
	And finally a last example:
	
	Which give us the two diagrams:
	\begin{figure}[H]
		\centering
		\includegraphics{img/arithmetics/anova_principal_effects_with_interactions_seventh_example.jpg}
	\end{figure}
	\begin{figure}[H]
		\centering
		\includegraphics{img/arithmetics/anova_interactions_effects_with_interactions_seventh_example.jpg}
		\caption{Main effects and interactions plot with interactions in Minitab 15}
	\end{figure}
	\begin{tcolorbox}[title=Remark,colframe=black,arc=10pt]
	A belief (commonly spread) of people who have a lack experience in laboratories is to think that for an interaction to be significant it is necessary that the both factors that compose it are also significant.
	\end{tcolorbox}
	After all these diagrams and tables, let us turn back to the mathematical part:
	
	We saw earlier how to perform a one-way fixed factor analysis of variance. To recap, this consists to a test for equality of means for $k$ independent samples of $n$ random variables each (in the case where all samples have the same number of measures: a balanced ANOVA). Each sample is regarded as an experiment on a different subject then considered therefore as an independent variable factor!!!
	
	However it happens in the reality that we do for each sample vary a second parameter, then considered a second variable factor but independent of the first one. We speak then of course analysis of variance with two factors. In addition, we will consider in the first instance to simplify the calculations that the random variables are independent! Therefore a factor has no influence over the other has we already mention it !!! In other words there is no interaction between factors. We speak then of a "\NewTerm{two-factor ANOVA without interactions}\index{two-factor ANOVA without interaction}".
	
	To determine the formulation of the test to be performed, remember that for the analysis of variance with factor, we decomposed the total variance in the sum of the variance of average (inter-class) and of the residual variance (intraclass) such that:
	
	explicitly we comparee the samples $i=1...k$:
	
	which had given us in the end:
	
	For the ANOVA with two fixed factors we will start form the table below:
	
	for which a laboratory, factor kept fixed while we will vary the other will be named the "\NewTerm{blocking factor}\index{ANOVA!Blocking factor}" and the other will be named "\NewTerm{treatment factor}\index{treatment factor}" and in practice we will ensure that it does is not always performed in the same order to eliminate potential inertia when changing from one treatment to the other effects (some authors designate the two fixed factor ANOVA without interactions  by the terms "\NewTerm{randomized block design R.B.D.}\index{randomized block design}").
	
	To continue the whole trick is to break down the total variance by comparing the average of the rows (observations) indexed this time with $i=1...k$ and of columns (samples) indexed with $j=1\ldots r$ relatively to the total average such as:
	
	But we have in a first time:
	
	So it remains:
	
	But we also have:
	
	For the rest, let us indicate first that relatively to our above table, we have:
	
	It follows then that:
	
	and then  it comes immediately that we have likewise:
	
	So it remains in the end:
	
	what we will note that on this site in the following condensed form:
	
	where $Q_A,Q_B$ are obviously associated with the main effects (comparison of marginal means with the total average).
	
	Therefore compared to the one-way fixed factor ANOVA we have an additional term for the total variance!
	
	In order it is clear that the first sum of the differences from the first column factor:
	
	will have just as for the one-way fixed factor ANOVA $k-1$ degrees of freedom. That is to say, under the same assumptions as the one-way fixed factore ANOVA:
	
	
	The second sum of the differences from the second factor (line factore):
	
	is new but we prove in a perfectly identical way to the first one it will have $k-1$ degrees of freedom. That is to say, under the same assumptions as the one-way fixed factor ANOVA:
	
	For the third sum that follows mandatory also a chi-square law (since the total variance follows a chi-square law and that the first two terms of the sum also!):
	
	it's a little bit trickier ... but there's a trick in the physicist way of life...! We know from our study of the one-way fixed factor ANOVA that the sum of the degrees of freedom of each term must equal the total number of degrees of freedom. In other words, we must have for the two-way fixed factor ANOVA:
	
	So obviously what is missing is equal to:
	
	Therefore:
	
	We then have the following table:
	
	Finally, the rest is exactly the same as for the one-way fixed factor ANOVA just that we have two tests to do this time that are:
	
	The choice above seems intuitively practical...
	
	All calculations we have made above are often represented in software in a standard form of a table whose form and content is presented below (this is how Minitab 15.1.1 and Microsoft Excel 11.8346 present it for example):
	
	and the condition of acceptance of the null hypothesis of equality of means for each factor is the same as for the one-way fixed factor ANOVA (see the server of exercises for a detailed practical example with Microsoft Excel 11.8346).
	
	So we have two Fisher tests to know whether if factor $A$ (respectively $B$) have a significant influence on the measures or not.
	
	Obviously, in the above developments, factors $A$ and $B$ are interchangeable in the developments by symmetry!
	
	\paragraph{Analysis of Variance with two fixed factors with repetitions}\mbox{}\\\\
	So far we have examined ANOVA on experiments with one or two fixed factors : one or two categorical variables not sampled from a populations). In the case of two factors, we considered that for each combination of factors we only had one measure (cell). But it can happen (and this is better!) that have several measures for one combination of factors! 
	
	We name this type of study of "\NewTerm{experimental design with repeated measures}\index{experimental design with repeated measures}" and the results will be processed with an analysis of variance for repeated measures with two fixed factors and interactions! This is an extremely important tool as it allows validation studies by several independant laboratories (or employees) and is also associated with many other statistical tools such as the study of reproducibility and repeatability (R\&R Study) to name only the best known cases in the industrial field.
	
	You must  understand that it is mandatory in the field of statistics to involve interactions between factors systematically when we are dealing with an experience having repeated measures. This for the simple reason that the mathematical interaction term only appears in this situation!
	
	Thus, it may be intuitive (even before the proof) that a two fixed factor ANOVA with repeated measures (some authors designate the two fixed factor ANOVA with interactions under the terms "\NewTerm{general randomized block design GRBD}\index{general randomized block design}") contains one double interaction, and two main effects. A three fixed factor ANOVA with repeated measures will have one triple interaction, three double interactions and 3 main effects. And so on...
	
	Before we start, we will consider the measurement table on the next page:
	
	with the usual properties of the mean (for reminder):
	
	And remember that that the two fixed factor ANOVA without replications (and therefore without possibility to analyze mathematically the interactions), all the trick was to to break down the total variance by comparing the average of the lines indexed  with $i=1...k$ and of columns indexed with $j=1...r$ compared with the total average.
	
	The idea will now be just about the same except that we will compare the mean of the lines indexed with $i=1...k$ and columns indexed with $j=1...r$ not only in comparison to the total average but also that of each row and of each column.
	
	For this purpose with start from we got for the two fixed factor ANOVA without replication:
	
	but whose notation will just be adapted to the context:
	
	Obviously, with this notation the two fixed factor ANOVA without replication become:
	
	But in the present case, we must add a term for the replication and adapt the notation for the measurements. So, without repeating all the developments (it is a bit cheeky but...) we get directly:
	
	where in order, $m$ is the replicaton of the sample $i$ of the factor $A$ and of sample $j$ of factor $B$.
	
	It comes then obviously the interclass variances for factors $A$ and $B$ that are immediate:
	
	where $Q_A,Q_B$ are obviously again associated with main effects (comparisons of marginal averages with the total average).
	
	Now let's play a little by introducing in the sum, in positive and negative in the last term:
	
	the average of replications:
	
	that we will find in fine in the total sum of squares:
	
	Of course, we recognize rather quickly the intra-classes variance (often also named "\NewTerm{residual error}\index{residual error}" or simply in the particular case of the two fixed factor ANOVA with repetitions "\NewTerm{repeatability error}\index{repeatability error}"):
	
	and the term we can interpret (by comparison with the two factor ANOVA without repetitions) as the variance of interaction:
	
	If the ANOVA is balanced the term:
	
	must vanish. Let us check this:
	
	and therefore for $i$ and $j$ fixed it comes:
	
	And therefore the summation over all $i$ and $j$ will be zero also by extension. Those who have a doubt about the cancellation of the two terms of the development above, may be able to reassure themselves by making a numerical application (enjoy!...).
	
	Then finally:
	
	where for reminder, $n$ is therefore the number of replications, $r$ the number of samples of factor $A$ and $k$ and the number of samples of factor B (the latter two parameters are often mixed by those who make calculations by hand) . Result is sometimes noted as following in the literature:
	
	So in comparison to the two fixed factor ANOVA without replication, we have an additional term for the total variance.
	
	In the order it is clear that the first sum of the differences from the first factor column:
	
	will have just like for one way fixed factor ANOVA and two way fixed factor ANOVA without repetition $k-1$ degrees of freedom. That is to say, under the same assumptions of these latter ANOVA, we have:
	
	The second sum of the differences relatively to the second factor line:
	
	will have under the same assumptions the property:
	
	Thanks to the reasoning performed with two fixed actor ANOVA without repetition, we know that for the interaction term:
	
	We have:
	
	It remains to determine the number of degrees of freedom of the last term:
	
	To do this, we proceed in the same way as for the two fixed factor ANOVA without repetitions. We know from our study of the one way fixed factor ANOVA that the sum of the degrees of freedom of each term must be equal tot the total number of degrees of freedom. In other words, we must have for the two fixed factors ANOVA:
	
	So what is missing is obvious:
	
	Therefore:
	
	We have then the following table:
	
	Finally, the rest is exactly the same as for the two-way fixed factor ANOVA without replication with the difference that now we just have three tests this time:
	
	Again the choice of the ratios is relatively intuitive!
	All calculations we have made above are often represented in software in a standard form of a table whose form and content is presented below (this is how Minitab 15.1.1 and Microsoft Excel 11.8346 present it for example):
	
	and the null hypothesis of equality of means for each factor is the same as for one-way fixed factor ANOVA (see the exercise server for a detailed practical example with Microsoft Excel 11.8346).
	
	So we have three Fisher tests for to know whether for each if factor $A$, respectively $B$ or interaction $AB$ have a significant influence on the measures or not.
	
	Obviously, in the above developments, factors $A$ and $B$ are interchangeable by symmetry!
	
	\paragraph{Multifactor ANOVA with Repeated measures}\mbox{}\\\\
	The "\NewTerm{multifactor categorical variables ANOVA with repeated measures}\index{multifactor ANOVA with repeated measures}" is simply the name by which specialists refer to the following fixed-factors ANOVA:
	\begin{itemize}
		\item three factors (fixed) ANOVA with or without repetition
		\item four factors (fixed) ANOVA with or without repetition
		\item five factors (fixed) ANOVA with or without repetition
		\item etc.
	\end{itemize}
	Obviously, the ANOVA with one and two fixed factors are also part of the family of multifactor ANOVA. Also be aware that most statistical software manages up to ANOVA with $15$ fixed factors fixed factors (categorical variables) on the assumption condition that the plan is balanced (i.e. for each level of each factor, there is the same number of measurements). A spreadsheet software (like Microsoft Excel for example) by default ANOVA with maximum two fixed factors.
	
	Okay now the reader may be disappointed (well I am also disappointed to have only one life ...) because honestly I do not want to redo all the developments seen above for the one-way and two-way fixed factors ANOVA for $3$, $4$ and up to $15$ factors because it would take more than 100 A4 pages to do in a pedagogical and clear form plus it is always based on the same mechanical development. Sadly the generalized theory of ANOVA well being much shorter, it is indigestible for my taste.
	
	\paragraph{Latin Square ANOVA}\mbox{}\\\\
	A Latin Square is a three factor nested ANOVA but with the subtility that the third nested one has not each of its levels repeated in the levels of the two first but only one time and in such a way that each of it level appears only one time at each row and each column of the parent level:
	\begin{figure}[H]
		\begin{center}
		\begin{tabular}{|ccc|}
		\hline
		1&2&3\\
		3&1&2\\
		2&3&1\\
		\hline
		\end{tabular}
		\hspace{10pt}
		\begin{tabular}{|cccc|}
		\hline
		4&3&1&2\\
		3&4&2&1\\
		1&2&4&3\\
		2&1&3&4\\
		\hline
		\end{tabular}
		\hspace{10pt}
		\begin{tabular}{|ccccc|}
		\hline
		1&2&4&3&5\\
		4&5&2&1&3\\
		3&4&1&5&2\\
		2&3&5&4&1\\
		5&1&3&2&4\\
		\hline
		\end{tabular}
		\caption{Latin squares of orders 3, 4, and 5}
		\end{center}
	\end{figure}

	A complete illustration of such as design can be given by the following example:
	
	Therefore we can already observe that:
	\begin{enumerate}
		\item The order of performing the trials is random
		\item We save many trials (as the example above should have $5\cdot \cdot 5 \cdot 5=125$ trials if it were complete)
	\end{enumerate}
	
	
	
	
	
	
	
	
	
	
	
	
	
	
	
	
	
	
	
	
	
	
	
	
	
	
	
	
	
	
	
	
	
	
	
	
	
	
	
	
	
	
	
	
	
	
	
	
	
	
	
	
	\paragraph{Greaco-Latin Square ANOVA}\mbox{}\\\\
	A Graeco-Latin square ANOVA is an extension of a Latin square used when the number of block is greater than $2$. Concretely, a Graeco-Latin square ANOVA can control up to two nuisance factors and one treatment factor (three sources of extraneous variability...) as represented below:
	
	Use of these designs results in exceptional saving of time and means, as the total number of design points has been drastically reduced as if we consider $n$ level for each of the three factor we have only $n^2$ experimental points, when compared to $n^4$ (since there are for factors!!!!) design points in full factorial design. However similar to Latin squares, these designs may be used only when interactions are statistically insignificant.
	
	To figure out a Graeco-Latin square, we consider initially a Latin square $K \otimes K$ on which we superimposed another Latin square $K \otimes K$ which this time is denoted by Greek letters. Afterwards in a second time during the superposition, we must ensure that the following property is respected, "each row and each column can only contain couple of letters (Latin, Greek) that are distincts". Thus, if this property is verified (as shown in the diagram below) we say that the two orthogonal Latin squares are orthogonals:
	
	As you can notice, the Graeco-Latin square design allows the analysis of $4$ factors (row, column, Latin letter, Greek letter), each factor having $K$ levels for only $ K \otimes K $ or $K^2$ possible combinations.

	For the need of the analysis of variance, we have to define certain variables:

	\begin{itemize}
		\item $K$ the number of levels for each factor
		
		\item $ N = K ^ 2 $, the total  number of observations $ y_ {i, j, k, l} $ in the Graeco-Latin square, where $ i $ is the $ i ^ e $ level of the factor represented in row, $ j $ the $ j ^ e $ level of the factor represented in column, $ k $ the $ k ^ e $ level of the factor represented by Latin letters and the $ l $ the $ l ^ e$ level of the factor represented by Greek letters.
	\end{itemize}
	
	Ainsi, on peut formuler comme suit :
	\begin{itemize}
	  \item La somme et la moyenne des observations:
	 
	  \item La somme et la moyenne pour chaque ligne:
	   
	  \item La somme et la moyenne pour chaque colonne:
	   
	  \item La somme et la moyenne pour chaque lettre latine:
	  
	  \item La somme et la moyenne pour chaque lettre grecque:
	   
	\end{itemize}
	Dans la suite, nous allons préférer cette notation $\sum_{i,j,k,l}^{K}$ à celle-ci $\sum_{i=1}^{K}{\sum_{j=1}^{K}{\sum_{k=1}^{K}{\sum_{l=1}^{K}{y}_{ijkl}}}}$.
	
	
	Décomposition de la variance totale

	Nous allons démontrer que la variance totale :
	
	peut être écrite:
	
	Commençons par poser :
	 
	intercalons comme ceci
	
	A ce niveau, on retrouve bien notre fameuse identité remarquable $(a + b)^2 =  a^2 +b^2+ 2ab$, qui dans le cadre notre développement, s'applique comme ceci :
	
	Considérons le $3^{e}$ membre de cette expression $2\sum ab$ :
	
	et démontrons que le résultat de cette dernière est égal à $0$. Commençons par le développer comme ci-dessous :
	
	Développons la première sous-expression $^{(a)}$:
	
		Évaluons chaque expression issue du développement de $^{(a)}$ 
	
	Somme toute nous obtenons que la sous-expression $^{(a)}$ :
	
	est égale à :
	
	ce qui donne :
	
	Ainsi, par le même procédé nous pouvons prouver que chacune des 4 sous-expressions restante est nulle:
	
	
	
	
	Maintenant que nous avons prouvé que $2\sum ab  = 0$ c'est-à-dire :
	
	Ainsi, si l'on remonte à :
	
	Vu que le $3^{e}$ membre de l'équation, $2\sum ab$  est égal à zéro, il ne nous reste que $\sum a^2 + \sum b^2$ :
	
	Prenons le $2^{e}$ membre $\sum b^2$, de l'expression ci-dessus :
	
	Dans le dernier développement ci-dessus, considérons la sous expression suivante :
	
	et démontrons qu'elle est nulle.
	
	Par conséquent, chacune des expressions suivantes :
	
	
	
	est égale à $0$.
		Finalement il nous reste :
	
	
	
	Si l'on note SSE la somme des carrés des écarts alors on peut réécrire notre résultat sous cette forme :
	
	où:
	
	
	
	
	
	
	Avec comme degrés de liberté :
	$$\text{SCE}_{T} = \text{SSE}_{L} + \text{SSE}_{C} + \text{SSE}_{l} + \text{SSE}_{g} +\text{SSE}_{r}$$
	$${K}^{2}-1 = \left(K-1 \right) + \left(K-1 \right) + \left(K-1 \right) + \left(K-1 \right) + \left(K-1 \right)\left(K-3 \right)$$
	Le degré de liberté pour les carrés des écarts résiduels s'obtient en posant :
	$${K}^{2}-1 -\left[\left(K-1 \right) + \left(K-1 \right) + \left(K-1 \right) + \left(K-1 \right) \right] = \left(K-1 \right)\left(K-3 \right)$$
	
	\paragraph{Multivariate Analysis of Variance (MANOVA)}\mbox{}\\\\
	The "\NewTerm{Multivariate Analysis of Variance}\index{Multivariate Analysis of Variance}" is an extension of ANOVA where we assume that the categorical variables are linearly dependent and we test whether there is a significant difference in the averages (variables supposed to be of the continuous type) through the different groups under the interdependence of the categorical explanatory variables.

	If it is not clear (...), the mathematical notation may be more explicit to the reader. Let us recall first that in the case of the canonical ANOVA with $1$ fixed factor at $k$ levels, the null hypothesis was of the form:
	
	For the one-way MANOVA, the null hypothesis is:
	
	where we have therefore "mean vectors" for a given number of dependent group variables in a quantity $k$.

	Thus explicitly for $k$ groups and $p$ categorical variables:
	
	As the reader will see in the developments below, the MANOVA e one-way MANOVA has nine assumptions (hypothesis):
	\begin{itemize}
		\item[H1.] Your two or more dependent variables should be measured at a continuous level (i.e., they
are interval or ratio variables)
		\item[H2.]  Your independent variable should consist of two or more categorical, independent (unrelated)
groups
		\item[H3.] You should have independence of observations, which means that there is no relationship
between the observations in each group or between the groups themselves
		\item[H4.] You should have an adequate sample size. At least need to have more cases (e.g., participants)
in each group of the independent variable than the number of dependent variables
you are analysing.
		\item[H5.] There are no univariate or multivariate outliers. First, there can be no (univariate) outliers
in each group of the independent variable for any of the dependent variables.
		\item[H6.] There is multivariate normality.
		\item[H7.] There is a linear relationship between each pair of dependent variables for each group of
the independent variable.
		\item[H8.] There is a homogeneity of variance-covariance matrices.
		\item[H9.] There is no multicollinearity
	\end{itemize}
	Let us now recall that in the one-factor ANOVA we compute the $p$-value of the ratio of the estimated variances assuming the true variances as equals:
	
	The next step in understanding MANOVA is the recognition that mathematicians are lazy and sometimes do engineering statistics... So first we simplify this ratio by multiplying left and right by $(k-1)/(N-1)$. This gives what we name and "$A$-statistics":
	
	But this is not enough as we are still with independent variables! So the idea is to used variance-covariance
like matrix and to get a similar expression of as a ratio we take the determinant... This brings us to write a multivariate like Fisher test for variances (......):
	
	better known in the following form:
	
	where $W$ and $T$ are the respectively the determinants of the square matrices of the sum of the squares of the within ($W$) and global (Total: $T$) deviations. Which means that if the part "within" (Between: $B$) the independent variables is really large, then $\Lambda$ tends to zero. In contrast, if $B$ is very small, tends $\Lambda$ tends to $1$.

	To prove that this ratio does not depend on the covariance matrix of the dependent variables, let us recall that considering the structure of $W$ and $B+W$ we can make a spectral decomposition (see further below our study of principal component analysis) in the form:
	
	It can be demonstrated with some approximations that $\Lambda$ follows a distribution named "Wilk's distribution" or "Samuel Stanley Wilks'".

	It should be noted that Wilk's $\Lambda$ can be expressed as a function of the eigenvalues of $W^{-1}B$. From the definition of $\Lambda$, it follows using the properties of the determinant (\SeeChapter{see section Linear Algebra}) that:
	
	We recognize here the determinant of the "eigenvalue equations" (\SeeChapter{see section Linear Algebra}) on $W^{-1}B$ but with $\lambda=1$. Indeed:
	
	So in our case here we have $\lambda=-1$ therefore:
	
	Therefore:
	
	and consequently:
	
	Also, it follows that:
	
	When Wilk's $\Lambda$ approaches $1$, we showed that it means that the difference in means is negligible. This is the case when $\ln(\Lambda)$ approaches $0$. However, when $\Lambda$ approaches $0$ or $\ln(\Lambda)$ approaches $1$, it means that the difference is large. Therefore, a large value of $\ln(\Lambda)$ (i.e., close to $0$) is an indication of the significance of the difference between the means.
	\begin{tcolorbox}[title=Remark,colframe=black,arc=10pt]
	The bad news is that there are in fact four different multivariate tests that are made from the above relation. The reason for four different statistics and for approximations is that the mathematics of MANOVA get so complicated in some cases that no one has ever been able to solve them.
	\end{tcolorbox}
	Let us see now a companion detailed example:
	\begin{tcolorbox}[colframe=black,colback=white,sharp corners]
	\textbf{{\Large \ding{45}}Example:}\\
	We consider the following set of two dependent variables on three groups of independent variables:
	 
	with:
	
	Now we calculate:
	
	With then for example for the first group:
	
	Which give us in our case (very easy to check with any spreadsheet software):
	
	\end{tcolorbox}
	\begin{tcolorbox}[colframe=black,colback=white,sharp corners]
	And we have (we notice that we assume that over all the $x_1$ and $x_2$ the number of measurement are $N_1=N_2=N$):
	
	and therefore:
	
	After the calculation of the p-value is more delicate and we will ask the reader to refer to the Minitab or R companion books for the computational details and the final conclusion.
	\end{tcolorbox}
	So we could use a one-way MANOVA to determine whether students’ short-term and long-term recall of facts differed based on three different lengths of lecture (i.e., the two dependent variables are "short-term memory recall" and "long-term memory recall", whilst the independent variable is "lecture duration", which has four independent groups: "$30$ minutes", "$60$ minutes", "$90$ minutes" and "$120$ minutes"). Alternately, a one-way MANOVA could be  used to determine whether there is a difference in salary and bonuses based on degree type (i.e., the two dependent variables are "salary" and "bonuses", whilst the independent variable is "degree type", which has five groups: "business studies", "psychology", "biological sciences", "engineering" and "law").

	When there is a statistically significant difference between the groups of the independent variable, it is possible to determine which specific groups were significantly different from each other using post hoc tests. You need to conduct these post hoc tests because the one-way MANOVA is an omnibus test statistic and cannot tell you which specific groups were significantly different from each other; it only tells you that at least two groups were different.
	
	\pagebreak
	\subsubsection{Equivalence tests}
	As the reader will probably have understood it from the preceding paragraphs, hypotheses tests (NHST) applied to the search for differences between groups do not technically allow us to conclude an equivalence simply because we do not reject the null hypothesis $H_0$. Yet we also showed that if the power of the test was typically above $80\%$ it did not make a problem to consider the null hypothesis $H_0$ as true but the problem is that for this we need most of times to consider large samples and in practice this is not always feasible and it can also lead to absurdities since the maths show that with large samples we almost systematically reject the null hypothesis $H_0$.

	The $p$-value can be used only to statistically reveal a posteriori the rejection of the null hypothesis $H_0$ in the vast majority of cases. 	To summarize this problem (I like this sentence!...): \textit{The absence of evidence, does not imply evidence of absence}. Expressed in other words in a very common case this gives (particular example!) that the non-rejection of the null hypothesis $H_0$ of equality of two means (for example) does not imply the equality of means!!!

	With the advent of generic medicines from the 1960s in the pharmaceutical field, the importance of highlighting equivalences (we are talking instead of "\NewTerm{bioequivalences}\index{bioequivalence}") has taken a growing place and organizations like the F.D.A. (Federal Drug Administration\index{Federal Drug Administration} in U.S.A.), EMA (European Medicines Agency\index{European Medicines Agency}) or WHO (World Health Organization\index{World Health Organization}) vave common guidelines that lead to the approval of generic only if the equivalence is shown under certain empirical conditions but at least standardized according to a consensus of experts.

	Thus, most often the following three conditions are required\footnote{still we the scientific publication rules introduced at the beginning of this book}:
	\begin{enumerate}
		\item A comparison criterion is required

		\item A confidence interval is required

		\item An a priori limit of bioequivalence is required
	\end{enumerate}
	Let us see a companion example to see the concept:
	\begin{tcolorbox}[colframe=black,colback=white,sharp corners]
	\textbf{{\Large \ding{45}}Example:}\\
	Let us consider the Student's T test of the difference of the averages of two unpaired samples $\{$Test, Reference$\}$ which we had demonstrated above:
	
	We then know the following two bounds of the confidence interval:
	
	\begin{tcolorbox}[title=Remark,colframe=black,arc=10pt]
	We speak in this particular case of "TOST" test for "\NewTerm{two-one sided test}\index{two-one sided test}".
	\end{tcolorbox}
	We will assume that the F.D.A. request:
	
	Consider that the measurement have given us:
	
	Therefore, we have the equivalence limits that will be:
	
	The confidence intervals are given by:
	
	\end{tcolorbox}
	\begin{tcolorbox}[colframe=black,colback=white,sharp corners]
	Therefore:
	
	As we have:
	
	The equivalence is then considerated by experts consensus as "proven".
	\end{tcolorbox}
	

	\subsubsection{Cochran C test}
	The Cochran's C test has for purpose to verify the homogeneity of variances for several populations. This is a preliminary or subsequent test (post hoc) helpful test to do before making a balanced ANOVA (balanced) analysis and is recommended by the standard ISO 5725 (as the Tukey's test we will see much further) !
	
	Although the idea of the Cochran's C test is empirical, it is nevertheless intuitive as are the definitions of Dixon and Grubbs tests. Why then do we present in this book in details the proof of the Cochran's C test when we have mentioned that we would not do this for the test of Grubbs and Dixon because the approach of these latter was also empirical? The reason is simple in fact: the test of Grubbs and Dixon require, at least as far as we know, Monte Carlo simulations to determine the critical values of rejection or acceptance of the null hypothesis, while the critical value of Cochran's C test can be obtained relatively easily analytically.
	
	That said ... we define the Cochran's C test by the ratio:
	
	where the $S_i$ are the unbiased variance of the $N$ different sources of data, each composed each of $n$ samples and the null hypothesis is intuitively the equality of variances against the alternative hypothesis that one of the variance is too big (that is to say: bad) and dismissed because considered as an outlier variance.
	
	The ISO 5725 recommends to repeat this test until there is no longer any aberrant variance (therefore too large AND far from other variances).
	
	To determine the critical value, let us invert the definition of Cochran's C test and do some basic algebraic manipulations:
	
	We note that that pretty much that the second term of the last equality looks like a Fisher law. As the Fisher is not stable by the addition, we should find a way to turn the term:
	
	into a single variance. The idea is then relatively simple but still had someone to think about it... We know the $S_i$ are non-biased variances equation with a factor $1/(n-1)$. Therefore if the $N$ samples (levels) are independent, the overall variance is then for by stability of the Normal distribution and by taking the traditional notations  of the ANOVA:
	
	Therefore:
	
	We recognize in the last equality the ratio of two squared variances. We then have identically to what we've proved in our study of the one-way fixed factor ANOVA without replication:
	
	and therefore it comes:
	
	which is therefore independent of $j$ and therefore the left-tail Cochran's C test (since by by definition the Cochran's ratio should be as small as possible) will have for critical value:
	
	If we consider a test with a significance level of $1-\alpha$ (thus corresponding to the cumulative probability of not making a Type I error) and we reiterate it independently again as second time. Then, if the tests are independent, following the axiom of probabilities, the probability of not making a Type I error will be given by the product of probabilities:
	
	and so on for $n$ tests. We notice so quickly that the cumulative probability of not making a Type I error decreases very quickly. For example, for $10$ independent repeated tests with a $5\%$ level, then we have:
	
	which is catastrophic! So if we want a level of confidence on repeated tests of a certain value which we will denote $\alpha_r$, it is clear that we must solve the following equation:
	
	Therefore (relation sometimes named "Šidàk equation"):
	
	and with a Taylor expansion to the second order it comes (\SeeChapter{see section Sequences and Series}):
	
	that we name "\NewTerm{Bonferroni approximation}\index{Bonferroni approximation}" or sometimes also "\NewTerm{Boole approximation}\index{Boole approximation}" or "\NewTerm{Dunn approximation}\index{Dunn approximation}". So in the end, we have for the Cochran's C test:
	
	that we can compute with the English version of Microsoft Excel 14.0.6123 using the following formula:
	
	\begin{center}
	\texttt{=1/(1+(N-1)/FINV(ALPHA/N,n-1,(N-1)*(n-1)))}
	\end{center}
	\begin{tcolorbox}[title=Remark,colframe=black,arc=10pt]
	We benefit here from assuming that all tests are independent of each other. In practical applications, that is often not the case. Depending on the correlation structure of the tests, the Bonferroni correction could be extremely conservative, leading to a high rate of false negatives!!!
	\end{tcolorbox}
	
	\subsubsection{Adequation Tests (goodness of fit tests)}
	The goodness of fit (GoF) of a statistical model describes how well it fits a set of observations. Measures of goodness of fit typically summarize the discrepancy between observed values and the values expected under the model in question. Such measures can be used in statistical hypothesis testing, e.g. to test for normality of residuals, to test whether two samples are drawn from identical distributions (see Kolmogorov–Smirnov test further below), or whether outcome frequencies follow a specified distribution (see Pearson's chi-squared test below). 
	
	\paragraph{Pearson's chi-squared GoF test}\mbox{}\\\\
	We will study here our first GoF nonparametric test, certainly one of the most known and most simple one (which applies only to non-censored data as far as we know).
	
	To introduce this test, assume that a random variable follow a probability distribution $P$. If we draw a sample from the population corresponding to this law, the observed distribution, named "\NewTerm{sampling distribution}\index{sampling distribution}", always deviate more or less of the theoretical distribution, taking into account the sampling fluctuations.
	
	Generally, we do not know the shape of the law $P$, nor the value of its parameters. It is the nature of the studied phenomenon and the analysis of the observed distribution that allow to choose a law likely to be adequate and afterwards to estimate the parameters.
	
	The differences between the theoretical law and the observed distribution can be attributed either to sampling fluctuations, or to the fact that the phenomenon does not follow, in reality, the supposed law.
	
	Basically, if the gaps are small enough, we will assume they are due to random fluctuations and we will accept (not reject in fact!) the supposed law. On the contrary, if the gaps are too high, we conclude that they can not be explained solely by the fluctuations and the phenomenon does not follow the supposed law (reject the null hypothesis).
	
	To assess these gaps and to make a decision, we need:
	\begin{enumerate}
		\item Define the measure of the distance between the empirical distribution and the theoretical resulting from the retention law.
		
		\item Determine the probability law followed by the random variable giving the distance (in fact sadly not reject the null hypothesis).
		
		 \item State a decision rule to tell from the observed distribution, if the law chosen is acceptable or not.
	\end{enumerate}
	First, we will need for this purpose the central limit theorem and secondly recall that during the construction of the Normal distribution, we have proved that the variable:
	
	follow a Normal distribution centered reduced when $n$ approaches infinity (Laplace condition) and the probability $p$ was very small.
	
	In practice, the approximation is quite acceptable.,. in some companies... and when $np>5$ and $p\leq 0.5$ therefore (it was one of the terms that needed to tend to zero when we made the proof):
	
	For example in the two figures below where we represented the binomial laws approached by the Normal associated laws, we have on the left $n=60,p=1/6,np=10$ and on the right $n=40,p=0.05,np=2$:
	\begin{figure}[H]
		\centering
		\includegraphics{img/arithmetics/binomial_normal_approximation.jpg}
		\caption{Approach of binomial functions by associated Normal functions}
	\end{figure}
	Finally, remember that we have proved that the sum of squares of $n$ linearly independent reduced centered Normal random variable follows a chi-square with $n$ degrees of freedom denoted by $\chi^2(n)$.
	
	Now consider a random variable $X$ that follows a theoretical distribution function (continuous or discrete) $P$ and let us draw a sample of size $n$ in the population corresponding to this law $P$.
	
	The $n$ observations are distributed along $k$ terms (class values) $C_1, C_2, ..., C_k$, whose probabilities $p_1, p_2, ..., p_k$ are determined by the distribution function $P$ (refer to the example with the Henry straight line).
	
	For each modality $C_i$, the empirical sample size is a binomial random variable $k_i$ of parameters:
	
	This number $k_i$ corresponds indeed to the number of successful events: "result equal to the modality $C_i$" of probability $p_i$, obtained during the sampling on $n$ items of the experimental batch (and not in the population of the theoretical law as before!).
	
	We have proved earlier above during our the study of the binomial law that his expected mean was:
	
	represents the expected theoretical sample size of the modality $C_i$ and its variance is given by (when $n$ is very big and $p_i$ very small):
	
	Its standard deviation is therefore:
	
	Under these conditions, provided that the modality $C_i$ has a size $np_i$ of  at least equal to $5$, the reduced centered variable:
	
	between empirical and theoretical sample size can be approximately regarded as a reduced centered Normal variable as we prove earlier above.
	
	We now define now the variable:
	
	where $k_i$ is often named "\NewTerm{experimental frequency}\index{experimental frequency}" and $np_i$ "\NewTerm{theoretical frequency}\index{theoretical frequency}".
	
	If we take the square it is because that in a simple sum certain terms would cancel by opposing effects and thus would mask the differences, if we take the sum of the absolute values the Statistical Table of $D$ would be difficult to build and the test would be not very robust because of the small gap of distances. The square allows not only to have an easy statistical table of $D$ which is simple since it is based on a law with a single parameter, as we shall see, and the square also sufficiently increases the test's robustness.
	
	Note that this variable is also sometimes (somewhat unfortunately) denoted by:
	
	or more often:
	
	This variable $D$, sum of squared variables $E_i$, provides a measure of a "distance" between empirical and theoretical distribution and empirical distribution. Let us note however that this is not a distance in the usual mathematical sense (topological).
	
	Recall that $D$ can therefore also be written:
	
	$D$ is therefore the sum of squares of $N$ reduced centered Normal random variables linked by the single linear relation:
	
	where $n$ is the sample size. So $D$ follows a chi-square distribution, but with $N-1$ degrees of freedom, so a degree less because of the unique linear relation between them! Indeed, recall that the degree of freedom is the number of independent variables in the sum and not just the number of summed terms.
	
	Therefore:
	
	We name this test the "\NewTerm{non-parametric Chi-square test}\index{non-parametric Chi-square test}" or "\NewTerm{Pearson's Chi-square test}\index{Pearson's Chi-square test}" or "\NewTerm{Chi-square test of adjustment}\index{Chi-square test of adjustment}" or "\NewTerm{Karl Pearson's test}\index{Karl Pearson's test}" or "\NewTerm{goodness of fit Chi-square test}\index{goodness of fit Chi-square test}"...
	
	Then, the usage is to determine the value of the Chi-square distribution with $N-1$ degrees of freedom with a $5\%$ probability of being exceeded. Thus, in the hypothesis that the studied phenomenon, follows the theoretical distribution $P$, so there is a $95\%$ cumulative probability that the variable $D$ takes a value less than the one given by the Chi-Square distribution.
	
	If the value of the law of Chi-square obtained from the sample is smaller than that corresponding to $95\%$ of cumulative probability, we do not reject the null hypothesis that the phenomenon follows the law $P$.
	
	\begin{tcolorbox}[title=Remarks,colframe=black,arc=10pt]
	\textbf{R1.}  The fact that the assumption of the law $P$ is accepted does not mean that this hypothesis is true, merely that the information given by the sample does not allow to reject it. Similarly, the fact that the assumption of the law $P$ is rejected does not necessarily mean that this assumption is false but that the information provided by the sample rather lead to the conclusion that the inadequacy of such a law.\\
	
	\textbf{R2.} For the variable $D$ to follow a chi-square law, it is necessary that the theoretical values $np_i$ of the different modalities $C_i$ are at least equal to $5$, that the sample was drawn randomly (no correlation) and that the probabilities $p_i$ are not  close to zero.
	\end{tcolorbox}	
	This goodness of fit test however suffers from a major issue: it requires to group the measurements in classes $C_i$ and in practice there is no absolute theorem (at least as far se know) to choose the number of classes (and in full width) excepted the Sturge Rule proved earlier. It is this reason that make the Chi-2 goodness of fit test is reserved for discrete distributions where the problem of the choice of classes not arise.
	
	However, we will need to create goodness of tests that do not require the use of classes and we will see just after ad hoc tools for this purpose (Kolmogorov-Smirnov, or Anderson-Darling to name only the most important one).
	\begin{tcolorbox}[colframe=black,colback=white,sharp corners]
	\textbf{{\Large \ding{45}}Example:}\\
	Suppose that the births in a hospital for a period of time, are as follows:
		
	We note that there were a total of $728$ births. We ask then the following question: How many should there be births, in theory, every day if there is no difference between days? This represents the null hypothesis $H_0$. In fact the null hypothesis states that the differences between the observed frequencies and conceptual frequencies are relatively small. We take for granted that if there is no difference there should be the same number of births each day. Since there are a total of $728$ births for $7$ days in theory there should be $728/7 = 104$ births every day. So now we have the following table:
		
	\end{tcolorbox}
	\begin{tcolorbox}[colframe=black,colback=white,sharp corners]
	The total of observed frequencies equals the total expected frequencies. The purpose is therefore to examine the difference between the observed and expected frequencies (assumed to follow a uniform law in this special case) using the Chi-square relation. In other words, we do a fit test between an empirical distribution function (observed) and the uniform distribution function. Then we have:
	
	The $\chi^2$ is therefore $43.49$. As such this number means little. This result should be interpreted with the help of the table of critical values of the $chi_N^2$. Without using the table we understand that it is very unlikely that the observed frequency and theoretical frequency are identical. We accept that there may be some difference (we therefore reject the null hypothesis $H_0$ in favor of the alternative hypotheses $H_A$).
	
	So do not forget that this test only applies to uncensored data, that is to say for which the intervals are closed!
	\end{tcolorbox}

	\paragraph{Kolmogorov-Smirnov GoF test}\mbox{}\\\\
	In statistics, the Kolmogorov-Smirnov test is also a goodness of fit test based on an empirical distance used to determine whether the distribution of a sample follows a well known law given by a continuous distribution function (or for comparing two samples and check if they are dependent or not as similar or not). This test, as well as that of Chi-2 GoF test is only valid for non-censored data (at least not without correction obtained by numerical simulations).
	
	To introduce this test, we chose the Lilliefors approach that give the possibility to avoid complex calculations. Furthermore, softwares that provide the "\NewTerm{Lilliefors GoF test}\index{Lilliefors Goodness of Fit test}" do not offer the Kolmogorov-Smirnov test since the latter is correct only asymptotically (which is the case of the software Tanagra 4.14).
	
	Imagine we want to build a nonparametric GoF test who works for both discrete and continuous laws without suffering the same problem as the Chi-square GoF test (clustring in classes).
	
	To build this test, we start from the empirical distribution function already defined at the beginning of this section and given for reminder by:
	
	that obviously belong to the interval $[0,1]$.
	
	Let us now denote by $F(x)$, the true supposed law which analytical expression is known and with which we would like to compare $\hat{F}(x)$ and build the distance:
	
	\begin{tcolorbox}[title=Remark,colframe=black,arc=10pt]
It seems that the letters to represent numbers were used for the first time by Viète in the middle of the 16th century (but the notation of exposants did not exist at this time).
	\end{tcolorbox}
	
	The reference distribution may, however, also originate from another measurement sample. The idea is then simply to compare two empirical distributions. We speak then of "\NewTerm{Kolmogorov-Smirnov test for $2$ independent samples}\index{Kolmogorov-Smirnov test for $2$ independant samples}". Some softwares also manage empirically the case where the two samples do not have the same size.
	
	The problem with this choice of distance is ... what $x$ should we then choose to make a test? Well to answer it is simple to see that it would be foolish to take a $x$ for which this distance is minimal, because have a $D_n$ that can be almost zero does not add much information... Therefore, we rather postponed towards the greatest absolute deviation. Which brings us to redefine the distance $D_n$ as follows:
	
	where $D_n$ is named "\NewTerm{Kolmogorov-Smirnov empirical distribution}\index{Kolmogorov-Smirnov empirical distribution}" (of good course we should prove rigorously that it is really a distribution ... but for now it is too complex in terms of the content of this book however this can be verified by numerical simulations). Before going further with respect to the theory, let's look at a practical example (because the example is long we will not put it into the conventional box).
	
	Suppose we measured the following five values:
	
	thus ordered:
	
	We want to test the following null hypothesis:
	
	where $\Phi(x)$ is as usual the distribution function of the Normal centered reduced distribution.
	
	The empirical distribution function will be given by:
	
	Then we traditionally build the following table:
	
	Often associated with the following graph comparing empirical (in red) and theoretical (in blue) distributions:
	\begin{figure}[H]
		\centering
		\includegraphics{img/arithmetics/ks_test_empirical_vs_theoretical.jpg}
		\caption{Representation of the approach of the Kolmogorov-Smirnov GoF}
	\end{figure}
	We then observed that the maximum deviation is $0.326$. We will denote that for after:
	
	that some softwares such as Minitab or R denote with the abbreviation: KS.
	
	The reader will have noticed that the biggest deviation above the curve is measured by:
	
	The largest deviation below the curve is measured by:
	
	The biggest deviation is then:
	
	But what can we do with this value? To what can we compare it? Well the idea is relatively simple and involves generating $n$ values (thus $5$ in our example) from the distribution law $F(x)$ of the null hypothesis $H_0$ and compare them to themselves. In other words, to make a Monte Carlo simulation (\SeeChapter{see section Theoretical Computing}).
	
	Thus, in our example, we generate 5 random value values of $\mathcal(0,1)$ which gives us example with Microsoft Excel 11.8346:
	
	\begin{center}
	\texttt{=NORM.S.INV(RANDBETWEEN(0,1000000)/1000000)}
	\end{center}
	
	We obtain then $5$ values of $Z$ (remember it is the usual notation of a random variable of a Normal centered reduced distribution ) which ordered will be for example:
	
	and we repeat the same table as before:
	
	And so we have the maximum deviation that is $0.333$. Either with Microsoft Excel 14.0.6123:
	\begin{figure}[H]
		\centering
		\includegraphics{img/arithmetics/ks_ExcelValuesList.jpg}
		\caption[]{Calculations in Microsoft Excel 14.0.6123}
	\end{figure}
	with the explicit formulas:
	\begin{figure}[H]
		\centering
		\includegraphics[scale=0.7]{img/arithmetics/ks_ExcelValuesListFormulas.jpg}
		\caption[]{Explicit calculations in Microsoft Excel 14.0.6123}
	\end{figure}
	with the small corresponding VBA routine quickly and poorly made that will take the number of iterations required in the cell K1 and will put the empirical distribution of Kolmogorov-Smirnov in column G of the active sheet:
	\begin{figure}[H]
		\centering
		\includegraphics[scale=0.9]{img/arithmetics/ks_VBA.jpg}
		\caption[]{Small VBA script for K-S GoF test}
	\end{figure}
	We therefore reiterate the procedure a thousand times and we get the following distribution function (obtained simply by making a scatter chart type in Microsoft Excel 14.0.6123 with $2,000$ simulations):
	\begin{figure}[H]
		\centering
		\includegraphics{img/arithmetics/ks_distribution_simulation.jpg}
		\caption[]{K-S GoF distributin simulation}
	\end{figure}
	and applying a one-sided test with a threshold of $\alpha=5\%$ we get for the $95$th percentile:
	
	The reader will find the same value in the Kolmogorov-Smirnov tables available in many books. A few thousand simulations are therefore sufficient to restore the values of the tables!
	
	And now, we compare:
	
	and therefore we do not reject the null hypothesis.
	
	But ... we must still take care with only five values, it is quite likely that the null hypothesis $H_0$ is not rejected for other distribution laws that the Normal distribution.
	
	Thus, as the reader will have noticed, for each null hypothesis $H_0$ associated with a particular distribution law, we must tabulate the empirical distribution of Kolmogorov-Smirnov for different values of $n$ and of $\alpha$ using numerical methods. In the majority of books there is only one table with a powerful theorem that shows that in reality, the critical values will be the same.
	
	\begin{tcolorbox}[title=Remark,colframe=black,arc=10pt]
	Kolmogorov and Smirnov have proved that when $n$ goes very large and that the law of the null hypothesis is continuous, it is no longer necessary to tabulate a Kolmogorov-Smirnov table for each law because we have then:
	
	therefore $D_n$ is independent of the distribution law of the null hypothesis $H_0$. By simulating with the Monte Carlo method, we observe an effective convergence when $n$ exceeds a hundred. But in practice, the vast majority of the time, it is unthinkable to have such a number of measures. Hence the fact that this theoretical result is little used in practice and justifies the absence of proof in this book.
	\end{tcolorbox}
	To conclude on the K-S GoF test, let us notice to the reader that he will find the mathematical proof of the Anderson-Darling GoF further below.
	
	\paragraph{Ryan-Joiner GoF test}\mbox{}\\\\
	Consider a random variable $X$ which we know the sampling distribution and for which we would like to check the Normality or not and consider an ordered random variable $Y$ generated by a Normal centered reduced distribution. To compare $X$ and $Y$, we will center $X$ and order its values in ascending order.
	
	For a same given sample size, if the ordered values of $X$ and $Y$ taken in pairs follow the same law, a linear regression based on the other should give a fairly close correlation coefficient of $1$. Taking the definition of the squared correlation coefficient, we have then:
	
	$Y$ is assumed to follow a centered reduced Normal distribution. It comes then:
	
	and if we take the estimator of the correlation coefficient:
	
	Therefore after simplification:
	
	This is the Ryan-Joiner approach (implemented in Minitab for example) of the Shapiro-Wilk test. The results of both tests are very similar. The coefficients $a_i$ can be easily obtained using any spreadsheet software by using a Monte Carlo simulation (\SeeChapter{see section Theoretical Computing}). If our reader wish it we will detail how to get the coefficients $a_i$ with Microsoft Excel for a given $n$.
	
	\begin{tcolorbox}[colframe=black,colback=white,sharp corners]
	\textbf{{\Large \ding{45}}Example:}\\
	Consider the $10$ measures of column $A$ already sorted in ascending order:
	\begin{figure}[H]
		\centering
		\includegraphics[scale=0.7]{img/arithmetics/rj_gof_table_excel_values.jpg}
		\caption[]{Example of ordered measures, ranks, RJ coefficient and $Z$-score}
	\end{figure}
	\end{tcolorbox}
	
	\pagebreak
	\begin{tcolorbox}[colframe=black,colback=white,sharp corners]
	The corresponding formulas are:
	\begin{figure}[H]
		\centering
		\includegraphics[scale=0.62]{img/arithmetics/rj_gof_table_excel_formulas.jpg}
		\caption[]{Formulas in  Microsoft Excel 14.0.6123 of previous screen shot}
	\end{figure}
	and therefore we have in a sheet named \textit{CoeffMonteCarlo} Monte Carlo simulations to determine the $10$ coefficients $a_i$ traditionally denoted in the case of $10$ measurements in the tables as following: $\left\lbrace a_{1/10},a_{2/10},...,a_{10/10} \right\rbrace$.
		First we must create $10$ columns with Normal centered reduces variables  on almost $10,000$ rows with the following Microsoft Excel formula:
	\begin{figure}[H]
		\centering
		\includegraphics[scale=0.75]{img/arithmetics/rj_normal_centered_reduced_variables_for_coeff.jpg}
		\caption[]{Centered reduced normal variables for RJ coefficients}
	\end{figure}
	and then you have to build the ranks of all these values row by row such as:
	\begin{center}
		\texttt{=NORM.S.INV(RANDBETWEEN(1;99999999)/100000000)}
	\end{center}
	and then we have to build the ranks (ascending order) of all these values row by row such as:
	\begin{figure}[H]
		\centering
		\includegraphics{img/arithmetics/rj_normal_centered_reduced_variables_ordered.jpg}
		\caption[]{Ascending row by row sorting of simulations to determine the RJ coefficients}
	\end{figure}
	\end{tcolorbox}
	\pagebreak
	\begin{tcolorbox}[colframe=black,colback=white,sharp corners]
	with the following formulas (given only for the first four $i$ because of lack of space on the screenshot):
	\begin{figure}[H]
		\centering
		\includegraphics{img/arithmetics/rj_normal_centered_reduced_variables_ordered_formulas.jpg}
		\caption[]{Microsoft Excel 14.0.6123 formulas of previous screenshot}
	\end{figure}
	Finally, it only remains to calculate the correlation coefficient between the columns E and E of the first screenshot:
	\begin{figure}[H]
		\centering
		\includegraphics{img/arithmetics/rj_correlation_coefficient.jpg}
		\caption[]{Final calculation of the RJ correlation coefficient with Microsoft Excel 14.0.6123}
	\end{figure}
	That gives:
	\begin{figure}[H]
		\centering
		\includegraphics{img/arithmetics/rj_correlation_coefficient_value.jpg}
	\end{figure}
	where the square of this value is very very close to the Shapiro-Wilk test.\\
	
	Then, to know if we reject or not the null hypothesis $H_o$ of Normality assumption, we should repeat the same procedure with in place of the measurements use random values generated from a Normal distribution and then determine the critical value of acceptance/rejection (it is very simple to make but we can detail on readers request).\\
	
	After calculations in this special case we sadly do not reject the null hypothesis.
	\end{tcolorbox}
	
	\pagebreak
	\paragraph{Anderson-Darling GoF test}\mbox{}\\\\
	It is surprising that a test reasonably strong (robust) as the Kolmogorov-Smirnov test can be designed based on only a single observation and a single point of the function of candidate distribution!!! It would seem, with more hindsight, more efficient to measure the difference between the two distribution functions by comparing these functions on their entire domain, that is to say from $-\infty$ to $+\infty$.
	
	It exist a family of tests which the statistics are based on the integral of the square of the difference (these tests are often considered as nonparametric but in my opinion wrongly and same for the Kolmogorov-Smirnov GoF test that is also considered as nonparametric):
	
	between the empirical distribution function and the reference distribution function. The simplest of these statistics is:
	
	which is simply the area between the empirical distribution function and the reference distribution function. Either, by taking the above graph used during our study of the Kolmogorov-Smirnov GoF:
	\begin{figure}[H]
		\centering
		\includegraphics{img/arithmetics/anderson_darling_concept.jpg}
		\caption{Idea behing the Anderson-Darling GoF}
	\end{figure}
	However, arbitrarily, we can choose something other than the measurement $x$ for the integral. Thus, a conventional choice is to take the theoretical distribution function itself as a basic measure of the integral. It comes as follows:
	
	Statistic resulting therefrom is named "\NewTerm{Cramer-von Mises statistic}\index{Cram-von Mises statistic}". However it suffers from a major flaw of robustness when measuring points are on the tails of the distribution.
	
	It then was proposed the following measure that is somewhat less sensitive to measurement points on the tails:
	
	named "\NewTerm{Anderson-Darling statistic}\index{Anderson-Darling statistic}" which was the most used in the late 20th century and remains dominant in the early 21st century also (at least while the sample is a fair size!). It is more robust by construction that the Cramer-von Mises statistics , and Kolmogorov-Smirnov simulations, but numerical studies have shown that it is less robust than the Shapiro-Wilk test or Ryan-Joiner test.
	
	Remembering that the definition of the empirical distribution $\hat{F}_n(x)$ in our study of the Kolmogorov-Smirnov GoF test implies that:
	
	if $x\in[-\infty,x_1]$ and:
	
	if $x\in[x_k,x_{k+1}]$ and:
	
	if $x\in[x_n,x_{+\infty}]$. We then have assuming in addition that $F$ is continuous:
	
	Then we make the change of variable $u=F(x)$ and therefore:
	
	and without forgetting that the integral bound change as:
	
	Therefore we get:
	
	where we of course have put:
	
	We must now calculate these integrals. So we look for the primitive of a function of the type:
	
	The primitive of the following terms:
	
	have been proved in their general form in the section of Differential and Integral Calculus and primitives are respectively:
	
	because in view of the possible values of $u$, then it is unnecessary to indicate the absolute values.
	
	So we have only to calculate the primitive:
	
	where an obvious change of variables (if you want details feel free to ask us) gives us the primitive without the constant:
	
	Then we have finally:
	
	Therefore we have:
	\thickmuskip=0mu
	\medmuskip=0mu
		
	\thickmuskip=3mu
	\medmuskip=3mu
	We can already notice that in the last equality:
	
	Therefore it remains:
	
	We now make some tricky algebraic manipulations (but simple) to condense the writing of the latter equality.
	
	First, notice that we can write the first sum as (the reader can check the two sums  are equal by developing for a small value of $n$ to make an example):
	
	which is equivalent to put $j=j-1$.

	We transform also the second sum:
	
	and the reader can check that the equality below for the third sum is verified (if you problems for this don't hesitate to contact us):
	
	which also is equivalent to put $j=j-1$.
	
	Finally, we transform the fourth sum (since in any case when $j$ is equal to $n$term of the sum is zero ...):
	
	Therefore we have:
	
	Either by eliminating the terms that vanish:
	
	And by grouping the terms with the same form of logarithm:
	
	Therefore:
	
	This is one of the closed form of the Anderson-Darling GoF test and that in the context of a Normal distribution is expressed by tradition in the following form:
	
	But there is another very common simplified expression. To establish it, we start agin from the prior-previous expression:
	
	By the doing the change of variable $j=1+n-i$ in the last sum of the expression:
	
	becomes:
	
	and the bounds of the sum becomes:
	
	Therefore:
	
	Therefore:
	
	Finally:
	
	\begin{tcolorbox}[colframe=black,colback=white,sharp corners]
	\textbf{{\Large \ding{45}}Example:}\\\\
	Suppose we have measured the following five values:
	\begin{gather*}
		-1.2,0.2,-0.6,0.8,-1.0
	\end{gather*}
	thus ordered:
	\begin{gather*}
		x_1=-1.2,x_2=-1.0,x_3=-0.6,x_4=0.2,x_5=0.8
	\end{gather*}
	We want to test the following null hypothesis:
	
	where $\Phi(x)$ if for recall the distribution function of the Normal centered reduced distribution $\mathbb{N}(0,1)$. \\

	To implement the calculation of the AD in a spreadsheet software like Microsoft Excel 14.0.6123 we first do:
	\begin{figure}[H]
		\centering
		\includegraphics{img/arithmetics/ad_gof_excel_initial_values.jpg}
		\caption[]{Initial values for AD GoF test}
	\end{figure}
	Explicitly:
	\begin{figure}[H]
		\centering
		\includegraphics[scale=0.5]{img/arithmetics/ad_gof_excel_initial_values_formulas.jpg}
		\caption[]{Formulas in previous figure values for AD GoF test}
	\end{figure}
	So we get the same value of the indicator AD that as statistical softwares that allows to choose the law to be compared (and therefore the parameters relating). However, for very small samples statistics software uses the following correction (which we were not able to re-obtain by simulation...):
	
	thus in our case AD* is about $0.789$ as $\text{AD}=0.636$.
	\end{tcolorbox}
	
	\pagebreak
	\begin{tcolorbox}[colframe=black,colback=white,sharp corners]
	Then to calculate the $p$-value we need to investigate a curiosity... Indeed if we determine it by a Monte Carlo simulation as we did during our practical application of the Kolmogorov-Smirnov test by first writing in the column A from an new sheet the following formulas:
	\begin{figure}[H]
		\centering
		\includegraphics{img/arithmetics/ad_gof_excel_initial_pvalue_sorting.jpg}
		\caption[]{Formulas to sort Monte Carlo simulated values for AD GoF test}
	\end{figure}
	values coming from the column O where we have:
	\begin{figure}[H]
		\centering
		\includegraphics{img/arithmetics/ad_gof_excel_mc_simulated_values.jpg}
		\caption[]{Monte Carlo simulated formulas values for AD GoF test}
	\end{figure}
	The reader will notice that this is same as comparing the sample with a uniform distribution!!!\\
	
	Then having prepare the following columns H, I (see the figure below) that will contain the simulated values reported by the VBA code given a little further and columns L, M that allow us after to have the distribution of AD and AD\* values thanks to the calculation of the percentile:
	\end{tcolorbox}
	
	\pagebreak
	\begin{tcolorbox}[colframe=black,colback=white,sharp corners]
	\begin{figure}[H]
		\centering
		\includegraphics[scale=0.8]{img/arithmetics/ad_gof_excel_reported_ad_values_formulas.jpg}
		\caption{Columns for writing the AD/AD* values computed by VBA and the percentiles calculations}
	\end{figure}
	with the little VBA code below quickly and dirty made:
	\begin{lstlisting}[language={[Visual]Basic}, caption={VBA AD GoF test VBA code}]
	Sub SimulAndersonDarling()

		Const intSimulations As Integer = 1000
		Dim i As Integer

		For i=1 To intSimulations
			Calculate
			Cells(i+1, 8).Value = Cells(9, 2).Value
			Cells(i+1, 9).Value = Cells(10, 2).Value
			Cells(8, 17).Value = i
		Next i
	End Sub
	\end{lstlisting}
	we then have with 10,000 simulations the following distribution of AD and AD* values:
	\end{tcolorbox}
	
	\pagebreak
	\begin{tcolorbox}[colframe=black,colback=white,sharp corners]
	\begin{figure}[H]
		\centering
		\includegraphics[scale=0.9]{img/arithmetics/ad_gof_percentiles.jpg}
		\caption{AD GoF test percentiles}
	\end{figure}
	So whether it is for AD or AD* the $p$-value is in our particular case almost equal to $1-40\%=60\%$ which corresponds to the tabulated values by Peter A.W. Lewis at IBM (1961).\\
	
	What is curious and that we need to investigate is that the majority of software use the following formulas (RB D'Augostino and MA Stephens, Eds., 1986, Goodness-of-Fit Technology, Marcel Dekker) to avoid Monte Carlo simulations:
	
	and in our case, the application of these formulas give a p-value of about $9\%$ !!! Value actually given by statistical softwares! We will try to find where this difference comes from as our calculation seem subjectively more accurate than this empirical formulas... We asked the support of an American statistical package editor to explain the reason for the difference between the values tabulated by Peter A.W. Lewis and those of R.B D'Augostino and M.A. Stephens but they were not able to answer. We also contacted M.A. Stephens himself so to he communicate us how he had obtained these formulas but we never get any answers ...
	\end{tcolorbox}
	
	\subsubsection{Likelihood-ratio tests}
	Many statistics software returns, in complementary of classical statistical results, an output that is sometimes named "\NewTerm{maximum likelihood test}\index{maximum likelihood test}" or "\NewTerm{maximum likelihood ratio test}\index{maximum likelihood ratio test}" for a test whose result is already given. This maximum likelihood test is also sometimes the unique output as classical methods does not provide a computable or accurate output (this is the case of the G-test and Poisson test of means). As we will see just below, the principle is basically ... extremely simple. 
	Let us recall that we have proved that for a Normal law during our study of likelyhood estimators that:
	 
	For example, if we know the standard deviation but we try to estimate the mean then this last relation can be written:
	 
 	Then nothing avoid us from writing:
	
	Therefore after some very small elementary algebra and arithmetics manipulations:
	
	In the case of following hypothesis (two sided Student $T$-test) with the assumption of homoscedasticity:
	
	We have:
	
	Therefore under the null hypothesis:
	
	That is often denoted:
	
	and is the "\NewTerm{likelihood ratio test for the comparison of the means from a Normal law}".
	
	Therefore we see already, before continuing, that a likelihood ratio test is a statistical test used for comparing the goodness of fit of two models, one of which (the null model) is a special case of the other (the alternative model). The test is based on the likelihood ratio, which expresses how many times more likely the data are under one model than the other. This likelihood ratio, or equivalently its logarithm, can then be used obviously to compute a $p$-value, or compared to a critical value to decide whether to reject the null model in favour of the alternative model. When the logarithm of the likelihood ratio is used, the statistic is known as a "\NewTerm{log-likelihood ratio statistic}\index{log-likelihood ratio statistic}", and the probability distribution of this test statistic, assuming that the null model is true, can be approximated using "\NewTerm{Wilks' theorem}\index{Wilk's theorem}" (approximation by a chi-squared distribution).
	
	\begin{tcolorbox}[title=Remark,colframe=black,arc=10pt]
	In the case of distinguishing between two models, each of which has no unknown parameters, use of the likelihood ratio test can be justified by the Neyman–Pearson lemma, which demonstrates that such a test has the highest power among all competitors. But this is out of the scope of this book (at least actually...).
	\end{tcolorbox}
	In practice we find more often the test of the ratio of maximum likelihood with the Poisson's law. We recall that for the latter we have also proved during our study of likelihood estimators:
	
	Therefore:
 	
	As for the case with the Normal law, we can write:
	
	Therefore:
	
	What it is customary to write:
	
	Suppose we want to test the fact that our measurements do not deviate too much from the true expectation, that is to say:
	
	We have the under the null hypothesis:
	
 	So.... arrrrgh. We cannot make inference on such a result. So it's a very annoying to deal with such a likelihood ratio ... Fortunately we know that in the case where:
	
	the Poisson's law tends to a Normal law (if $\mu$ is greater than about $10$, then the Normal distribution is a good approximation if an appropriate continuity correction is performed). Then in these conditions we use the previous result of the Normal law!
	
	More generally we can (sadly) found the above final relations in the following forms in various textbooks:
	
	So many common test statistics such as the $Z$-test, the $F$-test, the $T$-test, the Pearson's chi-squared test and the $G$-test are tests for nested models and can be phrased as log-likelihood ratios or approximations thereof.
	
	\subsection{Robustness}
	In the area of inferential statistics and hypothesis testing, robustness is a recurring concept (the banks are compelled to stress testing / crash testing of their risk models for example). We have already mentioned this fact previously.
	
	\textbf{Definitions (\#\mydef):}
	\begin{enumerate}
		\item[D1.] A test is named "\NewTerm{robust test}\index{robust test}" if it remains valid while the assumptions for the application are not all met. This can be a somewhat low sample size or a probability distribution (Normal distribution for parametric tests) that is not very well satisfied, or the outliers that influence too much the result of the test. For example, ANOVA is robust with respect to the Normality assumption but not compared to that of homoscedasticity.
		
		\item[D2.] An indicator is named "\NewTerm{robust indicator}\index{robust indicator}" if it is not very sensitive to the presence of outliers (the correlation coefficient or the average, for examples, are not very robust indicators).
		
		\item[D3.] More generally, a model is named "\NewTerm{robust model}\index{robust model}" when it allows an extension of the results (in time or for a population). The robustness is equally applicable to a multiple regression, statistical tests or to a scorecard, or project planning.
	\end{enumerate}
	Therefore, unless to be only descriptive, your studies will have to respect some rules so that their findings are generalisable.
	
	First condition of a good robustness: the data! Intuitively, we all know that we do not transform a special case in generality (which would not fall in the field of statistics but to countertop discussions). Sufficient data builds reliable and solid models. For example, forecasts derived from a time series showing seasonality require at least three or four years of history.
	
	However, the amount is not enough, we need quality (for most manager Quality is a very difficult concept)!! Indeed, it is best to avoid doing a study on unreliable information that can lead to costly decisions. Furthermore, it is recommended to be very careful with the manipulation of outliers. If this is not possible, we turn to more appropriate methods, such as those that using the median rather than the mean, ranks rather than values, or filtering techniques.
	
	\pagebreak
	\subsubsection{Rank Statistics}
	\textbf{Definition (\#\mydef):} The Rank Statistics, also named "\NewTerm{Order Statistics}\index{Order statistics}" are defined as all the techniques of statistical calculations and statistical inference that have for main objective of getting rid of the knowledge of a parametric distribution and for using it ranks only (order) of the measurements. This is a very powerful and practically used tool for non-parametric statistic!
	
	As we have already mentioned, we are talking about parametric tests when we stipulate that the data are from a given known distribution. In this case, the characteristics of the data can be summarized using the parameters estimated on the sample, the subsequent test procedure then only focus on these parameters.
	
	\begin{tcolorbox}[title=Remark,colframe=black,arc=10pt]
	The reader interested to learn more but without detailed is referred to the excellent book of Gopal K. Kanji which contains a summary presentation of the most used 100 parametric and non-parametric test with for each a small practical example. Sadly this book should be published in a new edition with the 200 most used statistical tests to be almost complete and perfect.
	\end{tcolorbox}	
	
	Non-parametric tests (such as the chi-square tests already seen) eliminate the necessary step consisting to estimate the parameters of the distribution prior to doing the hypothesis test. In general, conclusions drawn from non-parametric tests are not as powerful as the parametric ones. However, as non-parametric tests make fewer assumptions, they are more flexible, more robust, and applicable to non-quantitative data.
	
	When the data are quantitative, most common non-parametric tests transforms values into ranks. The name "\NewTerm{ranks test}\index{rank test}" is then often encountered. When the data are qualitative, only the non-parametric can be used.
	
	\paragraph{L-Statistics}\mbox{}\\\\
	Before addressing non-parametric distributions and tests, let give some definitions that the ready may see be in the hyperspecialized literature and we avoided the use (at least until now ...).
	
	The median, mean and range suggest the use of linear combinations of the components of the order statistics vector.
	
	\textbf{Definition (\#\mydef):} Thus, Let us denote by $X_{(1)}\geq X_{(2)},...,\geq X_{(n)},$ the order statistics (values ordered in descending order by their rank and numbered). We define then the "\NewTerm{L-statistic}\index{L-statistic}" by:
	
	
	The first most known "L-estimator" is the arithmetic mean for which:
	
	The second most known L-estimator is the median for which we have when $n$ is odd:
	
	and when $n$ is even:
	
	The third estimator best known is the range for which we have:
	
	There are other empirical $L$-estimators but we will stop here because the list is quite long.
	
	\paragraph{Ranks Distribution Law}\mbox{}\\\\
	The $k$-th order statistic is the $k$th smallest value in sample of $n$ from a random distribution $F(x)$. Let the random variables $X_1,X_2,...,X_n$ be independent and identically distributed. Label the smallest value $X_{(1)}$, the next smallest $X_{(2)}$ and so on up to $X_{(n)}$. Then the value of the $k$th order statistic is $X_(k)$, for $1\leq k\leq n$.
	
	Let $F(x)$ denote the distribution function of the $X_i$'s and $f(x)$ the density function. Let us first determine the distribution of the $k$th order statistic, that is:
	
	To have $X_{(k)}\leq x$ we need to have at least $k$ of the $n$ $X_i$'s less than or equal to $x$. Each of the $n$ independent trials has probability $F(x)$ of being less than or equal to $x$, thus the number of trials less than or equal to $x$ has a Binomial distribution with $n$ trials and probability $F(x)$. That is:
	
	To find the density function we take the derivative with respect to $x$:
	
	Written in this form, it is obvious that the majority of terms will cancel. This leaves:
	
	
	\paragraph{Wilcoxon Rank Sum Test}\mbox{}\\\\
	The idea of the "\NewTerm{Wilcoxon rank sum test}\index{Wilcoxon rank sum test}" is the following: if we collect two samples of measurement, and that we store the values in order, the alternance of the $X_i$ (of size $n_x$) and $Y_i$ (of size $n_y$) should be fairly steady if both samples distribution law $F$ and $G$ respectively follow the same probability distribution. It is therefore like a "fit check".
	
	It is therefore not such as Chi-square adequation test to compare measurements at a theoretical law, but to other measurement of a hypothesized same law.
	
	\begin{tcolorbox}[title=Remark,colframe=black,arc=10pt]
	The Wilcoxon rank sum text is a nonparametric test because we have no need of use of any measure of dispersion or position of the random variables. Moreover, it is a test say to be "robust" in the sense that it does not assume the Normality of the data.
	\end{tcolorbox}	
	Let us take an example before tackling the theoretical aspect. Here are two samples of size $10$ ($n_x=n_y$) of quantitative variables:
	
	\begin{tcolorbox}[title=Remark,colframe=black,arc=10pt]
	The Wilcoxon rank sum test may well be used for ordinal variables (but only as long as they are in an acceptable number). Typically, the Wilcoxon rank sum test is also used to analyze the response to business surveys using a Likert scales of $7$ points.
	\end{tcolorbox}	
	Here are the order statistics of the sample of size $20$ ($N=n_x+n_y$) regrouped and ordered (the $10$ values of the first sample are underlined):
	\begin{gather*}
		1.6,\underline{1.7},\underline{2.5},\underline{3.2},\underline{3.2},3.4,\underline{4.1},4.6,\underline{5.3},5.5,\underline{5.7},5.7,\underline{6.9},7.1,\underline{7.4},7.9,8.1,\underline{8.4},8.5,8.7
	\end{gather*}
	The values of the first sample $X$ (named "\NewTerm{treatment values}\index{treatment values}") in this example looks like to be smaller than those of the second  sample $Y$ (named "\NewTerm{control values}\index{control values}") that we often represent in a diagram as the following (by cheating a bit with Microsoft Excel 11.8346):
	\begin{figure}[H]
		\centering
		\includegraphics{img/arithmetics/wilcoxon_rank_sum_test_ordered_values.jpg}
		\caption[]{Comparing ordered values of two samples in Microsoft Excel 11.8346}
	\end{figure}
	The idea is then to find out if this trend is statistically significant? That is to say, whether we have a difference of the kind $F<G$ between their respective distributions laws:
	\begin{figure}[H]
		\centering
		\includegraphics{img/arithmetics/wilcoxon_rank_sum_distribution_comparison.jpg}
		\caption[]{Generic example of the comparison of two distributions in Microsoft Excel 11.8346}
	\end{figure}
	or whether they can be considered as identical. For this, we must examine the concept of "\NewTerm{rank}\index{rank}":
	
	\textbf{Definition (\#\mydef):} Given a random $n$-sample $X_1,X_2,...,X_n$ of any continuous statistical law, we denote by $R_i$ the rank of the $X_i$ ordered in a sample population. The rank $i$ is a non-zero integer strictly positive and between $1$ and $n$.
	
	\begin{tcolorbox}[colframe=black,colback=white,sharp corners]
	\textbf{{\Large \ding{45}}Example:}\\\\
	In:
	\begin{gather*}
		1.6,\underline{1.7},\underline{2.5},\underline{3.2},\underline{3.2},3.4,\underline{4.1},4.6,\underline{5.3},5.5,\underline{5.7},5.7,\underline{6.9},7.1,\underline{7.4},7.9,8.1,\underline{8.4},8.5,8.7
	\end{gather*}
	We have respectively the following "order statistics" :
	\begin{gather*}
		X:\; R_2=1.7,R_3=2.5,R_4=3.2,R_5=3.2,R_7=4.1,...\\
		Y:\; R_1=1.6,R_6=3.4,R_8=4.6,R_{10}=5.5,R_{11}=5.7,...
	\end{gather*}
	Once the concept of "rank" defined and calculated, let us look at the sum in the context of our example with two samples:
	\begin{gather*}
		1.6,\underline{1.7},\underline{2.5},\underline{3.2},\underline{3.2},3.4,\underline{4.1},4.6,\underline{5.3},5.5,\underline{5.7},5.7,\underline{6.9},7.1,\underline{7.4},7.9,8.1,\underline{8.4},8.5,8.7
	\end{gather*}
	The sum rank denoted traditionally $W_x$ ($W$ for Wilcoxon) for the first sample is thus:
	
	and for the second sample:
	
	\end{tcolorbox}
	The values $W_x,W_y$ are named "\NewTerm{Wilcoxon Statistics}\index{Wilcoxon statistics}".
	
	We can therefore already see that there is indeed a difference that seems a priori significant in terms of ranking between both samples. The problem remains to build a rigorous mathematical tool to infer a fact with some certainty.
	
	For this, let us first introduce the average of the ranks using the result shown in the section of Sequences and Series by considering only one sample:
	
	By calculating this, we notice quickly that this is the expected mean of the uniform discrete distribution study earlier in this section for a discrete random variable with values between $1$ and $n$, which is exactly the definition of rank! Thus, we have the rank that will have for mean and variance for the entire population:
	
	For those who find this analogy questionable we give just right below the proof of the variance using the Huygens relation and the sum of squares of positive integers proved in the section of Sequences And Series:
	
	But obviously for a single sample this has no interest! Let us take again our two series $X_i,Y_i$ respectively of equal size $n_x=10,n_y=10$ without distinction:
	\begin{gather*}
		1.6,1.7,2.5,3.2,3.2,3.4,4.1,4.6,5.3,5.5,5.7,5.7,6.9,7.1,7.4,7.9,8.1,8.4,8.5,8.7
	\end{gather*}
	We then have the statistical indicators of ranks without distinction (you must remember that we still do not know at this level of the mathematical development if this will be helpful or not):
	
	and the statistical indicators of ranks but this time with distinction:
	\begin{gather*}
		1.6,\underline{1.7},\underline{2.5},\underline{3.2},\underline{3.2},3.4,\underline{4.1},4.6,\underline{5.3},5.5,\underline{5.7},5.7,\underline{6.9},7.1,\underline{7.4},7.9,8.1,\underline{8.4},8.5,8.7
	\end{gather*}
	We then have the following local statistical indicators:
	
	These calculations now done, we have nothing concretely yet rigorous regarding the Wilcoxon's rank sum test which purpose is reminder to check whether the two samples follow the same law or not (and therefore have the same moments as the expected mean, variance, median, etc.)
	
	To move forward, let us consider the $n_x$ values of the sample $X$. We know (\SeeChapter{see section Probabilities}) then that there is:
	
	number of possible arrangements of the $X_i$ in the population of the samples and if the Wilcoxon's rank sum test is verified (that is to say, the probability laws are the same for both samples), the various arrangements should be equally likely.
	
	For example, if we take $2$ samples with respectively each $2$ measures ($2$ random variables of treatment and $2$ control random variables), we have:
	
	arrangements all different:
	
	But... sadly... this is not what we want in our case because we would like already to be able to distinguish the two samples and also do not take into account the arrangements that consist only of a permutation of the variables of the same sample. We then have (\SeeChapter{see section Probability}):
	
	possible combinations! Effectively with two samples having two treatment variables ($X$) and two control variables ($Y$), we have ($W_S$ is the sum of ranks of last column):
	
	If the null hypothesis of the Wilcoxon's sum test rank is not rejected, the $6$ rankings are equally likely. We conclude the following table:
	
	If the hypothesis of the Wilcoxon's rank sum test is right, the $6$ rankings should be equally likely. We conclude the following table:
	
	This table being constructed, suppose that we observe for the sum of rank of treatment variables: $W_S=7$. The threshold of a one-sided test would then give in conformity with the table above:
	
	or if we get $W_S=3$:
	
	So we would reject the null hypothesis of an identical distribution between the two samples at any upper limit (or lower, respectively) fixed in advance by laboratory policy ... in unilateral or bilateral test (reason why some statistical softwares give unilateral test values + bilateral test values at the same time).
	
	Two very important things you need to notice for what will follow is that:
	\begin{enumerate}
		\item First in the construction of the above table (where we take again the first part here):
		
		we see there is a symmetry at the value $5$, which means that the law $W_S$ is symmetrical in this particular case. But if we take another example with two samples respectively including two control variables and three of treatments (two random variables) we would have:
		
		the reader can check that whatever the number of samples and the number of variables and control treatment, the probability table above is always balanced (well there is a mathematical proof of this but I find it inelegant). But in fact it is pretty intuitive, like combinations $C_{n_x}^{n_x+n_y}$ are independent of the fact that the ranks are sorted in ascending or descending order, it is therefore necessary that there is a symmetry.
		
		\item Secondly the values of the measured variables do not come into account in this parametric statistics but only the tabulated values of the ranks with their associated probabilities. Indeed, as you may have noticed, we did not need explicit values of the random variables to build the table above!
	\end{enumerate}
	
	Now, knowing that the law $W_S$ is symmetrical and discreet we would like to calculate its expected mean.
	
	The smallest possible value of $W_S$ is assuming it is in the sample $X$ (computer algorithms automatically determine in which sample but anyway in practice, samples almost always have the same size):
	
	The largest possible value is naturally (remember that $N=n_x+n_y$):
	
	The expected mean of the sum of the ranks of one of the two samples is then:
	
	Then finally:
	
	To calculate the variance, which will be useful to us to make if necessary an approximation that we will see later below, appears (unfortunately) the covariance because knowledge of the ranks give partial information about other ranks. So we have:
	
	We already know with what we have just proved above that:
	
	The problem remains the term with covariance. For its calculation there are rigorous techniques into several pages and... a tip that is much shorter. The trick is to use the global ranking variable which we denote $T_i$ with $i=1...n+m$. As the sum of the $T_i$ is a constant, then we have:
	
	It comes then:
	
	We can then take back the initial calculation by replacing the covariances by their expression, the last relation obtained for the covariances calculated on the $T_i$  applying also (which is not necessarily intuitive ... but the trick works) to the $R_i$:
	
	Finally:
	
	This is the same result as the rigorous methodology that can be found in some rare references.
	\begin{tcolorbox}[colframe=black,colback=white,sharp corners]
	\textbf{{\Large \ding{45}}Example:}\\
	Let us turn to a practical case for the exact case. So consider $2$ samples having $2$ treatments variables $(X)$ and two control variables $(Y)$ (it is a bit simplistic and absurd as an example but it facilitates the educational aspect ...) we have:
	\begin{gather*}
		X:5.7,3.2\\
		Y:8.1,5.5
	\end{gather*}
	Thus (the treatment variable therefore has the ranks $1$ and $3$ which makes a sum of rank of $4$):
	\begin{gather*}
		\underline{3.2},5.5,\underline{5.7},8.1
	\end{gather*}
	Therefore:
	\begin{gather*}
		X:R_1=3.2,R_3=5.7\\
		Y:R_2=5.5,R_4=8.1
	\end{gather*}
	We have the following table as we have shown above:
	
	with in this case:
	
	If we choose the traditional bilateral confidence threshold level to $5\%$ we have according to the table above that:
	
	So in other words we see that there is:
	
	\end{tcolorbox}
	
	\pagebreak
	\begin{tcolorbox}[colframe=black,colback=white,sharp corners]
	of cumulative probability that $W_S$ is between $3$ and $7$ (the bar above the $6$ means for reminder that this digit is repeated to infinity). So $4$ is necessarily included in the bilateral range of $95\%$ ... and we do not the reject the null hypothesis as what the two samples are not different. The corresponding $p$-value in bilateral is then the half of $33.\bar{3}\%$.\\
	
	\begin{tcolorbox}[title=Remark,colframe=black,arc=10pt]
	In fact if we wanted to make an interesting computational example by playing with a bilateral threshold of $5\%$ (or $2.5\%$ on either side) we should have at least $2$ samples with $4$ random variables, that is to say $70$ possible combinations of  ranks. Below $4$ random variables per sample, it is clear that the two-tailed test at a level of $95\%$  will be such that we will almost never reject the null hypothesis of equality...
	\end{tcolorbox}
	\end{tcolorbox}
	If the size of the two samples is large enough (most practitioners consider that each sample must be at least $20$ individuals), it has been shown by simulations that we can make the following approximation (used by many statistical software):
	
	obviously always determining the $p$-value in bilateral. With the previous example (with only $4$ measurement in total), we have therefore:
	
	Which corresponds to a cumulative probability of $21.93\%$. So the corresponding bilateral $p$-value bilaterally is about $(1-22)\cong 44\%$ (compared to the value of about $33\%$ with the exact case).
	
	\paragraph{Mann-Witheny Rank Sum Test}\mbox{}\\\\
	The "\NewTerm{Mann-Whitney rank sum test}\index{Mann-Withney rank sum test}" is also a non-parametric adequation test, very simple, which can be deduced from the Wilcoxon rank sum test. Furthermore it is inspired to the point that we sometimes name in the industry "\NewTerm{Wilcoxon-Mann-Whitney test}\index{Wilcoxon-Mann-Whitney test}" or "\NewTerm{Wilcoxon-Mann-Whitney adequation test}\index{Wilocoxn-Mann-Withney adequation test}" or "\NewTerm{MWW test}" (without specifying each time in the name that it is based on the sum of ranks).
	
	The purpose of this test, as for Wilcoxon rank sum test, is to find a way to verify that two independent samples not necessarily of the same size are from the same law or not (verbatim come from a same population or not) but with a different approach!
	
	\begin{tcolorbox}[title=Remark,colframe=black,arc=10pt]
	Just as the  Wilcoxon rank sum test, the Mann-Whitney rank sum test may well be used for ordinal variables (categorical but so long as they are in an acceptable number).
	\end{tcolorbox}
	Some software also generates confusion because they propose the Wilcoxon rank sum test under the name of Mann-Whitney test... and vice-versa ... and most do not indicate or do not offer always the choice between the exact or approximate version... And furthermore the Wilcoxon rank sum test and that of the Signed rank test that we will see further below are not differentiated .... so be careful! This is typically a problem whose source is the absence of an ISO standard to define the terminology and options that should be available in scientific softwares...
	
	To see what is this test, let us build the rank table using two samples including two control variables and three measurement variables, then we have:
	
	from which we deduce the following table:
	
	Now imagine that we have another experience to be analyzed using two samples having $3$ control variables and $2$ treatment variables (therefore the symmetric or the previous case!!!), then we have:
	
	from which we deduce the following table (the reader will notice that this is exactly the same table as the previous regarding the probabilities!!):
	
	Well the idea of the Mann-Whitney rank sum test is quite simple. Indeed, rather than tabulate symmetric situations, it is enough to simply subtract at each value $W_S$, the value $W_{S,\min}$ so that each table is identical and only one of the two is helpful. Let us see this with the first table:
	
	We deduce the following table:
	
	Now imaging that we have another experiment to analyze using two sample having $3$ control variable and $2$ treatments variables, we have therefore using the same idea:
	
	From which we deduce this time  again exactly the same table as before:
	
	this is why the literature mention that we can take whatever we want!
	
	So to summarize, the Mann-Whitney variant (in the specific case here it is the variant named "\NewTerm{exact Mann-Whitney variant}\index{exact Mann-Whitney test}") consists to tabulate for symmetrical situations a variable denoted by $W_{XY}$ naturally defined by:
	
	Denoted also often in the literature by:
	
	since $W_{XY}\in \left\lbrace 0,1,2,...\right\rbrace$ and therefore:
	
	In the tables that we can found in books, the probabilities are given with the normalized value of $U$. Thus, if we take our previous example, but with the usual notations in practice ($U$ instead of $W_{YX}$):
	
	We see that the cumulative probability that $U=2$ is of $0.4$. The above table is sometimes given in literature as follows:
	
	where we put in bold the value corresponding to our example ($U=2,n_1=2,n_2=3$). Then the practitioner has to choose what he wants to do with these tables if he wishes to make a bilateral or unilateral test.
	
	\begin{tcolorbox}[title=Remarks,colframe=black,arc=10pt]
	\textbf{R1.} It is important to remember that we have showed with an example that we can also take:
	
	than:
	
	as they generate the same tables!\\
	
	\textbf{R2.} $W_{XY}$ is traditionally denoted $U$ by practitioners as we already have seen it, this is why in some books this test is given under the name "\NewTerm{Mann-Whitney $U$ test}\index{Mann-Whithney $U$ test}" with the associated tables under the same name. But take care not to make confusion with the "\NewTerm{Wilcoxon $U$ test}\index{Wilcoxon $U$ test}" sometimes named "\NewTerm{Wilcoxon inversion text}\index{Wilcoxon inversion test}" which is base on the alternation of sample values when grouped (this test will not be developed in this book).
	\end{tcolorbox}
	To see the approximate version (asymptotic) of the Mann-Withney $U$ test we need the expression of the expected mean and variance. For this, remember that we have seen that the sum of normalized ranks was given by:
	
	But we can also use as we saw:
	
	and as:
	
	with for reminder:
	 
	We therefore have:
	
	The average of the both $U$ is therefore the arithmetic average of the sum. We have therefore:
	
	This means that $U_1$ or $U_2$ must be different enough of the latter average so that we reject the null hypothesis $H_0$ that suppose (for recall) that both samples are issued from the same distribution law. But to determine the $p$-value, we also need the standard deviation. So let us search it!
	
	The standard deviation is the same a for the Wilcoxone rank sum test (as the second term in the expression of $U$ is an constant and therefore is variance is equal to $0$). Therefore, it only remains the variance of the sum of ranks that we have already proved earlier as having for value:
	
	\begin{tcolorbox}[colframe=black,colback=white,sharp corners]
	\textbf{{\Large \ding{45}}Example:}\\
	Let us take the same example as done with the Wilcoxon rank sum test but slightly modified (for the example to be more easy to understand) that is to say:
	\begin{gather*}
		X:\; 5.7,3.2\\
		Y:\; 8.1,5.5,1.2
	\end{gather*}
	That is to say grouped and sorted:
	\begin{gather*}
		\underline{1.2},3.2,\underline{5.5},5.7,\underline{8.1}
	\end{gather*}
	We then have:
	\end{tcolorbox}
	
	\pagebreak
	\begin{tcolorbox}[colframe=black,colback=white,sharp corners]
	
	So we can choose anyone for the test since both $U$ are equal. If we look at the table created above, with $(U=3,n_1=2,n_2=3)$, we have therefore a cumulative probability of $60\%$ that $U$ is equal to $3$. So we do not reject the null hypothesis $H_0$ (in unilateral) that the two samples are from the same distribution law.\\
	
	The approximation by a Normal distribution gives then:
	
	Then the cumulative probability is $50\%$ with the Normal approximation which corresponds to a $p$-value of $50\%$ in unilateral. Again we do not also not reject the null hypothesis $H_0$ here.
	\end{tcolorbox}
	
	\paragraph{Treatment of equalities}\mbox{}\\\\
	When we do a Wilcoxon-Mann-Withney rank sum test Mann-Whitney or other, equality of ranks may occur.
	
	As theoretical introduction let us take an example:
	
	with the following data:
	
	A conventional solution (among others ...) is to assign to each "?" the average rank. So in this case we have:
	
	The table:
	
	becomes in this case:
	
	where $W_S^{*}$ (notice the small upper right $^{*}$!) is the Wilcoxon statistics when we are in the presence of statistical equalities. The law $W_S^{*}$ can be more or less different from $W_S$. Indeed:
	
	
	\pagebreak
	\paragraph{One sample Wilcoxon rank sum signed test}\mbox{}\\\\
	The purpose  of the "\NewTerm{Wilcoxon signed rank sum test}\index{Wilcoxon signed rank sum test}" also sometimes named "\NewTerm{Wilcoxon median test}\index{Wilcoxon median test}", is to use a non-parametric technique for checking the symmetry or not of a distribution, and therefore make verbatim a hypothesis on the value of the median. The idea is both simple and subtle.
	
	The principle is that if we compare the differences denoted $D_i$ between individuals of a sample relatively to the median, we know that if we have (for example) an odd number of individuals all different  (not equal), then we have $50\%$ of the data above and below the median. Then, to control that distribution of the values satisfies a certain symmetry, the idea (simple but clever) then consists in:
	\begin{enumerate}
		\item Calculate the differences $|D_i|$ in absolute values relatively to the median.
		
		\item Order these absolute differences by ascending order and assign them a respective rank.
		
		\item Calculate the sum of the ranks of the differences $D_i$ that are negative.
		
		\item Calculate the sum of the ranks of the differences $D_i$ that are positive.
	\end{enumerate}
	and if the sample has a symmetric unimodal distribution (so the median is then confused with the average), there should be a sum of negative ranks $S_{-}$ that is not statistically significantly different from the sum of positive ranks $S^{+}$.
	
	We notice therefore that this hypothesis test for it to work is that the statistical distribution is symmetrical and unimodal!!
	
	\begin{tcolorbox}[title=Remark,colframe=black,arc=10pt]
	For recall, during our study of Wilcoxon or Mann-Withney tests independent samples  seen above (samples which do not necessarily have the same size), we ordered all the values of the two samples and we make a calculation on the ranks of these values. In tests for paired samples (i.e. of the same size), we ordered the differences values (not the original values!) and works on the ranks of the \underline{differences}!!!
	\end{tcolorbox}
	According to the idea (principle) described above, the sum of the rank bearing the sign $-$ has then for average:
	
	But we have already shown that the expected mean of the binomial distribution is:
	
	And as in our case $N$ is equal to $1$ (only one value ...) and $p$ is equal to $1/2$ (one in two chance of having a negative sign), it comes immediately using the proofs of the section Sequences and Series:
	
	and for the variance using also the results of the section Sequences And Series:
	
	and again using the variance of the binomial distribution and the results of section Sequences And Series:
	
	Obviously the sum of ranks of negative differences (respectively positive) will be at the minimum equal to zero and at maximum $n(n+1)/2$. Therefore, the expected mean in the case of a bilateral test should not be too close to one of these two extreme values.
	
	In the case where $n$ is large enough (more than thirty), we can use the approximation of the reduced centered Normal distribution for the variable:
	
	where $S_{-}$ is the sum of ranks of negative differences.
	
	Finally let us notice that empirically if some differences from the median are zero, they will not be included in the ranks. If differences are equal we will take an average rank...
	\begin{tcolorbox}[colframe=black,colback=white,sharp corners]
	\textbf{{\Large \ding{45}}Example:}\\
	Let us start with the case with one sample compared to its experimental median (at the opposite with the comparison to a hypothesized a priori median when we consider the distribution symmetric and unimodal). Consider that we measured the following values for the diameter of a piece:
	\begin{gather*}
		39, 20.2, 40, 32.2, 30.5, 26.5, 42.1, 45.6, 42.1, 45.6, 42.1, 29.9, 40.9
	\end{gather*}
	We wish to know if the calculated experimental median (which value is equal to $40$ in this case) of this sample can not be rejected as a main indicator to threshold level of $5\%$ in bilateral (which will be the case if the number of positive and negative differences is fairly balanced). We then construct the following table:
	\end{tcolorbox}
	
	\pagebreak
	\begin{tcolorbox}[colframe=black,colback=white,sharp corners]
	
	At first glance the equality of ranks does not look great but nevermind... we will going a little bit further...\\
	\begin{tcolorbox}[title=Remark,colframe=black,arc=10pt]
	Following the books the sum of ranks does not give the same value as there are several techniques to calculate the ranks of values that are the equals... However, we have chosen the one used by Minitab software that is customary in the scientific community and which corresponds to that which we have already presented the rules earlier.
	\end{tcolorbox}
	If we consider that the number of individuals is sufficient ... we use the approximation (even if in this case the conditions are not satisfied):
	
	Thus with this example:
	
	and respectively:
	\end{tcolorbox}
	
	\pagebreak
	\begin{tcolorbox}[colframe=black,colback=white,sharp corners]
	
	The first case corresponds with the approximation using a Normal distribution to a cumulative probability of $29.13\%$ obtained with the English versions of Microsoft Excel 14.0.6123 using the function:
	
	\begin{center}
	\texttt{=NORM.S.DIST(-0.549,TRUE)}
	\end{center}
	
	and thus obviously a bilateral $p$-value of about $2\cdot 29.13\%\cong 58.26\%$ bilaterally.\\
	
	The second case corresponds with the approximation using a Normal distribution to a cumulative probability of $84.62\%$ obtained with with the English versions of Microsoft Excel 14.0.6123 using the function:
	\begin{center}
	\texttt{=NORM.S.DIST(1.02,TRUE)}
	\end{center}
	which corresponds to a bilateral $p$-value of about $2\cdot(1-0.8462)/2=30.76\%$. Minitab gives à bilateral $p$-value of $32\%$ as it does not use the Normal approximation.\\
	
	Therefore for the both calculations we can prudently (and sadly) do not reject the null hypothesis $H_0$ as $40$ (the median) is in the middle of the confidence interval (also the sign test leads to the same conclusion).
	\end{tcolorbox}
	\begin{tcolorbox}[title=Remark,colframe=black,arc=10pt]
	As software like Minitab even if offering the Wilcoxon median test for 1 sample gives for the median a value of $36.5$ and gives for the median confidence interval the values $31.1$ and $42.1$. If we apply the bootstrapping method presented in detail in section of Theoretical Computing we get for estimated median $40$ (and an average of $38.733$) and as interval $30.50$ and $42.10$... Well in any case we anyway not to reject the null hypothesis $H_0$ but it is still boring not to know how these values are calculated in Minitab...
	\end{tcolorbox}
	
	\pagebreak
	\paragraph{Wilcoxon rank sum signed test for two paired samples}\mbox{}\\\\
	The "\NewTerm{Wilcoxon sum signed rank test for two paired samples}\index{Wilcoxon sum signed rank test for two paired samples}" is $100\%$ based on the principle of the test sample with one sample. The only difference is that the null or alternative hypothesis are based on the difference in the median of the data taken in pairs (two by two) of each of the samples. In most cases, the null hypothesis $H_0$ is that the median of the differences is zero against the alternative hypothesis $H_A$ that it is statistically significantly different from zero.
	
	Like the T-test for correlated samples, the Wilcoxon rank sum test applies to two-sample designs involving repeated measures, matched pairs, or "before" and "after" measures!!
	
	As the mathematical developments are the same as for one sample test let us attack it directly by an example.
	
	First let us just emphasize that by extension, that an hypothesis for test to work is that the statistical distribution of differences is therefore symmetrical and unimodal!!
	
	\begin{tcolorbox}[colframe=black,colback=white,sharp corners]
	\textbf{{\Large \ding{45}}Example:}\\
	We have two different softwares ($S1, S2$) to compare and we want to submit to $12$ tasks ($T1, T2, T3, ..., T12$) of specific but similar calculations for each of the software. We would like to know if the software have a processing time that is statistically significantly different or not and if so which one is better.
	
	We already see that the software $L1$ is generally faster than $L2$ without using any tables of the Wilcoxon's exact sign test, we can therefore already say that qualitatively the difference is statistically significant.\\
	
	If we consider that the number of individuals (sample sizes) is sufficient ... we use the Normal approximation (even in this special example the conditions are not satisfied):
	\end{tcolorbox}
	
	\pagebreak
	\begin{tcolorbox}[colframe=black,colback=white,sharp corners]
	
	That is to say in this example:
	
	and respectively:
	
	The first case corresponds in approximation to a Normal distribution with a cumulative probability of $0.836\%$ obtained with with the English version of Microsoft Excel 14.0.6123 using the function:
	\begin{center}
	\texttt{=NORM.S.DIST(-2.392,TRUE)}
	\end{center}
	and thus a bilateral $p$-value of $2\cdot 0.836\%$.\\
	
	The second case corresponds in approximation to a Normal distribution with a cumulative probability of $92.774\%$ obtained with with the English version of Microsoft Excel 14.0.6123 using the function:
	\begin{center}
	\texttt{=NORM.S.DIST(1.453,TRUE)}
	\end{center}
	and thus a bilateral $p$-value of $2\cdot (1-0.92774)\%\cong 14.452\%$.With a software such as Minitab 15.1.2 that does not offer yet the Wilcoxon test for paired samples but for which there is a trick to get it anyway, we get a bilateral $p$-value of $3.3\%$. Other software still give a $p$-value less than $5\%$ (but the are different from one software to another ...).\\
	
	Therefore, with the calculations by hand and using the Normal approximation we do not (sadly) reject the null hypothesis $H_0$. But with softwares using the exact text, the null hypothesis is rejected!
	\end{tcolorbox}
	
	\pagebreak
	\paragraph{Kruskal-Wallis test}\mbox{}\\\\
	The Kruskal-Wallis test is a nonparametric test often compared (a bit faster ...) to a nonparametric one-way canonical ANOVA with fixed to compare if two or more populations have the same median (null hypothesis $H_0$) except that it does not requires the assumptions necessary for the parametric version of the ANOVA. When several compared populations pass through this test, the KW-test does not say which population is statistically significantly different but only that there is at least one who is. In fact, as we will show, the KW-test is only an extension of the Mann-Whitney U test seen  earlier for a number of populations greater than or equal to three.
	
	To study this test, we will assume that we have only two populations and we will afterwards make an intuitive generalization. This approach is that one that would have used Wilcoxon before that Kruskal and Wallis make the rigorous proof of the generalized case.
	
	To study this test, let us first recall that (relations whose origin and verbatim the proof have already been explained during our study of the Mann-Whitney test seen above) the average of the sum of ranks and the variance of the sum of ranks is given by:
	
	in the case where there is no duplicate values. Under this assumption, remember that $\bar{R}$ can be assimilated to the rank of the median value (in the case of an odd number of values).
	
	Let us recall that the average values of $n$ draws without replacement among $N$ will be close to a Normal distribution when $N$ big enough, and we have already proved at the beginning of this section that:
	
	and if the population is not very large, the variance must be corrected by the correction factor on finite population that we have already prove earlier:
	
	Therefore it comes:
	
	We then have if we are concerned with the ranks (the variance of the ranks being the true variance: there is no estimator!):
	
	Now, to form a reduced centered Normal variable $Z$ we can center and reduce the random variable $\hat{\bar{R}}$ obtained by sampling by writing:
	
	where $\hat{\bar{R}}$ is the average of the sum of the ranks of a sample of the population. And in fact all the trick of the idea behind  the Kruskal-Wallis test is here: the statistical distribution of the average of the sum of the ranks of a large number of samples of $N$ values has approximately a Normal distribution (read again if necessary our study of limits of drawings samples of a population at the beginning of this section)!
	
	Let us take the square:
	
	The approximation by the chi-square law is only valid if $n$ is large enough as we have already discussed in detail during our study of the test chi-square adjustment.
	
	And so the parenthesis of the first equality is equal to the square of the deviation of the rank to the median. This is why we often say that this is a median test (but that is an abusive shortcut).
	
	Before continuing, let us insist on the fact that the scenario in which we find ourselves is that of a random sample of $n$ items from $N$, which is equivalent to end up with two samples (one of size $n$ and the other of size $N-n$) of the same law (verbatim from the same population/distribution). It then comes that (relations that we will use later):
	
	and by extension of the case with one sample, if we denote by $R_i$ the sum of ranks of the sample numbers $i$, we also have:
	
	It follows that if we note for the following equation where R is the sum of the ranks of the sample equation, we have:
	
	If we now write the relation proved above:
	
	as follows (this is a clever development in reverse order... starting from the third line):
	
	and we find ourselves at the end with the fact that we worked from the beginning with two samples, one of size $n$ and therefore the other (verbatim by sampling) of size $N-n$.
	
	The previous result (which was the one desired from the beginning) can be generalized as follows under the named "\NewTerm{Kruskal-Wallis H test}\index{Kruskal-Wallis H test}" to a given confidence level in unilateral (sometimes this relation is written without the parentheses around the sum which can lead to a bad reading):
	
	and if all the $n_i$ are equals, we fall back on a well know notation of the previous relation:
	
	The approximation according to a chi-square law, however, can be discussed when the sample size $(c)$ is small (refer to our study of the Chi-square law earlier).
	
	\pagebreak
	\begin{tcolorbox}[colframe=black,colback=white,sharp corners]
	\textbf{{\Large \ding{45}}Example:}\\
	Let us take the original example of Kruskal-Wallis. We consider that we have three machines that are the same at the origin but two of them have undergone some modifications. We measured the daily production a number of times and got the following table:
	
	We then have well:
	
	and:
	
	Be we have:
	
	That can be obtained easily with Microsoft Excel 14.0.7166 in English with the function:
	\begin{center}
	\texttt{=1-CHISQ.DIST(5.656,1,TRUE)}
	\end{center}
	In the present case, to a threshold level of $5\%$, so we are at the limit with the approximation by a chi-square distribution. As shown by Kruskal and Wallis, a Monte Carlo simulation gives a $p$-value of $0.049$.\\
	
	In short, in this situation it would be better have to reject the null hypothesis as that  the productions are similar. And therefore preferred the alternative hypothesis such that these are rather different. A recommendation is to redo the test by pairs to see what is statistically significantly one by one.
	\end{tcolorbox}
	
	\pagebreak
	\paragraph{Friedman Test}\mbox{}\\\\
	The Friedman test, recommended by the norm NF ISO 8587 for sensory analysis (ranking test), consider an experiment with two factors (the first being considered as the treatment and the second as the blocks of tests as well as the ANOVA with two-fixed factor without repetition) that is analyzed using the ranks, as the measurement values do not satisfy the conditions for the application of ANOVA. However, instead of ANOVA, Friedman test applies to paired data as we shall see now.
	
	Let us associate, as we have already done it several times, the theory to an example starting from the following table where $8$ subjects (blocks) denoted by $B$ under hypnosis were subjected to $4$ emotions (treatments) denoted by $T$. Their epidermal electrical potential was measured (millivolts) in each case (and the order of the treatments was randomized):
	
	The central and subtle idea is not to assign a rank to the entire population of the measures as is the case for the Kruskal-Wallis test (we would then lose the concept of the blocks: verbatim the second factor) but block by block all supposed therefore independent of each other.
	\begin{tcolorbox}[title=Remark,colframe=black,arc=10pt]
	We will not deal (as well as in our study of the Kruskal-Wallis test) the situation where some measurements are equal with some others in the same block, the existing proofs being not really convincing in my point of view.
	\end{tcolorbox}
	So, to every value $\left\lbrace x_{tb}\right\rbrace_{T\times B}$ of the table we now associate the rank $\left\lbrace r_{tb}\right\rbrace_{T\times B}$ corresponding to each treatment. Which will give us:
	
	Well now that we have built such a kind of non-parametric ANOVA table with two fixed factors without repetitions what do we do? What is the idea? Well the basic idea is the same as the Kruskal-Wallis test: we use the property of the average of the sum of the ranks but while having in mind that this time that the numbering (ranking) was not made on all measures of the table but block by block.
	
	In the context of our particular example we have therefore:
	\setlength\extrarowheight{5pt}
	
	\setlength\extrarowheight{0pt}
	and in case of no influence of treatments, we expect to have:
	
	or alternatively (this is equivalent):
	
	If there is no influence of the treatments this last four values should be equal and fluctuate around:
	
	We can feel that that the fluctuation of the  $\bar{R}_t$ around $\mu_R$ must follow a Normal centered  distribution if thery is not real any influence (there is a proof of this in the original article of Friedman but it has sometimes some unexplained gaps that refrain us to present it). We can the also reduce the Normal law such as:
	
	It is not always intuitive that the standard error is obtained by dividing the root of $B$ (the number of blocks) because the majority of practitioners have intuition to divide by the root of $T$ of the number of treatments when they study the theoretical aspect of Friedman test. But this can be verified with a numerical application or by remembering that the calculation of the variance $\sigma_R$ is done from the ranks of $B$ for a given treatment, ranks which the values (in the example above these values are $8$ time between $1$ and $4$) are obviously assumed for a given treatment to be independent and identically distributed.
	
	So we have:
	
	Unlike the Kruskal-Wallis, we don't do any sampling, so we should not correct the deviation with the correction factor on finite population (fpc) to decrease its value.
	
	The idea of Friedmann (at least that is how we will present it) is to say that the standard deviation of the sum of ranks of treatments obtained in the same way as in the Kruskal-Wallis test (which origin has been detailed in our study of the Mann-Whitney test):
	
	is this time only an estimator of the true standard deviation and we must used the relation between the unbiased and biased estimator to correct this estimate (relation proved during our study of the estimators):
	
	Therefore:
	
	where we have removed a degree of freedom to the chi-square for the reason already met many times in this section.
	
	Then after some elementary simplifications we get the "\NewTerm{Q Friedman test}\index{Q Friedman test}" (which is therefore a non-parametric test):
	
	\begin{tcolorbox}[colframe=black,colback=white,sharp corners]
	\textbf{{\Large \ding{45}}Example:}\\
	Going back to our example, then it comes:
	
	The critical value of reaction at the $5\%$ threshold is $7.65$. So we sadly do not reject the null hypothesis $H_0$ as what treatments have no effect (no difference between the treatments). The cumulative probability corresponding to $7.65$ (thus the $p$-value) is $9\%$.
	\end{tcolorbox}
	
	\pagebreak
	\paragraph{Spearman Rank Correlation Coefficient}\mbox{}\\\\
	The rank correlation coefficient Spearman, denoted $R_S$ is the correlation coefficient of the sequence $\left(R(i),S(i)\right),i=1...n,$ of the ranks naturally inspired by the linear correlation coefficient of Pearson that we saw at the beginning of this section:
	
	Let us also take an example before we tackle the theoretical aspect. Consider that the measurement of a sample of size $10$ (we took the same values as that taken for the previous non-parametric rank test studies):
	
	with their respective ranks as the approach idea of Kendall (simple idea but that he had to be found!):
	
	Now let us show that the above given relation is simplified drastically because the values of $R$, such as those of $S$, browse the first $n$ integers. For this, remember that we have proved in the section of Sequences and Series, that:
	
	Therefore:	
	
	Hence:
	
	We also have proved in the section of Sequences and Series that:
	
	therefore:
	
	Therefore it comes:
	
	Now let us play a little bit more to get a more simplified expression by observing that:
	
	Then it comes:
	
	Then we have:
	
	But we proved that:
	
	Therefore:
	
	Thus we find the famous relation available in all (good) statistics books:
	
	The Spearman coefficient takes the same essential properties as the Pearson coefficient namely that:
	
	and is equal to $0$ when the variables are correlated (dot not forget the important subtleties already mentioned in our study of the Pearson coefficient!!!).
	
	Note that this coefficient seems to be defined only for a pair of variables (I have never seen a generalization to a multivariate case at this day).
		
	\subsubsection{Range Statistics}
	The "\NewTerm{range statistics}\index{range statistics}" is a very important tool in finance and quality engineering (to mention only the two best-known examples). As the reader will see in what follows below, these statistics are by construction a subdomain of order statistics and the result that we will obtain here will be absolutely useful to us in the context of quality control charts (\SeeChapter{see section Engineering}) and therefore also in trading signals in finance.
		
	Given $X_1,X_2,...,X_n$ supposed independent and identically distributed random variables from a law of distribution $F$ and density $f$. Let us recall that we define the order statistic $X_{(i)}$ by:
	
	By writing:
	
		The variables $W_n$ and $M_n$ define the extreme order statistiques and their difference:
	
	is named the "\NewTerm{extreme deviation}\index{extreme deviation}". 
	
	For what will follow, we will consider as obvious the relation:
	
	Let us now determine the distribution function of the maximum $M_n$:
	
	because write that the maximum $M_n\leq$ is equivalent to write that for every $X_{(i)}$ we have $X_{(i)}\leq n$ (not easy to guess that you must have this approach...).
	
	We then have as the variables are independent (\SeeChapter{see section Probabilities}):
	
	and therefore we obviously have the distribution function:
	
	Respectively based on the same idea:
	
	and therefore we obviously have the distribution function:
	
	Hence:
	
	having used the linearity of the expected mean and using the fact that for the two distribution functions we are working on  the same random variable.
	
	Using an integration by parts (\SeeChapter{see section Differential and Integral Calculus}):
	
	remembering that $F(-\infty)$ and $F(+\infty)=1$.

	Now let us consider the special case where the distribution function follows a Normal centered reduced distribution such as:
	
	Then we have:
	
	Let us do a change of variable:
	
	Then we have:
	
	and then we found the relation given (almost $99\%$ of the time without proof) in the (good) books about statistical process control:
	
	named "\NewTerm{Hartley constant}\index{Hartley constant}" and therefore:
	
	This constant is as far as know not possible to calculate to calculate formally. Either we have to use Taylor series approximations of the terms of the integral, which becomes a nightmare for large $n$, either through a calculation using Monte Carlo simulation (\SeeChapter{see section Theoretical Computing}). As it is relatively long to implement in a spreadsheet, quality engineers prefer to use tables in which we find for example:
	
	Let us now see the variance of the range using always the Huygens theorem:
	
	The calculation of $\text{E}(R^2)$ is not nice (at least I have found nothing that satisfies the pedagogical goal of this book), the smallest complete proof held on $3-4$ A4 pages and formally bring anything because we end with an integral that cannot be calculate by hand (by cons if anyone has a simple, elegant and detailed proof do not hesitate to send it to us!). It is for this reason that after having write:
	
	if we write now as do many technical books:
	
	Then we have:
	
	But as we do not know the  maximum likelihood unbiased estimator of the standard deviation $\sigma$, we will use the proven relation:
	
	To finally have a biased estimator of the variance of the range:
	
	Here are some tabulated values of $d_3(n)$:
	
	
	\paragraph{Tukey's Range Test}\mbox{}\\\\
	Let us suppose that we have we have $Z_k$ independant centered reduced random variables. And let us denote by $U$ a a random variable following a chi-square law with $v$ degrees of freedom.

	Let us now define, for reasons that will seem obvious a little further below, the "\NewTerm{Studentized range}\index{Studentized range}" (the origin of the name comes from its resemblance with the definition of the Student law) by:
	
	and let us try to determine if this relation follows a known law and has a possible application (the find in the numerator what we defined earlier as the "extreme deviation" but with a different notation).
	
	For this, let us show that we fall on the above definition by considering a somewhat more general case where we have $X_i$ independent random variables that follow a Normal distribution $\mathcal{N}\left(\mu,\sigma\right)$ and the standard deviation:
	
	And let us study the ratio:
	
	Let us now proceed to the classical transformations already seen and proved and used many times since the beginning of this section:
	
	and therefore we have:
	
	So that's it already for the first step For the moment, even if we do do not know if this definition follows a known distribution law, we can already write the following very interesting definition (notice that the term on the left is always positive):
	
	or written in another way:
	
	and therefore we can calculate what is the cumulative probability of a range of measurements compared to a critical range $R_{X,\text{crit}}$ directly corresponding to a prescribed threshold $\alpha$. Which brings us to the possibility to write:
	
	Let us now recall that we have seen that the distribution function of the extreme deviation was given by a relation that was to our knowledge not calculable analytically:
	
	So the distribution function $Q_{k,v,1-\alpha}$ can therefore not be assimilate to a known law when $F(x)$ is any law. We must therefore unfortunately tabulate this distribution using Monte Carlo simulations (\SeeChapter{see section Theoretical Computing}) or refer to existing tables.
	
	Now to continue, we make a detour to the one-way fixed factor ANOVA that we studied earlier. Let us first recall that we have shown that for independent and identically Normal distributed random variables we had:
	
	and since a the ANOVA with fixed factor is also based on the assumption that:
	
	this implies that asymptotically the estimators have the same property:
	
	We also know that the standard deviation of the mean of a sample of a fixed-factor ANOVA is given in the framework and assumptions of the one-way fixed factor ANOVA by:
	
	But within the framework of one-way fixed factor ANOVA, we have also proved that under the assumptions imposed, we had:
	
	Therefore it comes:
	
	that is an estimator of:
	
	And as we have proved that:
	
	Therefore it comes:
	
	Therefore, we are naturally led to conclude that the relation we defined earlier:
	
	may be used in the study of the one-way fixed factor ANOVA under the form:
	
	to do a pre- or posterior-test (post hoc) to a one-way fixed factor ANOVA to check the hypothesis of equality of means and identify which are the aberrant means (multiple comparison test). So the Tukey's test is often accompanied by the Cochran C test that we have already discussed earlier when we make a one-way fixed factor ANOVA.
	
	Therefore within the ANOVA framework, we must reject the hypothesis of equality of the sample means if:
	
	or written in another way:
	
	In this case, it is then almost immediate we can build the following confidence interval:
	
	It should be known now that there is a post-hoc test for the one-way fixed factor ANOVA then when the following relation is applied:
	
	will not take the two most extreme means but will compare all the means pairwise (and why not after all!) with the highest average (well we could also have fun making all possible combinations as do some statistical software). In this case the relation to use is the same as above except that if we have for example a one-way fixed factor ANOVA with 4 levels then we will have 3 pairwise comparisons (average differences must always be positive!!!). Thus, imagining that the third average is the largest and that in descending order the biggest averages are the 4th, 2nd and 1st (the first is therefore smaller) then it comes:
	
	This approach (to extend the basic principle of Tukey's test), is named the "\NewTerm{Newman-Keuls test}\index{Newman-Keuls test}" or "\NewTerm{Student-Newman-Keuls test (SNK)}\index{Student-Newman-Keuls test}".
	
	\pagebreak
	\subsubsection{Extreme Value Theory}
	"\NewTerm{Extreme value theory}\index{extreme value theory}" or "\NewTerm{extreme value analysis}\index{extreme value analysis}" (EVA) is a branch of statistics dealing with the extreme deviations from the median of probability distributions. It seeks to assess, from a given ordered sample of a given random variable, the probability of events that are more extreme than any previously observed. Extreme value analysis is widely used in many disciplines, such as structural engineering, finance, earth sciences, traffic prediction, and geological engineering. For example, EVA might be used in the field of hydrology to estimate the probability of an unusually large flooding event, such as the 100-year flood. Similarly, for the design of a breakwater, a coastal engineer would seek to estimate the 50-year wave and design the structure accordingly.
	
	\pagebreak
	\subsection{Multivariate Statistics}
	Obviously Multivariate statistics is a subdivision of statistics encompassing the simultaneous observation and analysis of more than one  random variable.
	\begin{tcolorbox}[title=Remark,colframe=black,arc=10pt]
	Certain types of problem involving multivariate data, for example simple linear regression and multiple regression that we saw earlier above, are not usually considered as special cases of multivariate statistics because the analysis is dealt with by considering the (univariate) conditional distribution of a single outcome variable given the other variables.
	\end{tcolorbox}	 
	There are many different models, each with its own type of analysis that we will try to address in this book as always with a maximum of details and as accessible as possible (with softwares examples in the companion book for sure!):
	\begin{enumerate}
		\item Principal components analysis (PCA) that creates a new set of orthogonal variables that contain the same information as the original set. It rotates the axes of variation to give a new set of orthogonal axes, ordered so that they summarize decreasing proportions of the variation.
		
		\item Factor analysis that that is similar to PCA but allows the user to extract a specified number of synthetic variables, fewer than the original set, leaving the remaining unexplained variation as error. The extracted variables are known as latent variables or factors; each one may be supposed to account for covariation in a group of observed variables.

		\item Canonical correlation analysis that finds linear relations among two sets of variables; it is the generalised (i.e. canonical) version of bivariate correlation.

		\item Correspondence analysis (CA), or reciprocal averaging, that finds (like PCA) a set of synthetic variables that summarise the original set. The underlying model assumes chi-squared dissimilarities among records (cases).

		\item Canonical (or "constrained") correspondence analysis (CCA) for summarising the joint variation in two sets of variables; combination of correspondence analysis and multivariate regression analysis. The underlying model assumes chi-squared dissimilarities among records (cases).

		\item Multidimensional scaling comprises various algorithms to determine a set of synthetic variables that best represent the pairwise distances between records. The original method is principal coordinates analysis (PCoA based on PCA).

		\item Discriminant analysis (linear or quadratic), or canonical variate analysis, attempts to establish whether a set of variables can be used to distinguish between two or more groups of cases.

		\item Simultaneous equations models involve more than one regression equation, with different dependent variables, estimated together.
		
		\item Multivariate analysis of variance (MANOVA) that extends the analysis of variance to cover cases where there is more than one dependent variable to be analyzed simultaneously

		\item Multivariate analysis of covariance (MANCOVA) that an extension of analysis of covariance (ANCOVA) methods to cover cases where there is more than one dependent variable and where the control of concomitant continuous independent variables (covariates) is required. 
		
		\item Mixed model that contains both fixed effects and random effects. These models are useful in a wide variety of disciplines in the physical, biological and social sciences. They are particularly useful in settings where repeated measurements are made on the same statistical units (longitudinal study)
	\end{enumerate}
	
	\subsubsection{Principal Component Analysis}
	Principal component analysis (P.C.A.) is a mathematical method of graphical data analysis that consist to look for directions in space that best represent the correlations between $n$ random variables (supposed to have linear relations between them). In other terms it is a dimension reduction process (as for the single value decomposition process proved in the section of Linear Algebra) as it gives the possibility to the analyst to choose what are the variables in a model that explain the best the variability.
	
	Simply said, a P.C.A. allows for example to find in a dataset buying behavior similarities between observed classes.
	
	Even if the PCA is mainly used to visualize data, we must not forget that this is also a way to:
	
	\begin{itemize}
		\item To decorrelate the data. In the new base, consisting in new axis, the points have a zero correlation (we will prove it).
		
		\item To classify data into correlated clusters (in the industry it is mainly this possibility that is interesting!).
	\end{itemize}
	\begin{tcolorbox}[title=Remarks,colframe=black,arc=10pt]
	\textbf{R1.} There are several versions of P.C.A. known under the name of "\NewTerm{Karhunen-Loeve transformation}\index{Karhunen-Loeve transformation}" or "\NewTerm{Hotelling transformation}\index{Hotelling transformation}" and that can also be applied without programming in spreadsheet softwares or in specialized one (where the computing time will be shorter by cons ... and the results most accurate too...).\\
	
	\textbf{R2.} Depending on the authors and the point of view P.C.A. belong to the field of statistics named "Explanatory statistics".
	\end{tcolorbox}
	When we consider only two effects, it is customary to characterize their joint effect via the correlation coefficient. When we stands in two dimension, the available points (the sample of points drawn following the joint distribution) can be represented in a plane. The result of the P.C.A. in this plane is the determination of the two axes that best explain the dispersion of the available points.

	When there are more than two effects, for example three effects, there are three coefficients of correlations to be taken into account. The issue that gave rise to the P.C.A. is how to have a quick intuition of the joint effects?

	In dimension larger than two, a P.C.A. will always determine the axes that best explain the dispersion of the cloud of available points.

	The objective of the P.C.A. is to graphically describe a data array of individual with large quantitative variables:
	
	In order not to complicate the presentation of this method and to allow the reader to completely redo the calculations, we will work on the theory with a direct famous example.
	
	Let us consider for example a study of a botanist who measured the dimensions of $15$ iris flowers (the P.C.A. Is also widely used in finance to determine the elements that most influences the volatility of a portfolio). The three  variables ($p=3$) measured are:
	\begin{itemize}
		\item $x_1$: Length of sepal

		\item $x_2$: Width of sepal

		\item $x_3$: Length of the petal
	\end{itemize}
	
	The data are the following:
	
	
	For us, such a data table is simply a real components matrix with $n$ rows (the individuals) with $p$ columns (the variables):
	
	Thereafter the index $i$ will correspond to the line index and therefore to individuals. We will identify the individual $i$ with point line $x_{i.}=(x_{i1},\ldots,x_{ip})$ which will be considered as a point in an affine space (\SeeChapter{see section Vector Calculus}) of dimension $p$. The index $j$ will correspond to the index column so the variables. We will identify the variable $j$ with the column vector:
	
	this is therefore a vector in the vector space of dimension $n$ in $\mathbb{R}^n$.
	
	We will place ourselves in what will follow in two perspectives: either we will take the table data as $n$ points an affine space of dimension $p$, or we will take this table as $p$ points in a vector space of dimension $n$. We will see that there are dualities between these two perspectives.
	
	The mathematical tool that we will use here is linear algebra (\SeeChapter{see section Linear Algebra}), with the concepts of dot product, Euclidean norm and Euclidean distance.
	
	To simplify the presentation, we will initially consider that each individual as each variable has the same importance, the same weight. We also consider the case of the Euclidean distance.
	
	We'll start by centering the data, that is to say to put origin of the reference frame on the center of gravity of the point cloud. This does not change the appearance of the cloud, but allows us to get the coordinates of a point $M$, equal to the coordinates of the vector $\overrightarrow{GM}$, and thus to place ourselves in the vector space to be able to do the math! Since we assume throughout what will follow that the weights of the individuals are identical, we will take $m_i=1/n$ with $i=1\ldots n$.
	
	We consider the orthonormal reference frame $(\text{O},\vec{e}_1,\vec{e}_2,\ldots,\vec{e}_p)$ in the canonical basis $(\vec{e}_1,\vec{e}_2,\ldots,\vec{e}_p)$ of $\mathbb{R}^p$. Given $G$ being therefore the center of gravity of the point cloud. As each variable and each individual is assumed to have the same weight, then $G$ has for coordinates the reference frame $(\text{O},\vec{e}_1,\vec{e}_2,\ldots,\vec{e}_p)$:
	
	with:
	
	We then have yet graphically:
	\begin{figure}[H]
		\centering
		\includegraphics{img/arithmetics/pca_gravitypoint.jpg}
		\caption[]{Measurement points and center of gravity}
	\end{figure}
	We name "\NewTerm{centered matrix}\index{centered matrix}" the matrix:
	
	\begin{tcolorbox}[title=Remark,colframe=black,arc=10pt]
	The matrix of centered data contains the centered coordinates (which we will denote by $xc_{ij}$) of the individuals in the reference frame $(G,\vec{e}_1,\vec{e}_2,\ldots,\vec{e}_p)$. We place ourselves for what will follow always in this reference frame for point cloud of the individuals and we will take $\text{O}=G$.
	\end{tcolorbox}	
	for our example we have:
	
	and for the centered matrix:
	
	and graphically:
	\begin{figure}[H]
		\centering
		\includegraphics{img/arithmetics/pca_centered_measured_points.jpg}
		\caption[]{Centered measured points }
	\end{figure}
	To give an equal importance to each variable so that the type of units of the measurement does not influence the analysis, we will work with reduced centered data (\SeeChapter{see section Statistics}). For this, we will denote first:
	
	we the reader will have perhaps notice that we take the biased variance. But in the reality, we will take obviously the variance estimator and we will therefore divide by $n-1$ rather than by $n$.
	
	The variance of the sample centered variable is equal near to a factor of $1 / n$ to the norm of this same variable but centered. The matrix of reduced centered data (dimensionless) is therefore:
	
	If we denote by $D_{1/\sigma}$ the following diagonal matrix:
	
	Then we have:
	
	\begin{tcolorbox}[title=Remark,colframe=black,arc=10pt]
	Each component of the matrix $Y$ is therefore of zero mean and unit variance (which is equivalent to say that the norm of the standard reduced variable is equal to the unit as we will prove it immediately).
	\end{tcolorbox}	
	We define the "\NewTerm{matrix of centered normalized data}\index{matrix of centered normalized data}" by (we then speak of "\NewTerm{normalized PCA}\index{normalized PCA}" which is not mandatory but simplifies interpretation):
	
	Or also (it is simply the mean squared error that we introduced in the section Statistics):
	
	The terminology comes of course from the fact that the sum of the square of the components of each column of the matrix $Z$ has a unit norm. Indeed:
	
	Which gives:
	
	We have graphically:
	\begin{figure}[H]
		\centering
		\includegraphics{img/arithmetics/pca_centered_reduced_measured_points.jpg}
		\caption[]{Centered reduced measured points }
	\end{figure}
	Represent the point cloud of the reduced centered data or reduced normalized data don't change the shape of it. Indeed, the difference between the two is only a change of scale.
	
	The interesting information on individuals is the distance between the points! Indeed more this distance is great between two individuals $z_{i.}$ and $z_{i'.}$ plus the two individuals will be different and better we can characterize them. But we must first choose a distance. We will take the Euclidean distance (\SeeChapter{see chapter Topology}):
	
	The following figures show the orthogonal projections in space of this scatter cloud respectively in the planes $(\text{O},\vec{e}_1,\vec{e}_2),(\text{O},\vec{e}_2,\vec{e}_3)$ and finally in $(\text{O},\vec{u}_1,\vec{u}_3)$ that is the best projection, named "\NewTerm{factorial plane}\index{factorial plane}" (or sometimes "\NewTerm{scores diagram}\index{scores diagram}"), in the sense that it best respect the distances between individuals (verbatim, it deforms the least the cloud of points in space). The objective of the principal component analysis is to determine this best plan and we will prove now how!
	\begin{figure}[H]
		\centering
		\includegraphics{img/arithmetics/pca_projection_cloud_xy.jpg}
		\caption[]{Projection of the points on the horizontal plane of the center reduced basis}
	\end{figure}
	\begin{figure}[H]
		\centering
		\includegraphics{img/arithmetics/pca_projection_cloud_yz.jpg}
		\caption[]{Projection of the points on the vertical plane of the center reduced basis}
	\end{figure}
	\begin{figure}[H]
		\centering
		\includegraphics{img/arithmetics/pca_projection_cloud_best.jpg}
		\caption[]{Projection of the points on the factorial plane of the center reduced basis}
	\end{figure}
	And the plane view of each of the projections:
	\begin{figure}[H]
		\centering
		\includegraphics[scale=0.7]{img/arithmetics/pca_plane_views.jpg}
		\caption[]{Plane view of each of the projections}
	\end{figure}
	Before determining the factorial plane, we will first now try to detect possible links between variables.
	
	We recall (\SeeChapter{see section Statistics}) that the covariance between two variables $x_{.j}$ and $x_{.j'}$ is given by:
	
	and that the linear correlation coefficient (\SeeChapter{see section Statistics}) is:
	
	We will denote for later:
	
	the matrices of variance-covariance and of correlation (both being for recall square and symmetric matrices) with $j=1\ldots p,j'=1\ldots p$.
	
	We see quite easily that the matrix of covariance is at a given coefficient $1 / n$ near, the matrix of canonical dot products of vectors of the centered reduced matrix $X_c$ (in other words, each component of the variance-covariance matrix is equal to the dot product of the centered variables). We deduce the following relation:
	
	The matrix of variance-covariance (since, as we was it in the section Statistics, the diagonal contains the variances ... for recall!) is a well known interpretation in this book. By cons, what is new and we will be very useful to determine the factorial plane is the matrix of linear correlations that can also be written as follows:
	
	Which gives for our example where we have three variables (very easy to calculate using a spreadsheet software like Microsoft Excel), the following square matrix (the data are centered or not the components of the matrix are the same):
	
	To continue, always with the aim to determine the factorial plane, let us define the concept of inertia of a point cloud.
	
	\textbf{Definition (\#\mydef):} We name "\NewTerm{inertia of a point cloud}\index{inertia of a point cloud}" the quantity:
	
	where $G$ is the center of gravity of the point cloud and $M_i$ a point of $\mathbb{R}^p$ of coordinates $x_i^t$.
	\begin{tcolorbox}[title=Remark,colframe=black,arc=10pt]
	The square of the distance is taken by anticipation of the developments that will follow.
	\end{tcolorbox}
	Then we prove the following relation:
	
	\begin{dem}
	
	\begin{flushright}
		$\square$  Q.E.D.
	\end{flushright}
	\end{dem}
	We will in all the remaining text with normalized centered data, verbatim with the matrix $Z$. The points $M_i$ will therefore have here for coordinates $z_i^T$.
	
	The problem now is to find the best affine space of dimension $p$ in the sense that it respects the best the distances between points. For this, we will seek the best vectorial line $\Delta_{\vec{u}}$ that is perfectly determined by the vector $\vec{u}$. Let us denote $H_i$ the orthogonal projection of $M_i$ on the line $\Delta_{u}$. So our problem is to find the line (verbatim the vector $\vec{u}$) that makes the sum of the squares of the distances between the points $H_i$ is maximized. We write the problem as a quadratic programming problem (\SeeChapter{Numerical Methods}):
	\begin{equation}
		\begin{aligned}
		& \underset{\vec{u}}{\text{maximize}}
		& & \sum_{i,i'} d^2(H_i,{H'}_i) \\
		& \text{subject to}
		& & \vec{u}\in \mathbb{R}^p \\
		&&& ||\vec{u}||=1
		\end{aligned}
	\end{equation}
	But here we have:
	
	Indeed, the projected centroid (barycenter) of the cloud points  is also the origin. Consequently, our problem can be written:
	\begin{equation}
		\begin{aligned}
		& \underset{\vec{u}}{\text{maximize}}
		& & I \\
		& \text{subject to}
		& & \vec{u}\in \mathbb{R}^p \\
		&&& ||\vec{u}||=1
		\end{aligned}
	\end{equation}
	Itself therefore being equivalent to:
	\begin{equation}
		\begin{aligned}
		& \underset{\vec{u}}{\text{maximize}}
		& & \sum_{i,i'} d^2(\text{O},H_i,) \\
		& \text{subject to}
		& & \vec{u}\in \mathbb{R}^p \\
		&&& ||\vec{u}||=1
		\end{aligned}
	\end{equation}
	Let us solve this problem:
	First, since $H_i$ is the orthogonal projection of the point $M_i$ on $\Delta_{\vec{u}}$ we have $\overrightarrow{OH}_i=\alpha_i \vec{u}$ for all $i$ with $\alpha_i =\overrightarrow{OM}_i\circ \vec{u}$. Following this, the coordinates of the points $H_9$ on the straight line $\Delta_{\vec{u}}$ are:
	
	If follows that we have:
	
	Here we seek the unit vector $\vec{u}$. The matrix $Z$ is perfectly known to us. But, we have:
	
	The correlation matrix $R$ is symmetrical therefore, following the spectral theorem proved in the section of Linear Algebra, it is diagonalizable in an orthonormal basis of eigenvectors. Thus, we have proved that in the spectral theorem:
	
	is diagonal (up to us to choose the content) if $R$ is symmetrical and $S$ is orthogonal (which is in our special example a square $3\times 3$ matrix). The we deduce the following relation that it is common to name "\NewTerm{spectral decomposition}\index{spectral decomposition}":
	
	and as $S$ has been proven as being orthogonal (and that there exists a family of eigenvectors for this!), we have (\SeeChapter{see section Linear Algebra}):
	
	Therefore:
	
	where we choose for $\Lambda$ the diagonal matrix of the eigenvalues sorted in descending order: $\lambda_1 \geq \lambda_2 \geq \ldots \geq \lambda_p$.
	
	We then have:
	
	In the literature, this sum is often denoted as follows (in statistical software often named "\NewTerm{spectral decomposition}"):
	
	But $S$ being orthogonal, we have therefore:
	
	and this is comes from the fact that the orthogonal matrix is as we proved in the section Linear Algebra an isometry (it conserves the norm!).
	
	Since the eigenvalues are in descending order, we will write:
	
	Or the term in brackets is strictly less than or equal to $1$ by  the previous involvement. Therefore:
	
	Therefore:
	
	But remember that our objective is to maximize this inequality. In other words to seek $w_1$ such that the equality is respected. We see quickly enough that will be the case if $w_1=1$ and that the other terms are zero. Thus, a trivial solution to our maximization problem is:
	
	either because:
	
	which is then the first eigenvector of the matrix $R$ (since $R$ is diagonalized in this base) associated with the largest eigenvalue $\lambda_1$. Hence the fact that this solution is often denoted as:
	
	always with $\Lambda=S^{-1}RS$ (it is therefore quite easy to determine $S$ using softwares when $R$ and $\Lambda$ are known).
	
	Once we have found the first vector line, we seek a second orthogonal line in the subspace to the vector line that maximizes the inertia of the projected point cloud. We prove, and guess, that the solution is given by the vector line directed by the eigenvector associated to the second eigenvalue of the correlation matrix and so on ...
	
	Thus, we obtain a new basis $(\vec{u}_1,\ldots,\vec{u}_p)$ whose one of the plane is the factorial plane. However, we need to know the components of $Z$ in this base. As this base was built under the condition that $R$ is diagonalizable via the matrix $S$ then  the latter matrix is the linear application that will allow us to express $Z$ in the base $(\vec{u}_1,\ldots,\vec{u}_p)$ via the relation:
	
	Thus, in our example the three eigenvalues of the correlation matrix $R$ are (\SeeChapter{see section Linear Algebra}):
	
	and therefore:
	
	\begin{tcolorbox}[title=Remark,colframe=black,arc=10pt]
	Some softwares indicate the weights in respective and cumulated $\%$ for each of the eigenvalues. Thus we have in our special case the following respective weights in $\%$ of the total:
	
	So the first component explains $66.67\%$ of the effect. The first two components explain $96.15\%$, etc. This is why, for example, that in finance that among ten or more components, we will take only those that "explain" for example the $95\%$.
	\end{tcolorbox}
	By having the three eigenvalues, to determine the three eigenvectors $(\vec{u},\vec{u}_2,\vec{u}_3)$ that form the main base, we need to solve the following system of three equations with three unknowns (\SeeChapter{see section Linear Algebra}) for each eigenvalue:
	
	Which gives therefore (expected if someone request the details we will not provide them here as they can be simply made by hand or with any spreadhsheet software) for the matrix of eigenvectors:
	
	which satisfies therefore:
	
	or written in another way (following the remark  from a reader who wanted to check the calculations and was trapped):
	
	We then have for coordinates of the points $M_i$ in the base $(\vec{u}_1,\vec{u}_2,\vec{u}_3)$ using :
	
	The following matrix:
		
	The coordinates of the projection of the point cloud in the best plane defined by the vectors $(\vec{u}_1,\vec{u}_2)$ are then the first two columns of the previous matrix (thus corresponding the sepal length and sepals width).

	Indeed, we immediately see that these are the two columns that will maximize the sum of the norms in the given plane:
	\begin{figure}[H]
		\centering
		\includegraphics{img/arithmetics/pca_factorial_plane.jpg}
		\caption[]{Factorial plane already shown above ...}
	\end{figure}
	A software like Minitab 15.1 (reference in quality management industry) gives the following information for the eigenvalues (not very useful info graphically ... in my opinion):
	\begin{figure}[H]
		\centering
		\includegraphics[scale=0.8]{img/arithmetics/pca_eigenvalues.jpg}
		\includegraphics{img/arithmetics/pca_eigenvalues_details_minitab.jpg}
		\caption[]{Eigenvalues for the ACP as given by Minitab 15.1 ("scree plot")}
	\end{figure}
	and the following factorial plane (remains the question how the values are calculated in Minitab because they are not identical to those we got here my manual calculations... but the graphic is corresponding and this is the most important!):
	\begin{figure}[H]
		\centering
		\includegraphics[scale=0.8]{img/arithmetics/pca_factorial_plane_minitab.jpg}
		\caption[]{Factorial plane as given by Minitab 15.1}
	\end{figure}
	To close this subject, the reader must know that many software use the fact that the vectors $z_{.j}$ are of unit norm to do the scalar product which corresponds in this case simply the cosine between vectors such as:
	
	and as have have proved earlier above that:
	
	Therefore it comes:
	
	and as in our special example we have $3$ vectors $z_{.j}$, there is therefore $3$ possible dot products if we omit the scalard products of the vectors with themselves. Therefore the matrix:
	
	also contains the angles (in radian) between the vectors $z_{.j}$.

	Finally, let us indicate that the PCA being sensitive to outliers, it is sometimes better to transform the values of the original array in their ranks (see the study of rank statistics earlier above!) and applying exactly the same algorithm or using the Spearman Rank correlation coefficient. We then speak of a ""\NewTerm{PCA by ranks}\index{PCA by ranks}".
	
	\paragraph{SVD and PCA}\mbox{}\\\\
	Now we will prove and important theorem something important that is not quite obvious.
	\begin{theorem}
	The PCA is a special case of the SVD!!!
	\end{theorem}
	\begin{dem}
	We have prove before that:
	
	Using the properties of the transposed matrices:
	
	And that:
	
	And we have build in the section of Linear Algebra the singular value decomposition:
	
	That is in our case here:
	We have prove before that:
	
	Using the properties of the transposed matrices:
	
	And that:
	
	And we have build in the section of Linear Algebra the singular value decomposition:
	
	That is in our case here:
	
	Therefore:
	
	and since $V$ is an orthogonal matrix ($V^TV=\mathds{1}$):
	
	So if we compare:
	
	 the correspondence is easily seen!

	In fact, using the SVD to perform PCA makes much better sense numerically than forming the covariance matrix to begin with, since the formation of $ZZ^T$ can cause loss of precision. Indeed in some cases the PCA can diverge very quick and give inaccurate results.
	\begin{flushright}
		$\square$  Q.E.D.
	\end{flushright}
	\end{dem}
	
	\pagebreak
	\subsubsection{Correspondence Factorial Analysis (AFC)}
	The factorial correspondence  analysis, abbreviated CFA, is a statistical method of data analysis (widely used in biostatistics and survey analysis). The CFA technique is mainly used for large tables to compare all data (if possible all expressed in the same unit, like a currency, a dimension, a frequency or any other measurable quantity). It may in particular allow the study of contingency tables (or co-occurrence cross tables) and describe the connection between two variables. It serves to identify and prioritize all the dependencies between the rows and columns of the table.
	
	If more than two variables are to be taken into considerations we speak of multiple correspondence analysis (M.C.A.).
	
	Let us tackle now directly the theory with an example. For this we consider the following table (with two variables) of the areas of the types of tree that stands in Picardy in 1984 hectares:
	
	The experts in the domain sometimes name the totals for rows and columns, respectively, the "\NewTerm{line margines}\index{line margines}" and "\NewTerm{column margins}\index{column margins}". When the whole table is put in percentages, relatively to the total of the totals, it is named "\NewTerm{joint frequency representation}\index{joint frequency representation}":
	
	We wish to analyze whether there are degrees of similarity and difference between the variables. Let us notice that we are not trying to compare equality of means or variances therefore statistical tools seen in previoulsy are not suitable for this kind of analysis.
	
	If we choose the Euclidean distance (\SeeChapter{see section Vector Calculus}):
	
	on the raw data to measure the differences between departments, we get the following differences:
	
	and so on for the other regions. We then get:
	
	We see by looking at the table and before any calculation that the departments of Aisne and Oise seems to be similar and that the department of the Somme differs significantly. The distances put in evidence that observation.

	But in the above table the profiles of the Oise and the Somme, with a very small mixed forest, are however very similar in proportion.

	In this context, we see that the Euclidean distance transcribed quite well the mass differences between the departments. In other words, the Aisne and the Oise are similar because their surfaces are close. To eliminate the artifact related to the orders of magnitude, we need to transform the data into percentages (percentages of regions). We then get:
	
	where the experts in the field sometimes name the column \%Area "\NewTerm{marginal profile rows}\index{marginal profile rows}" or "\NewTerm{mass}" (and respectively when they indicate the rows of the percents for the trees).

	If we choose the Euclidean distance on the proportions (relative data), we get:
	
	Therefore:
	
	This time, the Oise and Somme appear well as have the most similar forests. We see that work with relative data seems more relevant in this case!

	Now let us borrow an idea to economists who, when they have tables of the same kind as the previous one, calculate what they named the "\NewTerm{index}\index{index (economy)}" or "\NewTerm{elasticity}" (also often name "\NewTerm{specificity index}\index{specificity index}" in statistics) and which is given by the ratios between the joint frequency and the marginal frequency:
	
	Here is an example obtained with the PivotTables in Microsoft Excel 11.8346 which includes natively the Index function. First the starting table:
	\begin{figure}[H]
		\centering
		\includegraphics{img/arithmetics/index_starting_pivottable.jpg}
		\caption[]{Microsoft Excel 11.8346 PivotTable for Index analysis}
	\end{figure}
	and by activating the Index function in the PivotTable properties we get:
	\begin{figure}[H]
		\centering
		\includegraphics{img/arithmetics/index_final_pivottable.jpg}
		\caption[]{Microsoft Excel 11.8346 PivotTable with Index values}
	\end{figure}
	To understand where these values comes from, let us look for example the article \textit{Desk} in the region \textit{Alberta}. It has a return (joint frequency) of:
	
	compared to all regions, which is above the value of $33.33\%$ that this article would have for yield for all regions if there is no region of preference!

	The \textit{Alberta} region a yield (marginal rate) of:
	
	with respect to all regions which is below to the yield of $33.33\%$ that it would have if there were no regions of preference. Thus, this index table shows whether the differences are qualitatively significant!!
	
	The ratio therefore gives:
	
	which shows a strong shift between the value obtained and the value we would have if the proportions were respected (overrepresentation of $283\%$).
	
	So it is a kind of conformity compliance: if the ratio was equal to $1$, it is that the regional sales performance of this particular article would be consistent with respect to all sales of this region with respect to the national market. There would be no anomalies. Let us see this example for our trees where we had seen the effectives:
	
	and for which we get the following PivotTable Next of the effective index in Microsoft Excel 14.0.7166:
	\begin{figure}[H]
		\centering
		\includegraphics{img/arithmetics/index_trees_forests.jpg}
		\caption[]{Microsoft Excel 14.0.7166 PivotTable for Index analysis}
	\end{figure}
	and we still see clearly with this table that it is well the Oise and Somme are most alike!

	Before continuing, we might ask ourselves the following most important question: What are the theoretical numbers that we would have been obtained if the proportions of the trees in the regions were strictly equivalent to the overall proportions (ie so that the index are all unitary)?

	Well simply by making the following calculations (it is merely a rule of three calculated in each cell) where the reader must - if possible - understand the meaning without applying this foolishly:
	
	And we get with this new values of the table the following theoretical index with in Microsoft Excel 14.0.7166:
	\begin{figure}[H]
		\centering
		\includegraphics{img/arithmetics/index_trees_forests_theoretical.jpg}
		\caption[]{Microsoft Excel 14.0.7166 PivotTable for Theoretical Index analysis}
	\end{figure}
	which shows now that the proportions are now observed! Parenthesis closed (but on which will come back later)!
	
	Well when we want to make a factorial correspondence analysis, our relation:
	
	Therefore becomes:
	
	hence:
	
	Again, the Oise and the Somme appear as the most similar.
	
	The distance above is named the "\NewTerm{Chi-square metric}\index{Chi-square metric}" because it looks like (but that's all!) at the distance used in the adjustment test of the same name (\SeeChapter{see section Statistics}) but here, it only helps to establish a hierarchy in the context of a contingency table and to observe similar the variables in a more easy way!!
	
	\begin{tcolorbox}[title=Remark,colframe=black,arc=10pt]
	There is another way to calculate an CFA based on an Euclidean distance but by taking care beforehand to transform in a special way to the contingencies table so that the calculation are the same that when we use the Chi-square metric.
	\end{tcolorbox}	
	
	\subsubsection{Chi-2 Test of Independence}
	The $\chi^2$-test is applied when you have two categorical variables from a single population. It is used to determine whether there is a significant association between the two variables.
	
	For example, in an election survey, voters might be classified by gender (male or female) and voting preference (Democrat, Republican, or Independent). We could use a chi-square test for independence to determine whether gender is related to voting preference. 
	
	Suppose that a categorical variable $A$ has $r$ levels, and another categorical variable $B$ has $c$ levels. The null hypothesis $H_0$ states that knowing the level of variable $A$ does not help you predict the level of variable $B$. That is, the variables are independent.
	
	We will present this test with a companion example because we know from experience that it is more effective for learning and understanding.

	Let us recall that during the introduction of the previous method for comparing numbers (values) and detect which were the closest we gave the following observed effectives:
	
	and we also showed how to find the table of theoretical effectives (rounded to the nearest whole number) where the proportions should have possibly be respected:
	
	We know that building the last table above assumes that the three regions are in identical conditions for everything related to the growth and multiplication of trees and the number of trees is in direct causal relation (!!!!) with regions and there is no other intermediate causes ... what is a strong assumption!
	
	But under this assumption, let us suppose that we would like know if the differences observed between the number of trees and the regions are statistically significant or purely random because of the experimental sample? In other words, we want to know if the number of trees actually depends on the regions in which they grow or if these values are only due to the sample chance? This is why this test is name the "\NewTerm{Chi-square test of independence}\index{Chi-square test of independence}". 
	\begin{tcolorbox}[title=Remark,colframe=black,arc=10pt]
	The chi-square test of independence is recommended in sensory analysis by the norm ISO 8588-1987 under the name "test A-Not A".
	\end{tcolorbox}
	To answer this questions we must first have a reference. And this reference is precisely the assumption of direct causal link (proportions respected) that we have given just previously.

	If we consider that each cell of the table of observed effective corresponds to a random variable of unknown law and that each celle of theoretical table is considered as following a random binomial variable (and asymptotically as a Normal distribution) then as we proved it earlier we know that we can use the chi-square test of adjustment:
	
	to have a good idea (but still pretty rough considering the assumptions!) if the differences between the the observed effective values are due to chance or are significant. But, if $D$ is small, the probability that it is due to chance is great but if $D$ is large so we have a real - significant - difference (so we use the chi-square test of adjustment but in reverse way in fact!).
	
	It remains to determine the number of degrees of freedom of the $\chi^2$ law that follows this sum in this configuration!
	
	In the particular case (but quite easily generalized by recurrence) of a table with two inputs with two categorized variables $X$ with $l$ levels and $Y$ with $c$ we will have respectively the $l$ rows and $c$ columns.
	
	Thus, the table will have obviously $l\cdot c$ cells. In the table of theoretical effectives (where each cell is treated as a random variable) each cell will be entirely determined by the sum of the other so that the number of degrees of freedom will be logically as we have seen during our study of degrees of freedom:
	
	Thus, taking our example of forests, the total of total is $272,650$ which allows us to write this last relation and thus determine the value of a possibly empty cell, all other data being known!

	A chi-square test on this type of table tests the hypothesis of independence against the alternative hypothesis of non-independance . Under the assumption of independence we believe we need only:
	
	values on the $N$ one to determine all  of them (assuming implicitly known the sum of all rows and columns).

	So if we have a table of $2$ rows by $2$ columns, we only need if we know the total rows and columns, to know $2$ values (i.e. $(2-1) + (2-1)=2$) to determine the $2$ missing values. The reasoning is the same for a table of $3$ rows by $3$ columns where we just have to know at least $4$ values (i.e. $(3-1) + (3-1)=4$) to determine the missing $5$ values.
	
	The degrees of freedom for the chi-square is then:
	
	It is this relation that we say us (trivially!) that if in a table of $2$ rows by $2$ columns so with $4$ cells (total of rows and columns being known!) that being known only one single values ($ddI$ being equal to $1$ ), we can determine the other $3$ missing values.
	
	As we know a possible definition of the number of degrees of freedom $\text{df}$ is that it is the maximum number of values of the model such that none of them is calculable from the others.
	
	Similarly, for a table of $3$ rows by $3$ columns with $9$ cells as is the case of our example above with forests, the knowledge of only $4$ cells allows us thanks to total lines and columns to determine the other $5$ that would possibly not be known.
	
	Hence the relation in the context of the application of chi-square for the final equation:
	
	by making use of notations used in the industry. The term:
	
	is often named "\NewTerm{square of standardized residual}\index{square of standardized residual}". And the ratio:
	
	is often named the "\NewTerm{contribution to the Chi-square of independence}\index{contribution to the Chi-square of independence}".
	\begin{tcolorbox}[title=Remark,colframe=black,arc=10pt]
	To use this test properly, the practitionner should check first that the differences (numerator) follow a normal distribution or that all terms of the sum follow a Chi-square distribution or approximately (asymptotically) a Normal centered reduced distribution centered and that the effective in each cell are greater than $5$ otherwise Monte Carlo simulations have to be used to determine the $p$-value.
	\end{tcolorbox}	
	In our example, we have:
	
	and the $p$-value of this value with the chi-square distribution with $4$ degrees of freedom:
	
	is so close to zero (not statistically significant) that we have no chance of being mistaken in asserting that the observed differences in the table are statistically significant between the $3$ and so that there is very likely independence.
	
	\pagebreak
	\subsubsection{Cramér's V}
	The "\NewTerm{Cramér's V}\index{Cramér's V} (sometimes referred to as "\NewTerm{Cramér's phi}\index{Cramér's phi}" and denoted as $\varphi_c$) is a measure of association between two nominal variables, giving a value between 0 and +1 (inclusive). It is based on Pearson's chi-squared statistic and was published by Harald Cramér in 1946.
	
	As we will prove it, $\varphi_c$ is the intercorrelation of two discrete variables and may be used with variables having two or more levels. $\varphi_c$ is a symmetrical measure, it does not matter which variable we place in the columns and which in the rows. Also, the order of rows/columns doesn't matter, so $\varphi_c$  may be used with nominal data types or higher (ordered, numerical, etc.)
	
	We have seen just previously that the chi-square test of independence may be used to measure the degree of association of two categorical variables in a contingency table of $l$ rows and $c$ columns:
	
	and that this distance follows a chi-square distribution with $(l-1) (c-1)$ degrees of freedom. We will prove intuitively that the maximum value of the distance $D$ is given by:
	
	and that maximum value is achieved if and only if each row or column contains exactly one non-zero value. Under the latter condition, we can always rearrange the contingency table so to have all nonzero terms on the diagonal of the table.
	
	Obviously if the table is not square as below:
	
	The case that minimze $D$ request diagonal terms on the smallest dimension in row or column (row dimension in the example above) denoted by tradition $q$. In the special example above we have:
	
	In this special case and obviously theoretical one, the rows that have only zero values can be omitted and therefore the previous table can be reduced to:
	
	Of course, ignoring the rows or columns that have only zero values bring us to assume that for the distance $D$ of the chi-square we request that:
	
	which is quite indisputable ... For what will follow, we will need the following relations:
	
	and:
	
	Therefore it comes:
	
	We then define the following value:
	
	as being the "\NewTerm{Cramér's V coefficient}\index{Cramér's V coefficient}" (the majority of software, however, give the value of $V$ squared). The latter is such that it never exceeds $1$ and allows a more intuitive interpretation of the degree of association in a contingency table.
	
	In the case where the table size is of $2$ rows and $2$  columns, the previous relation is then reduced immediately to:
	
	Relation that is traditional to note in this case as follows:
	
	and to name "Cramèr's phi".
	\begin{tcolorbox}[colframe=black,colback=white,sharp corners]
	\textbf{{\Large \ding{45}}Example:}\\
	Consider the following table (even if the conditions are not met for a chi-square test of independence):
	
	with the theoretical effectives:
	
	Then we have:
	
	And to a threshold level of $95\%$, we get with Microsoft Excel 14.0.6123:
	
	or with the corresponding $p$-value:
	
	It then comes immediately:
	
	We are a bit at the limited here ... since the $p$-value is very close to the traditional $0.05$. However, take a decision in this case where the effectives are so low would be equivalent to conclude anything and even more that this statistical tool is built on an accumulation of approximations.
	\end{tcolorbox}
	
	\subsubsection{Exact Fisher Test}
	When the effective in the contingency table are too small or that the values are really too irregular, the use of chi-square (Pearson test) is not possible because the application conditions are no longer valid. We will see that the Fisher exact test can be formalized analytically in $2\times 2$ contingency tables  (the majority of statistical software only support this particular scenario for the Fisher exact test) otherwise what must be used Monte Carlo simulations.
	
	The principle of the "\NewTerm{Fisher exact test}\index{Fisher exact test}" (usable both in bilateral or unilateral even if this latter is far more common in practice), so based on the crossover frequency, is to determine whether the pattern observed in the table contingency is an extreme situation with respect to all possible situations. As we will prove it, this test has the special property that any cell of the table can be referred for the test because the underlying distributions (marginals) of probability are equivalent.
	
	To study this test, as often in this section, we use directly an example as theoretical companion. 
	
	Let us consider the following contingency table (which now is known to us ...):
	
	which is not really suitable for chi-square test of independence since the content of the cells is less than $10$ units and the number of degrees of freedom would be equal to unity.
	
	The same table gives in percentages will give (even if it is unnecessary for the study of this test test statistical software communicate frequently these values):
	
	The theoretical effectives are given by (even if it is as useless to study this test but statistical software also frequently communicate these values):
	
	The question we will begin to ask ourselves is the following: knowing the totals for each row and each column, what is the probability of having the values present in each cell?!
	This question can be reformulated if we change the the table in the following generic form:
	
	Explicitly and relatively to our example by adopting the notation in use of the hypergeometric law, the question is to know what is the probability of having $8$ ($a = k$) projects among the $18$ ($n$) whose deadlines were met by certified project managers knowing that there are $9$ projects ($m$) at total whose deadlines were respected and $12$ projects $(p$) at total managed by certified project managers.
	
	We then saw in the section Statistics that in this case it is an exhaustive drawing, we then have to use the hypergeometric law given by for recall by as proven earlier above by:
	
	Thus with of Microsoft Excel 11.8346:
	
	\begin{center}
	\texttt{=HYPGEOM.DIST(k,p,m,n,false)}\\
	\texttt{=HYPGEOM.DIST(8,12,9,18,FALSE)=0.06108597}
	\end{center}
	
	where for recall, $k$ is the number of successes in the sample, $p$ is the sample size, $m$ the number of successes in the population and $n$ the size of the population.

	In fact, as already mentioned, the probabilities are all equal regardless of the selected cell of the contingency table !! This can be checked numerically for skeptics using again a spreadsheet software like Microsoft Excel 14.0.6123 by creating the following structure: 
	\begin{figure}[H]
		\centering
		\includegraphics[scale=0.85]{img/arithmetics/exact_fisher_text_excel.jpg}
		\caption[]{Exact Fisher test symmetry with Microsoft Excel 14.0.6123}
	\end{figure}
	and therefore each time that the reader presses the F9 keyboard key it can check that all probabilities are always equal.
	
	This can also be checked formally by selecting a cell of the table and writing:
	
	and for another cell of the same column, we will have:
	
	and therefore:
	
	and so on...

	Anyway, that being said, so we have in the upper left cell the value $8$ while the theoretical effective is $6$. The first thing we can answer is whether this value of $8$ is unusually large or not compared the theoretical effective. For this, we calculate by example the unilateral cumulative probability of being less than or equal to $8$. We have then with Microsoft Excel 14.0.6123 and later (the last parameter of $1$ (equivalent to \texttt{TRUE})of the function indicate to the software we want the cumulative probability):
	\begin{center}
	\texttt{=HYPGEOM.DIST(8,12,9,18,1)=0.995475113}
	\end{center}
	It therefore appears with a threshold of $5\%$ in unilateral that this value is unusually large. We are then in an extreme situation.
	
	By cons, even if the probabilities are equal for all the cells, the cumulative probability is not! Thus we have for example for the value of the lower left cell (to check if it is abnormally small compared to the theoretical effective of $6$):
	\begin{center}
	\texttt{=HYPGEOM.DIST(4,12,9,18,1)=0.06561086}
	\end{center}
	So this is a value that is not abnormally small. However, we would like to have a test to conclude if the entire table is or is not an extreme configuration. However, by making the calculation cell by cell, we will not get too much...
	
	The idea then is (at least on the "paper") to build all tables whose marginal frequencies are $9;9$ and $12;6$ and to calculate the probability of a given cell (the advantage of this technique is that the conclusion will be the same regardless the chosen cell as reference for the calculations):
	
	\begin{center}
	\texttt{=HYPGEOM.DIST(9,12,9,18)=0.004524887}
	\end{center}
	
	
	\begin{center}
	\texttt{=HYPGEOM.DIST(8,12,9,18)=0.06108597}
	\end{center}
	
	
	\begin{center}
	\texttt{=HYPGEOM.DIST(7,12,9,18)=0.244343891}
	\end{center}
	
	
	\begin{center}
	\texttt{=HYPGEOM.DIST(6,12,9,18)=0.380090498}
	\end{center}
	
	
	\begin{center}
	\texttt{=HYPGEOM.DIST(5,12,9,18)=0.244343891}
	\end{center}
	
	
	\makebox[\textwidth]{\texttt{=HYPGEOM.DIST(4,12,9,18)=0.061085973}}
	
	
	\makebox[\textwidth]{\texttt{=HYPGEOM.DIST(3,12,9,18)=0.004524887}}
	
	To summarize with the values of $k$ (corresponding to the top left cell):
	
	As in the column of the original table which we just worked with the smallest value is $4$ and the biggest $8$, we will take the tail probabilities to get what is the $p$-value to be greater or equal $8$ and less than or equal $4$ (this is therefore a bilateral test). Then we have:
	
	Therefore the $p$-value is $13.12\%$. We can therefore not say that our original table is in an extremal configuration if we chose an empirical threshold value of $5\%$. Many softwares only communicate the $p$-value.
	\begin{tcolorbox}[title=Remark,colframe=black,arc=10pt]
	The choice of the bounds in this test are subject to debate. Indeed, if we chose for example to focus on the probability to be in the closed bounded interval of $4$ and $8$ (inclusive!), we would have a result of $99.09\%$ and then we should consider that we are in an extreme configuration if we made the $5\%$ empirical threshold choice. Therefore the choice of the bounds with a discrete distribution like the Hypergeometric law many times something not obvious at the opposite of a test based on a continuous distribution that suffers absolutely not of this kind of problems. The majority of softwares that we know at this date take an open bounder interval (this corresponds therefore to the first calculation we made with the $p$-value of $13.12\%$.
	\end{tcolorbox}	
	Finally, to close this subject, the reader can check that he will find the same results whatever he takes for reference cell. 

	Let us also indicate that the Fisher exact test can obviously be used in machine learning to measure the performance of a binary classification!!!!!
	
	\subsubsection{Cohen's kappa agreement}
	If our judgments reflect our thinking, they are rarely in agreement with those of others.
	
	This inter-individual variability beneficial to humans, however, is disadvantageous in many scientific disciplines, where it is often necessary to assess and improve the agreement between similar information applied to the same object in the context of a quality control requirement or in sensory analysis.
	
	The non-parametric Cohen's kappa test allows for example to quantify the binary (dichotomous) agreement between two or more researchers or technicians when the judgments are qualitative.
	
	There this kappa statistic is frequently used to test "\NewTerm{inter-rater reliability}\index{inter-rater reliability}". The importance of rater reliability lies in the fact that it represents the extent to which the data collected in the study are correct representations of the variables measured. Measurement of the extent to which data collectors (raters) assign the same score to the same variable is named "interrater reliability". While there have been a variety of methods to measure interrater reliability, traditionally it was measured as percent agreement, calculated as the number of agreement scores divided by the total number of scores. In 1960, Jacob Cohen critiqued use of percent agreement due to its inability to account for chance agreement. He introduced the Cohen's kappa, developed to account for the possibility that raters actually guess on at least some variables due to uncertainty. Like most correlation statistics, the kappa can range from $-1$ to $+1$. While the kappa is one of the most commonly used statistics to test interrater reliability, it has limitations. Judgments about what level of kappa should be acceptable for health research are questioned. Cohen's suggested interpretation may be too lenient for health related studies because it implies that a score as low as 0.41 might be acceptable. Kappa and percent agreement are compared, and levels for both kappa and percent agreement that should be demanded in healthcare studies are suggested.
	
	Many situations in the healthcare industry rely on multiple people to collect research or clinical laboratory data. The question of consistency, or agreement among the individuals collecting data immediately arises due to the variability among human observers. Well-designed research studies must therefore include procedures that measure agreement among the various data collectors. Interrater reliability is a concern to one degree or another in most large studies due to the fact that multiple people collecting data may experience and interpret the phenomena of interest differently. 
	\begin{tcolorbox}[title=Remark,colframe=black,arc=10pt]
	There are a number of statistics that have been used to measure interrater and intrarater reliability. A partial list includes percent agreement, Cohen's kappa (for two raters), the Fleiss kappa (adaptation of Cohen's kappa for 3 or more raters) the contingency coefficient, the Pearson r and the Spearman Rho, the intra-class correlation coefficient, the concordance correlation coefficient, and Krippendorff's alpha (useful when there are multiple raters and multiple possible ratings). Use of correlation coefficients such as Pearson's $R$ may be a poor reflection of the amount of agreement between raters resulting in extreme over or underestimates of the true level of rater agreement
	\end{tcolorbox}	
	Let us take the case in the medical field where two or more practitioners examining the same patient or even offer different diagnoses or different therapeutic decisions. In the absence of a reference, this proliferation of advices does not bring the expected security of a perfect diagnostic or therapeutic agreement for the physician and the patient. It is therefore important that the agreement in a team or between teams is the best possible guarantee for the quality and continuity of care.
	
	One solution here is to realize a session of "concordance experiment" between physicians to estimate their rate of agreement using the kappa coefficient and study their disagreements to address them.

	To illustrate the concepts, let us consider the very important case of two quality managers that analyzed $11$ pieces to reject or accept them. They get:
	
	The theoretical frequencies being obtained always by a rule of three (same calculation of rule of three that for the  chi-square and Fisher test of independence exact seen above):
	
	Cohen's kappa is defined by the ratio:
	
	Therefore in our example:
	
	This value of $0.441$ indicates a moderate agreement between the two individuals (Alice and Bob).

	If instead of having the frequencies, we work in percentages (proportions) of the total, the Kappa is then:
	
	Which will give in our example:
	
	That it is usage to write in the following condensed form:
	
	with:
	
	where $+1$ corresponds to a perfect agreement and $-1$ to a perfect non-agreement. Obviously to have a perfect agreement, we must have the cells (Rejected, Rejected) and (Accepted, Accepted) that are equal and that the other cells are zero.
	
	The following table of interpretation of the positive part of the $\kappa_C$ is of common usage (the negative one having not too much interest...):
	
	However, the practitioner must be very critical (as always!) by using this type of tool. Understanding its construction also helps - also as always - to identify its weaknesses and assumptions that are quite questionable.

	Let us finally indicate that the Cohen's Kappa can obviously be used in machine learning to measure the performance of a binary classification!
	
	\subsubsection{McNemar's test}
	McNemar's test may well be calculated along the Cohen's Kappa (the first being a statistical hypothesis test and the second one only an empirical point estimator of concordance). The idea is that under the null hypothesis $H_0$ (in this case named "symmetry hypothesis"), one of the diagonals of the table should have equal values. In other words in the form of proportions and only by focusing on one of the two diagonals:
	
	or of frequencies:
	
	Knowing that:
	
	and under the condition that $n$ is large enough, we can write based on a binomial law whose behavior is asymptotically Normal:
	
	We can realize that this is equivalent to write:
	
	In the literature, we often find this last relation in the form:
	
	To return to our original relation, some take the square and then approximate the square of $Z$as a chi-square law (... !) with one degree of freedom (but good in reality approximate a Normal centered reduced distribution by the chi-square aw with one degree of freedom is bulls... in our point of view...):
	
	which is often the relation defined in the books (without proof...) as the "\NewTerm{McNemar test}\index{McNemar test}".

	The test is normally conducted bilaterally. The advantage of the McNemar test is the ease with which we can build a confidence interval of the difference in the diagonal. Indeed, starting from the estimator of the difference:
	
	Therefore using the variance (and covariance) proved in our study of the multinomial distribution, we get:
	
	and therefore we can do an approximate confidence intervals of the following if the usual conditions are met :
	
	That is to say a little bit more explicitly:
	
	with as we just proved it:
	
	
	\pagebreak
	\begin{tcolorbox}[colframe=black,colback=white,sharp corners]
	\textbf{{\Large \ding{45}}Example:}\\\\
	During a social audit, a survey is conducted on $200$ employees about the work organization. After reorganization, the same question is asked. Can we consider that there has been real changes?
	
	We then have taking arbitrarily  the diagonal $(25, 38)$ the for analysis:
	
	The majority of statistical software will not give you 2,683 as they apply an empirical correction due to the rough approximation of that is the chi-square law. You will then have on the softwares display the value of $2.285$ for $Z^2$.\\

	The $p$-value it is given (without correction) with a spreadsheet software like the Microsoft Excel14.0.6123:

	\begin{center}
	\texttt{= 1-CHISQ.DIST(2.683, 1, 1)} $= 10.14\%$
	\end{center}
	and with the correction use by statistical software we would get about 13\%. So in both cases the $p$-value is greater than the traditional $5\%$ threshold, we therefore can not reject the null hypothesis $H_0$ as the difference is large.\\

	We will not calculate in this example the confidence interval of the difference because almost all statistical software have different methods for calculating this value.\\
	
	Finally, to close the subject on the McNemar test, let us indicate an empirical indicator often used that is named the "\NewTerm{Yule coefficient}\index{Yule coefficient}" and defined by:
	
	\end{tcolorbox}
	
	\pagebreak
	\subsection{Survival Statistics}
	Survival analysis is a branch of statistics for analyzing the expected duration of time until one or more events happen, such as death in biological organisms and failure in mechanical systems. This topic is named "\NewTerm{reliability theory}\index{reliability theory}" or "\NewTerm{reliability analysis}\index{reliability analysis}" in engineering (\SeeChapter{see section Industrial Engineering}), "\NewTerm{duration analysis}" or "\NewTerm{duration modeling}\index{duration modeling}" in economics, and "\NewTerm{event history analysis}\index{event history analysis}" in sociology. Survival analysis attempts to answer questions such as: what is the proportion of a population which will survive past a certain time? Of those that survive, at what rate will they die or fail? Can multiple causes of death or failure be taken into account? How do particular circumstances or characteristics increase or decrease the probability of survival?

	To answer such questions, it is necessary to define "lifetime". In the case of biological survival, death is unambiguous, but for mechanical reliability, failure may not be well-defined, for there may well be mechanical systems in which failure is partial, a matter of degree, or not otherwise localized in time. Even in biological problems, some events (for example, heart attack or other organ failure) may have the same ambiguity. The theory outlined below assumes well-defined events at specific times; other cases may be better treated by models which explicitly account for ambiguous events.

	More generally, survival analysis involves the modelling of time to event data; in this context, death or failure is considered an "event" in the survival analysis literature – traditionally only a single event occurs for each subject, after which the organism or mechanism is dead or broken.
	
	In practice we consider:
	\begin{itemize}
		\item The Life tables (\SeeChapter{see section of Population Dynamics}) to describe the survival times of members of a group.
		
		\item The Kaplan-Meier method to also describe the survival times of members of a group.

		\item The Cochran-Mantel-Haenzel test (Log-Rank test) to compare the survival times of two or more groups.

		\item The Cox's proprotional hazard model (Cox regression) to describe the effect of categorical or quantitative variables on survival

	\end{itemize}

	\pagebreak
	\subsubsection{Kaplan-Meier Survival Rate}
	OK... we don't like to do this in this book. But for convenience four our reader and because of some readers feedback let us copy paste here the theory about the Kaplan-Meier Survival Rate as introduction in the section of Industrial Engineering:
	
	In areas such as high-level industry, high-level medicine or high-level biology, we often interested in the:
	\begin{enumerate}
		\item Survival duration after a serious event
		\item Duration of remission after treatment or surgery
		\item Duration of symptoms after an abnormality
		\item Duration of a symptomless infection
	\end{enumerate}
	We seek very often to distinguish at least "the event of interest":
	\begin{enumerate}
		\item System shutdown after serious event
		\item End of remission 
		\item End of a symptom after anomaly
		\item Beginning of a symptom during an infection
	\end{enumerate}
	of the variable to explain "duration before the event of interest":
	\begin{enumerate}
		\item Elapsed duration before shutdown
		\item Elapsed duration before the end of remission
		\item Elapsed duration before the end of the symptom
		\item Elapsed duration without symptoms
	\end{enumerate}
	\textbf{Definitions (\#\mydef):}
	\begin{enumerate}
		\item[D1.] We name "\NewTerm{remission}\index{remission}", the reduction of a disease or a temporarily malfunction.
		
		\item[D2.] The "\NewTerm{survival time}\index{survival time}" or "\NewTerm{lifetime}\index{lifetime}" $T$ means the time which elapses from an initial time (start of treatment, diagnosis or failure, etc.) until the occurrence of a final event of interest (patient death, relapse, remission, cure, repair, etc.). We say that the object of the study "survives at time $t$" if at this moment the final interest event has not yet occurred.
	\end{enumerate}
	\begin{tcolorbox}[title=Remark,colframe=black,arc=10pt]
	Although this type of study can be associated with preventive maintenance (\SeeChapter{see section Industrial Engineering}), statisticians associate this type of study rather to the domain of "\NewTerm{survival analysis}\index{survival analysis}".
	\end{tcolorbox}
	
	We will focus in this book on a particular context but that can be easily generalized:
	\begin{itemize}
		\item Cohort/Clinical trial where we study the survival time of each patient (machine).
		
		\item All patients (machines) do not have the same observation time (different instants of entry into the study).
		
		\item We have information on the survival time of each patient (machine) but we do not know exactly when it happens.
	\end{itemize}
	From the last two points, we conclude that the survival time can be censored. So the usual statistical techniques does not apply directly as censored data require special treatment (of course, if we remove the censored data we lose information). It goes without saying that this situation is extremely common and therefore the study of the Kaplan-Meier estimate is very important.
	
	\textbf{Definition (\#\mydef):} The duration $T$ is said to be censored if the duration is not observed completely. The different types of censoring are:
	\begin{itemize}
		\item Type I censoring: fixed right. In this situation, the time is not observable beyond a maximum, fixed, named "\NewTerm{fixed-censoring}\index{fixed-censoring}" and imposed. So either we have the opportunity to observe the real duration of the event of interest for the item if it occurs before the fixed-censorsing, or we limit ourselves to the fixed-censoring time if the vent of interest has not occurred before.
		
		\item Type II censoring: wait. In this situation, we observe the lives of $n$ individuals until $m\leq n$ individuals have seen the event to occur (deceased). The duration considered is therefore that of the beginning of the experiment until the event of interest for the $m$-th.
		
		\item Type III censoring: random left. In this situation, we do not know when the event of interest has occurred (because we started to observe the subject of study too late). We can not then deal with "durations" in the measurable sense and we have to limit ourselves to a simple count.
		
		\item Type III censoring: random right. In this situation, we do not know when the event will take place (because we stopped to observe the subject before it occurs  for any reason). We can not then also not deal with "times" in the measurable sense and we must limit ourselves to a simple count.
		
		\item Type IV censoring: random intervals. In this situation, we have a mixture of random left and right censoring. That is to say that for some study subjects, we do not know when the event of interest has begun, and for others we do not know when it will occur (if any. ...).
	\end{itemize}
	In the machinery industry, we often deal with the type I censoring: fixed right. In the medical field, in clinical trials, we often deal with a type III censoring: random right. In the case of pandemics, we are dealing with type III censoring: random left.
	
	To introduce this subject, rather than doing obscure theory, as always in this book we prefer at pragmatic approach. Suppose that the study is a clinical trial involving two groups of patients receiving two types of treatments. Two important questions raised to the physicians in a Phase II clinical trial (phase I is for non-toxicity approval to human and phase 0 for animals):
	\begin{enumerate}
		\item Is one of the two treatments more effective than the other in terms of improving patient survival?
		
		\item Can we highlight prognostic factors, that is to say that improve / deteriorate survival?
	\end{enumerate}
	
	To answer the first question we can develop statistical methods that will allow us to compare the two groups of patients who receive both types of treatment.
	
	To answer the second question we propose a model that links patient survival time for explanatory variables and highlight the prognostic factors.
	
	Let us as always assist the theory with an example. For this consider the following table where two cohorts of patients (we move from mechanical engineering to human engineering...) of same initial size with acute leukemia drug test (6-MP ) against a placebo (of course blindly).
	
	We have the following remission times for $21$ patients (the table of $21$ lines therefore indicates the number of weeks for where patient is considered cured after treatment before falling again ill):
	
	The $+$ sign corresponds to patients who left the study for that week. They are therefore censored. For example, the fourth patient was lost of view for any reason after $6$ weeks of treatment with 6-MP: it has therefore has a duration of remission greater than $6$ weeks. So in the study 6-MP, there are $21$ patients and $12$ with censored data.
	
	\begin{tcolorbox}[title=Remark,colframe=black,arc=10pt]
	The theoretical model assumes that censorsing is independent of the survival time (not informative censoring). But if censoring is due to the discontinuation of treatment, the independence assumption is not valid anymore!
	\end{tcolorbox}
	For the placebo group it is simple to make a survival curve. It is sufficient to produce the following table (for the omitted weeks, obviously we impose the number of remission as constants):
	
	So if the data are not censored, the survival $S(t)$ can be estimated by the proportion of individuals surviving at time $t$, that is customary to write under the following mathematical form:
	
	The idea is therefore to estimate:
	
	by the proportion of patients who survived until time $t$.
	
	If the data are censored, the estimated survival function requires specific tools. Kaplan and Meier have proposed in this particular case the following calculation:
	
	Let's see it in a slightly more mathematical form:
	
	With of course:
	
	If we denote by $X(1)\leq X(2)\leq ...\leq X(n)$ the moments (sorted) where an event occurred (death or censored), then we have:
	We estimate:
	
	where $d_k$ is the number of deaths (failures) observed in the time corresponding to the event $X(k)$ and $R_k$ is the number of individuals at risk (at risk of death/failure) just before $X(k)$.
	
	We define the Kaplan-Meier estimator for any $X(0)\leq t <X(k)$ by:
	
	Therefore we get doing now for 6-MP group (not the placebo group!!!!!) the following :
	
	We thus find the same values as those given for example by Minitab Statistical Software 15.1.1 (see the companion book on Minitab for the details).
	
	\subsubsection{Cochran–Mantel–Haenszel tests}
	In statistics, the Cochran–Mantel–Haenszel tests are a collection of test statistics used in the analysis of stratified categorical data. One of these test statistics is the "\NewTerm{Cochran–Mantel–Haenszel (CMH) test}\index{Cochran–Mantel–Haenszel test}", which allows the comparison of two groups on a dichotomous/categorical response.
	
	This test, also named "\NewTerm{log rank test}\index{log rank test}" or "\NewTerm{Cochran-Mantel(-Heanzel) test with stratified time}\index{Cochran-Mantel(-Heanzel) test with stratified time}" or "\NewTerm{Mantel-Cox test}\index{Mantel-Cox test}", or "\NewTerm{Cochran-Mantel-Heanzel Chi-square test}\index{Cochran-Mantel-Heanzel Chi-square test}" or furthermore "\NewTerm{multi-strata proportions test}\index{multi-strata proportions test}" ... has for main objective in practice to test the null hypothesis $H_0$ as what two survival curves (control group versus test group), such as those visible in the figure below (where the surviving population was normalized on the $y$-axis), are significantly different or not under the assumptions that:

	\begin{enumerate}
		\item[H1.] Each stratum(layer) is independent of the previous one

		\item[H2.] For each stratum we are expecting that the expected proportion of survivors / death is the same all things being equal... or "ceteris paribus" for those that prefer Latin (see below if this is not clear).

		\item[H3.] Each stratum is distributed according to the hypergeometric law.

		\item[H4.] The hypergeometric distribution can approximated by a Normal distribution (what we will recall, can only be done under certain conditions!!!).
	\end{enumerate}
	\begin{figure}[H]
		\centering
		\includegraphics{img/arithmetics/survival_typical_survival_curves.jpg}
		\caption[]{Typical survival curves}
	\end{figure}
	In the CMH test, the data are arranged in a series of associated $2\times 2$ contingency tables, the null hypothesis is that the observed response is independent of the treatment used in any $2\times 2$ contingency table. The CMH test's use of associated $2\times 2$ contingency tables increases the ability of the test to detect associations (the power of the test is increased).
	
	In other words, the null hypothesis of the test is that treatment (medical, mechanical or other) has no influence between the control group and the test group in one or multiple stratum, this is written as:
		
	In other words, the null hypothesis $H_0$ of the test is that treatment (medical, mechanical or other) has no influence between the control group and the test group in one or multiple stratum (corresponding to different hospitals, cohorts, laboratories, etc.).
	
	To introduce this test we create a $2\times 2$ contingency table for each stratum, that can correspond to a time interval $[t_i,t_{i+1}]$ where $i=1,2,...,n$ and that can also be assimilated to a different hospital/laboratory for a same clinical/reliability study, which have typically the following structure:
	
	We speak then of "\NewTerm{general stratified $2\times 2\times n$ table}\index{general stratified $2\times 2\times n$ table}" or of "\NewTerm{contingency table in three dimensions}\index{contingency table in three dimension}"...
	\begin{tcolorbox}[title=Remark,colframe=black,arc=10pt]
	Let us recall that if we have a single table and that we only want to compare the proportion of survivors or deaths for both groups, a test for differences in proportions will be applied as seen earlier above. We can also do a chi-square test if we always have only one single table (see also earlier above) and the ad hoc conditions are met or also a Fisher exact test if the conditions of the chi-square test are not met. This is why in good  statistical software (like Minitab for example), these three tests are available next to each other.
	\end{tcolorbox}
	Thus, under the assumption that all other things remaining equal (H2), and this is the core of this test (!!!!), the expected number $E$ of individuals for each cell in the period $i$ will like as for the exact Fisher test  or the Kohen kappa agreement coefficient be equal to\footnote{$L$ states for "Line" and $C$ for "Column"}:
	
	So it must be well understood that, for example, $E_{a,i}$ then represents the expected number of dead individuals from the control group if it would behaved like this of all individuals of the Control + Test groups. So if the two groups (or survival curves) behave identically, the expected value should be equal to the observed value. To be sure that the reader understand, let us illustrate this concept with a specific example:
	\begin{tcolorbox}[colframe=black,colback=white,sharp corners]
	\textbf{{\Large \ding{45}}Example:}\\\\
	We consider the following observed values:
	
	Here, the expected values gives:
	
	\end{tcolorbox}
	We see in the above this particular case, the expected values are equal to the observed one (simply for the reason that the ration $800 / 1'000$ and $1'120 / 1'400$ represent in the table the same percentage (the same proportion ) of $80\%$). Obviously, if the proportion of survivors observed for the two groups are all other things remaining equal, it is the same for the observed dead. So what the reader must understand when we do a MCH test is that we are free to choose the column to analyze (because ultimately it is the same!).
	
	Now we can notice that each of the relationships:
	
	represents in fact the mean of a binomial or hypergeometric distribution (see the proofs earlier above during our study of the corresponding lawas) since the two laws have the same expression for the mean. However, as the size of individuals in the cells could be significant relatively to the total of rows or columns, the drawing could not be independent. Then we must come back on the hypergeometric law (that is to say the third hypothesis enumerated before).
	
	We also see that by symmetry of the above relation, that the variable of interest is the column attribute or line attribute does not change anything in reality the result of the calculation as (for example):
	
	The variance for each cell will be that of the hypergeometric law we have already proved during it study and that was given by:
	
	with for recall $l=n-k$.
	
	If we adapt the notation of the previous relation to that of our table above, this provides us for all cells (by the analytic form of variance of the hypergeometric law the variance has the same expression for each cell!):
	
	with respectively (for those who want to make the analogy with our detailed study of hypergeometric law):
	
	Here again, we see that whether the variable of interest in either line or column, the calculated value of the variance remains the same!!!
	\begin{tcolorbox}[title=Remark,colframe=black,arc=10pt]
	A common question that arise frequently for those studying this test is to know why we can not simply do the sum of all stratum (layers) in a single table (because it would be much easier, of course...)? Well we can not put together the tables into a single one because they do not necessarily follow the same law (not identically distributed in the sense that the hypergeometric law has different settings from one table to another) and unfortunately, the hypergeometric distribution is not stable by the addition.
	\end{tcolorbox}	
	And so what...??? How will this help us to compare if the two survival curves are identical through time (or across different hospitals/labs if this is a clinical/reliability test) ?!

	Well let us take as companion example  the table with the following observed values:
	
	We then get for the expected values:
	
	Thus the differences:
	
	We understand then better why it has no influence to choose a particular cell to make the test. We simply take the one that we will arrange us the more (depending on what will follow...).
	
	So taking randomly (since the choice does not affect - at least for now .... - the test result as we have shown in our previous calculations) the column Survivors and particularly the observed group Test. Its expected value is then:
	
	The difference between observed and expected gives:
	
	That difference will be obviously zero if the observed was equal to the expected! With for variance:
	
	Now, under the conditions:
	
	as seen during our study of the hypergeometric law, we can approximate it by a Normal law (hypothesis H4).

	So in our case, this approximation is not acceptable (the third condition is disqualifying) and so the test can not be performed (usually we use to to take in the table the column and the line that allow this approximation since the choice does not influence on the value of the test but on the autorization to use the mentioned approximation!). But if it had been, we could approximate the hypergeometric distribution by a Normal distribution such that:
	
	Which can be obviously reduced to a centered reduced Normal distribution if we take for the cell of our table:
	
	And therefore it is sufficient for a stratum (layer) $i$ to know if we are outside of the confidence interval that we have set. But ... we have several strata (layers)! The idea then is to summ on the $T$ independent strata (then we fall back on the hypothesis H1) such that:
	
	which was not quite practical at the time when not everyone had a computer since only $\mathcal{N}(0,1)$ tables we available. This is why we prefer to do the following sum:
	
	Or in condensed form (again regardless of the choice of the cell!):
	
	And if the differences between observed and expected is not too great across all strata, the value of this expression well be found within a certain confidence interval of the Normal distribution. If it is outside, the hypothesis of equality of the two survival curves will be rejected.
	
	However, the majority of statistical softwares take the square of that relation.

	It comes then that the square follows a chi-2 law with one degree of freedom (we have proved this earlier in this section) such that we fall back on the final form such that the log-rank test  can be found in many books:
	
	So this is a Wald statistic (see earlier above) and therefore the CMH test belongs to the family of Wald tests.!

	For practical reasons, we add a 0.5 term to the sum, this provides a better approximation to the Normal distribution. Therefore we can see sometimes in books:
	
	\begin{tcolorbox}[colframe=black,colback=white,sharp corners]
	\textbf{{\Large \ding{45}}Example:}\\\\
	Let us consider the following observed values of two hospitals $A$ and $B$ for a clinical trial:
	
	
	We don't want to know if the two hospitals are different or not (this is not the purpose!), but whether the differences between the control group and test group across all hospitals is significantly different or not.
	\end{tcolorbox}
	
	\begin{tcolorbox}[colframe=black,colback=white,sharp corners]
	Well we already intuitively guess the result when we see the values ... but still let us do the calculations.\\

We are already seeing that for Hospital $A$, the column Survivors meets the three conditions for the approximation by a Normal distribution (which is not the case for the column of the Deaths):
	
	Similarly for the hospital $B$ (and this is fortunate because once a column selected in one stratum, it requires that the approximation condition is applicable to the same column of all other strata!!!):
	
	
	
	Which gives for the differences:
	
	
	We then quickly see why once the column chosen, the choice of the line does not matter anymore (either a sum of negative values or a sum of positive values and as we take the square of the sum, it changes nothing finally!). Let us arbitrarily choose the second line. Then we have:
	\end{tcolorbox}
	
	\begin{tcolorbox}[colframe=black,colback=white,sharp corners]
	
	And we have with a spreadsheet software like Microsoft Excel 14.0.6123 in left-sided with a threshold risk of $5\%$:

	\begin{center}
		$\chi_{95\%}^2(1)$\texttt{=CHISQ.INV(95\%,1)}$\cong 3.841$
	\end{center}
	So on the cumul of the two hospitals (strata) the Control group is significantly different from the Test group. The $p$-value of the test is typically given with Microsoft Excel 14.0.6123:
	\begin{center}
		\texttt{=1-CHISQ.DIST(31.845,1,1)}$\cong 0.000002 \%$
	\end{center}
	\end{tcolorbox}
	
	\pagebreak
	\subsection{Propagation of Errors (experimental uncertainty analysis)}
	It is impossible to know (measure) the exact value of a physical quantity experimentally, it is very important therefore to determine its uncertainty.
	
	We obviously name "\NewTerm{error}\index{error}", the difference between the measured value and the exact value. However, since we do not know the exact value, we can not know the error anyway .... The result is still uncertain. That is why we speak of "\NewTerm{measurement uncertainty}\index{measurement uncertainty}".
	
	We distinguish two main types of uncertainty:
	\begin{enumerate}
		\item The "\NewTerm{systematic errors}\index{systematic error}": they affect the result and this constantly in the same direction (errors of measurement devices, accuracy limits, etc.). It must then be eliminate, or correct the outcome, if possible!

		\item The (statistical) "\NewTerm{accidental errors}\index{accidental error}": we must the repeat the measurements and calculate the average estimate uncertainty using statistical techniques.
	\end{enumerate}
	
	The second type of error makes a very big use of all statistical tools we have presented so far. So we will not repeat them here and then we will focus only on a few new concepts.
	
	\subsubsection{Absolute and Relative Uncertainties (Direct  calculation of bias)}
	If the true value of a variable $x$ is (theoretically supposed to be known) and the measured value $x_0$, then $\delta x_0$ is the "\NewTerm{absolute uncertainty}\index{absolute uncertainty}" (uncertainty due to measurement devices) or "\NewTerm{absolute error}\index{absolute error}".

	The interval of fluctuation is therefore denoted by:
	
	or:
	
	The "\NewTerm{uncertainty}\index{uncertainty}" or "\NewTerm{relative error}\index{relative error}" is itself defined by:
	
	The absolute uncertainty is used to find an approximation of the last significant digit thereof. By cons, when we want to compare two measurements having absolute uncertainties to identify which was the largest margin of error, we calculate the relative uncertainty of this number by dividing the absolute uncertainty by the measurement itself, and transform the result typically in percentage.
	
	In other words, the relative uncertainty gives an idea of the accuracy of the measurement in \%. If we make a measurement with an absolute uncertainty of $1$ [mm], we will not know if this is a good measure or not. It depends if we measured the size of a coin, of our neighbor, or of the Paris-Marseille distance of the Earth-Moon distance. In short, it depends on the relative uncertainty (that is to say the ratio of the absolute uncertainty and of the measurement).
	
	Therefore in the case of a law of the type $f(x,y,z,t)$ for which we seek the error we would calculate the following (one time with everything positive and another with all terms negative):
	
	This method is quite boring as it must be calculated for each value $(x,y,z,t)$.
	
	\subsubsection{Statistical Errors}
	In most measurement, we can estimate the error due to random phenomena, named "\NewTerm{random error}\index{random error}", by a series of $n$ measurements $x_1,x_2,...,x_i, ..., x_n$ and this opposed to the "\NewTerm{systematic error}\index{systematic error}" which is the not  random part of the error.
	
	Random error allows to introduce the concepts of:
	\begin{itemize}
		\item Repeatability: This is defined as the closeness of agreement between the results of successive measurements of asame itm, made with the same method, by the same operator with the same measuring instruments, in the same laboratory (conditions), and in rather short time intervals  (see a little further below a little bit more rigorous definition in line with international standards).
		
		\item Reproducibility (sometimes named a "rightness"): which is defined as the closeness of agreement between the results of successive measurements of the same quantity, in the case where individual measurements are made: by different methods, using different instruments extent by different operators in different laboratories!
	\end{itemize}
	These two type of errors are almost always grouped under the labels "\NewTerm{R\&R}" or "\NewTerm{R\&R Study}\index{R\&R Study}" in the industry. In general, the agreement is less good when it comes to reproducibility.
	
	\begin{tcolorbox}[title=Remark,colframe=black,arc=10pt]
	There are softwares running two-way fixed factor ANOVA with repetition as Minitab that generate very detailed reports for R\&R analysis.
	\end{tcolorbox}	
	These two types of errors can be illustrated by the target shooting in a more general way:
	\begin{figure}[H]
		\centering
		\includegraphics{img/arithmetics/type_of_errors.jpg}
		\caption{Type of measurement errors}
	\end{figure}
	As we have seen earlier above, the arithmetic mean value will be in the univariate case:
	
	and the standard deviation (biased estimator as proved earlier) always in the univariate case (the $n$th case has already been proved during our detailed study of variance):
	
	and the unbiased standard deviation (as also proven earlier):
	
	and we have proved that the standard deviation of the mean was given by (under some assumptions!):
	
	and as we have prove it, after a large number of independent measurements, the distribution of errors on a measure follows a Normal distribution so that we can write for the fluctuation interval (if we do not have enough measurement, we then use the fluctuation interval based on the Student law):
	
	In short, we can use all statistical tools seen so far in the field of measurement in laboratories or elsewhere!
	
	The result of a measurement (or estimation and this even in the field of project management!!!) must include rigorously at least 4 elements. For example:
	
	Where we have:
	\begin{enumerate}
		\item The numerical value with the correct number of decimal places.

		\item Unit of measurement according to the international measurement system.

		\item Expanded uncertainty of $k\cdot \sigma$ (fluctuation interval)

		\item The integer value of $k$ used for fluctuation interval.
	\end{enumerate}
	This method is quite more useful than the previous one as we don't need to recalculate it each time for each experimental value. It's this method that for example the famous CERN (European Organization for Nuclear Research) use\footnote{Add to this the Six Sigma measurement methodology studied in the Industrial Engineering section}.
	
	\subsubsection{Repeatability}
	The repeatability $r$, the likely measurement difference between two measurements of similar objects in the same laboratory under similar operating conditions, is normatively defined (in the norms ISO 5725:1987 and AFNOR NF X 06-041) in the one-dimensional (univariate) case by:
	
	where $p$ is a high value probability, usually equal to $95\%$ and $X_1,X_2$ two independent and identically distributed random variables according to a Normal distribution with unkown mean and unknown variance $\mathcal{N}(\mu,\sigma)$. By the stability of the Normal distribution, then it comes:
	
	But, we saw earlier in this sectionin the context of the study of the confidence interval of the mean:
	
	So verbatim:
	
	and therefore using the tables, we have:
	
	and therefore:
	
	Either with the notation respecting the norms for laboratories:
	
	But in the present case, we have a double variance. So it comes:
	
	So we fall back on the relation available in the norms with the famous $2.77$ coefficient. Obviously after it is clear that the value of $r$ should be minimized!
	
	\subsubsection{Error propagation (linearized approximation)}
	Given a measurement $x+\delta x$ and $y=f(x)$ a function of $x$. What is the uncertainty on $y$ it if we know only the uncertainty of a measurement device, but that would not be given as a statistical standard deviation?
	
	In this type of situation we speak of "\NewTerm{indirect measurement}\index{indirect measurement}". This is typically the case if we want to measure an intensity $I$ by measuring it indirectly by making the ratio of the voltage $U$ by the resistance $R$ used for the measurement as $I=U/R$. It is indeed obvious that in the latter situation we can not make the sum of the voltage incertitude and resistance incertitude because the system is therefore not homogeneous at the unit level!!!
	When $\delta x$ is small, $f (x)$ is replaced in the neighborhood of $x$ by its tangent (it is simply the derivative of course!):
	
	but if $y$ depends of several variables $x, z, t$ measured with the uncertainties $\delta x,\delta z,\delta t$:
	
	the maximum possible error is then the exact total differential (\SeeChapter{see section of Differential and Integral Calculus}):
	
	Using first order Taylor expansion series (\SeeChapter{see section Sequences and Series}) we can write:
	
	What we denote often as the sum of the partial derivatives with their respective uncertainty:
	
	and this works very well as long as the increments $\Delta x$ are sufficiently small. Even highly curved functions are nearly linear over a small enough region. The fractional change is then
	
	The latter relation is the "\NewTerm{propagation law}\index{propagation law}" of the studied problem. The partial derivative in factor of the uncertainty is in the science of measurement, named the  "\NewTerm{uncertainty coefficient}\index{uncertainty coefficient}".
	
	For sure we can also write to get the relative error:
	
	
	Which bring us to:
	
	and:
	
	It is thus clear that a mathematical operation cannot improve the uncertainty of the data.
	\begin{tcolorbox}[title=Remark,colframe=black,arc=10pt]
	The result of multiplication, division, subtraction or addition is rounded to as many significant digits that the data that has the smallest number of them.
	\end{tcolorbox}	
	Obviously this propagation law (linear) is valid only in the range where the function can be approximated as linear. So be careful in its use! Otherwise you have to take a Taylor series approximation of higher order.
	
	If the uncertainty of the measurement is given in a statistic form (standard deviation), it is evident therefore that we will use the variance properties already seen at the beginning of this section... at least for simple cases.
	
	Once again the pitfall of this method in comparison to the statistical one is that you have the calculate the error for each measurement...
	
	\subsubsection{Significant Numbers}
	In small schools (and sometimes of higher level and in corporations), it is required to transform a measurement expressed in a certain unit into another unit.
	
	For example, taking the tables, we can have the following conversion:
	\begin{gather*}
		140\; \text{[lb]}=140\cdot 0.45349237\; \text{[kg/lb]}=63.5029318 \; \text{[kg]}
	\end{gather*}
	Then comes the question (that the student or practitionne may have forgotten ...). Starting from a measurement with an accuracy of about $1$ [lb] (therefore of the order of $0.5$ [kg]), could a simple unit conversion lead to an accuracy of $1/10$ [mg]????
	
	From this example it is necessary to retain that a margin of uncertainty is associated with all measured values and any values calculated from measured values.
	
	In the exact sciences (and also soft skills science as management), all reasoning, any analysis must take this uncertainty into account!!!
	
	But why are some digits significant and others not? Because in sciences we only report what has been observed objectively (principle of objectivity). Accordingly, we limit the writing of a number to the digits reasonably reliable figures despite the uncertainty: the significant digits! The accuracy that additional digits could then seem to bring is then illusory.
	
	We must then know rounding according to some rules and conventions:
	\begin{itemize}
		\item When the digit of the highest rank being dropped is greater than $5$, the last digit is increased by $1$ (example: $12.66$ rounds to $12.7$). In the English version of Microsoft Excel 11.8346 this is given with:\\
		
		\begin{center}
		\texttt{=ROUND(12.66,1)=12.7}
		\end{center}
		
		\item When the digit of the highest rank being dropped is less than $5$, the previous digit remains unchanged (example: $12.64$ rounds to $12.6$). In the English version of Microsoft Excel 11.8346 this gives:
		
		\begin{center}
		\texttt{=ROUND(12.64,1)=12.6}
		\end{center}
		
		\item When the digit of the highest rank being dropped is equal to $5$ if one of the digits that follow is not zero, the preceding digit is increased by $1$ (example: $12.6502$ rounds to $12.7$). In the English version of Microsoft Excel 11.8346 this is given with:
		
		\begin{center}
		\texttt{=ROUND(12.6502,1)=12.7}
		\end{center}
		
		\item When the digit of the highest rank we drop is a terminal $5$ (which is not followed by any number!) or is followed only by zeros, we increase of $1$ the previous digit of the number rounded if it is odd, otherwise we leave it unchanged (examples: $12.75$ rounds to $12.8$ and $12.65$ to $12.6$). In the latter case, the last digit of the rounded number will always be an even number. Spreadsheets softwares do not really respect this last rule, actually with the English version of Microsoft Excel 11.8346 we have:
		
		\begin{center}
		\texttt{=ROUND(12.75,1)=12.8\\
		=ROUND(12.65,1)=12.7}
		\end{center}
	\end{itemize}
	In fact, in practice these rules are rarely used because software (spreadsheets softwares manly) do not incorporate appropriate these rules. It is then customary to just to round up to the value of the nearest decimal.
	
	Significant digits of a value include all its digits determined with certainty and the first digit which carries an uncertainty (this latter significant occupies the same rank as the order of magnitude of the uncertainty).
	
	Often, data sources do not mention fluctuation interval (that is to say the indication $\pm \ldots$). For example, when we write $m=25.4\;\text{[kg]}$ we conventionally consider that uncertainty is of the same order of magnitude as the rank of the last significant digit (thus: the uncertain digit).
	
	In fact, only the decimal place of uncertainty is implicit: the real margin is unspecified.
	
	However, additional information about precision can be conveyed through additional notations. It is often useful to know how exact the final digit(s) are. For instance, the accepted value of the unit of elementary charge can properly be expressed as $1.602176487(40)\times 10^{-19}$ [C], which is shorthand for $1.602176487\pm 0.000000040 \times 10^{-19}$ [C].
	
	\subsection{A World without statistics}
	The following essays illustrate the importance of statistics in research and everyday life. To do this we ponder the question of what would things be like in A World (business, job) without statistics\footnote{the critics (bias) of the "\textit{statistics always lie}" statement has already been treated in the Introduction (and the non-constructive arguments of statistics hater as "\textit{your statistics are funny}" or "\textit{I go prepare the meal to my 1.2314 children...}" belongs to the same bias!)}?
	
	Science would be pretty much ok. Newton didn't need statistics for his theories of gravity, motion, and light, nor did Einstein need statistics for the theory of relativity. Thermodynamics and quantum mechanics are fundamentally statistical, but lots of progress could have been made in these areas without statistics. The second law of thermodynamics is an observable fact, ditto the two-slit experiment and various experimental results revealing the nature of the atom. The A-bomb and, almost certainly, the H-bomb, maybe these would never have been invented without statistics, but on balance I think most people would feel that the world would be a better place without these particular scientific developments. Without statistics, we could forget about discovering the Higgs boson etc, but that doesn't seem like such a loss for humanity.
	
	\begin{itemize}
		\item Statistics helped to win World War II, most notably in cracking the Enigma code (\SeeChapter{see section Cryptography})

		\item Statistics helps to assess observational data facts (weather change, new medication approval, new management method, new agricultural method, etc.) be eliminating the human cognitive bias and therefore to get a general consensus more quick than people debating with subjective bias.

		\item Statistics with automated softwares gives the possibility to big business to automate decisions and analysis and therefore to more competitive and sell less expensive services than the competitors do have multiple job to do it manually (\SeeChapter{see section Quantitative Management}).

		\item Statistics help to search optimum in R\&D by reducing the number of experiments that can be huge or very expensive in some industrial fields. Therefore those that use statistics will be more competitive.

		\item Statistics helps to build survey pool in an efficient way to minimize the costs and maximize the reliability of the result (this also include Quality Sampling for lot rejection in the industry!). They help also to calculate the tolerance interval (error) of the resulting survey.

		\item Statistics is used in insurance as they have be proven to be more efficient that human chosen criteria (\SeeChapter{see section Quantitative Management Techniques}).

		\item Statistics are used in fraud detection (extreme value analysis tools) as manually the analysis would be impossible on populations of more than a hundred millions (\SeeChapter{see section Theoretical Computing}).

		\item Statistics are used in modern finance as they are automated with computer (therefore less expensive than a broker) a more reliable on the long term (\SeeChapter{see section Econometry}).

		\item Statistics are used on critical large size project management to estimate margin errors on budgeting and time planning  (\SeeChapter{see section Quantitative Management}). In some projects the customer will not accept a proposal without modern estimations methods (otherwise the supplier will be suspected as being highly amateur).

		\item Statistics is used in Data Mining and Machine Learning to anticipate trends on financial markets or also on social networks behaviors or publicity targeting (\SeeChapter{see section Theoretical Computing}).

		\item Statistics help to predict catastrophic events (earthquakes, tsunami, extreme weather conditions) to evacuate the population and also to optimize the agricultural methods.

		\item Statistics help computer to understand natural language input in search engine or also in typing error correction.

		\item Statistics would have helped to have better modern computer keyboards (as originally they were adapted for typewriters.

		\item Statistics are used to automate the analysis of the stability (capability) of production methods and therefore to be more competitive and sell less expensive products that competitors that do this manually.

		\item Statistics are used in Quantum Mechanics ans Statistical Mechanics helped to the possibility to imagine devices that would not have been discovered so quickly without (LASER, Transistor, etc.)

		\item Statistics are used in genetics in the context of microarray analysis (see the R companion book) and help to do discover faster and develop vaccines sometimes very quickly.

		\item Statistics are used in Neural Netwoks, Deep learning and therefore Artificial Intelligence (\SeeChapter{see section Theoretical Computing}) that will change (it's still starting) radically the future in the next decades.

		\item Statistics are used in some court judgments when the evidence of an unprobable correlation seems to appear (see Lucie de Berk case on Internet for example).

		\item Statistics are used in astronomy to assume the observation of new planets.

		\item Statistics are used for products guarantee demonstrations plans and the calculation of optimal lifetime guarantee of new products that are sell on the mark in huge quantities.

		\item Statistics are used in modern countries to makes decisions based on evidence presented by Economists who cannot convince anyone with anecdotal evidence but with conclusions derived from Econometric models or Mathematical Programming models or variants of both.
		
		\item Statistics are used nation-wide to get information about its people: unemployment, gender proportion, wage distribution, health trends, aging, and so one and it would be impossible to take political decisions and do accurate communication without such statistics in modern countries.
	\end{itemize}
	
	\begin{flushright}
	\begin{tabular}{l c}
	\circled{90} & \pbox{20cm}{\score{3}{5} \\ {\tiny 125 votes, 62.5\%}} 
	\end{tabular} 
	\end{flushright}
	
	
\chapter{Algebra}

	\textit{\textbf{Algebra is the science of calculating the quantities or structures represented by letters.}} (Larousse)
	\minitoc
	%to make section start on odd page
	\newpage
	\thispagestyle{empty}
	\mbox{}
	\section{Calculus}
	\lettrine[lines=4]{\color{BrickRed}I}n the Arithmetic chapter of this book we have written extensively on various theorems using abstract numbers in order to extend the scope of the validity of the latter. However we have only discussed a few on how we should handle these abstract numbers. This is what we will see now.

As you may know already know, a number may be considered by making an abstraction of the nature of the objects that constitutes the group that it characterizes and as well as how to codify it (Indo-Arabic numbers, Roman numbers or other system...). We then say that the number is an "\NewTerm{abstract number}\index{abstract number}" and when we handle these kinds of objects we say that we are doing "algebra calculus" or also "\NewTerm{literal calculation}".

\textbf{Definition (\#\mydef):} "\NewTerm{Literal calculation}\index{Literal calculation}" is the fact to calculate with variables (that is to say with letters) as you would with numbers.

For the mathematicians it is often not advantageous to work with numerical values (1,2,3,...) because they represent only specific cases. What look for physicists, engineers and mathematicians are universally applicable relations in a most general framework as possible.

These abstract numbers today commonly named "\NewTerm{variable}\index{variables}" are often represented by the Latin alphabet (for which the first letters of the Latin alphabet $a, b, c, ...$ often denote imposed values and the last $..., x, y, z$ variables values), Greek alphabet (also much used to represent more or less complex mathematical operators) and the Hebrew alphabet (to a lesser extent).

Although these symbols can represent any number, there are however some as in  physics or mathematics which may represent constants named "\NewTerm{Universal constants}\index{Universal constants}" such as the speed of light $c$, the gravitational constant $G$, the value of $\pi$, the Euler number $e$, etc.

	\begin{tcolorbox}[title=Remark,colframe=black,arc=10pt]
It seems that the letters to represent numbers were used for the first time by Viète in the middle of the 16th century (but the notation of exponents did not exist at this time).
	\end{tcolorbox}	

	A variable is therefore likely to take different numerical values. All these values can vary according to the character of the problem considered. Let us recall (we had already defined this in the section on Numbers of the Arithmetic chapter) that given two numbers $a$ and $b$ such that $a<b$, then:

\begin{enumerate}
	\item[R1.] We name "\NewTerm{domain of definition}\index{domain of definition}" of a variable, all numerical values it is likely to take between two specified limits (endpoints) or on a set (like $\mathbb{N}, \mathbb{R},\mathbb{R}^+,$ etc.).
	
	\item[R2.] We name "\NewTerm{closed interval with endpoints $a$ and $b$}\index{closed interval}", the set of all numbers $x$ between these two values and we denote as example as follows:
	
	
	\item[R3.] We name "\NewTerm{open interval with endpoints $a$ and $b$}\index{open interval}", the set of all numbers $x$ between these two values not included and we denote it as example as follows:
	
	
	\item[R4.] We name "\NewTerm{interval closed, left open right}\index{semi-open interval}" or "\NewTerm{semi-closed left}\index{semi-closed interval}" the following relation as example:
	
	
	\item[R5.] We name "\NewTerm{interval open left, closed right}" or "\NewTerm{semi-closed right}" the following relation as example:
	
\end{enumerate}

	\begin{tcolorbox}[title=Remark,colframe=black,arc=10pt]
If the variable $x$ can take all possible negative and positive values we write therefore: $\left] -\infty,+\infty \right[$ where the symbol "$\infty$" means "infinite". Obviously there can be combinations of open infinite right intervals with left endpoint and vice versa.
	\end{tcolorbox}	

\textbf{Definition (\#\mydef):} We name "\NewTerm{neighborhood of $a$}\index{neighborhood (functional analysis)}", any open interval containing $a$ (it's a simple concept that we will use later to define a continuous function). So:
	
is a neighborhood of $a$.

\subsection{Equations and Inequations}

Elementary algebra consists starting from the definitions of addition, subtraction, multiplication, and power and their properties (associativity, distributivity, commutativity, identity element, inverse, ...) - this is according to which set we are working with a body or a commutative abelian group or not (\SeeChapter{see section Set Theory}) - to handle within a fixed goal "\NewTerm{algebraic equations}\index{algebraic equations}" linking together variables and constants.

We will define afterwards what an equation and an inequality are but first we want to define some of their properties:

Let $A$ and $B$ be any two polynomials (or monomials)  - see definitions a little further - the expressions:
	
satisfy the following properties:
	\begin{enumerate}
		\item[P1.] We can always add or substract from the two members of an equation or inequality same polynomial by obtaining an equivalent equation or inequality (i.e. with the same solutions or reductions). We then say that the equality or inequality remain "true" by the operations of addition or subtraction member to member.
		\item[P2.] If we multiply or if we divide both members of an equation or inequality by the same positive number we also get an equivalent equation or inequality (we have already seen this in the sections befow). We then say that the equality or inequality remains "true" by the operation of multiplication or division member to member.
		\item[P3.] If we multiply or if we divide both sides of an inequality by the same negative number and if we reverse the direction of inequality, then we get an inequality or equivalent equation.
	\end{enumerate}
	
	\subsubsection{Equations}
	\textbf{Definition (\#\mydef):} An "\NewTerm{equation}\index{equation}" is a relation of equality between all abstract values (i.e.: two algebraic expressions) or not all abstract (since we're talking about equations with one unknown, two unknowns, three unknowns and some constants...) interconnected by various operators.

	The perfect mastery of elementary algebra is fundamental in physics and mathematics and in the industry!!! Since there are an infinite number of types of equations, we will not present them all here. It is the role of the teacher/trainer in classes to train the brain of his audience for several years (2-3 years on average) to solve a lot of different configurations of algebraic equations (exposed as every day, purely mathematical problems or geometric problems) and this so that students manipulate these equations without errors in a logical and rigorous reasoning (practice makes perfect...)!!!

	In other words: A teacher/trainer and an ad hoc institution are irreplaceable to acquire knowledge and experience and to have a feedback on experience!!!

	\begin{tcolorbox}[title=Remark,colframe=black,arc=10pt]
We have attempted below to make a simple generalization of the basic rules of elementary algebra. This generalization will be more easy to understand to the reader that already have the habit of manipulating abstract quantities.
	\end{tcolorbox}	

	Thus, either $a, b, c, d, e, ..., x, y$ abstract numbers can take any numerical value (we stay within the high-school classical numbers ...).

	Let $\Xi$ (the Greek capital letter ruling "Xi") representing one or more abstract numbers (variables) operating between them in any way as we have or not different and distinguishable algebraic monomials (one abstract number) or polynomials (poly = many). Therefore we do here a kind of abstract of the abstraction or if you prefer a variable of several variables.

	Properties (in fact these are more examples that properties...):
	\begin{enumerate}
		\item[P1.] We will always have $\Xi=\Xi$ if and only if the term left $\Xi$ of equality is the same term as the one $\Xi$  that is on the right of the equality. If this condition is satisfied then we have:
		
		Otherwise:
		
		where we therefore exclude the case where all the $\Xi$ above are identical to each other (otherwise we get again the property P1).
		\item[P2.] We have if $\Xi\neq 0$:
		
		which verifies the symbolism of the equation $\Xi=\Xi$ in the case only where the elements are identical between them (we obviously exclude the case with zero denominator).
	
		\item[P3.] 	If all the $\Xi$  are identical, then:
				
		Otherwise we have:
		
		which cannot be written in a simple condensed form. It may also happen that:
		
		with the $\Xi$ on the right of equality identical to none, one or more $\Xi$ of the left member of equality.

		\item[P4.] We can have:
		
		without necessarily having the numerator or denominator to be equals (we exclude for sure the case where the denominator is equal to zero).

		Otherwise we can also have:
		
		But don't forget (\SeeChapter{see section Operators}) that in the general case where the numerator and denominator are not equal that:
		
		
		\item[P5.] We have if all denominators are strictly equals:
		
but it is however not impossible to have:
		
with the $\Xi$ at the right of equality identical to any one or more of the left member of the equality or even it is quite possible to have:
		

		\item[P6.] Let represent exclusively the addition or the subtraction symbol we have (at a given sign variation):
		
if all the $\Xi$ are identical to each other or if the combination of an indeterminate number of $\Xi$ are equal to the $\Xi$ on the right of equality.

		Otherwise we will have (meaning that the result will give any undetermined monomial or polynomial):
		

		\item[P7.] We have if all $\Xi$ (based) raised at the power are strictly identical:
		
	\end{enumerate}
	if and only if the bases $\Xi$ are equal (or could be decomposable to be equal) and the powers $\Xi$ are not  necessarily equals.

	From the knowledge of these basic rules/properties 7 examples, we can solve, simplify or show that a simple equation has solutions or not relatively to a given  problem or statement.

	Thus, given an operand or a sequence of any operations on one or more abstraction of abstracts and among all , one (or more) of which the numeric values is or are unknown (the others are known). So we could be able to prove that a relation (statement) like:
	
	has existing solutions or not (that is to say: is True or False).
In the case of an equation with the absolute value (\SeeChapter{see section Arithmetic Operators}) of the type:
	
	with the second member being strictly positive (otherwise the previous relations would be a nonsense) this is equivalent  of course from the definition of absolute value to write:
	

	\begin{tcolorbox}[title=Remark,colframe=black,arc=10pt]
	\textbf{R1.} The presence of the absolute value in an algebraic equation in which we seek solutions often doubles the number of solutions.\\
	
	\textbf{R2.} An equation is named "\NewTerm{conditional equation}\index{condition equation}" when there are numbers in the set of definitions that are not solutions (which is actually the most common case). Conversely, if any number of the definition set is solution of the equation then the equation is named an "\NewTerm{identity equation}\index{identity equation}".
	\end{tcolorbox}	
	
	We sometimes have to solve a "\NewTerm{system of equations}\index{system of equations}".
	\begin{itemize}
		\item What is it? It is a set of at least two equations to solve (solving is not always equivalent as to simplify an expression!).

		\item What is the specificity of such a system? All solutions of the system are the intersection of all the solutions of the equations to be solved (for detailed examples see the sections of Linear Algebra and Numerical Methods). 

		\item What is the usage? It is endless (see the different chapters of the book), these systems solve problems involving applications of mathematics to other fields (finance, engineering, operational research, etc.).
	\end{itemize}		
	Because of the unlimited variety of applications, it is difficult to establish precise general rules for solutions (see section of Theoretical Computing) and the existing procedures to follow will for sure be useful for sure only if the problem can be formulated into equations and the following procedures can help a little bit at least to avoid some errors:
	\begin{enumerate}
		\item If we have a problem statement already written, we read it several times carefully and we consider the given facts and the amount of unknowns to find and their domain of definition (summarizing statement on a sheet of paper or anywhere else is often useful for large problems!).
		\item Select letters that represents the unknown quantities. This is one of the decisive steps in the search for solutions. Sentences containing words such as: find, what, how, where, when should help you to identify the unknown quantity.
		\item When make a drawing (in your head or on paper) with captions. This is for sure possible most of time only with problems of 1, 2 or 3 unknowns.
		\item List the known facts and relations about the unknown quantities. A relation can be described by an equation in which appear in one or both sides of the equal sign statements written with normal sentences.
		\item After the previous step, write one or more equations that accurately describe what was stated with sentences.
		
		\item Solve the equation or the system of equations formulated in the previous steps using of the multiple heuristic existing techniques.
		\item Check the solutions obtained in the previous step by referencing to the initial problem statement (check that the solution is consistent with the conditions of the statement).
	\end{enumerate}
	Some methods of resolutions of systems of equations are treated in detail in the section of Theoretical Computing and also in the section Linear Algebra (you will thus therefore better understand the procedure above).
	 
	 \subsubsection{Inequations}
	 Previously we have seen that an equation was composed from an equality of various calculations with different terms (with at least one  "unknown" or an "abstract number") and that:
 	\begin{enumerate}
 	  \item "Solve an equation" is to process consisting to calculate the  value of the unknown to satisfy the equality (when a solution exists!)
 	  
 	  \item "Simplify and equation" is the process consisting to mathematically minimize the number of terms (factoring or eliminiate...)
 	  
 	  \item "Develop an equation" is the process consisting to flatten all the terms.
 	\end{enumerate}
	Why do we need to recall the definition of an equation? Just because for the inequality, it is almost the same intellectual process! The difference ? If the equation is an equality, inequality is an inequality (...): As the equation, the inequality is composed of various calculations with different terms interconnected by any operators with at least one unknown.
	
	Main differences between equality and inequality equations:
	
	\begin{enumerate}
		\item Equality: Symbolized by the symbol $=$
		\item Inequality: Symbolized by the strict or large order relations: $<, \leq, \geq, >$
	\end{enumerate}
		
	When we solve an inequality, our unknown may have a range of values that satisfy the inequality. We say then that the inequality of the solution is a "\NewTerm{set of values}\index{set of values}". This is the fundamental difference between equality (\underline{several} solutions) and inequality (\underline{range} of solutions)!
	
	Let us make a refresh about the signs that we can meet in an inequality:
	\begin{itemize}
		\item $<$: Must be read "\NewTerm{strictly inferior to}\index{strictly inferior symbol}" or "\NewTerm{strictly less than}\index{strictly less than}". In this case the target numerical value is not included in the range and we can then represent the range (interval) with an open left square bracket ] ... or right square bracket ... [ next to the target value.
	\begin{tcolorbox}[colframe=black,colback=white,sharp corners]
	\textbf{{\Large \ding{45}}Example:}\\
	\begin{flushleft}
	Writing $x<5$ means $x \in ]-\infty,5[$ and $x<-5$ means $x \in ]-\infty,-5[$ .
	\end{flushleft}
	\end{tcolorbox}
	\item $>$: Must be "\NewTerm{strictly superior to}\index{strictly superior to}" or "\NewTerm{strictly greater than}\index{strictly greater than}". In this case the numerical target value is also not included in the range (interval) and we can then represent the range with an open left square bracket ] ... or right square bracket ...[  next to the target value.
	\begin{tcolorbox}[colframe=black,colback=white,sharp corners]
\textbf{{\Large \ding{45}}Example:}\\
	\begin{flushleft}
	Writing $x>5$ means $x \in ]5,+\infty[$ and $x>-5$ means $x \in ]-5,+\infty[$ .
	\end{flushleft}
	\end{tcolorbox}
	\item $\leq$: Must be read "\NewTerm{inferior or equal to}\index{symbol inferior or equal to}" or "\NewTerm{less than or equal to}\index{symbol less than or equal to}". In this case, the numerical target value is in the range (interval) and we can then represent the range  with a closed left square bracket [... or right square bracket...].
	\begin{tcolorbox}[colframe=black,colback=white,sharp corners]
\textbf{{\Large \ding{45}}Example:}\\
	\begin{flushleft}
	Writing $x\leq 5$ means $x \in ]-\infty,5]$ and $x<-5$ means $x \in ]-\infty,-5]$ .
	\end{flushleft}
	\end{tcolorbox}
	item $\geq$: Must be read "\NewTerm{superior or equal to}\index{symbol superior or equal to}" or "\NewTerm{greater than or equal to}\index{symbol greater than or equal to}". In this case, the numerical target value is in the range (interval) and we can then represent the range  with a closed left square bracket [... or right square bracket...].
	\begin{tcolorbox}[colframe=black,colback=white,sharp corners]
\textbf{{\Large \ding{45}}Example:}\\
	\begin{flushleft}
	Writing $x\geq 5$ means $x \in [5,+\infty[$ and $x>-5$ means $x \in [-5,+\infty[$ .
	\end{flushleft}
	\end{tcolorbox}
	\end{itemize}
	The objective of inequalities is most of the time (except aesthetics purpose) to have at least one numeric value that defines the domain of solution (of all the abstract terms - variables - of the inequality) that satisfies the inequality.
	
	There are many ways to represent the domains of definition of variables which satisfy an inequality. We will see through small examples what are some of the  most common possibilities:
	\begin{tcolorbox}[colframe=black,colback=white,sharp corners]
\textbf{{\Large \ding{45}}Example:}\\
	Given a linear inequality (of first degree) on $x$ with a single unknown on which we impose an arbitrary particular constraint for example (of course, the expression can contain more terms...):
	
	we have in the above inequation already simplified all the terms that were superfluous.
	
	Solve the inequality is like looking for the $x$ values less than $2$. Of course, there is not only one solution in $\mathbb{R}$ but a set (interval) of solutions and this is the principle of inequalities!\\
	
	To solve the inequality, we first observe the type of inequality imposed ("strict" or "equal"). Then, in high-schools classes (and not only sometimes ...) we represent the set $\mathbb{R}$ traditionally by a table such as:
	
	We intuitively know that the solution of our inequality includes all values below $2$ ($2$ being itself excluded from the solutions) until $-\infty$. Then we write this interval or domain as follows:
	
	Then we can represent in a tabular way the set of solutions (it helps to understand and prepares the student for solving systems of equations and inequalities and to the study of functions variations). For this, we take back the template of the previous table and place in it our target value (we only have one in this special example but sometimes there can be several because there is a singularity or roots for certain values of the definition domain), that is to say the value $2$:
	
	and finally, we delimitate with color (...) the set of solutions to from $-\infty$ to $+2$ excluded:
	
	At the value $2$, we do not forget to mark the sign $....[$ to show that this value is excluded from the solutions. And that's it, and the concept can be extrapolated to much more complex inequalities.
	\end{tcolorbox}
	\begin{tcolorbox}[title=Remark,colframe=black,arc=10pt]
	\textbf{R1.} Sometimes instead of representing tables as we have done above, some teachers (this is a completely artistic choice) ask their students to shade the boxes of the table and to draw small circles inside..., or they also use small arrows, or draw the graph of the inequality (the latter method is certainly aesthetic but takes time in complex case but an example is given below...).\\
	
	\textbf{R2.} As part of inequalities of degree greater than 1, it is necessary (see later what that means exactly) first too determine the roots of the inequality that determine the intervals and then by trial and error, determine which intervals are to reject or keep.
	\end{tcolorbox}
	We can also (just like equations) sometimes have to solve a "system of inequalities". What is it ?: It is a set of at least two inequalities to solve. The peculiarity of the system ?: The set of solutions of the system is the intersection of all the solutions of every inequality.
	
	For example the system of three inequalities:
	
	In other words, the method is the same as the previous one, with the difference that our table (representing the areas of solutions) will include an additional line for each additional inequality in the system plus one line of synthesis which is the projection of the possible solutions areas of the system.
	
	Thus, a system of inequalities will have obviously a summary table with $n+1$ lines.
	
	Mathematically, the areas (there may be several which are disjoint) can be written as a set of domains:
	
	Systems of inequalities are very common in many problems of mathematics, physics, econometrics, etc. It is important to practice solve them during your studies with the help of your teacher.
	
	For example, here is a possible representation of the solution domain of the previous system of inequalities:
	\begin{center}
	\begin{tikzpicture}

    \draw[gray!50, thin, step=0.5] (-1,-3) grid (5,4);
    \draw[very thick,->] (-1,0) -- (5.2,0) node[right] {$x_1$};
    \draw[very thick,->] (0,-3) -- (0,4.2) node[above] {$x_2$};

    \foreach \x in {-1,...,5} \draw (\x,0.05) -- (\x,-0.05) node[below] {\tiny\x};
    \foreach \y in {-3,...,4} \draw (-0.05,\y) -- (0.05,\y) node[right] {\tiny\y};

    \fill[blue!50!cyan,opacity=0.3] (8/3,1/3) -- (1,2) -- (13/3,11/3) -- cycle;

    \draw (-1,4) -- node[below,sloped] {\tiny$x_1+x_2\geq3$} (5,-2);
    \draw (1,-3) -- (3,1) -- node[below left,sloped] {\tiny$2x_1-x_2\leq5$} (4.5,4);
    \draw (-1,1) -- node[above,sloped] {\tiny$-x_1+2x_2\leq3$} (5,4);

	\end{tikzpicture}
	\end{center}
	
	\subsection{Remarkable Identities}
	The remarkable identities are some kind of magical relations that we use most often for factoring or solving algebraic equations in all fields relative to Applied Mathematics. They have an important place and must be absolutely be mastered by the reader.
	
	Let us recall some notions that have already study in the section of Set Theory and of the chapter Arithmetic (we assume the concept of neutral element known as already defined):
	\begin{itemize}
		\item Commutativity:
		
		
		\item Associativity:
		
		
		\item Distributivity:
		
	\end{itemize}
	Similar properties are valid with the subtraction operation of course with the adequate domain of definition.
	
	\pagebreak
	We can check with numerical values (by replacing each abstract number a randomly chosen number), or by development (it would be better, so you are sure you have understood what we were talking about until now), that the following algebraic identities are satisfied (they are the most known one):
	\begin{enumerate}
		\item Second degree identity::
		
		\item Third degree identity::
		
	\end{enumerate}
	\begin{tcolorbox}[title=Remark,colframe=black,arc=10pt]
	We can very well put that in practice that $(a+(c+d))^2:=(a+b)^2$ where we obviously putted that $b:=c+d$ (we do an "abstract of the abstraction" or more commonly: a "change of variable"). And therefore:
	
	\end{tcolorbox}
	We can thus notice that in generality, to calculate the development $(a+b)^n$, we use the development of $(a+b)^{n-1}$, that is to say with the identity calculated with the previous value of $n$.
	
	We can also see a pattern if we put all together:
	
	We thus notice the following properties for $a$ and $b$:
	\begin{enumerate}
		\item The powers of $a$ decreasing from $n$ to $0$ ($a^0=1$, so it is not noted in the last term).
		
		\item The powers of $b$ increase from $0$ to $n$ ($a^0=1$, so it is not noted in the first term).
		
		\item In each term, the sum of the powers of $a$ and $b$ is equal to $n$.
		
		\item The multiplier coefficients in front of each term are calculated by summing the multipliers of two terms of development achieved with the previous value of $n$ (see figure below).
	\end{enumerate}
	The so named "\NewTerm{binomial coefficients}\index{Binomial coefficient}" can then be obtained by the construction of the "\NewTerm{Pascal triangle}\index{Pascal triangle}" below:
	\begin{figure}[H]
		\centering
		\includegraphics{img/algebra/pascal_triangle.jpg}
		\caption{Pascal Triangle}
	\end{figure}
	Where each element is given by (\SeeChapter{see chapter Probability}):
	
	with $n,p\in \mathbb{N}^{+}$.
	\begin{theorem}
	We can then prove that:
	
	which is the famous "\NewTerm{Newton binomial}\index{Newton binomial}" (that we will reuse many times in different chapters and thus sections of this book) or also named "\NewTerm{Binomial theorem}\index{Binomial theorem}".
	\end{theorem}
	\begin{dem}
	This relation can be proved quite simply by induction assuming true the previous relation and calculating it for the rank $1$:
	
	This relation can be simply proved by induction assuming true the previous relation and for calculating for the rank $1$:
	
	The relation is true for rank $n + 1$, so it is true for any $n$.
	\begin{flushright}
		$\square$  Q.E.D.
	\end{flushright}
	\end{dem} 
	Regarding remarkable identities with negative values, there is no need to memorize the location of the sign "$-$". Just make a change of variable and once the development is made we replace the variable back (inverse change of variable).
	\begin{tcolorbox}[colframe=black,colback=white,sharp corners]
	\textbf{{\Large \ding{45}}Example:}\\
	
	and so on for all finite power $n$.
	\end{tcolorbox}
	We can of course mix genders (...) such as (particularly famous example):
	
	and some remarkable additional practical relation that are often used in small classes for exercises:
	
	\begin{tcolorbox}[title=Remark,colframe=black,arc=10pt]
	When, from the right-hand side (in a simplified numerical form) the teacher asks his students as an exercise to get the factoring of the left of the equality, there is no other way but to proceed by successive tests.
	\end{tcolorbox}
	For information we also get following famous development that is immediately deducible from what we have seen before:
	
	which is valid for any value of $b$ and is named the "\NewTerm{binomial expansion}\index{binomial expansion}". 
	
	For small $b$ we could neglect all terms involving $b^2$ or higher powerf of $b$, giving us the approximation relation:
	
	of $b\ll 1$.
	
	Of course there is still a lot of more useful relations with monomial (from which the biggest part is only a generalization of those presented above) that the reader will discover by his own reasoning and according to his practice and in this book through the different chapters.
	\begin{tcolorbox}[title=Remark,colframe=black,arc=10pt]
	It is of course possible to multiply polynomials between them and distribute the multiplicative terms. Conversely, it is often asked to students in small classes to do the reverse procedure ("factoring" or "decompose" a polynomial) so they get used to the handling of remarkable identities. Decomposed into a product of factors is an important operation in mathematics, since it is thus possible to reduce the study of complicated expressions to the study of several simpler expressions having interesting properties depending on the context.
	\end{tcolorbox}
	
	\subsection{Polynomials}
	\textbf{Definition (naive version \#\mydef):} We name "\NewTerm{algebraic univariate polynomial $P (x)$}\index{Algebraic univariate polynomial}" a function of degree $n\in \mathbb{N}$ which is written:
	
	or in a more condensed manner by:
	
	 where the "\NewTerm{dominant factor}\index{dominant factor}" of a polynomial, also named "\NewTerm{leading coefficient}\index{dominant factor}", is the coefficient of the monomial of highest degree $a_n$ and the "\NewTerm{leading term}\index{leading term}" is simply $a_nx^n$.
	 
	\begin{tcolorbox}[title=Remarks,colframe=black,arc=10pt]
	\textbf{R1.} The subscript $n$ of $P (x)$ is most of time omitted as explicitly defined in the expression.\\
	
	\textbf{R2.} The reader who has read the section of Set Theory, probably remember that the set of all polynomial of degree $n$ or lower form a vector space structure!
	\end{tcolorbox}
	\textbf{Definition (set theory version \#\mydef):} Let $k$ be a ring (\SeeChapter{see section Set Theory}) and $n\in \mathbb{N}^{*}$, the "\NewTerm{polynomial ring}\index{polynomial ring}" in $n$ indeterminate (or variables) $k[X_1,...,X_n]$ is constructed from an elementary polynomial, named "\NewTerm{monomial}\index{monomial}" of the form:
	
	where $\lambda \in k$ is the "\NewTerm{coefficient of the monomial}\index{coefficient of the monomial}", $m_1,m_2,...,m_n$ are positive non-null integers and where $X_1^{m_1}...X_m^{m_n}$ forms the "\NewTerm{literal part of the monomial}\index{literal part of a monomial}". Thus, by construction, a polynomial is a sum of a finite number of monomials named the "\NewTerm{terms of the polynomial}\index{terms of a polynomial}".
	
	Therefore, the common special case used in small classes and presented at the beginning is $k[X]$, that is to say the ring of univariate polynoms with coefficients in $k$. Indeed most of time engineers and students deals with "ring of univariate polynoms with coefficient in $\mathbb{R}$ and denoted by $\mathbb{R}[X]$. Any element of $k[X]$ is therefore written as:
	
	with $a_i\in k$ (most of time  $a_i\in \mathbb{R}$ ), $i=0...n$ and $n\in \mathbb{N}$.
	\begin{tcolorbox}[title=Remarks,colframe=black,arc=10pt]
	\textbf{R1.} Notice that the powers $i$ are always positive or null in $k[X]$!!!\\
	
	\textbf{R2.} We say that two polynomes are "similar" if they have the same litteral part.
	\end{tcolorbox}
	\begin{figure}[H]
		\centering
		\includegraphics{img/algebra/polynomials.jpg}
		\caption{Some polynomials plotted with R.3.2.1 (see my book on R)}
	\end{figure}
	The "\NewTerm{limiting behavior}\index{limiting behavior}" of a function describes what happens to the function as $x\rightarrow \pm\infty$. The degree of a polynomial and the sign of its leading coefficient dictates its limiting behavior. In particular:
	\begin{itemize}
		\item If the degree of a polynomial $f(x)$ is even and the leading coefficient is positive, then $f(x)\rightarrow +\infty$ as $x\rightarrow \pm\infty$.
		\item If $f(x)$ is an even degree polynomial with negative leading coefficient, then $f(x)\rightarrow -\infty$ as $x\rightarrow \pm\infty$. 
		\item If $f(x)$ is an odd degree polynomial with positive leading coefficient, then $f(x)\rightarrow -\infty$ as $x\rightarrow +\infty$ and $f(x)\rightarrow +\infty$ as $x\rightarrow +\infty$.
 
		\item If $f(x)$ is an odd degree polynomial with negative leading coefficient, then $f(x)\rightarrow +\infty$ as $x\rightarrow -\infty$ and $f(x)\rightarrow -\infty$ as $x\rightarrow +\infty$.
	\end{itemize}
	These results are summarized in the table below.
	
	
	\textbf{Definition (\#\mydef):} We name "\NewTerm{turning point}\index{turning point}" a point at which the graph changes direction from increasing to decreasing or decreasing to increasing.
	\begin{figure}[H]
		\centering
		\includegraphics[scale=0.5]{img/algebra/turning_points.jpg}
		\caption{Turning point explicitly highlighted}
	\end{figure}
	The function $f$ above is a $4^{\text{th}}$ degree polynomial and has $3$ turning points. Thex maximum number of turning points of a polynomial function is always one less than the degree of the function.
	
	\textbf{Definition (\#\mydef):} We name "\NewTerm{root}\index{root of a polynomial}" or "\NewTerm{zero of (univariate) polynomial}\index{zero of a univariate polynomial}", the $x$ values such as the "\NewTerm{polynomial equation}\index{polynomial equation}" $P(x)=0$ is satisfied at the condition that at least one of the $a_n$ with $n>0$ is not null.
	
	If the polynomial admits one or more roots $r_n$ we can then obviously factorize it as (we will prove it more rigorously further below):
	
	so that when $x$ takes the value of one of the roots $r_n$, the expression above is zero. This is what we name by convention "\NewTerm{factorize a polynomial}\index{factorize a polynomial}".
	
	Algebraic identities are particular forms of polynomial functions. Indeed, consider a constant $c$ and a variable $x$ and:
	
	We see that if we put:
	
	we fall back on:
	
	\textbf{Definitions (\#\mydef):}
	\begin{enumerate}
		\item[D1.] A polynomial in one indeterminate is named an "\NewTerm{univariate polynomial}\index{univariate polynomial}", a polynomial in more than one indeterminate is named a "\NewTerm{multivariate polynomial}\index{multivariate polynomial}". A polynomial with two indeterminates is named a "\NewTerm{bivariate polynomial}\index{bivariate polynomial}".
		
		A famous example in pure mathematics of a multivariate polynomial given many times in undergraduates course is: 
		
	
		\item[D2.] In the case of polynomials in more than one indeterminate, a polynomial is named  "\NewTerm{homogeneous of degree $n$}\index{homogeneous polynomial of degree $n$} if all its non-zero terms have degree $n$ (the example just above is such a polynomial!).
	\end{enumerate}

	\pagebreak
	\subsubsection{Euclidean Division of Polynomials}
	Let us put ourselves now in the ring $k[X]$. If $P(x)\in k[X]$, we denote by $\text{deg}(P)$ the degree of the polynomial $P(X)$ with coefficients in a ring $k$ (real or complex ... whatever!)
	\begin{tcolorbox}[title=Remark,colframe=black,arc=10pt]
	By convention:
	
	\end{tcolorbox}
	\begin{theorem}
	Given:
	
	with $k,m>0$. Then there are two unique polynomial $q(X),r(X)\in k[X]$ such as:
	
	and:
	
	where $q(X)$ is the "\NewTerm{quotient polynomial}\index{quotient polynomial}" and $r(X)$ the "\NewTerm{residual polynomial}\index{residual polynomial}".
	\end{theorem}
	\begin{dem}
	If $u (X) = 0$ the result is obvious. Let us suppose that $u(X)\neq 0$ and let us prove the existence by induction on the degree $k$ of $u (X)$.
	
	If $k = 0$ then $q (X) = 0$ (since $m>0$) and therefore $r (X) = u (X)$ will do the job.
	
	Now let us suppose the statement for any $k\leq n$... (this supposition is completely free of charge...):
	
	Let $u (X)$ be of degree $k=n1$. If $m>n+1$ then $q (X) = 0$ and $r (X) = u (X)$ can also do the job.
	
	Otherwise, if $m\leq n+1$then by writing ($u_{n+1}$ is the $n+1$-th coefficient of the polynomial $u(X)$ and $v_m$ the $m$-th coefficient of $v(X)$:
	
	we reduce then $ u (X) $ to a polynomial of degree $ \ leq n $ since $v (X)$ is of degree $m$ (and that it exists)!
	
	Indeed, the term:
	
	removed (at least) the term of highest degree $u_{n+1}X^{n+1}$.
	
	By the induction hypothesis, there are $f(X)$ and $g(X)$ such as:
	
	with $\text{deg}(g)<m$. So after rearranging::
	
	and therefore:
	
	do the job!
	
	So by induction we see that the Euclidean division exists in the polynomial ring $k[X]$.
	\begin{flushright}
		$\square$  Q.E.D.
	\end{flushright}
	\end{dem}
	\begin{tcolorbox}[title=Remark,colframe=black,arc=10pt]
	This proved allowed us in the section of Set Theory to show that this ring is "principal".
	\end{tcolorbox}
	\begin{tcolorbox}[colframe=black,colback=white,sharp corners]
	\textbf{{\Large \ding{45}}Example:}\\\\
	We will seel only one example as the idea is always same. We want to divide $x^3+x^2$ by $x-1$ we get:
	\begin{equation}
		\renewcommand{\arraystretch}{1.2}
		\renewcommand{\arraycolsep}{2pt}
		  \begin{array}{rrrr|rrr}
		 x^3&+x^2 &   &\cdot(-1)&x  &-1 &  \\
		\cline{5-7}
		-x^3&+x^2 &   &  &x^2&+2x&+2\\
		\cline{1-2}
		    &2x^2 &   &  &   &   &  \\
		    &-2x^2&+2x&  &   &   &  \\
		    \cline{2-3}
		    &     &2x &  &   &   &  \\
		    &     &-2x&+2&   &   &  \\
		              \cline{3-4}
		    &     &   &+1&   &   &  \\ 
		  \end{array}
	\end{equation}	
	\end{tcolorbox}
	
	
	\pagebreak
	\subsubsection{Factorization Theorem of Polynomials}
	We will now prove an important theorem that is in fact originally illustrated (among others) by the remarkable identities we saw above:
	
	
	\begin{theorem}
	If a polynomial $P(X)\in k[K]$ with coefficients in $k$ of degree $n\geq 1$ has a root $x=r$ in the ring $k$, then we can factorize $P(x)$ by $(x - r)$ such that:
	
	where $Q$ is a polynomial of degree $n-1$ (and therefore can be a simple monomial).
	
	In other words, "\NewTerm{factorize a polynomial}\index{factorization of polynomial}", it is written it as a product of monomials (in the general case: of polynomial). When not only applied to polynomials or also simply to numbers, factorization is an operation that transforms a sum into a product!
	\end{theorem}
	\begin{dem}
	The idea is to perform the Euclidean division of $P(x)$ by $(x-r)$. According to the previous theorem, there exists a pair $(Q, R)$ of polynomials such as:
	
and according to the result of the previous theorem on the Euclidean division:
	
	But $\text{deg}(x-r)=1$, so $\text{deg}(R)=0$ (or $-\infty$ by convention). $R(x)$ is therefore a constant polynomial function. Moreover, by hypothesis, $r$ is a root of $P(x)$. We have therefore:
	
	So $R(r)=$. Therefore $R(x)$ is the zero polynomial and the theorem is practically proved. It remains to prove that $\text{deg}(Q)=n-1$, which is an immediate consequence of the relation:
	
	Hence:
	
	\begin{flushright}
		$\square$  Q.E.D.
	\end{flushright}
	\end{dem}
	From this property to factorize a polynomial, named sometimes "\NewTerm{factorization theorem}\index{factorization theorem}", we can give a foretaste of a much more important theorem:
	\begin{theorem}
	Let us show that if we have a polynomial  function $P(X)\in k[X]$ of degree $n\in \mathbb{N}$ with coefficients in $k$, then it has at most a finite number $n$ of roots (some being possibly confused) in $k$.
	\end{theorem}

	\begin{dem}
	First, because $P(x)$ has a degree (order), $P(x)$ is not a zero polynomial function. Then, let us argue by the absurd:
	
	If the function $P(x)$ has $n$ roots with $p>n$ (more roots than degree....), by noting these roots $r_1,...,r_p$, we have, by the previous  factorization theorem (applied $p$ times):
	
	where $Q$ is a polynomial of degree:
	
	Now, since by definition a polynomial is a polynomial if and only if its degree (order) belongs to  $\mathbb{N}$, the polynomial $Q$ must be the zero polynomial such that:
	
	It follows that:
	
	This contradicts the initial hypothesis that $P$ is not the zero polynomial, hence:
	
	\begin{flushright}
		$\square$  Q.E.D.
	\end{flushright}
	\end{dem}
	
	\subsubsection{Diophantine equation}
	If we generalize the concept of univariate polynomial with several variables such as:
	
	then we call "\NewTerm{Diophantine equation}\index{Diophantine equation}" an equation of the form:
	
	where $P$ is a polynomial with integer (or rational) coefficients for which seeks the radicals (roots) strictly in $\mathbb{N}$ or $\mathbb{Q}$. Conventional Diophantine equations are for examples:
	\begin{itemize}
		\item  The linear Diophantine equations:
		
		
		\item The Pythagorean triples:
			
	\end{itemize}
	For the general proof of the latter, the reader will have to wait a little bit the time for the as authors of this book to have time understand to understand the proof (...).
	
	\subsubsection{First order univariate Polynomial and Equations}
	Given the linear function:
	
	If $a\neq 0$ the this first the equation:
	
	has a simple closed formed root given obviously by:
	
	such that $P_1(r)=0$.
	If $b=0$ this polynomial is named an "\NewTerm{affine function}\index{affine function}".
	
	\begin{tcolorbox}[title=Remarks,colframe=black,arc=10pt]
	\textbf{R1.} If the coefficients of the univariate polynomial of defree $1$ are all such that  $a,b\in \mathbb{R}$ then the root also belongs to $\mathbb{R}$.\\
	
	\textbf{R2.} If one of coefficients of the univariate polynomial of defree $1$ belongs to $\mathbb{C}$ then the root also belongs to $\mathbb{C}$.\\
	
	\textbf{R3.} If the both coefficients of the univariate polynomial of defree $1$ belongs to $\mathbb{C}$ then the root also belongs to $\mathbb{C}$ or $\mathbb{R}$.\\
	
	\textbf{R4.} We say that two polynomial equations are "\NewTerm{equivalent}\index{equivalent polynomials}" if the admit the same solutions.
	\end{tcolorbox}
	Here are also some properties for univariate first order polynomials that we give without proof as they seem very very intuitive to us (except on reader request):
	\begin{enumerate}
		\item[P1.] If we add (or respectively subtract) a same number to each member of the question (left from the "$=$" sign, and also right), we get an equation that has the same solutions as the original equation (and this whatever is its degree).
	
		\item[P2.] If we multiply (or respectively divide) each me member of an equation (left from the "$=$" sign, and also right) by a same non-null number, we get an equation that has the same solutions as the original equation (and this whatever is its degree).
	\end{enumerate}
	The method should be general enough to by applied to all equations of the same kind, be build on the four arithmetic basis operations (addition, subtraction, multiplication and division) and the extraction of roots. We we can find the solutions (roots), of an equation thanks to its coefficients, using only the previous operations (that is to say in a "\NewTerm{closed form}\index{closed form}"), we then say that the equation can be solved by "\NewTerm{radicals}\index{radicals}".
	
	\subsubsection{Second order univariate Polynomial and Equations}
	Given the univariate polynomial with coefficient in $\mathbb{R}$ (trinomial of second degree):
	
	If we represent this univariate polynomial on the plan, this give us:
	\begin{figure}[H]
		\centering
		\includegraphics{img/algebra/polynomial_orientation.jpg}
		\caption{Typical orientation for second degree polynomials}
	\end{figure}
	If we derivative this function (\SeeChapter{see section Differential and Integral Calculus}) and we search in what point the derivative is equal to zero, we will always found the optimum on the inflection point of the parabola (which corresponds also to its symmetry axis):
	\begin{figure}[H]
		\centering
		\includegraphics{img/algebra/polynomial_optimum.jpg}
		\caption{Inflexion point of the tangent?}
	\end{figure}
	If $a\neq 0$, then we have:
	
	We then have a "\NewTerm{double root}\index{double root}" (or "\NewTerm{root of multiplicity $2$}") that we denote by:
	
	such that $P(r_{1,2})=0$ and we define a new term named sometimes the "\NewTerm{determinant of the polynomial}" and most commonly the "\NewTerm{discriminant of the polynomial}\index{discriminant of the polynomial}":
		
	Finally:
	
	If the second order univariate polynomial on $x$ has two root, we can then factorize it in an irreductible form (following the factorization theorem proved earlier) in the following form:
	
	We also prove we easily from the expression of the root by doing simple algebra the "\NewTerm{Vieta relations}\index{Vieta relations}" (on request of the readers we can detail the developments in necessary):
	
	
	Depending on the sign of $2$ and this of the discriminant $\Delta$, we have:
	\begin{figure}[H]
		\centering
		\includegraphics{img/algebra/polynomial_second_order_signature.jpg}
		\caption{Inflexion point of the tangent?}
	\end{figure}
	Therefore:
	\begin{itemize}
		\item If $\Delta<0$ our polynomial has no real roots and cannot be factorized in a multiplication of monomial with real ($
\mathbb{R}$) factor but with complex one ($\mathbb{C}$). Therefore (it is recommended to have read first the part about Complex Numbers in the section Numbers of this book):
		
		and we know that we can write any complex number in a condensed form (Euler formula) and as the complex roots of a polynomial of the second degree are conjugate (we already know this jargon) we have:
		
		where (recall) $r$ is the module of the complex roots (module that is equal for the both) and $\varphi$ the argument of the complex roots (equal in absolute value).
		
		\item If $\Delta=0$ the polynomial equation then has a single solution that is obviously:
		
		
		\item If $\Delta>0$ the polynomial equation then has two solutions defined by the general relations which we have already given above:
		
	\end{itemize}
	About the complex case, let us take as an example the following quadratic polynomial:
	
	that admits only two complex roots that are $\mathrm{i}$ and $-\mathrm{i}$. In the real plane this polynomial will be represented with Maple 4.00b by:
	
	\texttt{>plot(x\string^2+1,x=-5..5);}
	\begin{figure}[H]
		\centering
		\includegraphics{img/algebra/polynomial_complex_solutions_in_real_plane.jpg}
		\caption[]{Plot example of a polynomial of degree $2$ which admits only complex solutions}
	\end{figure}
	where we see well that there is no real solutions (zeros). While  placing us in the complex $\mathbb{C}$, we have:
	
	\texttt{>plot3d(abs((re+I*im)\string^2+1),re=-2..2,im=0..2,view=[-2..2,-2..2,0..2],\\
	orientation=[-130,70],contours=50,style=PATCHCONTOUR,axes=frame,\\
	grid=[100,100],numpoints=10000);}
	\begin{figure}[H]
		\centering
		\includegraphics{img/algebra/polynomial_complex_solutions_in_complex_plane.jpg}
		\caption[]{The same polynomial but playing with the complex representation}
	\end{figure}
	where the two zeros are visible on the imaginary axis at $-1$ and $+1$. Obviously when it's the first time we see a function shown in a figure taking into account the complex values we try to find where is the corresponding parabola of the purely real case. To do this, we simply cut the surface above on two on the imaginary axis and we get then:
	
	\texttt{>plot3d(abs((re+I*im)\string^2+1),re=-2..2,im=0..2,view=[-2..2,-2..2,0..2],\\
	orientation=[-130,70],contours=50,style=PATCHCONTOUR,axes=frame,\\
	grid=[100,100],numpoints=10000);}
	\begin{figure}[H]
		\centering
		\includegraphics{img/algebra/polynomial_complex_solutions_in_complex_plane_cutted.jpg}
		\caption[]{A little zoom on the same polynomial}
	\end{figure}
	where we find again clearly our parabola visible on the cutted surface. So we can ask ourselves whether the complex numbers are a natural extension of our conventional space beyond our physical senses and our common measuring devices...!

	\paragraph{Irrational Equations}\mbox{}\\\\
	The practitioner must always take the habit to verify the solution in the original equation to be sure of the validation of the definition domain of the function. Indeed, there are solutions to the resolution of equations that do not satisfy the original equation and this is what we name "\NewTerm{extraneous solutions}\index{polynomial extraneous solution}" and this is typically the case of irrational equations. 

	\textbf{Definition  (\#\mydef):} An "\NewTerm{irrational equation}\index{irrational equation}" is an equation where the unknown is under a radical (that is to say in typical case: under a square root).
	
	\begin{tcolorbox}[colframe=black,colback=white,sharp corners]
	\textbf{{\Large \ding{45}}Example:}\\\\
	Consider the following equation
	
	To solve it first we can redistribute:
	
	With take the power of $2$:
	
	We simplify little bit:
	
	again:
	
	We take the power of $2$ again:
	
	We simplify a last time:
	
	We get two trivial solution that are $x_1=-2$ and $x_2=2$. But only the solution $x_1=-2$ satisfy the proposed equation. Indeed, if you put the first solution in the original equation we get:
	
	But if we put the second one:
		
	\end{tcolorbox}
	
	\pagebreak
	\paragraph{Gold Number}\mbox{}\\\\
	There is a second order univariate polynomial whose solution is famous around the world. The solution is a value named the "\NewTerm{golden ratio}\index{golden ratio}" or "\NewTerm{Divine proportion}\index{divine proportion}" (...) and is found in architectural, aesthetic or in phyllotaxis (that is to say, in the arrangement of the leaves around the stem of the plants).
	
	This number is:
	
	The golden ratio is also an algebraic number (any complex number that is a root of a non-zero polynomial in one variable with rational coefficients) and even an algebraic integer (complex number that is a root of some monic polynomial with coefficients in $\mathbb{Z}$) as it is solution of:
	
	There is a very elegant way to bring out this polynomial of using the $\mathcal{Z}$ transform and that we will discuss in the section of Functional Analysis.
	
	\subsubsection{Third order univariate Polynomial and Equations}
	Even if rare to solve in theoretical physics or in engineering, solving an univariate polynomial of the $3$rd degree is quite recreational and shows a good example of an already mature mathematical reasoning (we have these developments thanks Scipione del Ferro and Jerome Cardan mathematicians of the 16th century...).
	
	Given the equation:
	
	with the coefficients in $\mathbb{R}$ (to begin...). In a first time, the reader will be able to see that the reasoning that we have applied for the polynomial a smaller order than $3$ stuck when the order is greater (excepted for special simple cases obviously...).

	We will avoid the problem by using a change of variables subtle but quite justified.
	
	Thus, nothing prevents us from putting that:
	
	and that by dividing the polynomial of degree 3 by $a$ to write:
	
	By grouping the terms of the same order:
	
	and let us write (nothing, but really nothing prevent us to do this):
	
	where (1) est known if and only if $X$ is known and where $p,q$ are anyway unknowns.

	The polynomial\footnote{The first time a had to solve such a polynomial was to calculate the nutation of a gyroscope and the second time was for the calculation of the horizon of a Black Hole based on the Schwarzschild metric with cosmological constant that was in natural units: $\dfrac{\Lambda}{3}r^3-r-2M=0$}:
	
	being of odd degree, it admits (as it can be seen from any visual plot of such a polynomial with real coefficients) at least one real root, named the "\NewTerm{certain root}\index{certain root}" and at maximum three roots. The reader will easily check by himself that with a graphical representation of an odd degree polynomial this is trivial!
	
	Now let us make another subtle change of variable (we have the right to do this):
	
	by imposing the condition that $u, v$ must be such that $3uv=-p$ (nothing prevents us from imposing such a constraint), then we have:
	
	Therefore we have:
	
	We can very well make an analogy between the two equations (1') and (2') and the Vieta relations we had obtained for the polynomial of degree 2 which we recall were:
	
	excepted that we have now (we adopt another notation for these intermediate roots):
	
	which gives us for the polynomial $P$ by imposing (always by analogy) $a=1$ a new equation:
	
	for which $z_1,z_2$ are the roots.
	The latter equation has for discriminant:
	
	Let us take now scenario by scenario:
	\begin{enumerate}
		\item If $\Delta >0$, the equation on $Z$ admits two solutions $z_1,z_2$ whose sum will give us indirectly the value of $X$ since by definition $X=u+v$ and $z_1=u^3$ and $z_2=v^3$. We see that we have all the ingredients to find the first root of the original equation that will be the certain root (or "\NewTerm{certain zero}\index{certain zero}"). So:
		
		as $\Delta>0$ and the superior roots are cubic we necessarily have $X_1\in \mathbb{R}$ if all coefficient of the equation are well in $\mathbb{R}$.
		
		\item If $\Delta=0$, we know it, the equation on $Z$ admit a double root and as the discriminant has a square power of $q$ this means necessarily that $p$ is negative.

		The polynomial $P$ therefore also has a double root and the same for the original equation. We saw also that for a second degree polynomial if the discriminant is zero roots are:
		
		Then by analogy:
		
		
		\item If $\Delta<0$ we must again use complex numbers as we did in our study of the polynomial of degree $2$. Thus, we know that the equation $Z$ admits two complex solutions such as:
		
		and once again as the roots are conjugated we can write in condensed form:
		
		and as:
		
		we therefore have:
		
		As $u_k,v_k$ are conjugated, we have necessarly $X_k\in \mathbb{R}$.
	\end{enumerate}
	
	\pagebreak
	\begin{tcolorbox}[colframe=black,colback=white,sharp corners]
	\textbf{{\Large \ding{45}}Example:}\\\\
	Consider the following equation:
	
	We therefore have:
	
	and therefore:
	
	We therefore have:
	
	\end{tcolorbox}
	The polynomials of degree three are therefore well solvable by radicals.
	
	\subsubsection{Fourth order univariate Polynomial and Equations}
	The univariate polynomial equation to solve here is:
	
	with $a\neq 0$.
	\begin{tcolorbox}[title=Remark,colframe=black,arc=10pt]
	We own the method of resolution of $4$th degree polynomial to the Italian mathematician  Ludovico Ferrari (also) of the 16th century.
	\end{tcolorbox}
	If we divide by $a$ we have:
	
	And by putting:
	
	the equation will be reduce to:
	
	where we see that the coefficient in front of the $y^3$ vanishes. Thus, any polynomial of the type:
	
	can be written in the following form:
	
	By putting:
	
	\begin{tcolorbox}[title=Remark,colframe=black,arc=10pt]
	If $d''=0$ the equation to solve is reality a "\NewTerm{bisquare equation}\index{bisquare equation}". The change of variable $X=y^2$ then allows to focus to a polynomial equation of the second degree (what we know is easily to solve).
	\end{tcolorbox}
	We now introduce a parameter $t$ (that we will choose wisely afterwards) and we rewrite the polynomial equation as follows:
	
	\begin{tcolorbox}[title=Remark,colframe=black,arc=10pt]
	If the reader develop and distribute all the terme of the previous relation he will fall back obviously on $x^4+c''x^2+d''x+e''=0$.
	\end{tcolorbox}
	The underlying idea is to try to ensure that the bracketed part of the previous expression can be written as a square as:
	
	Because in this case, using:
	
	Our polynomial equation can then be written:
	
	and we would have only to solve two polynomial equations of the second degree (what we know already how to do).
	
	But for us to write:
	
	the expression of second degree to the left of equality should have only one root. But we saw in our study of polynomial equations of the second degree that meant since the discriminant is zero:
	
	and that the root was given by:
	
	Which corresponds in our case to:
	
	and therefore that:
	
	with:
	
	So finally, if $t$ is such that $4(2t-c'')(t^2-e'')={d''}^2$, then we have:
	
	as the fundamental polynomial theorem give us for a polynomial of the second degree a unique root:
	
	To conclude, it is enoughe to find a number $t$ satisfying the following relation:
	
	which is a degree $3$ polynomial that we already know how to solve using the Cardan method.

	Such general methods doesn't exist anymore for polynomial of degree higher or equal to $5$ as we will prove during our study of Galois Theory (\SeeChapter{see section Set Algebra}).
	
	\subsubsection{Trigonometric Polynomials}
	\textbf{Definition (\#\mydef):} We name "\NewTerm{trigonometric polynomial of degree $N$}\index{trigonometric polynomial}" any finite sum of the type:
	
	where $c_n\in \mathbb{C}$.

	A trigonometric polynomial can also be written using the usual trigonometric functions with the following changes:
	
	Either using Euler's formula (\SeeChapter{see section Numbers}):
	
	What we can also rewrite as:
	
	By putting:
	
	It comes:
	
	We will extensively discussed in the section Sequences and Series how to use these polynomials in the context of the study of Fourier series.
	
	\subsubsection{Cyclotomic Polynomials}
	If $n$ is an integer (belongs to $\mathbb{N}$) and $x$ a complex number (belongs to $\mathbb{C}$), we name "\NewTerm{cyclotomic polynomial}\index{cyclotomotic polynomial}" that we denote traditionally by $\Phi_n$ and that we define as being the product of all monomials:
	
	where $\alpha$ is a $n$th primitive root of $\mathbb{C}$. In other words:
	
	To recall an $n$th unit root  (sometimes named "\NewTerm{Moivre number}\index{Moivre number}") is a complex number whose $n$-th power is equal to $1$.
	
	Thus, the set of all $n$-th unit roots is given by:
	
	which is a cyclic group (see the section Set Theory and also the section Set Algebra).

	Then we name "\NewTerm{$n$-th primitive root of unity}\index{primitive root of unity}" any element of this group generating it.
	
	The elements of $G_n$ are of the type:
	
	with $k\in \mathbb{Z}$. We write then the set of the $G_n$ in the form:
	
	A small example of cyclotomic polynomial (more example will be given below):
	
	with:
	
	which are therefore the $4$th roots of unity (in other words: each of these number set to the power of $4$ is equal to $1$). They form the group $G_4$ and this can be generated only by $\mathrm{i}$ and $-\mathrm{i}$ (field generator according to what was seen in the section Set Theory).
	
	So a cyclotomic polynomial is the product of factors that is written:
	
	with $k\in \{0,\ldots,n-1\}$.
	
	We will see with the examples below that if $n$ is even then:
	
	and if $n$ is odd:
	
	
	\pagebreak
	\begin{tcolorbox}[colframe=black,colback=white,sharp corners]
	\textbf{{\Large \ding{45}}Example:}\\\\
	For $n$ up to $30$, the cyclotomic polynomials are:
	
	\end{tcolorbox}
	
	\subsubsection{Legendre Polynomials}
	\textbf{Definition (\#\mydef):} Legendre polynomials are define by (it is strongly recommended to read the sections of Differential and Integral Calculus and also of Functional Analysis before continuing):
	
	where $P_n$ is therefore a polynomial of degree $n$. We will see again these polynomials in the resolution of differential equations in physics (heat propagation, quantum physics, quantum chemistry, etc.). In most book the Legendre polynomial are written in the following equivalent form:
	
	We will focus here only and uniquely on the properties that will are used actually in the other physic sections of this book!!!
	
	Let us prove that following definition of functional scalar product (see the sections Functional Analysis and Calculation Vector) Legendre polynomials are orthogonal. This is a very important property for our study of Quantum Chemistry!
	
	\begin{dem}
	Let $P$ be a polynomial of degree $\leq n-1$. It suffices to prove that $\langle P_n | P \rangle =0$, that is to say that $P_n$ is orthogonal to the space of polynomials of degree less than $n$. Indeed, we have:
	
	integrating by parts we get:
	
	Caution!!! For the above zero term, only the term $(1-x^2)^n$ is derived there! So since $x$ is squared, whatever the derivative the value will always be the same. What justifies this term is equal to zero.
	
	Continuing in this way we get after $n$ integration by parts:
	
	\begin{tcolorbox}[title=Remark,colframe=black,arc=10pt]
	The derivative term is zero, since the derivative polynomial is of degree $n-1$.
	\end{tcolorbox}
	\begin{flushright}
		$\square$  Q.E.D.
	\end{flushright}
	\end{dem}	
	Here are some useful properties for the section of Quantum Chemistry of the Legendre polynomials:
	\begin{enumerate}
		\item[P1.] We have $P_n(1)=1$:
		\begin{dem}
		
		and using the Leibniz Formula (\SeeChapter{see section Differential and Integral Calculus}) we have:
		
		Therefore:
		
		\begin{flushright}
			$\square$  Q.E.D.
		\end{flushright}
		\end{dem}

		\item[P2.] We have $P_n(-x)=P_n(-x)$ if $n$ is even:
		\begin{dem}
		If $n$ is even:
		
		is an even function and therefore:
		
		is even.
		\begin{flushright}
			$\square$  Q.E.D.
		\end{flushright}
		\end{dem}

		\item[P3.] We have $P_n(-x)=-P_n(x)$ if $n$ is odd:
		\begin{dem}
		If $n$ is odd:
		
		is an odd function and therefore:
		
		is odd.
		\begin{flushright}
			$\square$  Q.E.D.
		\end{flushright}
		\end{dem}
	\end{enumerate}
	\begin{theorem}
	We will now prove the validity of the following recurrence relation for the $P_n$ (relations that we use in physics):
	
	for $n \geq 1$.
	\end{theorem}
	\begin{dem}
	$xP_n(x)$ is a polynomial of degree $n+1$, there exists therefore $a_j\in \mathbb{R}$ such that this polynomial can be expressed as a linear combination of the family of polynomials constituting the orthonormal basis (basis that enables us to generate/build the $xP_n(x)$):
	
	Therefore we can write:
	
	but if we choose $k\leq n-2$ (because $xP_k$ is therefore of degree $n-1$):
	
	Therefore:
	
	that is to say that $a_k=0$. Than it follow:
	
	By the properties of the Legendre polynomials proved previously, we can write the equalities:
	
	and:
	
	hence:
	
	The dominant coefficient of $P_n$ denoted by $\text{dom}(P_n(x))$ is defined (for recall) as the coefficient of the monomial of the highest degree. Thus, it is given by:
	
	Therefore:
	
	\begin{tcolorbox}[title=Remark,colframe=black,arc=10pt]
	The reader will verify if needed for a given $n$ that:
	
	\end{tcolorbox}
	The relation:
	
	we got earlier impose us that the dominant coefficient of the polynomial of the linear combination to be equal to the dominant coefficient of the polynomial $xP_n$ (we have eliminated the $(-1)^n$ which simplifies):
	
	After simplification we get:
	
	and which finally gives easily:
	
	The relation:
	
	becomes therefore:
		
	\begin{flushright}
		$\square$  Q.E.D.
	\end{flushright}
	\end{dem}
	The first Legendre polynomials are:
	

	The graphs of these polynomials (up to $n = 5$) are shown below:
	\begin{figure}[H]
		\centering
		\includegraphics{img/algebra/legendre_polynomials.jpg}
		\caption{Five first Legendre Polynomials (source: Wikipedia)}
	\end{figure}
		

	\begin{flushright}
	\begin{tabular}{l c}
	\circled{90} & \pbox{20cm}{\score{3}{5} \\ {\tiny 70 votes,  56.29\%}} 
	\end{tabular} 
	\end{flushright}
	
	%to make section start on odd page
	\newpage
	\thispagestyle{empty}
	\mbox{}
	\section{Set Algebra}
	\lettrine[lines=4]{\color{BrickRed}W}e will approach in this book the study of set structures very pragmatically (since you must remember that this book is dedicated to engineers). Thus, it will be made use of the minimum of formalism and only the proofs of the elements that we consider as absolutely essential to the engineer will be presented in this. Moreover, numerous proofs will be made by example and we will focus largely on the algebraic theory of groups as it has a prominent place in physics almost more than for other set-structures.

	\subsection{Groups Algebra and Geometry}
	The symmetries of geometric figures, of crystals and all the other items of macroscopic physics, are subject for centuries of observations and studies. In modern terms, the symmetries of a given object form a "group".
	
	Since the mid-19th century, the group theory took a huge extension, and its applications to quantum mechanics and the theory of elementary particles have developed throughout the 20th century.
	
	In a letter of 1877 to the mathematician Adolf Meyer, Sophus Lie wrote that he created the theory of groups in January 1873. It is of the groups he named "\NewTerm{continuous groups}\index{continuous groups}" and that are named today "\NewTerm{Lie groups}\index{Lie groups}". Lie sought to expand the use of groups from the domain of algebraic equations, where Galois had introduced them, to the differential equations.
	
	Since 1871, the notion of infinitesimal generator of a one-parameter group of transformations appeared in his work. This is the set of infinitesimal generators of the subgroups with one parameter of a continuous group that forms what we name today a "\NewTerm{Lie algebra}\index{Lie algebra}".
	
	It was Weyl and Wigner that show the preeminent role of the theory of groups and their representations in particular in the new quantum mechanics that Heisenberg  and Dirac were developing. The general idea of representation theory is try to study a group by doing it acting on a vector space in a linear way: we try to see the group as as a group of matrices (hence the term "representation "). We can, from relatively well-known properties of the automorphism group of the vector space (\SeeChapter{see section Set Theory}), get to deduce some group properties of main interest.
	
	We can consider the theory of group representations as a vast generalization of the Fourier analysis. Its development is continuous and has, since the mid-20th century, countless applications in differential geometry, ergodic theory, probability theory, number theory, the theory of automorphic forms, in that of dynamic systems as well as in physics, chemistry, molecular biology and signal processing. Currently, entire branches of mathematics and physics depend on it.
	
	Before we begin, we refer the reader to the section on Set Theory to get a refresh of the structure and fundamental properties that make up a Group and also to the section of Linear Algebra (because we use some results proved in it).
	
	\subsubsection{Cyclic Groups}
	The cyclic group (whose definition of has already been given in the section of Set Theory) will serve us a basis for the study of finite groups. Moreover, rather than making generalized expansions we preferred to take specific examples to present the idea of cyclic group (more suitable for an engineering approach).
	
	We will take then the very nice example of the hours of a clock... with three different approaches that will successively (!) and simple address the concept of a cyclic group.
	\begin{enumerate}
		\item First approach:
		
		Let us imagine a clock with a needle which can take $12$ possible positions (but no intermediate positions). We will denote in a special way the $12$ possible positions: $\overline{0},\overline{1},\overline{2},...,\overline{11}$  (the line above the numbers is not innocent!).
		
		Nothing prevents us on all of these positions to define an addition, for example:
		
	  	which is similar to the results we get when we in our daily life we do calculations manipulating time.
	  	
	  	\item Second approach (first extension):
	  	
	  	If we observe well a Watch or a Clock, we notice that every time we add $12$ (or withdraw $12$...) to a value of hours of our Watch/Clock then we fall back on the set of numbers that are well defined that are also in $\mathbb{Z}$. Therefore (obviously as part of a Watch/Clock only the first positive values are of interest to us most of the time but here we do math so we generalize a bit...)
	  	
		Here we fall back on a concept we had already seen in the section of Numbers Theory. This are congruence classes and all of these classes form the quotient set $\mathbb{Z}/12\mathbb{Z}$. If we endow this quotient set of an addition law, it is normally easy to observe that it is an internal law to the quotient set, that it is associative, that there exist a neutral element and that every element has an inverse (opposite).
		
		Thus, this quotient set equipped only with the addition law (if adding the multiplication we can form a ring) is a commutative group.

		\item Third approach (second and last extension):
		
		Let us see a third and final approach which is why the quotient group is cyclic.
		
		If we project the rotation of the hands of our Watch/Clock (all rotations in the set-algebra are traditionally clockwise!) in $\mathbb{C}$ and that we define:
		
		We then have $x^{12}=x^0=1$ and:
		
		which is why the quotient group $(\mathbb{Z}/12\mathbb{Z},+)$ is named "\NewTerm{cyclic group}\index{cyclic group}" (by group isomorphism according to what has been seen in the section Set Theory). Its isomorphic is noted by $C_{12}$ and all elements are of modulus $1$. It is common to denote all complex number of modulus $1$ as following:
		
		If we represent in $\mathbb{C}$ the isomorphic set $C_{12}$ we get then on the unit circle a polygon with $n$ vertices as shown in the figure below:
		\begin{figure}[H]
			\centering
			\includegraphics{img/algebra/c_12_cyclic_group.jpg}
			\caption{$C_{12}$ Cyclic Group}
		\end{figure}
		Furthermore, the number of component elements of $\mathbb{Z}/12\mathbb{Z}$ being finite, $(\mathbb{Z}/12\mathbb{Z},+)$ is finite. Unlike the group $(\mathbb{Z},+)$ which is itself a discrete infinite group.
		
		This concept of finitude is perhaps most obvious with the example that we will do afterwards with $\mathbb{Z}/4\mathbb{Z}$ where the reader will observe that this set has the same number of elements than $C_4$.	
	\end{enumerate}
	
	\begin{tcolorbox}[title=Remark,colframe=black,arc=10pt]
	Mathematicians name $C_n$ the "group of $n$-th roots of unity". An $n$-th root of the unit (sometimes named "\NewTerm{Moivre's number}\index{Moivre's number}") is a complex number whose $n$-th power is equal to $1$. Moreover, for a given integer $n$, all $n$-th roots of unit are located on the unit circle and are the vertices of a regular $n$-sided polygon having an apex of affix $1$.
	\end{tcolorbox}
	What interests particularly physicists at first are the representations of finite groups (also the continuous groups that we will see later). Thus, the representative of $\mathbb{Z}/n\mathbb{Z}$ is know to us as the rotation in the complex plane is given as we have shown during our study of complex numbers in the section Numbers:
	
	with $k\in[0,n-1]$. This representative is a subgroup of the group of rotations $\text{O}(2)$ which will be discussed further. The group of rotations of the plane itself being a subgroup of the linear group $\text{GL}(2)$ (we will give a precise definition and an example further below).
	
	In fact, mathematicians are able to prove that all quotients groups $\mathbb{Z}/n\mathbb{Z}$ are cyclic to an isomorphism with $C_n$ and then they say that $\mathbb{Z}/n\mathbb{Z}$ is a finite quotient of the  monogenic group $\mathbb{Z}$...
	
	This approach is perhaps a bit abstract for the Padawan... So if the reader remembers the section Set Theory we saw a very precise definition of what is the cyclicity of a group: A group $G$ is said to cyclic if $G$ is generated by the power of at least one of its elements $a\in G$ named the "generator" such that:
	
	Let us check whether this is the case for the group:
	
	which is a school case.
	
	We will denote the elements that make up this group:
	
	This being done, we should be careful that in the set definition of a cyclic group we speak of "power" if the internal law of the group is multiplication but if the internal law is the addition, then we have:
	
	The first generator of the group $G=\left\lbrace \mathbb{Z}/4\mathbb{Z},+ \right\rbrace$ is $1$. Indeed:
	
	The second generating element of the same group is $3$:
	
	For cons, the reader can check that $2$ is not a generator of this group!
	
	In fact, regarding the groups $G=\left\lbrace \mathbb{Z}/n\mathbb{Z},+ \right\rbrace$ mathematicians are able to prove that only the elements of the group items that are prime with $n$ are generators (that is to say elements whose greatest common divisor is $1$).
	
	That's all four our introduction to cyclic groups for engineers. Let us turn now to another group category.
	
	\subsubsection{Transformations Groups}
	The transformations groups, that include the rotations groups, are the one that most interests physicists especially in the areas of continuum mechanics, chemistry, quantum physics and art ... Mathematicians appreciate obviously the study of rotations groups in the context of geometry (but not only!) and computer scientists equally linear groups. We have also seen an example of a rotations group just before.
	
	\textbf{Definition (\#\mydef):} We name "\NewTerm{linear group of order $n$}\index{linear group of order $n$}" and we note $\text{GL}(n)$ the invertible matrices $n\times n$ or also named "\NewTerm{regular matrices}\index{regular matrices}" (i.e. their determinant is not zero according to what we saw in the section Linear Algebra) which component are in any field or ring $K$ (the ring $\mathbb{R}$ or the grup $\mathbb{C}$ most of time):
	
	he group is so named because the columns of an invertible matrix are linearly independent.
	
	We will consider here as obvious that $\text{GL}(n)$ is a group: the matrix multiplication is associative array and each matrix of $\text{GL} (n)$ has an inverse by definition (as the determinant is not null). On the other hand, the product of two regular matrices gives a regular matrix that is again invertible.
	
	A simple and important example of linear group is that of the sub-"\NewTerm{group of affine transformations}\index{group of affine transformations}" of the plane that is traditionally noted (that is intuitive):
	
	with $a,b,c,d,\alpha,\beta\in \mathbb{R},ad-bc\neq 0$ (we will see why and how this latter inequality a little further below).
	
	\textbf{Definition (\#\mydef):} The "\NewTerm{affine group}\index{affine group}" or "\NewTerm{general affine group}\index{general affine group}" of any affine space $A$ over a field $K$ is the group denoted $\text{Aff}(A)$ or $\text{Aff}(n,K)$ of all invertible affine transformations from the space into itself. It is a "\NewTerm{Lie group}\index{Lie group}" if $K$ is the real or complex field or quaternions.
	
	Let us take a small practical example:
	
	which would apply to a circle gives:
	\begin{figure}[H]
			\centering
			\includegraphics{img/algebra/affine_group_transformation_example.jpg}
			\caption{Simple affine transformations on a circle}
		\end{figure}
	This transformation is a way to define the ellipses as images of a circle by an affine transformation.
	
	The coefficients $\alpha,\beta$ are irrelevant for the shape of the image figure. In fact, they obviously induce only translations to the figures. So we can do without them if we only seek to distort the original figure.
	
	Therefore it remains:
	
	which can be written in matrix form:
	
	The affine transformation is therefore reduced to the matrix:
	
	and as we have seen in the section of Linear Algebra, matrix multiplication is associative but is not commutative, so the linear transformation is not either.
	
	The neutral element is the matrix:
	
	and the inverse of of $F$ is:
	
	and as we have imposed $ad-bc\neq 0$ any element has therefore an inverse. Thus, the linear affine group is not commutative and... is therefore a group...
	
	As we will see it, all the "classics" Lie groups are subgroups of $\text{GL} (n)$.
	
	\textbf{Definition (\#\mydef):} We name "\NewTerm{special linear group of order $n$}\index{special linear group of order $n$}" and denote by $\text{SL} (n)$ the invertible (square $n\times n$) matrices with coefficients in an arbitrary field and with determinant equal to unity:
	
	This is obviously a subgroup of $\text{GL} (n)$.
	
	Returning to the previous example and remembering that the determinant of a square two-dimensional  matrix is (\SeeChapter{see section Linear Algebra}):
	
	we notice geometrically well what it means to have a unit determinant in this case! Indeed we will see in the section of Linear Algebra during our geometric interpretation of the determinant that having a determinant is equivalent to have a surface. Thus, having $ad-bc$ equal to the unit allows us regardless of the processing order, to have the area equal to $1$. Thus, the special linear is a group of transformation that keeps surfaces value.
	
	\textbf{Definition (\#\mydef):} We name "\NewTerm{orthogonal real group of order $n$}\index{orthogonal real group of order $n$}" and denote by $\text{O} (n)$ the invertible orthogonal (square $n\times n$) matrices (see the section Linear Algebra for a refresh of what are orthogonal matrices):
	
	Furthermore, we proved in the section Linear Algebra in our study rotation matrices that $A^TA=I_n$ implies that $\det(A)=\pm 1$.
	
	This is the case for example of the $\text{O}(2)$ matrix seen previously (it belongs to the orthogonal group but also to the group of rotations that we will see further below):
	
	which is orthogonal as it is easy to check (just multiple by the transposed to see if you get and identity matrix).
	\begin{tcolorbox}[title=Remark,colframe=black,arc=10pt]
	$\text{O}(1)$ is  made up of all trivial matrices... $[1] [-1]$ that are simply one component vectors or just... simple scalars.
	\end{tcolorbox}
	\textbf{Definition (\#\mydef):} If $A\in \text{O}(n)$ and that we have $\det(A)=1$ then we get a subgroup of $\text{O}(n)$ named "\NewTerm{special real group orthogonal of order $n$}\index{special real group orthogonal of order $n$}" and therefore define by:
	
	The rotation matrix given previously is part of this group since its determinant is equal to unity! Furthermore, this group occupies a very special place in physics and we will meet it again during our study of quantum physics.
	
	The subgroup $\text{SO}(2)$, also sometimes named "\NewTerm{circle group}\index{circle group}" and denoted $S^1$ that we had also studied in the section of Euclidean Geometry has a representative given by the matrix:
	
	and occupies a special place in the family of the $\text{SO}(n)$ groups  with $n$ greater than unity. Indeed it is the only one that is commutative. Moreover, it is isomorphic to $e^{\mathrm{i}\theta}$ either $\text{U}(1)$ the multiplicative group of complex numbers of modulus $1$. This is also the proper symmetry group of a circle and the equivalent continuous of $C_n$.
	
	The subgroup $\text{SO}(3)$ is given by the matrix (\SeeChapter{see section Euclidean Geometry}):
	
	for the rotation around the $x$-axis in the three-dimensional space is not commutative (the rotation matrices in the plane being commutative for recall!). Moreover the quaternions that we have seen in the section Numbers, whose representative is therefore $\text{SO}(3)$, form also a non-commutative group (relatively to the multiplication law) as we have seen in the section Numbers.
	
	Compared to a unit vector makes we can relatively easily visually speaking see that $\text{SO} (3)$ is a closed subgroup of $\text{GL} (3)$, that is to say, the set of linear groups of dimension $3$.
	
	\begin{tcolorbox}[title=Remark,colframe=black,arc=10pt]
	$\text{SO} (1)$ consists in the matrix $[1]$.
	\end{tcolorbox}	
	\textbf{Definition (\#\mydef):} We name "\NewTerm{linear group of order $n$}\index{linear group of order $n$}" and we denote it $\text{U}(n)$ the matrices whose components are complex (as part of this book the most often) or real and which are orthogonal:
	
	Notice also that any unitary matrix with complex components and of one dimension  (thus that belongs to $\text{U}(n)$...) is a complex number of unit module, which can always be written in the form $e^{\mathrm{i}\mathbb{R}}$.
	We have already seen an example in this book during our study of spinors (see section of the same name). These are the Pauli matrices (used in the section of Relativistic Quantum Physics) given by:
	
	\textbf{Definition (\#\mydef):} We name "\NewTerm{special unitary group of order $n$}\index{special unitary group of order $n$}" and we denote by $\text{SU} (n)$ the matrices whose entries are complex and are orthogonal and whose determinant is unity:
	
	\begin{tcolorbox}[title=Remark,colframe=black,arc=10pt]
	$\text{U}(1)$ is equal to $\text{SU}(1)$ and it is therefore the complex unit circle equals to $e^{\mathrm{i}\mathbb{R}}$. Moreover, $\text{SO} (2)$ is commutative and isomorphic to $\text{U}(1)$ because it is the set of the rotations of the plane.
	\end{tcolorbox}
	A well known example is still the one of the Pauli matrices but simply written in the form used in Relativistic Quantum Physics (see section of the same name):
	
	which are part of $\text{SU} (2)$ and as we have shown (implicitly) at the beginning of the section of Spinor Calculus isomorphic to the quaternion group $\text{SO} (3)$ of module $1$ Relations that mathematicians name in the present situation a "homomorphism's overlay" ...
	\begin{tcolorbox}[title=Remark,colframe=black,arc=10pt]
	The special unitary group has a particular importance in particle physics. If the unitary group $\text{U} (1)$ is the electromagnetic gauge group (think to the complex number appearing in the solutions of the wave equation!), $\text{SU} (2)$ is the group associated with the weak interaction, and $\text{SU} (3)$ the one of the strong interaction. This is for example due to the structure of representations of $\text{SU} (3)$ that Gell-Mann conjectured the existence of quarks!
	\end{tcolorbox}
	Let us see with a different approach to that used in the section of Spinor Calculus how to show that the Pauli matrices are the bases of $\text{SU} (2)$?
	
	First, the reader has to know that we prove in the section of Spinor Calculus  that any rotation in space of three dimensions could be expressed using the relation (for small angles!):
	
	And we saw in the section of Quantum Computing that an explicit decomposed formulation of  the previous relation was:
	
	and therefore that all $\text{SU}(2)$ element is produced from these three matrices which each make describe to the end of a vector a curvie on the surface of a sphere!
	
	Now, we notice that these three matrices are equal to when $\theta=0$:
	
	We then obtain the identity matrix. So if we search tangent at this point we can therefore build a base on it ($3$ orthogonal vectors).
	
	Let's look at this:
	
	Thus, $\text{SU}(2)$ has for basis:
	
	and are in other words the infinitesimal generators of the group $\text{SU}(2)$.$\text{SU}(2)$ has therefore a basis that is a Lie algebra according to the vocabulary of mathematicians.
	
	This result is quite remarkable ... Since $\text{SU} (2)$ and $\text{SO} (3)$ are isomorphic, then we can get the basis of the Lie algebra $\text{SO} (3)$ while using the same method !!!
	
	Let us see this! We proved in the section of Euclidean Euclidean that the rotations matrices were given by (we change the $R$ by a $U$ so to not confuse with the previous matrices):
	
	We notice again that on 
$\theta=\gamma=\phi=0$ the curve that is generated by the extremity of a vector from the three rotation matrices pass through:
	
	Then in the same manner as for $\text{SU} (2)$, we calculate the derivatives in these angles to determine the generating base matrices of $\text{SO} (3)$:
	
	The Lie algebra of $\text{SO} (3)$ has therefore for basis:
	
	In physics, we prefer to work with complex matrices. We then introduce the matrices:
	
	It must then been notice that if we define:
	
	We trivially have to the complex conjugate of the transposed matrix:
	
	and by the way ... we also have the following non-commutative relations (which we can develop on request):
	
	and also the relation of commutation:
	
	that the Pauli matrices also satisfy and ... for recall (or for information for those who have not yet read the section of Wave Quantum Physics) the $J_i$ are the operators of the total angular momentum of the spin-orbit coupling system!!!
	
	Most of the groups we have seen until so far can be resume with the following figure:
	\begin{figure}[H]
		\centering
		\includegraphics{img/algebra/special_linear_group.jpg}
	\end{figure}
	The rotations with the quaternions indicated in the figure above are studied in the section Numbers of the chapter Arithmetic.
	
	\pagebreak
	\subsubsection{Group of Symetries}
	The symmetry group of an object denoted $X$ (image, signal, etc. in 1D, 2D, 3D or other) is the group of all isometries (an isometry is a transformation that preserves length) under which it is invariant with composition as an operation.
	
	Any group of symmetries which elements have a common fixed point, which is true for all symmetry groups of limited figures, can be represented as a subgroup of the orthogonal group $\text{O} (n)$ by choosing the origin as a fixed point . The proper symmetry group is a subgroup of the special orthogonal group $\text{SO} (n)$, and hence it is also named the "rotations group" of the figure.
	
	In what follows, we will interpret the composition of two operation of symmetries  or rotations as the multiplication as well as for permutations.
	
	Let us see first two fundamental definitions!
	\textbf{Definitions (\#\mydef):}
	\begin{enumerate}
		\item[D1.] The "\NewTerm{group of symmetries}\index{group of symmetries}", also named "\NewTerm{group of invariants}\index{group of invariants}" of $X$ is the set of symmetries of $X$, generated by the multiplication structure given by the composition that leaves $X$ invariant.
		
		\item[D2.] The "\NewTerm{order}\index{order of a group}" of a group is the total number of all its symmetries only (including the identity!).
	\end{enumerate}
	\begin{tcolorbox}[colframe=black,colback=white,sharp corners]
	\textbf{{\Large \ding{45}}Examples:}\\\\
	E1. The heart (...):
	\begin{figure}[H]
		\centering
		\includegraphics{img/algebra/symmetries_heart.jpg}
	\end{figure}
	has a group of symmetries of $2$ elements, namely the identity application $\text{id}$ and the application $r_v$ that is the reflection in respect to the vertical axis (sub-group of symmetries with $1$ item). We observe also that the symmetrical is also provided via the relation:
	
	\end{tcolorbox}
	
	\pagebreak
	\begin{tcolorbox}[colframe=black,colback=white,sharp corners]
	E2. The letter phi (...):
	\begin{center}
	\[ \scalebox{8}{$\Phi$} \]
	\end{center}
	has a total symmetry group of $4$ elements, namely the identity application $\text{id}$, the both reflections $r_h$ and $r_v$ and the rotation of angle $\pi$ which we denote by $t_\pi$ (subgroup of rotations of $1$ item). This form thus has a group of symmetries of order $3$.\\
	
	In this group we have:
	
 	(which is commutative!), $t_{\pi}\circ t_{\pi}$ is the rotation of angle $2\pi$, which is the same application as the identity application, therefore $t_{\pi}\circ t_\pi=\text{id}$.\\
 	
 	Thus, the symmetry group of this letter is indeed commutative and the composition law is internal. It is indeed a group.\\
 	
 	E3. The regular pentagon:
 	\begin{figure}[H]
		\centering
		\includegraphics{img/algebra/symmetries_pentagon.jpg}
	\end{figure}
	has a group of symmetries of $10$ elements, namely, more precisley  $5$ rotations $\text{id},t_{2\pi/5},t_{4\pi/5},t_{6\pi/5},t_{8\pi/5}$ and also $5$ reflections along the $5$-axis symmetry. It is therefore a group of symmetries of order $5$ corresponding to cyclic group $\mathbb{Z}/ 5\mathbb{Z}$.\\
	
	\begin{tcolorbox}[title=Remark,colframe=black,arc=10pt]
More generally, the group of symmetries of a regular $n$-gon (where $n$ is odd) has exactly $2n$ elements. This group is named "\NewTerm{dihedral group of order $n$}\index{dihedral group of order $n$}" and is most often denoted by $D_{2n}$ (be careful because some authors do not multiply by a factor of $2$ so that the index is then directly the order and not the number of elements).
	\end{tcolorbox}	
	The pentagon has therefore $D_{10}$ for dihedral group and $\mathbb{Z}/ 5\mathbb{Z}$ is a "\NewTerm{distinct subgroups}\index{distinct subgroups}" (we'll comme on this notion of subgroup later below).
	\end{tcolorbox}
	
	\pagebreak
	\begin{tcolorbox}[colframe=black,colback=white,sharp corners]
	E4. The dihedral group $D_6$ of order $3$ of the isometries of an equilateral triangle (regular polygon) has $6$ elements which we will denote (so that the writing is less heavy):
	
	where $\sigma_1,\sigma_2,\sigma_3$ are the symmetries relatively to the three bisectors (mediators respectively). The compositions table below of this dihedral group also shows that this group is non-commutative:
	\begin{figure}[H]
		\centering
		\includegraphics{img/algebra/equilateral_dihedral_group_representation.jpg}
	\end{figure}
	
	We will return back later on this example when we will introduce a little further below the concept of a distinct group in our study of permutations groups.\\
	
	E5. Let's look at one last example applied to chemistry by enumerating the symmetry operations that leave the ammonia molecule NH$_3$ (tetrahedron) invariant.
	\begin{figure}[H]
		\centering
		\includegraphics{img/algebra/nh3.jpg}
	\end{figure}
	\end{tcolorbox}
	
	\pagebreak
	\begin{tcolorbox}[colframe=black,colback=white,sharp corners]
	The transformation group contains $6$ elements: the identity $\text{id}$, $C_3$ that is the rotation of $2\pi/3$, $C_3^2$ that is the rotation of $4\pi/3$ (which we will denote after by $C_{-3}$) both along the $z$-axis (therefore perpendicular to the $xy$-plane...) and $3$ axes of symmetry $\sigma_1,\sigma_2,\sigma_3$ of symetries/reflection each passing through the middle of one of the base edges in the middle of the opposite edge as shown in the figure below (pyramid view from above):
	\begin{figure}[H]
		\centering
		\includegraphics{img/algebra/tetrahedral_group_representation.jpg}
		\caption{Operations leaving invariant a tetrahedron}
	\end{figure}
	The combination of the various elements of symmetries show that the compositions table is (which proves that the law is internal and we are working effectively in a group):
	
	Attention to the order of operations in the above table, we first apply the line element and afterwards the column element!\\
	
	We note that the group is not commutative.
	\end{tcolorbox}

	\pagebreak
	\paragraph{Orbits and Stabilizers}\mbox{}\\\\
	We will now see two definitions we meet again in crystallography (their name is not innocent!).
	
	\textbf{Definition (\#\mydef):} The orbit of an element $x$ of $E$ is given by:
	
	The orbit of $x$ is the set of all positions (in $E$) likely to be occupied by the image of $x$ under the action of $G$. The orbits obviously form a partition of $E$.
	\begin{tcolorbox}[colframe=black,colback=white,sharp corners]
	\textbf{{\Large \ding{45}}Example:}\\\\
	Let us consider a set $E$ on which a group $G$ will act, by:
	
	the set of all $6$ vertices of a hexagon on which we do act the group $G=\{\text{id},t_{2\pi/3},t_{4\pi/3}\}$. We see already trivially see that $G$ is a group!  \\
	
	But now let us, consider an element of $E$, for example $S_0$.
	
	Its orbit will be by definition:
	
	\end{tcolorbox}
	\textbf{Definition (\#\mydef):} The stabilisator of $x$ of an element of $E$ is the set:
	
	of all elements that let $x$ invariant under their action. It is a subgroup of $G$.
	\begin{tcolorbox}[colframe=black,colback=white,sharp corners]
	\textbf{{\Large \ding{45}}Example:}\\\\
	To continue with our previous example. Its stabilisator is reduced to:
	
	\end{tcolorbox}
	
	
	\pagebreak
	\subsubsection{Permutations Groups}
	Symmetric groups have significant importance in some areas of Quantum Physics but also in mathematics as part of Linear Algebra (for the theory of the determinant as we will see in the corresponding section) and in Galois Theory (see further below) and also in Error Correcting Codes (see corresponding section for the example used with VISA credit cards). So these must be also pay special attention!
	
	Let us first recall (\SeeChapter{see section Probabilities}) that in a set $\{1,...,n\}$ there are $n!$ possible permutations. Mathematicians say, rightly, that there are $n!$ bijections and name this number "\NewTerm{order of the permutations of the group}\index{order of the permutations of the group}".
	
	\begin{tcolorbox}[colframe=black,colback=white,sharp corners]
	\textbf{{\Large \ding{45}}Example:}\\\\
	Given for example the set $\{1,2,3\}$. This set has $3!$ possible permutations that are denoted in the context of the permutation groups as follows:
	
	This must be read in the order: identity application id, $1$ takes on $2$ or $2$ on $1$ (in terms of positions!), $1$ takes on $3$ or $3$ on $1$, $2$ takes on $3$ or $3$ on $2$, $1$ takes on $2$ that takes on $3$ that takes on $1$, $1$ takes on $3$ that takes on $2$ that takes on $1$.
	
	We can easily observe that the composition of two permutations is not commutative:
	
	and that the composition of two permutations is an internal law:
	
	with a neutral element that is indeed the identity id. So we do have well a non-commutative group. Let us recall also the reader that certain elements of the group, if well chosen, can form a subgroup. This is the example of:
	
	which is a sub-group of $S_3$ (it must be easy to the reader to check that it has all the properties of a group).
	\end{tcolorbox}
	
	\pagebreak
	\textbf{Definition (\#\mydef):} A subgroup $H$ of a group $G$ is named "\NewTerm{distinct group}\index{distinct group}" if, for every $g$ of $G$ and every $h$ of $H$, we have that $ghg^{-1}$ is element of $H$. The mathematicians name this an "\NewTerm{inner automorphism}\index{inner automorphism}"...
	
	Let us first consider interesting introducing geometric example after which we will come back to this definition with $S_3$.
	\begin{tcolorbox}[colframe=black,colback=white,sharp corners]
	\textbf{{\Large \ding{45}}Example:}\\\\
	We have seen earlier the elements of the dihedral symmetry group of order $3$ of the equilateral triangle. Geometrically they all correspond to displacements in the plane in which the triangle is located. We got for recall of the following table of compositions:
	\begin{figure}[H]
		\centering
		\includegraphics{img/algebra/equilateral_dihedral_group_representation.jpg}
	\end{figure}
	
	First, we easily see using this table that we have:
	\begin{itemize}
		\item The sub-group made of $\{\text{id}\}$ of order $1$
	
		\item The sub-group made of $\{\text{id},t_{2\pi/3},t_{4\pi/3}\}$ of order $3$

		\item The sub-group made of $\{\text{id},\sigma_1\}$ of order $2$

		\item The sub-group made of $\{\text{id},\sigma_2\}$ of order $2$

		\item The sub-group made of $\{\text{id},\sigma_3\}$ of order $2$
	\end{itemize}
	Among these five subgroups, let us see which one are distincted sub-groups  (this is relatively easy to see using the table of compositions above):
	\begin{itemize}
		\item The sub-group made of $\{\text{id}\}$
	
		\item The sub-group made of $\{\text{id},t_{2\pi/3},t_{4\pi/3}\}$ 
	\end{itemize}
	\end{tcolorbox}
	
	\pagebreak
	\begin{tcolorbox}[colframe=black,colback=white,sharp corners]
	We will now see a remarkable thing! By numbering with $1, 2$ and $3$ the vertices of the equilateral triangle and taking the rotations clockwise, we can identify the elements of $D_6$ to the following elements of $S_3$:
	
	and rebuild the same table of compositions (copy of the previous one but just with the notation change ... hehe!):
	\begin{figure}[H]
		\centering
		\includegraphics{img/algebra/equilateral_dihedral_group_representation.jpg}
	\end{figure}
	
	\end{tcolorbox}
		Well ... this little interlude closed, let us return to the distincted subgroup of $S_3$ (as it will be important for our introduction to Galois groups) and let us first recall that:
	
	and we see that the distinct subgroup consists of:
	
	\textbf{Definition (\#\mydef):} For any subgroup $H$ stable by the inner automorphisms of a group $G$, we name "\NewTerm{index of $H$ in $G$}" the quotient of the order of the group $G$ by the order of the subgroup $H$ and we denote it by $[G / H]$.
	
	\begin{tcolorbox}[colframe=black,colback=white,sharp corners]
	\textbf{{\Large \ding{45}}Example:}\\\\
	The index of the subgroup $\{(1), (1 2)\}$ in the group $S_3$ is $6/2$ that is to say $3$.
	\end{tcolorbox}
	 This concept will be very helpful to us during our introduction to Galois Theory.
	
	Let us consider now the particular permutation $\sigma$ to introduce the subject from a different but equivalent angle:
	
	As we know the mathematicians are accustomed to note that, at first, in the form:
	 
	with:
	
	Hence:
	 
	Given $\sigma$ and $\tau$, two permutations, it is natural to look at their composition $\tau\circ \sigma$  (recall that this means that first we apply $\sigma$ and afterwards $\tau$ as for the composition of functions).
	
	Therefore if:
	
	Then:
	
	and:
	
	Now the idea is to interpret the composition as a multiplication of permutations. This multiplication is then non-commutative as we have seen in the previous example. We have generally $\sigma \circ \tau \neq \tau \circ \sigma$.
	
	Each bijection has in inverse (reciprocal function). In our example it is obviously:
	
	Geometrically, to calculate the inverse $\sigma^{-1}$ of an element $\sigma$, we just need to take the reflection of the drawing of $\sigma$ in a horizontal axis as shown in the left side of the figure below:
	\begin{figure}[H]
		\centering
		\includegraphics{img/algebra/permutations.jpg}
		\caption{Examples of compositions and inverse of permutations}
	\end{figure}
	We may represent a permutation $\sigma$ as a permutation matrix, i.e., the matrix that maps:
	
	An $n\times n$ matrix is a permutation matrix if and only if all its entries are zeros and ones, each column has exactly one $1$, and each row has exactly one $1$.  In this instance, we have:
	 
	 \textbf{Definitions (\#\mydef):}
	 \begin{enumerate}
		\item[D1.] The set of permutations of a set with $n$ elements, with this structure of multiplication, is named the "\NewTerm{group of permutations of order $n$}\index{group of permutations of order $n$}" or "\NewTerm{group of substitutions of order $n$}\index{group of substitutions of order $n$}", and is denoted by $S_n$ or $S(n)$.
		
		We say that an element $\sigma$  of $S_n$ is a "\NewTerm{cycle of order $k$}\index{cycle of order $k$}", or simply a "\NewTerm{$k$-th cycle"} if there exists $a_1,a_2,...,a_k\in \{1,...,n\}$ such that:
		\begin{itemize}
			\item $\sigma$ send $a_1$ on $a_2$, $a_2$ on $a_3$, ..., $a_{k-1}$, and $a_k$ on $a_1$.

			\item $\sigma$ fix all other elements of $S_n$
		\end{itemize}
		and we denote the cycle as following:
		
		\begin{tcolorbox}[colframe=black,colback=white,sharp corners]
		\textbf{{\Large \ding{45}}Example:}\\\\
		Perhaps, for a better understanding let us our previous example of $S_4$:
		
		This symmetric group is a $3$-cycle denoted $\sigma=(1\; 3\; 4)$ because in order: $1$ send on $3$, 3 send on $4$ and $4$ send on $1$ (and $2$ is not mentioned as it remains fixed). We can also write this in the following equivalent ways: $\sigma=(3\; 4\; 1)$ or also $\sigma=(4\; 1\; 3)$.
		\end{tcolorbox}

		\item[D2.] The order of a $k$-cycle is $k$ (hence the name!).
		\begin{tcolorbox}[colframe=black,colback=white,sharp corners]
		\textbf{{\Large \ding{45}}Example:}\\\\
		Indeed if we take again our $\sigma=(1\; 3\; 4)$, then we have:
		
		\end{tcolorbox}

		\item[D3.] We say that a permutation $\sigma$ is a "\NewTerm{cycle}\index{cycle (permutation)}" if there exists $k\in \mathbb{N}$ such as that $\sigma$ is a $k$-cycle.
		
		Warning! Any permutation must be written as a product of disjoint cycles (that is to say, a number that appears in a cycle should not appear in another cycle).
		
		\begin{tcolorbox}[colframe=black,colback=white,sharp corners]
		\textbf{{\Large \ding{45}}Example:}\\\\
		For example, in $S_9$, we have:
		
		So this service is a product of a $4$-cycle and of a $3$-disjoint cycle.\\
		
		We also let the reader see for himself that the cyclic group generated by $\sigma$ (which in this case is a subgroup of $S_9$) is of order $12$ ($12$-cycle)...
		\end{tcolorbox}
		\begin{tcolorbox}[title=Remark,colframe=black,arc=10pt]
		Mathematicians can prove that if $\sigma$ is an element that has a decomposition into $c$ disjoint cycles of length $n_1,n_2,...,n_c$ then the order of $\sigma$ is the least common multiple of the orders of all disjoint cycles that compose it.
		\end{tcolorbox}	
	\end{enumerate} 
	We also assume intuitive that in the common vocabulary, a $2$-cycle into $S_n$ is also named a "\NewTerm{transposition}\index{transposition}".
	
	Let us go a little further. We propose to show by example that the set of all transpositions generates $S_n$. In other words, any permutation is written as a product of transpositions!
	
	\begin{tcolorbox}[colframe=black,colback=white,sharp corners]
	\textbf{{\Large \ding{45}}Example:}\\\\
	Let us return to our example (it is an even permutation)
		
	\end{tcolorbox}
	In general, a $k$-cycle is thus written as the product of $k-1$ transpositions.
	
	\begin{theorem}
	As the permutations of a finite set form a group, this means (among other things) that there is therefore always an integer $k$, such that $p$ applied $k$ time is the identity transformation (that is to say, the operation that does not change anything).
	\end{theorem}
	\begin{dem}
	If $G$ is a finite group and that $g\in G$, we consider the sequence of items (recall that in a group there is only one an operation and therefore the square, cube, etc. means we compose this operation!):
	
	For example, in the permutation groups, the operation is the composition of permutations.
	
	Since $G$ is finite and that this sequence is infinite, there are necessarily two elements that are equal in the sequence... So there are two different $n$ and $m$ such that:
	
	Assuming that $n<m$, the previous equality is simplified and we get:
	
	where $e$ is the neutral element of the group.
	\begin{flushright}
		$\square$  Q.E.D.
	\end{flushright}
	\end{dem}
	We will see that the permutations being bijective, we can create on finite groups, compositions of permutations operations that always end up returning to the initial state (identity application id)!!!
	\begin{tcolorbox}[colframe=black,colback=white,sharp corners]
	\textbf{{\Large \ding{45}}Examples:}\\\\
	E1. In a list of $5$ items, we exchange the first and the third, and, at the same time, we pass the second in position $4$, this that is in position $4$ in put in position $5$ and the one in position $5$ is put on position $2$. And we reiterate. This gives:
	
	We returned to the starting point after $6$ steps.
	\end{tcolorbox}
	
	\pagebreak
	\begin{tcolorbox}[colframe=black,colback=white,sharp corners]
	E2. Let us consider the "\NewTerm{Photomaton transformation}\index{photomaton transformation}" of the image of Mona Lisa of size $256$ by $256$ pixels:
	\begin{figure}[H]
		\centering
		\includegraphics{img/algebra/mona_lisa.jpg}
		\caption{Photomaton transofrmation}
	\end{figure}
	We could thing that each image was obtained from the previous by reducing the size of the image half, which gave four similar pieces that we have placed in square to obtain an image having the same size as original image. But in fact it is not!!!! The number of pixels has been preserved (no pixel is duplicated!!!) and we actually just moved the pixels by permutation to get four images that do not actually contain all the information of the original image but only a part!\\
	
	By repeating the procedure $8$ times we always fall back on the original image regardless of the original image. The question is to understand why?\\
	
	Let us consider that the original image is a square with a size of $16$ pixels wide by 16 pixels high (but you can also apply what follows with a rectangular image of any size and you will see that it works also!). Each pixel of a line (the process is exactly the same for columns!) is identified by a coordinate along the $X$ axis going from $0$ to $15$.\\
	
	Thus we have at the beginning a sequence of numbers where the pixel coordinates correspond to their $x$ coordinate:
	\begin{gather*}
		0\; 1\; 2\; 3\; 4\; 5\; 6\; 7\; 8\; 9\; 10\; 11\; 12\; 13\; 14\; 15
	\end{gather*}
	We then do the permutation that consist to denote by $k$ the position of a pixel and to do:
	
	This then gives the first permutation:
	\end{tcolorbox}

	\pagebreak
	\begin{tcolorbox}[colframe=black,colback=white,sharp corners]
	\begin{gather*}
		0\; 1\; 2\; 3\; 4\; 5\; 6\; 7\; 8\; 9\; 10\; 11\; 12\; 13\; 14\; 15\\
		\downarrow \text{Permutation }1\\
		0\; 2\; 4\; 6\; 8\; 10\; 12\; 14\; 1\; 3\; 5\; 7\; 9\; 11\; 13\; 15\\
		\downarrow \text{Permutation }2\\
		0\; 4\; 8\; 12\; 1\; 5\; 9\; 13\; 2\; 6\; 10\; 14\; 3\; 7\; 11\; 15\\
		\downarrow \text{Permutation }3\\
		0\; 8\; 1\; 9\; 2\; 10\; 3\; 11\; 4\; 12\; 5\; 13\; 6\; 14\; 7\; 15\\
		\downarrow \text{Permutation }4\\
0\; 1\; 2\; 3\; 4\; 5\; 6\; 7\; 8\; 9\; 10\; 11\; 12\; 13\; 14\; 15\\
	\end{gather*}
	Thus, for an image of $16$ by $16$ pixels, it takes four permutations, which corresponds to $2^4=16$. So for an image of $256$ pixels, we have $256=2^8$, hence the fact that we need $8$ permutation to find back the original Mona Lisa with:
	
	Thus, in the general case of an image of width $L$ in number of pixels counting from $1$, the transformation is:
	
	where $E^+[...]$ is the upper (nearest) integer value in the case where $L$ is odd.\\
	
	The reader will also have maybe notice something interesting if we return to our example with the image of $16$ pixels ... Indeed, let us take the third pixel from the left of coordinate $x$ equal to $2$. In binary, its initial position is then $0010$. After the first permutation, its $x$ coordinate is equal to $1$, or in binary: $0001$. After the second permutation, the $x$ coordinate is equal to $8$, or in binary: $1000$, etc. In fact we see that every permutation can be summarized in binary by shifting the bits to the right.
	\end{tcolorbox}
	\textbf{Definition (\#\mydef):} Given $\sigma \in S_n$ a permutation. We say that $\sigma$ is an "\NewTerm{even permutation}\index{even permutation}" if, in a writing of $\sigma$ as product of transpositions, there is an even number of transpositions. We say the obviously that $\sigma$ is an "\NewTerm{odd permutation}\index{odd permutation}" if, in a writing of $\sigma$ as product of transposition, there is an odd number of transpositions.
	
	Let us end with a small complement... We know that $S_3$ is a group of permutations of order $3$ and therefore with $3!=6$ possible permutations.
	
	If we list the $6$ permutations we saw what we get:
	
	Among these only some can be written as an even product of transpositions:
	
	The even permutations form with the identity permutation, a subgroup (not commutative!) we name the "\NewTerm{alternating group of order $n$}\index{alternating group of order $n$}" and that we denote by $AN$. It is easy to check with the previous example.
		
	\subsection{Galois Theory}	
	In abstract algebra, Galois theory, named after Évariste Galois, provides a connection between field theory and group theory. Using Galois theory, certain problems in field theory can be reduced to group theory, which is, in some sense, simpler and better understood.
	
	Originally, Galois used permutation groups to describe how the various roots of a given polynomial equation are related to each other. 
	
	The birth and development of Galois theory was caused by the following question, whose answer is known as the "\NewTerm{Abel–Ruffini theorem}\index{Abel-Ruffini theorem}": Why is there no formula for the roots of a fifth (or higher) degree polynomial equation in terms of the coefficients of the polynomial, using only the usual algebraic operations (addition, subtraction, multiplication, division) and application of radicals (square roots, cube roots, etc)?.
	
	Galois theory not only provides a beautiful answer to this question, it also explains in detail why it is possible to solve equations of degree four or lower in the above manner, and why their solutions take the form that they do. Further, it gives a conceptually clear, and often practical, means of telling when some particular equation of higher degree can be solved in that manner.
	
	Galois theory originated in the study of symmetric functions:  the coefficients of a monic polynomial are (up to sign) the elementary symmetric polynomials in the roots. For instance:
	
 	where $1$, $a + b$ and $ab$ are the elementary polynomials of degree $0$, $1$ and $2$ in two variables.
 	
 	We tried to make this part of the book as easy as possible. So we hope our goal is reach if you understand what follow. Let us now begin!!!
 	
 	\subsubsection{Elementary symmetric and Invariant Polynomials}
 	\textbf{Definition (\#\mydef):} The "\NewTerm{$k$th elementary symmetric polynomial in $n$ variables}\index{elementary symmetric polynomial}", denoted $s_k(x_k,\ldots , x_n)$ is the sum of all possible degree $k$ monomials in $n$ variables with each $x_i$ appearing NO MORE THAN ONCE IN EACH MONOMIAL. Formally, for $k\leq n$:
	
	A polynomial is said "\NewTerm{invariant under $S_n$}\index{invariant polynomial}" if and only if it is a polynomial in the elementary symmetric functions $s_1,\ldots, s_n$.

	Therefore:
	
	\begin{tcolorbox}[colframe=black,colback=white,sharp corners]
	\textbf{{\Large \ding{45}}Example:}\\\\
	For $n = 1$:
	
	For $n = 2$:
	
	For $n = 3$:
	
	For $n = 4$:
	
	Now consider the equation:
	
	It can be rewritten as:
	
	or as we know the de Viete relations with two roots, the latter can be rewritten:
	
		\end{tcolorbox}
	
	\pagebreak
	\begin{tcolorbox}[colframe=black,colback=white,sharp corners]
	That is to say:
	
	Similarly, for a third-degree polynomial:
	
	and therefore:
	
	\end{tcolorbox}

	The elementary symmetric polynomials appear when we expand a linear factorization of a monic polynomial. We have the identity:
	
	That is, when we substitute numerical values for the variables $r_1,r_2,\dots,r_n$, we obtain the monic univariate polynomial (with variable $x$) whose roots are the values substituted for $r_1,r_2,\dots,r_n$ and whose coefficients are up to their sign the elementary symmetric polynomials. These relations between the roots and the coefficients of a polynomial are named "\NewTerm{General Vieta's formulas}\index{General Vieta's formulas}" for which we have already see two special cases in the section Calculus and that we will generalize further below.
	\begin{theorem}
	If $r_1,r_2,\ldots, r_n$ are the roots of a degree $n$ polynomial, then:
	
	\end{theorem}
	\begin{dem}
	We will prove this by induction on the degree of the polynomial. If our polynomial is of degree $n = 1$ with root $r$, the left hand side is $x-r$, and the right hand side is $x-s_1(r)= x-r$, so the equation holds for $n = 1$. Suppose the equation holds for all polynomials of degree $n$. Let $P(x)$ be of degree $n+1$ with roots $r_1,\ldots,r_{n+1}$.

    Then, we can write:
    
	where we let $s_i$ denote $s_i(r1, \ldots , a_n)$ for brevity. By multiplying out the right hand side:
	
	Since:
	
	and:
	
	So we get:
	
	Now it remains that if we can prove that:
	
	for all the other $i$, we conclude the equation holds for $n+1$, hence for all $n$.
	
	Remember that by definition:
	
	Then:
	
	By separating the sum with respect to monomials divisible by $r_{n+1}$, we see the above is equal to (most of time the best is to check this be using one the previous examples given earlier):
	
	so it is clear the relation we wanted holds.
	\begin{flushright}
		$\square$  Q.E.D.
	\end{flushright}
	\end{dem}
	
	\subsubsection{General Vieta's formulas}
	If we write the second degree equation as following:
	
	We already know that:
	
	And also for a third degree polynomial we saw just before in the examples that if we have:
	
	then:
	
	We can easily see a pattern emerging that is:
	
	
	\begin{flushright}
	\begin{tabular}{l c}
	\circled{90} & \pbox{20cm}{\score{4}{5} \\ {\tiny 16 votes,  81.25\%}} 
	\end{tabular} 
	\end{flushright}

	%to make section start on odd page
	\newpage
	\thispagestyle{empty}
	\mbox{}
	\section{Differential and Integral Calculus}

\lettrine[lines=4]{\color{BrickRed}D}ifferential calculus is one of the most exciting and vast field of mathematics and there is colossal literature on the subject. The results initiated by scientists as Fermat, Newton, Leibniz, Euler and company since the late 17th century found absolutely implications in all areas of physics, computer science, electronics, chemistry, finance, biology and mathematics itself.

Mathematicians have written such a lot of theorems since the birth of differential calculus in the mid-16th century that the validation of a sample of these theorems is sometimes tricky because requiring for some of them a whole life to be covered (it is a problem that the community of mathematicians recognizes ) and checked (so that sometimes nobody checks...).

This fact know, we have chosen here to present only the items absolutely necessary to the understanding of the fundamental tools for the engineer. The purists will excuse us for the moment not to present some theorems that seem indispensable to them but we will prepare once we will have more time...


We will mainly study in what follows what mathematicians like to precisely name (and they are right!): the general cases of real functions of one real variable. More complex functions (several real or complex variables, continuous or discrete) will come once this first part is finished. But the reader can already found theorems with complex variables in the section of Complex Analysis.

	\begin{tcolorbox}[title=Remark,colframe=black,arc=10pt]
We will not demonstrating all the primitives and derivatives all possible functions because as there is an infinite number of possible functions, there is also an infinite number of derivatives and primitives. It is the role of teachers in educational institutions to train students to apply and understand the reasoning of derivations and integration by applications on known functions (the Internet probably never replace the school at this level).
	\end{tcolorbox}	
	
	\subsection{Differential Calculus}

Let $f$ be a real function of one real variable $x$ denoted $f(x)$ (we restrict ourselves to this case for now and we will study the partial derivatives in any number of dimensions later) continue in at least one interval where the horizontal axis contains the value $a$.

\textbf{Definitions (\#\mydef):} 

\begin{enumerate}

\item[D1.] We name "\NewTerm{average slope}\index{average slope}", or "\NewTerm{directing coefficient}\index{directing coefficient}" the report of the orthogonal projection of two points $x_1 \neq x_2$ of the function $f$ not necessarily continuous on the $x$-axis and $y$-axis as:
	
	What can be represented graphically as follows with a specific function:

	\begin{figure}[H]
		\centering
		\begin{tikzpicture}[>=stealth',
                    dot/.style={circle,draw,fill=white,inner sep=0pt,minimum size=4pt},scale=1.25]

	    % draw axis lines
	    \draw[->,thick] (-0.5,0) -- ++(11,0) node[below left]{$x$};
	    \draw[->,thick] (0,-0.5) -- ++(0,7) node[below right]{$y=f(x)$};
	    \coordinate (O) at (0,0);
	
	    % create path for function curve
	    \path[thick,red] (-0.3,2) to[out=-25, in=200] coordinate[pos=0.2] (p) coordinate[pos=0.6] (q) (9,5);
	    % fill area
	    \fill[blue, opacity=.1] (p) -| (q);
	    % draw the secant line with fixed length
	    \draw[shorten <=-1.5cm] (p) -- ($ (p)!7.5cm!(q) $) node[below right, pos=0.9]{Secant};
	    % draw function curve
	    \draw[thick,red] (-0.3,2) to[out=-25, in=200] (9,5);
	
	    % draw all points
	    \node[dot,label={above:$P$}] (P) at (p) {};
	    \node[dot,label={above:$Q$}] (Q) at (q) {};
	    \node[dot] (p1) at (P |- O) {};
	    \node[dot] (p2) at (Q |- O) {};
	    \node[dot] (p3) at (P -| Q) {};
	
	    % draw lines between nodes and place text
	    \draw (P) -- node[left]{$f(x_{1})$} (p1) node[dot,label={below:$x_{1}$}]{};
	    \draw (p2) node[dot,label={below:$x_2=x_{1} + h$}]{} -- (p3);
	    \path (p1) -- node[below]{$h$} (p2);
	
	    % draw blue arrows between nodes
	    \draw[<->,blue,thick] (P) -- node[below]{$\Delta x=h$} (p3);
	    \draw[<->,blue,thick] (Q) -- node[right]{$f(x_{1} + h) - f(x_{1})=\Delta y=\Delta f$} (p3);
	
	    % draw the explanation for the y-value of point Q
	    \draw[help lines] (Q) -- (Q -| {(9.5,0)}) ++(-0.5,0) coordinate (p4);
	    \draw[help lines, <->] (p4) -- node[fill=white,text=black]{$f(x_{1} + h)=f(x_2)$} (p4 |- O);
		\end{tikzpicture}
	\end{figure}
	
	\begin{tcolorbox}[title=Remark,colframe=black,arc=10pt]
$\Delta$ named "delta" expresses the fact that we take a difference of the same amount.
	\end{tcolorbox}
We assume as obvious (without proof) that two functions whose slopes are the same in the same interval of definition, are parallel (on a plane).
	\begin{tcolorbox}[title=Remark,colframe=black,arc=10pt]
We will prove in the chapter of Analytic Geometry that two functions whose slopes multiplication is equal -1 are perpendicular.
	\end{tcolorbox}
	\item[D2.] We call "\NewTerm{derivative on $a$}\index{derivative}" or "\NewTerm{instantaneous slope}" or "\NewTerm{first derivative}\index{first derivative}", the limit when $h$ tends to $0$ (if the limits exists) of the ratio of the orthogonal projection of two points $x_1\neq x_2$ infinitely close  of a continuous function $f$ (in the sense that it does not contain "holes") on the $x$-axis and $y$-axis so that:
	
A graphic interpretation gives us that $f'(a)$ is the directing coefficient (the slope of the tangent at the point of abscissa $a$).
	\begin{tcolorbox}[title=Remarks,colframe=black,arc=10pt]
	\textbf{R1.} The letter $\mathrm{d}$ means here a "\NewTerm{differential}\index{differential}" and expresses the fact that we take an infinitesimal difference of the same amount.\\

	\textbf{R2.} We refer the reader to the section of Functional Analysis for the definition of a continuous function.
	\end{tcolorbox}
	\item[D3.] Let $f$ be a function defined on an interval I and differentiable at every point $a$ of $I$. The function that to any real number of I associates the number $f'(a)$ is named the "\NewTerm{derivative function of $f$ on I}" and is denoted by $f'$.
	\begin{tcolorbox}[title=Remark,colframe=black,arc=10pt]
About the notations of the derivatives... physicists adopt depending on their mood various possible notations. Thus, consider the real function $f(x)$  with one real variable $x$ you can find in the literature and in this book the following notations for the first derivative:
	
or implicitly assuming that f is a function of x (this gives the opportunity to reduce the size of developments)
	
	\end{tcolorbox}
\end{enumerate}
We can in the same way define 2nd order derivatives (derivative of a derivative), the derivatives of order 3 (derivative from a derivative of order 2) and so on. We will  frequently meet such derivatives in physics or in pure mathematics for functional analysis.

Note that the derivatives of order 2 have a very important interpretation in physics and in the areas of optimization (\SeeChapter{see section Theoretical Computing}). Indeed, if the sign of the first derivative is positive and then becomes negative (going trough the value of zero) when $x$ increases, then we easily guess we travel through a local maximum of a function (the point where the derivative is zero) and that if the sign of the first derivative becomes negative and positive when $x$ increases, so we travel a local minimum of the function (the point where the derivative is zero). In other words, when the slope changes sign (becomes zero by changing the sign) function passes through an extremum (maximum or minimum) and the tangent is "horizontal" in this point: parallel to the $x$-axis. But when the 2nd order derivative is zero, that means that the curvature of the function is reversing. We speak then about an "\NewTerm{inflection point}\index{inflection point}".

So a very important thing that should always keep in mind (!!!) when your write that the derivative of a function is null, is that we can have a derivative which vanishes at a point without being an extremum (remember that we call this an inflection point). To check whether it is really an extremum, we can calculate the second derivative to eliminate the case of an inflection point (because in a inflection point the second derivative will not be null). Otherwise you must use a table of variations to ensure that we are dealing with a maximum or minimum like for example with the function $x^3-3x^2+2$:

	\begin{minipage}{\linewidth}\centering
    \begin{variations}
     x      & \mI &    & 0 &    & 2 &    & \pI  \\
     \filet
     f'     & \ga +    & 0    &  -  &  0   & \dr+      \\
     \filet
     \m{f}  & ~  & \c  & \h{~} & \d & ~    &  \c       \\
     \end{variations}
	\end{minipage} 	
	
	Which corresponding plot is:
	\begin{figure}[H]
		\centering
		\includegraphics[scale=1]{img/algebra/variation_plot_example_1.jpg}
		\caption[]{Plot of  function $x^3-3x^2+2$}
	\end{figure}
	Another example with $f(x)=x^4-4x^3+11$:
	\begin{figure}[H]
		\centering
		\includegraphics[scale=0.8]{img/algebra/variation_plot_example_2.jpg}
		\caption[]{Plot of  function $x^4-4x^3+11$}
	\end{figure}
	With a more detailed variation table:
	
	\begin{center}
	\begin{tikzpicture}[t style/.style={solid}]
	\tkzTabInit[espcl=2]{$x$/.5,$f'(x)$/.5,$f''(x)$/.5,$f(x)$/3} {$-\infty$,$0$,$2$,$3$,$+\infty$}
	\tkzTabLine{,-,0,-,t,-,0,+, }
	\tkzTabLine{,+,0,-,0,+,t,+, }
	
	\node [below] (n1) at (N13){$+\infty$};
	\node [below=1cm](n2) at (N23){$11$};
	\node [below=2cm] (n3) at ([yshift=1em]N33){$-5$};
	\node [above] (n4) at ([yshift=1em]N44){$-16$};
	\node [below] (n5) at (N53){$+\infty$};
	
	\node[below=1ex]at(n2){$ \mathrm{\Sigma.K.} $};
	\node[below=1ex]at([xshift=.5ex]n3){$ \mathrm{\Sigma.K.} $};
	\node[below=1ex]at([xshift=1ex]n4){$ \mathrm{T.E.} $};
	
	\draw[>->] (n1) to [out=-90,in=180] (n2);
	\draw[>->] (n2) to [out=0,in=90] (n3.west);
	\draw[>->] (n3.east) to  [out=-90,in=180] (n4);
	\draw[>->] (n4) to [out=0,in=-90] (n5);
	
	\end{tikzpicture}
	\end{center}

Here is a very entertaining example of a function with its first and second derivatives with Maple 4.00:

\texttt{>plot([tanh(x),diff(tanh(x),x),diff(tanh(x),x\$2)],x=-5..5,\\color=[red,green,blue]);}

\begin{figure}[H]
\centering
\includegraphics[scale=0.75]{img/algebra/derivatives.eps}
\caption{Plot of the hyperbolic tangent function, its first and second derivatives}
\end{figure}

	\begin{tcolorbox}[title=Remark,colframe=black,arc=10pt]
Two good functions to easy remember the property of the second derivative is $f(x)=x^2$ and $f(x)=-x^2$. As you know the first has a global minimum and the second a global maximum and if you calculate the second derivative for the first one you get a positive constant and a negative constant for the second.
	\end{tcolorbox}

Now, following a problem of understanding of one of our reader in one of the chapters of this book, let us specify a technique often used by physicists. Consider a derivative of order $2$ as:
	
If we look at the $\dfrac{\mathrm{d}}{dx}$ as a differential operator (what it is!) we can obviously write:
	
Finally we have:
	
and so it comes after simplification by $f(x)$:
	
otherwise we can not have this equality if the operator acts explicitly on a function in a math or any physical relation.

This may seem obvious to some but sometimes less so for others ... and it is clearly useful to clarify this because it is often used in the sections of Special Relativity, General Relativity, Corpuscular Quantum Physics and Wave Quantum Physics.

Let us now indicate and prove two properties intuitively obvious of derivative which will be essential for us several times some proofs in this book (for example in the section  of Numerical Methods or just simple in this section...).

\begin{theorem}
Consider first two real numbers $a<b$ and $f$ a continuous real-valued function on the closed interval $[a, b]$ and differentiable on the open interval $]a, b[$ such that $f(a)=f(b)$. So we want to prove that there is obviously  at least one element $c \in ]a, b[$ such that $f'(c)=0$ (this is typically the case of polynomial functions!).

This property is named "\NewTerm{Rolle's theorem}\index{Rolle's theorem}" and therefore it explicitly shows that there is at least one element where the derivative of $f$ is zero when we browsing its path we return back to the same value images for two distinct values of the abscissas (pre-images), that is to say that there exists at least one point where the tangent is horizontal.
\end{theorem}

\begin{dem}
First, if $f$ is constant, the result is immediate... Otherwise, as $f$ is continuous on the closed interval $[a, b]$ it admits at least one minimum or maximum considering that we rely on the assumption that $f(a)=f(b)$ and that $f$ is not constant. The extrema is reached at a point $c$ belonging to the open interval $] a, b [$ (the fact of taking an open interval allows in some cases to avoid having an extrema again in $a$ or $b$).

Suppose as a first case that $f(c)$ is global maximun in the interval. The derivative of the function $f$ between $c$ and a second point $a$ has a known sign:

	\begin{itemize}
		\item For $h$ strictly positive and such that $c + h$ still belongs to the interval $[a, b]$:
	
Considering the limit when $h$ tends to $0$, the  derivative valuation on $f'(c)$ is thus negative.
	 \item For $h$ strictly negative and such that $c + h$ still belongs to the interval $[a, b]$:
	
Considering the limit when $h$ tends to $0$, the  derivative valuation on $f'(c)$ is thus positive.
	\end{itemize}
Ultimately, the derivative of $f$ is zero at point $c$.

The proof is analogous if $f(c)$ is a minimum in the interval, with the signs of derivatives that are opposite.
		\begin{flushright}
			$\square$  Q.E.D.
		\end{flushright}
\end{dem}
Let us now consider two real numbers $a,b$ and $f (x) $ a continuous function on $[a, b]$ and differentiable on $] a, b [$. 

\begin{theorem}
	So we propose to show that there is at least one real number $c \in ]a,b[ $ such that:
	
This can also be written as follows:
	
with $s\in ]0,1[.$

Since the term on the left represents a finite increase of the term right, then this result is named "\NewTerm{mean value theorem}\index{mean value theorem}" or better "\NewTerm{theorem of finite increments}\index{theorem of inite increments}".

Geometrically this means that on at least one point $c$ of the graph of the function $f (x)$, there is a tangent with a director coefficient of:
	
Graphically this gives:
\begin{figure}[H]
\centering
\includegraphics{img/algebra/mean_value_theorem.eps}
\caption{Graphical representation of Rolle's theorem}
\end{figure}

\end{theorem}
\begin{dem}
	We first have:
	
	because the slope of $h(x)$ is obviously:
		
 and as we must have $f(a)$ when $x=a$ it follows the relation given previously.

Then, to show that such a $c$ value exists, the idea is to bring the two points $a$ and $b$ in the same ordinate making this brings us back to Rolle's theorem and for that, we define a function $g(x)$ by:
	
which is such that indeed $g(a)=g(b)$ ... is in this case equal to $0$ (but this value is not relevant).

Therefore, the Rolle's theorem discussed above indicates that there is a point between $a$ and $b$ where the derivative of $g(x)$ is zero such that $g'(c)=0$. And by seing that:
	
we get:
	
	Therefore after simplification:
		
		\begin{flushright}
			$\square$  Q.E.D.
		\end{flushright}
\end{dem}
Using this little theorem and mathematical tools introduced earlier, we can build a little theorem useful and powerful physics.

\textbf{Definition (\#\mydef):} We name  "\NewTerm{Hôpital's rule}\index{Hôpital's rule}" (also named sometimes named "\NewTerm{Bernoulli's rule}\index{Bernoulli's rule}" or "NewTerm{Hospital rule}\index{Hospital rule}") the method that uses the derivative in order to determine the boundaries difficult to calculate with most quotients which often appear in physics.

	\begin{dem}
		Consider two functions $f(x)$ and $g(x)$ and such that $f(a)=f(b)=0$ so we can write:
		
		Then according to the definition of the derivative:
		
		\begin{flushright}
			$\square$  Q.E.D.
		\end{flushright}		
	\end{dem}
We can generalize this previous result initially based on a little too strong constraint:
	
	\begin{dem}
Let us recall that according to the mean value theorem, if $f(x)$ is differentiable on an interval $]a, b[$ and continuous on $[a, b]$ then there is a real $c$ in the interval $[a, b]$ such that:
		
		If the theorem is true for two functions satisfying the same constraints then we have two functions such as:
		
				If g '(c) is not zero then we have the right to write the ratio (some name this the "\NewTerm{generalized mean value theorem}\index{generalized mean value theorem}"...):	
		
		which without losing validity as $c$ is in the range $[a, x]$ can be written:
		
		Therefore, when $x \rightarrow a$ that implies the range $[a, x]$ is always smaller and thus $c \rightarrow a$ we have:
		
		So we just proved now that in the first simplified proof of the Hospital rule of the relation:
		
		we had is true in general and that it is not necessary that $f(a)=g(a)=0$ is true for the result to be fair!
		\begin{flushright}
			$\square$  Q.E.D.
		\end{flushright}
	\end{dem}
	
	\pagebreak
	\subsubsection{Differentials}
	
	We noted earlier what is a differential $\mathrm{d}$. But there are actually several different types of differential of a function (note that we distinguish the masculine and feminine gender of the word):
	\begin{enumerate}
		\item Differentials
		\item Partial differentials
		\item Total exact differentials
		\item Total inexact differentials
	\end{enumerate}
		Remember as seen at the beginning of this section hat we name  "\NewTerm{differential $\mathrm{d}f$}" of an univariate function the relation given by:
	
	However, to express the effect of changing all the variables of a multivariate function $f$, we must use another type of differential which we name the "\NewTerm{total differential}\index{total differential}" (derived into two subfamilies: total exact differential and total inexact differential).
	
	Take, for example, a function $f(x, y)$ of the two variables $x$ and $y$. The increase $df$ of the function $f$, for a finite increase of $x$ to $x+\Delta x$ and of $y$ to $y+\Delta y$ is obviously given by:
	
	we can also write:
	
	Or also:
	
	For infinitely small increments of $x$ and $y$:
	
	Let us therefore focus on the two terms when going to the limit:
	
	The first term on the left, we see it clearly, finally gives the change in $x$ of the function $f(x, y)$ by having $y$ fixed on the variation. We denote therefore this by (if the fixed variables are trivial we do note write them):
	
	and even:
	
	\begin{tcolorbox}[title=Remark,colframe=black,arc=10pt]
When a variable is fixed to study the variation of the other, some authors and teachers of older generations like to say, "all things being equal $f$ varies in function of ... as ...". In short, it is an expression that can be found in other areas of mathematics (such as multivariate linear regressions) but that is disappearing...
	\end{tcolorbox}
	Both expressions:
	
	are what we name "\NewTerm{partial differential}\index{partial differential}" or just "\NewTerm{partial derivative}\index{partial derivative}" (whose practical application case of the simplest and probably the most interesting and pedagogically relevant example available today on the entire book is the supply chain Wilson's model with rupture presented in the section of Quantitative Management).
	
	We have therefore:
	
	that is the "\NewTerm{differential of $f$}". The thermodynamicists them often talk of "\NewTerm{total exact differential of $f$}" or just "\NewTerm{exact differential of $f$}" or even simple "\NewTerm{total derivative}" and also "\NewTerm{exterior derivative}\index{exterior derivative}".
	
	The previous relation is a special case of what mathematicians name very generally a "\NewTerm{differential form}\index{differential form}":
	
	we will come back about this a little further below... 
	
	It is customary to write:
	
	that is to say under vector field way.
	
	It is important to remember the form of the total derivative because we will meet it again almost everywhere in special operators in quantum physics, in fluid mechanics, in electrodynamics, in thermodynamics, in economy, etc.
	
	Geometrically, the partial derivatives can be interpreted as follows: the function $f(x, y)$ defines typically a surface in $\mathbb{R}$ whose intersection with the plane $y=y_0$ is a curve given by $f(x,y_0)$.
	
	The partial derivative $\partial_x f$ is then the slope of this curve at every point $x$. We then naturally the following function for the slope at the point $(x_0,y_0)$:
	
	Similarly, the tangent to the curve $f(x_0,y)$ will be given by:
	
	The plane locally tangent to the point $(x_0,y_0)$ determined by its two tangents is then given by:
	
	Reorganizing the terms as:
	
	We recognize:
	
	Thus, for example, the surface represented by the function:
	
	is shown below with the two tangents passing through the point:
	
	and that respective equations are:
	
	and:
	
	\begin{figure}[H]
	\centering
	\includegraphics[scale=0.75]{img/algebra/total_derivative_two_tangents.jpg}
	\caption{Both partial derivatives tangents of the function at the point of interest}
	\end{figure}
	The "\NewTerm{tangent plane}\index{tangent plane}" at this point is then given by:
		
	\begin{figure}[H]
		\centering
		\includegraphics[scale=0.75]{img/algebra/total_derivative_tangent_plane.jpg}
		\caption[]{The two tangents of the function at the point of interest with the tangent plane}
	\end{figure}
	
	\begin{tcolorbox}[title=Remark,colframe=black,arc=10pt]
Similarly, for a function of more than two variables, for example $f(x,y,z)$, the total derivative $\mathrm{d}f$ is given by:
	
	In the above equation, the differential $\mathrm{d}f$  was calculated from the expression of the function $f$. Since there is a function $f$ satisfying an expression $\mathrm{d}f$ , the derivative $\mathrm{d}f$  differential is then, as we know, named the "total exact derivative".
	\end{tcolorbox}	
	
	We take the opportunity to make an important indication on the use of partial derivatives by physicists (and therefore in many chapter of this book). We have seen that if $f$ depends on two variables $x, y$ we have:
	
	and if it depends of only one variable we have then:
	
	And therefore in the univariate case:
	
	this is why many physicists mixed the two notations...
	
	Now you have to know however that there are also total exact derivative that no functions satisfies. In this case, we speak about "total inexact derivative" and for determining whether a total derivative is exact or inexact, we use the properties of partial derivatives (very important case in thermodynamics!!!).
	
	Given the famous general differential form (notation used typically in differential geometry):
	
	where $M (x, y)$ and $N (x, y)$ are functions of the variables $x$ and $y$. If $\mathrm{d}z$ is an exact total differential, then:
	
	This requires verbatim that:
	
	or, by performing a second partial derivative that:
	
	so that the differential form is an exact total derivative.
	
	Before continuing, we need a result given by the "\NewTerm{Schwarz theorem}\index{Schwarz theorem}" (but which was proved in the late 17th century by one of the Bernoulli brothers) which is the following:
	
	\begin{theorem}
	Given a function $f$, if:
	
	are continuous (we must really check that this assumption is true!) then we get a very important result in practice:
	
	for every $(x_0,y_0)\in U$ where $U$ is the domain of definition where $f$ is continuous (and therefore assumed to be differentiable).
	\end{theorem}
	\begin{dem}
	We consider the expression:
	
	Let us write:
	
	Then we have:
	
	By the mean value theorem:
	
	With $s,t \in ]0,1[$. By taking the definitions of $w$ and $g$ we get:
	
	by applying the again the mean value theorem to both sides in brackets we find:
	
	With $\tilde{s},\tilde{t} \in ]0,1[$. To finish we see that we have:
	
	and by continuity when $k,h\rightarrow 0$, we have:
	
	More simply written:
	
	So if $f$ is expressed in an total derivative  form therefore the cross differentials are equal (the reciprocal is not necessarily true!).
	\begin{flushright}
		$\square$  Q.E.D.
	\end{flushright}
	\end{dem}
	By induction on the number of variables we can prove in the general case (that is long but it is possible we will do it if it needs to be done and we have the time...).
	
	So finally getting back to our original problem, we have:
	
	Which finally gives us the "\NewTerm{Schwarz condition}\index{Schwarz condition}":
	
	This is the condition that must be met for a differential form to be an exact total differential and the condition that it should not meet to be an non-total eact differential!!! This is a very important property for the study of Thermodynamics (see corresponding section)!
	
	In order not to confuse the two types of differentials, we use the symbol $\delta$ to represent a non-total exact differential:
	
	and $\mathrm{d}$ for a total exact differential:
	
	The distinction is extremely important because only the total exact differentials that satisfy:
	
	have an integral that depends \underline{only} on the limits of integration (since all the variables change at the same time). Therefore non-total exact differentials depend not \underline{only} on the limits of integration, meaning that:
	
	and therefore on a closed path we can have:
	
	While for total exact differentials:
	
	and therefore (see detailed proof later when we will deal with curvilinear integrals):
	
	In other words, the variation of a function whose differential is total exact, does not depend on the path followed, but only of the initial and final states as it can be expressed as the gradient of a function (see the proof by example in the section of Electrostatics when we check that the electrostatic potential difference is independent of the path). We name such a function that satisfies an exact total differential in physics a "\NewTerm{state function}\index{state function}" and in mathematics a "\NewTerm{holomorphic function}\index{holomorphic function}" (see section Complex Analysis for details), that is, i.e. a function whose value depends only on the present and future state, not its history.
	
	This distinction is very important and especially in thermodynamics where it should be determined whether a physical quantity is a total exact differential (a "state function"!) or not to know how systems evolve.
	
	\begin{tcolorbox}[colframe=black,colback=white,sharp corners]
\textbf{{\Large \ding{45}}Example:}\\\\
An important example of differential form in thermodynamics, is the elementary work $\delta W$ of  force exerted on a body in motion in the plane $\text{O}xy$, we have:
	
	$F_x,F_y$ are not necessarily derived from the same potential $U(x, y)$ such that:
	
	therefore  $\delta W$ is indeed in this special case a non-total exact differential!
	\end{tcolorbox}
	 
	 \subsubsection{Usual Derivatives}
	 
	 We will prove in details here the most common univariate derivatives (around a small thirty) and some of their major properties that we will meet in theoretical physics and mathematics (in fact, we will used them all in respectively the chapters on Mechanics, Engineering, Atomistic, Social Mathematics, etc.). The list below is not exhaustive at this time but the proofs being generalized, they can be applied to many other similar cases (that we will apply / meet throughout almost all this book).
	 \begin{enumerate}
	 	\item Derivative of $f(x)=x^n$:
	 	
	 	First start with a particular case, the derivative of $x^3$:
	 	
		Therefore the derivative of the cubic function is $3a^2$.
		
		We can generalize this result for any positive or negative integer $n$ and we will see that the function $f$ defined on $\mathbb{R}$ by $f(x)=x^n$ is differentiable and that its derivative $f'$ is given by $f'(x)=nx^{n-1}$.
		 
		 Therefore we have (a few examples can be helpful to understand the scope of this result!):
		 
		 So we see that having determined the derivative of a function of the form $x^n$, we also determined the derivative of any function that written in such a form:
		 
		 However, the following functions:
		 
		 are not differentiable in the $x=0$ because the function is not more defined on this point (division by zero). Furthermore, relatively to the root function (not integer power), the derivative is not defined in $\mathbb{R}_{-}^{*}$.
		 
		 \item Derivative of $f(x)=c^{te}$:
		 
		 The previous result gives an interesting immediate result for constant functions such as:
		 
		 it is then not difficult to determine that the derivative is simply:
		 
		 So the derivative of any constant function is zero (it is important to remember that result when we will study the properties of integrals) !!!
		 \item Derivative of $f(x)=\cos(x)$:
		 
		 Given any real fixed number $a$, then (be careful it is useful to know the remarkable trigonometric relations that we proved in the section Trigonometry!):
		 
		 Because using Hospital rule (or by seeing that $\sin(x)$ can be assimilate by a straight line function $f(x)=x$ near $x=0$):
		 
		 So to summarize:
		 
		 
		 \item Derivative of $f(x)=\sin(x)$:
		 
		 Given any real fixed number $a$, then (be careful again it is useful to know the remarkable trigonometric relations that we proved in the section Trigonometry!):
		 
		 Because using Hospital rule (or by seeing that $\sin(x)$ can be assimilate by a straight line function $f(x)=x$ near $x=0$):
		 
		 So to summarize:
		 
		 \item Derivative of $f(x)=\log_b(x)$:
		 We begin by writing that:
		 
		 Therefore:
		 
		 Therefore:
		 
		 Multiply and divide by $x$ the term in the right member of the last previous equality:
		 
		 Denote the quantity $\dfrac{\Delta x}{x}$ by $\alpha$. It is obvious that $\alpha \rightarrow 0$ when $\Delta x$ tends to zero for a given $x$. Consequently:
		 
		 However, we find here again an another historical origin from Euler's constant (see the section Functional Analysis for the proof), that is:
		 
		 Therefore:
		 
		 An important special case is the case where $b = e$. Then we have the famous result:
		  
		 \item Derivative of a sum of functions:
		 
		 Let us now consider $u$ and $v$ two functions. The sum function $s=u+v$ is derivable over any interval where $u$ and $v$ are differentiable, we will denote by $s'$ the derivative of the sum. Let us now see what is its expression.
		 
		 Let $a$ be a real number and $u,v$ two defined and differentiable functions on $a$:
		 
		 
		 So the derivative of a sum is the sum of the derivatives.
		 
		 This result can be generalized for a sum of any number of functions.
		 
		 \item Derivative of a product of functions:
		 
		 Let us now consider $u$ and $v$ two functions. The product function $p=uv$ is derivable over any interval where $u$ and $v$ are differentiable, we will denote by $p'$ the derivative of the product. Let us now see what is its expression.
		 
		 Let $a$ be a real number and $u,v$ two defined and differentiable functions on $a$:
		 
		 
		 We add to this last relation two terms whose sum is zero such as:
		 
		 
		 But there is a more general formulation than the first derivative of a product:
		 \begin{theorem}
		 	Consider for this purpose always our two functions $u$ and $v$, $n$ times differentiable on an interval $I$. Then the $uv$ product is $n$ times differentiable on $I$ and:
		 		 	
		 	and this constitutes the "Leibniz formula" that we used in the section Calculus for the study of Legendre polynomials (which we are themselves essential to our study of Quantum Chemistry).
		 	The proof of this expression is very similar to that made for the Newton's binomial theorem (\SeeChapter{see section Calculus}).
		 \end{theorem}
		 \begin{dem}
		 	Either:
		 	
		 	On the other hand:
		 	
		 	The relation is thus at least well initialized.
		 	
		 	The proof is made by induction. Thus, the goal is to show that for $\forall n \geq 0 \in N$ if:
		 	
		 	then:
		 	
		 	We have therefore:
		 	
		 	We'll do now a change of variable in the first sum to not have the term in $k + 1$ again. We put for this purpose $j=k+1$:
		 	
		 	If we go back to the letter $k$, we have:
		 	
		 	So we have:
		 	
		 	We want to combine this two sums. For this, we discard the terms in excess in each:
		 	
		 	This therefore gives:
		 	
		 	According to Pascal's formula (\SeeChapter{see section Probabilities}), we have:
		 	
		 	Therefore:
		 	
		 	But we have at the same time:
		 	
		 	Therefore:
		 	
		 	\begin{flushright}
				$\square$  Q.E.D.
			\end{flushright}
		 \end{dem}
		 \item Derivative of a composite univariate function:
		 
		 Let the us consider the composite function $f=g \circ u=g(u(x))$ of two functions $g$ and $u$ differentiable, the first in $u (a)$, the second in $a$. We have therefore:
		 
		 Let us put now $k=u(a+h)-u(a)$ then we have:
		  
		 Let us continue our previous development:
		  
		 Thus the derivative of a composite function is given by the derivative of the function, multiplied by the "\NewTerm{inner derivative}\index{inner derivative}". Furthermore, this type of derivation is very important because often used in physics under the name of "\NewTerm{(univariate) chain derivation}\index{chain derivation (univariate)}" or simply "\NewTerm{(univariate)  chain rule}\index{chain rule (univariate) }".
		 
		 Let's see what it is. The previous obtained relation can be rewritten in another more common way:
		 
		 Or typically when we have multiply function multiplies to each other:
		 
		 
		 \item Derivative of a composite bivariate function:
		 \begin{theorem}
		 The $t$ derivative of the composite function $z=f(x(t),y(t))$ is:
		  
		 We assume in this theorem and its applications that $x=x(t)$ and $y=y(t)$ have first derivatives at $t$ and that $z=f(x,y)$ has continuous first order derivatives in an open circle centered at $(x(t),y(t))$.
		 \end{theorem}
		 \begin{dem}
		 	We fix $t$ and set $(x,y)=(x(t),y(t))$. We consider nonzero $\Delta t$ so small that $(x(t+\Delta t),y(t+\Delta t))$ is in the circle where $f$ has continuous first derivatives and set $\Delta x=x(t+\Delta t)-x(t)$ and $\Delta y=y(t+\Delta t)-y(t)$. Then by definition of the derivative:
		 	
			We can apply the "mean value theorem" that states (see the proof further below during our study of integral calculus):
						
			 to the expression in the first set of square brackets on the right of the last equality above where $y$ is constant and to the expression in the second set of square brackets where $x$ is constant. We conclude that there is a number $c_1$ between $x$ and $x+\Delta x$ and a number $c_2$ between $y$ and $y+\Delta y$ such that:
			
			We add the both relations above and divide by $\Delta t$ to get:
			
			The function $x=x(t)$ and $y=y(t)$ are continuous at $t$ because they have derivatives at that point. Consequently, as $\Delta t\rightarrow 0$, the numbers $\Delta x$ and $\Delta y$ both tend to zero and the circle including the constants $c_i$ collapses to the point $(x,y)$, Because the partial derivatives of $f$ are continuous, the term $\partial_x f(c_1,y+\Delta y)$ tends to $\partial_x f(x,y)$ and the term $\partial_y f(x,c_2)$ tends to $\partial_y f(x,y)$ as $\Delta t\rightarrow 0$. Moreover:
		
		as $\Delta t\rightarrow 0$, so the above relation becomes:
			
			 named the "\NewTerm{multivariate chain rule}\index{multivariate chain rule}" (but in reality it is only the bivariate case...) and that is veeeeery important for study physics.
			 
			 The latter relation sometimes written:
			 
		 	\begin{flushright}
			$\square$  Q.E.D.
			\end{flushright}
		 \end{dem}
		 
		 \item Derivative of an inverse function
		 \begin{theorem}
		 	If the function $f$ is continuous, strictly monotonic over an interval $I$, derivable over $I$, then the reciprocal function $f^{-1}$ is derivable on the interval $f(I)$ and admits for derivative function:
		 	
		 \end{theorem}
		 \begin{dem}
		 	Indeed we can write:
		 	
		 	That is to say (identity application):
		 	
		 	By application of the derivation of composite functions seen just above we have:
		 	
		 	Therefore:
		 	
		 	For a variable $x$, it is more common to write for the derivative of the inverse function as following:
		 	
		 	\begin{flushright}
			$\square$  Q.E.D.
			\end{flushright}
		 \end{dem}
		 \item Derivative of $\arccos (x)$:
		 
		 	Using the previous result of the reciprocal function and the derivative of $\cos (x)$ proved above, we can calculate the derivative of the function $\arccos (x)$:
		 	
		 	\item Derivative of $\arcsin (x)$:
		 
		 	Using the previous result of the reciprocal function and the derivative of $\sin (x)$ proved above, we can calculate the derivative of the function $\arcsin (x)$:
		 	
		 	\item Derivative from a quotient of two functions:
		 	
		 	Consider that function:
		 	
		 	is derivable over any interval where $u$ and $v$ are differentiable functions and wherein the function $v$ is not null.
		 	
		 	The function $f$ can be considered as the product of two functions: the function $u$ and the function $1/v$. A product of two functions is differentiable if each is differentiable, it is necessary that the function $u$ is differentiable and the function $1/v$ is also derivable which is the case when $v$ is differentiable and not null.
		 	
		 	\item Derivative of the function $\tan(x)$:
		 	
		 	By definition (\SeeChapter{see section Trigonometry}), $\forall x \neq k\dfrac{\pi}{2},k\in \mathbb{Z}$:
		 	
		 	and then applying the derivative rule for a quotient as proved above, we have:
		 	
		 	or:
		 	
		 	\item Derivative of the function $\cot(x)$:
		 	
		 	By definition (\SeeChapter{see section Trigonometry}),$\forall x \neq k\pi,k\in \mathbb{Z}$:
		 	
		 	and therefore (applying once again the rule of the derivative of a quotient as proved previously):
		 	
		 	or:
		 	
		 	\item Derivative of the function  $\arctan(x)$:
		 	
		 	We use the properties of the derivative of reciprocal functions proved previously:
		 	
		 	\item Derivative of the function  $\text{arccot}(x)$:
		 	
		 	We also the properties of the derivative of reciprocal functions proved previously:
		 	
		 	\item Derivative of $e^x$:
		 	
		 	We will prove in our study of Theoretical Computing (see section of the same) that the "Euler number" can be calculated from the series:
		 	
		 	which converges on $\mathbb{R}$. By derivating term by term this series that converges, we get:
		 	
		 	Thus the exponential is its own derivative. So now we can afford to study the derivatives of some hyperbolic trigonometric functions (\SeeChapter{see section Trigonometry}) and many other specific cases (see all other chapters of the book).
		 	\item Derivative of $\sinh(x)$:
		 	
		 	Remember (\SeeChapter{see section Trigonometry}):
		 	
		 	So trivially:
		 	
		 	
		 	\item Derivative of $\cosh(x)$:
		 	
		 	Remember (\SeeChapter{see section Trigonometry}):
		 	
		 	So trivially:
		 	
		 	
		 	\item Derivative of $\tanh(x)$:
		 	
		 	Remember (\SeeChapter{see section Trigonometry}):
		 	
		 	Therefore by applying the derivative from a quotient we obtain:
		 	
		 	or:
		 	
		 	\item Derivative of $\coth(x)$:
		 	
		 	Remember (\SeeChapter{see section Trigonometry}):
		 	
		 	Therefore by applying the derivative from a quotient we obtain:
		 	
		 	
		 	\item Derivative of $\text{arcsinh}(x)$:
		 	
		 	We also use the properties of the derivative of reciprocal functions proved previously:
		 	
		 	But (see again the section Trigonometry):
		 	
		 	and therefore:
		 	
		 	Since $\cosh(x)$ takes only positive values, we have:
		 	
		 	Then finally:
		 	
		 	\item Derivative of $\text{arccosh}(x)$:
		 	
		 	We also use the properties of the derivative of reciprocal functions proved previously:
		 	
		 	But (see again the section Trigonometry):
		 	
		 	and therefore:
		 	
		 	Since $\text{arccosh}(x)$ takes only positive values so do $\sinh(x)$, then we have:
		 	
		 	Then finally:
		 	
		 	\item Derivative of $\text{arctanh}(x)$:
		 	We also use the properties of the derivative of reciprocal functions proved previously:
		 	
		 	\item Derivative of $\text{arccoth}(x)$:
		 	We also use the properties of the derivative of reciprocal functions proved previously:
		 	
		 	\item Derivative of $a^x$ with $a>0$:
		 	
		 	So (derivative of a composite function):
		 	
		 \end{enumerate}
		 
		 \pagebreak
		\subsubsection{Implicit Differentiation}
	Thus far, the functions we have been concerned with functions that have been defined explicitly. A function is defined explicitly if {\it the output is given directly in terms of the input}. For instance, in the function:
		
	the value of $f(x)$ is given explicitly or directly in terms of the input. Just by knowing the input we can immediately find the output. A second type of function that is also useful for us to consider is an "\NewTerm{implicitly defined function}\index{implicitly defined function}". A function is defined implicitly if {\it the output cannot be found directly from the input}. For instance (stupid simple example):
	
	 is an implicitly defined function, because for each positive $x$ value there is a corresponding $f$ value, but we cannot find it directly from the function. We would need to square both sides, and then we would have the explicitly defined function:
	 	
	It is also possible for us to have implicitly defined functions that we cannot rewrite as an explicitly defined function!!! 
	
	For instance we might have the function:
	
	For a given $x$ value, there {\it may} be a corresponding output value $f(x)$ which makes this a true statement. In this way a function $f(x)$ would be defined for all such $x$ where there is a solution. For instance, we have $f(0) = 0$, because setting $x=f(x)=0$ in the above equation is a true statement. Right now we don't have the proper tools to solve such an equation, but the important concept here is that we can have a function defined in such a way. 
	
	When we speak of functions, we mean that we have a rule which provides us with at most one output for a given input (there is no output for inputs at which the function is not defined). In a more general sense we might want to look at rules that provide us with multiple outputs for a given input. Such an example would be the equation:
	
	which is the equation for the unit circle (\SeeChapter{see section Analytical Geometry}). It turns out that this object consists of two functions, namely:
	
	The equation of this circle does not define a function (because the output is multi-valued!!), but it does define some type of curve in the $x$-$y$ plane. In general, we should be able to describe an arbitrary curve as a combination of some number of functions. It is sensible (and useful) to consider the slope of certain points on the curve, which would simply correspond to the slope of the specific function that defines that part of the curve. 
	
	To solve the problem of finding the derivative of a function defined in a way such as $\sin(f(x)) + f(x) = x$ or by a curve like a circle, we employ the chain rule once again. The way in which we will employ it is named "\NewTerm{implicit differentiation}\index{implicit differentiation}". The process works as follows: we differentiate both sides of the equation with respect to $x$ (or the independent variable), and then we solve for the derivative of the dependent variable. Anywhere we find the function $f$ (or the dependent variable), we will use the chain rule to find the derivative. Let us begin with a few simple examples:
	\begin{tcolorbox}[colframe=black,colback=white,sharp corners]
	\textbf{{\Large \ding{45}}Examples:}\\\\
	E1. We want to found $\mathrm{d}y/\mathrm{d}x$ if $y^2 = x$.\\
	
	One method of solving this problem would be rewriting it in terms of an explicit function of $y$, and differentiating. Since we have $y = \pm \sqrt{x}$, we actually have two functions, and we would find:
	
	This works sufficiently well in this situation, but what about a function we cannot rewrite explicitly? We would need implicit differentiation. Let's apply implicit differentiation to this situation, as a means of exercise. 
	
	Were we unable to rewrite $y$ explicitly in terms of $x$, this is as far as we could go, but we know that $y = \pm \sqrt{x}$, and plugging in this result we find that 
	$$\frac{dy_1}{dx} = \frac{1}{2\sqrt{x}} \quad \text{and} \quad \frac{dy_2}{dx} = -\frac{1}{2\sqrt{x}}$$
	once again. In this way, we were able to calculate both derivatives at once.\\
	
	E2. We want to found $\mathrm{d}f/\mathrm{d}x$ for $\sin(f) + f = x$.\\
	
	Here we have no choice but to apply implicit differentiation. 
	
	\end{tcolorbox}
	A famous case of application in pure mathematics of implicit differentiation (we will see later for physics) is the one at the origin of this technique: the "\NewTerm{folium of Descartes}\footnote{The name comes from the Latin word folium which means "leaf".}" defined by:
	
	The curve was first proposed by René Descartes in... 1638.
	\begin{figure}[H]
		\centering
			\includegraphics[scale=0.9]{img/algebra/folium_of_descartes.jpg}
		\caption{The folium of Descartes (green) with asymptote (blue) (source: Wikipedia)}
	\end{figure}
	Its claim to fame lies in an incident in the development of calculus. Descartes challenged Pierre de Fermat to find the tangent line to the curve at an arbitrary point since Fermat had recently discovered a method for finding tangent lines. Fermat solved the problem easily, something Descartes was unable to do. Since the invention of calculus, the slope of the tangent line can be found easily using implicit differentiation in any point as we will show it:
	
	Consider now that we want to found the slope of the curve at the point $(2,4)$. Then we use implicit differentiation: 
	
	Evaluating the derivative at the point $(2,4)$ we find:
	
	Now that we have the slope of the tangent line at the point of interest, we use the point-slope form\index{point-slope form}:
	
	Now as far as the line normal to the curve at this point is concerned, we need to find the line perpendicular to the tangent line. This line will cross through the same point, but the slope will be the negative reciprocal of the slope of the tangent line. It follows that:
	
	
	Ok now let us focus on an example applied to physics (putting apart the famous case of looking for the tangent of an ellipse). We will prove in the section of Continuum Mechanics the Van der Waals equation:
	
	If $T$ remains constant, consider that we want to found the rate of change of the volume with respect to the pressure, that is to say $\mathrm{d}V/\mathrm{d}P$! We can challenge here to calculate this rate to transform this equation into an explicitly defined function... So we know we have to use implicit differentation. First:
	
	This gives immediately:
	
	Now we distribute:
	
	After simplification and rearrangement we get:
	
	Finally:
	
	The special case of the examples above with the Van der Waals equation and the folium of Descartes is in some text books introduced as following with a two variables case:
	
	Therefore:
	
	Reduce to:
	
	Finally:
	
	That's it for now. We will stop here on this topic as we don't need more techniques or example to our study of physics and engineering as introduced in this book.
	
		\pagebreak
		\subsubsection{Smoothness}
		 Smoothness has to do with how many derivatives of a function exist and are continuous. The term smooth function is often used technically to mean a function that has derivatives of all orders everywhere in its domain.
		 
		 \textbf{Definition  (\#\mydef):} A "\NewTerm{differentiability class}\index{differentiability class}" is a classification of functions according to the properties of their derivatives. Higher order differentiability classes correspond to the existence of more derivatives.
		 
		 Consider an open set on the real line $\mathbb{R}$ and a function $f$ defined on that set with real or complex values. Let $k$ be a non-negative integer. The function $f$ is said to be of (differentiability) class $\mathcal{C}^k$ if the derivatives $f', f'', ..., f(k)$ exist and are continuous (the continuity is implied by differentiability for all the derivatives except for $f(k)$). The function $f$ is said to be of class $\mathcal{C}^\infty$, or smooth, if it has derivatives of all orders. The function $f$ is said to be of class $\mathcal{C}^\omega$, or simply "analytic", if $f$ is smooth and if it equals its Taylor series expansion around any point in its domain (\SeeChapter{see section Sequences and Series}). $\mathcal{C}^\omega$ is thus strictly contained in $\mathcal{C}^\infty$.

		\pagebreak
		 \subsection{Integral Calculus}
		 We will discuss here the basic principles of integral calculus in $\mathbb{R}^n$. Most advanced topics will come depending the time that are available to the redactors of this book but the reader can already refer the Complex Analysis section for integration techniques based on the Residue Theorem that is almost powerful and useful especially for some integral in quantum physics.
		 
		 \subsubsection{Definite Integral}
		 The origin of Integral Calculus seems to come from Archimedes that was fascinated with calculating the areas of various shapes. He used a process that has come to be known as the \textit{method of exhaustion}, which used smaller and smaller shapes, the areas of which could be calculated exactly, to fill an irregular region and thereby obtain closer and closer approximations to the total area. In this process, an area bounded by curves is filled with rectangles, triangles, and shapes with exact area formulas. These areas are then summed to approximate the area of the curved region. This subsection introduces the Definite Integrals.
		 		 
		 The first idea of the concept of integral is to calculate the algebraic area (positive area if above the $x$-axis or negative below) between a curve and its support. See the figure below with a positive area and the notations for the developments that will follow:
		 \begin{figure}[H]
			\centering
			\includegraphics{img/algebra/integral_all_positive.jpg}
			\caption[]{Area $A$ to be calculated in a bounded continuous positive function}
		\end{figure}
		Or with different algebraic areas (the difference between the blue area and the yellow area is named the ""\NewTerm{net signed area}\index{net signed area}""):
		 \begin{figure}[H]
			\centering
			\includegraphics{img/algebra/integral_all_positive_and_negative.jpg}
			\caption[]{Area $A$ to be calculated in a bounded continuous positive and negative function}
		\end{figure}
		An approximate value of the area under a curve can be achieved by a division into $n$ vertical rectangular bands of the same width. In particular, we can achieve a framework of this area with the help a sum of upper bound areas (majorant) $A_M$ or a lower bound sum (minorant) $A_m$ for a given cutting:
		\begin{figure}[H]
			\centering
			\begin{subfigure}{0.4\textwidth}
				\includegraphics[width=\textwidth]{img/algebra/integral_minorant_sum.jpg}
				\caption{Minorant sum of areas $A_m$}
			\end{subfigure}
			\begin{subfigure}{0.4\textwidth}
				\includegraphics[width=\textwidth]{img/algebra/integral_majorant_sum.jpg}
				\caption{Majorant sum of areas $A_M$}
			\end{subfigure}				
		\end{figure}
		Suppose now that the number $n$ of bands tends to infinity. As the bands are of the same width, the width of each bands tends to $0$ (objectively it is not necessary that the width of the cutting of the subintervals is the same everywhere).
		
		If the sums $A_m$ and $A_M$ have both a limit when the number $n$ of bands tends to infinity, then the area $A$ under the curve is between these two limits. We write this:
		
		Obviously if these two limits are equal, their value is that of the area $A$ under the curve.
		
		Hence a first direct definition of the definite integral named also "\NewTerm{Riemann integral}\index{Riemann integral}".
		
		\textbf{Definition (\#\mydef):}
		Given an interval $[a, b]$, divided into $n$ equal parts, let $f$ be a continuous function on the interval $[a, b]$, consider $A_m$, the algebraic minorante sum of areas and $A_M$ the algebraic majorant sum. We name "\NewTerm{definite integral}\index{definite integral}" of $f$ from $a$ to $b$, denoted by:
		
		the number $A$ such that:
		
		provided that this limit exists. If this limit exists, then we say that $f$ is "integrable" on $[a, b]$ and the definite integral exists.
		
		The symbol:
		
		Is only the symbol of a discrete sum $\sum$ but applied to the case of infinitely small increments.
		
		The numbers $a$ and $b$ of the integral (that can also be functions sometimes!) are named the "\NewTerm{integration limits}\index{integral limits}" or "\NewTerm{integration bounds}\index{integral bounds}": $a$ is the "\NewTerm{lower bound}", $b$ is the "\NewTerm{upper bound}".
		
		\begin{center}
		\begin{tikzpicture}[scale=2.3]
		  \shade[top color=blue,bottom color=gray!50] 
		      (0,0) parabola (1.5,2.25) |- (0,0);
		  \draw (1.05cm,2pt) node[above] 
		      {$\displaystyle\int_0^{3/2} \!\!x^2\mathrm{d}x$};
		
		  \draw[style=help lines] (0,0) grid (3.9,3.9)
		       [step=0.25cm]      (1,2) grid +(1,1);
		
		  \draw[->] (-0.2,0) -- (4,0) node[right] {$x$};
		  \draw[->] (0,-0.2) -- (0,4) node[above] {$f(x)$};
		
		  \foreach \x/\xtext in {1/1, 1.5/1\frac{1}{2}, 2/2, 3/3}
		    \draw[shift={(\x,0)}] (0pt,2pt) -- (0pt,-2pt) node[below] {$\xtext$};
		
		  \foreach \y/\ytext in {1/1, 2/2, 2.25/2\frac{1}{4}, 3/3}
		    \draw[shift={(0,\y)}] (2pt,0pt) -- (-2pt,0pt) node[left] {$\ytext$};
		
		  \draw (-.5,.25) parabola bend (0,0) (2,4) node[below right] {$x^2$};
		\end{tikzpicture}
		\end{center}
		
		Intuitively, it is obvious that when $a=b$ we extend the definition as follows:
		
		Finally, notice that it is quite possible that the result of the integral to be negative or even complex since it is an algebraic surface! That is to say the result can be in general in $\mathbb{C}$.
	\begin{tcolorbox}[title=Remarks,colframe=black,arc=10pt]
	\textbf{R1.} Other letters rather than $x$ can be used in the evaluation of the definite integral. So if $f$ is integrable on $[a, b]$, then $\int\limits_a^bf(x)\mathrm{d}x=\int\limits_a^bf(t)\mathrm{d}t=\int\limits_a^bf(s)\mathrm{d}s$, etc. This is why the variable $x$ is sometimes named "\NewTerm{dummy variable}\index{dummy variable}".\\

	\textbf{R2.} As we will see below, it is essential not to confuse between a "\NewTerm{definite integral}\index{definite integral}" and a "\NewTerm{indefinite integral}\index{indefinite integral}". Thus, an indefinite integral, denoted $\int\limits f(x)\mathrm{d}x$ is a function, or more precisely, a family of functions also named "\NewTerm{primitives of $f$}\index{primitive of a f unction}" (see below) while a definite integral, denoted $\int\limits_a^b f(x)\mathrm{d}x$ is considered as a constant.
	\end{tcolorbox}
	Let us present a second approach for the definition of the integral, somewhat more rigorous than the previous one  (following the request of several readers). We will use this time, by tradition the $S$ surface instead of the area $A$.
	
	Let f be a bounded function on $[a, b]$. We consider a subdivision $\sigma$ of its support $[a, b]$ that we note:
	
	where the intervals are not necessarily of equal sizes.

	We write for $i=1,2,3,...,n$:	
	
	
	\textbf{Definitions (\#\mydef):}
	\begin{enumerate}
		\item[D1.] We name "\NewTerm{lower Darboux sum}\index{lower Darboux sum}" associated with $f$ and $\sigma$ the surface:
		
		\begin{figure}[H]
			\centering
			\includegraphics{img/algebra/darboux_inferior_sum_concept.jpg}
			\caption{Principle of calculating the lower Darboux sum}
		\end{figure}
		\item[D1.] We name "\NewTerm{upper Darboux sum}\index{upper Darboux sum}" associated with $f$ and $\sigma$ the surface:
		
		\begin{figure}[H]
			\centering
			\includegraphics{img/algebra/darboux_superior_sum_concept.jpg}
			\caption{Principle of calculating the upper Darboux sum}
		\end{figure}
	\end{enumerate}
	A function $f$ is said to be "\NewTerm{Riemann-integrable on $[a, b]$}" if and only if the above two surfaces coincide when the intervals become infinitely small.
	
	All Riemann-integrable functions on $[a, b]$ are denoted by $\mathcal{R}_{[a,b]}$.
	
	Darboux sums are not very useful for the effective calculation of an integral, for example using a computer, because it is usually quite difficult to find the $\inf$ and $\sup$ on sub-intervals. Rather, we consider:
	
	The "\NewTerm{Riemann sum}\index{Riemann sum}" is defined from the fact that we if denote a "\NewTerm{partition}\index{partition}" (or "\NewTerm{regular partition}" if they all have the same width) of the interval $[a,b]$ by:
	
	and that:
	
	where $\xi_i \in [x_{i-1},x_i]$, then:
	
	But as we must choose an $\xi_i$, we often takes either the right or the left one, thus taking randomly the "method of left rectangles":
	
	Which would give us the example below:
	\begin{figure}[H]
		\centering
		\includegraphics{img/algebra/left_rectangle_integral.jpg}
		\caption{Principle of calculating methods using left rectangles}
	\end{figure}
	either:
	
	But it is easy for a step function... but it is less so for a continuous function for which we will obtain only an approximate value of the actual surface! The idea is then to take intervals smaller and smaller:
	\begin{figure}[H]
		\centering
		\includegraphics{img/algebra/riemann_integral_left_rectangle.jpg}
		\caption{Principle of the calculation of the Riemann integral with left rectangles method}
	\end{figure}
	And then, at the limit, we obtain the desired quantity:
	
	The fact to search this limit is named "calculate the integral," and more specifically for the chosen method: "calculate the Riemann integral".
	
	\begin{tcolorbox}[colframe=black,colback=white,sharp corners]
	\textbf{{\Large \ding{45}}Example:}\\\\
	We want to use the construction of the definite integral to evaluation:
	
	using a right-endpoint approximation to generate the Riemann sum.\\
	
	For this purpose we first set up the Riemann sum. Based on the limits of integration, we have $a=0$ and $b=2$. For $i=0,1,2,\ldots,n$, let $P=\{x_i\}$ be a regular partition of $[0,2]$. Then:
	
	Thus, the function value at the right endpoint of the interval is:
	
	Then the Riemann sum takes the form:
	
	Using the Gauss summation relation of $\sum_{i=1}^n i^2$ (\SeeChapter{see section Sequences and Series}), we have:
	
	Now, to calculate the definite integral, we need to take the limit as $n\rightarrow+\infty$. We get:
	 
	\end{tcolorbox}
	
	\subsubsection{Indefinite Integral}
	We have seen before in our study of derivatives the following problem: given a function denoted by $F(x)$, find its derivative $f(x)$, that is to say the algebraic function:
	
	\textbf{Definition (\#\mydef):} We say that the function $F (x)$ is a "\NewTerm{primitive}\index{primitive}" or "\NewTerm{indefinite integral}\index{indefinite integral}" of the function $f (x)$ on any segment $[a, b]$ if at any point in any segment we have $F'(x)=f(x)$.
	
	Tow more explicit and less technical alternative definitions are: 
	\begin{itemize}
		\item An "indefinite integral" is a FUNCTION of $x$ (or another variable), while a "definite integral" is a VALUE!
		
		\item The collection of all antiderivatives of $f(x)$ is named the "indefinite integral" of $f$ with respect to $x$.
	\end{itemize}

	
	Another way to see the indefinite integral concept is to go through the "\NewTerm{fundamental theorem of integral and differential calculus}\index{fundamental theorem of integral and differential calculus}" also sometimes named "\NewTerm{fundamental theorem of calculus}\index{fundamental theorem of calculus}" whose two properties are stated as follows:
	\begin{enumerate}
		\item If $A$ (area) is the function defined by $A(X)=\displaystyle\int\limits_a^X f(t)\mathrm{d}t$ for each $X$ in any $[a, b]$, then $A$ is the primitive of $f$ on $[a, b]$ which is zero in $a$ (or in other words: $f (t)$ is the derivative of $A$).
		\item If $F$ is a primitive of $f$ on any $[a, b]$, then:
		
	\end{enumerate}
	Let us prove the first property of this fundamental theorem:
	\begin{dem}
	Given the function:
	
	If $f$ is positive, and $h>0$ (the proof in the case where $h<0$ is similar) and as $X>0$, we know that we can think of $A(X)$ as the area under the curve of $f$ from $t=a$ to $t=X$.
	\begin{figure}[H]
		\centering
		\includegraphics{img/algebra/area_primitive_representation.jpg}
		\caption{Graphical representation of the area}
	\end{figure}
	To show that $A$ is a primitive of $f$, we will prove that $A'=f$. According to the definition of the derivative:
	
	Let us study this quotient: $A(x+h)-A(x)$ is represented by the area of the strip of width $h$, sandwiched between two rectangles of width $h$.
	
		Given $M$ the maximum of $f$ on the interval $[X,X+h]$ and $m$ the minimum interval of $f$ over the same interval. The respective areas of the two rectangles are $Mh$ and $mh$.
	
	We then have the following double inequality:
	

	As $h$ is positive, we can divide by $h$ without changing the meaning of  the inequalities:
	
	When $h\rightarrow 0^+$ and if $f$ is a continuous function, then $M$ and $m$ have for limit $f (X)$, and the ratio:
	
	which is between $m$ and $M$, has effectively for limit $f(X)$.
	
	As $A'(X)=f(X)$ for all $X$, this shows that the derivative of the area function is $f$. And therefore $A$ is an indefinite primitive of $f$. As $A(a)=0$, $A$ is effectively the primitive of $f$ which vanishes on $a$.
	\begin{flushright}
		$\square$  Q.E.D.
	\end{flushright}
	\end{dem}
	Before starting the proof of the second properties of the fundamental theorem, let us give and prove the following theorem that will we be essential to us: If $F_1(x)$ and $F_2(x)$ are two primitives of $f (x)$ on any segment $[a, b]$, their difference is a constant (this result is very important in physics in terms of the study of what we name the "initial conditions").
	\begin{dem}
	We have after the definition of the concept of Primitive:
	
	for $\forall x \in [a,b]$.
	
	Let us write:
	
	We can write:
	
	So it comes from what we saw during our study of derivatives that:
	
	Then we have:
	
	\begin{flushright}
		$\square$  Q.E.D.
	\end{flushright}
	\end{dem}
	It follows from this theorem that if we know any primitive $F (x)$ of the function $f (x)$ any other primitive of this function will be of the form:
	
	So finally, we name "\NewTerm{indefinite integral}\index{indefinite integral}" of $f (x)$ and we denote by:
	
	any expression of the form $F (x)+c^{te}$ where $F(x)$ is a primitive of $f(x)$. Thus, by writing convention:
	
	if and only if $F'(x)=f(x)$.
	
	In this context, $f (x)$ is also named "\NewTerm{integrand function}\index{integrand function}" and $f (x) \mathrm{d}x$, the "\NewTerm{function under the sum sign}".
	\begin{tcolorbox}[title=Remarks,colframe=black,arc=10pt]
	An "\NewTerm{antiderivative}\index{antiderivative}" of a function $f$ is one function $F$ whose derivative is $f$. The indefinite integral of $f$ is the set of ALL antiderivatives of $f$. Therefore indefinite integral and antiderivative is not the same as the first is the set of all the seconds! If $f$ and $F$ are as described just now, the indefinite integral of $f$ has the form $\{F+c^{te}|c^{te}\in\mathbb{R}\}$ when an antiderivative is just one element of this set!! But not all teachers are according on the definition of antiderivatives...
	\end{tcolorbox}
	
	Geometrically, we can consider the indefinite integral as a set (family) of curves as we move from one to another by performing a translation in the positive or negative direction of the axis.
	
	Let us return to the proof of item (2) of the fundamental theorem of integral (and differential) calculus:
	\begin{dem}
	Let $F$ be a primitive of $f$. Since two primitives differ by a constant, we have well:
	
	that we can also write:
	
	for all $X$ in $[a, b]$. The particular case $X=a$ gives $\int\limits_a^a f(t)\mathrm{d}t$ and therefore $F(a)+c^{te}=0$ and we get obviously $c^{te}=-F(a)$. Substituting, we get:
	
	As this identity is valid for all $X$ in the interval $[a,b]$ it is true especially for $X=b$. Therefore:
	
	\begin{flushright}
		$\square$  Q.E.D.
	\end{flushright}
	\end{dem}
	This last result also shows something useful!: It is not necessary when we evaluate and integral to take into account the constant of the general primitive since it is canceled by the difference of the two primitives!!
	\begin{tcolorbox}[title=Remarks,colframe=black,arc=10pt]
	\textbf{R1.} The above fundamental theorem that shows the link between primitive and integral led us to use the same symbol $\int$ to write a primitive (introduced by Leibniz in the late 17th century), which is a function, and an integral that it is a number.\\
	
	\textbf{R2.} We also proved in the section of Analytical Mechanics using an integral how to calculate the full length of a curve in the plane if the function $f(x)$ is explicitly known.
	\end{tcolorbox}
	Here are some trivial properties of integration that is good to remember because often used elsewhere in this book (if it do not seem obvious to you, contact us and we will give the detailed proof):
	\begin{enumerate}
		\item[P1.] The derivative of an indefinite integral is equal to the integrand:
		
		\item[P2.] The differential of an indefinite integral is equal to the expression under the integral sign:
		
		\item[P3.] The indefinite integral of the differential of a given function is equal to the sum of this function and an arbitrary constant:
		
		\item[P4.] The indefinite integral of the sum (or subtraction) of two or more algebraic functions is equal to the algebraic sum of their integral (do not forget that we work with the set of all primitives and not a specific given one primitive!):
		
		\begin{dem}
		To prove this, following the request of a reader, we will prove that the derivative of the left-hand side allows us to find the derivative of the right-hand side and vice versa (reciprocal) with the above properties.
		
		According to P1 we have:
				
		Let us check if it is the same with the right-hand side (we assume known the properties of derivatives that we proved earlier in this section):
		
		\end{dem} 
		\begin{flushright}
			$\square$  Q.E.D.
		\end{flushright}
		\item [P5.] We can get out a constant factor from the integral sign, that is to say:
		
		We justify this equality by deriving the two members (and according to the properties of derivatives):
		
		\item[P6.] We can take out a constant factor in the argument of the integrated function (rather rarely used):
		
		Indeed, by differentiating both members of equality we have following the properties of derivatives:
		
		\item[P7.] The integration of a function whose argument is summed (or subtracted) is algebraically the primitive of the argument summed (respectively subtracted):
		
		This property can be showed identically to the previous one using also the derivatives properties.
		\item[P8.] The combination of properties P6 and P7 allows us to write:
		
		\item[P9.] Let $f$ be a continuous function on $[a, b]$, we have for all $c$ belonging to this interval:
		
		This theorem, sometimes named "\NewTerm{Chasles relation}\index{Chasles relation}" (by its vector equivalent) follows directly from the definition of the indefinite integral. $F$ being a primitive of $f$ on $[a, b]$ we have:
		
		\item[P10.] This is a property often used in the section of Statistics (we do not find an easy way to express this property by everyday language so...):
		
		 Let us see now two properties that will be helpful we sometimes to calculate difficult integrals:
		\item[P11.] If a function is even (\SeeChapter{see section Functional Analysis}), the integral on symmetrical bounds is equivalent to:
		
		\item[P12.] If a function is odd (\SeeChapter{see section Functional Analysis}), the integral on symmetrical bounds is equivalent to:
		
		
		\item[P13.] The integral of a periodic function is invariant under a shift of its integration. This is a property that we will use further below to finalize the proof of the integral representation of the zero order Bessel function of the first type:
	
		If $f$ is a periodic function of period $T$ we know that any value $a$:
		
		So now consider:
		
		We do the change of variable $y=t-T$, the we have for the last integral:
		
		Therefore:
		
	\end{enumerate}
	
	\pagebreak
	\subsubsection{Double Integral}
	The idea of double integrals is to measure the volume of the area bounded by the graph of a function of two variables over a domain $D$ of the plane (below $D$ is rectangular):
	\begin{figure}[H]
		\centering
		\includegraphics{img/algebra/double_integral_square_domain.jpg}
		\caption{Example of a continuous function of two variables over a squared domain}
	\end{figure}
	It could be obvious to the reader that double integrals are extremely important in the field of Applied Mathematics!
	
	Again, the idea is the same as the definite integral. If we adapt a simplistic approach, we decompose the continuous function like a a staircase and the volume to calculate is then reduced to the sum of the volumes of parallelepipeds:
		
	\begin{figure}[H]
		\centering
		\includegraphics{img/algebra/double_integral_square_domain_decomposition_into_big_parallelepipeds.jpg}
		\caption{Volume decomposition into big parallelepipeds}
	\end{figure}
	Therefore we have the double sum:
	
	For a continuous function, we proceed by successive approximations: we calculate the Riemann sums for subdivisions always thinner and thinner of the domain $D$:
	\begin{figure}[H]
		\centering
		\includegraphics{img/algebra/double_integral_square_domain_decomposition_into_thin_parallelepipeds.jpg}
		\caption{Decomposition of the volume in always thinner parallelepipeds}
	\end{figure}
	and therefore at the limit:
	
	But..., when want want to integrate an area that is not rectangular, things get a priori more complicated ... Let's see a workaround.
	
	For this purpose, we will build closed bounded domain $D$ as follows:
	
	where the reader will have noticed that the support is the variable $x$ by trough the two functions $u$ and $v$. This then is what we name a "\NewTerm{type I domain}\index{domain of definition!type I domain}" (and therefore if it is $y$ that parameterizes $x$ then it is "\NewTerm{type II domain}\index{domain of definition!type II domain}").

	This can be illustrated by the figure below:
	\begin{figure}[H]
		\centering
		\includegraphics{img/algebra/double_integral_type_I_domain_example.jpg}
		\caption{Example of a type I domain}
	\end{figure}
	where we notice that this simplistic approach (there are other possible approaches but that need the use of Measurement Theory) requires that the domain is simply convex\footnote{As we have seen in the section of Geometric Shapes an area is convex if any couple of points on the bounds of the area with a crossing line has this latter no outside of the area} (that is to say they are no holes outside the domain $D$ between $u (x)$ and $v (x)$) or decomposed into simply convex disjoint subdomains.

	
	To resume, we can integrate as follows:
	
	So we transform the double integral in two nested simple integrals.
	
	\paragraph{Fubini's theorem}\mbox{}\\\\
	We will see now an important theorem used repeatedly in different section of this book and which permits to reverse the order of integration.
	
	Remembering that:
	
	we can also use:
	
	So with this parametrization we can write:
	
	We can change the order of integration, the calculation is different, but the result is the same. But that's not what really interests us here!
	
	Consider a function such as (we say that the function can be "variable separable"):
	
	Therefore:
	
	Suppose that the domain is a rectangle (we do this simplification otherwise the proof complicates considerably). Meaning:
	
	Therefore the integral linearity property:
	
	
	\subsubsection{Integration by Substitution}
	When we can not easily determine the primitive of a given function, we can find sometimes, with a smart change of variables (sometimes very subtle...), bypass the difficulty. It does not work every time (because some functions are not formally integrable) but it is worth a try before taking out your computer.
	
	Again, we give only the general form of the method. It is the role of teachers in schools to train, and train, and train students to understand and master these techniques. In addition, the sections in this book that treats of exact sciences (physics, computer science, astrophysics, chemistry, ...) replete with examples using this technique and thus serve implicitly as exercises.
	
	Consider we want to calculate the integral (not bounded for the moment):
	
	although we do not know directly how to calculate the primitive of this function $f (x)$ (at least we imagine being in that situation) we know (in one way or another) that exists (we do treat of improper integrals at this level).

	The technique then consists in this integral to perform the following change of variable:
	
	where $\varphi (t)$ is a continuous function and also its derivative, and having an inverse function. Then $\mathrm{d}x=\varphi' (t)\mathrm{d}t$, and let us prove now that in this case the equality:	
	
	is satisfied.
	\begin{dem}
	We imply here that the variable $t$ will be replaced after integration of the right member by its expression in function of $x$. To justify this equality in this sense, it suffices to show that the considered amounts where each is a defined only with a difference of a given arbitrary constant have the same derivative with respect to $x$. The derivative of the left member is:
	
	We derive only the right member with respect to $x$ considering that $t$ is a function of $x$. We know that:
	
	We get therefore:
	
	\begin{flushright}
		$\square$  Q.E.D.
	\end{flushright}
	\end{dem}
	Obviously, the function $x=\varphi (t)$ must be chosen so that we know how to calculate the indefinite integral on the right side of equality.
	\begin{tcolorbox}[title=Remark,colframe=black,arc=10pt]
	Sometimes it is preferable to choose the change of variable in the form $t=\varphi (x)$ instead of $x= \psi(t)$ because as this to a large tendency to simplify the length of the equation instead of making it longer.
	\end{tcolorbox}
	It is obvious that this theorem will be more explicitly written:
	
	
	\pagebreak
	\paragraph{Jacobian}\mbox{}\\\\
	Consider a domain $D$ of the plane  $u,v$ delimited by a curve $L$. Suppose that the $x, y$ coordinates are functions of the new variables $u, v$ (always in the context of a change of variables!) by the following relations:
	
	where the functions $\varphi (x,y)$ and $\phi(u,v)$ are unique, continuous and have continuous derivatives in a given domain $D'$ which we will define later. It corresponds therefore following previous relations to any pair of values $u, v$ only one couple of values $x, y$ and vice versa.
	
	It follows from what was said just before that at any point $P(x,y)$ of the plane $\text{O}xy$  corresponds univocally a point $P '(u, v)$ of the plane $\text{O}uv$ of coordinates $u, v$ defined by the above relations. The numbers $v$ and $u$ will be named "\NewTerm{curvilinear coordinates}\index{curvilinear coordinates}" of $P$ and we will see concrete and schematics examples of these in the section of Vector Calculus.
	
	If in the $X\text{O}Y$ plane the set of points $P$ describes a closed curve $L$ defining a domain $D$, the corresponding set of points describes in $u\text{O}v$ a given domain $D'$. It corresponds to any point of $D'$ a point of $D$. Thus, the relations of transformation establish a bi-univocal correspondence between the points of the domains $D$ and $D'$.
	
	Now consider in $D'$ a straight line of equation $u=c^{te}$. In general, the relations of transformation make it corresponds in the plane $\text{O}xy$ a curved line (or vice versa). Therefore, let us cut the domain $D'$ by the multiple straight lines of equations $u=c^{te}$ and $v=c^{te}$ into small rectangular areas (we will not take into account at the limit, the rectangles on the boundary of $D'$). The corresponding curves of the domain $D$ then cut this latter it into quadrilateral (defined by curves therefore). Obviously, the reverse applies!
	
	Consider in the plane $\text{O}uv$ the rectangle $\Delta s'$ limited by the straight lines:
	
	and the curvilinear quadrilateral corresponding $\Delta s$ in the plane $\text{O}xy$. We will designate areas of these partial domains also  by $\Delta s'$ by $\Delta s$. We have obviously:
	
	The areas $\Delta s$ and $\Delta s'$ are different in general.
	 \begin{figure}[H]
		\centering
		\includegraphics[scale=0.8]{img/algebra/non_linear_map.jpg}
		\caption{A nonlinear map $f : \mathbb{R}^2\mapsto \mathbb{R}^2$ sends a small square to a distorted parallelogram}
	\end{figure}
	Suppose in $D$ a continuous function $z=f(x,y)$. It corresponds to any value of this function of the domain $D$ the same value $z=F(u,v)$ (what we want to check) in $D'$, where:
	
	Consider the integrals sum of the function $z$ in the domain $D$. We obviously have the following equality:
	
	Let us calculate $\Delta s$, that is to say the area of the curvilinear quadrilateral $P_1,P_2,P_3,P_4$ in the plane $\text{O}xy$:
	
	We determine the coordinates of its vertices:
	
	
	We will assimilate in the calculation of the are of the quadrilateral  $P_1,P_2,P_3,P_4$ the arcs $P_1P_2,P_2P_3,P_3P_4,P_4P_1$ to parallel line segments. We will also replace the increasing values of the functions by their differentials. This means that we ignore the infinitely small differentials of higher order than $\Delta u$ and $\Delta v$. The previous relations then become:
	

	Under these assumptions, the curvilinear quadrilateral $P_1P_2P_3P_4$ can be likened to a parallelogram. His area is therefore approximately equal to twice the area of the triangle $P_1P_2P_3$, area that we can calculate by using the properties of the determinant (as we will prove it in the section on Linear Algebra, in $\mathbb{R}$ the determinant  represents a parallelogram while in $\mathbb{R}^2$ it represents the volume of a parallelepiped):
	
	Such as (this is here that the best choice has to be done so that the final expression is the simplest and most aesthetic, for this purpose we proceed by trials and finally we make the choice below):
	\begin{figure}[H]
		\centering
		\includegraphics{img/algebra/graphical_representaton_of_determinant.jpg}
		\caption{Graphical representation of the determinant}
	\end{figure}

	Thus we have:
	
	Therefore the following relation (containing what is usually named the "\NewTerm{functional determinant}\index{functional determinant}"):
	
	with:
	
	that is the "\NewTerm{Jacobian matrix}\index{Jacobian matrix}" (its determinant is simply named the "\NewTerm{Jacobian}\index{Jacobian}" (for short)) of the coordinate transformation $\mathbb{R}^2 \rightarrow \mathbb{R}^3$. By applying exactly the same reasoning to $\mathbb{R}^3$, then the Jacobian is written (by changing some notations because otherwise it becomes unreadable):
	
	In short, what it is useful exactly? Well let us come back to our relation:
	
	which is finally only an approximation because in the calculations of the area $\Delta s$ we neglected the infinitely small differential of higher order. However, more the dimensions of the elementary domains $\Delta s$ and $\Delta s'$ are small, and more we are approaching the true equality. The equality finally taking place when we go to the limit (finally also in math we make approximations... eh!), the surfaces of the elementary domains tending to zero:
	
	We apply now the equality obtained to calculate the double integral (we can do the same with the triple of course). So we can finally write (this is the only way of putting the thing that makes sense):
	
	Passing to the limit, we obtain the strict equality:
	
	This is the coordinate transformation relation in a double integral! It allows us to reduce the calculation of a double integral in the domain $D$ to domain $D'$, which can simplify the problem.
	Similarly, for a triple integral, we write:
	
	
	\begin{tcolorbox}[colframe=black,colback=white,sharp corners]
	\textbf{{\Large \ding{45}}Examples:}\\\\
	Let us now determine the Jacobian for the most common coordinate systems (we send the reader back again to Vector Calculus section for more information about these systems):\\
	
	E1. Polar coordinates $x=r\cos(\phi),y=r\sin(\phi)$:
	
	Since $r$ is always positive, we simply write:
	
	\end{tcolorbox}
	\pagebreak
	
	\begin{tcolorbox}[colframe=black,colback=white,sharp corners]
	E2. In cylindrical coordinates $x=r\cos(\phi),y=r\sin(\phi),z=z$ (see section Linear Algebra for calculating the determinant):
	
	Since $r$ is always positive, we simply write:
	\\
	
	E3. In spherical coordinates $x=r\sin(\theta)\cos(\phi),y=r\sin(\theta)\sin(\phi),z=r\cos(\theta)$ (see section Linear Algebra for calculating the determinant):
	
	Since $r$ is always positive, we simply write:
	
	\end{tcolorbox}

	\subsubsection{Integration by Parts}
	When we seek to make integrations, it is very common that we have to use a tool (or calculation method) named "\NewTerm{integration by parts}\index{integration by parts}". There are different degrees of use of this tool and we'll start with the most simple and that is the most used in all chapters and sections in this book.
	
	First we start from the derivative of the product of two functions proved above:
	
	
	so we have:
	
	and we get:
	
	after a final simplification we finally get the famous very important equality:
	
	But sometimes we will need the generalization of that relation. We can show that if $f$ and $g$ are two applications (functions) of class $\mathcal{C}^n$ ($n$ times differentiable) on $[a, b]$ in $\mathbb{C}$, then:
	
	\begin{dem}
	Let us proceed by induction on n$ $(beware it is not necessarily easy to understand as often with demonstrations by induction!).
	
	Knowing the relation is true for $n = 1$, we assume it true for $n$ (as given in the above relation!) and we prove it for $n + 1$ (so we must laid on the previous relation but with $n + 1$ instead of $n$):
	
	\begin{tcolorbox}[title=Remark,colframe=black,arc=10pt]
	The trick (proposed by a reader) in this proof is to see that $-(-1)^n$ gives a minus sign when $n$ is even and plus sign when $n$ is odd and also that $+(-1)^{n+1}$ gives a minus sing when $n$ is even and plus sign when $n$ is odd.
	\end{tcolorbox}
	For $n = 1$ we fall back on the well known and very often used equality in this book:
	
	\begin{flushright}
		$\square$  Q.E.D.
	\end{flushright}
	\end{dem}

	\pagebreak
	\subsubsection{Usual Primitives}
	There are in math and physics many primitives or functions defined on integrals we see  quite frequently (but not exclusively). Furthermore, all primitive proved below will be used in the various chapters on Mechanics, Engineering, Atomistic, Social Mathematics, etc of this book. So, as in any formula booklet, we propose you the most sixty knows primitives but with... the proofs!
	
	However, we will omit the primitives that can be immediately deduced from the usual derivatives we have proved above. This means for example that we assume  known two very important primitives (certainly the most used in all pages of this book):
		
	Otherwise here is the list of the most common primitives (the reader will meet anyway many others - developed in the details - as it reads this book):
	\begin{enumerate}
		\item Primitive of $f(x)=\tan (x)$:
		
		By definition we have:
		
		We use the change of variable $u=\cos(x),\mathrm{d}u=-\sin(x)\mathrm{d}x$ and therefore:
		
		Therefore:
		
		\item Primitive of $f(x)=\cot(x)$:
		
		By definition we have:
		
		We use the change of variable $u=\sin(x),\mathrm{d}u=\cos(x)\mathrm{d}x$ and therefore:
		 
		Therefore:
		
		\item Primitive of $f(x)=\arcsin(x)$:
		
		We integrate by parts:
		
		If we put $u=1-x^2$, giving us $\mathrm{d}u=-2x\mathrm{d}x$, we get:
		
		Therefore:
		
		\item Primitive of $f(x)=\arccos(x)$:
		
		We integrate by parts again:
		
		If we put $u=1-x^2$, giving us $\mathrm{d}u=-2x\mathrm{d}x$, we get:
		
		Therefore:
		
		\item Primitive of $f(x)=\arctan(x)$:
		
		We integrate by parts again:
		
		If we put $u=1+x^2$, giving us $\mathrm{d}u=2x\mathrm{d}x$, we get:
		
		Therefore:
		
		\item Primitive of $f(x)=\text{arccot}(x)$:
		
		Once again... we integrate by part:
		
		If we put $u=1+x^2$, giving us $\mathrm{d}u=2x\mathrm{d}x$, we get:
		
		Therefore:
		
		\item Primitive of $f(x)=xe^{ax}$ with $a\in \mathbb{R}\left\lbrace 0 \right\rbrace$:
		
		An integration by part gives:
		
		Therefore:
		
		\begin{tcolorbox}[title=Remark,colframe=black,arc=10pt]
		Another very important primitive with the exponential in physics is that we had proved in our study of the law of Gauss-Laplace law (Normal Law) in the section of Statistics (determination of the expected mean).
		\end{tcolorbox}
		\item Primitive of $f(x)=\ln(x)$:
		
		We write:
		
		Integrating by parts we found:
		
		Finally:
		
		\item Primitive of $f(x)=x\ln(ax)$ with $a\in \mathbb{R}\left\lbrace 0 \right\rbrace$:
		
		An integration by part give us:
		
		Therefore:
		
		\item Primitive of $f(x)=a^x$ for $a>0,a\neq 1$:
		
		To begin we write:
		
		Therefore we get:
		
		and:
		
		Finally:
		
		\item Primitive of $f(x)=\log_a(x)$:
		
		For $a>0,a\neq 1$ knowing that (see the properties of logarithms in the section of Functional Analysis):
		
		we get using the primitive of $\ln(x)$:
		
		\item Primitive of $f(x)=\tanh(x)$:
		
		We have:
		
		we use the change of variables $u=\cosh(x),\mathrm{d}u=\sinh(x)\mathrm{d}x$ and we get:
		
		Finally:
		
		\item Primitive of $f(x)=\coth(x)$:
		
		We know we have:
		
		we use the change of variables $u=\sinh(x),\mathrm{d}u=\cosh(x)\mathrm{d}x$ and we get:
		
		Finally:
		
		\item Primitive of $f(x)=\text{arcsinh}(x)$:
		
		We integrate by parts:
		
		If we put $u=1+x^2,\mathrm{d}u=2x\mathrm{d}x$ we get:
		
		Finally:
		
		\item Primitive of $f(x)=\text{arccosh}(x)$:
		
		If we integrate by parts as before:
		
		If we put $u=1-x^2,\mathrm{d}u=-2x\mathrm{d}x$ we get:
		
		Finally:
		
		\item Primitive of $f(x)=\text{arctanh}(x)$:
		
		We integrate by parts:
		
		If we put $u=1-x^2,\mathrm{d}u=-2x\mathrm{d}x$ we get:
		
		Finally:
		
		\item Primitive of $f(x)=\text{arccoth}(x)$:
		
		We integrate by parts:
		
		If we put $u=1-x^2,\mathrm{d}u=-2x\mathrm{d}x$ we get:
		
		Finally:
		
		\item Primitive of $f(x)=\sin ^{n}(x)$ with $n \geq 2$:
		
		Let us put $I_n=\int \sin ^n(x)\mathrm{d}x$. An integration by parts give:
		
		substituting $\cos ^2(x)$ by $1-\sin ^2{x}$ in the last primitive, we obtain:
		
		and therefore:
		
		All primitives of the same form (recurrence relation) are named "\NewTerm{reduction formulas}\index{reduction formulas}".
		
		\item Primitive of $f(x)=\cos ^{n}(x)$ with $n \geq 2$:
		
		In this case we have the recursive formula:
		
		that is proved exactly in the same way as the previous recursive relation (the reader can request the details if required).
		\item Primitive of $f(x)=\tan ^{2}(x)$:
		
		Knowing that $\tan'(x)=1+\tan^2(x)$ we have:
		
		Therefore:
		
		\item Primitive of $f(x)=\cot ^{2}(x)$:
		
		Knowing that $\cot'(x)=-1-\cot^2(x)$ we have:
		
		Therefore:
		
		\item Primitive of $f(x)=\sin ^{-2}(x)$:
		
		Using remarkable trigonometric identities (\SeeChapter{see section Trigonometry}), we have:
		
		Thanks to the primitive of $\cot^2(x)$. Therefore:
		
		\item Primitive of $f(x)=\cos ^{-2}(x)$:
		Using once again remarkable trigonometric identities (\SeeChapter{see section Trigonometry}), we have:
		
		Thanks to the primitive of $\tan^2(x)$. Therefore:
		
		\item Primitive of $f(x)=\sin ^{-1}(x)$:
		
		We use the substitution $x=2\arctan(t),t=\tan(x/2)$. Knowing that (\SeeChapter{see section Trigonometry}):
		
		we then get:
		
		Therefore:
		
		Finally:
		
		\item Primitive of $f(x)=\cos ^{-1}(x)$:
		
		Knowing that $\cos(x)=\sin(x+\pi/2)$ (\SeeChapter{see section Trigonometry}) we have:
		
		We do the change of variable $x+\pi/2=u, \mathrm{d}u=\mathrm{d}x$:
		
		thanks to the knowledge of the primitive of $\sin ^{-1}(x)$. Finally:
		
		\item Primitive of $f(x)=\dfrac{1}{1+\cos(x)}$:
		
		We do the substitution $x=2\arctan(t),t=\tan(x/2)$, knowing that (\SeeChapter{see section Trigonometry}):
		
		we get:
		
		Therefore:
		
		Finally:
		
		\item  Primitive of $f(x)=\dfrac{1}{1-\cos(x)}$:
		
		We do again the substitution $x=2\arctan(t),t=\tan(x/2)$. Then we find:
		
		So that finally:
		
		\item  Primitive of $f(x)=\dfrac{1}{1+\sin(x)}$:
		
		Knowing that:
		
		we can write:
		
		By making the change of variables:
		
		we get:
		
		Finally:
		
		\item  Primitive of $f(x)=\dfrac{1}{1-\sin(x)}$:
		
		By the same reasoning as above using the cosine we get:
		
		
		\item Primitive of $f(x)=\sinh^n(x)$ with $n \geq 2$:
		
		Let us put:
		
		An integration by parts gives (we proved during our study of usual derivatives that the primitive of the hyperbolic sine was the hyperbolic cosine):
		
		by substituting $\cosh^2(x)$  and $1+\sinh^2(X)$ in the last integral, we obtain:
		
		and therefore:
		
		Therefore we obtain easily the special case:
		 
		\item Primitive of $f(x)=\cosh^n(x)$ with $n \geq 2$:
		
		In this case we also have the recurrence relation:
		
		that is proved in the same way as above. So we also get easily the special case:
		
		
		\item Primitive of $f(x)=\tanh^2(x)$:
		
		Knowing that (proved during our study of usual derivatives):
		
		we have:
		
		Therefore:
		
		
		\item Primitive of $f(x)=\coth^2(x)$:
		
		Knowing that (proved during our study of usual derivatives):
		
		we have:
		
		Therefore:
		
		
		\item Primitive of $f(x)=\dfrac{1}{\sinh^2(x)}$:
		
		We have using the primitive of $f(x)=\coth^2(x)$:
		
		Therefore:
		
		
		\item Primitive of $f(x)=\dfrac{1}{\cosh^2(x)}$:
		
		We have using the primitive of $f(x)=\tanh^2(x)$:
		
		Therefore:
		
		
		\item Primitive of $f(x)=\dfrac{1}{\sinh(x)}$:
		
		We do the substitution:
		
		We get using the derivative of $\text{arctanh}(x)$:
		
		
		and finally:
		
		
		\item Primitive of $f(x)=\dfrac{1}{\cosh^2(x)}$:
		
		We do the substitution:
		
		We get using the derivative of $\arctan(x)$:
		
		and finally:
		
		\item Primitive of $f(x)=\dfrac{1}{1+\cosh(x)}$:
		
		We do the substitution:
		
		We get:
		
		Finally we get:
		
		
		\item Primitive of $f(x)=\dfrac{1}{1-\cosh(x)}$:
		
		We do the substitution:
		
		We get:
		
		Finally:
		
		
		\item Primitive of $f(x)=\dfrac{1}{1+\sinh(x)}$:
		
		We do the substitution:
		
		We get:
		
		But:
		
		Therefore:
		
		Therefore:
		
		
		\item Primitive of $f(x)=\dfrac{1}{1-\sinh(x)}$:
		
		We still do the same substitution:
		
		We get:
		
		But:
		
		Therefore:
		
		Finally:
		
		
		\item Primitive of $f(x)=e^{ax}\sin(bx)$ with $a,b\in \mathbb{R},a^2+b^2\neq 0$:
		
		A first integration by parts gives:
		
		A second integration by parts gives
		
		So we have the equality:
		
		Therefore redistributing the previous relation:
		
		
		\item Primitive of $f(x)=e^{ax}\cos(bx)$ with $a,b\in \mathbb{R},a^2+b^2\neq 0$:
		
		A similar reasoning than before shows that (we can detail on demand as always!):
		
		
		\item Primitive of $f(x)=x\sin(ax)$ with $a \in \mathbb{R}^*$:
		
		And integration by parts gives us:
		
		
		\item Primitive of $f(x)=x\cos(ax)$ with $a \in \mathbb{R}^*$:
		
		And integration by parts gives us:
		
		
		\item Primitive of $f(x)=\dfrac{1}{(x-a)(x-b)}$ with $a \neq b$:
		
		We have the following relation (in integral calculus we name such decomposition a "\NewTerm{partial fraction decomposition}\index{partial fraction decomposition}"):
		
		Therefore:
		
		
		Finally:
		
		
		\item Primitive of $f(x)=\dfrac{1}{a^2-x^2}$ with $a \neq 0$:
		
		We have using the previous result:	
		
		Therefore:
		

		\item Primitive of $f(x)=\dfrac{1}{a^2+x^2}$ with $a \neq 0$:
		
		Doing a change of variable:
		
		We get using the derivative of $\arctan (x)$:
		
		
		\item Given:
		
		with $n \in \mathbb{N}$. We get:
		
		But this last primitive can be solved by parts:
		
		Therefore:
		
		What we find most frequently in the literature under the form:
		
		Identically to the next development, we have for (the sign change):
		
		the following relation:
		
		You can find an application of these two primitives in the Newtonian cosmological model of the Universe in the section of Astrophysics and also in the section of General Relativity in the study of the Shapiro effect!
		
		\item Primitive of $f(x)=\dfrac{1}{(1-x^2)^2}$:
		
		We have using the primitives of $\dfrac{1}{(1-x^2)^n}$ (proved before) and of $\dfrac{1}{1-x^2}$ (also proved above):
		
		
		\item Primitive of $f(x)=\dfrac{1}{(1+x^2)^2}$:
		
		We have using the primitives of $\dfrac{1}{(1+x^2)^n}$ (proved before) and of $\dfrac{1}{1+x^2}$ (also proved above):
		
		
		\item Primitive of $f(x)=\sqrt{x^2-a^2}$ with $a\in \mathbb{R}^*$ (case relative to the area under a hyperbola):
		
		We can assume without loss of generality $a>0$. Note that the domain of definition of $f$ is $]\infty,-a] \cup [a,+\infty[$.
		
		We will determine now a primitive of $f$ only on the interval $[a,+\infty[$ (because that is it we will need in some sections of this book).
		
		Let us make the change of variable:
		
		So with:
		
		where we consider the function $\cosh: \mathbb{R}^+ \rightarrow [1,+\infty[$ with for reciprocal the function $\text{arccosh}:[1,+\infty[ \rightarrow \mathbb{R}^+$ given by (\SeeChapter{see section Trigonometry}):
		
		We obtain then using then primitive of $\sinh^2(x)$:
		
		but \SeeChapter{see section Trigonometry}) as:
		
		Therefore:
		
		and using another identity proved in the section Trigonometry:
		
		we have therefore:
		
		as the primitive are given to a given constant, we can write:
		
		for $x\geq a$. $F$ is then a primitive of $\sqrt{x^2-a^2}$ on the interval $[a,+\infty[$.
		
		\item Primitive of $f(x)=\sqrt{a^2-x^2}$ with $a\in \mathbb{R}^*$:
		
		We can assume without loss of generality $a>0$. Note that the domain of definition of $f$ is $[-a, a]$.
		
		We make the substitution:
		
		we get:
		
		where we used the primitive $\cos^n (x)$ with $n=2$ proved above. Now we have:
		
		Then:
		
		and:
		

		\item Primitive of $f(x)=\sqrt{x^2+a^2}$ with $a\in \mathbb{R}^*$:
		
		We can assume without loss of generality $a>0$.
		
		Let us make the change of variable:
		
		with:
		
		We get:
		
		Therefore:
		
		But as we saw in the section Trigonometry:
		
		and:
		
		Finally we have:
		
		where $ln (a)$ has been omitted because the primitives are given to a given constant.
		
		\item Primitive of $f(x)=\left(\sqrt{a^2-x^2}\right)^{-1}$ with $a\in \mathbb{R}^*$:
		
		We can assume without loss of generality $a>0$.
		
		We do the substitution:
		
		We get:
		
		
		\item Primitive of $f(x)=\dfrac{1}{\sqrt{a^2+x^2}}$ with $a\in \mathbb{R}^*$:
		
		We can assume without loss of generality $a>0$.
		
		We do the substitution:
		
		We get in the same manner as the previous usual integrals:
		
		and knowing that (\SeeChapter{see section Trigonometry}):
		
		We then get finally the following important primitive:
		
		Proceeding in the same way, but using the hyperbolic cosine instead of hyperbolic sine, we get obviously (we can detail on demand as always):
		
		We will reuse these last two relation in important practical cases of the section of Analytical Mechanics, Civil Engineering (where the constant $a$ is equal $1$, $ln (a)$ will be equal to $0$) and General Relativity (where $a$ will be nonzero and therefore it will not be possible to omit the constant $ln (a)$).
		
		\item Let us consider an integral of the following form that we can use to improve the Stirling formula (\SeeChapter{see section Theoretical Computing}) and also that we need absolutely for the study of the circular Fresnel aperture diffraction (\SeeChapter{see section Wave Optics}):
		
		where
		\begin{itemize}
			\item $\lambda$ is large;
			\item $g(y)$ is a smooth function which has a local minimum at $y^*$ in the interior of the interval $[a, b]$;
			\item $h(y)$ is smooth.
		\end{itemize}
		The integral can be the moment generating function of the
		distribution of $g(Y)$ when $Y$ has density $h$ (\SeeChapter{see section Statistics}), it could be a posterior expectation of $h(Y)$, or just a "simple" integral.
		
		When $\lambda$ is large, the contribution to this integral is essentially entirely by construction originating from a neighborhood around $y^*$.
		
		We formalize this by Taylor expansion of the function $g$ around $y^*$ :
		
		Since $y^*$ is a local minimum, we have:
		
		 and therefore:
		
		Therefore:
		
		The fact that above the bounds have change from $[a,b]$ to $]-\infty,+\infty[$ is that we assume that the area of interest is around $y^*$ and that because $\lambda$ is soooo large that veeeery quickly a bit away from $y^*$ we can consider the curve as negligible!
		
		If we approximate $h(y)$ linearly around $y^*$, that is to say:
		
		such that:
		
		Therefore:
		
		The reader must not forget that here $\lambda g''(y^*)$ is a constant and that if we put:
		
		Then for the first integral above we see that we fall back on the integral of something very similar to the Gauss distribution (with mean $y^*$ and standard deviation $\lambda g''(y^*)$) and then it comes immediately (\SeeChapter{see section Statistics}):
		
		For the second integral:
		
		the change of variable $y-y^*=x$ give us:
		
		The primitive is of the form (\SeeChapter{see section Integral and Differential Calculus}):
		
		Therefore by symmetry the second integral is zero! We have finally:
		
		This calculation is named "\NewTerm{Laplace's Method of Integration}\index{Laplace's Method of Integration}" or simply "\NewTerm{Laplace Integration}".
	\end{enumerate}
	
	\subsubsection{Integral representation of first kind Bessel's function}
	A particularly useful and powerful way of treating Bessel functions employs their integral representation as we will see in the section of Wave Optics.
	
	Remember that in the section of Sequences and Series we have proved that the generating function of Bessel's function was:
	
	That is:
	
	Now remember that (\SeeChapter{see section Trigonometry}):
	
	So if we return to the generating function, and substitute $t=e^{\mathrm{i}\theta}$, we get:
	
	In which to condensate the result we have used first the property proved during our study of the generating function of Bessel's functions:
	
	and also:
	
	and so on...
	Now remember that:
	
	Therefore identifying real and imaginary part, we get:
	
	Remember also that we have proved during our study of Fourier series in the section of Sequences and Series that:
	
	\begin{center}
	\begin{tabular}{ccc}
	$\text{with }n,k\in \mathbb{N}\text{ and }n\ne k$
	&$\qquad$&
	$\text{with }n,k\in \mathbb{N}\text{ and }n = k$
	\end{tabular}
	\end{center}
	That is:
	
	where $\delta_{nm}$ is the Kronecker symbol (\SeeChapter{see section Tensor Calculus}).
	
	Now let us write:
	
	Let us focus on the first integral:
	
	So we see above that whatever the value for any $n$, excepte of $n=0$:
	
	Therefore in only remains for $n>0$:
	
	and we see above that if $n>0$ is odd all the integrals vanish but if $n>0$ is even, only the corresponding $J_n(x)$ remains!
	
	Exactly the same analysis can be done for can be done:
	
	We have therefore for each of the integrals above, especially for each the left term that (recall that $n=0,2,4,\ldots$ is even and $n=1,3,5,\ldots$ is odd):
	
	If these tow equations are added together we have using trigonometric identities (\SeeChapter{see section Trigonometry}):
	
	for $n=0,1,2,3,\ldots$.
	
	If we put $n=0$ in the above relation, we get:
	
	If we plot $\cos(x\sin(\theta))$ we see that it repeats itself in all four quadrants (it's an even function):\\\\
	\texttt{>plot([cos(sin(theta)),cos(2*sin(theta)),cos(3*sin(theta)),cos(5*sin(theta))]\\
	,theta=-2*Pi..2*Pi);}
	\begin{figure}[H]
		\centering
		\includegraphics[scale=0.55]{img/algebra/cos_sin_maple.jpg}
	\end{figure}
	So we can write:
	
	This is the real integral representation of the zero order Bessel function of the first type.
	
	But in many developments we don't use the above expression as there is not phasor that is visible. So the trick is to notice that $\sin(x\sin(\theta)$ reverses its sign in the third an fourth quadrant (it's and odd function):\\\\
		\texttt{>plot([sin(sin(theta)),sin(2*sin(theta)),sin(3*sin(theta)),sin(5*sin(theta))],\\
	theta=-2*Pi..2*Pi);}
	\begin{figure}[H]
		\centering
		\includegraphics[scale=0.6]{img/algebra/sin_sin_maple.jpg}
	\end{figure}
	So we have:
	
	Adding the both relations by multiplying the second by $\mathrm{i}$,  we have:
	
	Finally we get the complex representation of the zero order Bessel function of the first type:
	
	
	Let us do a change of variable $\theta=\varphi+\pi/2$, then:
	
	But we have proved earlier that for any periodic function:
	
	Therefore:
	
	This integral representation my be obtained in various ways but this one seems the most easy one to us. Many other integral representation exists.

	\subsubsection{Dirac Function}
	The Dirac function, also named "\NewTerm{Dirac peak}\index{Dirac peak}" or "\NewTerm{delta function}\index{delta function}", plays a very important practical role both in electronic and computer also  wave mechanics and quantum field theory (this allows to discretize a continuum!) and in the field of civil engineering (see section  of the same name for some examples).
	
	Before going further we could notice that it is wrong to speak about a "function" because a function is an application of a start set (usually the set of real or complex number with one or more dimensions) in an arrival set (usually the set of real or complex numbers in one or more dimensions). While the domain of definition of the Dirac function is not a set of numbers but strictly speaking a set of functions!
	
	More technically the Dirac delta function, or $\delta$ function, is a generalized function, or distribution, on the real number line that is zero everywhere except at zero, with an integral of one over the entire real line. The delta function is sometimes thought of as an infinitely high, infinitely thin spike at the origin, with total area one under the spike, and physically represents the density of an idealized point mass or point charge. It was introduced by theoretical physicist Paul Dirac. In the context of signal processing it is often referred to as the unit impulse symbol (or function). Its discrete analog is the Kronecker delta function, which is usually defined on a discrete domain and takes values 0 and 1.
	
	As always in this book we will focus here only on the properties we will need to study Applied Mathematics stuffs of other sections of the book.
	
	To represent mentally in an easy way this function, first consider the function defined by:
	
	The representation of $y=f(x)$ above is a rectangle of width $a$, and of height $1/ a$ and unit surface. The Dirac function can be considered as the boundary when $a\leftarrow 0$ of the  $f (x)$. So we have:		
	
	That is to say:
	\begin{figure}[H]
		\centering
		\includegraphics[scale=0.5]{img/algebra/common_dirac_peak_representation.jpg}
		\caption{Schematic representation of the Dirac delta function by a line surmounted by an arrow (source: Wikipedia)}
	\end{figure}
	with:
	
	where $\varepsilon$ is a number greater than $0$ and as small as we want.
	\begin{tcolorbox}[title=Remark,colframe=black,arc=10pt]
		As the reader will have probably noticed it when we introduced the initial function $f (x)$, the resulting delta Dirac function has therefore the dimension of the inverse of a length!
	\end{tcolorbox}
	For a function $g (x)$ continues in $x = 0$ we have:
	
	By extension we have:
	
	and for a function $g (x)$ coninue on $x_0$:
	
	It is then relatively easy to define the Dirac function in 3-dimensional space by:
	
	As as already mentioned we will prove properties of the Dirac function only if we will need them in other sections of this book.
	
	\subsubsection{Gamma Euler Function}
	We define the Euler Gamma function (Eulerian integral of the second kind) by the following integral:
	
	with $x$ belonging to the set of complex numbers whose real part is positive and non-zero (thus the positive real number are also included in the domain of definition)! Indeed, if we take complex numbers with a zero or negative real part, the integral diverges and is then undefined!
	
	\begin{tcolorbox}[title=Remark,colframe=black,arc=10pt]
		We have already met this integral and some of its properties (which will be proved here) in our study of the Beta, Gamma, Chi-square, Fisher and Student statistical distribution functions (\SeeChapter{see section Statistics}). We will also use this integral in maintenance (\SeeChapter{see section Management Techniques}), in string theory (\SeeChapter{see section String Theory}) and other engineering fields (see the corresponding chapter) as in the section of Theoretical Computing for the canonical negative binomial generalized linear regression.
	\end{tcolorbox}
	
	Here is a graphical plot of the module of the Euler Gamma function for $x$ browsing an interval of real numbers (take care in Maple 4.00b to write GAMMA capitalized!!!):
	
	\texttt{>with(plots):\\}
	\texttt{>plot(GAMMA(x),x=-Pi..Pi,y=-5..5);}
	\begin{figure}[H]
		\centering
		\includegraphics{img/algebra/maple_gamma_euler_2d_plot.jpg}
		\caption{Plot of the Euler Gamma function in Maple 4.00b}
	\end{figure}
	and always the same function with Maple 4.00b but now in the complex plane and always with in ordinate the module of the Gamma Euler function:
	
	\texttt{>with(plots):\\}
	\texttt{>plot3d(abs(GAMMA(x+y*I)),x=-Pi..Pi,y=-Pi..Pi,view=0..5, grid=[30,30],orientation=[-120,45],axes=frame,style=patchcontour);}
	
	\begin{figure}[H]
		\centering
		\includegraphics{img/algebra/maple_gamma_euler_3d_plot.jpg}
		\caption{Plot of the Euler Gamma function in the complex plane with Maple 4.0}
	\end{figure}
	This function is interesting if we impose the variable $x$ to belong to the set of integer numbers and that we write it as follows:
	
	Let us integrate by part the latter function:
	
	Since the exponential function decreases much faster than $t^x$ then we have:
	
	In literature, we frequently find the following notations (there are confusing):
	
	Which brings us to write the result in a more traditional form:
	
	From the relation $\Gamma_{0}(x)=x\Gamma_{0}(x-1)$, it comes by induction:
	
	But:
	
	That gives:
	
	Therefore:
	
	or written in another way for $x\in \mathbb{N}^*$
	
	Another interesting and useful result of the Euler gamma function is obtained when we replace $t$ by $y^2$ and calculate this latter for $x=0.5$.
	
	First we have:
	
	and after:
	
	But, as we have proved it in the section Statistics during our study of  the distribution, this integral is equal to:
	
	
	\pagebreak
	\paragraph{Euler-Mascheroni Constant}\mbox{}\\\\
	This small text is a just curiosity regarding to Euler's constant $e$ and to almost every Differential and Integral calculus tools that we have seen until now. This is a very nice example (almost artistic) of what we can do with mathematics as soon as we have enough tools at our disposal.
	
	Moreover, this constant is useful in certain differential equations which we see later.
	
	Remember that we saw in the section of Functional Analysis that the Euler constant $e$ is defined by the limit:
	
	
	In a more general case we can easily demonstrate in the same way that (you can ask us the details if needed):	
	
	This obviously suggests:
	
	by a change of variable $t=nu$ we write:
	
	And we use the definition of the Beta function:
	
	Therefore:
	
	To transform this expression we can write:
	
	But the quantity:
	
	tends to the limit $\gamma=0.5772$, named "\NewTerm{Euler-Mascheroni constant}\index{Euler-Mascheroni constant}" or also "\NewTerm{Euler Gamma constant}\index{Euler Gamma constant}" when $n$ tends to infinity.
	
	Therefore:
	
	We divide each term of the product $(x+1)...(x+n)$ by the corresponding integer taken into $n!$, so we get:
	
	
	\pagebreak
	\subsubsection{Curvilinear Integrals}
	The line integrals (curvilinear integrals) are also very important in physics. The reader will thus find see them again in the section of Classical Mechanics, Electrodynamics Magnetostatics and to calculate the work of a force or the "flow field", or in the chapter of Euclidean geometry to calculate the center of gravity of curves (functions), or in the section of Geometric Shapes to calculate the surface of some bodies of revolution but also in Corpuscular Quantum Physics for the famous "path integral" (which is only the term used by physicists to say "line integral") or for the specific calculation of integrals using the residue theorem proved in the section of Complex Analysis or for transformation states in the section of Thermodynamics. This is why there will not be here examples of line integrals because they are already so many applications in the other chapters of this book.
	
	With the definition of these integrals, we can prove two very important results detailed in section of Vector Calculus that are respectively the Green's theorem and Stokes' theorem or even the theorem of residues proved in the section Complex Analysis and already mentioned in the preceding paragraph (this is important enough to mention it twice!).
	
	More technically a line integral is an integral where the function to be integrated is evaluated along a curve. The terms "\NewTerm{path integral}\index{path integral}", "\NewTerm{curve integral}\index{curve integral}", and "\NewTerm{curvilinear integral}\index{curvilinear integral}" are also used; "\NewTerm{contour integral}\index{contour integral}" as well, although that is typically reserved for line integrals in the complex plane.
	
	\paragraph{Curvilinear Integral of a scalar field}\mbox{}\\\\
	Consider a parametrized curve $C$ (\SeeChapter{see section Differential Geometry}) by a vector function $\vec{r}(t)$ with $t \in [a,b]$ of class $\mathcal{C}^1$ piecewise (this condition is necessary so that we can integrate the curve without problems).
	
	\textbf{Definitions (\#\mydef):}
	 \begin{enumerate}
	 	\item[D1.] The curve is said to be a "\NewTerm{closed curve}\index{closed curve}" if $\vec{r}(a)=\vec{r}(b)$
	 	
	 	\item[D2.] The curve is said to be a "\NewTerm{smooth curve}\index{smooth curve}" if $\forall t \in [a,b]\; \vec{r}'(t)\neq 0$
	 \end{enumerate}
	 Recall that a parametric curve can be written as follows (all vector function can be written in this form):
	 
	Consider a function or a "\NewTerm{scalar field}\index{scalar field}" $f(x,y)$ defined in a neighborhood of $C$. We subdivide $[a,b]$ into $n$ subintervals of equal length $\Delta t$ as:
	
	We choose on each subinterval a point $t_i^*\in [t_i,t_{i+1}]$. Given $\delta s_i$ the length of the arc $C$ connecting the point $(x(t_i),y(t_i))$ and $(x(t_{i+1}),y(t_{i+1})$, the integral of $f$ along $C$ is defined as the "\NewTerm{line integral}\index{line integral}" or "\NewTerm{path integral}\index{path integral}":
	
	Which as we know, can be written (see section of Differential Geometry or of Geometric Shapes or even of Analytical Mechanics):
	
	and that can obviously be immediately extended to the case to 3 variables and more.
	
	Or in vector form:
	
	The line integral is linear, that is to say, if $C=C_1 \cup C_2$ and that $C_1 \cap C_2$ is a point, then (without going into the strict definition of the union of two curves...):
	
	
	\paragraph{Curvilinear Integral of a vector field}\mbox{}\\\\
	Consider a vector field (e.g. a force field) as:
	
	and an infinitesimal element of a curve (path) $\mathcal{C}^1$ piecewise as:
	
	The idea is then to consider that the dot product (vector field projection on the path element) represents the work along the differential element:
	
	Therefore the work on all the path will be given by (using the property of linearity of the curvilinear integral):
	
	This can obviously be generalized to $n$ dimensions. Let us indicate that when the line integral (path integral) of a vector field is extended to a closed curve, then we speak of "\NewTerm{circulation of the vector field}\index{circulation of the vector field}".
	
	As:
	
	We then have write a fairly common notation:
	
	\begin{tcolorbox}[title=Remark,colframe=black,arc=10pt]
	In physics many times problems are often in the plane and require the transition to polar coordinates because many academic physic problems are centro-symmetric, which also facilitates the calculations of path integrals.
	\end{tcolorbox}
	
	\begin{tcolorbox}[colframe=black,colback=white,sharp corners]
	\textbf{{\Large \ding{45}}Example:}\\\\
	Let us calculate the work of the force of gravity moving a mass $M$ of the point $M_1(a_1,b_1,c_1)$ to the point $M_2(a_2,b_2,c_2)$ along an arbitrary path $C$. The projections of the force of gravity on the coordinate axes are:
	
	The work accomplish is then:
	
	so we find a very known result of the section of Classical Mechanics.
	\end{tcolorbox}
	A line integral of a vector $\vec{F}$ field along a curve $C_1$ is independent of the path of integration if:
	
	for any non-null curve $C_2$ having only the same points of departure and arrival. Furthermore if the vector field satisfied (where $U$ in physics is typically a potential):
	
	as (the reader will recognize an exact total differential form):
	
	Then the integral path on an arbitrary curve only depends only on the difference of the values the function $U$ at the two ends! This is the "\NewTerm{fundamental theorem for line integrals}\index{fundamental theorem for line integrals}" or "\NewTerm{gradient theorem for line integrals}\index{gradient theorem for line integrals}".
	\begin{dem}
	If the differential form of the vector field satisfies an total exact differential, we have:
	
	That is:
	
	\begin{flushright}
		$\square$  Q.E.D.
	\end{flushright}
	\end{dem}
	So the line integral of an exact total differential does not depend on the path of integration but only the ends! We also conclude that if $\vec{F}$ isderived from a scalar potential and $A = B$, the line integral is then zero.
	
	In physics this result is interpreted by saying that the work provided by a force $\vec{F}$ derived from a scalar potential acting on an elementary particle in a finite displacement does not depend on the path followed.
	
	\textbf{Definitions (\#\mydef):}
	\begin{enumerate}
		\item[D1.] When the curve (path) $C$ is closed and the path integral has a result independent result of the direction in which this path is traveled, we use the following notation (the letter below the symbol representing the path can of course vary...):
		
		If this closed integral is always zero, we say that the integrated vector field is a "\NewTerm{conservative vector field}\index{conservative vector field}" and "\NewTerm{derives from a scalar potential}" (and therefore satisfies the Schwarz theorem to be written as exact total differential) since this stems from the proof given already just above.
		
		\item[D2.] When the value of the integral of a closed path depends on the orientation (clockwise not equal to counterclockwise) we use the following notation (the letter below the symbol representing the path can of course vary...):
		
		Thus if the direction is direct (that is to say "counterclockwise" or "trigonometric") as the notation on the above, its sign will be positive; if on the contrary the direction is clockwise his sign will be negative (see the proof in the section of Complex Analysis). Therefore we often speak about respectively "negative direction" or "positive direction".
		
		Thus, to summarize, a line integral (path integral) is fully defined by the expression under the symbol of the integral, the form of integration path and the direction of the integration.
	\end{enumerate}
	\begin{tcolorbox}[title=Remark,colframe=black,arc=10pt]
	The reader will find some the proofs of very important properties of curvilinear integrals  in section of Vector Calculus as the Green-Riemann theorem or a particular application to study holomorphic functions in the section of Complex Analysis.
	\end{tcolorbox}
	
	\subsubsection{Integrals involving parametric equations}
	Now that we have seen how to calculate the derivative of a plane curve, the next question is this: How do we find the area under a curve defined parametrically? 
	
	To derive an expression for the area under a parametric curve defined by the functions:
	
	with $a\leq t\leq b$.

	We assume that $x(t)$ is differentiable and start with an equal partition of the interval $a\leq t\leq b$. Suppose:
	
	and consider the following figure: $x(t_0),x(t_1),x(t_n)$
	\begin{figure}[H]
		\centering
		\includegraphics{img/algebra/integral_parametric_curve.jpg}
	\end{figure}
	We use rectangles to approximate the area under the curve. The height of a typical rectangle in this parametrization is $y(x(\overline{t_i}))$ for some value $\overline{t_i}$ in the $i$-th subinterval, and the width can be calculated as $x(t_i)-x(t_{i-1})$. Thus the area of the $i$-th rectangle is given by:
	
	Then a Riemann sum for the area is:
	
	Multiplying and dividing each area by $t_i-t_{i-1}$ gives:
	
	Taking the limit as $n$ approaches infinity gives:
	
	And it is obvious that applying Pythagoras's theorem on an infinitesimal length of the parametric curve we have:
	
	with $x=x(t)$, $y=y(t)$ and $t_1<t<t_2$. This gives the arc length\index{arc length} of the parametric curve between two points on the curve.
	
	It comes also immediately:
	
	The chain rule gives:
	
	Therefore:
	
	We will meet again this relation in the section of Analytical Mechanics.
	
	In astronomy we often have to deal with closed curved and calculated the distance travel on that curve. But working in cartesian coordinates is not always is the best. This is why it is better to change in polar coordinates.

	The idea is to suppose that we are able to express our curve of interest in the following form:
	
	where $\alpha\leq \theta \leq \beta$. In order to adapt the arc length relation for a polar curve, we use the relations:
	
	and we replace the parameter $t$ by $\theta$ Then:
	
	we replace $\mathrm{d}t$ by $\mathrm{d}\theta$, and the lower and upper limits of integration are $\alpha$ and $\beta$ respectively. Then the arc length formula becomes:
	
	So finally in polar coordinates:
	
	
	\subsubsection{Improper Integrals}
	Improper integrals are definite integrals where one or both of the boundaries is at infinity, or where the integrand has a vertical asymptote in the interval of integration. As crazy as it may sound, we can actually calculate some improper integrals using some clever methods that involve limits.
	
	By abuse of notation, improper integrals are often written symbolically just like standard definite integrals, perhaps with infinity among the limits of integration. When the definite integral exists (in the sense of either the Riemann integral or the more advanced Lebesgue integral), this ambiguity is resolved as both the proper and improper integral will coincide in value and this is what will the most occur through all applications of integrals in physics, chemistry and engineering through this book!
	
	For the Riemann integral (or the Darboux integral, which is equivalent to it as we have seen earlier above), improper integration is necessary both for unbounded intervals (since one cannot divide the interval into finitely many subintervals of finite length) and for unbounded functions with finite integral (since, supposing it is unbounded above, then the upper integral will be infinite, but the lower integral will be finite)!
	
	An "\NewTerm{improper integral}\index{improper integral}" of a function $f(x) > 0$ is:
	
	We say the improper integral converge if this limit exists and diverges otherwise.
	
	Geometrically then the improper integral represents the total area under a curve stretching to infinity. If the integral $\int_a^\infty f(x)\mathrm{d}x$ converges the total area under the curve is finite; otherwise it's infinite.
	
	How can an area that extends to infinity be finite?  Obviously the area between $a$ and $N$ (i.e. $\int_a^N f(x)\mathrm{d}x$) is finite.  As $N$ goes to infinity this quantity will either grow without bound or it will converge to some finite value. 
	
	The domains where improper integrals are the most used, without even be noticeable most of time by students or engineering practitioners, are respectively:
	\begin{itemize}
		\item Statistics (see corresponding section) when we normalize or check the condition of covergence to a cumulated probability of $1$ of a density function (most of time in statistics one or the both bounderies are equal to infinity)

		\item Wave Quantum Physics (see corresponding section) where we deal sometimes with free propagating particles from infinity to infinity (this also happens sometimes in General Relativity)

		\item Differential equations, especially when we solve them by using Fourier Transform or Laplace transform (we have many examples using this transforms accross the book) for practical application in physics and high level financial engineering.

		\item In astrophysics or electrostatics when dealing with any punctual potential field source of the type $1/r^2$  for which we want to calculate the work necessary to bring an object from infinity to that source (calculation of the type $\int_{+\infty}^{r} f(r)\mathrm{d}r$).
	\end{itemize}
	\begin{tcolorbox}[colframe=black,colback=white,sharp corners]
	\textbf{{\Large \ding{45}}Examples:}\\\\
	E1. We want compute the very important integral (to introduce Laplace Transforms):
	
	with $a\in\mathbb{R}$.\\
	
	Following the definition above we need to first compute a definite integral and the take a limit. So from the definition:
	
	We first compute the definite integral. We start with the case $a=0$:
	
	therefore for $a=0$ the improper integral $I$ does not exist. When $a\neq 0$ we have:
	
	In the case $a<0$, that is $a=-|a|$, we have that:
	
	therefore for $a<0$ the improper integral $I$ does not exist. In the case $a>0$ we have:
	
	
	E2. We want to compute:
	
	The integrand is not continuous at $x=0$ and so we’ll need to split the integral up at that point:
	
	
	\end{tcolorbox}
	
	\begin{tcolorbox}[colframe=black,colback=white,sharp corners]
	Now we need to look at each of these integrals and see if they are convergent
	
	At this point we're done.  One of the integrals is divergent that means the integral that we were asked to look at is divergent.  We don't even need to bother with the second integral.
	\end{tcolorbox}
	On a side note, notice that the area under the curve of this infinite interval was not infinity as the reader may have suspected it to be (perhaps).  In fact, it was a surprisingly small number.  Of course this won't always be the case, but it is important enough to point out that not all areas on an infinite interval will yield infinite areas.
 
	Let's now get some definitions out of the way.  We will call these integrals convergent if the associated limit exists and is a finite number (i.e. it's not plus or minus infinity) and divergent if the associated limit either doesn't exist or is (plus or minus) infinity.
	
	\pagebreak
	\subsection{Differential Equations}
	\textbf{Definition (\#\mydef):} In mathematics, a "\NewTerm{differential equation D.E.}\index{differential equation}"  is a relationb etween one or more unknown functions and their derivatives up to order $n$. The "\NewTerm{order}\index{order of a differential equation}" of a differential equation corresponding to the maximum degree of differentiation which one of the functions is subjected.
	
	Compared to our goal of trying to see how the math describes the sensible reality, differential equations are a great success but are also the source of many troubles. First there are modeling difficulties (see for example the differential equation system of General Relativity in the corresponding section of the book...) resolution difficulties (there is no general method even with numerical computer methods as you can see in the corresponding section!), then their are proper mathematical difficulties (that's why some D.E. have million dollar price in case or resolution), finally difficulties related to the fact that certain differential equations are unstable by nature and give chaotic solutions (see the section Population Dynamics or Meteorology for flagrant simple examples!).
	
	\begin{tcolorbox}[title=Remark,colframe=black,arc=10pt]
	The differential equations are used to construct mathematical models of physical orbiological phenomena, such as for the study of radioactivity, celestial mechanics, electronic circuits, populatin development or even financial stochastic process. Therefore, differential equations represent a vast field of study, both in pure and Applied Mathematics.
	\end{tcolorbox}
	The differential equation of order $n$ the more general can always be written as:
	
	We consider in this book only the cases where $x$ and $y$ have their values in $\mathbb{R}$. A solution to such a D.E. on the interval $I \in \mathbb{R}$ is a function $y \in \mathcal{C}^n (I,\mathbb{R})$ (a function $y:I \rightarrow \mathbb{R}$ which is $n$ times continuously differentiable) such that for any $x \in I$, we have:
	
	\begin{tcolorbox}[title=Remarks,colframe=black,arc=10pt]
	\textbf{R1.} For reasons that will be developed later, we also say "integrate a D.E." instead of saying "finding a solution to the D.E.". The first expression is particularly found in the american literature.\\
	
	\textbf{R2.} Since all this book is full examples of differential equations with initial conditions (we speak then about a "\NewTerm{Cauchy problem}\index{Cauchy problem}") and methods of resolutions in the section of Classical Mechanics, Atomic Physics, Cosmology, Econometry, Sequences and series, Industrial Engineering, Statistics, etc., we will not give any application examples here and will focus only on the minimum theoretical useful aspect.
	\end{tcolorbox}
	
	\pagebreak
	\subsubsection{First order Differential Equations}
	A differential equation of the first order is therefore a D.E. which involves only the first derivative $y'$.
	
	\textbf{Definition (\#\mydef):} A first order differential equation is named "D.E. of order 1 with separate variables" if it can be written as:
	
	Such a differential equation can be easily integrated. Indeed, we write:
	
	Then symbolically:
	
	\begin{tcolorbox}[title=Remark,colframe=black,arc=10pt]
	We write explicitly here the arbitrary integration constant $c^{te} \in \mathbb{R}$ (which is normally implicitly present in the indefinite integrals) to not forget it!
	\end{tcolorbox}
	So the purpose is first to find the primitives $F$ and $G$ of $f$ and $g$, and then to express it in terms of $x$:
	
	The integration constant is fixed when asked for a given $x=x_0$, we get a particular value of $y(x)=y(x_0)=y_0$. We speak then of "\NewTerm{initial value problem}\index{initial value problem}".
	
	\subsubsection{Linear Differential Equations}
	\textbf{Definition (\#\mydef):} A differential equation of order $n$ is named "\NewTerm{linear differential equation L.D.E.}\index{linear differential equation}" if and only if it is of the form:
	
	with:
	
	Let us now see a property that may seem insignificant at first glance but which will become very important later!
	
	We will prove now that $L$ is a linear application:
	
	and for all $\lambda \in \mathbb{R}$
	
	Then we say that the linear D.E. represent a linear model if the multiples of this function (or any linear combination) are also a solution. Thus, in physics, for a linear system, the amplification of the cause involves an amplification of the effect (the systems are often linear in high-school problems but in reality they are rather the exception!).
	
	For example, the ordinary differential equation of order $2$ of the simple pendulum proved in the section of Classical Mechanics is not linear because it contains a sine term that is not separable.

	\textbf{Definition (\#\mydef):} The linear differential equation (which is the most common in physics):
	
	is named "\NewTerm{homogeneous equation H.E.}\index{homogeneous equation}" or " \NewTerm{equation without second member E.W.S.M.}\index{equation without second member}" (and sometimes "\NewTerm{complementary equation}\index{complementary equation}") associated to:
	
	\begin{theorem}
	We will now prove an important property of H.E.: the set $\left\lbrace S_0 \right\rbrace$ of solutions of the H.E. is the kernel of the linear application $L$ (which means for refresh: $L(S_0)=0$) and the set  $\left\lbrace S \right\rbrace$ of solutions to $L(y)=f(x)$ is given by:
	
	that is to say that the solutions of the form:
	
	where $y_p$ is a "particular/specific solution" to $L(y)=f(x)$ and $y_h$ the "\NewTerm{homogeneous solution}\index{homogeneous solution}" give all the D.E. solutions.
	\end{theorem}
	\begin{dem}
	The first statement will be assumed obvious.
	
	As regards to the second part, any function of the form $y_p+y_h$ is solution of $L(y)=f(x)$.
	
	Indeed it is trivial and it follows from the definition of the kernel concept (\SeeChapter{see section Set Theory}):
	
	\begin{flushright}
		$\square$  Q.E.D.
	\end{flushright}
	\end{dem}
	What is important also to understand with the linear D.E. with second member, it is that if we find solutions to $L(y)$ with a second given member and solutions to the same D.E. with another different second member, then the sum of all these solutions will be a solution of the D.E. with the sum of the second members!!!
	
	\subsubsection{Resolution Methods of Differential Equations}
	There are many ways to solve accurately or approximately linear or non-linear differential equations. Let us give the list of the few methods we will analyze further below by the example (but who are already many, many times in the chapters of Mechanics, Cosmology, Social Sciences and Quantum Physics):
	\begin{itemize}
		\item The "\NewTerm{method of characteristic polynomial of D.E.}\index{Differential equations!method of characteristic polynomial}" (see below) used a bit in every section of the chapter of Mechanics/Quantum Physics/Cosmology/Chemistry and Social Sciences of this book.
		
		\item The "\NewTerm{method of integrating factor}\index{Differential equations!method of integrating factor}" (see also below) for general knowledge but used to this date for practical cases in this book.
		
		\item The "\NewTerm{method of variation of the constant}\index{Differential equations!method of variation of the constant}" (see below) and used to this date only in the section of Industrial Engineering in this book.
		
		\item The "\NewTerm{method of disturbances of D.E.}\index{Differential equations!method of disturbances}" (see below) useful for the wave quantum physics and quantum field theory.
	\end{itemize}
	
	Note also other widely used methods (classical high-school technics) but that are treated case by case in the individual sections of this book because solving approaches are too numerous and specific to each problem:
	
	\begin{itemize}
		\item The "\NewTerm{separation of variables method of D.E.}\index{Differential equations!separation of variables}" (the heat equation in the section of Thermodynamics, wave equation in the section Marine \& Weather Engineering, Schrödinger evolution equation in the section of Wave Quantum Physics, vibration of a drum in the section of Wave Mechanics, etc.), whose we will see a very specific and simple case lower but for which it is best to refer to the sections mentioned for concrete examples.
		
		\item The "\NewTerm{matrix method for solving D.E.}\index{Differential equations!matrix method}" and "\NewTerm{trivial solution of D.E.}" (Lotka-Volterra model in the section of Populations Dynamics, electron or nuclear spin resonance in the section of Relativistic Quantum Physics, Lorenz model in the section of Marine \& Weater Engineering, etc.).
		
		\item The "\NewTerm{spectral method}\index{Differential equations!spectral method}" using the spectral theorem proved in the section of Linear Algebra (see the section of Industrial Engineering for the calculation of system reliability by Markov chains for a concrete example).
		
		\item The "\NewTerm{method of the Fourier transform of the D.E.}\index{Differential equations!Fourier transform method}" or "\NewTerm{method of the Laplace transform of the D.E.}\index{Differential equations!Laplace transform method}" (heat equation in the section Thermodynamics, resolution of the Black \& Scholes equation in the section of Economy, beam equation  under point load in the section of Civil Engineering).
		
		\item "\NewTerm{Numerical methods for D.E.}\index{Differential equations!Numerical method}" to solve the differential equations using computer when the D.E. have no known analytic solutions or when they have but we need a visual three dimensional view of the solutions (heat equation in the section of Theoretical Computing).
		
		\item The "\NewTerm{Frobenius method}\index{Differential equations!Frobenius method} named after Ferdinand Georg Frobenius and  also "\NewTerm{power series solutions}\index{Differential equations!power series solutions}, that is a way to find an infinite series solution for a second-order ordinary differential equation of a special form. We will use this technique in the section of Sequences and Series for our study of the Bessel series and also introduce Bessel series by solving in the section of Mechanical Engineering the probel of the self-buckling column with the power series solutions.
	\end{itemize}
	\begin{tcolorbox}[title=Remark,colframe=black,arc=10pt]
	The first differential equations were solved around the end of the 17th century and beginning of the 18th century. By the middle of the 18th century people realized that the first methods we listed above had reached a dead end. One reason was the lack of functions to write the solutions of differential equations. The elementary functions we use in calculus, such as polynomials, quotient of polynomials, trigonometric functions, exponentials, and logarithms, were simply not enough. People even started to think of differential equations as sources to find new functions. It was matter of little time before mathematicians started to use power series expansions to find solutions of differential equations. Convergent power series define functions far more general than the elementary functions from calculus.
	\end{tcolorbox}
	
	\paragraph{Method of characteristic polynomial}\mbox{}\\\\
	Solving simple differential equations (with constant coefficients and without second member most of the time...) uses a technique using a characteristic polynomial of the differential equation which we will see the details in the developments that follow on  few special cases very frequent in physics.
	
	It is a relatively simple method to implement when we seek solutions to the homogeneous differential equation without second member (E.W.S.M.). In the contrary case, the presence of a second member, we add the solutions of the homogeneous equation to the particular solutions.
	
	\subparagraph{Resolution of the H.E. of the first order L.D.E. with constant coefficients}\mbox{}\\\\

	Consider the following L.D.E. with constant coefficients:
	
	which is a simplified version of the following general L.D.E.  with constant coefficients:
	
	where:
	
	We write its associated homogeneous equation (E.W.S.M.):
	
	Which can be written:
	
	Therefore:
	
	There is behind this homogeneous solution infinite solutions: to each value given to the constant $C$ there is a solution.
	
	We still need to add to this homogeneous solution the particular solution $y_p$ and for that we have a collection of recipes, depending on the type of the function $f (x)$ of the second member of the differential equation. We will see in each case in the various chapters this book as already mentioned.
	
	\subparagraph{Resolution of the H.E. of the first order L.D.E. with non-constant coefficients}\mbox{}\\\\
	The general solution of homogeneous linear differential equations (E.W.S.M.) of order $1$ with non constant coefficients:
	
	can always be reduced to the following form:
	
	where:
	
	Well obviously there is the solution $y=0$... but let us try to do better. So we have:
	
	It comes therefore:
	
	where $G (x)$ is a primitive of $g (x)$. Since then:
	
	It is also common to find these developments in another notation a little bit more explicit.
	
	So we start again of the differential equation without second member with non-constant coefficients:
	
	after rearrangement:
	
	And then:
	
	Therefore:
	
	This result will be very useful to calculate the Fourier transform of a Gaussian function (\SeeChapter{see section Sequences And Series}), Fourier transform, which is essential to resolve in a fairly general way the heat equation (\SeeChapter{see section Thermodynamics}) resolution that will finally allow us to prove the Black \& Scholes equation (\SeeChapter{see section Economics}).
	
	\subparagraph{Resolution of the H.E. of the second order L.D.E. with constant coefficients}\mbox{}\\\\
	Consider the L.D.E. with constant coefficients:
	
	which is a simplified version of the following general L.D.E. with constant coefficients:
	
	where:
	
	We write is homogeneous associated equation (E.W.S.M.):
	
	wherein the function of the second member is zero. We can quite quickly consider a solution of the type (inspiring of the form of the solutions of the first order differential equations):
	
	where $\tau$ is a constant. Which give us therefore:
	
	What we can simplify by:
	
	If our starting assumption is correct, we only have to solve in $K$ this "\NewTerm{characteristic equation (CHARE)}\index{characteristic equation}"  or "\NewTerm{characteristic polynomial}\index{characteristic polynomial}" of the homogeneous equation to find the homogeneous solution:
	
	whose solutions depend on the sign of the discriminant of the characteristic polynomial:
	
	\begin{itemize}
		\item If the discriminant is strictly positive, that is to say $\Delta>0$:
		
		So we know that the characteristic polynomial has two distinct roots and then we have:
		
		where $K_1\tau=c^{te}$ and $K_2\delta=c^{te'}$. Then we say that the solution is "\NewTerm{delayed}\index{Differential equations!delayed solution}" or "\NewTerm{advanced}\index{Differential equations!advanced solution}" by the values of these constants. But the key is to note that if $y_h(x)$ is a solution, then $y_h(x\pm \Delta x)$ is always a solution!
		
		We then speak of "\NewTerm{general solution of the homogeneous equation}\index{general solution of the homogeneous equation}". There is behind this result an infinity of solutions: to each value given to the constants $A, B$ corresponds a solution.
		
		Physicists also write sometimes this in a particular form by putting
		
		with then:
		
		And using the hyperbolic trigonometric functions (\SeeChapter{see section Trigonometry}):
		
		where finally the possibility to write the homogeneous solution in the form (when we omit the advance or delay $\delta=\tau=0$):
		
		In addition, let us show that the solutions of the E.W.S.M. form a vector space of dimension $2$ (corresponding to the order of our differential equation)!
		
		Indeed:
		\begin{itemize}
			\item The zero function: $y=0$ is a solution of the E.W.S.M. (this is unnecessary to prove because obvious...!)
			
			\item The sum or subtraction of solutions remains a solution (this we have already proved it before)
			
			\item The elements of the basis of a vector space (the solutions of the E.W.S.M.) are linearly independent (that's an interesting property that we will need later!)
		\end{itemize}
		Let us put:
		
		Then:
		
		These relations injected into the E.W.S.M. in generalized form:
		
		Then gives:
		
		We do have indeed a vector space structure.
		
		Let us recall that conversely two functions are linearly dependent if:
		
		\item If the discriminant is strictly positive, that is to say $\Delta=0$: 
		
		The characteristic equation has a real double root $K$.
		
		By going a little fast we would say then:
		
		and that it is over... but that it is forget that the vector base must be formed of two independent solutions!
		
		So the second option is probably... of the form:
		
		Then:
		
		If we inject it into the E.W.S.M. in generalized form:
		
		Then:
		
		That is to say in our case:
		
		But, both actual real values of $K$ are precisely solutions of:
		
		The prior-previous relation is reduced to:
		
		and as we are in the case of study where the discriminant is zero, we have:
		
		Therefore and finally the prior-previous relation reduces to:
		
		We deduce from it that:
		
		Therefore finally:
		
		Which gives for the general solution of the E.W.S.M.:
		
		
		\item If the discriminant is strictly positive, that is to say $\Delta<0$:
		The characteristic equation has two complex conjugate roots (\SeeChapter{see section Algebra}):
		
		Therefore:
		
		But if we look instead for real solutions, we can always put $A$ and $B$ as being equal such as:
		
		And if we set the delay and advance respectively as being zero ($\delta=\tau=0$), so we find the relation available in most books without proof:
		
		where $A'$ and $B'$ are any two real constants. There is another important form of this last relation (often used in electronics, for example). Indeed, it is possible for any $A'$ and $B' \in \mathbb{R}$ , to find $C'$ and $\phi$ also in $\mathbb{R}$ such as the following equality holds:
		
		We put:
		
		Then:
		
		It is then possible to find $\phi$ such as:
		
		Therefore our initial expression (proposition) can be written as:
		
		Finally:
		
	\end{itemize}
	So we can make the following summary:
	To resume:
	
	
	\paragraph{Integrating Factor Method (Euler's Method)}\mbox{}\\\\
	The technique of integrating factor is useful when it comes to solve differential equations of the form:
	
	We have not to this day practical application of this technique in the other chapters of this book. You must therefore see this as a presentation for general culture.
	
	The basic idea is to find a function $M(x)$, named "\NewTerm{integration factor}\index{integration factor}", by which can be multiplied by our differential equation to bring the left-hand side of equality to a simple derivative. For example, for a linear differential equation as the one above, we choose often the following integration factor (but this is by far not the only possibility and this choice does not solve everything!):
	
	Therefore we have:
	
	or by distributing:
		
	Which can therefore be seen as:
	
	or even better (and therein lies the whole trick)...:
	
	We can then take the primitive with respect to $x$:
	
	We can then take the primitive with respect to $x$:
	
	and trivially (!) we have the left primitive that is immediate:
	
	Which is sometimes written as:
	
	\begin{tcolorbox}[colframe=black,colback=white,sharp corners]
	\textbf{{\Large \ding{45}}Example:}\\\\
	Consider the following differential equation:
	
	That we we will rewrite as:
	
	We see then that (assuming $x$ is positive):
	
	Then we have:
	
	Hazard making  sometimes things good (the example is purposely very simple), we have this equality that simplified as:
	
	in:
	
	Which may be condense in:
	
	By integrating:
	
	It then comes immediately:
	
	Finally:
	
	\end{tcolorbox}
	
	\pagebreak
	\paragraph{Method of separation of variables}\mbox{}\\\\
	The method of separation of variables (also known as the "\NewTerm{Fourier method}\index{Fourier method}") is any of several methods for solving ordinary and partial differential equations, in which algebra allows one to rewrite an equation so that each of two variables occurs on a different side of the equation.
	
	In mathematics, a "\NewTerm{partial differential equation PDE}\index{partial differential equation}" is a differential equation that contains unknown multivariable functions and their partial derivatives (a special case are ordinary differential equations, which deal with functions of a single variable and their derivatives). PDEs are used to formulate problems involving functions of several variables, and are either solved by hand, or used to create a relevant computer model.
	
	PDEs can be used to describe a wide variety of phenomena such as sound, heat, electrostatics, electrodynamics, fluid dynamics, elasticity, or quantum mechanics. These seemingly distinct physical phenomena can be formalized similarly in terms of PDEs. Just as ordinary differential equations often model one-dimensional dynamical systems, partial differential equations often model multidimensional systems. PDEs find their generalization in stochastic partial differential equations.
	
	A partial differential equation (PDE) for the function $U(x_{1},\cdots ,x_{n})$ is an equation of the form:
	
	If $f$ is a linear function of $U$ and its derivatives, then the PDE is named a "\NewTerm{linear partial differential equation}\index{linear partial differential equation}". Common examples of linear PDEs include the heat equation, the wave equation, Laplace's equation, Helmholtz equation, Klein–Gordon equation, and Poisson's equation (see the chapters of Mechanics, Electrodynamics and Quantum Physics for the study of most of them!).
	
	The method of separation of variables is a very common technique used in physics when we have second-order differential equations. Many useful examples and very detailed are already in the various chapters already mentioned above. Here we will just present a special case just for doing things good but with the minimum subsistence level!
	
	Consider the common case of physical partial differential equation of the type:
	
	The solution of this equation therefore requires finding a function $U$ which depends on $x$ and $y$ such that:
	
	In physics, the idea is then to put that we can always find a said a "separable" solution of the form:
	
	Thus, the differential equation can be written as:
	
	Which can be rewritten as:
	
	Or:
	
	After rearrangement it is use in physics to note this last equality in condensed form:
	
	This equality can only take place if each term is a constant since $X$ depends only on $x$ and Y only on $y$. It comes then:
	
	And each differential equation can then be solved independently of the other and once the solutions found we multiplied them determine the expression of $U$.
	\begin{tcolorbox}[title=Remark,colframe=black,arc=10pt]
	Probably the most beautiful example is the section of Quantum Chemistry.
	\end{tcolorbox}
	
	\paragraph{Method of constant variation}\mbox{}\\\\
	The variation of constants, is a general method to solve inhomogeneous linear ordinary differential equations.

	For first-order inhomogeneous linear differential equations it is usually possible to find solutions via integrating factors or undetermined coefficients with considerably less effort, although those methods leverage heuristics that involve guessing and don't work for all inhomogeneous linear differential equations.

	Variation of parameters extends to linear partial differential equations as well, specifically to inhomogeneous problems for linear evolution equations like the heat equation, wave equation, and vibrating plate equation. In this setting, the method is more often known as "\NewTerm{Duhamel's principle}\index{Duhamel's principle}", named after Jean-Marie Duhamel who first applied the method to solve the inhomogeneous heat equation.
	
	The idea of the method of variation of the constant is as follows: if we have a particular solution affected by constants, we know that depending of the initial conditions thereof are well determined. The idea is then to generalize by putting that these constants are functions. In some cases obviously mathematical developments will show that the functions are necessarily constant.
	
	The idea behind this method is to say that the solutions of the (linear) differential equation with the second member will look like the solutions of the homogeneous equation. As the term on the right will disrupt this solution, we vary only constants (which will therefore no longer be constants), but we remain on the "base" of homogeneous solutions, to seek close solutions. After we check that this "physicist reasoning" gives out all the solutions of the differential equation.
	
	Let us see before studying to the general case a simple example by considering the following differential equation:
	
	for which the particular solution of the homogeneous equation (E.W.S.M.) is (if you need the details not hesitate to ask!):
	
	The method of variation of the constant consist then to put that:
	
	and therefore:
	
	But by the differential equation with the second member, we have:
	
	
	and it follows that:
	
	where we eliminated the integration constant because what we want is a particular solution! The particular general solution (pg) is then the sum of the particular solution and the homogeneous one and this with the variation of the constant:
	
	Thus, generalizing the previous example, so we have a differential equation of the form:
	
	General particular solution will be:
	
	Then we have:
	
	hence injected into the original differential equation:
	
	Therefore after factoring similar terms:
	
	So we have the above relation and the particular solution to the homogeneous differential equation (therefore without second member):
	
	Therefore we find:
	
	and it sufficient then to integrate this equation to find $C_0(t)$. Then, the particular general solution (pg) will be the sum of the particular homogeneous solution and that with the variation of the constant.
	
	\pagebreak
	\subsubsection{Classification of partial differential equations}
	Before we begin, the reader wonders what classifying differential equations can be used for, well, here are the two main arguments for the usefulness of a classification in the order of importance:
	\begin{itemize}
		\item Many books have authors who systematically speak of a differential equation by categorizing it, so it is more pleasant to know what they are talking about

		\item Some finite element modeling software (including MATLAB™) requires that the differential equation category be chosen before anything else can be done as a calculation.
	\end{itemize}
	So this being said, formally, we name "\NewTerm{partial differential equation PDE}\index{partial differential equation}" of order less than or equal to $2$ in a domain $\Omega \subset \mathbb{R}^n$ and of unknown:
	
	An equation of the following general type:
	
 	where $s(x)$ is often named a "\NewTerm{source term}\index{source term}" in analogy with the main situation in physics concerned.

	It is now important to generalize the latter equation in vector form. Thus, we introduce the following notations:
	\begin{itemize}
		\item $A=[a_{ij}]$ the $n\times n$ symmetric matrix of the coefficients in front of the terms of order $2$

		\item $B=(f_i(x))$ The vector of size $n$ of the coefficients in front of the terms of order $1$

		\item $[H\Phi(x)]$ the $n\times n$ symmetric Hessian matrix (\SeeChapter{see section Sequences and Series}) of $\Phi$:
		

		\item $\vec{\nabla}(\Phi(x))$ the vector of size $n$ of $\Phi$:
		

		\item The notation (named "\NewTerm{Frobenius (matrix) dot product}\index{Frobenius (matrix) dot product}"):
		
	\end{itemize}
	With this, the previous relation:
	
	can be rewritten as:
	
	What is often (abusively) written in a very condensed way:
	
	\begin{tcolorbox}[colframe=black,colback=white,sharp corners]
	\textbf{{\Large \ding{45}}Example:}\\\\
	The following PDE:
	
	First, it is easy to determine (because it is a definition) that:
	
	It is also trivial that:
	
	and that:
	
	It is also trivial that:
	
	Finally, the minor difficulties are to find:
	
	\end{tcolorbox}
	For linear PDE of order $2$, the matrix $A$ is non-zero and is symmetric. It is therefore diagonalizable with real eigenvalues (\SeeChapter{see section Linear Algebra}) and their study provides elements of classification of linear PDE systems of order $2$ under the denomination of "Elliptic", "hyperbolic" or "parabolic" PDE (\SeeChapter{see section Analytical Geometry}).

	It is often customary to say in physics that the elliptics PDE characterize problems of equilibrium or stationarity, that hyperbolics PDE characterize problems of wave propagations and finally that parabolic PDE characterize diffusion problems (see examples further below).
	
	That latter terminology comes from the fact that when the matrix $A$ is constant, the curves:
	 
	are respectively ellipsoids, hyperboloid and paraboloid (\SeeChapter{see section Analytical Geometry}).

	Indeed, we have proved in the section of Analytic Geometry that following the determinant of $A$, we had:
	\begin{itemize}
		\item If $\det(A)=ac-b^2>0$, the curve $\Gamma$ is either empty, reduced to a point, or an ellipse.

		\item If $\det(A)=ac-b^2<0$, the curve $\Gamma$ is either the union of two intersecting lines, that is to say an hyperbola.

		\item If $\det(A)=ac-b^2=0$, the curve $\Gamma$ is either empty, a line, or two distinct parallel lines, or a parabola.	
	\end{itemize}
	More explicitly what we have seen above can be reformulated as following. If we have the following PDE:
	
	where $a$, $b$, $c$, $d$, $e$ and $f$ are real constants is said to be:
	\begin{itemize}
		\item An "\NewTerm{elliptical partial differential equation}\index{elliptical partial differential equation}" if $ac-b^2>0$ that is to say if the eigenvalues are all positive or all negative (\SeeChapter{see section Analytical Geometry})

		\item An "\NewTerm{hyperbolic partial differential equation}\index{hyperbolic partial differential equation}" if $ac-b^2<0$ that is to say there only one negative eigenvalue and all the rest are positive, or there is only one positive eigenvalue and all the rest are negative (\SeeChapter{see section Analytical Geometry})

		\item A "\NewTerm{parabolic partial differential equation}\index{hyperbolic partial differential equation}" if $ac-b^2=0$ that is to say if the eigenvalues are all positive or all negative, save one that is zero (\SeeChapter{see section Analytical Geometry})
	\end{itemize}

	\begin{tcolorbox}[colframe=black,colback=white,sharp corners]
	\textbf{{\Large \ding{45}}Example:}\\\\
	The Laplace's equation:
	
	is an elliptic PDE.\\

	The wave equation:
	
	is a hyperbolic PDE.\\

	The heat equation:
	
	is a parabolic PDE.
	\end{tcolorbox}
	
	\pagebreak
	\subsection{Systems of Differential Equations}
	Let us study now special developments that will also be useful for the study of quantum physics or for the resolution of particular systems of differential equations (see corresponding section in this book for the details on these examples) and especially one which is well known in chaos theory!
	
	Let us first indicate to the reader before going further that more complex inhomogeneous case (with second member) and with unknown coefficients is treated directly by an example in the section of Industrial Engineering during the study of the reliability of a repairable system as a Markov chain with resolution using the determinants and eigenvalues/eigenvectors.
	
	To start this first approach, we will have to introduce the concept of exponentiation of a matrix:

	The set of matrices $n \times n$ with coefficients in $\mathbb{C}$ denoted $M_n(\mathbb{C})$ is a vector space for the addition of matrices and multiplication by a scalar. We will as always denote by $\mathds{1}_n$ the identity matrix of dimension $n$.
	
	We will admit that a sequence of matrices $A_n$ converges to a matrix $A$ if and only if the sequences of coefficients of the matrices $A_n$ converge towards the corresponding coefficients of $A$.
	
	\begin{tcolorbox}[colframe=black,colback=white,sharp corners]
	\textbf{{\Large \ding{45}}Example:}\\\\
	In $M_2(\mathbb{C}$ the sequence of matrice:
	
	converge to:
	
	when $n\rightarrow +\infty$.
	If $x\in \mathbb{C}$, we saw in our study of complex numbers (\SeeChapter{see section on Numbers}) that the series:	
	
	converges and its limit is denoted by $e^x$. In fact here there is no difficulty in replacing $x$ by a matrix A$ $since we know (we have proved it during our study of complex numbers) than any complex number can be written as follows (the body of complex numbers is isomorphic to the field of real matrices of square dimensions $2$ having this form):
	
	\end{tcolorbox}
	\pagebreak
	\begin{tcolorbox}[colframe=black,colback=white,sharp corners]
	and that a complex number is equivalent to put his matrix form also to the square:
	
	Indeed:
	
	\end{tcolorbox}
	We then define the exponential of a matrix $A\in M_n(\mathbb{C})$ as the matrix limit of the sequence:
	
	If the matrix $A$ is diagonal obviously its exponential is easy to calculate. Indeed, if:
	
	It follows:
	
	However, it is clear that a non-diagonal matrix will be much more complicated to deal with! We will then use the diagonalization technique or reduction of endomorphisms (\SeeChapter{see section Linear Algebra}).
	
	So note that if $S\in M_n\mathbb{C}$ is reversible and if $A\in M_n\mathbb{C}$  so:
	
	This follows from the fact that (think of the change of base of a linear application as has been studied in the section of Linear Algebra):
	
	Therefore:
	
	This development will enable us to bring the computing of the exponential of a diagonalizable matrix in search of its eigenvalues and its eigenvectors.
	\begin{tcolorbox}[colframe=black,colback=white,sharp corners]
	\textbf{{\Large \ding{45}}Examples:}\\\\
	Let us calculate $e^A$ where:
	
	The eigenvalues of $A$ are $\lambda_1=-3,\lambda_2=7$, and associated eigenvectors are:
	
	Indeed:
	
	By putting:
	
	We get:
	
	with:
	
	Therefore:
	
	\end{tcolorbox}
	Now, let us recall that in the case of real numbers we know that if $x,y\in \mathbb{R}$ then:
	
	In the case of matrices we can prove that if $A,B\in M_n(\mathbb{C})$ are two matrices that commute with one another, that is to say such that $AB=BA$, then:
	
	The condition of commutativity comes from the fact that the addition in the exponential is itself commutative. The proof is therefore intuitive.
	
	An important corollary of this proposition is that for any matrix $A\in M_n\mathbb{C}$, $eÂ$ is reversible. Indeed the matrices $A$ and $-A$ commute and therefore:
	
	We recall that a matrix $A$ with complex coefficients is unitary if:
	
	The following theorem will serve us later:
	\begin{theorem}
	Let us prove that if $A$ is a Hermitian matrix (also named "self-adjoint") (\SeeChapter{see section Linear Algebra}) then for any $t\in\mathbb{R}$, $e^{\mathrm{i}tA}$ is unitary.
	\end{theorem}
	\begin{dem}
	
	Therefore:
	
	\begin{flushright}
		$\square$  Q.E.D.
	\end{flushright}
	\end{dem}
	Remember that this condition for a self-adjoint matrix is linked to the definition of unit group of order $n$ (\SeeChapter{see section Set Algebra}).
	
	One of the first applications of the exponentation of matrices is the resolution of ordinary differential equations. Indeed, from the linear differential equation below using as initial condition $y(0)=0$ and where $A$ is a matrix:
	
	the solution is given by (as seen previously):
	
	We frequently find that kind of systems of differential equations in biology (population dynamics), astrophysics (study of plasmas) or in fluid mechanics (chaos theory) and in classical mechanics (coupled systems), astronomy (coupled orbits), in electrical engineering, etc.
	
	\begin{tcolorbox}[colframe=black,colback=white,sharp corners]
	\textbf{{\Large \ding{45}}Example:}\\\\
	Suppose we have the following homogeneous system of differential equations (without constant terms):
	
	The associated matrix is then:
	
	and its exponential (see developments made above):
	
	The general solution of the system is:
	
	So we have:
	
	By calculating the derivative of the previous relations and comparing to:
	
	we easily determine the constants to get:
	
	which finally gives us:
	
	\end{tcolorbox}

	\subsection{Regular Methods of Perturbations}
	Very frequently in physics (high level physics) or financial engineering, a mathematical problem can not be solved exactly. Even if the solution is known sometimes there are such a dependency of parameters that the solution is difficult to use as such.
	
	Sometimes, however, it happens that an identified parameter of the differential equation, which we denote by tradition with the Greek letter $\varepsilon$, is such that the solution is available and reasonably simple for $\varepsilon=0$.
	
	The problem then is to know how the solution is altered for a non-zero $\varepsilon$ but still small. This study is the center of "\NewTerm{perturbation theory}\index{perturbation theory}" that we will use, for example in the section of General Relativity to calculate the precession of the perihelion of Mercury.
	
	As the perturbation theory within the general framework is too complex for this book purpose, we propose an approach by example with first a simple algebraic equation and then with what interests us: a differential equation.
	
	\subsubsection{Perturbation theory for algebraic equations}
	Consider the following polynomial equation:
	
	We know from our study of the section Functional Analysis, that this polynomial equation has two roots that are trivially:
	
	For small $\varepsilon$, these roots can be approximated by the first term of the Taylor series expansion (\SeeChapter{see section Sequences and Series}):
	
	The question is whether we can get the two previous relation without a priori knowledge of the exact solution of the initial polynomial equation? The answer is obviously: YES with the help of perturbation theory.
	
	The technique is based on four steps:
	\begin{enumerate}
		\item In the first step, we assume that the solution of the polynomial equation is an expression of the type Taylor series on $\varepsilon$. Then we have:
		
		where $X_1,X_2,X_3$ are obviously to be determined.
		\item In the second step, we inject the solution in our hypothetical polynomial equation:
		
		As:
		
		and:
		
		It finally comes that the polynomial equation can be written as:
		
		
		\item In the third step we successively equalize the terms with 0 such as:
		
		\item Fourth and last step, we solve successively the polyinomial equations above to get:
		
		By injecting these results in the hypothetical solution:
		
		it is obvious to observe that we fall back on the certain solution:
		
	\end{enumerate}
	
	\pagebreak
	\subsubsection{Perturbation theory of differential equations}
	Perturbation theory is therefore also often used to resolve numerous differential equations. This is the case for example in fluid mechanics, in General Relativity or quantum physics.
	
	Again, rather than doing a super abstract and general theory, we will see the concept with an example as previously.
	
	Consider the following ordinary differential equation with second member and constant coefficients:
	
	Or written in another way?
	
	with the boundary conditions:
	
	The exact resolution is relatively easy to obtain:
	
	First we start with the homogeneous equation:
	
	So it is a linear differential equation of order 2 with constant coefficients, equation that it is relatively easy to solve in the general case. Given the equation:
	
	Assume that the function $y$ which satisfies this differential equation is the of the form $y=e^{Kx}$ where $K$ may be a complex number. Then we have:
	
	provided, of course, that $e^{Kx}\neq 0$. This last relation is the auxiliary quadratic equation of the differential equation (characteristic polynomial in other words). It has two solutions/roots (it's a simple resolution of a polynomial of the second degree) which we denote in the general case: $K_1,K_2$. Which means that:
	
	are satisfied for the two roots. If we do the sum, since both are equal to the same constant:
	
	Thus, it is immediate that the general solution of the homogeneous equation is of the type:
	
	where $A, B$ are obviously constants to be determined. Now we solve the characteristic polynomial:
	
	It comes immediately that:
		
	Therefore:
	
	Now a particular solution to:
	
	is relatively trivially a solution of the type:
	
	where $B$ is of course a constant to be determined and which is equal (once injected into the differential equation):
	
	Therefore:
	
	Hence finally the general solution:
	
	Then, with the initial conditions that are a for reminder:
	
	it is very easy to find A:
	
	We also have:
	
	We are free to choose that $c^{te}=0$ which gives us:
	
	Then:
	
	becomes:
	
	Now that we have the general solution if $\varepsilon$ is small we can take the development of order 4 in Maclaurin series of the exponential (\SeeChapter{see section Sequences and Series}). Such as:
	
	Injected into $y$ this gives (you will notice that we sometimes express explicitly... the term of order 5 by anticipation...):
	
	Now that we have this development, what we want to show is that from a perturbative expansion we can find the same result in series and this without any prior knowledge on the solution.
	
	Again, the development is done in 4 steps:
	\begin{enumerate}
		\item In the first step, we assume that the solution of differential equation is an expression of the type Taylor series on  $\varepsilon$. Then we have:
		
		where $y_0,y_1,y_2$ are obviously to be determined.
		
		\item In the second step, we inject the hypothetical solution of our differential equation in itself with the initial conditions and we develop the whole.
		
		then the initial conditions:
		
	
		\item In the third step we equalize successively the terms with $0$ such as:
		
		
		\item In the fourth step we solve the differential equations listed above (if you do not see how we solve them do not hesitate to contact us!):
		
		By injecting relations in the supposed solution developed in Taylor series and injected into the differential equation:
		
		We fall back on:
		
	\end{enumerate}
	
	\begin{flushright}
	\begin{tabular}{l c}
	\circled{95} & \pbox{20cm}{\score{4}{5} \\ {\tiny 119 votes,  75.45\%}} 
	\end{tabular} 
	\end{flushright}
	
	%to make section start on odd page
	\newpage
	\thispagestyle{empty}
	\mbox{}
	\section{Sequences and Series}
	
\lettrine[lines=4]{\color{BrickRed}S}equences and series have a great importance in Applied Mathematics and that is why we devote to them a whole section. We will also see them often in various sections of the Mechanics chapters when we need to make some minor approximations (...) as well as in the sections of Economy and Quantitative Management Techniques. The reader should try to not confuse in what follows the concept of "sequences" with that of "series" which, while being similar in substance, are not always analyze mathematically in the same way.

We wanted to study in this section simple things without going to far within the topological concepts of sequences and series. However, those interested in more rigorous definitions can read the sections Fractals (see chapter of Theoretical Computing) and Topology where many concepts about series are  (supremum, infimum, subsequence, Bolzano-Weierstrass' theorem, etc.).

\subsection{Sequences}

\textbf{Definition (\#\mydef):} A "\NewTerm{sequence}\index{sequence}" of a set is a family of elements indexed by the set of natural numbers (\SeeChapter{see section Numbers}) or by a part of it. In a vulgarized way we say that a sequence is a list of objects put in order, each with a order number. We typically write a sequence as:
	
where indexing is sometimes (by tradition...) without the 0.

For some sequences, we provide the first term $u_1$ (if indexing starts with 1 instead of 0), and a formula for any term $u_{n+1}$ from the previous term $u_n$ regardless $n \geq 1$. We call such a formulation a "\NewTerm{recurring definition}\index{recurring definition}" and the sequence is defined "\NewTerm{recursively}\index{recursively defined sequence}" (and even if it is indexed from 0 instead of 1).

Before seeing some examples of sequences families that will be used in the various sections of the book (Population Dynamic, Economy, Nuclear Physics, etc.) let us see a small set of definitions as it is the tradition in mathematics...

\textbf{Definitions (\#\mydef):}
	\begin{enumerate}
		\item[D1.] Numbers (as sequence) are in "\NewTerm{arithmetic progression}\index{Sequence!arithmetic progression}" if the difference of two consecutive terms is equal to a constant $r$ named the "\NewTerm{reason}".
		
		\item[D2.] Numbers (as sequence) are in "\NewTerm{geometric progression}\index{Sequence!geometric progression}" if the ratio of two consecutive terms is equal to a constant $r$ also named the "\NewTerm{reason}".
		
		\item[D3.] Numbers (as sequence) are in "\NewTerm{harmonic progression}\index{Sequence!harmonic progression}" if the inverse of two consecutive terms are in arithmetic progression.
	\end{enumerate}

	Therefore, a number $b$ is respectively the arithmetic mean, geometric, harmonic of $a$ and $c$ if the numbers $a, b, c$ are respectively in  an arithmetic, geometric or harmonic progression.

	\begin{tcolorbox}[title=Remark,colframe=black,arc=10pt]
For the definitions of the averages listed above see the section Statistics.
	\end{tcolorbox}
	
\textbf{Definitions (\#\mydef):}
	\begin{enumerate}
		\item[D1.] A "\NewTerm{majorated sequence}\index{majorated sequence}" is a sequence such that there is a real number $M$ such that:\\ $\forall n \in \mathbb{N}, \; u_n \leq M$
		
		\item[D2.] A "\NewTerm{minorated sequence}\index{minorated sequence}" is a sequence such that there is a real number $m$ such that:\\ $\forall n \in \mathbb{N}, \; u_n \geq m$
		
		\item[D3.] A "\NewTerm{bounded sequence}\index{bounded sequence}" is a sequence that is both majorated and minorated.
		
		\item[D4.] A sequence $(u_n)$ is named  "\NewTerm{increasing sequence}\index{increasing sequence}" if $\forall n \in \mathbb{N}, \; u_{n+1}-u_n > 0$
		
		\item[D5.]  A sequence $(u_n)$ is named  "\NewTerm{decreasing sequence}\index{decreasing sequence}" if $\forall n \in \mathbb{N}, \; u_{n+1}-u_n < 0$
		\begin{tcolorbox}[title=Remark,colframe=black,arc=10pt]
	If a sequence is increasing or decreasing, we sometimes just say it is a "\NewTerm{monotonous sequence}" (without specifying if its increasing or decreasing).
		\end{tcolorbox}
		
		\item[D6.]  A sequence $(u_n)$ is named  "\NewTerm{constant sequence}\index{constant sequence}" if $\forall n \in \mathbb{N}, \; u_{n+1}=u_n$
	\end{enumerate}
	
	We will now see some practical important arithmetic and geometric sequences that will be used later in other sections of this book.
	
\subsubsection{Arithmetic Sequences}

\textbf{Definition (\#\mydef):} We say that numbers or "\NewTerm{terms}\index{term of a sequence}" are in an "\NewTerm{arithmetic sequence}" when the difference of their sequential value is equal to a constant $r$ named the "\NewTerm{reason}\index{reason of an arithmetic sequence}" of the sequence so that:
	
where $r$ is the "reason" of the progression. We then obviously have if the indexing starts from $0$:
	
	\begin{tcolorbox}[colframe=black,colback=white,sharp corners]
\textbf{{\Large \ding{45}}Examples:}\\\\
E1. The sequence:
	
where $n$ is a constant and the reason is equal $1$.\\\\
E2. The sequence:
	
is an arithmetic sequence of reason $x$.
	\end{tcolorbox}
Thus, if we write $u_n$ any term of the sequence $(u_n)$ of reason $r$, we have:
	
We have the following properties for this type of sequences:
	\begin{enumerate}
		\item[P1.] A term whose rank is the average of the ranks of the other two terms is the arithmetic mean of these two terms.
		\begin{dem}
		Consider now $(u_n)$ an arithmetic sequence of reason $r$ given by the previous development:
			
and $a,b,k \in \mathbb{N}$ such as $a+b=2k$, then we have:
			
and so:
		
with $k=\dfrac{a+b}{2}$.
		\begin{flushright}
			$\square$  Q.E.D.
		\end{flushright}
\end{dem}
	\item[P2.] For three consecutive terms $u_n,u_{n+1},u_{n+2}$ in an arithmetic sequence of reason $r$, the second term is the arithmetic mean of the other two.
		\begin{dem}
			Let us write:
				
				\begin{flushright}
					$\square$  Q.E.D.
				\end{flushright}
		\end{dem}
	\item[P3.] If $u_1,u_2,u_3,...,u_n,...$ is an arithmetic sequence of ratio $r$, then the $n$-th partial sum $S_n$ (that is to say, the sum of the first $n$ terms to the power of $1$) is given by:
		
when indexing is starts from $1$.
		\begin{dem}
			We can write the sequence:
				
		Playing with the second line, we get:
			
What can be simplified even more:
			
Considering that we will prove a little bit later that the simple following Gauss series:
			
is equal to:
			
We then have for:
			
the following relation:
			
We thus get:
			
We see with the latter relation that if $u_1=r=1$ we fall back on the simple Gauss series.

As:
			
when the indexation starts from 1 we thus get:
			
			\begin{flushright}
				$\square$  Q.E.D.
			\end{flushright}
		\end{dem}
We will see other types of summations a little bit further below during our study of series!
	\end{enumerate}
	
	\pagebreak
	\subsubsection{Harmonic Sequences}

\textbf{Definition (\#\mydef):} We say that numbers $\dfrac{1}{a}, \dfrac{1}{b}, \dfrac{1}{c},...$ generates an "harmonic progression" when their inverses are in arithmetical progression (also with a "reason" $r$. We represent this progress by:
	
We then obviously if the indexing starts from 0:
	
Moreover, we assume, in what follows, that there is no zero denominator.

By sharing this type of sequences successively in groups containing $2^n$ terms, we observe that each of them is bigger than the last of his group. For example:
	
And we can see that the sum of the terms of each group is larger than 1/2.

We can also see that each term is the harmonic mean of the previous and consecutive one:

\begin{dem}
	
Thus:
	
So finally:
		
	\begin{flushright}
		$\square$  Q.E.D.
	\end{flushright}
\end{dem}

	\subsubsection{Geometric Sequences}
\textbf{Definition (\#\mydef):} A "\NewTerm{geometric sequence}\index{geometric sequence}" is a sequence of numbers such that each of them is equal to the previous $n$ multiplied by a constant number $q$ that we also name the "\NewTerm{reason}\index{term of a geometric sequence}" of the sequence. We will denote by:
	
Thus, if we denote by $u_n$ any term of the sequence $u_n$, we have (trivial):
	
Here are some properties for such a type of sequence (without proof until now... except if some readers ask for them because most are really trivial):
	\begin{enumerate}
		\item[P1.] (trivial) The quotient of two terms of the same sequence is a power of the reason $q$ whose exponent equals the difference in rank of the two terms chosen (simple ratio of two same bases with different powers).
		\item[P2.] (trivial) If we multiply or divide term by term two geometric sequences, we get a third geometric sequence whose raison equal the product (respectively the quotient) of the reasons of the two chosen sequences (simple operation with the reasons of the two original sequences).
		\item[P3.] In a geometric sequence, a term whose rank is the average of the ranks of the other two terms is the geometric mean (see section Statistics) of these two terms (reread many times if needed...).
	\end{enumerate}
Let us prove the property P3:
\begin{dem}
	Given a geometric sequence with real positive reason $q$, we have for recall:
	
Let $a, b$ be the ranks of two terms of the geometric sequence, then we have:
	
and thus:
	
	\begin{flushright}
		$\square$  Q.E.D.
	\end{flushright}
\end{dem}

	\begin{corollary}
	We have as corollary that for three consecutive terms $n,n+1,n+2$in a geometric progression, the second term is the geometric mean of the other two.
	\end{corollary}
	\begin{dem}
		We have:
		
Thus:	
		
	\begin{flushright}
		$\square$  Q.E.D.
	\end{flushright}
	\end{dem}
However, there are some special sequences that have special properties that we find very frequently in mathematical or theoretical physics in this book. Without going into  too much detail, here's a partial list of with the important proofs that we will have to use later:

	\subsubsection{Cauchy Sequence}

It is often interesting for the mathematician, as much as for the physicist, to know the properties of a sequence with a given type of progression. The most important property is the limit to which it tends.

	\begin{tcolorbox}[title=Remark,colframe=black,arc=10pt]
The reader who is not comfortable with topology can skip the text that follows... and whoever wants to know more about Cauchy sequences may read the section Topology or also particularly the section on Fractals (\SeeChapter{chapter Theoretical Computing}).
	\end{tcolorbox}

\textbf{Definition (\#\mydef):} Let $(X, d)$ a metric space (\SeeChapter{see section Topology}), we say that the sequence:
	
converges to $x \in X$ if (by definition!):
	
In other words, more we go far in the sequence, the more points are close (in the sense of the metric $d$) to each other.

If we chose a particular metric (the Euclidean one for example) and a discrete sequence the above definition will look like this:
	
	Where the convergence point is therefore $a$ and we have:
	
	In the example of the figure below where the sequence seems to converge to $1.13$ we observe that for a given non-zero positive $\varepsilon$, there there is a particular $n$ which we denote $N$ ($n=17$ in the figure below) from which the sequence converges:
	\begin{figure}[H]
		\centering
		\includegraphics[scale=0.75]{img/analysis/cauchy_convergence.eps}
		\caption{Illustration of the principle of convergence of a sequence}
	\end{figure}

	However, the above definition of the convergence makes problem because the number $x$ should be known. In most cases of interest $x$ is unfortunately not known. To break this deadlock, Cauchy had the idea to propose the following definition:

	\textbf{Definition (\#\mydef):} We say by definition that a sequence $(x_n)_{n \in \mathbb{N}}$ of elements of $X$ is a "\NewTerm{Cauchy sequence}\index{Cauchy sequence}" if:
	
	The reader must notice that it is not sufficient for each term to become arbitrarily close to the preceding term. This is why require that $|a_{N+1} - a_{N}| < \varepsilon$ is not sufficient!

	It is almost obvious then that any convergent sequence is a Cauchy sequence (well there are some subtleties that we will not reference for now).

	\begin{tcolorbox}[title=Remark,colframe=black,arc=10pt]
This criterion therefore facilitates some proofs because it helps to show the existence of a limit without involving its value, generally unknown.
	\end{tcolorbox}
	
	\begin{theorem}
	Let us now prove that the theorem that asses that any convergent sequence is Cauchy sequence.
	\end{theorem}
	\begin{dem}
	Consider a sequence $u_n$ converging to the value $l$ (which is unknown to us) and $\varepsilon>0$ (randomly selected). Then there exists according to the definition of a convergent sequence, $n \in \mathbb{N}$ such that:
	
	The choice to write $\dfrac{\varepsilon}{2}$ is completely arbitrary but in fact we anticipate the result of the demonstration so that it is more aesthetic...
	
	Therefore for $p,q>N$ (in fact know the value of $N$ is irrelevant, since it should work for any value... well don't forget that $N$ depends on $\dfrac{\varepsilon}{2}$) we have using the triangle inequality (\SeeChapter{see section Vector Calculus}):
	
	and because $d(u_n,l)\leq\dfrac{\varepsilon}{2}$ we can write:
	
	Therefore:
	
	That may be a bit abstract so let's see an example with the harmonic sequence as an example to close the proof:
	
	First, nothing it is not vorbidden to us to take $n \geq 2$ (otherwise it will be hard to make a difference between two terms...).
	Therefore we take the Euclidean distance:
	
	First, the reader will note that in all cases since $k\leq 2n$ is between $n+1$ and $2n$. Which brings us to write:
	
	So from this inequality it comes automatically that each term of the sum on the left below will be greater than each term of the sum on the right:
	
	With (just do a particular example)
	
	Therefore:
		
	Now the idea is to see if the sum on the left is therefore greater than or equal to $\varepsilon=\dfrac{1}{2}$ and this for any $n$. Thus the suite is not convergent!
	
	Thus, the idea is that we found an $\varepsilon$ for which the Cauchy criterion is deficient.
	\begin{flushright}
		$\square$  Q.E.D.
	\end{flushright}
	\end{dem}
	So it is not because the points are always closer to each other that they converge to a given point, because this point may not exist.
	
	The best example is probably the following (it is also a little bit stupid example but...):
	\begin{tcolorbox}[colframe=black,colback=white,sharp corners]
\textbf{{\Large \ding{45}}Example:}\\\\
Let us take $X=\mathbb{Q}$ and the absolute difference as distance:
	
	Given $z$ an irrational number and $q_j \in \mathbb{Q}$ with $j \in \mathbb{N}^{*}$ such that:
	
	The idea is that greater is $j$, more the rational number $q_j$ is near the irrational $z$ and we know we can found such a sequence.
	Let us show that the $q_1,q_2$ we could be able to build form a Cauchy sequence! Indeed using triangular inequality:
	
	and therefore is a Cauchy sequence if and only if $\vert q_m-q_n \vert\leq \varepsilon$ if:
	
	We have thus found a $N$ (equal to $\dfrac{1}{2\varepsilon}$) which satisfies our definition of a Cauchy sequence. But this sequence does not converge in $\mathbb{Q}$ otherwise $z$ would be rational.
	You can check this with $\pi$ and the sequence:
	
	\end{tcolorbox}
	
	\begin{tcolorbox}[title=Remarks,colframe=black,arc=10pt]
Mathematicians use such results to define the set of irrational and also by using some additional topological concepts.
	\end{tcolorbox}
	We have just seen that a Cauchy sequence is not necessarily a convergent sequence in $X$. The inverse is however true: any convergent sequence is a Cauchy sequence!!
	
	\subsubsection{Fibonacci Sequence}
	
	If we calculate a sequence of numbers starting with 0 and 1, such that each term is equal to the sum of the two previous ones, we can form the following sequence:
	
	therefore, if we designate the different terms by:
	
	We build therefore the following sequence law:
	
	The Fibonacci sequence has many interesting strong properties, which will be developed later. However, it seems to be the first "\NewTerm{recurring sequence}\index{recurring sequence}" known in history (hence the fact that we were talking about it in this book). 
	
	The origin of this sequence seem to come from a rabbit problem asked to Fibonacci in 1202. Starting with a couple of rabbits, how much couples of rabbits will we get after a given number of months knowing that each pair produces a new pair every month (and no couples die...), which becomes productive only after two months. Therefore we have:
	\begin{itemize}
		\item Beginning: We have nothing $(0)$
		\item 1st month: We buy a couple of baby rabbits $(1)$.
		\item 2nd month: The couple of rabbits are now adults $(1)$.
		\item 3rd month: We have the couple of rabbits that make a new couple of baby rabbits. We have two couples $(2)$.
		\item 4th month: We have two couple of adults with a new couple of babies. We have three couples $(3)$.
		\item 5th month: We have three couples of adults rabbits and two new couples of baby rabbits. We have five couples $(5)$
		\item and so on...
	\end{itemize}
	
	Let us take now a "in real life" example (this is typically a biased scientific example because you will always finish to find in Nature what your are looking for to argue your theories with a least on particular example...): the heart of some flowers! The scales of a pineapple or pinecone form two families of spirals wound in opposite directions. On a pine cone, you will count 5 spiral in one direction and 8 in the other, on pineapples, 8 and 13, on sunflowers 21 and 34. Each time we get Fibonacci numbers!
	
	A famous illustration of this is to do draw the following simple figure (named"Fibonacci Spiral") which reproduces the Fibonacci numbers on a grid plan with squares and corners connect with arc circles:
	
\begin{figure}[H]
\centering
\includegraphics[scale=0.75]{img/algebra/fibonacci.jpg}
\caption{Fibonacci spiral}
\end{figure}

We also use this kind of sequence to show the usefulness of the principle of induction as presented in the section on Numbers Theory and as a simple practical case the $\mathcal{Z}$ transform (\SeeChapter{see section Functional Analysis}).

	\subsubsection{Logic Sequences/Psychologist Sequences}
	\textbf{Definition (\#\mydef):} Psychologists name "\NewTerm{logical sequences}\index{logical sequence}", sequences that they write with an idea in mind, and they call "logical" people who find their idea, although there are other possibilities mathematically speaking (but psychologist don't know anything about real logic).
	
	For example if you have to find the next number $X$ to the logic sequence:
	
	In fact, you make the difference between the last and prior-previous number then you multiply by $10$ therefore the next number is $X=31000$.
	
	From a mathematical point of view, any number is suitable to replace $X$, also exists for each value of $X$, a polynomial in $n$ that takes the values $4, 5, 10, 50, 400, 3500$ for $n = 0, 1 , 2 ..6$.
	
	For the example above we can take for example:
	
	and is such that $P(0)=4, P(1)=5, P(2)=10, ...,P(6)=0$.

	\pagebreak
	\subsection{Series}

The physicists often needs to simply and formally solve problems, to approximate some given "terms" of their equations. For this purpose, they will use the properties of some given series. Also statisticians and financial analysts often face to series they need to simplify.

\textbf{Definition (\#\mydef):} Let be given an infinite number sequence:
	
	The expression:
	
	is named a "numeric series".
	
	\textbf{Definition (\#\mydef):} The partial sum of the first $n$ terms of the series is named "partial sum" and denoted by:
	
	If the following limit denoted $S$ exists and is finite:
	
	we name it "\NewTerm{sum of the series}\index{sum of a series}" and we say that the "\NewTerm{series converges}\index{convergent series}" (it is therefore a  Cauchy series). However, if the limit does not exist, we say that the "\NewTerm{series diverges}\index{divergent series}" and has no sum (for details see further below when we will deal with some empirical convergence criterias).
	\begin{theorem}
		Also let us prove for fun (because it is almost trivial) that if $\displaystyle \sum_{k\geq 0} u_k $ is a convergent numerical series then:
		
		But the opposite is not necessarily true!! In fact remember the example during our study above of Cauchy sequences with the harmonic series $\sum_{k=1}^n \dfrac{1}{k}$ that is not convergent even if the terms tends to zero when $k \rightarrow +\infty$.
		\end{theorem}
	\begin{dem}
		We assume first that $\displaystyle \sum_{k\geq 0} u_k $ is a convergent series and denote its limit by $S$. Let:
		
		Therefore:
		
		However, if the series is really convergent:
		
		So finally:
		
		\begin{flushright}
			$\square$  Q.E.D.
	\end{flushright}
	\end{dem}
	Let's see how to calculate the partial sum of some classic series that are important in physics, statistics and finance:
	
	Arithmetic series Gauss are an expression of the sum of the $n$ first nonzero integers raised to a given power $k$ in a condensed form. The application of this condensed form of a series has an important practical use in physics, statistics and finance when we wish to simplify the expression of certain results.
	
	\subsubsection{Gauss Series}
	
	Gauss arithmetic series are an expression of the sum of the $n$ first nonzero integers raised to a given power $k$ in a condensed form $S_k$. The application of this condensed form of a series has an important practical use in physics, statistics and finance when we wish to simplify the expression of certain results.
	
	It is said that Gauss have found an attractive method in 1786 to determine the arithmetic sum of the first $n$ integers at the power of 1 when he was nine years old (...):
	
	To simplify, we find easily:
	
	for $n \geq 0$. Let us indicate that each intermediate sum of the series (1, 3, 6, 10, 15, etc.) is named "\NewTerm{triangular number}\index{triangular number}" since it is possible to represent it in the following form:

	\begin{figure}[H]
		\centering
		\includegraphics[scale=1]{img/algebra/triangular_number.jpg}
		\caption{Triangular number}
	\end{figure}

	We can continue with higher powers bit not as exercises because these relations are very useful!

	Now let us calculate the very important case that we find ourselves in a number of other sections (Economy, Quantum wave Physics, etc.) and that  the sum of the first $n$ square integer numbers (still non-zero!).

	Let us write for this:
	
	We know from Newton's binomial theorem (\SeeChapter{see Section Calculus}):
	
	so we can write and add a member to member the $n$ following equalities:
	
	And the sum can be simplified as:
	
	After some elementary algebra manipulations we get:
	
	Therefore:
	
	Finally:
	
	We continue with the sum of the first $n$ cubes (non-zero) integers. The principle is the same as before, we write:
	
	We know from Newton's binomial theorem (\SeeChapter{see Section Calculus}):
	
	We get by varying $k$ from $1$ to $n$, $n$ relations that we can add a member to member
	
	And the sum can be simplified as:
	
	Giving after development:
	
	And after an fist simplification:
	
	And a second simplification:
	
	The result result is therefore:
	
	or written differently:
	
	For sure, we can continue like this during a long time, but from a certain value of the power things get a bit more complicated (furthermore the method is a little bit boring). Thus, one of the members of the Bernoulli family (it was a family of very talented mathematicians... as you can see int the Biographies chapter) founded a general relation working for any power by defining what we name the "Bernoulli polynomial" (see further below).
	
	Let us conclude with one last case we will need during our study of Fourier series. We put:
	
	We want express this expression (series) as rational fraction. To do this, we multiply all by $x^2$. So we have two expressions:
	
	We subtract the first from the second:
	
	Finally:
	
	Most of times, to indicate that this is for odd powers we prefer to write:
	
	Similarly, for the needs of the section of Economy, we have:
	
	Therefore:
	
	Finally:
	
	Most of times, to indicate that this is for even powers we prefer to write:
	
	\paragraph{Bernoulli's Numbers and Polynomials}\mbox{}\\\\
	As we have seen above, it is possible to express the sum of the first $n$ nonzero integers to a given power (the first four have been proved previously) following the below relations where we put now $n:=n+1$ as we want now $n$ to be the number of terms we want the sum including 0 (hence the negative sign in the relations below that we did not have earlier):
	
	It is said that Jacob Bernoulli  then noticed that the polynomials $S_p$ had the form:
	
	In this expression, the numbers $(1,-1/2,1/12,0,...)$ seem not to depend on $p$. More generally, after trial and error we see that the polynomial can be written as:
	
	Giving by identification the "\NewTerm{Bernoulli numbers}\index{Bernoulli numbers}":
	
	\begin{theorem}
	Thereafter, it seems that mathematicians in their research fell randomly (???) on the fact that the Bernoulli numbers could be expressed by the series:
	
	with $\vert z\vert<2\pi$.
	\end{theorem}
	\begin{dem}
	We have seen during our study of complex numbers (\SeeChapter{see section Numbers}) that:
		
		Therefore:
		
		Let us write now:
		
		Then we must have:
		
		We see (by distributing) that:
		
		for all this to be equal to unity we must have:
		
		From the second equation we get:
		
		and from the third equation we get:
		
		etc. Continuing this way we show that:
		
		It is obvious that this method allows us to calculate by hand only the first terms of this series.
		Thus, based on:
		
		we find that the first Bernoulli numbers are:
			
		The reader will have noticed that $B_k=0$ when k is odd and different from $1$.
		\begin{flushright}
			$\square$  Q.E.D.
		\end{flushright}
	\end{dem}
	We see easily that the values of the Bernoulli numbers can not be described in a simple way. In fact, they are essentially values of the zeta Riemann function (see below) for negative integer values of the variable, and these numbers are associated with profound theoretical properties that go beyond the study of this book. Furthermore, the Bernoulli numbers also appear in the Taylor series expansion of  circular and hyperbolic trigonometric tangent functions in the Euler-Maclaurin formula (see below).
	
	With a small modification it is possible to define the "\NewTerm{Bernoulli polynomials $B_k(x)$}\index{Bernoulli polynomials}" by:
	
	with:
	
	\begin{theorem}
	Furthermore, it is normally easy to observe that:
	
	and therefore it normally easy to deduce that:
	
	\end{theorem}
	\begin{dem}
	On one side we have:
	
	and another we have:
	
	So:
	
	\begin{flushright}
		$\square$  Q.E.D.
	\end{flushright}
	\end{dem}
	And by identification of the coefficients we deduce:
	
	and for $k \geq 1$:
	
	It is then easy to deduce that the polynomials $B_k(x)$ are of degree $k$:
	
	Here is a plot of these polynomials:
	\begin{figure}[H]
		\centering
		\includegraphics{img/algebra/bernoulli_polynomials.jpg}
		\caption{Some Bernoulli polynomials (source: Wikipedia)}
	\end{figure}
	What is remarkable is that using the Bernoulli polynomials, we see that it is possible to write the $S_n$ as follows after some trials:
	
	
	Some write this relations even otherwise. Indeed, from previous relation, we can write:
	
	using:
	
	We have:
	
	So we just demonstrated:
	
	
	
	However, we can now ask ourselves what happens to the partial sum of arithmetic and geometric sequences as presented earlier in this section.
	\subsubsection{Arithmetic Series}
	We have shown above that the partial sum of a Gauss series (analogous to the sum of the terms of an arithmetic progression of reason $r = 1$) was given by:
	
	if not denote the value of the $n$-th term by $u_n$ instead of $n$, the development that we made for the series of Gauss then brings us to:
	
	and if we denote the first term $1$ of the Gauss series $u_0$ then we have:
	
	which gives us simple the partial sum of the $n$-terms of an arithmetic sequence of reason $r$.
	\begin{tcolorbox}[colframe=black,colback=white,sharp corners]
	\textbf{{\Large \ding{45}}Examples:}\\\\
	E1. A simple Gauss series with of reason 1 starting at 4, finishing at 6:
	
	E2. Now an arithmetic partial sum series of reason 2 starting at 4, finishing at 8:
	
	\end{tcolorbox}
	\begin{tcolorbox}[title=Remarks,colframe=black,arc=10pt]
	The reader will have observed that the reason $r$ does not appear in the latter relation. Indeed, by taking (always) the same development that for the Gauss series, the term $r$ is simplified and vanish.
	\end{tcolorbox}
	
	\subsubsection{Geometric Series}
	Similarly, with a geometrical sume where we have for recall:
	
	we have therefore:
	
	The last relation is written (after simplification):
	
	and if $q\neq 1$ we get:
	
	which can be written by factoring $u_0$:
	
	If $q$ is positive and less than $1$, as $n$ approaches infinity we have the result that will be used extensively in the section Economy:
	
	
	\begin{tcolorbox}[colframe=black,colback=white,sharp corners]
	\textbf{{\Large \ding{45}}Example:}\\\\
	Consider the following geometric serie of reason $q = 2$:
	
	to calculate the sum of the first four terms $\left\lbrace 1,2,3,4 \right\rbrace$, we take the power of $2$ equivalent of $n=2$ (zero is not taken into account). We then get well: $S_3=15$.
	\end{tcolorbox}
	
	\paragraph{Zeta function and Euler's identity}\mbox{}\\\\
	The German mathematician Riemann named "zeta" a function already studied before him, but that he examend when the value is a complex number (\SeeChapter{see section Numbers}). This function is represented as a series of inverse powers of integers. This is the series:

	
	\begin{tcolorbox}[title=Remarks,colframe=black,arc=10pt]
	It is traditional to note $s$ the variable upon which this series depends.
	\end{tcolorbox}
	This series has an interesting property but if we remain within the framework of positive, non-zero integer powers:
	
	when $n\longrightarrow +\infty$ then we have:
	
	If we put $x=2^s$, we obtain the sum of the inverse of powers of $2$ and similarity with $x=3^s$ such that:
	
	If we do the product of these two expressions, we obtain the sum of the powers of all fractions whose denominator is a product of $2$ and $3$:
	
	If we take all primes left, we'll get on the right all integers, since every integer is the product of prime numbers according to the fundamental theorem of arithmetic (\SeeChapter{see section Number Theory}), and this is Euler fundamental identity: what we now name "\NewTerm{Riemann zeta function}\index{Riemann zeta function}" is both a finished product and the sum of inverse powers of all integers:
	
	In condensed notation, "\NewTerm{Euler's identity}\index{Euler's identity}" is given by:
	
	where $p$ are the prime numbers.
	
	We now recommend most readers to skip what follows on the Riemann zeta function and return back here once the Fourier series presented later in this section mastered and understood...
	
	We assume in what follows that the Fourier series are known and mastered and that the Parseval equality was studied (since it is also proved further below). We will seek to determine the Riemann zeta function for two values ($s$ respectively having values $2$ and $4$) that will be useful in the valuation of integrals in certain section of the chapter on Mechanics.
	
	To determine the value of $\zeta (2)$, we will express the function:
	
	in Fourier series form (see a little further below in this section). During our study of Fourier series we will see that there are two traditional ways to define a Fourier series and we have done here the choice of the definition of the most commonly used among physicists and engineers:
	
	As we prove it in our study of Fourier series, Fourier coefficients $a_0,a_n,b_n$ are obtained by solving:
	
	and using the integration by parts (\SeeChapter{see section Differential and Integral Calculus}) we have:
	
	It comes then:
	
	But the Parseval theorem that we will prove in our study of Fourier series a little bit further below gives us too (depending on the choice of the definition of the Fourier series and associated coefficients, the Parseval theorem is expressed a little bit differently!):
	
	Therefore we get immediately:
	
	But we will also see during our proof of Parseval theorem that:
	
	Therefore it comes in our case:
	
	Therefore:
	
	and finally:
	
	
	To determine the value of $\zeta(4)$, we will do the same, but with the function:
	
	in the form of Fourier series:
	
	For this purpose, we will calculate Fourier coefficients using the integration by parts (\SeeChapter{see section Differential and Integral Calculus}):
	Then we have:
	
	Therefore we have:
	
	But the Parseval theorem that so we will prove below gives us also:
	
	It then comes immediately:
	
	But we will see also see later below during our Parseval theorem proof that:
	
	Then it comes in our case:
	
	Therefore:
	
	That is to say:
	
	Finally:
	
	
	\subsubsection{Telescoping Series}
	A "\NewTerm{telescoping series}\index{telescoping series}" is a series in which most of terms cancel in each of the partial sums, leaving only some of the first terms and some of the last terms:
	
	For example, the series:
	
	simplifies as:
	
	We will encounter such as series for business purposes (management) in our study of Queuing Theory in the section of Quantitative Management!!!
	
	\subsubsection{Grandi's Series}
	The "\NewTerm{Grandi's series}\index{Grandi's series}" (after Italian mathematician, philosopher, and priest Guido Grandi, who gave a memorable treatment of the series in 1703) is defined as the following arithmetic series:
	
	It is a very famous series in mathematics and physics because:
	\begin{itemize}
		\item It highlights in a very simple way the fact (see below) that it is dangerous to manipulate infinite series
		
		\item Its result seems completes non-intuitive but in fact it opens the door to a more general definition of what is a "sum"
		
		\item It is a beautiful example of a series that seems useless and purely mathematics but that has in fact important application in quantum physics (Casimir Effect as seen in the section of Corpuscular Quantum Physics) and String Theory (number of dimensions as seen in the section of String Theory).
	\end{itemize}
	and this is why we dedicate to it a special subsection in this book!
	
	It seem quite obvious at a first glance that it is a divergent series, meaning that it lacks a sum in the usual sense (the sequence of partial sums of Grandi's series clearly does not approach any number). But the other hand, its Cesàro sum is $1/2$!!?? So what the hell is a Cesàro sum.
	
	\textbf{Definition (\#\mydef):} In mathematical analysis a "\NewTerm{Cesàro sum}\index{Cesàro sum} assigns values to some infinite sums that are not convergent in the usual sense. The Cesàro sum is defined as the limit of the arithmetic mean of the partial sums of the series.
	
	Let $\{a_n\}$ be a sequence, and let:
	
	be the $k$th partial sum of the series:
	
	The series $\sum _{n=1}^{\infty }a_{n}$ is say to be "Cesàro summable", with Cesàro sum $S\in\mathbb{R}$, if the average value of its partial sums $s_k$ tends to $S$:
	
	In other words, the Cesàro sum of an infinite series is the limit of the arithmetic mean (average) of the first $n$ partial sums of the series, as $n$ goes to infinity. If a series is convergent, then it is Cesàro summable and its Cesàro sum is the usual sum. For any convergent sequence, the corresponding series is Cesàro summable and the limit of the sequence coincides with the Cesàro sum.
	
	One obvious method to attack the Grandi's series:
	
	is to treat it like a telescoping series and perform the subtractions in place:
	
	On the other hand, a similar bracketing procedure leads to the apparently contradictory result:
	
	Thus, by applying parentheses to Grandi's series in different ways, one can obtain either $0$ or $1$ as a "value". It can be shown that it is not valid to perform many seemingly innocuous operations on a series, such as reordering individual terms, unless the series is absolutely convergent. Otherwise these operations can alter the result of summation.

	Treating Grandi's series as a divergent geometric series we may use the same algebraic methods that evaluate convergent geometric series to obtain a third value:
	
	so:
	
	Therefore:
	
	Finally:
	
	The same conclusion results from calculating $-S$, subtracting the result from $S$, and solving $2S = 1$.

	The above manipulations do not consider what the sum of a series actually means. Still, to the extent that it is important to be able to bracket series at will, and that it is more important to be able to perform arithmetic with them, one can arrive at two conclusions:
	\begin{itemize}
		\item The series $1-1 + 1-1 + \ldots$ has no sum

		\item ...but its sum should be $1/2$ (see further below)
	\end{itemize}
	In fact, both of these statements can be made precise and formally proven, but only using well-defined mathematical concepts that arose in the 19th century. After the late 17th-century introduction of calculus in Europe, but before the advent of modern rigor, the tension between these answers fueled what has been characterized as an "endless" and "violent" dispute between mathematicians. The funnies is that the violent discussions still continue today... a YouTube video on this subject have more than $5,000$ comments... and blog post more than $200$ comments and a forum thread more than $600$... So this is quite a hot topic...
	
	Let us also recall that at the beginning of our study of Geometric series we have proved that:
	
	Therefore if $u_0=1$ this reduce to:
		
	where as $n$ goes to infinity, the absolute value of $|q|$ must be less than one for the series to converge!

	Now notice that if $q=-1$ we fall back on Grandi's series and therefore that latter is a special case of the geometric series $1+q^1+q^2+q^3+\ldots$ and then we would perhaps write a bit too quick:
	
	But as we have just mention it, we are not authorized to write the latter fraction if $q=\pm 1$ otherwise the series diverge. 
	\begin{tcolorbox}[title=Remark,colframe=black,arc=10pt]
	There have been very interesting studies about the reaction of high-school level students to Grandi's series presentation. The reactions and analysis are very interesting and I can personally only recommend every teacher to introduce this series in classes but without giving the result in a first time!
	\end{tcolorbox}
	But there is a "one more thing"... We will now calculate a sum to think it really gives infinite:

	
	To do this, let's do another trick of mathematical magician:
	
	Therefore:
	
	So we can compute:
	
	Our first concrete result, squared, can be rewritten as follows:

	
	Or well:
	
	Explicitly:
   \begin{eqnarray*}
	   (-1 + 1-1 + 1-1 + \ldots)\\
	   \underline{\times (-1 + 1-1 + 1-1 + \ldots)} \\
	    =1-1 + 1-1 + 1-1 + 1-1 + \cdots \\ -1 + 1-1 + 1-1 + 1-1 + \cdots\\ + 1-1 + 1-1 + 1-1 + \cdots\\ \cdots
	\end{eqnarray*}
	Summing each column we see that we fall back on:
   
	Therefore:
	
   But as $S = 1/2$, then:
   
   Therefore:
   
   That is (to freak out a last time), we have shown that
   
	Astonishing! However, all this stuff is not new. It was known for many people, and it was Srinivasa Ramanujan and later Godfrey Harold Hardy in a book titled \textit{Divergent Series} where you can find fine theorems about this crazy subject.
	
	\pagebreak
	\subsubsection{Taylor and Maclaurin Series}
	Taylor and Maclaurin series provide a convenient and powerful tool to simplify theoretical models and computer calculations (fluid modeling or fields in space). They are used heavily in all fields of physics but they are also found in the industry including engineering (design of experiments, numerical methods, quality management), statistics (integral approximations), finance (stochastic processes ), complex analysis... We strongly advise the reader to read carefully the developments that follow.
	
	Given a polynomial (with one variable/univariate):
	
	We trivially have for this latter:
	
	Given now the derivative of the polynomial $P (x)$:
	
	Therefore:
	
	and so on with $P''(x), P'''(x), ...$ such that:
	
	Then:
	
	Therefore:
	
	relation that we name "\NewTerm{limited Maclaurin series}\index{limited Maclaurin series}" or simply "\NewTerm{Maclaurin series}\index{Maclaurin series}" of order $k + 1$.
	\begin{tcolorbox}[title=Remark,colframe=black,arc=10pt]
	In practice, as we will see in many other sections of this book, we often use limited developments of order $1$ (also named "\NewTerm{affine approximations}\index{affine approximation}", or "\NewTerm{affine tangent approximations}\index{affine tangent approximations}"), which can facilitate the calculations, when we do not expect too much precision in the final result.
	\end{tcolorbox}
	Now by applying the same reasoning but by centering the value of the polynomial on $x=x_0$, we have:
	
	and so the previous development becomes more general:
	
	which is no other than the general expression of a polynomial expression in a form named "\NewTerm{limited Taylor series}\index{limited Taylor series}" of order $k + 1$. This function can be assimilate to a polynomial as $n$ is finite. But if $n$ is infinite, as we shall see later, this series converges to the function we are seeking the representation in the form of a sum of terms.
	
	Thus, some functions $f (x)$ of class $\mathcal{C}^n$ that can be approximated by a polynomial $P (x)$ (a sum of powers in other words...) centered on the value $x_0$ can be expressed as:
	
	Relation often referred to as "\NewTerm{Taylor's theorem}\index{Taylor's theorem}".
	
	But this last relation is not correct for all functions that can not be expressed as a polynomial. Therefore we say that the series is not convergent for them. We will see an example later below.
	
	The latter relations is sometimes also written ... more conventionally:
	
	In finance (and not only!), we will often use the following rearrangement:
	
	Let us return briefly to the approximation of $f (x)$ near and centered in $x_0$:
	
	Some people do not like using this formulation because we have the risk to forget that the approximation for a few terms is only good as long as we are not too far from $x_0$ with $x$. This is why it often happens that we write:
	
	with $x_0$ fixed and a $h$ variable but small (!) and so it then comes a current form of notation of Taylor series:
	
	with $x_0$ fixed and $h$ variable but small and therefore it comes a common other notation of Taylor series (!):
	
	Let's see an application example with Maclaurin series (with $x_0$ being zero) of the function $sin (x)$ and Maple 4.00b:
	
	\texttt{>p[n](x) = sum((D@@i)(f)(a)/i!*(x-a)\string^i,i=0..n);\\
	>p11:= taylor(sin(x),x=0,12);\\
	>p11:= convert(p11,polynom);\\
	>with(plots):\\
	>tays:= plots[display](sinplot):\\
	for i from 1 by 2 to 11 do\\
	tpl:= convert(taylor(sin(x), x=0,i),polynom):\\
	tays:= tays,plots[display]([sinplot,plot(tpl,x=-Pi..2*Pi,y=-2..2,\\
	color=black,title=convert(tpl,string))]) od: \\
	>plots[display]([tays],view=[-Pi..2*Pi,-2..2]);}

	\begin{figure}[H]
		\centering
		\includegraphics{img/algebra/maclaurin_sinus_serie.jpg}
		\caption{Approximation of the sine function by a Maclaurin development Maple 4.00b}
	\end{figure}
	We see well in this example that the Maclaurin series only allows to approach a function at a point with a limited number of points. But more terms we take (put $100$ terms in the Maple code above) more the validity is big on the whole domain of definition of the function. In fact it is possible to prove that the function $sin (x)$ is exactly expressible in Maclaurin series when the number of terms is infinite. We say then that its "rest" is zero.
	
	But this is not true for all functions! For example the function:
	
	
	\texttt{>p[n](x) = sum((D@@i)(f)(a)/i!*(x-a)\string^i,i=0..n); \\
	>p10:= taylor(1/(1-x\string^2),x=0,10);\\
	>p10:= convert(p10,polynom);\\
	>with(plots):\\
	>tays:= plots[display](xplot):\\
	for i from 1 by 2 to 10 do\\
	tpl:= convert(taylor(1/(1-x\string^2), x=0,i),polynom):\\
	tays:= tays,plots[display]([xplot,plot(tpl,x=-2..2,y=-2..2,
	color=black,title=convert(tpl,string))]) od: \\
	>plots[display]([tays],view=[-2..2,-2..2]);}
	\begin{figure}[H]
		\centering
		\includegraphics{img/algebra/maclaurin_nonconvergent_serie.jpg}
		\caption{Example of non-convergent Maclaurin serie Maple 4.00b}
	\end{figure}
	We see above that regardless of the number of terms that we take the Maclaurin series converges only in one area of definition between $] -1,1 [$. This interval is named the "\NewTerm{radius of convergence}\index{radius of convergence}" and it determination (the singularity) is crucial in many areas of engineering, physics and analysis. We will return in mure more detail on this example in the section of Complex Analysis.
	
	But we can shift the Maclaurin series of the previous function to approximate the function with a Taylor series in other non-singular point such as in $x_0$ having for value $2$:
	
	\texttt{>p[n](x) = sum((D@@i)(f)(a)/i!*(x-a)\string^i,i=0..n);\\
	>p10:= taylor(1/(1-x\string^2),x=2,10);\\
	>p10:= convert(p10,polynom);\\
	>with(plots):\\
	>tays:= plots[display](xplot):\\
	for i from 1 by 2 to 10 do\\
	tpl:= convert(taylor(1/(1-x\string^2), x=2,i),polynom):\\
	tays:= tays,plots[display]([xplot,plot(tpl,x=0..5,y=-2..2,\\
	color=black,title=convert(tpl,string))]) od: \\
	>plots[display]([tays],view=[-0..5,-2..2]);}
	
	\begin{figure}[H]
		\centering
		\includegraphics{img/algebra/maclaurin_nonconvergent_serie_shifted.jpg}
		\caption{Shift possibility of Maclaurin serie in Maple 4.00b}
	\end{figure}
	
	We will study a generalization to the complex plane of Taylor series in the section of Complex Analysis to get a veeeeery powerful result for physicists to calculate complicated curvilinear integrals.
	
	\pagebreak
	\paragraph{Usual Maclaurin developments}\mbox{}\\\\
	We will prove here the most frequent Maclaurin developments (about ten) to the second order that we can meet in theoretical and mathematical physics (in fact we heve developed here only use almost everywhere in the book). The list is not exhaustive for the time being but as the proof below are generalized, they can be applied to many other cases (that we will apply/meet throughout this book).
	
	\begin{tcolorbox}[title=Remarks,colframe=black,arc=10pt]
	The Taylor expansions (that is to say elsewhere than on zero) are very rare (there are one or two in this entire book but they are detailed in their respective sections), we will omit them.
	\end{tcolorbox}
	
	\begin{enumerate}
		\item Taylor-Maclaurin development of $f(x)=e^x$:
		
		First remember that we have proved in the section of Differential and Integral Calculus that:
		
		Therefore we have:
		
		More generally:
		
		And therefore we have the famous result for $x=1$:
		
		that is sometimes named the "\NewTerm{exponential sequence}\index{exponential sequence}".
		
		\item  Taylor-Maclaurin development of $f(x)=\sin(x)$:
		
		First remember that we have proved in the section of Differential and Integral Calculus that:
		
		Therefore we have:
		
		
		\item  Taylor-Maclaurin development of $f(x)=\cos(x)$:
		
		First remember that we have proved in the section of Differential and Integral Calculus that:
		
		Therefore we have:
		
		\item  Taylor-Maclaurin development of $f(x)=\tan(x)$:
		
		First remember that we have proved in the section of Differential and Integral Calculus that:
		
		Therefore we have:
		
		
		\item  Taylor-Maclaurin development of $f(x)=\arctan(x)$:
		
		First remember that we have proved in the section of Differential and Integral Calculus that:
		
		Therefore we have:
		
		
		\item  Taylor-Maclaurin development of $f(x)=\displaystyle\frac{1}{1+x}$:
		
		First remember that we have proved in the section of Differential and Integral Calculus that:
		
		Therefore we have:
		
		It then follows immediately another Taylor series we will also meet again many  number of times:
		
		
		\item  Taylor-Maclaurin development of $f(x)=\sqrt{1+x}$:
		
		First remember that we have proved in the section of Differential and Integral Calculus that:
		
		Therefore we have:
		
		It then also follows immediately another Taylor series we will also meet again many  number of times:
		
		
		\item  Taylor-Maclaurin development of $f(x)=\ln(1+x)$:
		
		First remember that we have proved in the section of Differential and Integral Calculus that:
		
		Therefore we have:
		
		
		\item  Now consider the important case for the Langevin model of paramagnetism that is approximated Taylor expansion of the hyperbolic cotangent function (\SeeChapter{see section Trigonometry}), that is for refresh defined by the relation:
		
		For this, we will use the Landau notation, with expressions like $\mathcal{O}(x^n)$ remembering that we proved a little before above:
		
		when $x \rightarrow 0$.
		For the hyperbolic cotangent we have then:
		
		Now we must be remember as we have just proved a little earlier that:
		
		for $\vert x \vert < 1$. Therefore:
		
		and finally replacing this in the previous expression we find:
		
		
		\item  Another famous Maclaurin series used thousand of times in the world for business application is the computation of the numerical values of the Normal distribution:
		
		So first to simplify this integral, we typically let:
		
		the we know already (\SeeChapter{see section Statistics}) as being the $z$ score of a data value. With this simplification, the integral above becomes:
		
		The Maclaurin series for $e^{-x^2/2}$ is given by:
		
		Therefore:
		
		Therefore:
		
		Is is obvious that the constant will eliminate itself. Therefore!
		
		and in the common case in business where $a=0$ we get (with two terms only):
		
	\end{enumerate}
	
	\pagebreak
	\paragraph{Taylor series of bivariate functions (multivariate Taylor series)}\mbox{}\\\\
	We will see now how to approach a function $f (x, y)$ of two real variables by a sum of powers (Taylor series). This type of approximation is widely used in many fields of engineering (see sections of Industrial Engineering and Numerical Methods in this book).
	
	We are looking for an approximation of $f (x, y)$ at point $f(x_0+h,y_0+h)$. For this, let us write (a priori nothing prohibits us from doing so) that:
	
	Then we have:
	
	The value of (the trick is here!):
	
	can be approximated using a Taylor series around the value $0$ such that:
	
	But we have:
	
	and:
	
	According to Schwarz's theorem (\SeeChapter{see section Differential and Integral Calculus}):
	
	Then we have:
	
	and we show by induction that:
	
	Therefore we finally get:
	
	or in another equivalent simplified form:
	\begin{empheq}[box=\fbox]{align}
		\begin{split}
  		f(x_0+h,y_0+k)&=f(x_0,y_0)+\dfrac{1}{1!}\dfrac{\partial f}{\partial x}(x_0,y_0)h+\dfrac{1}{1!}\dfrac{\partial f}{\partial y}(x_0,y_0)k\\
		&+\dfrac{1}{2!}\left[\dfrac{\partial^2 f}{\partial x^2}(x_0,y_0)h^2+2\dfrac{\partial^2 f}{\partial x\partial y}(x_0,y_0)hk+\dfrac{\partial^2 f}{\partial y^2}(x_0,y_0)k^2\right]
		\end{split}
	\end{empheq}
	Or if we define a matrix H named "\NewTerm{Hessian matrix}\index{Hessian matrix}" given by:
	
	we can also write:
	
	In Maple 4.00b we use the following command to make a development of order $3$ around $0$:
	
	\texttt{>readlib(mtaylor):\\
	>mtaylor(f(x,y), [x,y], 3);}
	
	\paragraph{Quadratic Form}\mbox{}\\\\
	Now we will need for the section of Theoretical Computing to state an important property (which would have also its place only in the section of Differential and Integral Calculus):
	
	Let $f$ be a function defined and derived over an interval $I$ and given $a$ an element of $I$. If $f$ is such that $f'(a)=0$ then we say it has a local extremum on $a$.
	
	\begin{tcolorbox}[title=Remarks,colframe=black,arc=10pt]
	The reciprocal is false, the function $x^3$ is such an example. Its derivative is zero at $0$ but there is no local extreme at this point. So be careful!
	\end{tcolorbox}
	
	However, let $f$ be a function defined and derived over an interval $I$ and given $a$ an element of $I$. If $f$ is such that $f'(a)=0$ and if $f'$ changes sign in $a$ then $f$ has a local extremum at $a$.
	
	To return now to our bivariate development Taylor, we know that if $(x_0,y_0)$ is a local extremum of $f$ then we have in a first time (\SeeChapter{see section Differential and Integral Calculus}):
	
	However we have seen that this condition is not sufficient to ensure that $(x_0,y_0)$ is a local extremum.
	
	Reconsider the Taylor expansion of $f$ above taking into account the above condition. Development simplified then to:
	
	Then we know that by definition so that the $(x_0,y_0)$ is a local minimum (respectively a local maximum) it is sufficient that the expression in brackets is positive (respectively negative). Since the second derivatives of $f$ are continuous, it is sufficient that the expression:
	
	to be positive (negative resp.) regardless of $h$ or $k$ and it is zero only if $h=k=0$. Then we say that $q$ is a "\NewTerm{positive definite quadratic form (resp. negative definite)}\index{positive definite quadratic form}".
	
	To simplify writing and to comply with traditions we put now:
	
	Then we can rewrite $q$ as follows:
	
	where $H$ remains the Hessian matrix of $f$ evaluated on $(x_0,y_0)$.
	
	So we see that $q$ is positive definite (local minimum) and if $a>0$ and $\det(H)>0$, negative definite (local maximum) if $a<0$ and $\det(H)>0$.
	
	Returning to the partial derivatives these conditions are described as follows:
	\begin{itemize}
		\item Positive definite (local minimum) if:
		
		
		\item Negative definite (local maximum) if:
		
	\end{itemize}
	Finally we see that the sign of the determinant of Hessian matrix and that of $\dfrac{\partial^2 f}{\partial x^2}(x_0,y_0)$ allow us to obtain a sufficient condition to determine if we are in the presence of a local extremum.
	
	\begin{tcolorbox}[colframe=black,colback=white,sharp corners]
	\textbf{{\Large \ding{45}}Example:}\\\\
	Let us see an example with the famous humpback whale:\\
	
	\texttt{>with(plots): with(plottools):\\
	>readlib(mtaylor):\\
	>fct:=x\string^2*(4-2.1*x\string^2+1/3*x\string^4)+x*y+y\string^2*(-4+4*y\string^2);\\
	>poly2 :=mtaylor(fct,[x=1,y=1],6);\\
	>\#Convert all the coefficients to floating point numbers\\
	>poly2n := map(evalf,poly2):\\
	>gr1:= plot3d(poly2n,x=-2..2,y=-1..1,color=red):\\
	>gr2:= plot3d(fct,x=-2..2,y=-1..1,color=blue):\\
	>display3d({gr1,gr2},view=-3..8,axes=framed);
	}
	\begin{figure}[H]
		\centering
		\includegraphics{img/algebra/taylor_multivariate.jpg}
		\caption{Bivariate Taylor example with Maple 4.00b}
	\end{figure}
	\end{tcolorbox}
	
	\paragraph{Lagrange Remainder}\mbox{}\\\\
	There may be an interest in certain numerical applications (\SeeChapter{see section Theoretical Computing}) to know the approximation error of the polynomial $P_n(x)$ in relation to the function $f(x), \forall x$.
	
	Let us define for this purpose a "remainder", such that:
	
	The function $R_n(x)$ is named "\NewTerm{Lagrange rest}\index{Lagrange rest}" or "\NewTerm{Lagrange remainder}\index{Lagrange remainder}" or "\NewTerm{Lagrange error}\index{Lagrange error}".
	
	\begin{dem}
	Given a function $g(t)$ defined by the difference of a function $f(x)$ assumed to be known and a Taylor approximation of the same function:
	
	with, of course:
	
	We see that $g (t)$ vanishes as expected for value $t=x$.
	
	Now let us derive $g(t)$ with respect to $t$, we find:
	
	After simplification:
	
	According to Rolle's theorem (\SeeChapter{see section Differential and Integral Calculus}), there exist a value $t=z$ for which the derivative $g'(t)$ is zero. So:
	
	We can simplify the equation by $(x-z)^n$:
	
	which can also be written as:
	
	so we find for the maximum of $R_n$:
	
	\begin{flushright}
		$\square$  Q.E.D.
	\end{flushright}
	\end{dem}
	We see that as the polynomial $P_n(x)$ is of high degree, the more it approximates the function $f (x)$ with accuracy. What will happen when $n\rightarrow +\infty$?:
	
	Suppose that $f (x)$ has derivatives of all orders (what we denote for reminder $\mathcal{C}^n$) for all values of any interval containing $x_0$ and let the rest of Lagrange $R_n$ of f (x) of $f(x)$ on $x_0$. If, for any $x$ in the range:
	
	then $f (x)$ is exactly represented by P $(x)$ on the interval.
	\begin{dem}
	The proof simply stems from the expression of $P_n(x)$ when $n\rightarrow +\infty$.
	
	Indeed, if we take an infinity of terms for $P_n(x)$, the correspondence with the approximated function is perfect and so the rest is zero.
	\begin{flushright}
		$\square$  Q.E.D.
	\end{flushright}
	\end{dem}
	The polynomial:
	
	is named "\NewTerm{Taylor polynomial}\index{Taylor polynomial}" or "\NewTerm{Taylor series}\index{Taylor series}". If $x_0=0$, it is named "\NewTerm{Maclaurin polynomial}\index{Maclaurin polynomial}" or "\NewTerm{Maclaurin series}\index{Maclaurin series}".
	
	\paragraph{Taylor Series with Integral Remainder}\mbox{}\\\\
	We'll see if a theorem that will be useful in the section of Statistics to link the Poisson and Chi-2 laws and that is used in statistical software for Poisson test of rare events (that is the only business practical application that is known to us at this day).
	\begin{tcolorbox}[title=Remark,colframe=black,arc=10pt]
	If anyone has a more educational proof whose beginning is a little less "formula fell from the sky", we are takers!
	\end{tcolorbox}
	\begin{theorem}
	Let $f(x)$ be $n + 1$ times differentiable on the interval $[a, b]$. Then we have:
	
	where it is important (for the good understanding of what we will do in the section of Statistics) that the reader notices in the development that when the derivative stops at the $n$-th term in the series, the integral (the remainder) has a factor of $1 / n !$, a power $n$ and a derivative of order $n + 1$. So verbatim, as we shall prove it below, if we stop the development of the terms to $n-1$, the integral (the remainder) will have a factor of $1 / (n-1) !$, a power $n-1$ and a derivative of order $n$-th.
	\end{theorem}
	\begin{dem}
	The proof is mady by induction. We first consider the formula fallen from the sky:
	
	We show that it is correct for $k = 0$, then we do an induction on $k$ for $k\in \mathbb{N}$.
	
	For $k = 0$, we have the well-known relation (\SeeChapter{see section Differential and Integral Calculus}):
	
	Suppose the property true for $k<n$:
	
	We integrate by parts (\SeeChapter{see section Differential and Integral Calculus}) the term:
	
	Then we have:
	
	Therefore:
	
	\begin{flushright}
		$\square$  Q.E.D.
	\end{flushright}
	\end{dem}
	
	\subsubsection{Fourier Series (trigonometric series)}
	We name by definition "\NewTerm{trigonometric series}\index{trigonometric series}" a series of the form:
	
	or in a more compact form:
	
	The constants $a_n,b_n$ with $n\in \mathbb{N}^{*}$ are the coefficients of the trigonometric series usually named "\NewTerm{Fourier coefficients}\index{Fourier coefficients}".
	
	\begin{tcolorbox}[title=Remark,colframe=black,arc=10pt]
	We have already mentioned this type of series in our study of the types of existing polynomials since Fourier series are in fact only trigonometric polynomials (\SeeChapter{see section Calculus}). Furthermore, we saw as example in the section of Functional Analysis during our study of scalar functional product that the sine and cosine functions were the bases of a vector space!
	\end{tcolorbox}
	If the series converges, its sum is a periodic function $f (x)$ of period $T=2\pi$, since $\sin (nx)$ and $\cos (nx)$ are periodic functions of period $2\pi$. So that:
	
	Let us now state the following problem: We give ourselves a known periodic function $f(x)$, piece-wise continuous of period $2\pi$. We ask ourselves if there is a trigonometric series converging to $f (x)$ under some conditions that must be satisfied on this series.
	
	Suppose now that the function $f (x)$, periodic and of period $2\pi$, can be effectively represented by a trigonometric series converging to $f (x)$ in the interval $[0, T]$, that is to say it the sum of this series:
	
	Suppose that the integral of the function of the left member of this equality is equal to the sum of the integral of all the terms of the above series. This will occur, for example, if we assume that the proposed trigonometric series converges absolutely, that is to say, the numerical series converges (by the property that the trigonometric functions are bounded):
	
	The serie:
	
	is then majorable and can be integrated term by term from $0$ to $T$ (where $T=2\pi$) which allows us to determine the different Fourier coefficients. But before we start let us present the following integrals that will be very useful later:
	
	\begin{center}
	\begin{tabular}{ccc}
	$\text{with }n,k\in \mathbb{N}\text{ and }n\ne k$
	&$\qquad$&
	$\text{with }n,k\in \mathbb{N}\text{ and }n = k$
	\end{tabular}
	\end{center}
	Before continuing, let us prove the value taken by these six integrals (following the request of readers). But first, remember that as $n,k \in \mathbb{N}$ then:
	
	
	\begin{enumerate}
		\item We proceed using the remarkable trigonometric relations (\SeeChapter{see section Trigonometry}) and the primitive of elementary trigonometric functions (\SeeChapter{see section Differential and Integral Calculus}):
		
		because as we have seen it in the section Trigonometry $\sin(k\pi)=0,k\in\mathbb{Z}$ and as $T=2\pi$ the two previous differences have all terms equal to zero such that at the end:
		
		
		\item For the second integral, we proceed using the same techniques and the same properties of trigonometric functions:
		
		
		\item And we continue like this also for the third one, according to the same properties:
		
		
		\item Once gain using the same methods (this becomes routine ...) first for $k\neq 0$:
			
			and for $k=0$ it comes immediately:
			
			
			\item Again ... (soon finish...) first for $k\neq 0$:
			
			and for $k=0$ it comes immediately:
			
			
			\item And finally the last (...):
				
		\end{enumerate}
		This small work done let us now come back on our topic... To determine the coefficients $a_n$ both members of equality:
	
	by $\cos(kx)$:
	
	The series of the second member of equality is majorable, since its terms do not exceed in absolute to the terms of the positive convergent series. So we can integrate term by term on every bounded segment $0$ to $T$:
	
	We have proved above that whatever the integer values that take $k$ or $n$ the second term in the parenthesis is always zero. It then remains only:
	
	But we have proved above that the integral on the right is always zero if $n$ and $k$ are different. This leaves only the case where $n$ and $k$ are equal. Meaning:
	
	In this situation, we first the special case where $k$ is zero. In that case:
	
	Therefore:
	
	It is obvious that the coefficient $a_0$ represents the average of the signal or of its DC component, if it exists.
	
	In the case where $k$ it is not zero, we have:
	
	Therefore:
	
	To determine the coefficients $b_n$ we proceed the same way but this time multiplying both members of equality by $\sin(kx)$:
	
	The series of the second member of equality is majorable because its terms are not higher in absolute values to the terms in the convergent positive series. So we can integrate term by term on every bounded segment from $0$ to $T$:
	
	We have shown proved before that whatever are the integer values that taken by $k$ or $n$ the first term of the parenthesis is always zero. It remains then only:
	
	
	But we have proved before that the integral on the right is always zero if $n$ and $k$ are different. This leaves only the case where $n$ and $k$ are equal. Meaning:
	
	In this situation, we first have the special case where $k$ is zero. But we see now that we have a zero indeterminacy. It is better to consider the general case from which have:
	
	Hence we easily derive that:
	
	Therefore, for the situation where $k$ is zero the coefficient is therefore equal to zero!
	
	So finally the Fourier coefficients are determined by the integrals:
	
	But as it's annoying to have three results for the coefficients we'll play a little with the definition of the Fourier series.
	
	Indeed by summing from $1$ to $+\infty$, rather than $0$ to $+\infty$, we have:
	
	This then allows us only to have to remember ($a_0$ therefore included!):
	
	Physicists have for habit to write the last two relations as follows:
	
	The possible decomposition of any periodic piecewise continuous function approximated by an infinite sum of trigonometric functions (sine or cosine) consisting of a basic function and its harmonics is named "\NewTerm{Fourier theorem}\index{Fourier theorem}" or "\NewTerm{Fourier-Dirichlet theorem}\index{Fourier-Dirichlet theorem}".
	\begin{figure}[H]
		\centering
		\includegraphics{img/algebra/fourier_series_examples.jpg}
		\caption{Examples of some Fourier series (source: Mathworld)}
	\end{figure}
	It can also happen that sometimes we know the Fourier series and we are looking for the original function $f(x)$. As a companion example consider that we want to calculate:
	
	So this is like searching the original $f(x)$ of the above Fourier series.

	It follows therefore that $a_0=0$ and $b_n=0$ and:
	
	But as far as we know there is no easy way to extract $f(x)$ that seems accurate! So using hyperbolic trigonometry (\SeeChapter{see section Trigonometry}), we write:
	
	Now these power series may be identifies as Maclaurin expansions of $-\ln(1-z)$ (see proof above) with $z=e^{\mathrm{i}x}$ for the first terme and $z=e^{-\mathrm{i}x}$ for the second term.

	Therefore:
	
	
	The Fourier series allows implicitly to represent all the frequencies in a periodical signal whose function is known mathematically (closed form). We can wonder why talk about Fourier series when, in practice, we do not really know the mathematical representation of a signal? This will bring us to a better understanding of the concept of the Fourier transform in discrete-time that we will see a little further, which does not need a mathematical representation of a continuous and periodic signal.
	
	We note also that if $f (x)$, that is to say the periodic function of which we seek expression in trigonometric Fourier series, is even then the series will also be even and thus contain only cosine terms (the cosine function being an even function) implying that $b_n=0$ and otherwise in the case of an odd function $a_n=0$ (the sine being for reminder an odd function)!
	
	It should be noted, and this is important for what will follow, that as we have seen in the section Calculus during our study of trigonometric polynomials, Fourier series could be written in the following complex form (by changing some notations and passing the sum to infinity):
	
	and we have seen that (always in the section Calculus) that:
	
	Therefore:
	
	This gives us:
	
	Therefore:
	
	\begin{tcolorbox}[colframe=black,colback=white,sharp corners]
	\textbf{{\Large \ding{45}}Example:}\\\\
	Upon decomposition of a continuous signal, we say (improperly at our point of view) that the coefficients $a_n,b_n$ are each (implicitly) a separate frequency associated with an amplitude that we visualize on a graph by vertical lines. This graph shows the "\NewTerm{frequency spectrum}\index{frequency spectrum}" of the decomposed signal. We can also add another representation which is named "\NewTerm{phase spectrum}\index{phase spectrum}". This spectrum gives us the phase of the harmonic signal (in phase advance or delay).
	\begin{figure}[H]
		\centering
		\includegraphics{img/algebra/fourier_spectrum_graph.jpg}
		\caption{Example of amplitudes and frequencies associated to the different coefficients}
	\end{figure}
	Let us see now how to decompose a known periodic signal into several distinct amplitudes and frequencies signals
	Let us take for example, a periodic square wave signal defined over a period $T = 2$ and of amplitude $A$ such that:
	
	At period $T = 2$ corresponds as we know a pulsation:
	
	\end{tcolorbox}
	
	\pagebreak
	\begin{tcolorbox}[colframe=black,colback=white,sharp corners]
	Let us calculate first the coefficients $c_k$ thanks to the integral that determine the coefficients (the choice of the bounds of the integral is therefore assumed that the signal is periodic by construction!):
	
	Taking $k = 2$, we have:
	
	Similarly for $k = 4,6,8$ and for any even number.\\
	
	About odd numbers, we will have:
	
	The coefficients will then be:
	
	There is only problem in this relations, the coefficient $c_0$ cannot be calculated according to this relation because you can see that if $k = 0$ in the result above, we have an infinite value and it is at least impossible. The coefficient is null or not null  but never infinite (at least in physics because this implies and infinite energy).\\
	
	To find the coefficient $c_0$, we must calculate the integral for $k = 0$. The coefficient $c_0$ is then determined by:
	
	\end{tcolorbox}
	
	\pagebreak
	\begin{tcolorbox}[colframe=black,colback=white,sharp corners]
	The "frequency" spectrum (caution to the abuse of language!) and amplitude will be of the following form for $k=-5...+5$ and $A=1$ null frequencies not being shown:
	\begin{figure}[H]
		\centering
		\includegraphics{img/algebra/fourier_coefficients_example.jpg}
		\caption{Frequency spectrum of the example Fourier series coefficients}
	\end{figure}
	\end{tcolorbox}
	The abuse to talk about frequencies for Fourier coefficients thus leads us to have negative frequencies on the  $x$-axis... but it's only a question of vocabulary (there is no direct relation with the real frequencies) with which you must be familiar.
	
	The amplitude spectrum and phase is calculated according to the following relations:
	
	It is then relatively easy to notice that if $T$ tends to a larger and larger number, the spectrum peaks approach increasingly. So when $T$ tends to infinity the spectrum becomes continuous!!!
	
	The phase spectrum of the above example will give the following for the odd values:
	\begin{figure}[H]
		\centering
		\includegraphics{img/algebra/fourier_phase_diagram.jpg}
		\caption[]{Phase spectrum for the Fourier Series}
	\end{figure}
	It is even possible for example to obtain relatively easily the frequency spectrum in a software like Microsoft Excel 11.8346 (the reader will find an example much more detailed and interesting on the companion exercise server in the section Sequences and Series) !!!
	
	Indeed, it is enough for this purpose to sample for example our signal $128$ times (Microsoft Excel 11.8346 needs $2^n$ samples and works only under this condition!). Then we divide the interval $-1<t<0$ in $64$ samples and ditto for the interval $0>t>1$:
	\begin{figure}[H]
		\centering
		\includegraphics{img/algebra/signal_sample_01.jpg}\\
		\includegraphics{img/algebra/signal_sample_02.jpg}\\
		\includegraphics{img/algebra/signal_sample_03.jpg}
		\caption[]{Signal sample}
	\end{figure}
	Which gives in graphical form (be careful because for the discrete Fourier transform works well in Microsoft Excel 11.8346, it is necessary that the sampling frequency - corresponding to the number of measurements in a second - is at least 100 times higher than the frequency of the original signal otherwise the result can be absurd!):
	\begin{figure}[H]
		\centering
		\includegraphics{img/algebra/signal_fourier_transform_excel.jpg}
		\caption[]{Graphical representation of the data series in Microsoft Excel 11.8346}
	\end{figure}
	Afterwards in Microsoft Excel you simply go to the menu \textbf{Tools/Utility Analysis} and choose the \textbf{Fourier Analysis} option:
	\begin{figure}[H]
		\centering
		\includegraphics{img/algebra/excel_data_analysis_tool.jpg}
		\caption[]{Screenshot of \textbf{Utility Analysis} dialog box of Microsoft Excel 11.8346}
	\end{figure}
	Then comes the following dialog box that must be fill-in as indicated below (we see that the $x$-axis does not matter!):
	\begin{figure}[H]
		\centering
		\includegraphics{img/algebra/excel_fourier_analysis_dialog_box.jpg}
		\caption[]{Parameters of the \textbf{Fourier Analysis} tool in Microsoft Excel 11.8346}
	\end{figure}
	Then comes the following generated list for the coefficients:
	\begin{figure}[H]
		\centering
		\includegraphics{img/algebra/fourier_coefficients_excel_list_01.jpg}
	\end{figure}
	\begin{figure}[H]
		\centering
		\includegraphics{img/algebra/fourier_coefficients_excel_list_02.jpg}
	\end{figure}
	\begin{figure}[H]
		\centering
		\includegraphics{img/algebra/fourier_coefficients_excel_list_02.jpg}
		\caption[]{Corresponding Fourier coefficients to sampled signal with Microsoft Excel 11.8346}
	\end{figure}
	It remains to calculate the module of the complex numbers with the native function  \texttt{IMABS( )} function in Microsoft Excel 11.8346 and divide the result by $128$ for each of the coefficients $c_n$ but we already see that each pair even coefficient  is zero and this match well the theoretical result obtained previously.
	
	We then have putting the index $n$ in front of each module:
	\begin{figure}[H]
		\centering
		\includegraphics{img/algebra/fourier_coefficients_excel_list_completed_01.jpg}
	\end{figure}
	\begin{figure}[H]
		\centering
		\includegraphics{img/algebra/fourier_coefficients_excel_list_completed_02.jpg}
	\end{figure}
	\begin{figure}[H]
		\centering
		\includegraphics{img/algebra/fourier_coefficients_excel_list_completed_03.jpg}
		\caption[]{Module of complex coefficients of the example Fourier Transform with Microsoft Excel 11.8346}
	\end{figure}
	By plotting a customized scatter diagram (still with  Microsoft Excel 11.8346) of columns D and E, we finally get (we restricted the $x$-axis to $[-5, +5]$ for easier reading:
	\begin{figure}[H]
		\centering
		\includegraphics{img/algebra/excel_fourier_spectrum_frequencies.jpg}
		\caption[]{Frequency spectrum of the transformed with Microsoft Excel 11.8346}
	\end{figure}
	To compare with the theoretical calculations (chart already presented previously) ...:
	\begin{figure}[H]
		\centering
		\includegraphics{img/algebra/fourier_coefficients_example.jpg}
		\caption{Theoretical frequency spectrum of the example Fourier series coefficients}
	\end{figure}
	\begin{tcolorbox}[colframe=black,colback=white,sharp corners]
	\textbf{{\Large \ding{45}}Example:}\\\\
	Let us now consider another example identical to the previous with a different approach. We define a periodic function of period $T=2\pi$ as follows:
	
	Let us calculate the Fourier coefficients (we translate the bounds of the integral since the function is periodic this change nothing but facilitate the calculations!):
	
	and:
	
	We notice that $b_n$ is equal to $0$ for $n$ even and equal to $4\pi/n$ when $n$ is odd.
	The Fourier series of the function under consideration is thus written:
	
	What in Maple 4.00b will be written:\\
	
	\texttt{>S:=(4/Pi)*Sum(sin((2*n+1)*x)/(2*n+1),n=0..N);}
	\end{tcolorbox}
	
	\pagebreak
	\begin{tcolorbox}[colframe=black,colback=white,sharp corners]
	and that we can plot using the command:\\
	
	\texttt{>plot({subs(N=4,S),subs(N=8,S),subs(N=16,S)},x=-Pi..Pi,\\
	color=[red,green,blue],numpoints=200);}\\
	
	What gives three plots for $4$, $8$ and $16$ of the series in red, green and blue:
	\begin{figure}[H]
		\centering
		\includegraphics{img/algebra/fourier_series_with_various_terms.jpg}
		\caption{Example of Fourier series in Maple 4.00b with $4$, $8$ and $16$ terms}
	\end{figure}
	For $50$ terms we get:
	\texttt{> plot(subs(N=50,S),x=-Pi..Pi,numpoints=800);}\\
	\begin{figure}[H]
		\centering
		\includegraphics{img/algebra/fourier_series_with_fifty_terms.jpg}
		\caption{Example of Fourier series in Maple 4.00b with $50$ terms}
	\end{figure}
	\end{tcolorbox}
	We see on the example above the side effects named "\NewTerm{Gibbs phenomenon}\index{Gibbs phenomenon}". It is possible to prove they occur to the value of the abscissa corresponding to $x=\pi/2n$ and matching equation and the peak rises to $\pm 1.179$ for all values of $n$. Let's see this!
	
	We proved just before for our example that:
	
	Which can be written:
	
	using the proof made much more earlier at the beginning of this section as that:
	
	Then we have:
	
	Remember that during our study of complex numbers (\SeeChapter{see section Numbers}) we proved that:
	
	Which brings us to:
	
	We will now focus on small values of $x$. So we can then make a Maclaurin  development of first order at the denominator (but not the numerator because of the presence of the $n$):
	
	We make a change of variable:
	
	where we used the traditional notation of the "cardinal sinus" in the last relation as defined in the section trigonometry (remember that this fraction is common in physics this is why it has a specific notation).
	
	As what interest us is to determine the maximum of the Gibbs phenomenon (the disturbance), we see that it takes place in this particular case that we presented (see figure above) for each multiple of $\pi$ and as the denominator of the expression of the integral will decrease as the multiple is higher, it follows that the greatest maximum is at the point where $2nx=\pi$  (the point $0$ at the opposite will cancel the integral therefore we must this latter of our choice). Then we have:
	
	the evaluation of this integral can be done only numerically as we know, therefore we get:
	
	That is about $18\%$ above the expected threshold value.
	
	\paragraph{Power of a signal}\mbox{}\\\\
	A periodic signal has an infinite energy and null average power (\SeeChapter{see secton Electrokinetics}). Its average power over a period is then defined by:
	
	If we develop this equation, we have:
	
	This means that the power of a continuous-time periodic signal is equal to the sum of the squared Fourier coefficients. This is what we name the "\NewTerm{Parseval theorem}\index{Parseval theorem}". This means that if we have any signal that can be decomposed in Fourier series, we can know the power of that signal using only the spectral coefficients.
	
	In reality, we can't mathematically determine the expression of this signal, we use therefore discretization or sampling and then we use a discrete Fourier transform, we can calculate the power of this signal using only the spectral coefficients. This gives us a characteristic of the signal.
	
	Let us also indicate the following result which will be very useful to us in the section of Thermodynamics for the study of the black body and is also very closely related to important properties of the zeta function Riemann:
	
	The following relation:
	
	is named "\NewTerm{Parseval equality}\index{Parseval equality}".
	
	According to the definition of the Fourier series and the definition of the coefficient $a_0$ which follows immediately, we also have frequently in the literature:
	
	
	\pagebreak
	\paragraph{Fourier Transform}\mbox{}\\\\
	Fourier series are a very powerful tool for the analysis of periodic signals for example, but the set of periodic functions is small compared to all the functions that we encounter in physical and engineering problems. So, will we introduce a new extremely powerful analytical tool that extends to a class of more general functions that have very important applications in signal processing, image processing, sound processing, finance and markets advanced statistics!!!
	
	\begin{tcolorbox}[title=Remark,colframe=black,arc=10pt]
	Many teachers and authors associate Fourier Series and Fourier Transform in the field of Functional Analysis. This is right in fact but it seemed to us more appropriate to put the study of this two subjects in this section because closely related to Sequences and Series. However, the subjects that follows normally the study of Fourier Transform, that is to say: Laplace Transfoms, Hilbert Transform and others will be given in the section Functional Analysis of this book. The Fast Fourier Transform study can be found in the section of Theoretical Computing.
	\end{tcolorbox}
	The Fourier transform (FT) is then used for both periodic signals and for aperiodic signals.
	
	For this, we start from study of Fourier series with the complex notation of a periodic function of period $T$ by considering that the period is becoming increasingly big to such that $T\rightarrow +\infty$. Therefore the spectral lines gradually approach to turn into a continuous spectrum.
	
	Therefore, let us resume the expressions proved just earlier:
	
	that we can write equivalently in the following traditional form (wherein it is customary to put the factor $1 / T$ rather in $f (t)$):
	
	and let us write this for future needs in the following form:
		
	and let us put naturally that:
	
	Thus, when $T\rightarrow +\infty$, the pulsation tends to zero and we have $\omega\rightarrow \omega$ because we move from discrete values in continuous values that browse through the set of real number $\mathbb{R}$ (for all $k$). Therefore:
	
	we pass to the limit that is to say:
	
	This implies that:
	
	Therefore we obtain for the coefficients (we change the notation because the previous one is inadequate)
	
	and for the infinite series (the sum becomes an integral)
	
	Caution!!! To make the difference between the given function and its equivalent in which we seek expression in infinite sum, we will note them differently now. Thus, we get:
	
	Thus the discrete Fourier series becomes a continuous function.
	
	\textbf{Definitions (\#\mydef):}
	\begin{enumerate}
		\item We name "\NewTerm{Fourier transform (FT)}\index{Fourier transform }" of $f$ the relation:
		
		sometimes also denoted as follows:
		
		sometimes also named "\NewTerm{spectral density amplitude}\index{spectral density amplitude}".
		
		\item We name "\NewTerm{inverse Fourier transform (IFT)}\index{inverse Fourier transform}" of $F$ the relation:
		
	\end{enumerate}
	Any such transformation technique (as there are many as we will see in the section of Functional Analysis!) is named an "\NewTerm{integral transformation}\index{integral transformation}".
	\begin{tcolorbox}[title=Remark,colframe=black,arc=10pt]
	There are many way of writing the Fourier transform according to the choice of the initial value of $T$!
	\end{tcolorbox}
	Some physicist and engineers prefer to make the two previous relations symmetrical by putting the same coefficient in both directions, which will be for example $1/\sqrt{2}$. This will give:	
	
	Let us also give the corresponding three-dimensional version that will serve us many times in wave mechanics, electrodynamics, wave optics or in the various sections of quantum physics of this book:
	
	To make things perhaps clearer (at least we hope so), let us prove generally that the previous Fourier transform $\mathcal{F}$ is isometric (retains the "norm" - or "modulus" if you prefer...)
	
	\begin{theorem}
	For any functions $f, g$ we have the functional inner product:
	
	But since the functions are in the complex space, as we saw in the section of Vector Calculus, then we must use the notation of the hermitian product:
	
	Remember that:
	
	\end{theorem}
	\begin{dem}
	Then we want to prove the equality:
	
	Explicitly:
	
	But the variable to integrate but must be the same and for $\mathcal{F}(g)$ to be implicitly dependent of $\vec{r}$ it is necessary to take the Fourier transform on $\vec{k}$. Such as:
	
	Therefore:
	
	Therefore using the Fubini theorem (\SeeChapter{see section Differential and Integral}):
	
	Thanks to this result, we have also proved (this is immediate)
	
	We have not specified the bounds: they are infinite in every definition (we include all possible $\vec{k}$ or $\vec{r}$).
	\begin{flushright}
		$\square$  Q.E.D.
	\end{flushright}
	\end{dem}
	Let us now see and prove three interesting properties of the Fourier transform:
	\begin{enumerate}
		\item[P1.] If the function $f$ is an even function (\SeeChapter{Functional Analysis}), it comes a simplification of the Fourier Transform such that:
		
		\item[P2.] If $f$ is odd, we proceed in the same manner as above, and we get:
		
		\item[P3.] Very important property of the Fourier transforms which will be useful to us in finance (\SeeChapter{see section Economy}), and also as part of the study of the heat equation (\SeeChapter{see sectionof Thermodynamics}).
		
		First remember that the Fourier transform is given by:
		
		We want to see what happens if:
		
		By doing an integration by parts (\SeeChapter{see section Integral and Differential}):
		
		we get:
		
		where we put ourself in the situation where:
		
		Therefore:
		
		More generally:
		
	\end{enumerate}
	\begin{tcolorbox}[title=Remark,colframe=black,arc=10pt]
	The branch of "\NewTerm{harmonic analysis}\index{harmonic analysis}", or "\NewTerm{2D Fourier analysis}\index{2D Fourier analysis}", is the branch of mathematics that studies the representation of functions or signals as a superposition of basic waves. It deepens and generalizes the notions of Fourier series and Fourier transform. The basic waves are named "harmonics", hence the name of discipline. During the last two centuries it has had numerous applications in physics and economics under the name "spectral analysis" and knows recent applications including signal processing, quantum mechanics, neuroscience, stratigraphy, statistics, etc.
	\end{tcolorbox}
	\begin{tcolorbox}[colframe=black,colback=white,sharp corners]
	\textbf{{\Large \ding{45}}Examples:}\\\\
	E1. Let us see now a first example (among the two fundamental) of a Fourier transform that we use again in the sections about quantum physics as well as in wave optics.
	
	We will calculate the Fourier transform (spectrum) of the following function (rectangular pulse):
	\begin{figure}[H]
		\centering
		\includegraphics{img/algebra/fourier_transform_rectangular_pulse.jpg}
		\caption[]{Rectangular pulse example for Fourier Transform}
	\end{figure}
	We have therefore:
	
	where $\text{sinc}$ is the sine cardinal as we already know (\SeeChapter{see section Trigonometry}). So we fall back on the $\text{sinc}$ and if we take the squared modulus squared we therefore get the decomposition of a theoretical monochromatic wave diffracted by a rectangular slot!!! Thus, it seems possible to study the diffraction phenomena using the Fourier transform and this field is named "\NewTerm{Fourier optics}\index{Fourier optics}". We'll come back on this later in the section of Functional Analysis when we deepen the Fourier transforms.\\
	
	We the know that the spectrum (described by the $\text{sinc}$ function) crosses zero every time that the sine function is zero, that is to say, every time the frequency is a multiple of $1 / a$.\\
	
	The spectrum of this pulse illustrates two important points regarding the limited time signals:
	\begin{enumerate}
		\item[P1.] A short signal has a broadband spectrum.
		\item[P2.] To a narrow spectrum correspond a long-term signal.
	\end{enumerate}
	\end{tcolorbox}

	\pagebreak
	\begin{tcolorbox}[colframe=black,colback=white,sharp corners]
	E2. The Fourier transform of an integrable function $f$ is given as we know now by:
	
	Consider the integrable Gaussian function of the type:
	
	with $a>0$ define on $\mathbb{R}$.\\
	
	We want to compute its Fourier transform because it is a very important case and particularly useful for solving the heat equation that we will treat in the section of Thermodynamics and also to solve the differential equation of Black \& Scholes in the section Economy.\\
	
	The brilliant trick, if we want to avoid making complex analysis on $3$ A4 pages, is to notice that $F(\omega)$ is the solution of the following linear differential equation:
	
	where $y$ is a function of $\omega$.\\
	
	Indeed deriving  $F(\omega)$  we get:
	
	Integration by parts gives us:
	
	\end{tcolorbox}
	
	\pagebreak
	\begin{tcolorbox}[colframe=black,colback=white,sharp corners]
	We recognize the expression of the Fourier transform of $f$. Therefore:
	
	This shows that the $F(\omega)$ is solution of the differential equation above.\\
	
	We have proved in the section of Differential and Integral Calculus  that the general solution of this differential equation is given by:
	
	where $A \in \mathbb{R}$. And as in the present case:
	
	The primitive $G (x)$ is therefore easy to calculate and we get:
	
	Therefore:
	
	To determine the constant $A$ it suffices to notice that:
	
	To determine the constant $A$ it suffices to note that:
	
	and therefore:
	
	It is then usual to say that the Fourier transform of a Gaussian is another Gaussian!!
	\end{tcolorbox}
	
	\pagebreak
	\subsubsection{Bessel Series}
	Bessel functions are very useful in many advanced fields of physics involving delicate differential equations to solve. The areas in which we find them most often are calorimetry (heat conduction), nuclear physics (physics of reactors), optics and fluid mechanics.
	
	This series are still not study too much in the graduate curriculum and it is often the role of the student to seek the additional information it needs on this subject in the library of his school. We wanted to present here the developments that avoid this approach while staying at home in front of our computer (furthermore books on the subject are quite rare...).
	
	\begin{tcolorbox}[title=Remark,colframe=black,arc=10pt]
	We usually speak by abuse of language of "\NewTerm{Bessel functions}\index{Bessel functions}" instead of "\NewTerm{Bessel series}\index{Bessel series}".
	\end{tcolorbox}
	There is a significant amount of Bessel functions but we will restrict ourselves to the study of those most used one in physics.
	
	\paragraph{Zero order Bessel's Functions}\mbox{}\\\\
	The function known as the "\NewTerm{Zero order Bessel's function}\index{Zero order Bessel's function}", is defined by the power series:
	
	It is during the study of the properties of derivation and integration that Friedrich Bessel found that this series of power is a solution to a differential equation that is found frequently in physics. That is why it bears his name.
	
	If $u_r$ represents the $r$-th term of the series, we easily see that:
	
	which tends to zero as $r\rightarrow +\infty$, regardless of the value of $x$. This has for consequence that the series converges for all values of $x$. Since this is a series of positive power, the function $J_0(x)$ and all its derivatives are continuous function for all values of $x$, real or complex.
	
	\paragraph{$n$ order Bessel's Functions}\mbox{}\\\\
	The function $J_n(x)$, known as the "\NewTerm{$n$ order Bessel's function}\index{$n$ order Bessel's function}", is defined, when $n$ is a positive integer, by the power series:
	
	which converges for all values of $x$, real or complex.
	\begin{figure}[H]
		\centering
		\includegraphics{img/algebra/bessel_functions.jpg}
		\caption{Plot of few Bessel functions (source: Wikipedia)}
	\end{figure}
	and in Microsoft Excel or Maple 4.00b the previous function can be found under the name \texttt{BESSELJ( )}. For example for the previous graph in Maple, we just write:
	
	\texttt{>plot([BesselJ(0,x),BesselJ(1,x),BesselJ(2,x),BesselJ(3,x)],x=0..20);}
	
	Let us see in particular, that for $n=1$ we have:
	
	and when $n=2$:
	
	We can notice that $N_n(x)$ is an even function of $x$ when $n$ is even and odd if $n$ is odd (\SeeChapter{see section Functional Analysis}).
	
	If we play to do engineering maths we notice by trial and errors that:
	
	Based on this trial and error approach we have using the above expression by factorizing the $(x/2)^{2k}$ term only:
	
	So finally we see after trial and errors again (instead than doing 5 pages of mathematical developments as do mathematicians) that:
	
	More generally for $n$ that is non-integer we use the Euler Gamma function (\SeeChapter{see section Differential and Integral Calculus}):
	
	That is also sometimes written by pure mathematicians (...)
	
	Now by differentiating the function $J_0(x)$ and comparing the result with the series $J_1(x)$, we see that without much pain that:
	
	We also find without too much difficulty, the following relation:
	 
	\begin{tcolorbox}[title=Remark,colframe=black,arc=10pt]
	In general by recursive reasoning we get:
	
	Therefore:
	
	This last relation will be useful to us in the section of Wave Optics for our study of the circular aperture diffraction (Airy disk).
	\end{tcolorbox}
	Using the fact that:
	
	and including it in the previous relation, we find:
	
	written in another way:
	
	$y=J_0(x)$ is therefore a solution of the differential equation of the second order:
	
	written otherwise:
	
	or even:
	
	A solution to a an equation of parameter $n$ which is not a multiple of $J_n(x)$ is named "\NewTerm{Bessel function of the second kind}\index{Bessel function of the second kind}". Let us suppose now that $u$ is such a function and let us put $v=J_0(x)$; then according to the relation:
	
	we have:
	
	Multiplying the first relation by $v$ and the second by $u$ and after subtracting, we get:
	
	we therefore also have:
	
	we can therefore write:
	
	Indeed, because if we develop, we find:
	
	For the equality:
	
	is satisfied, we have:
	
	Dividing by $xv^2$, we have:
	
	which is equivalent to:
	
	immediately, by integrating it comes:
	
	where $A$ is a constant. Consecutively we have since $v=J_0(x)$:
	
	where we recall, $A$ and $B$ are constants, and $B\neq 0$ if $u$ is not a multiple of $J_0(x)$ by definition!
	
	If in the last relation, $J_0(x)$ is replaced by its expression in terms of series we have:
	
		For those who want to check this last relation (I do not like this kind of algebraic calculations) with Maple 4.00b just write:
	
	\texttt{>1/x*taylor(1/(series(BesselJ(0,x),x))\string^2,x=0,5);}
	
	Therefore:
	
	consecutively if we put:
	
	where $Y_0(x)$ is a particular Bessel function of the second type named "\NewTerm{Bessel-Neumann function of the second kind of zero order}\index{Bessel-Neumann function of the second kind of zero order}".

	Identically to the fact that when $J_0(x)\rightarrow 0$ when $x \rightarrow 0$, the expression $Y_0(x)$ because of the term $\log(x)$ when $x$ is small approaches $Y_0(x) \rightarrow -\infty$ when $x\rightarrow +0$.
	
	Finally, it comes from what we have seen that $J_0(x)$ and $Y_0(x)$ are independent solutions of the differential equation:
	
	The general solution being therefore:
	
	where $A$, $B$ are arbitrary constants and $x>0$ so that $Y_0(x)$ is real.

	If we replace $x$ by $k$, where $k$ is a constant, the differential equation becomes:
	
	by multiplying the whole by $k^2$ we find the general form of the differential equation:
	
	whose general solution is:
	
	where $k>0$ such that $Y_0(kx)$ is real when $x>0$.
	
	In fact, the Bessel functions are solutions of the differential equation previously studied and solved by the "\NewTerm{Frobenius method}\index{Frobenius method}". Indeed, let us write:
	
	and let us make the substitution:
	
	substituting in $Ly$, we get:
	
	Now let us choose the $c_i$ to satisfy the differential equation such as:
	
	Therefore, unless $\rho$ is a negative integer, we have:
	
	By substituting these values in the relation:
	
	we get:
	
	Therefore:
	
	If we put $\rho=$ in the prior-previous relation, we get:
	
	
	\paragraph{Bessel's Differential Equations of order $n$}\mbox{}\\\\
	We have defined the Bessel series as:
	
	Let us put:
	
	and let us derivate as follows:
	
	But we also have:
	
	By subtraction:
	
	Which finally gives:
	
	This is also written:
	
	which is named the "\NewTerm{Bessel differential equation of order $n$}\index{Bessel differential equation of order $n$}" or more simply "\NewTerm{Bessel equation}\index{Bessel equation}". In fact, most schools or Internet sites give this differential equation as a definition but now it is clear that there is rigorous reasoning behind this equation.
	
	The solution is therefore of the type:
	
	which is still sometimes written using the gamma Euler function (\SeeChapter{see section Differential and Integral Calculus}):
	
	It follows that:
	
	and therefore that $y=J_n(x)$ is the solution of this differential equation.
	
	We will fall back on such a differential equation during our study of the wave equation of a circular drum (\SeeChapter{see section Wave Mechanics}), during our study of the physics of nuclear reactors (\SeeChapter{see section Nuclear Physics}) and finally during our study of self-buckling (\SeeChapter{see section Mechanical Engineering}).
	
	\subsection{Convergence Criteria}
	When we study a series, one of the fundamental questions is that of the convergence or divergence of this series.
	
	If a series converges, its general term approaches zero as $n$ approaches infinity:
	
	or obviously generally:
	
	This criterion is necessary but insufficient to establish the convergence of a series. By cons, if this criterion is not met, we are absolutely sure that the series does not converge (so it diverges!).
	
	Three methods are proposed to deepen the convergence criteria:
	\begin{enumerate}
		\item The integral test

		\item The d'Alembert rule

		\item The Cauchy rule
	\end{enumerate}
	In the following paragraphs, we will assume the series with positive terms. The case of alternating series will be seen later.
	
	\subsubsection{Integral Test}
	The integral test for convergence is a method used to test infinite series of non-negative terms for convergence. It was developed by Colin Maclaurin and Augustin-Louis Cauchy and is sometimes known as the "\NewTerm{Maclaurin–Cauchy test}\index{Maclaurin–Cauchy test}".
	
	Given the series with decreasing positive (monotone decreasing) terms:
	
	That is to say:
	
	and given a continuous decreasing function such that:
	
	Then the infinite series
	
	converges to a real number if and only if the improper integral:
	
	is finite. In other words, if the integral diverges, then the series diverges as well.
	\begin{tcolorbox}[title=Remark,colframe=black,arc=10pt]
	In no case does the integral gives the value of the sum of the series! The full test only gives an indication of the convergence of the series. Before making the test of the integral, it is important to check that the terms of the series are strictly decreasing to fill the condition $a_1\ge a_2\ge a_3\ge \cdots\ge a_n\ge \cdots$.
	\end{tcolorbox}
	\begin{tcolorbox}[colframe=black,colback=white,sharp corners]
	\textbf{{\Large \ding{45}}Example:}\\\\
	The harmonic series:
	
	diverges because (\SeeChapter{see section Differential and Integral Calculus}):
	
	So this harmonic series does not converge.
	\end{tcolorbox}
	
	\subsubsection{D'Alembert Rule}
	The "\NewTerm{ratio test}\index{ratio test}" is also a test (or "criterion") for the convergence of a series of the type:
	
	where each term is a real or complex number and $a_n$ is nonzero when $n$ is large. The test was first published by Jean le Rond d'Alembert and is sometimes known as "\NewTerm{d'Alembert's ratio test}\index{d'Alembert's ratio test}" or as the "\NewTerm{Cauchy ratio test}\index{Cauchy ratio test}".
	
	The usual form of the test makes use of the limit:
	
 	The ratio test states that:
	\begin{enumerate}
		\item if $L < 1$ then the series converges absolutely;
		\item if $L > 1$ then the series does not converge;
		\item if $L = 1$ or the limit fails to exist, then the test is inconclusive, because there exist both convergent and divergent series that satisfy this case
	\end{enumerate}
	and we define the radius of convergence by:
	
	For the proof suppose that:
	
	\begin{tcolorbox}[title=Remark,colframe=black,arc=10pt]
	In reality this rule is normally without the absolute value. The case with the absolute value as above is named the "\NewTerm{absolute convergence test}\index{absolute convergence test}" and applied for the more general case of an alternate series such that:
	
	If an alternating series of terms is absolutely convergent, the absolute series that follows also converge.\\
	
	Therefore the absolute convergence test is a generalization of the d'Alembert rule but most of time we don't make any distinctions between the both.
	\end{tcolorbox}
	We can then show that the series converges absolutely by showing that its terms will eventually become less than those of a certain convergent geometric series. To do this, let:
	
	Then $r$ is strictly between $L$ and $1$, and:
	
	 for sufficiently large $n$ (say, $n$ greater than $N$).

	Hence:
	
	for each $n > N$ and $i > 0$, and so:
	
	That is, the series converges absolutely.

	On the other hand, if $L > 1$, then:
	
	for sufficiently large $n$, so that the limit of the sum is non-zero. Hence the series diverges.
	\begin{tcolorbox}[colframe=black,colback=white,sharp corners]
	\textbf{{\Large \ding{45}}Example:}\\\\
	Given the following geometric series:
	
	We get quotient:
	
	Therefore the series converge!
	\end{tcolorbox}
	Obviously some practical applications will (can) give for example:
	
	and some practitioners name this the "\NewTerm{Cauchy convergence rule}\index{Cauchy convergence rule}"... (do not confuse with the Cauchy convergence test that will be study in the section Fractals).
	
	\begin{tcolorbox}[colframe=black,colback=white,sharp corners]
	\textbf{{\Large \ding{45}}Example:}\\\\
	Let us do now a last important example. We have study earlier above Bessel series. But it can be not obvious that these series converge. Let us prove that this is indeed the case for $J_0$, that is to say for:
	
	Therefore:
	
	\end{tcolorbox}
	
	\subsubsection{Alternating Series Test}
	The alternating series test i a method used to prove that an alternating series with terms that decrease in absolute value is a convergent series. The test was used by Gottfried Leibniz and is sometimes known as "\NewTerm{Leibniz's test}\index{Leibniz's test}", "\NewTerm{Leibniz's rule}\index{Leibniz's rule}, or the "\NewTerm{Leibniz criterion}\index{Leibniz criterion}".
	
	A series of the form
	
	where either all an are positive or all an are negative, is named an "\NewTerm{alternating series}\index{alternating series}".

	The alternating series test then says: if $a_n$ decreases monotonically and:
	
 	then the alternating series converges.
	
	There a lot of other tests as the Raabe's test, the Bertrand's test, the Gauss's test, the Kummer's test.
	
	\subsubsection{Fixed Point Theorem}
	The fixed point theorem is not really useful in physics and for the engineers (but implicitly it is essential but physicists and engineers often use math tools whose properties have already been approved in advance by mathematicians), however we find it in chaos theory and in theoretical computing (see the sectiona on Fractals especially the topic on the Sierpinski triangle). We can therefore only recommend the reader to take the time to read and understand the explanations and developments that follow.
	
	Let $(X,d)$, be a complete metric space (see sections Topology or Fractals) and $T:X\mapsto X$ an application strictly contracting of constant $L$ (see the Lipschitz functions in the section Topology), then there exists a unique point $\omega\in X$ such that:
	
	$\omega$ is then named the "\NewTerm{fixed point}\index{fixed point}" of $T$ (think to the case $\cos(x)=x)$. Furthermore, if denote by:
	
	the image of $x$ by the $n$-th iterate of $T$, then we have:
	
	and the convergence speed can also be estimated by:
	
	By the fact that we restrict our study to iterating a function, we speak of "\NewTerm{Banach fixed-point theorem}\index{Banach fixed-point theorem}" that gives a general criterion guaranteeing that, if it is satisfied, the procedure of iterating a function yields a fixed point.
	\begin{tcolorbox}[title=Remarks,colframe=black,arc=10pt]
	You can have fun with your pocket calculator or your operating system by choosing a random number and taking the cosine iteratively. You will find you will tend to $0.74$ and therefore it is verbatim the solution of $\cos (x) = x$.
	\end{tcolorbox}
	\begin{dem}
	Given $x\in X$. We consider the following sequence $(T^n(x))_{n\in \mathbb{N}}$ as defined above. First we will prove that this sequence is a Cauchy sequence (see above what is a Cauchy sequence).

	Applying the triangle inequality (\SeeChapter{see section Vector Calculus}) several times we have:
	
	But:
	
	Therefore:
	
	and as:
	
	Therefore:
	
	To finish:
	
	that is to say, that in a first time $(T^n(x))_{n\in \mathbb{N}}$ converge, and we put:
	
	Now we check that $\omega$ is a fixed point of $T$. Indeed $T$ is uniformly continuous (as Lipschitz - see section Topology) therefore a fortiori continues:
	
	It remains to check that $\omega$ is the only fixed point (therefore we will have proved that $\omega$ does not depend on the choice of $x$). Suppose that we also have $T(y)=y$ then:
	
	An estimate of the speed of convergence is given by:
	
	$\mathrm{d}(,)$ is continuous with respect to each variable so:
	
	and the limits preserve the inequalities (not strict one) thus:
	
	\begin{flushright}
		$\square$  Q.E.D.
	\end{flushright}
	\end{dem}
	
	\subsection{Generating Functions (transformation of a sequence into a series)}
	For some mathematical financial risk management models (\SeeChapter{see section Economics}) and also integration transformation of Bessel functions (\SeeChapter{see section Differential and Integral Calculus}) we will need in this book a gentle introduction to generating functions.
	
	\subsubsection{Ordinary Generating Functions (transformation of a sequence into a series)}
	Remember first that we have proved earlier above that the general Maclaurin expansion of a function was given by:
	
	That is for $x=\cong 0$
	
	In the case of our study, let us write the latter relation as:
	
	We say then that the above relation is the "\NewTerm{ordinary generating function}\index{ordinary generating function}" of the sequences of numbers $a_0,a_1,a_2,a_3,\ldots$ in the formal parameter $x$. And we don't care if it diverge or not!
	
	Because the generating function is an algebraic expression that encodes the sequence and allows us to manipulate it in ways that are not possible in other forms. Many times if the sequence you are looking at is "interesting" (and this word has lots of interpretations), the generating function has a short simple form.

	The generating function allows us to derive formulas for the sequence, identities involving the sequence, estimate the values and so much more as we will see further below.	
	
	One thing to never forget: The generating function is not the sequence and the sequence is not the generating function. They are not the same thing. One is a sequence, the other is an algebraic expression!

	If you have a sequence you can say "the generating function of the sequence" to refer to the algebraic object. If you have a generating function you might say "the sequence of coefficients of the generating function" in order to refer to the sequence.
	
	Let us try now a companion example, the sequence consisting of all $1$:
	
	The generating function in therefore the geometric series that we have proved earlier above:
	
	The next simple companion example would be the positive integers:
	
	This has generating function:
	
	Now we observer that the derivative of the this generating function is the derivative of the generating function:
	
	Indeed:
	
	That is we have therefore:
	
	We conclude that:
	
	We can't use the exactly same trick to figure out the generating function for the sequence:
	
	because if we take the derivative of:
	
	then we do not quite have the square integers. But the attentive reader will notice that if we first multiply the latter relation by $x$ and then take the derivative then we have:
	
	Therefore:
	
	It is easy with a software like Maple to control that the Maclaurin series of $(1+x)/(1-x)^3$ is equal to the generating function and therefore give us the coefficients $a_i$ of the sequence.
	
	Let us see now a non-trival example... It's the Fibonacci sequence given by (each term is equal to the sum of the previous two):
	
	We will give the generating function for this sequence a name $F(x)$ so then:
	
	where $F_0=F_1=1$ and $F_k=F_{k-1}+F_{k-2}$ for $k\geq 2$. Then we can see that:
	
	Since we have figured out that:
	
	then:
	
	and this can be rewrittten as:
	
	and hence:
	
	It is always surprising that the generating function for the Fibonacci numbers has such a compact formula. But once again, using for example a software like Maple and doing the Maclaurin expansion of the above function, you will get a series whose coefficients $a_i$ corresponds to the Fibonacci sequence!
		
	\paragraph{Composition of Generating functions}\mbox{}\\\\
	Now if we have two generating functions:
	
	for two sequences of integers $a_0,a_1,a_2,a_3,\ldots$ and $b_0,b_1,b_2,b_3,\ldots$ then there are several ways that we can combine the sequences and get generating functions fr new sequences.

	\begin{itemize}
		\item Sum: If we add the generating functions we have that:
		
		is a generating function for the sequence:
		
		
		\item Product: However if we multiply the two generating function, we ha that:
		
		This can be summarized in the expression:
		
		Another special case of the product of generating function is the product:
		
		It is the product of two generating functions, the first one being the generating function:
		
		By:
		
		the product of these is a generating function for the sequence (as all $a_i=1$):
		
	
		\item Derivative: We have already seen a couple examples of the
use of the derivative in previous examples.
	\end{itemize}
	Remember now that we have proved that:
	
	Therefore by the fact that:
	
	has for generating sequence:
	
	we have then that:
	
	is then a generating function for the sequence of the sum of the first $n$ positive integers:
	
	In particular, the coefficient of $x^k$ is:
	
	We also know by taking the derivative of $1/(1-x)^2$ we have:
	
	Therefore if we divide this equation by two we have:
	
	It must be that the coefficient of $x^k$ in $1/(1-x)^3$ is equal to $(k + 1)(k + 2)/2$ and it is equal to the sum of the first $k+1$ integers, so:
	
	
	\subsubsection{Multivariate Generating Functions}
	Remember that in the section Calculus we have proved that:
	
	can be expressed by a condensed form that involves the binomial coefficient:
	
	then this is what we name a "\NewTerm{multivariate generating function}\index{multivariate generating function}". It works just as the other generating functions we have previously worked with except that it has two parameters.
	
	\subsubsection{Functional Generating Functions}
	So far we have seen generating functions that gives scalar values. But a more general family are the "\NewTerm{functional generating functions}\index{functional generating functions}" that gives functions instead of simple scalars.
	
	A generating function for a sequences of functions $\{f_n(x)\}$ is obviously a power series of the type:
	
	whose coefficient are now functions of $x$.
	
	Let us see the both examples that we be useful to us in finance and in optics (but also in differential and integral calculus!).
	
	Let us start with the most important example involving probabilities and especially the Poisson distribution as used in the First Boston Credit Risk Metric model!
	
	Let us recall that the Poisson distribution mass, that is, the probability that $N$ is equal to $n$, is given by (\SeeChapter{see section Statistics}):
	
	for $n=0,1,2,\ldots$.

	The probability generating function is equal to:
	
	We know also the following Maclaurin infinite series expansion:
	
	And if we rewrite:
	
	in the following way:
	
	And further assuming the above Maclaurin expansion we can write:
	
	Which implies:
	
	This is the probability generating function of a Poisson distribution that we will use for our study of the CreditRisk model.
	
	And now for our study of optics let us found the functional generating function of the first kind Bessel functions!
	\begin{theorem}
	The generating function for the sequence of Bessel function of first kind, on integer order, is:
	
	\end{theorem}
	\begin{dem}
	To obtain an expression for $J_n(x)$, we use the Maclaurin series for $e^x$ to get:
	
	Now let us  make the change of variable: $n=r-s$.

	Therefore the expression in the sum becomes:
	
	Then it is obvious that as the range of $r$ and $s$ is $]-\infty,+\infty[$ we have for the range of $n$ also: $]-\infty,+\infty[$. So the first sum is easy to determine:
	
	But as we can see in the expression in the sum above, we don't get rid of the summation variable $s$. So there is obviously a second missing sum on the variable $s$! The interval of summation for $s$ is not obvious at a first glance...

	What is sure is that $s$ is still in the range $[0,+\infty]$ (positivie values) if we look at the term $s!$ in the denominator. But what makes problem is the lower bound of the sum. So if we look closely to the relation in the sum above we have a term $(n+s)!$. Obviously $n+s$ must never be negative! It comes therefore that if $n<0$, we must have $n+s\geq 0$, that is when $n<0$, $s$ must start $-n$. Finally:
	
	That is:
	
	where the $J_n$ are the Bessel function of the first kind.
	\begin{flushright}
		$\square$  Q.E.D.
	\end{flushright}
	\end{dem} 
	Therefore, for $n\geq 0$:
	
	And for $n<0$:
	
	Using an index shift, we obtain:
	
	
	\begin{flushright}
	\begin{tabular}{l c}
	\circled{95} & \pbox{20cm}{\score{3}{5} \\ {\tiny 44 votes,  64.09\%}} 
	\end{tabular} 
	\end{flushright}
	
	%to make section start on odd page
	\newpage
	\thispagestyle{empty}
	\mbox{}
	\section{Vector Calculus}

	\lettrine[lines=4]{\color{BrickRed}V}ector calculus or "vector analysis" is a branch of mathematics that studies the scalar or vector fields that are sufficiently regular of Euclidean spaces (see definition further below).\\\\

The importance of vector calculus comes from its extensive use in physics and in the engineering sciences. It is from this perspective that we will present it, and this is why we limit ourselves mostly to the case of the usual three-dimensional spaces. In this context, a vector field associates to each point of the space a vector (with three real components), while a scalar field associates just a unique real number to such a point.

There is a phenomenal amount of series and theories about these, but we will mention especially the Taylor series (used almost everywhere in applied science), Fourier series (signal theory, statistics, wave mechanics or quantum physics) and Bessel series functions (very important in nuclear physics!) that we will make a brief study here and that will continue in the section of Functional Analysis.

	\begin{tcolorbox}[colframe=black,colback=white,sharp corners]
\textbf{{\Large \ding{45}}Example:}\\\\
For example, imagine the water from a lake. The temperature data at each point forms a scalar field, that of his speed at each point, a vector field (see definition further below).
	\end{tcolorbox}

Physical concepts such as strength or speed are characterized by a direction, an orientation and an intensity. This triple character is highlighted by arrows. These are the source of the concept of vector and are the most suggestive example. Although their nature is essentially geometric, it is their ability to bind to each other, so their algebraic behavior, which mostly retain our attention. Split into equivalence classes their set represents the classic model of a "\NewTerm{vector space}\index{vector space}" (\SeeChapter{see section Set Theory}). One of our primary goals here is the detailed description of this model.

	\begin{tcolorbox}[title=Remarks,colframe=black,arc=10pt]
	\textbf{R1.} Before reading what follows, the reader is advised to have at least read diagonally the section on Set Theory in the Arithmetic chapter. We define there what is a "vector space" using the tools of Set Theory. Even if this concept  is although not absolutely essential it is still interesting to see how two areas of mathematics fit together and also just for the reason... of introducing vector stuffs with a least a little bit rigour.\\
	
	\textbf{R2.} Vectorial analysis contains many terms and definitions that must be learned by heart. This work is hard but unfortunately necessary...
	\end{tcolorbox}
	
	\pagebreak
	\subsection{Concept of Arrow}	

\textbf{Definition (\#\mydef):} We denote by $U$ the ordinary space of elementary geometry and $P, Q,...$ its points. We will call "\NewTerm{arrow}\index{arrow}" all directed line segment (in space). The arrow of origin $P$ (origin point) and extremity $Q$ (terminal point) will be denoted $\overrightarrow{PQ}$ or abbreviated by a single letter (Latin or Greek) arbitrarily chosen as example: $\overrightarrow{F}$.

	\begin{tcolorbox}[title=Remark,colframe=black,arc=10pt]
In the norm ISO 80000-2: 2009 it is authorized to represent the vectors with a letter in bold.
	\end{tcolorbox}	

We will consider as obvious that any arrow is characterized by its direction, its orientation (because of a given direction it can point in both directions), its intensity or magnitude (length) and its origin.

In vector (or multivariable) calculus, we will deal with functions of two or three variables (usually $x, y$ or $x, y, z$, respectively). The graph of the arrow of coordinates $(x, y)$, lies in Euclidean space, which in the Cartesian coordinate system consists of all ordered doublets of real numbers $(a,b)$. Since Euclidean can be 3-dimensional (and more or less for sure!), we denote it by $\mathbb{R}^3$.

The graph of the arrow consists of the points $(a, b, c)$. The 3-dimensional coordinate system of Euclidean space can be represented on a flat surface, such as this page or a blackboard, only by giving the illusion of three dimensions, in the manner shown in the figure below:

\begin{figure}[H]
\centering
\includegraphics[scale=0.75]{img/algebra/euclidian_vector.eps}
\caption{Example of arrow in $\mathbb{R}^3$ euclidian space}
\end{figure}

Euclidean space has three mutually perpendicular coordinate axes ($x$, $y$ and $z$), and three
mutually perpendicular coordinate planes: the $xy$-plane, $yz$-plane and $xz$-plane:

\begin{figure}[H]
\centering
\includegraphics[scale=0.75]{img/algebra/euclidian_planes.eps}
\caption{Mutually perpendicular planes in $\mathbb{R}^3$}
\end{figure}

The coordinate system shown above is known as a right-handed coordinate system, because it is possible, using the right hand, to point the index finger in the positive direction of the $x$-axis, the middle finger in the positive direction of the $y$-axis, and the thumb in the positive direction of the $z$-axis, as below:

\begin{figure}[H]
\centering
\includegraphics[scale=0.75]{img/algebra/right_hand.eps}
\caption{Right hand system}
\end{figure}


	\subsection{Set of Vectors}	

\textbf{Definitions (\#\mydef):}

	\begin{enumerate}
		\item[D1.] We say that two arrows are "\NewTerm{equivalent arrows}" \index{equivalent arrows}  if they have the same direction, the same orientation and the same intensity.
		\item[D2.] We say that two arrows are "\NewTerm{colinear arrows}"\index{colinear arrows} only if they have the same direction.
	\end{enumerate}
Let us now split the set of all arrows in equivalence classes: two arrows belong to the same class if and only if they are equivalent.

So:

\textbf{Definitions (\#\mydef):} 

\begin{enumerate}
	\item[D1.]Each equivalence class of arrows whose origin point and terminal point are distinct is a "\NewTerm{vector}"\index{vector} or rather a "\NewTerm{free vector}"\index{free vector} because its origin is not taken into account (if its origin is well defined, then we have a "\NewTerm{bounded vector}")\index{bounded vector}.
	\item[D2.]Degenerated arrows (that is to say of the form $\overrightarrow{PP}$) are named "\NewTerm{zero vector}"\index{zero vector} and written $\vec{0}$ when they have an undefined direction and orientation and zero intensity (origin and terminal point are not distinct).
\end{enumerate}	

The set of vectors will be commonly referred  by $\mathbb{V}$. Note that the elements of $\mathbb{V}$ are (equivalence) classes arrows and not individual arrows. It is however clear that any arrow is sufficient to determine the class of equivalence to which it belongs and it is natural to name the corresponding class: "\NewTerm{representative class}"\index{representative class} of the vector.

Let us now draw a representative of a vector $\vec{y}$ from the end of a representative vector $\vec{x}$. The arrow whose origin is that of the representative $\vec{x}$ and the end of representative $\vec{y}$ determines a new vector which we write: $\vec{x}+ \vec{y}$. The operation that combines any two vectors by their sum is named "\NewTerm{vector addition}"\index{vector addition}.

\begin{figure}[H]
\centering
\includegraphics[scale=1]{img/algebra/vector_addition.jpg}
\caption{Example of a sum of two vectors}
\end{figure}

\begin{figure}[H]
\centering
\includegraphics[scale=1]{img/algebra/vector_addition_robotics.jpg}
\caption{Example of a sum of two vectors with robot dynamics notation}
\end{figure}

Using a figure, it is easy to show that the operation of vector addition is associative and commutative, i.e. that:
	
and:
	
It is also evident that the zero vector $\vec{0}$ is the neutral element of the vector addition. Formally:
	
where $-\vec{x}$ means the opposite of vector $\vec{x}$, that is to say the vector whose representatives have the same direction and the same intensity as those of $\vec{x}$, but the opposite orientation. 

\textbf{Definitions (\#\mydef):}
	\begin{enumerate}
		\item[D1.] Two vectors whose sum is zero are then named "\NewTerm{opposed vectors}"\index{opposed vectors} since the only thing that differentiates them is their orientation...
		\item[D2.] It follows that if two or more vectors have the same direction, the same intensity and the same orientation then  we say that they are "\NewTerm{equal vectors}"\index{equal vectors}.
	\end{enumerate}
As we can see the reverse operation of vector addition is the vector subtraction. Subtract a vector is equivalent to adding the opposite vector.

	\begin{tcolorbox}[title=Remarks,colframe=black,arc=10pt]
\textbf{R1.} The addition extends, by induction, to the case of any finite family of vectors. Under associativity, these successive additions can be performed in any order, which justifies the writing without brackets.\\\\
\textbf{R2.} The multiplication between two vectors is a concept that does not exist. But, as we shall see it a little further, we can multiply the vectors by some other vector properties that which we call the "norm" or simply by scalars and still by some other things...
	\end{tcolorbox}	

\subsubsection{Pseudo-Vectors}

In physics, in the statement named the "\NewTerm{Curie's principle}"\index{Curie's principle} (\SeeChapter{see section Principa}), physicists mention of what they name "\NewTerm{pseudo-vectors}"\index{pseudo-vector}. This is the simple vocabulary to talk about something equally trivial but basically only a few people actually do use. But it can still be useful to present what it is.

In fact, vectors and pseudo-vectors are transformed in the same way for a rotation or translation (we will see in our study of Linear Algebra how mathematically perform this type transformations). It is not the same in symmetry with respect to a plane or at one point. In these transformations we have by definition the following properties:
	\begin{enumerate}
		\item[P1.] A vector is transformed into its symmetrical.
		\item[P2.] A pseudo-vector is converted into the opposite of its symmetrical.
	\end{enumerate}
Here is a figure with typical examples (the choice of letters representing the vectors and pseudo-vectors is not due to chance; they are a wink to the properties of electric and magnetic fields as studied in the Electromagnetism chapter):
\begin{figure}[H]
\centering
\includegraphics[scale=0.75]{img/algebra/pseudo_vector.eps}
\caption{Differences of transformations between a vector and a pseudo-vector}
\end{figure}

A well know practical example is a pseudo-vector that we will study in detail much further below and resulting of an operation named "\NewTerm{cross product}"\index{cross product}:
	
And to see why the result is a pseudo-vector, consider the special simple case:
	
Now if we do a symmetric operation of the $X\text{O}Z$ plan we get:
	
So as we can see the vector resulting of the cross product is a pseudo-vector because under transformation of the plan it's orientation change!

A vector resulting of a mathematical operation of symmetry that does not change its orientation is named a "\NewTerm{polar vector}"\index{polar vector} but in fact almost everybody say just "vector".

Now that we have an idea of what vectors are, we can start to perform some of the usual algebraic operations on them and this is what we name "\NewTerm{vector algebra}"\index{vector algebra}.

\pagebreak
\subsubsection{Multiplication by a scalar}

The vector expression $\alpha\cdot \vec{x}$ named "\NewTerm{product of vector $\vec{x}$ by scalar $\alpha$}"\index{vector scalar product} is defined as follows:

Take a representative arrow $\vec{x}$ and construct a same direction arrow in the same or opposite orientation , depending on whether $\alpha$ (scalar) is positive or negative, and of intensity $\mid \alpha \mid$ times the intensity of the initial arrow. The arrow thus obtained is a representative of the vector of relation:
	
If $\alpha=0$ or $\vec{x}=0$ we write:
	
The operation consisting of performing the product of a scalar by a vector is named "\NewTerm{scalar multiplication}\index{scalar multiplication}".

We easily check that the scalar multiplication is associative and distributive with respect to the vector numerical addition, formally:
	

	The multiplication of a vector by non-null scalar doesn't change its direction if the scalar is positive but if the scalar is negative the vector will still have the same direction but it orientation will be opposite.
	
	From this definition he have that two vectors $\vec{v}$ and $\vec{w}$ are parallel (denoted by $\vec{v}\mid\mid\vec{w})$) if one is a scalar multiple of the other. You can think of scalar multiplication of a vector as stretching or shrinking the vector, and as flipping the vector in the opposite direction if the scalar is a negative number.
	
	
	Let's see a concrete example worldwide known of use of vectors with scalars (probably also the simplest example):
	
	\paragraph{Rule of three}\mbox{}\\\\
	Let us go back to the "\NewTerm{rule of three}"\index{rule of three} (sometimes also named "\NewTerm{rules of ratios and proportions}\index{rules of ratios and proportions}" or "\NewTerm{unit reduction method}\index{unit reduction method}") often define in small classes (middle-school) intuitively but without nice proof. This rule is probably the most widely used algorithm in the world that identifies a fourth number when are given three and the four numbers are linearly dependent.

The rule of three is derived most of time in two versions:
	\begin{enumerate}
		\item[V1.] Simple an direct if the magnitudes are directly proportional.
		\item[V2.] Simple and reverse if the quantities are inversely proportional.
	\end{enumerate}
and when two variables $X$ and $Y$ are proportional we note for recall:
	
\begin{theorem}
Suppose now that $X$ can take the values $x_1,x_2$. $Y$ will take  the values linearly dependent $y_1,y_2$ then the following proportional relation applies:
	
is named "\NewTerm{simple and direct ratio}\index{simple and direct ratio}".
\end{theorem}
\begin{dem}
	Given two collinear vectors $\vec{x}=(x_1,x_2),\vec{y}=(y_1,y_2)$ and therefore proportional to a given factor $\lambda$ such that:
	
	\begin{flushright}
		$\square$  Q.E.D.
	\end{flushright}
\end{dem}
	\begin{tcolorbox}[title=Remark,colframe=black,arc=10pt]
If this ratio is not equal (thus: not proportional), then we must switch to other tools such as simple and inverse ratio, or regression techniques and verbatim: extrapolation.
	\end{tcolorbox}	
	\begin{tcolorbox}[colframe=black,colback=white,sharp corners]
\textbf{{\Large \ding{45}}Example:}\\\\
In Lausanne (Switzerland), in 2011, garbage bags ar taxed and following rates apply: a bag of $17 [L]$ is $1.-$ and the bag of $110 [L]$ is $3.80.-$. Reported to $17 [L]$ the price of the garbage bag of 110 [L] is thus:
	
That is to say approximately $60\%$ of the price of the bag of $17 [L]$ (then go search for an explanation... ???).
	\end{tcolorbox}
\begin{theorem}
The following proportional relation:
	
is named "\NewTerm{simple and inverse ratio}\index{simple and inverse ratio}".
\end{theorem}
\begin{dem}
	Given two collinear vectors $\vec{x}=(x_1,x_2),\vec{y}=(y_1,y_2)$ and therefore proportional to a given factor $\lambda$ such that:
	
	\begin{flushright}
		$\square$  Q.E.D.
	\end{flushright}
\end{dem}
	\begin{tcolorbox}[title=Remark,colframe=black,arc=10pt]
\textbf{R1. }If this ratio is not equal (thus: not inverse proportional), then we must switch to other tools such as simple and direct ratio, or regression techniques and verbatim: extrapolation.\\
\textbf{R2.} We also name "\NewTerm{simple or inverse joint rule}\index{simple or inverse joint rule}", a series of direct or inverse rule of three.
	\end{tcolorbox}	

	Basically, it is enough that we knew three of the four variables to solve this simple equation of the first degree.

	In such calculations, the agents of the exchange market have noticed that most of the time the ratio values were close to unity. They were thus naturally led to define the "percentage" as the proportion of a quantity or magnitude relative to another, measured with hundred (at least most of time...). Remember (\SeeChapter{see section Numbers}):

	\begin{itemize}
		\item Given a scalar $x \in \mathbb{R}$ then expressed in percentage it will denoted by:
			
		\item Given a scalar $x \in \mathbb{R}$ then expressed in per-thousand it will denoted by:
			
	\end{itemize}

	\subsection{Vector Spaces}

\textbf{Definition (\#\mydef):} We name "\NewTerm{vector space}\index{vector space}" a set $E$ of elements designated by $\vec{x},\vec{y},...$ and named (as we know) "vectors", with a "vector algebraic structure" defined by the operation of vector addition (and thus vector subtraction) and scalar multiplication. These two operations satisfy the laws of associativity, commutativity, distributivity, neutral element and opposing element as we have already seen in the section of Set Theory.

	For more information about what a vector space set is exactly  the reader will have therefore to refer to the section of Set Theory where this concept is defined more strictly (it would be redundant to repeat it here an anyway it is not crucial because the properties are intuitive).

	For every positive integer $n$, $a_i$ means all the $n$-tuples of numbers arranged in a vector column:
	
	or as line vector (vector column that has been \textbf{T}ranslated):
	
	and $\mathbb{R}^n$ provides clearly a vector space structure. The vectors of this space will be named as we already know: "vectors". They are often denoted more briefly by:
	
	or even more briefly by:
	
	The number $a_i$ is sometimes name "\NewTerm{term}\index{vector term}" or "\NewTerm{component of index $i$}\index{vector component}" of $(a_i)$.

Now, unless stated otherwise, the vectors will always be the elements of a vector space $E$.

	\subsubsection{Linear Combinations}

\textbf{Definition (\#\mydef):} We name "\NewTerm{linear combination}\index{linear combination}" of vectors any vector relation of the form:
	
When a vector can be expressed in the above way we say that the vector is in "\NewTerm{component form}\index{vector component form}".

	The null vector $\vec{0}$ is a linear combination of the $\alpha_1\vec{x}_1+\alpha_2\vec{x}_2+...+\alpha_n\vec{x}_n$ with all coefficients equal to zero. We speak therefore of "\NewTerm{trivial linear combination}\index{vector trivial linear combination}".

	\textbf{Definition (\#\mydef):} We name "\NewTerm{convex combination}\index{convex combination}", any linear combination whose coefficients are non-negative and sum equal to 1. The set of convex combinations of two points $P$ and $Q$ of a punctual space $P_0$ (with an origin) is the line segment $P$ and $Q$. To realize this, we just write:
	
	and we make $\alpha$ vary from 0 to 1 and to find that all the points of the segment are thereby obtained.
	
	If the vector $\vec{v}$ is a linear combination of $\alpha_1\vec{x}_1+\alpha_2\vec{x}_2+...+\alpha_n\vec{x}_n$ and each of these vectors $\vec{x_i}$ is a linear combination of a set of independent vectors $\vec{y}_1,\vec{y}_2,...,\vec{y}_n$, then it could be obvious that $\vec{v}$ is also a linear combination of  $\vec{y}_1,\vec{y}_2,...,\vec{y}_n$.
	
	\textbf{Definition (\#\mydef):} A number $n$ of non-zero vector are "\NewTerm{coplanar}\index{coplanar}" if one of them is a linear combination of the others. For example, three vectors are coplanar if one of them is in the plane defined by the two others.
	
	\subsubsection{Sub-vector spaces}
	
	\textbf{Definition (\#\mydef):} We name "\NewTerm{vectorial subspace $V$ of $E$}\index{vectorial subspace}" any subset of $E$ that is itself a vector space for the operations of addition and scalar multiplication defined in $E$.
	
	A vectorial sub-space $V$, as a vectorial space, can not be empty as it includes at least one vector, i.e. its zero vector, this being also necessarily also the zero vector of $E$. In addition, together with the vectores $\vec{x}$ and $\vec{y}$ (if it contains other vectors than the zero vector), it also includes all their linear combinations $\alpha\vec{x}+\beta\vec{y}$.
	
	Conversely, as soon as we see any subset having these properties is a vectorial subspace. We have thus established the following proposition:
	
	A subset $V$ of $E$ is a subspace of $E$ if and only if $V$ is not empty and $\alpha\vec{x}+\beta\vec{y}$ belongs to $V$ for every pair $(\vec{x},\vec{y})$ of $V$ and all any pair $(\alpha,\beta) \in \mathbb{R}$.
	
	\subsubsection{Generating families}
	
	It follows that if we have a family of vectors $(\vec{x}_1,\vec{x}_2,...,\vec{x}_k)$ the set of linear combinations of $\vec{x}_1,\vec{x}_2,...,\vec{x}_k$ with $k<n$ can be a subspace $S$ of $E$, more specifically the smallest subspace of $E$ consisting of $\vec{x}_1,\vec{x}_2,...,\vec{x}_k$.
	
	The $\vec{x}_1,\vec{x}_2,...,\vec{x}_k$ vectors that satisfy the above condition are named "\NewTerm{generators}\index{generators of a family of vectors}" of $S$ and the family $(\vec{x}_1,\vec{x}_2,...,\vec{x}_k)$ the "\NewTerm{generating family}\index{generating family}" of $S$. We also say that these vectors or family "generate $S$".
	
	\begin{tcolorbox}[title=Remark,colframe=black,arc=10pt]
The subspace generated by a nonzero vector consists of all multiples of this vector. We name such a subspace a "\NewTerm{vector line}\index{vector line}". A subspace generated by two vector non multiple of each other is named a "\NewTerm{vector map}\index{vector map}" or "\NewTerm{vector plane}\index{vector plane}".
	\end{tcolorbox}
	\subsubsection{Linear Dependance or Independance}
	What follows is very important in physics: we advise future physicists or engineer really take the time to read the developments below.
	
	If $(\vec{e}_1,\vec{e}_2,\vec{e}_3)$ are three vectors of $e^3$ whose representatives are not parallel to the same plane (by convention a zero-vector is parallel to any plane), so any vector $\vec{x}$ of $E^3$ can be written by the linear combination:
	
	where $\alpha_1,\alpha_2,\alpha_3$ are typically in $\mathbb{R}$.
	For example the above vector $\vec{x}$ (but can also be obtained for different values of $\alpha_i$!):
	\begin{figure}[H]
		\centering
		\includegraphics[scale=0.75]{img/algebra/vector_linear_combination.jpg}
		\caption{Example of a construction of a vector in a three-dimensional space}
	\end{figure}
	In particular, the only possibility to get the zero vector $\vec{0}$ as a linear combination of $(\vec{e}_1,\vec{e}_2,\vec{e}_3)$ is to assign the trivial value $0$ to the coefficients $\alpha_1,\alpha_2,\alpha_3$.

	Conversely, if for three vectors  $\vec{e}_1,\vec{e}_2,\vec{e}_3$ of $E^3$ the relation:
		
	implies $\alpha_1=\alpha_2=\alpha_3=0$, any vectors may be linear combination of the other two, in other words, their representatives are not parallel to the same plane.
	
	Based on these observations, we will extend the notion of absence of parallelism to a same plane in the case of any number of vectors of a given vector space $E$.
	
	We say that the vectors $\vec{x}_1,\vec{x}_2,...,\vec{x}_k$ are "\NewTerm{linearly independent}\index{linearly independent vectors}" if the relation:
	
	necessarily implies  $\alpha_1=\alpha_2=...=\alpha_k=0$, in other words, if the trivial linear combination is the only linear combination of $\vec{x}_1,\vec{x}_2,...,\vec{x}_k$ which is zero. Otherwise, we say that the vectors $\vec{x}_1,\vec{x}_2,...,\vec{x}_k$ are "\NewTerm{linearly dependent}\index{linearly dependent vectors}".
	
	If the intention is fixed on the family $(\vec{x}_1,\vec{x}_2,...,\vec{x}_k)$ rather than the terms of which it is made, we say that the latter is a "\NewTerm{free family}\index{free family (vector calculus)}" or "\NewTerm{linked family}\index{linked family (vector calculus)}" following that the vectors are linearly independent or dependent.
	
	\subsubsection{Base of a vectorial space}
	\textbf{Definition (\#\mydef):}We say that a family of finite vectors is a basis of $E$ if and only if:
	\begin{enumerate}
		\item If it is free.
		
		\item It generates $E$.
	\end{enumerate}
	Following this definition, every free family $\vec{x}_1,\vec{x}_2,...,\vec{x}_k$ is a basis of the subspace it generates.
	
	\begin{tcolorbox}[colframe=black,colback=white,sharp corners]
	\textbf{{\Large \ding{45}}Example:}\\\\
	If we consider $\mathbb{C}$ as a $\mathbb{R}$-vector space (\SeeChapter{see section Set Theory}), then since all the elements of $\mathbb{C}$ are written $a+\mathrm{i}b$, the elements that generate $\mathbb{C}$ are $1$ and $\mathrm{i}$ (both are free).\\
	
	A base of $\mathbb{C}$ (which is 2-dimensional) as a $\mathbb{R}$-vector space is therefore the free finite set $\left\lbrace 1, i \right\rbrace$.
	\end{tcolorbox}
	For a family of vectors $(\vec{e}_1,\vec{e}_2,...,\vec{e}_n)$ to be a basis of $E$, then it is necessary and sufficient that every vector $\vec{x}$ of $E$ is expressed uniquely as a linear combination of the vectors $(\vec{e}_1,\vec{e}_2,...,\vec{e}_n)$:
	
	The above relation is decomposition of $\vec{x}$ following the base $(\vec{e}_1,\vec{e}_2,...,\vec{e}_n)$ where the coefficients $x_1,x_2,...,x_n$ are the components of $\vec{x}$ in this base. In the presence of a base, each vector is determined entirely by its components.
	
	Proposition:
	
	If $x_1,x_2,...,x_n$ are the components of $\vec{x}$ and $y_1,y_2,...,y_n$ those of equation then: 
	
	are the components of $\vec{x}+\vec{y}$.
	
	In other words, add two vectors is equivalent to add their components and multiply a vector by a scalar obviously equivalent to multiplying its components by the same scalar. The basis is an important tool because it allows you to perform operations on vectors through operations on numbers.
	
	\begin{tcolorbox}[colframe=black,colback=white,sharp corners]
	\textbf{{\Large \ding{45}}Example:}\\\\
	The following column vectors of $\mathbb{R}^n$:
	
	generate a base that we name "\NewTerm{canonical basis}\index{canonical basis}" of $\mathbb{R}^n$ (we will work in complex spaces in another section of this book).
	\end{tcolorbox}
	\begin{tcolorbox}[title=Remark,colframe=black,arc=10pt]
	As part of the three-dimensional space, bases are very often treated as a triad (actually if you connect the ends of the three vectors by features you will get an imaginary triad).
	\end{tcolorbox}	
	
	\pagebreak
	\subsubsection{Direction Angles}
	It is clear that only one standard angle cannot describe the direction of a vector in space. We then use the concept of "\NewTerm{direction angles}\index{direction angles}". This is to measure the angle of the vector $\vec{U}$ with respect to each of the positive axis of the base:
	\begin{figure}[H]
		\centering
		\includegraphics{img/algebra/direction_angles.jpg}
		\caption{Representation of direction angles}
	\end{figure}
	if:
	
	Then by definition:
	
	The values:
	
	are named the "\NewTerm{cosines directions}\index{cosines directions}" of $\vec{x}$.
	
	The three angles mentioned are not completely independent. Indeed, two are enough to completely determine the direction of a vector in space, the third can be deduced from the following equality (obtained from the calculation of the sum of squares of previous relations):
	
	Therefore the direction cosines are the scalar components of a unit standard vector  $\vec{u}$ having the same direction as $\vec{U}$:
	
	
	\subsubsection{Dimensions of a vector space}
	We say that a basis $E$ is of "\NewTerm{finite size}\index{finite size basis}" if it is generated by a finite family of vectors. Otherwise, we say that $E$ is of "\NewTerm{infinite dimension}\index{infinite dimension basis}" (we'll discuss this type of spaces in another section). Any finite dimensional vector space and not reduced to the zero vector has a basis. In fact, from any generating family of such a vector space we can extract a basis.
	
	The dimension of a vector space is denoted by:
	
	Any vector space $E$ of nonzero finite dimension $n$ can be mapped in one to one correspondence (that is to say in bijection) with $\mathbb{R}^n$. We just need to choose a basis of $E$ and to match to any vector $\vec{x}$ of $E$ the column vector whose terms are the components of $\vec{x}$ in the chosen basis (this is  mathematician blah blah but it will be useful when we will discuss more complex spaces):
	
	This correspondence preserves the operations of addition and multiplication by a scalar as we have already seen; in other words, it can perform operations on vectors by operations on numbers.
	
	\begin{tcolorbox}[title=Remark,colframe=black,arc=10pt]
For "classic" resolution methods of such systems, we refer the readers to the section on Numerical Methods of the chapter on Computing Science.
	\end{tcolorbox}	
	Then we say that $E$ and $\mathbb{R}^n$ are "\NewTerm{isomorphic}\index{isomorphic basis}" or that the correspondence is an isomorphism (\SeeChapter{see section Set Theory}).
	
	\subsubsection{Extension of a free family}
	\begin{theorem}
	Given $(\vec{x}_1,\vec{x}_2,...,\vec{x}_k)$ a free family and $(\vec{v}_1,\vec{v}_2,...,\vec{v}_m)$ a generating family of $E$. If $(\vec{x}_1,\vec{x}_2,...,\vec{x}_k)$ is not a basis of $E$, we can extract a subfamily $$(\vec{v}_{i1},\vec{v}_{i2},...,\vec{v}_{il})$$ of $(\vec{v}_1,\vec{v}_2,...,\vec{v}_m)$ so that the family $(\vec{x}_1,\vec{x}_2,...,\vec{x}_k,\vec{v}_{i1},\vec{v}_{i2},...,\vec{v}_{il})$ is a basis of $E$.
	\end{theorem}
	\begin{tcolorbox}[title=Remark,colframe=black,arc=10pt]
	Such a theorem is useful when going from a mathematical space passage having given properties to another space with different mathematical properties.
	\end{tcolorbox}	
	\begin{dem}
	We assume that at least one of the vectors $\vec{v}_i$ is not a linear combination of vectors $(\vec{x}_1,\vec{x}_2,...,\vec{x}_k)$, otherwise $(\vec{x}_1,\vec{x}_2,...,\vec{x}_k)$ would generate $E$ and would therefore be a possible basis of $E$. Let us note that vector $\vec{v}_{i1}$. The family $(\vec{x}_1,\vec{x}_2,...,\vec{x}_k,\vec{v}_{i1})$ is then a free family. Indeed, the relation:
	
	then implies first that $\beta_1=0$, otherwise $\vec{v}_{i1}$ would be a linear combination of the vectors $\vec{x}_1,\vec{x}_2,...,\vec{x}_k$, and then all $\alpha_i=0$ since the vectors $\vec{x}_1,\vec{x}_2,...,\vec{x}_k$ are linearly independent.
	
	If the family $(\vec{x}_1,\vec{x}_2,...,\vec{x}_k,\vec{v}_{i1})$ generates $E$, it is then a possible base for $E$ and the theorem is proved. Otherwise, the same reasoning ensures the existence of another vector $\vec{v}_{i2}$ .... If the new resulting family is not a basis of $E$, then the extraction process vectors $\vec{v}_i$ of $(\vec{v}_1,\vec{v}_2,...,\vec{v}_m)$ continues. When it stops, we will get an "extension" of $(\vec{x}_1,\vec{x}_2,...,\vec{x}_k)$ in a free family generating $E$, that is to say a base of $E$.
	\begin{flushright}
		$\square$  Q.E.D.
	\end{flushright}
	\end{dem}
	It returns a corollary: 

	Every finite dimensional vector space and not reduced to zero vector has a basis! In fact, from any generating family of such a space, we can extract a base.
	
	\subsubsection{Rank of a finite family}
	\textbf{Definition (\#\mydef):} We name "\NewTerm{rank of a family of vectors}\index{rank of a family of vectors}" and denote by $\text{rk}(S)$ the dimension of the subspace $S$ of $E$ it creates.
	
	\begin{theorem}
	The rank of a family of vector $(\vec{x}_1,\vec{x}_2,...,\vec{x}_k)$ is less than or equal to $k$ and is equal to $k$ if and only if the family is free.
	\end{theorem}
	\begin{dem}
	Let us set aside the trivial first case where the rank of the family $(\vec{x}_1,\vec{x}_2,...,\vec{x}_k)$ is zero. By the previous corollary, then we can extract from this family a base of the subspace it generates. The rank is than less or equal to $k$ following that $(\vec{x}_1,\vec{x}_2,...,\vec{x}_k)$ is a linked family or not.
	\begin{flushright}
		$\square$  Q.E.D.
	\end{flushright}
	\end{dem}
	
	\pagebreak
	\subsubsection{Direct Sums}
	\textbf{Definition (\#\mydef):} We say that the sum $S + T$ of two subspaces $S$ and $T$ of $E$ is a "\NewTerm{direct sum}\index{direct sum of subspaces}" if (special case applied to a 2 dimensional space!):
	
	In this case, we note it:
	
	In other words, the sum of two vector subspaces $S$ and $T$ of $E$ is direct if the decomposition of all element $S + T$ into a sum of an element of $S$ and of $T$  is unique.
	
	\begin{tcolorbox}[colframe=black,colback=white,sharp corners]
	\textbf{{\Large \ding{45}}Example:}\\\\
	For example, the $XY$-plane, a two-dimensional vector space, can be thought of as the direct sum of two one-dimensional vector spaces, namely the $X$ and $Y$ axes. In this direct sum, the x and y axes intersect only at the origin (the zero vector). Addition is defined coordinate-wise, that is:
	 
	which is the same as vector addition.
	\end{tcolorbox}	
	This concept of trivial decomposition will be very useful in some theorems, the most important in this book is definitely the spectral theorem (\SeeChapter{see section of Linear Algebra}) that has important implications in statistics!!!
	
	From the direct sum we can introduce the concept of "\NewTerm{complementary subspace}\index{complementary subspace}" also named "\NewTerm{subspace}\index{subspace}" (depending on countries ...):
	\begin{theorem}
	Suppose that $E$ is of finite dimensions. For any subspace $S$ of $E$, there exists a subspace $T$ (not unique) of $E$ such that $E$ is the direct sum of $S$ and $T$. We say then that $T$ is a "\NewTerm{supplementary subspace}\index{supplementary subspace}" of $S$ into $E$.
	\end{theorem}
	\begin{dem}
	First let us set aside the trivial case where $S=\left\lbrace \vec{0} \right\rbrace$  and $S = E$. The subspace $S$ admits a basis $(\vec{e}_1,\vec{e}_2,...,\vec{e}_k)$, where $k$ is less than the dimension $n$ of $E$. By the theorem of extension of a free family, this basis can be extended in a basis $(\vec{e}_1,\vec{e}_2,...,\vec{e}_k,\vec{e}_{k+1},...,\vec{e}_n)$ of $E$. Let $T$ be the subspace vector generated by the family $(\vec{e}_{k+1},...,\vec{e}_n)$ . If $\vec{x}$ is any vector of $E$, then $\vec{x}=\vec{s}+\vec{t}$, where $\vec{s}$ is a vector of $S$ and $\vec{t}$ a vector of $T$. In addition $S\cap T=\left\lbrace \vec{0} \right\rbrace$, because no vector, excepted the zero vector may be a linear combination of the vectors $\vec{e}_1,...,\vec{e}_k$ and of the vectors $\vec{e}_{k+1},...,\vec{e}_n$. We therefore conclude that:
	\begin{flushright}
		$\square$  Q.E.D.
	\end{flushright}
	\end{dem}
	
	\subsubsection{Affine spaces}
	In mathematics, an affine space $G=\text{A}\mathbb{R}^n$ is a geometric structure that are independent of the concepts of distance and measure of angles, keeping only the properties related to parallelism and ratio of lengths for parallel line segments as their is not origin point $\text{=}(0,0)$.
	
	The space $G$ of elementary geometry is both common and the source of the concept of "affine space" that we will introduce because when high-school student begins learn geometry they learn it without any reference point $\text{=}(0,0)$.

	In an affine space, there is therefore no distinguished point that serves as an origin. Hence, no vector has a fixed origin and no vector can be uniquely associated to a point. In an affine space, there are instead "\NewTerm{displacement vectors}\index{displacement vectors}", also named "\NewTerm{translation vectors}\index{translation vectors}" or simply translations, between two points of the space.
	
	This space $G$ is associated with the "\NewTerm{geometric vector space}\index{geometric vector space}" $V$ by the correspondence between vectors and arrows studied so far! The following definition is only to highlight the main common points of this correspondence:
	
	\textbf{Definition (\#\mydef):} Let $G$ be a non-empty set of elements that we name "\NewTerm{points}\index{points in a vector space}" and let us  denote them by the letters $P, Q, ...$; given also $E$ a vector space. Suppose that to any two points $(P, Q)$ corresponds a vector denoted $\overrightarrow{PQ}$ (typically the point $P$ is choosen as fictive origin). We say then that $U$ is an "\NewTerm{affine space}\index{affine space}" of directed space $E$ if the following conditions are met:
	\begin{enumerate}
		\item[C1.] For any fixed point $P$, the correspondence between pairs $(P, Q)$ and and vectors $\vec{x}$ defined by only one point plus the origina point is bijective, ie, for every vector $\vec{x}$ it exists a point $Q$ such that we can define a vector $\overrightarrow{PQ}$.
		
		\item[C2.] For each triple of points $(P, Q, R)$:
		
		This is the famous "\NewTerm{Chasles relation}\index{chasles relation}" (which we will see later a pseudo-equivalent in the section of Differential and Integral Calculus).
		
		\item[C3.] If $P$ is a point and a $\vec{x}$ vector, to express that $Q$ is the unique point such as $\vec{x}=\overrightarrow{•}{PQ}$, we write:
		
		Although being a bit excessive, this writing is consistent with the usage and suggests well the idea of the operation it designates.
	\end{enumerate}
	\begin{tcolorbox}[colframe=black,colback=white,sharp corners]
	\textbf{{\Large \ding{45}}Example:}
	Below an "artistic" example of an affine space $G=\text{A}\mathbb{R}^2$ where there is no origin and any extremity of a line can be considered as the origin of a vector:
	\begin{figure}[H]
		\centering
		\includegraphics{img/algebra/affine_space.jpg}
		\caption{Artistic but real example of an $G=\text{A}\mathbb{R}^2$ affine space}
	\end{figure}
	\end{tcolorbox}
	
	The following properties follow directly from the definition of affine space:
	\begin{enumerate}
		\item[P1.] For any point $P$, $P+(\vec{x}+\vec{y})=(P+\vec{x})+\vec{y}$
		
		\item[P2.] For any point $P$, $\overrightarrow{PP}=\vec{0}$. This results from the $\overrightarrow{PQ}+\overrightarrow{QR}=\overrightarrow{PR}$ provided in the case where we have $P=Q=R$.
		
		\item[P3.] $\overrightarrow{PQ}=-\overrightarrow{QP}$. Just put $R = P$ in the De Chasles relation $\overrightarrow{PQ}+\overrightarrow{QR}=\overrightarrow{PR}$.
		
		\item[P4.] Parallelogram rule:
		Given the polygon with the vertices (clockwise) $P,P',Q,Q'$ and edges $\overrightarrow{PP'},\overrightarrow{P'Q'},\overrightarrow{QQ'},\overrightarrow{PQ}$:
		\begin{figure}[H]
			\centering
			\includegraphics{img/algebra/affine_parallelogram.jpg}
			\caption{Vector polygon in $\text{A}\mathbb{R}^2$}
		\end{figure}
		We have:
		
		if and only if:
		
		which would then give a parallelogram!
	
		Indeed, replacing $R$ with $Q'$ in the Chasles relation we have:
		
		and by doing the same but replacing $R$ with $Q$ and $Q$ by $P'$ we get:
		
		We then have by equalizing the last two relations:
		
	\end{enumerate}
	Earlier we saw what made that a space $G$ could be provided with a vector space structure (we saw then that it was therefore "vectorialized"). In the general case of an affine space $G$, the process is the same:
	
	We choose any point $\text{O}$ of $G$. The correspondence between pairs $(\text{O},P)$ and vectors of director space $E$ being therefore biunivocal we define then the addition of points and multiplication of a point by a scalar by the corresponding operations on the vectors of $E$. Armed with these two operations, $G$ becomes a vector space, named "\NewTerm{vectorialized space $G$ regarding to $Q$}\index{vectorialized space}". We denote this space by $V$ and named the point $\text{O}$ "\NewTerm{origin}\index{origin of a vector space}".
	
	Given how operations have been defined, it follows that $V$ is isomorphic to the space vector $E$:
	
	However, this isomorphism depends on the choice of the origin $\text{O}$ and in practice this origin is selected on the basis of the data inherent to the studied problem. For example, if an affine transformation allows an invariant point (which does not move), it is advantageous to select that point as the origin.
	\begin{tcolorbox}[title=Remarks,colframe=black,arc=10pt]
	\textbf{R1.} When we talk about dimension of an affine space, we talk about the size of its director space.\\
	
	\textbf{R2.} The space $G$ of elementary geometry is an affine space of type $\text{A}\mathbb{R}^2$. Indeed, its direction is the geometrical space $G$ and the conditions of definition of affine space are met.\\
	
	\textbf{R3.} An affine space is a set of elements with a difference function. This difference is a binary function, which takes two points $p$ and $q$ (both in $G$) and yields an element (a vector) $\vec{v}$ of a vector space $E$. We write $\vec{v}=p-q.$ Additionally, this difference function must ensure that, for any point $p$ in $E$, it holds$p-p=0$, where $\vec{0}$ is the null vector of $E$.
	\end{tcolorbox}
	
	\pagebreak
	\subsection{Euclidean Vectore Spaces}
	Before defining what is an Euclidean vector space, let us first define some mathematical tools and some concepts.
	
	We can, by choosing a unit length, measure the intensity of each arrow, in other words, determine its length. We can also measure the angular distance of two arrows (or vectors) of any common origin (not necessarily distinct) taking as the unit of angle measurement for example the radian (\SeeChapter{see section Trigonometry}). The measurement of this difference is then a number between $0$ and $\pi$ named "\NewTerm{angle}\index{angle between two vector}" of the two arrows (see the section of Euclidiean Geometry for more details). If both arrows have same direction and orientation, their angle is zero and if they have same direction and the opposite orientation, this same angle is $\pi$.
	
	The representing arrows of a same vector $\vec{x}$ all have the same length. We denote this length by the notation:
	
	\begin{figure}[H]
		\centering
		\includegraphics{img/algebra/details_norm_calculation.jpg}
		\caption{Details of the calculation of the norm in an orthogonal coordinate system $\mathbb{R}^3$}
	\end{figure}
	If $\vec{x}$ is a nonzero vector we can build an unit norm vector $\vec{u}$ of same direction and orientation (colinear) by the following operation that is used a lot in physics:
	
	We will name "\NewTerm{non-zero angle of vectors $\vec{x}$ and $\vec{y}$}" the angle of two arrows of common origin representing one being $\vec{x}$ and the other $\vec{y}$.
	
	However, more strictly speaking a "norm" is defined on a real vector space (or complex) $E$, so that we speak then of "\NewTerm{normalized vector space}\index{normalized vector space}" is an application:
	
	satisfying the following properties:
	\begin{enumerate}
		\item[P1.] Positivity:
		
		\item[P2.] Linearity:
		
		where we take the modulus of the constant if this is not in the set of real numbers $\mathbb{R}$ but in the set of complex numbers $\mathbb{C}$.
		\item[P3.] Nullity (often associated with the property P1):
		
		\item[P4.] Minkowski inequality (triangle inequality):
		
		That we will prove further below.
	\end{enumerate}
	\begin{tcolorbox}[title=Remark,colframe=black,arc=10pt]
	\textbf{R1.} These properties are mainly imposed by our intuitive approach of Euclidean space (vector space of finite dimension over the field of real number $\mathbb{R}$ and with a scalar product that we will see later) and its geometric interpretation (through the fact that it is also an affine space $\text{A}\mathbb{R}^n$).\\
	
	\textbf{R2.} We will prove a little further below the property P4 under the name of "triangle inequality" and we will do a little more general study of this inequality under the name "Minkowski inequality" in the section Topology.
	\end{tcolorbox}	
	
	\pagebreak	
	\subsubsection{Scalar Product (Dot Product)}
	\textbf{Definition (\#\mydef):} An "\NewTerm{Euclidean vector space}\index{Euclidean vector space}" is a vector space (real and of finite dimensional for the purists) with a specific operation, the "\NewTerm{scalar product}\index{scalar product}" also named "\NewTerm{dot product}\index{dot product}" which we denote (notation specific to this website) regarding to the special case of vectors:
	
	\begin{tcolorbox}[title=Remarks,colframe=black,arc=10pt]
	\textbf{R1.} We find in some books (for information) the notation $\left( \vec{x}|\vec{y}\right)$ or $\langle \vec{x} | \vec{y} \rangle$ in the generalization of this definition as we shall see a little further below. According to the standard ISO 80000-2: 2009 we should write the dot product as $a\cdot b$.\\
	
	\textbf{R2.} The scalar product has a huge importance in the whole field of mathematics and physics; and we will see it again in all following chapters of this book. It is therefore necessary to carefully understand what follows.\\
	
	\textbf{R3.} The scalar product may be viewed as a projection of the length of a vector along the length of another one as we will see later.
	\end{tcolorbox}	
	This scalar product has the following properties (most of which stem from the definition itself) in a Euclidean space:
	\begin{enumerate}
		\item[P1.] Commutativity: $\vec{x}\circ\vec{y}=\vec{y}\circ\vec{x}$
		\item[P2.] Associativity: $\alpha(\vec{x}\circ\vec{y})=(\alpha\vec{x})\circ\vec{y}=\vec{x}\circ(\alpha\vec{y}) $
		\item[P3.] Distributivity: $\vec{x}\circ(\vec{y}+\vec{z})=\vec{x}\circ\vec{y}+\vec{x}\circ\vec{z}$
		\item[P4.] Non-degerated: $\vec{x}\circ\vec{y}=0$ then $\forall\vec{x}\neq\vec{0}\Rightarrow \vec{y}=\vec{0}$
		\item[P5.] Squared scalar: $||\vec{x}||^2=\vec{x}\circ\vec{x}$ and $\vec{x}\circ\vec{x}>0$ if $\vec{x}\neq \vec{0}$
		\item[P6.] Bi-linearity: $(\alpha\vec{x}+\beta\vec{y})\circ\vec{z}=\alpha(\vec{x}\circ	\vec{z})+\beta(\vec{y}\circ\vec{z})$
	\end{enumerate}
	Only the latter property requires perhaps a proof (and one of the results of the proof  will be useful to us later to prove another very important property of the scalar product):
	\begin{dem}
	Given:
	
	which is the "\NewTerm{orthogonal projection vector}\index{orthogonal projection vector}" (the $x$ on index of $\text{project}$ meaning "the vector $\vec{x}$") of the vector $\vec{y}$ of standardization at the unit of vector $\vec{x}$.
	
	Using the scalar product, the vector $\text{proj}_x\vec{y}$ can be expressed otherwise, we just need to take the relation that we have seen above:
	\begin{figure}[H]
		\centering
		\includegraphics{img/algebra/dot_product.jpg}
		\caption{Geometrical representation of dot product (projection)}
	\end{figure}
	
	and introduce it into $\text{proj}_x\vec{y}$ with to obtain:
	
	The norm of $\text{proj}_x\vec{y}$ is written:
	
	If $\vec{x}$ has a unit norm, the relations of previous projections are simplified and become obviously:
	
	By elementary geometric considerations (distributivity of the scalar product), it is easy to realize that:
	
	If we now return to the proof of the bi-linearity property:
	
	We have in a first time:
	
	and, from the definition of the property of the orthogonal projection, it comes immediately by a one-to-one correspondence:
	
	hence the property $P6$ that follows by multiplying the two members of equality by $\vec{z}\circ\vec{z}$ and after by simplification by $\vec{z}$.
	\begin{flushright}
		$\square$  Q.E.D.
	\end{flushright}
	\end{dem}
	\begin{enumerate}
		\item[D1.] A vector space $E$ is said to be an "\NewTerm{proper Euclidean vector space}\index{proper Euclidean vector space}" if $\forall \vec{x} \in E \qquad ||\vec{x}||>0$.
		
		\item[D2.] We say that the vectors $\vec{x}$ and $\vec{y}$ are "\NewTerm{orthogonal vectors}\index{orthogonal vectors}" if they are non-null and that their scalar product is equal to zero (their angle is equal to $\pi/2$).
		
		\item[D3.] A basis of vectors $(\vec{e}_1,\vec{e}_2,...,\vec{e_n})$ is said to be an "\NewTerm{orthonormal basis}\index{orthonormal basis}" if all the vectors $\vec{e}_1,\vec{e}_2,...,\vec{e_n}$ are pairwise orthogonal and their norm is equal to the unit (thus constituting a: free family).
	\end{enumerate}
	\begin{tcolorbox}[title=Remark,colframe=black,arc=10pt]
	We will see in the section Tensor Calculus (we could have done here too but we don't use it for a practical case in this section therefore...) how from a set of independent vectors build an orthogonal basis. This is what the reader will find under the name  "(Gram-)Schmidt orthogonalization method".
	\end{tcolorbox}	
	By a a simple geometric argument, we see that every vector is the sum of its orthogonal projections on the vectors of an orthonormal basis, that is, if $(\vec{e}_1,\vec{e}_2,\vec{e}_2)=(\vec{u},\vec{v},\vec{w})$ is an orthonormal basis in $\mathbb{R}^3$ for example:
	
	\begin{figure}[H]
		\centering
		\includegraphics{img/algebra/orthonormal_basis_projection.jpg}
		\caption{Orthonormale basis projection example}
	\end{figure}
	This decomposition is also obtained by the P6 property of the scalar product. Indeed, consider the components $(\vec{x}_1,\vec{x_2},\vec{x}_3)$ of a vector $\vec{x}$ in our orthonormal basis:
	
	since $\vec{e}_1\circ\vec{e}_1=1$ and $\vec{e}_1\circ\vec{e}_2=0$. Therefore we get immediately:
	
	hence the decomposition.
	
	Given the respective components $(\vec{x}_1,\vec{x_2},\vec{x}_3)$ and $(\vec{y}_1,\vec{y_2},\vec{y}_3)$ of the vectors  $\vec{x}$ and $\vec{y}$ vectors in a canonical orthonormal basis $(\vec{e}_1,\vec{e_2},\vec{e}_3)$ we know now that we can write the scalar product in the form:
	
	by the property P6 of the scalar product:
	
	using the properties P1 and P6 again:
	
	Which finally gives us the very famous and important decomposition:
	
	This is one of the most important relation in the field of vector calculus, which we name "\NewTerm{canonical scalar product}\index{canonical scalar product}" or "\NewTerm{canonical dot product}\index{canonical dot product}".
	
	\begin{tcolorbox}[title=Remark,colframe=black,arc=10pt]
	Then angle $\theta$ of the dot product is sometimes denoted by:
	
	\end{tcolorbox}
	Now let us prove with a simple two dimensional case a property that physicist like a lot to characterize an orthogonal linear application (\SeeChapter{see section Linear Algebra}): that the dot product is invariant under any orthogonal transformation (then abusively said "invariant under basis change...").

	For this let us consider a vector $\vec{x}=(x_1,x_2)$ and the 2D rotation matrix (\SeeChapter{see section Numbers}):
	
	Now let us calculate:
	
	So now let us consider two vectors $\vec{a}$ and $\vec{b}$ we have proved just above that their dot product is given by:
	
	And after the chosen orthogonal transformation we get:
	
	So the dot product is indeed invariant under this rotation that is a special 2D case of orthogonal transformation  and in fact under any other orthogonal transformation.		
	
	\pagebreak
	\paragraph{Cauchy–Schwarz inequality}\mbox{}\\\\
	In mathematics, the Cauchy–Schwarz inequality is a useful inequality encountered in many different settings, such as linear algebra, analysis, probabilities, statistics and other areas (just read this book entirely to have an idea...). It is considered to be one of the most important inequalities in all of mathematics!!!
	
	The relation
	
	 can also be trivially written as follows if we use the concept of the norm and the definition of the scalar product:
	 
	 It is interesting to notice that if both $\vec{x}$ and $\vec{y}$ are orthogonal vectors, we fall back on the result of a famous theorem: the Pythagorean theorem!

	Indeed, therefore we have if the two vectors are orthogonal:
	 
		This gives us:
	 
	This relations is very important in physics and mathematics. It must be remembered!
	 
	\begin{theorem}
	We name "\NewTerm{Cauchy-Schwarz inequality}\index{Cauchy-Schwarz inequality}", the inequality, valid for any choice of the vectors $\vec{x}$ and $\vec{y}$, the relation:
	
	Which can also be written as:
	
	\end{theorem}
	First we will consider as obvious that equality only occurs when the two vectors are collinear.	
	\begin{dem}
	We put ourself in the case where $\vec{x},\vec{y}\neq\vec{0}$. So then $\lambda\in\mathrm{R}$ we have obviously according to the properties of the scalar product:
	
	So this is a simple equation of the second degree where variable is $\lambda$. Remembering what we saw in our study of polynomials of second degree (see section Calculus), the previous relation (that it is always greater than or equal to zero) is satisfied if the discriminant $b^2-4ac$ is negative or zero. In other words, if:
	
	Thus after simplification:
	
	\begin{flushright}
		$\square$  Q.E.D.
	\end{flushright}
	\end{dem}
	When $E$ is $\mathbb{R}^n$, the Cauchy-Schwarz inequality is writtent with the vector components:
	
	In the particular case where $\forall i,b_i=1$ it becomes:
	
	or even:
	
	which shows that the square of the arithmetic mean is less than or equal to the arithmetic mean of the squares. This result is important for the study of Statistics!
	
	Furthermore, using the property of the cosine and the Cauchy-Schwarz inequality we can write immediately:
	
	relation that we will see again in the context of the study of Statistics (\SeeChapter{see section Statistics}).
	
	\paragraph{Triangular Inequalities}\mbox{}\\\\
	By majoring $2\vec{x}\circ\vec{y}$ by $2||\vec{x}||\cdot||\vec{y}||$ (using the Cauchy-Schwarz inequality!) in the relation already establish previously:
	
	we get:
	
	which take us immediately by taking the root square the "\NewTerm{triangular inequality}\index{triangular inequality}" that is very useful for the study of Sequences and Series and also in Topology:
	
	\begin{tcolorbox}[title=Remark,colframe=black,arc=10pt]
	The generalization of this inequality relatively to the choice of the norm (that is to say: the way we define a distance) as we will see in the section of Topology, gives what we name the "\NewTerm{Minkowski inequality}\index{Minkowski inequality}".
	\end{tcolorbox}	
	By applying one thime the triangular inequality to the vectors $\vec{x}$ and $(\vec{y}-\vec{x})$ and another time to vectors $(\vec{y})$ and $(\vec{x}-\vec{y})$ we get the variant:
	
	
	\paragraph{General Scalar/Dot Product}\mbox{}\\\\
	Let us see now another and little more general, formal and abstract way to define the dot product while trying to stay as simple as possible (caution! in the general case the notation of the scalar product changes! ).
	
	First the reader must know that, as we will see it in the section of Functional Analysis, all the concepts studied until now in this section can also be applied to a special category of functions! Yeeesss!!! Add, subtract functins like vector is obvious but you must know that some more or less complicated functions are colinear or orthogonal (think to affine functions!) and furthermore there are not limite to $\mathbb{R}$ but can be extended to $\mathbb{C}$ easily and hence the scalar product departure set. 
	
	To make thinks to too much complicate we will focus here only on a gentle generalization of the dot product to vectors (we will come back on functions in the section of Functional Analysis later).
	
	\textbf{Definition (\#\mydef):} Let $E$ be a real vector space (once again we focus here only on simple vectors for the moment). A "\NewTerm{positive symmetric bilinear form}\index{positive symmetric bilinear form}" on $E$ also named "\NewTerm{inner product}\index{inner product}", is an application:
	
	\begin{enumerate}
		\item[P1.] Positivity: 
		
	  	
	  	\item[P2.] Nullity (defined): 
	  	 
	  	
	  	\item[P3.] Symetry (defined): 
	  	
	  	
	  	\item[P4.] The bilinearity (bilinear form) with, in order, the "\NewTerm{linearity on the left}\index{linearity on the left}" and "\NewTerm{linearity on the right}\index{linearity on the right}":
	  	
		
		\item[P...] And so on... we have the same six properties as the scalare product as the both are the same if we focus only on vector in $\mathbb{R}$.
	\end{enumerate}
	\begin{tcolorbox}[title=Remark,colframe=black,arc=10pt]
	Again, these properties are mainly imposed by our intuitive approach of the Euclidean space and its geometric interpretation.
	\end{tcolorbox}	
	
	\textbf{Definition (\#\mydef):} A space $E$ provided with a scalar product is named in general (with the departure set in $\mathbb{C}$) a "\NewTerm{pre-Hilbert space}\index{pre-Hilbert space}" or "\NewTerm{inner product space}\index{inner product space}". If $E$ is of finite dimension, then we speak of "Euclidean space".
	
	We will see in our study of Topology (see section of the same name) that the properties of the scalar product are the foundation bricks to set a norm and therefore a distance in a metric space. This distance will be given according to what we will see in the section of Topology:
	
	
	\textbf{Definition (\#\mydef):} We say that a space $E$ having a dot product (inner product) $ \langle \cdot | \cdot \rangle$ is a "\NewTerm{Hilbert space}\index{Hilbert space}" if this space is complete for the metric defined above.
	
	In other words, having a metric space provided with a distance generated by a scalar product is one thing. Then having a measurable distance is another one!!! A Hilbert space has thus distances measurable in the topological sense because the set we are working on is continuous and any point can be approached indefinitely (imagine having a rule and you can not approach ont this rule the points that define the dimensions of your object... it would be embarrassing...). So without complete space a lot of theorems of functional analysis (that is strongly linked to vector calculus) could not be used in the study of vector spaces and this would be very embarrassing in quantum wave physics for example...
	
	Formally, remember that a metric space is complete if all Cauchy sequences (\SeeChapter{see section Sequences and Series}) of this space are converging (\SeeChapter{see section Fractals}) in a metric space (\SeeChapter{see section Topology}).
	
	\subsubsection{Cross Product}
	The cross product of two vectors is a proper operation to the dimension $3$. To introduce it, it is first necessary to orient the space intended to receive it. The orientation is defined by the concept of "determinant", therefore we will begin with a brief introduction to the study of this concept. This study will be repeated later in more details in the analysis of linear systems in the section of Linear Algebra.
	
	\textbf{Definition (\#\mydef):} We name basically "\NewTerm{determinant}\index{determinant}" of two column vectors of $\mathbb{R}^2$ (for the general form of the determinant see the section of Linear Algebra):
	
	and we denote it:
	
	the number:
	
	We name determinant of three column vectors of $\mathbb{R}^2$ (once again see the section Linear Algebra for a generalization):
	
	and we denote it:
	
	the number:
	
	Thus, the function that associates to each pair of column vectors of $\mathbb{R}^2$ (or respectively to each triplet of column vectors of $\mathbb{R}^3$) has a determinant named "determinant of order $2$" (respectively "determinant of order $3$")
	
	As we will prove it in the section of Linear Algebra the determining has the property of being multiplied by $-1$ if one of the column vectors is replaced by its opposite or two of its column vectors are exchanged. In addition, the determinant is nonzero if and only if its column vectors are linearly independent (the proof - that has a great importantce in Applied Mathematics - is a few lines further below and a generalization  can be found in the section of Linear Algebra).
	
	\textbf{Definition (\#\mydef):} Given $x_1,x_2,x_3$ and $y_1,y_2,y_3$the respective components of the vectors $\vec{x}$ and  $\vec{y}$ in the orthonormal basis $(\vec{e}_1,\vec{e}_2,\vec{e}_3)$. We name "\NewTerm{cross-product}\index{vector cross-product}" of $\vec{x}$ and $\vec{y}$, and we denote it in most books by:
	
	and in a minority of books:
	
	the vector:
	
	or as components:
	
	The matrix form above will be very useful to us in the section Mechanics for the construction of the Inertial Matrix.
	\begin{tcolorbox}[title=Remark,colframe=black,arc=10pt]
	\textbf{R1.} The first notation is the international notation due to Gibbs (which we will use throughout this book), the second is the French notation due to Burali-Forti (quite annoying because confusing with the notation of the operator AND in Proof Theory or Logical Systems).\\
	
	\textbf{R2.} It is usually quite annoying to remember by heart the relations that form the cross product. But fortunately there are at least three good mnemonics techniques:
	\begin{enumerate}

		\item The first and probably the fastest method is to remember by heart only one of the expressions of the components of the cross product and after by decrement of the indices (by starting again from $3$ when we reaches $0$) get all the other components. But we must still find a simple way to remember by heart one of the components... A good way is the following mathematical property of two collinear vectors giving an easy with to find back the third component (the one along the $z$-axis):
		
		Given two colinear vectors in the same plane, then:
		
		We fall back on the expression of the third component of the cross product of two vectors.\\
		
		Or if you want to remember only the first component given in letters by $z_x = x_y y_z - x_z y_y$ the indices gives "xyzzy" (like a name of a person...). The second and third equations can be obtained from the first by simply vertically rotating the subscripts, $x\rightarrow y \rightarrow  z \rightarrow x$. 
		
		\item The second method that we will see in details during our study of the section of Tensor Calculus is to use the Levi-Civita anti-symmetry symbol. This method is certainly the most aesthetic of all but not necessarily the fastest to develop and the easiest to remember. We give here just the expression without explanations at the moment as we will study this later (but t is also useful to get the general expression of the determinant):
		
		
		\item The latter method is quite simple and trivial but it implicitly uses the first method as you must remember how the calculate at least a $2\times 2$ determinant. The idea is the following: the $i$-th component of $\vec{x}\times\vec{y}$ is the determinant of the two column vectors from which we have removed the $i$-th term, the second determinant is, however, with a "$-$" sign such that:
			  
	\end{enumerate}
	\end{tcolorbox}
	
	\pagebreak
	It is important, even if it is relatively simple to remember, that the different cross products for orthogonal basis vectors are (and especially for the canonical basis) first:
	
	and also:
	
	The vector product also has the following properties that we will prove just now:
	  \begin{enumerate}
	  	\item[P1.] Antisymmetry:
	  		
	  		
	  	\item[P2.] Linearity:
	  		
	  		
	  	\item[P3.] If and only if $\vec{x}$ and $\vec{y}$ are linearly independent (very important!):
	  	
	  	
	  	\item[P4.] Non associativity:
	  	
	  	
	  	\item[P4.] Distributivity over the sum:
	  	
	  \end{enumerate}
	  The first two properties directly derived from the definition and the property P4 is easily verified by developing the components and comparing the results.
	  
	 Then let us prove the third property which is very important in linear algebra (next section) and the fifth one (because requested by a reader).
	\begin{theorem}
	If and only if $\vec{x}$ and $\vec{y}$ are linearly independent (very important!):
	
	\end{theorem}
	\begin{dem}
	Given two vectors $\vec{x}(x_1,x_2,x_3)$ and $\vec{y}(y_1,y_2,y_3)$. If the two vectors are linearly dependent then there exists an $\alpha \in \mathbb{R}$ such that we can write:
	
	If we develop the cross product of two vectors that a dependent to a given factor, we get:
	
	It goes without saying that the above result is equal to the zero vector $\vec{0}$ if indeed the two vectors are linearly dependent.
	\begin{flushright}
		$\square$  Q.E.D.
	\end{flushright}
	\end{dem}
	
	\begin{theorem}
	The cross product is distributive over the sum.
	\end{theorem}
	\begin{dem}
	
	\begin{flushright}
		$\square$  Q.E.D.
	\end{flushright}
	\end{dem}
	
	If we now assume that both vector $\vec{x}$ and $\vec{y}$ are linearly independent and non-zero vector, we must prove that the cross product two properties:
	\begin{theorem}
	The resulting of a cross product results in a vector orthogonal (perpendicular) to $\vec{x}$ and $\vec{y}$ if they are not null.
	\end{theorem}
	\begin{dem}
	To prove this we simply write the development using the dot product:
	
	This equation shows that the vector $\vec{x}$ is perpendicular to the resulting vector of cross product between $\vec{x}$ and $\vec{y}$.
	\begin{flushright}
		$\square$  Q.E.D.
	\end{flushright}
	\end{dem}
	\begin{theorem}
	The cross product has for norm (module):
	
	where $\theta$ is the angle between $\vec{x}$ and $\vec{y}$.
	\end{theorem}
	\begin{dem}
		To prove this we simply write the development of the norm of the cross product:
		
		Finally:
		
	\begin{flushright}
		$\square$  Q.E.D.
	\end{flushright}
	\end{dem}
	We then notice that in the case where $E$ is the Euclidean vector space, the norm of the vector product is the area (surface) of the parallelogram constructed on representatives of vectors $\vec{x}$ and $\vec{y}$ of common origin:
	\begin{figure}[H]
		\centering
		\includegraphics{img/algebra/cross_product_parallelogram.jpg}
		\caption{Geometrical representation of cross product}
	\end{figure}
	If $\vec{x}$ and $\vec{y}$ are linearly independent, the triplet $(\vec{x},\vec{y},\vec{x}\times \vec{y})$ and also the triplet $(\vec{x},\vec{y},\vec{y}\times \vec{x})$ are direct.
	
	Indeed, $(\vec{z}_1,\vec{z}_2,\vec{z}_3)$ being the components of $\vec{x}\times \vec{y}$ (in the basis $(\vec{e}_1,\vec{e}_2,\vec{e}_3)$), the determinant of passage of $(\vec{e}_1,\vec{e}_2,\vec{e}_3)$ to $(\vec{x}\times \vec{y},\vec{x},\vec{y}$ (form example) will be written:
	 
	 This determinant is positive, as at least one of the $z_i$ is not zero, according to the third property of linear independence of the cross product.
	 
	 Here are a few very important properties of practical utility of the cross product (particularly in physics) that are trivial to check whether the developments with explicit components are done (we can make them on  request if needed!):
	\begin{enumerate}
		\item[P1.] $\vec{x}\times(\vec{y}\times\vec{z})=(\vec{x}\circ\vec{z})\vec{y}-(\vec{x}\circ\vec{y})\vec{z}$
		\begin{tcolorbox}[title=Remark,colframe=black,arc=10pt]
		The latter relation is sometimes named the "\NewTerm{Grassman rule}\index{Grassman rule}", or more commonly "\NewTerm{dual vector product}\index{dual vector product}" and it is important to note that without the parentheses the result is not unique.
		\end{tcolorbox}	
		
		\item[P2.] $(\vec{x}\times\vec{y})\circ (\vec{z}\times\vec{v})=(\vec{x}\circ\vec{z})(\vec{y}\circ\vec{v})-(\vec{x}\circ\vec{v})(\vec{y}\circ\vec{z})$
		
		\item[P3.] $(\vec{x}\times\vec{y})\circ\vec{z}=-(\vec{x}\times\vec{z})\circ \vec{y}$
		
		\item[P4.] $(\vec{x}\times\vec{y})\circ\vec{z}=\vec{x}\circ(\vec{y}\times\vec{z})$
		
		\item[P5.] $||\vec{x}\times\vec{y}||^2=(\vec{x}\circ\vec{x})^2(\vec{y}\circ\vec{y})^2-(\vec{x}\circ\vec{y})^2$
	\end{enumerate}
	The last identity is related to the Pythagorean theorem (\SeeChapter{see section Euclidean Geometry}). Indeed, we wee it better by rewriting:
	
	This can be seen from the definitions of the cross product and dot product, as
	
	
	\pagebreak
	\subsubsection{Mixed Product (triple product)}
	We can extend the definition of the vector product to another type of mathematical tool we name the "\NewTerm{mixed product}".
	
	\textbf{Definition (\#\mydef):} We name "\NewTerm{mixed product}\index{mixed product}" of vectors $\vec{x},\vec{y},\vec{z}$ the double product:
	 
	 often condensed under the following notation:
	  
	 From what we saw in the definition of the dot and cross product, mixed product can also be written:
	 
	 We note that in the case where $E$ is the Euclidean vector space $\mathbb{R}^3$, the absolute value of the mixed product symbolize the oriented volume of the parallelepiped, built on the representatives $\vec{x},\vec{y},\vec{z}$ of common origin.
	 
	It is quite trivial that the mixed product is an extension to the three-dimensional case of the cross product. Indeed, in the expression of the mixed product, the vector product is the base surface of the parallelepiped and the scalar product project the vectors on the resulting vector from the cross product which gives the height $h$ of the parallelepiped.
	
	\begin{figure}[H]
		\centering
		\includegraphics{img/algebra/mixed_product.jpg}
		\caption{Mixed product illustration}		
	\end{figure}
			
	If we develop:
	
	so triple product can also be understood as the determinant of a $3\times 3$ matrix (thus also its inverse) having the three vectors either as its rows or its columns (\SeeChapter{see section Linear Algebra}):
	
	 By the commutative properties of the scalar product, we have:
	 
	and the reader will check without any trouble (we can write the details on request) that by developing components we have:
	
	The triple product has also the following properties that the reader could be able to check easily by developing just the components of each expression excepted perhaps the third one (we can detailed as always on request if needed):
	\begin{enumerate}
		\item[P1.]  $[\vec{x},\vec{y},\vec{z}] =
	   [\vec{z},\vec{x},\vec{y}] =
	   [\vec{y},\vec{z},\vec{x}] =
	  -[\vec{y},\vec{x},\vec{z}] =
	  -[\vec{z},\vec{y},\vec{x}] =
	  -[\vec{x},\vec{z},\vec{y}]$
	  
	  \item[P2.] $[\alpha\vec{x}+\beta\vec{y},\vec{z},\vec{v}] =
	  \alpha[\vec{x},\vec{z},\vec{v}] +
	    \beta[\vec{y},\vec{z},\vec{v}]$
	    
	  \item[P3.] $[\vec{x},\vec{y},\vec{z}] \ne 0$ if and only if $\vec{x},\vec{y},\vec{z}$ are independent.
	  
	  \item[P4.] $(\vec{x}\times\vec{y})\times(\vec{z}\times\vec{v}) =
	  [\vec{x},\vec{y},\vec{v}] \cdot \vec{z} - [\vec{x},\vec{y},\vec{z}] \cdot \vec{v}$ 
	\end{enumerate}
	\begin{tcolorbox}[title=Remark,colframe=black,arc=10pt]
	We will come back on the triple product during our study on tensor calculus as it gives the opportunity to get a very interesting result concerning a future application in General Relativity.	
	\end{tcolorbox}
	
	
	\subsection{Vectorial Functional Space}
	Given $\mathcal{C}_{[a,b]}^k$ the set of real functions that can be $k$-times derivates (\SeeChapter{see section Differential and Integral Calculus}) in the closed bounded interval $[a,b]$. We will designate the elements of this set by the letters $\vec{f},\vec{g},...$.
	
	The value of $\vec{f}$ at the point $t$ will be obviously denoted by $\vec{f}(t)$. Say that $\vec{f}=\vec{g}$ is therefore equivalent than to say that:
	 
	 In a condensed way some practitioners denote this $\vec{f}(t) \equiv \vec{g}(t)$, the symbol $\equiv$ indicating obviously that the two members are equals for any $t$ in the bounded interval $[a,b]$.
	 
	 Consider the two following operations:
	\begin{itemize}
		\item $\vec{f}(t)+\vec{g}(t)$ defined by the relation $(\vec{f}+\vec{g})(t)\equiv \vec{f}(t)+\vec{g}(t)$
		\item $\alpha\vec{f}$ defined by the relation $(\alpha \vec{f})(t)=\alpha\vec{f}(t)$
	\end{itemize}
	These two operations satisfy to all conditions of the vectors of a vector space as we have already defined at the beginning of this section (associativity, commutativity, null vector, opposed vector, distributivity, neutral element) and therefore gives us the possibility to assign to $\mathcal{C}_{[a,b]}^k$ of a vector space structure! Le null vector of this space being obviously the null function (equal to zero everywhere) and the opposite of $\vec{f}$ being $-\vec{f}$.

	It is interesting to notice that $\mathcal{C}_{[a,b]}^k$  as a vector space is a generalization of $\mathbb{R}^n$ to the continuous case. Indeed, we can consider any vector $\vec{v}=(a_i)$ of $\mathbb{R}^n$ in the form of a real function defined on the set $\left\lbrace 1,2,...,n \right\rbrace
$: the value of this function at the point $i$ is simply $a_i$.
	
	\begin{tcolorbox}[title=Remark,colframe=black,arc=10pt]
	The polynomials of order $n$ with one unknown form as an example of functional vector space of dimension $n + 1$ such that for each coefficient of the polynomial corresponds a vector component such that:
		
	\end{tcolorbox}
	The preferred application field of the abstract theory of the dot product (inner product) is formed by the functional vector spaces. We name therefore "\NewTerm{canonical scalar product}\index{canonical dot product}" in $\mathcal{C}_{[a,b]}(\mathbb{R}^2)$ the operation defined by the relation:
	
	This operation defines indeed a scalar product, the properties of the latter being verified (on reader request we can add the proof if necessary), and furthermore, the integral:
	
	is positive if the continuous function $\vec{f}$ is not identically zero.
	
	Technically the latter relation is written when in $\mathbb{R}^2$:
	
	We will give more precision about this norm and its associated scalar product and with example in the section of Functional Analysis.
	
	\pagebreak
	\subsection{Hermitian Vector Space}
	The purpose of what will follow is, as always in this book, not to give a detailed study about vector spaces in $\mathbb{C}$ but just to give the minimal knowledge and vocabulary necessary to the lecture of some theoretical models in physics and especially those presented in this book in the section of Wave Quantum Physics.
	
	When the scalars that appears in the definition of vector spaces are complex numbers (in $\mathbb{C}$ as seen in the section Numbers), and not only real numbers, then we speak obviously about "\NewTerm{complex vectorial spaces}\index{complex vectorial spaces}".
	
	\begin{tcolorbox}[title=Remark,colframe=black,arc=10pt]
	Rigourlsy in the common communication, people should always precise if we speak of real vectorial space or complex vectorial space...
	\end{tcolorbox}
	Let us give some expamples of famous complex vectorial spaces (as many people think the are useless):

	\begin{tcolorbox}[colframe=black,colback=white,sharp corners]
	\textbf{{\Large \ding{45}}Example:}\\\\
	E1. The space vector $\mathbb{C}^n$ of column-vectors with $n$ components ($\mathbb{C}^1$ being obviously identified to $\mathbb{C}$). We will meet, among others, such vector space in the section of Relativistic Quantum Physics.\\
	
	E2. The vectorial space of univariate polynomial with coefficient in $\mathbb{C}$. We will meet such spaces in the section of Wave Quantum Physics or even Quantum Chemistry.\\
	
	E3. The vectorial space of complex functions of one real or comple variables continuous or not. We will meet such vector spaces frequently in the section of Wave Mechanics and especially in the section of Electrodynamics.
	\end{tcolorbox}
	The purpose here is to adapt what we have seen so far to complex vectorial space. The following example, show us that we can transpose as it the previous definitions. Indeed, let us consider the vector space $\mathbb{C}^n$. As for $\mathbb{R}^n$, we could have the tentation to define a dot product on $\mathbb{C}^n$ by:
	
	with $x_i,y_i\in \mathbb{C}$.
	Sadly, we see that this definition is not satisfactory because we could have therefore:
	
	and this quantity is in general not a real number in a complex vector space and violates the property of positivity of the dot product and therefore prevent us to introduce and use the concept of distance. What is obviously a big problem in our actual perception of the world.
	
	We could therefore not define anymore a norm in $\mathbb{C}^n$ by writing:
	
	For $\langle \vec{x}|\vec{x}\rangle$ to be a positive real number we see that it would be better to define the scalar product like this:
	
	In this case we have therefore:
	
	which is well a positive real number. From there, we can once again define a norm for complex vector space $\mathbb{C}^n$ by putting:
	
	We will now show how to define an inner product on a complex vector space in the general case.
	
	\subsubsection{Hermitian Inner Product}
	\textbf{Definition (\#\mydef):} Let $\mathcal{H}$ be a complex vector space (!). We name "\NewTerm{scalar product}\index{scalar product}" or more accurately "\NewTerm{Hermitian inner product}\index{Hermitian inner product}" on $\mathcal{H}$ (that is to say: a dot product in complex space...), an application:
	
	That satisfies (they are more properties but we will focus only about what we need for practical application in this book and especially quantum physics):
	\begin{enumerate}
		\item[P1.] Positivity:
		
	  	
		\item[P2.] Nullity:
		 
	  	
		\item[P3.] Hermitian symmetry:
		
	  	
	  	\item[P4.] Bilinearity (bilinear form) changes a little bit too ... so that we speak now of "\NewTerm{sesquilinearity}\index{sesquilinearity}". We speak then, in order, of left anti-linearity and of right linearity such as:
	  	
	\end{enumerate}
	\begin{tcolorbox}[title=Remarks,colframe=black,arc=10pt]
	\textbf{R1.} It seems that some mathematicians put the anti-linearity on the right. It's just a matter of agreement that does not matter and exists because of a lack of international norms in mathematics.\\
	
	\textbf{R2.} The reader may notice easily that if the elements of the above definitions are all in the set $\mathbb{R}$ then the sesquilinearity is reduced to the bilinearity and the hermitian character to a simple symmetry. So the hermitian inner product reduces to the scalar product.\\
	
	\textbf{R3.} We want to give for now only the minimum on the vast subject that is complex vector spaces so that the reader can read without too much trouble the beginning of the section of Wave Quantum Physics.
	\end{tcolorbox}	
	When we join to a complex vector space a scalar product then just as a real vector space becomes an Euclidean vector space or Prehilbertien vector space, the complex vector space becomes what we name a "\NewTerm{hermitian vector space}\index{hermitian vector space}" (term often used in the section of Wave Quantum Physics).
	
	\textbf{Definition (\#\mydef):} Again, we say that a space $\mathcal{H}$ provided with an Hermitian product $\langle \vec{x}|\lambda \vec{y} +\mu \vec{z}\rangle$ is a "\NewTerm{Hilbert space}\index{Hilbert space}" if this space is complete for the metric defined above.
	
	Thus, Hilbert spaces is a generalization of spaces including dot products and Hermitian dot product of Euclidean and Prehilbertien spaces.
	
	\subsubsection{Types of Vectors Spaces}
	To sum it all up:
	\begin{itemize}
		\item We name "pre-Hilbert space" (real or complex) any vector space of finite dimension or not, provided with a dot (scalar) product.
		
		\item We name "Hilbert space" (real or complex) any complete prehilbertian space (as space provide with a norm).
		
		\item We name "Euclidean space" any real vector space of finite dimension with a dot (scalar) product and denoted by $\mathcal{E}^n$.
		
		\item We name Hermitian space any complex vector space of finite dimension with a dot (scalar) product and denoted by $\mathcal{H}^n$.
	\end{itemize}
	
	\pagebreak
	\subsection{System of Coordinates}
	We will address here the aspect of coordinates changes of vector components not from a basis to another one (for that you need to go see the section of Linear Algebra) but from one coordinate system to another. That means that in any case we will stay in an Euclidean space. This type of transformation has strong implication in physical (and a little bit less in pure mathematics) when we want to simplify the study of physical systems whose equations become easier to handle in other coordinate systems.
	
	\textbf{Definition (\#\mydef):} In mathematics, a "\NewTerm{coordinate system}\index{coordinate system}" is used to match to each point of an $n$-dimensional space, a $m$-tuple of scalars.
	
	Although we are in a chapter and a section of this book that is suppose to be pure maths oriented..., we will allow ourselves in what follows to make a direct connection with physics relatively the terms of the speed and acceleration in different coordinate systems (sorry for the "math skills only" people...). Our teaching experience has show that this helps the readers (most of time students) to better understand the various abstract concepts.
	
	\subsubsection{Cartesian (rectangular) Coordinate System}
	We do not want to take too much time on this system as it is well known to everyone usually. However, let us recall that most of the time, in physics, the Cartesian systems in which we are working are in $\mathbb{R}^2 $(two real spatial dimensions), or $\mathbb{R}^3$ (three real spatial dimensions) or even $\mathbb{R}^4$ or $\mathbb{C}^4$ (three spatial dimensions and one of time) when we work in relativity. The number of dimensions can be higher as for example with the Kaluza-Klein theory (five dimensions) merging General Relativity and Electromagnetism or much more with String Theory (above 20 dimensions!!!).
	
	In $\mathbb{R}^3$ (the most common case), there are three basic vectors traditionally denoted by:
	
	Or more explicitly:
	
	
	In this system, the position of a point $P$ (identifiable by a vector $\vec{x}$ for example) is defined by the three numbers named  "\NewTerm{coordinates}\index{coordinates}" (more generally "\NewTerm{components}") denoted (typically in Tensor Calculus):
	
	and in physics denoted more conventionally by:
	
	where usually the component $(z)$ represents the height (vertical), the component $(x)$ is the width and the component  $(y)$ is the length (obviously these are completely arbitrary choice).
	
	This point $P$ can be spotted by a vector arbitrarily designated $\vec{r}$ in the basis  $\vec{e}_i$ by the relation (using Tensor notation):
	
	and if the basis is canonical (orthonormal) such that:
	
	we write:
	
	In physics, if we work with coordinates, it is always to be able to determine the location of an item. Or, as we shall see it more rigorously in the section of Analytical Mechanics, the physicist works with the following concepts (each element being often time-dependent):
	\begin{itemize}
		\item Positions: $\vec{r}=\biggl(x(t),y(t),z(t),t\biggr)$
		
		\item Velocity: $\displaystyle\frac{\mathrm{d}\vec{r}}{\mathrm{d}t}=\dot{\vec{r}}=\vec{v}
	=\biggl(\dot{x}(t),\dot{y}(t),\dot{z}(t),t\biggr)$
	
		\item Acceleration: $\displaystyle\frac{\mathrm{d}\vec{v}}{\mathrm{d}t}=\dot{\vec{v}}=\vec{a}=
	\biggl(\ddot{x}(t),\ddot{y}(t),\ddot{z}(t),t\biggr)$
	\end{itemize}
	Now let us see how the different concepts are expressed in systems such as spherical, cylindrical and polar coordinates (remember that we remains for all of them in a flat Euclidean space!!!).
	
	\pagebreak
	\subsubsection{Spherical Coordinate System}
	The choice to start with this coordinate system is not a coincidence. It has the advantage of being a generalization of cylindrical and polar systems that we will meet thereafter and will help us easier to determine the expressions of position, velocity and acceleration.
	
	We traditionally represent (in Switzerland ... and in accordance with the standard ISO 31-11) a spherical coordinate system as follows:
	\begin{figure}[H]
		\centering
		\includegraphics{img/algebra/coordinate_system_spherical.jpg}
		\caption{Representation of the spherical coordinate system}
	\end{figure}
	We see very clearly if we know the basic trigonometric relations and identities (see the section of the same name in the Geometry chapter) we have the transformations:
	
	
	where the two angles $\theta, \phi$ are respectively the latitude and colatitude (longitude):
	\begin{figure}[H]
		\centering
		\includegraphics[scale=0.7]{img/algebra/latitude_longitude.jpg}
		\caption{Latitute and Longitude concepts illustrated (source: OpenStax)}
	\end{figure}
	We have inversely:
	
	Now we must find the expressions that connect the vectors of the spherical basis that we choose to denote by $\vec{e}_r,\vec{e}_\theta,\vec{e}_\phi$ with the vectors of the Cartesian basis $\vec{e}_x,\vec{e}_y,\vec{e}_z$:
	
	Let us indicate that by dividing by $\sin(\theta)$ the second basis vector $\vec{e}_\phi$, we make sure that by the properties of the norm of the vector product that:
	
	will be well normalized to unity as expected (as we know from start that as we toke a direct orthogonal coordinate system the product of norms of the basis vectors must be equal to $1$)!
	
	
	We will also use later (for the study of vector operators further below and the geodesic of the sphere in the section of Analytical Mechanics) the variation $\mathrm{d}\vec{r}$ expressed in spherical coordinates:
	
	Or more explicitly:
	
	To express the velocity and acceleration in spherical coordinates, we will also need the derivatives with respect to time:
	
	So if we do now a little bit of physics, we have:
	
	This brings us to (we will need this relation mainly in the chapter of Astrophysics):
	
	It is interesting that we get the same result through the following method that may be less intuitive:
	
	and substituting the derivative obtained above:
	
	Concerning the acceleration we get:
	
	But we have:
	
	Therefore it comes:
	

	Thus finally:
	
	
	\pagebreak
	\subsubsection{Cylindrical Coordinate System}
	The cylindrical coordinate system (very useful in the study of helical motion systems) is quite similar to spherical coordinates as it can be seen as a slice of the sphere. 

	Given the figure:
	\begin{figure}[H]
		\centering
		\includegraphics{img/algebra/coordinate_system_cylindrical.jpg}
		\caption{Representation of the cylindrical coordinate system}
	\end{figure}
	Warning!!! The vector $\vec{r}$ is unlike the previous spherical case defined only in the $XY$ plane or a plane which is parallel to it!
	
	It comes easily in cylindrical coordinates for $r>0$:
	
	and vice versa:
	 
	Now we must find the expressions that connect the vectors of the cylindrical base that we choose to denote by $\vec{e}_r,\vec{e}_\phi,\vec{e}_z$ (instead of $\vec{r}, \vec{\phi},\vec{z}$ as it is done traditionally) with the vectors of the Cartesian base $\vec{e}_x,\vec{e}_y,\vec{e}_z$. We have identically to what we did for the spherical coordinates:
	
	Or more explicitly:
	
	Let us indicate that by dividing by $\sin(\phi)$ the second vector base $\vec{e}_\phi$, we ensure us that by the properties of the norm of the cross product we have:
	
	will be well normalized to unity as expected (as we know from start that as we toke a direct orthogonal coordinate system the product of norms of the basis vectors must be equal to $1$)!! In the case of cylindrical coordinates the angle being anyway right, we would not be obliged to indicate this division, but we have made this choice for consistency with previous developments...
	
	\begin{tcolorbox}[title=Remark,colframe=black,arc=10pt]
	It is important to notice that the cross product of two basis vectors always gives the third perpendicular  basis vector (like the Cartesian and spherical coordinates so!).
	\end{tcolorbox}
	For future needs, let us determine the partial differential of each of these coordinates:
	
	We will also use later (for the study of vector operators) the variation $\mathrm{d}\vec{r}$ expressed in cylindrical coordinates:
	
	To express the speed and acceleration in cylindrical coordinates, we will also need the derivatives with respect to time:
	
	So if we now do a little bit physics, we get (let us recall that the $z$ component is independent of other cylindrical components):
	
	which brings us to:
	
	For acceleration we get (exactly the same approach as for the expression of the speed):
	
	
	\subsubsection{Polar Coordinate System}
	The polar coordinate system is very similar to the cylindrical coordinates as it can be seen as an entrenchment of one dimension (the height) of the cylindrical system (we will often encounter this system in the section of Classical Mechanics, Corpuscular Quantum Physics and Astronomy).

	Given the figure:
	\begin{figure}[H]
		\centering
		\includegraphics{img/algebra/coordinate_system_polar.jpg}
		\caption{Representation of the polar coordinate system}
	\end{figure}
	Thus, it comes easily in polar coordinates for $r>0$:
	
	and vice versa:
	
	Now we must find the expressions that connect the vectors of the polar base that we choose to denote by $\vec{e}_r,\vec{e}_\phi$ (instead of $\vec{r}, \vec{\phi}$ as it is done traditionally) with the vectors of the Cartesian base $\vec{e}_x,\vec{e}_y$. We have identically to what we did for the spherical coordinates:
	
	Or more explicitly:
	
	Once again, dividing by $\sin(\theta)$ the second basis vector $\vec{e}_\phi$, we ensure the properties of the norm of the vector product that:
	
	will be well normalized to unity (as we know from start that as we toke a direct orthogonal coordinate system the product of norms of the basis vectors must be equal to $1$)!. In the case of polar coordinate the angle being a anyway right, we would not be obliged to indicate this division, but we have made this choice for consistency with the previous developments.
	
	For future needs, Let us determine the partial differential of each of these coordinates:
	
	We will also use later (for the study of vector operators) the variation $\mathrm{r}$ expressed in polar coordinates:
	
	To express the speed and acceleration in polar coordinates, we will also need the derivatives with respect to time:
	
	So if we now do a little bit physics, we have:
	
	and therefore:
	
	where the first term is the radial velocity component and the second term the tangential component of the (angular) velocity. The velocity expression in polar coordinates is very important in astronomy as it allows quite easily calculate the calculation of the kinetic energy:
	
	For the acceleration we get:	
	
	where the first term is the radial acceleration, the second term the centripetal acceleration, the third the Coriolis acceleration and finally the fourth one the tangential acceleration.
	
	\pagebreak
	\subsection{Differential Operators}
	\textbf{Definition (\#\mydef):} Define a scalar field, vector field or tensor field in a volume $V$, it is define an application that for any point $\vec{x}$ of this volume $V$ associates respectively a scalar, a vector or a tensor.
	
	Thus, the application $f$ that at any point $\vec{x}$ of $V$ of spatial coordinates $(x, y, z)$ associates the scalar value $f(\vec{x})=f(x,y,z)$ is a scalar field in $V$.
	
	At each point of a volume traversed by a moving fluid, the vector that coincides at every moment with the speed of the particle which pass through this point at this same time defines a 3D vector field, optionally variable in time. The fields thus defined are a basic mathematical tool in physics.
	\begin{tcolorbox}[title=Remark,colframe=black,arc=10pt]
	When we plot a scalar field, the set of continuous dots of equal value is so-named "\NewTerm{isolines}\index{isolines}" or more commonly "\NewTerm{contours}\index{contours}".
	\end{tcolorbox}
	Vector fields are especially an important tool for describing many physical concepts, such as gravitation and electromagnetism, which affect the behavior of objects over a large region of a plane or of space. They are also useful for dealing with large-scale behavior such as atmospheric storms or deep-sea ocean currents.

	For example the figure below shows a gravitational field exerted by two astronomical objects, such as a star and a planet or a planet and a moon. At any point in the figure, the vector associated with a point gives the net gravitational force exerted by the two objects on an object of unit mass. The vectors of largest magnitude in the figure are the vectors closest to the larger object. The larger object has greater mass, so it exerts a gravitational force of greater magnitude than the smaller object.
	\begin{figure}[H]
		\centering
		\includegraphics[scale=0.5]{img/algebra/vector_field_gravitation.jpg}
		\caption{Gravitation vector field example (source: OpenStax)}
	\end{figure}
	Another example is the figure below that shows the velocity of a river at points on its surface. The vector associated with a given point on the river's surface gives the velocity of the water at that point. Since the vectors to the left of the figure are small in magnitude, the water is flowing slowly on that part of the surface. As the water moves from left to right, it encounters some rapids around a rock. The speed of the water increases, and a whirlpool occurs in part of the rapids.
	\begin{figure}[H]
		\centering
		\includegraphics[scale=0.5]{img/algebra/vector_field_speed.jpg}
		\caption{Velocity vector field example (source: OpenStax)}
	\end{figure}
	The gradient, divergence and the rotational are the three main linear differential operators of the first order that will be introduced here. This means they only involve the partial first derivatives (or simply "differentials") of the fields, unlike, for example, the Laplace operator which involves partial derivatives of order $2$.	
	
	\pagebreak
	\subsubsection{Gradients of Scalar Field}
	The gradient is an operator that applies to a scalar field and transforms it into a vector field. Intuitively, the gradient indicates the direction of the greater variation of the scalar field, and the intensity of this variation. For example, the gradient of the altitude is directed along the maximum slope line and its norm increases with the slope.
	
	Given a three-dimensional scalar field $f(x,y,z)$, wherein x and y and z are the Cartesian coordinates of a point $M$ in space. When $M$ moves in space according to the $\mathrm{d}\vec{r}$ of components $\mathrm{d}x, \mathrm{d}y$ and $\mathrm{d}z$, the scalar field $f$ varies according to the total differential $\mathrm{d}f$:
	
	From this relation, we can define the "\NewTerm{gradient operator}\index{gradient operator}" of a scalar field such as:
	
	where:
	
	is a vector operator named "\NewTerm{gradient of the scalar field $f$}\index{gradient of a scalar field}". To condense the writing, we sometimes use the symbol $\vec{\nabla}$ named the "\NewTerm{nabla of the scalar field $f$}\index{nabla of a scalar field }".
	
	The vector obtained by the gradient calculation has the following three fundamental properties:
	\begin{enumerate}
		\item[P1.] The components of the gradient represent by construction the variation (slope) of the function $f$ in the different directions of space.
		
		\item[P2.] The gradient is perpendicular to the isolines of the function $f$.

		\item[P3.] The direction of the gradient (and therefore its norm) is the maximum variation of $f$.

		\item[P4.] The direction of the gradient shows the values where $f$ increases.
	\end{enumerate}
	Following the request of some readers let us prove some of these properties.
	Given $t\mapsto C(t)$ an isoline. Then $t\mapsto f(c(t))=c^{te}$ and therefore:
	  
	which proves that the gradient is orthogonal to the tangent of the isoline (property P2).
	
	Let us come back on:
	
	That it is tradition to write as:
	
	named the "\NewTerm{directional derivative}\index{directional derivative}". Its value is maximum obviously if $\theta=0$, that is to say that the gradient is colinear to the variation $\mathrm{d}\vec{r}$. Hence, the direction (and therefore the norm) of greatest increase of $f$ is the same direction as the gradient vector!! Thus we proved the property P3.
	
	Obviously the directional derivative takes on its greatest negative value if $\theta=\pi$ (or $180$ degrees). Hence, the direction of greatest decrease of $f$ is the direction opposite to the gradient vector.

	The property P4 can be explains without formalism as following:
	\begin{itemize}
		\item If the function is decreasing in one variable, then the partial derivative is negative, so the component vector of the gradient for that variable points in the negative direction - which means increasing function value.

		\item If the function is increasing in one variable, then the partial derivative is positive, so the component vector of the gradient for that variable points in the positive direction - which means increasing function value.
	\end{itemize}
	Then is doesn't matter how the function profile is, the gradient, by definition, points in the increasing direction. Indeed, when 	
$f(x,y)$ is decreasing in $x$, the function decreases as you move forward in $x$. But because the partial derivative with respect to $x$ is negative, the $x$-component of the gradient points towards origin (move backward in $x$), that is to say in the direction which makes f to increase.
	
	From the definition and from the total differential, we get
	
	This leads us to put that:
	
	and so that finally the operator of the "\NewTerm{gradient in Cartesian coordinates}\index{gradient in Cartesian coordinates}" is given by:
	
	Finally we see that the gradient of a scalar field $f(x,y,z)$ is the vector field whose components at each point are the three derivatives of the scalar field $f$ with respect to the three-dimensional coordinates, denoted here by $x, y, z$ and that by its direction, and its norm, the gradient vector of a scalar field at a point therefore includes indications on how the field varies around this point.
	
	\begin{tcolorbox}[title=Remark,colframe=black,arc=10pt]
	One of the necessary and sufficient conditions for a vector field to be the gradient of a scalar field $f$ is that this vector field is irrotational (see below the rotational operator of a vector field).
	\end{tcolorbox}
	\begin{tcolorbox}[colframe=black,colback=white,sharp corners]
	\textbf{{\Large \ding{45}}Example:}\\\\
	Let us find the direction for which the directional of:
	
	at $(-2,3)$ is a maximum and what is its maximum value?
	\begin{figure}[H]
		\centering
		\includegraphics{img/algebra/directional_derivative_search_plot_maple.jpg}
		\caption[]{Maple plot of $f(x,y,)=3x^2-4xy+2y^2$}
	\end{figure}
	The maximum value of the directiona derivative occurs as we have just proved it when $\vec{\nabla}f$ and the unit vector $\mathrm{d}\vec{r}$ point in the same direction.\\

	Therefore we start by calculating $\vec{\nabla}f(x,y)$:
	
	Next we evaluate the gradient at $(-2,3)$:
	
	\end{tcolorbox}
	\begin{tcolorbox}[colframe=black,colback=white,sharp corners]
	We need to find a unit vector that points in the same direction as $\vec{\nabla}f(-2,3)$, so the next step is to divide $\vec{\nabla}f(-2,3)$ by its norm (the value of that norm being also the maximum value of the directional derivative at point $(-2,3)$), which gives:
	
	As the unit vector above is a vector in the plane, nothing avoid us to calculate the angle it does with the axis. By applying elementary trigonometry, we get if we denote this angle by $\theta$:
	
	Since cosine is negative and sine is positive, the angle must be in the second quadrant. Therefore:
	
	that is to say approximately $114.5916$ degrees (or seen from the point of view of the first quadrant: $39.792$ degrees).
	
	With Maple 4.00b we can have a more detailed and general investigation of the calculation we just did:\\

	\texttt{>with(plots):\\
	>with(linalg):\\
	>Pa:=contourplot(3*x\string^3-4*x*y+2*y\string^2,x=-3..-1,y=2..4,filled=true):\\
	>Pb:=fieldplot(grad(3*x\string^3-4*x*y+2*y\string^2,vector([x,y])),x=-3..-1,y=2..4):\\
	>display(Pa,Pb);
	}
	\begin{figure}[H]
		\centering
		\includegraphics[scale=0.9]{img/algebra/directional_derivative_gradient_plot_maple.jpg}
	\end{figure}
	\end{tcolorbox}
	After having defined the gradient in Cartesian coordinates $x, y, z$ we have to address the expression of this operator in other coordinate systems. It is common in physics to have to use cylindrical, polar and spherical coordinates to simplify the formal study of physical systems. So if we refer to the previous study of coordinate systems, we have (recall) first in polar coordinates:
	
	But, with the definition of gradient in Cartesian coordinates, in polar coordinates we have the following definition:
	
	If we express the total exact differential (\SeeChapter{see section of Differential and Integral Calculus}) of $\mathrm{d}f$ we obtain the following relation:
	
	This allows us to get the relation:
	
	therefore:
	
	which bring us to:
	
	Thus the "\NewTerm{gradient in polar coordinates}\index{gradient in polar coordinates}" is expressed as
	
	Let us now tackle the expression of the gradient in cylindrical coordinates. Let us recall that  during our study of different coordinate systems we obtained for cylindrical coordinates:
	
	So we already know that the expression of the gradient in cylindrical coordinates will be the same in polar coordinates with the exception of the addition of the vertical $z$ component that is independent of other coordinates. Thus we get the "\NewTerm{gradient in cylindrical coordinates}\index{gradient in cylindrical coordinates}":
	 
	Let us now tackle on the expression of the gradient in spherical coordinates. Let us recall that during our study of the different coordinate systems we obtained for the spherical coordinates:
	
	But, with the definition of gradient in Cartesian coordinates, we have in spherical coordinates the following definition:
	
	If we express the total differential of $\mathrm{d}f$ we get the following relations:
	
	This allows us to obtain the relation (we now use the notation that uses the operator "nabla"):
	
	The relation:
	
	requires that:
	
	Thus the "\NewTerm{gradient in spherical coordinates}\index{gradient in spherical coordinates}" is expressed as:
	 
	So we finally saw all the expressions of the gradient operator in the Cartesian, polar, cylindrical and spherical systems.
	
	\begin{tcolorbox}[colframe=black,colback=white,sharp corners]
	\textbf{{\Large \ding{45}}Example:}\\\\
	Let us see now a visual example of the previous developements with Maple 4.00 and a special case function $f(x,y)=\sin(x)\sin(y)$.\\
	
	\texttt{> with(linalg):\\
	> with(plots):\\
	> plot3d(sin(x)*sin(y),x=-3..3,y=-3..3,axes=framed);}\\
	\begin{figure}[H]
		\centering
		\includegraphics{img/algebra/gradient_function_of_example.jpg}
		\caption[]{Plot of the function taken as example}
	\end{figure}
	And now we show the isolines:\\
	
	\texttt{>contourplot3d(sin(x)*sin(y),x=-3..3,y=-3..3,filled=true,\\
	axes=framed,coloring=[red,blue],style=patch);}
	\begin{figure}[H]
		\centering
		\includegraphics{img/algebra/gradient_function_with_isolines.jpg}
		\caption[]{Function with is isolines}
	\end{figure}
	\end{tcolorbox}
	
	\pagebreak
	\begin{tcolorbox}[colframe=black,colback=white,sharp corners]

	And now we plot the gradient vector with a plane projection and we see they are indeed perpendicular to the isolines:\\\\
		\texttt{>Pa:=contourplot(sin(x)*sin(y),x=-3..3,y=-3..3,contours=10,\\
		coloring=[red,blue],filled=true):
\\
	>Pb:=fieldplot(grad(sin(x)*sin(y),vector([x,y])),x=-3..3,y=-3..3,\\
	arrows=THICK):\\
	>display(Pa,Pb);}
	\begin{figure}[H]
		\centering
		\includegraphics{img/algebra/gradient_function_with_isolines_and_projected_gradient.jpg}
		\caption[]{Function with its isolines and projected gradient}
	\end{figure}
	And with a 3D perpsective:\\
	
	\texttt{>campo:=fieldplot3d([diff(sin(x)*sin(y),x),diff(sin(x)*sin(y),\\
	y),0],x=-3..3,y=-3..3,
z=-3..3,axes=framed,arrows=THICK);\\
>superf:=plot3d(sin(x)*sin(y),x=-3..3,y=-3..3):
\\
	> display({campo,superf});}
	\begin{figure}[H]
		\centering
		\includegraphics[scale=0.8]{img/algebra/gradient_function_with_isolines_and_gradient.jpg}
		\caption[]{Function with its isolines and gradient in 3D}
	\end{figure}
	\end{tcolorbox}

	\pagebreak
	\begin{tcolorbox}[colframe=black,colback=white,sharp corners]
	Seen from above with a small rotation:\\
	\begin{figure}[H]
		\centering
		\includegraphics{img/algebra/gradient_function_with_isolines_and_projected_gradient_rotation.jpg}
		\caption[]{Function rotation with its isolines and gradient}
	\end{figure}
	\end{tcolorbox}
	
	\subsubsection{Gradients of Vector Field}
	The gradient of a vector field $\vec{f}(x,y,z): \mathbb{R}^3\mapsto\mathbb{R}^3$ is the field named "\NewTerm{tensor field}\index{tensor field}" defined by the following nine relations in Cartesian coordinates:
	 
	 We will use such a gradient in our study in the section of Marine \& Weather Engineering the Papillon effect whose origin comes from the determination of the Navier-Stokes equations of the section of Continuum Mechanics and we will also use this type of gradient in the section of Theoretical Computing in our study of the Gauss-Newton optimization method.
	
	We have the following 4 components in polar coordinates:
	 
	We have the following 9 components in cylindrical coordinates:
	 
	We have the following 9 components in spherical coordinates:
	 
	 So we finally saw all the expressions of a gradient vector field in Cartesian , polar, cylindrical and spherical system.
	 
	\subsubsection{Divergences of a Vector Field}
	The diverence is applied to a vector field and turns it into a scalar field. Therefore it is an application from $\mathbb{R}^3\mapsto \mathbb{R}$. Intuitively, and in the most common case, the divergence of a vector field expressed its tendency to come from or converge to some points.
	\begin{tcolorbox}[title=Remark,colframe=black,arc=10pt]
	Non-initated people often confuse the gradient and divergence operator. To make the difference we must remember that the divergence of a vector is a number and that the gradient is a vector! The gradient indicates the direction in which the change is the most important and its amplitude. The divergence simply say what comes in or out from a given point.
	\end{tcolorbox}	
	However, we must distinguish two contributions to the divergence that we will rigorously define a little further below: one due to the variations named the "\NewTerm{directional divergence}\index{directional divergenc}" and the other due to variations in modules (norm) named the "\NewTerm{modular divergence}\index{modular divergence}". Thus, for simple fields, we can imagine cases where the divergence would only be modular and others, where it would only be directional. We could also build a field where the two types of divergence coexist, but having adverse effects.
	
	Let us consider for example a vector $\vec{f}$ of space and we make it pass through any surface $S$. Physicists then assimilate the quantity $\vec{f}$ which moves along the normal vector to the surface as a flow of $\vec{f}$ through $S$.
	
	To be convinced of this analogy we can imagine a fluid flowing on a flat surface, the flow through the surface is obviously zero in this cas, by cons if the fluid flows vertically through a horizontal surface the flow will be maximal. It is then immediate that we want to represent the flow by the scalar product of $\vec{f}$ with the normal $\vec{n}$ to  the surface $S$.
	\begin{tcolorbox}[title=Remark,colframe=black,arc=10pt]
	We must always pay attention to the direction of $¨\vec{n}$ because at any point of a surface $S$ we have in general two normal vectors $\vec{n}$ that are colinear but of opposite directions.
	\end{tcolorbox}	
	If the surface is planar  then the normal $\vec{n}$ is the same everywhere, but if it changes from place to place, then we will look at a small surface element $\mathrm{d}S$.
	
	If a small flow element is defined by:
	
	then the total flow will be given by:
	
	which is sometimes written (it's a little bit abusive but why not ...)
	
	Let us now suppose that our vector $\vec{f}$ moves a point $M(x,y,z)$ in space to  $M'(x+\mathrm{d}x,y+\mathrm{d}y,z+\mathrm{d}z)$ through a rectangular parallelepiped of sides of $\mathrm{d}x, \mathrm{d}y$ and $\mathrm{d}z$:
	
	\begin{figure}[H]
		\centering
		\includegraphics{img/algebra/ostrogradsky_box_vector_displacement.jpg}
		\caption[]{Move of a vector through a parallelepiped}
	\end{figure}
	We can decompose the movement (flow) through each face of the parallelepiped (decompositions in the orthonormal basis). For example, if we are interested at the decomposed part of the flow through the face $(\mathrm{d}y, \mathrm{d}z)$ described by the peaks vertices $BCFG$ we have obviously $\vec{n}=(1,0,0)$.
	
	We still need to determined how to represent the flow $\vec{f}$ for this direction. As the flow is a function, that is to say that each of its components may be dependent of the three components of the space (if we take the case of a function in $\mathbb{R}^3$) we have:
	
	\begin{tcolorbox}[title=Remark,colframe=black,arc=10pt]
	Those who are not convinced can go read the beginning of section Electrodynamics where we take the electric field as an (excellent) example.
	\end{tcolorbox}	
	While the variation of the flow according to $x$ is given by:
	
	which give us:
	
	therefore by summing:
	
	Compared to the first expression of $\Phi$, the term $\mathrm{d}x\mathrm{d}y\mathrm{d}z$ is then a volume element and not more of surface. We also have an interesting result:
	
	whose more explicit and rigorous writing  should be (to highlight well that the considered closed surface is the boundary of the closed studied volume):
	
	or more commonly written:
	
	\begin{tcolorbox}[title=Remark,colframe=black,arc=10pt]
	See the practical examples in the section Electrodynamics where for example the electric field divergence is zero for a free spherical charge as the vectors point in different directions (directional divergence) and where the norms decrease as the inverse of the square of the radius (modular convergence). Both contributions are in opposition and so the total divergence is zero.
	\end{tcolorbox}
	The development above is named "\NewTerm{Ostrogradsky theorem}\index{Ostrogradsky theorem}" or "\NewTerm{Gauss-Ostrogradsky theorem}\index{Gauss-Ostrogradsky theorem}" or more simply "\NewTerm{divergence theorem}\index{divergence theorem}" and actually defines the total divergence of $\vec{f}$ in a volume as the flow $\vec{f}$ through the "walls" of the closed volume (Gauss closed surface), which expresses well the name "divergence".
	
	Now reconsider the previous relation but extracting and unitary vector from the vector field $\vec{f}$ such that:
	
	where now $f$ is a scalar field. This can be rewritten obviously using chain rule derivatives an dot product commutativity:
	
	In the special case of a uniform field we have:
	
	Then it remains:
	
	The dot product being distributive on the sum of vectors we can rewrite this as:
	
	and therefore we get the "\NewTerm{gradient theorem\footnote{As it is applicable only for uniform field it is not used a lot in practice}}\index{gradient theorem}":
	
	
	We define the operator "\NewTerm{divergence}\index{divergence operator}" by the following relation (the tensor notation has been used to shorten the writing) in an $n$-dimensional space:
	
	Thus we have for the operator "\NewTerm{divergence in Cartesian coordinates}\index{divergence in Cartesian coordinates}":
	
	If the divergence of a vector field is identically zero in all the points of an Eulerian frame \footnote{The Eulerian specification of the flow field is a way of looking at fluid motion that focuses on specific locations in the space through which the fluid flows as time passes.[1][2] This can be visualized by sitting on the bank of a river and watching the water pass the fixed location.}, the triple integral flux of this field through a volume $V$ will be:
	
	It follows that the flow of this vector field through the edges of the volume is zero, that is to say that the incoming flow compensates the output flow. We say that such a field vectors having a null divergence has a "\NewTerm{conservation flow}\index{conservation flow}".
	
	To determine the expression of the divergence operator in polar coordinates let us recall the relations proved earlier above:
	
	Given now a vectorial function $\vec{f}:\mathbb{R}^2\rightarrow \mathbb{R}^2$. We have:
	
	Knowing the expression of $\vec{e}_r,\vec{e}_\phi$ depending on $\vec{e}_x,\vec{e}_y$, from the expression above we deduce:
	
	The divergence of $\vec{f}$ being defined in the two dimensional case by:
	
	we then have:
	
	The first term is (application of the gradient in polar coordinates!):
	
	in the same way we get for the second term (we can as always give more details on request):
	
	By adding the two terms and expressing the partial derivatives of the functions $f_x,f_y$ in function of the partial derivatives of the functions $f_r,f_\phi$ using the relations:
	
	We get:
	
	After simplification:
	
	The expression of the operator "\NewTerm{divergence in polar coordinates}\index{divergence in polar coordinates}" is then:
	
	To determine the expression of the divergence operator in cylindrical coordinates let us recall the relations:
	
	Given now a vector function $\vec{f}:\mathbb{R}^3\rightarrow \mathbb{R}^3$. We have:
	
	As we know the expressions of $\vec{e}_r,\vec{e}_\phi,\vec{e}_z$ in function of the $\vec{e}_x,\vec{e}_y,\vec{e}_z$, from the above expressions we deduce:
	
	The divergence of $\vec{f}$ being defined in the three dimensional case by:
	
	we then have:
	
	The first term is equal to (application of the gradient in cylindrical coordinates):
	
	in the same way we get for the second component (we can give the details on request):
	
	and for the last one:
	
	By adding the three terms and expressing the partial derivatives of the functions $f_x,f_y,f_z$ in function of the partial derivatives of the functions $f_r,f_\phi,f_z$ using the relations:
	
	we get:
	
	After simplification:
	
	The expression of the operator "\NewTerm{divergence in cylindrical coordinates}\index{divergence in cylindrical coordinates}" is then:
	
	To find the expression of the divergence in spherical coordinates, let us recall the relations:
	
	Given now a vector function $\vec{f}:\mathbb{R}^3\rightarrow \mathbb{R}^3$. We have:
	
	Knowing the expression of $\vec{e}_r,\vec{e}_\theta,\vec{e}_\phi$ depending on $\vec{e}_x,\vec{e}_y,\vec{e}_z$, from the expression above we deduce:
	
	The divergence of $\vec{f}$ being defined in the three dimensional case by:
	
	we then have:
	
	The first component is equal to (application of the gradient in spherical coordinates):
	
	in the same way we get for the second component (we can give the details on request):
	
	and finally for the third and last one:
	and:
	
	By adding the three terms and expressing the partial derivatives of the functions $f_x,f_y,f_z$ in function of the partial derivatives of the functions $f_r,f_\theta,f_\phi$ using the relations:
	
	we get (we can develop the intermediate details on request):
	
	We notice that we can regroup terms depending on the same variable using the property of the derivative, so we get for the expression of the divergence in spherical coordinates:
	
	and therefore the operator of "\NewTerm{divergence in spherical coordinates}\index{divergence in spherical coordinates}" is:
	
	So we finally we saw all the expressions of the divergence operator of a vector field in Cartesian, polar, cylindrical and spherical systems.
	
	\pagebreak
	\subsubsection{Rotationals of a Vector Field (Curl)}
	The "\NewTerm{curl}\index{curl}" or "\NewTerm{rotationnal}\index{rotational}" of a vector field can be seen (this is a simplification!) as the vector field whose field lines are perpendicular to those we have calculated the rotational as shown in the special example below (we will see more academic detailed further below):
	\begin{figure}[H]
	\centering
		\includegraphics{img/algebra/curl.jpg}
		\caption{Example of rotational of a vector field}
	\end{figure}
	In a little bit more technical way the rotational is a vector operator that describes the infinitesimal rotation of a $3$-dimensional vector field. At every point in the field, the rotational of that point is represented by a vector. The attributes of this vector (length and direction) characterize the rotation at that point. he direction of the rotational is the axis of rotation, as determined by the right-hand rule, and the magnitude of the rotational is the magnitude of rotation. 
	
	The rotational transforms a vector field in another vector field. For most people it is more difficult to accurately represent than the gradient and divergence, it intuitively reflects the tendency of a field to rotate around a point (the way it is twisted).
	
	Let us give before tackling with the mathematical stuff and also mathematical examples two every-day life examples:
	\begin{tcolorbox}[colframe=black,colback=white,sharp corners]
	\textbf{{\Large \ding{45}}Examples:}\\\\
	E1. In a tornado, the wind turns around the eye of the storm and the wind velocity vector field has a non-zero rotational around the eye.\\

	E2. The rotational of the velocity field of a disc that rotates at a constant speed is constant, directed along the axis of rotation and oriented such that the rotation takes place, in relation to it, in the direct sense.\\
	\end{tcolorbox}
	A vector field is said to by "\NewTerm{irrotational}\index{irrotational}" when the rotational of this field is identically zero at all points of space. Otherwise, we say it is a "\NewTerm{vortex}\index{vortex}".
	
	In the usual case where $\mathrm{d}x$ is an element of length, the measurement unit of the rotational is then the unit of the considerated field divided by a unit length. For example, in fluid mechanics: the unity of the rotational of a velocity field is radians per unit time, as an angular velocity as we divide a velocity ([ms$^{-1}$] by the length [m]!
	
	The divergence gives some indication of the behavior of a vector or a vector field: how it moves in relation to the normal and how it crosses the surface, but it is insufficient. Take a field which would have the shape of a cylinder and another field which have a helicoidal form of the same diameter as the cylinder. If the move  in the same direction the divergence will be the same even if the movements are quite different This requires that we determine how the field is bent as it passes through a surface: this will be determined by the circulation (as the work of a force, for example) of the vector along a closed curve, obtained with the sum of dot products $\vec{f}\circ \mathrm{d}\vec{r}$ on the closed contour (\SeeChapter{see section Differential and Integral Calculus}):
	
	in fact it's the same to look at how twisted is the vector with respect to the normal vector of the surface which leads us to define the "rotational" or "swirl vector" by writing:
	
	that thus establishes a relation between the line integral and the surface integral (we then transform a line integral on a closed path in a surface integral delimited by the given path).
	
	\begin{theorem}
	In other words, the rotational is calculated by using the fact that the flow around a closed basic path of a vector field is equal to the flux of its rotational through the immediate elementary surface generated by this path.

	This is the "\NewTerm{Stokes theorem}\index{Stokes theorem}" (which is more rigorously demonstrable with a heavy mathematical formalism) which is in fact a definition of the rotational operator which we will seek the explicit mathematical expression right now!
	\end{theorem}
	\begin{dem}
	Given $\vec{f}$ a vector field defined in a given space. We want to calculate the circulation of $\vec{f}$ around a closed path (contour) $C$:
	
	We choose for contour $C$ the edged of an infinitesimal rectangle $(\mathrm{d}x,\mathrm{d}y)$ that is into $\mathbb{R}^3$and parallel to the $xy$-plane (note that we travel the contour so as to always have the surface to our left!):
	\begin{figure}[H]
		\centering
		\includegraphics{img/algebra/rotational_contour_path.jpg}
		\caption[]{Contour (path) of integraton}
	\end{figure}
	For the two horizontal sides (edges), the contribution to the circulation is:
	
	Which authorize us to write:
	
	Same for the vertical sides  (edges) we have also:
	
	Therefore we have the circulation following $z$:
	
	Which can also be written in the following more general and important form:
	
	and is the no less than the famous "\NewTerm{Green theorem}\index{Green theorem}" or also known as the "\NewTerm{Green-Riemann theorem}\index{Gree-Riemann theorem}" that we will see again in the section of Complex Analysis.
	
	And that will will write in the situation that interest us:
	
	By circulation permutation we then get:
	
	Either in vector condensed form:
	
	This allows us to better understand the notation, or the non intuitive definition of the rotational in many books and that is:
	
	that is to say the cross product of the gradient operator by the vector field!
	
	So finally we have proved the Stokes theorem or also named in this form "\NewTerm{curl theorem}\index{curl theorem}" that gives well:
	
	and at the same time the rotational in Cartesian coordinates.
	\begin{flushright}
		$\square$  Q.E.D.
	\end{flushright}
	\end{dem}
	\begin{tcolorbox}[colframe=black,colback=white,sharp corners]
	\textbf{{\Large \ding{45}}Example:}
	Take the vector field, which depends on $x$ and $y$ linearly:
	
	Its plot look like this in Maple 4.00b:\\
	
	\texttt{
	>with(DEtools): with(plots):\\
	>fieldplot([y, -x], x=-5..5, y=-5..5,arrows = medium, \\
	color = sqrt(x \string^2 + y\string^2),thickness=2,labels=[`x`,`y`],\\
	title=`Simple vector field`);	
	}
	\begin{figure}[H]
		\centering
		\includegraphics{img/algebra/vector_field_01.jpg}
		\caption[]{Vector field example with Maple 4.00b}
	\end{figure}
	Simply by visual inspection, we can see that the field is rotating. If we place a paddle wheel anywhere, we see immediately its tendency to rotate clockwise. Using the right-hand rule, we expect the rotational to be into the page. If we are to keep a right-handed coordinate system, into the page will be in the negative $z$ direction. The lack of $x$ and $y$ directions is analogous to the cross product operation.\\

	If we calculate the rotational:
	
	\end{tcolorbox}
	
	\pagebreak
	\begin{tcolorbox}[colframe=black,colback=white,sharp corners]
	Avec Maple 4.00b nous obtenons ce résultat algébrique avec les commandes suivantes:\\
	
	\texttt{>with(linalg):\\
	>f:=[y,-x,0];v:=[x,y,z];\\
	>curl(f,v);\\
	}
	
	As we did yet not have the time to found an easy way to plot the resulting rotational vector field in the release 4.00b of Maple we will take the picture provided by Wikipedia:
	\begin{figure}[H]
		\centering
		\includegraphics[scale=0.65]{img/algebra/rotational_of_vector_field_01.jpg}
		\caption[]{Rotational of previous vector field (source: Wikipedia)}
	\end{figure}
		
	E2. Suppose we now consider a slightly more complicated vector field:
	
	Its plot:
	\begin{figure}[H]
		\centering
		\includegraphics[scale=0.65]{img/algebra/vector_field_02.jpg}
		\caption[]{second vector field example (source: Wikipedia)}
	\end{figure}
	\end{tcolorbox}
	
	\begin{tcolorbox}[colframe=black,colback=white,sharp corners]
	We might not see any rotation initially, but if we closely look at the right, we see a larger field at, say, $x=4$ than at $x=3$. Intuitively, if we placed a small paddle wheel there, the larger "current" on its right side would cause the paddlewheel to rotate clockwise, which corresponds to a rotational in the negative $z$ direction. By contrast, if we look at a point on the left and placed a small paddle wheel there, the larger "current" on its left side would cause the paddlewheel to rotate counterclockwise, which corresponds to a rotational in the positive $z$ direction. Let's check out our guess by doing the math:\\
	
	Indeed the rotational is in the positive $z$ direction for negative $x$ and in the negative $z$ direction for positive $x$, as expected. Since this rotational is not the same at every point, its plot is a bit more interesting:
	\begin{figure}[H]
		\centering
		\includegraphics[scale=0.65]{img/algebra/rotational_of_vector_field_02.jpg}
		\caption[]{Rotational of previous vector field (source: Wikipedia)}
	\end{figure}
	\end{tcolorbox}
	\pagebreak	
	Let us now determine the expression of the rotational in cylindrical coordinates (the rotational in polar coordinates is not defined in $\mathbb{R}^2$).
	
	Using the same technique as for the rotational in Cartesian coordinates, we write the circulation of $\vec{f}$ along a contour corresponding to a small piece $P_1P_2P_3P_4$ of an orthogonal cylinder (oriented in the direction of the $z$-axis):
	\begin{figure}[H]
		\centering
		\includegraphics{img/algebra/rotational_cylindrical.jpg}
		\caption[]{Representation of the cylinder piece $P_1P_2P_3P_4$}
	\end{figure}
	We have then by fixing $z$ (caution! the $\mathrm{d}\vec{r}$ has nothing to do with the cylinder radius $r$... the notation can be confusing I'm sorry!):
	
	the total circulation thus gives after regrouping terms:
	
	We can not at this stage compare with the rotational because it is difficult to us to make appearing the differential of the surface if we look at the differentials that currently appear in circulation. The best is then to divide everything by $r\mathrm{d}\phi\mathrm{d}r$:
	
	Therefore:
	
	Now we determine the rotational by fixing $\varphi$. The problem is like having a rectangle in the space that we travel to determine the circulation. But we already know what is the result of the rotational for a rectangle in Cartesian coordinates following the $z$-axis:
	
	except that in cylindrical coordinates we have to replace $z$ by $\varphi$, $x$ by $y$, $y$ by $r$ and $f_y$ by $f_r$ and finally $f_x$ by $f_z$ (this choice is always appropriate simply because the circulation is such that the surface is always on our left). This gives us:
	
	It therefore only remains to us to find the component of the rotational on $r$ (therefore when $r$ is fixed). The calculation is more difficult as we have to follow (positively always!) a curved surface by the variation of the angle $\varphi$.
	
	We then have by fixing $r$:
	
	the total circulation thus gives after regrouping terms:
	
	We can not at this stage compare with the rotational because it is difficult to us to make appearing the differential of the surface if we look at the differentials that currently appear in circulation. The best is then to divide everything by $r\mathrm{d}\phi\mathrm{d}z$:
	
	Then finally:
	
		And finally we have the "\NewTerm{rotational in cylindrical coordinates}\index{rotational in cylindrical coordinates}" given in its globality by:
	
	The reader can check verify that this result is simply the gradient in cylindrical coordinates applied to the vector field $\vec{f}$.
	
	To be convinced, let us now show directly the expression of the rotational in spherical coordinates by showing this through the cross product of the gradient in spherical coordinates with the vector field $\vec{f}$.
	
	First let us recall that we have obtained for the gradient in spherical coordinates:
	
	Therefore we have:
	
	what we can write using the decomposition in basis vectors:
	
	Thanks to the partial derivatives that we proved earlier during our introduction to the spherical coordinates, it comes:
	
	The cross products with the colinears vectors canceled. Therefore it remains:
	
	As the cross product of two basis vectors give the corresponding orthogonal vector (positively or negatively) then we have:
	
	By regrouping the terms it comes:
	
	Thus by simplifying:
	
	Thus finally:
	
	
	\pagebreak
	\subsubsection{Laplacians of Scalar Field (Laplace Operator)}
	The Laplacian of a scalar field $\phi(x_1,x_2,x_3)$ give also a scalar field that gives the difference between the value of the function $\phi$ on one point and it average around this point. In other words: the second partial derivative measure the variations of the slope on the study point in its immediate neighborhood and following one direction at a time. If the second partial derivative is null following $x$, then the slope is constant in its immediate neighborhood and following this dimension (direction), this implies that the value of the function at the study point is the average of its neighborhood (following one dimension).
	
	The reader will be able to see again major practical applications of this differential operator in the sections of Complex Analysis, Quantum Chemistry, Astronomy, Electrodynamics, Weather \& Marine Engineering, Wave Mechanics, Wave Quantum Physic and Quantum Field Theory.
	
	This operator is defined from the divergence and the gradient and we denote it by (tensor notations):
	
	The Laplacian is null, or quite small, when the function varies. The functions satisfying the "\NewTerm{Laplace equation}\index{Laplace equation}":
	
	are named "\NewTerm{harmonic function}\index{harmonic function}".
	
	Thus the "\NewTerm{scalar Laplacian operator in cartesian coordinates}\index{scalar Laplacian operator in cartesian coordinates}" is by this definition, given by:
	
	The Laplacian of a scalar field in other coordinate systems is a little bit more hard to get that for the other differential operators. There are more than one possible proof but among the existing one we have try to choose (as always) this that seem to us the most interesting in the point of view of the tools used (and not of simplicity!).

	Given the Laplacian in cartesian coordinates in $\mathbb{R}^2$ of a scalar field $f$:
	
	To determine this expression in polar coordinates, we will use the total exact differential and the rule chain in polar coordinates (\SeeChapter{see section Differential and Integral Calculus}):
	
	therefore for a second derivative:
	
	but we know that we have in polar coordinates:
	
	hence for the first derivative:
	
	and for the second derivative:
	
	therefore:
	
	and given that the second partial derivatives are continuous, then the cross derivatives are equal according to the Schwarz theorem (\SeeChapter{see section Differential and Integral Calculus}):
	
	Therefore:
	
	Similarly, we will have:
	
	hence the expression of the Laplace operator in polar coordinates by adding the last two expressions:
	
	Therefore the "\NewTerm{scalar Laplacian in polar coordinates}\index{scalar Laplacian in polar coordinates}" is finally given by:
	
	To find the expression of the Laplacian operator in spherical coordinates, we will use the intuition of the physicist and the concepts of similarity.
	
	We will first of all help us with the below figure to find out what we mean:
	\begin{figure}[H]
		\centering
		\includegraphics{img/algebra/coordinate_system_spherical_for_Laplacian_study.jpg}
		\caption[]{Recall of the spherical coordinate system representation}
	\end{figure}
	Recall that the relation between cartesian and spherical coordinates are given by the relations:
	
	We will now consider the following similarities:
	\begin{enumerate}
		\item Cylindrical coordinates:
			
		
		\item Spherical coordinates:
			
	\end{enumerate}
	Let us build a correspondence table:
	
	The goal is to play with this correspondence with in a first time the Laplacian in cylindrical coordinates where we have subtracted from both sides of the equality the term $\dfrac{\partial^2 f}{\partial z^2}$. Therefore:
	
	let us now use our small correspondence table and we get:
	
	The second term of the equality of the latter relation is the spherical equivalent of the term \#1 of the Laplacian in cylindrical coordinates:
	
	Now let us examine and focus on the term: $\dfrac{1}{\rho}\dfrac{\partial f}{\partial \rho}$
	
	Identically as when we determined the relation:
	
	we get:
	
	with:
	
	Which give us the possibility to write:
	
	If we play again with our small correspondence table we get:
	
	We divide the latter relation by $\rho$ and we get:
	
	We have therefore above the spherical equivalent of the second term \#2 of the Laplacian in cylindrical coordinates:
	
	The last and third term is quite simple to determine. We just have to replace $\rho$ by $r\sin(\theta)$ to get:
	
	By bringing together all terms obtained previously, we finally get the extended form of the Laplacien in spherical coordinates used so much in physics (see corresponding sections of this book):
	
	We can shorten this expression by factoring the terms:
	
	If we condense even a little bit more, we get the final expression of the "\NewTerm{scalar Laplacian in spherical coordinates}\index{scalar Laplacian in spherical coordinates}" named also "\NewTerm{spherical Laplacian}":
	
	
	\subsubsection{Laplacians of a Vector Field}
	As to the Laplacian of a scalar field, the Laplacian of a vector field is only a very convenient notation system for condensing the writing of the components of a vector field.
	
	The reader will also find practical applications of this operator in the sections of Electrokinetics, Electrodynamics and Continuum Mechanics.
	
	Thus, the vector Laplacian is often defined by:
	
	We also prove that in the specific case of the Cartesian coordinates, the Laplacian of a vector field has the components the scalar Laplacian of each of the components.
	
	We also prove that in the specific case of the Cartesian coordinates, the Laplacian of a vector field has the components the scalar Laplacian of each of the components.
	
	So therefore have in Cartesian coordinates:
	
	So we therefore in Cartesian coordinates:
	
	and thus the "\NewTerm{vector Laplacian of a vector field in Cartesian coordinates}\index{vector Laplacian of a vector field in Cartesian coordinates}" is indeed the scalar Laplacien of each component:
	
	Or more explicitly:
	
	The Laplacian of a vector field, frequently named "\NewTerm{vectorial laplacien}\index{vectorial laplacien}", in other coordinates systems is quite simple to get once we know the Laplacian of a scalar field in the same coordinates!
	
	We have first in cylindrical coordinates:
	
	To simplify (because of a lack of space) let us focus first on the first line:
	
	and after for the second line:
	
	and finally the third one:
	
	So what gives for us for the "\NewTerm{vector Laplacian in cylindrical coordinates}\index{vector Laplacian in cylindrical coordinates}" as we can find it in tables or forms:
	
	To finish and in the joy (...) let us make the merry and detailed calculations of the vector Laplacian operator in spherical coordinates (it's quite long but it's just to make sure that we fall back on what is in tables and forms):
	
	Let us focus on the first line (Caution! this will be quite long...):
	
	\pagebreak
	
	That's it for the first line ... Let us get on to the second line always with joy...:
	
	
	\pagebreak
	and finally a last effort for the third and last line:
	

	\pagebreak
	
	So what gives for the "\NewTerm{vector Laplacian in spherical coordinates}\index{vector Laplacian in spherical coordinates}" as we can find it in tables or forms:
	
	that's it... for the skeptics...
	
	\subsubsection{Remarkable Identities}
	The scalar and vector differential operators have some very simple remarkable identities that we will find very often in physics in this book.
	
	Let us first see the relation that make no sense (in case you would fall on them without purpose...) :
	
	For the relation above, the rotational (curl) of a divergence does not exist since the rotational operator applies to a vector field while the divergence is a scalar!
	
	For the above relation the rotational (curl) of a scalar Laplacian does not exist since the rotational operator applies to a vector field while by construction, the Laplacian is a scalar.
	
	Let us now see some remarkable identities without proof for the majority (if there is a proof this is because one reader did the request to have all the details...):
	\begin{enumerate}
		\item By construction the scalar Laplacian is the divergence of the gradient of the scalar field:
		
		
		\item The rotational (curl) of the gradient is equal to zero:
		
		Therefore if the rotational (curl) of a vector variable (vector field) is zero, this same variable can be expressed as the gradient of a scalar potential!!!!!!!!!!!! This is a veeeeerrrrry important property (or trick depending of the point of view...) in Electromagnetics and Fluid Mechanics and Quantum Physics!
		\begin{dem}
		
		\begin{flushright}
			$\square$  Q.E.D.
		\end{flushright}
		\end{dem}
		
		\item The dot product of two rotational is equal to something boring to say just with words...:
		
		
		\item The divergence of the rotational (curl) of a vector field is always equal to zero:
		
		\begin{dem}
		
		\begin{flushright}
			$\square$  Q.E.D.
		\end{flushright}
		\end{dem}
		
		\item The rotational (curl) of the rotational  of a vector field is equal to the gradient of the divergence of this vector field less its vector Laplacian:
		
		\begin{dem}
		
		It is then easy to check that this last equality is equal to:
		
		\begin{flushright}
			$\square$  Q.E.D.
		\end{flushright}
		\end{dem}
		
		\item The multiplication of the nabla operator by the dot product of two vectors is equal to ... (see below), which provides a very useful relation in Fluid Mechanics:
		
		
		\item The scalar product of the rotational (curl) of a vector is the difference of the commutated operators such that (we can provide the detail proof on request):
		
		We will use this last relation in our study of electromagnetic radiation pressure in the section of Electrodynamics (among others...).
		
		\item The gradient of a cross product is the difference of the commutated operators such that (we can provide the detail proof on request)
		
		We will use this last relation in our study of superconductors in the section of Electrokinetics.
	\end{enumerate}
	
	\pagebreak
	\subsubsection{Summary}
	As part of this book, we will use the different notations presented and summarized in the table below. Their usage gives us the possibility in the context of the different theories to avoid confusion with other mathematics being (tools). It's annoying but we have to do with it.
	
	
	Now let us do a quick summary of main differential operators:
	\begin{itemize}
		\item The gradient can be assimilated to the "slope" (example: the electric field is the "slope" of the electrostatic potential).
		
		The various expressions of the gradient operator (placed under the form of the nabla operator) in Cartesian, polar, cylindrical, spherical coordinates are the following:
		
		
		\item The divergence characterizes a flow of something that comes from somewhere, a source, or who goes to it. If the divergence is is different from zero, it means that there is concentration around a point, so the density increases (or decreases, it depends on the sign). It could be the density of electric charges or the mass density. Hence the famous theorem that says that the flow (that which passes through a surface) is equal to the integral of the divergence (what remains).
		
		The various expressions of the divergence operator (placed under the form of the nabla operator) in Cartesian, polar, cylindrical, spherical coordinates are the following:
		
		
		\item The rotational characterizes the existence of a vortex (Widely used in fluid mechanics). If there is a whirlwind, we can follow a flow line on a closed curve (closed: in the differential point of view, not in the geometrical one!) without it change of direction: the circulation will the not be equal to zero (it is equal to integral of the rotational (curl)).
		
		The various expressions of the rotational (curl) operator (placed under the form of the nabla operator) in Cartesian, cylindrical, spherical coordinates are the following:
		
		
		\item The Laplacian of a scalar field gives a scalar field that measures the difference between the value of the function at a point and its average around that point. In other words, the partial second derivative measure the variations of the slope at the point examined in the immediate surroundings and in one dimension at a time. If the partial second derivative is zero in one direction, then the slope is constant in the immediate surroundings and according to this dimension, this means that the value of the function on the study is equal to the average of his neighborhood (in one dimension).
		
		The different expressions of the scalar Laplacian operator (placed under the form of the nabla operator) in Cartesian, polar and spherical coordinates are:
		
		
		\item As for the Laplacian of a scalar field, the Laplacian of a vector field is only a very convenient notation system for condensing the writing of the components of a vector field.
		
		The different expressions of the scalar Laplacian operator (placed under the form of the nabla operator) in Cartesian, polar and spherical coordinates are:
		
	\end{itemize}
	And also we have the following list of remarkable identities):
	
	
	\pagebreak
	Finally let us finis this summary with all the theorem that we have obtained so far in this section that are named "\NewTerm{$1$st order Integral Theorems}\index{First order Integral Theorems}":
	\begin{itemize}
		\item Gradient theorem (only for uniform field):
		
	
		\item Ostrogradsky theorem (divergence theorem):
		

		\item Green theorem (Green-Riemann theorem):
		
	
		\item Stokes theorem (curl theorem):
		
	\end{itemize}
	
	\begin{flushright}
	\begin{tabular}{l c}
	\circled{95} & \pbox{20cm}{\score{4}{5} \\ {\tiny 99 votes,  84.44\%}} 
	\end{tabular} 
	\end{flushright}
	
	%to make section start on odd page
	\newpage
	\thispagestyle{empty}
	\mbox{}
	\section{Linear Algebra}

	\lettrine[lines=4]{\color{BrickRed}T}here are several approaches to learn Linear Algebra. First a pragmatic way (we'll start with this one because our experience has shown us that it is the one that seemed to work best for students) and a more formal way that we will present in after. We should first warn the reader that linear algebra is a powerful calculation tool that we use enormous in economic and industrial practice in the following areas (see the respective chapters of the book for specific examples): Statistics , Electrotechniques, Finance market, Numerical optimization methods, Optics, Quantum Physics, Electrodynamics, Relativity, Fluid mechanics, etc. It is then necessary to attach a special attention to this subject.
	
	First let us answer at  two questions from a reader: Why Linear Algebra is named like this? And is there a Non-Linear Algebra?
	
	Here are my answers:

	\begin{enumerate}
		\item This is named "Linear Algebra" because it was first necessary to choose a name... and also because it is a generalization of the scalar algebra but with vectors where applications are no longer scalar functions but matrix applications whose effect is to act as a linear sum of a base vectors (at least this can be interpreted as such).
		\item Officially and to my knowledge there is no non-linear algebra in the same philosophy as what we will see in this section. It seem that they are mathematicians who have created "non linear algebra" but these have nothing to do with matrices.
	\end{enumerate}

Now, remember that we studied in section Calculus how to determine the intersection (if any exist) of the equation of two lines in $\mathbb{R}^2$  (we can extend the problem obviously to more than two lines), as it is equivalent to solve a polynomial of order $1$, given by:
	
where $a_i,b_i \in \mathbb{R}$.

Thus seeking the value for which:
	
leads us to write:
	
However, there is another way of presenting the problem as we have seen in the Numerical Methods section (\SeeChapter{Theoretical Computer chapter}). Indeed, we can write the problem in the form of a block of equations:
		
and as we seek $y_1=y_2=y$, we have:
		
This writing is named as we have seen in the section of Theoretical Computing (\SeeChapter{see chapter of Theoretical Computing}) a "\NewTerm{linear system}\index{linear system}" that we can solve by subtracting or adding the lines between them (all the solutions are always equal), which gives us:
	
and we see that we fall back on the solution:
	
So there are two ways to present a problem of intersection of lines:
	\begin{enumerate}
		\item In the form of an equation
		\item In the form of a system
	\end{enumerate}
We will focus in a part of this section to the second method that will allow us with tools seen in the section of Vector Calculus to resolve not only the intersections of one or more straight lines but one or more straight, plans, hyperplanes, etc. in respectively $\mathbb{R},\mathbb{R^2},...,\mathbb{R^n}$. 

	Obviously we will see that Linear Algebra is not only use for this purpose but can also be used the generalized some physics mathematical models or to express geometrical transformations of vectors or figures in 2D or 3D or also express Markov Chains, special properties of Multivariate Statistics, calculation of some differential equation (for reliability engineering for example) and much more! Many examples are given in this book about the application of Linear Algebra in real life.

Before attacking the theoretical part, let us presented a very interesting example but that requires a concept - the determinant - that we will prove rigorously much further in detail in this section (it seemed to us more pedagogical to approach this subject now rather than the reader must wait to browse dozens of mathematical developments of pages before reaching the rigorous definition of the determinant).

Consider the system of two linear equations with two unknowns (system of intersection of planes):
	
If we solve this, we quickly get (technique named "\NewTerm{substitution method}\index{substitution method}"):
	
It comes then:
	
and so at the end:
	
and if we define a little bit fast something that is named the "\NewTerm{determinant}\index{determinant}" that we will see further below rigorously as follows:
	
or with another very more common notation:
	
we thus have:
	
And by proceeding in the same way we get:
	
It comes then:
	
and so finally we get:
	
It then appears clear that if:
		
the system has infinitely many solutions. In contrast, the system has no solution if:
	
And if the reader repeats (happily...) the procedure for a system of three equations with three unknowns of the type (intersection of hyperplanes):
	
We then get (after some basic boring algebraic operations):
	
with:
	
	It then appears clear that if:
	
	the system has infinitely many solutions. In contrast, the system has no solution if:
	
	and so on for $n$ equations with $n$ unknowns.

	However, there was a condition to satisfy: as we have seen in the previous example, we could not solve a system of equations with two unknowns if we have only one equation. That is why it is necessary and sufficient for a system of equations with $n$ unknowns to have at least $n$ equations. Thus, we speak of "\NewTerm{systems with $n$ equations in $n$ unknowns}\index{systems with $n$ equations in $n$ unknowns}". We also prove that it is necessary and sufficient that the determinant is non-zero for a linear system, whose matrix equivalent is square, to have a unique solution (the concept of "determinant" and "matrix" will be defined further below robustly) and therefore that the corresponding matrix to the system is invertible (non-singular).

\pagebreak
\subsubsection{Linear Systems}

\textbf{Definitions (\#\mydef):}
\begin{enumerate}
	\item[D1.] We name "\NewTerm{linear system}\index{linear system}", or simply "\NewTerm{system}" any family of equations of the form:
	
where each line represents the equation of a line, plane or hyperplane (\SeeChapter{see section Analytical Geometry}), and $a_{mn}$ are the "\NewTerm{system coefficients}\index{linear system coefficients}", the $b_m$ the "\NewTerm{coefficients of the second member}\index{coefficients of the second member}" and the $x_n$ the "\NewTerm{unknowns of the system}\index{unknowns of the system}".

	\item[D2.] If the system has $n$ unknown and $n$ equations and has a unique solution, we then name it a "\NewTerm{Cramer system}\index{Cramer system}" (1750).

	\item[D3.] If the coefficients of the second member are all zero, then we say that the system is a "\NewTerm{homogeneous system}\index{homogeneous system}" so it has at least the trivial solution where all $x_n$ are equal to zero.
	
	\item[D4.] We name "\NewTerm{homogeneous system associated to the system}\index{homogeneous system associated to the system}", the system of equations we get by substituting zeros to the coefficients of the second member ($b_m$).
	\end{enumerate}
Let us now recall the following items:
	\begin{itemize}
		\item The equation of a line (\SeeChapter{see section Functional Analysis}) is given by:
			
		by defining $x=x_1$ and $y=x_2$.
		\item The equation of a plane (\SeeChapter{see section Functional Analysis}) is given by:
			
		by defining $x=x_1, y=x_2, z=x_3$.
	\end{itemize}
We often write a linear system in the following condensed form:
	
We name "\NewTerm{system solution}\index{linear system solution}" or "\NewTerm{vector system solution}\index{vector system solution}" any $t$-uple $(x_1^0,x_2^0,...,x_n^0)$ such as:
	
Solve a system means finding all the solutions of this system (we find many such systems in economic, operational research or design of experiments). Two system with $n$ unknowns are named  "\NewTerm{equivalent systems}\index{equivalent linear systems}" if all of the solution of one system is the solution of the other, i.e., if they have the same set of solutions. We sometimes say that the equations of a system are "\NewTerm{compatible equations}\index{compatible equations}" or "\NewTerm{incompatible equations}\index{incompatible equations}", depending on whether the system has at least one solution or not admit any.

We can also give for sure a geometric interpretation to these systems. Suppose that the first members of the equations of the system are not zero. So we know that each of these equations represents a hyperplane of an affine space (\SeeChapter{see section Vector Calculus}) of dimension $n$. Therefore, all solutions of the system, considered as set of $n$-tuples of coordinates is a finite intersection of hyperplanes.

	\begin{tcolorbox}[colframe=black,colback=white,sharp corners]
	\textbf{{\Large \ding{45}}Example:}
	
	noted conventionally in high-school classes in the form:
	
	This system has for solutions the points representing the intersection of the three planes defined by the three equations. But as we can see it visually with Maple 4.00b using the following commands:\\

	\texttt{>with(plots):}\\
	\texttt{>implicitplot3d({x-3*z=-3,2*x-5*y-z=-2,x+2*y-5*z=1},x=-3..3,}
	\texttt{y=-3..3,z=-3..3);}
	
	\begin{figure}[H]
	\centering
	\includegraphics[scale=0.35]{img/algebra/system_3_equations_3_unknowns.eps}
	\caption{Graphical representation of the linear system}
	\end{figure}
	
	There is no solutions. You can be checked by hand or with Maple 4.00b by writing:\\
	
	\texttt{>solve({x-3*z=-3,2*x-5*y-z=-2,x+2*y-5*z=1},{x,y,z});}
	\end{tcolorbox}

	\begin{tcolorbox}[title=Remark,colframe=black,arc=10pt]
For "classic" resolution methods of such systems, we refer the readers to the section on Numerical Methods of the chapter on Computing Science.
	\end{tcolorbox}	
	
Finally, note an important case in practice: "\NewTerm{overdetermined systems}\index{overdetermined systems}" where we have more equations than unknowns. The first situation of this type dates from the 18th century through the study of the lunar oscillations but we also find this frequently in R\&D laboratory in the context of design of experiments (\SeeChapter{see section Industrial Engineering}) or in structural equation models (\SeeChapter{see section Theoretical Computing}).

	\begin{tcolorbox}[colframe=black,colback=white,sharp corners]
\textbf{{\Large \ding{45}}Example:}\\\\
Consider the special case but very telling following examples of three equations with two unknowns:
	
system that will be written in the form of matrix and vector as following:
	
or said in an "\NewTerm{augmented form}\index{linear system augmented form}" as follows
	
and with Maple 4.00b:\\

\texttt{>with(plots):}\\
\texttt{>implicitplot({2*x+3*y=-1,-3*x+y=-2,-x+y=1},x=-3..3,y=-3..3);}
	\end{tcolorbox}

	\begin{tcolorbox}[colframe=black,colback=white,sharp corners]
\begin{figure}[H]
\centering
\includegraphics[scale=0.7]{img/algebra/overdetermined_system.eps}
\caption{Representation of a system of 3 equations with two unknowns with Maple 4.00b}
\end{figure}

We can see on the chart above that this particular system has no complete solution but thus has a  solution if we take only the problem by pair of equations... which does not necessarily help in facts...
	\end{tcolorbox}

Notice, that the we have just seen that the system could be written as following:
	

	
This looks like a multiple linear regression system (see section of Theoretical Computing) whose column vector of unknowns can be viewed as the coefficient vector $\vec{\beta}$ of the regression line such that:
	
We then proved in detail in the section of Theoretical Computing that:
	

at the condition that the square matrix $X^TX$ is, as we will see further below, invertible (non-singular). If this is satisfied, we find a "\NewTerm{pseudo-solution}\index{pseudo linear solution}" (this is the official terminology...) by making calculations quickly by hand (or with a spreadsheet software like Microsoft Excel):
	
and injecting these values in the initial overdetermined system, the reader will quickly understand why we talk about "pseudo-solution"...

	\begin{tcolorbox}[title=Remark,colframe=black,arc=10pt]
A reader asked us why we don't use for the example above just the relation $\vec{\beta}=\vec{y}X^{-1}$? The answer is simple! Because $X$ is not a square matrix or in other words... it is in our example over-determined this is why we can only found a "pseudo-solution".
	\end{tcolorbox}

This was the pragmatic way of looking at it ... now let us turn to the second slightly more mathematical way ... (but still relatively simple):

\subsection{Linear Transformations}
	
\textbf{Definition (\#\mydef):} A "\NewTerm{linear transformation}\index{linear transformation}" or "\NewTerm{linear application}\index{linear application}" $A$ is a mapping of a vector space $E$ to a vector space $F$ such that with $K$ being in $\mathbb{R}$ or in $\mathbb{C}$:
	
this constitutes, as a reminder, an endomorphism (\SeeChapter{see section Set Theory}).

The first property specifies that the transformation of a sum of vectors must be equal to the sum of the transformed vectors, so that it is linear. The second property specifies that the transformation of a vector to  which we applied a scale factor (scaling) must also be equal to the same factor applied on the transformation of the original vector. If either of these two properties is not met, the transformation is therefore not linear.

We now show that any linear transformation can be represented by a matrix!

Given the basis vectors $\left(\vec{v}_1,\vec{v}_2,...,\vec{v}_n \right)$ of $E$ and $\left(\vec{u}_1,\vec{u}_2,...,\vec{u}_n \right)$ for those of $F$ with $m \geq n$. With these bases, we can represent any vectors $\vec{x} \in E, \vec{y} \in F$ with the following linear combinations (\SeeChapter{see section Vector Calculus}):
	
	Consider the linear transformation $A$ which applies $E$ to $F$ ($A:E\mapsto F$). So:
			
	that we can rewrite as follows:
	
	But since $A$ is a linear operator by definition, we can also write:
	
	Considering now that the vectors $A(\vec{v}_j)$ are elements of $F$, we can rewrite them as a linear combination of its basis vectors:
	
	Therefore, we get:
	
	By reversing the order of summations, we can write:
	
	and rearranging the latter relation, we produce the result:
	
	Finally, remembering that the basis vectors $\vec{u}_i$  must be independent, we can conclude that their coefficients must necessarily be zero, so:
	
	Which corresponds to the "\NewTerm{matrix product}\index{matrix product}":
	
	That we can write:
	
	In other words, any linear transformation can be described by a matrix $A$ that is multiplied with the vector that we want to transform, to obtain the vector resulting from the transformation.
	
	\pagebreak
	\subsection{Matrices}
	So we call a "\NewTerm{matrix}\index{matrix}" with $m$ rows and $n$ columns, or "\NewTerm{type $m\times n$ matrix}" (the first term always correspond to the rows and the second to columns to the second, to remember this there is a good mnemonic trick: President Lincoln - abbreviation of Lin(e) and Col(um)n ...), any array of numbers in a ring $\mathbb{K}$ (that is most of times $\mathbb{R}$):
	
	We often denote a matrix of type $m\times n$ briefly by:
	
	or simply $(a_{ij})$. In a more formal way:
	
	
	The number $a_{ij}$ is named "\NewTerm{term or coefficient of index $i, j$}\index{term of a matrix}". The index $i$ is named the "\NewTerm{line index}\index{row index}" and the index $j$ the "\NewTerm{column index}\index{column index}".
	
	We denote by $M_{mn}(\mathbb{K})$ all matrices $m\times n$ whose coefficients take values in $\mathbb{K}$ (typically $\mathbb{R}$ or $\mathbb{C}$ for example).
	
	When $m=n$, we say that $(a_{ij})$ is a "\NewTerm{square matrix of order $n$}\index{square matrix}":
		
	In this case, the terms $a_{11},a_{22},...,a_{nn}$ are named "\NewTerm{diagonal terms}\index{matrix diagonal terms}" denoted by: 
	
	
	We will assign to the matrices special symbols, ie uppercase Latin letters: $A, B, ...$ for matrices and for columns-matrices symbols that will be vectorial lowercase letters $\vec{a},\vec{b},...$.
	
	We also name a matrix with a single row a "\NewTerm{line-matrix}\index{line -matrix}" and a matrix with a single column a "\NewTerm{column-matrix}\index{column-matrix}". It is clear that a matrix column is nothing but a "\NewTerm{column vector}\index{column-vector}" or simply a "\NewTerm{vector}\index{vector}" (\SeeChapter{see section Vector Calculus}). Thereafter, the rows of a matrix will be treated as lines-matrices and the columns to columns-matrices.
	
	We name "\NewTerm{zero matrix}\index{zero matrix}", and we denote by $0_{mn}$ or simply $\mathbf{0}$ any matrix in which each term is zero:
		
	Null columns-matrices are also designated by the symbol vector: $\vec{0}$.
	
	We name "\NewTerm{identity matrix of order $n$}\index{identity matrix}" or "\NewTerm{unit matrix of order $n$}\index{unit matrix}", and denote it by $I$, or simply $\mathds{1}$, the square matrix of order $n$:
	
	It can also be written using the Kronecker delta notation $\delta_{ij}$ (\SeeChapter{see section Tensors Calculus}).
	
	Caution! When we work with matrices having complex coefficients we must always use the term "identity matrix" rather than "unitary matrix" because in the field of complex numbers the unitary matrix is another mathematical object that should not be confused!
	
	We will see later that the zero matrix acts as a neutral element of the matrix addition and the unit matrix of neutral element for the matrix multiplication.
	
	The purpose of the concept of matrix will appear throughout the texts that will follow but the immediate reason for this notion is simply to allow some finite families of numbers to be designed as a rectangular array and to generalize physics theorems to multidimensional spaces.
	
	\subsubsection{Rank of a matrix}
	We will now briefly review the definition of "\NewTerm{rank of a finite family}\index{rank of a finite family}" that we saw in the section of Vector Calculus.
	
	As a reminder we name "\NewTerm{rank}" of a free family of vectors the dimension (positive integer number) of the vector subspace $S$ of $E$ that it generates.
	
	\textbf{Definition (\#\mydef):} Given $(\vec{a}_1,\vec{a}_2,...,\vec{a}_n)$ the columns of a matrix $A$, we name "\NewTerm{rank of $A$}\index{rank of a matrix}", and denote by $\text{rk}(A)$, the rank of the family of vectors $(\vec{a}_1,\vec{a}_2,...,\vec{a}_n)$. More precisely, in linear algebra, the rank of a matrix $A$ is the dimension of the vector space generated (or spanned) by its columns. This is the same as the dimension of the space spanned by its rows as we will see late. It is a measure of the "nondegenerateness" of the system of linear equations and linear transformation encoded by $A$ (see earlier above!). There are multiple equivalent definitions of rank. A matrix's rank is one of its most fundamental characteristics!
	
	In a slightly more familiar language (...) the rank of a matrix is given by the number of columns-matrices that can't be expressed by the combination and scalar multiplication of other columns-matrices of the same matrix!!
	
	Given this definition it is almost obvious that the rank of a matrix is zero if and only if the matrix is the zero matrix $\mathbf{0}$.
	
	Before we enter in more formal calculations let us see already some introducing examples:
	\begin{tcolorbox}[colframe=black,colback=white,sharp corners]
	\textbf{{\Large \ding{45}}Examples:}\\\\
	E1. The matrix:
	
	has rank $2$: the first two rows are linearly independent, so the rank is at least $2$, but all three rows are linearly dependent (the first is equal to the sum of the second and third) so the rank must be less than $3$.\\
	
	E2. The matrix:
	
	has rank $1$: there are nonzero columns, so the rank is positive, but any pair of columns is linearly dependent. Similarly, the transpose:
	
	of $A$ has rank $1$. Indeed, since the column vectors of $A$ are the row vectors of the transpose of $A$, the statement that the column rank of a matrix equals its row rank is equivalent to the statement that the rank of a matrix is equal to the rank of its transpose, i.e.:
	
	\end{tcolorbox}
	The above example show the statement that the column rank of a matrix equals its row rank is equivalent to the statement that the rank of a matrix is equal to the rank of its transpose.
	
	Before continuing we would like to indicate to the reader that later we will prove that if the rows of a matrix are independant, its determinant is non zero $\det(A)\neq 0$ and therefore $\text{rk}(A)=n$ and obviously $\text{rk}(A)=1$ when $\det(A)=0$.
	
	\textbf{Definition (\#\mydef):} A matrix is said to have "\NewTerm{full rank}\index{full rank}" if its rank equals the largest possible for a matrix of the same dimensions, which is the lesser of the number of rows and columns. A matrix is said to be "\NewTerm{rank deficient}\index{rank deficient}" if it does not have full rank.
	
	\begin{tcolorbox}[title=Remark,colframe=black,arc=10pt]
	If there is difficulty in determining the rank of a matrix there is a technique named "\NewTerm{matrix scaling}\index{matrix scaling}" that we will see just below that can do this work very quickly.
	\end{tcolorbox}
	\textbf{Definition (\#\mydef):} We name "\NewTerm{system associated matrix}\index{system associated matrix}":
	
	The mathematical object define by:
	
	that is to say, the matrix $A$ in which the terms are the coefficients of the system. We name "\NewTerm{matrix of the second member of the linear system}\index{matrix of the second member of the linear system}", or simply "\NewTerm{second member of the system}", the matrix-column $\vec{b}=(b_i)$ whose terms are the coefficients of the second member of this system. We also name "\NewTerm{augmented matrix associated of the system}\index{augmented matrix associated of a system}" the matrix $A$ obtained by adding $\vec{b}=(b_i)$ as the $(n + 1)$-th column.
	
	If we now consider an associated  system matrix $A$ and of second member $\vec{b}$. Let us always denote the columns of $A$ by $(\vec{a}_1,\vec{a}_2,...,\vec{a}_n)$. The system can then be written equivalently as a linear vector equation:
	
	Now remember a theorem that we saw and proved in the section of Vector Calculus: for the rank of a family of vectors $(\vec{x}_1,\vec{x}_2,...,\vec{x}_n)$ to be the equal to rank of augmented family $(\vec{x}_1,\vec{x}_2,...,\vec{x}_n,\vec{y})$, it is necessary and sufficient that the vector $\vec{y}$ is a linear combination of the vectors $\vec{x}_1,\vec{x}_2,...,\vec{x}_n)$.
	
	It follows that our linear system in a vector form has at least one solution $(\vec{x}_1^0,\vec{x}_2^0,...,\vec{x}_n^0)$ if the rank of the family $(\vec{a}_1,\vec{a}_2,...,\vec{a}_n)$ is equal to the rank of augmented family $(\vec{a}_1,\vec{a}_2,...,\vec{a}_n,\vec{b})$ and this solution is unique if and only if the rank of the family $(\vec{a}_1,\vec{a}_2,...,\vec{a}_n)$ is equal to $n$.
	
	Thus, for a linear system matrix of associated matrix $A$ and of second member $\vec{b}$ admits at least one solution, it is necessary and sufficient that the rank of $A$ is equal to the rank of the augmented matrix $(A|\vec{b})$. If this condition is met, the system admits a unique solution if and only if the rank of $A$ is equal to the number of unknowns in other words: if the columns of $A$ are linearly independent!!!
	
	We say that a matrix is "staggered" if its rows (lines) meet the following two conditions:
	\begin{enumerate}
		\item[C1.] Any null line is followed by lines full of zeros
		
		\item[C2.] The leading coefficient (the first nonzero number from the left, also named the "\NewTerm{pivot}") of a nonzero row is always strictly to the right of the leading coefficient of the row above it 
	\end{enumerate}
	These two conditions imply that all entries in a column below a leading coefficient are zeros.
	
	A non-zero row echelon matrix is therefore of the form (by adding and subtracting rows between them):
	
	where $j_1<j_2<...<j_r$ and $a_{1j_1},a_{2j_2},...,a_{rj_r}$ are nonzero terms. The terminal zero lines may be missing.
	
	The columns of index $j_1,j_2,...,j_r$ of an echelon matrix are clearly linearly independent. Considered as vectors-columns of $\mathbb{R}^2$, so they form a basis of this vector space. Considerating the other columns also as vectors-columns of $\mathbb{R}^n$, we deduce that they are necessarily linear combination of those of index  $j_1,j_2,...,j_r$ and therefore that the rank of the echelon matrix $M$ is $\text{rk}(A)=r$.
	
	We will note that $r$ is also the number of nonzero lines of the echelon matrix and also the rank of the lines of this matrix, since the nonzero lines are therefore clearly independent (we proved in the section Vector Calculus that the rank of lines and columns has same value with the same properties of independence).
	
	We can therefore allow ourselves to do a given number of elementary (extra) operations on the lines of matrices that we will be very useful, without changing its rank:
	\begin{enumerate}
		\item[P1.] We can swap the lines.
		\begin{tcolorbox}[title=Remark,colframe=black,arc=10pt]
		As we know the matrix can be seen just an aesthetic graphic representation of a linear system. So swap two lines does not change the system.
		\end{tcolorbox} 
		
		\item[P2.] Multiply a row by a nonzero scalar.
		\begin{tcolorbox}[title=Remark,colframe=black,arc=10pt]
		This obvioiusly not altering the linear independence of the vectors lines.
		\end{tcolorbox} 
		
		\item[P3.] Adding to an original line, a multiple of another line.
		\begin{tcolorbox}[title=Remark,colframe=black,arc=10pt]
		The original line will disappear in favor of the new that is independent of all (former) others. The system thus remains linearly independent.
		\end{tcolorbox} 
	\end{enumerate}
	
	Any matrix can be transformed into a row echelon form by a finite sequence of the previous properties here's how.
	\begin{enumerate}
		\item Pivot the matrix
		\begin{enumerate}
			\item Find the pivot, the first non-zero entry in the first column of the matrix.
			\item Interchange rows, moving the pivot row to the first row.
			\item Multiply each element in the pivot row by the inverse of the pivot, so the pivot equals $1$.
			\item Add multiples of the pivot row to each of the lower rows, so every element in the pivot column of the lower rows equals $0$.				
		\end{enumerate}
		\item To get the matrix in row echelon form, repeat the pivot.
		\begin{enumerate}
			\item Repeat the procedure from Step 1 above, ignoring previous pivot rows.
			\item Continue until there are no more pivots to be processed.
		\end{enumerate}
		\item To get the matrix in reduced row echelon form, process non-zero entries above each pivot.
		\begin{enumerate}
			\item Identify the last row having a pivot equal to 1, and let this be the pivot row.
			\item Add multiples of the pivot row to each of the upper rows, until every element above the pivot equals $0$.
			\item Moving up the matrix, repeat this process for each row.
		\end{enumerate}
	\end{enumerate}
	It is therefore obvious that the elementary operations on the rows of a matrix do not change the rank of the rows of the matrix. However, we know that the rank of lines of a row matrix matrix is equal to the rank of the columns, that is to say to the rank of this matrix (once again see the section Vector Calculus for the proof). We conclude that the column rank of any matrix of the type $m\times n$ is also equal to the rank of the rows of this matrix.
	
	As a corollary of this conclusion, it appears that:
	
	When solving linear systems of $m$ equation with $n$ unknowns it appears, as we have already noted at the beginning of this section (and with practical example in the section of Theoretical Computing), there there must be at least an equal number of equations than unknowns or more rigorously: the number of unknowns must be less or equal to the number of equations such as:
	
	
	\pagebreak
	\subsubsection{Matrix Algebra}
	Remember that we saw during our study of Vector Calculus that the algebraic operations of multiplication of a vector by a scalar, addition or subtraction of vectors between them and the operation of scalar product formed in the context of set theory a "vector space "(\SeeChapter{see section Set Theory}) possessing therefore a "vector algebraic structure". This under the condition that the vectors of course have the same dimensions (this observation, for recall, is not valid if instead of the scalar product we take the cross product).
	
	Just as vectors, we can multiply a matrix by a scalar and add (subtract) them together (as long as they have the same dimensions..) but in addition, we can also multiply two matrices together under certain conditions which we will define below. This will also make the set of matrices in the set-sense, a vector space on $K$ (being most of time $\mathbb{R}$) having also therefore a "vector algebraic structure".
	
	Thus, a vector may also be viewed as a particular matrix of dimension $m\times n$  and operate in the vector space of matrices. Basically..., vector calculus is only a special case of linear algebra!!! This is way at school people learn (after Calculus) Vector Calculus first and Linear Algebra later (and some will lean after Tensor Calculus).
	
	\begin{enumerate}
		\item[D1.] Given $A,B\in M_{mn} (\mathbb{R})$. We name "sum of $A$ and $B$" the matrix $C\in M_{mn}(\mathbb{R})$ whose coefficients are:
		
		That is to say explicitly:
		
		
		\item[D2.] Given $A\in M_{mn} (\mathbb{R})$ a matrix and a $\lambda\in \mathbb{R}$ a scalar (we can also take in $\mathbb{C}$ if we want). We name "\NewTerm{product of $A$ by $\lambda$}" the matrix whose coefficients are:
		
		That is to say explicitly:
		
		In two previous definitions so we can actually conclude that the space / set of matrices is a vector space and thus has a vector algebraic structure.
		
		\item[D3.] Let $E, F, G$ be three  vector spaces of basis respectively $\mathcal{E},\mathcal{F},\mathcal{G}$ and two linear application $f$ and $g$ (see the section Set Theory for a refresh).
		
		Let us denote by $A$ the matrix of $f$ with respect to basis $\mathcal{E},\mathcal{F}$, and $B$ the matrix of $g$ with respect to basis $\mathcal{F},\mathcal{G}$. Then the matrix $C$ of $g\cdot g$ (see the definition of a composite function in the section of Functional Analysis) relating to the basis $\mathcal{E},\mathcal{G}$ is equal to the product of $B$ by $A$ denoted simply by $BA$.
		
		So let $B\in M_{mn} \in \mathbb{R}$ and $A\in M_{np} \in \mathbb{R}$, we name "\NewTerm{matrix product}\index{matrix product}" or "\NewTerm{matrix multiplication}" of $A$ and $B$ and we denote it by $BA$, the matrix $C\in M_{mp}\mathbb{R}$ whose components are:
		
		It is important to notice that at the opposite to the addition, $A$ and $B$ may have different dimensions. However! the number of rows of $A$ must be equal to the number of columns of $B$, as indicated by the index $n$ of the two matrices. So in the product $BA$, if $B$ is a matrix $m\times n$, $A$ must be a matrix $n\times p$, for any $p$!!!
		
		Schematically:
		{\Huge{
		\[
		\framebox[2.5cm]{\clap{\raisebox{0pt}[1.5cm][1.5cm]{$\mat C$}}\subdims{-2.5cm} n p} =
		\framebox[1.5cm]{\clap{\raisebox{0pt}[1.5cm][1.5cm]{$\mat B$}}\subdims{-2.5cm} m n} \ 
		\framebox[2.5cm]{\clap{\raisebox{5mm}[1.5cm]{$\mat A$}}     \subdims{-1cm} n p} 
		\]}}\\\\
		
		or even more explicity (\NewTerm{Falk's scheme}\index{Falk's scheme}):
		\begin{figure}[H]
		\centering
			\includegraphics[scale=0.9]{img/algebra/falks_scheme.jpg}
			\caption{Matrix Product Falk's scheme (credit: Alain Matthes)}
		\end{figure}
	\end{enumerate}
	By noting with a capital Latin letters matrices and with lowercase Greek letters scalars, the reader can easily verify with what we have seen \underline{until now} (we can add proofs on request) the following properties of matrix algebra (the matrices are assumed to have adequate dimensions):
	\begin{enumerate}
		\item[P1.] Left distributivity: $A(B+C)=AB+AC$
		\item[P2.] Right distributivity: $(A+B)C=AC+BC$
		\item[P3.] Scaling association: $(\lambda A)B=A(\lambda A)$
		\item[P4.] Associativity: $(AB)C=A(BC)$
		\item[P5.] Non-commutativity: $BA\neq AB$
		\item[P6.] Absorbing-Element: $A\mathbf{0}=\mathbf{0}$
		\item[P7.] Neutral Element for addition: $A+\mathbf{0}=A$
	\end{enumerate}
	It is especially important to remember of the property P5 that shows that the multiplication is obviously not commutative (for dimensions greater than $1$ of course!) and also the property P4 such that the matrix multiplication is associative.
	
	Concerning the general proof that the assertion of commutativity if false we must pass through a numerical example (because even the general case without replacing algebraic terms by numerical values will not show you much in our point of view...).
	
	\begin{tcolorbox}[title=Remark,colframe=black,arc=10pt]
	The set of square matrices $M_{nn}\in \mathbb{R}$ of order $n$ with components in $\mathbb{R}$ provided with the sum and the usual matrix multiplication forms a ring (\SeeChapter{see section Set Theory}). This is true more generally if the coefficients of the matrices are taken in any ring: for example, all the matrices $M_{nn}\in \mathbb{Z}$  with integer components is a ring.
	\end{tcolorbox}
	One reader asked us to prove the property of associative. So let us begin!
	
	Let $A\in M_{kl}(\mathbb{C}),B\in M_{lm}(\mathbb{C}),C\in M_{mn}(\mathbb{C})$, then for $1\leq i\leq k,1\leq j\leq n$, we have well (we use the explicit expression of matrix components multiplication seen above many times as you can see):
	
	
	\subsubsection{Type of Matrices}
	To simplify the notations and length of calculations we introduce now the most common types of matrices that the reader will encounter throughout his reading of this book (and not just in the chapters on pure mathematics!).
	
	Some definitions will only be recalls!
	
	We denote by $M_{mn}(\mathbb{K})$ all matrices $m\times n$ whose coefficients take values in $K$ (typically $\mathbb{R}$ or $\mathbb{C}$ for example).
	
	\textbf{Definitions (\#\mydef):}
	\begin{enumerate}
		\item[D1.] When $m=n$, we say that $(a_{ij})$ is a "\NewTerm{square matrix of order $n$}\index{square matrix}":
			

		\item[D2.] We name "\NewTerm{zero matrix}\index{zero matrix}", and we denote by $0_{mn}$ or simply $\mathbf{0}$ any matrix in which each term is zero:
			
	
		\item[D3.] We name "\NewTerm{identity matrix of order $n$}\index{identity matrix}" or "\NewTerm{unit matrix of order $n$}\index{unit matrix}", and denote it by $I$, or simply $\mathds{1}$, the square matrix of order $n$:
		
	
		It can also be written using the Kronecker delta notation $\delta_{ij}$ (\SeeChapter{see section Tensors Calculus}).
		
		\item[D4.] We name "\NewTerm{diagonal matrix}\index{diagonal matrix}" any square matrix $A\in M_{nn}\in\mathbb{C}$ which only the diagonal has non-null elements:
		
		Formally:
		
		The usual notation of a diagonal matrix is:
		
		
		\item[D5.] We name "\NewTerm{lower triangular matrix}\index{lower triangular matrix}" a square matrix if all the entries above the main diagonal are zero:
		
		Formally:
		 
		 Similarly, a square matrix is named "\NewTerm{upper triangular matrix}\index{upper triangular matrix}" if all the entries below the main diagonal are zero:
		 
		Formally:
		 
		A "\NewTerm{triangular matrix}\index{triangular matrix}" is one that is either lower triangular or upper triangular. A matrix that is both upper and lower triangular is named a "diagonal matrix".
		
		If an upper triangular matrix was obtained by a "staggered" matrix we will write it as following:
		
		
		\item[D6.] Given $M_{nn}$ a square matrix. The matrix $M_{nn}$ is named "\NewTerm{invertible matrix}\index{invertiable matrix}" or "\NewTerm{regular matrix}\index{regular matrix}" or "\NewTerm{non-singular matrix}\index{non-singular matrix}" if and only if $M_{nn}^{-1}$ is such that:
		
		If this is not the case, we say that $M_nn$ is a "\NewTerm{singular matrix}\index{singular matrix}". We will prove later that for a square matrix to invertible (non-singular) on sufficient condition is that its determinant is equal to zero.
		
		This definition is fundamental, it has extremely important consequences in all linear algebra and also in physics (solving linear systems, determining, eigenvectors and eigenvalues, etc.), statistics and finance and it is appropriate to remember it.
		
		\item[D7.] Given a matrix $A_{mn}:=A$:
		
		We name "\NewTerm{transposed matrix}\index{transposed matrix}" of $A=A_ {MN}$, the matrix denoted by $A^T=A_{nm}$  (the superscript $T$ is depending of the books and teachers uppercase or lowercase and either left or right but the standard ISO 80000-2: 2009 recommends the capital and superscript on the top right) the matrix for which we transpose the rows into columns and the into rows:
		
		Here are some interesting properties of the transpose of a matrix (which we will be useful to us later in this section for a famous theorem and also in the study of the multiple linear regression methods in the section of Numercial Methods!):
		\begin{enumerate}
			\item[P1.] $(A^T)^T$
			\item[P2.] $(\lambda A+B)^T=\lambda A^T+B^T,\lambda\in \mathbb{R},\lambda\in \mathbb{R}$
			\item[P3.] $(AB)^T=B^TA^T$
			\item[P4.] $(A^{-1})^T=(A^T)^{-1}\exists A^{-1}$
			\item[P5.] $A\vec{x}\circ\vec{y}=\vec{x}A^T\vec{y}$
		\end{enumerate}
		The transposed matrix is very important in physics, statistics and finance and obviously in the field of mathematics for example in the context of the theory of groups and symmetries! So it is also worth remembering its definition.
		
		As the third property is the most used one in the various sections of this book let us demonstrate it by considering $A\in M_{lm}\in\mathbb{C},B\in M_{mn}\in \mathbb{C}$:
		\begin{dem}
		 Remembering the explicit relation of matrix multiplication seen earlier:
		
		But in this last equality, we note that we browse $B$ by line and $A$ in column for a $i$ and a $j$ fixed and this we know then corresponds to the matrix multiplication $AB$, therefore:
		
		Finally we have well:
		
		\begin{flushright}
			$\square$  Q.E.D.
		\end{flushright}
		\end{dem}
		And for the same reasons let us prove the before last property.
		\begin{dem}
		First, it is trivial that if $A$ is invertible:
		
		and taking the transpose on both sides of the equality we find (we use the property proved just before):
		
		The latter equality show obviously that $(A^{-1})^T$ is the inverse of $A^T$, that is to say:
		\begin{flushright}
			$\square$  Q.E.D.
		\end{flushright}
		\end{dem}
		
		\item[D8.] Given:
		
		a matrix of $M_{mn}(\mathbb{C})$. We name "\NewTerm{adjoint matrix}\index{adjoint matrix}" of $A$, the matrix of $M_{mn}(\mathbb{C})$ defined by:
		
		which is the complex conjugate of the transposed matrix or if you prefer ... the transposed matrix of the complex conjugate (in the case of real components... we obviously don't need to take the conjugate!). To simplify the notations we simply note this matrix $A^\dagger$ (notation frequently use in Quantum Physics and Set Algebra).
		\begin{tcolorbox}[title=Remark,colframe=black,arc=10pt]
		Trivial relation (which is often used in Quantum Field Physics) already prove juste before and obviously right when the component are in $\mathbb{C}$:
		
		\end{tcolorbox}
		
		\item[D9.] By definition, a matrix is named "\NewTerm{Hermitian matrix}\index{Hermitian matrix}" or "\NewTerm{self-adjoint matrix}\index{self-adjint matrix}"... if it is equal to its own adjoint (conjugate transpose matrix) such that:
		
		
		\item[D10.] Given $A$ as square matrix $M_{nn}(\mathbb{R})$, the "\NewTerm{trace}\index{trace of a matrix}" of $A$ denoted $\text{tr}(A)$ is defined by the sum of the terms of the diagonal (very useful in some statistical techniques):
		
		Some useful related relations (we can add the detailed proof on the demand of the readers):
		
		and:
		

		\item[D11.] A matrix $A$ is named "\NewTerm{nilpotent matrix}\index{nilpotent matrix}" if by multiplying it successively by itself it can give zero. Explicitly, if there exists an integer $k$ such that:
		
		If the matrix $A$ multiplied by itself gives $A$ ... then we talk about an "\NewTerm{idempotent matrix} \index{idempotent matrix}".
		
		Such matrices are for example very common in Markov Chains when the transition matrix contains probabilities (see the sections of Probabilities and Graph Theory).
		\begin{tcolorbox}[title=Remark,colframe=black,arc=10pt]
		To remember this name, we can decompose it into "nil" that means "zero" and "potent" that means "potential". So something "nilpotent" is therefore something that is potentially zero....
		\end{tcolorbox}
		
		\item[D12.] A matrix $A$ is named "\NewTerm{orthogonal matrix}\index{orthogonal matrix}" if its elements are real and if it obeys to:
		
		which can be translate into (where $\delta_{ij}$ the Kronecker symbol):
		
		The matrix column vectors thus orthogonal to each other as the resulting operation above can be seen as a row-column dot product. Therefore an orthogonal matrix represents also an orthogonal basis!
		
		A typical mathematical example is the matrix of the canonical orthonormal basis (\SeeChapter{see section Vector Calculus}):
		
		or a well known matrix in quantum physics (\SeeChapter{see section Quantum Computing}):
		
		\begin{tcolorbox}[title=Remarks,colframe=black,arc=10pt]
		\textbf{R1.} Therefore this is typically the case of the canonical basis matrix, or any diagonalized matrix.\\
		
		\textbf{R2.} If instead of just taking a matrix with real coefficients, we take complex coefficients with its complex transposed matrix (adjoint matrix). So we say (sadly... because it makes confusion with the name of another matrix already define) that $A$ is a "unitary matrix" if it satisfies the previous relation!
		\end{tcolorbox}
		We will come back later, after having introduced the concepts of eigenvectors and eigenvalues, a particular and very important case of orthogonal matrices (named "translations matrices").
		
		Let us also mention another important property in geometry, physics and statistics of orthogonal matrices.
		
		\begin{theorem}
		Given $f(\vec{x})=A\vec{x}+\vec{b}$, where $A$ is an orthgonal matrix and $\vec{b}\in \mathbb{R}^n$. Then $f$ (respectively $A$) is an isometry. That is to say:
		
		So in other words: Orthogonal matrices are linear mappings which preserve the norm (the distance)!!!
		\end{theorem}
		\begin{dem}
		
		and we have well:
		
		\begin{flushright}
			$\square$  Q.E.D.
		\end{flushright}
		\end{dem}
		
		\item[D13.] Given a square matrix $A \in M_{nn}$. The matrix $A$ is named "\NewTerm{symmetric matrix}\index{symmetric matrix}" if and only if:
		
		We will meet this definition again in the section of Tensor Calculus.
		
		\item[D14.] Given a square matrix $A \in M_{nn}$. The matrix $A$ is named "\NewTerm{anti-symmetric matrix}\index{anti-symmetric matrix}" if and only if:
		
		which requires that:
		
		
		\item[D15.] Let $E$ be a vector space of dimension $n$ and $ \mathcal{B},\mathcal{B}'$ two basis of $E$:
		
		We name "\NewTerm{transition matrix}\index{transition matrix}" of the basis $\mathcal{B}$ to the basis  $\mathcal{B}'$, and we denote by $P$ the square matrix of $M_{nn}(\mathbb{K})$ which columns are formed of components of vectors of the basis $\mathcal{B}'$ on the basis $\mathcal{B}$ (see further below the detailed treatment of basis changes for more information).
		
		We consider the vector $\vec{x}(x_1,x_2,...,x_n)$ of $E$ which is written in the basis $\mathcal{B}(\vec{e}_1,\vec{e}_2,...,\vec{e}_n)$ and $\mathcal{B}'(\vec{e}_1^{'},\vec{e}_2^{'},...,\vec{e}_n^{'})$ following the relations:
		
		With:
		
		the vector of $K^n$ formed of the components of $\vec{x}$ in the basis $\mathcal{B}$ and of the vector $\vec{x}$ formed of the components in the basis $\mathcal{B}^{'}$. So:
		
		relation for which the detailed proof will be given later in our study of basis changes. We also have obviously:
		
		\begin{tcolorbox}[title=Remarks,colframe=black,arc=10pt]
		\textbf{R1.} When a vector is given and its basis is not specified, remember that it is therefore implicitly in the canonical basis:
		
		which remains invariant by the multiplication by any vector and when the basis used is denoted by $(\vec{e}_i)$ and is not specified, then it is also that of the canonical basis.\\
		
		\textbf{R2.} If a vector is given relative to the canonical basis, its components are named "\NewTerm{covariant}\index{covariant components}",  if they are expressed in another noncanonical base, then we say that the components are "\NewTerm{contravariant}\index{contravariant components}" (for details on the subject see the section of Tensor Calculus).
		\end{tcolorbox}
		
		\item[D16.] A matrix is named "\NewTerm{positive-defined matrix}\index{positive-defined matrix}" (which will be useful in the section of Theoretical Computing for some important engineering techniques and in quantitative finance for qualitative estimation of the correlation matrix) if:
		
		and "\NewTerm{positive matrix}\index{positive matrix}" or "\NewTerm{semi-positive matrix}\index{semi-positive matrix}" if:
		
		We have proved in our study of the covariance matrix in the section Statistics that a semi-positive matrix has its eigenvalues which are all positive OR null, while if it is positive definite its eigenvalues are all positive AND not null.
		
		\item[D17.] A symmetric matrix having all it components being positive and only zeros on the diagonal is named a "\NewTerm{distance matrix}\index{distance matrix}" (we will meet several times this type of matrix in Data Mining techniques during our study of the Numerical Methods section).
		
		\item[D18.] A matrix is named "\NewTerm{sparse matrix}\index{sparse matrix}" if it contains a significant number of null values. In numerical methods, there are algorithms that use this specificity to optimize the storage of this type of matrices (used in OLAP cubes and financial engineering).
	\end{enumerate}
	
	\subsubsection{Determinant}
	We will look at determinants in the point of view of the physicist or of the engineer (the mathematician point of view is rather off-putting ...). In physics (whether in classical mechanics and quantum field physics), chemistry or engineering, we frequently have to solve linear systems. But we have now seen that a linear system:
	
	can be written as:
	
	and we know that the only soluble linear systems (in the sense that they have a unique solution !!!) are those that have as many equations as unknowns and their determinant is not zero! Thus, the matrix must be a square matrix $M_{mm}$.
	
	If a solution exists, then there is a column matrix (or "vector") $X$ such that $AX=B$ which involves:
	
	What imposed this relation? Well this is relatively simple, but at the same time very important: for a linear system to have a unique solution, it is necessary that the matrix is invertible (non-singular)! What relation with the  concept of "determinant" then? It's simple: mathematicians have sought how to write inverse matrices of linear systems for which they knew there was a unique solution and they arrived after trial and error to determine a kind of formula to assess if the matrix is invertible (non-singular) or not. Once this formula found, they formalized (as they know so well how to do it...), with a great rigor, the concept surrounding this formula that they named the "\NewTerm{determinant}\index{determinant}". They did it so well that in fact we sometimes forgot that they have found it by trial and error...
	\begin{tcolorbox}[title=Remarks,colframe=black,arc=10pt]
	If a matrix of a linear system is not invertible (non-singular), this has the consequence that there is no solution or an infinity of solutions (as usual what ...)
	\end{tcolorbox}
	We below first focus on how to build the determinant bydefining a particular type of application. Then, after seeing a simple and readable example of the calculation of a determinant, we will focus on determining the formula of it in the general case. Finally, once this is done, we will see what is the relation between the inverse of a matrix and the determinant!!!
	
	In what will follow all vector spaces will be considered of finite dimension and the field of complex numbers $\mathbb{C}$ (those who prefer the can take $\mathbb{R}$ as basis field, in fact we could take any field).
	
	First of all we will do a little bit of pure math (a bit off putting) before moving on  concrete stuff.
	
	Given $V$ a vector space, we write will write as usual $V^n$ instead of $V\times V\times... \times V$. $(\vec{e}_1,\vec{e}_2,...,\vec{e}_3)$ designate the canonical basis of $\mathbb{R}^n$. $M_n(\mathbb{R})$ is the set of square matrices $n\times n$ with coefficients in $\mathbb{R}$.
	
	\textbf{Definitions (\#\mydef):}
	\begin{enumerate}
		\item[D1.] A "\NewTerm{multilinear application}\index{multilinear application}" on a space $V$ is defined by an $\varphi: V^n \rightarrow \mathbb{R}$ which is linear in each of its components. Meaning:
		
		for any $\lambda,\mu\in K$ and $\vec{x}_i,\vec{u},\vec{v}\in V$ where the $\vec{x}_i$ are vectors.
		\begin{tcolorbox}[title=Remark,colframe=black,arc=10pt]
		A non-null multilinear application is not a linear application of the space $V^n$ in $\mathbb{R}^n$. Excepted if $n=1$. Indeed, this can be verified by the definition of a linear application  versus the one of the multilinear application:
		
		\end{tcolorbox}
		\item[D2.] An "\NewTerm{alterneted multilinear application}\index{alterneted multilinear application}" on $V$ is by definition a multilinear application that satisfies:
		
		for any $j=1...n,\vec{x}_j\in V$. Therefore the permutation of two vectors that follows change the sign of $\varphi$.
		\begin{theorem}
		Therefore, if $\varphi$ is a multilinear alterneted application, then $\varphi$ is multilinear if and only if $\forall \vec{j}\in V,j=1...n$ we have:
		
		or in a more simple case:
		
		\end{theorem}
		\begin{dem}
		If $\varphi$ is alternated we have by definition:
		
		Therefore by rearranging:
		
		And if $\varphi$ is a multilinear application we can write:
		
		\begin{flushright}
			$\square$  Q.E.D.
		\end{flushright}
		\end{dem}
		Now comes the interesting stuff:
		
		\item[D3.]  A "\NewTerm{determinant}\index{determinant}" is by definition a multilinear alterneted application:
		
		satisfying as well:
		
		\begin{tcolorbox}[title=Remark,colframe=black,arc=10pt]
		The columns of a matrix form $n$ vectors and we then see that a determinant $D$ on $\mathbb{R}^n$ inducte an application $D$ of $M_n(\mathbb{R})\mapsto \mathbb{R}$ (where ase we know $M_n(\mathbb{R})$ is the set of squatre matrices $n\times n$ with components in $\mathbb{R}$) defined by:
		
		where $\vec{m}_i$ is the $i$-th column of $M$. 
		\end{tcolorbox}	
	\end{enumerate}
	Let us study the case $n=2$. If $D$ is a determinant, for any vector:
	
	we have:
	
	As $D$ is multilinear, we have:
	
	as it is alterneted:
	
	it remains:
	
	and we have:
	
	And finally:
	
	In fact, we just prove that if the a determinant exists, it is unique and of the form indicated previously, we should also check that the defined application satisfy the properties of a determinant, but the latter is immediate.
	
	Thus, if:
	
	is a matrix we have then:
	
	Let us give now a geometric interpretation of the determinant. Given $\vec{v}_1,\vec{v}_2$ two vectors of $\mathbb{R}^2$:
	\begin{figure}[H]
		\centering
		\includegraphics{img/algebra/determinant_parallelelogram.jpg}
		\caption{Geometric interpretation of the determinant}
	\end{figure}
	The vector $\vec{w}$ is obtained by projecting $\vec{v}_1$ on $\vec{v}_2$ and we have therefore:
	The vector $\vec{w}$ is obtained by projecting $\vec{v}_1$ on $\vec{v}_2$ and we have therefore:
	
	The area of the above parallelogram is therefore:
		
	if:
	
	then:
	
	and finally:
	
	Therefore the determinant, in absolute value, represents the area of the parallelogram defined by the vectors $\vec{v}_1,\vec{v}_2$ when these vectors are linearly independent. We can generalize this result to a $n$-dimensional space, in particular, for $n=3$, the determinant of three linearly independent vectors represents the volume of the parallelepiped defined by these as we already prove it during our study of the mixed product in the section of Vector Calculus.
	
	The more general case of the expression of the determinant is a little trickier to ascertain. This requires that we define a particular bijective application but simple that we have already met in the section Statistics.
	
	\textbf{Definition (\#\mydef):} Given $F_n=\left\lbrace 1,2,...,n\right\rbrace,n\in \mathbb{N}^{*}$ we name "\NewTerm{permutation}\index{permutation}" of $F_n$ any bijective application of $F_n$ in $F_n$:
	
	Given $\mathcal{S}_n$ the set of possible permutations (bijective applications) of $\left\lbrace 1,2,...,n\right\rbrace$.  $\mathcal{S}_n$ obviously contains ... (see our study of Combinatorics in the section of Probabilities) $n!$ elements. The information an element $\sigma$ of $\mathcal{S}_n$ is defined by the successive information of:
	
	Given an ordered sequence of elements (ascending) $\left\lbrace 1,2,...,n\right\rbrace,n \in \mathbb{N}^{*}$ we name "inversion", any permutation of elements in the ordered sequence (so the result will not be ordered at all...). We denote by $I(\sigma)$ the number of inversions.
	
	We say that the permutation $\sigma$ is even (odd) if $I(\sigma)$ is even (odd). We name "\NewTerm{signature}\index{signature of a matrix}" of $\sigma$, the number $\varepsilon(\sigma)$ defined by $\varepsilon(\sigma)=(-1)^{I(\sigma)}$, that is to say:
	
	We now have the necessary tools to set up the general relation of the determinant:
	\textbf{Definition (\#\mydef):} Given:
	
	We name "\NewTerm{determinant of a square matrix $A$}\index{determinant of a square matrix}" of dimension $n$, and we denote by $\det (A)$, the scalar defined by (we'll see an example further below):
	
	sometimes named "\NewTerm{Leibniz formula}\index{Leibniz formula}" or "\NewTerm{Laplace's formula}\index{Laplace's formula}". This relation was obtained in the past by trial and error and by induction for larger dimensions.
	
	\pagebreak
	\begin{tcolorbox}[colframe=black,colback=white,sharp corners]
	\textbf{{\Large \ding{45}}Examples:}\\\\
	E1. Given $A=(a_{ij})_{1\leq i,j\leq 2}\in M_{22}(K)$, let us consider the $2!=2$ permutations of the second indices (of integers $1,2$) taken in their whole:
	\begin{gather*}
		12 \qquad 21
	\end{gather*}
	We calculate the signature of $\sigma$. Here is the scheme of this rule (recall: we say that there is an "inversion", if in a permutation, a greater integer preceed a smallest integer):
	
	Therefore we have:
	
	This corresponds well to what we saw initially. Remember also on the way that we will soon prove that the determinant of a square matrix must be zero so that the matrix is invertible (non-singular)!\\
	
	E2. Given $A=(a_{ij})_{1\leq i,j\leq 3} \in M_{33}(K)$, let us consider the $3!=6$ permutations of the second indices (integers $1,2,3$) taken in their whole:
	\begin{align*}
	123 \quad 132 \quad 213 \quad 31 \quad 312 \quad 321
	\end{align*}
	We calculate the signatures of $\sigma$. Here is a scheme of this rule (recall: we say that there is an "inverse", if in a permutation, a greatest integer precedes a lower integer):
	
	\end{tcolorbox}
	
	\pagebreak
	Therefore we have
	\begin{tcolorbox}[colframe=black,colback=white,sharp corners]
		
	\end{tcolorbox}
	\begin{tcolorbox}[title=Remark,colframe=black,arc=10pt]
	Some people learn by heart a method named "\NewTerm{rule of Sarrus}\index{rule of Sarrus}" to calculate the determinants of order three as the previous one given by:
	\begin{center}
	\begin{tikzpicture}
    \matrix [%
      matrix of math nodes,
      column sep=1em,
      row sep=1em
    ] (sarrus) {%
      a_{11} & a_{12} & a_{13} & a_{11} & a_{12} \\
      a_{21} & a_{22} & a_{23} & a_{21} & a_{22} \\
      a_{31} & a_{32} & a_{33} & a_{31} & a_{32} \\
    }; 

    \path ($(sarrus-1-3.north east)+(0.5em,0)$) edge[dotted] ($(sarrus-3-3.south east)+(0.5em,0)$)
          (sarrus-1-1)                          edge         (sarrus-2-2)
          (sarrus-2-2)                          edge         (sarrus-3-3)
          (sarrus-1-2)                          edge         (sarrus-2-3)
          (sarrus-2-3)                          edge         (sarrus-3-4)
          (sarrus-1-3)                          edge         (sarrus-2-4)
          (sarrus-2-4)                          edge         (sarrus-3-5)
          (sarrus-3-1)                          edge[dashed] (sarrus-2-2)
          (sarrus-2-2)                          edge[dashed] (sarrus-1-3)
          (sarrus-3-2)                          edge[dashed] (sarrus-2-3)
          (sarrus-2-3)                          edge[dashed] (sarrus-1-4)
          (sarrus-3-3)                          edge[dashed] (sarrus-2-4)
          (sarrus-2-4)                          edge[dashed] (sarrus-1-5);

    \foreach \c in {1,2,3} {\node[anchor=south] at (sarrus-1-\c.north) {$+$};};
    \foreach \c in {1,2,3} {\node[anchor=north] at (sarrus-3-\c.south) {$-$};};
  \end{tikzpicture}
	\end{center}
	We prefer in this book the general formulation of the determinant because applicable to all orders.
	\end{tcolorbox}
	Let's us see some properties and corollaries of this formulation of the determinant:
	\begin{enumerate}
		\item[P1.] Given a square matrix of order $n$, we do not change the value of its determinant if:
		\begin{enumerate}
			\item By performing an elementary operation on the columns of $M_n$.

			\item By performing an elementary operation on the rows of $M_n$.
		\end{enumerate}
		\begin{dem}
		If $M_n=(a_{ij})_{i,j=1,...,n}$ the $M_n$ is composed of $n$ column vectors:
		
		Doing an elementary operation on the columns of $M_n$ is equivalent to add $\lambda v_i,i \in \{1,...,n\}$ to one of the columns $v_n$ of $M_n$. Given $M_n^{'}$ the matrix obtained by adding $\lambda v$ to the $j-th$ column of $M_n$, we get:
		
		By multilinearity (finally the proof in not really difficult):
		
		and as the determinant is alternated:
		
		About the elementary operations on the rows we just need to consider the transpose (that is to cry such it is simple, but we had to thing about this trick).
		\begin{flushright}
			$\square$  Q.E.D.
		\end{flushright}
		\end{dem}
		
		\item[P2.] Given $M_n(K)$ a squared matrix of order $n$ and given $\lambda \in K$:
		
		\begin{dem}
		As before, it is enough to simply noticed that if $v_1,...,v_n$ are the column vectors forming the matrix $M_n$ then $\lambda v_1,...,\lambda v_n$ are those that constitute $\lambda M_n$ and:
		
		The application being $n$-linear, we arrive at the equality:
		
		\begin{flushright}
			$\square$  Q.E.D.
		\end{flushright}
		\end{dem}
		
		\item[P3.] Given is a square matrix of order $n$. We change the sign of the determinant of $M_n¨$ if:
		\begin{itemize}
			\item We permute two of its columns
	
			\item We permute tow of its rows
		\end{itemize}
		\begin{dem}
		$M_n$ is constituted by $n$ vectors $v_1,..,v_n$. The determinant of $M_n$ is equal to the determinant of these $n$. Permute two columns of $M_n$ is same as permuting the two corresponding vectors. Let us suppose that the permuted vectors are the $i$-th and $j$-th, the determinant being an alternate application, we have:
		
	About the rows, we just have to consider the transposed of $M_n$ to arrive to the same result!
		\begin{flushright}
			$\square$  Q.E.D.
		\end{flushright}
		\end{dem}
		
		\item[P4.] Given $A,B\in M_m\in (\mathbb{C})$ then:
		
		As far as we know the proof can be done in at least two ways, the first is rather indigestible and abstract... so we will let it to mathematicians (...) even if it has the advantage of being general, the second one easiest is to check this assertion for various squared matrices.
		\begin{dem}
		
		and:
		
		The calculations therefore produce results that are identical. We can check for square matrices of higher dimensions.
		\begin{flushright}
			$\square$  Q.E.D.
		\end{flushright}
		\end{dem}
		
		\item[P5.] A square matrix $A\in M_n(\mathbb{C})$ is invertible (non-singular) if and only if $\det(A)\neq 0$ (this is the most important property among all).
		\begin{dem}
		If $A$ is invertible (non-singular), we have:
		
		\begin{flushright}
			$\square$  Q.E.D.
		\end{flushright}
		\end{dem}
		As we alread say it, this is the most important property of matrices as part of theoretical physics because if $A$ is a linear system, the calculation of the determinant indicates whether it has unique solutions or not. Otherwise, as we have already mentioned and study it, the system has no solution, or an infinity of solutions!
		
		We must consider an important special case! Given the following system:
		
		where $A\in M_(K)\neq =0$ and $B\in M_n(K)$ are to be determined. It is obvious that $A$ is invertible (non-singular) or not, the trivial solution is $A\cdot B=0$. However, let us imagine a case of theoretical physics where we have $A\cdot B=0$ but for which we know that $A\in M_n(K)\neq 0$ for which we impose $B\in M_n(K)\neq 0$. In this case, we must eliminate the trivial solution $B=0$. Furthermore, calculate the inverse (if it exists) of the matrix $A$ will bring us to nothing concrete except that $B=0$ which obviously does not satisfy us. The only solution is then to play such that the coefficients $a_{ij}$ of the matrix $A$ are such that its determinant is zero and therefore the matrix in invertible! The advantage? Just to to have an infinite number of possible solutions (of $B$ then!) that satisfy $A\cdot B=0$. We will need this methodology in the section of Wave Quantum Physics, when we will determine the existence of antiparticles through the linearized Dirac equation. It must therefore be remember.
		
		\item[P6.] Two "\NewTerm{conjugated}\index{conjugated matrices}" matrices (be careful! it is not the "conjugate" in the complex sense) have the same determinant.
		\begin{dem}
		
		
		\begin{flushright}
			$\square$  Q.E.D.
		\end{flushright}
		\end{dem}
		
		\item[P7.] For any matrices $A\in M_n(\mathbb{C})$:
		\begin{dem}
		
		But as (trivial... simple product of all coefficients):
		
		As (trivial) $\varepsilon(\sigma^-1)=\varepsilon(\sigma)$ and that $x,y\in\mathbb{C}:\bar{x}\cdot \bar{y}=\overline{x\cdot y}$ (\SeeChapter{see section Numbers}) then we can write:
		
		\begin{flushright}
			$\square$  Q.E.D.
		\end{flushright}
		\end{dem}
		
		\item[P8.] For any matrix $A\in M_n(\mathbb{R})$:
		
		\begin{dem}
		Well... it's the same as the previous property but without the conjugate values... In fact, we prove in the same way, the same property for $A\in M_n(\mathbb{C})$.
		\begin{flushright}
			$\square$  Q.E.D.
		\end{flushright}
		\end{dem}
		
		\item[P9.] Given a matrix $A=(a_{ij})\in M_n(\mathbb{C})$ we denote by $A_{ij}$ the matrix obtained from $A$ by removing the $i$-th row and the $j$-th column (very important notation to remember for what will follow!!!). The $A_{ij}$ belongs therefore to $M_{n-1}(\mathbb{C})$. Then for any $i=1...n$:
		
		where the term:
		
		is named the "\NewTerm{cofactor}\index{cofactor}".
		\begin{dem}
		For the proof let us define the application:
		
		It could be almost easy to see that $\varphi$ is multilinear (you just have to consider that $(-1)^{i+j}a_{ij}$ as a simple constant and after by extension of the definition of the determinant... too easy...).

		Let us show however that this application is alternated (in this case it is a determinant hat has all the properties of a... determinant!).
	
		Given $a_k,a_{k+1}$ two column vectors $A$ that follows each other. Let us suppose that $a_k=a_{k+1}$, we have to show in this case that $\varphi(A)=0$ (which comes from the definition of an alternate application).
	
		We have first (it is mandatory by the definition itself) if we don't erase any of the columns $j$ being $k$ or $k+1$:
		
		and we have obviously if we don't remove respectively the column $k$ and the column $k+1$:
		
		Therefore:
		
		It is therefore OK. The application $\varphi$ is alternated and multilinear, it is indeed well a determinant.
		
		We have just prove that $\varphi$ is a determinant and by unicity we have $\varphi(A)=\det(A)$ for any $A\in M_n(\mathbb{C})$.
		\begin{flushright}
			$\square$  Q.E.D.
		\end{flushright}
		\end{dem}
		\begin{tcolorbox}[colframe=black,colback=white,sharp corners]
		\textbf{{\Large \ding{45}}Example:}\\\\
		Let us see an example of this method by calculating the determinant of: 
		
		Let us develop the second line ($i=2$). We get:
		\end{tcolorbox}
		
		\pagebreak
		\begin{tcolorbox}[colframe=black,colback=white,sharp corners]
			
		Let us develop following the first column for verification (we never know...):
		
		The calculation determined above is therefore "exponential" as if for example we must calculate the determinant of a square matrix of order (dimension) $n=10$, then the determinant will be developed in a sum of $10$ terms, which each contains the determinant of a matrix of dimension $n=9$, which is a cofactor of the starting matrix. If we develop any of this determinant, we get a sum of $9$ determinants where each contains the determinant of a matrix of dimension $n=8$. At this level, there is therefore $90$ determinants of matrices of dimension $8$ to calculate. The process could continue until it remains only determinant of order $2$. And therefore we guess that the number of determinants of order $2$ is important!
		\end{tcolorbox}
	\end{enumerate}

	\textbf{Definition (\#\mydef):} Given $m,n$ two positive integers and $A$ a $m\times n$ matrix with coefficients in $\mathbb{C}$. For any $k\leq \min(m,n)$ a "\NewTerm{minor of order $k$}\index{minor}" is a determinant of the type:	
	
	with $1\leq i_1< \ldots \leq m$ and $1\leq j_1 <\ldots <j_k\leq n$.
	
	In the particular case of a matrix of order $n>1$ the definition is simpler: the minor $M_{ij}$ of the element $a_{ij}$ is the determinant of the matrix of order $n-1$ that we get by remove the row $i$ and the column $j$. Therefore, to calculate the minor of an element, we remove the line and the column to which the element belongs to, and we calculate the determinant of the remaining square matrix.
	
	\pagebreak
	\paragraph{Derivative of a Determinant}\mbox{}\\\\
	Let us see now a result that will be quite useful to us in the section General Relativity:

	Given a square matrix $n\times n$ with functions $g_{ij}:\mathbb{R}\mapsto \mathbb{R}$ that can be derivate at least one time. Let us put $g:=\det(G)$ with $G=(g_{ij})$. We want to calculate $\mathrm{d}_t g$. Given $g_i$ the $i$-th column vector of the matrix $G$. Let us use the formula:
	
	Knowing that the derivative of $g_{\sigma(1),1}\cdot \ldots \cdot g_{\sigma(n),n}$ is (derivative of $n$ products):
	
	Therefore we have:
	
	If we take a look closely to the first above sum, we notice that:
	
	where $g_1^{'}$ is the derivative of the vector $g_1$. Same for the following sums. Therefore:
	
	Let us develop again. Let us consider the term $\det(g_1^{'},g_2,\ldots,g_n)$ above. If we develop it relatively to the first column, we get:
	
	 Also, by developing the $j$-term of the above sum relatively to the $j$-th column, we get:
	
	If we put:
	
	We get:
	
	Which is written in tensor notation (\SeeChapter{see section Tensor Calculus}):
	
	We also have:
	
	where $b_{ji}$ is the coefficient being at the row $j$-th, columen $i$-th of the matrix $G^{-1}$. If we denoted $g^{ij}$ the coefficient $i,j$ of the matrix $(G^{-1})^t$ then:
	
	The expression of the derivative is then finally:
	
	which is written in tensor notation:
	
	This result, finally quite simple, we will be helpful to us in the section of Tensor Calculus to build the tools necessary for the study of General Relativity and in the context of the determination of Einstein field equations. It is therefore appropriate to remember it!
	
	\paragraph{Determinant Cofactor and Matrix Inverse}\mbox{}\\\\
	Let us finish our study of the determinants with the "icing on the cake" by giving a very important relation in many fields of engineering, physics and mathematics that connects the inverse of the coefficients of a matrix with miners of order $n$ (we will use this relation further below).
	
	Given $A\in M_n(\mathbb{C})$ an invertible matrix (non singular). Let us write $A=(a_{ij})$ and $A^{-1}=(b_{ij})$. Then:
	
	\begin{dem}
	Let us denote by $a_k$ the $k$-th column vector of the matrix $A$. Knowing that $A\cdot A^{1}=\mathds{1}$ (under known assumptions), we have (trivial):
	
	Let us calculate now $\det(a_1,\ldots,a_{k-1},e_j,a_{k+1},\ldots,a_n)$. First by developping relatively to the $k$-the column we found (as only one of the coefficient of $e_j$ is not nul and that the unique non-null one is equal to the unit):
	
	Furthermore (properties of the determinant):
	
	Therefore:
	
	That is to say:
	
	\begin{flushright}
		$\square$  Q.E.D.
	\end{flushright}
	\end{dem}
	For a simple, detailed, and important practical application in the industry (because otherwise in this entire book we will rarely inverse small matrices), the reader can refer to the section of Theoretical Computing in the part concerning the multiple linear regression.
	
	Let us also indicate the following important properties where $A$ and $B$ are a square invertible matrix $M_{n}(\mathbb{C})$ and $\lambda\in\mathbb{C}$ (the first should be obvious, the second has already been presented earlier but unproven and the third one is important for the proof of the variance inflation factor that we we will prove in the section of Theoretical Computing) :
	
	Let us prove the last property using the property of associativity:
	\begin{dem}
	
	Which prove that $B^{-1}A^{-1}$ is indeed the inverse of $AB$ where $I_n$ (also denoted $\mathds{1}$ for recall) is a diagonal matrix (also square) of dimension $n$.
	\begin{flushright}
		$\square$  Q.E.D.
	\end{flushright}
	\end{dem}
	
	\pagebreak
	\subsection{Change of Basis (frames)}
	A basis for a vector space of dimension $n$ is a sequence of $n$ vectors $(\vec{e}_1, …, \vec{e}_n)$ with the property that every vector in the space can be expressed uniquely as a linear combination of the basis vectors (\SeeChapter{see section Vector Calculus}). The matrix representations of operators are also determined by the chosen basis! Since it is often desirable to work with more than one basis for a vector space, it is of fundamental importance in linear algebra to be able to easily transform coordinate-wise representations of vectors and operators taken with respect to one basis to their equivalent representations with respect to another basis. Such a transformation is named a "\NewTerm{change of basis}\index{change of basis}".
	
	Let us now suppose that we move from a frame $\mathcal{E}=(\vec{e}_1,\vec{e}_2,...,\vec{e}_n)$ of a space $V^n$ to another space $\mathcal{F}=(\vec{f}_1,\vec{f}_2,...,\vec{f}_n)$ of this same space sharing the same origin O. Thus in two dimension:
	\begin{figure}[H]
		\centering
		\includegraphics[scale=0.75]{img/algebra/basis_change.jpg}
		\caption{A vector can be represented in two different bases (purple and red arrows) (source: Wikipedia)}
	\end{figure}
	Let us decompose the $\vec{f}_i$ in the basis $\mathcal{E}$:
	
	\textbf{Definition (\#\mydef):} We name "\NewTerm{transition matrix}\index{transition matrix}" the matrix (linear application) that allows to pass from $\mathcal{E}\mapsto \mathcal{F}$ given by:
	
	\begin{theorem}
	Now let us consider the vector given by:
	
	So we intend to prove that the components of $y_1,y_2,...,y_n$ of $\vec{v}$ in the basis $\mathcal{F}$ are given by:
	
	Thus explicitly:
	
	\begin{tcolorbox}[title=Remark,colframe=black,arc=10pt]
	The matrix $P$ is invertible (non-singular), because its columns are linearly independent (they are the vectors $\vec{f}_i$ decomposed in the basis $\mathcal{E}$ and the $\vec{f}_i$ base and the $\vec{f}_i$ are linearly independent as they form a base!).
	\end{tcolorbox}
	\end{theorem}
	\begin{dem}
	Let us take the case to simplify the case $n=2$ (the proof being quite easily generalized) with $\mathcal{E}=(\vec{e}_1,\vec{e}_2)$ and $\mathcal{F}=(\vec{f}_1,\vec{f}_2)$.
	Then we have:
	
	We therefore have $\vec{v}=x^i\vec{e}_i$ and we seek to express $\vec{v}$ in the basis $\mathcal{F}$ as $\vec{v}=y^i\vec{f}_i$. We'll search the linear application that link these two relation such that:
	
	Thus written in an explicit way:
	
	Therefore:
	
	That is to say:
	
	So $P$ (if it exists) is indeed the matrix that can express the components of a vector of a basis in those of another basis such that we write in vector notation:
	
	\begin{flushright}
		$\square$  Q.E.D.
	\end{flushright}
	\end{dem}
	\begin{theorem}
	Let us now consider a linear application $g:V^n\mapsto V^n$. Let $A$ be the matrix in the basis $\mathcal{E}$, and $B$ its matrix in the basis $\mathcal{F}$ (of same dimension). Then we might have:
	
	which is equivalent:
	
	or even:
	
	If there exists such a matrix $P$ satisfying these relations, we say that $A$ and $B$ are "\NewTerm{similar matrices}\index{similar matrices}".
	\end{theorem}
	\begin{dem}
	Let us take back the fact that we proved that it was eventually possible to build a transition matrix $P$ from the fact that:
	
	and let us put:
	
	We have then a function that bring us to write:
	
	On the other hand, we have (that we proved earlier):
	
	Therefore:
	
	hence:
	
	and as we saw it in our study of the determinant, the determinants of $A$, $B$ are equal and therefore invariant. We will  return later back on a similar formulation in our study of the Spectral Theorem below.
	\begin{flushright}
		$\square$  Q.E.D.
	\end{flushright}
	\end{dem}
	At the vocabulary level we say when we are in the presence of a such a matrix relation that: the matrix $A$ is "\NewTerm{conjugated}" to the matrix $B$.
	
	\pagebreak
	\subsection{Eigenvalues and Eigenvectors}
	\textbf{Definition (\#\mydef):} An "\NewTerm{eigenvalue}\index{eigenvalue}" is by definition (we will find again this definition in the introduction to quantum algebra in the section of Wave Quantum Physics) a value $\lambda$ belonging to a field $K$ such that given a squared matrix $A\in M_{mm}(K)$ we have:
	
	and conversely a vector $\vec{X}\in M_{m1}(K)$ is an "\NewTerm{eigenvector}\index{eigenvector}" if and only if:
	
	The major advantage of these concepts will be able the possibility to study a linear application, or any other item linked to a matrix representation, in a simple representation through a basis change on which the restriction of $A$ is a single homothetic transformation (typically solving simple systems of differential equations).
	
	Thus, all the eigenvalues of a matrix $A\in M_{mm}(K)$ is named "\NewTerm{spectrum of $A$}\index{spectrum of a matrix}" and satisfies the homogeneous system:
	
	or (whatever it is the same!):
	
	where $I_n$ (also denoted for recall that $\mathds{1}$) is for recall a diagonal unit matrix (and therefore also square) of dimension $n$. This system we know (proved above) has nontrivial solutions, therefore $\vec{X} \neq=0$ or $(\lambda I_n-A)\neq 0$, if and only if (we'll see many examples in various section related to physics in this book):
	
	that is to say that the matrix $A-\lambda I_n$ is not inversible (singular).
	
	The determinant $\det(A-\lambda I_n)$ is a polynomial on $\lambda$ of degree $n$ and can have at maximum $n$ solutions/eigenvalues as we have proved it in our study of polynomials (\SeeChapter{see section Calculus}) and is named "\NewTerm{characteristic polynomial}\index{characteristic polynomial}" of $A$ and the equation $\det(A-\lambda I_n)$ is named "\NewTerm{characteristic equation}\index{characteristic equation}"  of $A$ or "\NewTerm{eigenvalue equations}\index{eigenvalue equations}".
	
	For the small parenthesis, it is nice to notice that we always have in the development of $\det(A-\lambda I_n)$ the trace of the matrix $\text{tr}(A)$ and the determinant $\det (A)$ that appear. Let us see two examples of this:
	
	\pagebreak
	\begin{tcolorbox}[colframe=black,colback=white,sharp corners]
	\textbf{{\Large \ding{45}}Examples:}\\\\
	E1. Let us begin with the case $n=2$:
	
	Therefore for a square matrix of dimension $2$, the eigenvalues are (simple resolution of a polynomial of the 2nd degree):
	
	E2. For a matrix of dimension $n=3$, we have:
	
	and here... the final solution (roots) are quite less easy in the general case...
	\end{tcolorbox}
	On the path let us notice (we will generalize the result coming from this during our study of the spectral theorem) that as multiplying the homogeneous system:
	
	by $-1$ on the both sides of the equality doesn't change anything to the problem, then we get:
	
	So we can see that is multiplication doesn't change anything to the final result!
	
	Thus by a term by term correspondence it comes the very important result in Statistics (and also in Numerical Methods!) that we will prove later in a more general way with the spectral theorem:
	
	If we look at $(\lambda I_n-A)$ as a linear application $f$, as it is non-trivial solutions that interest us, we can say that the eigenvalues are the elements $\lambda$ such that:
	
	and that the kernel constitutes the eigenspace  of $A$ of the eigenvalue $\lambda$ from which non-null elements are the eigenvectors!
	
	It corresponds to the study of the main axes, according to which the application behaves like an expansion (homothetic application) multiplying the vectors by the same constant. This homothetic ratio is then the "eigenvalue", the vectors to which it applies the "eigenvectors" are asembled in a "eigenspace".
	
	Another way of looking at it:
	\begin{itemize}
		\item A vector is said to be an "eigenvector" by a linear application if it is not zero and if the application does only change its size (norm) without changing its direction.

		\item An "eigenvalue", associated to an "eigenvector", is the size modification factor (homothetic ratio), ie the number by which we must multiply the vector to get its image. This factor can be negative (reverse direction of the vector) or zero (vector transformed into a vector of zero length).
		
		We can say that therefore that the eigenvalue $\lambda$ "scalar" the application $A$ for the eigenvector $\vec{X}$.

		\item An "eigenspace" associated to an "eigenvalue" is the set of eigenvectors that have the same eigenvalue value and a zero vector. They suffer all from the multiplication by the same factor.
	\end{itemize}
	\begin{tcolorbox}[title=Remark,colframe=black,arc=10pt]
	In mechanics, we study the eigen-frequencies and eigenmode of oscillating systems (\SeeChapter{see section Wave Mechanics}). In Functional Analysis, an eigenfunction is an eigenvector for a linear operator, that is to say a linear application acting on a space of function (\SeeChapter{see section Functional Analysis}). In geometry and optics, we speak of eigendirections to take into account the curvature of the surfaces (\SeeChapter{see section Non-Euclidean Geometry}). In graph theory, an eigenvalue is simply an eigenvector of the adjacency matrix of the graph (\SeeChapter{see section Graph Theory}).
	\end{tcolorbox}So as the determinant of $\det(A-\lambda I_n)$ is a polynomial on $\lambda$ then the $\lambda_i$ are also the roots of the characteristic polynomial:
	
	Therefore:
	
	This is a relation used sometimes in some statistical models (form example the MANOVA!). 
	
	Before closing this short introduction to the eigenvalues and eigenvectors (we will discussed this further below), let us indicate that since an eigenvector must satisfy the homogeneous system:
	
	
	Thus in practice it is customary that if the eigenvector is given for example by:
	
	To normalize it at the unit by writing:
	
	
	For the section of Wave Quantum Physics and more especially for the study of the angular momentum and spin we need to prove that given an operator acting on an eigenvector, then if we square the operator, this result in squaring the eigenvalue.
	\begin{dem}
	Given:
	
	Then:
	
	\begin{flushright}
		$\square$  Q.E.D.
	\end{flushright}
	\end{dem}
	
	\subsubsection{Rotation Matrices and Eigenvalues}
	Now that we have seen what was an eigenvalue and an eigenvector, let us come back on a particular type of orthogonal matrices that we will be particularly useful to us in our study of quaternions (\SeeChapter{see section Numbers}), of groups and symmetries (\SeeChapter{see section Set Algebra}) and particle physics (\SeeChapter{Elementary Particle Physics}).
	
	We denote, as what has been seen in the section of Set Algebra, $\text{O}(n)$ the set of $n\times n$ (square) orthogonal matrices with coefficients in $\mathbb{R}$, that is to say, satisfying:
	
	That will denote also sometimes for recall sometimes as:
	
	The columns and rows of an orthogonal matrix the form the orthonormal basis of the usual space $\mathbb{R}^2$ for the usual dot product.
	
	The determinant of an orthogonal matrix is equal to $\pm 1$ (rotation conserves angles and volumes), indeed $A^T A=I$  leads to:
	
	A rotation matrix with determinant $+1$ is a "\NewTerm{proper rotation}\index{eigenvalue equations}", and one with a negative determinant $-1$ is an "\NewTerm{improper rotation}\index{improper rotation}", that is a reflection combined with a proper rotation.
	
	\pagebreak
	\begin{tcolorbox}[colframe=black,colback=white,sharp corners]
	\textbf{{\Large \ding{45}}Example:}\\\\
	Let us calculate now explicitly the determinant of a $2\times 2$ rotation matrix (\SeeChapter{see section Numbers}) and $3\times 3$ rotation matrix (\SeeChapter{see section Euclidean Geometry}) as it is ask by many student on various Internet forums.\\

	So first we consider the $2\times 2$ rotation matrix and using the relation of the determinant proved earlier, we get:
	
	And for one randomly chosen rotation matrix of the three $3\times 3$ rotation matrices (\SeeChapter{see section Euclidean Geometry}), we get:
		
	\end{tcolorbox}
	
	We denote by $\text{SO}(n)$ the set of orthogonal matrices of determinant $1$ (for more details see the section of Set Algebra). Let us show in three points that if $A\in \text{SO}(3,\mathbb{R})$  then $A$ is the rotation matrix relative to an axis passing through the origin.
	
	\begin{enumerate}
		\item Any eigenvalue of a rotation matrix $A$ (real or complex) is of module $1$. In other words it conserve the norm:

		Indeed, if $\lambda$ is an eigenvalue of eigenvector $\vec{X}$, we have:
		
		or noting the dot product with the book usual notation:
		
		
		\item  It exists a straight line in the space that is used a rotation axes and any vector on this line is not modified by any rotation.

		Let us denote by $\vec{X}$ an eigenvector  of eigenvalue $1$ (that is to say such that $A\vec{X}=\vec{X}$). As the reader may have perhaps already understand it (read until the end please!), the straight line generated by $\vec{x}$ that we will denote by $\langle \vec{X} \rangle$ constitutes our rotation axes.
	
		Indeed, any vector $\langle \vec{X} \rangle$ is send on itself by the application $A$. In this case, the orthonormal space denoted by $\langle \vec{X} \rangle^\perp$ that is of dimension $2$ is the perpendicular plane to the rotation axes.
	
		\item Any vector perpendicular to the rotation axes remains, after rotation, perpendicular to this axes. In other words,  $\langle \vec{X} \rangle$ in invariant through the application of $A$.
		
		Indeed, if $\vec{w}\in \langle \vec{x} \rangle$ the, $w=A^TA w=A^Tw$ and for all $\vec{y}\in \langle \vec{X} \rangle^\perp$:
		
		that is to say $A \vec{y} \langle \vec{X} \rangle^\perp$. Therefore $\langle \vec{X} \rangle^\perp$ is invariant by $A$.
		
		Finally, the restriction of $A$ to the space $\langle \vec{X} \rangle^\perp$ is a rotation!
	\end{enumerate}
	\begin{tcolorbox}[colframe=black,colback=white,sharp corners]
	\textbf{{\Large \ding{45}}Example:}\\\\
	Given $e^{\mathrm{i}\alpha}$ (see the section Numbers where the rotation by the complex number is proven) an eigenvalue (which module is $1$ as we proved it during our study of complex numbers) of $A$ restraint to $\langle \vec{X} \rangle^\perp$.\\
	
	Let us write $\vec{w}=\vec{u}+\mathrm{i}\vec{v}$ an eigenvector with $\vec{u},\vec{v}\in \mathbb{R}^2$ such as:
	
	with (as we already proved it in our study of complex numbers):
	
	where we know by our study of complex numbers, that the vectors $\vec{u},\vec{v}$ generate an orthogonal basis (not necessarily normalized at the unit!) of $\langle \vec{X} \rangle^\perp$.
	\begin{tcolorbox}[title=Remark,colframe=black,arc=10pt]
	We think that it could by easy at this level of the reader to check that this matrix is orthogonal (if it not the case contact us and this will be detailed!).
	\end{tcolorbox}
	\end{tcolorbox}
	
	\pagebreak
	\subsection{Spectral Theorem}
	Let us now see a very important theorem relatively to the eigenvalues and eigenvectors which is named the "\NewTerm{spectral theorem}\index{spectral theorem}" which will be very useful to us for the various sections of physics of this book and also the section  Statistics as well as in the section of Theoretical Computing and Industrial Engineering.
	
	To summarize, mathematicians say in their language that the spectral theorem give the possibility to affirm the diagonability of endomorphism (of matrices) and also justify the decomposition in eigenvalues (also named "\NewTerm{singular value decomposition S.V.D.}\index{singular value decomposition }").
	\begin{tcolorbox}[title=Remark,colframe=black,arc=10pt]
	The singular value decomposition theorem (S.V.D.) is however very general, in the sense that it applies to any rectangular matrices. The eigenvalue decomposition, however, only works for some square matrices.
	\end{tcolorbox}
	To simplify the proof, we will deal here only real matrices (component in $\mathbb{R }$) and also avoiding up the language of mathematicians.
	
	We will note in a first time $M_n(\mathbb{R})$ the set of all $n\times n$ matrices with real coefficients.
	
	We will confuse the matrix $M\in M_n (\mathbb{R})$ with the linear application on the vector space $\mathbb{R}^n$ by:
	
	with $\vec{v}\in \mathbb{R}^n$.
	
	Reminder: We have seen above during our study of basis changes that if $(\vec{c}_1,\ldots,\vec{c}_n)$ is a basis of $\mathbb{R}^n$ and $M\in M_n(\mathbb{R})$ then the matrix of the linear map $M$ in the basis $(\vec{c}_1,\ldots,\vec{c}_n)$  is:
	
	where $S$ is the matrix formed by the column vectors $\vec{c}_1,...,\vec{c}_n$.
	
	First, we simply check that if $A$ is a symmetric matrix then (this should be trivial but it can be verified with an example of dimension $2$ very quickly):
	
	\begin{enumerate}
		\item[P1.] All eigenvalues of $M$ are reals.
		\begin{dem}
		Given:
		
		an a priori complex eigenvector of the eigenvalue $\lambda \in \mathbb{C}$. Let us denote:
				
		the conjugate vector of $\vec{z}$. Then we have:
		
		On the other hand since $M=\overline{M}$ we have:
		
		As $\vec{z} \neq \vec{0}$, we have $\lambda=\vec{\lambda}$ and therefore, $\lambda\in \mathbb{R}$.
		\begin{flushright}
		$\square$  Q.E.D.
		\end{flushright}
		\end{dem}
		Before going further, we also have to prove that if $M\in M_n(\mathbb{R})$ is a symmetrical matrix and $V$ and vectorial subspace of $\mathbb{R}^n$ invariant relatively to $M$ (that is to say that satisfies for any $\vec{v}\in V: \; M\vec{v}\in V$) the we have the following properties:
		
		\item[P3.] The orthogonal of $V$ denoted obviously by $V^\perp$ (obtained by applying the Gram-Schmidt method seen in the section of Vector Calculus) is also invariant through $M$.
		
		\begin{dem}
		Given $\vec{v}\in V$ and $\vec{w}\in V^\perp$ then:
		
		this show well that $M\vec{w}\in V^\perp$.
		\begin{flushright}
		$\square$  Q.E.D.
		\end{flushright}
		\end{dem}

		\item[P4.] If $(\vec{w}_1,\ldots,\vec{w}_k)$ is an orthonormal basis of $\vec{V}^\perp$ then the restriction matrix of $M$ to $V^\perp$ in the basis $(\vec{w}_1,\ldots,\vec{w}_k)$  is also symmetrical.
		\begin{dem}
		
		Let us denote $A=(a_{ij})_{1\leq i,j\leq k}$ the matrix of the restriction of $M$ to $V^\perp$ in the basis $(\vec{w}_1,\ldots,\vec{w}_k)$. We have by definition for any $j=1...k$ (as the vector resulting of a linear application such as $M$ can be express in its basis):
		
		Or:
		
		as:
		
		if $i\neq m$ in the orthonormal basis.
		
		On another side:
		
		Therefore:
		
		This shows that:
				
		\begin{flushright}
		$\square$  Q.E.D.
		\end{flushright}
		\end{dem}
	\end{enumerate}
	\begin{theorem}
		We will now be able to show that any symmetric matrix $M \in M_n(\mathbb{R})$ is diagonalizable. That is to say that there is an invertible matrix $S$ such that the result of the calculation:
	
	gives a diagonal matrix! This result is in this text a particular form of the more general case (thus also applicable to rectangular matrices) named "\NewTerm{Eckart-Young theorem}\index{Eckart-Young theorem}".
	\begin{tcolorbox}[title=Remark,colframe=black,arc=10pt]
	In fact we will see, to be more precise, that there exists an orthogonal matrix $S$ such that $S^{-1}MS$ is diagonal.
	\end{tcolorbox}
	Reminder: Say that $S$ is orthogonal means that $SS^T=I$ (where $I$ is the identity matrix) which is equivalent to say that the columns of $S$ form an orthonormal basis of $\mathbb{R}^n$.
	\end{theorem}
	\begin{dem}
	We prove the assertion by induction on $n$. If $n=1$ there is nothing to prove. Let us suppose that the assertion is satisfied for $k\leq n$ and let us prove it for $k=n+1$. Then given $M\in M_{n+1}(\mathbb{R})$ a symmetric matrix and $\lambda$ an eigenvalue of $M$.
	
	We easily verify that the eigenspace:
	
	is invariant by $M$ (just take any numerical application) and that by the proof seen earlier, that $W^\perp$ is also invariant by $M$. Moreover, we know (\SeeChapter{see section Vector Calculus}) that $\mathbb{R}^{n+1}$ can be decomposed into a direct sum:
	
	If:
	
	then:
	
	and it is sufficient to take an orthonormal basis of $W$ to diagonalise $M$. Indeed, if $(\vec{w}_1,\ldots,\vec{w}_{n+1})$ is such a basis, the matrix $S$ formed by the column vectors $\vec{w}_j$ ($j=1\ldots n+1$) is orthogonal and satisfies:
	
	and $S^{-1}MS$ is indeed diagonal.
	
	Let us now suppose that $\dim(W^\perp)>0$ and given $(\vec{u}_1,\ldots,\vec{u}_m)$ with $m\leq n$ an orthonormal basis of $W^\perp$. Let us denote by $A$ the restriction matrix of $M$ to $W^\perp$ in the basis  $(\vec{u}_1,\ldots,\vec{u}_m)$ . $A$ is also  symmetric (as proved in one of the preceding properties).
	
	By induction hypothesis there exists an orthogonal matrix $H\in M_m(\mathbb{R})$ such that $H^{-1}AH$ is diagonal.
	
	Let us denote by $(\vec{w}_1,\ldots,\vec{w}_{n+1-m})$ an orthonormal basis of $W$ and $G$ the matrix formed by the column vectors: $\vec{w}_1,\ldots,\vec{w}_{n+1-m},\vec{u}_1,\ldots\vec{u}_m$. So we can write that:
	
	and $G$ is also orthogonal by construction.
	
	Let us consider the following block matrix (matrix of matrices):
	
	and let us put:
	
	It is almost obvious that $S$ is orthogonal as $G$ and $L$ are also orthogonal. Indeed, if:
	
	then (remember hat matrix multiplication is associative !!!):
	
	Also $S$ satisfies:
	
	and then:
	
	is indeed diagonal.
	\begin{flushright}
		$\square$  Q.E.D.
	\end{flushright}
	\end{dem}
	Finally here is finally the famous "\NewTerm{spectral theorem}\index{spectral theorem}" (real case):
	\begin{theorem}
	Given $M\in M_n(\mathbb{R})$ a symmetric matrix, then there exists an orthonormal basis made of eigenvectors of $M$.
	\end{theorem}
	\begin{dem}
	So we have seen in the preceding paragraphs that there exists an orthogonal matrix $S$ such that $S^{-1}MS$ is diagonal if $M$ is symmetric! Let denote by $\vec{c}_1,\ldots,\vec{c}_n$ the columns of $S$. The basis $(\vec{c}_1,\ldots,\vec{c}_n)$ is an orthonormal basis of $\mathbb{R}^2$ as $S$ is orthogonal. Denoting the $\vec{e}_i$ the $i$-th vector of the canonical basis of $\mathbb{R}^n$ and $\lambda_i$ and the $i$-th diagonal coefficient of $S^1{M}S$ we have without directly supposing that $\lambda_i$ is an eigenvalue for now:
	
	by multiplying by $S$ on both sides of the equality we have:
	
	and therefore:
	
	\begin{flushright}
		$\square$  Q.E.D.
	\end{flushright}
	\end{dem}
	To finish about the spectral theorem in this book, let us so reprove a result we get earlier but that was presented in a quite ugly way and poorly  rigorous (the sum of the eigenvalues equals the trace of a matrix):
	
	Remember that spectral theorem therefore tells us that for any symmetric matrix $M$, there exists an orthogonal matrix $S$ such that:
	
	is diagonal. Nothing prevents us to choose the resulting diagonal matrix as a matrix of eigenvalues in the diagonal. What we denote usually:
	
	and as $S$ is a real orthogonal matrix and that by definition we have that a matrix is orthogonal if and only if $A^{-1}=A^T$, then we find the following relation as frequently as follows:
	
	So obviously therefore have we will have to found $S$ if $M$ is known or vice versa. Anyway, let us come back on our topic and take track of this relation:
	
	Then by using the property of the trace $\text{tr}$, of the associativity of the matrix multiplication, and the orthogonality of $S$ we have:
	
	This reprove the results seen earlier above with a condition that was not trivial at this time: the matrix must be symmetrical (or symmetrizable)!
	
	We also have by extension:
	
	and therefore by using the proven property relatively to the determinant (during our proofs of the main determinant properties) and  the conjugated matrices we get:
	
	and therefore if $M$ is symmetric we have the property:
	
	
	\begin{tcolorbox}[colframe=black,colback=white,sharp corners]
	\textbf{{\Large \ding{45}}Example:}\\\\
	We want to show that:
	
	we assume that we know that the eigenvalue-eigenvector pairs are:
	
	We therefore introduce $S$ and $\Lambda$ as follows:
	
	We must show that $M=S\Lambda S^{-1}$. This is indeed the case, since:
	
	Therefore $M$ is indeed diagonalizable.
	\end{tcolorbox}
	
	\pagebreak
	\subsection{Singular Value Decomposition (SVD)}
	The singular value decomposition of a matrix $M$ ($m\times n$) is the factorization of $M$ into the product of three matrice:
	
	where the columns of $U$ ($m\times r$) and $V$ ($r\times n$) are orthonormal such that:
	
	and the matrix $D$ ($n\times n$) is diagonal with positive real entries. 

	The SVD is useful in many tasks as Data Mining, Image Processing and Advanced Numerical Methods.

	To gain insight into the SVD, we treat the rows of an $m\times n$ matrix $M$ as $m$ points in a $n$-dimensional space and consider the problem of finding the best $k$-dimensional subspace with respect to the set of points. Here "best" means minimize the sum of the squares of the perpendicular distances of the points to the subspace. 

	Let us begin with a special case of the problem where the subspace is 1-dimensional: a line through the origin. We will see later that the best-fitting $k$-dimensional subspace can be found by $k$ applications of the best fitting line algorithm. Finding the best fitting line through the origin with respect to a set of points $\{x_i|1 \leq i \leq m\}$ in the plane means minimizing the sum of the squared distances of the points to the line. Here distance is measured perpendicular to the line. The problem is then named as we know: the "best least squares fit" (\SeeChapter{see section Theoretical Computing}).
	
	Consider projecting a point $\vec{x}_1$ onto a line through the origin:
	\begin{figure}[H]
		\centering
		\includegraphics[scale=1]{img/algebra/svd.jpg}
		\caption{The projection of the point $\vec{x}_i$ onto the line through the origin in the direction of $\vec{v}$}
	\end{figure}
	Then us Pythagorean theorem we get:
	
	That is:
	
	Therefore (see figure):
	
	To minimize the sum of squares for the distance to the line, one could minimize $\sum_{i=1}^m (x_{i1}^2+x_{i2}^2+\ldots+x_{in}^2)$ minus the sum of the square of the lengths of the projections of the points to the line. However,  $\sum_{i=1}^m (x_{i1}^2+x_{i2}^2+\ldots+x_{in}^2)$ is constant! (independent of the line), so minimizing the sum of the squares of the distances is equivalent to maximizing the sum of the squares of the lengths of the projections onto the line. Similarly for best-fit subspaces, we could maximize the sum of the squared lengths of the projections onto the subspace instead of minimizing the sum of squared distances to the subspace.
	
	\subsubsection{Singular Vectors}
	We now build the "\NewTerm{singular vector}\index{singular vector}" of a $m\times n$ matrix $M$. 

	Consider the rows of $M$ as $m$ points in a $d$-dimensional space. Consider the best fit line through the origin. Let $\vec{v}$ be a unit vector along this line.

	The length of the projection of $\vec{x}_i$, the $i$-th row $M$, onto $\vec{v}$ is (\SeeChapter{see section Vector Calculus}):
	
	That we will denote for what will follow as (\SeeChapter{see section Vector Calculus}):
	
	So in our case:
	
	From this we denote the sum of all lengths of the projections by:
	
	The best fit line is the on maximizing $|M\vec{v}|^2$ (ie: $|M\vec{v}|$) and hence minimizing the sum of the squared distances of the points to the line.
	
	With this in mind, we define the "\NewTerm{first singular vector $\vec{v}_1$}\index{first singular vector}", of $M$, which is is a vector, as the best fit through the origin for the $m$ points in $n$-space that are the rows of $M$. Thus:
	
	The scalar value:
	
	is named the "\NewTerm{first singular value}\index{first singular value}" of $M$. Notice that $\sigma_1^2$ is therefore implicitly the sum of the squares of the projections of the points to the line determined by $\vec{v}_1$.
	
	The greedy approach to find this time not the best fit $1$-dimensions but $2$-dimensional subspace for a matrix $M$, takes $\vec{v}_1$ as the first basis vector for the $2$-dimensional subspace and finds the best $2$-dimensional subspace containing $\vec{v}_1$.

	Thus, instead of looking for the best $2$-dimensional subspace containing $\vec{v}_1$, look for a unit vector, denoted $\vec{v}_2$, perpendicular to $\vec{v}_1$ that maximizes $|M\vec{v}|^2$ among all such unit vectors.

	Using the same strategy to find the best three and higher dimensional subspaces, defines $\vec{v}_3,\vec{v}_4,\ldots$ in similar manner.
	
	The "\NewTerm{second singular vector $\vec{v}_2$}", is defined by the best fit line perpendicular to $\vec{v}_1$:
	
	The value:
	
	is named the "\NewTerm{second singular value}" of $M$. 

	The "\NewTerm{third singular vector $\vec{v}_3$}" is defined similarly by:
	
	and so on...
	
	The process stop theoretically when we have found $\vec{v}_1,\vec{v}_2,\ldots,\vec{v}_r$ as singular vector that satisfies:
	
	
	As the $\vec{v}_i$ are perpendiculars, if we apply $M$ on all this vectors, the resulting vectors will also be perpendicular between them!

	Therefore we build the vectors:
	
	that are all perpendiculars vectors between them as already mentioned and named "\NewTerm{left singular vectors}\index{left singular vectors}" of $M$ when the $\vec{v}_i$ will be named "\NewTerm{right singular vectors}\index{right singular vectors}". The SVD theorem will fully explain the reason for these terms.
	
	\begin{theorem}
	Let $M$ be an $m\times n$ matrix with right singular vector $\vec{v}_1,\ldots,\vec{v}_r$, left singular vectors $\vec{u}_1,\ldots,\vec{u}_r$, and corresponding singular values $\sigma_1,\ldots,\sigma_n$. Then the "\NewTerm{singular value decomposition theorem}\index{singular value decomposition theorem}" states that:
	
	\end{theorem}
	\begin{dem}
	We start naturally from:
	
	\begin{tcolorbox}[title=Remark,colframe=black,arc=10pt]
	Don't forget that $\vec{x}^T\vec{x}$ in linear algebra gives a scalar that is equivalent to the "dot product" ("inner product"), when instead $\vec{x}\vec{x}^T$ gives a square matrix named the "\NewTerm{outer product}\index{outer product}".
	\end{tcolorbox}
	Now let us take the a special case with (as i don't like the general proof):
	

	and let us wee what gives:
	
	So if we look closely the result is the same as if we define the matrix;
	
	Therefore we can see that:
	
	leads to the same result. Therefore we can write:
	
	But we must not forget that if  $V$ is an orthogonal matrix, then it represents an orthonormal basis and then we have proved already earlier above that in this case:
	
	Therefore:
	
	And this finish the proof!
	\begin{flushright}
		$\square$  Q.E.D.
	\end{flushright}
	\end{dem}
	The reader should also notice that:
	
	and:
	
	are equivalent notation for the same thing (just develop the last one explicitly and you will see you fall back on the same result\footnote{On request we can write the details}! The difference is that the notation with the sum is most used in Data Mining and that with the matrices in Statistics.
	\begin{figure}[H]
		\centering
		\includegraphics[scale=1]{img/algebra/svd_multiplication.jpg}
		\caption{The SVD decomposition of a $m\times n$ matrix}
	\end{figure}
	It is usage to build the matrix $D$ such that the diagonal is in descending order of amplitude and to order the vectors $\vec{v}_i$ in the corresponding order. The reason is quite easy to understand as you can see in the example below:
	
	This is important to know that this not the only possible decomposition of a matrix. The are many other one but we will focus in this book only the decomposition that are directly useful for engineering topics presented in this book.
	\begin{tcolorbox}[title=Remark,colframe=black,arc=10pt]
	Some authors prefers to work with the norm of the singular values, that is, with $\sqrt{\sigma_i}$, therefore the left singular value are defines as:
	
	Without that it change our previous proof result that will just be:
	
	That in Europe is frequently written (but can bring to confusion with the notation of eigenvalues...):
	
	or in matrix form (...):
	
	\end{tcolorbox}
	
	\pagebreak
	Take this example in image processing (made by Jason Liu in MATLAB™), the image below is an image made of $400$ unique row vectors (the reader can found the equivalent example in our R companion book):
	\begin{figure}[H]
		\centering
		\includegraphics[scale=0.6]{img/algebra/svd_feynman_original.jpg}
		\caption{SVD MATLAB™ example original image}
	\end{figure}
	What happens if in the sum:
	
	we take only the first biggest singular vector?:
	\begin{figure}[H]
		\centering
		\includegraphics[scale=0.6]{img/algebra/svd_feynman_r_equal_1.jpg}
		\caption[]{SVD MATLAB™ SVD with $r=1$}
	\end{figure}
	What happens if I take the first two singular vectors?:
	\begin{figure}[H]
		\centering
		\includegraphics[scale=0.6]{img/algebra/svd_feynman_r_equal_2.jpg}
		\caption[]{SVD MATLAB™ SVD with $r=2$}
	\end{figure}
	...and if we take the first ten singular vectors?:
	\begin{figure}[H]
		\centering
		\includegraphics[scale=0.6]{img/algebra/svd_feynman_r_equal_10.jpg}
		\caption[]{SVD MATLAB™ SVD with $r=10$}
	\end{figure}
	...and if we take the first fity singular vectors?:
	\begin{figure}[H]
		\centering
		\includegraphics[scale=0.6]{img/algebra/svd_feynman_r_equal_50.jpg}
		\caption[]{SVD MATLAB™ SVD with $r=50$}
	\end{figure}
	There we have it! Using $50$ unique values and we get a decent representation of what $400$ unique values look like.
	
	So as we can see SVD is a great space reduction technique!
	
	\begin{flushright}
	\begin{tabular}{l c}
	\circled{95} & \pbox{20cm}{\score{4}{5} \\ {\tiny 36 votes,  79.94\%}} 
	\end{tabular} 
	\end{flushright}
	
	%to make section start on odd page
	\newpage
	\thispagestyle{empty}
	\mbox{}
	\section{Tensor Calculus}
	\lettrine[lines=4]{\color{BrickRed}T}he conventional vector calculus is a simple and effective technique that adapts perfectly to the study of mechanical and physical properties of matter in an Euclidean space of three dimensions. However, in many fields of physics, it appears experimental quantities that can't be easily represented by simple column vectors of Euclidean vector spaces. This is the case for example in continuum mechanics (fluids or solids), electromagnetism, General Relativity, etc.
	
	Thus, since the late 19th century, the analysis of forces acting within a continuous medium led to highlight the physical quantities characterized by nine numbers representing the pressure forces or internal stress (see section Continuum Mechanics for the details). The representation of these quantities required the introduction of a new mathematical tool who was named "\NewTerm{tensor}\index{tensor}", by reference to its physical origin. Subsequently, starting the years 1900, it was R. Ricci and T. Levi-Civita who developed the tensor calculus; then the study of tensor allowed a deepening of the theory of vector spaces and contributed to the development of differential geometry (see section of the same name).
	
	Tensor calculus, also sometimes named "\NewTerm{absolute differential geometry}\index{absolute differential geometry}" also has the advantage to free itself from all coordinate systems and the results of  the mathematical developments are thus invariant (huge simplification in calculations but in compensation we have a huge increase in abstraction and notation complexity). We therefore don't need to be concerned in what type of referential frame we work and this is very interesting in General Relativity.
	
	We advise strongly the reader to very well master the basics of vector calculus and linear algebra as they have been presented before in previous sections (especially because linear algebra forms the skeleton of tensor calculus!). If necessary, we have chosen when writing this section to come back on certain points seen in the section of Vector Calculus and Linear Algebra (covariant components, contravariant components, etc.).
	
	Furthermore, if the reader has already covered the study of constraints in solids (\SeeChapter{see section Continuum Mechanics}) or of the Faraday tensor (\SeeChapter{see section Electrodynamics}) or the energy-momentum tensor (\SeeChapter{see section General Relativity}) this will be a practical advantage before reading what follows. Furthermore, the redaction of the above items (tensors) was made so that the concept of tensor is  introduced if possible (...) intuitively.
	
	We will only do a very few practical examples in this section. Indeed the examples, you have probably already guess it..., will come when we will study the continuum mechanics, General Relativity, quantum field theory, electrodynamics, etc.
	
	An advice maybe: you have seen many time in the section of Statistics that write with vectors and after think with matrix was a powerful way to generalize some important results. For this section on tensors remember that the idea is the same but we think matrix and we write tensor! (you will better understand this little adage once you will be finished to read this whole section).
	
	\subsection{Tensor}
	\textbf{Definition (simplistic \#\mydef):} The "\NewTerm{tensors}\index{tensor}" are mathematical objects generalizing the concepts of vectors and matrices. They were introduced in physics to represent the state of stress and deformation of a volume subjected to forces, hence their name (tensions).
	
	The rigorous definition requires (I personally think...) to have first read this section in its whole. But you must know that in fact a tensor is roughly like a determinant... (\SeeChapter{see section Linear Algebra}). Eh yes! It is simply a multilinear application on a space of a given size (corresponding to the number of columns of the matrix/tensor) which finally gives a scalar (of a given field).
	
	For example, we have proved in the section of Continuum Nechanics that normal and tangential forces in a fluid were given by the relation:
	
	what was noted in the traditional condensed form as following (where we no longer distinguish what is tangential to what is normal so there is a loss of clarity):
	
	We thus make appear a mathematical quantity $\sigma_{ij}$ with $9$ components, while a vector in the same space $\mathbb{R}^3$ has $3$ components.
	
	This notion is also much used in the section of General Relativity where we have proved that the energy-momentum tensor in a particularly simple case was given by:
	
	and satisfies the non-less important equation of conservation:
	
	Or otherwise, still in the section of General Relativity, we have shown that the tensor of the Schwarzschild metric was given by:
	
	and therefore gives us the equation of the metric (\SeeChapter{see section Differential and Integral Calculus}):
	
	Note also that in the section of Special Relativity we have shown that the Lorentz transformation tensor is given by:
	
	which in a condensed form gives the following components transformation:
	
	As regards the transformation of the electromagnetic field we have also proved that the Faraday tensor is given by:
	
	and therefore permits switch from one repository frame to another using the relation:
	
	But these are very simple tensor that can be represented in the form of matrices. You should also remember that it is not because you are reading a variable with indices suggests that we are dealing with a tensor that it is necessarily one. For example, the famous relation (widely used in the section of General Relativity and we will prove far further below):
	
	might suggest that the first member of the far left is a tensor but in fact it is not... this is just a symbol... hence its name: Christoffel \underline{symbol} (not: Christoffel \underline{tensor}).
	
	The interest of tensors in physics is that their characteristics are independent of the chosen coordinates. Thus, a relation between tensors in a base will be true regardless of the base used thereafter. This is a fundamental and powerful characteristic of General Relativity (among others)!
	
	\pagebreak
	\subsection{Indicial Notation}
	We will use thereafter many mathematical symbols: coordinates, components of vectors and tensors, matrix components, etc., whose number in each category is large or indeterminate. To distinguish the various symbols of a category we use indices. For example, instead of the traditional variables $x, y, z$ we will use the variables $x_1,x_2,x_3$ (as we have already done in the section of Linear Algebra). This rating becomes essential when we an undetermined number of variables.
	
	Thus, if we have $n$ variables, we denote them by $x_1,x_2,...,x_n$.
	
	We will also use superscripts, when required; eg $x^1,x^2,x^3$. To avoid confusion with writing powers, the quantity $xî$ to the power $p$ will be written $(x^i)^p$. When the context eliminates any potential ambiguity, the use of parentheses however is not fundamentally necessary.
	
	In tensor calculus there is a summation convention using the fact that the repeated index, below for example the index $i$, will become itself an indication of the summation. We write then, with this convention:
	
	thereby this condense relatively well the notations!
	
	Thus, to represent the linear system:
	
	we will write (notice carefully how are written the components of the associated matrix!):
	
	specifying that's for $n=3$.
	
	We see in this example, how the summation convention allows a condensed and thus powerful writing.
	
	The summation convention covers all the mathematical symbols having repeating indices. Thus the decomposition of a vector $\vec{x}$ on a basis $(\vec{e}_1,\vec{e}_2,\vec{e}_3)$ will be therefore written for $n=3$:
	
	In summary, any term that has a repeated index represents a sum over all possible values of the repeated index.
	
	\begin{tcolorbox}[title=Remark,colframe=black,arc=10pt]
	We name, for obvious reasons we will detail below, the $x^i$ "\NewTerm{contravariant component}\index{contravariant component}" of the vector $\vec{x}$.
	\end{tcolorbox}
	
	\pagebreak
	\subsubsection{Summation on multiple index}
	The summation convention (due to Einstein) extends to the case where we have, in general, several repeated indices in upper and lower positions so-named "\NewTerm{dumb indices}\index{dumb indices}" in the same monomial (often physicists omit the rule of set their position opposite as it will be the case often on this book too!). Thus, for example, the quantity $A_i^jx^ix^j$, represents the following sum for $i$ and $j$ taking the values from $1$ to $2$:
		
	Thus we see easily that an expression with two summation indices that take values respectively $1,2,...,n$, will have $n^2$; will have $n^2$ terms, $n^3$ if there are three sommation indices, etc.
	
	However, we must be careful to substitutions with this kind of notation because if we assume that we have the relation:
	
	then to get the expression of $A$ only in function of the variables $y^j$ we cannot write:
	
	because it does not return to the same expression as the dumb indices after development are systematically sum in an identical and rigid way (we leave to the reader make a little application case to see this, if need you can contact us and we will do an example). In other words, a dumb index can not be repeated more than $2$ times.
	
	\subsubsection{Kronecker Symbol}
	This symbol introduced by the mathematician Kronecker, is the following (often used in physics and in many other fields):
	
	This symbol is named "\NewTerm{Kronecker symbol}\index{Kronecker symbol}". It conveniently allows you to write, for example, the dot product of two vectors $\vec{e}_1$ and $\vec{e}_2$, of unit norm and orthogonal to each other, in the form:
	
	We will find this symbol in many examples of theoretical physics in this book (wave quantum physics, quantum field theory, general relativity, fluid mechanics, etc.).
	
	It should be noted that there is a generalized version of the Kronecker symbol:
	
	We have also, for example:
	
	where $\varepsilon_{ijk}$ is the Levi-Civita symbol that will define right now:
	
	\subsubsection{Antisymmetric Symbol (Levi-Civita symbol)}
	Another useful symbol is the "\NewTerm{symbol of anti-symmetry}\index{symbol of anti-symmetry}" or also named "\NewTerm{antisymmetry tensor}\index{antisymmetry tensor}" that we will find in the sections of Electrodynamics,  General Relativity and Relativistic Quantum Physics in this book.
	
	In mathematics, particularly in linear algebra, tensor analysis, and differential geometry, the Levi-Civita symbol represents a collection of numbers; defined from the sign of a permutation of the natural numbers $1, 2, …, n$, for some positive integer $n$.
	
		 The values of the Levi-Civita symbol are independent of any metric tensor and coordinate system. Also, the specific term "symbol" emphasizes that it is not a tensor because of how it transforms between coordinate systems, however it can be interpreted as an antisymmetric tensor (a tensor is antisymmetric on (or with respect to) an index subset if it alternates sign (+/-) when any two indices of the subset are interchanged).
	 
	 In the case $n=2$ the Levi-Civita symbol is defined by:
	 
	The values can be arranged into a $2\times 2$ antisymmetric matrix (we can see we fall back on the definition of an antisymmetric tensor):
	
	Use of the 2D symbol is relatively uncommon, although in certain specialized topics like supersymmetry and twistor theory it appears in the context of 2-spinors. The 3D and higher-dimensional Levi-Civita symbols are used more commonly.
	
	In three dimensions, the Levi-Civita symbol is defined as follows:
	
	i.e.  $\varepsilon_{ijk}$  is $1$ if $(i, j, k)$ is an even permutation of $(1,2,3)$ or in the natural order $(1,2,3)$, $-1$ if it is an odd permutation, and $0$ if any index is repeated. In three dimensions only, the cyclic permutations of $(1,2,3)$ are all even permutations, similarly the anticyclic permutations are all odd permutations. This means in 3D it is sufficient to take cyclic or anticyclic permutations of $(1,2,3)$ and easily obtain all the even or odd permutations.

	It can also be express with the Kronecker symbol:
	
	
	An illustrative representation gives:
	\begin{figure}[H]
		\centering
		\includegraphics[scale=0.75]{img/algebra/levi_civita_symbol.jpg}
		\caption{3D Levi-Civita symbol illustration (source: Wikipedia)}
	\end{figure}
	By using this symbol, a determinant of order two (\SeeChapter{see section Linear Algebra}) is then written in the advantageous form:
	
	and the vector cross product:
	
	where of course, $j$ and $k$ are summed and where the dummy index $i$ is the line number of the resulting vector (if requested we will make the developments). In particular, the rotational of a vector field (\SeeChapter{see section Vector Calculus}) is then:
	   
	As an example, let us calculate in index notation the double vector product $\vec{A}\times\vec{B}\times\vec{C}$:
	 
	where again, the dumy index $i$ is the line number of the resulting vector. Let us eee detailed demonstration of these equalities (the order of the equalities below does not need to follow the sequence of equalities of the previous relation).
	\begin{dem}
	We have proved in the section of Vector Calculus the following identity:
	
	\begin{tcolorbox}[title=Remark,colframe=black,arc=10pt]
	The latter relation is sometimes named the "\NewTerm{Grassmann rule}\index{Grassmann rule}" or more commonly "\NewTerm{dual vector product}\index{dual vector product}".
	\end{tcolorbox}	
	To prove the relation:
	
	to a change of indices let us first prove that:
	
	which give us:
	
	We do the development only for the first line (this is already bor... euh long enough...):
	
	This is the first step that was necessary to be proven.
	
	Now let us prove that for the $n$th line we have well:
	
	with the help of a result obtained in the section of Vector Calculus (vector product of three different vectors) we have the first term (the first line of the vector resulting from the calculation):
	
	It is immediate that ($i$ being equal to $1$):
	
	Let us show now that for $i$ equal $1$ we also have:
	
	Indeed:
	
	\begin{flushright}
		$\square$  Q.E.D.
	\end{flushright}
	\end{dem}
	As a second example, letus prove that the divergence of a curl vanishes:
	
	By the Schwarz theorem (\SeeChapter{see section Differential and Integral Calculus}) $\partial_i\partial_j$ is symmetric (so invert the indices has no impact) in the indices and that $\varepsilon_{ijk}$ is antisymmetric (by definition) in the same indices, the sum on $i$ and $j$ must necessarily cancel. For example, the contribution to the sum of the $i=1,j=2$ is the opposite of this with $i=2,j=1$.
	
	\begin{tcolorbox}[title=Remarks,colframe=black,arc=10pt]
	\textbf{R1.} The symbol of antisymmetry is often named "\NewTerm{Levi-Civita tensor}\index{Levi-Civita tensor}" in the literature. In fact, although it is a tensor in the form of its notation, it's more of a mathematical tool that a mathematical "being" hence the preference of some physicists to name it "symbol" rather than "tensor". But it's up to you ...\\
	
	\textbf{R2.} By abuse of writing we do not write the basic vector but rigorously, and to avoid forgetting it, remember that in order to balance the members of the equality and in order to clarify that the vectors are expressed in the same base, we should write:
	
	\end{tcolorbox}
	Let us now see the first simple practical applications of this index notation using the example of the base change that we have seen in the section of Vector Calculus:
	
	Given two bases $(\vec{e}_1,\vec{e}_2,\ldots,\vec{e}_n)$ and $(\vec{e}_1^{'},\vec{e}_2^{'},\ldots,\vec{e}_n^{'})$ of an Euclidean vector space $\mathcal{E}^n$. Each vector of a base can be decomposed on the other base in the form of a linear application (base change matrix - see section Linear Algebra):
	
	where we obviously use the summation convention for $i,k=1,2,\ldots,n$.
	
	Let us recall that the base change matrix (or "\NewTerm{transformation matrix}\index{transformation matrix}") should have as many columns as the basic vector have lines (dimensions or components). Small example with three dimensions gives:
	
	and obviously it is much more funny to write this as:
	
	so where on $A$, we have the $k$ that represents the column of the matrix and $i$ the row of the matrix.

	Any vector $\vec{x}$ of $\mathcal{E}^n$ can be decomposed (we have already prove this in the section of Vector Calculus) on each basis $\mathcal{E}^n$ under the form:
	
	If we seek for the relations between the components $x^i$ and ${x'}^k$ it is enough to take again the relations prove in the section of Linear Algebra and we have then:
	
	Immediately by the uniqueness of the decomposition of a vector on a base, we can equalize the components of the basis vectors and we get (the must be careful to rearrange again the order of the terms because the matrix multiplication is in general, not commutative as we already know!):
	
	By construction we also the trivial relation (\SeeChapter{see section Linear Algebra}):
	
	A way to prove in a quite general way the previous relation using tensor calculus notation is to remember the following result proved in the section of Linear Algebra:
	
	and by using:
	
	We therefore have:
	
	The basis vectors being linearly independent, this last relation implies that when $i\neq j$:
	
	and when $i=j$:
	
	Therefore it comes:
	
	And for the dot product, the results obtained with the index notation are very interesting and extremely powerful. We have already defined the scalar product in the section of Vector Calculus but let us see how we handle this with the index notation:

	Let us consider an Euclidean vector space $\mathcal{E}^n$ reported any basis ${\vec{e}_i}$. We already know that vectors are written on this basis:
	
	The scalar product with respect to its properties and the index notation is then written:
	
	This is a fundamental relation for advanced physics (General Relativity and String Theory) that makes appear "\NewTerm{metric covariant tensor}\index{metric covariant tensor}" (\SeeChapter{see section Non-Euclidean Geometry}):
	
	and to satisfy the commutative property of the dot product (\SeeChapter{see section Vector Calculus}) we must obviously have the equality (at least in Euclidean space or approximated as...):
	
	The prior-previous relation is sometimes written in the form:
	
	\begin{tcolorbox}[title=Remark,colframe=black,arc=10pt]
	When the basis vectors $\vec{e}_i$ form an orthogonal vector space (not necessarily orthonormal) the quantities:
	
	are obviously zero when $i \neq j$. The dot product of two vectors $\vec{x}$ and $\vec{y}$ is then reduce to:
	
	We the have in this particular case:
	
	and therefore when the basis vectors form an orthonormal vector space it is clear that $g_{ij}$ is equal to the Kronecker symbol alone such that:
	
	\end{tcolorbox}
	
	\subsection{Metric and Signature}
	As we have seen in it in the section of Vector Calculu (and Topology), the dot product of a vector $\vec{x}$ can be used to define the concept of norm of a vector (and also the concept of distance).
	
	Let us recall that we have by definition the norm of a vector which is given by (\SeeChapter{see section Vector Calculus}):
	
	where the numbers $g_{ij}$ define somehow a "measure" of the vectors; we then say in the language of tensor calculus that they are the "metrics" of the selected vector space.
	
	In the space of classical geometry, the norm is a number that is always strictly positive and which becomes zero if the measured vector is also zero. By cons the previous expression of the norm of a vector, may eventually be negative for any numbers $g_{11},g_{12},\ldots,g_{nn}$ (complex spaces for example). So we can distinguish two kinds pre-Euclidean vector spaces (Euclidean space in which we define a scalar product for recall) depending on the fact that the norm is positive or not. However when in theoretical physics we want to make the analogy with a vector space structure we need that the condition:
	
	is satisfied ($g_{ij}$ can be written as a matrix, nothing avoid us to do it).
	
	Explanations: We know that the dot product must satisfy the commutative property such that:
	
	On the other hand, if for any nonzero $y^j$ we have:
	
	this implies $x^i=0$ (that is one of the properties of the norm we saw in the section of Vector Calculus). We can then write:
	
	We are here simply with a system of $n$ equations with $n$ unknowns (having to admit by hypothesis that only the solution $x^i=0$), it is necessary and sufficient for this that the determinant of the system, denoted $g$, to be is different from zero (\SeeChapter{see section Linear Algebra}). So we must have:
	
	It is one of the condition for an expression comparable to a norm under a tensor index notation form in the context of theoreticla physicsa vector space of the states of the system !!
	
	\begin{tcolorbox}[title=Remarks,colframe=black,arc=10pt]
	\textbf{R1.} The number of $+$ and $-$ signs found in the expression of the dot product is a is a characteristic of a given vector space $E^n$. It is named the "\NewTerm{signature of the vector space}\index{signature of a vector space}".\\
	
	\textbf{R2.} A practical application of calculation of the metric is presented in details in the section of General Relativity.\\
	\end{tcolorbox}
	From the coefficients of the covariant metric tensor $g_{ij}$ defining the metric of the space $E^n$, we can introduce the coefficients of the "\NewTerm{contravariant metric tensor $g^{ij}$}\index{contravariant metric tensor}" defining the metric of a "\NewTerm{dual space}\index{dual space metric}" $E_{*}^n$ by the relation:
	
	In other words, the metric tensor twice covariant is its own inverse by its equivalent twice contravariant tensor. We will prove it explicitly later by showing during our study of the of the Gram's determinant that contravariant and covariant components of a Euclidean space are equal and both space have the same number of dimensions.

	A well know special case that meets the above equality is the Minkowski's metric tensor of (\SeeChapter{see section General Relativity}), where we have:
	 
	\begin{tcolorbox}[title=Remark,colframe=black,arc=10pt]
	The space $E^n$ is also named "primal space" and if it is of Euclidean type let us recall that it is denoted $\mathcal{E}^n$.
	\end{tcolorbox}
	The dual space is underpinned by $n$ basis vectors $\vec{e}^i$ constructed from the vectors $\vec{e}_i$ such that:
	
	It is therefore easy to see that the scalar product of the vectors $\vec{e}^i$ defines the metric $g^{ij}$ of the dual space:
	
	while the vectors $\vec{e}^i$ (contravariant) and $\vec{e}_j$ (covariates) are orthogonal:
	
	We can also express a vector in the dual base by the next writing by noting that obviously the position of the dummy indices is reversed:
	
	\begin{tcolorbox}[title=Remark,colframe=black,arc=10pt]
	The components $x_i$ (orthogonal projections of the vector on the axes) are named, for reasons that we will see further below, the "covariant" components.
	\end{tcolorbox}
	So we finally have the possibility to change the vectors of a base to another one:
	
	where it is important to remember that to make a contravariant a covariant  component, we bring up the index:
	
	and conversely, to make it covariant:
	
	So, still in the case of the example of the Minkowski metric, if we consider the contravariant four-vector:
	
	Then we have:
	
	
	\subsection{Gram's Determinant}
	Let us see another approach to determine the base vectors of the dual space that can allow also a better understanding of the concept and will allow us to get an interesting result that we will use during certain calculations of General Relativity and String Theory (mainly its study using to the Lagrangian formalism).
	
	So we have for $i=j=1$:
	
	This scalar product can be seen as a normalization condition for the two bases and two scalar products $\vec{e}^2\circ\vec{e}_1=0$,$\vec{e}^2\circ\vec{e}_1=0$ as orthogonalization conditions . Thus, as $\vec{e}_1$ is perpendicular to $\vec{e}^2,\vec{e}^3$ we can write:
	
	where $c^{te}$ is a constant of proportionality. Now let us play around with the prior previous relation:
	
	Then we get:
	
	where we see appear the mixed product as we had defined it in the section of Vector Calculus.
	
	Thus we get very easily:
	
	and even for contravariant vectors (without proof as may be too obvious?):
	
	\begin{tcolorbox}[title=Remarks,colframe=black,arc=10pt]
	\textbf{R1.} The reader will have perhaps noticed that the relations above are only valid for a three-dimensional space.\\
	
	\textbf{R2.} The notation of the two previous relations is mathematically a bit unfair because in reality it is not an equality between two vectors but an application of a vector space in the other one!\\
	
	\textbf{R3.} As in physics is very frequently considered Cartesian , cylindrical and spherical orthonormal base and that denominator of the two previous relations is always equal to $1$ in these bases than the contravariant basis vectors are identified with covariant basis vectors (and vice versa ). So the covariant coordinates are equal to the contravariant coordinates for these special cases!!!!!!!!!! 
	\end{tcolorbox}	
	Now let us come back on something that will seem very old of us... In the section of Vector Calculus, we have defined and studied what were the cross product and mixed product. We will now see another way of representing them and see that this representation provides a result for the less quite relevant!
	We saw in the section of Vector Calculus that the vector product was given by:
	
	But what we did not see and we will now that we can trivially this expression is only the vector determinant of the following matrices:
	
	Yes ... so the result does not give a scalar! It is just a usage rating.
	
	But as we do the tensor calculus, we must now properly distinguish covariant and contravariant components. We'll rewrite it properly with contravariant components:
	
	Similarly, the mixed product can be written using this relation and notation:
	
	Or, looking at the expression of the determinant we see quite easily, without having to do developments, that
	
	Indeed (we calculate the determinant making use of the demonstration of the determinant with three components proved in the section of Linear Algebra):
	
	The prior-previous relation is also is frequently written:
	
	with obviously:
	
	named "\NewTerm{Euclidean volume}\index{Euclidean volume}" (indeed let us recall that the mixed product is a volume as we show it in the section of Vector Calculus!)
	\begin{tcolorbox}[title=Remark,colframe=black,arc=10pt]
	Let us recall again that if the basis vectors are orthonormal, whether they are expressed in Cartesian , cylindrical or spherical coordinates then:
	
	\end{tcolorbox}
	Moreover, we also have the important relation:
	
	Moreover, we also have the important relation:
	
	Indeed, using the relation see in the of Vector Calculus:
	
	But we have seen previously that $\vec{e}^i\circ\vec{e}_j=\delta_j^i$:
	
	and thus finally:
	
	This having been done let us come back our relation of the vector product:
	
	and let express the components of the 1st line of the determinant in their dual basis (in contravariant coordinates):
	
	Obviously, if the cross product is expressed in covariant components then we have:
	
	Now let us apply the mixed product:
	
	knowing the expression of the determinant of a square $3\times 3$ matrix  (\SeeChapter{see section Linear Algebra}) it comes immediately (we can detail on request as always in this book):
	
	Conversely, it comes almost immediately:
	
	But, we have proved in the section of Vector Calculus that $x_i=\vec{x}\circ\vec{e}_i$. It comes then:
	
	and therefore:
	
	The latter relation is often named "\NewTerm{Gram determinant}\index{Gram determinant}". A special very interesting case gives us (we use the relation between the metric components and the dot products of the vector basis we proved during our study of the metric just earlier above):
	
	written in another way:
	
	Thus, the Euclidean volume  is given by what name call the "\NewTerm{functional determinant}\index{functional determinant}" of the system (expression that we will see again and use in the section of General relativity to calculate the real volume and also in the section of String Theory):
		
	that is without units (so you have to multiply it by a factor of elementary volume to get volume units). If we note in another way the determinant:
		
	We get the common relation we can found in many books on General Relativity and String Theory but given without proof and named the "\NewTerm{Riemannian volume}\index{Riemannian volume}" form or simply "\NewTerm{volume form}":
		
	or written in the following "\NewTerm{invariant volume element}\index{invariant volume element}" form:
	
	The reader can verify normally easily enough that for the orthonormal Cartesian reference frame we fall back on the volume of a cube and that for the spherical case we fall back well on the expression of the infinitesimal volume of the sphere as used in the section of Geometric Shapes (but on request we can add the details here).
	
	If we use the result we get in the section of Differential Geometry we have therefore for any surface patch:
	

	
	\subsection{Contravariant and Covariant Components}
	So far we wrote the dummy indices arbitrarily on superscript or subscript at our discretion. However, this is not always allowed and sometimes the fact that a dummy index is in superscript or subscript has a special significance! This is often the major difficulty in the study of some theorems, because if we do not study those inddices from the beginning, we do not really know how to interpret the position of the dummy indices. The reader should then be extremely careful at this level.

	For an Euclidean vector space $\mathcal{E}^n$ reported to any base $(\vec{e}_i)$, the scalar product of a vector $\vec{x}=x^i\vec{e}_i$ by a vector its base is written as we know by (remember that this is equivalent as projecting the components on the axis corresponding to $\vec{e}_j$):
	
	Therefore:
	
	This relation is of major importance in theoretical physics and tensor calculus. It is important to remember it when we will study the contraction of indexes later (you can observe in the previous relation that we have "lowered" in the left side the index of the component of the right member of the equality).
	
	These scalar products denoted $x_j$, are named "\NewTerm{covariant components}\index{covariant components}" in the base $(\vec{e}_i)$, of the vector $\vec{x}$. These components are therefore defined by:
	
	They will denoted by lower indices !!! We will see later that these components are naturally introduced for some vectors of physics, for example the gradient vector. Moreover, the notion of covariant component is essential for tensors.
	
	\begin{tcolorbox}[title=Remarks,colframe=black,arc=10pt]
	\textbf{R1.} Never forget that this is therefore the projection of a vector on a vector of its own base!!!!\\
	
	\textbf{R2.} The basic vectors always have their indices noted down because they are their own covariant components (they project on themselves by scalar product). This is the main trick used by beginners to remember when to put lower indices (and therefore they know when to put the upper one...)!
	\end{tcolorbox}
	Conversely, the "\NewTerm{contravariant components}\index{contravariant components}" (in other words: the non-projected components) can be calculated by solving with respect to the $n$ unknowns, the system of $n$ equations:
	
	The previous relations show that the covariant components $x_j$ are related to the conventional components $x^i$ and that the contravariant components $x^i$ are therefore numbers such that:
	
	They will be indicated with superscripts !! The study of the basis changes will further justify the appellation of these different components.
	
	In a canonical orthonormal basis (very special case and corresponding to the classical cartesian, polar, cylindric and shperical coordinates), the covariant and contravariant components are the same as we already know after our study of Gram's determinant. Indeed:
	
	\begin{tcolorbox}[title=Remark,colframe=black,arc=10pt]
	We see above, that the incessant writing  of dummy superscript or subscript indices  can sometimes lead to some confusion and serious headaches ...
	\end{tcolorbox}
	
	\pagebreak
	\subsection{Operation in Basis}
	The interest of physicist for the tensor calculus, is passing parameters from one base to another for some given reasons (often the aim is to simplify the study of problems or simply because the studied states depend - or may depend - on the geometry of the space in question). It is therefore necessary to introduce the main tools relating thereto. We will also take this opportunity to present the developments that could have been already addressed in the section of Vector Calculus.
	\begin{tcolorbox}[title=Remark,colframe=black,arc=10pt]
	As I like to say... Tensor Calculus is to physics with is XML to computing science. A background independant language!
	\end{tcolorbox}
	
	\subsubsection{Gram-Schmidt Orthogonalization Method}
	The "\NewTerm{Schmidt orthogonalization method}\index{Schmidt orthogonalization method}" (also named "\NewTerm{Gram-Schmidt orthogonalization method}\index{Gram-Schmidt orthogonalization method}") allows the actual determination of an orthogonal basis for any pre-Euclidean vector space $\mathcal{E}^n$ (we could introduce this method in the section of Vector Calculus but it seemed more interesting to us to present this method in a general and aesthetic framework using tensor calculus).
	
	For this, let us consider a set of $n$ linearly independent vectors $(\vec{x}_1,\vec{x}_2,\ldots,\vec{x}_n)$ of $\mathcal{E}^n$ and suppose that for each vector we have the dot product (square of the norm):
	
	Let us seek $n$ vectors $\vec{e}_i$ orthogonal between them. Let us start for this with $\vec{e}_1=\vec{x}_1$ and let us seek for $\vec{e}_2$ orthogonal to to $\vec{e}_1$ under the form (this is a choice!!!):
	
	The mental visualization of the process is not quit easy so the reader has to trust (anyway...) the mathematical results (if once we have the time we will draw the process of the classical three dimension case).
	
	The coefficient $\lambda_1$ is calculated by writing the orthogonality relation:
	
	We deduce without too much troubles:
	
	The parameter $\lambda_1$ being determined, we get the vector $\vec{e}_2$ that is orthogonal to $\vec{e}_1$ and not zero, since the system is linearly independent $(\vec{e}_1,\vec{x}_2,\ldots,\vec{x}_n)$.
	
	Thus so far we have:
	
	The parameter $\lambda_1$ being determined, we get the vector $\vec{e}_2$ that is orthogonal to $\vec{e}_1$ and not zero, since the system $(\vec{e}_1,\vec{x}_2,\ldots,\vec{x}_n$ is linearly independent.
	
	The vector $\vec{e}_2$ is sought in the form:
	
	The two relations of orthogonality: $\vec{e}_1\circ\vec{e}_e=0$ and $\vec{e}_2\circ\vec{e}_3=0$, enables the calculation of the coefficients $\mu_1$ and $\mu_2$ . We get therefore:
	
	what determines the vector $\vec{e}_3$, orthogonal to $\vec{e}_1$ and $\vec{e}_2$, and not zero, since the $(\vec{e}_1,\vec{e}_2,\vec{x}_3,\ldots,\vec{x}_n)$ are independent. By continuing the same type of calculation, we get step by step a system of orthogonal vectors $(\vec{e}_1,\vec{e}_2,\vec{x}_3,\ldots,\vec{e}_n)$ between them and none of them are zero.
	
	In case where some vectors are such like $\vec{x}_i\circ\vec{x}_i=0$ (their norm is zero), we replace then $\vec{x}_i$ by $\vec{x'}_i+\lambda\vec{x}_j$, choosing a vector $\vec{x}_j$ so that we get $\vec{x'}_i\circ\vec{x'}_i\neq 0$.
	
	We therefore conclude that any pre-Euclidean space admits orthogonal bases!

	This system of calculation of bases is of primary importance! It can be used to study physical systems from a pre-Euclidean repository whose properties change over time. Which is typical in General Relativity.

	\subsubsection{Change of Basis}
	Given two bases $(\vec{e}_1,\ldots,\vec{e}_n)$ and $(\vec{e'}_1,\ldots,\vec{e'}_n)$ of a vector space $\mathcal{E}^n$. Each vector of a base can be decomposed on the other basis as follows (we have already prove it):
	
	A vector $\vec{x}$ of $\mathcal{E}^n$, when its contravariant components known, can be decomposed in each base in the form:
	
	and we have already proved that:
	
	We notice that the transformation relations of the components of a contravariant vare the opposite of those of the basis vectors, the quantities $A$ and $A '$ being permutted, hence also the origin of the name "\NewTerm{contra}"-"\NewTerm{variants}" of these components!

	Let $x_i$ and ${x'}_k$ be the covariant components of a vector $\vec{x}$ respectively in the bases $(\vec{e}_i)$ and $(\vec{e'}_k$. Let us replace the basis vectors expressed by the relations:
	
	in the expression of the definition of the covariant components, therefore we get:
	
	Hence the relation between the covariant components in each base:
	
	We get also:
	
	We notice that the covariant components transform as the basis vectors, hence also the name of these components!
	
	Once again, unless the basis is orthonormal, never forget that the covariant and contravariant components are different!!!
	
	\subsubsection{Reciprocal Basis (Dual Basis)}
	Now let us come back on the concept of dual space but as seen in the vector calculation. This second approach can perhaps help some readers to better understand the concepts seen previously but against hides the underlying reasoning for the origin of the names "covariant" and "contravariant". But it is still the most common presentation used in the literature...
	
	Given a basis ($\vec{e}_i$) of an Euclidean vector space $\mathcal{E}^n$. By definition, $n$ vectors $\vec{e}^k$ which satisfy the following relations:
	
	are named "\NewTerm{reciprocal vectors}\index{reciprocal vectors}" of the vectors $\vec{e}_i$. They will be denoted with higher indices. By definition, each reciprocal vector $\vec{e}^k$ must therefore be orthogonal to all the vectors $\vec{e}_i$, except for $k=i$.
	
	Let us first show that the reciprocal vectors $\vec{e}^k$ of a given base $(\vec{e}_i)$ are linearly independent. For this, we must show that a linear combination $\lambda\vec{e}^k$ gives a zero vector if and only if each coefficient $\lambda_k$ is zero.
	\begin{dem}
	Given $\vec{e}=x^i\vec{e}_i$ any vector of $\mathcal{E}_n$. Let us make a dot product by $\vec{x}$ the previous linear combination $\lambda_k \vec{e}^k$, we get:
	
	The latter equality must be verified whatever the $x^i$, it is therefore necessary that each $\lambda_i$ is zero and thus the vectors $\vec{e}^k$ are indeed linearly independent vectors.
	\begin{flushright}
		$\square$  Q.E.D.
	\end{flushright}
	\end{dem} 
	The system of $n$ reciprocal vectors forms a basis named the "\NewTerm{reciprocal basis}\index{reciprocal basis}" (which is just the dual basis) of the vector space $\mathcal{E}^n$.
	\begin{tcolorbox}[colframe=black,colback=white,sharp corners]
	\textbf{{\Large \ding{45}}Example:}\\\\
	Given three vectors $\vec{e}_1,\vec{e}_2,\vec{e}_3$ forming a basis (not necessarily orthonormal!) of an euclidean space. We decide to denote by:
	
	where, for recall, the symbol $\times$ represents the cross product (\SeeChapter{see section Vector Calculus}) and the whole is mixed product as also seen in the section Vector calculus and represents an oriented volume.\\
	
	The following vectors:
	
	\begin{tcolorbox}[title=Remark,colframe=black,arc=10pt]
	We recognize here the relations we have just proved earlier above during our study of the Gram's determinant!!!
	\end{tcolorbox}
	\end{tcolorbox}
	Now, let us consider a vector on the original base $\vec{e}_1,\vec{e}_2,\vec{e}_3$ that we will denote by (as seen above):
	
	with therefore by definition the contravariant components of the vector that appear as we defined earlier above (and that we had at the same time explained the origin of the name). We also saw above that each contravariant component will also (naturally and by extension) be given by:
	
	Similarly, so we have the covariant components that appear:
	
	In this approach, we then define the contravariant  metric tensor and respectively covariant:
	
	It comes therefore for example for the contravariant components (in the case of a three-dimensional space), knowing that the approach is the same for the covariant components:
	
	And so we find the transformation relations between the  contravariant and covariant components already seen above with the difference that it seems coming out of a hat by successive definitions and that therefore hides the origin of the name of these components (at least in our point of view). But perhaps some readers prefer this approach...???
	
	\begin{tcolorbox}[colframe=black,colback=white,sharp corners]
	\textbf{{\Large \ding{45}}Example:}\\\
	As an example, consider the basis:
	
	Notice that it is not orthogonal because $\vec{e}_1\circ \vec{e}_2=4\neq 0$.

	In this case we have by applying the previous Gram's relations:
	
	As we can see in the figure below where $\vec{e}_1$ and $\vec{e}_2$ are shown in green, and the reciprocal vectors  $\vec{e}^1$ and  $\vec{e}^2$ are shown in blue:\\
	\begin{figure}[H]
		\centering
		\includegraphics[scale=0.75]{img/algebra/reciprocal_basis_contravariant_components.jpg}
		\caption[]{Basis vectors, reciprocal basis vectors and contravariant components}
	\end{figure}
	Notice that by construction we have indeed that $\vec{e}^1$ is orthogonal to $\vec{e}_2$ and $\vec{e}^2$ is orthogonal to $\vec{e}_1$!
	\end{tcolorbox}
	
	\begin{tcolorbox}[colframe=black,colback=white,sharp corners]
	For a given vector $\vec{a}$, say:
	
	we can use the relation proved earlier to find its contravariant components in the basis $(\vec{e}_1,\vec{e}_2,\vec{e}_3)$. We get:
	
	So that (see figure above):
	
	Now observe that the original basis vectors are reciprocal of the reciprocal ones. Thus we can just as well expand the same vector $\vec{a}$ alogn the basis vectors:
	
	with $a_i=\vec{a}\circ\vec{e}_i$.

	The components $a_i$ are the covariant components of $\vec{a}$. In our example, we obtain:
	
	so we have:
	
	\begin{figure}[H]
		\centering
		\includegraphics[scale=0.75]{img/algebra/reciprocal_basis_covariant_components.jpg}
		\caption[]{Basis vectors, reciprocal basis vectors and covariant components}
	\end{figure}
	\end{tcolorbox}
	
	\subsection{Euclidean Tensors (cartesian tensor)}
	The generalization of the concept of vector has led us to the study of vector spaces to $n$ dimensions. Tensors are also one-dimensional vectors but possess additional properties compared to vectors.

	For the theoretical physicist, tensor calculus is interesting primarily in how the components of the tensor are transformed during a change of basis vector spaces from which they come. We will begin to study them vis-à-vis the properties of bases changes (because it is the most interesting case).

	A tensor is in practice often only defined and used in the form of its components. These can be expressed in covariant or contravariant form like any vector. But a new type of components will appear in the tensor, it is the "mixed components". These three types of components are decomposition of Euclidean tensor on different bases.
	
	\textbf{Definition (\#\mydef):} A "\NewTerm{Cartesian tensor}\index{Cartesian tensor}" uses an orthonormal basis to represent a tensor in a Euclidean space in the form of components.

	Use of Cartesian tensors occurs in physics and engineering, such as with the Cauchy stress tensor (\SeeChapter{see section Continuum Mechanics}) and the moment of inertia tensor in rigid body dynamics (\SeeChapter{see section Classical Mechanics}). Sometimes general curvilinear coordinates are convenient, as in high-deformation continuum mechanics, or even necessary, as in General Relativity (\SeeChapter{see section General Relativity }).
	
	\subsubsection{Fundamental Tensor}
	During the theory view earlier above, we used the quantities $g_{ij}$, defined from the scalar product of the basis vectors $(\vec{e}_i)$ of a $n$-dimensional pre-Euclidean  vector space $\mathcal{E}^n$, by:
	
	These $n^2$ quantities are the covariant components of a tensor named the "\NewTerm{fundamental tensor}\index{fundamental tensor}" or as we already know the "\NewTerm{metric tensor}\index{metric tensor}".
	
	Let us study how vary the quantities $g_{ij}$ when we make a basis change:

	Given $({e'}_k)$ another based linked to the previous by the known relation:
	
	Substituting the relation $\vec{e}_i={A'}_i^k\vec{e'}_k$ in the expression of $g_{ij}$, it come (we change the indices as it should be done during a substitution):
	
	In the new base $(\vec{e'}_k)$, the dot products of the basis vectors are therefore quantities such that:
	
	So we finally have for the expression of the covariant components $g_{ij}$ in a basis change:
	
	Identically we have:
	
	In general, a sequence of $n^2$ quantities $t_{ij}$ that transforms, during a base change of $\mathcal{E}^n$, according to the two previous relations, namely:
	
	are, by definition, the "\NewTerm{covariant components of a tensor of order two}" (with two indices) on $\mathcal{E}^n$.

	We can therefore manipulate quantities expressing the intrinsic properties of bases as standard tensors!
	
	\subsubsection{Tensor product (dyadic) of two vectors and matrices}
	Let us consider an Euclidean vector space $\mathcal{E}^n$ of base $(\vec{e}_i)$ and given two vector of $\mathcal{E}^n$:
	
	Let us form the two by two products of contravariant components $x^i$ and $y^j$, namely:
	
	We thus get $n^2$ quantities, if the two vectors have the same number of components, which are also the contravariant components of a tensor of order two named the "\NewTerm{tensor product}\index{tensor product}" of the vector $\vec{x}$ by the vector $\vec{y}$.
	
	For example for $\vec{x}$ of dimension $2$ and $\vec{y}$ of dimension $3$ we have:
	
	We can also tensorally mutliply two matrices $A$ and $B$. Then, the matrix describing the tensor product $A\otimes B$ is the Kronecker product of the two matrices.
	
	For example, if:
	
	Then:
	
	The most famous case is the covariant tensor of rank $2$ in a space of $4$ dimensions as it is the most used one in tensor calculus:
	
	The reader can also now better understand the origin of the name of the gradient of a vector field (giving a "tensor field") as we saw in the section of Vector Calculus because we can rewrite it now:
	
	 The reader will have certainly notice that through the examples above, the tensor product is non-commutative. That is:
	
	We can obviously build tensor products of order three (thus with $n^3$ terms) such as with the following tensor three times  contravariant vectors:
	
	etc.

	Let us study the properties of the basis changes of these components. Let us use for the basis changes relations of contravariant components of a vector, namely:
	
	Let us replace in the relation $u^{ij}=x^iy^j$ the components $x^i$ and $y^i$ by their basis change expression, we get:
	
	The quantities ${u'}^{kl}$ are the new components:
	
	The transformation formula of the $n^2$ quantities $u^{ij}$ on a change of basis change of $\mathcal{E}^n$ is finally (very similar to metric tensor):
	
	Such a change basis relation characterizes the contravariant components of a tensor of order two. Conversely, we get:
	
	So the $n^2$ quantities are the "\NewTerm{contravariant components of a tensor of order two}\index{contravariant components of a tensor of order two}".

	We can the build the same products by pairs for covariant components $x_i$ and $y_i$ of the vectors $\vec{x}$ and $\vec{y}$ thus:
	
	The formulas of basis change of the covariant components of the vectors are given by the following relations that we have already proved previously:
	
	Substituting the first relation in the product $u_{ij}=x_iy_i$, we get:
	
	This is the basis change relation of covariant components of a tensor of order two. We also easily check that we have:
	
	Identically we have of course ${u'}_{kl}={x'}_k{y'}_l$ since $u_{ij}=x_iy_i$.

	So the $n^2$ quantities are then the "\NewTerm{covariant components of a tensor of order two}\index{covariant components of a tensor of order two}".

	Let us now create $n^2$ quantities my multiplying two by two the covariant components of a vector $\vec{x}$ by contravariant components of a vector $\vec{y}$, we get:
	
	Let us perform a basis change in this last relation taking into account the expressions $x_i={A'}_i^k{x'}_k$ and $x^i={A}_k^i {x'}^k$, we get:
	
	This basis change relation characterizes the "\NewTerm{mixed components}\index{mixed components}" of an order two tensor. Conversely, we can verify that we have:
	
	These mixed components also constitutes the components of a tensor product of $\vec{x}$ by $\vec{y}$, according to a given basis.
	
	In general, a sequence of  $n^2$ quantities that transforms, during a basis of $\mathcal{E}^n$, just as previously established relation are therefore, by definition, "\NewTerm{mixed components of a tensor of order two}\index{mixed components of a tensor of order two}".
	
	\pagebreak
	\subsubsection{Tensor Spaces}
	In the previous study, we used as system of $n^2$ number, created from a vector space $\mathcal{E}^n$. When these numbers satisfy some basis change relations, we name these quantities, by definition, "\NewTerm{components of a tensor}\index{components of a tensor}".

	We have seen that any linear combination of these components constitutes the components of a new tensor. We can therefore add together the components of the tensor and multiply by a scalar, to get other components of a new tensor. These addition and multiplication properties mean that we can use these tensors quantities as vector components.
	
	To clarify how we define a tensor on a base, let us study the particular case of a tensor product of two vectors formed by triplets of numbers (that is to say in $\mathbb{R}^3$ typically). Consider therefore the Euclidean vector space $\mathcal{E}^3$ whose vectors are triplets of number of the form $\vec{x}=(x_1,x_2,x_3)$. The canonical orthonormal basis of $\mathcal{E}^3$ consists of three vectors that we know very well but written in tensor calculs as:
	
	with $i=1,2,3$ (nice way to write simple things isn't it...).
	Vectors of $\mathcal{E}^3$ gives the possibility to form the nine quantities that we have named the "\NewTerm{components of the tensor product}\index{components of a tensor product}" of the the vectors $\vec{x}$ and $\vec{y}$.
	
	If we make all possible tensor products between vectors of $\mathcal{E}^3$, we get sequences of $9$ numbers that can be used to define the following vector:
		
	\begin{tcolorbox}[title=Remark,colframe=black,arc=10pt]
	We see immediately with the above relation and the previous relation that the tensor product is therefore not commutative.
	\end{tcolorbox}
	We are left then with the elements of a vector space $\mathcal{E}^9$ with nine-dimensional space, whose elements all combinations by pairs of three numbers.
	
	We then say that $\mathcal{E}^9$ has a "\NewTerm{tensor product structure}\index{tensor product structure}" which is denoted obviously and in standard calculus by $\mathcal{E}^9:\; \mathcal{E}^3\otimes \mathcal{E}^3$ or sometimes $\mathcal{E}_3^{(2)}$.

	These vectors can be decomposed, for example, on an orthonormal canonical basis:
	
	with $k=1,2,\ldots,9$.

	If we rewrite the quantities $x^iy^j$ according to their place in the expression of $\vec{U}$, ie:
	
	with $k=1,2,\ldots,9$ and $i,j=1,2,3$, the vectors $\vec{U}$ are then written:
	
	and is an example as we know of tensor of order $2$ (obviously we can generalize this approach).
	How do these tensor differ from ordinary vectors? Although they are identical to some vectors of $\mathcal{E}^9$ in our example but were formed by the vectors $\vec{x}$ and $\vec{y}$ of $\mathcal{E}^3$. To remember this fact, we write then as we already know:
	
	and they are named as we already know "tensor products of order two" of the vectors $\vec{x}$ and $\vec{y}$. The symbol $\otimes$ is defined in the way we have formed the quantities $x^iy^j=u^{ij}$ and in the order in which they were classified them to form the vector $\vec{U}$.

	To recall the dependence between a quantity $x^iy^j=u^{ij}$ and the basis vector $\vec{e}_i$ to which he is assigned, les us rewrite these vectors by putting in place of the index $k$ the two indices $i$ and $j$, relative to the components, namely:
	
	The latter can be written in the form:
	
	The vectors $\vec{e}_i\otimes\vec{j}$ generates a basis of $\mathcal{E}^9$ in the case of our example with is same the "\NewTerm{tensor associated basis}\index{tensor associated basis}".
	\begin{tcolorbox}[colframe=black,colback=white,sharp corners]
	\textbf{{\Large \ding{45}}Example:}\\\\
	Consider:
	
	We then have for example:
	
	That is to say:
	
	\end{tcolorbox}
	It follows that as element of a space $\mathcal{E}^n\otimes \mathcal{E}^n$, a tensor $\vec{U}$ is a vector of the general form:
	
	Let us study its properties vis-à-vis a base change of $\mathcal{E}^n$ such that:
	
	During such a base change, the base $(\vec{e}_i\otimes\vec{e}_j)$ associated to $\vec{e}_i$  becomes another base $(\vec{e}_k^{'}\otimes\vec{e}_l^{'})$ associated to $\vec{e}_k^{'}$, that is:
	
	It follows that the tensor product $\vec{U}$ has for components in the new basis:
	
	We have the following properties for the tensor product given:
	
	\begin{enumerate}
		\item[P1.] Right/Left distribituvity relatively to the addition of vectors:
		
		The proof of these relation is simple deduce from the definition of the tensor product. Indeed, we have for example:
		
		
		\item[P2.] Associativity with multiplication by a scalar:
		
		Indeed, we have:
		
	
		\item[P3.] When we choose a base in each of the vector spaces $(\vec{e}_i)$ for $\mathcal{E}^n$, $(\vec{f}_i)$ to for $\mathcal{F}^m$, the $n\cdot m$ elements of $G_{nm}$ that we denote by $\vec{e}_i \otimes\vec{f}_i$ also form a basis of $G_{nm}$.
		\begin{dem}
			Already made in the particular example we used earlier above.
		\begin{flushright}
			$\square$  Q.E.D.
		\end{flushright}
		\end{dem}
	\end{enumerate}
	\begin{tcolorbox}[title=Remark,colframe=black,arc=10pt]
	In practice, we often have to use tensor formed from vectors belonging to the same vector spaces $\mathcal{E}^n$.
	\end{tcolorbox}
	We can of course generalize the tensor product to any number of vectors. Gradually, given the property P1, we can consider $p$ vectors $\vec{x}_1,\vec{x}_2,\ldots,\vec{x}_p$ each belonging to different vector spaces $\mathcal{E}^{n_1},\mathcal{E}^{n_2},\ldots,\mathcal{E}^{n_p}$. If we have:
	
	we can form the tensor product:
	
with $i_1=\{1,\ldots,n_1\},i_2=\{1,\ldots,n_2\},\ldots,i_p=\{1,\ldots,n_p\},$.

	We build thus tensor products of oder $p$ belonging to the vector space $\mathcal{E}^{n_1}\otimes\mathcal{E}^{n_2}\otimes\ldots \otimes\mathcal{E}^{n_p}$, space that has a product structure tensor. The elements of this space are by definition tensor of order $p$.

	In order to unify the classification, the elementary vector spaces, which can not be fitted with a tensor product structure can be regarded as having components of a tensor of order $1$. In general, we name these elements "\NewTerm{vectors}\index{vector}", reserving the term "\NewTerm{tensor}\index{tensor}" to elements of tensor spaces of order equal or greater than $2$!
	\begin{tcolorbox}[title=Remark,colframe=black,arc=10pt]
	It is of usage to name "\NewTerm{tensor of order zero}\index{tensor of order zero}" scalar quantities. It is also rare to meet tensor of order tensor greater than $2$.
	\end{tcolorbox}
	It is quite obvious and we will not do the proof  that we absolutely can redefine all the concepts (base, decomposition on a base, reciprocal basis, dot product, tensor product) that we have seen so far considering tensor of order $1$ as a vector (we should therefore rewrite everything that was already written above... which is useless in our point of view).

	It is also quite possible to repeat all these definitions for higher order tensor and thus generalize the concept of space tensor for all dimensions.

	From these considerations, we can state the "\NewTerm{tensoriality criterion}\index{tensoriality criterion}":
	
	So that the elements of a sequence of $n^p$, relative to a base of a vector space $\mathcal{E}_{(p)}^n$, can be considered as the components of a tensor, it is necessary and sufficient that these quantities to be linked together, in two different bases of $\mathcal{E}_{(p)}^n$, by the relations of transformation of the components.
	\begin{tcolorbox}[title=Remark,colframe=black,arc=10pt]
	A vector can be represented in any base by a sequence of $n$ components. However, we can not conclude that any sequence of $n$ numbers is a vector. Indeed, when we put ourselves in another base of space, the components must also change to represent the same object, then we say that the vector is an intrinsic object (whose existence does not depend on the choice of the base). It remains then to know that a vector is a tensor of order $1$.	
	\end{tcolorbox}

	\subsubsection{Linear combination of tensors}
	We can form other tensor by combining together the components of various tensor products defined using vectors of the same vector space. For example, let us consider the contravariant components of the tensor products of the vectors $\vec{x},\vec{y}$ and $\vec{w},\vec{z}$ :
	
	Let us form the following quantities:
	
	The $n^2$ quantities $t^{ij}$ also satisfied the general formulas for basis change. We have indeed by substituting the relations of transformation of the  contravariant components of a tensor product in the previous expression:
	
	The $n^2$ quantities $t^{ij}$, satisfying the relations of basis change also constitutes components of a tensor of order two.

	\subsubsection{Contraction of indices}
	Let us consider the mixed tensor product of two vectors $\vec{x}$ and $\vec{y}$ of respective contravariant $x^i$ and covariant $y_j$ components . The mixed components of the tensor product $\vec{V}$ of these two vectors are:
	
	Let us perform the addition of the various components of the tensor $\vec{V}$ such as $i=j$, ie:
	
	We thus get the expression of the dot product of vectors $\vec{x}$ and $\vec{y}$. The quantity $v$ is a scalar (tensor of order zero). Such an addition on different variance indices constitutes, by definition, the operation of "\NewTerm{contraction of indices}\index{contraction of indices}" of the tensor $\vec{V}$. This operation allowed us to move from a tensor of order two to a tensor of order zero. The tensor $\vec{V}$ has been amputated and of a covariance and a contravariance.
	
	Let us also take the example of a tensor $\vec{U}$ whose mixed components are one time covariant and one two times contravariant $u_k^{ij}$ (caution ... it is not a three-dimensional matrix but simply an indication that the components of this tensor are expressed from three other variables!!!). Let us consider some of its components such as $k=j$, that is the components $u_j^{ij}$ and let us perform the addition of the latter. We then get:
	
	These new quantities $v^i$ form the components of a tensor $\vec{V}$ of order one (thus a vector!) and constitute what we name then "\NewTerm{contracted components}\index{contracted components}" of the tensor $\vec{U}$ and of course meet the basis relations change (on request we can prove it but you have to know that it is similar to the one we made for vectors). So we have indeed change form a tensor of order three to a tensor of order one!
	
	So we can see that the underlying idea of the contraction is to allow us to facilitate the resolution of a purely mathematical problem and depending on the situation it may be good  to raise or reduce the order of a tensor. This is often a choice that is made by trial and error based on a specific context or that naturally arises from a purely mathematical or mathematical-physical development (as we will see examples further below).

	\paragraph{Raising and lowering indices}\mbox{}\\\\
	If we start with a contravariant or covariant components tensor , we can lower / raise one or more indices by multiplication (if repeated) by $g_{ij}$ or $g^{ij}$ (unitary diagonal and positive signature: of canonical type) to get mixed components on which we can then perform contraction operations.
	
	Let us consider an Euclidean tensor $\vec{U}$ of contravariant components $u^{i_1i_2i_3\ldots i_p}$

	If we want to perform a contraction on this tensor, we will first need to transform it into a mixed tensor. This transformation we be done using a fundamental tensor.

	Let us write $\vec{U}$ in mixed components by lowering at the covariant position the index $i_1$ by example (it is therefore equivalent to express this in contravariant component in covariant component). So:
	
	We see well that in this case to take down a contravariant index in a tensor using a fundamental tensor, we must first go search in the covariants components of the fundamental tensor the one who is itself in contravariant position in the original tensor and replace its position (but this time in covariance) by the other index of the fundamental tensor (it is the same idea when it we desire to operate a contraction on a covariant tensor ).
	
	Indeed, let us recall that we have proved that:
	
	Also remember that raising and then lowering the same index (or conversely) are inverse operations, which is reflected in the covariant and contravariant metric tensors being inverse to each other:
	
	Now that we got a mixed tensor components, we can very well getting contract the indices. Let us choose for example the index $i_2$ and let us perform the contraction with the index $j_1$, let us put $i_2=j_1=k$ (we are then concerned only to some specific terms), then just writing the whole process from the beginning:
	
	So we get after lowering the index and one contraction, a tensor of order $p-2$.
	\begin{tcolorbox}[colframe=black,colback=white,sharp corners]
	\textbf{{\Large \ding{45}}Examples:}\\\\
	E1. Let us see a first example of raising down and contracting a the covariant 4-position first order tensor given by (\SeeChapter{see section Special Relativity}):
	
	in components:
	
	(where $x_j$ are the usual Cartesian coordinates) and the Minkowski metric tensor with signature $(-+++)$ given by (\SeeChapter{see section Special Relativity}):
	
	in components:
	In components:
	
	To raise the index, multiply by the tensor and contract:
	
	Then for $\lambda = 0$:
	
	and for $\lambda = j = 1, 2, 3$:
	
	So the index-raised contravariant $4$-position is:
	
	\end{tcolorbox}
	
	\begin{tcolorbox}[colframe=black,colback=white,sharp corners]
	E2. For a second order tensor example let us consider the contravariant electromagnetic tensor in the $(+---)$ signature is given by (\SeeChapter{see section Electrodynamics}):
	
	in components:
	
	To obtain the covariant tensor $F_{\alpha\beta}$, we multiply by the metric tensor and contract:
	
	and since $F^{00} = 0$ and $F^{0i}=-F^{i0}$, this reduces to:
	
	Now for $\alpha = 0, \beta = k = 1, 2, 3$:
	
	and by antisymmetry, for $\alpha = k = 1, 2, 3$, $\beta = 0$:
	
	then finally for $\alpha = k = 1, 2, 3$, $\beta = \ell = 1, 2, 3$:
	
	The (covariant) lower indexed tensor is then:
	
	\end{tcolorbox}
	
	\begin{tcolorbox}[colframe=black,colback=white,sharp corners]
	E3. We will also see further below an example where we contract a tensor of order $1$ (one of the contravariant components of the vectors of the spherical base) having already a lowered index:
	
	\end{tcolorbox}
	\begin{tcolorbox}[title=Remarks,colframe=black,arc=10pt]
	\textbf{R1.} In the relation:
	
	the equality is an abusive notation that we can found in some books (because strictly speaking we should do the calculation in two steps).\\
	
	\textbf{R2.} As a result of the symmetry of the quantities $g_{ij}$ (the dot product is commutative), this latter tensor is identical to what we would get to the position by lowering to the covariant position the index $i_2$, and then by doing the contraction of the index $i_1$ with the index $j_2$.\\

	Let see this:\\

	The symmetry $g_{ij}=g_{ji}$ takes the form (this may seem confusing but let us remember that the number of the component $i$ indicates the place of this component):
	
	Therefore it comes:
	
	and putting $i_1=j_2=k$:
	
	\end{tcolorbox}
	In general, the contraction of a tensor allows to form a tensor of order $p-2$ from a tensor of order $p$. We can of course repeat the operation of contraction. Thus, an even tensor, $2p$, will become a scalar after $p$ contractions and an odd order tensor $2p+1$, will become a vector.

	We can extend after this definition of the contraction of indices, the tensoriality criterion. We have seen until now two ways to recognize the character of a tensor of a sequence of quantities:
	\begin{itemize}
		\item The first is to show that these quantities are formed by the tensor product of component vectors or by a sum of tensor products;

		\item The second is to study how these quantities are converted during a basis change and to check the conformity of the relations of transformation;

		\item The third and new one that brings to put that for a set of $n^{p+1}$ quantities, having $p$ upper and $q$ lower indexes to be a tensor, it is necessary and sufficient that their product fully contracted by the contravariant components of any $p$ vectors and the covariates components of any $q$ vectors, to be a quantity (the norm in fact...) that remains invariant under basis change.
	\end{itemize}
	
	\subsection{Special Tensors}
	We may face in theoretical physics and engineering with tensors that have interesting properties. To avoid redundant work in each case, we will list here and proved the various existing properties used in this book in other sections and discuss their possible implications briefly (the detailed analysis being reserved for their application in the other same sections of the book).
	
	\subsubsection{Symmetric Tensor}
	Consider a tensor $\vec{T}$ of order two of contravariant components $T^{ij}$. Let us suppose that, following a base $(\vec{e}_i)$, all these components satisfy the relations:
	
	On another base $(\vec{e}_j^{'})$, related to the previous by the known transformation relations, the new components of ${T'}^{lm}$ satisfy the relation:
	
	We see that the property $T^{ij}=T^{ji}$ is therefore an intrinsic characteristic of the tensor $\vec{T}$, independent of the base! We then say that the tensor is a "\NewTerm{symmetric tensor}\index{symmetric tensor}" (we will come back again on this concept a little further below) also named "\NewTerm{totally invariant tensor}\index{totally invariant tensor}" (implicitly meaning: by base change).

	The symmetry property is also true for the covariant components of a symmetric tensor since we have:
	
	Conversely, the symmetry of the covariant components implies that of the contravariant components.

	For higher order tensor, the symmetry may be partial, being valid only for two covariant indices or two contravariant indices. Thus, a fourth order tensor, of mixed components $T_l^{ijk}$, may also be symmetrical in $i$ and $j$, for example, given:
	
	We check, as above, that such a property is intrinsic.
	
	A tensor is said to be "\NewTerm{completely symmetrical tensor}\index{completely symmetrical tensor}" if any transposition of two indices with the same variance, changes the corresponding component into itself. For example, for a tensor of order three $T^{ijk}$, completely symmetric, we have the following components that are equal:	
	
	There are many examples of symmetric tensors. Some include:
	\begin{itemize}
		\item the metric tensor $g_{\mu \nu }$ (\SeeChapter{see section General Relativity}) 
		\item the Einstein tensor,$G_{\mu \nu }$ (see further below) 
		\item the Ricci tensor $R_{\mu \nu }$ (see also further below).
		\item the stress and strain tensor for fluids or solids  $\sigma_{ij}$ (\SeeChapter{see section Continuum Mechanics})
		\item  the Lorentz boost tensor $\Gamma_\nu^\mu$ (\SeeChapter{see section Special Relativity}) 
		\item ...
	\end{itemize}
	We can also (very interesting curiosity) obtain a geometric representation of the values of the components of a symmetric tensor of order two!! 
	
	For this let us, consider in the ordinary geometric  space coordinates $x^i$, the following equation:
	
	where, for recall, $x^ix^j$ can be seen as a tensor with $i,j=1,2,3$ and where the $a_{ij}$ are real given coefficients. Let us suppose that coefficients are such that:
	
	The above equation is then:
	
	Here we fall back on the equation of a surface of the second degree of a quadratic similar to of the plan that we saw in the section of Analytical Geometry. We know by extension in to three dimensions that these surfaces are ellipsoids or hyperboloids, depending to the values of the quantities $a_{ij}$.
	
	Let us study how the quantities $a_{ij}$ transform  when we make a change of coordinates such as:
	
	The equation of the quadric is written in this new coordinate system:
	
	Hence the expression of the coefficients in the new system of axes:
	
	The coefficients $a_{ij}$ therefore transform as the covariant components of a tensor of order two. Conversely, if the quantities $a_{ij}$ are the components of a symmetric tensor, these components define the coefficients of a quadric!! There is therefore a certain equivalence between a symmetric tensor and the coefficients of a quadratic...!!! We say then that the equation of the quadric is the "\NewTerm{quadric representation of a symmetric tensor}\index{quadric representation of a symmetric tensor}" or "\NewTerm{representation surface}".
	
	So the representation surface (or representation quadric) is a geometrical representation of a second rank symmetric tensor and is useful for giving us a visual image of the tensor as well as being useful for example in calculating magnitudes of material properties described by second rank symmetric tensors!!
	
	We know from our study of quadrics in the section of Analytical Geometry (by extending it to the three-dimensional case) that we can always find a coordinate system relative to which the equation of a quadratic takes a simpler form:
	
	In this case, the basis vectors are supported by the main axes of the quadric. In this coordinate system, the components of the tensor equation reduce to:
	
	and $a_{ij}=0$ for the other components. The quantity $b_i$ equation are named the "\NewTerm{principal values}\index{principal values}" of the tensor $a_{ij}$.

	If the quantities $b_1,b_2,b_3$ are positive, the surface is an ellipsoid, if two quantities are strictly positive and the third strictly negative, we have a one sheet hyperboloid, if two quantities are strictly negative and the third positive, we have two sheet hyperboloid (\SeeChapter{see section Analytical Geometry}).
	
	Comparing the expression of the quadric obtained previously with the classic equation:
	
	where $a$, $b$, $c$ are the semi-axes of an ellipsoid shows that we have:
	
	Below we can see a screen shot of an interactive tool of Cambridge University to play with the ellipcity (only!) representation quadric of a rank tow symmetric tensor:
	\begin{figure}[H]
		\centering
		\includegraphics[scale=1]{img/algebra/symmetric_tensor_represntation_quadric_cambridge.jpg}
		\caption[]{Visual link between a tensor and its representative surface \\(source: \href{http://www.doitpoms.ac.uk/tlplib/tensors/representation.php}{http://www.doitpoms.ac.uk})}
	\end{figure}
	
	\subsubsection{Antisymmetric Tensor}
	When the contravariant or covariant components of a tensor of order two, satisfy the property:
	
	we the say that the tensor is a "\NewTerm{anti-symmetric tensor}\index{anti-symmetric tensor}".  In other words a tensor is antisymmetric on (or with respect to) an index subset if it alternates sign ($+/-$) when any two indices of the subset are interchanged.
	
	It follows from this definition that an antisymmetric tensor must obviously satisfy the fact that its diagonal components are all zero, such as for example for a rank $2$ tensor:
	
	
	A well know antisymmetric tensor is the electromagnetic tensor $F_{\mu \nu }$ (\SeeChapter{see section Electrodynamics}).
	
	For example a covariant tensor of order three $T_{ijk}$ will be say to be symmetric in $i$ and $k$ if for all values that can take the indices, we have:
	
	Or the fourth order covariant tensor $T_{ijkl}$ will be said in antisymmetric on $i$ and $l$ if for all values that can take the indices, we have:
	 
	
	A tensor will "\NewTerm{totally anti-symmetric}\index{totally anti-symmetric tensor}" if any transposition of index of same variance (covariant/contravariant) changes the corresponding component into its opposite.

	A tensor $T_{ij}$ can be put in the form of a sum of a symmetric tensor and an antisymmetric tensor. Indeed, we have:
	
	The first term of the sum above is a symmetric tensor and the second, an antisymmetric tensor.
	\begin{dem}
	Consider first that $T_{ij}$ is symmetric, then we have:
	
	So this proves that the left term is indeed symmetric for this special case.
	
	Consider secondly that $T_{ij}$ is symmetric, then we have:
	
	So this proves that the left term is indeed antisymmetric for this special case.

	Now if $T_{ij}$ is neither symmetric or antisymmetric we have:
	
	\begin{flushright}
		$\square$  Q.E.D.
	\end{flushright}
	\end{dem}
	A tensor $T_l^{ijk}$ will be "\NewTerm{partially anti-symmetric}\index{partially anti-symmetric tensor} if for example we have:
	
	That is to say anti-symmetric only for a subset of indices.
	Let s now consider two vectors $\vec{x}=x^i\vec{e}_i$ and $\vec{y}=y^j\vec{e}_j$ of a vector space $\mathcal{E}^n$. Let us form the following antisymmetric quantities (we can see in it two tensor products):
	
	where we see immediately that the components $T^{ij}$ are those of an antisymmetric tensor $\vec{T}$ by construction as:
	
	The decomposition of the vector $\vec{T}$ in the base $\vec{e}_i\vec{e}_j$ is:
	
	The tensor $\vec{x}\otimes\vec{y}$ (written as it in analogy to the vectors cross product for $n=3$ ) is named the "\NewTerm{outer product}\index{outer product}" of the vectors $\vec{x}$ and $\vec{y}$. We say that this tensor is a "\NewTerm{bivector}\index{bivector}".
	
	The outer product is an antisymmetric tensor which satisfies the following properties:
	\begin{enumerate}
		\item[P1.] Anticommutativity:
		
		the result is:
		

		\item[P2.] Left and right distributivity for the vector addition:
		

		\item[P3.] Associativity for the multiplication by a scalar:
		

		\item[P4.] Outer products:
		
	\end{enumerate}
	constitute a base for all bivectors.
	\begin{dem}
	An antisymmetric tensor $\vec{T}$ of order two, element of $\mathcal{E}_{(2)}^2$ can, as we have proved it earlier, be written as:
	
	Exchanging, in the last sum of the above relation, the name of the indices and considering that $T^{ij}=-T^{ji}$, we get:
	
	The elements:
	
	are linearly independent vectors since the vectors $\vec{e}_i\otimes\vec{e}_j$ also are it. These elements constitute a base on which the antisymmetric tensor can be decomposed.
	\begin{flushright}
		$\square$  Q.E.D.
	\end{flushright}
	\end{dem}
	The number of distinguishable vectors $\vec{e}_i\otimes\vec{e}_j-\vec{e}_j\otimes\vec{e}_i$ is equal to the number of combinations of vectors taken two by two and distinguishable among $n$ such that (\SeeChapter{see section Probabilities}):
	
	Indeed among the $n^2$ components, $n$ components are equal to zero and $n(n-1)$ other components have opposed values to two by two. So we can consider that half of the latter is sufficient to characterize the tensor.

	In the context of the outer tensor product where we have:
	
	the number of distinguishable components is also equal to $n(n-1)/2$ and they are named "\NewTerm{strict components}\index{strict components}".
	
	We notice that for $n=3$, the strict number of components of the outer product of two vectors is also equal to three. This allows to form, with the components of the bivector, the components of a cross product $\vec{z}$.

	Thus, a cross product therefore exists only for a subspace of bivectors whose number of dimensions is equal to $3$ and the pre-images that are antisymmetric tensors.

	If all these conditions are satisfied, we say that the vector 
$\vec{z}$ is the "\NewTerm{adjoint tensor}\index{adjoint tensor}" of tensor $\vec{T}$.

	\subsubsection{Fundamental Tensor}
	We saw at the beginning of our study of Tensor Calculus the definition of the components of the fundamental covariant tensor $g_{ij}$, that is for recall:
	
	These quantities are involved, as we know it (see previous topics), in the expression of the dot product of two vectors $\vec{x}$ and $\vec{y}$ of contravariant components $x^i$ and $y^i$, given by the relation (for recall):
	
	Let us use the general test of tensoriality to highlight the tensor character of the $g_{ij}$. The previous expression is a product fully contracted of the quantities $g_{ij}$ with the contravariant components  $x^iy^i$ of an arbitrary tensor. As the dot product is an invariant quantity (in this case a scalar) with respect to the base changes, it follows that the $n^2$ quantities $g_{ij}$ are the covariant components of a tensor.

	This tensor is also symmetrical as a result of the symmetry property of the dot product of the basis vectors such that:
	
	We have the same for the contravariant components of the fundamental tensor:
	
	If we denote by $g_j^i$ the mixed components of a fundamental tensor to himself:
	
	obviously with the canonical basis:	
	
	
	\subsection{Curvilinear Coordinates}
	The conventional concepts of coordinate system can be generalized as we know to any specific $n$-dimensional (\SeeChapter{see section Principia}). We name "\NewTerm{coordinate system}\index{coordinate system}" in $\mathcal{E}^n$, any mode of definition of a point $M$ in the considered system.
	
	\textbf{Definitions (\#\mydef):} 
	\begin{enumerate}
		\item[D1.] For a given coordinates system (Cartesian, spherical, cylindrical, polar ...) system we name "\NewTerm{coordinate line}\index{coordinate line}" the "place" of the points $M$ when a only single coordinate of $M$ varies, the other being keep as constant.

		\item[D2.] A "\NewTerm{curvilinear coordinates}\index{curvilinear coordinates}" are a coordinate system for Euclidean space in which the coordinate lines may be curved. These coordinates may be derived from a set of Cartesian coordinates by using a transformation that is locally invertible (a one-to-one map) at each point (\SeeChapter{see section Vector Calculus}).
	\end{enumerate}
	 This means that one can convert a point given in a Cartesian coordinate system to its curvilinear coordinates and back. The purpose as we already know is that depending on the application, a curvilinear coordinate system may be simpler to use than the Cartesian coordinate system. For instance, a physical problem with spherical symmetry defined (for example, motion of particles under the influence of central forces) is usually easier to solve in spherical polar coordinates than in Cartesian coordinates.
	 
	 Well-known examples of curvilinear coordinate systems are as we know in three-dimensional Euclidean space the Cartesian, cylindrical and spherical polar coordinates.
	 
	 Let us first study the generalization of a coordinate system relative to a fixed reference frame (we urge the reader to read first the subsection about Coordinate Systems in the section of Vector Calculus and the subsection of Analytical Mechanis in the section Principia).
	 
	 Let us consider a punctual space $\mathcal{E}^n$ and $(\vec{e}_i)$ a reference frame of that space. Given $x_i$ the rectilinear coordinates of a point $M$ of $\mathcal{E}^n$ relatively to this reference fame. Any curvilinear coordinate system $u^k$ (with $k=1,2,\ldots,n$ is obtained by giving $n$ arbitrary functions $f^i$ of parameters $u^k$, such that:
	 
	
	\begin{figure}[H]
		\centering
		\begin{tikzpicture}[x=(10:4cm),y=(90:4cm),z=(225:4cm),>=Triangle,scale=1.5]
		\coordinate (O) at (0,0,0); 
		\draw [->] (O) -- (1,0,0) node [at end, right] {$x^2$ axis};
		\draw [->] (O) -- (0,1,0) node [at end, above] {$x^3$ axis};
		\draw [->] (O) -- (0,0,1) node [at end, left]  {$x^1$ axis};
		
		\draw [draw=blue, -Circle] (O) to [bend left=8] 
		  coordinate [pos=7/8] (q2n) 
		  (1,-1/4,0) coordinate (q2) node [right] {$u^2$};
		\draw [draw=blue, -Circle] (O) to [bend right=8] 
		  coordinate [pos=7/8] (q3n) 
		  (0,1,1/2) coordinate (q3) node [left] {$u^3$};
		\draw [draw=blue, -Circle] (O) to [bend right=8] 
		  coordinate [pos=7/8] (q1n) 
		  (1/4,0,1) coordinate (q1) node [right] {$u^1$};
		
		\begin{pgfonlayer}{background}
		\begin{scope}
		\clip (O) to [bend left=8] (q2) -- (1,1,0) -- (q3n) to [bend right=8] (O);
		\shade [left color=green, right color=green!15!white, shading angle=135]
		  (O) to [bend left] (q3n) to [bend left=16] (3/4,1/2,0) to [bend left=16] (q2n) -- cycle;
		\end{scope}
		
		\begin{scope}
		\clip (O) to [bend left=8] (q2) -- (1,0,1) -- (q1) to [bend left=8] (O);
		\shade [left color=red, right color=red!15!white, shading angle=45]
		  (O) to [bend right] (q1n) to [bend left=16] (1,0,1) to [bend left=16] 
		  (q2n) to [bend right] (O);
		\end{scope}
		
		\begin{scope}
		\clip (O) to [bend right=8] (q1) -- (0,1,1) -- (q3) to [bend left=8] (O);
		\shade [left color=cyan, right color=cyan!15!white, shading angle=225] 
		  (O) -- (q1n) to [bend right=16] (0,1,1) to [bend left=16] (q3n) 
		to [bend left] (O);
		\end{scope}
		\end{pgfonlayer}
		
		\node at (1/3,1/3,0) {$q_1=\mbox{const}$};
		\node at (0,1/2,1/2) {$q_2=\mbox{const}$};
		\node at (1/2,0,1/3) {$q_3=\mbox{const}$};
		\end{tikzpicture}
		\caption{Coordinate surfaces, coordinate lines, and coordinate axes of general curvilinear coordinates}
	\end{figure}
	We will assume thereafter that these $n$ functions satisfy the following three properties:
	\begin{itemize}
		\item[P1.] They must be of class at least $\mathcal{C}^2$ (differentiable at least twice for the needs of physics: speed and acceleration). This assumption implies, that at any point where it is satisfied, that we have the permutation of derivations (with respect to the two partial derivaties as seen in the section of Differential and Integral Calculus):
		

		\item[P2.] These functions are such that we can solve the system of $n$ equations of coordinate system change relatively to the variables $u^k$ and express them in function of the parameters $x^i$, thus:
		
		still with $k=1,2, \ldots,n$.

		\item[P3.] When the variables $u^k$ vary in a domain $\Delta$, the variables $x^i$ vary in a domain $\Delta'$ (think to cartesian$\leftrightarrow$ spherical coordinates where in cartesian the variable range is infinite when in spherical both of them are limited a typical $2\pi$ width range ). 

		\item[P4.] The Jacobian of the functions $x^i=f^i(u^1,u^2,\ldots,u^n)$, defined by (\SeeChapter{see section Differential and Integral Calculus}):
		
		will be assumed different from zero in the domain $\Delta$ (and also the Jacobian $D(\partial_i u^k)$ of the functions $u^k=g^k(x^1,x^2,\ldots,x^n)$) and is the inverse of the Jacobian of $u^k=g^k(x^1,x^2,\ldots,x^n)$. If tje jacobians exist, they are not zero as a result primarily of the second property above and implicitly the first.
	\end{itemize}
	As we have already mention if, if If we fix $(n-1)$ parameters $u^k$ by varying only one parameter, $u^i$ for example, we get the coordinates $x_{(1)}^i$ of a set of points $M$ that constitute a "coordinate line line". In general, the coordinate lines are not straight but curved as we know. These coordinates $u^k$ are named for this reasons the "curvilinear coordinates". On a point $M$ of $\mathcal{E}^n$ intersect elsewhere $n$ coordinate lines (see figure above).
	
	We prove in the section of Analytical Mechanics, during our study of punctual spaces, that partial derivatives of a vector $\overrightarrow{\text{O}M}$ are independent of the point O (origin) of a given reference frame. If $\mathcal{E}^n$ is assimilated to a system of curvilinear coordinate, we write:
	
	\begin{tcolorbox}[colframe=black,colback=white,sharp corners]
	\textbf{{\Large \ding{45}}Example:}\\\\
	A famous example of curvilinear coordinates, where each $u^k$ is a uniform function of the rectilinear coordinates $x^k$, the being moreover $u^k$  continuous functions at the current point $M$, is that of the spherical coordinates where we have (\SeeChapter{see section Vector Calculus}):
	
	and where for recall:
	
	Let us also recall that during our study of the system of spherical coordinates in the section of Vector Calculus we obtained after normalization of the basis vector:
	
	Therefore:
	
	with:
	
	\end{tcolorbox}
	In a non-Euclidean space, we can not define a unique valid basis over the whole space. Thus, we construct a base at each point separately and for this purpose we use the curvilinear coordinates such that at each point $M$, the base vectors $\vec{e}_k$ are tangent to the corresponding coordinate line equation via the relation given above:
	
	Given now $u^1, u^2, \ldots, u^n$ the curvilinear coordinates of the point $M$ with respect to a Cartesian basis $(\vec{e}_i^0)$. In this reference frame, we obviously have:
	
	where the Cartesian coordinates are functions $x^i=x^i(u^1,u^2,\ldots,u^n)$.
	
	The vector $\vec{e}_k$ therefore also be expressed by:
	
	The reader can check with the example of spherical coordinates (by looking to the explicit version of the corresponding vectors in the section of Vector Calculus) that this relation is correct but at the condition that we work with the non-normalized version of the orthogonal vectors $\vec{e}_r$, $\vec{e}_\theta$, $\vec{e}_\phi$!
	
	From the components $\partial_k x^i$ of the vector $\vec{e}_k$, we can form a determinant $D(\partial_kx^i)$ which is precisely the Jacobian of the of the functions $x^i$ we have defined previously. Since this determinant is different from zero (at least imposed as such), it follows that the $n$ vectors $\vec{e}_k$ (as functions) are linearly independent (we have proved in the section of Differential and Integral Calculus that this Jacobian is in absolute value equal to $r^2\sin(\theta)$ for the spherical coordinates).
	\begin{tcolorbox}[title=Remark,colframe=black,arc=10pt]
	Let us recall that in the section of Differential and Integral Calculus we have prove that the determinant of the Jacobian appears when calculating the surface of a parallelogram in a non-euclidean space. Obviously is the determinant is zero, then the surface is zero as it means that the vector basis of the parallelogram are all collinear (linearly dependents). Hence the fact that if the determinant is not null then some of the vectors (or all) are independent!
	\end{tcolorbox}
	These $n$ vectors, defined by the relation:
	
	are named the "\NewTerm{natural basis}\index{natural basis}" at the point $M$ of the vector space $\mathcal{E}^n$. They are collinear to the tangents of the $n$ coordinated lines which intersect at the point $M$ where they are defined.
	
	We will not insist on the quite obvious fact that at each system of curvilinear coordinates there is an associated natural basis whose base is expressed by these same coordinates (\SeeChapter{see section Vector Calculus}).
	\begin{tcolorbox}[colframe=black,colback=white,sharp corners]
	\textbf{{\Large \ding{45}}Example:}\\\\
	In spherical coordinates, the vectors of the natural basis are those that we obtained in our study of the spherical coordinate system in the section of Vector Calculus and that are orthogonal but not orthonormal.
	\end{tcolorbox}
	
	Let us associate at the point $M$ of $\mathcal{E}^n$ a reference frame formed by the point $M$ and by the vectors of the natural basis\footnote{For recall that natural basis in an Euclidean space is the set of unit vectors pointing in the direction of the axes of a Cartesian coordinate system}. This reference frame is named the "\NewTerm{natural reference frame}\index{natural reference frame}" on $M$ of the coordinate system $u^k$. It will be denoted by:
	
	The differential of the vector $\overrightarrow{OM}$ is then expressed as:
	
	The quantities $\mathrm{d}u^k$ are (obviously) the contravariant components of the vector $\mathrm{d}\vec{M}$ in the natural reference frame $(M,\vec{e}_k)$ of the coordinate system $u^k$.
	
	Let us now consider any two curvilinear coordinate systems $u^i$ and $u^k$ (thinks for example to the spherical and cylindrical coordinate systems), linked between them by the relations:
	
	where the functions $u^i=u^i({u'}^1,{u'}^2,\ldots,{u'}^n)$ are assumed as we already know to be several times continuously differentiable with respect to the ${u'}^k$ and same for the functions ${u'}^k={u'}^k(u^1,u^2,\ldots,u^n)$ with respect the coordinates $u^i$. When we move from one coordinate system to another, we say that we make a "\NewTerm{change of curvilinear coordinates}"\index{change of curvilinear coordinates}.
	
	We saw that in the sections of Differential Geometry and General Relativity that the squared distance $\mathrm{d}s$ between two points $M$ and $M'$ infinitely close was given by the relation:
	
	where the $\mathrm{d}x^i$ are the components of the vector $\mathrm{d}\vec{M}=\overrightarrow{MM}$, assimilated to a fixed reference frame of a punctual space $\mathcal{E}^n$. When this space is assimilated to a system of curvilinear coordinates $u^i$, we have seen that the relation:
	
	Shows that the vector $\mathrm{d}\vec{M}$ has for curvilinear contravariant components the quantities $\mathrm{d}u^i$ with respect to the natural base $(M,\vec{e}_i$. The square of the distance $\mathrm{d}s^2$ is then written in the natural reference frame:
	
	Where the quantities $g_{ij}=\vec{e}_i\circ\vec{e}_j$ are the components of the "fundamental tensor" or of "metric tensor" defined using a natural base. The previous expression is named the  "\NewTerm{linear element of the point space}" $\mathcal{E}^n$ or sometimes the "\NewTerm{metric}" of this space.
	
	The vectors $\vec{e}_i$ of the natural reference frame generally vary from one point to another. This is the case, for example, of the spherical coordinates whose quantities $g_{ij}$ (we show it afterwards with a detailed example) are variable since depending on the parameters $r$, $\theta$ or $\phi$!
	
	A curve $\Gamma$  of $\mathcal{E}^n$ can be defined by the data of the curvilinear coordinates $u^i(\alpha)$ of the locus of the points $M(\alpha)$ as a function of a parameter $\alpha$ (\SeeChapter{see section Differential Geometry}). The elementary distance $\mathrm{d}s$ on this curve $\Gamma$ is then written:
	
	
	\subsubsection{Natural basis in spherical coordinates (curvilinear basis in spherical coordinates)}
	Let us determine the natural basis of the vector space $E^3$ associated with the point space $\mathcal{E}^3$ of ordinary geometry, in spherical coordinates. Let us write the expression of the vectors $\overrightarrow{\text{O}M}$ in a fixed cartesian basis $(\vec{e}_i^{\,0})$ which is by definition (see the section Vector Calculus for more details):
	
	The vectors of the natural base are given by:
	
	Therefore we have:
	
	The derivative of $\overrightarrow{\text{O}M}$ with respect to $\theta$ gives the vector $\vec{e}_2$:
	
	The derivative with respect to $\varphi$ gives the vector $\vec{e}_3$:
	
	These three vectors are orthogonal to each other as we can easily verify it by performing the dot products $\vec{e}_i\circ\vec{e}_j$. When this is the case, we say that the coordinates are "\NewTerm{orthogonal curvilinear coordinates}\index{orthogonal curvilinear coordinates}" (\SeeChapter{see section Differential Geometry}).

	We thus find the same result as in the section of Vector Calculus! These vectors, however, are not all normalized (they're norm is not equal to $1$) since we have:
	
	The natural basis in spherical coordinates is thus formed by vectors that are variable in direction and in modulus at each point of $M$. The quantities $g_{ij}$ constitute an example of a metric tensor attached to each of the points $M$ of the space $\mathcal{E}^3$.
	\begin{figure}[H]
		\centering
		\includegraphics[scale=1]{img/algebra/spherical_natural_basis.jpg}
		\caption{oordinate surfaces, coordinate lines, and coordinate axes of spherical coordinates (source: Wikipedia)}
	\end{figure}
	The linear element of the surface is then given by (the details of the calculations can be found in the section of General Relativity):
	
	
	\subsubsection{Natural basis in polar coordinates (curvilinear basis in polar coordinates)}
	Let us determine the natural basis of the vector space $E^2$ associated with the point space $\mathcal{E}^2$ of ordinary geometry, in polar coordinates. Let us write the expression of the vectors $\overrightarrow{\text{O}M}$ in a fixed cartesian basis $(\vec{e}_i^{\,0})$ which is by definition (see the section Vector Calculus for more details):
	
	The vectors of the natural basis are given by:
	
	We have:
	
	The derivative of $\overrightarrow{\text{O}M}$ with respect to $\phi$ gives the vector $\vec{e}_2$:
	
	These two vectors are orthogonal to each other as we can easily verify by performing the dot products $\vec{e}_1\circ\vec{e}_2$. We thus find the same result as in the section of Vector Calculus.
	
	We then have:
	
	The linear element of the plane is then given by (the details of the calculations can be found in the section of General Relativity):
	
	
	\subsubsection{Natural basis in cylindrical coordinates (cylindrical basis in polar coordinates)}
	Let us determine the natural basis of the vector space $E^3$ associated with the point space $\mathcal{E}^3$ of ordinary geometry, in cylindrical coordinates. Let us write the expression of the vectors $\overrightarrow{\text{O}M}$ in a fixed cartesian basis $(\vec{e}_i^{\,0})$ which is by definition (see the section Vector Calculus for more details):
	
	The vectors of the natural basis are given by:
	
	We have:
	
	The derivative of $\overrightarrow{\text{O}M}$ with respect to $\varphi$ gives the vector $\vec{e}_2$:
	
	and finally:
	
	These three vectors are orthogonal to each other as we can easily verify by performing the dot products $\vec{e}_i\circ\vec{e}_j$. We thus find the same result as in the section of Vector Calculus.
	
	The linear element of the surface is then given by (the details of the calculations can be found in the section of General Relativity):
	
	
	\pagebreak
	\subsection{Christoffel symbols}
	The study of tensor fields constitutes, for the physicist, the essential element of the tensorial analysis. The generic tensor $\vec{U}$ of this field is a function of the point $M$ and we will denote it simply by:
	
If the tensor $\vec{U}$ is a function only of $M$, the field considered is named a "\NewTerm{fixed field}". If $\vec{U}$ is moreover a function of one or more equation parameters other than the coordinates of $M$, then we say that this is a "\NewTerm{variable field}" and we note it:
	
	The different algebraic operations on the tensors $\vec{U}(M)$ associated with a same point $M$ do not generates any particular difficulty. The derivative of $\vec{U}(M)$ with respect to a parameter $\alpha$ leads to the use of the classical results relating to the derivation of the vectors.
	
	However, a difficulty appears when we try to calculate the derivative of a tensor $\vec{U}(M)$ with respect to the curvilinear coordinates. Indeed, the components of the tensor are defined at each point $M$ with respect to a natural coordinate system which varies from one point to another.

	Consequently, the calculation of the elementary variation, named "\NewTerm{elementary transport}":
	
	When we pass from a point $M$ to an infinitely neighboring point $M'$ this can be done in physics only if we use to the same basis. In order to compare the the tensors $\vec{U}(M')$ and $\vec{U}(M)$ each other, we are led to study how a natural coordinate system for a given coordinate system varies when we pass from a point $M$ to an infinitely close point $M'$.
	
	For a system of curvilinear coordinates $u^i$ given a punctual space $\mathcal{E}^n$, a fundamental problem of tensor analysis consists in determining, with respect to the natural reference point $(M,\vec{e}_k)$ at the point $M$, the natural reference point $(M',\vec{e}_k^{'})$ at the infinitely close point $M'$. We then say that we are looking for an "\NewTerm{affine connection}\index{affine connection}".
	
	On the one hand, the point $M'$ will be perfectly defined with respect to $M$ if we determine the vector $\mathrm{d}\vec{M}$ such as $\overrightarrow{MM'}=\mathrm{d}\vec{M}$. For curvilinear coordinates $u^k$, the decomposition of an elementary vector $\mathrm{d}\vec{M}$ is given by the relation that we have previously proved:
	
	the quantities $\mathrm{d}u^k$ being for recall the contravariant components of the vector $\mathrm{d}\vec{M}$ on the natural basis $(\vec{e}_k)$.
	
	And now to make things physically comparable, we must guarantee that vectors of the both basis are also parallel! So the idea is that the derivative we are looking for allows one to transport vectors of a manifold (surface) along curves so that they stay parallel with respect to the connection (derivative). Such an idea is named in physics "\NewTerm{parallel transport}\index{parallel transport}".

	Therefore, the idea is to determine the vector $\vec{e}_k^{'}$ with elementary variations $\mathrm{d}\vec{e}_k$ of the vector $\vec{e}_k$ relatively to the natural reference frame $(M,\vec{e}_k)$, when we go from $M$ to $M'$. We then have:
	
	The computation of the vectors $\mathrm{d}\vec{e}_k$ then remains the main problem to solve. We will first consider an example of this type of calculation in spherical coordinates for pedagogical purposes as it can help to understand what will follows.

For this, let us return the expression of the vectors $\vec{e}_k$ of the natural base in spherical coordinates, that is:
	
	Since the basis vectors $\vec{i}$,$\vec{j}$,$\vec{k}$ of the fixed Cartesian reference being constant in modulus and in direction, the differential of the vector $\vec{e}_1$ is written:
	
	We notice that the terms in parentheses represent respectively the vectors $\vec{e}_2/r$ and $\vec{e}_3/r$, hence:
	
	We also compute, by differentiating the vector $\vec{e}_2$:
	
	with:
	
	we have:
	
	So finally:
	
	And:
	
	After a few elementary and very tricky algebraic operations (...), we get:
	
	The differentials $\mathrm{d}\vec{e}_k$ are thus decomposed on the natural basis $(\vec{e}_k)$. If we denote by $\omega_i^k$ the contravariant components of the vector $\mathrm{d}\vec{e}_1$, that latter is written:
	
	The components $\omega_i^k$ of the vector $\mathrm{d}\vec{e}_i$ are differential forms (linear combinations of differentials). We have, for example:
	
	If we denote by in a general way by $u^i$ the spherical parameters, we have:
	
	The coordinate differentials are then denoted:
	
	and the components $\omega_i^j$ are then written in a general way:
	
	Where the quantities $\Gamma_{ki}^j$ are functions of $r$,$\theta$,$\phi$ that will be explicitly obtained by identifying each component $\omega_i^j$. For example, the component $\omega_3^3$ is written with the notation of the previous relation:
	
	Identifying the coefficients of the differentials, it comes:
	
	By doing the same with the $9$ components $\omega_i^j$, we get the $27$ (...) terms for which the detailed calculations for the $3\cdot 9=27$ are given much further below in the text. For any curvilinear coordinate system, these quantities $\Gamma_{ki}^j$ are named "\NewTerm{Christoffel symbols of the second kind}\index{Christoffel symbols of the second kind}" or "\NewTerm{Euclidean functions of affine connection}\index{Euclidean functions of affine connection}".
	
	Thus, for a punctual space $\mathcal{E}^3$ and a system of any curvilinear coordinates $u^j$, the differential $\mathrm{}\vec{e}_i=\omega_i^k\vec{e}_k$ of the vectors $\vec{e}_i$ of the natural basis is written on this basis
	
	We have just seen, on the example of the spherical coordinates, that a direct calculation makes it possible, by identification, to obtain explicitly the quantities $\Gamma_{ki}^j$. We shall see that we can also obtain the expression of these quantities as a function of the components of $g_{ij}$.
	
	The calculation of the quantities $\Gamma_{ki}^j$ as a function of the $g_{ij}$ will lead us to introduce other Christoffel symbols. For this, let us write the covariant components, denoted $\omega_{ji}$, of the differentials $\mathrm{d}\vec{e}_i$, thus (it's kind a definition):
	
	\begin{tcolorbox}[colframe=black,colback=white,sharp corners]
	\textbf{{\Large \ding{45}}Example:}\\\\
	With our example above:
	and:
	
	We get (most dot products are orthogonal vectors, therefore they vanish):
	
	\end{tcolorbox}
	The covariant components are also a linear combinations of the  differentials $\mathrm{d}u^k$ that we can write in the form:
	
	The quantities $\Gamma_{kji}$ are named the "\NewTerm{Christoffel symbols of the first kind}".

	We see quite clearly by going through the definitions and examples of the Christoffel symbols above again that:
\begin{enumerate}
	\item For the Christoffel symbols of the second kind, they are symmetrical with respect to their lower indices and therefore if the metric is symmetric, we have:
	
	
	\item For the Christoffel symbols of the first kind, they are also symmetrical with respect to their extremal indices and therefore if the metric is symmetric, we have:
	
	\end{enumerate}
	Indeed (following the request of a reader), since we have:
	
	it then comes:
	
	and by swapping the indices $i$ and$ j$:
	
	The term-to-term identification of the development on a concrete case of the last two relations will give (necessarily) the equality:
	
	that we wanted to prove.

	Since the covariant components are related to the contravariant components by the relations (contraction of indices):
	
	We get the expression linking the Christoffel symbols of each kind:
	
	Conversely:
	
	\begin{tcolorbox}[title=Remark,colframe=black,arc=10pt]
	Various notations are used to represent the Christoffel symbols. The most common are:
	\begin{itemize}
		\item Christoffel symbol of the first kind:
		

		\item Christoffel symbol of the second kind:
		
	\end{itemize}
	\end{tcolorbox}
	Let us consider now a punctual space $\mathcal{E}^n$ and given a linear element $\mathrm{d}s^2$ of this space:
	
	Starting from:
	
	we get by differentiation:
	
	By injecting in it the expression of the differential $\mathrm{d}\vec{e}_j=\omega_j^l\vec{e}_l$ this gives us:
	
	where the term represents $\omega_j^l$ represents the mixed component of the vector $\mathrm{d}\vec{e}_j$. We can make this component covariant by multiplying it by the metric tensor $g_{il}$ so as to form an $\omega_{ij}$ quantity which can in turn be expressed by means of the Christoffel symbols as follows as we already know:
	
	Substituting the relation $\Gamma_{kji}=g_{jl}\Gamma_{ki}^l$ in the preceding expression (the indices used in this relation are not those of the expression in question, but mutatis mutandis it is equivalent), we then get:
	
	The differential $\mathrm{d}g_{ij}$ is then written:
	
	On the other hand, the differential of the function $g_{ij}$ is then written:
	
	hence by identifying the coefficients of the differentials $\mathrm{d}u^k$ in these two last expressions (much further below in this section there is a detailed example of all the relations which will follow with several coordinate systems):
	
	Relation that the reader can (if he doubt about its veracity) check with the detailed practical examples which are given much further below.

	As we have (in the case of a symmetric metric):
	
	where it is strongly recommended to the reader to remember that the permutation of the indices respecting this last relation generally only  works on the extreme indices!

	We can therefore write the prior-previous relation as:
	
	and by by performing a circular permutation on the indices (hence it is not a permutation of the extreme indices!), we get:
	
	By making the sum:
	
	and by subtracting:
	
	Simplifying it comes:
	
	therefore:
	
	It is the expression of the Christoffel symbols of the first kind as a function of the partial derivatives of the components $g_{ij}$ of the fundamental tensor named "\NewTerm{first Christoffel identity}\index{first Christoffel identity}". We thus understand why in a locally inertial frame (of the Minkowski type) the Christoffel symbols are all zero (given that the metric is constant).
	
	We get those of the second kind from the following relation (by definition) named "\NewTerm{fundamental theorem of Riemannian geometry}\index{fundamental theorem of Riemannian geometry}" or "\NewTerm{Levi-Civita connection}\index{Levi-Civita connection}" or "\NewTerm{first Christoffel identity}\index{first Christoffel identity}":
	
	The last two expressions listed above allow the actual calculation of the Christoffel symbols for a given metric (hence an enormous gain in calculations). When the quantities $g_{ij}$ are given a priori, we can study the properties of the punctual space defined by the data of this metric, which is the case of the Riemann spaces we will see later.
	
	\pagebreak
	\begin{tcolorbox}[colframe=black,colback=white,sharp corners]
	\textbf{{\Large \ding{45}}Example:}\\\\
	We propose to calculate the Christoffel symbols of the second kind $\Gamma_{kj}^i$ corresponding to the polar coordinate system in the plane  (it will be sufficiently long...) that we will write this time (at the opposite of the section of Vector Calculus) in index notation:
	
	We will calculate the Christoffel symbols of the second kind from our last relation:
	
	Let us determine the components of the metric. By the way, they are the same as those we had calculated for the cylindrical coordinates above with the obvious difference that $g_{33}$ does not exist. Therefore, we have:
	
	Let us then calculate the $g_{ji}$. In this example it is rather trivial, it is enough to apply the relation demonstrated at the beginning of this section:
	
	Or by dealing with as with standard matrices (\SeeChapter{see section Linear Algebra}):
	
	We then have immediately:
	
	Now let us develop the Christoffel symbols writing for these coordinates:
	\end{tcolorbox}
	
	
	\begin{tcolorbox}[colframe=black,colback=white,sharp corners]
	
	Hence due to the properties of symmetry:
	
	In the same way:
	
	\end{tcolorbox}
	
	
	\begin{tcolorbox}[colframe=black,colback=white,sharp corners]
	To sum up:
	
	\end{tcolorbox}
	
	\subsection{Ricci Theorem}
	\begin{tcolorbox}[colback=red!5,borderline={1mm}{2mm}{red!5},arc=0mm,boxrule=0pt]
	\bcbombe Before reading what follows ... We want to remind the reader that the writing of this section is not finished (as most chapters in the book)! Thus, we still have to illustrate the abstract notions that will follow with concrete practical examples but that are example outside of the special case of General Relativity!
	\end{tcolorbox}
	This being said, we have seen in the section of General Relativity that geodesics are the shortest distances between two points in any type of space. What will interest us now is to study the variations of a vector during such a displacement. Let us first remind that the geodesics equation\index{geodesic equation} for any curvilinear coordinate system $y^i$ of the punctual $\mathcal{E}^n$ (\SeeChapter{see section Principles}) is given by (\SeeChapter{see section General Relativity}):
	
	Let us consider now a vector $v$ of $\mathcal{E}^n$ of covariant components $v_i$ and let us form the dot product of the vector $\vec{v}$ and $\vec{n}=\mathrm{d}y/\mathrm{d}s$ (the latter vector, denoted here abusively with the indices, gives the components tangent to the geodesic on which the first vector flows), we then have the following quantity:
	
	When moving along the geodesic from a point $M$ to an infinitely near point $M'$, the scalar undergoes the variation:
	
	and as:
	
	Hence:
	
	Let us replace in this last expression, on the one hand, the differential of $\mathrm{d}v_K$ by its exact total differential that we rewrite a bit:
	
	and on the other hand, the second derivative $\mathrm{d}^2y^i/\mathrm{d}s$ by its expression taken from the equation of geodesics. We are getting:
	
	which can also be written:
	
	Where we have put:
	
	which are by definition the absolute differentials of the covariant components of the vector $\vec{v}$. We also define the "\NewTerm{covariant derivative}\index{covariant derivative}" (also named "\NewTerm{connection}\index{connection}" or "\NewTerm{affine connection}\index{affine connection}") by the relation:
	
	\begin{tcolorbox}[title=Remark,colframe=black,arc=10pt]
	In ancient or American textbooks this is often written in the form (which we will not use in this book):
	
	making use of the "$;$" to denote the covariant derivative and the "$,$" for the partial differential.
	\end{tcolorbox}
	Since the derivative of the product of two functions is the sum of the partial derivatives, then we also have:
	
	If we put $t_br_c\equiv \nabla_j v_i$ then we have (result which we will use after having proved the Ricci theorem to further determine the Einstein tensor necessary in the section of General Relativity):
	
	In curvilinear coordinates, in order for the differential of a vector to be a vector, the two vectors of which we take the difference must be in the same point of space. In other words, one of the two infinitely close vectors must be transported in one way or another to the point where the second is located, and only after making the difference between the two vectors which are now in one and only one point of the same punctual space. The "\NewTerm{parallel transport}\index{parallel transport}" operation must be defined in such a way that in cartesian coordinates (for the small example...), the difference of the components coincides with the ordinary difference $\mathrm{d}v_k$. 
	
	Thus, we have indeed in Cartesian coordinates:
	
	since in this system: $\Gamma_{kj}^i=0$.
	
	Thus, in curvilinear coordinates, the difference of the components of the two vectors after the transport of one of them to the point where the other is is denoted $\delta v_k$ such that we have:
	
	This brings us by identification to write:
	
	But also to write the principle of least action (variational principle) in the tensorial form:
	
	Let us consider now a tensor of order two, product of two tensors of order one such that (as we have seen in our study of tensor compositions):
	
	Therefore:
	
	hence (we take out the last two equalities just for aesthetics!):
	
	A parallel transport is therefore an operation that takes a tangent vector and moves it along a path in space without turning it (relative to the space) or changing its length. In flat space we can say that the transported vector is parallel to the original vector at every point along the path. In curved space we cannot say such a thing. Let's use the spherical surface of Earth to show this! We start at the equator at longitude $0^\circ$ holding an arrow pointing north:
	\begin{figure}[H]
		\centering
		\includegraphics[scale=1]{img/algebra/parallel_curvature.jpg}
		\caption{Parallel transport illustration on a sphere}
	\end{figure}
	We go along longitude $0^\circ$ up to the North Pole, keeping our arrow parallel to the ground and pointing forward all the time. When we get to the North Pole our arrow is pointing in the direction of longitude $180^\circ$. Now we go south along longitude $90^\circ$ east keeping our arrow perpendicular to our path as it was at the North Pole. When we get to the equator our arrow is pointing to the east. Finally, we go west along the equator until we get back to our starting point. We keep the arrow pointing backwards all the way. Though we did not turn the arrow all the way, we are now at the starting point with our arrow turned $90^\circ$ relative to its original position. We surely can't say that it is now parallel to the original position. This means that the term "direction" cannot be defined globally in a curved space. We can only compare the direction of two vectors if they are at the same point.

	The fact that parallel transport along a closed loop changes the direction of a vector in a curved space but not in a flat space may lead to the idea of using it as a way to measure curvature. It turns out that if we choose a loop that is small enough around a point in a curved space, the amount of change in the direction of a vector that is parallel transported along it is proportional to the area enclosed by the loop. So, the ratio between the area of the loop and the amount of change in the direction of the vector (whatever way we chose to measure it) can be used as a measure to the curvature of the surface that includes the loop. Actually we define curvature by the value of this ratio.
	
	The previous relation leads us to be able to write the metric in its variational form named "\NewTerm{Ricci identity}\index{Ricci identity}":
	
	But we also have since $g_{ij}=\vec{e}_i\circ\vec{e}_j$:
	
	hence the identity:
	
	With the both relations:
	
	and the absolute differential (which is simply generalized for a tensor of order two):
	
	we get:
	
	Now, let us recall that we have by definition:
	
	Hence finally:
	
	The absolute differential on a geodesic in the approximation of an infinitesimal transport of the fundamental tensor is therefore (as we could expect) zero. This is the "\NewTerm{Ricci theorem}\index{Ricci theorem}". Some theoretical physicists then say that \textit{the covariant derivative kills the metric} in the sense that the metric does not change on a space differential.

	Finally, we also see that for a tensor of order two (the metric in particular) we have:
	
	We can therefore write the absolute differential which in this particular case is zero:
	
	And therefore the "\NewTerm{covariant derivative}\index{covariant derivative}" of the metric is null:
	
	\begin{tcolorbox}[title=Remark,colframe=black,arc=10pt]
	The reader will have to remember for the definition of the Einstein tensor that:
	
	and that this is another way of expressing that an infinitesimal variation on a geodesic according to the principle of least action kills the metric. We will therefore work from now on (as before) with nonlinear differential equations that must be integrated to find the behavior of matter in a given space.
	\end{tcolorbox}
	Let us now determine an expression which will be very useful in General Relativity when we will determine the Einstein field equations (another way of expressing that the covariant derivative of the metric is null):

	Let us perform the contracted multiplication of:
	
	by $g^{ij}$, it then comes by using the relation $g^{ij}g_{jl}=\delta_i^l$ (which we had proved much earlier above) that:
	
	hence the relation:
	
	The quantities $\Gamma_{jh}^j$ and $\Gamma_{ih}^h$ representing the same sums, we then have:
	
	\begin{theorem}
	Let us consider now the determinant $g$ of the quantities $g_{ij}$. The derivation of the determinant gives us:
	
	\end{theorem}
	\begin{dem}
	Given any variable that we choose here as being the time $t$ only to simplify the notations of the calculations that will follow. When the main part of the development is completed, the result can be adapted to any other variable! We will write for what will follow $g_j$ the elements of the $j$-th column of $g_{ij}$.

	For the following developments, we define the notations:
	
	The rule of derivation of a functional determinant is given for recall (\SeeChapter{see section Linear Algebra}) by:
	
	By considering the first determinant, by using the minors (\SeeChapter{see section Linear Algebra}) for the development of its first column:
	
	For the $j$-th determinant, it comes:
	
	Thus:
	
	Now, we have demonstrated much earlier that the metric tensor is its own inverse. Therefore:
	
	This allows us to write:
	
	and therefore:
	
	what is also written (following the conventions defined at the beginning of the proof):
	
	where the reader must therefore be careful not to read badly, typically thinking that the derivative $\mathrm{d}_t$ in the right-hand term derives everything ($g_{ij}g^{ij}$)... while it only derives $g_{ij}$.
	
	However, we can adopt another variable. Let $h$ be this other variable:
	
	Either by rearranging:
	
	This is what we wanted to prove.
	\begin{flushright}
		$\square$  Q.E.D.
	\end{flushright}
	\end{dem}
	Now by combining:
	
	Demonstrated earlier above and the result we have just proved:
	
	it comes:
	
	Therefore we have:
	
	Let us prove that it is possible to derive this last relation from the following important equality:
	
	Indeed:
	
	This relationship does not mean much until we will make a more explicit use of it in our study of General Relativity (\SeeChapter{see section General Relativity}).

	Let us now determine the second covariant derivative of the metric tensor. Let us remember before we go further (because it is important) that we had obtained:
	
	
	\subsection{Riemann-Christoffel symbols}\begin{tcolorbox}[colback=red!5,borderline={1mm}{2mm}{red!5},arc=0mm,boxrule=0pt]
	\bcbombe Before reading what follows ... We want to remind the reader \underline{again} that the writing of this section is not finished (as most chapters in the book)! Thus, we still have to illustrate the abstract notions that will follow with concrete practical examples but that are example outside of the special case of General Relativity!
	\end{tcolorbox}
	
	Let us recall that we have demonstrated earlier above that:
	
	This relation means, exception of an an interpretation error from the writer of those linear as sometimes interpreting Tensor Calculus result is a pain in the a.., that the covariant derivative of a tensor of order two - such as the metric - on a geodesic path in two perpendicular directions (the second covariant derivative making it possible to the "perpendicular geodesic" between the two geodesics infinitely close to the first covariant derivative). We know already that such a shift is "parallel transport".
	
	By substituting in it:
	
	We then have:
	
	Let us now switch the indices $j$ and $k$ in the previous expression to have a differential with respect to another path:
	
	Assuming that the components satisfy the classical properties $\partial_{kj}v_i=\partial_{jk}v_i$, we get by subtraction of the two previous expressions:
	
	And since we have proved that in the case of a symmetric metric we have:
	
	We have therefore:
	
	\begin{tcolorbox}[title=Remark,colframe=black,arc=10pt]
	The fact of having in the case of a symmetric metric $\Gamma_{jk}^r=\Gamma_{kj}^r$ and which vanishes in the preceding relation, leads many practitioners to define what we name the "\NewTerm{torsion tensor}\index{torsion tensor}" or "\NewTerm{torsor}" (but in reality it is a particular case of a more general relation which belongs to the domain of differential geometry). Thus, we define the torsion tensor as:
	
	and in the case of a symmetric metric (Euclidean space), the torsion is zero by extension as we have already seen it! In fact, the Einstein field equations which we shall prove later implicitly implies a symmetric metric with zero torsion. However, it is possible to proved that a non-symmetric tensor can always be decomposed into a symmetric and non-symmetric tensor (this is trivial because it is like separating a complete matrix in the sum of a matrix having a diagonal of non-zero component that are and another matrix having the diagonal zero).
	\end{tcolorbox}
	It then remains:
	
	Since parallel transport takes place on infinitely close geodesic paths, we take the limit:
	
	Which mainly underlies the fact that the velocity field is almost equal in two infinitely close parallel points.

	It then remains:
	
	This relation expresses the fact that, like gravity, the curvature of space-time causes a mutual acceleration between the geodesics! Moreover, it is easy to see that the mutual acceleration between the geodesics is zero if the Riemann-Christoffel tensors are null (typically in Cartesian coordinates, in extenso it means for a flat time-space). This is exactly what we expect of gravity: if we do not observe any acceleration, the curvature (we shall now define what it is) is zero and if the curvature is zero, we observe no acceleration. Morale of the story: the gravity is curvature and the curvature is gravity !!

	We see that the quantity in parentheses is a tensor of order four that we will write on this site as following (because there are several traditions in the way to write it...):
	
	and which summarizes itsel the parallel transport and the fact that gravity and geometry of space are linked together. Obviously, if the metric is of Minkowski type (or tends to a metric of Minkowski under certain conditions), then $R_{i,jk}^l$ is zero! Very rare authors write this last equality in the (unhappy ....) form:
	
	The tensor $R_{i,jk}^l$ is named the "\NewTerm{Riemann-Christoffel tensor}\index{Riemann-Christoffel tensor}" or "\NewTerm{Riemannian space tensor}\index{Riemannian space tensor}". The curvature of a Riemannian space can also be characterized using this tensor.

	If we multiply the tensor $R_{i,rs}^k$ by $g_{jk}$, then we have the covariant components of this tensor such that:
	
	and given the following relations that we proved earlier above:
	
	Therefore we get:
	
	and let us replace the quantities $g_{jk}\partial_r\Gamma_{is}^k$ by $\partial_r\left(g_{jk}\Gamma_{is}^k\right)-\Gamma_{is}^k\partial_r g_{jk}$. We then get:
	
	We also proved earlier before that:
	
	Hence:
	
	and as:
	
	we get:
	
	and we also proved that:
	
	Hence:
	
	And by reporting them into the prior-previous relation, we get:
	
	Finally, we get for the covariant expression of the Riemann-Christoffel tensor:
	
	It should be noticed that the Riemann-Christoffel tensor is therefore antisymmetric:
	
	and that in the parenthesis of the prior-previous relation we have only double partial derivatives, while outside of the parenthesis the Christoffel symbols contain only simple partial derivatives!

	Finally, the permutation of the indices $ij$ and $rs$ as a block gives us as a consequence of the symmetry of the $g_{ij}$ and by inverting their derivation order:
	
	Let us now perform a circular permutation on the indices $j$, $r$, $s$ in the expression (obtained just above)
	
	then we get:
	
	and we get (it is very simple to control by summing the three lines above):
	
	The previous identity is named the "\NewTerm{first Bianchi identity}\index{first Bianchi identity}" or also "\NewTerm{Bianchi algebraic identity}\index{Bianchi algebraic identity}" and it highlights the cyclicity property of the  Riemann-Christoffeltensor. In reality, we should not use the word "identity" since it is verified only (at least to my knowledge) in the case of a symmetric metric tensor (otherwise, the torsion is not zero for recall!).

	The reader will observe that it is immediate that this last relation is satisfied in the case of the Minkowski metric, since if at all points the partial derivative of the metric is zero we have:
	
	And we will see in the sectin of General Relativity that this first identity will serve as a basis for the construction of the Schwarzschild metric.

	If the metric is of the Minkowski type (we change the notations of the indices to be more in conformity with the usual writings in general relativity) then it is immediate that we also have:
	
	But in the case where the metric is not of the Minkowski type, this latter relation can be satisfied and has an interest only if and only if the chosen metric is decomposable in Taylor series whose first partial derivatives are zero at $0$ (see the section of Differential Geometry for such Taylor series!).

	This relation is valid only in the case of a "\NewTerm{locally inertial frame (LIF)}\index{locally inertial frame}" in which all Christoffel symbols cancel each other but not their derivatives.

	By extension:
	
	Let us recall that implicitly, this relation, named "\NewTerm{Bianchi second identity}\index{}" or "\NewTerm{Bianchi differential identity}\index{Bianchi differential identity}", always expresses simply (if one may say ...) the fact that gravity and geometry of space are linked together.

	Following a reader request let us detail much more how to derivate this identity!

	The identity is easiest to derive at the origin of a locally inertial frame (LIF) as already mention, where the first derivatives of the metric tensor, and thus the Christoffel symbols, are all zero. At this point, we have
	
	If the Christoffel symbols are all zero, then the covariant derivative becomes the ordinary derivative
	
	Therefore, we get, at the origin of a LIF:
	
	By cyclically permuting the index of the derivative with the last two indices of the tensor, we get
	
	By adding up all linear with the covariant derivative and using the commutativity of partial derivatives, we see that the terms cancel in pairs, so we get
	
	As usual we can use the argument that since we can set up a LIF with its origin at any non-singular point in spacetime, this equation is true everywhere and since the covariant derivative is a tensor, this is a tensor equation and is thus valid in all coordinate systems.
	
	
	\subsection{Ricci curvature (Ricci tensor)}
	Before we can see the consequences of second Bianchi's identity, we need to define the "\NewTerm{Ricci tensor}\index{Ricci tensor}":
	
	which is therefore simply the contraction of the first and third indices of the Riemann-Christoffel tensor which we have given above:
	
	in other words it is just a more condensed notation ... and then the letters for the upper or lower indices as well as the presence of the comma are at the free choice of the writer (according to the mood and especially if the context makes it possible to avoid any confusion).

	For example with the Riemann-Christoffel tensor we have just given, the Ricci tensor could be written in the following two ways (we keep the indices with latin letters):
	
	The Ricci tensor can be taken as the trace of the Riemann tensor, hence it is of lower rank, and has fewer components. If you have a small geodesic ball in free fall, then (ignoring shear and vorticity) the Ricci tensor tells you the rate at which the volume of that ball begins to change, whereas the Riemann tensor contains information not only about its volume, but also about its shape. 
	
	Other contractions of other indices may also be possible but because $R_{\alpha\beta,\mu\nu}$ is antisymmetric on $\alpha,\beta$ and $\mu,\nu$ then the contraction on these indices are equivalent to have $\pm R_{\alpha\beta}$.
	\begin{tcolorbox}[title=Remark,colframe=black,arc=10pt]
	There is no widely accepted convention for the sign of the Riemann
curvature tensor, or the Ricci tensor, so check the sign conventions of whatever book you are reading!
	\end{tcolorbox}
	Similarly, we define the "\NewTerm{Ricci scalar}\index{Ricci scalar}" (also sometimes named "\NewTerm{Riemann scalar}\index{Riemann scalar}") by the relation:
	
	which has the following properties:
	\begin{itemize}
		\item If the space is flat, the Ricci scalar is zero
	
		\item If space is curved like a sphere, the Ricci scalar is positive
	
		\item If the space is curved like a horse saddle, the Ricci scalar is negative
	\end{itemize}
	Either explicitly (by changing the notation for the indices in order to insist on the fact that it has no impact!):
	
	The Ricci scalar is the trace of the Ricci tensor, and it is a measure of scalar curvature. It can be taken as a way to quantify how the volume of a small geodesic ball (or alternatively its surface area) is different from that of a reference ball in flat space.

	We will have concrete practical examples in the section of General Relativity for the first two cases, but let us look at simplified examples for the first two cases (we will not, however, prove the reciprocal).
	
	\begin{tcolorbox}[colframe=black,colback=white,sharp corners]
	\textbf{{\Large \ding{45}}Example:}\\\\
	E1. Let us start with the metric of flat space (without the temporal component). We have (\SeeChapter{see section General Relativity}):
	
	By taking the definition of Ricci's scalar in explicit form:
	
	It is immediate that R is zero since the partial derivatives will all be zero. So a flat space has a null Ricci scalar.\\
	
	E2. Let us now look at the metric of the plane expressed in spherical coordinates (without the temporal component). We have (\SeeChapter{see section General Relativity}):
	
	and:
	
	with:
	
	We know that to compute the Ricci scalar (or Ricci curvature), we must therefore compute the contraction of the Riemann-Christoffel tensor (that is, the Ricci tensor) which itself depends on the Christoffel symbols of the second kind which themselves depend on the symbols of Christoffel of the first kind (argh!).

	We shall therefore begin with the lowest level, that is to say by determining all the Christoffel symbols of the first kind given for recall by:
	
	
	\end{tcolorbox}
	\begin{tcolorbox}[colframe=black,colback=white,sharp corners]
	So we have an $3^3$, that is $27$ possible Christoffel symbols of the first kind! Even if some symbols are equal (we have proved it!), We will still calculate everything.\\

	Let's start with joy and good humor ...:
	
	\end{tcolorbox}
	
	\begin{tcolorbox}[colframe=black,colback=white,sharp corners]
	
	Let us now calculate all the symbols of Christoffel symbols of the second in the details:
	
	Again, as the metric tensor is diagonal, this will simplify the calculations!\\

	We have then:
	
	\end{tcolorbox}
	\begin{tcolorbox}[colframe=black,colback=white,sharp corners]
	Let us now calculatethe $9$ components of the Ricci tensor in details according to:
	
	We then have (we calculate them all, even if we know that subsequently those which do not have $\alpha=\beta$ will be useless by the fact that the metric is diagonal):
	
	\end{tcolorbox}
	
	\begin{tcolorbox}[colframe=black,colback=white,sharp corners]
	
	\end{tcolorbox}
	
	\begin{tcolorbox}[colframe=black,colback=white,sharp corners]
	
	\end{tcolorbox}
	
	\begin{tcolorbox}[colframe=black,colback=white,sharp corners]
	
	\end{tcolorbox}
	
	\begin{tcolorbox}[colframe=black,colback=white,sharp corners]
	
	Let us now calculate the Ricci scalar:
	
	we then have:
	
	The Ricci scalar is therefore also zero. This result may be surprising, but in reality it is logical since we have only calculated the scalar curvature of a flat space expressed in spherical coordinates.
	\end{tcolorbox}
	
	\begin{tcolorbox}[colframe=black,colback=white,sharp corners]
	E3. Let us now impose ourselves the diagonal metric of the 2-sphere surface $\mathcal{S}^2$ (without the temporal component). We have then in accordance with what we have seen in the section of Differential Geometry:
	
	and:
	
	Therefore (\SeeChapter{see section Differential geometry}):
	
	often written as:
	
	where $r$ is a constant!

	We shall therefore begin with the lowest level, that is, by determining all the Christoffel symbols of of the first kind given for recall by:
	
	We have therefore $2^3$, that is, $8$ possible Christoffel symbols of the first kind! Even if some symbols are equal (we have demonstrated!), we will still calculate everything:
	
	\end{tcolorbox}
	
	\begin{tcolorbox}[colframe=black,colback=white,sharp corners]
	
	Let us now calculate all the Christoffel symbols of the second kind in the details:
	
	Again, as the metric tensor is diagonal, this will simplify the calculations!\\

	We have then:
	
	Let us now calculate the $4$ components of the Ricci tensor in the details according to:
	
	We then have (we calculate them all, even if we know that subsequently those which do not have $\alpha=\beta$ will be useless by the fact that the metric is diagonal):
	
	\end{tcolorbox}
	
	
	\begin{tcolorbox}[colframe=black,colback=white,sharp corners]
	
	Let us now calculate the Ricci scalar :
	
	We then have:
	
	We notice that:
	\begin{enumerate}
		\item The Ricci scalar is a constant. This means that the hypersurface has a constant curvature at all points of the surface (we know that the sphere by symmetry has a constant curvature at all points). It thus possesses a form of symmetry, with respect to its curvature. We are then dealing with a "\NewTerm{maximally symmetrical variety}".

		\item This scalar is positive which describes a domed space (ball, sphere)
	\end{enumerate}
	\end{tcolorbox}
	\begin{tcolorbox}[title=Remark,colframe=black,arc=10pt]
	Be careful not to confuse the value of the Ricci curvature of and that of the Gauss curvature.
	\end{tcolorbox}
	
	\subsection{Einstein Tensor}
	Let us apply a contraction to the second Bianchi  identity (valid for recall with the "$+$" only if the metric is positive):
	
	Let us recall that $\nabla_\lambda g_{\alpha\mu}=0$ and similarly by extension that $\Delta_\lambda g^{\alpha\mu}=0$. So finally this leads us to write by the property of the derivatives (product in sum):
	
	and therefore to get:
	
	 Using the antisymmetry property of the Riemann-Christoffel tensor we write:
	
	What ultimately brings us to write from the definition of Ricci's tensor:
	
	This last relationship being named "\NewTerm{contracted Bianchi identity}".

	Let us contract this relation once more:
	 
	That which is identical to write using the properties of the Einstein summation (which allows to freely change the indices):
	
	Which is equivalent to:
	
	As $\nabla_\lambda=g_{\lambda}^\mu \nabla_\mu R$, we have:
	
	By raising the index $\lambda$ by multiplication with $g^{\nu\mu}$, we get the "\NewTerm{Einstein's identity}\index{Einstein's identity}":
	
	The "\NewTerm{Einstein tensor}\index{Enstein tensor}" (tensor of order two and contravariant in the present case) which is therefore a constant in a given Riemannian space is therefore defined by:
	
	And expresses in the shortest possible way, parallel transport under all assumptions seen so far.

	Identically, we can obtain the covariant form:
	
	The tensor is therefore constructed for a Riemannian metric only (which nevertheless makes a lot of possible spaces ...), and is automatically non-divergent:
	
	It must be remembered, however, that a large part of the latest developments consider a symmetrical metric. This is why some speak of "\NewTerm{symmetrical gravitational theory}" when we deal with General Relativity.

	We shall find this tensor naturally in the sectionof General Relativity when, by making use of the variational principle, we decompose the action into two terms:
	\begin{itemize}
		\item the action of mass in the gravitational field

		\item the action of the gravitational field in the absence of mass
	\end{itemize}
	By expressing the whole in a Riemannian space we will then get the no less famous Einstein field equations (without further explanations in this section):
	
	the details being given in the section of General Relativity.
	\begin{tcolorbox}[title=Remark,colframe=black,arc=10pt]
	As we see, we can very well add a constant term to the expression of Einstein's tensor, without changing the nullity of its divergence. This fact, used in Astrophysics, makes it possible to construct models of particular Universes that we will deal with in the section of Cosmology.
	\end{tcolorbox}
	\begin{tcolorbox}[colframe=black,colback=white,sharp corners]
	\textbf{{\Large \ding{45}}Example:}\\\\
	Let us calculate the order $2$ covariant Einstein tensor:
	
	based on the diagonal metricsurface of the 2-sphere $\mathcal{S}^2$ (without the temporal component):
	
	Since the metric is diagonal, we have proved earlier above and in detail by the example that:
	
	And as in the present case, we also have:
	
	It comes:
	
	So we have to focus only on two components:
	
	which confirms what we have said earlier above.
	\end{tcolorbox}
	The complexity of the Einstein Tensor expression can be shown using the formula for the Ricci tensor in terms of Christoffel symbols:
	
	where ${\displaystyle \delta _{\beta }^{\alpha }}$ is the Kronecker tensor and the Christoffel symbol $\Gamma ^{\alpha }{}_{\beta \gamma }$ is given for recall by:
	
	Before cancellations, this formula results in $2\times (6+6+9+9)=60$ individual terms. Cancellations bring this number down somewhat.
	
	\begin{flushright}
	\begin{tabular}{l c}
	\circled{95} & \pbox{20cm}{\score{4}{5} \\ {\tiny 40 votes,  82.5\%}} 
	\end{tabular} 
	\end{flushright}
		
	%to make section start on odd page
	\newpage
	\thispagestyle{empty}
	\mbox{}
	\section{Spinor Calculus}
	\lettrine[lines=4]{\color{BrickRed}A}s we will see first in relativistic quantum physics, spinors play a major role in quantum theory, and therefore in all contemporary physics (quantum field theory, standard model, string theory, etc.). The actual purpose of this section on the spinors is only to give the tools to the reader that are necessary to a deep understanding of what will be done in the chapter about Atomistic.
	
	It was starting 1927 that the physicists Pauli, and after Dirac introduced spinors  for the representation of the wave functions (see section Relativistic Quantum Physics). However, in their mathematical form, spinors were discovered by Elie Cartan in 1913 during his research on the representations of the groups following the general theory of Clifford spaces (introduced by the mathematician W.K. Clifford in 1876). He showed, as we will see it, that in spinors provide in fact a linear representation of the group of rotations of a space with any number of dimensions. Thus, spinors are closely related to geometry but they are often introduce in an abstract way without intuitive geometric meaning. Thus, we will try (as always on this book) in this section to introduce this tool in the most simple and intuitive possible way with a maximum of details.
	
	The spinor formalism is not interest only of interest for quantum physics and its related developments, among others, Roger Penrose showed that the spinor theory was an extremely fruitful approach to the theory of General Relativity. Even if the most commonly used tool for the treatment of General Relativity is the tensor calculus, Penrose seems to have shown that in the specific case of the four-dimensional space and in the metric Lorentz the formalism of two components spinors was more appropriate.
	
	The theory of spinors named, "\NewTerm{spinor calculus}\index{spinor calculus}" or sometimes "\NewTerm{spin geometry}\index{spin geometry}" is extremely broad but as we know this book aims to address the physicists and engineers therefore we will limit ourselves to spinors properties useful in quantum physics (at least actually).
	
	\begin{tcolorbox}[title=Remark,colframe=black,arc=10pt]
We strongly recommend the reader to have previously read the subsubsection on quaternions (\SeeChapter{see section Numbers}), the subsubsection on rotations in space (\SeeChapter{see section Euclidean Geometry}) and finally, if possible, for a physical practice example, the section on Relativistic Quantum Physics.
	\end{tcolorbox}
	
	\pagebreak
	\subsection{Unit Spinor}
	
	We will give here a first simplified and special definition (or example) of spinors. Thus, we will show that it is possible from such a tool to represent a vector equation of a space $e^3$ of three components using a two-component spinor. The method is extremely simple and the reader who has already read the part of the section on Quantum Wave Physics dealing with the Dirac equation and the section on Quantum Computing will see a rather beautiful analogy.
	
	Consider to start the sphere of radius of the following equation (\SeeChapter{see section Analytic Geometry}):
	
	And consider the following figure:
\begin{figure}[H]
\centering
\includegraphics[scale=0.75]{img/algebra/spinor_unit_sphere.jpg}
\caption[]{Unit sphere}
\end{figure}
Let us put on this sphere of center on O and unit radius a point $P$ of coordinates $(x,y,z)$  and denote by $N$ (north) and $S$ (south) the points of the sphere intersecting with the $Z$ axis.

The point $S$ will have by convention for coordinates:
	
We obtain a projection so-named "\NewTerm{stereographic projection} \index{stereographic projection}" $P'$ of the point $P$ by tracing the straight line $SP$ that passes through the complex equatorial plane $x\text{O}y$  (yes! we chose a $\mathbb{C}$ as plane!) at point $P'$ of coordinates $(x', y', z')$.

The similar triangles $SP'\text{O}$ and $SPQ$ (with $Q$ being the orthogonal projection onto the $Z$-axis of the point $P$) give us the following relations by simply applying Thales' theorem (\SeeChapter{see section Euclidean Geometry}):
	
	
	\begin{tcolorbox}[title=Remark,colframe=black,arc=10pt]
The last two relation (ratios) are obtained by simply applying Thales' theorem (see section Euclidean Geometry) in the complex equatorial plane.
	\end{tcolorbox}
	
	Let us write now to simplify the notations:
	
	We have from the prior-previous relation that:
	
	Taking the squared modulus (see the study of complex numbers in the section on Numbers):
	
	and from  the equation of the sphere it follows:
	
	we finally get:
	
	Let us write now the complex number $\xi$ under the form:
	
	
	 where $\phi,\psi$ are two complex numbers that we can always impose verify the following condition of unitarity (nothing prohibits to do that but for theoretical physics purpose this choice suit us well...):
	
	\begin{tcolorbox}[title=Remark,colframe=black,arc=10pt]
		The following complex numbers satisfy for example (hazard!!??)  the above condition:
		
	\end{tcolorbox}	
	Remember before continuing that we have proved in our study of complex numbers (\SeeChapter{see section Numbers}) that:
	
	Therefore it comes by injecting the last two relations in the equation given above:
	
	So we get:
	
	Rearranged:
	
	Therefore:
	
	Finally we have a simple expression for the $z$ coordinate of the point $P$ because in the last equality above you must remember that $\psi\bar{\psi}+\phi\bar{\phi}=1$ then:
	
	As we have:
	
	Then by summing and respectively by substracting the two above relations and using previous results we get:
	
	To resume for the point $P$ we have with our choices:
	
	Thus, for any point $P$ on the sphere of radius unity, we can match a pair of complex numbers satisfying the imposed unitary identity!
	
	\pagebreak
	Therefore in complete and explicit form we finally have using what we know about complex numbers (\SeeChapter{see section Numbers}) and remarkable trigonometric functions (\SeeChapter{see section Trigonometry}):
	
	We notice easily that the norm of this vector is equal to $1$.
	
	This last relation also indicates us that $2\alpha$ is the angle between $\text{O}z$ and $\overrightarrow{OP}$ (since the hypotenuse of vector's angle has a unit norm) and therefore by deduction $\gamma-\beta$ represents the angle between $\text{O}x$ and the plane $(\text{O}z, \overrightarrow{OP})$:
	
	\begin{figure}[H]
		\centering
		\includegraphics{img/algebra/spinor_simple_rotation.jpg}
		\caption[]{Representation of the rotation}
	\end{figure}

	The pair of complex numbers:

	

	is by definition a "\NewTerm{unitary spinor}\index{unitary spinor}" (it contains also all the information about $z$). Thus, as we have seen, a unitary spinor can also be expressed in the form:
	
	As well any spinor can be written in the more general form:
	
	The spin is essentially measured from the oriented $z$-axis as we have seen it yet with the previous figure.
	
	The stereographic projection led us to represent certain vectors of the Euclidean space $\mathcal{E}^3$ with the elements of a complex two-dimensional vector space that is the "\NewTerm{space of spinors}\index{space of spinors}".
	
	\begin{tcolorbox}[title=Remark,colframe=black,arc=10pt]
This representation is not unique because the arguments of complex number are (in trigonometric form) determined at a given offset constant!
	\end{tcolorbox}	
	
	The reader who has already read a little bit the section on Wave Quantum Physics  will probably noticed the strange (not innocent) similarity of the following identity and relations:
	
	compared to the de Broglie normalization condition (the integral over the entire space of the sum of the products of the complex wave functions and its conjugate is equal to the unit) and to the developments determining the continuity equation also in the section of Wave Quantum Physics .
	
	Let us now see that for future needs, we can find two new vectors $\vec{v}_1,\vec{v}_2$ of Euclidean space $\mathcal{E}^3$ associated with a unitary spinor $(\psi,\phi) $ determined on the unit sphere. These vectors will have to be orthogonal to each other and with unit norm, each orthogonal to the vector $\overrightarrow{OP}$.	
	
	To simplify the notations let us write  $\vec{v}_3=\overrightarrow{OP}$ and $\vec{v}_i=(x_i,y_i,z_i),i\in \left\lbrace 1, 2 ,3\right\rbrace$.
	
	The $\vec{v}_1,\vec{v}_2,\vec{v}_3$ vectors are of course bounded by the cross product (\SeeChapter{see section Vector Calculus}):
	
	hence taking into account the expression of $\overrightarrow{OP}$ components based on its associated spinor, and the fact that $\psi\bar{\psi}+\phi\bar{\phi}=1$, we obtain:
	
	Writing the orthogonality of vectors we get them obviously six additional equations. However the orientation of vectors $\vec{v}_1,\vec{v}_2$ being not fixed, there is some uncertainty in the values of their components. Let us select values such that:
	
	Taking the complex conjugate quantities of previous relation and summing, to have only real parts we have to write:
	
	We can control the norm is equal to the unit. Just check with the squared norm:
	
	In the same way we get:
	
	We can easily check that these values restore well the relations of the vector cross product of $\vec{v}_2$. At any unitary spinor $(\psi,\phi)$ we can therefore associate three vectors $\vec{v}_1,\vec{v}_2,\vec{v}_3$. We can directly check that the vectors thus calculated are mutually orthogonal and with unit norm.
	
	A reader has make us the request to show in much details as possible this affirmation for $x_x$. So let's go:
	 
	
	\subsection{Geometric Properties}
	We will study the transformations of vectors associated to a spinor to derive the corresponding properties of spinor transformation. As we know some special (trivial) rotations in space can always be expressed as the product of two plane symmetries, therefore we begin by studying these latter.
	
	\subsubsection{Plane Symmetries}
	Let us consider first the plan symmetry of a vector:
	
	During a symmetry relative to a plane $P$, any vector $\overrightarrow{OM}$ is transformed into a vector $\overrightarrow{OM'}$. Let us determine a matrix $S$ representing this symmetry with respect to this plane! 
	
	Given $\overrightarrow{\text{O}A}$ a unit vector normal to the plane $\mathcal{P}$ and $H$  the root of the perpendicular projection from any given point $M$ of space on the plane $\mathcal{P}$:
	
	\begin{figure}[H]
		\centering
		\includegraphics{img/algebra/plane_symmetry.jpg}
		\caption[]{Plane symmetry of a vector relatively to a plane}
	\end{figure}
	Let $M'$ be the symmetric point $M$ with respect to the plane $\mathcal{P}$, we have:
	
	Given $a_1,a_2,a_3$ the cartesian components of $\overrightarrow{\text{O}A}$ and $(x,y,z),(x',y',z')$ the respective components of the vectors $\overrightarrow{OM},\overrightarrow{OM'}$, the above equation gives us the linear relations:
	
	The matrix $S$ that take us from vector $\vec{X}(x,y,z)$ to the vector $\vec{X}'(x,y,z)$ has therefore the following expression:
	
	We keep in mind this result and let us now consider two vectors $(\vec{X}_1,\vec{X}_2)$ orthogonal to each other and unitary, defining as we have above  seen a unitary spinor $(\psi,\phi)$ (we used the notation $\vec{v}_1,\vec{v}_2$ before). A symmetry with respect to a plane $\mathcal{P}$ transforms the vectors $\vec{X}_1,\vec{X}_2$ into vectors $\vec{X'}_1,\vec{X'}_2$ which are associated the spinor $(\psi',\phi')$. 
	
	\begin{theorem}
	We will now show that the following transformation of the spinor $(\psi,\phi)$ into spinor $(\psi',\phi')=(x'_3,y'_3,z'_3)$ is:
	
	\end{theorem}
	and transforms the vectors $\vec{X}_1,\vec{X}_2$ into vectors $\vec{X}_1^{'},\vec{X}_2^{'}$, these vectors being deduced respectively - as we will just show it - from each other by a single plane symmetry and the matrix $\mathcal{A}$ represents well the sought transformation.
	
	The previous relation gives us therefore:
	
	In all we have so far the previous set of relations and:
	
	Therefore we can deduce:
	
	After, using the fact that $||\vec{A}||=a_1^2+a_2^2+a_3^2=1$, we get:
	
	So we fall well back on the symmetry matrix:
	
	Thus, the matrix that we will again in the section of Relativistic Quantum Physics:
	
	$\mathcal{A}$ therefore generates the transformation of a spinor $(\psi,\phi)$ into a spinor $(\psi',\phi')$ such that the associated vectors $(\vec{X}_1,\vec{X}_2)$ can be deduced respectively from $(\vec{X}_1^{'},\vec{X}_2^{'})$  by a planar symmetry.
	
	\subsubsection{Rotations}
	As we have saw it in the section of Euclidean Geometry, it is possible to rotate a vector in the plane or in space using matrices. Similarly, by extension, it is clear that the multiplication of two rotations is a rotation (that is the elementary linear algebra - at least we consider it as is).
	
	Consider therefore two planes $P, Q$ whose intersection generates a line $L$ and let us denote $\vec{A}(a_1,a_2,a_3)$ and $\vec{B}(b_1,b_2,b_2)$ the unit vectors carried by the respective normal vectors (\SeeChapter{see section Analytical Geometry}) to these two intersecting planes in $L$:
	\begin{figure}[H]
		\centering
		\includegraphics{img/algebra/spinor_intersecting_planes.jpg}
		\caption[]{Illustrated intersection of two planes}
	\end{figure}
	Let us denote by $\theta/2$ the angle of the vectors $\vec{A},\vec{B}$ between them (the reason for this notation comes from our study of quaternions (\SeeChapter{see section Numbers})). Given $\vec{L}$ the unit direction vector carried by the line $L$ resulting from the intersection of planes $P, Q$ and such that:
	
	Explanations: $\vec{A},\vec{B}$ are unitary but not necessarily perpendicular and we still need to ensure that $\vec{L}$ is a unit vector (the norm equal to unity!). Therefore, the above relation ensures that:
	
	The previous vector product gives us for the components of $\vec{L}$:
	
	On the other hand, the scalar product can be written:	
	
	\begin{tcolorbox}[title=Remark,colframe=black,arc=10pt]
	We will use these two planes as symmetry planes for our rotations.
	\end{tcolorbox}	
	As we have noticed previously, a rotation in $\mathcal{E}^3$ can always be done with more than two plane symmetries. Thus, a rotation can be denoted by the application (multiplication) of two matrices of symmetry according to the results obtained previously:
	
	Developing the product of these two matrices and taking into account the relations arising from vector and dot product we get:
	
	Thus, we can write the transformation of a spinor $(\psi,\phi)$ and a spinor $(\psi',\phi')$ with a matrix of the form:
	
	whose parameters are named "\NewTerm{Cayley-Klein parameters}\index{Cayley-Klein parameters}".
	
	The matrix $R\left(\vec{L},\frac{\theta}{2}\right)$ can be written in another form if we do a limited development for infinitely small rotations $\theta/2=\varepsilon/2$ (that's where the physics comes back...):
	
	Using only the first order terms, the rotation matrix is finally written:
	
	This matrix is the limited development of the matrix of rotation in the neighborhood of the identity matrix, the latter obviously corresponding to the zero rotation. We note also the latter in the form:
	
	where the matrix $\sigma_0$ is the identity matrix of order two and $\chi(\vec{L})$ is named the "\NewTerm{infinitesimal rotation matrix}\index{infinitesimal rotation matrix}". Now, if we put $L_1=1,L_2=L_3=0$ in $\chi(\vec{L})$ we get:
	
	How to interpret this result? Well it's quite simple, choose $L_1=1,L_2=L_3=0$ gives us a collinear vector $\vec{L}$ to the axis $\text{O}x$. Therefore, we can very well imagine the planes generating the axis $\text{O}x$ that carries $\vec{L}$. As $\varepsilon/2$ (verbatim $\theta/2$) is generated by the vectors $\vec{A},\vec{B}$ perpendicular to $\vec{L}$ and thus to $\text{O}x$, then the angle $\varepsilon/2$ (or its variation) represents a variation of the direction of the normal planes to $\vec{A},\vec{B}$ which by symmetry are used to construct the rotation (recall that $\vec{A},\vec{B}$are not necessarily mutually orthogonal). So by extension, having  $L_1=1,L_2=L_3=0$ allows us then only to make rotations (symmetries) around the $x$-axis.
	
	Similarly, a rotation about the $y$-axis corresponds to $L_2=1,L_1=L_3=0$, which gives:
	
	and the same with $L_3=1,L_1=L_2=0$ we finally have:
	
	The three matrices:
	
	are rotation matrices in the space of "\NewTerm{two-dimensional spinors}\index{two-dimensional spinors}". Physicists and mathematicians say that these matrices are an irreducible representation of dimension two of the group "$\NewTerm{\text{SU} (2)}$" or named  "\NewTerm{special group of spatial rotations $\text{SU}(2)$}\index{special group of spatial rotations}" (\SeeChapter{see section of Set Algebra}).
	
	The previous infinitesimal matrices therefore show us in a skillful way the following matrices:
	
	These matrices are named "\NewTerm{Pauli matrices}\index{Pauli matrices}" and we will find them again in the section of Relativistic Quantum Physics as part of the study of the Dirac equation and the determination of its explicit solutions (using spinors).
	
	Using Pauli matrices, the infinitesimal rotations matrix can finally be written:
	
	Let us define a vector $\vec{\sigma}$, named "\NewTerm{Pauli vector}\index{Pauli vector}", whose components are the Pauli matrices:
	
	The expression $L_1\sigma_1+L_2\sigma_2+L_3\sigma_3$ can be written as a sort of dot product which represents a sum of matrices (the arrow above the sigma is sometimes omitted if no confusion is possible):
	
	The limited development is then written:
	
	The rotations matrix:
	
	can using Pauli matrices be written in the remarkable form under the assumption of small angles:
	
	Expression that we will a lot in the section of Quantum Computing to express the $R$ matrices explicitly and also in the section of Set Algebra.
	
	What is written sometimes:
	
	Which can also be written in an extensive form:
	
	which has the form of a quaternion of angle $\theta$ (don't remember that this is only for small angles!) and of axis $\vec{L}$. Hence the reason that  have from the beginning chosen the notation $\varepsilon/2$.
	
	It is clear, so that the analogy with quaternions to be stronger, that the $2\times 2$ Pauli matrices are a set of four linearly independent matrices! As the canonical basis for quaternions!
	
	If we denote by $\vec{L}(L_1,L_2,L_3)=\vec{X}(x,y,z)$  then the "\NewTerm{spinor product}\index{spinor product}" is finally defined by:
	
	This matrix constitutes as we have already mentioned, to the limited development of the rotation matrix in the neighborhood of the identity matrix, the components of $\vec{x}$ being associated with a spinor whose rotation is through the double symmetry defined by two planes whose intersection is defined by the vector $\vec{X}$.
	
	We can also notice the interesting consequence that a rotation of $2\pi$ ($360^\circ$) rotation does not restore the object to its original position!!!
	
	Indeed:
	
	Therefore we need a rotation of $4\pi$ ($720^\circ$) to make a full turn! This corresponds to the spin of ${}^1{\mskip -5mu/\mskip -3mu}_2$. It takes two turns to find that the object reappears equivalently (this is counter intuitive and can make thing we are dealing with object having higher dimensions that what we exepect!). We then say that the representation of rotations is "bivaluated".
	
	Schematically this can be represented as:
	\begin{figure}[H]
		\centering
		\includegraphics{img/algebra/full_rotation.jpg}
		\caption{Spinor full rotation (special example)}
	\end{figure}
	Or you can consider the analogy that is to hold a tea cup on the palm of your hand and you turn your hand by maintaining it flat to regain its (the hand, not the cup!) original position. You will see that your hand has to do two laps!
	
	\pagebreak
	\subsubsection{Properties of Pauli Matrices}
	The reader can easily check (if this is not the case he can always contact us and we will write the details) the following main properties of the Pauli matrices, some of which will be used in the section of Relativistic Quantum Physics:
	\begin{enumerate}
		\item[P1.] Unitarity:
		

		\item[P2.] Anticommutativity:
		
		or $i\neq j$ and $i,j=1,2,3$.

		The last tow properties gives us:
		
		with $i,j=1,2,3$.

		\item[P3.] Cyclicity:
		

		\item[P4.] Commutativity:
		
		
		\item[P5.] Vector product:
		
		Given the square of the different $\sigma$ by noting abusively by "$1$" the unitary matrix (we change the indices to give you the habit to use other common notations):
		
		Leading us to write that (square norm of the Pauli vector):
		
		Let us consider now the following products:
		
		Let us consider now the following products:
		
		All these relation can be summarized into a unique one (!):
		
		where for recall (\SeeChapter{see section Tensor Calculus}) the Kronecker symbol is defined by:
		
		and the antisymmetric symbol by:
		In three dimensions, the Levi-Civita symbol is defined as follows:
		
		i.e.  $\varepsilon_{ijk}$  is $1$ if $(i, j, k)$ is an even permutation of $(1,2,3)$ or in the natural order $(1,2,3)$, $-1$ if it is an odd permutation, and $0$ if any index is repeated. In three dimensions only, the cyclic permutations of $(1,2,3)$ are all even permutations, similarly the anticyclic permutations are all odd permutations. This means in 3D it is sufficient to take cyclic or anticyclic permutations of $(1,2,3)$ and easily obtain all the even or odd permutations.
		
		We also have:
		
		We fall back here on the components of the vector product:
		
		Now let us develop an important spinor identity which will be useful to us in the section of Relativistic Quantum Physics:
		
		But we also have:
		
		So finally:	
		
	
		\item[P6.] We note that these matrices are also hermitian (let us recall that a Hermitian matrix is a transposed matrix followed by its complex conjugate according to what we saw in the section of Linear Algebra) such that:
		
		It is therefore in the language of quantum physics: Hermitian operators!!!
	\end{enumerate}
	Let us now see what are the eigenvectors and eigenvalues of the Pauli matrices because this result is very useful for the section of Relativistic Quantum Physics and of Quantum Computing!

	Let us recall that when a transformation (application of a matrix) act on a vector, it changes the direction of the vector except for specific matrices that have eigenvalues. In this case, the direction is conserved but not their length. This property is exploited in quantum mechanics (and not only as we will see in many other sections of this book).
	
	Let us determine in a first time, the associated eigenvectors and eigenvalues (\SeeChapter{see sectionLinear Algebra}) using the most common method:
	
	The eigenvalues equation (\SeeChapter{see section Linear Algebra}) is thus written:
	
	Which gives us as characteristic equation:
	
	hence the eigenvalues $\lambda=\pm 1$. Which gives us the possibility to determine the eigenvectors as following:
	
	Therefore for $\lambda=1$:
	
	This impose that $y=x$. The eigenvector is therefore:
	
	whatever is the value of $x$.
	
	Conclusion: The proper direction of the vector is conserved but not its length (norm) because it depends on the value of $x$.
	
	For $\lambda=-1$:
	
	This impose that $y=-x$ and therefore that the eigenvector is  equal to:
	
	The previous eigenvectors written with the Dirac formalism (\SeeChapter{see section of Relativistic Quantum Physics}) give for $\lambda=1$:
	
	with a norm ($1$ since we formalize to the unit):
	
	\begin{tcolorbox}[title=Remark,colframe=black,arc=10pt]
	In the formalism of Dirac $\langle v |$ is the is named a "Bra" and $|v \rangle$ a "Ket".
	\end{tcolorbox}
	This being only valid for components that are real numbers. The normalized eigenvector has therefore for expression:
	
	For $\lambda=-1$, we have:
	
	and:
	
	and the normalized eigenvector has thus for expression:
	
	Let us now determine the eigenvectors and eigenvalues associated to $\sigma_y$ by following the same procedure:
	
	So we have for the eigenvalues:
	
	The eigenvectors are determined as following:
	
	and therefore for $\lambda=1$:
	
	The eigenvector is therefore:
	and therefore for $\lambda=1$:
	
	The associated norm:
	
	The normalized vector is therefore expressed as:
	
	Let us now determine the eigenvectors and eigenvalues associated with $\sigma_z$ by doing the same again.
	
	We have therefore:
	
	The eigenvectors are then for $\lambda=1$:
	
	Which makes problem to us for say anything ... the only possibility is to choose $y=0$ and therefore:
	
	and the associated norm:
	
	The normalized vector has then for expression:
	
	and for $\lambda=-1$ we will have the same choice to do by chosing this time $x=0$ and therefore:
	
	hence the associated norm:
	
	The normalized eigenvector has then for expression:
	
	Therefore the normalized eigenvectors of $\sigma_z$ are on the directions of the Cartesian coordinate axes. It is for this particular reason that the eigenvectors of $\sigma_z$ are denoted in quantum computing by:
	
	and the reader should also know that we write also:
	
	
	
	\begin{flushright}
	\begin{tabular}{l c}
	\circled{100} & \pbox{20cm}{\score{4}{5} \\ {\tiny 18 votes,  84.44\%}} 
	\end{tabular} 
	\end{flushright}
	
		
\chapter{Analysis}

	\textit{\textbf{The analysis is the rigorous formulation of calculus.}}(Wikipedia)
	\minitoc
	\pagebreak
		%to force start on odd page
	\newpage
	\thispagestyle{empty}
	\mbox{}
	\section{Functional Analysis}\label{functional analysis}
\lettrine[lines=4]{\color{BrickRed}F}unctional analysis is the branch of mathematics and specifically of the analysis that is related to the study of function spaces. It takes its historical roots in the study of transformations such as the Fourier transform and in the study of differential equations and integrals. As such it encompasses so many areas that it is difficult to justify that it be a section of this book because it is rather a field of study. Moreover, it is because of this difficulty to accurately identify the area it covers that the reader will find the Fundamental Theorem of Calculus in the chapter of Integral and Differential Calculus rather than here... 

	Why do we use the term "analysis" in the particular case of functions? The reason lies in the historical study of various phenomena of nature and resolution of various technical problems and therefore mathematics, which often lead us to consider the variation of a parameter correlated with the variation of another or several other variables. To study these variations, many tools are available to each of us:
\begin{itemize}
	\item The engineer, for example, frequently use charts (in cartesian, polar or logarithmic coordinate system... concepts which are discussed further in more detail) to determine the mathematical relations (or "law") linking variables between them. Certainly, this kind of method is (sometimes ...) aesthetic but students know well how it is sometimes painful to transcribe measures points on a sheet of paper or on a computer and consultants know how dangerous can be a chart when not build in a scientific way. This is unfortunately a necessary step (but should avoid an abusive usage) to understand how our predecessors worked and got the results that help us today in our advances in theoretical physics.
	
	\item The mathematician and theoretical physicist usually hate to use the paper-pencil-scrawl methods. Nevertheless, the role of the mathematician or physicist is to develop new theories with mathematical axioms or principles which should require no usage of graphical representation nor access to experimental  measures that are often attached to it.
\end{itemize}

	\begin{tcolorbox}[title=Remark,colframe=black,arc=10pt]
Before starting to read what follows, it may be useful to remind the reader that the definition of the concept of "function" (and the basic properties thereto) is given in the section on Set Theory.
	\end{tcolorbox}	

	Function analysis is also strongly linked to Vector Calculus (and not only...). Thus for people who want to increase their knowledge about the fundamentals of function analysis we strongly recommend the reader to have a look to the Vector Calculus section.

\pagebreak
\subsection{Representations}

	We will see in what follows, firstly, how to represent different values related by tables and charts (yes! We must because it helps to understand more complicated stuff) and secondly how to mathematically analyse the properties of these representations only by using abstract mathematical tools.

	\textbf{Definition (\#\mydef):} A function is named "\NewTerm{univalent function}\index{univalent function}" or  "\NewTerm{unary function}\index{unary function}" if the number of its arguments (parameters or variables) is equal to one. In the case of a function of two arguments, we speak about a "\NewTerm{bivalent function}\index{bivalent function}" or "\NewTerm{bivalent function}\index{bivalent function}", etc. Formally a function is $n$-ary if:
	

\subsubsection{Tabular Representation}

Among the possible visual representation of functions, the most intuitive and the oldest is the one where we have in the column or the row of a table in an orderly way the values of the independent variable $x_1,x_2,...,x_n$ and the corresponding values, namely the "\NewTerm{transformed variables}\index{transformed variables}" of the function $y_1,y_2,...,y_n$ in another column or aligned row:

	\begin{table}[H]
	\begin{center}
		\definecolor{gris}{gray}{0.85}
			\begin{tabular}{|p{2cm}|p{2.5cm}|}
				\hline
				\multicolumn{1}{c}{\cellcolor{black!30}\textbf{$x$}} & 
  \multicolumn{1}{c}{\cellcolor{black!30}\textbf{$y=f(x)$}} \\ \hline
				\centering\arraybackslash\ $x_1$ & \centering\arraybackslash\ $y_1=f(x_1)$ \\ \hline
				\centering\arraybackslash\ $x_2$ & \centering\arraybackslash\ $y_2=f(x_2)$  \\ \hline
				\centering\arraybackslash\ $...$ & \centering\arraybackslash\ $...$  \\ \hline
				\centering\arraybackslash\ $x_n$ & \centering\arraybackslash\ $y_n=f(x_n)$  \\ \hline
		\end{tabular}
	\end{center}
	\caption[]{Values and corresponding transformed variables}
	\end{table}	
	
	In the expression:
	
	we say that the $a_1,a_2,...,a_n$ are the "\NewTerm{arguments}\index{arguments}" of $f$.

Such are, for example, tables of trigonometric functions, logarithmic tables, etc. and during the experimental study of certain phenomena tables which express the existing functional dependence between the measured physical quantities such as the readings of the temperature of the air stored in a meteorological station during one day.

Of course, this concept can be generalized to any multivalent function regardless its definition domain.

However, this method is laborious and does not permit to directly see the behaviour of the function and therefore a simple and attractive visual analysis of its qualitative properties. It still has the advantage of not requiring any special tools or advanced mathematics.

	\pagebreak
	\subsubsection{Graphical Representation}
	The natural, relative, real or purely imaginary numbers  (\SeeChapter{see section Numbers page \pageref{type of numbers}}) can all be represented as simply by points on a numerical infinite axis (straight line).

	To this purpose, we choose on this axis:
	\begin{enumerate}
		\item A point O named "\NewTerm{origin}\index{origin}"
		\item A positive direction, that we indicate by a horizontal arrow
		\item A unit of measure (usually represented by small vertical lines: the "\NewTerm{graduation}\index{graduation}")
	\end{enumerate}
Such that:
\begin{figure}[H]
\centering
\includegraphics[scale=0.75]{img/analysis/representative_1d.eps}
\caption{Typical representation example of an oriented infinite axis with origin}
\end{figure}
	In most cases we put (traditionally) the axis horizontally and choose the direction from left to right (at least when there is only on axis...).
	\begin{tcolorbox}[title=Remark,colframe=black,arc=10pt]
	The point (letter) $O$, frequently represents the number zero in mathematics but we might very well choose to put the origin elsewhere. For example, in physics, the point $O$ is often positioned at the location of the centroid of a system. 
	\end{tcolorbox}
	It is obvious that the fact that the sets of numbers that we discussed in the section Numbers are ordered implies that every number is represented by a single point on this axis. Thus, two distinct real numbers correspond two different points on the axis.

	Thus, there is a correspondence between all numbers and all the points of the axis (in the case of real or complex numbers, it corresponds not a number to each graduation, but a number at each \underline{point} of the axis!). Thus, each number represents a point or a unique graduation and back to each point or graduation is a single number which is the image.

\pagebreak
\paragraph{2D representations}\mbox{}\\\\
There are besides the one dimensional representations, other of higher dimensions (phew!...) like the "\NewTerm{planar representation}\index{planar representation}" that allow us to draw much more than simple points on a one-dimensional straight line but functions of one variable (but also points!). Let's see what this is and looks like:
\begin{figure}[H]
	\centering
	\includegraphics[scale=0.5]{img/analysis/cartesian_plane.jpg}
	\caption{Position of a point in a cartesian plane}
\end{figure}
In the above figure we have added two graduation that helps us to identify uniquely the position of the point $P$ given by $P(x,y)=P(+4,+3)=P(4,3)$. We then speak of "cartesian coordinates of a point".

In single-variable calculus, the functions that one encounters are functions of a variable (usually $x$ or $t$) that varies over some subset of the real number line (which we denote by $\mathbb{R}$). For such a function, say, $y = f (x)$, the graph of the function $f$ consists of the points 
	
These points lie in the Euclidean plane, which, in the Cartesian or rectangular
coordinate system, consists of all ordered pairs of real numbers $(a,b)$. We use the word "Euclidean" to denote a system in which all the usual rules of Euclidean geometry hold (\SeeChapter{see section Euclidean Geometry page \pageref{hilbert axioms}}). We denote the Euclidean plane by $\mathbb{R}^2$. where the exponent "$2$" represents the number of dimensions of the plane.

Thus we can for each of a variable $x$ on a horizontal axis, named commonly "\NewTerm{$x$-axis\index{$x$-axis}}" match a value $y$ through a function $f$ such that:
	
plotted on a vertical axis, named commonly the "\NewTerm{$y$-axis}\index{$y$-axis}" which passes through the junction defined by the origin $O$ such tat (arbitrary example):
\begin{figure}[H]
	\centering
	\includegraphics[scale=0.75]{img/analysis/representative_2d_planar.eps}
	\caption{Typical example of a planar representation with orthogonal axes, origin O and the $4$ quadrants}
\end{figure}
All points of the plane (that latter being denoted with the variations $X\text{O}Y$, $XY$ or $x\text{O}y$, $\text{O}xy$, $xy$) have for "\NewTerm{abscissa}\index{abscissa}" traditionally the $x$-values corresponding to the independent variable (horizontal axes by tradition) of the function and for "\NewTerm{ordinate}\index{ordinate}" the corresponding value of the function (vertical axes by tradition). All these generated what is named the "\NewTerm{planar graph}\index{planar graph}" of the function. If there is no confusion, we just say "\NewTerm{graph}\index{graph}".
\begin{figure}[H]
	\centering
	\includegraphics[scale=0.4]{img/analysis/vocabulary_planar_graph.jpg}
	\caption{Vocabulary planar graph}
\end{figure}
An interesting detailed example for middle schools students that may help is the planar representation of the following line equation:

That give in the range $x\in [-3.3,+3.3]$ the following graph:
\begin{figure}[H]
	\centering
	\includegraphics[scale=0.5]{img/analysis/line_equation_plan_representation.jpg}
\end{figure}
with its equivalent tabular representation for some arbitrary chosen points:
\begin{table}[H]
	\begin{center}
		\definecolor{gris}{gray}{0.85}
			\begin{tabular}{|p{2cm}|p{3.5cm}|p{2.5cm}|}
				\hline
				\multicolumn{1}{c}{\cellcolor{black!30}\textbf{$x$}} & 
  \multicolumn{1}{c}{\cellcolor{black!30}\textbf{$2x-3$}} &  \multicolumn{1}{c}{\cellcolor{black!30}\textbf{Point $(x,2x-3)$}}\\ \hline
				\centering\arraybackslash\ $-1$ & \centering\arraybackslash\ $2\cdot(-1)-3=-5$ & \centering\arraybackslash\ $(-1,-5)$ \\ \hline
				\centering\arraybackslash\ $0$ & \centering\arraybackslash\ $2\cdot (0)-3=-3$ & \centering\arraybackslash\ $(-1,-5)$  \\ \hline
				\centering\arraybackslash\ $+1$ & \centering\arraybackslash\ $2\cdot(+1)-3=-1$ & \centering\arraybackslash\ $(-1,-5)$  \\ \hline
				\centering\arraybackslash\ $+3$ & \centering\arraybackslash\ $2\cdot(+3)-3=+3$ & \centering\arraybackslash\ $(-1,-5)$  \\ \hline
		\end{tabular}
	\end{center}
	\caption[]{Values and corresponding transformed variables}
	\end{table}	
In the case of representation by a rectangular coordinate system (cartesian, polar or logarithmic) as the figure above, we can see that the entire coordinate plane is divided into four areas that by tradition we name "\NewTerm{quadrants}\index{quadrants}" as already mentioned just earlier.

	\begin{tcolorbox}[title=Remark,colframe=black,arc=10pt]
	When we wish to highlight a particular point on the graph representing the function, we draw most of time a small round as presented in the prior-previous figure for the point of coordinates $(x_n,y_n)$.
	\end{tcolorbox}	

Another classic case of plane graph representation  known by a large number of students is the plot of polynomials (\SeeChapter{see section Calculus page \pageref{polynomial}}) with real coefficients or trigonometric functions (\SeeChapter{see section Trigonometry page \pageref{trigonometry}}).

Indeed, to solve polynomial equations of the second degree (\SeeChapter{see section Calculus page \pageref{second order polynomials}}), it is common in small classes that the teacher asks his students in addition to give an algebraic expression of the roots of:
	
given by for recall (see section Calculus page \pageref{double root} for the proof):
	
a graphics resolution where the two roots (in the case where there are two distinct real roots) are given by the intersection of the parabola with the $x$-axis (of course, if the equation has no solution, there are no intersections...):

\begin{figure}[H]
\centering
\includegraphics[scale=0.75]{img/analysis/roots_parabola.eps}
\caption{Representation of roots on a planar graph}
\end{figure}

The graphical representation can be generalized to polynomial equations of the 3rd, 4th and 5th degree (we will prove much further, using Galois theory that it is not possible to get a general algebraic expression of the roots of a polynomial equation of the 5th degree and higher).

There is another well-known and interesting example of special graph because when most young people think that after high-school they will never do maths again, in Switzerland many employees are faced to calculate in spreadsheet softwares what we name the "coordinate wage" that is a "\NewTerm{step-wise function}\index{step-wise function}" defined in year 2013 by the government as:

	

Where $R$ is a minimal value defined also by the government as being equal to 25,800.- in 2013 and the wage is denoted by the letter $S$ (for \textbf{S}alary).

When we plot such a stepwise function with for example Maple 4.00b we get:

\begin{figure}[H]
\centering
\includegraphics[scale=0.6]{img/analysis/step_wise_function.eps}
\caption{Example of step-wise function for swiss coordinate wage with Maple 4.00b}
\end{figure}

And therefore it is obvious thank to this chart representation that the previous definition can be simplified as:

	

	That is much easier to write in any spreadsheet software or also with Maple 4.00b:

\texttt{>R:=258000;}\\
\texttt{>plot(min(max(R/8,S-7/8*R),17/8*R,S=3/4*R..100000);}

	Also, graphs\index{graphs}\index{charts} are as we know powerful qualitative tools in the field of statistics (\SeeChapter{see section Statistics page \pageref{statistics}}) but also of data mining (\SeeChapter{see section Numerical Methods page \pageref{data mining}}) as a starting point for data analysis (histograms, cheese, box plots, radar, scatter plots, etc.). The assumptions and ideas that are generated by graphical analysis can be investigated with advanced statistical tools (for a few hundred of examples see the \texttt{R} or MATLAB™ softwares companion book).

Below for example, a graph (histogram) taken from the Industrial Engineering section that is very common in the field of statistics and project management in the global industry:

\begin{figure}[H]
\centering
\includegraphics[scale=0.75]{img/analysis/six_sigma.eps}
\caption{Example of typical histogram in engineering companies (Six Sigma)}
\end{figure}

Histograms allow to observe distributions and determine qualitatively if it fits a particular theoretical model.

Graphics can also be used to observe changes over time (time series, control charts, residual analysis, etc.):

\begin{figure}[H]
\centering
\includegraphics[scale=0.75]{img/analysis/time_serie.eps}
\caption{Example of OHLC time series with moving averages in financial trading}
\end{figure}
or another type of OHLC (Open-High-Low-Close) trading plot:
\begin{figure}[H]
\centering
\includegraphics[scale=0.75]{img/analysis/plot_OHLC.jpg}
\end{figure}
There are different rules for the colors of an OHLC plot! We can first define a color depending if the close price is lower than the open price (the open price is always on the left and the closure price always on the right):
\begin{figure}[H]
\centering
\includegraphics[scale=0.75]{img/analysis/plot_OHCL_color_first_rule.jpg}
\end{figure}
Or with the following rules:
\begin{figure}[H]
\centering
\includegraphics{img/analysis/plot_OHCL_color_second_rule.jpg}
\end{figure}
And still many other charts... that we have already seen and other we will see throughout the pages of this book.

\paragraph{3D representations}\mbox{}\\\\
Of course, in the case of a trivalent function (three-dimensional), that is to say a parameter which depends on two other, the idea is the same as for 2D except that the number of quadrants doubles:

\begin{figure}[H]
\centering
\includegraphics[scale=0.75]{img/analysis/quadrants_3d.eps}
\caption[Quadrants in a 3D orthogonal system]{Quadrants in a 3D orthogonal system (source: Wikipedia)}
\end{figure}

This 3D method of representation and analysis of a trivalent function was time consuming at the beginning of the 20th century but with the help of computers in the end of the 20th century this time consuming problem was almost solved...

In 3D functional analysis, we will deal with functions of two or three variables (usually  $x, y, z$, respectively). The graph of the arrow of coordinates $(x, y, z)=(x,y,f(x,y))$, lies in Euclidean space. Since Euclidean can be 3-dimensional (and more or less for sure!), we denote it by $\mathbb{R}^3$.

Euclidean space has three mutually perpendicular coordinate axes ($x$, $y$ and $z$), and three mutually perpendicular coordinate planes: the $xy$-plane, $yz$-plane and $xz$-plane:

\begin{figure}[H]
\centering
\includegraphics[scale=0.75]{img/algebra/euclidian_planes.eps}
\caption{Mutually perpendicular planes in $\mathbb{R}^3$}
\end{figure}

The coordinate system shown above is known as a right-handed coordinate system, because it is possible, using the right hand, to point the index finger in the positive direction of the $x$-axis, the middle finger in the positive direction of the $y$-axis, and the thumb in the positive direction of the $z$-axis, as below:

\begin{figure}[H]
\centering
\includegraphics[scale=0.75]{img/algebra/right_hand.eps}
\caption{Right hand system}
\end{figure}

What we are going to represent now further below (special example), purists mathematicians would notice it as follows (it's nice to have seen at least once this notation as you could meet it in other books):
	
and let us see what it gives with  Maple 4.00b:

\texttt{>restart:}\\
\texttt{>with(plots):}\\
\texttt{>f:=(x,y)->12*x/(1+x\string^ 2+y\string^ 2);}\\
\texttt{>xrange:=-10..10;yrange:=-5..5;}\\
\texttt{>plot3d(f,xrange,yrange);}

This will give:

\begin{figure}[H]
\centering
\includegraphics[scale=0.75]{img/analysis/representation_grid_function.eps}
\caption{Grid representation of a 3D function with Maple 4.00b}
\end{figure}

Let us improve the visual by adding a shading interpolation color with warm color to high positions and cold colors to low positions:

\texttt{>plot3d(f,xrange,yrange, style=patchnogrid, grid=[80,50], shading=ZHUE, axes=FRAME, tickmarks=[3,3,3], labels=[`x`,`y`,`f(x,y)`], labelfont=[TIMES,BOLD,12], title=`Graphique rempli`, titlefont=[TIMES,BOLD,12], scaling=unconstrained, orientation=[-107,68]);}

This will give:

\begin{figure}[H]
\centering
\includegraphics[scale=0.6]{img/analysis/representation_shading_interp_function.eps}
\caption{Isolines representation of a 3D function with Maple 4.00b}
\end{figure}

Let us plot now the "\NewTerm{contour lines}\index{contour line}", also named "\NewTerm{isoline}\index{isoline}\label{isoline}" (or "\NewTerm{isoquant}\index{isoquant} in Econometry), that represents lines of the same height on the function surface\footnote{It is a cross-section of the three-dimensional graph of the function $f(x, y)$ parallel to the $x, y$ plane. In cartography, a contour line (often just named a "contour") joins points of equal elevation (height) above a given level, such as mean sea level. A contour map is a map illustrated with contour lines, for example a topographic map, which thus shows valleys and hills, and the steepness of slopes. The contour interval of a contour map is the difference in elevation between successive contour lines.} (see section of Differential Geometry page \pageref{isoline} for a rigorous definition):

\texttt{>plot3d(f,xrange,yrange,style=patchcontour);}

This will give:

\begin{figure}[H]
\centering
\includegraphics[scale=0.75]{img/analysis/representation_isoline.eps}
\caption{Shading interpolation representation of a 3D function with Maple 4.00b}
\end{figure}

It's not very nice so let us improve this a little bit:

\texttt{>plot3d(f,xrange,yrange,style=patchcontour,contours=[seq(-7+k/4,k=0..60)],\\
grid=[80,50],shading=ZHUE,axes=FRAME, tickmarks=[3,3,3],\\ scaling=unconstrained,orientation=[-107,68]);}

This will give:

\begin{figure}[H]
\centering
\includegraphics[scale=0.75]{img/analysis/representation_nice_3d_function.eps}
\caption{Better representation of a 3D function with Maple 4.00b}
\end{figure}

With a small rotation to view from above:

\texttt{>plot3d(f,xrange,yrange, style=patchcontour, contours=[seq(-7+k/4,k=0..60)], grid=[80,50], shading=ZHUE, axes=FRAME, tickmarks=[3,3,3], scaling=unconstrained, orientation=[-90,0]);}

\begin{figure}[H]
\centering
\includegraphics[scale=0.75]{img/analysis/representation_nice_3d_function_above.eps}
\caption[]{Above representation of a 3D function with Maple 4.00b}
\end{figure}

And in section view (side view):

\texttt{>plot(f(x,2),x=xrange);}

\begin{figure}[H]
\centering
\includegraphics[scale=0.5]{img/analysis/representation_nice_3d_function_section.eps}
\caption{Representation of a section of the pseudo-3D surface}
\end{figure}

Or with multiple sections views:

\texttt{>display([seq(plot(f(x,y),x=xrange),y=yrange)]);}

\begin{figure}[H]
\centering
\includegraphics[scale=0.5]{img/analysis/representation_nice_3d_function_multiple_sections.eps}
\caption{Representation of multiple sections of the pseudo-3D surface}
\end{figure}

The reader can also animate the graph above with the following command:

\texttt{>display([seq(display([plot(f(x,k/5),x=xrange),}\\ \texttt{textplot([6,5,cat('y=',convert(evalf(k/5,2),string))],font=[TIMES,BOLD,16])])}\\
\texttt{,k=-25..25)],insequence=true, title='Animation',titlefont=[TIMES,BOLD,18]);}

That's all for typical and simple example of standard manipulations of an engineer hired in a company and using graphics (in practice it will instead use MATLAB™ instead of Maple but the reader can refer to the free companion book on MATLAB™ with a few hundreds of pages graphics).

\paragraph{2D Vector representations}\mbox{}\\\\
It is also frequently made use of graphic representations in the context of analytical geometry to simplify analysis or to prove theorems with the help of visual representations (do not abuse of this method!).

Thus, we can easily introduce the concept of "norm" (\SeeChapter{see section Vector Calculus page \pageref{vector norm}}) in a very easy way by plotting the distance between two points (in 2D or in 3D) and applying the Pythagorean theorem that will be assumed to be known (\SeeChapter{see section Euclidean Geometry page \pageref{pythagorean theorem}}).

The main idea of a planar vector representation in physics and engineering labs is that  a point $P_1$ of coordinates $(x_1,y_1)$ that has some physical properties (typically a velocity) will be after a given time at the point $P_2$ of coordinates $(x_2,y_2)$ supposed to be always in the same plane. In this way, the straight line between $P_1$ and $P_2$ is a visualization of the "intensity" of the velocity (and implicitly of the force). When doing that for many points we get a planar representation of a planar vector field (for more example see the companion book on MATLAB™):

\begin{figure}[H]
\centering
\includegraphics{img/analysis/vector_field.jpg}
\caption[]{Typical planar vector field with MATLAB™}
\end{figure}


Now let us represent three points $P_1,P_2,P_3$ on a plane graph in which has been defined a referential as presented below:

\begin{figure}[H]
\centering
\includegraphics{img/analysis/vector_plane.jpg}
\caption[]{Scenario of three points in a plane}
\end{figure}

We can consider the straight line $\overline{P_1P_2}$ as a vector but not translated at the origin of the referential (\SeeChapter{Vector Calculus}).

If $x_1\neq x_2$ and $y_1\neq y_2$ (as in the figure above), the points $P_1,P_2,P_3$ are the vertices of a perpendicular triangle. By applying the Pythagorean theorem (\SeeChapter{see section Euclidean Geometry page \pageref{pythagorean theorem}}) we can easily calculate the metric distance $d$ as:
	
	On the figure, we see that:
	
	Since $\forall x \in \mathbb{R} \; \vert x \vert ^2 =x^2$, we can write:
	
	If $x_1=y_1=0$, we end up with a relation named "\NewTerm{norm}", "\NewTerm{module}" or "\NewTerm{distance}\index{distance}" that we have already defined as part of our study of Vector Calculus when the origin of the vector is translated on the origin of the referential (see section of the corresponding name page \pageref{vector norm}).
	
	\begin{theorem}
	Obviously, if we consider two points $P_1(x_1,y_1),P_2(x_2,y_2)$, we can determine if a third point $P_3(x_3,y_3)$ is on the mediator (\SeeChapter{see section Euclidean Geometry page \pageref{mediator}}) of the first two and for this that it is obviously sufficient that (by definition of the mediator!):
	
	\end{theorem}
	\begin{tcolorbox}[title=Remark,colframe=black,arc=10pt]
	We hesitated to put this proof in the section of Analytical Geometry but at then end we have decided that it was a nice example of showing how visual representation can help readers to better understand some subjects.
	\end{tcolorbox}	
	\begin{dem}
	As $(x_1,y_1),(x_2,y_2)$ are known, we can easily express an "\NewTerm{analytic expression}" property of the mediator that is that for each point on the mediator we have:
	
	where $a, b$ are therefore constants and wherein any point that satisfies this relation, which is in this case the equation of a straight line, lies on the mediator.
	\begin{flushright}
		$\blacksquare$  Q.E.D.
	\end{flushright}
	\end{dem}
	Furthermore, it is easy to see that the midpoint of the line segment that coincide with the mediator is given by:
	
	So we see that with a simple visual representation, we can achieve results that are sometimes (...) more obvious for students.
	
	Let us use this example to define some concepts on that we will come back further and do some reminders.
	
	\textbf{Definition (\#\mydef):} Any function of the form of a polynomial (\SeeChapter{see section Calculus page \pageref{polynomial}}) of degree $1$ with constant real coefficients:
	
	is the analytic expression of what we name a "\NewTerm{straight line}\index{straight line}\label{straight line}" "\NewTerm{linear equation}\index{linear equation}" of "\NewTerm{slope}\index{slope}" $a$ and "\NewTerm{intercept}\index{intercept}" $b$ (when $x=0$).
	
	Obviously, if:
	
	the line is horizontal if we graphically represent it since $y$ is constant for all $x$ and is equal to $b$. Conversely, if:
	
	the straight line will be vertical in the $x\text{O}y$ referential.
	
	\paragraph{Properties of visual representations}\mbox{}\\\\
	Depending on the type of graph we visualize (especially graphics planes) it is possible to extract some basic properties. Let us see the most important one to know for univariate functions:
	
	\begin{enumerate}
		\item[P1.] The graph of a function is "\NewTerm{symmetrical about the $y$-axis}\index{graph symmetric about the $y$-axis}" if the change in from $x$ to  $-x$ in the function does not change the value of $y$ such that:
		\begin{figure}[H]
		\centering
		\includegraphics{img/analysis/function_property_symetry_y.jpg}
		\caption{Example of symmetry through the $y$-axis of a function}
		\end{figure}
		
		\item[P2.] The graph of a function is "\NewTerm{symmetrical about the $x$-axis}\index{graph symmetric about the $x$-axis}" if the change from $y$ to $-y$ does not change the value of $x$ such that:
		\begin{figure}[H]
		\centering
		\includegraphics{img/analysis/function_property_symetry_x.jpg}
		\caption{Example of symmetry through the $x$-axis of a function}
		\end{figure}
		
		\item[P3.] The graph of a function is "\NewTerm{symmetrical about the origin $\text{O}$}\index{graph symmetrical about the origin}" if the simultaneous change of $y$ to $-y$ and from $x$ to $-x$ gives the following result (that is to say that the change in the sign of one variable change the sign of the other):
		\begin{figure}[H]
		\centering
		\includegraphics{img/analysis/function_property_symetry_o.jpg}
		\caption{Example of symmetry through the origin $\text{O}$ of a function}
		\end{figure}
		
		\item[P4.] Given a function $y=f(x)$, if we add a constant $c^{te} \geq 0$ to this function as:
		
		then the function $f(x)$ is shifted (or "translated") vertically upwards of a distance $c^{te}$ as presented in the figure below:
		\begin{figure}[H]
		\centering
		\includegraphics{img/analysis/function_property_positive_translated.jpg}
		\caption{Example of a positive vertical translation of a function}
		\end{figure}
		And conversely if $c^{te} \geq 0$ but:
		
		then the function $f(x)$  is obviously translated vertically downwards:
		\begin{figure}[H]
		\centering
		\includegraphics{img/analysis/function_property_negative_translated.jpg}
		\caption{Example of a negative vertical translation of a function}
		\end{figure}
		We can also consider horizontal translations of functions. Specifically, if we have still $c^{te}$, then $y=f(x)$ is translated horizontally to the right if we write:
		
		which graphically is represented by:
		\begin{figure}[H]
		\centering
		\includegraphics{img/analysis/function_property_negative_horizontal_translated.jpg}
		\caption{Example of negative horizontal translation of a function}
		\end{figure}
		and conversely, translated horizontally to the left, if we write:
		
			as shown in the graph below:
		\begin{figure}[H]
		\centering
		\includegraphics{img/analysis/function_property_positive_horizontal_translated.jpg}
		\caption{Example of positive horizontal translation of a function}
		\end{figure}
		To stretch or compress vertically a function, we simply multiply $y=f(x)$ by a constant $c^{te}>1$ and respectively $0\leq c^{te}<1$ as:
		
		and don't forget that if a function is linear then we have the special property $f(\lambda x)=\lambda f(x)$.
		This is graphically represented for the case $c^{te}>1$ by:
		\begin{figure}[H]
		\centering
		\includegraphics{img/analysis/function_property_upscaled.jpg}
		\caption{Example of vertical stretch of a function}
		\end{figure}
		and when $0\leq c^{te}<1$ by:
		\begin{figure}[H]
		\centering
		\includegraphics{img/analysis/function_property_downscaled.jpg}
		\caption{Example of vertical compression of a function}
		\end{figure}
		To stretch or compress a function horizontally, by the same way, we just need to multiply the variable $x$ by a constant by a constant $c^{te}>1$ and respectively $0\leq c^{te}<1$ as:
		
		This is graphically represented for the case $c^{te}>1$ by:
		\begin{figure}[H]
		\centering
		\includegraphics{img/analysis/function_property_upscaled_horizontal.jpg}
		\caption{Example of horizontal stretch of a function}
		\end{figure}
		and when $0\leq c^{te}<1$ by:
		\begin{figure}[H]
			\centering
			\includegraphics{img/analysis/function_property_downscaled_horizontal.jpg}
			\caption{Example of horizontal downscale of a function}
		\end{figure}
	\end{enumerate}
	
	\begin{tcolorbox}[title=Remark,colframe=black,arc=10pt]
	Translate, stretch, compress a function or apply it a symmetry is transforming it. The plot resulting from these transformations is named the "\NewTerm{transformed}\index{transformed graph}" from the initial plot.
	\end{tcolorbox}	
	
	\textbf{Definitions (\#\mydef):} We say that a function $f$ is (we simplify the definition using an univariate function):
	\begin{itemize}
		\item[D1.] A function is a "\NewTerm{constant function}\index{constant function}" on an interval $I$ if for each pair $(x_1,x_2)$ of elements of $I$ such that $x_1\neq x_2$, we have $f(x_1)=f(x_2)$. What we denote in a condensed manner by:
		
		
		\item[D2.] A function is an "\NewTerm{increasing function}\index{increasing function}" or an "\NewTerm{increasing function in the broadest sense}" on the interval $I$ if for each pair $(x_1,x_2)$ of elements of $I$ such that $x_1\leq x_2$, we have $f(x_1)\leq f(x_2)$. What we denote in a condensed manner by:
		
		
		\item[D3.] A function is an "\NewTerm{decreasing function}\index{decreasing function}" or an "\NewTerm{decreasing function in the broadest sense}" on the interval $I$ if for each pair $(x_1,x_2)$ of elements of $I$ such that $x_1\leq x_2$, we have $f(x_1)\geq f(x_2)$. What we denote in a condensed manner by:
		
		\begin{tcolorbox}[title=Remark,colframe=black,arc=10pt]
		A function is a "\NewTerm{monotonic function}\index{monotonic function}" or "\NewTerm{monotonic function in the broadest sense}" on an interval $I$ if it is increasing or decreasing in this interval.
		\end{tcolorbox}
		
		\item[D4.] A function is a "\NewTerm{strictly increasing function}\index{strictly increasing function}"  on the interval $I$ if for each pair $(x_1,x_2)$ of elements of $I$ such that $x_1\leq x_2$, we have $f(x_1)< f(x_2)$. What we denote in a condensed manner by:
		
		
		\item[D5.] A function is an "\NewTerm{strictly decreasing function}\index{strictly decreasing function}"  on the interval $I$ if for each pair $(x_1,x_2)$ of elements of $I$ such that $x_1\leq x_2$, we have $f(x_1)> f(x_2)$. What we denote in a condensed manner by:
		
		\begin{tcolorbox}[title=Remark,colframe=black,arc=10pt]
		A function is a "\NewTerm{strictly monotonic function}" on an interval $I$ if it is strictly increasing or decreasing in this interval.
		\end{tcolorbox}
	\end{itemize}
	
	\subsubsection{Analytical Representation}
	The analytical method of representation is by far the most used and consists of representing any function in an "\NewTerm{analytic expression}\index{analytic expression}" or "\NewTerm{closed form}\index{closed form}" which is a symbolic and abstract mathematical notation of all known mathematical operations that must be applied in a certain order to numbers and letters expressing constants or variables that we seek to analyse.
	
	Note that by "all known mathematical operations", we consider not only the mathematical operations seen in the chapter of Arithmetics (addition, subtraction, root extraction, etc.) but also all the operations that will be defined later in this book.
	
	If the functional dependence $y=f(x)$ is such that $f$ is an analytic expression, then we say that the "\NewTerm{function $y$ of $x$}" is "given analytically ". 

	Here are some examples of simple analytical expressions:
	
	When we have determined the equation of the mediator, we have obtained an analytical expression of the visual straight line that characterize it as a function of the type:
	
	which we recall, is the analytical expression of the equation of a straight line, also named "\NewTerm{linear equation}\index{linear equation}" or "\NewTerm{affine function}\index{affine function}", on a plane where two points are known $P_1(x_1,y_1),P_2(x_2,y_2)$, the slope is given by the ratio of vertical growth on the horizontal growth as:
	
	A friendly and trivial application is to prove analytically that two non-vertical lines are parallel if and only if they have the same slope. Thus, given two lines with the equations:
	
	The lines intersect at a point $(x, y)$ if and only if values of $y$ are equal for a certain $x$, that is to say:
	
	The last equation can be solved with respect to $x$ if and only if $a_2-a_2\neq 0$. We have therefore proved that the lines $y_1,y_2$ intersect if and only if $a_1\neq a_2$. Therefore, they do not intersect (are parallel) if and only if $a_1=a_2$.
	
	In a quite simple way by applying the Pythagorean theorem, it is not difficult (\SeeChapter{see section Analytical Geometry page \pageref{conics}}) to determine that the equation of a circle with center $C (h, k)$ has for equation (it is of use in mathematics not explain $y$ for the equation of the circle therefore the equation of the latter is much more visually aesthetic and speaking):
	
	In these examples the functions are expressed analytically by a single formula (equality between two analytical expressions) which defines at the same time the "natural domain of definition" of the functions.
	
	\textbf{Definition (\#\mydef):} The "\NewTerm{natural domain of definition}\index{natural domain of definition}\label{natural domain of definition}" of a function given by an analytical expression is the set of $x$ values for which the expression on the right-hand side has a definite value.
	
	For example the function:
	
	is defined for all values of $x$ except the value $x=1$ where we have a singularity (division by zero).
	\begin{tcolorbox}[title=Remark,colframe=black,arc=10pt]
	There are an infinite number of functions and we can not expose them all here, however we will meet more than a thousand on this entire book and should amply suffice to get an idea of their study.
	\end{tcolorbox}
	
	And we have the famous following "\NewTerm{table of variations}\index{table of variations}\label{table of variations}" that is also considered as an analytical tool and also used by some teachers to study the basics of the derivative $f'$ of a function $f$ (\SeeChapter{see section Differential and Integral Calculus page \pageref{differential calculus}}). For example with the function $x^3-3x^2+2$ (already seen in the previous mentioned section):

	\begin{minipage}{\linewidth}\centering
    \begin{variations}
     x      & \mI &    & 0 &    & 2 &    & \pI  \\
     \filet
     f'     & \ga +    & 0    &  -  &  0   & \dr+      \\
     \filet
     \m{f}  & ~  & \c  & \h{~} & \d & ~    &  \c       \\
     \end{variations}
	\end{minipage} 	
	
	Whose corresponding plot is:
	\begin{figure}[H]
		\centering
		\includegraphics{img/algebra/variation_plot_example.jpg}
		\caption[]{Plot of  function $x^3-3x^2+2$}
	\end{figure}
	
	\pagebreak
	\subsection{Functions}\label{functions}
	In mathematics, a function is a relation between a set of inputs and a set of permissible outputs with the property that each input is related to exactly one output.
	
	Functions of various kinds are the central objects of investigation in most fields of modern mathematics. There are many ways to describe or represent a function. Some functions may be defined by a formula or algorithm that tells how to compute the output for a given input. Others are given by a picture, named the "graph" of the function. In science, functions are sometimes defined by a table that gives the outputs for selected inputs. A function could be described implicitly, for example as the inverse to another function or as a solution of a differential equation.
	
	First remember the definitions already given earlier during our study of graphical representation of functions:
	
	\textbf{Definitions (\#\mydef):} We say that a function $f$ is (we simplify the definition using an univariate function):
	\begin{itemize}
		\item[D1.] A function is a "\NewTerm{constant function}\index{constant function}" on an interval $I$ if for each pair $(x_1,x_2)$ of elements of $I$ such that $x_1\neq x_2$, we have $f(x_1)=f(x_2)$. What we denote in a condensed manner by:
		
		
		\item[D2.] A function is an "\NewTerm{increasing function}\index{increasing function}" or an "\NewTerm{increasing function in the broadest sense}" on the interval $I$ if for each pair $(x_1,x_2)$ of elements of $I$ such that $x_1\leq x_2$, we have $f(x_1)\leq f(x_2)$. What we denote in a condensed manner by:
		
		
		\item[D3.] A function is an "\NewTerm{decreasing function}\index{decreasing function}" or an "\NewTerm{decreasing function in the broadest sense}" on the interval $I$ if for each pair $(x_1,x_2)$ of elements of $I$ such that $x_1\leq x_2$, we have $f(x_1)\geq f(x_2)$. What we denote in a condensed manner by:
		
		\begin{tcolorbox}[title=Remark,colframe=black,arc=10pt]
		A function is a "\NewTerm{monotonic function}\index{monotonic function}" or "\NewTerm{monotonic function in the broadest sense}" on an interval $I$ if it is increasing or decreasing in this interval.
		\end{tcolorbox}
		
		\item[D4.] A function is a "\NewTerm{strictly increasing function}\index{strictly increasing function}"  on the interval $I$ if for each pair $(x_1,x_2)$ of elements of $I$ such that $x_1\leq x_2$, we have $f(x_1)< f(x_2)$. What we denote in a condensed manner by:
		
		
		\item[D5.] A function is an "\NewTerm{strictly decreasing function}\index{strictly decreasing function}"  on the interval $I$ if for each pair $(x_1,x_2)$ of elements of $I$ such that $x_1\leq x_2$, we have $f(x_1)> f(x_2)$. What we denote in a condensed manner by:
		
		\begin{tcolorbox}[title=Remark,colframe=black,arc=10pt]
		A function is a "\NewTerm{strictly monotonic function}\index{strictly monotonic function}" on an interval $I$ if it is strictly increasing or decreasing in this interval.
		\end{tcolorbox}
	\end{itemize}
	And let us add now complementary definitions:
	
	\textbf{Definitions (\#\mydef):}
	\begin{enumerate}
		\item[D1.] We say that $y$ is a function of $x$ and we will write $y=f(x),y=\varphi(x)$, etc., if for every value of the variable $x$ belonging to a certain domain of definition (set) $D$, corresponds a value of the variable $y$ in another target domain of definition (set) $E$. What we denote in various ways (the third one being the most recommended):
		
		The variable $x$ is named "\NewTerm{independent variable}\index{independent variable}" or "\NewTerm{input variable}" or even "\NewTerm{exogenous variable}\index{exogenous variable}" and $y$ the "\NewTerm{dependent variable}\index{dependent variable}" or "\NewTerm{endogenous variable}\index{endogenous variable}".
		
		The dependence between the variables $x$ and $y$ is named a "\NewTerm{functional dependency}\index{functional dependency}". The letter $f$, which in the symbolic notation of functional dependence, indicates that we need to apply some operations to $x$ to obtain the corresponding $y$ value.
		
		Sometimes we write:
		
		rather than:
		
		In the latter case the letter $y$ expresses at the same time the value of the function and the symbol of operations applied to $x$.
		\begin{tcolorbox}[title=Remark,colframe=black,arc=10pt]
		As we saw it during our study in the section Set Theory, an application (or function) may be injective, surjective or bijective: 
		\begin{figure}[H]
			\centering
			\includegraphics[scale=0.75]{img/analysis/functions_type.jpg}
			\caption{Quick summary of applications/functions types}
		\end{figure}
		It is therefore necessary that the reader for whom these concepts are unknown goes in priority read these definitions.
		\end{tcolorbox}
		
		\item[D2.] The set of $x$ values (inputs) for which the value of the function $y$ is given by the function $f (x)$ is named the "\NewTerm{range of existence}\index{range of existence}" of the function or "\NewTerm{domain of definition}\index{domain of definition}\label{domain of definition}" of the function and denoted in this book by the letter $D$.
		
		The set of outputs of $f(x)$ is named the "\NewTerm{image}\index{image}" or sometimes the "\NewTerm{codomain}\index{codomain}" and denoted in this book by the letter $E$. When study of the point of view of the knowledge of the output values only, the set of $x$ is named the "\NewTerm{pre-image}".
		
		\item[D3.] A function $f(x)$ is named a "\NewTerm{periodic function}\index{periodic function}" if there is a constant $c^{te}$ such that the function's value does not change when we add (or subtract we) that constant to the independent variable such as:
		
		which corresponds to a translation along the $x$-axis. The smallest constant satisfying this condition is named the "\NewTerm{period}\index{period}" of the function. It is frequently denoted by the letter $T$ in physics.
		
		The most common periodic functions know by students and engineers are the trigonometric functions (see section of the corresponding name page \pageref{trigonometry}):
		\begin{figure}[H]
			\centering
			\includegraphics[scale=0.4]{img/analysis/periodic_function.jpg}
			\caption{Example of periodic function with period and amplitude}
		\end{figure}
		 \begin{tcolorbox}[title=Remark,colframe=black,arc=10pt]
		The sum of two periodic functions with different periods is not necessarily periodic and there is no general formula to get the period of a function that is the sum of $n$ other functions!
		\end{tcolorbox}
		
		\item[D4.] In differential calculus (\SeeChapter{see section Differential and Integral Calculus page \pageref{differential calculus}}), the expression:
		
		with $h\neq 0$ is of particular interest. We name it a "\NewTerm{growth quotient}\index{growth quotient}" (we discuss this in much more detail in our study of differential and integral calculus).
		
		\item[D5.] We use certain properties of functions for easy graphical representation and analysis or mathematical simplifications. In particular, a function $f (x)$ is named "\NewTerm{even function}\index{even function}\label{even function}" if:
		
		for all $x$ in its definition domain.
		
		That is to say as we already seen previously, it is symmetric relatively with the $y$-axis:
		\begin{figure}[H]
			\centering
			\includegraphics{img/analysis/function_property_symetry_y.jpg}
			\caption{Example of even function}
		\end{figure}
		A function $f (x)$ is named "\NewTerm{odd function}\index{odd function}\label{odd function}" if:
		
		for all $x$ in its definition domain.
		
		That is to say as we already seen previously, it is symmetric relatively with the origin:
		\begin{figure}[H]
			\centering	\includegraphics{img/analysis/function_property_symetry_o.jpg}
			\caption{Example of odd function}
		\end{figure}
		So, to summarize, an even function is a function that is independent of the sign of the variable and an odd function change of sign when we change the sign of the variable (the spiral of Cornus in the section Civil Engineering is a good practical example of odd function). This concept will be very useful to us to simplify some very useful developments in physics (such as Fourier transforms of odd or even functions for example, or the calculation of certain integrals!).
		\begin{theorem}
		Remember that this type theorem linking a general concept to a particular case and its opposite is often found in mathematics. We will see such examples in tensor calculus with the symmetric and antisymmetric tensor (\SeeChapter{see section Tensor Calculus page \pageref{symmetric tensor} and page \pageref{antisymmetric tensor}}) or in quantum physics with the Hermitian or non-Hermitian operators (\SeeChapter{see section Wave Quantum Physics page \pageref{hermitian operator} and page \pageref{non-hermitian operator}}).
		\end{theorem}
		\begin{dem}
		Let us write:
		
		Then:
		
		If we sum then we get:
		
		and by subtracting:
		
		So there is really and odd and even decomposition of any function!!!
		\begin{flushright}
			$\blacksquare$  Q.E.D.
		\end{flushright}
		\end{dem}
		Finally, it is important to note that:
		\begin{itemize}
			\item The product of two even functions is an even function
			\item The product of two odd functions is an even function
			\item The product of an even and odd function is an odd function
		\end{itemize}
		Let us see a short proof of the last property because we will need it in the chapter on Geometry.
		\begin{dem}
		Let $g(x)$ be an even function and $h(x)$ an odd function such as:
		
		Therefore:
		
		\begin{flushright}
			$\blacksquare$  Q.E.D.
		\end{flushright}
		\end{dem}
		\item[D6.] In general, if $f (x)$ and $g (x)$ are arbitrary functions, we use the terminology and notations given in the following table:
		\begin{table}[H]	
			\begin{center}
				\begin{tabular}{|c|c|}
				\hline
				  \rowcolor[gray]{0.75}Terminology&Value of the function\\
				  \hline
				  % after \\: \hline or \cline{col1-col2} \cline{col3-col4} ...
				  Sum $f+g$ & $(f+g)(x)=f(x)+g(x)$ \\\hline
				  Difference $f-g$ & $(f-g)(x)=f(x)-g(x)$ \\\hline
				  Product $f \cdot g$ & $(f \cdot g)(x)=f(x)g(x)$ \\\hline
				  Quotient $\displaystyle\frac{f}{g}$&$\left(\displaystyle\frac{f}{g}\right)(x)=\displaystyle\frac{f(x)}{g(x)}$ \\\hline
				\end{tabular}
			\end{center}
			\caption{Terminology about functions}
		\end{table}
		The definition domains of $f+g,f-g,f\cdot g$ are the intersection $I$ of the definition domain of $f (x)$ and g $(x)$, that is to say, the numbers which are common to both domains of definition. The domain of definition of $g/g$ is meanwhile the subset $I$ of all $x$ such that  $g(x)\neq 0$.
		
		\item[D7.] Let $y$ be a function of $f$ of $u$ such that $y=f(u)$ and $u$ a function $g$ of $x$ such that $u=g(x)$, then $y$ depends on $x$ and we have what we name a "\NewTerm{composite function}\index{composite function}\label{composite function}" that we denote:
		
		The last equality should be read "\NewTerm{$f$ round $g$}" and not confuse with the "round" symbol with the notation of the dot product that we have seen during our study of the section Vector Calculus page \pageref{dot product}.
		
		The domain of definition of the composite function is either identical to the entire domain of definition of the function $u=g(x)$ or the part of the domain in which the values of $u$ are such that the corresponding values $f (u)$ belong to the domain of definition of this function.
		
		Obviously the principle of composite function can be applied not only once, but an arbitrary number of times such that $y=f(g(h(t)))$ and so on...
		
		In computing science a function may compose with itself a given number of times $n$, such that $f(f(f(f(f...)))))=f^n$ that must not be confuse with the notation $f^2(x)$.
		
		If $u$ does not depend on another variable (or it is not itself a composite function), then we say that $f(x)$ is an "\NewTerm{elementary function}\index{elementary function}".

		Obviously there are an infinite number of elementary functions but most can be classified into families whose expression is similar to one of the following:
		
		\begin{itemize}
			\item "\NewTerm{Linear functions}\index{linear function}":
			
			The are simply functions representing straight lines of slope $a$ passing through the origin of the axis.
			
			\item "\NewTerm{Affine functions}\index{affine function}":
			
			The are simply functions representing straight lines of slope $a$ passing through the origin of the axis or not (linear function with a translation).
						
			\item "\NewTerm{Power functions}\index{power function}":
			
			where $m\in\mathbb{R}$. Functions involving roots are often named "\NewTerm{radical functions}\index{radical functions}".
			\begin{figure}[H]
				\centering
				\includegraphics{img/analysis/power_function.jpg}
				\caption{Different plots of simple power functions}
			\end{figure}
			
			\item "\NewTerm{Absolute value functions}\index{absolute value function}\label{absolute value plot}" (see section Arithmetic Operators page \pageref{absolute value} for the definition and the study of the "absolute value"):
			
			For example the plots with Maple 4.00b that we get with the command:\\
			
			\texttt{>plot([(x),(cos(x)),(x\string^2-3),(x\string^3-4*x\string^2+2*x)],x=-6..6,y=-4..3,\\
			thickness=3);}	
			
			\begin{figure}[H]
				\centering
				\includegraphics{img/analysis/pre_absolute_plot_functions.jpg}
			\end{figure}
			
			and taking the absolute value:\\
			
			\texttt{>plot([abs(x),abs(cos(x)),abs(x\string^2-3),abs(x\string^3-4*x\string^2+2*x)]\\
			,x=-6..6,y=-0.5..3,thickness=3);}
			\begin{figure}[H]
				\centering
				\includegraphics{img/analysis/post_absolute_plot_functions.jpg}
			\end{figure}
		
		\item "\NewTerm{Exponential functions}\index{exponential function}":
			
			where the famous $e^x$ is only a special case and also $a\in\mathbb{R}$.
			
			When $a\geq 0$ we have typically:
			\begin{figure}[H]
				\centering
				\includegraphics{img/analysis/exponential_functions.jpg}
				\caption{Different plots of simple exponential functions $(1^2,2^x,3^x,4^x)$ with Maple 4.00b}
			\end{figure}
			where $m$ is a positive number different from $1$ (otherwise it is simple a linear function):
			
			If $a<0,x\in\mathbb{R}$ the function is not defined. Indeed for $(-1)^(0.5)=\left\lbrace \mathrm{i},	-\mathrm{i}	\right\rbrace$ therefore it is an application from $f:\mathrm{R}\mapsto\mathbb{C}^2$ and as far as we know there is no nice way to represent it visually and anyway this is not a function in the traditional way.
			
			\item "\NewTerm{Logarithmic functions}\index{logarithmic function}":
				
			with $a\in\mathbb{R}^{+}$ and that by construction of the logarithm (see further below) are of the type $f:\mathbb{R}^{+}\mapsto \mathbb{R}$.
			We have typically:
			\begin{figure}[H]
				\centering
				\includegraphics{img/analysis/logarithm_functions.jpg}
				\caption{Different plots of logarithm $\ln(x)=\ln_e(x)$ in green and $\log_{10}(x)$ in red with Maple 4.00b}
			\end{figure}
			
			\item "\NewTerm{Periodic/Trigonometric functions}\index{period function}\index{trigonometric function}":
			
			We already defined previously what is a periodic function. For the trigonometric functions the reader can see below a plot of the main one but for more details it is strongly recommended to read the section Trigonometry page \pageref{trigonometry}:
			\begin{figure}[H]
				\centering
				\includegraphics[scale=0.9]{img/analysis/trigonometric_functions.jpg}
				\caption{Different plots of trigonometric functions with Maple 4.00b}
			\end{figure}
			
			\item "\NewTerm{Polynomial functions}\index{polynomial function}":
			
			
			where as we already know $a_0,a_1,...,a_n$ are constant numbers named "\NewTerm{coefficients}\index{coefficients}" and $n$ is a positive integer that we name "\NewTerm{degree of the polynomial}\index{degree of a polynomial}". Obviously this function is defined for all values of $x$, that is to say, it is define on an infinite interval.
			
			If follows that functions the power functions of the type $x^m$ and linear functions of the type $f(x)=x$ are a subclass of polynomial for $m\in \mathbb{N}$.
			
			We have already study more deeply polynomials in the section Calculus with their main properties but let us give us again the plot of some of them as recall: 
			\begin{figure}[H]
				\centering
				\includegraphics{img/algebra/polynomials.jpg}
				\caption[Some polynomials plotted with R.3.2.1]{Some polynomials plotted with R.3.2.1 (see our \texttt{R} companion book)}
			\end{figure}
			We will see and study in this book some famous polynomials as: Legendre polynomials (\SeeChapter{see section Quantum Chemistry page \pageref{legendre polynomial}}), Bernoulli polynomials (\SeeChapter{see section Sequences and Series page \pageref{bernoulli polynomials}}), Bernstein polynomials (\SeeChapter{see section Numerical Methods page \pageref{bernstein polynomial}}), Hermite polynomials (\SeeChapter{see section Functional Analysis page \pageref{hermite polynomial}}), ...
			
			\item "\NewTerm{Rational fractions}\index{rational fractions}" are polynomials divisions (\SeeChapter{see section Calculus page \pageref{polynomials division}}):
			
			\begin{tcolorbox}[title=Remark,colframe=black,arc=10pt]
			Obviously two rational fractions are equal, if one is obtained from the other by multiplying the numerator and denominator by the same polynomial.
			\end{tcolorbox}
			The rational function:
			
			is not defined at $x^2=5 \Leftrightarrow x=\pm \sqrt{5}$. It is asymptotic (see further below) to $\frac{x}{2}$ as $x$ approaches infinity:
			\begin{figure}[H]
				\centering
				\includegraphics{img/analysis/rational_function.jpg}
				\caption[Example of rational function]{Example of rational function $f(x) = \frac{x^3-2x}{2(x^2-5)}$}
			\end{figure}
			The rational function:
			
			 is defined for all real numbers, but not for all complex numbers, since if x were a square root of -1 (i.e. the imaginary unit or its negative), then formal evaluation would lead to division by zero!
			 
			 A constant function such as is a rational function since constants are polynomials. Every polynomial function $f(x) = P(x)$ is a rational function with $Q(x) = 1$. The power functions $f(x)=x^m$ are also rational functions when $m\in\mathbb{N}$.
			 
			 \item "\NewTerm{Algebraic functions}\index{algebraic function}" are defined by the fact that the function $f(x)$ is the result of addition, subtraction, multiplication, division, of variables put to an integer or non-integer power. Therefore most of the functions defined previously can be included in this definition: linear functions, affine function, power function, polynomial function, rational functions.
			 
			 \item A "\NewTerm{piecewise function}\index{piecewise function}" is a function defined by different formulas on different parts of its domain. The absolute value is a famous example of a piecewise-defined function because the formula changes with the sign of $x$:
			 
			 
			 \item A "\NewTerm{step function}\index{step function}" $f:[a,b]\in \mathbb{R}$ is defined if and only if there exists a subdivision $(a_i)_{0\leq i \leq n}$ of $[a, b]$ such that $a_0=a$ and $a_n=b$ and $(\lambda_0,...,\lambda_n)\in \mathbb{R}^n$ such as:
			 
			\begin{figure}[H]
				\centering
				\includegraphics{img/analysis/step_function.jpg}
				\caption{Example of a step function}
			\end{figure}
			Such functions can be found in signal processing and also in statistics for survival analysis.
		\end{itemize}
		
		However, there are a very large number of other elementary functions that will meet in the individual sections of this book. Examples include the "Bessel functions" (\SeeChapter{see section Sequences and Series page \pageref{bessel functions}}), the "Lipschitz functions" (\SeeChapter{see section Topology page \pageref{lipschitz functions}}), the "Dirac functions" (\SeeChapter{see section Differential and Integral Calculus page \pageref{dirac function}}), the "distribution functions" (\SeeChapter{see section Statistics page \pageref{distribution function}}), the "Euler gamma function" (\SeeChapter{see section Differential and Integral Calculus page \pageref{gamma euler function}}), etc. The reader will notice that the Dirac function also belongs to the family of distribution functions.
	\end{enumerate}
	
	Here is a quite good summary (non exhaustive but good!):
	\begin{figure}[H]
		\centering
		\includegraphics{img/analysis/functions.jpg}
		\caption[Visual representation of various functions]{Visual representation of various functions (source: ?)}
	\end{figure}
	
	\subsubsection{Limits and Continuity of Functions}\label{limits}
	We will now consider ordered variables of a special type, which we define by the relation "the variable tends to a limit." In what will follow, the concept of limit of a variable will play a fundamental role, being intimately related to the basic notions of mathematical analysis, derivatives, integrals, etc.
	
	\textbf{Definition (\#\mydef):} The number $a$ is named the "\NewTerm{limit}\index{limit}" of variable magnitude $x$, if for any arbitrarily small positive number $\varepsilon$ we have:
	
	If the number $a$ is the limit of the variable $x$, we say that "\NewTerm{$x$ tends to the limit $a$}".

	We can also define the concept of limit from geometrical considerations (this can help to better understand ... but not always ...):
	
	The constant number $a$ is the limit of the variable $x$, if for any given neighbourhood, no matter how small, of center $a$ and of radius $\varepsilon$, we can find a value $x$ such that all the points corresponding to the following values of the variable belong to this neighbourhood (notions that we defined earlier). We represent geometrically this as:
	\begin{figure}[H]
		\centering
		\includegraphics{img/analysis/limit_geometric_representation.jpg}
		\caption{Geometric concept of limit in $\mathbb{R}^1$}
	\end{figure}
	\begin{tcolorbox}[title=Remarks,colframe=black,arc=10pt]
	\textbf{R1.} It should be trivial that the limit of a constant value is equal to this constant, since the inequality $|x-c^{te}|=|c^{te}-c^{te}|=0<\varepsilon $ is always satisfied for an arbitrary $\varepsilon>0$.\\
	
	\textbf{R2.}  Not all variable have limits. For example $y=\sin(x)$ as this trigonometric function fluctuates between $[-1,+1]$ from $[-\infty,+\infty]$.
	\end{tcolorbox}
	
	\textbf{Definition (\#\mydef):} A variable $x$ tends to infinity if for any positive chosen $M$, we indicate one value of $x$ from which all successive values of the variable $x$ (values in the neighbourhood of the previous chosen value) satisfy the inequality $|x|>M$. Formally:
	
	\begin{itemize}
		\item A variable $x$ "\NewTerm{tends to $+\infty$}" if for any positive chosen $M>0$, we indicate one value of $x$ from which all the successive values of the variables $x$ satisfies the inequality $M<x$.
	
		It is typically the type of consideration that we have for divergent sequences (divergent to infinity) where for a given term of value $M$ of the sequence all the other terms are greater ant $M$.
		
		
		\item A variable $x$ "\NewTerm{tends to $-\infty$}" if for any negative chosen $M<0$, we indicate one value of $x$ from which all the successive values of the variables $x$ satisfies the inequality $x<M$.
		
	\end{itemize}
	\textbf{Definition (\#\mydef):} Given $y=f(x)$ a function defined in a neighbourhood of $a$ or on some point of this neighbourhood. The function $y=f(x)$ tends to the limit $b$ (that is to say $y\rightarrow b$) when $x$ tends to $a$ (that is to say $x a$) if for any positive number $\varepsilon$ as small as possible, we can indicate  positive number $\delta$ such that all $x$ different from $a$ satisfying the inequality $|x-a|<\delta$ also satisfy $|f(x)-b|<\varepsilon$. Formally a function has a limit $b$ on $a$ when in a domain $E$ if:
	
	The inequality $|x-a|<\delta$ gives the possibility to have the distance from which we come with our $x$ without taking care of the direction (left or right) as we take for measurement of distance the absolute values. Indeed on a system of axis representing ordinates values, we can, for a given value, coming from the left or from the right (if necessary you can imagine a bus coming to a bus stop that can from the left or from the right only since the absolute distance from it to the bus stop is less than or equal to $\delta$).
	
	If $b$ is the limit of the function $f (x)$ when $x\rightarrow a$ we then write in this book in any case:
	
	Obviously the above definition is available when $a=\pm \infty$ or/and $b=\pm \infty$!	
	
	To define the direction from which we come from by applying the limit, we use a special notation (recall that this will give us the information of which side of the road comes our bus from...). Thus, if $f (x)$ tends towards the limit $b_1$ when $x$ approaches a number $a$ by taking only values smaller than $a$, then we write:
	
	(notice the small $-$ subscript) and we name $b_1$ the "\NewTerm{left limit}\index{left limit}" of the function $f (x)$ at point $a$ (because remember that the horizontal axis goes from left to right from $-\infty$ to $+\infty$, so small values compared to a given value, are on the left).
	
	If $x$ takes values greater than $a$, then we will write:
	
	(notice the small $+$ subscript) and we name $b_2$ the "\NewTerm{right limit}\index{right limit}" of the function $f (x)$ at point $a$.
	
	In the figure below we have for example:
	
	\begin{figure}[H]
		\centering
		\includegraphics{img/analysis/limits.jpg}
		\caption{Left and Right limit examples}
	\end{figure}
	It is not always easy (or even possible) to calculate limits of some functions. Let us see some typical examples:
	\begin{tcolorbox}[colframe=black,colback=white,sharp corners]
	\textbf{{\Large \ding{45}}Examples:}\\\\
	E1. Let us prove that:
		
	is true. For this purpose we have to prove that for any small $\varepsilon$ the inequality:
	
	will be satisfied as soon as $|x|>M$ where $M$ is defined by the choice of $\varepsilon$. The previous inequality is obviously equal to:
	
	which is satisfy if we have $x$:
	
	We admit that the example and the method can be discussed.... But in fact it is only an application of the Hospital rule (ratio of the derivatives) already proved in the section of Differential and Integral Calculus. The reader must also know that we will see also other techniques to determine limits further below with better examples.\\
	
	E2. Now using Taylor series and change of variables consider we want to calculate:
	
	The method is quite to intuitive. Indeed, first we do a change of variable:
	
	Now consider the Taylor series about $x=0$ for the function $f(x)=\sqrt{1+ax}$. We have:
	
	Which gives:
	\end{tcolorbox}
	
	\begin{tcolorbox}[colframe=black,colback=white,sharp corners]
	
	as a Taylor expansion about $x=0$. Applying this to our limit we see that:
	
	E3. We want to calculate the limit of:
	
	How can we deal with something like this? An idea is the to remember that it also implicitly means:
	
	Hence:
	
	And to answer what is the value of $\theta$, we refer to the plot of the function $\tan(\theta)$ and we see then that the first corresponding value is $\pi/2$, therefore:
	
	\end{tcolorbox}
	
	The signification of the symbols $x\rightarrow -\infty$ and $x\rightarrow +\infty$ makes obvious the signification of the expressions:
	
	and:
	
	that we denote formally by:	
	
	We have defined the case where the function $f (x)$ tends to a certain limit $b$ when $x\rightarrow a_{+,-}$ or $x\rightarrow \pm \infty$. Now let us consider the case where the function tend to infinity when the variable $x$ change in a certain way.
	
	We then have typically and obviously:
	
	Or when we need to indicate the direction:
	
	If the function $f(x) \rightarrow +\infty$ when $a \rightarrow +\infty$ then we write:
	
	And as we have four possibilities for the sign, we write:
	
	that is to say the four following possibilities:
	
	And once again don't forget as we already mentioned before that some function such as for example $f(x)=\sin(x)$ don't have any finite limit when $x\rightarrow \pm \infty$. Then we say that the function is just "bounded" (\SeeChapter{see section Set Theory page \pageref{closed bounded interval}}).
	\begin{figure}[H]
		\centering
		\includegraphics{img/analysis/limite_delucq.jpg}
	\end{figure}
	Now that we've roughly an overview of the concept of limit, we will give an extremely important definition that has a very important place of many areas of high-level mathematics, theoretical physics and computing science (numerical methods).
	
	\textbf{Definition (\#\mydef):} Given a function $f(x)$ and one of its subdomain (or whole one) $E$ (most of time $E \subseteq \mathbb{R}$ and $x_0\in E$, we say that we have a "\NewTerm{continuous function}\index{continuous function}" on $x_0$ if and only if:
	
	That is to say more formally (you have to be able to read the fact that we are going close in an infinitely small way of a limit an this allows the continuity):
	
	In other words: a function is continuous if for every point $x_0$ in the domain $E$, we can make the images of that point ($f(x_0)$) and another point ($f(x)$) arbitrarily close (of a distance $\varepsilon$) if we move the other point ($x$) close enough (distance $\delta$) to our given point.
	
	The latter relation will be generalized a little bit in the section of Topology and completed with the concept of... "uniform continuity"!
	\begin{tcolorbox}[title=Remarks,colframe=black,arc=10pt]
	\textbf{R1. }$f$ is "\NewTerm{continuous on the left}\index{continuous on the left}" or respectively "\NewTerm{continuous on the right}\index{continuous on the right}", if we add to the definition above the condition $x>x_0$, respectively $x<x_0$.\\
	
	\textbf{R2.} A continuous function with a continuous inverse function is named a "\NewTerm{homeomorphism}\index{homeomorphism}".\\
	
	\textbf{R3.} Instead of saying when necessary that a function is not continuous on $x_0$ or on a given domain, some practitioners prefer to say that the function has an "\NewTerm{oscillation}\index{oscillation}".
	\end{tcolorbox}	
	We have the following trivial corollaries:
	\begin{enumerate}
		\item[C1.] $f(x)$ is continuous on $x_0$ if and only if $f(x)$ is continuous on the left right and on the right left.
		
		\item[C2.] $f(x)$ is continuous on $E$ if and only if $f(x)$ is continuous on any point of $E$.
	\end{enumerate}
	
	\paragraph{Limit laws}\mbox{}\\\\
	We now take a look at the "\NewTerm{limit laws}\index{limit laws}", the individual properties limits in the univariate case. The proofs will be omitted as it is quite intuitive but any reader can request us the proof of one of them if needed!
	
	Let $f(x)$ and $g(x)$ be defined for all $x\neq a$ over some open interval containing $a$. Assume that $L$ and $M$ are real numbers such that:
	
	Let $c^{te}$ be a constant. Then, each of the following statements holds:		
	\begin{itemize}
		\item The sum law for limits gives:
		
		
		\item The difference law for limits gives:
		
		
		\item Constant multiple law for limits:
		
		
		\item Product law for limits:
		
		
		\item Quotient law for limits:
		
		for $M\neq 0$.
		
		\item Power law for limits:
		
		for every positive integer $n$.
		
		\item Root law for limits:
		
		for all $L$ if $n$ is odd and for $L\geq 0$ if $n$ is even.
	\end{itemize}
	
	\subsubsection{Asymptotes}
	The term "\NewTerm{asymptote}\index{asymptote}" is used in mathematics to precise possible properties of an infinite branch of curve which growth tends to an infinitesimal value.
	
	In analytic geometry, an asymptote of a curve is simply said to be a line such that the distance between the curve and the line approaches zero as they tend to infinity. In some contexts, such as algebraic geometry, an asymptote is defined as a line which is tangent to a curve at infinity.
	\begin{tcolorbox}[title=Remark,colframe=black,arc=10pt]
	The word asymptote is derived from the Greek and means "not falling together".
	\end{tcolorbox}	
	
	\textbf{Definitions (\#\mydef):}
	
	\begin{enumerate}
		\item[D1.] When the limit of a function $f(x)$ tends to a constant 
$c^{te}$ when $x \rightarrow \pm \infty$, then the graphical representation of this function leads us to draw a horizontal line that we name "\NewTerm{horizontal asymptote}\index{horizontal asymptote}" which equation is satisfies:
		
		
		\item[D2.] When the limit of a function $f(x)$ tends to  
$\pm \infty$ when $x \rightarrow a_{+,-}$, then the graphical representation of this function leads us to draw a vertical line that we name "\NewTerm{vertical asymptote}\index{vertical asymptote}" which equation is satisfies:
		
		Vertical asymptotes is the typical symptom of a division by zero in a fraction and has a very important place in physics. The syndrome is also named a "\NewTerm{singularity}\index{singularity}".
		\begin{tcolorbox}[colframe=black,colback=white,sharp corners]
		\textbf{{\Large \ding{45}}Example:}\\\\
		The graph of the function:
		
		 has the straight line of $x=1$ and $y=0$ as horizontal asymptote:
		 \begin{figure}[H]
			\centering
			\includegraphics{img/analysis/asymptote_vertical_horizontal_example.jpg}
			\caption{Graphical representation of a  horizontal and vertical asymptote}
		\end{figure}
		\end{tcolorbox}
		
		\item[D3.] The straight line of equation is an "\NewTerm{oblique asymptote}\index{oblique asymptote}" of a curve of the function $f (x)$ if:
		
		the values of $a$ and $b$ can be easily found using the following relations:
		
		
		\begin{tcolorbox}[colback=red!5,borderline={1mm}{2mm}{red!5},arc=0mm,boxrule=0pt]
		\bcbombe Caution! A curve may have two distinct oblique asymptotes in $+\infty$ and $-\infty$.
		\end{tcolorbox}
		
		To find a possible oblique asymptote, one must already be certain that the function $f(x)$ admits an infinite limit in $+\infty$ or $-\infty$ then only we look for the limits at $-\infty$ and $+\infty$ of  $f (x) / x$ and $f(x)-ax$.
		
		Three typical cases can be considered for oblique asymptotes:
		\begin{enumerate}
			\item The representative curve of $f(x)$ has for asymptotical direction the affine equation $y=ax$:
			
			\begin{tcolorbox}[colframe=black,colback=white,sharp corners]
			\textbf{{\Large \ding{45}}Example:}\\\\
			The graph of the function:
			
			 has the straight line of $y=x$ as oblique asymptote:
			 \begin{figure}[H]
				\centering
				\includegraphics{img/analysis/asymptote_oblique_affine_example.jpg}
				\caption{Graphical representation of an oblique affine asymptote}
			\end{figure}
			\end{tcolorbox}
			
			\item The representative curve of $f(x)$ has an infinite branch (this branch has not close form asymptote) and the only one thing we can say is that $x$-axis is the direction of this asymptote. Such an asymptote exists when:
			
			\begin{tcolorbox}[colframe=black,colback=white,sharp corners]
			\textbf{{\Large \ding{45}}Example:}\\\\
			The functions $f(x)=\sqrt{x}$ (in red) or $\ln(x)$ (in green) have a limit $f(x)/x$ equal to $0$ and both have a "parabolic branch"  of direction following the $x$-axis:
			 \begin{figure}[H]
				\centering
				\includegraphics{img/analysis/asymptote_parabolic_branche_example_x.jpg}
				\caption[]{Graphical representation of an parabolic branch example following $x$-axis}
			\end{figure}
			\end{tcolorbox}
			
			\item The representative curve of $f(x)$ has an infinite branch (this branch has not close form asymptote) and the only one thing we can say is that $y$-axis is the direction of this asymptote (we then also speak of "parabolic branch"):
			
			\begin{tcolorbox}[colframe=black,colback=white,sharp corners]
			\textbf{{\Large \ding{45}}Example:}\\\\
			The function $f(x)=x^2$ has an infinite $f(x)/x$ limit and therefore has a parabolic branch of direction following the $y$-axis.
			 \begin{figure}[H]
				\centering
				\includegraphics{img/analysis/asymptote_parabolic_branche_example_y.jpg}
				\caption[]{Graphical representation of a parabolic branch example following $y$-axis}
			\end{figure}
			\end{tcolorbox}
			
			\item A function $f(x)$ is say to have a "\NewTerm{curvilinear asymptote}\index{curvilinear asymptote}" if it satisfies:
			
			for $n>1 $where for recall $P_n(x)$ is a polynomial of degree $n$.
			\begin{tcolorbox}[colframe=black,colback=white,sharp corners]
			\textbf{{\Large \ding{45}}Example:}\\\\
			The function :
			
			has a curvilinear asymptote that is:
			
			Indeed:
			
			 \begin{figure}[H]
				\centering
				\includegraphics{img/analysis/asymptote_curvilinear_example.jpg}
				\caption[]{Graphical representation of curvilinear asymptote}
			\end{figure}
			\end{tcolorbox}
		\end{enumerate}
	\end{enumerate}
	There are many techniques for finding limits that apply in various conditions. It's important to know all these techniques, but it's also important to know when to apply which technique. Some basic techniques who doesn't involve derivation are:
	\begin{figure}[H]
		\centering
		\includegraphics{img/analysis/finding_limits.jpg}
		\caption[Basic techniques for finding limits]{Basic techniques for finding limits (source: Khan Academy)}
	\end{figure}
	
	
	\pagebreak
	\subsubsection{Concavity/Convexity of a function}
	We will define now a property which at first sight may seem of no interest as it is so trivial but which we shall find in the section of Statistics for the demonstration of an important relation named "Jensen inequality" and which is of major importance Finance and Insurance for the valuation of options and premiums (\SeeChapter{see section Economy page \pageref{finance convex function}}) and especially for the application of the Jensen's inequality (\SeeChapter{see section Statistics page \pageref{jensen inequality}}).

	Consider the following figure:
	\begin{figure}[H]
		\centering
		\includegraphics{img/analysis/concavity_convexity.jpg}
		\caption[]{Graphical representation of curvilinear asymptote}
	\end{figure}
	\textbf{Definition (\#\mydef):} In mathematics, a real function of a real variable is say to a "\NewTerm{convex function}\index{convex function}\label{convex function}" if, viewed from below, its graph is convex (in bump); we mean that if $A$ and $B$ are two points of the graph of the function, the segment $[AB]$ is entirely situated above the graph. It is the same to say that the "\NewTerm{epigaph}" (the set of points above the graph) is a concave set. Conversely, a function whose graph, as seen from below, is seen as a cave, is say to be a "\NewTerm{concave function}\index{concave function}\label{concave function}". It is the same to say that the "\NewTerm{hypograph}" (the set of points below) is a convex set.
	
	By specifying by the values of the function what are the points $A$ and $B$ above, we get often an equivalent definition of the convexity of a function: a function defined on a real interval $I$ is convex when, for any $x_1$ and $x_2$ of $I$ and all $t$ in $[0,1]$ we have:
	
	When the inequality is strict, then we obviously speak of a "\NewTerm{strictly convex function}".
	
	\begin{tcolorbox}[title=Remarks,colframe=black,arc=10pt]
	Without proof, just by looking to the above chart, we will assume quite obvious that convexity implies $f''(x)\geq 0$ for all $x$. Just as before, strict convexity occurs when
the inequality is strict.
	\end{tcolorbox}

	By extension (common sense from my point of view), a function $f$ is concave if $-f$ is convex (which is trivial with the pay-off function - see section Economy page \pageref{finance convex function} - profile of options from seller or buyer point of view ).
	
	\begin{tcolorbox}[colframe=black,colback=white,sharp corners]
	\textbf{{\Large \ding{45}}Examples:}\\\\
	E1. Consider $f(x)=x^{2}$ . The first derivative of $f(x)$ is given by $\frac{\mathrm{d}}{d\mathrm{d} x} f=2 x$ and its second derivative by $\frac{\mathrm{d}^{2}}{\mathrm{d} x^{2}} f=2$. Since this is  always strictly greater than $0,$ we have proven that $f(x)=x^{2}$ is strictly convex.\\
	
	E2. $f(x)=\log (x)$ . The first derivative is $\frac{\mathrm{d}}{\mathrm{d} x} f=\frac{1}{x}$ and its second derivative is given by $\frac{\mathrm{d}^{2}}{\mathrm{d} x^{2}} f=-\frac{1}{x^{2}}$. Since this is negative for all $x>0$, we have proven that $\log(x)$ is a concave function over $\mathbb{R}_{+}$ .
	\end{tcolorbox}
	
	\subsubsection{Euler theorem for homogeneous functions}
	We have to introduce now a definition and a theorem that will be very important for quite advanced concepts in our study of Physics.
		
	\textbf{Definition (\#\mydef):} A "\NewTerm{homogeneous function}\index{homogeneous function}\label{homogeneous function}" is one with multiplicative scaling behaviour: if all its arguments are multiplied by a factor, then its value is multiplied by some power of this factor. For example, a homogeneous real-valued function of two variables $x$ and $y$ is a real-valued function that satisfies the condition $f(t x,t y)=t ^{k}f(x,y)$ for some constant $k$ and all real numbers $t\in \mathbb{R}^*$. The constant $k$ is named  the "\NewTerm{degree of homogeneity}".
	
	More generally, if $f:\mathbb{R}^n\mapsto \mathbb{R}$ is a function between two vector spaces over a field $F$, and $k$ is an integer, then $f$ is said to be homogeneous of degree $k$ if:
	
	or more commonly written:
	
	
	\begin{tcolorbox}[colframe=black,colback=white,sharp corners]
	\textbf{{\Large \ding{45}}Example:}\\\\
	The function $f(x,y)=x^{2}+y^{2}$ is homogeneous of degree $2$. Indeed:
	
	\end{tcolorbox}
	
	Now let us suppose $f:\mathbb{R}^n\mapsto \mathbb{R}$ is continuously differentiable on $\mathbb{R}$. We know that a function is homogeneous of degree $k$ if:
	
	Differentiating both sides with respect to $t$, we get:
	
	by the total exact differential chain rule of $f$. 
	
	So the "\NewTerm{Euler theorem for homogeneous functions}\index{Euler theorem for homogeneous functions}\label{Euler theorem for homogeneous functions}" can be summarized as following:
	
	\begin{tcolorbox}[title=Remark,colframe=black,arc=10pt]
	Remember that we have proved in the section of Differential and Integral Calculus (see page \pageref{total exact differential}) that:
	
	Therefore:
	
	\end{tcolorbox}
	
	Then if we choose to set with the special case $k=1$ (it's the case that will interest us the most in Physics) then the above becomes:
	
	or explicitly for $n=2$:
	
	if $f(tx)$ is homogeneous of degree $1$.
	
	This is an important result we will need in Lagrangian Mechanics that will be useful for the study of the Lagrangian of a free particle in Special Relativity and in Quantum Cosmology.
	
	
			
	\pagebreak
	\subsection{Logarithms}\label{logarithms}
	We hesitated to put the definition of logarithms in the section Calculus. After a moment of reflection, we decided it was better to put it in this section because to understand it well, we must be aware of the concept of limits, of definition domain and of the power function. We hope that our choice will suit you best.
	
	Given the power (bijective) function of any base where $a \in \mathbb{R}_{+}^{*}/1$ (we exclude $1$ otherwise it is not bijective) and denoted for recall by:
	
	for which it corresponds to each real number $x$, exactly one positive number $a^x$ (the image set of the function is in $\mathbb{R}$) such as the powers calculations rules are applicable (\SeeChapter{see section Calculus page \pageref{power rules calculations}}).
	
	We know that for such a function that if $a>1$, then $f (x)$ is an increasing and positive (monotone) in $\mathbb{R}$, and if $0<a<1$, then $f(x)$ is positive and decreasing (monotone) in $\mathbb{R}$.
	
	\begin{tcolorbox}[title=Remarks,colframe=black,arc=10pt]
	\textbf{R1.} If $a>1$, when $x$ decreases to negative values, the graph of $f (x)$ approaches the $x$-axis. Thus, the $x$ axis is a horizontal asymptote. When $x$ increases in positive values, the graph rises quickly. This type of change is characteristic of the "\NewTerm{law of exponential of growth}\index{law of exponential of growth}" and $f(x)$ is sometimes named "\NewTerm{growing function}\index{growing function}"... If $0<a<1$, when $x$ increases, the graph tends asymptotically to the $x$-axis. This type of variation is known as an "\NewTerm{exponential decay}\index{exponential decay}".\\
	
	\textbf{R2.} By studying $a^x$, we exclude the case where $a\leq 0$ and $a=1$. Notice that if $a<0$, then $a^x$ is not a real number for many values of $x$ (we recall that the whole image set is forced to $\mathbb{R}$ in our previous definition). If $a=0$, the $a^0=0$ is not defined. Finally, if $a=1$, then $a^x=1$ for all $x$ and the graph of $f(x)$ is a horizontal line.
	\end{tcolorbox}
	As the power function $f (x)$ is bijective then there exists an inverse function $f^{-1}(x)$ and is named "\NewTerm{logarithm function}\index{logarithm function}" of base $a$ and is denoted by:
	
	and therefore:
	
	if and only if $y=a^x$.
	
	More generally it is defined by:
	
	
	Considering $\log_a(x)$ as an exponent, we have the following properties:	
	\begin{table}[H]
	\begin{center}
		\begin{tabular}{|c|c|}
			  \hline
			  \rowcolor[gray]{0.75}Properties & Justification \\ \hline
			  $\log_a1=0$ & $a^0=1$ \\ \hline
			  $\log_aa=a$ & $a^1=a$ \\ \hline
			  $\log_aa^x$ & $a^x=a^x$ \\ \hline
			  $a^{\log_a(x)}=x$ & $a^{\log_a(x)}=a^y=x$ \\
			  \hline
		\end{tabular}
		\end{center}
		\caption{Properties of the logarithm in base $a$}
	\end{table}
	\begin{tcolorbox}[title=Remarks,colframe=black,arc=10pt]
	\textbf{R1.} The word "logarithm" means "number of logos", "logos" meaning "reason" or "ratio".\\
	
	\textbf{R2.} The logarithm and power functions are defined by their bases (the number $a$). When using a power of $10$ as a base ($10, 100, 1000, ...$) then we speak of "\NewTerm{common system}\index{common system}" because they have for $\log$ successive integers.\\
	
	\textbf{R3.} The integer part of the logarithm is named the "\NewTerm{characteristic}"\index{characteristic of a logarithm}.
	\end{tcolorbox}
	There are two types of logarithms that we find almost exclusively in mathematics and physics: the logarithm of base $10$, logarithm of base $e$ (the latter often named "\NewTerm{natural logarithm}\index{natural logarithm}") and logarithm of base $2$ for information theory.
	
	First the on in base $10$ (the most used on graphical representations):
	
	abusively noted:
	
	and the base (Eulerian) $e$:
	
	historically noted:
	
	the "$n$" meaning "Napierian".
	
	\begin{tcolorbox}[title=Remark,colframe=black,arc=10pt]
	Historically, it is John Napier (1550-1617) whose name was Latinized "Napier" that we own the study of logarithms and the name of "natural logarithms" which aimed to facilitate greatly the time for manual calculations.
	\end{tcolorbox}
	In English for the logarithm function in base-$10$ logarithmic we need to calculate:
	
	ask the following question: at what power $n\in \mathbb{R}$ should we raise $10$ to get $x$?
	
	Formally, this consist to solve the equation:
	
	or written in another way:
	
	with $x$ being known and therefore in base $10$:
	
	The logarithm in base $10$ is used a lot in graphical representations in the scientific perspective when we look at amplitudes variations. For example with Maple 4.00b  we have for two sine function  having respectively for their respective mean the same amplitude variation of $50\%$ visible below that do not highlights necessarily this fact trivially:
	
	\texttt{>plot({10+0.5*10*sin(x),100+100*0.5*sin(x)},x=1..10);}
	
	\begin{figure}[H]
		\centering
		\includegraphics{img/analysis/two_sinus_for_comparison_without_logarithm_scale.jpg}
		\caption[]{Plot with Maple 4.00b with two sine functions having same amplitude change compared to their average}
	\end{figure}
	While in logarithmic scale, this gives:
	
	\texttt{>with(plots):\\
	>logplot({10+0.5*10*sin(x),100+100*0.5*sin(x)},x=1..10);}
	
	\begin{figure}[H]
		\centering
		\includegraphics{img/analysis/two_sinus_for_comparison_with_logarithm_scale.jpg}
		\caption[]{Same plot with Maple 4.00b but with the $y$-axis in logarithm (base $10$) scale}
	\end{figure}
	For the logarithmic function in Eulerian base $e$ it is necessary to calculate:
	
	to ask ourselves the following question: at what power $n\in \mathbb{R}$ we must raise the number $e$ to get $x$?
	
	Formally this consists to solve the equation:
	
	with $x$ being known and therefore:
	
	Technically, we say that the exponential function (see below for details):
	
	is the inverse bijection of the $\ln (x)$ function.
	\begin{figure}[H]
		\centering
		\includegraphics{img/analysis/bijection_ln_x_exp_x.jpg}
		\caption{Graphical representation of the correspondence between the natural logarithm and the exponential}
	\end{figure}
	But what is that "Eulerian" number also named "\NewTerm{Euler number}\index{Euler number}\label{Euler number}"? Why do we find so often in physics and mathematics? Let us first determine the origin of its value:
	
	with $\alpha \in \mathbb{N}$ and when $\alpha \rightarrow +\infty$.
	\begin{tcolorbox}[title=Remark,colframe=black,arc=10pt]
	The second term of the equality is typically the type of expression that we find in compound interest in finance (\SeeChapter{see section Economy page \pageref{compound interest}}) or in any other type of identical increase factor. And what interests us in this case is when this type of increase tends to infinity.
	\end{tcolorbox}
	The interest we have to pose the problem as in this way is that if we do tend $\alpha \rightarrow +\infty$ the function written above tends to $e$ and this function has the special property of being calculable more or less easily for historical reasons using Newton's binomial.
	
	So according to the development of the Newton binomial (\SeeChapter{see section Calculus page \pageref{binomial coefficient development}}) we can write:
	
	This development is similar to the Taylor expansion (\SeeChapter{see section Sequences and Series page \pageref{taylor series}}) of some given functions for particular cases of development values (hence the reason why we find this eulerian number in many places that we will later).
	
	By performing some algebraic transformations that should now be obvious to the reader, we find:
	
	We see in this last equality that the function $\left(1+\frac{1}{\alpha}\right)^\alpha$ is increasing when $\alpha$ increases. Indeed, when we move from $\alpha$ to the value $\alpha+1$ each term of this sum increases:
	
	Let us prove now that the variable $\left(1+\frac{1}{\alpha}\right)^\alpha$ is bounded. By seeing that:
	
	So we get by analogy with the extended expression of Newton binomial determined just previously the following order relation:
	
	On the other hand:
	
	We then can write the inequality:
	
	The underlined terms constitute a geometric sequence of reason $q=1/2$ (\SeeChapter{see section Sequences and Series page \pageref{geometric sequence}}) and whose first term is $1$. If follows using the result obtained in the section of Sequences ans Series, that we can write:
	
	Therefore, we have:
	
	We have therefore proved that the function $\left(1+\frac{1}{\alpha}\right)^\alpha$ is bounded.
	
	The limit:
	
	tends to this limited value that is the number $e$ whose value is:
	
	The prior previous relation is also know in the following form after an obvious change of variable:
	
	\begin{tcolorbox}[title=Remark,colframe=black,arc=10pt]
	As we have proved it in the section Numbers, this number is irrational.
	\end{tcolorbox}
	We can then define the "\NewTerm{natural exponential function}\index{natural exponential function}\label{natural exponential function}" (reciprocal of the natural logarithm function) by:
	
	also sometimes denoted by:
	
	The number $e$ and the function that determines it are very useful. We find them in all areas of mathematics and physics and thus in almost all the chapters of this book.
	
	As we have proved it in the section of Differential and Integral Calculus the functions $e^x$ has for remarkable property that its derivative is equal to itself:
	
	and this is used a lot for the resolution of differential equations in physics and finance.
	
	Logarithms have several properties. Here are the most important one in our point of view (we are referring to a given base $X$) and that are very useful in physics, electronics, chemistry and so on...
	
	Let us begin. First:
	
	If we put $X^m=a$ and $X^n=b$ we get:
	
	If we have the special case when $a=b$ then:
	
	Now let us try to express:
	
	in another way. For this we put first:
	
	which leads us to the development:
	
	Now let us try to express:
	
	with $n\in \mathbb{N}^{*}$ in another way. For this we put first:
	
	which leads us to the development:
	
	There is  also another relation used a lot of time in physics in respect to the change of logarithm basis. The first relation is trivial and follows from the algebraic properties of logarithms:
	
	the second one:
	
	is a bit less trivial and requires perhaps a proof (we used it for our study of continued fractions in the section Number Theory).
	\begin{dem}
	We first use the equivalent equations (of the first relation above):
	
	and we proceed as follows:
	
	What finally brings us to:
	
	\begin{flushright}
		$\blacksquare$  Q.E.D.
	\end{flushright}
	\end{dem}
	
	\pagebreak
	\subsection{Convolutions}\label{convolution}
	A convolution is a mathematical operation on two functions to produce a third function, that is typically viewed as a modified version of one of the original functions, giving the integral of the point-wise multiplication of the two functions as a function of the amount that one of the original functions is translated.
	
	There are different type of convolutions and as always, we will focus in this book only on the one that are actually used in other sections of this books.
	
	\subsubsection{Continuous and Discrete Linear Convolution Product}
	\textbf{Definition (\#\mydef):} The "\NewTerm{continuous convolution}\index{continuous convolution}\label{continuous convolution}" of two continuous signals $x(t)$ and $h(t)$ is defined as:
	
	
	\begin{tcolorbox}[title=Remark,colframe=black,arc=10pt]
	The convolution is also sometimes denoted with different symbols:
	
	or:
	
	\end{tcolorbox}
	We will now prove some properties of the convolutions on focusing only on those use in the other sections of this book!
	\begin{itemize}
		\item[P1.] Convolution is commutative:
		
		\begin{dem}
		By making the change of variable $\lambda=t-\tau$, in one form of the definition of convolution:
		
		it becomes:
		
		proving that convolution is commutative.
		\begin{flushright}
			$\blacksquare$  Q.E.D.
		\end{flushright}
		\end{dem}
		
		\item[P2.] Convolution is associative:
		
		\begin{dem}
		The proof is easier to understand if we consider a limited integral (but you can change the bounds to the infinity one and you will fall back on the general result):
		
		 The proof may not be obvious for many readers. So we recommend to see the equivalent proof for the discrete version further below.
		\begin{flushright}
			$\blacksquare$  Q.E.D.
		\end{flushright}
		\end{dem}
		
		\item[P3.] Convolution is distributive:
		
		\begin{dem}
		
		\begin{flushright}
			$\blacksquare$  Q.E.D.
		\end{flushright}
		\end{dem}
		
		\item[P4.] Relation with differentiation:
		
		\begin{dem}
		
		\begin{flushright}
			$\blacksquare$  Q.E.D.
		\end{flushright}
		\end{dem}
	\end{itemize}
	We will assume as obvious that these properties also apply to the discrete convolution that we will introduce now!
	
	Typically, $y(t)$ is the output of a system characterized by its impulse response function $h(t)$ with input $x(t)$.
	
	\textbf{Definition (\#\mydef):} The "\NewTerm{discrete convolution}\index{discrete convolution}" of two discrete signals $x[n]$ and $h[n]$ is defined as:
	
	If $h[m]$ is finite, e.g.:
	
	the convolution becomes
	
	If the system in question were a causal system in time domain:
	
	the above would become:
	
	This is also written:
	
	\begin{tcolorbox}[colframe=black,colback=white,sharp corners]
	\textbf{{\Large \ding{45}}Examples:}\\\\
	E1. Consider:
	
	We get visually (it's not obvious!):
	\begin{figure}[H]
		\centering
		\includegraphics[scale=0.6]{img/analysis/discrete_convolution.jpg}
		\caption{Discrete convolution example}
	\end{figure}
	The details steps are given by first considering that:
	
	and the fact that:
	And we have for recall:
	
	And in tabular form:
	\begin{table}[H]
	\centering
		\begin{tabular}{lccccc}
		$n$: & $0$ & $1$ & $2$ & $3$ & $4$ \\
		$m=0$ & ${\color{blue}{2}}\cdot {\color{red}{3}}=6$ & ${\color{blue}{1}}\cdot {\color{red}{3}}=3$ & ${\color{blue}{3}}\cdot {\color{red}{3}}=9$ & $0$ & $0$ \\
		$m=1$ & $0$ & ${\color{blue}{2}}\cdot {\color{red}{2}}=4$ & ${\color{blue}{1}}\cdot {\color{red}{2}}=2$ & ${\color{blue}{3}}\cdot {\color{red}{2}}=6$ & $0$ \\
		$m=2$ & $0$ & $0$ & ${\color{blue}{2}}\cdot {\color{red}{1}}=2$ & ${\color{blue}{1}}\cdot {\color{red}{1}}=1$ & ${\color{blue}{3}}\cdot {\color{red}{1}}=3$ \\ \hhline{|=|=|=|=|=|=|}
		$c[n]$ & $6$ & $7$ & $13$ & $7$ & $3$
		\end{tabular}
	\end{table}
	The $c[n>4]$ being all equal to zero.\\
	\end{tcolorbox}
	\begin{tcolorbox}[colframe=black,colback=white,sharp corners]
	
	
	The final result will then be written:
	
	Such a calculations can obviously be done very simply with softwares like \texttt{R} for example (see the corresponding companion book).\\
	
	E2. A well known example is the convolution of two gaussians (\SeeChapter{see section Statistics page \pageref{sum of two random normal variables}}) that also result in a... gaussian. And obviously the discrete version of the convolution also give a gaussian as illustrated below with \texttt{R} (see the corresponding companion book for more details):
	\begin{figure}[H]
		\centering
		\includegraphics[scale=0.55]{img/analysis/discrete_convolution_gaussians_R.jpg}
	\end{figure}
	where we have as the reader have noticed above, the following convolution $\mathcal{N}(5,20)*\mathcal{N}(10,3)$ and we see obviously that the result gives a gaussian of $\mathcal{N}(5+10=15,\sqrt{20^2+3^2}=15)$.
	\end{tcolorbox}
	Let us now prove that the discrete convolution is also associative for the discrete convolution (as promised earlier!):
	\begin{dem}
	
	\begin{flushright}
		$\blacksquare$  Q.E.D.
	\end{flushright}
	\end{dem}
	However, in image processing, we often consider convolution in spatial domain where causality does not apply.
	
	If $h[m]=h[-m]$ is symmetric (almost always true in image processing),
	then replacing $m$ by $-m$ we get:
	
	We see that now the convolution is the same as the {\em correlation} of the two functions. 
	
	If the input $x[m]$ is finite (always true in reality), i.e.:
	
	its index $n+m$ in the convolution has to satisfy the following for $x$ to be in the valid non-zero range:
		
	or correspondingly, the index $n$ of the output $y[n]$ has to satisfy:
	
	When the variable index $m$ in the convolution is equal to $k$, the 
	index of output $y[n]$ reaches its lower bound $n=-k$; when $m=-k$, 
	the index of $y[n]$ reaches its upper bound $n=N+k-1$. In other words,
	there are $N+2k$ valid (non-zero) elements in the output:
	
	
	Assume the size of the input signal $x[n]$ is $N$ ($n=0,\cdots,n=N-1$) and the size of $h$ is $M=2k+1$ (usually an odd number), then the size of the resulting convolution $y=x*h$ is $N+M-1=N+2k$. However, as it is usually desirable for the output $y$ to have the same size as the input $x$, we can drop $k$ components at each end of $y$. When the size of $h$ is even,we can drop $k$ components at one end and $k-1$ from the other of $y$.
	
	The code segment for this $1$D convolution $y=x*h$ is given below. 
	
	In particular, if the elements of the kernel are all the same (an average operator or a low-pass filter), then we can speed up the convolution process while sliding the kernel over the input signal by taking care of only the two ends of the kernel.
	
	\subsubsection{Matrix Convolution}\label{matrix convolution}
	In image processing, all of the discussions above for one-dimensional convolution are generalized into two dimensions!
	
	"\NewTerm{Matrix convolution}\index{matrix convolution}" is the treatment of a matrix by another one which is named the "\NewTerm{kernel}". Most of the time, the convolution matrix filter uses a first matrix which is the image to be treated. The image is a bi-dimensional collection of pixels in rectangular coordinates. The used kernel $h$ of always \underline{odd} $k\times k$ dimensions  depends on the effect you want!
	
	
	For example, if we have the following matrices:
	
	Then (don't forget that the columns and row of the kernel matrix are flipped!):
	
	That can be illustrated as following ($h$ has already been double-flipped in the figure below):
	\begin{figure}[H]
		\centering
		\includegraphics[width=0.8\textwidth]{img/analysis/matrix_convolution.jpg}
		\caption{Matrix Convolution}
	\end{figure}
	The full example can be run and reproduced with a free software like R:
	\begin{figure}[H]
		\centering
		\includegraphics[width=1.0\textwidth]{img/analysis/matrix_convolution_r.jpg}
	\end{figure}
	Or for people who may prefer a slow implementation in C++:
	\begin{lstlisting}[language={C++}, caption={C++ for matrix convolution}]
	// find center position of kernel (half of kernel size)
	kCenterX = kCols / 2;
	kCenterY = kRows / 2;

	for(i=0; i < rows; ++i)              // rows
	{
	    for(j=0; j < cols; ++j)          // columns
	    {
	        for(m=0; m < kRows; ++m)     // kernel rows
	        {
	            mm = kRows - 1 - m;      // row index of flipped kernel
	
	            for(n=0; n < kCols; ++n) // kernel columns
	            {
	                nn = kCols - 1 - n;  // column index of flipped kernel
	
	                // index of input signal, used for checking boundary
	                ii = i + (kCenterY - mm);
	                jj = j + (kCenterX - nn);
	
	                // ignore input samples which are out of bound
	                if( ii >= 0 && ii < rows && jj >= 0 && jj < cols )
	                    out[i][j] += in[ii][jj] * kernel[mm][nn];
	            }
	        }
	    }
	}
	\end{lstlisting}

	\pagebreak
	\subsection{Integral Transforms}
	An "\NewTerm{integral transform}\index{integral transform}" is an operator that maps functions from one space to another. Formally:
	
	Now the practical motivation for an integral transform is to reduce the complexity of the problem i.e the mathematical operations will be much easier to handle in the image space (typically the resolution of differential equations!).
	
	However, as much as it is fun to do work in the image space, one has to be able to interpret the results in the original space. To do so requires the study of the operator $K$. Usually one knows a priori the nature of the function $f$ by the nature of the problem one is dealing. Hence the study of integral transforms is the study of the operator $\mathcal{T}$. Two properties come very easily:
	
	To ensure invertibility, one has to show that the kernel space only contains the null function.

	The Fourier and Laplace transforms are for example continuous (integral) transforms of continuous functions (even if there exist a discrete version of the Fourier transform!).

	The Laplace transform maps a function $f(t)$ to a function $\mathcal{F}(s)$ of the complex variable $s$, where:
	
	Since the derivative:
	
	maps to $s\mathcal{F}(s)$, the Laplace transform $\mathcal{L}$ of a linear differential equation is an algebraic equation. Thus, the Laplace transform is useful for, among other things, solving linear differential equations.
	
	If we set the real part of the complex variable $s$ to zero, $\sigma=0$, the result is the Fourier transform $\mathcal{F}(\mathrm{i}\omega)$ which is essentially the frequency domain representation of $f(t)$ (note that this is true only if for that value of $\sigma$ the formula to obtain the Laplace transform of $f(t)$ exists, i.e., it does not go to infinity).
	
	The $\mathcal{Z}$-transform is essentially a discrete version of the Laplace transform and, thus, can be useful in solving difference equations, the discrete version of differential equations. The $\mathcal{Z}$-transform maps a sequence $f[n]$ to a continuous function $F(z)$ of the complex variable $z=re^{\mathrm{i}\Omega}$.
	
	If we set the magnitude of $z$ to unity, $r=1$, the result is the Discrete Time Fourier Transform (DTFT) $\mathcal{F}(\mathrm{i}\Omega)$ which is essentially the frequency domain representation of $f[n]$.
	
	\begin{tcolorbox}[title=Remarks,colframe=black,arc=10pt]
	\textbf{R1.} The three integral transformations mentioned above (Fourier, Laplace, $\mathcal{Z}$) are only a sample of what exists in practice. Let us also mention the Hartley transform, the Mellin transform, the Weierstrass transform, the Hankel transform (Fourier-Bessel), the Abel transform, the Hilbert transform, the Gauss-Weierstrass transform, etc.\\
	
	\textbf{R2.} For information we will calculate cases and prove properties of the Fourier, Laplace, $\mathcal{Z}$ and Hilbert transforms only used in concrete applications in the industry and mainly useful in other chapters of this book! Indeed a rather general presentation would require a few hundred pages and it is not the objective of this book, as you already know it, to prove mathematical properties not associated with concrete cases.
	\end{tcolorbox} 
	
	\subsubsection{Fourier Transform} \label{fourier transform analysis}
	So we have already introduced Fourier Transforms in the section of Sequences and Series (page \pageref{fourier transform}), we will come back here more in details on this topic in (we wish) are more structured way...
	
	\paragraph{Continuous Time Fourier Transform}\mbox{}\\\\	
	The Fourier expansion coefficient $X[k]$ of a continuous periodic  signal $x_T(t)=x_T(t+T)$ is:
	
	and the Fourier expansion of the signal is:
	
	which can also be written as:
	
	where $X(k\omega_0)$ is defined as:
	
	
	When the period of $x_T(t)$ approaches infinity $T \rightarrow +\infty $, the 
	periodic signal $x_T(t)$ becomes a non-periodic signal $x(t)$ and the following 
	will result:
	\begin{itemize}
	
	\item Interval between two neighbouring frequency components becomes zero:
	
	
	\item Discrete frequency becomes continuous frequency:
	
	
	\item Summation of the Fourier expansion becomes an integral:
	
	the second equal sign is due to the general fact:
	
	
	\item Time integral over $T$ becomes over the entire time axis:
	
	\end{itemize}
	
	In summary, when the signal is non-periodic $x(t)=\lim_{T\rightarrow +\infty}x_T(t)$, the Fourier expansion becomes Fourier transform. The forward transform (analysis) is:
	
	and the inverse transform (synthesis) is:
	
	
	Comparing Fourier coefficient of a periodic signal $x_T(t)$ with Fourier spectrum of a non-periodic signal $x(t)$:
	
		
	The spectrum of a time signal can be denoted by $X(\omega)$ or $X(f)$ to emphasize the fact that the spectrum represents how the energy contained in the signal is distributed as a function of frequency $\omega$ or $f$. Moreover, if $X(f)$ is used, the factor $1/2\pi$ in front of the inverse transform is dropped so that the transform pair takes a more symmetric form. On the other hand, as Fourier transform can be considered as a special case  of Laplace transform when the real part $\sigma$ of the complex argument $s=\sigma+\mathrm{i}\omega=\mathrm{i}\omega$ is zero:
	
	it is also natural to denote the spectrum of $x(t)$ by $X(\mathrm{i}\omega)$.
	
	Ok this done let us see now some important example for physics (especially quantum physics) and signal processing!
	\begin{tcolorbox}[colframe=black,colback=white,sharp corners]
	\textbf{{\Large \ding{45}}Examples:}\\\\
	E1. Consider the unit impulse function (Dirac function):
	
	Therefore:
	
	and if $a=0$ we have then:
	
	E2. If the spectrum of a signal $x(t)$ is a delta function in frequency domain $X(\mathrm{i}\omega)=2\pi\;\delta(\omega)$, the signal can be found to be:
	
	i.e.:
	
	E3. We consider:
	
	The spectrum is:
	
	This is the sinc function with a parameter $a$.\\
	
	Note that the height of the main peak is $2a$ and it gets taller and narrower as	$a$ gets larger.
	
	\end{tcolorbox}
	
	
	\begin{tcolorbox}[colframe=black,colback=white,sharp corners]
	 Also note:
	
	When $a$ approaches infinity, $x(t)=1$ for all $t$, and the spectrum becomes:
	
	Recall that the Fourier coefficient of $x(t)=1$ is:
	
	which represents the energy contained in the signal at $k=0$ (DC component at zero frequency), and the spectrum $X(\mathrm{i}\omega)=X[k]/\omega$ is the energy density or distribution which is infinity at zero frequency.\\
	
	The integral in the above transform is an important formula to be used frequently later:
	
	which can also be written as:
	
	Switching $t$ and $f$ in the equation above, we also have:
	
	representing a superposition of an infinite number of cosine functions of all
	frequencies, which cancel each other any where along the time axis except at
	$t=0$ where they add up to infinity, an impulse. \\
	
	E3. Let us now consider:
	
	The spectrum of the cosine function is:
		
	\end{tcolorbox}
	
	\begin{tcolorbox}[colframe=black,colback=white,sharp corners]
	
	The spectrum of the sine function:
	
	can be similarly obtained to be:
	
	Again, these spectra represent the energy density distribution of the sinusoids, while the corresponding Fourier coefficients:
	
	and:
	
	represent the energy contained at frequency $\omega=\omega_0$.
	\end{tcolorbox}
	
	
	
	
	\subparagraph{Properties of Fourier Transform}\mbox{}\\\\	
	The properties of the Fourier transform are summarized below. For some of them already proved in the section of Sequences and Series we will not give the poof. But for others we will do it again! 
	
	In the following, we assume 
	$\;\;{\cal F}(x(t))=X(\mathrm{i}\omega)$ and ${\cal F}(y(t))=Y(\mathrm{i}\omega)$.
	
	\begin{itemize}
	
	\item[P1.] Linearity:
	
	
	\item[P2.] Time shift:
	
	\begin{dem} Let $t'=t\pm t_0$, i.e., $t = t' \mp t_0$, we have:
	\begin{eqnarray}
	{\cal F}(x(t \pm t_0))&=&\int\limits_{-\infty}^{+\infty} x(t\pm t_0)
		e^{-\mathrm{i}\omega t} \mathrm{d}t
		=\int\limits_{-\infty}^{+\infty} x(t')e^{-\mathrm{i}\omega(t'\mp t_0)} \mathrm{d}t'
		\nonumber \\
		&=& e^{\pm \mathrm{i}\omega t_0}
		\int\limits_{-\infty}^{+\infty} x(t')e^{-\mathrm{i}\omega t'} \mathrm{d}t'=X(\mathrm{i}\omega)e^{\pm \mathrm{i}\omega t_0}
		\nonumber
	\end{eqnarray}
	\begin{flushright}
		$\blacksquare$  Q.E.D.
	\end{flushright}
	\end{dem}
	
	\item[P3.] Frequency shift:
	
	\begin{dem} Let $\omega'=\omega\pm \omega_0$, i.e., $\omega = \omega'\mp\omega_0$,
	we have:
	
	\begin{flushright}
		$\blacksquare$  Q.E.D.
	\end{flushright}
	\end{dem}
	
	\item[P4.] Time reversal:
	
	\begin{dem}
	
	Replacing $t$ by $-t'$, we get:
	
	\begin{flushright}
		$\blacksquare$  Q.E.D.
	\end{flushright}
	\end{dem}
	
	\item[P5.] Even and Odd Signals and Spectra:
	
	If the signal $x(t)$ is an even (or odd) function of time, its spectrum $X(\mathrm{i}\omega)$ is an even (or odd) function of frequency:
	
	and:
	
	\begin{dem} If $x(t)=x(-t)$ is even, then according to the time reversal property, we have:
	
	i.e., the spectrum $X(\mathrm{i}\omega)=X(-\omega)$ is also even. Similarly, if $x(t)=-x(-t)$ is odd, we have:
	
	i.e., the spectrum $X(\mathrm{i}\omega)=-X(-\omega)$ is also odd.
	\begin{flushright}
		$\blacksquare$  Q.E.D.
	\end{flushright}
	\end{dem}
	
	\item[P6.] Time and frequency scaling:
	
	\begin{dem}
	Let $u=at$, i.e., $t=u/a$, where $a>0$ is a scaling factor, we have:
	
	Note that when $a<1$, time function $x(at)$ is stretched, and $X(\mathrm{i}\omega/a)$ is compressed; when $a>1$, $x(at)$ is compressed and $X(\mathrm{i}\omega/a)$ is stretched.	This is a general feature of Fourier transform, i.e., compressing one of the $x(t)$ and $X(\mathrm{i}\omega)$ will stretch the other and vice versa. In particular, when $a\rightarrow 0$, $x(at)$ is stretched to approach a constant, and $X(\mathrm{i}\omega/a)/a$ is compressed with its value increased to approach an impulse; on the other	hand, when $a \rightarrow +\infty$, $ax(at)$ is compressed with its value increased to approach an impulse and $X(\mathrm{i}\omega/a)$ is stretched to approach a constant.
	\begin{flushright}
		$\blacksquare$  Q.E.D.
	\end{flushright}
	\end{dem}
	
	\item[P7.] Complex Conjugation:
	
	
	\begin{dem} Taking the complex conjugate of the inverse Fourier transform, we get:
	
	Replacing $\omega$ by $-\omega'$ we get the desired result:
	
	\begin{flushright}
		$\blacksquare$  Q.E.D.
	\end{flushright}
	\end{dem}
	We further consider two special cases:
	\begin{itemize}
	\item If $x(t)=x^*(t)$ is real, then:
	
	i.e., the real part of the spectrum is even (with respect to frequency $\omega$), and the imaginary part is odd:
	
	\item If $x(t)=-x^*(t)$ is imaginary, then:
	
	i.e., the real part of the spectrum is odd, and the imaginary part is even:
	
	\end{itemize}
	
	If the time signal $x(t)$ is one of the four combinations shown in the table (real even, real odd, imaginary even, and imaginary odd), then its spectrum $X(\mathrm{i}\omega)$ is given in the corresponding table entry:
	\vskip0.2in
	\begin{table}[H]
		\centering
		\begin{tabular}{c||c|c} \hline
			& if $x(t)$ is real		& if $x(t)$ is imaginary	\\ 
			& $X_r$ even, $X_i$ odd	& $X_r$ odd, $X_i$ even \\ \hline \hline
		if $x(t)$ is Even	&			&		\\
		$X_r$ and $X_i$ even	& $X_i=0$, $X=X_r$ even & $X_r=0$, $X=X_i$ even	\\ \hline
		if $x(t)$ is Odd	& 			&		\\
		$X_r$ and $X_i$ odd	& $X_r=0$, $X=X_i$ odd	& $X_i=0$, $X=X_r$ odd	\\ \hline
		\end{tabular}
	\end{table}
	Note that if a real or imaginary part in the table is required to be both even 
	and odd at the same time, it has to be zero.
	
	These properties are summarized below:
	\vskip 0.1in
	\begin{table}[H]
		\centering
		\begin{tabular}{l|l|l} \hline
		  & $x(t)=x_r(t)+\mathrm{i}x_i(t)$	& $X(\mathrm{i}\omega)=X_r(\mathrm{i}\omega)+\mathrm{i}X_i(\mathrm{i}\omega)$	\\ \hline
		1 & real $x(t)=x_r(t)$ 		& even $X_r(\mathrm{i}\omega)$, odd $X_i(\mathrm{i}\omega)$ \\
		2 & real and even $x(-t)=x_r(t)$ 	& real and even $X_r(\mathrm{i}\omega)$ \\
		3 & real and odd $x(-t)=-x_r(t)$ 	& imaginary and odd $X_i(\mathrm{i}\omega)$ \\
		4 & imaginary $x(t)=x_i(t)$  	& odd $X_r(\mathrm{i}\omega)$, even $X_i(\mathrm{i}\omega)$ \\ 
		5 & imaginary and even $x(-t)=x_i(t)$ 	& imaginary and even $X_i(\mathrm{i}\omega)$ \\
		6 & imaginary and odd $x(-t)=-x_i(t)$ 	& real and odd $X_r(\mathrm{i}\omega)$ \\ \hline
		\end{tabular}
	\end{table}
	
	As any signal can be expressed as the sum of its even and odd components, the first three items above indicate that the spectrum of the even part of a real signal is real and even, and the spectrum of the odd part of the signal is  imaginary and odd. 
	
	\item[P8.] Symmetry (or Duality):
	
	
	Or in a more symmetric form:
	
	\begin{dem} As ${\cal F}(x(t))=X(\mathrm{i}\omega)$, we have:
	
	Letting $t'=-t$, we get:
	
	Interchanging $t'$ and $\omega$ we get:
	
	or:
	
	In particular, if the signal is even:
	
	then we have:
	
	\begin{flushright}
		$\blacksquare$  Q.E.D.
	\end{flushright}
	\end{dem}
	For example, the spectrum of an even square wave is a sinc function, and the spectrum of a sinc function is an even square wave. 
	
	\item[P9.] Multiplication theorem:
	
	
	\begin{dem} 
	
	\begin{flushright}
		$\blacksquare$  Q.E.D.
	\end{flushright}
	\end{dem}
	
	\item[P10.] Parseval's equation\index{Parseval's equation} for the Fourier transform (for more details see page \pageref{Parseval theorem}):
	
	In the special case when $y(t)=x(t)$, the above becomes the Parseval's equation:
	
	where:
	
	is the energy density function, commonly named "\NewTerm{power spectrum}\index{power spectrum}\label{power spectrum}", representing how the signal's energy is distributed along the frequency axes. The total energy contained in the signal is obtained by integrating $S(\mathrm{i}\omega)$ over the entire frequency axes.
	
	The Parseval's equation also indicates that the energy or information contained in the signal is reserved, i.e., the signal is represented equivalently in either the time or frequency domain with no energy gained or lost!
	
	The latter relation is more commonly written:
	
	or more simply:
	
	\begin{tcolorbox}[colframe=black,colback=white,sharp corners]
	\textbf{{\Large \ding{45}}Example:}\\\\
	We have already proved in the section of Sequences and Series that the Fourier transform of the square pulse was given by (see page \pageref{fourier transform pulse square}):
	
	Hence:
	
	\end{tcolorbox}
	
	\begin{tcolorbox}[colframe=black,colback=white,sharp corners]
	The energy spectrum of the pulse we have just calculated shows a great similarity with the Fraunhofer diffraction pattern due to a narrow slit (\SeeChapter{see section Wave Optics page \pageref{fraunhofer diffraction}}). In reality, it is more than a similarity because it is possible to prove in physics that any diffraction pattern is the Fourier transform of the object that is the cause!
	\end{tcolorbox}
	
	
	\item[P11.] Correlation:
	
	The "\NewTerm{cross-correlation}\index{cross-correlation}" of two real signals $x(t)$ and $y(t)$ is defined as (notice that it is strictly equivalent to the definition of the continuous convolution seen just earlier above at page \pageref{continuous convolution}):
	
	Specially, when $x(t)=y(t)$, the above becomes the "\NewTerm{auto-correlation}\index{auto-correlation}" of signal $x(t)$:
	
	Assuming ${\cal F}(x(t))=X(\mathrm{i}\omega)$, we have ${\cal F}(x(t-\tau))=X(\mathrm{i}\omega)e^{-\mathrm{i}\omega\tau}$ and according to multiplication theorem, $R_x(\tau)$ can be written as:
	
	i.e.:
	
	that is, the auto-correlation and the energy density function of a signal $x(t)$ are a Fourier transform pair.
	
	\item[P12.] Convolution Theorems:
	
	The "\NewTerm{convolution theorem}\index{convolution theorem}\label{convolution theorem}" states that convolution in time domain corresponds to multiplication in frequency domain and vice versa:
	
	
	Let us start with the proof of ($a$)!
	
	\begin{dem}
	
	\begin{flushright}
		$\blacksquare$  Q.E.D.
	\end{flushright}
	\end{dem}
	And now let us continue with the proof of ($b$)!

	\begin{dem}
	
	
	\begin{flushright}
		$\blacksquare$  Q.E.D.
	\end{flushright}
	\end{dem}
	
	\item[P13.]  Time Derivative\label{fourier transform time derivative}:
	
	\begin{dem} 
	Differentiating the inverse Fourier transform $X(\mathrm{i}\omega)$ with respect to $t$ we get:
	
	Repeating this process we get:
	
	\begin{flushright}
		$\blacksquare$  Q.E.D.
	\end{flushright}
	\end{dem}
	
	\item[P14.] Time Integration\label{fourier transform time integration}:
	
	First consider the Fourier transform of the following two signals:
	
	
	According to the time derivative property above:
	
	we get:
	
	and:
	
	Why do the two different functions have the same transform?
	
	In general, any two function $f(t)$ and $g(t)=f(t)+c^{te}$ with a constant difference $c^{te}$ have the same derivative $\mathrm{d}\;f(t)/\mathrm{d}t$, and therefore they have the same transform according the above method. This problem is obviously caused by the fact that the constant difference $c^{te}$ is lost in the derivative operation.
	
	To recover this constant difference in time domain, a delta function 
	needs to be added in frequency domain. Specifically, as function $\mathrm{sgn}(t)$ does not have DC component, its transform does not contain a delta:
	
	To find the transform of $u(t)$, consider:
	
	and:
	
	The added impulse term $\pi \delta(\omega)$ directly reflects the constant $c=1/2$ in time domain.
	
	Now we show that the Fourier transform of a time integration is:
	
	
	\begin{dem}
	
	First consider the convolution of $x(t)$ and $u(t)$:
	
	Due to the convolution theorem, we have:
	
	\begin{flushright}
		$\blacksquare$  Q.E.D.
	\end{flushright}
	\end{dem}
	
	\item[P15.] Frequency Derivative:
	
	\begin{dem} We differentiate the Fourier transform of $x(t)$ with
	respect to $\omega$ to get:
	
	i.e.:
	
	Multiplying both sides by $\mathrm{i}$, we get:
	
	Repeating this process we get:
	
	\begin{flushright}
		$\blacksquare$  Q.E.D.
	\end{flushright}
	\end{dem}
	
	\end{itemize}
	
	\pagebreak
	\subparagraph{Usual Fourier transforms}\label{usual Fourier transforms}\mbox{}\\\\
	There are in math and physics many Fourier transforms of signals we see quite frequently (but not exclusively). Furthermore, all Fourier transforms proved below will be used in the various chapters on Physics, Engineering, Atomistic, Social Mathematics, etc of this book. So, as in any formula booklet, we propose you the most Fourier transforms but with... the proofs!
	\begin{enumerate}
	
	\item Impulse:
	
	As shown above:
	
	or:
	
	It is therefore immediate that the inverse Fourier transform of the complex exponential is the Dirac delta:
	
	
	\item Unit Step:
	
	As shown above:
	
	
	\item Constant:
	
	As shown above:
	
	This is a useful formula.
	
	\item Complex exponential:
	
	The spectrum of a complex exponential can be found from the above due to the frequency shift property:
	
	Let us write this more explicitly:
	
	Now let us try to solve this using the physicist way... Using what we have just seen before (the inverse Fourier transform of the Dirac pulse), we have:
	
	it comes obviously after rearranging (multiplying both sides by $2\pi$):
	
	
	\item Sinusoids:
	
	
	Therefore:
	
	Similarly, we have:
	
	
	\item Exponential decay (right-sided):
	
	Therefore:
	
	
	\item Exponential decay (left-sided):
	
	Due to the time reversal property, we also have (for $a>0$):
	
	or:
	
	
	\item Exponential decay (two-sided):
	
	As the two-sided exponential decay is the sum of the right and left-sided 
	exponential decays, its spectrum of $x(t)$ is the sum of their spectra due 
	to linearity:
	
	
	\item Comb function:
	
	The comb function is defined as:
	
	Its Fourier series coefficient is:
	
	and its spectrum is:
	
	We see that the spectrum of an impulse train with time interval $T$ is also an impulse train with frequency interval $\omega_0=2\pi/T$. Also, according to the definition of the Fourier transform, we have:
	
	Therefore we have this equation:
	
	which can be compared with the equation in continuous case:
	
	
	
	\item Square wave (for another detailed derivation see page \pageref{fourier transform pulse square}):
	
	A square wave or rectangular function of width $a$ can be considered as the  difference between two unit step functions:
	
	and due to linearity, its Fourier spectrum is the difference between 
	the two corresponding spectra:
	
	
	\item Sinc function:
	
	The spectrum of an ideal low-pass filter is:
	
	and its impulse response can be found by inverse Fourier transform:
	
	
	\item Triangle function:
	
	As $x(t)$ is an even function, its Fourier transform is:
	
	Alternatively, as the triangle function is the convolution of two square functions
	($a=1/2$), its Fourier transform can be more conveniently obtained according to the
	convolution theorem as: 
	
	
	\item Gaussian function:
	
	The Fourier transform of a Gaussian or bell-shaped function $x(t)=e^{-\pi t^2}$ is:
	
	Here we have used the identity:
	
	We see that the Fourier transform of a bell-shaped function is also a bell-shaped function:
	
	Note that the area underneath either $x(t)$ or $X(\mathrm{i}\omega)$ is unity. Moreover, due to the property of time and frequency scaling, we have:
	
	(Note that if $a=1/\sqrt{2\pi \sigma^2}$, then $a\;x(at)$ above is a normal 	distribution with variance $\sigma^2$ and mean $\mu=0$.) If we let $a \rightarrow \infty$, $x(t)$ becomes narrower and taller and  approaches $\delta(t)$, and its spectrum $e^{-\pi (f/a)^2}$ becomes wider and approaches constant $1$. On the other hand, if we rewrite the above as:
	
	and let $a \rightarrow 0$, $x(t)$ approaches $1$ and $X(\mathrm{i}\omega)$ approaches $\delta(\omega)$.
	
	\end{enumerate}
	
	Now that we have seen quite a number of properties and usual Fourier Transform, let us go back to our heat equation\index{heat equation} determined in the section of Thermodynamics (see page \pageref{heat equation}) in the form:
	
	And let us also be interested in the case where:
	
	To simplify the writing, we will write the differential equation in the following form:
	
	Let us take the Fourier transform relatively to $x$ of this equality. Let us recall that for this purpose we have proved in the section of Sequences and Series and just earlier at page \pageref{fourier transform time derivative} that (time derivative property):
	
	Let us put to simplify the notations:
	
	As we take the Fourier transform with respect to $x$, we can take out the partial derivative of the integral of the Fourier transform such as:
	
	Our differential equation is then reduced to:
	
	Thus explicitly:
	
	We see then immediately with the simplified version that a particular solution is:
	
	The constant has to be determined by the initial condition:
	
	Therefore:
	
	Then it comes by doing the inverse Fourier transform in $x$:
	
	Now let's make a small change of notation by putting:
	
	Then we have:
	
	We then fall much more quickly on the same integral that we obtained in the section of Thermodynamics in our study of the heat equation, the finishing work being the same, the reader can refer to it!

	We had also proved in the section Thermodynamics that we obtained:
	
	We then notice that the Fourier transform is not only a tool for analysing a frequency domain signal but also for solving some differential equations more quickly.
	
	But as many times in mathematics, we must be careful using such a tool. They may be some trap and subtilities. Let us see on famous example!

	We know that the set of solutions of the differential equation:
	
	we $y$ is a continuous function from $\mathbb{R} \mapsto \mathbb{C}$ is made of the set of constant functions!
	
	Let us try to solve this equation using the Fourier transform. By taking the Fourier transform on the left and on the right we get:
	
	An error would be to believe that we can divide left and right by $\omega $ and thus get:
	
	In this case, taking the inverse transform would get $ y = 0 $ as the only solution to the equation, then we would lose all other solutions.
	
	In fact, what we must remember is that the Fourier transform is defined on the space of "temperate distributions" and that therefore, as long as we decide to use this integral transform to solve the differential equation  above, we also decide, implicitly, to look for solutions in this space. Now in the space of temperate distributions, the equation $\omega\mathcal{F}(y)=0$ possesses an infinity of solutions that are given by (without proof) $c\cdot \delta$ where $c$ is a complex number and $\delta$ is the Dirac distribution. As a result:
	
	and taking the inverse transform we get indeed:
	
	We therefore conclude that when we solve differential equations with the Fourier transform it must be remembered that in the space of temperate distributions the usual algebraic calculation rules are to be handled with care. This remark is obviously valid only for people who know the Fourier transform but have only a vague idea of what is a temperate distribution (which is normally the case for engineers....
			
	\paragraph{Discrete Time Fourier Transform}\mbox{}\\\\
	A discrete-time signal can be considered as a continuous signal $x(t)$ 
	sampled at a rate $F=1/t_0$ or $\Omega=2\pi/t_0$, where $t_0$ is the 
	sampling period (time interval between two consecutive samples). The
	corresponding sampling function (comb function) is:
	
	The sampling process can be represented by:
	
	where $x[m]=x(mt_0)$ is the value of $x(t)$ at $t=mt_0$. The Fourier transform of this discrete signal (treated as a special case of continuous signal) is:
	
	This is the forward Fourier transform (analysis) of a discrete signal $x_s(t)$. The spectrum $X(\mathrm{i}\omega)$ is periodic with period $\Omega=2\pi F=2\pi/t_0$:
	
	as :
	
	
	To get back the time signal $x[m]$ from its spectrum:
	
	we multiply the equation by $e^{\mathrm{i}\omega nt_0}/\Omega$ and integrate both sides with respect to $\omega$ over the period $\Omega=2\pi F=2\pi/t_0$	to obtain the inverse Fourier transform (synthesis):
	
	Note that here we used:
	
	which can be compared this with:
	
	To summarize, the spectrum of a given discrete signal:
	
	can be found by the "\NewTerm{forward discrete Fourier transform}\index{forward discrete Fourier transform}\index{discrete Fourier transform}" to be:
	
	and the signal can be expressed by inverse Fourier transform:
	
	It is interesting to compare this discrete time Fourier transform pair with the Fourier series expansion (the Fourier transform of a periodic signal): 
	
	
	with discrete spectrum:
	
	We see symmetry between these two different forms of Fourier transform. If the  signal $x(t)=x(t+T)$ is periodic, its spectrum $X(\mathrm{i}\omega)$ is discrete, the coefficients of the Fourier series with interval $\omega_0=2\pi/T$. On the other hand, if $x(t)$ is discrete with interval $t_0=2\pi/\Omega$, its spectrum $X(\mathrm{i}\omega)=X(\mathrm{i}\omega+\Omega)$ is periodic.
	
	In particular, if the unit of time is so chosen that the sampling period is $t_0=1$, then $\Omega=2\pi/t_0=2\pi$, and the forward Fourier transform of a discrete signal becomes:
	
	The inverse transform becomes:
	
	The spectrum $X(\mathrm{i}\omega)=X(\mathrm{i}\omega+2\pi)$ is periodic.
	
	\begin{tcolorbox}[title=Remark,colframe=black,arc=10pt]
	The spectrum of a time signal (continuous or discrete) can be denoted by $X(\mathrm{i}\omega)$ or $X(f)$ to emphasize the fact that the spectrum represents how the energy contained in the signal is distributed as a function of frequency $\omega$ or $f$. Moreover, if $X(f)$ is used, the factor $1/2\pi$ in front of the inverse transform is dropped so that the transform pair takes a more symmetric form. On the other hand, as Fourier transform of discrete signal can be considered as a special case of Z transform when the real part of $s=\sigma+\mathrm{i}\omega$ is zero, i.e., $z=e^s=e^{\mathrm{i}\omega}$:
	
	it is also natural to denote the spectrum of $x[n]$ by $X(e^{\mathrm{i}\omega})$.
	\end{tcolorbox}
	
	
	\paragraph{Properties of Discrete Fourier Transform}\mbox{}\\\\
	As a special case of general Fourier transform, the discrete time transform  shares all properties (and their proofs) of the Fourier transform discussed above, except now some of these properties may take different forms. In the following, we always assume ${\cal F}[x[m]]=X(e^{\mathrm{i}\omega})$ and ${\cal F}[y[m]]=Y(e^{\mathrm{i}\omega})$. 
	
	\begin{enumerate}
	\item[P1.] Linearity:
	
	
	\item[P2.] Time Shifting:
	
	\begin{dem}
	
	If we let $m'=m-m_0$, the above becomes:
	
	\begin{flushright}
		$\blacksquare$  Q.E.D.
	\end{flushright}
	\end{dem}
	
	
	\item[P3.] Time Reversal:
	
	
	
	\item[P4.] Frequency Shifting:
	
	
	\item[P5.] Differencing:
	
	Differencing is the discrete-time counterpart of differentiation.
	
	\begin{dem}
	
	\begin{flushright}
		$\blacksquare$  Q.E.D.
	\end{flushright}
	\end{dem}
	
	
	\item[P6.] Differentiation in frequency:
	
	
	\begin{dem}
	Differentiating the definition of discrete Fourier transform with respect to 	$\omega$, we get:
	
	\begin{flushright}
		$\blacksquare$  Q.E.D.
	\end{flushright}
	\end{dem}
	
	\item[P7.] Convolution Theorems:
	
	The convolution theorem states that convolution in time domain corresponds to multiplication in frequency domain and vice versa:
	
	
	Recall that the convolution of periodic signals $x_T(t)$ and $y_T(t)$ is:
	
	Here the convolution of periodic spectra $X(f)$ and $Y(f)$ is similarly defined as:
	
	
	Proof of ($a$): 
	
	
	Proof of ($b$):
	
	\begin{flushright}
		$\blacksquare$  Q.E.D.
	\end{flushright}
	
	\item[P8.] Parseval's relation\index{Parseval's relation} for the discrete Fourier transform:
	
	\end{enumerate}
	
	\begin{tcolorbox}[colframe=black,colback=white,sharp corners]
	\textbf{{\Large \ding{45}}Examples:}\\\\
	E1. The discrete Fourier transform of the Dirac delta:
	
	E2. The spectrum of:
	
	is:
	
	If the signal is two-sided:
	
	Due to the time reversal property, its spectrum is:
	
	E3. Consider a LTI system (Linear Time-Invariant) with impulse response:
	
	and input:
	
	The output $y[n]$ can be found in either time domain by convolution or in frequency domain by multiplication. In time domain, we have:
	
	\end{tcolorbox}
	
	\begin{tcolorbox}[colframe=black,colback=white,sharp corners]
	When $a=b$, we have:
	
	In frequency domain, we first find the spectra of both $x[n]$ and $h[n]$ to be:
	
	and the spectrum of the output is:
	
	To find $y(n)$ in time domain by inverse transform of $Y(e^{\mathrm{i}\omega})$, we use partial fraction expansion to rewrite the above as:
	
	By equating the coefficients of $e^{-\mathrm{i}\omega}$ and the constants, we get:
	
	which can be solved to get:
	
	In this form, $Y(\mathrm{i}\omega)$ can be easily inverse transformed to yield:
	
	same as the result from convolution. Again when $a=b$, we have:
	
	But since:
	
	by the frequency differentiation property, we have:
	
	and the output in time domain is obtained as:
	\end{tcolorbox}
	
	\begin{tcolorbox}[colframe=black,colback=white,sharp corners]
	
	Note that the time-shifting property is used due to the factor $e^{\mathrm{i}\omega}$. Also note that $u[n+1]$ (starting at $n=-1$) is replaced by $u[n]$ (starting at $n=0$) as $n+1=0$ when $n=-1$.\\
	
	E4. The impulse response of a discrete LTI system is:
	
	where $|a|<1$ so that the system is stable. The output $y[m]$ of the system with an input:
	
	can be found in three different ways.
	\begin{itemize}
		\item Time domain convolution: 
		The output is the convolution of $x[m]$ and $h[m]$:
		
		
		\item The eigenequation method:\\ 
	
		We first get the frequency response function from $h[m]$:
		
		which is the eigenvalue of the system when the input is a complex exponential $e^{\mathrm{i}n\omega}$. Now the system's response to: 
		
		can be found to be:
		
	\end{itemize}
	
	\end{tcolorbox}
	
	\begin{tcolorbox}[colframe=black,colback=white,sharp corners]
	\begin{itemize}
		\item Frequency domain multiplication:
		
		If we find the spectra of both $h[m]$ and $x[m]$ in the frequency domain, the spectrum of $y[m]$ can be found by multiplication. We already know:
		
		We next find the spectrum of $x[m]$:
		
		Now the spectrum of the output $y[m]$ can be found:
		
		and the output $y[m]$ is obtained by inverse Fourier transform:
		\begin{eqnarray}
		y[m] &=& \frac{1}{2\pi} \int\limits_0^{2\pi} \left[\frac{\pi}{1-ae^{-\mathrm{i} \omega}}
		 \sum_{k=-\infty}^{+\infty} \left[\delta(\omega-2k\pi-\frac{2\pi}{N})+\delta(\omega-2k\pi-\frac{2\pi}{N})\right]\right]
			e^{\mathrm{i}m\omega} \mathrm{d}\omega
			\nonumber \\
		 &=& \frac{1}{2}e^{\mathrm{i}2\pi m/N}\frac{1}{1-ae^{-\mathrm{i}2\pi /N}}
			+\frac{1}{2}e^{-\mathrm{i}2\pi m/N}\frac{1}{1-ae^{\mathrm{i}2\pi /N}}	
			\nonumber
		\end{eqnarray}
	\end{itemize}
	
	The physical meaning of this result will be clear if we write $H(2\pi/N)$ in polar form:
	
	and the output becomes:
	
	That is, the output of the system is also a sinusoidal signal of the same 
	frequency as the input, but with different magnitude $r$ and a phase angle
	$\theta$. For example, if $N=4$, we have:
	
	and the output is:
	
	\end{tcolorbox}
	
	
	
	\StickyNote[2.5cm]{\LARGE To finish depending on donations}[6.5cm]
	
	\pagebreak
	\subsubsection{Laplace Transform}\label{Laplace transform}
	The Laplace transform (LT), as we have already mention it, is a generalization of the Fourier transform (TF), however, although it is so called in his honour because he used it in his work on the theory of probability, seems to have been originally discovered by Leonhard Euler. The Laplace transform also appears in all branches of mathematical physics (mechanical engineering, electronics, quantitative finance, etc.) and is used extensively in order to solve  differential equations that arise in many modelling situations of real life.
	
	As for the Fourier transform, the Laplace transformation allows us to get rid of differentiations. The transform can do this because it has the wonderful property of converting the operation of differentiation into the far simpler one of multiplication. That is, it transforms a differential equation into an algebraic equation. This process is analogous to how logarithms transform multiplication into the simpler operation of
addition (because $\log (xy)=\log(x)+\log(y)$).

	A power series about an origin is any series that can be written in the following form:
	
	where $a_n$ are  numbers and $n$ is a non-negative integer. One can think of $a_n = a(n)$ as a function of $n$ for each non-negative integer $n = 0, 1, 2, \ldots$. In order to give birth to Laplace transformation technique, we  make some associations. The discrete variable $n$ is converted into a real variable $t$. The coefficient term $a_n$ is written as $f(t)$. The term $x^n$ can equivalently be written as $e^{(\ln (x^t))}$. Finally, summation notation can be replaced by its continuous analogue, that is, integration. By doing so, we have following:
	 
For convergence, it is obviously important to have following condition for the above integral (yes think for this about the original sum!):
	 
	Therefore:
	 
	Since $0<x<1 $ so it implies that:
	
	Thus $\ln(x)$ has to be negative for the integral to converge, in this regard, we suppose $\ln(x)=-s$ where $s>0$. Thus, the final integral takes the form:
	
	In this way, we can say that Laplace Transform is simply stretching a discrete (infinite series)  into a continuous (integration) analogue. 
	
	Let us recall that if $s=\mathrm{i}\omega$ (the real part of $s$ is purely imaginary), Laplace transform becomes Fourier transform! In general, any continuous time signal $x(t)$ can be Laplace transformed to get:
	
	provided the integral converges, i.e., the function $X(s)$ exists. This general form of "\NewTerm{bilateral Laplace transform}\index{bilateral Laplace transform}\index{Laplace transform}\label{bilateral Laplace transform}" is related to the Fourier transform:
	
	i.e., Laplace transform of a generic function is Fourier transform of the same function multiplied by $e^{-\sigma t}$. This exponential factor has the effect of forcing the signals to converge. This is why the Laplace transform can be applied to a broader class of signals than the Fourier transform, including exponentially growing signals shown in the following two cases:

	\begin{itemize}
		\item Right sided:
		
		The Fourier transform does not exist as the signal grows exponentially when $t\rightarrow +\infty$, i.e., the transform integral does not converge (not integrable). However, its Laplace transform exists if $\Re[s]=\sigma>a$ (i.e., $\sigma-a>0$), as the modified signal $x(t)e^{-\sigma t}=e^{-(\sigma-a)t}u(t)$ will converge.
		
		\item Left sided:
		 
		Again the Fourier transform does not exist as the signal grows exponentially when $t\rightarrow -\infty$, i.e., the transform integral does not converge. But if $\Re[s]=\sigma<a$ (i.e., $\sigma-a<0$), the modified signal $x(t)e^{-\sigma t}=e^{-(\sigma-a)t}u(-t)$ will converge and its Laplace transform exists.
	\end{itemize}

	In Fourier transform, both the signal $x(t)$ in time domain and its spectrum $X(\mathrm{i}\omega)$ in frequency domain are a one-dimensional (1D) complex function. However, the Laplace transform $X(s)$ of the 1D signal $x(t)$ is a complex function defined over a two-dimensional {\em complex plane}, called the s-plane,	spanned by the two variables $\sigma$ (for the horizontal real axis) and $\omega$ (for the vertical imaginary axis). 
	
	In particular, if this 2D function $X(s)=X(\sigma+\mathrm{i}\omega)$ is evaluated along the imaginary axis $\Re[s]=\sigma=0$, it becomes a 1D function $X(\mathrm{i}\omega)$, the	Fourier transform of $x(t)$. Graphically, the Fourier transform, the spectrum of the signal, can be found as the cross section of the 2D function $X(s)=X(\sigma+\mathrm{i}\omega)$ along the line $\Re[s]=\sigma=0$.
	
	Before we go further another way to introduce the Laplace may be useful and help the reader. But if you have understand the above introduction you can go over this alternate presentation.
	
	The Fourier transform works quite well only when the argument of the integral does not diverge, where the kernel (the multiplication by $e^{-\mathrm{i}\omega t}$) is purely complex, and its integral sweeps the set of reals (bilateral infinity $]-\infty,+\infty[$).

	Typically, mathematicians, that like to generalize stuff...., have generalized the Fourier transform to functions kernel that are not purely imaginary and whose integral could be one-sided $[0,+\infty[$. Thus, the Laplace transform converges for a larger set of functions than the Fourier transform.
	
	Indeed, let us see a simple example with a diverging function:
	
	This last integral does not converge (at least as far as I know ...) whatever the value of $\alpha$ different from zero!
	
	Taking a generalized version of the Fourier transform and denoting it (we already know that it is the bilateral Laplace transform):
	
	with (we know that already!):
	
	where in general, the convergence of the integral is not guaranteed for all $s$. We then name the "\NewTerm{abscissa of absolute convergence}" of the Laplace transform the set of values that $\sigma$ must take for the integral to converge.
	
	Let us take again the non convergent Fourier transform seen just above but where the function is defined only for the positive times only (unilateral Laplace transform):
	
	The result converges if $\omega\geq 0$ (which is always the case in physics) for $t>0$ and if and only if $\sigma>1$. Then in this case we have:
	
	
	\paragraph{Inverse Laplace Transform}\mbox{}\\\\
	The inverse Laplace transform can be obtained from the corresponding Fourier	transform:
	
	The inverse Fourier transform of the above is:
	
	Multiplying both sides by $e^{\sigma t}$, we get the inverse Laplace transform:
	
	As we want to represent the inverse transform in terms of $s$ (instead of $\omega$), we realize that:
	
	We also realize that the integral in the inverse transform is along a vertical line in the $s$-plane from $\mathrm{i}\omega=-\mathrm{i}\infty$ to $\mathrm{i}\omega=+\mathrm{i}\infty$ while the other variable $\sigma$ is fixed. The "\NewTerm{inverse Laplace transform}\index{inverse Laplace transform}" above can therefore be written as:
	
	Now we have the Laplace transform pair:
	
	
	The forward (bilateral) and inverse Laplace transform pair can also be represented as:
	
	
	\pagebreak
	\paragraph{Region of Convergence}\mbox{}\\\\
	An essential issue of Laplace transform of $x(t)$ is whether the transform $X(s)$ even exists, and under what condition it exists. To see this, consider the following examples.
	
	\begin{tcolorbox}[colframe=black,colback=white,sharp corners]
	\textbf{{\Large \ding{45}}Examples:}\\\\
	The Fourier transform of a signal $x(t)=e^{-at}u(t)$ is:
	
	This integral does not converge unless $a>0$. In other words, only when the signal $x(t)$ decays (instead of grows) exponentially, will its Fourier transform  $X(\mathrm{i}\omega)$ exist:
	
	
	Now consider Laplace transform of the same signal:
	
	Similar to the Fourier transform, for this integral to converge, i.e., for Laplace transform $X(s)$ to exist, it is necessary for $\sigma=\Re[s]$ to satisfy:
	
	in which case the Laplace transform is:
	
	As a special case where $a=0$, $x(t)=u(t)$ and we have:
	
	When $ \Re[s]=\sigma=0$, Laplace transform $X(s)$ becomes Fourier transform $X(\mathrm{i}\omega)$. \\
	
	E2. The non-causal version of the signal above is $x(t)=-e^{-at}u(-t)$ and its Laplace transform is:
	
	\end{tcolorbox}
	
	\begin{tcolorbox}[colframe=black,colback=white,sharp corners]
	Only when:
	
	will this integral converge and Laplace transform $X(s)$ exists
	
	Again as a special case when $a=0$, $x(t)=-u(-t)$ we have
	
	Comparing the two examples above we see that two different signals can have identical Laplace transform $X(s)$ but $s$ may have to satisfy different conditions for $X(s)$ to exist. In general, the set of all $s$ values satisfying the conditions for the integral of Laplace transform to converge is called the "\NewTerm{region of convergence}\index{region of convergence}" (ROC) in the complex s-plane. In the first case above, the ROC is $\Re[s]>0$, and in the second case, the ROC is $\Re[s]<0$.\\
	
	E3. Consider the signal:
	
	The Laplace transform of this signal is:
	
	Following example 1, we see that the conditions for the three integrals to converge are, respectively:
	
	i.e., when $\Re[s]>-1$, $X(s)$ exists and can be written as:
	
	\end{tcolorbox}
	
	\begin{tcolorbox}[colframe=black,colback=white,sharp corners]
	E4. Let us consider:
	
	As the Laplace integration converges independent of $s$, the ROC is the entire $s$-plane. In particular, when $T=0$, we have:
	
	\end{tcolorbox}
	\paragraph{Zeros and Poles of Laplace Transform}\mbox{}\\\\
	All Laplace transforms in the above examples are rational, i.e., they can be written as a ratio of polynomials of variable $s$ in the general form:
	
	where $N(s)$ is the numerator polynomial of order $M$ with roots $s_{z_k}, (k=1,2, \cdots, M)$, and $D(s)$ is the denominator polynomial of order $N$ with roots $s_{p_k}, (k=1,2, \cdots, N)$. In general, we assume the order of the numerator polynomial is lower than that of the denominator polynomial, i.e.,  $M < N$. If this is not the case, we can always expand $X(s)$ into multiple terms so that $M<N$ is true for each of terms.	
	\begin{tcolorbox}[colframe=black,colback=white,sharp corners]
	\textbf{{\Large \ding{45}}Example:}\\\\
	Consider:
	
	As the order of the numerator $M=2$ is higher than that of the denominator $N=1$, 
	we expand it into the following terms
	
	and get
	
	Equating the coefficients for terms $s^k$ $(k=0, 1, \cdots, M)$ on both sides, we get
	
	Solving this equation system, we get coefficients
	
	and
	
	Alternatively, the same result can be obtained by carrying out a long-hand division
	
	\end{tcolorbox}
	
	\textbf{Definitions (\#\mydef):}
	\begin{enumerate}
		\item[D1.] Any complex value $s_z$ of $s$ for which $H(s)|_{s=s_z}=H(s_z)=0$ is a "\NewTerm{zero}" of $H(s)$.

		\item[D2.]  Any complex value $s_p$ of $s$ for which $H(s)|_{s=s_p}=H(s_p)=\infty$ is a "\NewTerm{pole}\index{pole}" of $H(s)$.
	\end{enumerate}
	Obviously, all roots of the numerator polynomial $N(s)$ are zeros of $H(s)$ and all roots of the denominator polynomial $D(s)$ are poles of $H(s)$. Moreover, if the order of $D(s)$ exceeds the order of $N(s)$ (i.e., $N>M$), then $H(\infty)=0$, i.e., there is a zero at infinity. On the other hand, if the order of $N(s)$ exceeds that of $D(s)$ (i.e., $M>N$), then $H(\infty)=\infty$, i.e, there is a pole at infinity. On the $s$-plane zeros and poles can be indicated by $o$ and $x$ respectively. Most essential behaviour properties of an LTI system can be obtained graphically from the ROC and the zeros and poles of its transfer function $H(s)$ on the $s$-plane.
	
	\paragraph{Properties of region of convergence}\mbox{}\\\\
	Whether the Laplace transform $X(s)$ of a function $x(t)$ exists depends on whether or not the transform integral converges:
	
	which in turn depends on the duration and magnitude of $x(t)$ as well as the real part of $s$ $\Re[s]=\sigma$. When $x(t)$ is right sided (i.e., $x(t)=0$ for $t<t_0$), it may have infinite duration for $t>0$, and the larger $\sigma$ the more quickly $x(t)e^{-\sigma t}$ decays as $t \rightarrow +\infty$. On the other hand, if $x(t)$ is left sided (i.e., $x(t)$ for $t<t_0$), it may have infinite duration for $t<0$, and the smaller $\sigma$ the more quickly $x(t)e^{-\sigma t}$ decays as $t \rightarrow -\infty$. The imaginary part of $s$ $Im[s]=\mathrm{i}\omega$ determines the frequency of a sinusoid which is bounded and has no effect on the convergence of the integral. Based on these observations, we can get the following properties for the ROC:	
	\begin{itemize}
	
	\item If $x(t)$ is absolutely integrable and of finite duration, then the ROC is the entire $s$-plane (the Laplace transform integral is finite, i.e., $X(s)$ exists, for any $s$).
	
	\item The ROC of $X(s)$ consists of strips parallel to the $\mathrm{i}\omega$-axis in the $s$-plane.
	
	\item If $x(t)$ is right sided and $\Re[s]=\sigma_0$ is in the ROC, then any $s$ to	the right of $\sigma_0$ (i.e., $\Re[s]>\sigma_0$) is also in the ROC, i.e., ROC is a right sided half plane.
	
	\item If $x(t)$ is left sided and $\Re[s]=\sigma_0$ is in the ROC, then any $s$ to the left of $\sigma_0$ (i.e., $\Re[s]<\sigma_0$) is also in the ROC, i.e., ROC is a left sided half plane.
	
	\item If $x(t)$ is two-sided, then the ROC is the intersection of the two one-sided ROCs corresponding to the two one-sided parts of $x(t)$. This intersection can be either a vertical strip or an empty set.
	
	\item If $X(s)$ is rational, then its ROC does not contain any poles (by definition $X(s)|_{s=s_p}=\infty$ dose not exist). The ROC is bounded by the poles or extends to infinity.
	
	\item If $X(s)$ is a rational Laplace transform of a right sided function $x(t)$, then the ROC is the half plane to the right of the rightmost pole, and if $X(s)$ is a rational Laplace transform of a left sided function $x(t)$, then the ROC is the half plane to the left of the leftmost pole.
	
	\item A signal $x(t)$ is absolutely integrable, i.e., its Fourier transform $X(\mathrm{i}\omega)$ exists, if and only if the ROC of the corresponding Laplace transform $X(s)$ contains the imaginary axis $\Re[s]=0$ or $s=\mathrm{i}\omega$.
	
	\end{itemize}
	
	\begin{tcolorbox}[colframe=black,colback=white,sharp corners]
	\textbf{{\Large \ding{45}}Examples:}\\\\
	E1. Consider the Laplace transform of a two-sided signal:
		$x(t)=e^{-b|t|}$:
	
	\end{tcolorbox}
	\begin{tcolorbox}[colframe=black,colback=white,sharp corners]
	The Laplace transform of the two components can be obtained from the two examples discussed earlier above. Then we get:
	
	and let $b=-a$,  we then have:
	
	Combining the two components, we have:
	
	Whether $X(s)$ exists or not depends on $b$. If $b>0$, i.e., $x(t)$ decays exponentially as $|t| \rightarrow +\infty$, then the ROC is the strip between $-b$ and $b$ and $X(s)$ exists. But if $b<0$, i.e., $x(t)$ grows exponentially as  $|t| \rightarrow +\infty$, then the ROC is an empty set and $X(s)$ does not exist.\\
	
	E2. Given the following Laplace transform, find the corresponding signal:
	
	Given the two poles $s_{p_1}=-1$ and $s_{p_2}$ of the expression, there are three associated ROCs: 
	\begin{itemize}
		\item The half plane to the right of the rightmost pole $s_{p_2}=-1$, with the corresponding right sided time function:
		
		\item The half plane to the left of the leftmost pole $s_{p_1}=-2$, with the corresponding left sided time function:
		
		\item The vertical strip between the two poles $-1 < \Re[s] < -2$, with the corresponding two sided time function:
		 
	\end{itemize}
	In particular, note that only the first ROC includes the $\mathrm{i}\omega$-axis and the corresponding time function has a Fourier transform. Fourier transform of the other two functions do not exist.
	\end{tcolorbox}
	
	\pagebreak
	\paragraph{Properties of Laplace Transform}\label{properties of Laplace Transform}\mbox{}\\\\ 
	The Laplace transform has a set of properties in parallel with that of the Fourier transform. The difference is that we need to pay special attention to the ROCs. In the following, we always assume:
	
	and:
	
	
	\begin{enumerate}
		\item[P1.] Linearity:
		
		While it is obvious that the ROC of the linear combination of $x(t)$ and $y(t)$ should be the intersection of the their individual ROCs $R_x \cap R_y$ in which both	$X(s)$ and $Y(s)$ exist, we note that in some cases when zero-pole cancellation occurs, the ROC of the linear combination could be larger than $R_x \cap R_y$, as  shown in the example below.
		
		\begin{tcolorbox}[colframe=black,colback=white,sharp corners]
		\textbf{{\Large \ding{45}}Example:}\\\\
		Assume:
		
		and:
		
		then:
		
		\end{tcolorbox}
		
		\item[P2.] Time Shifting:
		
		
		\item[P3.] Shifting in $s$-Domain:
		
		Note that the ROC is shifted by $s_0$, i.e., it is shifted vertically by $Im[s_0]$ (with no effect to ROC) and horizontally by $\Re[s_0]$. If $R_x$ is $\Re[s]>0$, the new ROC is $\Re[s+s_0]>0$, i.e., $\Re[s]>-\Re[s_0]$.
		
		\item[P4.] Time Scaling:
		
		Note that the ROC is horizontally scaled by $1/a$, which could be either positive ($a>0$) or negative ($a<0$) in which case both the signal $x(t)$ and the ROC of its Laplace transform are horizontally flipped. 
		
		\item[P5.] Conjugation:
		
		\begin{dem} 
		
		\begin{flushright}
			$\blacksquare$  Q.E.D.
		\end{flushright}
		\end{dem}
		
		\item[P6.] Convolution:
		
		Note that the ROC of the convolution could be larger than the intersection of $R_x$ and $R_y$, due to the possible pole-zero cancellation caused by the convolution, similar to the linearity property.
		
		\begin{tcolorbox}[colframe=black,colback=white,sharp corners]
		\textbf{{\Large \ding{45}}Example:}\\\\
		Assume:
		
		then:
		
		\end{tcolorbox}
		
		\item[P7.] Differentiation in Time Domain:
		
		This can be proven by differentiating the inverse Laplace transform:
		
		Again, multiplying $X(s)$ by $s$ may cause pole-zero cancellation and therefore the resulting ROC may be larger than $R_x$. For example, when $x(t)=u(t)$ and $X(s)=1/s$ with $\Re[s]>0$ as the ROC, $\mathrm{d} x(t)/\mathrm{d}t=\delta(t)$ and with $sX(s)=1$ whose ROC is the entire $s$-plane. In general, we have:
		
		
		\begin{tcolorbox}[colframe=black,colback=white,sharp corners]
		\textbf{{\Large \ding{45}}Example:}\\\\
		The ROC of ${\cal L}[\delta(t)]=1$ is the entire s-plane, and we have:
		
		and more generally:
		
		\end{tcolorbox}
		
		\item[P8.] Differentiation in $s$-Domain:
		
		This can be proven by differentiating the Laplace transform:
		
		Repeat this process we get:
		
		
		\item[P9.] Integration in Time Domain:
		
		This can be proven by realizing that:
		
		and therefore by convolution property we have:
		
		Also note that as the ROC of ${\cal L}[u(t)]=1/s$ is the right half plane $\Re[s]>0$, the ROC of $X(s)/s$ is the intersection of the two individual ROCs $R_x \cap \{\Re[s]>0\}$, except if pole-zero cancellation occurs (when $x(t)=\mathrm{d}\delta(t)/\mathrm{d}t$ with $X(s)=s$) in which case the ROC is the entire $s$-plane.
	\end{enumerate}
	
	\paragraph{Usual Laplace transforms}\mbox{}\\\\
	As we always do in this book, let us see some common usual Laplace Transforms that are useful in some physics and finance well known problems (and also used as example in the MATLAB™ companion book):
	\begin{enumerate}
		\item $\delta(t)$, $\delta(t-\tau)$
		
		
		Moreover, due to time shifting property, we have:
		
		
		\item $u(t)$ (Heaviside function\index{Heaviside function}), $t\;u(t)$, $t^n\;u(t)$
		
		Due to the property of time domain integration, we have:
		
		Let us have another more explicit approach as the previous one may be a bit hard.... The function $u(t)$ (for recall it is the Heaviside function defined as $H(t)=0,\forall t<0$) for the usage of the unilateral Laplace transform as it non-null only for $t\geq 0$. Then we have:
		
		This integral converse only if $\sigma \geq 0$ Hence:
		
		
		Applying the $s$-domain differentiation property to the above, we have:
		
		Or more explicitly (it sometimes good to see different ways...!) using integration by part:
		
		By the first term we also see that we must have $\sigma>0$. Which gives us:
		
		
		And in general (useful especially in Mechanical Engineering even if so far we were able to avoid its usage):
		
		Indeed, using integration by parts:
		
		From the first term we also see that we must have $\sigma >0$. Hence it remains:
		
		In then comes under the condition $\sigma>0$:
		
		
		\item $e^{-at}u(t)$, $te^{-at}u(t)$
		
		Applying the $s$-domain shifting property to:
		
		we have:
		
		If you don't like using the shift property for that proof, here is a more explicit and long way to get the same result (we change the notation a bit to avoid any confusion):
		
		We see that the integral will converge if and only if:
		
		Therefore:
		

		Applying the same property to:
		
		we have:
		
		
		\item $e^{-\mathrm{i}\omega_0 t}u(t)$, $\sin(\omega_0 t)u(t)$, $\cos(\omega_0 t)u(t)$ 
		
		We could make it by integration by parts but it's quite long. The tick is letting $a=\pm \mathrm{i}\omega_0$ in:
		
		we get:
		
		and therefore:
		
		and:
		
		
		\item $t\;\cos(\omega_0 t)u(t)$, $t\;\sin(\omega_0 t)u(t)$ 
		
		Letting $a=\pm \mathrm{i}\omega_0$ in:
		
		we get:
		
		Furthermore we have:
		
		and:
		
		
		\item $e^{-at}\cos(\omega_0 t) u(t)$,  $e^{-at}\sin(\omega_0 t) u(t)$  
		
		Applying $s$-domain shifting property to:
		
		and:
		
		we get, respectively:
		
		and:
		
		
		\item \label{Laplace pair for finance and telegrapher equation}Let us see an important transform pair that we will find invaluable when we get to transients in transmission lines and in advanced quantitative finance\footnote{ Notice that the entire development that will follow is inspired from the excellent book: \textit{Transients for Electrical Engineers} \pageref{nahin2018transients}.}. Specifically, if we define the "\NewTerm{Gauss error function}\index{Gauss error function}\index{error function}":
		
		then we will derive here the pair of unilateral Laplace transform:
		
		\begin{dem}
		We will start by computing the Laplace transform of:
		
		How do we know we have to start from this? Because this function appears in the book \textit{Analytical Theory of Heat} (1822) of Joseph Fourier where he already did the job before us but in a different painful way\footnote{You can found it in the English translation of 1878 \cite{fourier2003analytical} at the page 384 for $\mathbb{R}^3$.}!
		
		In any case what we have is:
		
		Next, let us make the change of variable:
		
		and so:
		
		Therefore:
		
		With this our unilateral Laplace transforms becomes:
		
		Or:
		
		where:
		
		Now, concentrate on the integral, alone. Multiplying out the exponent
	of the integrand, we have:
		
		where:
		
		The integral $I(b)$, despite its perhaps complicated appearance, can be evaluated as follows.
		
		Differentiating  with respect to $b$ we get:
		
		Now, make the change of variable:
		
		which leads to:
		
		and so:
		
		Thus:
		
		which the primitive gives:
		
		where $c^{te}$ is some constant. We can determine it easily with (\SeeChapter{see section Statistics page \pageref{Gauss integral}}):
		
		the reader may have indeed recognized the Gauss integral here.
		
		So:
		
		And putting into:
		
		We get:
		
		And putting also back into:
		
		Which give us indeed the pair\label{laplace pair for telegrapher equation} (that will be useful to us for the study of the Telegrapher equation):
		
		Which is equivalent:
		
		Finally, recall that we have proved during our studies of main Laplace transform properties the (\SeeChapter{see section Analysis page \pageref{properties of Laplace Transform}}) following pair (integration in time domain):
		
		Therefore:
		
		In the integral above let us make the following change of variable:
		
		Therefore:
		
		and so:
		
		In the first integral in the square brackets we recognize the half on the Gauss integral. Therefore:
		
		In the second one we recognize the Gauss error function (see page \pageref{error function}), therefore:
		
		Therefore our pair:
		
		can be rewritten:
		
		and so, at last, we have the claimed pair of the beginning!
		\begin{flushright}
			$\blacksquare$  Q.E.D.
		\end{flushright}
		\end{dem}
	\end{enumerate}
	
	\paragraph{Solving differential systems}\mbox{}\\\\
	 Let us see well known simple academic companion example of the application of the Laplace transform to deal with differential equations.
	 
	 Let us consider that a voltage $x(t)$ is applied as the input to a resistor $R$, a capacitor $C$ and an inductor $L$ connected in series (the case without the use of the Laplace transform is introduced in details in the section of Electrical Engineering page \pageref{rlc circuit}). The output $y(t)$ is the voltage across one of the three elements. The system can be described by a differential equation in time domain:
	
	or an algebraic equation in $s$-domain:
	
	and the overall impedance of the circuit is defined as the ratio between voltage $V(s)$ and the current $I(s)$:
	
	which is composed of the individual impedance of the three elements
	

	\begin{table}[H]
		\centering
		\begin{tabular}{l|c|c|c} \hline
			& resistor $R$ & capacitor $C$ & inductor $L$  \\ \hline
		time domain & $i=\frac{v}{R}$ & $i=\frac{1}{C}\frac{\mathrm{d}v}{\mathrm{d}t}$ & $v=\frac{1}{L}\frac{\mathrm{d}i}{\mathrm{d}t}$ 
		\\ \hline 
		$s$-domain    & $V_R=IR$ & $V_C=I/Cs$ & $V_L=IsL$	\\ \hline
		impedance $Z=V/I$   &    $R$   &   $1/sC$   &   $sL$    \\ \hline
		\end{tabular}
	\end{table}	
	If the output is the voltage across one of the three elements ($V_L$, $V_R$, or $V_C$), the transfer function $H(s)$ can be easily obtained by treating the series circuit as a voltage divider: 
	\begin{itemize}
		\item Output is voltage across the capacitor $v_C(t)$
		
		\item Output is voltage across the resistor $v_R(t)$
		
		\item Output is voltage across the inductor $v_L(t)$
		
	\end{itemize}
	If we define
	
	the common denominator of the transfer functions can be written in standard (canonical) form:
	
	with two roots
	
	and the transfer functions above can be written in standard forms:
	\begin{itemize}
		\item 
		
		with two poles $p_1, p_2$ and no zeros. 
		\item 
		
		with two poles $p_1, p_2$ and one zero at the origin. 
		\item 
		
		with two poles $p_1, p_2$ and two repeated zeros at the origin. 
	\end{itemize}
	The three transfer functions behave like low-pass, band-pass and high-pass filter, respectively. Moreover, when the common real part $-\zeta \omega_n$ of the two complex conjugate poles is small (i.e., $0<\zeta < 0.5$), there will be a narrow pass-band around $\omega=\omega_n$ for all three transfer functions. The magnitude and phase angle of the corresponding frequency response function $|H(\mathrm{i}\omega)|$ can be 
	qualitatively determined in the s-plane, as to be discussed later.
	
	\paragraph{Unilateral Laplace Transform}\mbox{}\\\\
	The Laplace transform so far discussed is the "bilateral Laplace transform" as it can be applied to left sided signals as well as right sided ones. Now we will consider the "\NewTerm{unilateral Laplace transform}\index{unilateral Laplace transform}\index{Laplace transform}" of an arbitrary signal $x(t)$ defined as:
	
	where for recall $\cal U$ and $u(t)$ are another typical notation of the Heaviside function.
	
	This definition always assumes $x(t)=0$ for $t<0$. When the unilateral Laplace transform is applied to find the transfer function $H(s)={\cal UL}[h(t)]$ of an LTI system, it is always assumed to be causal. And the ROC is always right sided in $s$-plane.
	
	By definition, the unilateral Laplace transform of any signal $x(t)=x(t)u(t)$ is identical to its bilateral Laplace transform. However, when $x(t) \ne x(t)u(t)$, the two Laplace transforms are different. 
	
	\label{properties of unilateral Laplace Transform}Unilateral Laplace Transform shares all the properties of bilateral Laplace transform, except some of the properties are expressed in different forms. Here we only consider the differentiation in time domain:
	
	\begin{dem}
	
	\begin{flushright}
		$\blacksquare$  Q.E.D.
	\end{flushright}
	\end{dem}
	We can further get Laplace transform of higher order derivatives
	
	and in general:
	

	\StickyNote[2.5cm]{\LARGE To finish depending on donations}[6.5cm]
	
	\pagebreak
	\subsubsection{$\mathcal{Z}$-Transform}
	Laplace transforms (whose Fourier transform is for recall a special case) is applicable only for functions say ... "continuous" to make it simple.... But since the advent of computing the vast majority of functions (signals) are samples at a time interval $T_s$ such that the functions are actually discrete.

	From then on it becomes useful to define a transformation similar to the Laplace transform but in the discrete case! Thus, the Laplace transform becomes a special case of the $\mathcal{Z}$-transform when the sampling period $T_s$ tends to $0$.

	The discrete Fourier transform of a discrete signal $x[n]$ is defined for recall as:
	
	provided $x[n]$ is absolutely summable:
	
	Obviously some signals may not satisfy this condition and their Fourier transform do not exist. To overcome this difficulty, we can multiply the given $x[n]$ by an exponential function $e^{-\sigma n}$ so that $x[n]$ may be forced to be summable for certain values of the real parameter $\sigma$. Now the discrete time Fourier transform becomes:
	
	The result of this summation is a function of a complex variable defined as:
	
	This is the "\NewTerm{forward (bilateral) $\mathcal{Z}$-transform}\index{forward $\mathcal{Z}$-transform}" of the discrete signal $x[n]$:
	
	
	There is another way to introduce the $\mathcal{Z}$-transform that result to the same result but with different notation and that is more explicit. As it may help the reader to better understand, let us show that but by limiting ourselves to positive times only (the case of interest in engineering is the unilateral $\mathcal{Z}$-transform!).

So we know that a continuous function $f(t)t$ can be discretized (sampled) in a way to take at each time step $nT_s$ with $n\in \mathbb{N}$ the value of the function of interest, multiplied by a Dirac pulse on this same point such that:
	
	Now let us calculate the Laplace transform of this discretized function at the limit and using the fact that the $f(nT_e)$ are henceforth independent from the time and the linearity property of the Laplace transform:
	
	As we have shown above in our study of the usual Laplace transforms, we have:
	
	Then we have:
	
	That is usage as we have just seen before to write as following and define as the "\NewTerm{unilateral $\mathcal{Z}$-transform}" (on which we will come back later below):
	
	with:
	
	That's it for the second approach!
	
	Now, given the $\mathcal{Z}$-transform $X(z)$, the original time signal can be obtained by the inverse $\mathcal{Z}$-transform, which can be derived from the corresponding discrete Fourier transform. As shown above, we have:
	
	Now $x[n]e^{-\sigma n}$ can be obtained by the inverse Fourier transform:
	
	Multiplying both sides by $e^{\sigma n}$, we get:
	
	To represent the inverse transform in terms of $z$ (instead of $\omega$), we note:
	
	i.e.:
	
	and the "\NewTerm{inverse $\mathcal{Z}$-transform}\index{inverse $\mathcal{Z}$-transform}" can be obtained as:
	
	Note that the integral with respect to $\omega$ from $0$ to $2\pi$ becomes an integral with respect to $z=e^{\sigma+\mathrm{i}\omega}$ in the complex $z$-plane, along a circle with a fixed radius $e^\sigma$ and a varying angle $\omega$ from $0$ to $2\pi$. Now we have the $\mathcal{Z}$-transform pair:
	
	
	The forward and inverse $\mathcal{Z}$-transform pair can also be represented as:
	
	In particular, if we let $\sigma=0$, i.e., $z=e^{j\omega}$, then the $\mathcal{Z}$-transform becomes the discrete-time Fourier transform:
	
	This is the reason why sometimes the discrete Fourier spectrum is expressed as a function of $e^{\mathrm{i}\omega}$.
	
	
	Different from the discrete-time Fourier transform which converts a 1-D signal $x[n]$ in time domain to a 1-D complex spectrum $X(e^{\mathrm{i}\omega})$ in frequency domain, the $\mathcal{Z}$-transform $X(s)$ converts the 1-D signal $x[n]$ to a complex function defined over a 2-D complex plane, named the  "\NewTerm{$z$-plane}\index{$z$-plane}", represented in polar form by radius $|z|=|e^{\sigma+\mathrm{i}\omega}|=e^\sigma$ and angle 
	$\angle z=\angle(e^{\sigma+\mathrm{i}\omega})=\omega$. 
	
	In particular, when this 2-D function $X(z)=X(e^{\sigma+\mathrm{i}\omega})$ is evaluated along the unit circle $|z|=e^0=1$ corresponding to $\sigma=0$, it becomes a 1-D periodic function $X(e^{\mathrm{i}\omega})$, the discrete Fourier transform of $x[n]$. 
	
	\pagebreak
	\paragraph{Transfer function of LTI system}\mbox{}\\\\
	The output $y[n]$ of a discrete LTI (Linear Time-Invariant) system with input $x[n]$ can be found by convolution (see page \pageref{convolution}):
	
	where $h[n]$ is the impulse response function of the system. In 	particular, if the input is a complex exponential:
	
	then the output $y[n]$ can be found to be:
	
	This is the eigenequation with the complex exponential $x[n]=z^n=e^{sn}$ being the eigenfunction of any discrete LTI system, corresponding to its eigenvalue defined as:
	
	which is the $\mathcal{Z}$-transform of its impulse response $h[n]$, called the "\NewTerm{transfer function}\index{transfer function}" of the LTI system. In particular, when $\sigma=0$, i.e., $z=e^s=e^{\mathrm{i}\omega}$, the transfer function $H(z)$ becomes the frequency response function, the Fourier transform of the impulse response:
	
	
	\paragraph{Conformal mapping between $s$-plane to $z$-plane}\mbox{}\\\\
	The $s$-plane and the $z$-plane are related by a conformal mapping (ie. conserves the angles) specified by the analytic complex function:
	
	where :
	
	Even if, as far as i know, the is no important application of mapping, it is more a mathematical curiosity the may help some students to understand how the $s$-plane and $z$-plane are related together.
	
	The mapping is continuous, i.e., neighbouring points in $s$-plane are mapped to neighbouring points in $z$-plane and vice versa. Consider the mapping of these specific features: 
	\begin{itemize}
		\item The origin $s=0$ of $s$-plane is mapped to $z=e^0=1$ on the real axis in $z$-plane.
		\item Each vertical line $\Re[s]=\sigma_0$ in $s$-plane is mapped to a circle $|z|=e^{\sigma_0}$ centered about the origin in $z$-plane. In particular,
			\begin{itemize}
			\item Leftmost vertical line $\Re[s]=\sigma=-\infty$ is mapped as the origin $|z|=e^{-\infty}=0$
			\item Imaginary axis $\Re[s]=0$ is mapped as the unit circle $|z|=e^0=1$
			\item Rightmost vertical line $\Re[s]=\sigma={+\infty}$ is mapped as a circle of infinite radius $|z|=e^{{+\infty}}={+\infty}$.
			\end{itemize}
		\item Each horizontal line $\Im[s]=\mathrm{i}\omega_0$ in $s$-plane is mapped to $\angle{z}=\omega_0$, a ray from the origin in $z$-plane of angle $\omega_0$ with respect to the positive horizontal direction. 
		\item A right angle formed by a pair vertical and horizontal lines in $s$-plane is conserved by the mapping, as the corresponding circle and ray in $z$-plane also form a right angle (in fact any angle is conserved, an important property of the conformal mapping).
	\end{itemize}
	The infinite range $-\infty < \omega < {+\infty}$ for frequency $\omega$ along a vertical line $\Re[s]=\sigma_0$ in $s$-plane is mapped repeatedly to a finite range $0 \le \omega < 2\pi$ around a circle $|z|=e^{\sigma_0}$ in $z$-plane, corresponding to the conversion of a continuous signal $x(t)$ with non-periodic spectrum $X(\mathrm{i}\omega)$ for $-\infty < \omega < {+\infty}$ to a discrete signal $x[n]$ with periodic spectrum $X(e^{\mathrm{i}\omega})$ for $0 \le \omega < 2\pi$.
	
	\paragraph{Region of Convergence}\mbox{}\\\\
	Whether the $\mathcal{Z}$-transform $X(z)$ of a signal $x[n]$ exists depends on the complex variable $z=e^s$ as well as the signal itself. $X(z)$ exists if and only if the argument $z$ is inside the region of convergence (ROC) in the $z$-plane, which is composed of all $z$ values for the summation of the $\mathcal{Z}$-transform to converge. The ROC of the $\mathcal{Z}$-transform is determined by $|z|=|e^s|=e^{\sigma}$ (a circle), the magnitude of variable $z$, while the ROC for the Laplace transform is determined by $\sigma=\Re[s]$, (a vertical line), the real part of $s$. 
	
	\begin{tcolorbox}[colframe=black,colback=white,sharp corners]
	\textbf{{\Large \ding{45}}Examples:}\\\\
	E1. Let us see our first $\mathcal{Z}$-transform with at the same time its corresponding region of convergence. For this purpose, let us first recall the geometric series (see page \pageref{sum of powers}):
	
	for $|x|<1$. The $\mathcal{Z}$-transform of the following right sided signal $x[n]=a^n u[n]$ is:
	
	\end{tcolorbox}
	
	\begin{tcolorbox}[colframe=black,colback=white,sharp corners]
	
	If we put $a=1$, we simply get the $\mathcal{Z}$-transform of the Heaviside function (as a simple geometric series: $1+z^{-1}+z^{-2}+\ldots$\\
	
	For this summation to converge, i.e., for $X(z)$ to exist, it is necessary to have $| az^{-1} |<1$, i.e., the ROC is $|z| > |a|$. As a special case when $a=1$, $x[n]=u[n]$ and we have:
	
	E2. Let us see now a simple trivial example of an inverse of the following given $\mathcal{Z}$-transform:
	
	Comparing this with the definition of $\mathcal{Z}$-transform:
	
	we get:
	
	In general, we can use the time shifting property:
	
	to inverse transform the $X(z)$ given above to $x[n]$ directly.\\
	
	E3.  Let us determine the obvious inverse transform of:
	
	As we know that:
	
	which converges if the ROC is $|z|>|a|$, i.e., $|az^{-1}|<1$ then we get:
	
	\end{tcolorbox}
	
	\pagebreak
	\paragraph{Zeros and Poles of $\mathcal{Z}$-Transform}\mbox{}\\\\
	All $\mathcal{Z}$-transforms in the above examples are rational, i.e., they can be written as a ratio of polynomials of variable $z$ in the general form:
	
	where $N(z)$ is the numerator polynomial of order $M$ with roots $z_{z_k}, (k=1,2, \cdots, M)$, and $D(z)$ is the denominator polynomial of order $N$ with roots $z_{p_k}, (k=1,2, \cdots, N)$. In general, we assume the order of the numerator polynomial is lower than that of the denominator polynomial, i.e.,  $M < N$. If this is not the case, we can always expand $X(z)$ into multiple terms so that $M<N$ is true for each of terms.
	
	The "\NewTerm{zeros}\index{zeros}" and "\NewTerm{poles}\index{poles}" of a rational $X(z)=N(z)/D(z)$ are defined as:
	
	\textbf{Definitions (\#\mydef):} 
	\begin{enumerate}
		\item[D1.]  Each of the roots of the numerator polynomial $z_z$ for which:
		
		is a "\NewTerm{zero}" of $X(z)$.
	
	  	If the order of $D(z)$ exceeds that of $N(z)$ (i.e., $N>M$), then $X({+\infty})=0$, i.e., there is a zero at infinity:
	  	
	
		\item[D2.] Each of the roots of the denominator polynomial $z_p$ for which :
		
		is a "\NewTerm{pole}" of $X(z)$.
	
		If the order of $N(z)$ exceeds that of $D(z)$ (i.e., $M>N$), then $X({+\infty})={+\infty}$, i.e, there is a pole at infinity: 
	  
	\end{enumerate}
	Most essential behaviour properties of an LTI system can be obtained graphically from the ROC and the zeros and poles of its transfer function $H(z)$ on the $z$-plane.
	
	\paragraph{Properties of $\mathcal{Z}$-Transform}\mbox{}\\\\
	The $\mathcal{Z}$-transform has a set of properties in parallel with that of the Fourier transform (and Laplace transform). The difference is that we need to pay special attention to the ROCs. In the following, we always assume:
	
	and:
	
	
	\begin{itemize}
		\item Linearity:
		
		While it is obvious that the ROC of the linear combination of $x[n]$ and $y[n]$ should be the intersection of the their individual ROCs $R_x \cap R_y$ in which both $X(z)$ and $Y(z)$ exist, note that in some cases the ROC of the linear combination could be larger than $R_x \cap R_y$. For example, for both $x[n]=a^n u[n]$ and $y[n]=a^n u[n-1]$, the ROC is $|z|>|a|$, but the ROC of their difference $a^n u[n]-a^n u[n-1]=\delta[n]$ is the entire $z$-plane.
		
		\item Time shifting:
		
		\begin{dem}
		
		Define $m=n-n_0$, we have $n=m+n_0$ and:
		
		The new ROC is the same as the old one except the possible addition/deletion of the origin or infinity as the shift may change the duration of the signal.
		\begin{flushright}
			$\blacksquare$  Q.E.D.
		\end{flushright}
		\end{dem}
		A slightly different approach lead us to a different result that is more explicit and by the way more useful especially in business applications (we change the notation a bit to be in conformity with the tradition in the financial field). First let us consider the special case (shift of one unit only in the positive direction\footnote{This is then a "unilateral $\mathcal{Z}$-transform" and as we will see it on page , the properties then differ slightly from the bilateral version.}):
		
		Repeating exactly the same type of procedure we get quite simply in the general way for $m>0$:
		
		
		\item Time Expansion (Scaling):
		
		
		The discrete signal $x[n]$ cannot be continuously scaled in time as $n$ has to be an integer (for a non-integer $n$ $x[n]$ is zero). Therefore $x[n/k]$ is defined as:
		
		\begin{tcolorbox}[colframe=black,colback=white,sharp corners]
		\textbf{{\Large \ding{45}}Example:}\\\\
		If $x[n]$ is ramp:
		\begin{table}[H]
		\centering
		\begin{tabular}{c|cccccc} \hline
		 $n$ & 1 & 2 & 3 & 4 & 5 & 6 \\ \hline 
		 $x[n]$ & 1 & 2 & 3 & 4 & 5 & 6 \\ \hline 
		\end{tabular}
		\end{table}
		
		then the expanded version $x[n/2]$ is :
		\begin{table}[H]
		\centering
		\begin{tabular}{c|cccccc} \hline
		 $n$ & 1 & 2 & 3 & 4 & 5 & 6 \\ \hline
		 $n/2$ & 0.5 & 1 & 1.5 & 2 & 2.5 & 3 \\ \hline
		 $m$ &	     & 1 &     & 2 &     & 3 \\ \hline
		 $x[n/2]$ & 0 & 1 & 0 & 2 & 0 & 3 \\ \hline 
		\end{tabular}
		\end{table}
		
		where $m$ is the integer part of $n/k$.
		\end{tcolorbox} 
		
		\begin{dem}
		 The $\mathcal{Z}$-transform of such an expanded signal is:
		
		Note that the change of the summation index from $n$ to $m$ has no effect as the terms skipped are all zeros.
		\begin{flushright}
			$\blacksquare$  Q.E.D.
		\end{flushright}
		\end{dem}
		
		\item Convolution:
		
		The ROC of the convolution could be larger than the intersection of $R_x$ and $R_y$, due to the possible pole-zero cancellation caused by the convolution.
		
		\item Time Difference:
		
		\begin{dem}
		
		Note that due to the additional zero $z=1$ and pole $z=0$, the resulting ROC is the same as $R_x$ except the possible deletion of $z=0$ caused by the added pole and/or addition of $z=1$ caused by the added zero which may cancel an existing pole.
		\begin{flushright}
			$\blacksquare$  Q.E.D.
		\end{flushright}
		\end{dem}
		
		\item Time Accumulation:
			
		\begin{dem}
		The accumulation of $x[n]$ can be written as its convolution with $u[n]$:
		
		Applying the convolution property, we get:
		
		as ${\cal Z}[u[n]]=1/(1-z^{-1})$.
		\begin{flushright}
			$\blacksquare$  Q.E.D.
		\end{flushright}
		\end{dem}
		
		\item Time Reversal:
		
		\begin{dem}
		
		where $m=-n$.
		\begin{flushright}
			$\blacksquare$  Q.E.D.
		\end{flushright}
		\end{dem}
		
		\item Scaling in $z$-domain:
		
		
		\begin{dem}
		
		In particular, if $a=e^{j\omega_0}$, the above becomes:
		
		The multiplication by $e^{-\mathrm{i}\omega_0}$ to $z$ corresponds to a rotation by  angle $\omega_0$ in the $z$-plane, i.e., a frequency shift by $\omega_0$. The rotation is either clockwise ($\omega_0>0$) or counter clockwise ($\omega_0<0$) corresponding to, respectively, either a left-shift or a right shift in frequency domain. The property is essentially the same as the frequency shifting property of discrete Fourier transform.
		\begin{flushright}
			$\blacksquare$  Q.E.D.
		\end{flushright} 
		\end{dem}
		
		\item Conjugation:
		
		\begin{dem}
		Complex conjugate of the $\mathcal{Z}$-transform of $x[n]$ is
		
		Replacing $z$ by $z^*$, we get the desired result.
		\begin{flushright}
			$\blacksquare$  Q.E.D.
		\end{flushright}
		\end{dem}
		
		\item Differentiation in $z$-domain:
		
		\begin{dem}
		
		i.e.:
		
		\begin{flushright}
			$\blacksquare$  Q.E.D.
		\end{flushright}
		\end{dem}
		\begin{tcolorbox}[colframe=black,colback=white,sharp corners]
		\textbf{{\Large \ding{45}}Example:}\\\\
		Taking derivative with respect to $z$ of the right side of:
		
		we get:
		
		Due to the  property of differentiation in $z$-domain, we have:
		
		Note that for a different ROC $|z|<|a|$, we have:
		
		\end{tcolorbox} 
	\end{itemize}
	
	\paragraph{Usual $\mathcal{Z}$-Transforms}\mbox{}\\\\
	\begin{enumerate}

		\item $\delta[n]$, $\delta[n-m]$
		
		Due to the time shifting property, we also have:
		
		
		\item $u[n]$, $a^n u[n]$, $n a^n u[n]$
		
		
		Due to the scaling in $z$-domain property, we have:
		
		Or for those that don't like using this property, we have simply using the previous result:
		
		Applying the property of differentiation in $z$-domain to the above, we have:
		
		
		\item $u[n-k]$

		Using the time shifting property that is for recall given by:
		
		we have:
		
		\begin{tcolorbox}[colframe=black,colback=white,sharp corners]
		\textbf{{\Large \ding{45}}Example:}\\\\
		A useful case for a simple business application that we will use later is:
		
		\end{tcolorbox}
		
		\item $e^{\pm jn\omega_0}u[n]$, $\cos[n\omega_0]u[n]$, $\sin[n\omega_0]u[n]$
		
		Applying the scaling in $z$-domain property to ${\cal Z}[u[n]]=1/(1-z^{-1})$, we have:
		
		and similarly, we have:
		
		Moreover, we have:
		
		Similarly we have:
		
		
		\item $r^n \cos[n\omega_0]u[n]$, $r^n \sin[n\omega_0]u[n]$
		
		Applying the $z$-domain scaling property to the above, we have:
		
		and:
		
	
	\end{enumerate}
	
	
	\paragraph{Unilateral $\mathcal{Z}$-Transform}\mbox{}\\\\
	The "\NewTerm{unilateral $\mathcal{Z}$-transform}\index{unilateral $\mathcal{Z}$-transform}" of an arbitrary signal $x[n]$ is defined as:
	
	where for recall $\cal U$ and $u[n]$ are another typical notation of the Heaviside function.
	
	Some of the properties of the unilateral $\mathcal{Z}$-transform that differ slightly from the bilateral $\mathcal{Z}$-transform are listed below:
	\begin{enumerate}
		\item[P1.] Time Advance:
		
		where we have assumed $m=n+1$.
		
		\item[P2.] Time Delay:
		
		where $m=n-1$. Similarly, we have:
		
		where $m=n-2$. In general, we have:
		
		
		\item[P3.] Convolution:
		
		If both $x[n]$ and $y[n]$ are causal, i.e., $x[n]=y[n]=0$ for $n<0$, the unilateral and bilateral $\mathcal{Z}$ transforms are identical.
		
		\item[P4.] Time Difference:
		
		\begin{dem}
		
		\begin{flushright}
			$\blacksquare$  Q.E.D.
		\end{flushright}
		\end{dem} 
		
		
		\item[P5.] Time Accumulation:
		
		
		\item[P6.] Initial Value Theorem:
		
		If $x[n]=x[n]u[n]$, i.e., $x[n]=0$ for $n<0$, then:
		
		\begin{dem}
		
		All terms with $n>0$ become zero as $z^{-n}=1/z^n \rightarrow 0$ as 
		$z \rightarrow {+\infty}$, except the first one which is always $x[0]$.
		\begin{flushright}
			$\blacksquare$  Q.E.D.
		\end{flushright}
		\end{dem}		
		
		\item[P7.] Final Value Theorem:
		
		If $x[n]=x[n]u[n]$, i.e., $x[n]=0$ for $n<0$, then:
		
		\begin{dem}

		i.e.:
		
		Letting $z\rightarrow 1$ in the above, we get:
		
		where $x[-1]=0$.
		\begin{flushright}
			$\blacksquare$  Q.E.D.
		\end{flushright}
		\end{dem}
	\end{enumerate}
	
	
	\StickyNote[2.5cm]{\LARGE To finish depending on donations}[6.5cm]
	
	\pagebreak
	\subsubsection{Hilbert Transform}\label{hilbert transform}
	The information about the Hilbert transform is often scattered in most textbooks about signal processing. Their authors frequently use mathematical formulas without explaining them thoroughly to the reader at the opposite of the Fourier, Laplace or $\mathcal{Z}$-transform. Our purpose is to make a more stringent presentation of the Hilbert transform but still with the signal processing application in mind.
	
	\textbf{Definition (\#\mydef):} The "\NewTerm{Hilbert transform}\index{Hilbert transform}" of a function $f(t)$ is defined for all $t$ by:
	
	when the integral exists.
	
	It is normally not possible to calculated the Hilbert transform as an ordinary improper integral because of the pole at $\tau=t$. However, the $P$ in front of the integral denotes the "\NewTerm{Cauchy principal value}\index{Cauchy principal value}" which expanding the class of function for which the integral definition above exist. It can be defined as following:
	
	\textbf{Definition (\#\mydef):} The "\NewTerm{Cauchy principal value}\index{Cauchy principal value}" is a method for assigning values to certain improper integrals (\SeeChapter{see section Differential and Integral Calculus page \pageref{improper integral}}) which would otherwise be undefined. Depending on the type of singularity in the integrand $f$, the Cauchy principal value is defined according to the following rules:
	\begin{itemize}
		\item For a singularity at the finite number $b$:
		
		where $b$ is a point at which the behaviour of the function $f$ is such that:
		
	
		\item For a singularity at infinity:
		
		where:
		
	\end{itemize}
	In some cases it is necessary to deal simultaneously with singularities both at a finite number $b$ and at infinity. This is usually done by a limit of the form:
	
	The Cauchy principal value can also be defined in terms of contour integrals of a complex-valued function $f(z)$; $z = x + \mathrm{i}y$, with a pole on a contour $C$. Define $C(\varepsilon)$ to be the same contour where the portion inside the disk of radius $\varepsilon$ around the pole has been removed. Provided the function $f(z)$ is integrable over $C(\varepsilon)$ no matter how small $\varepsilon$ becomes, then the Cauchy principal value is the limit:
	
	\begin{tcolorbox}[colframe=black,colback=white,sharp corners]
	\textbf{{\Large \ding{45}}Example:}\\\\
	Consider the ill-defined expression:
	
	As the primitive of $1/x$ is equal to $\ln(x)$ there is obviously a problem... So the idea is to write instead:
	
	but as the $1/x$ function is odd, we can write this:
	
	that is obviously equal to $0$! So $0$ is the principal value of the initial integral.
	\end{tcolorbox}
	
	\begin{tcolorbox}[title=Remark,colframe=black,arc=10pt]
	Sadly different authors use different notations for the Cauchy principal value of a function $f$, among others:
	
	\end{tcolorbox}

	\StickyNote[2.5cm]{\LARGE To finish depending on donations}[6.5cm]
	
	\pagebreak
	\subsection{Functional dot product (inner product)}\label{functional dot product}
	The "\NewTerm{functional dot product}\index{function dot product}\index{orthogonality of functions}\index{functions orthogonality}" (very strong analogy with the dot product in seen in the section Vector Calculus) may seem unnecessary when examined for the first time outside of an application context or only as generalization purpose, but in fact it has many practical applications. We will make such direct use in the section of Wave Quantum Physics and Quantum Chemistry, or even more important in the context of trigonometric polynomials through the Fourier series and transforms (\SeeChapter{see section Sequences and Series page \pageref{fourier series}}) that we find everywhere in contemporary physics and computer science.
	
	However, if the reader has not travelled the section of Vector Calculus and the part treating the vector dot product, we would highly recommend reading it otherwise what follows may be a little incomprehensible.
	
	We put ourselves in the space $\mathcal{C}([a,b],\mathbb{R})$ of continuous functions in the interval $[a, b]$ into $\mathbb{R}$ with the inner product defined by (we find here again the specific notation of the dot product in its functional - ie integral - version as we had mentioned during our definition of the vector dot product in the section of Vector Calculus):
	
	
	A family of orthogonal polynomials, as we can make the analogy with the dot product in the section Vector Calculus, is therefore a polynomial family $(p_0,...,p_n,...)$ such as:
	
	if $j \ne k$. We recall that an orthogonal family is a free family. We also saw in the section of Vector Calculus that in the space $\mathcal{C}([a,b],\mathbb{C})$ the only possible coherent choice was:
	
	We name the two previous relations "\NewTerm{$L^2$-dot product}\index{dot product!$L^2$-dot product}".
	Remember now that if we want to show something is an inner product, we need to show three things for all $f,g\in \mathbb{R}^2$ and $\alpha\in \mathbb{R}$:
	\begin{enumerate}
		\item Symmetry: $\langle f|g\rangle=\langle g|f\rangle$ (or, if the field is the complex numbers $\mathbb{C}^n$, $\langle f|g\rangle=\overline{\langle g|f\rangle}$, i.e. "conjugate symmetry.)
	
		\item Linearity: $\langle \alpha f|g\rangle=\alpha \langle f|g\rangle$. Notice this also implies $\langle f|\alpha g\rangle=\alpha \langle f|g\rangle$ ($\bar{\alpha}$ in the complex case) by symmetry
	
		\item Positive-definite: $\langle f,f\rangle\geq 0$ with equality if and only if $f=0$, the zero function
	\end{enumerate}
	The first two properties follow directly from the definition of an integral. For the third property, it's quite obvious (the integral of $f^2$ can be zero only if $f$ is zero).
	\begin{tcolorbox}[title=Remark,colframe=black,arc=10pt]
	If such a dot product exist, we can then the Cauchy-Schwarz inequality also applies to it. We name it the "\NewTerm{Cauchy-Schwarz inequality for integrals}\index{Cauchy-Schwarz inequality for integrals}".
	\end{tcolorbox}
	
	Therefore we can build the "\NewTerm{$L^2$-norm}\index{norm!$L^2$-norm}":
	
	\begin{tcolorbox}[title=Remark,colframe=black,arc=10pt]
	We will see further below that the definition above is not the most general one as especially physicists and engineers say that functions are orthogonal under a more constraint situation!
	\end{tcolorbox}
	The development that will follow will remind us the Gram-Schmidt procedure (\SeeChapter{see section Vector Calculus page \pageref{gram-schmidt procedure}}) to build an orthogonal family.
	
	\begin{theorem}
	Given $(p_0,...,p_n,...)$ a family of linearly independent polynomial defined on $[a,b]$ and $V$ the vector space defined by this family. The family $(y_0,...,y_n,...)$ defined by recurrence in the following way:
	
	and $y_0=p_0$ is orthogonal and generates $V$.
	\end{theorem}
	\begin{dem}
	Let us show by induction on $n$ that $(y_0,y...,y_n,...)$ is an orthogonal family which generates the same space as $(p_0,...,p_n,...)$. The assertion holds for $n=0$. Let us suppose that the assertion holds for $n\geq 0$, for $0\leq k\leq n$ we have:
	
	$(y_0,...,y_n,...)$ is therefore orthogonal. Finally, the equality:	
	
	\begin{flushright}
		$\blacksquare$  Q.E.D.
	\end{flushright}
	\end{dem}
	
	\addcontentsline{toc}{paragraph}{Orthogonality of trigonometric functions}
	Let us see a first example very important is signal processing and statistics that is relatively to frequency analysis:
	\begin{tcolorbox}[colframe=black,colback=white,sharp corners]
	\textbf{{\Large \ding{45}}Example:}\\\\
	Let us consider the very important example in modern physics that is the set of continuous $2\pi$-periodic function denoted $P_{2\pi}$ that forms a vector space (\SeeChapter{see section Set Theory page \pageref{vector space}}).\\
	
	We define the dot product of two functions of this set by:
	
	The aim of this definition is to build an abstract functional basis $P_{2\pi}$ on which we can break down any $2\pi$-periodic function!!!\\
	
	The simplest idea is then to use the trigonometric functions sine and cosine:
	
	The relations below show that the basis chosen above are orthogonal and therefore form a free family, plus it's a generating family of the vector space $P_{2\pi}$ because as we have seen in our study of Fourier series (\SeeChapter{see section Sequences and Series page \pageref{fourier series}}), we have the following values:
	
	where $\delta_{km}$ is the kronecker symbol (\SeeChapter{see section Tensor Calculus page \pageref{kronecker symbol}}).\\
	
	Therefore it is also an orthogonal basis but not orthonormal. If we want to normalized the vectors of the basis we just need obviously to take:
	
	\end{tcolorbox}
	\begin{tcolorbox}[title=Remark,colframe=black,arc=10pt]
	If the reader remembers that for a random variable $X$ defined on $\mathbb{R}$, the mean was calculated as (\SeeChapter{see section Statistics page \pageref{expected mean continuous variable}}):
	
	The we can assimilate:
	
	where:
	
	to the expected mean of the function $g(x)$! Analogy sometimes very useful in practice!
	\end{tcolorbox}
	\addcontentsline{toc}{paragraph}{Orthogonality of complex exponential functions}
	Let us see now another example that is an extension of the previous one and that has also a great importance in signal processing but also in quantum physics and quantum chemistry:
	\begin{tcolorbox}[colframe=black,colback=white,sharp corners]
	\textbf{{\Large \ding{45}}Example:}\\\\
	Let us consider a basis of complex functions of the form $r e^{\mathrm{i}n\varphi}$ with $n\in\mathbb{Z}$. We therefore can write:
	
	We get:
	
	It is obvious that if we take for basis functions of the type:
	
	then we have an orthonormal basis (and not just an orthogonal one).
	\end{tcolorbox}
	\addcontentsline{toc}{paragraph}{Orthogonality of Bessel functions}\label{orthogonality of bessel functions}
	Another example that will be useful for us in the section of Wave Mechanics for our study of the ideal circular membrane of a drum.
	\begin{tcolorbox}[colframe=black,colback=white,sharp corners]
	\textbf{{\Large \ding{45}}Example:}\\\\
	We have proved in the section Sequences and Series that the Bessel function\index{Bessel function} $J_p(x)$ satisfies the following differential equation (Bessel's equation):
	
	which can be written as:
	
	The variable $p$ need not be an integer as we will see it in the section of Mechanical Engineering with the study case of the self-buckling column.\\
	
	It turns out to be useful to define a new variable $t$ by $x = a t$, where $a$ is a constant which we will take to be a zero of $J_p$, i.e. $J_p(a) = 0$. Let us define:
	
	which implies:
	
	and substituting into (\ref{eq}) gives:
	
	since $x \mathrm{d}/\mathrm{d}x$ is equivalent to $t \mathrm{d}/\mathrm{d}t$.
	We can also write down the equation obtained by picking another zero, $b$. Defining:
	
	which implies:
	
	we have then:
	
	To derive the orthogonality relation, we multiply (\ref{eq1}) by $v$, and (\ref{eq2}) by $u$. Subtracting and dividing by $t$ gives:
	\end{tcolorbox}
	
	\begin{tcolorbox}[colframe=black,colback=white,sharp corners]
	
	The first two terms in (\ref{combine}) can be combined as:
	
	since the extra terms present in (\ref{totalderiv}), but not in (\ref{combine}), when the derivatives are expanded out are equal and opposite and so cancel. Hence we have:
	
	We next integrate this over the range of $t$ from $0$ to $1$ ($0$ since the Bessel function is not defined for $t<0$ and to $1$ since it's the place where there is a zero by construction for recall!), which gives:
	
	The integrated term vanishes at the lower limit because $t=0$, and it also vanishes at the upper limit because $u(1) = v(1) = 0$, see (\ref{u10}) and (\ref{v10}). Hence, if $a \ne b$, (\ref{int01}) gives:
	
	which, using (\ref{ut}) and (\ref{v10}), can be written
	
	This is the desired orthogonality equation. Remember we require that $a$ and $ b$ are distinct zeroes of $J_p$, so both Bessel functions in (\ref{orthog}) vanish at the upper limit.
	\end{tcolorbox}
	
	\pagebreak
	\addcontentsline{toc}{paragraph}{Orthogonality of Hermite polynomial}
	Another that will be useful for us in the section of Wave Quantum Physics during our study of the harmonic oscillator:
	\begin{tcolorbox}[colframe=black,colback=white,sharp corners]
	\textbf{{\Large \ding{45}}Example:}\\\\
	We will introduce in the section of Wave Quantum Physics the following "physicist Hermite polynomial\index{Hermite polynomials}\label{hermite polynomial}":
	
	Therefore (see the plot in the section of Wave Quantum Physics):
	
	where we notice almost immediately that (useful for further below):
	
	And we need to prove that they are orthogonal (or even better: orthonormal!).\\
	
	For this purpose we introduce the weight function $w(x)=e^{-x^2}$ therefore:
	
	So we get (we use the Gauss integral as proved in the section Statistics page \pageref{Gauss integral}):
	
	\end{tcolorbox}
	
	\begin{tcolorbox}[colframe=black,colback=white,sharp corners]
	Finally, using Kronecker symbol (\SeeChapter{see section Tensor Calculus page \pageref{kronecker symbol}}):
	
	\end{tcolorbox}
	
	\addcontentsline{toc}{paragraph}{Orthogonality of Laguerre polynomial}\label{orthogonality of Laguerre polynomial}
	Now let us see a last example useful for our study of the non-rigid rotator in Quantum Chemistry (radial part solution of the hydrogenoid Schrödinger equation):
	\begin{tcolorbox}[colframe=black,colback=white,sharp corners]
	\textbf{{\Large \ding{45}}Example:}\\\\
	If $L_{m}(x)$ and $L_{n}(x)$ are Laguerre's polynomials (\SeeChapter{see section page \pageref{Laguerre polynomials}}) $m,n$ being positive integers) then:
	
	where
	
	Proof: The generating function for Laguerre's polynomial gives:
	
	We now multiply both sides of the above by $e^{-x}$ and integrate both sides from $0$ to $+\infty$ with respect to $x$, which gives:
	
	Therefore:
	
	\end{tcolorbox}
	
	\begin{tcolorbox}[colframe=black,colback=white,sharp corners]
	
	When $m\neq n$, equating coefficients of $t^{n} s^{m}$ on both sides of the above relation gives:
	
	When $m=n$, equating coefficients of $t^{n} s^{n}$ from both sides of the prior previous relation gives:
	
	Combining we get:
	
	where:
	
	\end{tcolorbox}
	
	\begin{tcolorbox}[title=Remark,colframe=black,arc=10pt]
	\addcontentsline{toc}{paragraph}{Orthogonality of Legendre polynomial}We have already proved in the section of Calculus at page \pageref{legendre polynomials} that the Legendre polynomials (useful for our study of Quantum Chemistry) are orthogonal.
	\end{tcolorbox}
	From what we have seen above we deduce that:
	
	is in fact generalized by:
	
	where $w(x)$ is the "\NewTerm{weight function}\index{weight function}". 
	
	So the engineer, physicist, mathematician must always be careful when he see in a textbook a sentence of the type: \textit{these functions are orthogonal}. Indeed the author/redactor should instead read: \textit{these functions are orthogonal with a given weight}.
	
	\subsubsection{Cauchy-Schwarz inequality for integrals}\label{Cauchy-Schwarz inequality for integrals}
	Let us now provide a detailed proof of what we have mentioned earlier above: the 	"\NewTerm{Cauchy-Schwarz inequality for integrals}\index{Cauchy-Schwarz inequality for integrals}\label{Cauchy-Schwarz inequality for integrals}". As this will be an important result with quite important application in Statistics and Quantum Physics.
	
	\begin{theorem}
	Let $f$ and $g$ be real functions which are continuous on the closed interval $[a,b]$. Then:
	
	\end{theorem}
	\begin{dem}
	For the proof we use the following clever trick! We put $\forall x\in\mathbb{R}$:
	
	Hence using relative sizes of definite integrals:
	
	Using the linear combination of integrals property:
	
	Let us put:
	
	where:
	
	The quadratic equation $Ax^2+2Bx+C$ is non-negative for all $x$. It follows (using the same reasoning as in Cauchy's Inequality) that the discriminant $(2B)^2-4AC$ of this polynomial must be non-positive.
	
	Thus:
	
	and hence the result!
	\begin{flushright}
		$\blacksquare$  Q.E.D.
	\end{flushright}
	\end{dem}
	
	\begin{flushright}
	\begin{tabular}{l c}
	\circled{100} & \pbox{20cm}{\score{4}{5} \\ {\tiny 47 votes,  71.49\%}} 
	\end{tabular} 
	\end{flushright}
	
	%to make section start on odd page
	\newpage
	\thispagestyle{empty}
	\mbox{}
	\section{Complex Analysis}\label{complex analysis}

	\lettrine[lines=4]{\color{BrickRed}B}efore starting this section on the study of differential and integral calculus in the generalized case of complex numbers, I should point out that I used many illustrations inspired by the PDF of E.~Hairer (with his permission). This text also contains many sentences and developments taken, homogenized and simplified from the same PDF (at the risk to make some purist readers climbing to the curtains...) according to the notations and educational objectives of the rest of this book.

	The subject of the complex analysis is the study of functions $\mathbb{C} \mapsto \mathbb{C}$ and their differentiability (which is different from that in $\mathbb{R}^2$). The "holomorphic functions" (that is to say differentiable in a subset of $\mathbb{C}$) have as we will see it later surprising and elegant properties that can be reused in the situation of the special case of functions in $\mathbb{R}^2$ (remember that $\mathbb{C}$ is a generalization of $\mathbb{R}^2$ ) that have important applications in advanced physics (we will use the results of this section for our study of quantum physics and also for some applications of fluid mechanics and also for advanced models in financial options pricing).

	Before we start let us first explain the interest of Complex Analysis in a simplified way!

	We studied in the chapter Algebra a part of the Differential and Integral Calculus with some useful and important theorems in physics and engineering. However, staying in $\mathbb{R}$ or $\mathbb{R}^2$ the list of theorems runs out somehow and we end up finding much relevant tools in practice that allows to simplify the integration calculation that we can sometimes found in industrial applications. So, when we remember that $\mathbb{R} \subset \mathbb{C}$ (thus the set of complex numbers generalizes the set of real numbers) and that we can also build a correspondence $\mathbb{R}^2 \mapsto \mathbb{C}$ as we shall see, then new theorems appear with very interesting results that can be exploit for the integrals in $\mathbb{R}$ or $\mathbb{R}^2$!! It is because of this reason that the engineer needs to know Complex Analysis!
	
	After studying this particular field of mathematics, it is common to say that the shortest path between two truths of the real domain often requires to pass trough the complex domain...

	\subsection{Linear Applications}

	A good introduction to complex analysis and its representation is to look at first (for educational purposes mainly) the special case of complex linear applications. Let us see this!

Let $U \subset \mathbb{C}$ be a set and $V \subset \mathbb{C}$ another set. A function that associates to each $z \in U$ an $w \in V$ such that:
	
is a "\NewTerm{complex function}\index{complex function}":
	
What is important is to remember (\SeeChapter{see section Numbers page \pageref{complex numbers}}) that we can identify:
	
and:
	
We have thus two functions of two real variables $x, y$:
	
which are the coordinates of the point $w$.

\textbf{Definition (\#\mydef):} An application is named "\NewTerm{$\mathbb{C}$-linear}\index{$\mathbb{C}$-linear function}" if for example a function of the type:
	
where $c$ is a fixed complex number and $z$ an arbitrary complex number, satisfying:
	

That is to say that $f(z)$ must me additive and homogeneous or just briefly when this two properties are satisfied we say that $f(z)$ is a "\NewTerm{linear map}\index{linear map}".

We have seen and proved in the section on Numbers during our study of complex numbers, that the multiplication of two complex numbers could be equivalent to an orthogonal rotation followed by a scaling and that this same multiplication could be represented in matrix form! Or the transcription into a matrix form involves as we saw in the section on Linear Algebra automatically  the property of linearity!

So the reader can easily check that a matrix of rotation/scaling is an example of an $\mathbb{C}$-linear application (on request we can detail) that we will now write:
	
Which can be typically represented as follows (we can clearly see a rotation and a scaling which conserve the angles and proportions):

\begin{figure}[H]
\centering
\includegraphics[scale=0.75]{img/analysis/clinear_application.eps}
\caption{$\mathbb{C}$-linear application example}
\end{figure}

It is the fact that the proportions and the angles are kept that makes a complex function $\mathbb{C}$-linear. Otherwise, we would say that the function is $\mathbb{R}$-linear.

So a matrix equation is $\mathbb{C}$-linear if and only if it is of the form:
	
Let us see some examples of quite remarkable $\mathbb{C}$ non-linear functions.

	\begin{tcolorbox}[colframe=black,colback=white,sharp corners]
\textbf{{\Large \ding{45}}Examples:}\\\\
E1. Consider the function:
	
In real coordinates this gives:
		
So let's look what this function do with the points of the complex plane which are coincident with the vertical lines of this same plane (which take us to write $x=a$). Then we have:
	
	and eliminating y, we find the equation of a parabola or rather a family of  parabolas (for several values of $b$) which are open to the left of the pictured complex plane.\\

	Here is a picture representation of the complex plane on which we have drawn a cat head:
	\begin{figure}[H]
		\begin{center}
			\includegraphics[scale=0.75]{img/analysis/c_linear_image_cat.eps}
		\end{center}	
		\caption{Complex representation of the image of the example function}
	\end{figure}
	\end{tcolorbox}

	\begin{tcolorbox}[colframe=black,colback=white,sharp corners]
and if we look at the corresponding pre-image complex plane  then we have two heads of cats that appear:
	\begin{figure}[H]
		\begin{center}
			\includegraphics[scale=0.75]{img/analysis/c_linear_pre_image_cat.eps}
		\end{center}	
		\caption{Pre-image representation of the example function}
	\end{figure}
The appearance of these two heads of cats is that this function has 2 possible pre-images for each image point (so it is a surjective function - see section Set Theory page \pageref{surjective application}).\\

Here is a nice Maple 17.00 script by Carl Ebehart to check this (shame that this can not be done in an easier way in Maple):\\

\texttt{> complextools[gridimage] := proc(p)\\
local llhc, width, height, xres, yres, clrs, V, H, i,j,k,l,pz,x,y,z,f,g,xtcs,ytcs,opts,margs;\\
llhc := [-1, -1];\\
width := 2;height := 2;\\
xres := .25;yres := .25;\\
xtcs := 1; ytcs := 1;\\
clrs := [red, black];\\
opts := NULL;\\
opts := op(select(type,[args],`=`));\\
margs:= remove(type, [args] ,`=`) ;\\
if nops(margs) >1 and margs[2] <> `` then llhc := margs[2] fi:\\
if nops(margs) >2 and margs[3] <> `` then width := margs[3] fi:\\
if nops(margs) >3 and margs[4] <> `` then height := margs[4] fi:\\
if nops(margs) >4 and margs[5] <> `` then xres := margs[5] fi:\\
if nops(margs) >5 and margs[6] <> `` then yres := margs[6] fi:\\
if nops(margs) >6 and margs[7] <> `` then xtcs := margs[7] fi:\\
if nops(margs) >7 and margs[8] <> `` then ytcs := margs[8] fi:}
	\end{tcolorbox}

	\begin{tcolorbox}[colframe=black,colback=white,sharp corners]
\texttt{if nops(margs) >8 and margs[9] <> `` then clrs := margs[9] fi:\\
z:= x + I*y;\\
pz := evalc(p(z));\\
f := unapply(evalc( Re(pz)),x,y); g := unapply(evalc(Im(pz)),x,y);\\
V:= plot( [
seq([seq(op([[f(llhc[1] + i*xres ,llhc[2]+(j-1)*yres/ytcs),g(llhc[1] + i*xres ,llhc[2]+(j-1)*yres/ytcs)], [f(llhc[1] + i*xres , llhc[2] +j*yres/ytcs),g(llhc[1] + i*xres , llhc[2] +j*yres/ytcs)]]),
j=1..ytcs*height/yres)], i = 0 .. width/xres)
],color=clrs[1]);\\
H := plot( [
seq([seq(op([[f(llhc[1]+(j-1)*xres/xtcs,llhc[2] + i*yres),
g(llhc[1]+(j-1)*xres/xtcs,llhc[2] + i*yres)], 
[f(llhc[1] +j*xres/xtcs, llhc[2] + i*yres),
g(llhc[1] +j*xres/xtcs, llhc[2] + i*yres)]]),
j=1..xtcs*width/xres)], i = 0 .. height/yres)
],color=clrs[2]);\\
plots[display]([V,H],scaling=constrained,opts);
end:\\
with(complextools);}\\

\texttt{>plots[display]([seq(plots[display]([gridimage(z->z), gridimage(z->z\string^2)]),i=10)],insequence=true);}
	\begin{figure}[H]
		\begin{center}
			\includegraphics[scale=0.75]{img/analysis/c_linear_maple_transform.eps}
		\end{center}	
		\caption{Practical Maple 17.00 example of simple $C$-linear application}
	\end{figure}
	\end{tcolorbox}
	
	\begin{tcolorbox}[colframe=black,colback=white,sharp corners]
E2. Another interesting feature is the "\NewTerm{Cayley transformation}\index{Cayley transformation}" used in some areas of physics and defined as:
	
having as domain definition: $\mathbb{C}/\left\lbrace 1\right\rbrace$.\\

	We notice that this is an involutive function since:
	
	and as we have proved in the section of Proofs Theory that any involution function is both injective and surjective, then the Cayley transform is a bijective function.\\

This function transforms the imaginary axis $\mi y$ in unit circle (and vice versa as it is involutive). Let us see that:
	
where:
	
satisfies:
	
That is to say:
	
This is the equation of a circle as proven in the section of Analytical Geometric.
	\end{tcolorbox}

\pagebreak
	\begin{tcolorbox}[colframe=black,colback=white,sharp corners]
	E3. As another example of function, consider the "\NewTerm{Joukovski transformation}\index{Joukovski transformation}" defined by:
	
If the definition domain is built in polar coordinates look at how a circle or ellipse transforms with this function:

	\begin{figure}[H]
		\begin{center}
			\includegraphics[scale=0.75]{img/analysis/joukovski_pre_image.eps}
		\end{center}	
		\caption{Transformation into polar coordinates of an ellipse with the example function}
	\end{figure}
Then the image plane will be:
	\begin{figure}[H]
		\begin{center}
			\includegraphics[scale=0.5]{img/analysis/joukovski_image.eps}
		\end{center}	
		\caption{Result of the Joukovski transformation in polar coordinates}
	\end{figure}
It thus transforms the circles respectively centered at $0$ and the rays passing through $0$ into a family cofocal ellipses and hyperbole . To prove this fact, we use the complex polar coordinates (Euler formula) seen in the section on Numbers (\SeeChapter{see chapter Arithmetics page \pageref{euler formula}}):
	\end{tcolorbox}

	\begin{tcolorbox}[colframe=black,colback=white,sharp corners]
	
and:
	
Then we have:
	
therefore:
	
and we immediately see that (\SeeChapter{see section Trigonometry page \pageref{remarkable trigonometric identities}}):
	
has the form of the equation of an ellipse (\SeeChapter{see section Analytical Geometry page \pageref{analytical expression ellipse}}) and we also have:
	
	which is the equation of a hyperbola (\SeeChapter{see section Analytical Geometry page \pageref{hyperbola}}).\\

	This function is useful in case we cleverly place a circle through the point $z=1$ (as in the case of the first figure) the plan represented in polar coordinates with a dotted line might looks like an airplane wing. This allowed a time ago in aerodynamics (but the technique is obsolete today)  to transpose the study of a vector field of an airplane wing profile to the study of a circle profile and to do after the Joukovski transformation.
	\end{tcolorbox}
	
	\pagebreak
	\begin{tcolorbox}[colframe=black,colback=white,sharp corners]
Indeed, let us see a part of this still with Maple 4.00:\\

\texttt{> assume(x,real,y,real);}\\
\texttt{> z:=x+I*y;}\\
\texttt{> F:=1/2*(z+1/z);}\\
\texttt{> u:=Re(F);}\\
\texttt{> u:=evalc(u);}\\
\texttt{> v:=Im(F);}
\texttt{> v:=evalc(v);}\\
\texttt{> with(plots):with(plottools):}\\
\texttt{> p1:=disk([0,0],1,color=black):}\\
\texttt{> p2:=implicitplot({seq(v=b8,b=-10..10)},x=-4..4,y=-2..2,color=black):}\\
\texttt{> display([p2,p1],scaling=constrained);}\\

We thus get:

	\begin{figure}[H]
		\begin{center}
			\includegraphics[scale=0.5]{img/analysis/joukovski_application.eps}
		\end{center}	
		\caption{Important application example of the Joukovski function}
	\end{figure}

	\end{tcolorbox}
	Let us see a last example that shows an electric dipole with its electric field and potential lines (\SeeChapter{see section Electrostatics page \pageref{equipotentials}}) can bee seen as the emergence of the $\mathbb{C}$-linear function $1/z$:

	\begin{tcolorbox}[colframe=black,colback=white,sharp corners]
\textbf{{\Large \ding{45}}Examples:}\\\\
E4. Always with Maple 4.00b we write:\\

\texttt{>assume(x,real,y,real);}\\
\texttt{> z:=x+I*y;}\\
\texttt{> F:=1/z;}\\
\texttt{> u:=Re(F);u:=evalc(u);}\\
\texttt{> v:=Im(F);v:=evalc(v);}\\
\texttt{> with(plots):}\\
\texttt{> p1:=implicitplot({seq(u=a,a=-5..5)},x=-1..1,y=-1..1,numpoints=1000):}\\
\texttt{> p2:=implicitplot({seq(v=b,b=-5..5)},x=-1..1,y=-1..1,numpoints=1000,}\\
\texttt{color=green):}\\
\texttt{> display([p1,p2],scaling=constrained);}\\
	\end{tcolorbox}
	
	\pagebreak
	\begin{tcolorbox}[colframe=black,colback=white,sharp corners]
	\begin{figure}[H]
		\begin{center}
			\includegraphics[scale=0.5]{img/analysis/dipole.eps}
		\end{center}	
		\caption{Another important application of a complex application}
	\end{figure}
	\end{tcolorbox}
	
	\subsection{Holomorphic Functions}\label{holomorphic functions}
	The definition of the derivative with respect to a complex variable is naturally formally identical to the derivative with respect to a real variable.
	
	We then have, if the function $f(z)$ is differentiable in $z_0$:
	
	and we say (abusively in this book) that function is "\NewTerm{holomorphic}\index{holomorphic function}" (while in $\mathbb{R}$ we say "differentiable") or "\NewTerm{analytical}\index{analytical function}" in its domain or in a subset of it if it is differentiable at any point.
	
	In other words a holomorphic functions is a complex-valued function of one or more complex variables that is complex differentiable in a neighbourhood of every point in its domain. The existence of a complex derivative in a neighbourhood is a very strong condition, for it implies that any holomorphic function is actually infinitely differentiable and equal to its own Taylor series.
	
	\begin{tcolorbox}[title=Remarks,colframe=black,arc=10pt]
	\textbf{R1.} A complex function is derived like a real function, we just have to put $z$ as being $x$... at the condition of what we will see in what follows is respected!\\
	
	\textbf{R2.} In fact if the function is holomorphic in a subset of the complex plane, we will see a little further below in our study of the convergence of power series that this is than always an open subset.
	\end{tcolorbox}
	Equivalently, we say that the function $f$ is $\mathbb{C}$-differentiable if the following limit exists in  $\mathbb{C}$:
	
	Let us now present and prove a central theorem for complex analysis named "\NewTerm{Cauchy-Riemann theorem}\index{Cauchy-Riemann theorem}"!
	
	If the function:
	
	is $\mathbb{C}$-differentiable on $z_0=x_0+\mathrm{i}y_0$, then we have:
	
	which is somewhat the equivalent to the Schwarz theorem (limited to $\mathbb{R}$) proved in the section of Differential and Integral Calculus. The above two relations are named "Cauchy conditions". So these are the two conditions that must verify a complex function to be differentiable on $z_0$. Thus, it is possible to use these relations to examine the points where the function is not analytic.
	
	\begin{theorem}
	If these conditions are satisfied (what will prove right below), then we deduce that $u$ and $v$ must both harmonic functions of $x$ and $y$.
	\end{theorem}
	\begin{dem}
	As:
	
	by choosing:
	
	with $x \in \mathbb{R}$ we get:
	
	and as $x$ approaches a small value $\mathrm{d}x$, we have (\SeeChapter{see section Differential and Integral Calculus page \pageref{differential calculus}}):
	
	by choosing:
	
	with $y \in \mathrm{R}$, we get:
	
	and when $y$ tends to a small value we have (\SeeChapter{see section Differential and Integral Calculus page \pageref{differential calculus}}):
	
	So now we have:
	
	But remember we proved in the section of Integral and Differential Calculus the following theorem:
	
	Therefore:
	
	Therefore using directly Schwartz theorem:
	
	Which can be written:
	
	A trivial solution is obviously to have:
	
	Therefore the right to write:
	
	By identifying real and imaginary parts, we finish the proof!
	\begin{flushright}
		$\blacksquare$  Q.E.D.
	\end{flushright}
	\end{dem}
	So for $f$ to be differentiable in the complex domain $\mathbb{C}$ (holomorphic) at a point, it is sufficient that it be differentiable as a function of two real variables ($\mathbb{R}^2$-differentiable on $(x_0,y_0)$) and that its partial first derivatives at this point satisfy the Cauchy-Riemann equations.
	
	But, for it to be $\mathbb{C}$-differentiable, Cauchy-Riemann's equations must valid at all points of the complex plane (we sometimes speaks about "\NewTerm{complete functions}\index{complete functions}") and not only in a subdomain thereof! Otherwise, it contains therefore "\NewTerm{singularities}\index{singularities}" and we then speak of "\NewTerm{meromorphic function}\index{meromorphic function}\label{meromorphic function}" (which is therefore a holomorphic function excepted on singularities points).
	
	The Gamma function studied in the section of Differential and Integral Calculus (see page \pageref{gamma euler function}) is such a well-known function:
	
	\begin{figure}[H]
		\centering
		\includegraphics[scale=0.6]{img/analysis/gamma_meromorphic.jpg}
		\caption[Gamma function is meromorphic in the whole complex plane]{Gamma function is meromorphic in the whole complex plane (source: Wikipedia)}
	\end{figure}
	\begin{tcolorbox}[title=Remark,colframe=black,arc=10pt]
	Geometrically, we will prove later that a holomorphic function has a possible interpretation in the sense that it is a conformal transformation (angles conservation).
	\end{tcolorbox}
	
	Notice therefore that if $f (z)$ is $\mathbb{C}$-differentiable it can be developed as Taylor series (\SeeChapter{see section Sequences and Series page \pageref{taylor series}}):
	
	Note an important thing too. If we rewrite:
	
	as following:
	
	Then we say that $f$ is "\NewTerm{irrotational}\index{irrotational}" (\SeeChapter{see section Vector Calculus page \pageref{irrotational}}) since the first relation can be seen as:
	
	which is an important analogy! Finally, the second equation:
	
	also let us say by analogy (but it stops at a simple analogy!) that the bivariate function $f$ is non-divergent (\SeeChapter{see section Vector Calculus page \pageref{divergence vector field}}) what is good mnemonic way to remember this equation.
	
	Let's also show something else in evidence. If we take the two Cauchy-Riemann equations:
	
	and that we derivate them once again we get:
	
	and that we sum these two relations, we get then:
	
	It is the same with v. Then we have:
	
	And we know very well this form of equations (Maxwell-Poisson equation in the section of Electrodynamics and Newton-Poisson equation in the section of Astronomy...). This is a wave equation also named "\NewTerm{Laplace equation}\index{Laplace equation}" (nothing to do with that of the same name seen in our study of the hydrostatic!) and given by the scalar Laplacian (\SeeChapter{see section Vector Calculus page \pageref{scalar laplacian}}):
	
	Then it is traditional to say that $u$ is harmonic and of course we can get the same result with $v$! Well obviously ... we knew it, since we have already studied in the section Numbers that the real and imaginary parts of a complex number could be put in trigonometric form.

	Thanks to this discovery, Riemann opened the application of holomorphic functions in many problems of physics, since these equations are satisfied by the gravitational potential (Newton-Poisson equation in the section of Astronomy page \pageref{newton-poisson equation}) by electric and magnetic fields (Maxwell-Poisson equation in the section of Electrodynamics page \pageref{maxwell-poisson equation}) by heat balance (no examples yet in this book) and by movements without rotational of certain fluids (no examples either in this book yet).

	
	\begin{tcolorbox}[colframe=black,colback=white,sharp corners]
	\textbf{{\Large \ding{45}}Example:}\\\\
	The potential of a dipole can be described by the following holomorphic function:
	
	The figure below:
	\begin{figure}[H]
		\centering
		\includegraphics{img/analysis/holomorphic_dipole_plot.jpg}
		\caption{Plane representation of a well known holomorphic function...}
	\end{figure}
	shows level-curves (iso-curves) of the given harmonic functions $u (x, y)$ and $v (x, y)$ as real and complex parts of the function $f (z)$ of this example.
	\end{tcolorbox}
	
	
	\pagebreak
	\subsubsection{Orthogonality of real and imaginary iso-curves}
	We will now prove a friendly property that have the functions that satisfy the Cauchy conditions (i.e. that are analytic functions!). Indeed, remember that we have already seen above the function:
	
	which gave the following diagram:
	\begin{figure}[H]
		\begin{center}
			\includegraphics[scale=0.75]{img/analysis/c_linear_image_cat.eps}
		\end{center}	
		\caption{Reminder of plane representation of a complex function seen earlier}
	\end{figure}
	
	\begin{theorem}
	Well he functions satisfying the conditions Cauchy have the simple following geometrical property following: the lines whose real part of the function is constant $\mathcal{R}(f(z))=c^{te}$ and lines whose imaginary part is constant $\mathcal{I}(f(z))=c^{te}$  are orthogonal to each other (think to the trigonometric form of complex numbers it helps to better visualize!).
	
	In other words, the analytical complex functions are transformation functions of an area of the plane into a new plane where the angles are preserved. Then we say that the function is a "\NewTerm{complete transformation}\index{complete transformation}".
	\end{theorem}
	
	\begin{dem}
	For the proof remember that we have proved in section of Vector Calculus that  gradient of a function $f$ of $\mathbb{R}^2$ is given by:
	
	and as part of our study of isolines in the section of Differential Geometry that the tangent vector to isolines of the function $f$ will always be parallel to the vector of the plane:
	
	and that the latter two vectors are perpendicular, such that:
	
	Now assimilate the tangent (parallel) vector $\vec{t}_u$ to the real isolines:
	
	with:
	
	and the normal vector to the imaginary isolines:
	
	with the gradient $v$ of components:
	
	Using the Cauchy conditions proved above, we have for this last relation:
	
	By comparing:
	
	we therefore see that $\vec{t}_u$ and $\vec{\nabla}(v)$ are parallel (collinear). And since $\vec{t}_u$ is colinear the real isolines and that $\vec{\nabla}(v)$  is perpendicular to the imaginary isolines we finished our proof.
	\begin{flushright}
		$\blacksquare$  Q.E.D.
	\end{flushright}
	\end{dem}
	The reader may take as an example the function:
	
	mathematically and schematically detailed earlier above! But to change a little bit, consider an example that will accompany us throughout the rest of this section and that is the following holomorphic function:
	
	That gives us with Maple 4.00b:
	
	\texttt{
	>assume(x,real,y,real);\\
	> z:=1/(1+(x+I*y)\string^2);\\
	> F:=1/z;\\
	> u:=Re(F);\\
	> u:=evalc(u);\\
	> v:=Im(F);\\
	> v:=evalc(v);\\
	> with(plots):\\
	> p1:=implicitplot({seq(u=a,a=-5..5)},x=-5..5,y=-5..5,numpoints=1000):\\
	> p2:=implicitplot({seq(v=b,b=-5..5)},x=-5..5,y=-5..5,numpoints=1000,color=green):\\
	> display([p1,p2]);
	}
	
	which gives:
	\begin{figure}[H]
		\begin{center}
			\includegraphics{img/analysis/holomorphic_isoclines.jpg}
		\end{center}	
		\caption{Representation of an important holomorphic function with its isolines}
	\end{figure}
	
	\subsection{Complex Logarithm}
	We need for all functions built into $\mathbb{R}$ found their equivalent in $\mathbb{C}$ while knowing that if we reduce the case of $\mathbb{R}$ to $\mathbb{C}$ we must get back on our feet!
	
	To do this, let us start with the most classical and academic function which is the logarithm and also the only one function for which we will need the complex version in other sections of this book. As always we will focus only on the properties that we will need later for practical applications and nothing more!
	
	In the same way that we built the logarithm as being by definition by the inverse function of the natural exponential $e^x$ in the section of Functional Analysis, we first start from:
	
	where $z$ is a complex number and we will define the complex logarithm that must be reduced to the natural logarithm if $z$ has no imaginary part!
	
	So by definition the complex logarithm will be:
	
	and in this entire book, the complex logarithm will be differentiated by the real logarithm by a capital L for the first letter!
	
	Let us write $z$ and $w$ in the Euler form as viewed in the section Numbers:
	
	Then we have:
	
	By correspondence, we find immediately
	
	with $k \in \mathbb{Z}$. Therefore we get:
	
	Therefore:
	
	or more explicitly:
	
	So if $w$ has no imaginary part, we fall back on our feet since $\text{arg} (w)$ becomes zero.
	
	A big difference is highlighted between the logarithm of the complex and real numbers: the complex numbers logarithms can take several values because of the argument!!
	
	For a function to have an inverse, it must map distinct values to distinct values, i.e., be injective. But the complex exponential function is not injective, because $e^{z+2\pi \mathrm{i} }= e^z$ for any $z$, since adding $\mathrm{i}\theta$ to $w$ has the effect of rotating $e^z$ counter-clockwise $\theta$ radians. So all the points of the form $z+\mathrm{i}k\theta$  are all mapped to the same number by the exponential function. So the exponential function does not have an inverse function in the standard sense.
	
	There are two solutions to this problem.
	
	\begin{enumerate}
		\item One is to restrict the domain of the exponential function to a region that does not contain any two numbers differing by an integer multiple of $2\pi \mathrm{i}$: this leads naturally to the definition of "\NewTerm{branches}\index{branches}" of $\text{Log}(w)$, which are certain functions that single out one logarithm of each number in their domains.
		
		\item Another way to resolve the indeterminacy is to view the logarithm as a function whose domain is not a region in the complex plane, but a "Riemann surface" that covers the punctured complex plane in an infinite-to-$1$ way.
	\end{enumerate}
	Branches have the advantage that they can be evaluated at complex numbers. On the other hand, the function on the Riemann surface is elegant in that it packages together all branches of $\text{Log}(w)$ and does not require an arbitrary choice as part of its definition.
	
	We can see this with Maple 4.00b easily:
	
	\texttt{>plot3d([r*cos(f),r*sin(f),f],r=0..1,f=-2*Pi..2*Pi,axes=boxed,style=patch,\\
	shading=ZHUE);}
	
	which gives:
	\begin{figure}[H]
		\begin{center}
			\includegraphics{img/analysis/complex_logarithm.jpg}
		\end{center}	
		\caption{Complex Logarithm plot with Maple 4.00b}
	\end{figure}
	
	For this reason, one cannot always apply $\text{Log}$ to both sides of an identity $e^{z_1}=e^{z_2}$ to deduce $z_1=z_2$ . Also, the identity $\text{Log} (z_1z_2)= \text{Log}(z_1) + \text{Log}(z_2)$ can fail: the two sides can differ by an integer multiple of $2\pi \mathrm{i}$.
	
	For each non-zero complex number $w = x + i\mathrm{y}$, the principal value $\text{Log}(w)$ is the logarithm whose imaginary part lies in the interval $[-\pi,+\pi]$. The expression $\text{Log}(0)$ is left undefined since there is no complex number $z$ satisfying $e^z = 0$.
	
	Then the principal value of the complex logarithm can be defined by (\SeeChapter{see section Trigonometry page \pageref{trigonometry}}):
	
	We see also obviously that the function $\text{Log}(w)$ is discontinuous at each negative real number (we can see it on the figure above), but continuous everywhere else in $\mathbb{C}^*$.
	
	Riemann surfaces\index{riemann surfaces} can be thought of as deformed versions of the complex plane: locally near every point they look like patches of the complex plane, but the global topology can be quite different. For example, they can look like a sphere or a torus or a couple of sheets glued together.
	
	The main point of Riemann surfaces is that holomorphic functions may be defined between them. Riemann surfaces are nowadays considered the natural setting for studying the global behaviour of these functions, especially multi-valued functions such as the square root and other algebraic functions, or the logarithm.
	
	Basically, a Riemann surface is simply just a surface, as far as the shape is concerned. Any normal surface you can think of (for example, plane, sphere, torus, ...) are all Riemann surfaces. We say it's Riemann surface, is due to the context, is that we define the surface using complex functions, and for use in studying complex functions.
	
	\subsection{Complex Integral Calculus}
	We have seen just above how to check if a complex function $f (z)$ was differentiable (it must at least respect the Cauchy-Riemann equations) at any point.
	
	Now let us see the opposite case... the integration that is absolutely fascinating in complex plane!
	
	We have obviously taking again the notations of the section of Differential and Integral Calculus:
	
	either in explicit form:
	
	Well once this expression established, let us give a little explanation about how to read it:
	\begin{enumerate}
		\item We know that $u$ and $v$ are dependent both in the general case of $x$ and $y$.
		
		\item We know that $ u $ and $ v $ represent (see examples at the beginning of this section) closed or open curves and also straight lines when $ x $ (or respectively $y$) is fixed and that the other associated variable varies!
	\end{enumerate}
	So each term have an integral in the above expression is in fact a line integral on a family of open or closed curves (including a specific case that is straight lines...)!
	
	This integral can be evaluated using the Green's theorem in the plane (\SeeChapter{see section Vector Calculus page \pageref{green theorem}}) if we consider the particular case of a closed curvilinear path such as:
	
	Let us first study the real part:
	
	Indeed we proved (it is strongly advised to read again this Green's theorem) in the section of Vector Calculus that:
	
	What will be written in our situation:
	
	However, if the function is holomorphic and thus satisfies the Cauchy-Riemann equations we get immediately:
	
	Thus our integral is reduced in the particular case of a closed path:
	
	and... reusing Green's theorem for this imaginary part:
	
	However, if the function is holomorphic (for reminder that is to say differentiable at every point of the complex plane or an open subset of it) and thus satisfies the Cauchy-Riemann equations we get immediately:
	
	and we thus obtain the "\NewTerm{Cauchy theorem}\index{Cauchy theorem}", or "\NewTerm{Cauchy-Goursat theorem}\index{Cauchy-Goursat theorem}" for its generalized version for non continuous functions, which says that if a function is holomorphic (thus satisfying the Cauchy-Riemann equations) and integrated on a closed contour then:
	
	As a corollary (without proof), any function that satisfies the above relation is holomorphic (in the whole complex plane or an open subset of it).
	
	This result gives the possibility in certain fields like quantum physics fields (we think of the Yukawa potential that is not yet treated in this book in detailed) to calculate complicated real definite integrals using the above property. The idea is when choosing the closed contour of the path integral to play to make the real definite integral only as a part only of the path (by generalizing to the complex case) and by equality with zero we deduce its value thanks to the other parts of the integrals of the chosen path (parts that are obviously simple to calculate).
	
	In other words, the idea is to calculate by difference! The difficulty residing in practice in finding the function $f (z)$ and the closed contour that permits to make appear the function $f (x) $ of the researched definite integral...
	
	Using this result, let us make a very important academic example which will be useful later (but who has no connection with the case of calculating a real definite integral).
	\begin{tcolorbox}[colframe=black,colback=white,sharp corners]
	\textbf{{\Large \ding{45}}Example:}\\\\
	Let us calculate:
	
	For this purpose, we will use the simplification that consist to remember (\SeeChapter{see section Numbers page \pageref{euler formula}}) that:
	
	Therefore:
	
	We can then write the path integral as:
	
	Or as on a closed path differentiable at any point (without nodes) the angle to make a full turn will necessarily be between $0$ and $+\pi$. It comes then:
	
	\end{tcolorbox}
	Before we continue by noticing a very interesting and important fact that we will detail later formally: An integral (we do not speak of primitive but of integral!) of the type $1/x$ in $\mathbb{R}$ would not be calculable. But now if we generalize the concept of $\mathbb{C}$, we see that we get go around the singularity via a path integral that enclose the singularity. And ... and ... in our previous calculation $z$ might have only the real value and not the imaginary one (so $z$ reduce to $x$). So the integral of $1/x$ becomes calculable and has a result in the set of complex numbers which is remarkable!
	
	Some mathematicians interpret this by figuring that $1/x$ is a flat projection of a three-dimensional space in which the imaginary axis is perpendicular to the plane $\mathbb{R}^2$ (see figures below). Hence the fact that $1/x$ can integrated in the set $\mathbb{C}$.
	
	Finally, let us indicate that $1/z$ is holomorphic on the whole complex plane except on $0$ (the derivative being the same as $1/x$). Then the function $1/z$ is thus not $\mathbb{C}$-differentiable!
	
	This being done, let us do an important and similar case with the following path integral:
	
	where $z_0$ is a constant complex number. Let us write:
	
	We can then write if we make one turn counter clockwise:
	
	which is valid only if our integration path avoids $z_0$ what otherwise there is a singularity. This latter integral is a little simplistic generalization of the previous one.
	
	Now let us show the important theorem that interests us since the beginning of this section using many proven results so far!
	
	We know that if a function $f (z)$ satisfies the Cauchy-Riemann equations, the if we carefully avoid the value $z_0$ (as in the above calculations), the expression:
	
	is differentiable at all points except in on $z_0$ (where the expression is no longer holomorph) is name a "\NewTerm{singularity}\index{singularity}".

	Indeed, take a holomorphic function $f (z)$ satisfying Cauchy-Riemann equation and subtract a constant ($f(z_0)$) does not change the fact that the expression (in this case the numerator in previous relationship) remain holomorphic. Finally, multiply it by a fraction (the denominator of the above equation) which is also holomorph gives a holomorphic function. But singularities can then appear, we then speak of "\NewTerm{meromorphic functions}\index{meromorphic functions}" (this is the ratio of two holomorphic functions).
	\begin{tcolorbox}[title=Remark,colframe=black,arc=10pt]
	A meromorphic function is a function holomorphic in the whole complex plane, except possibly on a set of isolated points each of which is a pole (singularity) for the function (see further below for the concept of pole/singularity). The gamma function (see the plot in the Differential and Integral calculus section) is a famous example of meromorphic function!
	\end{tcolorbox}	
	So if we take the path integral on a closed path avoiding $z_0$, the Cauchy theorem gives us immediately (remember the proof above):
	
	However, this can also be written after rearrangement of terms:
	
	Therefore:
	
	But we have proved above that:
	
	Then we get the result named "\NewTerm{Cauchy's integral theorem}\index{Cauchy's integral theorem}", or more rarely "\NewTerm{Cauchy formula}\index{Cauchy formula}" (of which there is a generalized result we will prove later below):
	
	In fact, in practice all the subtlety is to be able to take back a given holomorphic function $g(z)$ (which therefore satisfies the Cauchy-Riemann equations) by manipulating it in a form of the type:
	
	when its possible... then the calculation of its path integral (closed path) becomes extremely simple since it will be equal to:
	
	by the Cauchy's integral theorem!
	
	\begin{tcolorbox}[title=Remark,colframe=black,arc=10pt]
	So we know how to calculate the value of a path integral of an expression that is not holomorph but for which the numerator is holomorph! 
	\end{tcolorbox}
	
	\begin{tcolorbox}[colback=red!5,borderline={1mm}{2mm}{red!5},arc=0mm,boxrule=0pt]
	\bcbombe Caution! The sign of the value of a path integral will depend on the direction in which its integration path will be done. If the direction is straightforward (that is to say "counter-clockwise") its sign will be positive; if on the contrary the direction is clockwise his sign will be negative. You probably think that this information is irrelevant since this value is usually zero. Yes... it is, but we will see later the importance of this information when referring to the calculation of what we name the "residuals".
	\end{tcolorbox}
		
	\begin{tcolorbox}[colframe=black,colback=white,sharp corners]
	\textbf{{\Large \ding{45}}Example:}\\\\
	An important application example is named the "\NewTerm{Gauss' mean value theorem}\index{Gauss' mean value theorem}" that states given $f(z)$ an analytic function on and inside a disk $C_r:|z-z_0|=r$, then:
	
	Indeed, from the Cauchy's integral relation we have (we simply make the change of variable $z=z_0+re^{\mathrm{i}\theta}$):
	
	\end{tcolorbox}
	 There is a similar relation for the derivative $f'(z_0)$ to that given by the Cauchy's integral theorem. Let us see this:
	
	Therefore:
	
	thereby continuing, we have:
	
	In short, we therefore note that:
	
	which is the "\NewTerm{Generalized Cauchy's integral theorem}\index{Generalized Cauchy's integral theorem}\index{Cauchy's integral}".
	
	This result is very powerful because it shows that holomorphic functions are infinitely differentiable (because of the denominator), that is to say analytical, and it is much more difficult to find an equivalent theorem with such simple conditions for real functions.
	
	If we now return to our Taylor expansion of a complex function:
	
	um ... and what do we see here? Well this !:
	
	It follows the following relation named "\NewTerm{Laurent series in positive powers}\index{Laurent series in positive powers}" (there is a more generalized version will be prove later below):
	
	that gives us the formal expression of a complex function in the form of infinite series of integer powers near a point $z_0$ of the complex plane with therefore:
	
	Remembering that $d^{n}f(z_0)/\mathrm{d}z^n$ can be written equivalently $f^n(z_0)$, we see that all the two previous relations gives us the Taylor series expansion that we had obtained in real analysis (\SeeChapter{see section Sequences and Series page \pageref{taylor series}}) and that was:
	
	Thus, the Taylor series in $\mathbb{R}$ are a special case of Laurent series that are in $\mathbb{C}$!!!
	
	This result is quite remarkable because it also shows that we can use the path integral in the complex plane for calculating the coefficients $c_n$ of the Laurent series instead of calculating the derivatives of order $n$ of the function $f$ if these latter are too complicated to determine. Or vice versa... calculate a simple derivation instead of calculating a headache type path integral (typically the case in physics) using the fact that:
	
	The only unfortunate point being that the latter relation is calculable only if we can put the function in path line integral in the form:
	
	where $n$ is a positive or null integer. This is honestly far from to be easy in most cases! The idea would be to find a general path for line integral, valid for any function $f (z)$ such that the denominator (which additionally contains a singularity on $z_0$) disappears. That would be ideal ... but we need a track ... and it will come from the study of the convergence of series of complex powers. Let's see what it is with a qualitative approach!
	
	\pagebreak
	\subsubsection{Convergence of a complex series}
	We saw in the section of Sequences and Series that many real functions could be expressed in Maclaurin series (special case of the Taylor series on $x_0=0$) in the form:
	
	We also showed, by example the only, that this series expansion of infinite powers was valid for some real functions only in a certain domain of definition named "radius of convergence".
	
	Even if this radius of convergence can be determined more or less easily in each case, there are some baffling examples that could not in the early 19th century be understood without complex analysis.
	
	Let's see a simple example to understand what kind of problem it is. Consider for this the two functions:
	
	and before continuing our example, recall that we have proved in the section of Sequences and Series the relation:
	
	relative to a geometric series, that is to say a series whose terms are of the type:
	
	Therefore it comes immediately if $n \rightarrow +\infty$ and $q \in ]-1,+1[$:
	
	if $u_0=1$, we get:
	
	So if we change the notation, we have\label{sum of powers}
	
	Then it comes immediately:
	
	Therefore the two previous functions $g(x)$ and $h(x)$ are defined for a infinite series expansion in powers only in radius of convergence $x \in ]-1,+1[$.
	
	We would get the same result by making a Maclaurin series expansion!
	
	We see that there is trivially for $g(x)$ two singularities that are $x=\left\lbrace -1,+1\right\rbrace$ by cons, basically we do not see trivial singularities for $h (x)$ if we reason only in $\mathbb{F}$ so it can be hard for the latter function to understand the origin of the radius of convergence!
	
	Indeed, if we plot these two functions in $\mathbb{R}$ with Maple 4.00b we get respectively:
	\begin{figure}[H]
		\begin{center}
			\includegraphics{img/analysis/g_h_example_functions.jpg}
		\end{center}
	\end{figure}
	hence the problem of why there is still implicitly a radius of convergence $x \in ]-1,+1[$ for $h (x)$???
	
	An even more blatantly way to highlight the problem, is to show the approach of these two functions by a Maclaurin series expansion with ten terms.
	
	For $g(x)$ we get for example:
	
	\texttt{>with(plots):\\
	>xplot:= plot(1/(1-x\string^2),x=-5..5,thickness=2,color=red):\\
	>tays:= plots[display](xplot):\\
	>for i from 1 by 2 to 10 do\\
		tpl:= convert(taylor(1/(1-x\string^2), x=0,i),polynom):\\
		tays:= tays,plots[display]([xplot,plot(tpl,x=-5..5,y=-2..2,\\
		color=black,title=convert(tpl,string))])\\
		od:\\
	>plots[display]([tays],view=[-5..5,-2..2]);}
	
	\begin{figure}[H]
		\begin{center}
			\includegraphics{img/analysis/g_function_expansion_inspection.jpg}
		\end{center}	
		\caption[]{Plane representation of the function $g$ to visualize the problem}
	\end{figure}
	where we see well that the Maclaurin series (or expression in power series) does not converge outside $x \in ]-1,+1[$  which can be intuitive because of both singularities.
	
	For $h(x)$ we have by cons:
	
	\texttt{
	> with(plots):\\
	> xplot:= plot(1/(1+x\string^2),x=-5..5,thickness=2,color=red):\\
	> tays:= plots[display](xplot):\\
	> for i from 1 by 2 to 10 do\\
		tpl:= convert(taylor(1/(1+x\string^2), x=0,i),polynom):\\
		tays:= tays,plots[display]([xplot,plot(tpl,x=-5..5,y=-2..2,\\
		color=black,title=convert(tpl,string))])\\
	od:\\
	> plots[display]([tays],view=[-5..5,-2..2]);\\
	}
	
	\begin{figure}[H]
		\begin{center}
			\includegraphics{img/analysis/h_function_expansion_inspection.jpg}
		\end{center}	
		\caption[Divergent Maclaurin series]{Surprisingly, here the Maclaurin series (in black) does not converge}
	\end{figure}
	where we see well that the Maclaurin series (or the expression in power series) does not converge either outside $x \in ]-1,+1[$ which was unsettling and against-intuitive at the beginning of the history of real analysis.
	
	Today even a high school student knows that he can also think in $\mathbb{C}$ and that $\mathbb{R} \subset \mathbb{C}$. So the real analysis is just a special case and restricted of the field of complex analysis. The fact to extend the domain of a given analytic function is named "\NewTerm{analytic continuation}\index{analytic continuation}". As we will see just now analytic continuation often succeeds in defining further values of a function, for example in a new region where an infinite series representation in terms of which it is initially defined becomes divergent!
	
	The singularity for $h (x)$ in $\mathbb{C}$ comes that latter is then  written:
	
	and there are therefore two singularities $z=\left\lbrace{-\mathrm{i},+\mathrm{i} }\right\rbrace$ that we see well if we represent:
	
	with Maple 4.00b (fortunately we now have the equivalent of a microscope in mathematics with Maple...):
	
	\texttt{>plot3d(abs(1/(1+(re+I*im)\string^2)),re=-3..3,im=-3..3,view=[-2..2,-2..2,-2..2]\\
	,orientation=[-130,70],contours=50,style=PATCHCONTOUR,axes=frame,\\
	grid=[100,100],numpoints=10000);}
	
	\begin{figure}[H]
		\begin{center}
			\includegraphics{img/analysis/g_inspection_in_C.jpg}
		\end{center}	
		\caption{Complex representation of the function $h$ to highlight the reason for the divergence}
	\end{figure}
	where we can see the two singularities on the imaginary axis and the function $h (x)$ on the real axis (between the two peaks). So when we develop a function in power series, we conclude that the radius of convergence is defined by the whole complex plane and not by the traditional axis of the real analysis.
	
	This makes it more natural to understand why we were talking in section of Sequences and Series of "radius" as seen from above, we have in the complex plane:
	\begin{figure}[H]
		\begin{center}
			\includegraphics{img/analysis/h_various_radius_convergence.jpg}
		\end{center}	
		\caption{Representation of the various convergence of radius of $h(z)$}
	\end{figure}
	hence the fact that we are talking sometimes about (open) convergence disk and sometimes of (open) convergence radius. Moreover, we notice on the chart that the domain of convergence is convex (any couple of points of the domain can be connected by a straight line that is in the area of convergence).
	\begin{tcolorbox}[title=Remark,colframe=black,arc=10pt]
	Let us Recall that a subset, interval or "open" disc means that we do not take its border as we have seen in the section Topology.
	\end{tcolorbox}
	Then we understand better why the Taylor series does not converge trivially for $h(x)$: it must converge on the whole disc of the complex plane and not just converge on the real axis!
	
	From all this we deduce that our Laurent series in positive powers proved above:
	
	not necessarily converge, unsurprisingly... on the whole complex plane (just like the Taylor series on the real line as this is the equivalent!) but sometimes only in a opened subdomain (convex?) of this plane around $z_0$ (which in the particular example taken above was obviously: $0$).
	
	With our function $h(x)$ expressed using a development of Maclaurin with 5 terms, we see immediately with Maple 4.00b that on the borders of the square inscribed in the disc of convergence, the series does not converge and we're guessing the start of the two singularities:
	
	\texttt{>plot3d(abs(1-(re+I*im)\string^2+(re+I*im)\string^4-(re+I*im)\string^6+(re+I*im)\string^8),\\
	re=-0.7..0.7,im=-0.7..0.7,view=[-1.5..1.5,-1.5..1.5,0..1.5]\\
	,orientation=[-130,70],contours=50,style=PATCHCONTOUR,axes=frame,\\
	grid=[100,100],numpoints=10000);}
	
	\begin{figure}[H]
		\begin{center}
			\includegraphics{img/analysis/h_zoom_on_complex_representation.jpg}
		\end{center}	
		\caption{Focus on the complex representation to understand the reason for the divergence}
	\end{figure}
	a little outside the disc of convergence, we obviously have a little bit nonsense:
	
	\texttt{>plot3d(abs(1-(re+I*im)\string^2+(re+I*im)\string^4-(re+I*im)\string^6+(re+I*im)\string^8),re=-3..3,\\
im=-3..3,view=[-1.5..1.5,-1.5..1.5,0..1.5],orientation=[-130,70]\\
	,contours=50,style=PATCHCONTOUR,axes=frame,grid=[100,100],numpoints=10000);}
	
	\begin{figure}[H]
		\begin{center}
			\includegraphics{img/analysis/h_divergence.jpg}
		\end{center}	
		\caption[]{This diverges ... (stalactites ???)}
	\end{figure}
	There is still something interesting to try ... since we are now on a plane, not a straight line right (axis), it is possible for us to make the Taylor expansion around a singularity $z_0$ by deforming the disk in a convex crown/ring simply connected as shown below (the crown/ring being the simplest simply convex geometry arising from the deformation of a disk):
	\begin{figure}[H]
		\begin{center}
			\includegraphics{img/analysis/h_representation_transformation_disc_in_crow.jpg}
		\end{center}	
		\caption{Representation of the deformation of a disc in a crown/ring}
	\end{figure}
	The advantage of this is to deform the area of convergence on the whole complex plane by avoiding (bypassing) all the singularities. Thus, unlike the Taylor series that are only valid on an interval of the $x$-axis, we would have a new type of series describing a function absolutely everywhere, that is to say before AND after (so around...) singularities!
	
	So obviously we will require that in the deformed crown above the function is always holomorph and analytical (as in the initial convex disc). Before determining what we are going down (generalized Laurent series!), we must first do a study of the decomposition of path integral:
	
	\pagebreak
	\subsection{Path Decomposition}
	The path integrals as given previously can also be written in another form almost classical and used many times in the literature.
	
	Let us see this. First, remember that we have just proved in the special case of a holomorphic function that:
	
	But a closed path can be seen as a path having a round trip:
	\begin{figure}[H]
		\begin{center}
			\includegraphics{img/analysis/closed_path.jpg}
		\end{center}	
		\caption{Representation of a closed path with round trip}
	\end{figure}
	Therefore we can write:
	
	And now comes what interest us... for this purpose let us focus one  of the path integral of the type:
	
	We already well known (1st form of notation) that any complex number $z$ of the type:
	
	can be (2nd form of notation) written as (Euler form):
	
	and to integrate on a path, nothing prevent us to choose a path where $r$ (the module) would be fixed and $\theta$ variable (we could not have the possibility to do this with the 1st form because by modifying the imaginary or real part, we can't get be guarantee to get a nice smooth curve but this is possible with the Euler form of a complex number)!
	
	Therefore we have:
	
	We write then naturally:
	
	and as:
	
	Therefore:
	
	That we often find in the following form in the literature:
	
	
	
	\subsubsection{Inverse Path}
	If $C$ is a curve going from a point $P$ to a point $Q$, then we denote by $C^-$  the same curve but travelled from $Q$ to $P$.
	
	Let us parametrized $C^-$:
	
	If $C(t)$ it is the curve defined on $[a, b]$, then we define the curve $C^-(t)$ on $[a, b]$ by:
	
	Indeed we have with this parametrization:
	
	and when $t$ increases from $a$ to $b$, $a + b - t$ decreases from $b$ to $a$. $C^-$ is therefore only $C$ but travelled in the opposite direction.
	
	We then have using the last proof:
	
	Let us put:
	
	Therefore:
	
	Then we have:
	
	Therefore if $C^-$ and $C$ are the paths of the same function but travel in the opposite direction, we have by taking our conventional notation (caution! In the second term it is implicit that the parametrization is different from the first one!):
	
	Therefore:
	
	this is why we often say that the sign of the value of a line integral will depend on the direction in which its integration path is travel. If the direction is straightforward (that is to say "counter-clockwise") its sign will be positive; if on the contrary the direction is clockwise its sign will be negative (\SeeChapter{see section Differential and Integral Calculus page \pageref{closed path orientation}}).
	
	\pagebreak
	\subsection{Laurent Series}
	This last relation obtained, we can return to the deformation of our disc of convergence in a crown. We recall that initially the idea is to have the analytical expression of a function as an infinite series of powers in a limited area around a singularity point and all this... in the purpose to be able to calculate for physicists complex path integrals through a method using the properties of complex series!
	
	Let's start with the point (2) that is to say have an infinite power series for a path integral, which will take us more easily to point (1) that is to say get the an analytical expression of a function around a singularity point, by zooming on our crown:
	
	\begin{figure}[H]
		\begin{center}
			\includegraphics{img/analysis/crown.jpg}
		\end{center}	
		\caption[]{Zoom on our crown from our starting example}
	\end{figure}
	We therefore have if the function $f$ is analytic and holomorphic in the crown of outer radius $R$ and inside radius $r$, the following path curvilinear integral in the crown as we proved above (we change notation: $z=z'$ and $z_0=z$):
	
	
	therefore we denote now by $z$ the point where we want to know the function and $z'$ variable of which $f$ depends. This notation change will be justified later for a purely practical reason.
	
	The crown can be broken down into four paths:
	
	If both segments $C_c$ and $-C_c$ are infinitely close, they then correspond to the same path travelled once in a positive direction and once in the negative direction. As we have proved just above that:
	
	It therefore follows that:
	
	Which brings us to write:
	
	where we have put a "+" between the last two terms, because as we shall see immediately, the convergence criterion associated with the traditional notation in this field of study, makes automatically emerge the sign "-".
	
	For the two integral $f_1,f_2$, we know that the fraction can be written as a geometric series as already seen above. Effectively, starting from (now you will understand why we changed the notation):
	
	by assimilating:
	
	where as we have seen, the convergence requires that:
	
	so that $x$ is lower in absolute value to $1$.
	
	We then see the infinite geometric series appearing:
	
	Therefore:
	
	To come back to:
	
	we have in any point $z$ inside the circle of radius $R$ whose border is described by the variable $z'$ and of center $z_0$ the convergence that is assured because:
	
	Then we can write:
	
	Integrating term by term integration, we highlight the development (already known):
	
	with the definition of coefficients $c_n$, where $n$ is a positive or null integer:
	
	This development may do think to the development of Taylor in the sense that only positive (or zero) powers of $(z'-z_0)$ appear, but this is not Taylor development in the case of the crown! Indeed, $c_n$ can not be written this time as:
	
	since, by assumption, $f(z)$ is assumed analytic in the crown only and may therefore very well not be inside the small circle of radius $r$, in particular on $z_0$, in which case $f^{n}(z_0)$ may simply not exist (let us repeat that $z$ is strictly constrained to be in the crown, therefore $r<\vert z \vert <R$). We will see later what happens when $f (z)$ is holomorphic in this disk and that, in particular, $z_0$ is not a singular point.
	
	We still need to treat $f_2$. We then do the same type of development as for $f_1$, with the difference that now:
	
	when $z'$ browse the small circle of radius $r$. To make a geometric series appear, we must write this time:
	
	Therefore:
	
	So we have:
	
	Integrating term by term, we highlight the (new) development:
	
	with:
	
	By changing $n$ in $-n$ in the summation for $f_2$, we have for the sum $f_1(z)+f_2(z)$:
	
	with at this time two distinct $c_n$:
	
	We will now see that these two relations can be combined into one!
	
	For this purpose if we observe well the last two relations, we find that they do not depend at all of $z$ (!) and this is normal since the $c_n$ are the coefficients of the series expansion of $f(z)$ and these are the same at any point of the domain of definition of the function where it is analytic!
	
	So the two contours (circles) can be merged into only one circle since it is located in the crown and has for center $z_0$:
	
	Furthermore, the attentive reader will have noticed that this contour does not even need to be a circle finally! It may be any geometry as long as it is closed and is located in an analytical area!
	
	Thus, we get the two relations:
	
	The two previous relation define the "\NewTerm{general Laurent series}\index{general Laurent series}". It is remarkable and differs from a Taylor series in the sense that it contains all the positive and negative integer powers and the coefficients $c_n$ can a priori not be expressed with the derivatives of $f$.
	
	The power series of $n\geq 0$ is named "\NewTerm{regular part}\index{regular part of a power series}", the negative powers is commonly named "\NewTerm{main part}\index{main part of a power series}".
	
	The series of negative powers converges uniformly everywhere outside $\gamma_r$, that of positive powers within $\gamma_R$. In total the development of Laurent converges uniformly in the common area, which is the crown and therefore also on the unique path $\gamma$.
	
	Let us now show a point that we have mentioned above. If the circle contains no singularity, then all the coefficients:
	
	are zero. First note that $-n-1$ is a positive or zero integer, which we will denote by $p$ such as:
	
	We then have the following integrand along a closed path:
	
	But, if we remove the singularity that requires $f(z')$ is holomorphic (and in anyway this is required by all initial developments of the Laurent series).
	
	As $(z'-z_0)^p$ is polynomial with positive and not null integer powers and that as we know any polynomial satisfying these conditions is differentiable at least once without showing singularity. Thus this term is also holomorphic.
	
	Assuming that the product of two holomorphic functions is holomorphic and that contour $\gamma$ is closed, then we have using the following result proved above (for a holomorphic function):
	
	the following immediate consequence:
	
	if there is no singularity in the small circle of the crown. We then fall back again on a development with only positive powers, the $c_{n\geq 0}$ this time being equal to:
	
	according to the generalized Cauchy's integral theorem proved earlier above. Conversely, we see well that this is the main part (when it exists!) which contains the information on the fact that $f$ is not a priori holomorphic in the small disk. The existence of negative powers shows that $f$ is clearly not bounded on $z_0$.
	
	The classification of singularities of a function will be precisely based on the consideration of the characteristics of the main part of the Laurent  development centered on a singular point of this function.
	\begin{tcolorbox}[colframe=black,colback=white,sharp corners]
	\textbf{{\Large \ding{45}}Example:}\\\\
	Let us see to what looks like the Laurent series of our famous example function:
	
	on a simply convex domain that would be the crown rounding the singularity  $\mathrm{i}$ for example (we could have taken the second singularity $-\mathrm{i}$ but we had to choose one to not repeat twice the explanations below...). This is equivalent therefore to search the power series development of $z-\mathrm{i}$. \\
	
	We will proceed as following:
	
	For what will follow we will use:
	
	\end{tcolorbox}
	
	\pagebreak
	\begin{tcolorbox}[colframe=black,colback=white,sharp corners]
	The second fraction can be expressed as a geometric series if as we have already seen:
	
	Therefore it comes:
	
	Let us multiply both sides of this equality by $-i / $2 and then divide them by $z - i$ (the second term in the denominator of the original fraction) for obtain the left term:
	
	and for the right term:
	
	Finally we have the following geometric series:
	
	We see then on this Laurent series around $\mathrm{i}$ of the holomorphic function $f(z)$ that the following coefficient appears:
	
	and then we have with Maple 4.00b:\\
	
	\texttt{>plot3d(abs(-I/2*1/((re+I*im)-I)-(I/2)\string^2-(I/2)\string^3*(re-I*im)-\\
	(I/2)\string^4*(re-I*im)\string^2-(I/2)\string^5*(re-I*im)\string^3),\\
	re=-1.5..1.5,im=-1.5..1.5,view=[-2..2,-2..2,-1..2],\\
	orientation=[-130,70],contours=50,style=PATCHCONTOUR,axes=frame,\\
	grid=[100,100],numpoints=10000);}
	\end{tcolorbox}
	
	\pagebreak
	\begin{tcolorbox}[colframe=black,colback=white,sharp corners]
	This gives the following figure:
	\begin{figure}[H]
		\centering
		\includegraphics{img/analysis/laurent_series_representation.jpg}
		\caption{Laurent series representation of $f(z)$ with Maple 4.00b}
	\end{figure}
	where we see that the Laurent series allows us to express $f (z)$ in a neighbourhood close to the singularity $\mathrm{i}$ by taking five terms.\\
	
	Ditto if we make the sum of the two Laurent series for the two singularities with seven terms:\\
	
	\texttt{>plot3d(abs(-I/2*1/((re+I*im)-I)-(I/2)\string^2-(I/2)\string^3*(re-I*im)-\\
	(I/2)\string^4*(re-I*im)\string^2-(I/2)\string^5*(re-I*im)\string^3 -(I/2)\string^6*(re-I*im)\string^4\\
	-(I/2)\string^7*(re-I*im)\string^5+I/2*1/((re+I*im)+I)+(I/2)\string^2+
(I/2)\string^3*(re+I*im)+(I/2)\string^4*(re+I*im)\string^2+(I/2)\string^5*(re+I*im)\string^3\\
	+(I/2)\string^6*(re+I*im)\string^4
+(I/2)\string^7*(re+I*im)\string^5),re=-1.5..1.5, im=-1.5..1.5, view=[-2..2,-2..2,-1..2],orientation=[130,70], contours=50, style=PATCHCONTOUR, axes=frame,grid=[100,100],\\
numpoints=10000);}\\

	This gives the image visible on the next page:
	\end{tcolorbox}
	
	\pagebreak
	\begin{tcolorbox}[colframe=black,colback=white,sharp corners]
	\begin{figure}[H]
		\centering
		\includegraphics{img/analysis/sum_of_two_laurent_series.jpg}
		\caption{Sum of the two Laurent series of $f(z)$ for both singularities with Maple 4.00b}
	\end{figure}
	\end{tcolorbox}
	
	\subsection{Singularities}
	We have seen just before that it was possible to calculate the path integral of a function, on condition of analyticity, on the outline of a singularity. Our goal will now be to enhance this approach.
	
	We have already mentioned and highlighted in our previous proofs that the integrant in the "Cauchy's integral theorem" was of the form:
	
	where $f(z)$ is well defined in $z_0$.
	
	The point $z-z_0$ is of course a singularity of $f(z)$ and it is not defined there.
	
	As we saw during our proof of Laurent series, $f(z)$ can be expressed as a Laurent series in the form a positive power Laurent series in a convergence disk (or what remains the same: as a series of Laurent in a crown not centered on a singularity...) in the form:
	
	Before continuing, it is customary in mathematics to define a small conventional vocabulary regarding this time the possible singularities of $f(z)$!
	
	Let us first recall that we know, and that we have proved, that all information on the singularities of $f (z)$ are contained in the main part of the Laurent series (negative powers) defined on the crown surrounding $z_0$:
	
	The following classification focus on "\NewTerm{isolated singularities}\index{isolated singularities}", that is to say, a singular point where $f(z)$ is analytic everywhere in the neighbourhood excepted on $z_0$. This classification, as we will see permits us to distinguish three types of singular points, will be useful when developing the theory of residues further.
	
	\textbf{Definitions (\#\mydef):}
	\begin{enumerate}
		\item[D1.] When the limit of the function $\vert f(z) \vert$ exists on $z_0$, we say that the singularity is a "\NewTerm{removable singular point}\index{removable singular point}" or "\NewTerm{apparent singularity}\index{apparent singularity}".
		
		For example:
		
		does not seem to be defined on $z=z_0=0$ but we have a numerator having a Laurent series without negative powers (therefore a simple Taylor series). It then comes into by doing the Maclaurin series (that is to say the Taylor series on $z=z_0=0$...):
		
		We then see that $f (z)$ finally has no term with negative power and therefore we have eliminated the singularity (or that it contains simply no singularities... which can easily be check with Maple 4.00b).
		
		\item[D2.] When on $z_0$ the limit $\vert f(z) \vert $ does not exist we speak about "\NewTerm{essential singularity}".
		
		For example, $z_0=0$ is an essential singularity for the function:
		
		Indeed, if $z$ approaches zero coming from the positive real axis $\mathbb{R}_+$, the function diverges, more precisely, it tends to $+\infty$. If $z$ comes from $\mathbb{R}_-$, the function tends to zero as illustrated by the following Maple 4.00b plot:
		
		
		\texttt{>plot3d(abs(exp(1/(re+I*im))),re=-5..5,im=-5..5,\\
		view=[-3..3,-3..3,-0.5..3],orientation=[-130,70],contours=50,\\
		style=PATCHCONTOUR,axes=frame,grid=[100,100],numpoints=10000)\\
		>plot3d(abs(exp(1/(re+I*im))),re=-5..5,im=-5..5,\\
		view=[-3..3,-3..3,-0.5..3],orientation=[-130,70],contours=50,\\
		style=PATCHCONTOUR,axes=frame,grid=[100,100],numpoints=10000)}
		\begin{figure}[H]
			\centering
			\includegraphics{img/analysis/plot_essential_singularity.jpg}
			\caption{Essential singularity example with $e^{1/z}$ in Maple 4.00b}
		\end{figure}
		Indeed:
		
		So an equivalent way of defining an essential singularity, is to say that there are an infinite number of terms with negative powers in the main part of the Laurent series.
		
		\item[D3.] When on $z_0$ the limit of $\vert f(z) \vert$ is $+\infty$, we speak about a "\NewTerm{pole}\label{pole}".
		
		This is the last category (as far as we know...) in which we can store a function that is not classifiable neither in the first nor in the second definition above.
		
		So another equivalent way of defining a "pole", is to say that there is a finite number of terms with negative powers in the main part of the Laurent series. If the number of terms is $k$, then we speak of "\NewTerm{pole of order $k$}".
		
	\begin{tcolorbox}[title=Remarks,colframe=black,arc=10pt]
	\textbf{R1.} We sometimes say that an essential singularity is a " \NewTerm{pole of order $+\infty$}".\\
	
	\textbf{R2.} A pole of order 1 is named a "\NewTerm{simple pole}". One of order 2 is named a "\NewTerm{double pole}" and so on...
	\end{tcolorbox}
	\end{enumerate}
	If we come back on our example:
	
	We have proved previously that the Laurent series of the function was:
	
	This function has therefore a trivial pole of order $1$ on $z_0=\mathrm{i}$ and also on $z_0=-1$ because in this latter case this infinite series diverge to $+\infty$ and we can easily check this with the following Maple 17.00 command:
	
	\texttt{>sum(-(I*(1/2))\string^n*(-2*I)\string^(n-2), n = 0 .. infinity)}
	
	\subsection{Residue Theorem}\label{residue theorem}
	Consider a function $f(z)$ whose pole is of order less or equal to $k$.
	
	Let us make it  analytic:
	
	that is to say that we have take a function $f(z)$ that we have made analytic after elimination of the poles supposed in finite number - order - less than or equal to $k$ on $z_0$. 
	
	This function $\phi(z)$ has therefore a Laurent series development  in a disc center on $z_0$.
	
	As we have prove it previously, we can therefore by using the following relation:
	
	write:
	
	Using $f(z)$ under the integral it comes:
	
	You must deeply analyse this relation and understand that it link together the integral of a function having singularities with the value on one point of an analytical function having no more singularities!!!
	
	This latter relation can be rewritten by rearranging terms:
	
	And by expressing $\phi^{(k)}(z_0)$ by using (this is authorized because this latter function is analytical) the fact that by definition:
	
	We get obviously:
	
	Therefore by making $\phi(z)$ explicit again:
	
	This latter relation is valid only for ONE isolated singularity (in case you forget!) and where $k$ is equal at least to $1$!

	Mathematicians therefore define:
	
	as being the residue of the function $f(z)$ at the point being an isolated singularity of order $k$. Or respectively:
	
	where the path integral is centered on $z_0$.
	
	Now notice that the term on the right of the equality in the previous relation correspond to the coefficient $c_{-1}$ of the Laurent series. Indeed:
	
	Therefore:
	
	\begin{tcolorbox}[title=Remark,colframe=black,arc=10pt]
	Therefore it comes that on an isolated singularity that can be eliminated, the residue is null because as we saw it before, the path integral rounding a domain without singularity is equal to zero!
	\end{tcolorbox}
	To resume, the relation:
	
	is very interesting for the physicist... because this is a very elegant way for him to calculate the path integral of a non analytic function $f(z)$ having a unique isolated singularity and this just by knowing the order of its poles!
	
	For example if a function $f(z)$ has only a pole of order $1$, we have therefore:
	
	and we replace therefore $z_0$ by the desired value in the parenthesis $(z-z_0)$ and after we calculate the limit between brackets!
	
	Now to go more fare, let us remind that outline of the path integral:
	
	and the curvilinear path of the integral:
	
	are in fact combined (identical) and the coefficients $c_n$ do not depend on $z$! The only constraint on the path is that is closed and in an analytical domain centered on one point.
	
	So if we have several isolated singularities, surrounded by connected curvilinear paths as shown below on the complex plane of a function having a pole of order 3 (i.e. three non-removable singularities $z_0,z_1,z_2$):
	\begin{figure}[H]
		\centering
		\includegraphics{img/analysis/multiple_surrounded_singularities.jpg}
		\caption{Multiple isolated singularities surrounded by curvilinear paths}
	\end{figure}
	then we have still only one closed curvilinear path but whose different isolated singularities are connected by cross which as we know: the paths that are opposed  cancel themselves! And let us remind that the coefficients are the same throughout on all the path since it is on an analytical domain.
	
	We then have the generalized version of the residue theorem for a function $f$ with $n$ isolated singularities:
	
	with a rigorous approach that is specific to engineers ... who sometimes write this latter relation as following:
	
	where $r$ is therefore a residue. This is an important result in the field of solving differential equations associated with some inverse Laplace transforms (\SeeChapter{see section Functional Analysis page \pageref{Laplace transform}}). This intermediate result will give us the possibility to get an another one a little further of major importance for the section of Corpuscular Quantum Physics.
	\begin{tcolorbox}[colframe=black,colback=white,sharp corners]
	\textbf{{\Large \ding{45}}Example:}\\\\
	Let us take again our famous function:
	
	We know it has a pole of order $1$ on $z_0=\mathrm{i}$ and a pole of order $1$ on $z_0=-\mathrm{i}$. So if we take this time Laurent series with a path that surrounds the two singularities (and not only one) then we have a function with a pole of order $2$.
	
	It comes then for this particular case:
	
	with $n$ being equal to $2$.
	
	Then we have:
	
	and:
	
	\end{tcolorbox}
	
	\pagebreak
	\begin{tcolorbox}[colframe=black,colback=white,sharp corners]
	We can easily check this with Maple 4.00b:\\
	
	\texttt{>readlib(singular):\\
	>singular(1/(1+z\string^2),z);\\
	>readlib(residue):\\
	>residue(1/(1+z\string^2),z=-I);\\
	>residue(1/(1+z\string^2),z=I);\\}

	and therefore:
	
	In fact in this case, the residue theorem gives zero because the function has no poles to infinity which is true since in our example:
	
	Physicists meanwhile say that... the force does make any work on this path ...!
	\end{tcolorbox}
	
	\subsubsection{Pole at infinity}
	We have say before that any function that did not have poles at infinity had therefore the sum of the residues of all the poles that are equal to zeros. This result is very important in physics and merits to be study!
	
	It is almost trivial to recognize the number of poles... but to recognize the poles that are at infinity there are many times traps we can easily fall in.
	
	Let us consider the expression $f(z)\mathrm{d}z$. If $z$ is at the neighbourhood of the infinity then $1/z$ is near $0$. Let us write:
	
	Then we have:
	
	Then the residue at infinity is such that:
	
	with:
	
	Therefore with:
	
	The latter relation we will be indispensable to us in the section of Corpuscular Quantum Physics Corpuscular to build the relativistic Sommerfeld model of hydrogenous atom because we will need to calculate a path integral with a pole.
	
	Let's see an example with the function that accompanies us since the beginning of this section. That is to say:
	
	Therefore it comes:
	
	And we recognize immediately the initial function, in absolute value, and that has therefore no pole on $0$. Therefore $f(z)$ has no pole at infinity.
	
	\begin{flushright}
	\begin{tabular}{l c}
	\circled{100} & \pbox{20cm}{\score{3}{5} \\ {\tiny 14 votes,  61.43\%}} 
	\end{tabular} 
	\end{flushright}
	
	%to force start on odd page
	\newpage
	\thispagestyle{empty}
	\mbox{}
	\section{Topology}\label{topology}
	\lettrine[lines=4]{\color{BrickRed}T}opology is an extremely broad field of mathematics for which it is difficult to define precisely the object so the areas where it applies are varied (real line topology, graphs topology, differential topology, complex topology, symplectic topology, etc.). 

We mainly make a distinction with:
	\begin{itemize}
		\item "\NewTerm{General topology}" that establishes the foundational aspects of topology and investigates properties of topological spaces and investigates concepts inherent to topological spaces. It includes point-set topology, which is the foundational topology used in all other branches (including topics like compactness and connectedness).
		
		\item "\NewTerm{Algebraic topology"} tries to measure degrees of connectivity using algebraic constructs such as homology and homotopy groups.
		\item "\NewTerm{Differential topology"} is the field dealing with differentiable functions on differentiable manifolds. It is closely related to differential geometry (see section of the same name in the chapter about Geometry page \pageref{differential geometry}) and together they make up the geometric theory of differentiable manifolds.
		\item "\NewTerm{Geometric topology}" primarily studies manifolds and their embeddings (placements) in other manifolds. A particularly active area is low dimensional topology, which studies manifolds of four or fewer dimensions. This includes knot theory, the study of mathematical knots.	
	\end{itemize}
	\begin{tcolorbox}[title=Remark,colframe=black,arc=10pt]
	A "\NewTerm{manifold}\index{manifold}" is a higher dimensional analogue of a curve or a surface. A manifold of dimension $n$ is a space, you can think of a collection of points, that locally looks like $\mathbb{R}^{n}$ for some integer $n$. All curves in the two dimensional plane that do not intersect locally look like part of a line. All smooth surfaces in $\mathbb{R}^3$, that is surfaces that do not have sharp kinks, edges (boundaries), or points all locally look like the two dimensional plane. Thus such surfaces are two dimensional manifolds. One should imagine being able to tear any small piece off the smooth surface, then being able to stretch it, push down any "hills" and push up any "valleys" to end up with a flat piece of the $2$-plane. Therefore a manifold is a space that is locally Euclidean\\
	
	The picture to have in your mind when thinking about manifolds is the relation between a globe and a map. Small pieces of the globe can always be described by maps (pieces of the 2-plane). Moreover, the entire globe can be covered by a collection of maps: an atlas.
	\end{tcolorbox}
	What we can say at first is that in its foundations Topology is very closely related to the set theory, the study of convergence of sequences and series, functional analysis, analysis complex, the differential and integral calculus, vector calculus and the geometry to mention only the most important cases that the reader can already found in this book.

	The origin of Topology comes from the problems that laid to the progress of functional analysis in the rigorous study of continuous functions, their differentiability, their limits at a point (finite or note), the existence of extremums, etc. in higher-dimensional spaces (in fact, implicitly, the goal for topology is to create tools that easily allow to study the properties of functions in all dimensions). All these concepts, needed a rigorous mathematical definition of the intuitive idea of proximity, especially when doing operations on such functions.

	We will try in this section to identify the basis of the structures that allow us to speak about limits and continuity and this only for curiosity as consultants in R\&D and financial engineering we never saw a business application where the subjects below are absolutely necessary to develop a new business or solve a problem. 
	
	The majority of examples that we will take in this section will be in $\mathbb{R}$ (the $\mathbb{R}$ straight line to be more exact...) because it is the most used one by the engineers (and most of times the only one!) and the one we will have to use for the sections on Graph Theory, Statistics, Differential and Integral Calculus and also on Fractals. When we restrict our study of Topology on $\mathbb{R}$ we then speak of "\NewTerm{Real Analysis}".

	\subsection{General Topology}

General topology is the branch of topology dealing with the basic set-theoretic definitions and constructions used in topology. It is the foundation of most other branches of topology, including differential topology, geometric topology, and algebraic topology.

The fundamental concepts in point-set topology are "\NewTerm{continuity}", "\NewTerm{compactness}", and "\NewTerm{connectedness}".  
	\begin{itemize}
		\item Continuous functions take \underline{nearby} points to nearby points.
		
		\item Compact\label{compact} sets are those that can be covered by finitely many sets of \underline{arbitrarily small} size. 
		
		\item Connected sets are sets that cannot be divided into two pieces that are \underline{far apart}. 
	\end{itemize}		
	The words \underline{nearby}, \underline{arbitrarily small}, and \underline{far apart} can all be made precise by using open sets. If we change the definition of open set, we change what continuous functions, compact sets, and connected sets are. Each choice of definition for open set is named a "\NewTerm{topology}". A set with a topology is named a "\NewTerm{topological space}\index{topological space}\label{topological space}".

	"\NewTerm{Metric spaces}\index{metric space}\label{metric space}" are an important class of topological spaces where distances can be assigned a number named a "\NewTerm{metric}\label{metric}". Having a metric simplifies many proofs, and many of the most common topological spaces are metric spaces.
	
	An "\NewTerm{inner product}" (\SeeChapter{see section Vector Calculus page \pageref{inner product}}) induces a "\NewTerm{norm}"  (\SeeChapter{see section Vector Calculus page \pageref{vector norm}}) and the norm induces a metric space. 
	
	Therefore we understand better what we will study in this section that can be summarized by the following figure:
	\begin{figure}[H]
		\centering
		\includegraphics{img/analysis/topological.jpg}
	\end{figure}

	\subsubsection{Topological Spaces}
	Topological spaces form the conceptual foundation on which the concepts of limit, continuity or equivalence are defined.
	
	The framework is general enough to be applied in many different situations: finite sets, discrete sets, geometry spaces, $n$ dimensional numerical spaces and most complex functional areas. These concepts appear in almost all branches of mathematics, they are therefore central to the modern view of mathematics.
	
	\textbf{Definition (\#\mydef):} Consider a non-empty set $X$ (the length of a plastic ruler for example). A "\NewTerm{topology $\mathcal{T}$}" or "\NewTerm{topological space $(x,\mathcal{T})$}" on $X$ is a family $\mathcal{T}$ of parts of $X$ (of length of our rule...) named "\NewTerm{open $V$}"  (as the open intervals seen in the section  of Functional Analysis) such that the following axioms are true:
	\begin{enumerate}
		\item[A1.] The empty set $\varnothing$ and $X$ are considered as open $V$ and must belong to the family of the topology $\mathcal{T}$ (these only both open sets define what we name the "\NewTerm{trivial topology}" that is the most minimal one satisfying all the axioms):
		
		In other words, if we imagine our plastic ruler, the measure zero (strictly speaking: the empty set) must belong to the topology defined on the ruler and the ruler itself (seen as a subset).
		
		\item[A2.] Any finite intersection of open of $\mathcal{T}$ will be an open of $\mathcal{T}$:
		
		
		\item[A3.] Any union of open of $\mathcal{T}$ will be an open of $\mathcal{T}$:
		
		\begin{tcolorbox}[title=Remarks,colframe=black,arc=10pt]
		\textbf{R1.} Mathematicians frequently note by $O$ the family of open sets and by $F$ the family of closed sets. Convention we will not follow in this book.\\

		\textbf{R2.} The close sets of a topology are complementary of open sets. Therefore, the family of  close sets contains among other $X$ and the empty set $\varnothing$...\\
	
		\textbf{R3.} There is no difference between part and subset of a set.
		\end{tcolorbox}
		
		\item[A4.] The couple $(X,\mathcal{T})$ is a "\NewTerm{Hausdorff space}" or "\NewTerm{separate space}" if moreover the property named "\NewTerm{Hausdorff axiom}" is verified:
				
	\end{enumerate}
	\begin{tcolorbox}[title=Remarks,colframe=black,arc=10pt]
		\textbf{R1.} A well known example of topological space is $\mathbb{R}$ provided with the set $F$ generated by the open intervals (by the union law), that is to say the intervals of the type $] a, b [$.\\
		
		\textbf{R2.} We will see a very concrete and beautiful + nice application of Hausdorff spaces in our study of fractals in the chapter Theoretical Computing.
	\end{tcolorbox}
	
	\textbf{Definition (\#\mydef):} 
	\begin{enumerate}
		\item[D1.] If we denote by $(X, V)$ a topological space, $V$ designating the open sets of $X$, a "\NewTerm{base}", in the topological sense, of $(X, V)$ is a part $B$ of $V$ such that any open set of $V$ is a an union of open sets of $B$ (this is the same idea as vector spaces but in fact applied to sets ... nothing bad and difficult! If you want an example see the section of Measure Theory).
		
		\item[D2.] In topology a subset $A$ of a topological space $X$ is named "\NewTerm{dense}\index{dense set}\label{dense set}" (in $X$) if for every point $x$ in $X$ either belongs to $A$ or is a limit point of $A$. Informally, for every point in $X$, the point is either in $A$ or arbitrarily "close" to a member of $A$. For instance, every real number is either a rational number or has one arbitrarily close to it. Therefore $\mathbb{Q}$ is dense in $\mathbb{R}$.
	\end{enumerate}
	
	\pagebreak
	\subsection{Metric Space and Distance}\label{distance}
	\textbf{Definition (\#\mydef):} A "\NewTerm{metric space}" denoted by $(X, d)$ or sometimes $X_d$ (or even sometimes just $X$ if the type of distance $d$ cannot not be confused) is by definition a set $X$ with provided with an application:
	
	named "\NewTerm{distance}\index{distance}" or "\NewTerm{metric}", which satisfies the following axioms:
	\begin{itemize}
		\item[A1.] Positivity:
		
		
		\item[A2.] Separation:
		
		\item[A3.] Triangular inequality:
		
		
		\item[A4.] Symmetry: 
		
	\end{itemize}
	\begin{tcolorbox}[title=Remarks,colframe=black,arc=10pt]
		\textbf{R1.} Some readers will probably see immediately that some of these properties have already been seen in other sections of this book during our study of distances between functional points and during our study of norms (triangle inequality proved in the section of Vector Calculus - the symmetry, positivity, the separation have already been study in the section of Functional Analysis).\\
		
		\textbf{R2.} Some authors omit the axiom A1 which is strictly correct as it trivially follows from A3.\\
	\end{tcolorbox}
	The "distance function" of $\forall x,y \in X$ is thus usually denoted in the more possible sense in mathematics (at least as far as we know):
	
	we will see three examples much more further below with a schema.
	
	\textbf{Definition (\#\mydef):} If we do not impose the axiom A2, we say that $d$ is a "\NewTerm{semi-distance}" on $X$ and if we allow a semi-distance $d$ to take the value $+\infty$, we prefer to say that $d$is a "gap".
	\begin{tcolorbox}[title=Remarks,colframe=black,arc=10pt]
		\textbf{R1.} If a distance $d$ satisfies the property:
		
		property more restrictive that the triangle inequality in some spaces, we say that $d$ is "\NewTerm{ultrametric}".\\
		
		An example of ultrametric distance is the family tree (...):
		
		\begin{figure}[H]
			\centering
			\includegraphics{img/analysis/family_tree.jpg}
			\caption{Example of ultrametric distance with an orgchart}
		\end{figure}
		We have the following distances:
		
		We note that the distances above do not add up, but we have by cons:
		
		Therefore:
		
		
		\textbf{R2.} Let $(X, d)$ be a metric space and consider $F=\varnothing$ a part of the set $E$. The metric space $(F,\delta)$ where $\delta$ denotes the restriction $d_{F \times F}$ of $d$ is named "\NewTerm{metric subspace}" of $(X, d)$ (we should check that the distance $d$ is equivalent to the distance $\delta$). In this case, we also say that $F$ is provided with the distance induced by this of $X$. We therefore simply note $d$ the induced distance.
	\end{tcolorbox}

	\pagebreak
	Let us give now some examples:
	\begin{tcolorbox}[colframe=black,colback=white,sharp corners]
	\textbf{{\Large \ding{45}}Examples:}\\\\
	E1. If we take for $X$ the plane, or the three-dimensional space of Euclidean geometry and a unit of length, the "distance" in the usual sense is a distance within the meaning of the 4 axioms mentioned above. In these spaces, the three points $A, B, C$ satisfy as we have proved it in the section Vector Calculus:
	
	with other inequality obtained by circular permutation of $A, B, C$. These inequalities are well known, for example between the side lengths of a triangle.\\
	
	E2. If we take $X=\mathbb{R}^n$, $n \in \mathbb{N} \geq 1$ and that we equip $\mathbb{R}^n$ of an Euclidean vector space structure (and not non-Euclidean!) and we take two points:
	
	in $\mathbb{R}^n$, the distance is then given by (we have already proved this in the sections of Functional Analysis and Vector Calculus):
	
	\end{tcolorbox}
	\label{euclidean topology}This latter distance satisfies the five axioms of distance and we name it the "\NewTerm{Euclidean distance}" and is often denoted $L_2$. We can take (it is an interesting property for the general culture), any relation of the form:
	
	is also a distance in $\mathbb{R}^n$ (without proof) named " \NewTerm{$p$-norm}". In the particular case with $n=1$, we have of course:
	
	This is the usual distance on $\mathbb{R}$ and is often denoted $L_2$.
	
	Mathematicians are even stronger by generalizing ever more (the proof has little interest for now in this book) the prior-previous relation (taking into account the definition of the distance) in the form:
	
	which is named "\NewTerm{Hölder distance}". Also sometimes denoted for $L_p:\mathbb{R}^n\mapsto\mathbb{R}$:
	
	\begin{tcolorbox}[title=Remark,colframe=black,arc=10pt]
	Following the intervention of a reader we would like to point out that strictly speaking the above inclusion should be noted $[1,+\infty[ \subset \mathbb{\overline{R}}$ where $\mathbb{\overline{R}}$ is the achieved line (also valid for precision for the Minkowski inequality below).
	\end{tcolorbox}	
	As for the triangle inequality, then given by (\SeeChapter{see section Vector Calculus page \pageref{triangle inequality}}):
	
	The generalization, by the verification of the existence of the Hölder distance, gives us true "\NewTerm{Minkowski inequality}":
		

	Let us continue with our examples:
	\begin{tcolorbox}[colframe=black,colback=white,sharp corners]
	\textbf{{\Large \ding{45}}Examples:}\\\\
	E3. If we take $X=\mathbb{C}$ we will consider the distance:
	
	Therefore if $z=a+\mathrm{i}b=(a,b)$ and $z'=a'+\mathrm{i}b'=(a',b')$ we have the module that the same manner as norm in $\mathbb{R}^2$, forms a distance:
	\\
	
	E4. Let us consider $E=\varnothing$ an arbitrary set. Let us write:
	
	It is quite check that this distance satisfies the five axioms and that is furthermore an ultrametric distance. This distance is named "\NewTerm{discreet distance}" and the reader will notice that, by analogy, we choosed to express this distance by the Dirac symbol $\delta$ (this is not innocent !!) rather than the traditional $d$.
	\end{tcolorbox}
	
	\pagebreak
	\subsubsection{Equivalent Distances}
	Sometimes two different distances $d$ and $\delta$ on the same set $E$ are quite similar so that the related metric spaces $(E,d),(E,\delta)$ have the same properties for certain mathematical objects defined by $d$ on one hand, and by $\delta$ on the hand. There are several concepts of equivalences for example first (before the others that require mathematical tools that we have not yet defined):

	\textbf{Definition (\#\mydef):} Let $d$ and $\delta$ be two distances on the same set $E$, $d$ and $\delta$ are named "\NewTerm{equivalent distances}" if there are two real constants $c>0,C>0$ such that:
	
	Therefore:
	
	with $c\leq C$. We note this equivalence by:
	
	The advantage of this definition is the following: if we have convergence for one of the metric, then we have convergence for the other too. More clearly:
	
	verbatim:
	
	
	\subsubsection{Lipschitz Functions}\label{lipschitz functions}
	With respect to the above definitions, we can now assign some additional properties to functions such as we had define in the section of Set Theory or Functional Analysis and analysed (in part...) in the section of Differential and Integral Calculus. The idea is also mainly to build a set of tools enabling the study of differential properties of non differentiable functions.
	
	Let $(E, d)$ and $(F,\delta)$ be metric spaces, and $f:E \rightarrow E$ a function. We define the following properties:
	\begin{enumerate}
		\item[P1.] We say that $f$ is an "\NewTerm{isometry}" if (it is rather intuitive ...!):
		
		
		\item[P2.] If we take the usual distance, the $L$-Lipschitz or "\NewTerm{Lipschits function of order $L$}" is then defined by on a given interval by:
		
		that we can also write:
		
		or what remains the same: all line drawn between two arbitrary points of the graph must have a bounded and finite slope coefficient (derivative) between $L$ and $-L$.
		
		Any such $L$ is referred to as a "\NewTerm{Lipschitz constant}" for the function $f$. As $L$ can be define on intervals (not necessarily the whole domain of definition) smallest constant is sometimes named the "\NewTerm{best Lipschitz constant}".
		
		Otherwise, one can equivalently define a function to be Lipschitz continuous if and only if there exists a constant $L$ such that, for all $x\neq y$:
		
		For real-valued functions of several real variables, this holds if and only if the absolute value of the slopes of all secant lines are bounded by $k$. 
		
		Therefore all Lipschitz must be continuous and any function $f$ that has a bounded $L$ value is more restrictive than just simply being continuous! 
		
		\begin{tcolorbox}[colframe=black,colback=white,sharp corners]
		\textbf{{\Large \ding{45}}Examples:}\\\\
		E1. The function $f(x)=\sin (x)$ is $1$-Lipschitz as the derivative of the cosine is between $-1$ and $1$.\\
		
		E2. The function $f(x)=x^2$ is locally Lipschitz as for any interval closed and finite interval we can found a bound $L$ but is not globally Lipschitz as when $x\rightarrow \pm\infty$ then the derivative has also no bounds.\\
		
		E3. The function $f(x)=|x|$ has no derivatives on $x=0$ in the ordinary sense. But its derivative in the Lipschitz sense on $x=0$ is given by the a closed interval denoted $\partial_L f(0)=[-1,1]$ given by the bound of $L$ as the ordinary derivatives is less than or equal to $L$ in absolute value!\\
		
		This last example show us that the notion of a local minimum of a function $f(x)$ in the ordinary sense is generalized with Lipschitz condition. As we can now simply define the condition of local minimum of a non-smooth function $f(x)$ as:
		
		rather than in the ordinary sense (more restrictive):
		
	\end{tcolorbox}
		
		
		Schematically for a Lipschitz continuous function, there is a double cone (shown in white) whose vertex can be translated along the graph, so that the graph always remains entirely outside the cone.
		\begin{figure}[H]
			\centering
			\includegraphics{img/analysis/lipschitz.jpg}
			\caption[Example of Lipschitz function]{Example of Lipschitz function (source: Wikipedia)}
		\end{figure}
		
		Intuitively, a Lipschitz continuous function is therefore limited in how fast it can change: there exists a definite real number such that, for every pair of points on the graph of this function, the absolute value of the slope of the line connecting them is not greater than this real number; this bound is named a "Lipschitz constant" of the function (or "modulus of uniform continuity"). For instance, every function that has bounded first derivatives is Lipschitz.
		
		\item[P3.] If $L=1$, we say that the function $f(x)$ is a "\NewTerm{short map}". If $-1<L<1$, we say that $f(x)$ is "\NewTerm{strictly contracting}\label{strictly contracting}".
		
		\item[P4.] We say that two metric spaces are "\NewTerm{isometric spaces}" if there is a surjective isometry of one over the other (which is quite natural in geometry ...).
	\end{enumerate}
	\begin{tcolorbox}[title=Remarks,colframe=black,arc=10pt]
	\textbf{R1.} An isometry is always injective as:
	
	but in general it is not surjective.\\
	
	\textbf{R2.} If $(E,d)$ and $(F,\delta)$ are isometric, of the point of view of the theory of metric spaces they are not discernible, as all their properties are the same, but their elements can be  of very different nature (sequences in one and functions in the other).\\
	\end{tcolorbox}
	
	\pagebreak
	\subsubsection{Continuity and Uniform Continuity}\label{continuity and uniform continuity}
	As we already see it in the section of Functional Analysis, a continuous function is, roughly speaking, a function for which small changes in the input result in small changes in the output and that permits the analysis of limits. Otherwise, a function is said to be a discontinuous function. Formally it was defined by:
	
	In other words remember that this mean that a function is continuous if for every point $x_0$ in the domain $E$, we can make the images of that point ($f(x_0)$) and another point ($f(x)$) arbitrarily close (of a distance $\varepsilon$) if we move the other point ($x$) close enough (distance $\delta$) to our given point.
	
	Hence it is not continuous if:
	
	
	The previous definition is not so good as it make usage of a special case of distance (the absolute value). It is therefore more common to generalize by writing:
	
	
	Now let us state a more restrictive definition!
	
	\textbf{Definition (\#\mydef):} A function $f(x)$ is "\NewTerm{uniformly continuous}" if it satisfies:
	
	with $\lambda=\varepsilon/L$ and $L\neq 0$. In other words, if we can bring two points as close as we want in a space, so can we in the other way (which ensures somehow the derivation.
	
	Hence it is not uniformly continuous if:
	
	
	If we compare the two relations:
	
	the only difference is the order of the quantifiers. Indeed, for something to be continuous, you can check "one $x$ at a time", so for each $x$, you pick a $\varepsilon$ and then find some $\varepsilon>0$ that depends on both $x$ and $\varepsilon$ so that $|f(x)-f(x_0)|<\varepsilon$ if $|x-x_0|<\delta$. If we want uniform continuity, we need to pick first a $\varepsilon$, then find a $\delta$ which is good for ALL the $x$ values we might have.
	
	As the previous definition are not quite easy for everybody let us see the engineer version of these two definitions:
	
	\pagebreak
	\textbf{Definitions (\#\mydef):}
	\begin{enumerate}
		\item[D1.] A function $f(x):E \rightarrow \mathbb{R}$ is "\NewTerm{continuous}" at a point $x_0\in A$ if, for all $\varepsilon>0$, there exists a $\delta>0$ such that whenever $|x-x_0|<\delta$ (and $x\in E$) it follows that $|f(x)-f(x_0)|<\varepsilon$.
		
		\item[D2.] A function $f(x):E \rightarrow \mathbb{R}$ is "\NewTerm{uniformly continuous}" on $A$ if, for all $\varepsilon>0$, there exists a $\delta>0$ such that whenever $|x-x'|<\delta$ (and $(x,x')\in E$) it follows that $|f(x)-f(x')|<\varepsilon$.
	\end{enumerate}
	Therefore we see better the difference: continuity is defined at a point $x_0$, whereas uniform continuity is defined on a set $E$. Roughly speaking, uniform continuity requires the existence of a single $\delta>0$ that works for the whole set $E$, and not near the single point $x_0$.
	
	From this definition wee that any uniformly continuous function is continuous but the reciprocity is not true (any uniformly continuous function is not necessarily continuous):
	
	If the chosen distance is known (for example the absolute value for scalar functions such that $d=|\cdot|$ and $\delta=|\cdot|$) the previous definition notation change obviously a little bit:
	
	
	\begin{tcolorbox}[colframe=black,colback=white,sharp corners]
	\textbf{{\Large \ding{45}}Example:}\\\\
	The function $f(x) = x^2$ is continuous but not uniformly continuous
on the interval $E = [0,+\infty[$.\\

	We prove first that our function $f(x)$ is continuous on $E$. Remember first that:
	
	In our case we can therefore write and check that:
	
	
	Let us choose $x_0=a-1$ with $a>1$ and $\delta=\min(1,\varepsilon/2a)$ (note that $\delta$ depends on $x_0$ since $a$ does). Choose $x \in S$. Assume $|x-x_0|<\delta$. Then $|x-x_0|<1$ so $x<x_0+1$ so $x,x_0<a$ so:
	
	We prove now that $f(x)$ is not uniformly continuous on $E$, i.e.:
	
	Let $\varepsilon=1$. Choose $\delta>0$. Let $x_0=1/\delta$ and $x=x_0+\delta/2$. Then $|x-x_0|=\delta/2<\delta$ but:
	
	as required.
	\end{tcolorbox}
	
	\subsection{Opened and Closed Set}
	\textbf{Definition (\#\mydef):} Consider a set $E$ with a distance $d$. A subset $U$ of $E$ is named "\NewTerm{open subset}" if, for each element of $U$, there is a non-null distance $r$ for which all the elements of $E$ whose distance from this element is less than or equal to $r$, belong to $U$, which gives in mathematical language:
	
	In topology, an open subset is then only an abstract concept generalizing the idea of an open interval in the real line.
	\begin{tcolorbox}[title=Remark,colframe=black,arc=10pt]
	For recall, the symbol "|" means in this context: satisfies the property...
	\end{tcolorbox}	
	In practice, however, open sets are usually chosen to be similar to the open intervals of the real line. The notion of an open set provides a fundamental way to speak of nearness of points in a topological space, without explicitly having a concept of distance defined. 
	\begin{tcolorbox}[colframe=black,colback=white,sharp corners]
	\textbf{{\Large \ding{45}}Example:}\\\\
	The points $(x, y)$ satisfying $x^2 + y^2 = r^2$ are colored blue. The points $(x, y)$ satisfying $x^2 + y^2 < r^2$ are colored red. The red points form an open subset of the plane $\mathbb{R}^2$. The blue points form a boundary set. The union of the red and blue points is a closed set.
	\begin{figure}[H]
		\centering
		\includegraphics{img/analysis/opened_set.jpg}
	\end{figure}
	\end{tcolorbox}
	This definition may perhaps seem complicated but in fact, its real meaning is simpler than it seems. In fact, according to this definition, an open set in a topological space is nothing more than a set of contiguous points and without borders.
	
	The lack of border comes from the condition $r\neq 0$. Indeed, by reductio ad absurdum, if an open set $U$ had an edge, then for each point on it (the edge) it would still be possible to find a point not belonging to $U$ as close as we want from it. It follows that the distance $r$ becomes necessary therefore zero.

	\textbf{Definitions (\#\mydef):} 
	\begin{enumerate}
		\item[D1.] A "\NewTerm{closed subset}" is an "\NewTerm{open with edge}"

		\item[D2.] A "\NewTerm{neighbourhood}" of a point $E$ is a subset of $E$ containing an open subset containing this point.
	\end{enumerate}
	The definition of an open set can be simplified by introducing an additional concept, that of "open ball":
	
	\subsubsection{Balls}
	Given $x$ an element of $E$:
	
	\textbf{Definition (\#\mydef):} An "\NewTerm{open ball of center $x$ and radius $r>0$}" or "\NewTerm{metric ball of radius $r$ centered at $x$ without border}" is the subset of all the points of $E$ whose the distance $x$ is less than $r$, that we write in general:
	
	An open set can also be defined as a set for which it is possible to define an open ball at each point.
	
	Typically in the real plane where $d$ is the euclidean distance:
	
	\begin{figure}[H]
		\centering
		\includegraphics{img/analysis/open_set.jpg}
		\caption{An open ball of radius $r$, centered at the point $x$}
	\end{figure}
	An open set can also be defined as a set for which it is possible to define an open ball at each point.
	\begin{tcolorbox}[title=Remarks,colframe=black,arc=10pt]
	\textbf{R1.} The open such defined, form what we name an "\NewTerm{induced topology}" by the distance $d$ or also a "\NewTerm{metric topology}".\\
	
	\textbf{R2.} We name an "\NewTerm{open cover}" $U$ of $E$, a set of open of $E$ whose union is equal to $E$. In other words: A collection of open sets that collectively cover another set.\\
	
	Formally, if:
	
	is an indexed family of open sets $U_\alpha$, then $C$ is a cover of $X$ if:
	
	Visually in a naive way this gives:
	\begin{figure}[H]
		\centering
		\includegraphics{img/analysis/open_cover.jpg}
	\end{figure}
	\end{tcolorbox}	
	\textbf{Definition (\#\mydef):} A "\NewTerm{closed ball}" is similar to an open ball but differs in the sense that we include the elements located at a distance $r$ from the center:
	
	\begin{tcolorbox}[title=Remark,colframe=black,arc=10pt]
	For $0<r<r'$ the inclusions $_oB_x^r \subset B_x^r \subset B(x,r')$ are direct consequences of the definition of the open and closed ball.
	\end{tcolorbox}
	\begin{tcolorbox}[colframe=black,colback=white,sharp corners]
	\textbf{{\Large \ding{45}}Example:}\\\\
	The usual distance in $\mathbb{R}$ is given by $d(x,y)=|x-y|$. The balls are there simple intervals. For $x \in \mathbb{R}$ and $r\in \mathbb{R}_{+}^{*}$, we have:
	
	\end{tcolorbox}
	\textbf{Definition (\#\mydef):} A "\NewTerm{sphere}" is given by:
	
	\begin{tcolorbox}[title=Remark,colframe=black,arc=10pt]
	Since by definition $r>0$, open and closed balls are not empty because they contain at least their center. By cons, a sphere may be empty!
	\end{tcolorbox}
	\begin{tcolorbox}[colframe=black,colback=white,sharp corners]
	\textbf{{\Large \ding{45}}Example:}\\\\
	With $\mathbb{R}^n,\mathbb{C}^n$ we have seen in the previous examples we could set different distances. To distinguish them, we denote then by:
	
	So in $\mathbb{R}^2$ the closed balls with center O and of radius unit equivalent to the previous three formulations, have the following shapes (remember that $0<r\leq 1$ in this example):
	\begin{figure}[H]
		\centering
		\includegraphics{img/analysis/shape_some_distances.jpg}
		\caption{Examples of closed balls of unit radius with different distances}
	\end{figure}
	\end{tcolorbox}
	For example in statistics (see the section of the same name) we also use (among a lot of others) the Chi-2 distance given by:
	
	Or always in (multivariate) statistics (have a look to the Statistics section but also to the Industrial Engineering one) the "\NewTerm{Mahalanobis distance}"\index{Mahalanobis distance}\label{Mahalanobis distance}:
	
	\begin{tcolorbox}[title=Remarks,colframe=black,arc=10pt]
	The proof that the Mahalonobis distance is indeed a distance is straightforward. Indeed, without loss of generality, let us rewrite $x-\mu=\vec{x}$. Then:
	
	But as we have proved it in the section Statistics at page \pageref{positive semi-definitiveness of covariance matrix}, $\Sigma$ is positive semi-definite, hence $\Sigma^{-1}$ is most of time invertible (\SeeChapter{see section Statistics page \pageref{positive semi-definite matrix not always invertible}}). When it's invertible, all its eigenvalues are positive, then $\Sigma^{-1}$ is also positive semi-definite (\SeeChapter{see section Linear Algebra page \pageref{inverse definite positive matrix is also definite positive}}). This implies that:
	
	Therefore the Mahalonobis distance in indeed a distance. Another way to see it as engineer is just to consider that as $\Sigma^{-1}\vec{x}$ simply gives just another vector $\vec{y}$ such that:
	
	This it's just the euclidean distance between two different vectors!
	\end{tcolorbox}	
	For any other semi-positive definite matrix $A$ other than the variance-covariance matrix we also define the "\NewTerm{elliptic metric}"\index{elliptic metric}\label{elliptic metric}:
	
	
	Or in the Error Correcting Codes section we use the Hamming distance given by:
	
	and so on...
	
	\subsubsection{Partitions}
	Now that we have defined the concepts of balls, we can finally (almost) rigorously define the concepts of open and closed intervals (which in a space of more than one dimension are named "partitions") that we have so often used in the section of Functional Analysis and Differential and Integral Calculus.
	
	\textbf{Definition (\#\mydef):} Let $(X,d)$ of a metric space. We say that a subset $A$ of $X$ is "bounded" if there is a closed ball $_fB_{r_0}^r(X)$ such that $A \subseteq _fB_{r_0}^r(X)$:
	
	Given the previous note on balls inclusions, it is clear that we can replace the word "closed" by"open". Moreover the triangle inequality implies that the bounded character of $A$ does not depend on the choice of $x_0$ (with a $x_0^{\prime}$ we simply need to replace $r$ by $r'=r+d\left(x_0,x_0^{\prime}\right)$).
	\begin{tcolorbox}[colframe=black,colback=white,sharp corners]
	\textbf{{\Large \ding{45}}Example:}\\\\
	The odd–even topology is the topology where $X = \mathbb{N}$ and:
	
	the unbounded partitions of radius $r<1$. \\
	
	 Therefore we see that unless $P$ is trivial, at least one set in $P$ contains more than one point, and the elements of this set are topologically indistinguishable: the topology does not separate points!
	\end{tcolorbox}
	
	\textbf{Definitions (\#\mydef):}
	\begin{enumerate}
		\item[D1.] Let $X$ be a set and $(Y,d)$ a metric space. If $X$ is a set, we say that a function $f: X\mapsto Y$ is "bounded" if its image $f (X)$ is bounded (the case of the sine or cosine function, for example).
		
		\item[D2.] Given $(E, d)$ a metric space, and given $A$ a non-empty subset of $E$. For any $u\in E$ we note $d(u, A)$ and name "\NewTerm{distance $u$ to $A$}", the positive real number:
		
		We extend the concept by writing:
		
		If $A$ and $B$ are two parts (subsets) of $E$ we have respectively (perhaps this is more understandable in this way for some readers...)
		
		The reader must take care here to interpret $d(A,B)$ as the infinimum of the distance between the sets $A$ and $B$, because the distance between the parties does not always define a distance in the usual way on the part for example of $\mathcal{P}(\mathbb{R})$.
		
		Indeed, if we take again our famous example:
			
	 	we have $d(A,B)=0$ when $n \rightarrow 0$ while $A\neq B$.
	 	\begin{tcolorbox}[title=Remarks,colframe=black,arc=10pt]
		\textbf{R1.} If the reader has well understood the definition of "parts" (and especially the previous example) he has probably noticed that it does not necessarily always exist a $a\in A$ such that $d(u,A)=d(u,a)$. Accordingly, we write:
		
		Moreover, if such an $\alpha$ exists, it is obviously not necessarily unique.\\
		
		\textbf{R2.} It should be remembered that this distance also meets the $5$ axioms of distances (we can give the proof on request)!
		\end{tcolorbox}	
	
		\item[D3.] Given $(E, d)$ a metric space, and let $A$ be a part (subset) of $E$. We name "\NewTerm{adhesion}" of $A$ and denote by $\text{adh} (A)$ the subset of $E$ defined by:
		
		For example, the adhesion of the part (subset) of rational numbers $\mathbb{Q}$ (part $A$) of $\mathbb{R}$  (the metric space $E$) is a subset of $\mathbb{R}$ itself since any real number is the limit of a rational.
		
		Especially, since $\forall u\in A:\quad =+\infty$, we have $\text{adh}(\varnothing)=\varnothing$, and since $\forall u\in E:\quad d(u,E)=0$, we have $\text{adh}(E)=E$.
		
		\begin{tcolorbox}[title=Remarks,colframe=black,arc=10pt]
		\textbf{R1.} Any element of the set $\text{adh}(A)$ is named "adherent point" of $A$.\\
		
		\textbf{R2.} We say that a part $A$ of $E$ is and "\NewTerm{closed part}" if it is equal to its adherence.\\
		
		\textbf{R3.} We say that a part $A$ of $E$ is an "\NewTerm{open part}" if its complementary relatively to $E$:
		
		is closed.
		\end{tcolorbox}	
	\end{enumerate}
	It follows from the definitions that (without proof):
	
	and:
	
	with some properties (supposed as very obvious but we can give the proof on request):
	\begin{enumerate}
		\item[P1.] If $A\subset E$ and $B\subset E$ satisfies $A\subset B$, then we have:
		
		
		\item[P2.] For all $A\subset E$, any $u\in E$ we have:
		
		The latter property has for corollary (obvious and therefore without proof excepted on request):
		If for any $u\in E$, we have $d(u,A)=d(u,B)$ and $A,B\neq \varnothing$, we then have:
		
	\end{enumerate}
	
	\subsubsection{Formal Ball}
	The concept of distance from a point to a set gives the possibility to extend the notions of ball and sphere see previously. We will see now the basis concepts of a "\NewTerm{formal ball}" also named "\NewTerm{generalized ball}".
	
	\begin{enumerate}
		\item[D1.] Given $A\neq \varnothing$ and given a $r>0$. We name "\NewTerm{generalized open ball}" of center $A$ of radius $r$, the following set:
		
		and respectively "\NewTerm{generalized closed ball}":
		
		and respectively  a "\NewTerm{generalized sphere}":
		
		
		\item[D2.] Given $(E, d)$ a metric space and let $A, B$ be two non-empty parts (subsets) of $E$. We denote by $g (A, B)$ and name "\NewTerm{gap}" of $A$ to $B$, the real number greater than or equal to zero such that:
		
		\begin{tcolorbox}[title=Remark,colframe=black,arc=10pt]
		The triangle inequality $g(A,B)\leq g(A,C)+g(C,B)$ is not valid in the context of gaps. To prove it, a single example that contradicts this inequality is sufficient.
		\end{tcolorbox}
		\begin{tcolorbox}[colframe=black,colback=white,sharp corners]
		\textbf{{\Large \ding{45}}Example:}\\\\
		In $\mathbb{R}$ let us take $A=\{0,1\},B=\{2,3\},C=\{1,3\}$ then we have:
		
		\end{tcolorbox}	
	\end{enumerate}
	
	\subsubsection{Diameter}
	\textbf{Definition (\#\mydef)}: Given $(E, d)$ a metric space and $A$ a non-empty part (subset) of $E$. We denote $\text{diam}(A)$ and name "\NewTerm{diameter}" of $A$, the positive non-zero real number:
	
	Every non-empty part (subset) $A$ of a metric space satisfying $\text{diam}<+\infty$ will also be say "bounded".
	\begin{tcolorbox}[title=Remark,colframe=black,arc=10pt]
	We consider the empty set $\varnothing$ as bound set of diameter $A$.
	\end{tcolorbox}	
	If the whole metric space $(E,d)$ is bounded, we say that the distance $d$ is bounded. For example, the discrete distance is limited, the usual distance on $\mathbb{R}$ is not.
	
	We also have the following properties (the first two are usually trivial, the third one comes from the definition of the diameter itself):
	\begin{enumerate}
		\item[P1.] $\text{diam}(A)=0 \Leftrightarrow A=\{a\}$ or $A=\varnothing$
		
		\item[P2.] $A\subset B \Rightarrow \text{diam}(A)\leq \text{diam}(B)$
		
		\item[P3.] $\text{diam}(_fB_x^r)\leq 2r,\text{diam}(_oB_x^r)\leq 2r,\text{diam}(S_x^r)\leq 2r$
		
		\begin{tcolorbox}[colback=red!5,borderline={1mm}{2mm}{red!5},arc=0mm,boxrule=0pt]
		\bcbombe Caution!!! Concerning the latter property, the reader must take the habit of thinking with the Euclidean distance. The first common pitfall is to think that the second diameter (that of the open ball) should be strictly less but that would be forgetting that the board has no thickness strictly speaking! 
		\end{tcolorbox}
		
		There is also often a understanding problem with $\text{diam}(S_x^r)\leq 2r$. To be convinced just take the discrete distance (that for two points that are not confused is equal to $1$, otherwise $0$). Thus, in a metric space where we take $S_x^r$ with $r=1$, we have indeed $\text{diam}(S_x^1)\leq 1$ (that is an interesting case because almost completely counter-intuitive).
		
		\item[P4.] $\text{diam}(A\cup B)\leq \text{diam}(A)+g(A,B)+\text{diam}(B)$
		
		To be convinced, in $\mathbb{R}$ take $A=B$, then we have (trivial strict inferiority):
		
		
		\item[P5.] $A$ is bounded if and only if: $\exists r>0,\exists x\in E:\quad A\subset _oB_x^r$
	\end{enumerate}

	\textbf{Definition (\#\mydef):} We name "\NewTerm{Hausdorff excess}" or "\NewTerm{Hausdorff distance}" from $X$ to $Y$:
	
	that we found often in the literature with the more condensed notation:
	
	or much more explicitly:
	
	\begin{figure}[H]
		\centering
		\includegraphics{img/analysis/hausdorff_distance.jpg}
		\caption[]{Components of the calculation of $d_H$ between $X$ and $Y$ (source: Wikipedia)}
	\end{figure}
	\begin{tcolorbox}[colframe=black,colback=white,sharp corners]
	\textbf{{\Large \ding{45}}Example:}\\\\
	Let us take $X\subset \mathbb{R}^2$ as the unit radius circle centered at the origin and $Y$ to the square circumscribing it. Elementary geometry concepts obviously leads to finding that the Hausdorff distance between the circle and the square is therefore:
	\begin{figure}[H]
		\centering
		\includegraphics{img/analysis/hausdorff_distance_highschool_example.jpg}
		\caption{High-school example of a Hausdorff distance in the plane}
	\end{figure}
	technically:
	
	\end{tcolorbox}
	\begin{tcolorbox}[title=Remark,colframe=black,arc=10pt]
	We have generally $e(X,Y)\neq e(Y,X)$ and these quantities may not be finite.
	\end{tcolorbox}
	
	\subsection{Varieties}\label{varieties}
	We now introduce the "varieties". These are topological spaces that are "locally as $\mathbb{R}^2$" (our space for example ...), that is locally euclidean.
	
	\textbf{Definitions (\#\mydef):}
	\begin{enumerate}
		\item[D1.] A "\NewTerm{topological variety of dimension $n$}" is a Hausdorff space $M$ such that for every $p\in M$ there exists an open neighbourhood $U\subset M$ with $p\in U$, an open neighbourhood $U' \subset \mathbb{R}^n$ and a homeomorphism such that:
		
		\item[D2.] A "\NewTerm{homeomorphism}" between two spaces is a continuous bijection whose inverse is also continuous.

		\item[D3.] The pairs $(U,\varphi)$ are named "\NewTerm{maps}", $U$ being the "\NewTerm{domain of the map}" and $\varphi$ the "\NewTerm{coordinate application}". Instead of "map" sometimes we say also "coordinate system".

		\begin{tcolorbox}[title=Remark,colframe=black,arc=10pt]
		We will denote by $\dim(M)$ the dimension of a topological variety. Therefore:
		
		\end{tcolorbox}
	
		\item[D4.] Given $M$ be a topological variety of dimension $n$. A family $A$ of maps of $M$ family is named a  "\NewTerm{atlas}" if for each $x\in M$, there is exists a map $(U, \varphi)\in A$ such as $x\in U$.
	\end{enumerate}
	If $(U_1,\varphi_1),(U_2,\varphi_2)$ are two maps of $M$ such as $U_1\cap U_2\neq \varnothing$, then the application of map changes:
	
	\begin{figure}[H]
		\centering
		\includegraphics{img/analysis/homeomorphism_of_maps.jpg}
		\caption{Maps homeomorphism}
	\end{figure}
	is obviously also a homeomorphism. More "geometrically" it looks like this:
	\begin{figure}[H]
		\centering
		\includegraphics[scale=0.6]{img/analysis/maps_homeomorphism.jpg}
		\caption[Intuitive maps homeomorphism]{Intuitive maps homeomorphism (source: ?)}
	\end{figure}
	
	\pagebreak
	\subsubsection{Subvariety}
	\textbf{Definition (\#\mydef):}A subset of a variety is itself a variety named a "\NewTerm{subvariety}\index{subvariety}\label{subvariety}". 

	\begin{tcolorbox}[colframe=black,colback=white,sharp corners]
	\textbf{{\Large \ding{45}}Example:}\\\\
	A sphere of the three-dimensional Euclidean space $\mathbb{R}^3$ is an (algebraic) variety since it is defined by a polynomial equation. For example and is obviously smooth and locally euclidean:
	
	defines the sphere of radius $1$ centered at the origin. Its intersection with the $xy$-plane is a circle given by the system of polynomial equations:
	
	Hence the circle is itself an algebraic variety, and a subvariety of the sphere, and of the plane as well.
	\end{tcolorbox}
	

	\pagebreak
	\subsubsection{Surfaces Homeomorphism}
	\textbf{Definition (\#\mydef):} In the mathematical field of topology, a "\NewTerm{homeomorphism} or "\NewTerm{topological isomorphism}" is a continuous function between topological spaces that has a continuous inverse function. Roughly speaking, a topological space is a geometric object, and the homeomorphism is a continuous stretching and bending of the object into a new shape. Thus, a square and a circle are homeomorphic to each other, but a sphere and a torus are not. 
	
	More formally, remember that an application $\varphi: X \mapsto Y$ between two topological spaces is named a homeomorphism if it has the following properties:
	\begin{enumerate}
		\item $\varphi$ is a bijection (one-to-one and onto)
		
		\item $\varphi$ is continuous
		
		\item The reciprocal function $\varphi^{-1}$ is continuous 
	\end{enumerate}
	
	Can we say that a square (being a special map in $\mathbb{R}^2$) is homeomorph to circle (being another special map in $\mathbb{R}^2$), or a torus to a cup of tee... If this is possible we must be able to find a closed form bijective expression between the two surfaces.
	
	As the pure theoretical concepts are very not friendly in our point of view let us begin with a two dimension special case. Let us show (prove) first that we can transform all interior points of  square of side $1$ into all interior points circle of radius $1$. This is represented by the  know figure:
	\begin{figure}[H]
		\centering
		\includegraphics{img/analysis/isomorphism_circle_square.jpg}
	\end{figure}
	Such mappings have particular interest in industrial design or just simply for communication purposes (Photoshop effects or Statistics charts deformation as we do many times in the \texttt{R} Software):
	\begin{figure}[H]
		\centering
		\includegraphics[scale=0.5]{img/analysis/chessboard_isomorphic_circle_square.jpg}
	\end{figure}
	or for defishing fish eyes captors, picture or security mirrors:
	\begin{figure}[H]
		\centering
		\includegraphics[scale=1.25]{img/analysis/defishing.jpg}
	\end{figure}
	Recall that we defined unit disc as the set:
	
	If we think of the unit disc as a continuum of concentric circles with radii growing from zero to one, we can parametrize the unit disc as the set:
	
	In doing so, we introduced a parameter $t$ this is the distance of point $(u,v)$ to the origin.
	\begin{figure}[H]
		\centering
		\includegraphics{img/analysis/continnum_disc.jpg}
	\end{figure}
	In analogy to the circular continuum of the unit disc, one can write the square region $[-1,1] \times [-1,1]$ as the set:
	
	In other words, the square can be considered as a continuum of concentric shrunken FG-squircles (\SeeChapter{see section Analytical Geometry page \pageref{fg squircle}}).
	\begin{figure}[H]
		\centering
		\includegraphics{img/analysis/continnum_square.jpg}
	\end{figure}

	Topologists denote the proof that the interior points of two geometries are homeomorph in this special case using the following notation
	
	We will now show that $\mathring{\mathcal{D}}\mapsto \mathring{\mathcal{D}}$ as it is the most common case in practice in our point of view. That is to say:
	\begin{figure}[H]
		\centering
		\includegraphics{img/analysis/mapping_circle_to_square.jpg}
	\end{figure}
	We can establish a correspondence between the unit disc and the square region by mapping every circular contour in the interior of the disc to a squircular contour in the interior of the square. In other words, we map contour curves in the circular continuum of the disc to those in the squircular continuum of the square. This can be done by equating the parameter $t$ of both sets to get the equation:
	
	We name this equation the "\NewTerm{squircularity condition}" for mapping a circular disc to a square region.
	
	It is easy to derive the FG-Squircular mapping by combining the squircularity condition:
	
	That we can also write:
	
	Using radial to cartesian coordinates (\SeeChapter{see section Vector Calculus page \pageref{polar coordinates}}):
	
	Therefore by equivalence we get:
	
	After substitution of parameter $t$, we get:
	

	In other words, this is a radial mapping that converts circular contours on the disc to squircular contours on the square.
	
	We shall now derive the inverse equations for the FG-Squircular mapping. But as it is boring to write in \LaTeX{} and it is not used to much in practice we will omit the latter for the moment.
	
	\subsubsection{Differential Varieties}
	\textbf{Definitions (\#\mydef)}:
	\begin{enumerate}
		\item[D1.] A "\NewTerm{differentiable variety}" is a topological space $M$ where the applications $\varphi$ are of class $\mathcal{C}^{+\infty}$.

		\item[D2.] A "\NewTerm{diffeomorphism}" is an application where $\varphi: U\mapsto U' $ where $U,U'$ are open domains of $\mathbb{R}^n$ and if $\varphi$ is a homeomorphism and furthermore $U,U'$ are differentiable!
		\begin{tcolorbox}[title=Remark,colframe=black,arc=10pt]
		"Differentiable" in this context will always mean of class $\mathcal{C}^{+\infty}$.
		\end{tcolorbox}

		\item[D3.] Given a topological variety $M:=M^n$ (to simplify the notations), two maps $(U_1,\varphi_1),(U_2,\varphi_2)$ of $M$ are named \NewTerm{compatible maps} (more precisely: compatible of class $\mathcal{C}^{+\infty}$ if one of these two properties is satisfied:
		\begin{enumerate}
			\item[P1.] $U_1\cap U_2\neq \varnothing$ and the application $\varphi_2\circ \varphi_1^{-1}$ of map changes is a diffeomorphism.

			\item[P2.] $U_1\cap U_2 =\varnothing$
		\end{enumerate}
		An atlas $A$ of $M$ is differentiable if all maps of $A$ are compatible between them.
	\end{enumerate}
	\begin{tcolorbox}[title=Remark,colframe=black,arc=10pt]
	Given a differentiable atlas, it is sometimes necessary to complete it: we say that a map of $M$ is compatible with a differentiable atlas if it is compatible with every map of $A$. An atlas of $A$ is a "\NewTerm{maximal atlas}" if every compatible map of $A$ belongs already to $A$. A maximal atlas is named a "\NewTerm{differentiable structure}".
	\end{tcolorbox}

	\begin{flushright}
	\begin{tabular}{l c}
	\circled{70} & \pbox{20cm}{\score{4}{5} \\ {\tiny 12 votes,  71.67\%}} 
	\end{tabular} 
	\end{flushright}

	%to make section start on odd page
	\newpage
	\thispagestyle{empty}
	\mbox{}
	\section{Measure Theory}\label{measure theory}
	\begin{tcolorbox}[colback=red!5,borderline={1mm}{2mm}{red!5},arc=0mm,boxrule=0pt]
	\bcbombe Caution! The level of abstraction and of motivation required for reading and understanding this section is quite high for engineers (target audience of this book for recall). The reader should be comfortable with the concepts seen in the section Set Theory as well as the one one Topology. We also apologize for the actual lack of figures. 
	\end{tcolorbox}
	\lettrine[lines=4]{\color{BrickRed}T}he measure, in the topological sense, will allow us to generalize the elementary notion of measure of a segment or area (in the Riemann sense, for example) and is inseparable from the new theory of integration that will build Lebesgue from the years to 1901-1902 and we will address here to build mathematical tools much more powerful than the simple Riemann integral (\SeeChapter{see section Differential and Integral Calculus page \pageref{riemann integral}}) with practical and numerical example in MATLAB\textsuperscript{TM}.
	
	The philosophers of science who developed measurement theory were largely concerned with epistemic questions like: we can't observe correlations between physical objects and real numbers, so how can the use of real numbers be justified in terms of things we can observe? Indeed, privileges a single unit of mass, involves real numbers in the facts of
mass. Why is the latter bad?: 
	\begin{enumerate}
		\item Real numbers are abstract and therefore causally inert
		
		\item Real numbers don't fundamentally exist
		
		\item Real numbers are constructed entities, and constructed entities can't be involved  in fundamental facts
	\end{enumerate}

	The Measure Theory will also allow us to rigorously define the concept of measurement (no matter what is the measure) and so return to the important results of the study of probabilities (\SeeChapter{see section Probabilities page \pageref{probabilities}}). Indeed, we will see (we will define the vocabulary that follows just now further below) why $(U, A, P)$ is a "\NewTerm{probability space}" where $A$ is in fact a "\NewTerm{tribe}" on $U$ and $P$ a measure on the measurable space $(U , A)$.
	
	\subsection{Measurable Spaces}
	When in mathematics we calculate derivatives, primitives or simply count stuff, we carry implicitly a measure of an object or set of objects. Rigorously, mathematicians want to define how the measured thing can be structured, how to make a measurement of it and the properties resulting!
	
	\textbf{Definitions (\#\mydef):}
	\begin{enumerate}
		\item[D1.] Let $E$ be a set, a "\NewTerm{tribe}" on $E$ is a family $\mathcal{A}$ (this notation comes from the fact that many people speak of "\textbf{A}lgebra sets" instead of "tribe") of subsets of $E$ satisfying the following axioms:
		\begin{enumerate}
			\item[A1.] $E\in \mathcal{A}$ (see examples below - $E$ being one of the possible elements of $\mathcal{A}$).
			
			\item[A2.]  If $A$ is a member of a tribe then:
			
			This means that $\mathcal{A} $ is "\NewTerm{stable by transition to complementary}". This axiom implies that the empty set is always an element of a tribe!
			
			\item[A3.] For any sequence $(A_n)$ of elements of $\mathcal{A}$ we have:
			
			 We then say that $\mathcal{A}$ is then "\NewTerm{stable by countable union}".
		\end{enumerate}
		For example, the graduating from a simple ruler of measurement ... satisfies these three axioms!
	\begin{tcolorbox}[title=Remarks,colframe=black,arc=10pt]
	\textbf{R1.} We write $E\in \mathcal{A}$ because we consider with this notation $E$ not anymore as a subset of $\mathcal{A}$  but as an element of $\mathcal{A}$!\\
	
	\textbf{R2.} The uncountable cases are typical of topology, statistics or integral calculus!
	\end{tcolorbox}	
	
	\item[D2.] The pair $(E,\mathcal{A})$ is named "\NewTerm{measurable space}" and we say that the elements of $\mathcal{A}$ are "\NewTerm{measurable sets}".
	
	\item[D3.] If in the third axiom we require that $\mathcal{A}$ is stable under finite (uncountable) union then we impose the more general notion of "\NewTerm{$\sigma$-algebra}\label{sigma algebra}". Thus, a tribe is necessarily contained in an $\sigma$-algebra (but the opposite is not true just because the axiom is stronger) such that we can write:
	
	\begin{tcolorbox}[title=Remark,colframe=black,arc=10pt]
	In the field of probabilities, $E$ is assimilated to the Universe of events and $\mathcal{A}$ to a family of events and we speak then of "\NewTerm{probabilistic space}" or simply of... "\NewTerm{measurable space}".
	\end{tcolorbox}	
	\end{enumerate}
	\begin{tcolorbox}[colframe=black,colback=white,sharp corners]
	\textbf{{\Large \ding{45}}Examples:}\\\\
	E1. Given $E=\{1,2\}$ a set of cardinal 2... The only two tribe $\mathcal{A}$ that satisfy the three axioms are:
	
	There are no other tribes for the set $E$ as these two (the normal one, and the maximum one), because we must not forget that the union of each elements of the tribe must also be in the tribe (axiom A3), and also the complement of a member (axiom A2).\\
	
	We also see from this example that if $E$ is set then $\{E,\varnothing\}$ is indeed a tribe!\\
	
	E2. The set of parts of $E$, denoted $\mathcal{P}(E)$ is also a tribe (dixit previous example).
	\end{tcolorbox}
	A tribe $\mathcal{A}$ is also "\NewTerm{stable by the union of the finite complementaries}". Indeed, if $(A_n)$ is a sequence of elements of $\mathcal{A}$ we have (trivial when taking for example the previous first example):
	
	A tribe is also "\NewTerm{stable by finite intersection}", that is to say (trivial also by taking the previous first example):
	
	which brings to the property that a tribe is stable by finite unions and intersections. Especially, if we take two elements of a tribe $A,B\in \mathcal{A}$, then $A\setminus B\in \mathcal{A}$ with for recall (\SeeChapter{see section Set Theory page \pageref{symmetric difference}}):
	
	\begin{tcolorbox}[title=Remark,colframe=black,arc=10pt]
	Most readers should probably easily see with the previous first example that if $(\mathcal{A}_i)_I$ is a family of tribes on $E$ the $\bigcap_I \mathcal{A}_i$ is also a tribe (the verification is almost immediate).
	\end{tcolorbox}	
	Well it is nice to play with potatoes and sub-potatoes... and their complementary but let us continue...
	
	\textbf{Definition (\#\mydef):} Given $E$ a set and $\mathcal{B}$ a family of subsets of $\mathcal{P}(E)$ such that $\mathcal{B}\subset \mathcal{P}(E)$. We denote by definition:
	
	the "\NewTerm{generated tribe}" by $\mathcal{B}$. Therefore $\sigma(\mathcal{B})$ is by definition the smallest tribe containing $\mathcal{B}$ (and by extension the smallest tribe of $E$).

	Below are three small examples that gives the opportunity to check if what precedes has been well understood and that also gives the possibility to highlight important results for what will follow:
	\begin{tcolorbox}[colframe=black,colback=white,sharp corners]
	\textbf{{\Large \ding{45}}Examples:}\\\\
	Given a set $E$ and $A\subset E,A \neq E$ and also $\mathcal{B}=\{A\}$ then (when $A$ is seen as a subset of $E$ as given by the statement of a family of subsets!):
	
	
	E2. If $\mathcal{A}$ is a tribe on $E$ then:
	

	E3. Given $E=\{1,2,3,4\}$ and $A=\{\{1,2\},{3}\}$ we then have (take care because now $A$ is a family of parts (subsets) and not only a unique subset!) the following generated tribe:
	
	Rather than determining this tribe by seeking the smallest tribe $\mathcal{P}(E)$ containing $A$ (which would be laborious) we play with the axioms defining a tribe to easily find it.
	
	So therefore we find well in $\sigma(A)$ at least the obligatory empty set $\{\varnothing\}$ and also:
	
	following the axiom A1 and:
	
	itself by the definition of $\sigma(A)$ and the complementaries of:
	
	following the axiom A2 and also the unions:
	
	following the axiom A3.
	\end{tcolorbox}
	\textbf{Definition (\#\mydef)}: Let $E$ be a topological space (\SeeChapter{see section Topology page \pageref{topological space}}). We denote by $\mathcal{B}(E)$ the tribe generated by the open sets of $E$. $\mathcal{B}(E)$ is named the "\NewTerm{borelian tribe}" of $E$. The elements of $\mathcal{B}(E)$ are named the "\NewTerm{borelians}" of $E$. 
	
	\begin{tcolorbox}[title=Remarks,colframe=black,arc=10pt]
	\textbf{R1.} The notion of borelian tribe is especially interesting because it is necessary for the definition of "Lebesgue tribe" and afterwards to the "Lebesgue measure" that will lead us to define the  famous "Lebesgue integral"!\\
	
	\textbf{R2.} The tribe $\mathcal{B}(E)$ being stable by going to the complementary, it also contains all closed subsets.\\
	
	\textbf{R3.} If $E$ is a topological space with a finite basis, $\mathcal{B}(A)$ in generated by the opens of the basis.
	\end{tcolorbox}	
	\begin{tcolorbox}[colframe=black,colback=white,sharp corners]
	\textbf{{\Large \ding{45}}Example:}\\\\
	If $\mathbb{R}$ designates the space provided of real numbers with the Euclidean topology (\SeeChapter{see section Topology page \pageref{euclidean topology}}), the family of open intervals with rational bounds is a "\NewTerm{countable base}" (given the bounds...) of $\mathbb{R}$ and therefore generates $\mathcal{B}(\mathbb{R})$ . Same thing for $\mathbb{R}^d$ with for countable basis the family of open spaces with rational bounds.
	\end{tcolorbox}
	\begin{theorem}
	Let us now consider a dense set (\SeeChapter{see section Topology page \pageref{dense set}}) in $\mathbb{R}$. The following families generate $\mathcal{B}(\mathbb{R})$:
	
	\end{theorem}
	\begin{dem}
	Given (the family of open subsets):
	
	We have obviously:
	
	Furthermore:
	
	Therefore the intervals of the type $[a,b[$ with $a$ and $b$ in $\mathcal{S}$ also belongs to $\sigma(\mathcal{F})$. Therefore, if we generalize, with $x<y$, it exists a sequence $(a_n)$ of elements of $\mathcal{S}$ decreasing to $x$ and a sequence $(b_n)$ of elements of $\mathcal{S}$ increasing to $y$ such that:
	
	which bring in the same way as $E\in \mathcal{A}$ that $\mathcal{B}(\mathbb{R})\subseteq\sigma(\mathcal{F})$. Other cases can be treated analogously.
	\begin{flushright}
		$\blacksquare$  Q.E.D.
	\end{flushright}
	\end{dem}
	\begin{theorem}
	Given $(E, \mathcal{A})$ a measurable space and $A\subseteq E$ (and $A\in \mathcal{A}$) (where $A$ is therefore considerate as a subset and non as an element!). The family $\{A\cap B| B\in \mathcal{A}\}$ is a tribe on $A$ named "\NewTerm{trace tribe}" of $\mathcal{A}$ on $A$, that we will denote by $A\cap \mathcal{A}$. Furthermore, if $A\in \mathcal{A}$, the trace tribe is formed by the measurable elements contained in $A$.
	\end{theorem}
	\begin{dem}
	We will do a proof by the example (... yes it is not a real proof...). For this we check the three points that define a tribe:
	\begin{enumerate}
		\item $A=(E\cap A) \Rightarrow A\in (A\cap \mathcal{A})$
		
		\item Given $B\in\mathcal{A},A \setminus (A\cap B)=A\setminus B=A \cap B^c=A \cap B^c$ and therefore $A\setminus (A\cap B)\in (A\cap \mathcal{A})$
		\begin{tcolorbox}[colframe=black,colback=white,sharp corners]
		\textbf{{\Large \ding{45}}Example:}\\\\
		Given $E=\{1,2,3\}$ then (a tribe among others - do not forget the stability by union!):
		
		Let us choose $A=\{1,2\},B=\{2,3\}$ (it is obvious that $\{A\cap B|B\in \mathcal{A}\}$ is a tribe on $A$). Then:
		
		and we have well $\{1\}\in A$ and also $A\in (A\cap \mathcal{A})$.
		\end{tcolorbox}
		
		\item Given $(A\cap B_n)$ a sequence of elements of $A\cap \mathcal{A}\; (B_n\in \mathcal{A})$ then:
		
		The last statement of the proposition will be supposed as obvious (if not, let us know!).
	\end{enumerate}
	\begin{flushright}
		$\blacksquare$  Q.E.D.
	\end{flushright}
	\end{dem}
	Given now $E$ a set, $\mathcal{C}$ a family of subsets of $E$ and $A\subseteq E$ non empty. We denote by $A\cap \mathcal{C}$, the trace $\mathcal{C}$ on $A$ and $\sigma_A(A\cap \mathcal{C})$ the tribe generated on $A$. Therefore:
	
	\begin{tcolorbox}[colframe=black,colback=white,sharp corners]
	\textbf{{\Large \ding{45}}Example:}\\\\
	Given the set $E=\{1,2,3,4\},\mathcal{C}=\{\{1,2\},\{3\}\},A=\{3,4\}$ then:
	
	and let us check that $A\cap \sigma(\mathcal{C})=\sigma_A(A\cap \mathcal{C})$:
	
	So the equality is satisfied!
	\end{tcolorbox}
	A trivial corollary of this equality is that if we consider a topological space $ E$ and $A\subseteq E$ with the induced topology, then:
	
	We will study more in details $\sigma$-algebra in measurement theory but first let us recall that a tribe (sometimes named "algebra of sets") on $E$ must satisfy the following properties:
	\begin{enumerate}
		\item[P1.] Has to contain $E$
		\item[P2.] Must be stable by the complementary
		\item[P3.] Must be stable by countable union or intersection
	\end{enumerate}
	and a $\sigma$-algebra on $E$ is less restrictive than a tribe as it has to satisfy:
	\begin{enumerate}
		\item[P1.] Has to contain $E$
		\item[P2.] Must be stable by the complementary
		\item[P3.] Must be stable by finite (uncountable) union or intersection
	\end{enumerate}
	Let us recall (\SeeChapter{see section Set Theory page \pageref{symmetric difference}}) that if we have $E$ that is a set, then for every $A,B\subseteq E$ we define the symmetric difference $A\Delta B$ between $A$ and $B$ by:
	
	Trivial properties are as follows:
	\begin{enumerate}
		\item[P1.] A $\sigma$-algebra is stable by symmetric difference ($A,B\in \mathcal{A}$ we have $A\Delta B\in \mathcal{A}$)
		
		\item[P2.] $A\Delta B=B\Delta A$
		\item[P3.] $A^c\Delta B^c=A\Delta B$
			
		\item[P4.] $A\Delta B=(A\cup B)\setminus (A\cap B)$
	\end{enumerate}
	\begin{theorem}
	If $\mathcal{B}$ is a $\sigma$-algebra over $E$, then $(\mathcal{B},\Delta,\cap)$ is a "\NewTerm{Boolean ring}" (or "Boolean algebra" but be careful with the term "algebra" here which can cause confusion with the corresponding structure in Set theory) with $\varnothing$ and $E$ as neutral "additive" element ($\Delta$) and respectively" multiplicative" ($\cap$).
	\end{theorem}
	\begin{tcolorbox}[title=Remark,colframe=black,arc=10pt]
	For reminders on the items listed in the preceding paragraph, the reader can refer to the section of Set Theory page \pageref{set theory} and the subsection of Boolean Algebra (\SeeChapter{see section Formal Logic Systems page \pageref{boolean algebra}}).
	\end{tcolorbox}	
	\begin{dem}
	The "addition" $\Delta$ is associative because developing we get (this can verified by an arrow diagram if needed - the "potatoes"):
	
	and the latter expression is stable by permutation (commutation) of $A$ and $C$ (same method of verification). Therefore:
	
	We check that $\varnothing$ is neutral with respect to the symmetric difference (the proof that $E$ is neutral with respect to inclusion is obvious). It is trivial that:
	
	$(\mathcal{B},\Delta,\varnothing)$ is therefore well an Abelian group with respect to the law $\Delta$ (symmetric difference).
	
	Finally $\cap$ is distributive with respect to $\Delta$. Indeed:
	
	What makes $(\mathcal{B},\Delta,\varnothing)$ is indeed a ring (furthermore of a commutative ring!).
	\begin{flushright}
		$\blacksquare$  Q.E.D.
	\end{flushright}
	\end{dem}

	\pagebreak
	\subsubsection{Monotone Classes}
	\textbf{Definition (\#\mydef)}: Let $E$ be a set. A "\NewTerm{monotone class}" on $E$ is a family $\mathcal{C}$ of subsets of $E$ satisfying the following axioms:
	\begin{enumerate}
		\item[A1.] $E\in \mathcal{C}$
		\item[A2.] $A,B\in \mathcal{C}$ and $A\subseteq B \Rightarrow \setminus A\in \mathcal{C}$
		\item[A3.] If $(A_n)$ is an increasing sequence (take care to the word "increasing"!) of elements of $\mathcal{C}$ then $\displaystyle\bigcup_{i=1}^{+\infty} A_i\in \mathcal{C}$ (stable by countable increasing union).
	\end{enumerate}
	\begin{tcolorbox}[title=Remarks,colframe=black,arc=10pt]
	\textbf{R1.} An increasing sequence of sets is: $A_1\subseteq A_2\subseteq A_3 ...$\\
	
	\textbf{R2.} The first two axioms imply that $\mathcal{C}$ is by complementary.\\
	
	\textbf{R3.} The three axioms together leads to that the monotonous class is stable by decreasing intersection. A way to check this to take the complement of each element of the increasing sequence to fall back on the decreasing sequence and vice versa.
	\end{tcolorbox}
	Every $\sigma$-algebra is a monotone class, because $\sigma$-algebras are closed under arbitrary countable unions and intersections.
	
	Therefore:
	
	In the same way as for the tribes, if we consider a family $(\mathcal{C}_i)_I$ of monotone class on $E$. Then $\bigcap_I \mathcal{C}_i$ is a monotone class (the proof is verified immediately by the three previous axioms).
	\begin{tcolorbox}[colframe=black,colback=white,sharp corners]
	\textbf{{\Large \ding{45}}Example:}\\\\
	Given $E$ a set, $\mathcal{P}(E)$ is a monotone class on $E$. More generally, a tribe is a monotone class.

	Equivalently to tribes, let us consider a set $E$ and $\mathcal{C}\subseteq \mathcal{P}(E)$. Given $\mathcal{S}$ the family of all monotone class containing $\mathcal{C}$, $\mathcal{S}$ is not empty because $\mathcal{P}(E)\in \mathcal{S}$. We denote by:
	
	the monotone class generated by $\mathcal{C}$. Therefore $\mathcal{M}(\mathcal{C})$ is the smallest monotone class containing $\mathcal{C}$ (and satisfying obviously the previous axioms).
	\end{tcolorbox}
	\begin{tcolorbox}[title=Remark,colframe=black,arc=10pt]
	If $E$ is a set and $\mathcal{C}\subseteq \mathcal{P}(E)$ then $\mathcal{C}\subseteq \sigma(\mathcal{C})$, as $\sigma(\mathcal{C})$ is a monotone class (and also a tribe) containing $\mathcal{C}$ and therefore contains also $\mathcal{M}(\mathcal{C})$ (see the examples with tribes).
	\end{tcolorbox}	
	 \begin{theorem}
	Given $E$ as set. If $\mathcal{C}$ is a family of parts of $E$ that we impose as stable by finite intersection then $\mathcal{C}=\sigma{\mathcal{C}}$ (we then have to prove that the smallest tribe of $\mathcal{C}$ is equal to the smallest monotone class of $\mathcal{C}$. If we do not impose that $\mathcal{C}$ is stable by finite intersection we would not have necessarily the equality!
	\end{theorem}
	\begin{dem}
	As already said: $\mathcal{M}(\mathcal{C})\subseteq \sigma(\mathcal{C})$ (which is trivial). We will prove first that $\mathcal{M}(\mathcal{C})$ is a tribe on $E$. For this it is sufficient to show that $\mathcal{M}(\mathcal{C})$ is (also) stable by countable union (and not necessarily by an increasing sequence of elements!).
	
	Let us considerate following families for the proof:
	
	By the previous definitions $\mathcal{M}_1\subseteq \mathcal{C}$ but $\mathcal{C}$ being (imposed) stable by finite intersections implies that $\mathcal{C}\subseteq \mathcal{M}_1$ and therefore (it is the same reasoning as for the tribes):
	
	$\mathcal{M}_1$ is a monotone class, indeed $E\in \mathcal{M}_1$, if $A_1,A_2 \in \mathcal{M}_1$ and that $A_1\subseteq A_2$ (second axiom) then:	
	
	and therefore (which supports the fact that the other elements $(A_n)$ satisfy the previous relation):
	
	If $(A_n)$ is an increasing sequence of elements of $\mathcal{M}_1$ then:
	
	as $(A_n\cap B)$ is an increasing sequence.
	
	Therefore $\mathcal{M}_1$ is indeed a monotone class and by $\mathcal{C}\subseteq \mathcal{M}_1 \subseteq \mathcal{M}(\mathcal{C})$, we therefore have:
	
	The latter equality implies $\mathcal{C}\subseteq \mathcal{M}_2$. As for $\mathcal{M}_1$, we show that  $\mathcal{M}_2$ is a monotone class and therefore $\mathcal{C}= \mathcal{M}_2$, which means by extension that $\mathcal{M}(\mathcal{C})$ is therefore stable by finite intersections.
	
	$\mathcal{M}(\mathcal{C})$ being stable by complementary this take us to that $\mathcal{M}(\mathcal{C})$  is, we just proved it, stable by finite unions (but we want to prove that it is stable by countable union!).
	
	Given now a sequence $(A_n)$ of elements of $\mathcal{M}(\mathcal{C})$. We consider the sequence:
	
	$(B_n)$ is an increasing sequence of elements of $\mathcal{M}(\mathcal{C})$, therefore:
	
	but:
	
	Therefore:
	
	Therefore $\mathcal{M}(\mathcal{C})$ is stable by countable union and finally $\mathcal{M}(\mathcal{C})$ is a tribe. But as $\mathcal{C}\subseteq \mathcal{M}(\mathcal{C})$ this brings us to $\mathcal{M}(\mathcal{C})=\sigma(\mathcal{C})$.
	\begin{flushright}
		$\blacksquare$  Q.E.D.
	\end{flushright}
	\end{dem}
	We will later see some important applications of this theorem (but first we want to improve the above text with figure and more simple and practical examples!).
	
	\begin{flushright}
	\begin{tabular}{l c}
	\circled{50} & \pbox{20cm}{\score{4}{5} \\ {\tiny 17 votes,  76.47\%}} 
	\end{tabular} 
	\end{flushright}
		
\chapter{Geometry}

	\textit{\textbf{Geometry is the mathematical discipline that focuses on the rigorous study of spaces and forms}}. (Larousse)
	\minitoc
	\pagebreak
		%to force start on odd page
	\newpage
	\thispagestyle{empty}
	\mbox{}
	\section{Trigonometry}\label{trigonometry}
	\lettrine[lines=4]{\color{BrickRed}T}rigonometry is part of the science of geometry. Geometry whose etymological root is "measure of the earth", trigonometry has etymological root "measuring of body with three angles (trine)". Thus the trigonometry is a branch of mathematics that studies relations involving lengths and angles of triangles.
	
	There are currently four known trigonometries (defined) commonly used in mathematics: the "\NewTerm{circle trigonometry}\index{circle trigonometry}" (assimilated with the study of "circular functions"), the  "\NewTerm{hyperbolic trigonometry}\index{hyperbolic trigonometry}" and  "\NewTerm{spherical trigonometry}\index{spherical trigonometry}" and a more exotic "\NewTerm{elliptical trigonometry}\index{elliptical trigonometry}" (related to the elliptic integrals and their related elliptic function). We propose in this text an attempt to a fairly rigorous approach to all the most famous relations of the three first type of trigonometries (the last one - elliptical trigonometry - reserved to another section of this book).
	
	\begin{tcolorbox}[title=Remarks,colframe=black,arc=10pt]
	\textbf{R1.} We will not deal here on the quadratic and rhombic trigonometries that are used by the electricians and have little or no interest in theoretical physics. The same remark is valid for the lemniscatique trigonometry which is related to pure mathematics and in particular to the Riemann zeta function.\\
	
	\textbf{R2.} The reader who is looking for the proofs of derivatives and integrals of trigonometric functions defined below should read the Differential and Integral Calculus section (\SeeChapter{see chapter Algebra page \pageref{differential and integral calculus}}) where the derivatives and integrals of common trigonometric functions are all proven.
	\end{tcolorbox}	

	The purpose of this section will be to determine the most common relations in trigonometry and that are used extensively in all sections of this book (Mechanics, Astronomy, Statistics, Atomistic, etc.). Note that the majority of the relations (but not all!) that we will define and prove were determined in the 16th century by algebraists like Viète.

	\subsection{Radian}\label{radian}
	When we speak about trigonometry, the first thing that should come to mind and emerge as standard for plan angles measurements (see the section of Euclidean Geometry chapter for the definition of the concept of angle) is the notion of "radians ".

	\textbf{Definition (\#\mydef):} $1$ "\NewTerm{radian}\index{radian}" (denoted sometimes [rad]) is the angle described by a plane secant to a circle passing through its center as the arc thus defined by the horizontal axis passing through the center of the circle and the secant is equal in length to the radius of this circle.

	\begin{tcolorbox}[title=Remark,colframe=black,arc=10pt]
	The radian is widely used in physics, as we will see it in the corresponding chapters of this book, when angular measurements are required. For example, angular velocity is typically measured in radians per second ([rad/s]). One revolution per second is equal to $2\pi$ radians per second.
	\end{tcolorbox}
	
	For example, for a circle of radius $R=1$ therefore of circumference (or "perimeter") $P=2\pi R=2\pi$ the length of the arc defined by a secant having an angle of 1 radian with respect to the horizontal passing through the center of the circle will be equal 1.

	Hence it follows that the angle for a "round" of the circle will be:
	
	This can be resume in the following drawing:
	\begin{figure}[H]
		\centering
		\includegraphics[scale=0.6]{img/geometry/radians.eps}
		\caption[A chart to convert between degrees and radians]{A chart to convert between degrees and radians (source: Wikipedia)}
	\end{figure}
	The previous example can be generalized to any circle of radius $R$ as the angle for a complete tour will always be for $2\pi [\text{rad}]$ a whole turn-around, $\pi [\text{rad}]$  for a half turn-around and $\dfrac{\pi}{2} [\text{rad}]$ for fourth turn-around...
	
	Unfortunately in high-schools most teachers still teach children to measure angles in degrees. Fortunately conversion to do is not too difficult ... (it's a simple rule of three):
		
	Let $r$ be the measurement of an angle in radians, $d$ the measurement of the same angle in degrees, $g$ and the measurement of the same angle in gradients (old unit) we have by definition:
	
	Astronomers and astrophysicists like to talk in minutes or seconds of arc such as:
	
	For example if the Moon has a diameter angle of approximately $29'$ ($0.5^\circ$) in the sky and if our eyes would have a deep exposure capability we would be able to see that Andromeda (M31) has a diameter angle of $178'$ ($3^\circ$), therefore about $6$ times as wide as the Moon!
	\begin{figure}[H]
		\centering
		\includegraphics[scale=1]{img/geometry/angular_diameter.jpg}
		\caption[]{Earth's sky as seen if human eyes had deep exposure capabilities}
	\end{figure}
	Here is a  nice visual summarizing this concept of sub-angles units:
	\begin{figure}[H]
		\centering
		\includegraphics[scale=0.8]{img/geometry/degrees_minutes_seconds.jpg}
		\caption[Degrees, minutes of arc, seconds of arc]{Degrees, minutes of arc, seconds of arc (author: ?)}
	\end{figure}

	\pagebreak
	\subsection{Circle Trigonometry}\label{circle trigonometry}
	Consider the figure below showing any circle centered at the origin in a direct basis:		
	\begin{figure}[H]
		\centering
		\includegraphics[scale=0.8]{img/geometry/trigonometric_circles.eps}
		\caption{Construction idea of elementary (circle) trigonometric functions}
	\end{figure}

	First, by applying the Pythagorean theorem (\SeeChapter{see section Euclidean Geometry page \pageref{pythagorean theorem}}) in the visible right triangle above, we get:
	
where $R$ is the radius of the circle.

	To define the trigonometric functions for the angle $\alpha$, start with any right triangle that contains the angle $\alpha$. The three sides of the triangle are named as follows:
	\begin{enumerate}
		\item The "\NewTerm{hypotenuse}\index{hypotenuse}" is the side opposite the right angle, in this case side it is the segment $\overline{\text{O}M}$. The hypotenuse is always the longest side of a right-angled triangle.
		
		\item The "\NewTerm{opposite side}\index{opposite side}" is the side opposite to the angle we are interested in.
		
		\item The "\NewTerm{adjacent side}\index{adjacent side}" is the side adjacent to the angle we are interested in.
	\end{enumerate}

From this representation we can define mathematical entities named "\NewTerm{trigonometric functions of the circle}\index{trigonometric functions of the circle}", also sometimes named by the ancient (...) "\NewTerm{cyclometrics functions}\index{cyclometrics functions}\label{cyclometrics functions}" such as (for the most known one)\label{definition trigonometric functions}



Be careful because depending on the authors and on the context (and that's our case!) $\arccos, \arcsin, \arctan, $ are respectively denoted by $\cos^{-1}, \sin^{-1}, \tan^{-1}$.

Read "\NewTerm{cosine}\index{cosine}" for "cos", "\NewTerm{sine}\index{sine}" for "sin", "\NewTerm{tangent}\index{tangent}" for "tan", "\NewTerm{cotangent}\index{cotangent}" for "cot", "\NewTerm{secant}\index{secant}" for "sec", "\NewTerm{cosecant}\index{cosecant}" for "csc".

	\begin{tcolorbox}[title=Remarks,colframe=black,arc=10pt]
	\textbf{R1.} When the context allows it, and that there can not be any ambiguity, the parentheses after the name of the trigonometric function may be omitted (this is often the case in physics).\\
	
	\textbf{R2.} The arc... functions are obviously the reciprocal functions of the trigonometric functions!
	\end{tcolorbox}

In practice (school or high level engineering) you have to take care to two important properties: 
	\begin{enumerate}
		\item Trigonometric functions are far from being bijective. Indeed, they tend to repeat themselves periodically as we will see further below. Nonetheless with a little restriction on their definition domain, they may acquire this quality. In this context, the perspective of a reciprocal becomes possible.
		\item If the sign of the bracket of the arc tangent is negative, then we do not know exactly in which quadrant (I, II, III or IV) we are so we do not know the sign of the numerator or denominator! It is for this reason that most calculators and softwares have two arc tangent functions: \texttt{arctan} and \texttt{arctan2}. The first gives an angle between $\left[-\dfrac{\pi}{2},+\dfrac{\pi}{2}\right]$ and therefore we can not know precisely in which quadrant we are. The second gives us an angle between $\left[-\pi,\pi\right]$ so we can know in which quadrant we are but we must provide the sign of the numerator or denominator.
	\end{enumerate}

	From these functions, we can make some combinations and define simple but remarkable relations but whom usefulness is debatable (and which are very rarely used) such as:
	
	but you will never meet these functions in this book because we personally never use such notations (it is rather customary in some U.S. books).
	
	Here is a nice figure... that resume all that stuff:
	
	\begin{figure}[H]
	\centering
	\includegraphics[scale=0.75]{img/geometry/trigonometry_resume.eps}
	\caption{Design principle of common circle trigonometric functions}
	\end{figure}
	
	\textbf{Definition (\#\mydef):} We name "\NewTerm{phase-shift}\index{phase-shift}" and angle $\delta$ that shift any trigonometric function argument of a given value. For example as illustrated below:
	
	\begin{figure}[H]
	\centering
		\begin{tikzpicture}
	  	\begin{axis}[
	    trig format plots=rad,
	    axis lines = middle,
	    enlargelimits,
	    clip=false
	    ]
	    \addplot[domain=-2*pi:2*pi,samples=200,blue] {sin(x)};
	    \addplot[domain=-2*pi:2*pi,samples=200,red] {sin(x-2)};
	    \draw[dotted,blue!40] (axis cs: 0.5*pi,1.1) -- (axis cs: 0.5*pi,0);
	    \draw[dotted,red!40] (axis cs: 0.5*pi+2,1.1) -- (axis cs: 0.5*pi+2,0);
	    \draw[dashed,olive,<->] (axis cs: 0.5*pi,1.05) -- node[above,text=black,font=\footnotesize]{$\delta$}(axis cs: 0.5*pi+2,1.05);
	  	\end{axis}
		\end{tikzpicture}
	\end{figure}
	
	OK for people reading this book on a computer with Adobe Flash player installed and activated, here is an animated version (otherwise see here: \url{https://vimeo.com/575748635}):
	
	\begin{center}
	\centering
		\includemedia[activate=pageopen,width=\textwidth,height=500pt,
	]{}{swf/Trigonometric_circle.swf}
	\end{center}

	Let us see now some very simple angular properties of these common functions:
	\begin{enumerate}
		\item[P1.] if we remain in the study of the circle, hence named "\NewTerm{trigonometric circle}\index{trigonometric circle}\label{trigonometric circle}", we must put for the above definitions above $R=1$. Thus, it appears more clearly the physical meaning of these definitions and it will result in a number of properties and applications directly usable in theoretical physics and pure mathematics.
		
		Indeed, if $R=1$ we trivially have (\SeeChapter{see section Vector Calculus page \pageref{polar coordinates}}):
		
		and by applying the Pythagorean theorem (\SeeChapter{see section Euclidean Geometry page \pageref{pythagorean theorem}}):
		
		Therefore:
		
		
		\item[P2.] If $\alpha$ is a real number, and for $\forall k \in \mathbb{N}$, the real numbers $\alpha$ and $\alpha + 2k\pi$ are associated with the same point $M$ by the periodicity of the unit circle. Indeed, $\alpha$ and $\alpha + 2k\pi$ are two measures of the same oriented angle. So:
		
		Ditto for all trigonometric functions arising from the definition of sine and cosine functions.
	\end{enumerate}

	\begin{tcolorbox}[title=Remark,colframe=black,arc=10pt]
	In the measure of "oriented angles", we say that two measures are congruent modulo $2k\pi [rad]$ if and only if their difference is a multiple of $2k\pi [rad]$. This characterize two measurements of the same angle.
	\end{tcolorbox}
	
	By definition, the sine and cosine of any real number part of the 
 interval $[-1,+1]$. Specifically, the position of the point $M$ allows us to learn more about the cosine and sine equation. So:
 
\begin{figure}[H]
\centering
\includegraphics[scale=0.75]{img/geometry/trigonometry_angles.eps}
\caption{Remarkable angles of common circle trigonometric functions}
\end{figure}

There is also another representation of the trigonometric functions of the circle, a bit more technical but almost important to understand well later, at least visually, for the study of wave mechanics (see section of the same name page \pageref{wave mechanics}):

	\begin{figure}[H]
		\centering
		\begin{tikzpicture}[scale=1.5]
		\begin{axis}[domain=-3*pi:3*pi,samples=200,grid=major,
		    restrict y to domain=-3:3, 
		    legend style={font=\fontsize{4}{5}\selectfont},
		    legend pos=north west,
		    ytick={-2,-1,0,1,2},
		    xtick={-6.28,-4.71,-3.14,-1.57,0,1.57,3.14,4.71,6.28},
	     	xticklabels={-$2\pi\,\,\,$,-$\frac{3}{2}\pi\,\,\,$,-$\pi\,$,-$\frac{\pi}{2}$,$0$, +$\frac{\pi}{2}$,+$\pi$,+$\frac{3}{2}\pi$,+$2\pi$},
	     	axis lines=middle,
	     	x tick label style={font=\tiny},y tick label style={font=\tiny}
	     	]
		  \addplot[domain=-2*pi:2*pi,samples=200,red]{sin(deg(x))};
	      \addplot[domain=-2*pi:2*pi,samples=200,blue]{cos(deg(x))};
	      \addplot[domain=-2*pi:2*pi,samples=200,green]{tan(deg(x))};
	      \addplot[domain=-2*pi:2*pi,samples=200,purple]{cot(deg(x))};
		\legend{$\sin(x)$, $\cos(x)$, $\tan(x)$, $\cot(x)$}
		\end{axis}
		\end{tikzpicture}
	\end{figure}
	
	\begin{figure}[H]
		\centering
		\begin{tikzpicture}[scale=1.5]
		\begin{axis}[domain=-1:1,
		samples=500,
		axis lines=middle,
		legend style={font=\fontsize{4}{5}\selectfont},
		xtick={-2,1.5,-1,-0.5,0,0.5,1,1.5,2},
		ytick={-6.28,-4.71,-3.14,-1.57,0,1.57,3.14,4.71,6.28},
		yticklabels={-$2\pi\,\,\,$,-$\frac{3}{2}\pi\,\,\,$,-$\pi\,$,-$\frac{\pi}{2}$,$0$, +$\frac{\pi}{2}$,+$\pi$,+$\frac{3}{2}\pi$,+$2\pi$},
		x tick label style={font=\tiny},y tick label style={font=\tiny},
		legend pos=north east,
		grid=major
		]
    	\addplot[color = red]  {rad(asin(x))};
    	\addplot[color = blue]  {rad(acos(x))};
    	\addplot[domain=-2:2,color = green]  {rad(atan(x))};
		\legend{$\arcsin(x)$,$\arccos(x)$,$\arctan(x)$}
		\end{axis}
		\end{tikzpicture}
		\caption{Plots of some common circle trigonometric functions}
	\end{figure}

The reader should notice at this point without too much difficulties the following properties (often used in physics!) with $n \in \mathbb{Z}$\label{periodicity trigonometric functions}:
	
and easily recognize that the sine is an odd function and the cosine an even function (observation that we often will be useful in various mathematical developments on trigonometric series).

As we saw at the beginning of this section, following the definition of the trigonometric functions we obviously have:
	
	and also:
	
	In exactly the same way we prove that:
	
	From previous equalities we found without too much difficulty (elementary algebra):
	
	We have identically:
	
	By reverse reasoning we get just as easily:
	
	It could comes without difficulty by observing the unit circle with its remarkable angles that\label{remarkable angles}:
	
	\begin{tcolorbox}[title=Remark,colframe=black,arc=10pt]
	The minus-plus sign "$\mp$" is generally used in conjunction with the "$\pm$" sign in a same relation. So when the first is at the "$+$" level, then the other one is at "$-$" level and conversely.
	\end{tcolorbox}
	Here are the figures that summarize how to analyse some of these properties (for other relations, the method is the same):
	\begin{figure}[H]
		\centering
		\includegraphics[scale=0.65]{img/geometry/trigonometry_remarkable_angles.jpg}
		\caption{Visual equivalences of some trigonometric relations}
	\end{figure}
	Here are some remarkable angles associated with the values of cosine and sine functions that many students must memorize during their studies:
	\begin{figure}[H]
		\centering
		\includegraphics[scale=0.65]{img/geometry/trigonometry_angles_identity.jpg}
		\caption{Some identity angles and conventional trigonometric associated values for sine and cosine}
	\end{figure}

	\begin{figure}[H]
		\centering
		\includegraphics[scale=0.99]{img/geometry/trigonometric_angles.jpg}
		\caption{Some identity angles and conventional trigonometric associated values table}
	\end{figure}
	Let us now introduce one last definition concerning circle trigonometric function relation that we will meet again in the sections of Wave Optics or during our study of Fourier transforms in the section of Sequences and Series which is the "\NewTerm{sine cardinal}\index{sine cardinal}\label{sinc cardinal}":
	
	represented by (with Maple 4.00b):
	
	\texttt{>with(plots):\\
	>sinc:=sin(x)/x;\\
	>plot([sinc(x)], x=-5*Pi..5*Pi, Sinc);}
	\begin{figure}[H]
		\centering
		\includegraphics[scale=0.8]{img/geometry/sinus_cardinal_2d.jpg}
		\caption{2D plot of the sinc function with Maple 4.00b}
	\end{figure}
	Even if engineers know very well the previous figure it is especially its pseudo-3D representation which is known by common peoples because often used for marketing reasons as reminiscent of a drop of water falling into water (figure made with Maple 4.00b) and it is always pretty to look at ...:

\texttt{>plot3d(sin(sqrt(x\string^2+y\string^2))/(sqrt(x\string^2+y\string^2)),x=-20..20,y=-20..20);}

\begin{figure}[H]
\centering
\includegraphics[scale=0.8]{img/geometry/sinus_cardinal_3d.jpg}
\caption{Pseudo 3D plot of the sinc function with Maple 4.00b}
\end{figure}
In either case, the value at $x = 0$ is defined to be the limiting value $\mathrm{sinc(0)} = 1$ that is really easy to prove using Hospital's rule (\SeeChapter{see section Differential and Integral Calculus page \pageref{Hospital rule}}).

Keep in mind that "sinc" function is particularly important in signal processing and is the Fourier transform of a rectangular pulse and also in some Optics Phenomenon.

	\subsubsection{Remarkable trigonometric triangle identities}\label{remarkable trigonometric identities}
	The drawing below will allow us to build relations that will help to solve equations involving trigonometric functions (all these relations are of primary importance in physics to simplify problem solving!).

\begin{figure}[H]
\centering
\includegraphics{img/geometry/trigonometric_triangle.jpg}
\caption{Basic construction for determining remarkable circle trigonometric identities for triangles}
\end{figure}

	First we will note on the above figure the following relation:
	
	Therefore:
	
	To resume:
	
	\begin{tcolorbox}[title=Remark,colframe=black,arc=10pt]
	We hesitated to put the main triangle identities in boxes (frames) but in fact all the results above and below are important and therefore everything will have been into boxes... So we apology to the reader if the absence of boxes (frames) disturbed.
	\end{tcolorbox}
	This implies trivially if $\alpha=\beta$:
	
	and:
	
	Therefore:
	
	We also have:
	
	Therefore:
	
	This implies trivially if $\alpha=\beta$:
	
	With the already proven identity $\cos^2(\alpha)+\sin^2(\alpha)=1$ we also get the very important following relations:
	
	Relations with which we get very easily relations that seems to be named "\NewTerm{Carnot's Formula}\index{Carnot's Formula}":
	
	and: 
	
	We also have:
	
	This, to get the relation:
	
	which implies when $\alpha=\beta$:
	
	and obviously:
	
	Therefore to summarize:
	
	We also get trivially (ask us if you have problems!) from previous relations (we do a little mixing and we shake...):
	
	We also have using above results:
	
	using the relation proved far above :
	
	Therefore:
	
	Exactly similarly we get (ask us if you need the details):
	
	Using always:
	
	Therefore:
	
	Now let us determine additional trigonometric formulas that seems to be name "\NewTerm{Simpson's Formula}\index{Simpson's Formula}" or "\NewTerm{Trigonometric addition formulas}\index{trigonometric addition formulas}" which express the sum of sine and / or cosine in product of sine and / or cosine.
	
	Given the identities already previously proved:
	
	That can also be summarized as:
	
	Let us put $\alpha+\beta=p$ and $\alpha-\beta=q$ and this also give us:
	
	We get by summing (1) and (2):
	
	and by subtracting:
	
	In the same way we but starting from:
	
	That can also be summarized as:
	
	We get first by summing:
	
	and by subtracting:
	
	and re-injecting the angles we easily (normally but if you have difficulties tell us!) we fall back on the following "\NewTerm{Simpson's inverse formulas}\index{Simpson's inverse formulas}":
	
	
	We also have starting from:	
	
	and we divide by $\cos(\alpha)\cos(\beta)$ to get:
	
	And exactly in the same way we get:
	
	
	All these relations will help us in our study of general physics (especially in the section of Wave Mechanics and therefore in the sections of Electrodynamics and Acoustics) and particularly in the case of integral calculations or superposition of waves.
	
	\begin{tcolorbox}[title=Remark,colframe=black,arc=10pt]
It helps perhaps to resume that the following relations are the Simpson's relations:
	
	\end{tcolorbox}
	Let us also derive the following identity that will be useful to us in the section of Differential and Integral Calculus:
	
	and respectively:
	
	
	\paragraph{Laws of Cosines}\label{law of cosines}\mbox{}\\\\\
	Let us prove now the cosine theorem that will be very useful to us in the sections of Geometry, Cosmology, Electrodynamics and Mechanics.
	
	\begin{theorem}
	In any triangle, the square of one of the three sides is equal to the sum of the squares of the two others reduced by the product of these two sides by the cosine of the angle between them:
	\begin{figure}[H]
	\centering
	\includegraphics{img/geometry/law_of_cosines.jpg}
	\caption{Construction for the proof of the law of cosines}
	\end{figure}
	\end{theorem}
	\begin{dem}
	One possible proof has the following path by using the Pythagorean theorem:
	
	But in the $ABH$ triangle, rectangle in $H$, we have the relation $n/c=\cos(\chi)$ seen above therefore:
	
	So we get one of the relations of "\NewTerm{cosine theorem}\index{cosine theorem}":
	
	By circular permutation, we get the other two known relations of the cosine theorem.
	\begin{flushright}
		$\blacksquare$  Q.E.D.
	\end{flushright}
	\end{dem}
	\begin{tcolorbox}[title=Remark,colframe=black,arc=10pt]
	The cosine theorem is sometimes named  "\NewTerm{Al-Kashi's formula}\index{Al-Kashi's formula}". And you can note that if $a$ is the hypotenuse and its opposite angle a right angle such that $\cos(\chi)$ is zero, so we fall back on the Pythagorean theorem. Here's why we sometimes name the Al-Kashi's formula: "\NewTerm{generalized Pythagorean formula}\index{generalized Pythagorean formula}".
	\end{tcolorbox}
	An interesting case for applying the cosine theorem is to determine the distance $l$ between a point $b$ on the circumference of the circle and a point $a$ situated inside the circle depending on the direction (angle) in which the eye is directed:
	\begin{figure}[H]
	\centering
	\includegraphics{img/geometry/law_of_cosines_distance_point_circle.jpg}
	\caption{Distance from a point inside a circle at its circumference}
	\end{figure}
	We can therefore apply the cosine theorem that gives us (knowing $d, R$ and $\beta$):
	
	Therefore:
	
	It is therefore an equation of the second degree whose discriminant is (\SeeChapter{see section Calculus page \pageref{discriminant}}):
	
	and whose two solutions are (\SeeChapter{see section Calculus page \pageref{double root}}):
	
	We will also see an important application in the section of Geometric Shapes to calculate the area of any triangle (Heron's formula) page \pageref{heron formula}.
	
	\pagebreak
	\paragraph{Laws of Sines}\label{law of sines}\mbox{}\\\\\
	\index{law of sines}Consider the below triangle for whom we represent the two heights:
	\begin{figure}[H]
	\centering
	\includegraphics{img/geometry/law_of_sines.jpg}
	\caption{Construction for the proof of the law of sines}
	\end{figure}
	In the triangle above we have the relations:
	
	which leads us to the expression:
	
	Therefore:
	
	By similar reasoning we have:
	
	That gives in the same way:
	
	All combined provides us the "sine theorem", the most beautiful example of application in this books is certainly the determination of the Lagrange points L4 and L5 in the section Astronomy:
	
	Obviously, there is not here all the existing trigonometric circle relations as we have already said, but at least the most important you need to know to find we you study  physical systems.
	
	\pagebreak
	\subsection{Hyperbolic Trigonometry}\label{hyperbolic trigonometry}
	We have shown in the section of Functional Analysis that any function $f (x)$ can be divided into an odd and even function as:
	
	Thus, for the function $f(x)=e^x$, we get:
	
	Remember that in our study of complex numbers (\SeeChapter{see section Numbers page \pageref{de Moivre and Euler formulas}}) we proved the following Euler formulas:
	
	and we talked about the fact that even if $\phi$ should be most of time bin in $\mathbb{R}$, it can also be in $\mathbb{C}$ and therefore we generalize the definition of the sine and cosine by analogy with the hyperbolic sine and cosine (we will demonstrate the origin of this term later below) by:
	
	where $z \in \mathbb{C}$. Therefore we can also obviously define:
	
	And so on... like for circle trigonometry.

	We can therefore write:
	
	relation that we can again put in analogy with:
	
	Interestingly, we can now work with complex angles in trigonometry!!! Indeed, if we put $a+\mi b$ then we have:
	
	But as we know (\SeeChapter{see section Numbers page \pageref{power rules calculations}}):
	
	Therefore we can write the following equality:
	
	Therefore the hyperbolic trigonometric function of a complex angle exists and the image is also a complex number. We can see abusively hyperbolic trigonometry as a kind of generalization of trigonometry circle with real and complex angles.
	
	As opposed to the circle trigonometry where we have proved that:
	
	In hyperbolic trigonometry we have:
	
	The proof is easy:
	
	By rearranging we get:
	
	Before we prove why this is named "hyperbolic trigonometric" let us see some important identities that we will use in various sections of this book.
	
	First we will look for a closed form of the reciprocal hyperbolic sine and cosine functions. For this purpose remember that:
	
	and that the search for the inverse function is always to isolate the variable (in this case: $z$).
	
	Therefore:
	
	That is to say (multiplying by $-e^z$):
	
	by solving the polynomial this second degree polynomial on $e^{z}$ we get:
	
	As $e^{z}>0$ we must reject the solution with the sign "$-$" (yes! otherwise with a $z \in \mathbb{R}$ you could have a negative $e^z$). It comes then:
	
	Therefore:
	
	Using the same for:
	
	Therefore:
	
	That is to say (multiplying by $-e^z$):
	
	by solving the polynomial this second degree polynomial on $e^{z}$ we get:
	
	As $e^{z}>0$ we must reject the solution with the sign "-" (yes! otherwise with a $z \in \mathbb{R}$ you could have a negative $e^z$). It comes then:
	
	Therefore:
	
	Finally \label{inverse hyperbolic to logarithm}:
	
	
	\begin{tcolorbox}[colback=red!5,borderline={1mm}{2mm}{red!5},arc=0mm,boxrule=0pt]
	\bcbombe Caution! Following the authors $\text{arccosh}$ and $\text{arcsinh}$ are denoted by $\text{argcosh}, \text{argsinh}$ or $\cosh^{-1}, \sinh^{-1}$ (the two last notations are confusing with the corresponding inverse hyperbolic function!).
	\end{tcolorbox}
	
	
	To study now a simple important geometric representation of that stuff let us now write:
	
	With a restriction $\alpha,z in \mathbb{R}$ we have:
	
	This can be represented as follow:
	\begin{figure}[H]
	\centering
	\includegraphics{img/geometry/hyperbolic_functions.jpg}
	\caption[Hyperbolic functions domain]{Hyperbolic functions domain (source: Wikipedia)}
	\end{figure}
	With this new notation, we have:
	
	But as we shall see in our study on conical in the section of Analytical Geometry:
	\begin{enumerate}
		\item The first of these two relations is for the entire given domain of definition, a circle of unit radius centered on the origin. The reader will note that it is curious enough for circle trigonometry to get a circle \Winkey
		\item The second of these two relations is for the entire domain of a definition, an equilateral hyperbola oriented along the $X$ axis whose apex is $S (1,0)$ and of focus $F(\sqrt{2},0)$. The reader will note again that he is curious enough for hyperbolic trigonometry to get a hyperbola \Winkey
	\end{enumerate}
	For the second case this take us to have the possibility to represent the two main hyperbolic trigonometric functions as:
	\begin{figure}[H]
	\centering
	\includegraphics[scale=0.7]{img/geometry/hyperbola_hyperbolic_trigonometry.jpg}
	\caption[Equivalence of hyperbolic functions and hyperbola]{Equivalence of hyperbolic functions and hyperbola (source: Wikipedia)}
	\end{figure}
	This figure is very interesting because as we will prove it now, from the calculation of the red area $a/2$, where $a$ is assimilated to a "\NewTerm{hyperbolic angle}\index{hyperbolic angle}", we can also define the hyperbolic trigonometric functions.
	
	Let us denote by $x$ and $y$ the abscissa and ordinate of the red triangle (by imagining this is \underline{full} a triangle). We therefore have (\SeeChapter{see section Geometric Shapes page \pageref{unspecified triangle}}) therefore for its area (base multiplied by high divided by $2$):
	
	But we see that we have to subtract a part of an area under the hyperbola of equation $X^2-Y^2=1$. That is to say of equation:
	
	And the area to subtract is therefore given by:
	
	Therefore the read area is given by:
	
	Using the primitive proved in the section of Differential and Integral Calculus we get:
	
	Therefore:
	
	But we can rewrite this:
	
	Therefore:
	
	Finally:
	
	We recognize here the first relation of the both that we proved a little sooner above if we put $z=a,y=x$:

	So the interpretation is very interesting: the hyperbolic angle can be seen as the double of the red triangle surface (or if you prefer: the half of the hyperbolic angle is equal to the red surface)!
	
	These two relations:
	
	
	should help, we hope to understand the origin of the name of the hyperbolic trigonometry and note that the study of hyperbolic trigonometry, is analogous to the study of circle trigonometry on the circle.
	
	If we represent the trigonometric circle and the trigonometric hyperbole and we add some additional information, here is what we get:
	\begin{figure}[H]
	\centering
	\includegraphics{img/geometry/circle_hyperbolic_trigonometry_resume.jpg}
	\caption{Visual definition of circle and hyperbolic trigonometric functions}
	\end{figure}
	Explanations: 

	To trace with the ruler and compass the point P$(x, y)$ of the hyperbole, we give us a $x$, that is to say the point $A(x, 0)$. We draw the tangent to the circle (C) through A (x, 0) which gives us the tangent point T. We trace the circle (G) of center A$(x, 0)$ and passing through T. This circle intersects the hyperbola at the point P$(x, y)$ perpendicular to the A$(x, 0)$ to $\text{O}x$.
	
	We see appear on the figure several values of the hyperbolic functions corresponding to $x=\cosh(t)=\text{ch}(t),y=\sinh(t)=\text{sh}(t)$ but also $\tanh(t)=\text{th}(t),\coth(t)$, etc. Among others, the circle (L) intersects the axis $\text{O}x$ in two points whose abscissas are $e^t$ and $e^{-t}$.
	
	If the reader wants to check thanks to the figure, it can control that in all points of the  hyperbole, the relations (among others):
	
	are always verified.
	
	We get for the readers who wants to play with software using Maple 4.00b:\\\\
		\texttt{>plot([sinh(x),cosh(x),tanh(x)],x=-2..2,color=[red,black,blue]);}
	
	\begin{figure}[H]
	\centering
	\includegraphics{img/geometry/maple_hyperbolic_functions.jpg}
	\caption{Visual definition of hyperbolic trigonometric functions with Maple 4.00b}
	\end{figure}
	We will see again the $\cosh(x)$ in the section of Civil Engineering for example in the context of suspended cables. We also see again $\sinh (x)$ and $\tanh (x)$ as part of the study of gravity waves in the section of Marine and Weather Engineering.
	
	\subsubsection{Remarkable hyperbolic identities}
	
	The purpose now is as for the circle trigonometry to prove some remarkable identities but for hyperbolic trigonometry.
	
	So we start with the definition:
	
	and:
	
	From these definitions and using the basic algebra operations we can determine the following outstanding relations (this is much easier than determining the remarkable relations of circle trigonometry, so unless requested we give most of these relations without proof).
	
	First let us begin with the very easiest one:
	
	We also have:
	
	And we have the addition relations:
	
	Following the request of a reader, let us prove the first and third above relations.
	
	For the first:
	
	and for the third:
	
	Also let us mention other remarkable identities (once again don't hesitate to request us the detail proof if you need help):
	
 	and finally (we stop here because we don't need anything else to study the other chapters and sections of this book):
 	
	
	\pagebreak
	\subsection{Spherical Trigonometry}\label{spherical trigonometry}
	The aim of spherical trigonometry is to determine remarkable relations between the angles and sides of projected forms (also named "geodetic forms" because following the curvature of space) on the surface of a sphere. To determine this relations, we will look at the special case of a sphere of radius unity and to the relations between the sides of a triangle (planar elementary surface) and the various existing angles. We will see that the results are completely independent of the radius of the sphere.
	
	Consider the figure in which is located a geodesic triangle with vertices $A, B, C$ of respective opening angles $\hat{A}, \hat{B}, \hat{C}$ and opposite sides $a, b, c$ and three unit vectors $\vec{i}, \vec{j}, \vec{k}$ such as $\vec{i} \perp \vec{k}, \vec{j} \perp \vec{k}$ and that the end of $\vec{k}$ is merged with the vertices $A$:
	\begin{figure}[H]
	\centering
	\includegraphics{img/geometry/spheric_trigonometric_construction.jpg}
	\caption{Construction to introduce the spherical trigonometry}
	\end{figure}
	The angle between points $B$ and $C$, denoted by $\alpha$, could not be shown in the diagram above because of lack of space.
	
	Let us recall that the perimeter of a unit circle on the sphere of radius unity is obviously $P=2\pi R$ (\SeeChapter{see section Euclidean Geometry page \pageref{circle}}). The perimeter of the circle as a function of the opening angle of the latter being given by (relation very often used in physics !!!)\label{arc length trigonometry}:
	
	\begin{figure}[H]
	\centering
	\includegraphics{img/geometry/length_arc_circle.jpg}
	\end{figure}
	If the circle is of radius $R=1$ (as is the case for our sphere), the calculation of the arc length is simplified and becomes:
	
	We will keep this constraint of a unitary radius subsequently to simplify the expressions we'll get later on.
	
	Consequently regarding the points on our sphere; the sides of the triangle are given by:
	
	Now consider the scalar product (\SeeChapter{see section Vector Calculus page \pageref{dot product}}):
	
	and as $\overrightarrow{OB}= \overrightarrow{OC}$ (unit radius) we have:
	
	If we decompose the two vectors $\overrightarrow{OB}$ and $\overrightarrow{OC}$ vectors on the orthonormal vectors we have:
	
	This gives us:
	
	giving (distributive property of scalar product):
	
	As $\vec{k}\circ \vec{k}=1$ and $\vec{k}\circ \vec{i}=\vec{k}\circ \vec{j}=0$ the above relation reduces to:
	
	And as:
	
	We get:
	
	relation named "\NewTerm{fundamental relation}\index{trigonometric fundamental relation}" or "\NewTerm{cosine formula}\index{cosine formula}\label{cosine formula}" that we can therefore as well write (because of the unitary radius assumption):
	
	Also often rearranged as following (especially in statistics to introduce the partial correlation coefficient):
	
	The latter relation is invariant by circular permutation of the variables $a,b,c,\hat{A}$. It is also interesting to note before we continue that if the spherical triangle has right angle at $A$, the previous relation simplifies to:
	
	and if the triangle is sufficiently small relative to the radius and we make a second order Taylor expansion (\SeeChapter{see section Sequences and Series page \pageref{taylor series}}) near $0$  for each of the terms we get:
	
	Therefore:
	
	After simplification:
	
	we find back the Pythagorean theorem (\SeeChapter{see section Euclidean Geometry page \pageref{pythagorean theorem}}). Therefore:
	
	is the spherical geometry (non-Euclidean geometry) equivalent  of the Pythagorean theorem in plane geometry (Euclidean geometry)!!!
	
	This  parenthesis now closed, let's us go back on our main business. The sine of all angles being positive (as all angles are $[0,\pi]$), we can write:
	
	The latter relation is of course also invariant by circular permutation of the variables $a,b,c,\hat{A}$. So we get a remarkable relation of the spherical triangle, named "\NewTerm{sines relations}\index{sines relations}" or "\NewTerm{sines formula}\index{sines formula}" and given by:
	
	As spherical trigonometry is often used for land trails, often with two very specific and orthogonal circles: the Earth's equator and a meridian or any parallel, this case is of particular interest! Therefore in the case of a right triangle in $A$ we obviously get:
	
	All relations we have determined so far allow us in case of to get very interesting relation for geophysics:
	
	For sure, we have not presented here all the relations or remarkable identities existing about spherical trigonometry, but at least the most important one that should be know to read the other sections of this book.
	\begin{tcolorbox}[title=Remark,colframe=black,arc=10pt]
	We define the  "surplus" or "spherical excess" by the number:
	
	\end{tcolorbox}
	While we're at it, let us calculate a classic problem which is that of the area of a triangle on a sphere. Consider the following figure:
	\begin{figure}[H]
		\centering
		\includegraphics{img/geometry/spherical_triangle.jpg}
		\caption[]{Construction for the study of the surface of a spherical triangle}
	\end{figure}
	If we extend the geodesic arcs $AB$ and $AC$ until $A_1$ we get a slice of a sphere whose surface is proportional to the angle $\alpha$ in $A$. If this angle was equal to $2\pi$, we would have the whole sphere and the surface would be equal to $4\pi R^2$ (\SeeChapter{see section Geometric Shapes page \pageref{sphere}}). As the angle is equal to $\alpha$, the proportionality told us that $S_1$ is equal to:
	
	Similarly, if we extend the arcs $BC$ and $BA$ up to $B_1$ and if we extend the arcs $CA$ and $CB$ until $C_1$, we get two slices with surfaces $S_2$ and $S_3$ are equal to:
	
	Now suppose we add these three areas:
	
	we then get the half of the sphere $S_\odot$ (look to the figure to represent it yourself mentally) more two times the geodesic triangle area $S$ in pink on the figure (as counted two times in excess).
	
	We must subtract twice the blue triangle area to get the surface of the hemisphere:
	
	Therefore:
	
	as $S_\odot=4\pi R^2$, we get:
	
	After simplification we deduce that the area $S$ of the triangle $ABC$ is:
	
	where $\varepsilon$ is a solid angle (see below definition). That latter relation is sometimes named "\NewTerm{Girard's theorem}\index{Girard's theorem}". More commonly written:
	
	and $K=1/R^2$ is defined as the "curvature" (it indicates how much the geometry of the sphere deviates from the Euclidean plane geometry where $K=0$). On the basis of its definition $K$ seems to depend on the radius of the sphere considered as a two dimensional manifold embedded in the ambient Euclidean space $\mathbb{R}^3$. The boxed relation above however tells us that the curvature $K$ can be evaluated by just performing measures of angles and areas on the two-dimensional surface of the sphere. It follows that $K$ has an intrinsic geometric meaning, that is independent of its embedding in the flat 3-dimensional space, since it can be computed by only using measures on the spherical surface. 
	
	It is fairly simple to generalize this concept to other forms following the same philosophy (especially those composed of triangles...).
	
	\subsection{Solid Angle}\label{solid angle}
	In spatial geometry, arise the problem of opening angle of a portion of the space (in extension to the so named "plane angle" that applies to $\mathbb{R}^2$). , We define the "\NewTerm{solid angle}\index{solid angle}" by measuring the portion of space delimited by a conical surface of apex $\text{O}$ and we express it in "\NewTerm{steradian}\index{steradian}", given by the ratio (logical extension of the definition of the plane angle):
	
	$S$ being the area of the cap cutted by cone on a sphere of radius $r$.
	
	\begin{figure}[H]
		\centering
		\includegraphics{img/geometry/solid_angle_configuration.jpg}
		\caption{Configuration for the definition of the solid angle}
	\end{figure}
	If $\alpha$ is the half plane-angle of the cone, we get for this ratio (for calculation of the cap of a spherical surface see the section Geometric Shapes):
	
	Hence we conclude that the total solid angle is by definition on the whole space:
	
	We can also calculate the "\NewTerm{elementary solid angle}\index{elementary solid angle}" as shown below:
	\begin{figure}[H]
		\centering
		\includegraphics{img/geometry/elementary_solid_angle_configuration.jpg}
		\caption{Configuration for the definition of the elementary solid angle}
	\end{figure}
	Given an elementary solid angle $\mathrm{d}\Omega$ and $\overline{\text{O}M}$ the cone axis. We put:
	
	We consider any surface $\Sigma$ passing through the point $M$. $\mathrm{d}\Omega$ cut on this surface a portion $\mathrm{d}\Sigma$.
	
	If we draw the sphere $S$ of center $\text{O}$ and radius $r$, this solid angle cut on the sphere a cap of area $\mathrm{d}S$:
	
	Given $MN$ the normal to $\mathrm{d}\Sigma$ which make an angle $\theta$ with $OM$. We have assimilating $\mathrm{d}S$ and $\mathrm{d}\Sigma$ to portions of a plane:
	
	Therefore:
	
	This concept of solid angle will be very useful especially in the field of theoretical physics that deals with the thermal radiation (see sections on Optics page \pageref{solid angle optics} and Thermodynamics page \pageref{solid angle black body} and Atomic Physics page \pageref{solid angle atomic physics}).
	
	We can still calculate from the previous concepts, the elementary solid angle of revolution as presented in the figure below:
	\begin{figure}[H]
		\centering
		\includegraphics{img/geometry/solid_angle_of_revolution_configuration.jpg}
		\caption{Configuration for the definition of the elementary solid angle of revolution}
	\end{figure}
	It is between two solid angles of revolution whose half-angles at the apex differ of $\mathrm{d}\alpha$:
	
	Therefore:
	
	\begin{dem}
	In the section dealing with Geometric Shapes (\SeeChapter{see section Geometric Shapes page \pageref{sphere}}) we proved the different ways to calculate the area of a sphere. From these calculations it was concluded that the elementary surface with constant radius $R$ was:
	
	and since:
	
	the elementary solid angle is then:
	
	Thus, the solid angle defined by a cone of revolution, with an apex plane angle equal to $\alpha$ is given by:
	
	\begin{flushright}
		$\blacksquare$  Q.E.D.
	\end{flushright}
	\end{dem}
	
	\begin{flushright}
	\begin{tabular}{l c}
	\circled{50} & \pbox{20cm}{\score{3}{5} \\ {\tiny 186 votes,  80.54\%}} 
	\end{tabular} 
	\end{flushright}
	
	%to force start on odd page
	\newpage
	\thispagestyle{empty}
	\mbox{}	
	\section{Euclidean Geometry}\label{euclidean geometry}

	The purpose of the "\NewTerm{Euclidean geometry}\index{Euclidean geometry}" (more commonly named "\NewTerm{plane geometry}\index{plane geometry}") is, in principle, the study of forms and properties of natural bodies. The geometry is however not an experimental science, since its purpose is not to study certain aspects of nature, but a necessarily arbitrary reproduction of it.

We will present in this section implicitly,at first, the five postulates of the Euclidean geometry (the first four are nowadays regarded as axioms) and then develop around these basic geometry that will be necessary to the reader need to study the rest of this book. Once this is done, we will summarize our study by presenting explicitly the five postulates of Euclid's and then Hilbert's axioms.

	\begin{tcolorbox}[title=Remark,colframe=black,arc=10pt]
We tried to preserve Euclid's notations as best as possible but although with a little more modern approach to certain concepts and present only those that are useful to Applied Mathematics.
	\end{tcolorbox}
	
	\subsection{Objects of Euclidean Geometry}
	
	Before stating the five Euclid's postulates, it seems good to set some intuitive concepts first:
	
	\textbf{Definitions (\#\mydef):}
	\begin{enumerate}
		\item[D1.] The simplest experimental concept is that of "\NewTerm{volume}\index{volume}". We say that a body occupies a given volume when it occupies in the three-dimensional space a certain place (for spaces with higher dimensions, we talk about "\NewTerm{hyper-volumes}\index{hyper-volume}").
		
		\item[D2.] We assume as obvious that a volume is limited by a "\NewTerm{surface}\index{surface}"; but if the existence of the volume is physically controllable and measurable, the surface is a creation of the mind; it is something similar to a balloon, for example, wrapping any volume, but only similar. It is a two-dimensional geometric without thickness.
		
		\item[D3.] When a surface is limited, this limit is a "line". Again, the line is a creation of the mind, a line has no experimental existence; it is something similar to the figure formed by a wire. Still geometric entity but without width (otherwise it is a surface)... only a length.
		
		\item[D4.] A "\NewTerm{straight-line}\index{straight-line}" is defined as the unbounded line of shortest distance joining two points on a surface. In other words, A straight line is a line with no beginning and no end, and therefore infinitely long.
		
		\item[D5.] When a line is limited (bounded), its limit is a "\NewTerm{point}\index{point}": the point is something analogous to the intersection of two stretched wires. This is still a creation of the mind, a geometric being (because it is suppose to have now width, no length, no height... just no dimensions).


	\begin{tcolorbox}[title=Remark,colframe=black,arc=10pt]
It is customary in geometry to represent a point by a letter $A, B, ...$; a line or a surface by a letter in brackets (but this is rarely respected because we often assume the reader knows what we're talking about). Then we write, for example: the line $(L)$, the surface $(S)$.
	\end{tcolorbox}		
		
		\item[D6.] The term the "\NewTerm{segment $\overline{AB}$}\index{segment}" generally designates a line bounded by the points $A$ and $B$. We say that a point $M$ is on the segment $\overline{AB}$ is to translate the following fact: any segment $\overline{AB}$ can be split in an infinity of ways into two pieces limited by $A$ and $M$ on one hand, by $M$ and $B$ on the other - fact actually inspired by the experimental possibility to cut a piece of wire in half, and that in an infinite of ways (we will come back later about this).

	\begin{tcolorbox}[title=Remark,colframe=black,arc=10pt]
		The sentence: «The line $(L)$ is drawn on a surface $(S)$» means that the surface $(S)$ could be divided into several pieces, so that the line $(L)$ is the boundary or part of a boundary of one of these pieces. This definition is based on the fact that it is possible to cut a tissue, for example by following with scissors any line on this tissue.
	\end{tcolorbox}

When a line $(L)$ is drawn on a surface $(S)$, any point $M$ which is located on the line $(L)$ is, by definition, also located on the surface $(S)$. Then we say that it is a "point of this surface."

		\item[D7.] A "\NewTerm{semi-straight line}\index{semi-straight line}" is a line which has a boundary (or vertex) on one side and is infinite on the other side.
		
		\begin{figure}[H]
		\centering
		\includegraphics{img/geometry/lines_type.jpg}
		\caption{Difference between line, segment and semi-straight line}
		\end{figure}
		
		\item[D8.] We name "\NewTerm{angle}\index{angle}" (or "\NewTerm{plane angle}\index{plane angle}") or more strictly "\NewTerm{straight angle}\index{straight line}" the plane portion limited by two half-lines (see further below the definition of a half-line).
	\end{enumerate}
	
	\subsubsection{Dimensions}\label{dimensions}

We talked before about volume, surface and line to whom we can associate dimensions. But what is a dimension? 

We will try to try to define it as best as possible but first, it is important to know that there are several types of dimensions in geometry. The best known and common one is that we call "\NewTerm{topological dimension}\index{topological dimension}".

For example, the point (mathematical and geometric abstraction) has a topological dimension of $0$, the curve (continuous line of zero thickness) a dimension of $1$, a  surface a dimension of $2$, a volume of a dimensions $3$ and a hyper- volume a dimensions of $4$ (to represent a hyper-volume take a volume drawn on a paper (...) make it a translation and connect the vertices). These are all integer values by definition:

{\centering
\begin{table}[H]
\begin{tabular}{cc}
\begin{subfigure}{0.5\textwidth}\centering
\textbf{Dimension: $0$}\\
\includegraphics[width=0.5\columnwidth]{img/geometry/point.eps}\caption{A point}\end{subfigure}&
\begin{subfigure}{0.5\textwidth}\centering
\textbf{Dimension: $1$}\\
\includegraphics[width=0.5\columnwidth]{img/geometry/line.eps}\caption{A line (straight or curved)}\end{subfigure}\\
\newline
\begin{subfigure}{0.5\textwidth}\centering
\textbf{Dimension: $2$}\\
\includegraphics[width=0.5\columnwidth]{img/geometry/surface.eps}\caption{Plane figure/Surface (polygon, ellipse, etc.)}\end{subfigure}&
\begin{subfigure}{0.5\textwidth}\centering
\textbf{Dimension: $3$}\\
\includegraphics[width=0.5\columnwidth]{img/geometry/volume.eps}\caption{A solid or body (sphere, parallelepiped, etc.)}\end{subfigure}\\
\end{tabular}
\caption{Objects, representations and corresponding dimensions}
\end{table}
}

In physics and mathematics, the dimension of a mathematical space (or object) is informally defined as the minimum number of coordinates needed to specify any point within it. Thus a line has a dimension of one because only one coordinate is needed to specify a point on it. 

	In mathematics, the dimension of an object is an intrinsic property independent of the space in which the object is embedded. For example, a point on the unit circle in the plane can be specified by two Cartesian coordinates, but a single polar coordinate (the angle) would be sufficient (for more details on coordinate systems see the section of Vector Calculus), so the circle is 1-dimensional even though it exists in the 2-dimensional plane. This intrinsic notion of dimension is one of the chief ways the mathematical notion of dimension differs from its common usages.

	\begin{figure}[H]
		\centering
		\includegraphics[scale=0.6]{img/geometry/co_ordinates_systems.jpg}
		\caption[Example of some co-ordinate systems]{Example of some co-ordinate systems (source: Wikipedia)}
	\end{figure}

	To calculate the size of certain objects, we will use the method of the plane metric geometry by taking a standard measure of this object, that is to say the object itself but smaller, and to carry it over our object a given number of times:
	\begin{figure}[H]
		\centering
		\includegraphics{img/geometry/standard_measurement_1d.jpg}
		\caption{Concept of one-dimensional standard}
	\end{figure}
	Let $L$ be the total length of the segment. We will take a standard segment of length $n$ that we will report on the big segment. This standard will be reported $L / n$ times. We will write this:
	
	We can apply the same reasoning to a surface:
	\begin{figure}[H]
		\centering
		\includegraphics{img/geometry/standard_measurement_2d.jpg}
		\caption{Concept of two-dimensional standard}
	\end{figure}
	Let $L^2$ be the total area of the square. We will take a standard square of area $n^2$ that we will report into the big square. This standard will be reported $L^2 / n^2$ times. We note that:
	
	We can apply the same reasoning to a volume:
	\begin{figure}[H]
		\centering
		\includegraphics{img/geometry/standard_measurement_3d.jpg}
		\caption{Concept of three-dimensional standard}
	\end{figure}
	Let $L^3$ be the total area of the volume. We will take a standard cube of area $n^2$ that we will report into the big cube. This standard will be reported $L^3 / n^3$ times. We note that:
	
	and so on...
	
	Through these three examples, we make appear the number $1$ for the segment,  the number $2$ for the square, the number $3$ for the volume. These numbers are the "dimension" of the object (same applies for a sphere for example).
	
	So, to summarize:
	\begin{enumerate}
		\item The lines are of dimension $1$, because to measure a length with a finer division by a factor $n$ of a standard segment, the number of subdivisions will be multiplied by the same factor. So there is a unity power relation between the subdivision and the measurement.
		
		\item Surfaces are two-dimensional, as for measuring a surface with a finer division by a factor $n$ of a standard square, the number of subdivisions will be given by the square power of $n$. Therefore there is a power of two relation between the subdivision and the measurement (if we take the square two times smaller to cover a surface, we will need four times the standard).
		
		\item Volumes are 3-dimensional, as for measuring a volume with a finer division by a factor $n$ of a standard cube, the number of subdivisions will be given by the cubic power of $n$. So there is a power of three relation between the subdivision and the measurement (if we take cubes two time smaller to fill in a volume, we will need eight times the standard).
	\end{enumerate}
	Let us generalize: let $N$ be the number of times we report the standard of length $n$ on our object of length $L$, and given $d$ the dimension of the object, we have:
	
	In the case of fractals (\SeeChapter{see section Fractals page \pageref{fractals}}) the dimensions are variable and fractional. Consider the Von Koch curve (for example) after one iteration of the sequence defining it:
	\begin{figure}[H]
		\centering
		\includegraphics{img/geometry/vonKoch_whole.jpg}
		\caption{von Koch curve}
	\end{figure}
	Let $L$ be its size such that $L=1$. To calculate its dimension we choose the fundamental element of the curve in red below (yes... we could take the segment of line for sure...):
	\begin{figure}[H]
		\centering
		\includegraphics{img/geometry/vonKoch_standard.jpg}
		\caption{Standard choice for Von Koch curve}
	\end{figure}
	Let $n$ be the size of this standard such as $n=1/3$. We see very well that we can report it $4$ times on the curve. Therefore:
	
	The dimension of the Von Koch fractal therefore has a fractional value and is more a curve (because close to $1$), rather than a surface (which is of dimension $2$). Cauliflower are for example fractal with dimension $2.33$...
	
	So we can calculate the size of any fractal objects on condition of knowing their standard measurement.
	
	We don't venture fetching complex objects in some galaxies, while the most famous fractal is in your plate (the second being your lungs ...). Eh yes! Cauliflower is a fractal (as well as your lungs)! You've probably noticed that when we cut cauliflower (not something recommended to try with your lungs ...), we are breaking it instead of cutting it, and it gives lots of small cauliflower, which themselves may give other smaller cauliflower. This characteristic of self-similarity at different scales is makes the cauliflower a fractal.
	
	Let us calculate now the fractal dimension of cauliflower. When we break cauliflower, we get between 12 and 14 branches that resemble to the whole cauliflower close to a given scale... This expansion is, if we calculate it, of a factor 3. So according to the formula above, the fractal dimension of the cauliflower is approximately:
	
	So the cauliflower is closer of a surface than a volume.
	Let us do one last example with Sierpinski triangle fractal (\SeeChapter{see section Fractals page \pageref{sierpinski fractal}}):
	\begin{figure}[H]
		\centering
		\includegraphics{img/geometry/sierpinski_triangle.jpg}
		\caption{Sierpinski triangle fractal}
	\end{figure}
	First, it is clear that we need a square to cover it (round it) completely. With squares two times smaller, we will need three squares:
	\begin{figure}[H]
		\centering
		\includegraphics{img/geometry/sierpinski_triangle_three_squares.jpg}
		\caption{Three squares to cover the Sierpinski fractal}
	\end{figure}
	If we divide again once the size of the squares by two, we will need $9$ squares to cover the triangle:
	\begin{figure}[H]
		\centering
		\includegraphics{img/geometry/sierpinski_triangle_nine_squares.jpg}
		\caption{Nine squares to cover the Sierpinski fractal}
	\end{figure}
	If we divide once again the size of the square by two we will need $27$ of them:
	\begin{figure}[H]
		\centering
		\includegraphics{img/geometry/sierpinski_triangle_twentyseven_squares.jpg}
		\caption{Twenty seven squares to cover the Sierpinski fractal}
	\end{figure}
	And so we see that:
	
	And therefore:
	
	So the Sierpinski fractal is closer to a surface than a curve.
	
	There exist also other dimensions. Take for example the "\NewTerm{homothetic dimensions}\index{homothetic dimensions}" Here are some simple examples (see further below the strict definition of "homothetic transformation"):
	\begin{figure}[H]
		\centering
		\includegraphics{img/geometry/homothetic_dimension.jpg}
		\caption{Representation of homothetic dimensions}
	\end{figure}
	The segment (far left), of $1$-dimension, has by homothetic transformation seen its length doubled, we note that:
	
	The square (at the center), of $2$-dimension has by homothetic transformation seen its surface doubled and we note that:
	
	The cube (far right), of 3-dimension has by homothetic transformation seen its volume has quadruple and we note that:
	
	The duplication of scale factor (homothetic) is therefore equal to:
	
	As you can see, this is always an integer value but of a different kind of dimension.
	
	The concept of dimensions having been introduced let us look now to the Euclid's postulates which may seem vague at first but will be detailed as we go far in our reading of this section.
	
	\pagebreak
	\subsection{Euclid's Constructions}\label{euclid's postulates}
	The construction of the planar Euclidean geometry is based on five following postulates (the first four are today regarded as axioms as we have already mentioned):
	\begin{enumerate}
		\item[P1.] Drawing a straight line of any point to any point.
		
		In a modern way we would say that through two distinct points $A$ and $B$, it passes a straight line and it goes in only one.
		
		In other words, two straight lines $(D)$ and $(D ')$ which have two common points are coincident, any point of one is a point of the other and vice versa.
		
		It follows from this postulate that two lines $(D)$ and $(D ')$ have no common point or have a single common point named "\NewTerm{intersection}\index{intersection point}" and are then "\NewTerm{intersecting}" and "\NewTerm{distinct}\index{distinct lines}" or have more than one common point and are then "\NewTerm{combined}\index{combined lines}".
		
		\item[P2.] Extend indefinitely, in its direction, a finite straight line.
		
		In a modern way we will say that any finite segment $\overline{AB}$ can be extended in a straight line passing through $A$ and $B$ (given the first axiom, it is unique in Euclidean geometry).
		
		\item[P3.] From any point and with any interval, describe a circle circumference.
		
		In a modern way we will say that for any point $A$ and any separate point $B$ of $A$, we can draw a circle with center $A$ passing through $B$.
		
		\item[P4.] All right angles are equal.
		
		Under modern form we say that for every angle $\widehat{xyz}$ of the plane corresponds its  measurement $\theta$ (also denoted in high-school by $\measuredangle AOB$), performed with a unit chosen once for all where $\theta$ is a positive number, less than $2\pi$. Conversely, let $\theta$ be any positive number between $0$ and $2\pi$, we shall assume that there is an infinite number of angles $\widehat{xyz}$ equal to one another whose measurement with the chosen angle unit is $\theta$.
		
		 \item[P5.] If a straight line, falling on two straight lines, makes the interior angles on the same side smaller than two right angles, these lines, extended to infinity, will meet at the side where the angles are smaller than the two rights.
		 
		 In a modern way we will say that given a straight line and a point, there is a unique line through this point and not intersecting the initial right.		
	\end{enumerate}

	\begin{figure}[H]
		\centering
		\includegraphics{img/geometry/euclids_postulates.jpg}
		\caption{Summary of Euclid's postulates}
	\end{figure}
	The Euclid's postulates will allow us the development of the concept of measurement of a length, area, volume, angle, as we shall see below.
	
	The two fundamental theorems of Euclidean geometry is the Pythagorean theorem and the Thales theorem as we will prove later. A bit of analysis permits us to go further with trigonometry that we have already developed in the previous section.
	
	\subsubsection{Segments and Lines}
	First, the simplest geometric figure (excepted the point ...) in Euclidean geometry is the "straight line" and it is directly concerned by the first two Euclid's postulates.
	
	\textbf{Definitions (\#\mydef):}
	\begin{enumerate}
		\item[D1.] The "\NewTerm{straight line}\index{straight line}" is the image given by a tensioned wire of zero thickness and infinite length.
		\begin{tcolorbox}[title=Remark,colframe=black,arc=10pt]
		We can also define the "straight line" as an infinity of points placed next to each other in a same direction on a plane.
		\end{tcolorbox}
		
		\item[D2.] We name "\NewTerm{half-line}\index{half-line}" the portion of a line limited to a point O named "\NewTerm{origin}\index{origin}".
		\begin{tcolorbox}[title=Remark,colframe=black,arc=10pt]
		The expression the "\NewTerm{half-line $\overline{\text{O}A}$}\index{half-line}" means the half-line of origin O, point named the "first", which contains the point $A$.
		\end{tcolorbox}
		
		\item[D3.] We say that two half-lines $\overline{\text{O}A}$, $\overline{\text{O}B}$ are "\NewTerm{opposing half-lines}\index{opposing half-lines}" when they constitute the whole line $\overline{AB}$.
		
		\item[D4.] We name "\NewTerm{segment $\overline{AB}$}\index{segment}" the portion of a right straight line limited by two points $A$ and $B$. These points are called the "extremities" of the segment.
	\end{enumerate}
	
	\pagebreak
	\paragraph{Quantities of the same type}\mbox{}\\\\\
	We say that geometric figures (meaning straight lines) are of the "\NewTerm{same type quantities}\index{identical geometric quantities}" when it is possible to define:
	\begin{enumerate}
		\item In which case a figure $(A)$ is said to be equal to a figure $(B)$ and, if they are unequal, which is the smallest.
		
		\item What we need to understand when we sum a figure $(A)$ and a figure $(B)$.
	\end{enumerate}
	The definitions should be chosen such that if $(A)$ is smaller than said $(B)$ and $(B)$ smaller than $(C)$, then $(A)$ must be reported to be small than $(C)$.
	
	It is necessary, moreover, that the figure resulting of the sum of $(A)$ and $(B)$ is equal to that which is named: sum of $(B)$ and $(A)$.
	
	Finally, a substitution in a comparison, or an equal amount, of a figure by an identical figure should not affect the result of operations.
	
	To understand what are the quantities of the same type, take the example of line segments:
	\begin{itemize}
		\item We admit that it is possible to determine the equality of two segments $\overline{AB}$ and $\overline{A'B'}$ when we can make them coincide.
		
		\item We also admit that it is possible to replace the segment $\overline{A'B'} \in AB$ by an equal segment $\overline{AD}$ and putt on the line $AB$.
		
		\item Finally, we will assume that it is possible to distinguish between three points $A, B, C$, taken at random on a line or segment and to know which is between the other two.
	\end{itemize}
	
	We then agree to say that the segment $\overline{A'B'}$ is smaller than the segment $\overline{AB}$, which is written formally 
(\SeeChapter{see section Operators page \pageref{comparators}}):
	
	when the point $C$, obtained by putting it on the segment $\overline{AB}$ a segment  $\overline{AB}$ equal to $\overline{A'B'}$, falls between $A$ and $B$.
	
	If the point $C$ was on $B$, the segments $\overline{AB}$ and $\overline{A'B'}$  would be equal and then we would write:
	
	We agree to named "\NewTerm{sum of two segments}\index{sum of two segments}" $\overline{AB}$, $\overline{A'B'}$, the segment $\overline{AC}$ obtained by translating on the half-line opposed to the segment $\overline{BA}$ a segment $\overline{AB}$ equal to $\overline{A'B'}$. We translate this by writing:
	
	Let us still consider quantities of the same species ... Add therebetween several of these quantities is to add one of them to another, the sum determined to another third, etc. For example, add the segments $\overline{AB}, \overline{BC}, \overline{CD}$, is add $\overline{AB}$ and $\overline{BC}$ which gives $\overline{AC}$, then $\overline{AC}$ and $\overline{CD}$ which gives $\overline{AD}$. We summarize the operation by writing:
	
	Multiply by a quantity by an integer $n$, it is $n$ quantities to itself. For example, if we have $\overline{AB} = \overline{BC} =\overline{CD}$, the above relation would be written:
	
	We will now define what we name "compare two quantities $(A)$ and $(B)$ of the same kind". For this let us arbitrarily select a quantity $(C)$ of the same kind as $(A)$ and $(B)$ and smaller than each of them. Let us build a sequence of quantities such that:
	
	We find that the quantity $(A)$ is between two quantities $(C_p)$ and $(C_{p+1})$ equation and $(B)$ is located between two other $(C_q)$ and $(C_{q+1}$ by construction.
	
	\begin{tcolorbox}[colframe=black,colback=white,sharp corners]
	\textbf{{\Large \ding{45}}Example:}\\\\
	Consider then, for example, any two segments $\overline{AB}, \overline{A'B'}$ but different (e.g. $1.2\;[\text{cm}]$ and $3.5\;[\text{cm}]$):\\
	
	To perform the previous operation, let us use a graduated rule which unit will be the quantity $(C)$, an arbitrary segment (for example $1 \;[\text{cm}]$).\\
	
	We apply the zero of the rule on $A$, and $B$ will by design intercalate between two graduations of the rule numbered $p$ and $p + 1$ unless the size $(C)$ is equal to $\overline{AB}$ ... (either with the choice taken as an example, $B$ intercalate between the $1$st and $2$nd graduation).\\
	
	For $\overline{A'B'}$, we apply the zero of the rule also on $A'$, and $B'$ also intercalate between two values of the rule numbered $q$ and $q + 1$ unless the variable $C$ is equal to $\overline{A'B'}$ ... (either with the choice taken as an example, $B'$ will intercalate between the $3$rd and the $4$th one).\\
	
	We will express the results of these measurements and their ratio by writing:
	
	where the term on the left extremity is named a "\NewTerm{default measurement}\index{default measurement}" and this at the opposite a "\NewTerm{measurement by excesses}\index{measurement by excesses}".\\
	
	So with the measures taken as an example we have:
	
	\end{tcolorbox}
	\textbf{Definition (\#\mydef):} And in physics, we name "\NewTerm{measure of a quantity}\index{measure of a quantity}" $(A)$ the positive number which measures the ratio of this size and a quantity $(U)$ arbitrarily chosen as reference and which we name the "\NewTerm{unit}\index{unit}", the measuring unit being "$1$", by definition.
	
	We can show that if $a$ is a measure of $(A)$, $b$ that of $(B)$ evaluated both with the same unit $(U)$, the number $(A) / (B)$ is equal to the ratio $a / b$. This ratio is (and this should be obvious for the reader at this level) independent of the chosen unit of measurement.
	\begin{tcolorbox}[title=Remark,colframe=black,arc=10pt]
	We say that $(B)$ is an "\NewTerm{aliquot part}\index{aliquot part}" of $(A)$ if the ratio $(A)/(B)$ is an integer.
	\end{tcolorbox}
	We will agree once and for all, that in geometry all quantities of the same species that occur in a given figure are measured with the same unit.
	
	Given $(A), (B), (C), ..., (S)$ quantities of the same species and $(A'), (B'), (C'), ..., (S ')$, quantities of the same species, but that are not necessarily of the same kind as the previous. We say that these quantities are "\NewTerm{homologous}\index{homologous}" if we can group them in pairs, $(A ')$ homologous to $(A)$, $(B')$ homologous to $(B)$, ..., etc., so that following conditions are met:
	\begin{itemize}
		\item If $(A)$ is equal to $(B)$, $(A')$ is equal to $(B')$.
		\item If $(A)$ is smaller than $(B)$, $(A')$ is smaller than $(B')$.
		\item If $(S)$ is the sum of $(A)$ and $(B)$, $(S')$ is the sum of $(A')$ and $(B')$.
	\end{itemize}
	To calculate the ratio $(A) / (B)$, let us build the following sequence $(C_1),(C_2),...,(C_p)$ and to calculate the ratio $(A')/(B')$, let us build the sequence  $(C_1^{\prime}),(C_2^{\prime}),...,(C_p^{\prime})$ following equation, obtained as the previous one, but from the quantity $(C ')$ homolog of $(C)$.
	
	It is obvious that if $(A)$ can be intercalate between $(C_p)$ and $(C_{p+1})$, $(A')$ could therefore also be intercalate between $(C_p^{\prime})$ and $(C_{p+1}^{\prime})$; Similarly, if $(B)$ can be intercalated between $(C_q)$ and $(C_{q+1})$. The ratios $(A) / (B)$ and $(A ') / (B')$ will be bounded by the same numbers:
		
	Consequently, the ratio between two quantities $(A)$ and $(B)$ is equal to the ratio of the homolog quantities $(A')$ and $(B')$.

	If, in particular, the quantities $(A),(B),...$ are measured with a unity $(U)$, and if the quantities $(A'),(B'),...$ are measured with the sames unities $(U')$, homolog of $(U)$, the equal ratios:
	
	are only the measurement of $(A)$ and of $(A')$. Consequently:
	
	The measurements of two homolog quantities $(A)$ and $(A')$ are equal at the condition that the chosen units to measure them are also homolog quantities.

	Les us consider now on a half-line along $\text{O}x$-axis a point $M$. Given $x$ the measurement (following previous definition) of $\overline{\text{O}M}$. For every point $M$ of the half-line $\overline{\text{O}M}$ corresponds a unique positive $x$. We will admit that to any positive value $x$ randomly chosen correspond a unique point $M$ on the half-line.
	
	A consequence of this assumption is that there is a point, and only one, $C$, which divides the segment $\overline{\text{O}M}$ in equal parts. This point is the point of the half-line $\overline{\text{O}M}$, such that:
	
	We name this point the "\NewTerm{middle of the segment}\index{middle of a segment}" $\overline{\text{O}M}$.
	\begin{theorem}
	There exists a point $M$ and only one located on the segment $\overline{AB}$ such that the relation $\overline{MA}/\overline{MB}$ is equal to a given positive number.
	\end{theorem}
	\begin{tcolorbox}[title=Remark,colframe=black,arc=10pt]
	Obviously if $\lambda=1$, this point is the middle of the segment.
	\end{tcolorbox}
	\begin{dem}
	Given $M$ any point of the segment $\overline{AB}$ on a plane; given $x$, the measurement of $\overline{AM}$ and $a$  the measurement of $\overline{AB}$: the measurement of $\overline{MB}$ will be equal to $a-x$ since $M$ is placed between $A$ and $B$. We will have:
	
	For this ratio to be equal to $\lambda$, we have to, and it is sufficient to, that $x$ is solution of the equation:
	
	This equation has for unique solution:
	
	To this positive $x$ value that is less than $a$ corresponds a point $M$ and only one of the half-line $\overline{AB}$ such as $\overline{MA} = x$. This point $M$ satisfies, and satisfied alone, to the imposed conditions.
	\begin{flushright}
		$\blacksquare$  Q.E.D.
	\end{flushright}
	\end{dem}
	
	\begin{theorem}
	There is a point $M$ and only one of the line $\overline{AB}$, located outside the segment $\overline{AB}$, such as the ratio $\overline{MA}/\overline{MB}$  equals a given definite number $\lambda$ and from $1$.
	\end{theorem}
	\begin{dem} Let us prove this uniqueness in two steps:
	\begin{enumerate}
		\item Let us suppose $\lambda>1$. Given $M$ any point of the line $\overline{AB}$ located outside the segment $\overline{AB}$: even $A$ is on the $\overline{MB}$ segment, or $B$ is on the $\overline{MA}$ segment. If $A$ is the $\overline{MB}$segment, we have necessarily $\overline{MB}>\overline{MA}$, thus:
		
		The point $M$ thus does not respond to the question of uniqueness.
		
		If $B$ is on the segment $\overline{MA}$ and we denote$\overline{MA} = x, \overline{AB} = a, \overline{MB}=x-a$. So we have:
		
		For this ratio to be equal to $\lambda$, it is necessary, and sufficient that $x$ is a solution of the equation:
		
		This equation admits as unique solution:
		
		that gives a value of $x$ that will always be positive and greater than $\lambda$. To this positive $x$ value and greater than $a$ corresponds a point $M$ and only one of the half-line $\overline{AB}$. This point $M$ satisfies, and only it, to the conditions of uniqueness imposed.
		
		\item Let us suppose that $\lambda<1$. We will seek the point $M$ for which (we simply reversed the ratio):
		
		is a number greater than $1$. There is one and only one from the point (1). It is the only one that satisfies the imposed conditions.
	\begin{tcolorbox}[title=Remark,colframe=black,arc=10pt]
	There is no point $M$ located outside the segment $\overline{AB}$ for which $\overline{MA} / \overline{MB} = 1$. Indeed, if $A$ is on the $\overline{MB}$ segment, we have regardless of $M$, $\overline{MA}/\overline{MB}<1$, and if $B$ is on the segment $\overline{MA}$, $\overline{MA}/\overline{MB}>1$....
	\end{tcolorbox}
		
	\end{enumerate}
	\begin{flushright}
		$\blacksquare$  Q.E.D.
	\end{flushright}
	\end{dem}
	
	\pagebreak
	\subsection{Plane Geometry}
	Let us now study a geometrical object of greater dimension than that of the straight line that is the Plane and the Surface.
	
	A plane is a flat, two-dimensional surface that extends infinitely far. A plane is the two-dimensional analogue of a point (zero dimensions), a line (one dimension) and three-dimensional space. Planes can arise as subspaces of some higher-dimensional space, as with a room's walls extended infinitely far, or they may enjoy an independent existence in their own right, as in the setting of Euclidean geometry.
	
	When working exclusively in two-dimensional Euclidean space, the definite article is used, so, "the plane" refers to the whole space. Many fundamental tasks in mathematics, geometry, trigonometry, graph theory and graphing are performed in a two-dimensional space, or in other words, in: "the plane".
	
	Let us now consider a finite surface $(S)$ and two points $A$ and $B$ of this surface. At least two simple cases are possible:
	
	\begin{enumerate}
		\item There are points of the straight segment $\overline{AB}$ that are not on the surface $(S)$. We say in this case that: "the segment $\overline{AB}$ intersects the surface"; the common point to the segment $\overline{AB}$ and the surface $(S)$ are the intersection points of the surface $(S)$ and of the segment $\overline{AB}$. Among these common points there are especially the points $A$ and $B$.
		
		\item All points on the segment $\overline{AB}$ are points of  the surface $(S)$. We say then that: "the segment $\overline{AB}$ is on the surface $(S)$".
	\end{enumerate}

	\textbf{Definition (\#\mydef):} We name "\NewTerm{plane}\index{plane}" the surface such that any segment $\overline{AB}$ joining two points arbitrarily chosen on the surface, is on the surface.
	
	We will assume that such a surface exists and that in an Euclidean space of any number of dimensions, a plane is uniquely determined by any of the following:	
	\begin{itemize}
		\item Three non-collinear points (points not on a single line).
		\item A line and a point not on that line.
		\item Two distinct but intersecting lines.
		\item Two parallel lines.
	\end{itemize}
	
	The following properties hold in three-dimensional Euclidean space but not in higher dimensions, though they have higher-dimensional analogues:
	\begin{enumerate}
		\item[P1.] Two distinct planes are either parallel or they intersect in a line.
		\item[P2.] A line is either parallel to a plane, intersects it at a single point, or is contained in the plane.
		\item[P3.] Two distinct lines perpendicular to the same plane must be parallel to each other.
		\item[P4.] Two distinct planes perpendicular to the same line must be parallel to each other.
	\end{enumerate}
	
	The study of planes will be made later in the section (but mathematical vectorial analysis of a plane is done the in the section of Analytical Geometry). we will now focus on the study of geometric figures drawn in a given plane, figures named "\NewTerm{plane figures}\index{plane figures}". Their study is related to the field of "\NewTerm{plane geometry}\index{plane geometry}".
	
	All the shapes exist in a flat plane. A plane can be thought of an a flat sheet with no thickness, and which goes on for ever in both directions. It is absolutely flat and infinitely large, which makes it hard to draw.
	
	\begin{tcolorbox}[title=Remarks,colframe=black,arc=10pt]
	\textbf{R1.} The study of geometry can be broken into two broad types: plane geometry, which deals with only two dimensions, and solid geometry which allows all three. The world around us is obviously three-dimensional, having width, depth and height, Solid geometry deals with objects in that space such as cubes and spheres.\\
	
	\textbf{R2.} In practice, the figures are drawn either on a sheet of paper or on the surface of the blackboard (or on a computer screen).
	\end{tcolorbox}
	
	\subsubsection{Displacements and Turnarounds}
	Given $(F)$ a drawing made on a plane board. Imagine we do on a transparent layer, which front  is applied on the board, a copy $(C)$ of the drawing $(F)$. Let us perform afterwards, by moving this transparent layer on another point of the plane board, a new drawing a new drawing $(F ')$ identical to $(F)$.
	
	Two major cases have to be considered mathematically speaking:
	\begin{enumerate}
		\item If the recto side of the transparent layer is remained applied to the plane board, the drawing $(F')$ is obtained from $(F)$ by an operation named "\NewTerm{displacement}\index{displacement}" or more commonly "\NewTerm{translation}\index{translation}".
		
		\item If, however, the transparent layer was horizontally or vertically returned, and if it is the verso that is applied to the plane board, the operation is named "\NewTerm{turnaround}\index{turnaround}" or more commonly a "\NewTerm{vertical or horizontal symmetry}\index{vertical symmetry}\index{horizontal symmetry}".
	\end{enumerate}
	In the figure below the red triangle is a translation of the blue one. The green triangle is a horizontal symmetry of the blue one plus a translation.
	\begin{figure}[H]
		\centering
		\includegraphics{img/geometry/translation_symmetry.jpg}
		\caption{Illustrated example a translation and a symmetry}
	\end{figure}
	
	\textbf{Definition (\#\mydef):} We commonly say that the drawing $(F)$ is "\NewTerm{stackable}\index{stackable}" to drawing $(F '$) and that these drawings represent equal figures.

	\subsubsection{Plane angles}
	We have already defined the concept of "\NewTerm{angle}\index{angle}" in the previous section on Trigonometry (circle angle and spherical angle). We now have more the fourth Euclid's postulate at our disposal regarding the concept of plane angle.
	
	We will now come back more in detail about the concepts of plane angle and see the underlying concepts that will enable us to address further a particularly useful object which is the bisecting line!

	\textbf{Definitions (\#\mydef):}
	\begin{enumerate}
		\item[D1.] We name "\NewTerm{angle}\index{angle}" (or "\NewTerm{plane angle}\index{plane angle}") or more strictly "\NewTerm{linear angle}" the plane portion bounded by two half-lines O$A$, O$B$, for example. The point O is called the "\NewTerm{top}\index{top}" of the angle, the half-lines O$A$, O$B$ are named the "\NewTerm{sides}\index{sides}" of the angle.

		In highs-school, angles are measured with an instrument named a "\NewTerm{protractor}\index{protractor}":
		
		\begin{figure}[H]
			\centering
			\includegraphics[scale=0.375]{img/geometry/protractor.jpg}
			\caption{A half circle protractor marked in degrees ($180^circ$)}
		\end{figure}

		\item[D2.] We name "\NewTerm{angle between two half-lines $AB$, $AC$}\index{angle between two segments}", the angle of top $A$ which sides are the half-lines $AB$, $AC$.

		\item[D3.] The half-lines O$A$, O$B$ divide the plane into two regions: they therefore define two angles:
		\begin{enumerate}
			\item One consists of the covered hatch area (see figure on the far left below) is named "\NewTerm{salient angle}\index{salient angle}".
	
			\item The other consists of the covered hatch covered (see figure on the center below) is named "\NewTerm{reflex angle}\index{reflex angle}".
		\end{enumerate}
		\begin{figure}[H]
			\centering
			\includegraphics{img/geometry/angle_families.jpg}
			\caption{Salient angle, reflex angle, adjacent angle}
		\end{figure}
		The notation $\widehat{BOA}$ or $\widehat{AOB}$ denotes one of the two angles: the letter that designate the top should be (usually) written in the middle (often we do not mention the top if the context is clear). Where no precision accompanies this notation, it is by definition the salient angle!!

		\item[D4.] We name "\NewTerm{adjacent angles}\index{adjacent angle}" two angles which have the top and one side that are common and which are placed on either side of the common side. In the figure above the far right, the saillant angles $\widehat{AOB}$, $\widehat{BOC}$ are adjacent.
		
		Given $\widehat{AOB}$ and $\widehat{A'OB'}$  two angles of a sample plane. We have assumed previously that there is a displacement that brings $O'$ on $O$ and $A'$ on a point $A$ of the line $OA$. This movement causes $O'B'$ to move either on $OB_1$ so that the two angles $(\widehat{AOB},\widehat{AOB_1})$ are not adjacent, or on $OB_2$ so that $(\widehat{AOB},\widehat{AOB_2})$ are adjacent. 

		In the latter case, an additional half turn around $OA$ will bring $\widehat{AOB_2}$ into the position $\widehat{AOB_1}$. This movement and this rotation, if it occurs, replace $\widehat{A'O'B'}$ by an angle $\widehat{AOB_1}$ that is equal by definition.
		
		\begin{figure}[H]
			\centering
			\includegraphics{img/geometry/angle_deplacement.jpg}
			\caption{Representation of the movement of two angles}
		\end{figure}
		If $OB_1$ is merged with $OB$, there are points of one of the two angles $\widehat{AOB_1}$ and $\widehat{AOB}$ that are equal, any point of one being a point of the other, we say, in this case that: the angles $\widehat{AOB}$, $\widehat{A'O'B'}$ are "\NewTerm{equal angles}\index{equal angles}", which is expressed by the expression:
		
		If $OB_1$ is not merged with $OB$, there are points of one of the two angles $\widehat{AOB_1}$, for example, which are not points of $AOB$. On the figure above, the angle $\widehat{AOB}$ is covered by hatch, the angle $\widehat{AOB_1}$ also. The points we are speaking about are those of the angle $\widehat{BOB_1}$ covered only once by hatch. We will agree to say that the angle $\widehat{AOB_1}$ and, therefore, the angle equal to $\widehat{A'O'B'}$ are, in this case, greater than the angle $\widehat{AOB}$, which is expressed by the inequality:
		
		Now that we are able to compare angles, let us study how we can sum the up (and thus subtract respectively).
		
		Let us study first the case of the sum of two adjacent angles $\widehat{AOB}$, $\widehat{BOC}$. Two cases can occur following that the angles are salient angles or reflex angles:
		\begin{enumerate}
			\item Given $\widehat{AOB}$ and $\widehat{BOC}$ the two angles to sum (see left figure below). By definition, the sum of these angles is the angle $\widehat{AOC}$, what we express by the equality:
			
			
			\item Given $\widehat{AOB}$ a salient angle to sum up to the reflex angle $\widehat{BOC}$. If we cover of hatch successively the both angles (see right figure below), the salient angle $\widehat{AOC}$ is covered twice:
		\end{enumerate}
		\begin{figure}[H]
			\centering
			\includegraphics{img/geometry/angle_sum.jpg}
			\caption{Sum of two adjacent angles}
		\end{figure}
		In this case, the sum of the angles $\widehat{AOB}$ (saillant) and $\widehat{BOC}$ (reflex) is therefore equal to $\widehat{AOC}$ plus two "flat angles" (see below for the definition), which is expressed by writing:
		
		\begin{tcolorbox}[title=Remark,colframe=black,arc=10pt]
		This may seem confusing to some but those who have already read the section of Trigonometry know already that the angles of the trigonometric circle are equal to themselves modulo $2\pi$ [rad].
		\end{tcolorbox}
		Let us now study the case of the sum of any two angles:
		The sum of two angles $\hat{AOB}$, $\widehat{A'O'B'}$ is, by definition, equal to the sum of the angle $\widehat{AOB}$ and of an angle $\widehat{BOC}$ equal to the angle $\widehat{A'O'B'}$ and adjacent to the angle $\widehat{AOB}$.
		\begin{figure}[H]
			\centering
			\includegraphics{img/geometry/angle_sum_any.jpg}
			\caption{Sum of any two angles}
		\end{figure}
		Such an angle is obtained by a displacement which brings O on O' and $A$ at a point O$A$, followed or not by a rotation around O$A$.
		
		Let us study for last case the sum of more than two angles:
		
		The sum of several angles $\widehat{AOB}$, $\widehat{A'O'B'}$, etc., is by definition equal to the sum obtained by adding the first to the second, this sum to the third, and so on.
		
		Given $\widehat{AOB}$ the first angle, $\widehat{BOC}$ the second angle, $\widehat{KOL}$ an angle equal to the last angle to add and adjacent to the previous one. The result of the sum will be $\widehat{AOL}$ augmented as many times of two flat angles that the plane was recovered during this operation. We easily see that this result does not depend on the order of angles $\widehat{AOB},\widehat{A'O'B'}$ to add, etc.
	
		\item[D5.] Two angles formed by two lines cut by a secant are named "\NewTerm{alternate angles}\index{alternate angles}" if:
		\begin{enumerate}
			\item They are located on either side of the secant
	
			\item They are located between the two lines
			
			\item They are not adjacent angles
		\end{enumerate}
		In the example below, the straight lines $(X)$ and $(Y)$ are cut respectively in $A$ and $B$ by the secant $(S)$:
		\begin{figure}[H]
			\centering
			\includegraphics{img/geometry/alternate_angle.jpg}
			\caption{Example of alternate angle}
		\end{figure}
		and the two shown angles are alternate angles. In the case where $(X)$ and $(Y)$ are parallel, the alternate angles are equal. 

		It is also this which would have allowed Eratosthenes deduced the circumference of the Earth from a purely geometrical way (thus it is not useless!).
		
		\begin{tcolorbox}[colframe=black,colback=white,sharp corners]
		\textbf{{\Large \ding{45}}Example:}\\\\
		Indeed Eratosthenes used the observation he made on the shadow of two objects in two different places, Syene (now Aswan) and Alexandria, considered on the same meridian, the June 21 (summer solstice at his time) at local solar noon. It is at this precise time of the year that in the northern hemisphere the Sun holds the highest position above the horizon. In a previous observation, Eratosthenes noticed that there was no shadow in a water well at Syene. Thus, at that moment, the Sun was vertical and its light shone directly the bottom of the water well. Eratosthenes noticed however that the same day at the same hour, an obelisk located in Alexandria formed a shadow. The Sun was no longer vertical and the obelisk had an off centered shadow. Eratosthenes considered as parallel the light rays of the Sun at any point on the Earth (considered as spherical). He deduced that the angle between the sunlight and the vertical was of $7.2$ degrees.\\
		
		Eratosthenes then estimated the distance between Syene and Alexandria by using a bématiste which extrapolate upon the time of camel-day journey between the two cities: the distance achieved was estimated as $787.5$ kilometers (measure very close to reality).\\
		
		By the geometric theory of alternate congruent angles  Eratosthenes proposed a simple figure: it consisted of a single circle having a central angle of $7.2$ degrees which intercepts an arc (connecting Syene to Alexandria) of $800$ kilometers.
		\begin{figure}[H]
			\centering
			\includegraphics{img/geometry/erasthoten_alternate_angle.jpg}
			\caption{Alternate angle principle as used by Eratosthenes}
		\end{figure}
		By the ratio in the circles (already known at that time), he calculated the circumference of the Earth was by a simple rule of three equal to $39,375$ kilometers. What was a remarkably accurate measurement for that time (the current measures give an average of $40,075.02$ kilometers).
		\end{tcolorbox}
	\end{enumerate}
	
	Let us do a summary of most type on angles that the reader mus remember:
	\begin{figure}[H]
		\centering
		\includegraphics{img/geometry/angle_types.jpg}
		\caption{Angle types summary}
	\end{figure}
	
	\pagebreak
	\paragraph{Angle Measurements}\mbox{}\\\\
	We have defined the equality and the sum of two or more angles. These definitions meet the conditions of the same quantities variables we have already seen previously.

	So let us choose arbitrarily an angle of the plane $\widehat{aob}$, which will bet the unit angle for the plane. The measurement of the ratio:
	
	proceed as it has been explained previously during our study of the quantities of the same species, will be a positive number denoted most of time by the greek letters $\alpha$ or $\theta$ named by definition the "\NewTerm{angle measurement $\widehat{AOB}$, with the selected unit $\widehat{aob}$}\index{angle measurement}".
	
	We denote by to the lowercase Greek letter "pi" the irrational number:
	
	which is by definition the measurement value of a "\NewTerm{flat angle}\index{flat angle}" or also named "\NewTerm{plane angle}\index{plane angle}" (we have already see we this value comes from in the section of Trigonometry).
	\begin{tcolorbox}[title=Remark,colframe=black,arc=10pt]
	As all angles of the plane are smaller than two flat angles, the number $\theta$ (which is the measurement of $\widehat{AOB}$) must be less than $2\pi$.
	\end{tcolorbox}
	Having defined the flat angle, we can now define other types of commonly used angles:
	
	\textbf{Definitions (\#\mydef):}
	\begin{enumerate}
		\item[D1.] Two angles are "\NewTerm{perpendicular angles}\index{perpendicular angles}", denoted by the symbol $\perp$, when they are both rights and adjacent.

		\item[D2.] We name "\NewTerm{oriented angle}\index{oriented angle}" or "\NewTerm{vector angle}\index{vector angle}", the angle between two vectors or lines (\SeeChapter{see section Vector Calculus page \pageref{vector}}) having same origin and which value measured counter-clockwise will be taken as positive and negative if taken in the clockwise direction.
		
		The best known example of oriented angle is the trigonometric unit circle:
		\begin{figure}[H]
			\centering
			\includegraphics{img/geometry/angle_positive_oriented_or_null.jpg}
			\caption{Positive or null oriented angle}
		\end{figure}
		\begin{figure}[H]
			\centering
			\includegraphics{img/geometry/angle_negative_oriented_or_null.jpg}
			\caption{Negative or null oriented angle}
		\end{figure}
		
		\item[D3.] We say that two angles are "\NewTerm{additional angles}\index{additional angles}" when their sum is equal to right angles (i.e. a flat angle).

		\item[D4.] We say that two angles are "\NewTerm{complementary angles}\index{complementary angles}" when their sum is equal to a right angle (thus two complementary angles are always right...).
		
		The "\NewTerm{internal angle}\index{internal angle}" and "\NewTerm{external angle}" of a regular polygons are complementary angles as by construction:
		\begin{figure}[H]
			\centering
			\includegraphics[scale=0.5]{img/geometry/internal_external_angle.jpg}
			\caption[Internal + External angle]{Internal + External angle (source: Wikipedia)}
		\end{figure}
		
	\end{enumerate}
	Let us now consider the symbols $\alpha,\beta,\gamma$ as the measurements of several angles $widehat{AOB}$, $\widehat{BOC}$, $\widehat{COD}$.

	We will not insist on the obvious fact that the equalities $\widehat{AOB}=\widehat{BOC}$ or $\alpha=\beta$ are equivalent and thus the  inequalities $\widehat{AOB}<\widehat{BOC}$, or $\alpha<\beta$. These common sense remarks are applied whenever the same species of quantities will have been measured, of course with the same unit!
	
	However, we insist that, according to the definition of the sum of several angles, $\alpha+\beta+\gamma$ is the measure of the angle $\widehat{AOD}$ increased as many times of two flat angles $\pi$ that the plane was covered during the operations of addition:
	
	The relative integer $n$ which is introduce in this calculation has a value that can be specified, but that has no importance for the mathematician or physicist, as we will see later. We thus do not decide not to write the $2n\pi$ in geometry (as it will be always implicit). Also, we decide by convention to write:
	
	So we have:
	
	the notation convention means that $\theta$ is the measurement of the angle $\widehat{AOD}$ if we have $0\leq \theta <2\pi$.
	
	If $\theta$ is greater than $2\pi$, this equality means that the measurement of the angle $\widehat{AOD}$ is $\theta-2n\pi$, $n$ being a relative integer so that we have $0 \leq \theta-2n\pi <2\pi$.
	\begin{tcolorbox}[title=Remarks,colframe=black,arc=10pt]
	\textbf{R1.} If $\theta>2\pi$ there is only one unique positive integer $n$ and only one, as we have $0\leq \theta-2n\pi<2\pi$, that is to say:
	
	because the two numbers $(\theta/2\pi)-1$ and $\theta/2\pi$ are positive and different from $1$.\\
	
	\textbf{R2.} Saying that $\theta$ is the measurement of the angle $\widehat{AOD}$ suppose that we have $0\leq \theta<2\pi$ and leads to the equality $\widehat{AOD}=\alpha$. But write $\widehat{AOD}=\alpha$ does not necessarily mean that $\theta$ is the measurement of $\widehat{AOD}$; it is necessary for this that we have $0\leq \theta<2\pi$.
	\end{tcolorbox}
	
	\paragraph{Units of Angle Measurements}\mbox{}\\\\
	We have define the flat angle to be equal to $\pi$ without specifying the unit. This is what we will now apply to do. There are (still and sadly) several units of angle measurements which most important and common one are listed below:

	\textbf{Definitions (\#\mydef):}
	\begin{enumerate}
		\item[D1.] We name "\NewTerm{degree}\index{degree}" the $180$th part of a flat angle.
		
		\begin{tcolorbox}[title=Remark,colframe=black,arc=10pt]
		The original motivation for choosing the degree as a unit of rotations and angles is unknown.
		\end{tcolorbox}
		
		Many old former are still made in degrees (and still today in some areas of physics/astronomy). The sub-multiple of the degree is the: "\NewTerm{Sexagesimal minute}\index{Sexagesimal minute}" equal to the $60$th of 1 degree, and the "\NewTerm{Sexagesimal second}\index{Sexagesimal second}" is equal to the $60$th of the sexagesimal minute.
		
		\begin{tcolorbox}[colframe=black,colback=white,sharp corners]
		\textbf{{\Large \ding{45}}Example:}\\\\
		The value:
		
		 will be read as two hundred thirty six degrees, twenty minutes and forty two seconds.\\
		 
		 The conversion from degrees to fractional decimal angles is quite easy:
		
		\end{tcolorbox}
		Most countries around the world sadly still use today degrees in Kindergarten and High-schools but without making use of the notation of minutes and seconds (not very convenient for pedagogical reasons). 

		\item[D2.] As we have already see it in the section of Trigonometry, we name "\NewTerm{radian}\index{radian}" (denoted [rad]) the plane angle described by a secant to a circle, passing through its center, such that the arc as defined by the horizontal axis through the center of the circle and the secant be of equal length of the radius of the circle:
		\begin{figure}[H]
			\centering
			\includegraphics{img/geometry/radian_definition.jpg}
			\caption{Geometric definition of the radian}
		\end{figure}

		In most mathematical work beyond practical geometry, angles are typically measured in radians rather than degrees. This is for a variety of reasons; for example, the trigonometric functions have simpler and more "natural" properties when their arguments are expressed in radians (\SeeChapter{see section Trigonometry page \pageref{radian}}). 
		
		One complete turn ($360^\circ $) is equal to $2\pi$ radians, so a half turn of $180^\circ$ is equal to $\pi$ radians, or equivalently, the degree is a mathematical constant given by:
		
		Thus, in radians, a flat angle is equal to $\pi$ and all other angles are real multiples of this constant.
		\begin{figure}[H]
			\centering
			\includegraphics{img/geometry/degrees_radians.jpg}
			\caption[Degrees-Radians conversions figure]{Degrees-Radians conversions figure (source: Wikipedia)}
		\end{figure}

		\item[D3.] We name "\NewTerm{grad}\index{grad}" the $200$th part of the flat angle.
		
		The grade is also an old angle unit. Its sub-multiples are: the "centesimal minute," equal to $100$th of one grad, and the "centesimal second", equal to one hundredth of the centesimal minute.
		
		The value:
		
		 will be read as two hundred thirty six grads, eighteen  minutes and five second.
	\end{enumerate}
	
	\pagebreak
	\paragraph{Bisector}\mbox{}\\\\
	Now that we know how to compare, add and measure angles we will be able to look at an important concept in geometry that is the one of "bisection" and some relative properties that we will reuse later for quite important theorems.

	\textbf{Definitions (\#\mydef):}
	\begin{enumerate}
		\item[D1.] We name "\NewTerm{bisector}\index{bisector}" the straight line that divides an angle into two equal parts.

		\item[D2.] We name "\NewTerm{half-bisector}\index{half-bisector}"  the segment that divides an angle into two equal parts and start at the origin of the angle.

		\item[D3.] We name "\NewTerm{line segment bisector}\index{line segment bisector}" the (infinite) straight line that passes through the midpoint of a segment.

		Particularly important is the "\NewTerm{perpendicular bisector}\index{perpendicular bisector}" of a segment, which, according to its name, meets the segment at right angles. 
	\end{enumerate}
	
	Two straight lines $AB$ and $CD$ intersecting on a point $O$ form as we already know intuitively, four angles:
	
	The angles $\widehat{AOC},\widehat{BOD}$, as well as the angles $\widehat{COB},\widehat{DOA}$ whose sides are opposite, are known as "\NewTerm{opposing angles by the top}\index{opposing angles by the top}".
	\begin{figure}[H]
		\centering
		\includegraphics{img/geometry/opposite_angles.jpg}
	\end{figure}
	Obviously, if $\theta$ is the measurement of the angle $\widehat{AOC}$, the measurement of the adjacent angle $\widehat{COB}$ is $\pi-\theta$. That of the angle $\widehat{DOB}$ adjacent to the previous is $\pi-(\pi-\theta)$, that of $\widehat{DOA}$ is $\pi-\theta$.
	
	Given $OE$ the half-bisector of the angle $\widehat{AOC}$, $OG$ that of the angle $\widehat{COB}$, $OH$ that of the angle $\widehat{BOD}$, $OH$ that of the angle $\widehat{DOA}$, we have:
	
	and consequently:
	
	We would have also same:
	
	We summarize as following all these results as well as measurement properties:
	\begin{enumerate}
		\item[P1.] Two opposite angles that share the same top are equals.
		\item[P2.] Two opposite angles that share the same top have the same bisector.
		\item[P3.] The bisectors of two additional adjacent angles are rectangular bisector.
		\item[P4.] The bisectors of angles formed by two secants are two perpendicular lines (right angles).
	\end{enumerate}
	\begin{figure}[H]
		\centering
		\includegraphics[width=0.8\textwidth]{img/geometry/useful_triangle_angles_in_statics.jpg}
		\caption{Typical useful triangle angles in the study of Statics}
	\end{figure}
	
	\pagebreak
	\subsubsection{Triangles}
	We have study so far until now the concept of dimensions, point, straight segment, straight line, angle and of open (infinite) plan. However, a plane may be defined by several lines to thereby obtain geometric plane shapes  the simplest of which may be regarded as triangles (for many other shapes see the section Geometrical Shapes).

	\textbf{Definition (\#\mydef):} We name "\NewTerm{triangle}\index{triangle}" a figure (see below) formed by three segments (or lines) $AB$, $BC$, $CA$, the points $A$, $B$, $C$ being not aligned. The segments $AB$, $BC$, $CA$, are the "\NewTerm{sides}\index{sides}" of the triangle. The points $A$, $B$, $C$ are the "\NewTerm{vertices}\index{vertices}" of the triangle. The salient angle $\widehat{BAC}$, which contains all the points of $BC$, is named angle $\hat{A}$ of the triangle and $BC$ is then named the "\NewTerm{opposite side}\index{opposite side}".
	
	\begin{tcolorbox}[title=Remark,colframe=black,arc=10pt]
	We use the notation $\hat{A}$ when no confusion is possible. If necessary we use the notation $\widehat{BAC}$ with the same meaning.
	\end{tcolorbox}
	There are $6$ elements in a triangle, namely: three angles $\hat{A}$,  $\hat{B}$,  $\hat{C}$ and three finite sides $\overline{AB}$, $\overline{BC}$, $\overline{CA}$ that are most of time simply denoted by $AB$, $BC$, $CA$.

	We will designate by $a=BC$, $b=AC$, $c=AB$ the measured lengths of the sides with the same unit of measurement, and obviously by $\hat{A}$,  $\hat{B}$,  $\hat{C}$ the measurements of the angles.

	\begin{theorem}
	The sum of the angles of a plane triangle is always equal to $\pi$ radians (or $180^\circ$). The proof is quite simple and is named the "\NewTerm{angle sum theorem}\index{angle sum theorem}\label{angle sum theorem}".
	\end{theorem}
	\begin{dem}
	In the figure below, $ABCD$ is any triangle, and $D$ is the parallel to $\overline{BC}$ passing through $A$. We observe:
	\begin{enumerate}
		\item The blue angles have the same amplitude (value) because they are alternate angles (the segment $\overline{BC}$ and the line $D$ being parallels)

		\item Also, the green angles have the same amplitude (value) because they are alternate angles.

		\item We notice that the sum of the angles blue $+$ red $+$ green are equal to a flat angle on $A$.
	\end{enumerate}
	From the equalities seen in (1) and (2) previously, we deduce that:
	
	\begin{figure}[H]
		\centering
		\includegraphics{img/geometry/triangle_sum_angles.jpg}
		\caption{Sum of angles of a plane triangle}
	\end{figure}
	This proof is valid regardless the type of triangle drawn in the plane.
	\begin{flushright}
		$\blacksquare$  Q.E.D.
	\end{flushright}
	\end{dem}
	As for all other shapes that we will see in this section, the calculations of the perimeter and area (surface) are given in the section of Geometric Shapes page \pageref{known surfaces}.
	
	\paragraph{Equal Triangles (congruent triangles)}\mbox{}\\\\
	\textbf{Definition (\#\mydef):} We say that two triangles are "\NewTerm{equal triangles}\index{equal triangles}" when we can by movement or by a reversal or both combined, superimpose all vertices of the first triangle with that of the second. Then we say also that the triangles are "\NewTerm{homologous triangles}\index{homologous triangles}" or "\NewTerm{congruent triangles}\index{congruent triangles}".

	If a triangle $ABC$ is congruent to a triangle $DEF$, the relation can be written mathematically as:
	

	From this definition, it comes that two triangles are congruent when either the following two properties are satisfied:
	\begin{enumerate}
		\item[P1.] They have one equal side and two equal angles.
		\begin{dem}
		So we state that two triangles have equal side $\overline{BC} = \overline{B'C'}$ between two equal angles $\hat{B}=\hat{B}'$,$\hat{C}=\hat{C}'$ are equal.
		
		In other words, if $2$ triangles have $2$ equal angles $2$ by $2$, then the $3$rd angles are equal too. That said we can now take as equal side the one that is located between the $2$ equal angles without loss of generality.

	Since $\overline{BC} = \overline{B'C'}$ (see figure below), there is a movement which brings $B'$ to $B$ and $C'$ in $C$. This movement brings $A'$ on $A_1$ located on the same side relatively to the segment $\overline{BC}$, or on $A_2$ symmetric of $A_1$ relatively to this same segment.

	The two segments $\overline{BA}$ and  $\overline{BA_1}$ do by hypothesis, with  $\overline{BC}$  the same angle $\hat{B}=\hat{B}'$. As they are by construction on the same side of  $\overline{BC}$ , they are merged. The both segments  $\overline{CA}$ and $\overline{CA_1}$  are merged for the same reason and because  $\hat{C}=\hat{C}'$. $A_1$ is then merged with $A$. The two triangles $ABC$, $A'B'C'$ are then indeed congruent.
		\begin{figure}[H]
			\centering
			\includegraphics{img/geometry/congruent_triangles_one_side_equal_two_angles_equal.jpg}
			\caption{Two triangles having equal side and two equal angles}
		\end{figure}
		\begin{flushright}
			$\blacksquare$  Q.E.D.
		\end{flushright}
		\end{dem}
		This result is named sometimes the "\NewTerm{angle-side-angle theorem}\index{angle-side-angle theorem}".

		\item[P2.] They have a same angle between two sides of equal length.
		\begin{dem}
		So we state that two triangles with an equal angle $\hat{A}=\hat{A}'$ between two equal sides $\overline{AB} = \overline{A'B '}$ and $\overline{AC}= \overline{A'C'}$ are equal.
		
		Since $AB = A'B'$ (see figure below) there exist a shift located relatively to $\overline{AB}$ at the same side as the point. If it brought it on $C_2$ symmetric of $C_1$ with respect to $\overline{AB}$, a half turn around the $\overline{AB}$ would take it on $C_1$. The segments $\overline{AC}$ and $\overline{AC_2}$ located on the same side of $\overline{AB}$ do, by hypothesis the same angle with $\overline{AB}$, since $\hat{A}=\hat{A}'$. They are then merged. The hypothesis $\overline{AC}=\overline{AC_1}$ then make us conclude that $C_1$ and $C$ coincide. The two triangles $ABC$, $A'B'C'$ are then congruent.
		\begin{figure}[H]
			\centering
			\includegraphics{img/geometry/congruent_triangles_one_equal_angle_two_equal_sides.jpg}
			\caption{Two triangles having equal angles and two equal sides}
		\end{figure}
		\begin{flushright}
			$\blacksquare$  Q.E.D.
		\end{flushright}
		\end{dem}
	\end{enumerate}
	\begin{corollary}
	As corollary we have that two triangles are congruent if their three sides are equal. This is named sometimes the "\NewTerm{SSS (side-side-side) theorem}\index{side-side-side theorem}".
	\end{corollary}
	
	Now about drawing symbols, it is common to represent various equal angles or equal sides in triangle as following formation information:
	\begin{figure}[H]
		\centering
		\includegraphics{img/geometry/congruent_triangle_1.jpg}
		\caption{First typical notation for congruent triangles}
	\end{figure}
	\begin{figure}[H]
		\centering
		\includegraphics{img/geometry/congruent_triangle_2.jpg}
		\caption{Second typical notation for congruent triangles}
	\end{figure}
	
	\pagebreak
	\paragraph{Isosceles Triangles}\mbox{}\\\\
	\textbf{Definition (\#\mydef):} We say that $ABC$ is an "\NewTerm{isosceles triangle}\index{isosceles triangle}" when two of its sides $AB$ and $AC$ are equal ("iso" meaning "same"). The third side $\overline{BC}$ is then named the "\NewTerm{base}\index{base (triangle)}" of the triangle.
	
	\begin{tcolorbox}[title=Remark,colframe=black,arc=10pt]
	We say that a triangle is "\NewTerm{scalene}\index{scalene triangle}" when it has three unequal sides.
	\end{tcolorbox}
	\textbf{Definition (\#\mydef):} We name "\NewTerm{mediator}\index{mediator of a triangle}\label{mediator}" of a segment $\overline{BC}$, the perpendicular to $\overline{BC}$ at point $H$ of this segment, middle of $\overline{BC}$.
	\begin{figure}[H]
		\centering
		\includegraphics{img/geometry/mediator_isosceles_triangles.jpg}
		\caption{Representation of the mediator in isosceles triangle}
	\end{figure}
	\begin{theorem}
	In an isosceles triangle $ABC$ as shown above, the angles $\hat{B}$ and $\hat{B}$ opposed to the equal sides are equal.
	\end{theorem}
	
	\begin{dem}
	The both (right) triangles $BAH$ and $CAH$ defined by the bisector of  $\widehat{A}$ have an equal angle $\widehat{BAH}=\widehat{CAH}$ by construction between two equal sides: $\overline{AH}$ that is the shared (common) side and $\overline{AB} = \overline{AC}$ by hypothesis. As the angles $\widehat{BHA}=\widehat{CHA}$ and rights and equals and that the sum of angles of a triangle is equal to a straight angle ($180^\circ$ or $\pi$ radians), then the angles $\widehat{HBA}$ and $\widehat{ACH}$ are equal.
	\begin{flushright}
		$\blacksquare$  Q.E.D.
	\end{flushright}
	\end{dem}
	\begin{tcolorbox}[title=Remark,colframe=black,arc=10pt]
	This latest proof is one to which refers the comic Logicomix on page $55$ and it is this type of proof by logical induction which would have brought the young Bertrand Russell to became interested in logic to later become one of the most famous logicians in modern history.
	\end{tcolorbox}
	
	\pagebreak
	\begin{theorem}
	In an isosceles triangle $ABC$ as shown above the mediator of $\overline{BC}$ and the bisector of the angle $\hat{A}$ coincide (figure below).
	\begin{figure}[H]
		\centering
		\includegraphics{img/geometry/locus_mediator_bisector.jpg}
	\end{figure}
	In other word, the locus of equidistant points $M$ (at same distance) between of two given points $B$ and $C$ is given by the bisector $(D)$ of segment $\overline{BC}$.
	\begin{tcolorbox}[title=Remark,colframe=black,arc=10pt]
	In geometry, a "\NewTerm{locus}\index{locus}" is a set of points (commonly, a line, a line segment, a curve or a surface), whose location satisfies or is determined by one or more specified conditions. More formally we name "locus" of a point $M$, subject to some constraints (conditions), all positions occupied by this point $M$ that still satisfy the constraints (conditions).
	\end{tcolorbox}
	\end{theorem}
	\begin{dem}
	We will prove first that any point of the locus is on the line $(D)$ and afterwards that any point of $(D)$ is a point of the locus.
	\begin{enumerate}
		\item Any point of the locus is on the line $(D)$:

		In other words, the assumption, in the figure above, the assumption that $\overline{MB} =\overline{MC}$ implies that $M$ is the mediator of $\overline{BC}$. Indeed, if $\overline{MB} =\overline{MC}$, the triangle $MBC$ is isosceles and the top $M$ is on the mediator of $BC$ (that coincides as we know with the bisector).
			
		\item Any point of $(D)$ is a point of the locus:
		
		This means that if $M$ is on the mediator of $\overline{BC}$, we have $\overline{MB} = \overline{MC}$. Indeed, if $M$ is on the line $(D)$ that meets in $H$, midpoint of $\overline{BC}$, the line $\overline{B}$, the triangles $MHB$, $MHC$ are then equal (second case of equality: $\widehat{BHM}=\widehat{CHM}$ because these angles are straight, $\overline{HM}$ common; $\overline{HB} = \overline{HC}$ because $H$ is the midpoint of $\overline{BC}$): the sides $\overline{MB}$, $\overline{MC}$ are then also equal and we have indeed $\overline{MB} = \overline{MC}$. The point $M$ is a point of the locus.

	\end{enumerate}
	\begin{flushright}
		$\blacksquare$  Q.E.D.
	\end{flushright}
	\end{dem}
	Using this theorem we can state another one:
	\begin{theorem}
	By taking a point $A$ outside a line $\overline{BC}$, we can draw to this line a single perpendicular (or in other words: to a given point outside of a given line we can draw one and only one perpendicular line).

	This is an equivalent to the Euclid's fifth postulate and a way to build an isoceles triangle as the apex is on the perpendicular of the base.
	\end{theorem}
	\begin{dem}
	The proof is done in two steps: 
	\begin{enumerate}
		\item There exist a perpendicular line:
		
		Indeed, given a triangle $ABC$, let us subjecting this triangle to half a turn around $BC$ (horizontal symmetry) as in the figure below:
		\begin{figure}[H]
			\centering
			\includegraphics{img/geometry/isoceles_triangle_simetry.jpg}
		\end{figure}
		The apex $A$ comes in $A'$ symmetrical by definition, relative to $\overline{BC}$. Since the figures, $ABC$ and $A'BC$ are equal, $\overline{AB}=\overline{A'B}$ and $\overline{AC} = \overline{A'C}$. 

		$\overline{BC}$ is then the mediator of $\overline{AA '}$ and the lines $\overline{BC}$ and $\overline{AA'}$ are perpendicular. $\overline{AA'}$ is therefore well perpendicular to $\overline{BC}$ passing through $A$.
		
		\item The perpendicular line is unique:
		
		Given $\overline{AH}$s a perpendicular drawn from $A$ to $\overline{BC}$, it meets the line $\overline{BC}$ on a point $H$ that is different from either $B$ or $C$.

		Let us Suppose that $H$ is different from $B$. The angles $\widehat{AHB}$, $A'HB$ which are deduced from each other by rotation, are equal, and as each of them is a right angle, the angle $\widehat{AHA'}$ is a flat angle. The line $\overline{AH}$ thus coincides with the straight line $\overline{AA'}$ and is therefore the only perpendicular to $\overline{BC}$ which passes through the apex $A$.

	\end{enumerate}
	\begin{flushright}
		$\blacksquare$  Q.E.D.
	\end{flushright}
	\end{dem}
	\textbf{Definitions (\#\mydef):}
	\begin{enumerate}
		\item[D1.] We name "\NewTerm{orthogonal projection}\index{orthogonal projection}" from point $A$ on a line $\overline{BC}$ the point $H$ point where the perpendicular drawn from $A$ at this line meet $H$. The  point $H$ is also named the "\NewTerm{foot}\index{foot of a perpendicular}" of the perpendicular.

		We have already see the vector version of this projection in the section of Vector Calculus and that it was denoted by $\text{proj}_{\overline{BC} A}$

		\item[D2.] We name "\NewTerm{geometrical distance}\index{geometrical distance}" from $A$ to $\overline{BC}$ the length of the segment $\overline{AH}$.
	\end{enumerate}
	Since in the previous figure $\overline{BC}$ is the mediator of $\overline{AA'}$, $H$ is the midpoint of $\overline{AA'}$. So: A point $A$ and its symmetrical point $A'$ are with respect to a straight line $(D)$ are characterized by the following two properties that we will not prove as we consider them sufficiently intuitive:
	\begin{enumerate}
		\item[P1.] $\overline{AA'}$ is perpendicular to $(D)$
		
		\item[P2.]  The middle of $\overline{AA'}$ is on $(D)$
	\end{enumerate}
	The line $\overline{AB}$, which connects the apex $A$ to a point of the straight line $\overline{BC}$, other than the foot $H$ of the perpendicular drawn from $A$ to this line, is named an "\NewTerm{oblique line}\index{oblique line}". The point $B$ being named the "\NewTerm{oblique foot}\index{oblique foot}".
	
	\pagebreak
	\paragraph{Equilateral Triangles}\mbox{}\\\\
	\textbf{Definition (\#\mydef):} As triangle $ABC$ is say to be an "\NewTerm{equilateral triangle}\index{equilateral triangle}\label{equilateral triangle}" when all its three sides are equal. In the familiar Euclidean geometry, equilateral triangles are also equiangular; that is, all three internal angles $\hat{A},\hat{B},\hat{C}$ are also congruent to each other and are each equal to $\pi/3$ ($60^\circ$). They are regular polygons (see definition further below), and can therefore also be referred to as "\NewTerm{regular triangles}\index{regular triangle}":
	\begin{figure}[H]
		\centering
		\includegraphics{img/geometry/equilateral_triangle_definition.jpg}
		\caption{Equilateral Triangle}
	\end{figure}
	As for all other shapes that we will see in this section the calculation of the perimeter and area (surface) are given in the section of Geometric Shapes.
	
	\paragraph{Right Triangle}\mbox{}\\\\
	\textbf{Definition (\#\mydef):} A "\NewTerm{right triangle}\index{right triangle}" (American English) or "\NewTerm{right-angled triangle}\index{right-angle triangle}" (British English) is a triangle in which one angle is a right angle (that is, a $90^\circ$ angle or $\pi/2$ [rad]). The relation between the sides and angles of a right triangle is the basis for trigonometry.

	The side opposite the right angle is named the "\NewTerm{hypotenuse}\index{hypotenuse}" (side $c$ on the figure). The sides adjacent to the right angle are named "\NewTerm{legs}\index{legs}". Side $a$ may be identified as "the side adjacent to angle $\hat{B}$" and "opposed to (or opposite) angle $\hat{C}$", while side $b$ is the side adjacent to angle $\hat{C}$ and opposed to angle $\hat{B}$.
	\begin{figure}[H]
		\centering
		\includegraphics{img/geometry/right_angle_triangle_shape.jpg}
		\caption{Right triangle}
	\end{figure}
	To say that the triangle is right at $A$ means it is in $\hat{A}$ that is the right angle.
	\begin{tcolorbox}[title=Remark,colframe=black,arc=10pt]
	In a right triangle, the longest side is always the side opposite the right angle. We will prove this property with the Pythagorean theorem (see further below).
	\end{tcolorbox}
	
	\paragraph{Right Isosceles Triangle}\mbox{}\\\\
	\textbf{Definition (\#\mydef):} A "\NewTerm{right isosceles triangle}\index{right isosceles triangle}" $ABC$ is both right and isosceles, meaning that it has both a right angle and two sides of equal length.
	\begin{figure}[H]
		\centering
		\includegraphics{img/geometry/right_isoceles_triangle.jpg}
		\caption{Right isosceles triangle}
	\end{figure}
	The main vertex (apex) corresponds to the right angle ($A$). Indeed, as $\overline{BC}$, the hypotenuse must be the greatest side, the sides $\overline{AB}$ and $\overline{AC}$ have the same length (smaller)!
	
	So far we have for summary:
	\begin{figure}[H]
		\centering
		\includegraphics{img/geometry/type_of_triangles.jpg}
	\end{figure}
	
	\pagebreak
	\paragraph{Inequalities in the triangles}\mbox{}\\\\
	Let us now see some interesting inequalities (properties) in the triangle.
	\begin{enumerate}
		\item[P1.] First let us prove that in any triangle, a side opposite to a right or obtuse angle (greater than $90^\circ$ or $\pi/2$ [rad]) is greater than each of the other two sides of the triangle.

		\begin{dem}
		Consider the triangle $ABC$ below where $\hat{A}\geq \pi/2$ and $Cx$ the half-line extension of $\overline{BC}$. Let us draw, on the half-line $Bx$, a length $\overline{BD}=\overline{BA}$ to construct an isosceles triangle in the original triangle:
		\begin{figure}[H]
			\centering
			\includegraphics{img/geometry/property1_triangle.jpg}
		\end{figure}
		Therefore the triangle $BAD$ is an isosceles triangle whose angle at the base $BAD$ is obviously acute (less than $\pi/2$ or less than $90^\circ$).
		
		So by construction $\widehat{BAD}<\widehat{BAC}$. The straight line $\overline{AD}$ is interior by construction the angle $\widehat{BAC}$ and the it follows:
		
		and as $\overline{BD}=\overline{BA}$ by construction we have:
		
		Which finites our proof. As the reasoning is the same to prove that $\overline{BC}>\overline{CA}$.
		\begin{flushright}
			$\blacksquare$  Q.E.D.
		\end{flushright}
		\end{dem}

		\item[P2.] In any triangle whose sides have strictly increasing lengths, one side is always less than the sum of the other two.
		\begin{figure}[H]
			\centering
			\includegraphics{img/geometry/property2_triangle.jpg}
		\end{figure}
		\begin{dem}
		Let $D$ be the point of the side $\overline{BC}$ such that $\overline{BD}=\overline{BA}=c$ are the sides of the isosceles triangle $ABD$. We get:
		
		The triangle $ABD$ being isosceles, the angle at the base $\widehat{ADB}$ is acute and its complement $\widehat{ADC}$ is obtuse. In the ADC triangle, we get from the preceding equation P1 property that is to say:
		
		where:
		
		this is the famous "\NewTerm{triangle inequality}\index{triangle inequality}" in it geometrical version. We will see it again in many other sections of this book in spaces and more abstract mathematical concepts.
	
		The property is the immediate for other sides $b$ and $c$ by permutation of the method:
		
		\begin{flushright}
			$\blacksquare$  Q.E.D.
		\end{flushright}
		\end{dem}


		\item[P3.] In any triangle any side is greater than the difference of the other two.
		\begin{dem}
		Suppose we have $a>b>c$. Therefore we also have:
		
		Subtracting $c$ to both sides of the inequality gives:
		
		The property is the immediate for the other sides $b$ and $c$ by permutation of the method:
		
		And finally as:
		
		for any triangle with increasing sides we have:
		
		\begin{flushright}
			$\blacksquare$  Q.E.D.
		\end{flushright}
		\end{dem}
	\end{enumerate}
	
	\paragraph{Triangles remarkable interior lines}\label{triangles remarkable interior lines}\mbox{}\\\\
	A little summary about triangles remarkable interior lines is probably necessary at this point (before continuing on other triangles theorems) to avoid if possible to much confusion as we have given a lot of definitions so far:
	\begin{itemize}
		\item The "\NewTerm{perpendicular bisector}\index{perpendicular bisector}" of a segment is the line which is perpendicular to the segment and which passes through its center.

		\item The "\NewTerm{height}\index{height}" or "\NewTerm{altitude}\index{altitude}"  of a triangle is a line which passes through a vertex of the triangle and which intercept the opposite edge with right angle.

		\item The "\NewTerm{angle bisector}\index{angle bisector}" of an angle is the semi-straight line that divides the angle into two equal angles.

		\item The "\NewTerm{median}\index{median}" of triangle passes through a vertex and intercepts the opposite edge at its center.

		\item The "\NewTerm{mediator}\index{mediator}" is the perpendicular line to the crossing point of the median.
	\end{itemize}
	\begin{figure}[H]
		\centering
		\includegraphics{img/geometry/bissectors_median_mediator_height.jpg}
		\caption{Bisectors, Median, Mediator, Height in a triangle}
	\end{figure}
	Obviously in some geometries all a part of all the lines above are not discernible.
	In a triangle:
	\begin{itemize}
		\item The intersection of $3$ altitudes is named the "\NewTerm{orthocenter}\index{orthocenter}".
		\item The intersection of $3$ medians is named the "\NewTerm{centroid}\index{centroid}".
		\item The intersection of $3$ angle bisectors is the center of the inscribed circle.
		\item The intersection of $3$ perpendicular bisectors is the center of the circumscribed circle.
	\end{itemize}
	
	\pagebreak
	\paragraph{Pythagorean theorem}\label{pythagorean theorem}\mbox{}\\\\
	Now that we have see what was a triangle and reviewed some of its properties as well as the concept angle, we can prove the famous "Pythagorean theorem" that gives the relation that must satisfy three numbers that represent the sides of a right triangle in Euclidean Geometry and also permits to make circle trigonometry (\SeeChapter{see section Trigonometry page \pageref{circle trigonometry}}).
	
	\begin{theorem}
	The square of the hypotenuse (the side opposite the right angle) is equal to the sum of the squares of the other two sides
	\end{theorem}
	The theorem has been given numerous proofs - possibly the most for any mathematical theorem. They are very diverse, including both geometric proofs and algebraic proofs, with some dating back thousands of years. The theorem can be generalized in various ways, including higher-dimensional spaces, to spaces that are not Euclidean, to objects that are not right triangles, and indeed, to objects that are not triangles at all, but $n$-dimensional solids.
	
	Among all possible proof we chose to present the next one as it seemed to most easy one to us for readers (student mainly in this case):
	\begin{dem}
	Given a square (four right angles polygon) inside which is inscribed another square of smaller sides. We can calculate the surface of the inscribed square thanks to the right triangles resulting from the empty spaces between the two squares such as presented below:
	\begin{figure}[H]
		\centering
		\includegraphics{img/geometry/pythagorean_theorem.jpg}
	\end{figure}
	The surface of the white square is obviously:
	
	and that of the gray square:
	
	To get the surface of the gray square we can subtract from the with square the surface of the four right triangles having each for surface (\SeeChapter{Geometric Shapes}):
	
	The surface of the gray square is finally:
	
	We then get the final result of the famous "\NewTerm{Pythagorean theorem}\index{Pythagorean theorem}"
	
	In a right triangle, the square of the hypotenuse (side opposite the right angle) is equal to the sum of the squares of sides of the right angle.
	\begin{flushright}
		$\blacksquare$  Q.E.D.
	\end{flushright}
	\end{dem}
	In the particular case we have three integers $a$, $b$ and $c$ that satisfy the Pythagorean theorem (there are infinite combinations of integers satisfying the Pythagorean theorem), then we speak of "\NewTerm{Pythagorean triple}\index{Pythagorean triple}".
	
	\begin{tcolorbox}[title=Remark,colframe=black,arc=10pt]
	We own this possible proof to the Chinese Chao King (2nd century).
	\end{tcolorbox}
	It is often mentioned in smaller class of the reciprocal of Pythagorean theorem which states: In a triangle, if the square of one side is equal to the sum of the squares of the other two sides, then the triangle is rectangle and the hypotenuse will be the longest side of the triangle.
	
	\paragraph{Thales' Theorem (intercept theorem)}\mbox{}\\\\
	The "\NewTerm{intercept theorem}\index{intercept theorem}", also known as "\NewTerm{Thales' theorem}\index{Thales' theorem}\label{thales theorem}" (not to be confused with another theorem with that name), is an important theorem in elementary geometry about the ratios of various line segments that are created if two intersecting lines are intercepted by a pair of parallels.
	
	Having just proved the Pythagorean theorem and that now the concepts of parallels, segments, angles and others are quite well known to us, we can finally prove the Thales' theorem which is given here a possible proof that first requires the development of two lemmas:
	\begin{lemma}
	 Triangles with same surfaces: If two triangles have a common side and if the third vertices are on a parallel to this common side, then they have the same surface!
	\end{lemma}
	\begin{dem}
	Given the figure:
	\begin{figure}[H]
		\centering
		\includegraphics{img/geometry/thales_theorem_lemma_1.jpg}
		\caption[]{First construction for Thales' theorem proof}
	\end{figure}
	We have:
	
	$EFGH$ is a rectangle because its sides are parallel in pairs and have at least two right angles. So its opposite sides have the same length: $\overline{EH} = \overline{FG}$.
	
	$\overline{EH}$ is the relative height to $\overline{AB}$ in the triangle $EAB$ and $\overline{FG}$ is the relative height $\overline{AB}$ in the triangle $FAB$.
	
	The surface of the triangle depends only on the length of the side and the length of the height relative this side (\SeeChapter{see section Geometric Shapes page \pageref{unspecified triangle}}). For the both triangles $EAB$ and $FAB$, these lengths are equal, so they have the same surface!
	
	So we have indeed that of two triangles have a common side and if the third vertices are on a parallel to this common side, then they have the same surface!
	\begin{flushright}
		$\blacksquare$  Q.E.D.
	\end{flushright}
	\end{dem}
	
	\begin{lemma}
	We have:
	
	\end{lemma}
	\begin{dem}
	Given the proportions ratio ("proportional calculus" or "product in cross"):
	
	then:
	
	If $ad=bc$, then:
	
	where we added as same positive or negative number to the two members of the equality. Thus after factorization:
	
	and by applying contreversly the product in cross:
	
	\begin{flushright}
		$\blacksquare$  Q.E.D.
	\end{flushright}
	\end{dem}
	Let us now expose what is the Thales theorem.
	\begin{theorem}
	 If two intersecting lines are cut by parallel lines, the line segments cut by the parallel lines from one of the lines are proportional to the corresponding line segments cut by them from the other line and from this we can also deduce also other remarkable identities.
	\end{theorem}
	\begin{dem}
	Given the figure:
	\begin{figure}[H]
		\centering
		\includegraphics{img/geometry/thales_theorem_first_approach.jpg}
		\caption{First approach of Thales' theorem}
	\end{figure}
	With:
	
	We have proven previously that if two triangles have a common side and if the third vertices are on a parallel, then they have the same surface. So the triangles $\Delta ACD$ and $\Delta BCD$ have the same surface!

	Adding to each of these two surfaces that of the triangle $\Delta OCD$, we get that the triangles $\Delta ODA$ and $\Delta OCB$ have the same surface!

	We deduce then that using the product in cross that:
	
	Given $h_1$ the height issue from $D$ in the triangle $\Delta OCD$ and $h_2$ the height issue from $C$ in the triangle $\Delta OCD$:
	
	Conclusion:
	
	Given now the figure:
	\begin{figure}[H]
		\centering
		\includegraphics{img/geometry/thales_theorem_second_approach.jpg}
		\caption{Second approach of Thales' theorem}
	\end{figure}
	The triangles $\Delta IJD$ and $\Delta IDB$ have the same surface from our first lemma, also same for the triangles $\Delta OJD$ and $\Delta OIB$ therefore:
	
	hence:
	
	therefore:
	
	In the same way, in the triangles $OIA$ and $OCJ$, we get:
	
	From the lemma 2, as:
	
	Then:
	
	By applying to the first triangle $OJB$ what we did during the first approach, we also get:
	
	So finally by taking all previous results we get:
	
	which constitutes the "\NewTerm{Thales' theorem}" of ratios.
	\begin{flushright}
		$\blacksquare$  Q.E.D.
	\end{flushright}
	\end{dem}
	
	Here is an excellent summary of what we have seen so far about Thales' theorem:
	\begin{figure}[H]
		\centering
		\includegraphics{img/geometry/thales_summary.jpg}
		\caption[Thales' theorem]{Thales' theorem (source: Wikipedia)}
	\end{figure}
	
	\pagebreak
	\subsubsection{Parallelism}
	\textbf{Definition (\#\mydef):} We name "\NewTerm{parallels}\index{parallels}" two line (non-necessarily straight one!) equally distant from one another over their entire length.

	This concept is directly connected to Euclid's fifth postulate (parallel postulate) and is often considered as the most important being been proved that it can not be considered as an axiom as it is not respected in non-Euclidean geometries (\SeeChapter{see section Non-Euclidean Geometry page \pageref{non-euclidean geometry}}).
	
	The consequences of this postulate 	are as we know the following in a Euclidean geometry:
	\begin{enumerate}
		\item If two lines $AB$ and $CD$ are parallels, any straight line that cut one, cut the other one.
		\begin{dem}
		Given $F$ the common point to the straight line $CD$ and the straight line $E'F'$: if the straight line $E'F'$ did not cut the straight line $AB$, it would be parallel to it, and through the point $F$ would pass two straight lines $CD$ and $E'F'$ parallel to a third one $AB$, which is not the case. So the straight line $E'F'$, cuts the line $AB$.
		\begin{flushright}
			$\blacksquare$  Q.E.D.
		\end{flushright}
		\end{dem}

		\item Two straight lines $AB$ and $CD$ parallels to a third line $E'F'$ are parallel between them.
		\begin{dem}
		If the straight line $CD$ was not parallel to the straight line $AB$ it would cut (cross) it. It would also cut (cross) the straight line $E'F '$ parallel to the straight line $AB$, it would therefore not be parallel to $E 'F'$.
		\begin{flushright}
			$\blacksquare$  Q.E.D.
		\end{flushright}
		\end{dem}
	\end{enumerate}
	\begin{theorem}
	If two parallels lines are cut by a secant then:
	\begin{enumerate}
		\item The intern-alternate angles are equal;
		\item The external-alternate angles are equal;
		\item The corresponding angles are equal.
	\end{enumerate}
	\end{theorem}
	\begin{dem}
	Consider two (coplanar) parallel $AB$ and $CD$ and a secant $EF$:
	\begin{figure}[H]
		\centering
		\includegraphics{img/geometry/two_parallels_with_a_secant.jpg}
	\end{figure}
	\begin{enumerate}
		\item By the middle O of $\overline{EF}$ let us conduct the perpendicular $\overline{GH}$ to $\overline{AB}$, which is also perpendicular to $\overline{CD}$. The right triangles $E$O$G$ and $F$O$H$ an acute angle equal to $\widehat{G\text{O}E}=\widehat{F\text{O}H}$ and an equal hypotenuse: O$F = $O$E$. They are equal, and the angles $\widehat{\text{O}FH}$ and $\widehat{GE\text{O}}$ are equal.

		\item The alternate exterior angles $\widehat{E'EB}$ and $\widehat{F'FC}$ are equal, because $\widehat{E'EB}$ is opposed by the apex at the angle $\widehat{GE\text{O}}$, which an intern-alternate with the angle $\widehat{\text{O}FH}$.
	\end{enumerate}
	\begin{flushright}
		$\blacksquare$  Q.E.D.
	\end{flushright}
	\end{dem}
	... and vice versa if two straight lines are cut by a secant forming with these straight lines:
	\begin{enumerate}
		\item Two intern-alternate equal angles;
		\item Two external-alternate equal angles;
		\item Two corresponding equal angles.
	\end{enumerate} 
	\begin{tcolorbox}[title=Remark,colframe=black,arc=10pt]
	To prove the parallelism of two straight lines, it is necessary and sufficient that the intern-alternate angles, extern-alternate angles or corresponding angles, formed by these two straight lines with secant transversal, are all equal. That is to say equal to $\pi/2$.
	\end{tcolorbox}
	
	\pagebreak
	\subsubsection{Circle}\label{circle}
	\textbf{Definition (\#\mydef):} We name "\NewTerm{circle}\index{circle}" the locus of points $M$ of the plane that are a given distance $R$, named "\NewTerm{radius}\index{radius}" of the circle, of a fixed point O, named "\NewTerm{center of the circle}\index{center of the circle}". Or using the concept of "locus" defined earlier:  a circle is defined as the locus of a point that is at a given distance of a fixed point, the center of the circle:
	
	\begin{figure}[H]
		\centering
		\includegraphics{img/geometry/circle_definition.jpg}
		\caption{Circle and Radius}
	\end{figure}
	\begin{tcolorbox}[title=Remark,colframe=black,arc=10pt]
	The word "radius" means either the segment $\overline{\text{O}M}$, or its measurement $R$.
	\end{tcolorbox}
	Circles are directly concerned by the third Euclid that we stated earlier above.

	We name "\NewTerm{diameter}\index{diameter}", denoted $\varnothing$, of a circle every line passing through the center O of the circle. Any diameter meets the circle at two points $A$ and $B$, such that by construction:
	
	which we name "\NewTerm{extremities of the diameter}\index{extremities of the diameter}". We reserve the expression "\NewTerm{diameter $\overline{AB}$}" for the diameter of extremities $A$ and $B$. We say that two points of a circle are "\NewTerm{diametrically opposed}\index{diametrically opposed points}" when they are the two ends of the same diameter:
	\begin{figure}[H]
		\centering
		\includegraphics{img/geometry/circle_diameter.jpg}
		\caption{Circle and Diameter}
	\end{figure}

	A circle divides the plane into two parts: one which contains the center, which we name "\NewTerm{inner region}\index{inner region}" and one that does not contain it, which we call "\NewTerm{outer region}\index{outer region}":
	\begin{figure}[H]
		\centering
		\includegraphics{img/geometry/circle_inner_outer.jpg}
	\end{figure}
	
	\begin{theorem}
	The necessary and sufficient condition for a point $P$ to be strictly inside a circle $(C)$, of center O and radius $R$, is .
	\end{theorem}
	\begin{dem}
	There are two case to consider for the proof:
	\begin{enumerate}
		\item The condition is necessary: If, by hypothesis, $P$ is inside the circle $(C)$, it is located either on O, or between the ends $A$ and $B$ of the diameter defined by the locus of the points $M$. If it is on O the proposition is obvious, if it is not on O, then it is between O and $A$ for example, and we have $\overline{\text{O}P}<\overline{\text{O}A}$, that is to say equation.

		\item The condition is sufficient: If, by hypothesis $\overline{OP}<R$, $P$ is between the extremities $A$ and $B$ of the loci defined by the points $M$, so inside the circle $(C)$.
	\end{enumerate}
	\begin{flushright}
		$\blacksquare$  Q.E.D.
	\end{flushright}
	\end{dem}
	\begin{corollary}
	The necessary and sufficient condition for a point $P$ to be outside a circle $(C)$ is $\overline{\text{O}P}>R$
	\end{corollary}
	
	We name "\NewTerm{string of a circle}\index{string of a circle}" $C$ the segment  $\overline{CD}$ whose extremities $C$ and $D$ are on the circle (on the loci of $M$) as visible in the figure below.

	\begin{theorem}
	The mediator of a string $\overline{CD}$ is a diameter and obviously $C\text{O}D$ is an isosceles triangle.
	\end{theorem}
	\begin{dem}
	The mediator $\Delta$ of $\overline{CD}$ (see figure below), string of the circle $(C)$ of center O and radius $R$ pass through the point O because as proved it during our study of the triangles we have $\overline{\text{O}R}=\overline{\text{O}D}=R$.
	\begin{figure}[H]
		\centering
		\includegraphics{img/geometry/circle_string.jpg}
		\caption[]{Mediator of circle string}
	\end{figure}
	\begin{flushright}
		$\blacksquare$  Q.E.D.
	\end{flushright}
	\end{dem}
	\begin{corollary}
	The perpendicular drawn from the center O of a circle of any circle string pass by construction through the middle $H$ of this string.
	\end{corollary}
	
	\pagebreak
	\paragraph{Circumscribed circle theorem}\mbox{}\\\\
	\begin{theorem}
	By three points non-aligned $A$, $B$, $C$ we can draw a circle and only one. This is the "\NewTerm{circumscribed circle theorem}\index{circumscribed circle theorem}".
	
	Or in other words: All triangles are cyclic, i.e. every triangle has a circumscribed circle or "\NewTerm{circumcircle}\index{circumcircle}".
	\end{theorem}
	\begin{dem}
	 First, we draw the mediator $D$ of  the segment  $\overline{AB} $ and the mediator $\Delta$ of  the segment $\overline{AC}$. If $(D)$ and $\Delta$ were parallel, the perpendicular $\overline{AB}$ to $D$ would be perpendicular also to $\Delta$ so coinciding with $\overline{AC}$. The points $A$, $B$, $C$ would then be aligned. Therefore $D$ and $\Delta$ are non-parallel and intersect at a point O:
	\begin{figure}[H]
		\centering
		\includegraphics{img/geometry/circumscribed_circle_theorem.jpg}
	\end{figure}
	\begin{enumerate}
		\item There is one circle going through $A$, $B$, $C$:

		The point O being on $D$, mediator of $\overline{AB}$, $\overline{\text{O}A} = \overline{\text{O}B}$ by construction:  the point O being on $D$, mediator of $\overline{AC}$, $\overline{\text{O}A} = \overline{\text{O}C}$. The circle $(C)$, of center O and of radius $\overline{\text{O}A}$ passes through $B$ (since  $\overline{\text{O}A} = \overline{\text{O}B}$) and also by $C$ (since $\overline{\text{O}A} = \overline{\text{O}C}$). It therefore passes through $A$, $B$, $C$.

		\item It goes through the points $A$, $B$, $C$ only one circle: 

		If it was going through $A$, $B$, $C$ a different circle than that $(C)$ of center O and radius $\overline{\text{O}A}$, its center O' would be on the mediator of $\overline{AB}$ and $\overline{AC}$ that are two strings of the circle; it is then confused with O!
	\end{enumerate}
	\begin{flushright}
		$\blacksquare$  Q.E.D.
	\end{flushright}
	\end{dem}
	\begin{tcolorbox}[title=Remark,colframe=black,arc=10pt]
	The mediator of $\overline{BC}$, string of the circle $(C)$ above, also passes through the point O. We can say (important result!) that the three mediators of the sides of a triangle $ABC$ are concurrent in a same point named the "\NewTerm{orthocenter}\index{orthocenter}".
	\end{tcolorbox}
	We will see further below that using the law of sines it's quite easy to get the diameter of the circumscribed circle and therefore its surface if we know its intern-angles. But if we don't know its intern-angles we need the center angles theorem that we will prove further below to calculate its surface.

	\paragraph{Inscribed circle theorem}\mbox{}\\\\
	\begin{theorem}
	It is the opposite of the previous theorem and states therefore a unique circle can be inscribed in any triangle, i.e. every triangle has an incircle tangent to it.
	\end{theorem}
	\begin{dem}
	Given $ABC$, bisect the angles at the vertices $A$ and $B$.

	These angles bisectors must interact at a point O:
	\begin{figure}[H]
		\centering
		\includegraphics{img/geometry/inscribed_circle_theorem.jpg}
	\end{figure}

	We locate afterwards the point $D$, $E$ and $F$ on sides $\overline{AB}$, $\overline{BC}$ and $\overline{CA}$ respectively so that:
	
	Observe that the triangles $\Delta A\text{O}D$ and $\Delta A\text{O}F$ are congruent and also the triangles $\Delta B\text{O}D$ and $\Delta B\text{O}E$ by the angle-side-angle theorem.
	
	Since corresponding sides of congruent triangles are equal, we also know that:
	
	Hence:
	
	So the circle with center O and radius $R$ is an incircle for the triangle.
	\begin{flushright}
		$\blacksquare$  Q.E.D.
	\end{flushright}
	\end{dem}
	We can be interested to calculate the surface of the incircle of radius $R$ of the triangle $\Delta ABC$ knowing only the distances between the points $A$, $B$ and $C$.
	
	So following the above figure, we consider the triangle $\Delta ABC$ with incircle having center at O, we notice that:
	
	Thus:
	
	That is:
	
	So:
	
	Therefore:
	
	where $P$ is the perimeter of the triangle.
	
	\pagebreak
	\paragraph{Thales' theorem of the circle}\mbox{}\\\\
	\begin{theorem}
	If $A$, $B$ and $C$ are points on a circle where the segment $\overline{AC}$ is a diameter of the circle, then the angle $\widehat{ABC}$ is a right angle. This is the "\NewTerm{Thales' theorem for circle}\index{Thales' theorem for circle}".
	\end{theorem}
	\begin{dem}
	Given the figure:
	\begin{figure}[H]
		\centering
		\includegraphics{img/geometry/thales_circle.jpg}
	\end{figure}	
	Since $\overline{\text{O}A} = \overline{\text{O}B} = \overline{\text{O}C}$, the triangles $\Delta OBA$ and $\Delta OBC$ are therefore isosceles triangles, and by the equality of the base angles of an isosceles triangle, $\widehat{\text{O}BC} = \widehat{\text{O}CB}$ and $\widehat{BA\text{O}} = \widehat{AB\text{O}}$.

	Let $\alpha = \widehat{BA\text{O}}$ and $\beta = \widehat{\text{O}BC}$. The three internal angles of the  triangle $ABC$ are $\alpha$, $(\alpha + \beta)$, and $\beta$. Since the sum of the angles of a triangle is equal to $\pi$ [rad] ($180^\circ$), we have:
	
	\begin{flushright}
		$\blacksquare$  Q.E.D.
	\end{flushright}
	\end{dem}
	\begin{corollary}
	If and only if the inscribed triangle is a right triangle, the circumcenter lies at the center of the hypotenuse.
	\end{corollary}
	
	\pagebreak
	\paragraph{Central angle theorem}\mbox{}\\\\
	The "\NewTerm{central angle theorem}\index{central angle theorem}" is very useful in solving questions that deals with angles within circles as we will see just after. The use of this theorem often simplifies a complicated situation into a rather simple one.
	
	\begin{theorem}
	The central angle theorem states that the central angle from two chosen points $A$ and $B$ on the circle is always twice the inscribed angle from those two points. The inscribed angle can be defined by any point $P$ along the outer arc $AB$ and the two points $A$ and $B$.
	\end{theorem}

	\begin{dem}
	Consider the following figure:
	\begin{figure}[H]
		\centering
		\includegraphics{img/geometry/center_angle_theorem.jpg}
	\end{figure}
	If we denote $\widehat{AP\text{O}}=\alpha$ and $\widehat{BP\text{O}}=\beta$ then it follows as $\Delta P\text{O}A$ and $\Delta P\text{O}B$ are isosceles triangle that:
	
	Using the fact that the sum of the angles of a triangle is equal to $\pi$ (or $180^\circ$) then:
	
	Since all angles at center O must add up to $2\pi$ (or $360^\circ$), we have:
	
	Therefore:
	
	Don't forget that this result is valid whatever the position of $P$ on the circle!!!!!!
	\begin{flushright}
		$\blacksquare$  Q.E.D.
	\end{flushright}
	\end{dem}
	Using this theorem we can now prove that the relation that appears in the law of sines (\SeeChapter{see section Trigonometry page \pageref{law of sines}}) is the diameter of the circumcircle of the triangle $ABC$.

	\begin{theorem}
	
	\end{theorem}
	\begin{dem}
	For the proof consider the following figure:
	\begin{figure}[H]
		\centering
		\includegraphics{img/geometry/law_of_sines_diameter.jpg}
	\end{figure}
	Given the triangle $ABC$, let O denoted the center of its circumcircle. 

	Observe that $\widehat{B\text{O}C}=2\widehat{BAC}$ due to the central angle theorem.

	That is:
	
	Let $M$ be the midpoint of $\overline{BC}$.

	Then the triangle $\Delta B\text{O}M$ is congruent to the triangle $\Delta C\text{O}M$ by the side-side-side theorem.
	
	So we have that:
	
	since congruent angles in congruent triangles are equal.
	
	By construction we have:
	
	Let us now write:
	
	Now in the right triangle $B\text{O}M$ we see that:
	
	Therefore:
	
	Hence, by applying the law of sines, we conclude that:
	
	\begin{flushright}
		$\blacksquare$  Q.E.D.
	\end{flushright}
	\end{dem}
	So we can easily calculate the surface knowing the intern angles. But consider now that we don't know the intern-angles. So we also use the central angle theorem and the following figure:
	\begin{figure}[H]
		\centering
		\includegraphics{img/geometry/circumscribed_circle_radius.jpg}
	\end{figure}
	From triangle $\Delta BD\text{O}$ we have:
	
	As we will prove in the section of Geometric Shapes, the surface of the internal triangle is:
	
	So if we choose for basis of the triangle $b$ and knowing that therefore:
	
	we get:
	
	Therefore:
	
	and we can then easily get the surface of the circumscribed circle from its radius.
	
	\pagebreak
	\subsection{Hilbert's Axioms}\label{hilbert axioms}
	Euclid gathered all the geometric knowledge of his time in the form of the $5$ postulates. He left his name to the Euclidean geometry that uses its fifth postulate, to non-Euclidean geometry that does not use it, and to Euclidean spaces.

	This is postulated base, however, is imperfect, to rigorously prove theorems associated with this geometry, it is necessary to accept as true additional implicit assumptions. David Hilbert built a corresponding axiomatic to the idea that made Euclid  of the geometry by adding ad hoc assumptions (mathematicians make definitions with coffee...).

	The Hilbert's axioms are grouped into five categories: the association, the order, the congruence, the continuity and the parallels.

	Hilbert's axiom system is constructed with $6$ primitive notions: three primitive terms and three primitive relations.

	The three primitive terms are:
	\begin{enumerate}
		\item Point
		\item Line
		\item Plane
	\end{enumerate}
	\begin{tcolorbox}[title=Remark,colframe=black,arc=10pt]
	Points, lines and planes are considered a distinct by default.
	\end{tcolorbox}
	The three primitive relations are:
	\begin{enumerate}
		\item That of association defines the word "\NewTerm{contains}" in geometry! It corresponds to the ideas "is part of" and "is included in" of set theory.

		\item That of "\NewTerm{order}\index{order}" is a binary relation between a couple of points and one point, it appears in the terms "between" and to define the segments.

		\item That of "\NewTerm{congruence}\index{congruence}", which corresponds to the three "equivalence relations" for the couples of points, the triangles and the angles.
	\end{enumerate}

	\pagebreak	
	Here are the "\NewTerm{Hilbert's axioms}\index{Hilbert's axioms}":

	\subsubsection{Incidence Axioms (axioms of association)}
		\begin{enumerate}
			\item[A.A1.] Given two points, there is a line going through these two points (contains them both).
	
			\item[A.A2.] Given two points, there is one line and only  one passing through these two points (verbatim the line described in A.A1).
	
			\item[A.A3.] A line contains at least two points, and for a given line, there exists at least one point not contained into that line.
	
			\item[A.A4.] Given three points that are not contained in a line, there is a plane containing the three points. Any plane contains at least one point.
	
			\item[A.A5.] Given three points that are not contained in a line, there is only one plane containing the three points.
	
			\item[A.A6.] Given two points contained in a line $D$ and in a plane $A$, then $A$ contains all the points of $D$.
	
			\item[A.A7.] If two planes $A$ and $B$ both contain a point $C$, then the intersection of $A$ and $B$ contains at least one other point.
	
			\item[A.A8.] There are at least four non coplanar points.
		\end{enumerate}
		
	\subsubsection{Order Axioms}
	\begin{enumerate}
		\item[A.O1.] If point $B$ is between the points $A$ and $C$, $B$ is then also between the points $C$ and $A$, and there is a line containing the three points $A$, $B$, $C$.

		\item[A.O2.] Given two points $A$ and $C$, there is a point $B$ element of the line $\overline{AC}$ such that $C$ is between $A$ and $B$.

		\item[A.O3.]  Given three points contained in a line, then a one and only is located between the other two.

		\item[A.O4.] ("\NewTerm{Pasch's theorem}\index{Pasch's theorem}") Given three points $A$, $B$, $C$ non-collinear, and given a line $D$ in the plane $ABC$ but containing none of the points $A$, $B$, $C$: If $D$ contains a point of the segment $\overline{AB}$, then $D$ contains either a point of the segment $\overline{AC}$ or a point of the segment $\overline{BC}$.
	\end{enumerate}

	\pagebreak
	\subsubsection{Congruence Axioms}\label{congruence axioms}
	First let us recall that intuitively "\NewTerm{congruent}\index{congruent}" means in geometry "\NewTerm{stackable}\index{stackable}" (exactly the same size and shape) and the opposite of "\NewTerm{similar}" (only one of the both properties are satisfied).
	
	\begin{enumerate}
		\item[A.G1.] Given two points $A$, $B$ and a point $A'$ element of a line $d$, there are two and two only unique points $C$ and $D$, such that $A'$ is between $C$ and $D$, and $\overline{AB}$ is congruent to $\overline{A'C}$ and $\overline{AB}$ is congruent to $\overline{A'D}$.
	
		\item[A.G2.] The congruence relation is transitive, that is to say, if $\overline{AB}$ is congruent to $\overline{CD}$ and if $\overline{CD}$ is congruent to $\overline{EF}$, then $\overline{AB}$ is congruent to $overline{EF}$.	

		\item[A.G3.] Given a straight line $d$ containing the adjacent segments $\overline{AB}$ and $\overline{BC}$, and given a straight line containing the adjacent segments $\overline{A'B'}$ and $\overline{B'C'}$. If $\overline{AB}$ is congruent to $\overline{A'B'}$ and $\overline{BC}$ is congruent to $\overline{B'C'}$ then $\overline{AC}$ is congruent to $\overline{A'C'}$.	

		\item[A.G4.] Given an angle $\widehat{ABC}$ and a half-line $\overline{B'C'}$, there are two and only two half-lines, $\overline{B'D}$ and $\overline{B'E}$, such that the angle $\widehat{DB'C'}$ is congruent to the angle $\widehat{ABC}$ and the angle $\widehat{EB'C'}$ is congruent to the angle $\widehat{ABC}$.
 	
		\item[A.G5.] Given two triangles $ABC$ and $A'B'C'$ such that $\overline{AB}$ is congruent to $\overline{A'B'}$, $\overline{AC}$ is congruent to $\overline{A'C'}$, and the angle $\widehat{BAC}$ is congruent to the angle $\widehat{B'A'C'}$, then the triangle $ABC$ is congruent to the triangle $A'B'C'$.
	\end{enumerate}
	Therefore we see that these axioms are used to compare segments, and also angles, to define the center of a segment, the orthogonal lines, to talk about equilateral triangles, isosceles, etc. They also enable to strictly define the translations which was so often used by Euclid without having been defined.

	\subsubsection{Continuity Axioms}
	\begin{enumerate}
		\item[A.C1.] Also named "\NewTerm{Archimedes' axiom}\index{Archimedes' axiom}", this axiom assume that given two segments $\overline{AB}$ and $\overline{CD}$ then there is always a finite sequence of points $A_1,\ldots, A_n$ belonging to the line containing the segment $\overline{AB}$  and such that $A<A_1<\ldots<A_{n-1}<B<A_n$ that can satisfy $\overline{AA_1}\equiv \overline{AA_2}\equiv\ldots \overline{A_{n-1}A_n}\equiv \overline{CD}$.
	
		\item[A.C2.] Also named "\NewTerm{Cantor's Axiom}\index{Cantor's Axiom}" it assumes if $(A_n)$ and $(B_n)$ are two infinite sequence of points and that we build overlapping line segments $\overline{A_iB_i}$ such as $\overline{A_{k}B_{k}} \subset \overline{A_iB_i}$ if $i<k$ (also sometimes written $]A_{i+1},B_{i+1}[\subset]A_i,B_i[$) located on a line $L$ and such that $\forall \overline{CD},\exists i: \overline{A_iB_i}\equiv\overline{CD}$, then there exists a point $X$ belonging to all segments $\overline{A_iB_i}$. In other words, either a series of nested segments whose length tends to $0$ then there is a point common to all segments.
	\end{enumerate}
	It is not obvious but the Archimedean axiom is the result of the experience in a large number of measurements and guarantees the measurability of line segments. An arbitrary line segment can be attached a real number called the line segment length. The axiom does not guarantee the inverse relation, that an arbitrary real number can be related to some line segment as the length of this line segment. This fact is guaranteed by the Cantor axiom.


	\subsubsection{Parallels Axioms}
	\begin{enumerate}
		\item[A.P1.] Also know as "\NewTerm{Euclidean axiom}\index{Euclidean axiom}" it assumes that given of a straight line $L$ and a point $P$ not belonging to $L$, it passes one and only one straight line $L'$ through $P$ which is parallel to $d$.
		
		Other equivalent formulation is:
	
		\item[A.P1'.] Given a straight line $L$, a point $P$ not included on $L$, then there exists a plane containing $d$ and $P$. This plane contains one and only one straight line containing $P$ and containing no point of $d$.
		
		Any non-empty set of points (in the plane or in the space) consistent to the axiomatic system with the Euclidean postulate forms the Euclidean geometry (Euclidean plane $\mathcal{E}^2$, or Euclidean space $\mathcal{E}^3$) which is called also "\NewTerm{parabolic geometry}\index{parabolic geometry}". One of the most well known theorems following directly from the Euclidean axiom of parallelism is that the sum of the interior angles in an arbitrary triangle equal $\pi$.
	\end{enumerate}
	
	We can't prove until now the non-contradictory logic of all these axioms. However, we know two things if we make an analogy with what we studied in the chapters Arithmetic and Algebra of the books (especially the section on Sets Theory, Functional Analysis and Sequences and Series):
   \begin{enumerate}
      \item If these axioms are contradictory, then the theory of real numbers is also contradictory.

      \item If the system of axioms obtained by removing the axiom of Cantor is contradictory, then the theory of rational numbers is contradictory.
   \end{enumerate}
   Thus, the confidence we have in the strength of these axioms lies on the one we have in the theory of real numbers, which is already also quite big.
   
   \pagebreak
   \subsection{Barycentre (centroid)}\label{barycenter}
   Now that we have discussed the minimum of the construction of Euclid and Hilbert geometry, we can move to a higher level to the analysis of properties of geometric shapes. We will begin by studying the concept of "\NewTerm{center of gravity}", also referred to but rarely to "\NewTerm{centroid}" or "\NewTerm{barycentre}" and that is very important not only in geometry but also in classical mechanics, astronomy and many other domains where multiple masses must be reduced to a single virtual point.
   
   \textbf{Definition (\#\mydef):} We name "\NewTerm{center of gravity}\index{center of gravity}" or "\NewTerm{centroid}\index{centroid}" or "\NewTerm{geometric center}\index{barycentre}" of the points $A_i$ of the plane or of the space affected respectively by the coefficients $\alpha_1,\alpha_2,\ldots,\alpha_n$ (where the $\alpha_i$ are real such that $\sum \alpha_i\neq 0$) the single point $G$ (or also denotes sometimes CM for "Center of Mass") such that:
	
	The couple denoted $(A_i,\alpha_i)$ is named "\NewTerm{ponderated point}\index{ponderated point}" or "\NewTerm{massive point}\index{massive point}" in the context of study of physics and when $\alpha_i>0$ represents a mass. If all the $\alpha_i$ have the value that the centroid is assimilated obviously to the  mean position of all the points in all of the coordinate directions.  Therefore a physical object has uniform density, then its center of mass is the same as the centroid of its shape!
	\begin{figure}[H]
		\centering
		\includegraphics{img/geometry/centroid_triangle.jpg}
		\caption[Example of planar centroid]{Example of planar centroid (source: Wikipedia)}
	\end{figure}
	
	\begin{tcolorbox}[title=Remarks,colframe=black,arc=10pt]
	\textbf{R1}. In physics, the "\NewTerm{center of inertia}\index{center of inertia}" of a body corresponds to the centroid of the particles that make up the body in question. Each particle being weighted by its own mass. So this is the point from which the mass is evenly distributed. If the density is constant, the center of mass coincides with the center of gravity as already mention.\\

	\textbf{R2}. The "\NewTerm{center of gravity}\index{center of gravity}" of a body corresponds to the centroid of the particles that make up the body in question, each particle being weighted by its own weight! Very often in mechanics, the size of the body is small compared with the roundness of the earth, we consider a uniform gravity field. Under this assumption, the center of gravity and the center of mass coincide.\\

	\textbf{R3}. When for any massive point $(A_i,\alpha_i)$ we have $\alpha_i=\alpha_j\;\forall (i,j)\in \mathbb{N}^{*}$, then we speak of "barycentre".
	\end{tcolorbox}
	For an arbitrary point O, we obviously have by simple vector addition:
	
	hence the major result:
	
	Passing to the limit, if the domain is continuous (or can be considered as), we have:
	
	We may well also work with the surface or volumes elements (to mention only the more trivial case) to determine the centroid:
	
	\begin{tcolorbox}[colframe=black,colback=white,sharp corners]
	\textbf{{\Large \ding{45}}Example:}\\\\
	To calculate for example the centroid of the area between the 0$x$ axis and the parabola function $y=kx^2$ for $a\leq x\leq a$ we have (do not forget that when we calculate the centroid along the $y$ axis that we have to take it the right side of the curve!):
	
	in the first component we have $\mathrm{d}S=y\mathrm{d}x$ so the double integral becomes a simple integral!\\
	
	For the second component we can't use $\mathrm{d}S=x\mathrm{d}y$ as it represent an element of area between the $y$-axis and the parabola curve. Indeed, we want the outside area (above the parabola curve when see of the point of view of the $y$-axis!). Therefore the idea is to translate the parabola of $a$ so that the parabola surface relatively to $y$ will be the expected one. Therefore we have $\mathrm{d}S=(x-a)\mathrm{d}y$.\\
	
	And it comes:
	
	\end{tcolorbox}
	\begin{tcolorbox}[title=Remark,colframe=black,arc=10pt]
	When we calculate the center of gravity of a function as in the example above, then we speak of "\NewTerm{weighted curve}\index{weighted curve}".
	\end{tcolorbox}
	In space with a reference frame $(\text{O},\vec{i},\vec{j},\vec{k})$ by writing $(x_i,y_i,z_i)$ the coordinates of the weighted points 
$(A_i,\alpha_i)$ and $(x_G,y_G,z_G)$ those of $G$, then we have (in the discrete case):
	
	Let us now give and prove some properties of the centroid (center of mass/gravity):
	\begin{enumerate}
		\item[P1.] Given $(A_1,\alpha_1),(A_2,\alpha_2),\ldots,(A_n,\alpha_n)$, $n$ weighted points. If $\sum\alpha_i\neq 0$, we have then for any point $M$:
		
		\begin{dem}
		
		As by definition of the barycentre we have:
		
		Then we have indeed:
		
		\begin{flushright}
			$\blacksquare$  Q.E.D.
		\end{flushright}
		\end{dem}

		\item[P2.] For $\forall k\in\mathbb{R}^{*}$, the weighted points $(A_1,\alpha_1),(A_2,\alpha_2),\ldots,(A_n,\alpha_n)$ and $(A_1,k\alpha_1),(A_2,k\alpha_2),\ldots,(A_n,k\alpha_n)$ have same barycentre because (invariance of the barycentre):
	
		\begin{dem}
		The proof is in our point of view immediate (but if you don't see it don't hesitate as always to contact us and we will write it).
		\begin{flushright}
			$\blacksquare$  Q.E.D.
		\end{flushright}
		\end{dem}

		\item[P3.] The barycentre $G$ of $n$ weighted points is invariant when we replace $p$ of them, by their barycentre $G'$, assigned to the condition that their coefficients are such that $\sum_{i=1}^n \alpha_i\neq 0$, then $G$ is the barycentre of:
		
		\begin{dem}
		If $G$ is the barycentre of the $n$ then weighted points:
		
		For the particular case where $M = G$:
		
		But $G$ being the barycentre of the $n$ weighted points $(A_1,\alpha_1),(A_2,\alpha_2),\ldots,(A_n,\alpha_n)$ we have then:
		
		As:
		
		the previous equality proves that $G$ is the barycentre of weighted points:
		
		\begin{flushright}
			$\blacksquare$  Q.E.D.
		\end{flushright}
		\end{dem}

		\item[P4.] If $\sum \alpha_i=0$, for any point $M,N$:
		
		\begin{dem}
		For $M\neq N$ let us calculate:
	
		as $\sum_{i=1}^n \alpha_i=0$, we then have:
		
		\begin{flushright}
			$\blacksquare$  Q.E.D.
		\end{flushright}
		\end{dem}
	\end{enumerate}
	\begin{tcolorbox}[title=Remark,colframe=black,arc=10pt]
	When a body has a certain symmetry, the calculations are simplified because the center of gravity (barycentre) should coincide with the element of symmetry. If a body such as a sphere, parallelepiped, etc., has a center of symmetry, the centroid is coincides with it. If the body only has one axis of symmetry, the centroid is then on this axis.
	\end{tcolorbox}
	
	\subsection{Geometric Transformations}\label{geometric transformations}
	The transformations in the plane (and spaces of more dimensions) are usually strictly defined using group theory (\SeeChapter{see section Set Algebra page \pageref{set algebra}}). But in the context of Euclidean geometry, this set theory approach does not interest us. So we will do in this subsection only a simple formal approach (and therefore more intuitive) of elementary transformations in the plane that are: translation, scaling, rotation and skew (shear). For the translation we will see also the equivalent in space as it is very important for our study of Physics (especially for the study of the gyroscope).
	\begin{tcolorbox}[title=Remark,colframe=black,arc=10pt]
	By definition an "\NewTerm{isometry}\index{isometry}" is a transformation that preserves distances and areas ( distance-preserving injective map between metric spaces). As we will see below, the translation, rotation and reflection are isometries. The scaling being obviously not an isometry in the plane or the space.
	\end{tcolorbox}
	\begin{figure}[H]
		\centering
		\includegraphics{img/geometry/transformations_type_01.jpg}
		\includegraphics{img/geometry/transformations_type_02.jpg}
	\end{figure}
	
	\pagebreak
	\subsubsection{Translation}
	Given a straight line in a plane $P$ on which two points $A$ and $B$ define a segment of line denoted $\overline{AB}$.

	\textbf{Definition (\#\mydef):} A "\NewTerm{translation $T$}\index{translation}" ("displacement in a given direction" as said Euclid) of this line segment associates with each point $A$ and $B$ new points $A'$ and $B'$ such that $\overline{AB}=\overline{A'B'}$. So we can restrict the notion of translation to a point only if we want such that mathematically we can write:
	
	In other words, a transformation function (application) of the type transformation $T$ of the of the entire plane in itself associates with each pre-image not more than a single image. Translation is therefore a bijective function. So we can define a unique reciprocal transformation application denoted as $T^{-1}$ (recall of what was seen in the chapter Arithmetics):
	
	We say by definition a point is "\NewTerm{translationally invariant}\index{translationally invariant point}" if and only if:
	
	In another type of formalism, the displacement of the point $A$ to point $B$ following the vector $\overrightarrow{AB}$ is named "\NewTerm{translation vector $\overrightarrow{AB}$}\index{translation vector}" equation (\SeeChapter{see section Vector Calculus page \pageref{translation vector}}). It is expressed mathematically by the sum of the point coordinates and the matrix of the coordinates of the vector (see just below the details).
	
	For example in a three-dimensional space:
	
	The translation is not a linear transformation (\SeeChapter{see section Linear Algebra page \pageref{linear application}}), we can not allow ourselves to be represented by the multiplication of a square matrix as we will see it for following other transformations (scaling and transformation).

	For this purpose (be able to express this transformation as square matrix linear application) we must use a workaround consisting in using a system named "\NewTerm{homogeneous coordinates}\index{homogeneous coordinates}" (\SeeChapter{see section Projective Geometry page \pageref{homogeneous coordinates}}) where the points of the plane are represented by a vector having $3$ components (and respectively those of space by a vector with $4$ components):
	
	In the context of the study of translation we put $w=1$ because in this case:
	
	This system of homogeneous coordinates is applicable to all other transformation we will see later by adding each time a coordinate (\SeeChapter{see section Projective Geometry page \pageref{homogeneous coordinates}}).
	\begin{tcolorbox}[title=Remark,colframe=black,arc=10pt]
	Translation sends a line on a parallel line to the original line of course!	
	\end{tcolorbox}

	\subsubsection{Homothety (scaling)}\label{scaling}
	Given any shape in the plane or space (point, line, oval, polygon, etc.), a number $R$, and a point placed $C$ at a predefined location.

	\textbf{Definition (\#\mydef):} A "\NewTerm{homothetic transformation $H$}\index{homothetic transformation}" (also named "\NewTerm{scaling}\index{scaling}" or rarely "\NewTerm{enlargement}") of ratio $R$ and center $C$ is the application that at every point $M$ of the shape associates to the segment $\overline{CM}$ a new point collinear to $\overline{CM}$ but arranged at a greater or smaller distance of ratio $R$ from the center $C$ such that $\overline{CM'}=R\cdot \overline{CM}$.

	We can restrict the concept of scaling to a line segment such that we can write mathematically:
	
	In other words, a transformation of the type scaling in the entire plan in itself associates with each pre-image not more than a single image. Dilation is therefore as for the translation a bijective function. We can the define a reciprocal transformation application denoted $H^{-1}$ such that as:
	
	If $R\neq 1$, then $C$ is the only invariant point. If $R=1$, then all points are invariant and the scaling is say to be of the type "\NewTerm{scaling identity}\index{scaling identity}". If $R=-1$, then we say that we have a "\NewTerm{central symmetry}\index{central symmetry}". A central symmetry is then a rotation of $180^\circ$ (or $\pi$ [rad]).
	
	In another type of formalism, a scaling of center O and ratio $k$, associate to the point $A$ a point $B$ such that $\overline{\text{O}B}=k\overline{\text{O}A}$. The point $B$ being located on the straight line $\text{O}A$ and at a distance $\overline{\text{O}B}=k\overline{\text{O}A}$. The sign of $k$ determines the position of $B$ with respect to O:
	\begin{figure}[H]
		\centering
		\includegraphics{img/geometry/scaling.jpg}\\
		$\overline{\text{O}B}=k\overline{\text{O}A}\Leftrightarrow (x',y',z')=k(x,y,z)$
		\caption{Example of scaling of center O)}
	\end{figure}
	We now make a jump in the spatial geometry (the jump is not very large and require just the knowledge of elementary vector calculus and linear algebra)

	We can also replace the scalar $k$ by a square matrix such that:
	
	A trivial solution for a scaling is to put that $b=c=d=f=g=h=0$ from which the matrix form obvious diagonal k:
	
	This matrix is named "\NewTerm{transformation matrix by scaling of center O (in this case the origin of the reference frame) and ratio $k$}\index{transformation matrix}" and therefore the scaling being a diagonal matrix commutes with any linear application.
	
	Obviously for the two dimension case we have immediately:
	
	where $W$ is for "width" and $H$ is for "height".
		
	\begin{tcolorbox}[title=Remark,colframe=black,arc=10pt]
	It  should be noticed that as many planar shapes have their surface that depends on the multiplication of two of their geometrical parameters, as scaling of factor $k$ has a squared effect of factor $k$ on the surface. The same principle applied for a volume but it's cubic amplitude rather than a squared one.
	\end{tcolorbox}
	In the case presented above, the scaling conserves the forms in all directions (its geometry is invariant under this transformation) if we use indeed the same factor $k$ for all axes (we speak then of "\NewTerm{uniform scaling}\index{uniform scaling}"). But we may also use the following matrix:
	
	that would deforms the shape according to a different factor for each axis.
	The inverse transformation of the scaling is obviously the scaling of center O and ratio $k^{-1}$ thus in the form of a matrix:
	
	When the scaling center does not coincide with the origin of the selected coordinate system (which almost happens almost all the time), the procedure for calculating the coordinates of the image point is very simple. It is necessary to:
	\begin{enumerate}
		\item Carry out a translation to match the center of scaling with the origin of the frame and apply this translation to all points involved.

		\item Make the scaling as described above (the center being the origin of the reference frame).

		\item Make the reverse translation to take back the of center and the image in its original place.
	\end{enumerate}
	To conclude, notice that the successive operations of translation and scaling are not commutative in the general case. In practice, we often performs first scaling and then translation as described in the three steps above.
	
	\subsubsection{Shear (skew) transformation}
	\textbf{Definition (\#\mydef):} In plane geometry, a "\NewTerm{shear mapping}\index{shear mapping}" is a linear application that displaces each point in fixed direction, by an amount proportional to its signed distance from a line that is parallel to that direction.This type of mapping is also named "\NewTerm{shear transformation}\index{shear transformation}", "\NewTerm{transvection}\index{transvection}", or just "\NewTerm{shearing}\index{shearing}".
	
	Before formalize the general case of shear let us consider the following special case of a shearing factor $k$ along the $x$ axis:
	
	\begin{figure}[H]
		\centering
		\includegraphics{img/geometry/shearing_example.jpg}
		\caption{Scaling with shear}
	\end{figure}
	OK let us generalize that stuff! In fact for the shearing we must consider the case of shearing through the $x$-axis whose situation can be summarized by:
	\begin{figure}[H]
		\centering
		\includegraphics{img/geometry/shearing_x.jpg}
		\caption{Shearing trough $x$-axis}
	\end{figure}
	So it is immediate that we have:
	
	and the case of shearing through the $y$-axis whose situation can be summarized by:
	\begin{figure}[H]
		\centering
		\includegraphics{img/geometry/shearing_y.jpg}
		\caption{Shearing trough $y$-axis}
	\end{figure}
	So it is immediate that we have:
	
	
	\pagebreak
	\subsubsection{Rotation}\label{rotation}
	Given any some form in the plane (point, line, oval, polygon, etc.), a real number $\alpha$, and a point $C$ located at a predefined location.

	\textbf{Definition (\#\mydef):} A "\NewTerm{rotation $R$}\index{rotation}" of angle $\alpha$ and of center $C$ is the application that to every point $M$ of the shape associates to the segment $\overline{CM}$ a new point, but having had a positive or negative angle of rotation $\alpha$ of center $C$ such that $\overline{CM}$ and $\overline{CM'}$ have the same length but not same direction.
	
	From this definition, it is apparent that the axis of rotation of an object is the locus of points of the object which remain stationary!
	\begin{tcolorbox}[title=Remark,colframe=black,arc=10pt]
	The rotation is also, in more academic way, a bijective mapping in the plane, we can also define a reciprocal transformation application denoted $R^{-1}$. But the rotation in the space is not bijective so we can not define a unique reciprocal transformation.
	\end{tcolorbox}
	If $\alpha=0$, then $C$ is the only invariant point. If $\alpha=2k\pi$ (with $k\in\mathbb{Z}$), then all points are invariant and the rotation is say to be of the type "\NewTerm{identity rotation}\index{identity rotation}". If we choose an appropriate system of perpendicular axes such that their intersection coincides with $C$ and that $\pm \pi$ then $R$ is named a "\NewTerm{rotation of central symmetry}\index{rotation of central symmetry}".

	In another type of formalism, the rotation is expressed much more rigorously. We will help us with the drawing of a unit circle (in the plane) to study this type of transformation. We will consider the first case the origin of the reference and translation coincides: 
	\begin{figure}[H]
		\centering
		\includegraphics{img/geometry/rotation_transformation_in_the_plane.jpg}
		\caption{Example of rotation in the plane}
	\end{figure}
	where $A'$ is the image of $A$ by the rotation of center O and angle $\theta$.

	We have in the plane for the point $A$ (\SeeChapter{see section Trigonometry page \pageref{circle trigonometry}}):
	
	and identically for the point $A'$:
	
	with $\beta=\alpha+\theta$.

	Which brings us to write:
	
	Therefore using the Trigonometric identities proved in the section Trigonometry page \pageref{remarkable trigonometric identities}:
	
	Identically (based on the fact that the basic trigonometric identities presented in the section Trigonometry of this book are known), we find:
	
	This allows us to write the rotation matrix in the plane (imagining that the $z$-axis out of the sheet/screen)\label{rotation matrix in the plane}:
	
	Therefore the rotation matrix in the plane is given by:
	
	The inverse transformation is simply the rotation of center O (same as before) but of angle $-\theta$ that is (we use obviously again the trigonometric relations of opposite angles):
	
	An easy way to see this is to notice that it is immediate that:
	
	When we want to make a rotation of an object around its barycentre (centroid) we should proceed as for the scaling, that is to say: bring the barycentre of the shape back to origin O of the reference frame with a translation, apply the rotation on the shape and only after take back the shape to its original position with a translation.
	
	\label{3d rotation matrix around}During the rotation of an object in space (we take the opportunity to introduce this now... because we need it in several on physics concepts that will follow in this book), the transformation is quite similar to the previous one except that the product of two rotations does not give a rotation and that it is not commutative when the number of dimensions is strictly greater than $2$ (we can make the detailed calculation on request).
	
	Indeed, during a rotation of angle $\phi$ around the $z$-axis coordinate, the coordinate $z$ does not change. Which brings us to write the rotation matrix in three-dimensional space relatively to the plane $x-y$ as:
	
	OK for people reading this book on a computer with Adobe Flash player installed and activated, here is an animated version (otherwise see here: \url{https://vimeo.com/575753732}):
	\begin{center}
	\centering
		\includemedia[activate=pageopen,width=125pt,height=125pt,
	]{}{swf/Z_rotation.swf}
	\end{center}
	So therefore the rotation matrix application around the $z$ axis of angle $\phi$ is:
		
	 So we can see that $R_Z(\phi)$ contain the rotation matrix in the plane $x-y$ that we have proved earlier above but with one more component according to the idea of homogeneous coordinates!!!
	 
	 The philosophy is then always the same relative to other axes:
	 
	 The three-dimensional rotation around the $x$-axis (relatively to the plane $z-y$) of angle $\theta$:
	
	OK for people reading this book on a computer with Adobe Flash player installed and activated, here is an animated version (otherwise see here: \url{https://vimeo.com/575753223}):
	\begin{center}
	\centering
		\includemedia[activate=pageopen,width=125pt,height=125pt,
	]{}{swf/X_rotation.swf}
	\end{center}
	So therefore the rotation matrix application around the $x$ axis of angle $\theta$ is \label{3d rotation matrix around x}:
	
	 
	 And finally the three-dimensional rotation around the $y$-axis (relatively to the plane $z-x$) of angle $\gamma$:
	
	OK for people reading this book on a computer with Adobe Flash player installed and activated, here is an animated version (otherwise see here: \url{https://vimeo.com/575753505}):
	\begin{center}
	\centering
		\includemedia[activate=pageopen,width=125pt,height=125pt,
	]{}{swf/Y_rotation.swf}
	\end{center}
	So therefore the rotation matrix application around the $y$ axis of angle $\gamma$ is:
	
	 We finally have three rotation matrices $R_X(\theta)$, $R_Y(\gamma)$, $R_Z(\phi)$ each corresponding to one of the planes $(xy,yz,xz)$ of three-dimensional space. The three angles referred above are named "\NewTerm{Euler angles}\index{Euler angles}" and sadly there is no convention in the notation of the angles, they depends on the authors and teachers.

	These three matrices are part of the group of matrices of order three, denoted "\NewTerm{SO ($3$)}" and named by physicists and mathematicians "\NewTerm{group of spatial rotations SO($3$)}\index{group of spatial rotations}" as seen in the section of Set Algebra. Any rotation can be represented by the product matrix resulting from the product of these three (orthogonal) matrices (whose determinant is equal to $1$ as proved in the section of Linear Algebra).
	\begin{tcolorbox}[title=Remark,colframe=black,arc=10pt]
	There is not only one eigenvector for one rotation matrix. There can be one but also a maximum of three. Indeed, consider a rotation of an object around the $z$-axis of $\pi$ [rad] then obviously the $z$ eigenvector $(0,0,1)$ with eigenvalues $-1$, $+1$ is obviously an eigenvector but also in this special case $(1,0,0)$ is an eigenvector but with an eigenvalue of $-1$.
	\end{tcolorbox}
	Any rotation is a composition of these three rotations but it is important that the reader remembers the in the section of Linear Algebra where we saw that matrix multiplication is not commutative. Thus, rotation about the $x$-axis of $90^\circ$ ($\pi/2$ [rad]) and the about the $z$-axis of  $90^\circ$ ($\pi/2$ [rad]) is not equivalent to rotate first around the $z$-axis and then around the $x$axis by the same angle as shown in the figure below:
	\begin{figure}[H]
		\centering
		\includegraphics{img/geometry/rotation_transformation_non_commutativity.jpg}
		\caption{Example of non-commutativity of the rotation matrix}
	\end{figure}
	Finally, to get the matrix that makes up a particular case of the three rotations here for example are the commands to provide in Maple 4.00b:
	
	\texttt{>X:=array([[1,0,0],[0,cos(theta),sin(theta)],[0,-sin(theta),cos(theta)]]);\\
	>Y:=array([[cos(lambda),0,-sin(lambda)],[0,1,0],[sin(lambda),0,cos(lambda)]]);\\
	>Z:=array([[cos(phi),sin(phi),0],[-sin(phi),cos(phi),0],[0,0,1]]);
	>evalm(X\&*Y\&*Z);\\
	}\\
	It is not very interesting to provide to our reader the expression of $R_{XYZ}=R_XR_YR_Z$ as the rotations are not commutative and that therefore (excepted for some special cases) we have $R_{XYZ}\neq R_{XZY}\neq R_{YZX}\neq R_{YXZ}\neq R_{ZXY}\neq R_{ZYX}$. This is why most book don't give them (excepted some PDFs on Internet).
	
	But as a reader requested it for $R_{ZYX}$ let us give it:
	
	
	If we are looking to achieve the composition of a translation $T$ and a rotation $R$ and a scale scaling $H$ the transformation matrix will be typically:
	
	\begin{tcolorbox}[title=Remarks,colframe=black,arc=10pt]
	\textbf{R1.} Let us recall (\SeeChapter{see section Linear Algebra page \pageref{non-commutativity matrices}}) that the multiplication of two matrices is generally not commutative.\\
	
	\textbf{R2.} The direct similarity of center $C$, of ratio $R$ and angle $\alpha$ is the composite of the scaling of center $C$ with ratio $R$ and a rotation of center $C$ and angle $\alpha$. We refer the reader to the section Numbers to review that complex numbers allow to operate formally to similarities with the operations of addition and multiplication (direct or retrograde).\\
	
	\textbf{R3.} We can make much more powerful and variable rotations using quaternions numbers (or "hypercomplex numbers"). For more information about this type of numbers the reader should refer to the section Numbers where the quaternions are introduced with a lot of details that makes more clear why they are preferred sometimes in 3D computing for rotations instead of the rotation matrices.
	\end{tcolorbox}
	
	\paragraph{Gimbal lock}\mbox{}\\\\
	Gimbal lock is the loss of one degree of freedom in a three-dimensional, three-gimbal mechanism that occurs when the axes of two of the three gimbals are driven into a parallel configuration, "locking" the system into rotation in a degenerate two-dimensional space.
	
	The problem of gimbal lock appears when one uses Euler angles in applied mathematics. Developers of 3D computer programs, such as 3D modelling, embedded navigation systems, robotics and video games must take care to avoid it.
	
	Before going in the math stuff let us show an example of gimbal lock. Consider the following gyroscope like structure:
	\begin{figure}[H]
		\centering
		\includegraphics{img/geometry/gimbal_lock_01.jpg}
	\end{figure}
	In the above configuration we can rotate the Jet in any angle (that means that a jet will know its position wherever it go) Indeed, the inner disc can rotate uniquely, same for the inner and outer disc. But now if the following configuration happens to occur:
	\begin{figure}[H]
		\centering
		\includegraphics{img/geometry/gimbal_lock_02.jpg}
	\end{figure}
	The reader can then see that the inner disc then can only rotate around the same axis as the outer disc. So... yes!!! We lost one degree of freedom.
	
	A lot of confusion results sometimes from the term "lock" in this issue. Indeed, we see from the above figure the gimbals don't actually "lock". They're all still free to spin about like they normally would, no ring is being held up or forcibly locked into position. It's only that if two of the three axis are aligned we can't say anymore  the difference between rotating about one axis or another one. We can't because they're effectively the same axis in this special position. Both gimbals would rotate in the same direction at the same time.

	So in certain orientations a way, pitch or roll axis aligns with one of the others. A that point, a change in the first can't be distinguished from a change in the other. That's a big problem if we need to keep track of our orientation accurately. 
	
	\begin{tcolorbox}[title=Remark,colframe=black,arc=10pt]
	It is said that well-known gimbal lock incident happened in the Apollo 11 Moon mission
	\end{tcolorbox}
	In formal language, gimbal lock occurs because the map from Euler angles to rotations . Indeed, Euler angles provide a means for giving a numerical description of any rotation in three-dimensional space using three numbers, but not only is this description not unique, but there are some points where not every change in the target space (rotations) can be realized by a change in the source space (Euler angles). This is a topological constraint - there is no covering map from the 3-torus to the 3-dimensional real projective space; the only (non-trivial) covering map is from the 3-sphere, as in the use of quaternions.

	A rotation in 3D space can be represented numerically with matrices in several ways. One of these representations is as we have just seen it:
	
	Let us observe what happens if $\theta=0$ (that is to say we block the rotation around the $x$-axis):
	
	Carrying out matrix multiplication:
	
	And finally using some of our trigonometry identities (\SeeChapter{see section Trigonometry page \pageref{remarkable trigonometric identities}}):
	
	Changing the values of $\phi$ and $\gamma$  in the above matrix has the same effects: the rotation angle $\phi-\gamma$ changes, but the rotation axis remains in the $z$ direction! Indeed, we know that in a rotation matrix that when the first row and first column are fixed then we have a rotation around the $x$-axis, when the second row and second column are fixed then we have a rotation around the $y$-axis, when we the third a row and third column are fixed then we have a rotation around the $z$ axis. So when the third column is fixed but instead of the third row it is the first row that is fixed than the rotation
	
	the last column and the first row in the matrix won't change. The only solution for $\phi$  and $\gamma$  to recover different roles is to change $\theta$.
	
	In this specific situation, the gimbal lock behaviour can NOT be avoided. Quaternion can uniquely identify a rotation, but when it is converted into Euler rotation, it loses also one degree of freedom. 
	\begin{figure}[H]
		\centering
		\includegraphics[scale=0.6]{img/geometry/rotation_vocabulary_in_aeronautic_science.jpg}
		\caption{Rotation vocabulary in aeronautic science}
	\end{figure}
	
	\pagebreak
	\paragraph{Euler angles}\mbox{}\\\\
	Let us come back on the Euler angles for an important result for our study late of the gyroscope effect in Classical Mechanics (especially the "nutation" effect)!
		
		So we characterize a general orientation of a body with respect to the inertial system $xyz$ (see figure below) in term of the following $3$ rotations respectively by tradition the following order:
	\begin{enumerate}
		\item Rotation by angle $\phi$ around the $z$-axis
		\item Rotation by angle $\theta$ around the new $x_1'$ axis, which we name the "\NewTerm{line of nodes}\index{line of nodes}".
		\item Rotation by angle $\gamma$ about the new $x_3$-axis
	\end{enumerate}
	\begin{figure}[H]
		\centering
		\includegraphics[scale=0.8]{img/geometry/euler_angles_rotation_convention.jpg}
		\caption{Euler angles rotation convention}
	\end{figure}
	Formally as we know this is given by:
	
	Fortunately, we most of time not have to use the resulting matrix of the three matrix multiplication above. In most kinematic problems what we really need is to be able to write the components of the angular velocity $\vec{\Omega}$ in both systems of coordinates. Since $\vec{\Omega}$­ describes precisely how fast the angles vary in time, we have:
	
	since the three rotations are about these particular axes.

	Let us analyze each contribution to $\vec{\Omega}$:
	\begin{enumerate}
		\item $\dot{\vec{\phi}}=\vec{e}_z\dot{\phi}$ (with respect to $xyz$ system). Following the rotations, we find that with respect to $123$ system we have first (the reader can take the extreme situations where $\theta=0$ and $\theta=\pi/2$ to see that this is accurate):
		
		but we notice that we also have with the same idea:
		
		Therefore we have:
		
		Hence:
		
		with respect to the $123$ system.
		
		\item $\dot{\vec{\theta}}=\vec{e}_{1'}\dot{\theta}$ (with respect to $1'2'3'$ system). Following the rotations, we find that with respect to $123$ system we have obviously (the reader can take the extreme situations where $\gamma=0$ and $\gamma=\pi/2$ to see that this is accurate):
		
		Therefore:
		
		with respect to $123$ system.

		\item $\dot{\vec{\gamma}}=\vec{e}_{3}\dot{\gamma}$. Following the rotations, we have then immediately with respect to $123$:
		
	\end{enumerate}
	Therefore summing the relations:
	
	Therefore the sum finely regrouped gives:
	
	with respect to $123$ system. Many times denoted:
	
	The system:
	
	is named "\NewTerm{kinematic Euler equations}\index{kinematic Euler equations}" and is very useful to express the rotation speed of a body in its own reference frame (most of time its center of mass).
	
	\pagebreak
	\subsubsection{Reflection}
	\textbf{Definition (\#\mydef):} The "\NewTerm{reflection}\index{reflection}", also named "\NewTerm{axial symmetry}\index{axial symmetry}", denoted $S_\Delta$ (in geometry) relatively to the straight line $\Delta$ (in space the reflection is done relatively to a plane) is the application that associates to each point $M$ outside $\Delta$ the point $M'$ such that $\Delta$ is the mediator of $\overline{MM'}$. If $M$ belongs to $\Delta$, then $M=M'$.

	Mathematically it is written:
	
	In other words, a transformation function of the type reflection of the entire plan in itself associates with each pre-image not more than a single image. Reflection is then a bijective function. So we can define a reciprocal transformation denoted $S_\Delta^{-1}$ such that:
	
	\begin{tcolorbox}[title=Remark,colframe=black,arc=10pt]
	All points  of $\Delta$ are trivially invariant under reflection in the plane.
	\end{tcolorbox}
	In matrix form the reflections in the plane are extremely simple to formalize using linear algebra (see section of the same name page \pageref{linear algebra}) as shown in the situations below:
	\begin{itemize}
		\item Reflection relatively to the $y$-axis:
		
		\begin{figure}[H]
			\centering
			\includegraphics{img/geometry/reflection_y.jpg}
			\caption{Reflection around $y$-axis}
		\end{figure}

		\item Reflection relatively to the $x$-axis:
		
		\begin{figure}[H]
			\centering
			\includegraphics{img/geometry/reflection_x.jpg}
			\caption{Reflection around $x$-axis}
		\end{figure}


		\item Reflection relatively to the origin O:
		
		\begin{figure}[H]
			\centering
			\includegraphics{img/geometry/reflection_origin.jpg}
			\caption{Reflection around origin O}
		\end{figure}
		The reader have probably notice that every reflection matrix is in fact only the special case of the plane rotation matrix and this was not difficulty to guess as we can see on every figure that a reflection is only a rotation around a special center of rotation.
		
	\end{itemize}
	
	Here is a quite good summary of what he have seen so far:
	\begin{figure}[H]
		\centering
		\includegraphics{img/geometry/affine_transformation_matrix_summary.jpg}
		\caption[Summary of affine transformations]{Summary of affine transformations (source: Wikipedia)}
	\end{figure}
	
	
	\begin{flushright}
	\begin{tabular}{l c}
	\circled{70} & \pbox{20cm}{\score{4}{5} \\ {\tiny 31 votes,  71.61\%}} 
	\end{tabular} 
	\end{flushright}
	
	%to make section start on odd page
	\newpage
	\thispagestyle{empty}
	\mbox{}	
	\section{Non-Euclidean Geometry}\label{non-euclidean geometry}
	\lettrine[lines=4]{\color{BrickRed}O}ne may recall from their the section of Euclidean Geometry that the sum of the angles in a triangle is 180 degrees. This comes from the Euclidean or "flat" geometry, which includes something called the "parallel postulate" which states that if you were to draw two points next to one another, then extend from those points two lines that are parallel to one another, those lines could be extended to infinity without ever becoming closer together or further apart. This makes perfect intuitive sense; Euclidean geometry seems to be the structure of our world. However, our senses often deceive us. It turns out that gravity literally warps the geometry of space-time itself as we will see it in the section of General Relativity. That's right, because of the gravitational field, space-time is non-Euclidean (and there is some amount of gravity everywhere, since it is a force with infinite range). If not Euclidean, what else can geometry even be?As illustrated below, geometry on curved surfaces is a little different from geometry on flat (Euclidean) surfaces:
	\begin{figure}[H]
		\centering
		\includegraphics{img/geometry/curvatures.jpg}
		\caption{Positive (elliptic), negative (hyperbolic) and flat (euclidean) curves}
	\end{figure}
	
	In a negatively curved ("hyperbolic") geometry, the sum of the angles in a triangle add up to less than $\pi$ degrees, and parallel lines diverge from one another. In a positively curved geometry (also known as an "elliptic" geometry), the sum of the angles in a triangle add up to greater than $\pi$ degrees as proved in the section of Trigonometry, and parallel lines converge towards one another. A "flat" geometry, the Euclidean case, has zero curvature. It is a special case that is the exact point in between hyperbolic and elliptic geometries. Another illustration below shows the failure of the parallel postulate in non-Euclidean spaces:
	\begin{figure}[H]
		\centering
		\includegraphics{img/geometry/curvatures_plane_view.jpg}
	\end{figure}
	Therefore we guess obviously that non-Euclidean geometries are all geometries that not necessarily satisfy all the Hilbert's axioms (\SeeChapter{see section Euclidean Geometry page \pageref{hilbert axioms}}) but without any contradiction between them (unlike the old axioms of Euclid, particularly that on parallels).\\
	
	A particular representation of this type of geometry consists to define the points as being distributed over the surface of a sphere (that are the intersection of the diameters of the sphere with the surface), and the lines, to generalize the concept of straight line (now we say "geodesic"), as the intersections of the surface of the sphere with the planes containing the center of the sphere. Two points then define uniquely a line and a point is always given by the intersection of two lines. However, in this geometry, if we give ourselves a line $AB$ and a external point $P$ to the line, there is no line passing through $P$ and not intersecting $AB$. So Euclid's fifth postulate is not satisfied because in $P$ we can not draw any parallel to $AB$.
	
	\begin{figure}[H]
		\centering
		\includegraphics{img/geometry/euclid_paralle_violation.jpg}
		\caption{Illustrated example of the violation of the fifth Euclide postulate}
	\end{figure}
	
	\begin{tcolorbox}[title=Remark,colframe=black,arc=10pt]
	Before starting to read this section, we strongly recommend the reader to read and if possible to understand the content of the section on Trigonometry, Euclidean Geometry and Tensor Calculus as we will use many results, not necessarily trivial, we have proved in them.
	\end{tcolorbox}
	
	In section on Euclidean Geometry, we studied a number of theorems related to plans. Let us insist on the fact that the "plan" is a two-dimensional figure whose curvature is null and immersed in a 3-dimensional space (therefore the plane can be moved in it). This said, it should perhaps now necessary to define a little bit more rigorously the intuitive concept of "curvature(s)".
	
	\subsection{Curvature(s)}
	\textbf{Definition (\#\mydef):}	A figure is said to be "\NewTerm{curved}\index{curved}" if there is at least one point lying on the straight or straights defining the bounds of the figure (surfaces, volumes, etc.) defining a tangent on the bound and not crossing the bound elsewhere.
	
	This is Friedrich Gauss who in 1824 had formulated the possibility that there are alternatives to Euclidean geometries. We distinguish between geometries with "\NewTerm{negative curvature}\index{negative curvature}", like the one of the Russian Nikolai Lobachevsky (1829) and Bolyai (1832) (sum of angles of a triangle below $\pi$, infinite number of possible parallel to a line through a point), from geometries with "\NewTerm{positive curvature}\index{positive curvature}" as the Riemann (1867) one (sum of the angles of a triangle greater than $\pi$, parallel joining to the poles).
	
	We will see in this section various non-Euclidean geometries, the best known are the "\NewTerm{Riemannian geometries}\index{Riemannian geometries}" (constant curvature) and "Lobachevsky geometries" (hyperbolic type with non-constant curvature).
	
	\begin{tcolorbox}[title=Remark,colframe=black,arc=10pt]
	The geometry commonly named "Riemann geometry" is a three-dimensional spherical space, finite space but boundless, with regular curvature, alternative to  Euclidean postulate on parallels.
	\end{tcolorbox}
	
	The interest in the study of these geometries is that we can't determine whether the Universe we live in is made of a type of geometry rather than another because given our (physical) size, immersed we are in any geometry whatsoever with low curvature, any surface space seems locally Euclidean to us (two parallel lines do not intersect). However, in General Relativity, which makes an excessive us of tensor calculus (generalization of any geometry as seen in the section on Tensor Calculus) shows that there are region of the space where the geometry is strongly curved and therefore locally non-Euclidean and only the study of this kind of geometries allows us to draw theories explaining observations that are not exploitable only with human intuition.
	
	A distinction has also to be made between "\NewTerm{intrinsic curvature}\index{curvature!intrinsic curvature}" and "\NewTerm{extrinsic curvature}\index{curvature!extrinsic curvature}\footnote{As a common question on Internet is that a space can only be curved in relation to an absolute "straight space"? Otherwise we would never know that it is curved. There must be an absolute straight space geometry underlying the curved space?}". The easy to visualize curvature is extrinsic. The physically meaningful curvature is intrinsic! Intrinsic curvature is what General Relativity deals with! Imagine a sheet of paper laid flat on a table. On this paper is a bug. The bug walks along the shortest path on the paper from point $A$ to point $B$. We can all agree that the paper is flat and that the line is straight. Now roll up the paper into a tube. Again the bug walks along the shortest path from point $A$ to point $B$. The bug will have walked across the exact same path on the paper (assuming that the shortest path does not cross the seam). From the point of view of the paper, the path is straight. From our external point of view, the paper is curved and the path follows the paper. The above is an example of extrinsic curvature. We have a two dimensional space (the paper) and a larger three dimensional space (our ordinary three dimensional geometry) in which it is embedded. The curvature of the path depends on how we roll up the paper, ie how we choose to do the embedding. For an inhabitant living on the surface of the paper using measurements made only on that surface, extrinsic curvature is not detectable.
	
	 Extrinsic curvature is a property of the embedding. The bug traces out the same path on the paper, whether it is rolled or unrolled. It is only from our external viewpoint that we can see any "curvature". In the case of a sphere, things are not so simple. We cannot roll up a flat sheet of paper into a sphere without stretching or wrinkling it. If the bug on a sphere follows shortest paths and keeps careful track of distances, he can notice that the surface he lives on is not Euclidean. The interior angles of a triangle will sum to more than $180$ degrees as we have already proved in the section of Trigonometry. The above is an example of intrinsic curvature. For an inhabitant living on the surface of the paper using measurements made only on that surface, intrinsic curvature is detectable. Here comes the hard part... In the examples above, we talked about a two dimensional surface embedded within a three dimensional Euclidean space. That was just an aid to visualization. We can talk about the geometry of a two (or three or four or more) dimensional space without requiring that it be embedded in a higher dimensional space at all. A common way to do that is to imagine that the inhabitants of the space are able to measure distances. From any point in the space to any other point in the space, they can measure the distance between them. The distance measurement is the mathematical notion of a "metric". A space for which a metric exists is called a "metric space" as we know it. Given a metric space, one can define intrinsic curvature in terms of the metric. A standard way of handling a metric space in physics is to divide it up (if needed) into pieces and equip each piece with a Cartesian coordinate system. There are some rules about the boundaries and how to handle the seams between the pieces that we need not concern ourselves with. The result is a "manifold". A surface of a sphere can be modelled as a two dimensional manifold. There is no embedding in three dimensional space and no meaningful notion of extrinsic curvature. There is still non-zero intrinsic curvature for this space.
	
	Before we tackle formally and with an abstract manner to certain non-Euclidean geometries we will first make a pragmatic and specific introduction of certain concepts that we are not totally foreign for because already covered in other section theoretically. Once made this introduction, which will be very useful pedagogically, we will address these concepts more rigorously.
	
	\subsection{Axioms of non-euclidean geometry}
	To obtain a non-Euclidean geometry, the parallel postulate A.P1 (or its equivalent) that we have seen during our study of Euclidean Geometry must be replaced by its negation (A.NP1). Everything that follows is related to that figure:
	\begin{figure}[H]
		\centering
		\includegraphics{img/geometry/curvatures_plane_view.jpg}
	\end{figure}	
	Then the axiom of parallels becomes:
	\begin{enumerate}[leftmargin=2cm]
		\item[A.NP1.] The also named "\NewTerm{Lobachevsky axiom}\index{Lobachevsky axiom}" or "\NewTerm{Hyperbolic Parallel Postulate (HPP)}\index{Hyperbolic Parallel Postulate}" assume that there exist at least two lines passing through the given point $P$ not located on a given line and parallel to this line (we speak then of "two diverging ultra-parallel lines").
	\end{enumerate}
	An alternative and more easy formulation is:
	\begin{enumerate}[leftmargin=2cm]
		\item[A.NP1'.] There exist a line $L$ and a point $P$ not on $L$ such that at least two distinct lines parallel to $l$ pass through $P$. In other words:  Through a point not on a line there is more than one line parallel to the given line!!!
	\end{enumerate}
	The above form of the axiom of parallelism ensures the existence of an infinite number of parallel lines satisfying the given conditions. Geometric space defined by the axiomatic system including Lobachevsky axiom is named "\NewTerm{hyperbolic non-Euclidean geometry}\index{hyperbolic non-Euclidean geometry}". Geometry of the curved surface hyperboloid can be described as a hyperbolic geometry. In the hyperbolic space, the interior angles in an arbitrary triangle is less than $\pi$.
	
	A second possible form of the negation of axiom of parallelism is as follows:
	\begin{enumerate}[leftmargin=2cm]
		\item[A.NP1.] There exist no line passing through the given point not located on a given line and parallel to this line.
	\end{enumerate}
	Geometric space defined by this last axiomatic system with the named "\NewTerm{elliptic non-Euclidean geometry}\index{elliptic non-Euclidean geometry}". Geometry of the sphere (spherical geometry) can be regarded as elliptic geometry. In the elliptic space, the interior angles in an arbitrary triangle is greater than $\pi$ (see proof in the section Trigonometry).
	
	\subsection{Geodesic and Metric Equation}\label{geodesic and metric equation}
	Let us come back on the concepts of geodesic and curvature which we have often mentioned in the section of Tensor Calculus (the fact of not having read the section of Tensor Calculus is normally not a problem in understanding what follows).
	
	Consider, as an introduction example, the two-dimensional surface of a sphere of radius $R$. Given two points $B$ and $C$ diametrically opposed, we seek the shortest distance $s$ measured on the sphere between $B$ and $C$. The curve we get is as we know a "geodesic", a concept that generalizes therefore, for an arbitrary surface, the notion of the straight lines for planes:
	
	\begin{figure}[H]
		\centering
		\includegraphics{img/geometry/geodesic.jpg}
		\caption{Illustration of the problem to research the geodesic on a sphere}
	\end{figure}
	
	\begin{tcolorbox}[title=Remark,colframe=black,arc=10pt]
	We assume as intuitive that the length of a curve in the three-dimensional Euclidean space is always greater than or equal to the length of any planar projection of this curve. The geodesic curve is therefore necessarily a plane curve.
	\end{tcolorbox}
	
	The radius between the axis $\text{O}z$ and one of the points $B$ and $C$ is trivially given by some elementary trigonometry (\SeeChapter{see section Trigonometry page \pageref{remarkable trigonometric identities}}):
	
	And therefore the half circumference of the circle at the height of $B$ and $C$ is given by:
	
	And we have proved in the chapter of trigonometry that the perimeter of a circle depending on the opening angle of the latter was given by ($\alpha$ must be in radians!):
	
	It therefore comes automatically:
	
	Finally:
	
	A plot or Taylor approximation shows easily that $\pi\sin(\theta)\geq 2\theta$ on the interval $[0,\pi/2]$ then on the same interval $s_2 \geq s_1$ (there is equality in $\theta=0$ and $\theta=\pi/2$).
	
	The geodesics of the sphere are therefore the big circles arcs $s_1$, paths taken by aircraft for intercontinental flight (in empty space, without wind and perfect spherical planet), and correspond to the lines obtained between the surface of the sphere and a plane passing through the center thereof!
	
	The geometrical properties of figures drawn on the surface of a sphere are obviously no longer those of Euclidean Geometry. Thus, the shortest path from point $B$ to point $C$ on the spherical surface, consists of a great circle passing through the points $B$ and $C$ plotted on a plane passing through the center of the sphere. Great circle arcs play the same role for the sphere that lines in the plane. These are the "\NewTerm{geodesic}\index{geodesic}" of the sphere.
	
	Now consider two dimensional surfaces: the surface of the sphere and that of the cylinder. Given two points $B$ and $C$, we trace the geodesic curve between these points:
	\begin{figure}[H]
		\centering
		\includegraphics{img/geometry/some_geodesics.jpg}
		\caption{Some geodesics of trivial surface on trivial volumes}
	\end{figure}
	
	The cylinder may be cut parallel to its axis and unfolded flat. The geodesic thus appears as a straight line of the plane! We say then that the cylinder is "inherently flat" (even if its topology is different from the plane, we must especially avoid that the cut crosses the geodesic). This is intuitively obviously not the case of the surface of the sphere.
	
	In the case of the cylindrical surface, we can define the Cartesian coordinates of the plane $B(y_1,z_1)$ and $C(y_2,z_2)$ that permits to write the length $s$ of the curve (straight line) $BC$ with the Pythagorean theorem:
	
	The metric of the plane is Euclidean and under infinitesimal form we get the "\NewTerm{Euclidean metric equation}\index{Euclidean metric equation}":
	
	On the cylinder, the change of variable gives $y=r\theta$ (with the angle in radians!):
	
	Or under local form:
	
	The surface of the cylinder can thus be represented by Cartesian coordinates similar to those of the plane, the metric of the cylinder surface being Euclidean under infinitesimal form and under global form.
	\begin{tcolorbox}[title=Remark,colframe=black,arc=10pt]
	The previous relation is what we got in the section of Tensor Calculus for the metric equation in polar coordinates.
	\end{tcolorbox}
	We can interest us now to the problem to write the analogue of Pythagorean theorem to a spherical surface. The impossibility to cut the ball and flatten it to marry a plane suggests some difficulties...
	
	That is why the equation of the metric can't be written in general form as the Pythagorean theorem. Indeed, we proved in the section of Tensor Calculus that this latter was given for a spherical surface by:
	
	However, locally (that is to say in a small region of small dimension relatively to the radius of the sphere), the properties of the sphere can be described by Cartesian coordinates of a plane tangent to the surface (the essential property of Riemann spaces) as the metric equation is locally Euclidean!:
	
	By writing:
	
	We therefore have:
	
	with:
	
	While $(\theta,\phi)$ are the "\NewTerm{Gauss coordinates}\index{Gauss coordinates}",  $(\xi,\eta)$ are the " \NewTerm{Riemann coordinates}\index{Riemann coordinates}\label{riemann coordinates}" of the locally tangent plane. Thus, we notice that Euclidean space is a special case of Gauss coordinates for which we have (\SeeChapter{see section Tensor Calculus page \pageref{interval invariant}}):
	
	and thus the metric tensor is a identity matrix (the diagonal elements are equal to $1$ the rest being zero):
	
	The reader will have perhaps notice that if we take the vector basis of the polar coordinates as seen in the section of Vector Calculus but without normalizing them to then unit (that is to say without dividing the first vector by $r$ and the second by $\sin(\theta)$) we have:
	
	with $i,j=\{\theta,\phi\}$ such that and so on for every other coordinate system.
	
	\pagebreak
	\subsection{Riemann Spaces}\label{riemann spaces}
	To better understand what is a Riemann space, we will now go through a small example of a two-dimensional surface (classic example):
	
	Consider a sphere of radius $r$, of surface $S$, located in the ordinary three-dimensional space. The Cartesian coordinates $x, y, z$ of a point $M$ on the surface $S$ can be expressed, for example, depending on the spherical coordinate $(r,\theta,\phi)$ (\SeeChapter{see section Vector Calculus page \pageref{spherical coordinates}}). The sphere is fully described for a given radius $r$ and $0\leq \phi < 2\phi$ and $0 \leq \theta < \pi$.
	
	Three such parameters, giving the possibility to determine a point on the surface of a sphere, are as we know (\SeeChapter{see section Tensor Calculus page \pageref{curvilinear coordinates tensor calculus}}) curvilinear coordinates of the surface or also named "Gaussian coordinates" (Gauss being one of the first mathematicians interested in the study of bodies immersed in non-Euclidean spaces). Other arbitrary parameters $u, v, w$ may of course be chosen as curvilinear coordinates on the surface.
	
	The linear element $\mathrm{d}s^2$ of the surface, square distance between two infinitely close points $M, M'$, is written based on spherical coordinates, as we have seen in the section of Tensor Calculus and just above:
	
	We thus obtain an expression of the linear element based on only the three Gauss  coordinates $(r,\theta,\phi)$. We could of course impose a local study (tangent plane) so that the linear element is only function of $(\theta,\phi)$ as we have seen above:
	
	Written using the three parameters,the surface of the sphere (considered as a two-dimensional space) is an example of two-dimensional Riemann space whose linear element is in the general form well known (see the chapter of tensor calculus):
	
	where the $\mathrm{d}u^i,\mathrm{d}u^k$ are the contravariant components of the vector $\mathrm{d}\vec{M}=\overrightarrow{MM'}$ relatively to the referential denoted by $(M,\vec{e}_j)$.
	\begin{tcolorbox}[title=Remark,colframe=black,arc=10pt]
	The study of figures on Riemannian surfaces is part of differential geometry to which we devote a whole section in this chapter.
	\end{tcolorbox}
	Now consider any surface of coordinates $u^1,u^2$. The Cartesian coordinates $x, y, z$ of ordinary space where this surface is dived are written in general terms with the Gauss coordinates:
	
	Remember also that the metric equation can be written in tensorial notation:
	
	can be written in an expanded form as follows (this is proved in details with a geometric approach in the section of Differential Geometry):
	
	with:
	
	\begin{tcolorbox}[title=Remarks,colframe=black,arc=10pt]
	\textbf{R1.} The expression given above for the linear elements $\mathrm{d}s$ is named "\NewTerm{quadratic fundamental form}\index{quadratic fundamental form}" of the surface considered. The coefficients $E, F, G$ are functions of curvilinear coordinates. Generally this surface, considered a two-dimensional space, will be an example of Riemann space, for arbitrary curvilinear coordinates.\\
	
	\textbf{R2.} The different Riemann spaces are what we name in general form (because there is not only Riemannian spaces of constant curvature) a "\NewTerm{variety}\index{variety}" equipped with a Riemannian metric. A variety may be defined (not formally), for example, by a set of points into an existing space. Generally a surface gives the idea of a two-dimensional variety. The sphere and the torus are two-dimensional varieties without borders. A cylinder of revolution, a hyperbolic paraboloid are two-dimensional open varieties with borders to infinity. But we can also consider abstract varieties. This is the case for example of a configuration space. This is then an $n$-dimensional space represented by a set $q^i$ (or denoted by $u^i$) of generalized coordinates (see the introduction to the Lagrangian formalism in the section of Analytical Mechanics page \pageref{lagrangian formalism}), the latter can have values in a finite domain or not.
	\end{tcolorbox}
	
	We can now define in a little better way what is a Riemann space.
	
	\textbf{Definition (\#\mydef):}	A "\NewTerm{Riemann space}\index{Riemann space}" is a variety to which we attached a metric. This means that in every part of the variety analytically represented by a coordinate system $u^i$, we gave ourselves a quadratic differential form:
	
	which is the metric of space.
	
	The coefficients $g_{ij}$ are not entirely arbitrary and should verify, as we have shown in the section of Tensor Calculus, the following conditions:
	\begin{enumerate}
		\item[C1.] The components are symmetric as $g_{ij}=g_{ji}$
		
		\item[C2.] The determinant of the matrix of $g_{ij}$ is not zero
		
		\item[C3.] The differential form of the linear element  $\mathrm{d}s^2$, and therefore the concept of distance defined by the $g_{ij}$, is invariant vis-a-vis any change of coordinates.
		
		\item[C4.] All partial differential of second order of the $g_{ij}$ exist and are continuous and therefore of class $\mathcal{C}^2$.
		\end{enumerate}
		A Riemann space is a space of points, each being identified by a system of $n$ coordinates $u^i$, with a metric such as the differential form of the linear element that satisfies the above conditions. This metric is therefore named "\NewTerm{Riemannian metric}\index{Riemannian metric}".
		
	\begin{tcolorbox}[title=Remarks,colframe=black,arc=10pt]
	\textbf{R1.} If the metric is positive definite, that is to say, if $g_{ij}u^iu^j$ for any non-zero vector $\vec{v}$, we say that the space is "\NewTerm{strictly Riemannian}\index{strictly Riemannian}". In this case, the determinant of the matrix $g_{ij}$ is strictly positive and all the eigenvalues of the matrix are strictly positive (\SeeChapter{see section Analytical Geometry page \pageref{classification of conical by the determinant}}).\\
	
	\textbf{R2.} By definition, we say that a metric space is Euclidean when any fundamental tensor of this space may be reduced, by an appropriate change of coordinates, to a form such as (\SeeChapter{see section Vector Calculus page \pageref{canonical basis}}) the canonical orthonormal basis. That is to say $g_{ij} =\delta_{ij}$.\\
	
	\textbf{R3.} The definition of Riemannian spaces shows that Euclidean space is a very special case of these spaces. So there exists only one Euclidean space but we can create an infinite number of Riemannian spaces.
	\end{tcolorbox}
	It is important to see that:
	
	Can be rewritten as:
	
	that we denote by tradition:
	
	Therefore:
	
	Hence the length of a curve on any differential surface:
	
	
	\pagebreak
	\begin{tcolorbox}[colframe=black,colback=white,sharp corners]
	\textbf{{\Large \ding{45}}Example:}\\\\
	Let us consider the parametric equation of a special hyperbolic surface:
	
	We take now the parametric equation of a circle or radius $1$ on the plane:
	
	That is to say visually with Maple 4.00b:\\
	
	\texttt{>x:=u;y:=v;z:=u*v;\\
	>with(plots):\\
	>surface:=plot3d([x,y,z],u=-1.5..1.5,v=-1.5..1.5,grid=[30,30],\\
	scaling=constrained,axes=boxed,style=PATCH):\\
	>curve:=spacecurve([cos(t),sin(t),cos(t)*sin(t)],t=-4*Pi..4*Pi,\\
	numpoints=1000,scaling=CONSTRAINED,orientation=[50,60],style=PATCH,\\
	axes=NORMAL,color=black,thickness=3):\\
	>display(surface,curve,orientation=[-20,70]);
	}
	
	Which gives:
	\begin{figure}[H]
		\centering
		\includegraphics[scale=0.8]{img/geometry/curvilinear_length.jpg}
	\end{figure}
	\end{tcolorbox}
	
	\begin{tcolorbox}[colframe=black,colback=white,sharp corners]
	Therefore:
	
	Then we have:
	
	And we want to calculate the perimeter of the circle on the surface. Therefore:
	
	We make a change of variable $\tau=2t$, and therefore:
	
	and therefore:
	
	\end{tcolorbox}
	
	\begin{tcolorbox}[colframe=black,colback=white,sharp corners]
	As the integrand is even:
	
	We have then:
		
	With:
	
	and is referred to as the "\NewTerm{complete elliptic integral of second kind}\index{elliptic integral!complete elliptic integral of second kind}\label{elliptic integral riemann space}" (\SeeChapter{see section Differential and Integral Calculus page \pageref{elliptic integrals}}). This integral cannot be formally calculated. But a numerical integration gives:
	
	\end{tcolorbox}
	
	\begin{flushright}
	\begin{tabular}{l c}
	\circled{30} & \pbox{20cm}{\score{4}{5} \\ {\tiny 26 votes,  80.77\%}} 
	\end{tabular} 
	\end{flushright}
	
	%to force start on odd page
	\newpage
	\thispagestyle{empty}
	\mbox{}			
	\section{Projective Geometry}\label{projective geometry}
	\lettrine[lines=4]{\color{BrickRed}S}ince Chasles and this until the years 1930s, Projective Geometry was often synonymous of "higher geometry." It was opposed to Euclidean Geometry: that was said to be elementary and analytical. At the time of Monge, Carnot and von Staudt, we also spoke of "geometry of position" or "geometry of situation". These geometries study figures at the point of view of their respective positions and of their invariant properties which bind them in a geometric transformation (rotation, symmetry, scaling, etc.), homographic in particular. Besides the harmonic division, basic concept, it uses the famous anharmonic ratio (cross ratio), the inverse, the involution the transformation by reciprocal polar, stereographic projection\footnote{ie: projection on a plane}, the correlation, the homology, the duality, the conical, etc. (see later in this text for the definitions).
	
	\textbf{Definition (\#\mydef):} "\NewTerm{Projective geometry}\index{projective geometry}" is a topic of mathematics. It is the study of geometric properties that are invariant with respect to projective transformations. This means that, compared to elementary geometry, projective geometry has a different setting, "\NewTerm{projective space}\index{projective space}", and a selective set of basic geometric concepts. 
	
	Project Geometry is quite abstract apart from some general and basic principles outlined below (undergraduate level). It is therefore relatively difficult to understand and requires before study it: a good knowledge of elementary geometry even in space, a perfect mastery of three-dimensional space in the observation-representation-interpretation cycle, to master also complex analysis for advanced topics. This will led us to introduce some concepts that do not normally have their place in this section, but that we think, can greatly help the reader to understand this branch of mathematics.
	
	In a first place, we will discuss the basic concepts of perspective with a particular focus on the concept of "projective" representation (there are other empirical perspective methods: cavalier, cabinet, isometric, military, ... but these latter don't have mathematical or real meaning even if they represent quite properly volumetric objects). Then we will study the mathematical representations of some three-dimensional objects with some computer application to finally go study "hard" and "pure" projective geometry.
	
	\begin{tcolorbox}[title=Remark,colframe=black,arc=10pt]
	The "\NewTerm{descriptive geometry}\index{descriptive geometry}" is a rigorous artistic form of projective geometry but not a formal on (in the sense that it is not mathematically formalized... at least as far as we know...).
	\end{tcolorbox}
	
	\pagebreak
	\subsection{Conical Perspective (Central Perspective)}
	One problem with the study of three-dimensional volumes and their representation is the concept of "\NewTerm{perspective}\index{perspective}". Indeed, human beings can not see the $3$ dimensions of an object, it is the brain that interprets the shadows and reflections of an object so that we can interpret it as having a volume (there are optical illusions that go in this direction ...: the "trompe l'oeil").

	Here is an animated example of one of the most famous optical illusion, named the Shepard tabletop illusion, that shows that we should, statistically speaking, not trust our intuitions too much...:
	\begin{center}
	\centering
		\includemedia[activate=pageopen,width=\textwidth,height=400pt,
	]{}{swf/shepard_tabletops.swf}
	\end{center}
	The animation above will run for people having a PDF reader with Adobe Flash player installed and activated (otherwise see here: \url{https://vimeo.com/575741013}).
	
	We will focus in the following paragraphs on the "\NewTerm{conical perspective}\index{conical perspective}", also named "\NewTerm{central perspective}\index{central perspective}" or "\NewTerm{linear perspective}\index{linear perspective}".
	
	\begin{tcolorbox}[title=Remark,colframe=black,arc=10pt]
	In the field of projective geometry we are not talking about "\NewTerm{conical perspective}\index{conical perspective}" but "\NewTerm{conical projection}\index{conical projection}".
	\end{tcolorbox}
	
	\textbf{Definition (\#\mydef):} The "\NewTerm{conical perspective}\index{conical perspective}" or "\NewTerm{perspective projection}\index{perspective projection}" (to not be confused with the "conical projection"!!!!!) is by construction the closest representation of our human visual perceptions, it allows especially to see a sphere as a circle as the human vision is base on a visual cone that when looking to a sphere is equivalent to see a circle as illustrated below:
	\begin{figure}[H]
		\centering
		\includegraphics{img/geometry/visual_cone.jpg}
		\caption{Visual cone and corresponding projection circle}
	\end{figure}
	
	\begin{tcolorbox}[title=Remark,colframe=black,arc=10pt]
	The conical perspective is that of Renaissance painters. It is also the one that appears on photos.
	\end{tcolorbox}
	The difficulty of the representation in perspective is to translate it in a plane (for example that of the paper or the computer screen as our eyes makes a projection of a 3D world in a 2D image in our brain) a construction (object) that is defined - fairly simply, by the way - in space using mathematical tools.
	
	A conical perspective of the conventional $3$-dimensional space is a projective transformation that sends all the points of space on the same plane of this space. It requires the information of a point O (equivalent to the position of the observer's eye) and a projection plane named the "\NewTerm{table}\index{table}" or "\NewTerm{Dürer glass}\index{Dürer glass}" (the equivalent of our retina) as referring to a quite old copy painting method:
	\begin{figure}[H]
		\centering
		\includegraphics{img/geometry/durer_plane.jpg}
		\caption{Origin of Durer glass method illustration}
	\end{figure}
	We will see during the theoretical mathematical developments that unlike affine projections (see the following subsection), the conical projection does not retain the barycentre (therefore the ratios of lengths on a given straight line) but it conserve the alignment and cross ratio.
	
	When we speak of conical perspective, we use a few special planes and straight lines in three-dimensional space (see figure below).
	\textbf{Definitions (\#\mydef):}
	\begin{enumerate}
		\item[D1.] The "\NewTerm{picture plane}\index{picture plane}" or "\NewTerm{Dürer glass}\index{Dürer glass}", denoted T, is the plane on which we are drawing (projection plane).

		\item[D2.] The "\NewTerm{ground plane}\index{ground plane}", denoted S, is a fixed plane, perpendicular to the picture plane T.
	
		\item[D3.] The "\NewTerm{point of view}\index{point of view}" (or "\NewTerm{center of projection}\index{center of projection}"), denoted O, is a point out of T and of S: this is the point where we will have to place the eye for the drawing on the picture plane $T$ coincides with the real image.
	
		\item[D4.] The "\NewTerm{horizon plane}\index{horizon plane}" denoted H, is the plane parallel to the ground plane S passing through the point of view point O.
	
		\item[D5.] The "\NewTerm{horizon line}\index{horizon line}", denoted h, is the intersection of the horizon plane H and the picture plane T.
		
		\item[D6.] The "\NewTerm{Earth line}\index{Earth line}", denoted EL, is the intersection of the ground plane S plan and of the picture plane $T$.
		
		\item[D7.] A plane or a straight line parallel to the ground plane S that are named "\NewTerm{horizontal}\index{horizontal}".
		
		\item[D8.] A plane or a straight line perpendicular to the ground plane S that are named "\NewTerm{vertical}\index{vertical}".
		
		\item[D9.] A plane or a straight line parallel to the picture plane T that are named "\NewTerm{front}\index{front}".
		
		\item[D10.] A plane or a straight line perpendicular to the picture plane T that are named "\NewTerm{end}\index{end}".
	\end{enumerate}
	Here is a diagram showing these different concepts:
	\begin{figure}[H]
		\centering
		\includegraphics[scale=0.7]{img/geometry/perspective.jpg}
		\caption{Definition of the planes and lines of projective geometry}
	\end{figure}
	
	\pagebreak
	\subsubsection{Images of Points}
	Any volumetric object (which can only be determined by touching or at the level of mathematical abstraction) composed by a set of points $M$, is for our brain, the image of a plane projection $m$ whose support is a surface in the space between the observed object and our eye.
	
	In mathematics, this surface, we know, named the "table" is defined in its central view by the horizon line (where are located the vanishing points) and a physical repository so named the earth line (see figure below where the observer is at a higher point than the observed object):
	\begin{figure}[H]
		\centering
		\includegraphics{img/geometry/projective_table.jpg}
		\caption{Example of what is a "table" in projective geometry}
	\end{figure}
	The height between the ground line and the point view is named the "\NewTerm{horizon height}\index{horizon height}" and is denoted by $h$.
	
	The objective from this figure is to mathematically determine a representation of a solid object on a surface (on a plane in a simple case, otherwise any) by knowing the equation of the line between point $M$ (point of the observed object) and the point $V$ (point of view) in the purpose to determine the coordinates of intersection between this line and the table.

	We can from the above diagram draw the following relations using the Thales theorem (\SeeChapter{see section Euclidean Geometry}) in the different triangles:
	
	with $\Delta\geq 0$.

	If we put $y'=0$, according to what show the figure above, the coordinates of $m$ become:
	
	and:
	
	From these relations, the problem of a plane representation of a volumetric shape completely solved, since we can always project a point (or the distance between two points) on a table from the coordinates of the original.
	
	The term $\Delta$ is commonly named the "\NewTerm{focal length}\index{focal length}" from the point of view to the table and optical specialists usually denote it with the letter $f$.

	To understand this result, we can put ourselves in the context of a two-dimensional study where the observer is at the height of the ground line ($h = 0$) and arranged to represent a person looking at the table (comparable to a TV screen, computer screen or other screen) in which we ask the conventional $x$ and $y$ axes in the screen plane and the $z$ axes perpendicular to it (thus relative to the last figure, $Y$ becomes $Z$ and vice versa).

	Thus, the above relation of the ratios:
	
	becomes with this change of axis:
	
	and as $h = 0$ (which is often the case in front of screens):
	
	\begin{tcolorbox}[title=Remark,colframe=black,arc=10pt]
	This relation is a special form of what we name the "\NewTerm{homographic transformations}\index{homographic transformations}". We will come back on the latter further below and prove some of their properties.
	\end{tcolorbox}
	If the table is placed on the axis of reference (the projection table is assimilated to the screen), then we have $z '= 0$ which gives us:
	
	and by proceeding in the same way:
	
	\begin{tcolorbox}[title=Remark,colframe=black,arc=10pt]
	The last two relations are those that we use to make 3D animations programmed in the Macromedia Flash 6.0 (see example further below) or in Javascript (see example also further below) on in any other programming language.
	\end{tcolorbox}
	
	
	We see with the last two relation and identical term:
	
	this term corresponds to the "\NewTerm{depth}\index{depth}" of the perspective.

	In some works, this depth is denoted by (simply factorization):
	
	If we consider two points ($x_1,x_2$ or $y_1,y_2$) visible from the surface of a volume seen by an observer and their respective distance $\Delta x$ or $\Delta y$, these quantities are preserved if the two points coincide in the plane of the table because we have then:
	
	as $z=0,\forall \Delta$.

	It is interesting to study what should be the value of the focal length for $z\neq 0$ to have $\Delta x'=\Delta x$ or $\Delta y'=\Delta y$. Thus, if we take the limit:
	
	by applying the rule of the Hospital (derivative of the numerator and denominator as seen in the section of Differential and Integral Calculus) and remembering that $z$ is fixed, then:
	
	From this result we can conclude the following:
	
	For any real distance between two points that do not coincide in the table but located in a same plane have an equal projection distance, it is necessary that the equations of the two lines that determine their intersection with table are parallel. This implies, since the observer is a convergent point, that we have to take away the observer to an infinite distance from the plane to keep the quantities projected on the table: this is the "\NewTerm{orthogonal parallel projection}\index{orthogonal parallel projection}" also sometimes named "\NewTerm{orthographic parallel projection}\index{orthographic parallel projection}".
	
	A very good example to see these results is to program in pseudo-3D on a computer.
	\begin{tcolorbox}[colframe=black,colback=white,sharp corners]
	\textbf{{\Large \ding{45}}Example:}\\\\
	There are many ways to do pseudo 3D with computers. The best known are technically OpenGL or DirectX and C++ but are not very easy to introduce ... so we'll see how to turn a pseudo-sphere in the projective space with Macromedia Flash 6.0 in order to show how apply the different theoretical elements presented above, but also to show that these are not the only tools available.\\
	
	For this purpose, open the Macromedia Flash 6.0 software and save the new animation with the name \textit{Circle.fla}:
	\begin{figure}[H]
		\centering
		\includegraphics{img/geometry/mm_flash_gui.jpg}
		\caption[]{Macromedia Flash 6.0 GUI}
	\end{figure}
	With the \textbf{Circle} tool in the \textbf{Drawing} toolbar, draw a respectable size disc in the animation area:
	\begin{figure}[H]
		\centering
		\includegraphics[scale=0.8]{img/geometry/mm_flash_disc.jpg}
		\caption[]{Disc in the animation area}
	\end{figure}
	\end{tcolorbox}
	\begin{tcolorbox}[colframe=black,colback=white,sharp corners]
	After having selected the circle with the \textbf{Fill} tool \includegraphics{img/geometry/mm_flash_fillin_tool.jpg} choose a Radial Gradient:
	\begin{figure}[H]
		\centering
		\includegraphics{img/geometry/mm_flash_gradient_color.jpg}
		\caption[]{Color palette to fill-in the circle}
	\end{figure}
	Right-click on the circle and choose the option \textbf{Convert to Symbol} and enter the information as presented below:
	\begin{figure}[H]
		\centering
		\includegraphics{img/geometry/mm_flash_convert_to_symbol.jpg}
		\caption[]{Conversion dialogue box from of Object to Symbol}
	\end{figure}
	Rename the layer where lies your animation clip with the name \textit{3d clip}:
	\begin{figure}[H]
		\centering
		\includegraphics{img/geometry/mm_flash_rename_layer.jpg}
		\caption[]{Renaming the animation layer}
	\end{figure}
	Afterwards Double-click on your circle to enter your animation Clip.\\
	
	There select the circle again, right-click it and select \textbf{Convert to Symbol}:
	\begin{figure}[H]
		\centering
		\includegraphics{img/geometry/mm_flash_rename_layer.jpg}
		\caption[]{Renaming the animation layer}
	\end{figure}
	There select the circle again, right-click it and select \textbf{Convert to Symbol} and enter the information as presented below:
	\end{tcolorbox}
	
	\begin{tcolorbox}[colframe=black,colback=white,sharp corners]
	\begin{figure}[H]
		\centering
		\includegraphics{img/geometry/mm_flash_convert_to_symbol_circle.jpg}
	\end{figure}
	Then, in the properties of the circle there the name \textit{Point}:
	\begin{figure}[H]
		\centering
		\includegraphics{img/geometry/mm_flash_symbole_name.jpg}
	\end{figure}
	Now, in the animation Clip named \textit{Cercle}, we will insert three frames: the first to define the mathematical functions necessary for recalculating the variables, the second calling functions, the third that allows us to loop and go the second frame indefinitely.\\
	
	To make things in an almost clean way, we will create a second layer (by renaming the first one that contains our circle with the name \textit{Cercle}) that we will rename \textit{Code}:
	\begin{figure}[H]
		\centering
		\includegraphics{img/geometry/mm_flash_convert_create_layes.jpg}
	\end{figure}
	Do a right-click on the third image of the layer containing our circle:
	\begin{figure}[H]
		\centering
		\includegraphics{img/geometry/mm_flash_insert_first_frame.jpg}
	\end{figure}
	and choose the option \textbf{Insert Frame} and for the layer \textit{Code} made almost the same, but by choosing \textbf{Insert Key Frame}. You should then get the following display:
	\end{tcolorbox}
	
	\begin{tcolorbox}[colframe=black,colback=white,sharp corners]
	\begin{figure}[H]
		\centering
		\includegraphics{img/geometry/mm_flash_frame_and_keyframe.jpg}
	\end{figure}
	Then by selecting the first image of the layer \textit{Code} active the display of the Action to insert the following code:
	\begin{figure}[H]
		\centering
		\includegraphics{img/geometry/mm_flash_action_script_01.jpg}
	\end{figure}	
	Repeat the same with the second image of the layer \textit{Code}, but putting this time inside it:
	\begin{figure}[H]
		\includegraphics{img/geometry/mm_flash_action_script_02.jpg}
	\end{figure}
	and finally doing the same with the third image of the same layer, but by putting inside it:
	\begin{figure}[H]
		\includegraphics{img/geometry/mm_flash_action_script_03.jpg}
	\end{figure}
	We then get the following animated result (the animation is only visible on a PDF reader with Adobe Flash installed and activated otherwise go see here \url{https://vimeo.com/578782166}):	
	\end{tcolorbox}
	
	\begin{tcolorbox}[colframe=black,colback=white,sharp corners]
	\begin{center}
	\centering
		\includemedia[activate=pageopen,width=360pt,height=360pt,
	]{}{swf/Cercle1.swf}
	\end{center}
	We will now make appear the $z$-axis by fixing $y$ (and therefore by moving $z$). Obviously, we will not see anything happen with regard to $z$ until we define the homographic projection and this because a computer is unable to show basically the concept of depth... The code is then written:
	\begin{figure}[H]
		\centering
		\includegraphics{img/geometry/mm_flash_action_script_04.jpg}
	\end{figure}
	\end{tcolorbox}
	
	\begin{tcolorbox}[colframe=black,colback=white,sharp corners]
	The calculation of \texttt{Zpos} will enable us further below to calculate the depth of the movement of the object along the $z$-axis and this is where the homographic projection will play a role.\\
	
	We then get a pseudo-sphere which rotates around an axis in a plane perpendicular to the screen. This is why we see the pseudo-sphere make left / right trips (the concept of distance is not yet present because of a lack of presence of the depth factor):
	\begin{center}
	\centering
		\includemedia[activate=pageopen,width=300pt,height=300pt,
	]{}{swf/Cercle2.swf}
	\end{center}
	The animation above will run for people having a PDF reader with Adobe Flash player installed and activated (otherwise see here: \url{https://vimeo.com/578782447}).\\
	
	Now we will use the relations:
	
	proved earlier. If we seek to represent the depth of any point of the projection table, it is the distance ratio between two points in the table that will interest us to determine the scaling factor:
	
	\end{tcolorbox}
	
	\begin{tcolorbox}[colframe=black,colback=white,sharp corners]
	We will have therefore to apply this result as defining the scale of the projection table.\\

	We then have the following code where the depth $P$ plays on the height and width of the animated surface of the instance \textit{Point} of our pseudo-sphere:
	\begin{figure}[H]
		\centering
		\includegraphics{img/geometry/mm_flash_action_script_05.jpg}
	\end{figure}
	Which gives us:
	\begin{center}
	\centering
		\includemedia[activate=pageopen,width=245pt,height=245pt,
	]{}{swf/Cercle3.swf}
	\end{center}
	\end{tcolorbox}
	
	\begin{tcolorbox}[colframe=black,colback=white,sharp corners]
	and obviously it works regardless of the parametric equation of the path followed! We can then copy this instance animation, change the height, the starting angle for the get the $4$ vertices of a cube rotating in space.\\
	
	The animation above will run for people having a PDF reader with Adobe Flash player installed and activated (otherwise see here: \url{https://vimeo.com/578783079}).\\

	This is the next step:\\

	Indeed, let us change our code as shown below for $4$ pseudo-spheres rotating around an imaginary $z$-axis going out of the screen (we use the rotation matrices proved in the section of Euclidean Geometry) and sorry for the comments in French as I did first the code for french readers:\\
	
	\includegraphics{img/geometry/mm_flash_action_script_06.jpg}
	\includegraphics{img/geometry/mm_flash_action_script_07.jpg}\\
	
	This gives us ... four charming pseudo-spheres turning around a common center (the animation below will run for people having a PDF reader with Adobe Flash player installed and activated otherwise see here: \url{https://vimeo.com/578783905}):
	\end{tcolorbox}
	
	\pagebreak
	\begin{tcolorbox}[colframe=black,colback=white,sharp corners]
	\begin{center}
	\centering
		\includemedia[activate=pageopen,width=180pt,height=180pt,
	]{}{swf/Cercle4.swf}
	\end{center}
	Now we reverse $y$ and $z$ again and apply the homographic projection:\\
	
	\includegraphics{img/geometry/mm_flash_action_script_08.jpg}
	\includegraphics{img/geometry/mm_flash_action_script_09.jpg}
	\end{tcolorbox}
	
	
	\begin{tcolorbox}[colframe=black,colback=white,sharp corners]
	and we get (the animation below will run for people having a PDF reader with Adobe Flash player installed and activated otherwise see here: \url{https://vimeo.com/578784812}):
	\begin{center}
	\centering
		\includemedia[activate=pageopen,width=350pt,height=350pt,
	]{}{swf/Cercle5.swf}
	\end{center}
	For the remaining part, we will generate $8$ pseudo-sphere and instead to turn them always about the same axis, we will rotate around three axes $x$, $y$ or $z$ using the variables \texttt{XAngle}, \texttt{YAngle} or \texttt{ZAngle} and rotation matrices about each of these respective axes:\\
	
	\includegraphics{img/geometry/mm_flash_action_script_10.jpg}
	\end{tcolorbox}
	
	\begin{tcolorbox}[colframe=black,colback=white,sharp corners]
	\includegraphics{img/geometry/mm_flash_action_script_11.jpg}
	\includegraphics{img/geometry/mm_flash_action_script_12.jpg}
	
	and here is the final result (the animation below will run for people having a PDF reader with Adobe Flash player installed and activated otherwise see here: \url{https://vimeo.com/578785768}):
	\begin{center}
	\centering
		\includemedia[activate=pageopen,width=213pt,height=213pt,
	]{}{swf/Cercle6.swf}
	\end{center}
	\end{tcolorbox}
	Since the Adobe Flash technology is virtually dead since the time I wrote the code above... an anonymous reader rewrote the above code again into WebGL and here is the corresponding code (if you want to give the same example in other languages don't hesitate):
	
	\includegraphics{img/geometry/webgl_pseudo_sphere_01.jpg}
	
	\includegraphics{img/geometry/webgl_pseudo_sphere_02.jpg}
	
	\includegraphics{img/geometry/webgl_pseudo_sphere_03.jpg}
	
	\includegraphics{img/geometry/webgl_pseudo_sphere_04.jpg}
	
	\includegraphics{img/geometry/webgl_pseudo_sphere_05.jpg}
	
	To see this code work or copy it you can go visit the following URL:
	\begin{center}
	\url{http://www.sciences.ch/htmlen/cube.htm}
	\end{center}
	
	\subsubsection{Images of Straight Lines}
	Let us determine from the previous results, the image of a straight line parallel to the ground line (hence the $x$-axis) of the table $XY$. In this case, we have:
	
	where $a$ and $b$ are constants, and for any value of $h$. This gives us from the relations obtained previously:
	
	So as we could expect it, a straight line parallel to the ground line becomes in perspective still a straight line but at a height $y'$ in the $XY$ plane parallel to our screen (it was quite intuitive to guess...).

	For any line parallel to the $z$-axis of the screen (and therefore in its "depth"), we have:
	
	where $a$ and $b$ are constants for any value of $h$. Which gives us:
	
	The straight lines of equation:
	
	all pass through the point $P(x'=0,y'=h)$ when $z\rightarrow +\infty$ which is the "main vanishing point" and through the point $P(x'=a,y'=b)$ when $z\rightarrow  0$ as shown below in  the projection made in Adobe Photoshop (the horizontal lines corresponding to the ground live have been added to give the effect of perspective):
	\begin{figure}[H]
		\centering
		\includegraphics{img/geometry/perspective_unique_vanishing_point.jpg}
		\caption{Example of a single vanishing point}
	\end{figure}
	From the figure above, we can define the concept of "\NewTerm{departure angle}\index{departure angle}" given by the figure below:
	\begin{figure}[H]
		\centering
		\includegraphics{img/geometry/departure_angle.jpg}
		\caption{Departure angle}
	\end{figure}
	Another geometric representation can perhaps help to better understand the result. Let us recall that the vanishing point of a line $(D)$ is the intersection point $F$ of the table plane $T$ with the line parallel to $D$ passing through O. Two parallel lines $(D)$ and $(D')$ therefore have the same vanishing point (in the perspective point of view obviously!). As shown below:
	\begin{figure}[H]
		\centering
		\includegraphics{img/geometry/parallel_lines_conical_perspective.jpg}
	\end{figure}
	If we denote by $A$ the intersection point of $(D)$ with $T$, the perspective drawing of $D$ is the straight line $\overline{AF}$, intersection of $T$ with the plane containing $O$ and $D$. Since two parallel lines in a conical perspective point of view have the same vanishing point $F$, they are therefore represented by two intersecting straight lines on $F$.
	
	For any straight line being in the $XZ$ plane of the screen (i.e. in its "depth"), we have:
	
	and for $y=0$. Which gives us:
	
	From the latter equation we deduce:
	
	by putting $z$ in the expression of $x$:
	
	we replace $x$ and $y$ in the equation of the straight line $z=mx+p$, and once calculations and simplifications made, we find:
	
	Let us consider the particular case of straight lines passing through the opposite vertices of a tile, that is to say inclined at $\pm\pi/4$ therefore with a director coefficient (slope) of $\pm 1$.
	The image of these straight lines are then given by:
	
	If $y'=h$, then we have following the previous relation:
	
	this mean that any projection of lines of director coefficient $\pm 1$ lying in the $xy$-plane for all $x'$ constitutes secondary vanishing points situated on the same remote horizon line at an equal distance of the main vanishing point as shown below (in the context of a on Adobe Photoshop training, we added a cube is in this perspective):
	\begin{figure}[H]
		\centering
		\includegraphics{img/geometry/secondary_vanishing_point.jpg}
		\caption{Example of secondary vanishing point}
	\end{figure}
	We see well in the figure above the symmetry about the vertical axis and the two vanishing point that are causing the tiles.

	Now let us consider the straight lines parallel to the $y$-axis of the table (display) and their projection equation if $x=a$ and $z=b$:
	
	These right images thus remain straight lines parallel to the $y$ axis. In other words, the right images remain parallel to straight "object" as shown below (for different positions of the observation point):
	\begin{figure}[H]
		\centering
		\includegraphics{img/geometry/perspective_vertical_lines.jpg}
		\caption{Different perspective based on the values of the constants}
	\end{figure}
	Let us take for example of this last result, segment of height $Hi$ whose footer is on the horizontal plane (ground):
	
	The height of the segment is given by:
	
	Let us consider now $x=a$, that is to say the vertical columns of height $H$ are aligned on the line $x=a$:
	\begin{figure}[H]
		\centering
		\includegraphics{img/geometry/perspective_columns.jpg}
		\caption{Aligned columns perspective problem}
	\end{figure}
	Let us calculate the coordinates of the vertices of these images:
	
	This is obviously still the equation of a straight line and let us notice that all these line pass trough the point of coordinates $(x'=0,y'=h)$ that is just the main vanishing point!
	
	So as we experiment in real life, if we consider the above figure corresponding to the previous developments when $H<h$ the columns seems to have their height that decrease as the are more far from the observer and if $h>H$ the columns height seems to increase.
	
	From the previous discussion, we have therefore also obviously a new method for rotating object in three dimensional space. Instead of rotating the object around different axes, we can imagine using the above equations to rotate the viewer around the object (that is a point of view...).

	We have considered so far only the projective perspective on a plane. By the way, to work on any projection methods (sphere on a plane, plane on a sphere, sphere on sphere, anything about anything) it is simply enough to extend the analysis that we made above in a coordinate system suitable for the studied system (polar coordinates, cylindrical coordinates, spherical coordinates, ...). This is probably that way that some 3D simulation software use to project an image on a reflective surface such as a semi-transparent undulated glass at the difference that on computer a surface don't have an infinite number of points but a limited number of pixels and therefore the projection need some smoothing trough special interpolation techniques.
	
	\begin{tcolorbox}[title=Remark,colframe=black,arc=10pt]
	From the results we have obtained above, we can extract an interesting and intuitive conclusion For an observer of a picture or a table seethe image as it was originally, it must be placed at specific coordinates of the table (of the picture or of the table).
	\end{tcolorbox}
	Software such as Adobe Illustrator offers since the early 2010s a tool to create perspectives for one vanishing point, for two vanishing points and three vanishing points:
	\begin{figure}[H]
		\centering
		\includegraphics{img/geometry/adobe_illustrator_one_vanishing_point.jpg}
	\end{figure}
	\begin{figure}[H]
		\centering
		\includegraphics{img/geometry/adobe_illustrator_two_vanishing_point.jpg}
	\end{figure}
	\begin{figure}[H]
		\centering
		\includegraphics{img/geometry/adobe_illustrator_three_vanishing_point.jpg}
		\caption{Aligned columns perspective problem}
	\end{figure}
	And to finish a wink for all our favourite flat earthers (...):
	\begin{figure}[H]
		\centering
		\includegraphics[width=1\textwidth]{img/geometry/vanishing_point_flat_earth.jpg}
	\end{figure}
	

	\pagebreak
	\subsection{Affine projections}
	Although the best method of perspective representation is the conical perspective method, we can not always tolerate to do a large quantity of computation to represent a volume. Thus, it is possible to define projection techniques that result from an approximation of the mathematical results obtained previously to get two new techniques (there are more but these two perspectives we will see further below are by far the most used) which we see daily on many technical  papers or artistic works. These two techniques are respectively the "\NewTerm{isometric projection}\index{isometric projection}" and "\NewTerm{orthogonal projection}\index{orthogonal projection}" that are part of the family of  "\NewTerm{affine projections}\index{affine projection}" (in fact the use of the word "projection" must understand "projection on the Table of the observer" and therefore it is also a perspective).
	\begin{figure}[H]
		\centering
		\includegraphics{img/geometry/conical_perspective_vs_isometric_projection.jpg}
	\end{figure}
	An example of projective transformation that is not affine is photography through a Camera:
	 
	\begin{figure}[H]
		\centering
		\includegraphics{img/geometry/not_affine_camera.jpg}
	\end{figure}
	Like all affine transformations of space, an affine projection preserves:
	\begin{itemize}
		\item The parallelism between the lines

		\item The barycentre, that is all the proportions existing on a given line
	\end{itemize}
		Only the lengths and angles within a parallel plane to the projection plane are kept.
	
	Let us we present briefly these two techniques because they must be part of the general culture of the engineer.
	
	Like all projections and all perspective, the loss of the third dimension and the non-respect of real mathematical transformation induces possible errors of interpretation. This has been extensively used by the artist M. C. Escher to create impossible situations:
	\begin{figure}[H]
		\centering
		\includegraphics[scale=0.4]{img/geometry/mc_escher_waterfall.jpg}
	\end{figure}
	\textbf{Definition (\#\mydef):} An "\NewTerm{affine projection}\index{affine projection}" of the usual 3-dimensional space is an affine transformation that sends all the points of this space on a same plane of this space. If the point $M (x, y, z)$ is not on the projection plane, he and its image $m (x ', y', z ')$ form a line whose direction is constant: we name it the "\NewTerm{projection direction}\index{projection direction}". The corresponding perspective is named "\NewTerm{parallel perspective}\index{parallel perspective}" or "\NewTerm{cylindrical perspective}\index{cylindrical perspective}".
	
	
	\pagebreak
	\subsubsection{Isometric perspective}
	Isometric projection is a method for visually representing three-dimensional objects in two dimensions in technical and engineering drawings. It is a parallel projection (object is rotated along one or more of its axes relative to the plane of projection) in which the three coordinate axes appear equally foreshortened (hence the "iso") and the angle between any two of them is approximated to $120$ degrees ($2\pi/3$ [rad]):
	\begin{figure}[H]
		\centering
		\includegraphics[scale=0.5]{img/geometry/isometric_angles.jpg}
		\caption[Isometric angles]{Isometric angles (source: Wikipedia)}
	\end{figure}
	Indeed, as we will see during our study of Analytical Geometry, the angle of $30^\circ$ above is only an approximation of the angle between the $xy$-plane and the (isometric) plane vertical to the vector $\vec{n}=(1,1,1)$ passing through the corner of a cube of equal sides of length $1$. In fact the choice of the angle is empirical and for example in computer games it varies between $30^\circ$ and $60^\circ$ depending on the game editor.
		
	Let us consider some 3D shapes using the isometric drawing method. In the figure below coming from Wikipedia (year 2016) the black dimensions are supposed true lengths as found in an orthographic projection (see further below). The red dimensions are used when drawing with the isometric drawing method. The same 3D shapes drawn in isometric projection would appear smaller; an isometric projection will show the object's sides foreshortened, by approximately $80\%$.
	
	In fact we will see that the figure below coming from Wikipedia (year 2016) is in fact quite wrong or at least can bring the reader to some misunderstanding!!!!
	
	\begin{figure}[H]
		\centering
		\includegraphics[scale=0.7]{img/geometry/isometric_projection_volumes.jpg}
		\caption[Isometric volumes]{Isometric volumes (source: Wikipedia)}
	\end{figure}
	Ok let us see closer to the figure above. To prove where this values comes from we need to use the change of basis mathematics tools (\SeeChapter{see section Linear Algebra page \pageref{change of basis}}). This will be a veeery nice application of Linear Algebra and Vector Calculus!!!
	
	So let us consider the following situation that we already used in the section of Analytical Geometry page \pageref{isometric plane}:
	\begin{figure}[H]
		\centering
		\includegraphics{img/geometry/isometric_perspective_plane.jpg}
		\caption{Isometric plane to find the cube dimensions}
	\end{figure}	
	To simplify the determination of the new basis in the plane we well make pass our plane trough the origin (isometric affine plane) such that its equation becomes therefore:
	

	Obviously as our plane is perpendicular to $\vec{n}=(1,1,1)$ it comes immediately that a first basis vector of our orthonormal 2D basis will be the vector:
	
	We can easily control that this point belongs well to our isometric affine plane of equation:
	
	by just replacing the values and see that the equality holds.

		Now we are looking for the second orthonormal vector of our plane basis. So we first simply use the cross product:
	
	We can also easily control that this point belongs well to our isometric affine plane.
	
	And we must normalize the to the unit so finally:\\
	
	Now we calculate the images of the vectors of the canonical basis $(\vec{e}_1,\vec{e}_2,\vec{e}_3)$ by the orthogonal projection on our isometric affine plane $P$:
	
	Finally the projection matrix in the canonical basis is:
	
	Ok now let us consider the projection of the origin $(0,0,0)$ we then have obviously:
	
	and now one of its $X$ or $Y$ unit vector :
	
	So we get for the norm of the distance of a unit vector of the original base of the $XY$-plane:
	
	And finally the unit vector along the $z$-axis:
	
	So we get for the norm of the distance of a unit vector of the original base along the $Z$-axis:
	
	
	\pagebreak
	So now let us come back on Wikipedia's figure:
	\begin{itemize}
		\item First the cube... 
		\begin{figure}[H]
			\centering
			\includegraphics[scale=0.75]{img/geometry/isometric_cube.jpg}
		\end{figure}
		So the real sides of a cube of side length equal to $1$ transform same as the basis vector as seen just before. That is to say an isometric transformation through $Z$ gives for the sides in or parallel to the $XY$ plane:
		
		and along the $Z$-axis:
		
		So when Wikipedia says that the distances are shortened this is true... but they omit to say that this is only in two space directions. Also when on the figure they show that $1$ in real dimensions is equal to $1$ for the sides in isometric perspective is completely false.
		
		The distance between the two horizontal top vertices is given by the calculation of the norm of the projection:
		
		So:
		
		So the norm is:
		
		And therefore the original diagonal transforms to:
		
		So once again Wikipedia is wrong. In fact on Wikipedia they take for original sides size the inverse of $\sqrt{2/3}$, that is to say $\sqrt{3/2}$. Indeed we have in  this case:
		
		So the norm is:
		
		So we fall back on the value given by Wikipedia...
		
		\item Second the cylinder...
		\begin{figure}[H]
			\centering
			\includegraphics[scale=0.75]{img/geometry/isometric_cylinder.jpg}
		\end{figure}
		The height of the cylinder follows exactly the same procedure as for the cube and the remark relatively to Wikipedia's figure remains the same! 
		
		So what will interest us here is especially the "isometric ellipse" (also named "isocircle") visible in the figure above and we will focus on the minor and major axes of that latter. 
		
		First, to find these values we need to project the circle on the isometric plane. The figure (where we added a circle on the cube) is given now by (\SeeChapter{see section Analytical Geometry page \pageref{isometric plane}}):
		\begin{figure}[H]
			\centering
			\includegraphics{img/geometry/isometric_plane_isocircle.jpg}
			\caption{Isometric plane to find the isocircle dimensions}
		\end{figure}
		\StickyNote[2.5cm]{\LARGE To finish depending on donations}[6.5cm]
		
	\end{itemize}
	
	\subsubsection{Oblique perspective}
	Let us take a view (eg the front view), and let us name the axes "$x$" (horizontal) and $y$ (vertical) as usual. The $z$-axis being the axis perpendicular to the view (also as usual).

	In isometric projection, we plot the $z$-axis with an angle relative to the $x$ axis (for example $\pi/6$ ($30^\circ$) or sometimes $\pi/4$ ($45^\circ$)), and we report then the distances by multiplying by a coefficient less than $1$ that is empirical on by using trigonometric basic rules such as.
	\begin{tcolorbox}[title=Remark,colframe=black,arc=10pt]
	The oblique projections for all angles other than $\pi/4$ are named "\NewTerm{Cavalier projection}\index{Cavalier projection}". When the angle is equal to $\pi/4$ we then speak of "\NewTerm{Cabinet projection}\index{Cabinet projection}".	
	\end{tcolorbox}
	
	Oblique projection is a simple type of technical drawing of graphical projection used for producing two-dimensional images of three-dimensional objects. The objects are not in perspective, so they do not correspond to any view of an object that can be obtained in practice, but the technique does yield somewhat convincing and useful images:
	\begin{figure}[H]
		\centering
		\includegraphics{img/geometry/oblique_pictorial.jpg}
	\end{figure}

	In the oblique pictorials coordinate system only one axes is at an angle. The angle may range between $0$ and $90$ degrees ($\pi/2$ [rad]); however, the most commonly used angle is $45$ degrees.

 	It is thus often used when a figure must be drawn by hand, e.g. on a black board (lesson, oral examination).
 	
 	\pagebreak
 	\subsubsection{Orthogonal projection}
	If the projection direction is orthogonal to the plane of projection, then the perspective becomes transforms a sphere into a simple circle. This is a type of perspective drawing used as an alternative to the conical perspective (with which it will also coincide when the eye of the observer is placed infinitely far away from the "table").

	The simplest orthogonal projection to express is obviously the one that sends space on a plane parallel to the table, this is the "\NewTerm{parallel orthogonal projection}\index{parallel orthogonal projection}" or "\NewTerm{orthographic projection}\index{orthographic projection}", of distance $d$. In other words, such that for example for the "above view" we have for any point of the volume:
	
	So we trivially obtain all the coordinates in an orthonormal 2D frame proper to this plane itself where $x=x'$ and $y=y'$:
	\begin{figure}[H]
		\centering
		\includegraphics[scale=0.8]{img/geometry/orthographic_view.jpg}
		\caption{Orthogonal projection in real life...}
	\end{figure}
	Nowadays, softwares like AutoCAD, Inventor, Cathia and so on... generate automatically orthogonal projections from an isometric view of a drawing.
	\begin{figure}[H]
		\centering
		\includegraphics{img/geometry/technical_drawing_perspective.jpg}
		\caption[]{Orthogonal Projection with a software}
	\end{figure}
	In computer graphics, one of the most common matrices used for orthographic projection can be defined by a $6$-tuple, (left, right, bottom, top, near, far), which defines the clipping planes (that is, the region of eye space that contains all the geometry you want to display). These planes form a box with the minimum corner at (left, bottom, -near) and the maximum corner at (right, top, -far). The box is then translated so that its center is at the origin, then it is scaled to the unit cube which is defined by having a minimum corner at $(-1,-1,-1)$ and a maximum corner at $(1,1,1)$.

	The orthographic transform to get the canonical can be given by the following matrix:
	
	which is in fact obviously a scaling followed by a translation of the form (the positions are abbreviated now with their first letter):
	
	As visible illustrated in the figure below:
	\begin{figure}[H]
		\centering
		\includegraphics{img/geometry/canonical_view_orthogonal_projection.jpg}
		\caption[]{Canonical transformation for orthogonal projection}
	\end{figure}
	If the origin of the matrix $C_V$ is not obvious here are the details to get it:
	
	The easiest approach may be to consider each of the three axes separately, and compute how to map points along that axis from the original view volume into the canonical view volume. We begin with the $x$-coordinate. A point within our view volume will have an $x$-coordinate on the range $[l, r]$, and we want to transform it to the range $[-1, 1]$!

	So first we see that we have:
	
	Now, in preparation to scale the range down to the size we want, we subtract $l$ from all terms to produce a zero on the left-hand side. Another approach we could take here would be to translate the range so that it centers on zero, rather than having one of its endpoints at zero, but the algebra is a bit neater this way, so I'll do it like this for the sake of readability:
	
	Now that one end of your range is positioned at zero, we can scale it down to the size we want. We want the range of $x$-values to be two units wide, from $1$ to $-1$, so we multiply through by $2/(r - l)$. Notice that $r - l$ is the width of our view volume and is thus always a positive number, so we don't have to worry about the inequalities changing directions:
	
	Next, we subtract one from all terms to produce our desired range of $[-1, 1]$:
	
	A bit of basic algebra allows us to write the center term as a single fraction:
	
	Finally, we split the center term into two fractions so that it takes the form $px + q$. For this we need to group our terms this way so that the equations we derive can be easily translated into matrix form:
	
	The center term of this inequality now gives us the equation we need to transform $x$ into the canonical view volume:
	
	The steps required to obtain a formula for $y$ are exactly the same. We just have to substitute $y$ for $x$, $t$ for $r$, and $b$ for $l$. So rather than repeat them here, we will just show the result:
	
	Finally, we need to derive a relation for $z$. It's a little different in this case because we are mapping $z$ to the range $[0, 1]$ rather than $[-1, 1]$, but this should look very familiar. Here our starting condition, a $z$-coordinate on the range $[n, f]$:
	
	We subtract $n$ from all terms so the lower end of the range is positioned at zero:
	
	And now, all that's left is to divide through by $f - n$ to produce a final range of $[0, 1]$. As before, we notice that $f - n$ indicates the depth of our viewing volume and thus will never be negative:
	
	Finally, we split this into two fractions so it takes the form $pz + q$:
	
	This gives us our formula for transforming $z$:
	
	Now, we are ready to write our centering orthographic projection matrix. To recap our work thus far, here are the three projection equations we have derived:
	
	Once the canonical volume obtained we apply for example the simple orthographic projection onto the plane $z = 0$ that is intuitively defined by the following matrix:
	
	For each point $(x, y, z)$, the transformed point would be:
	
	Often, it is more useful to use homogeneous coordinates. The transformation above can be represented for homogeneous coordinates as
	
	For each homogeneous vector$(x,y,z)$, the transformed vector will be obviously:
	
	Although this is the most used method in the industry on paper, however, it has a drawback: the real effect of depth is lost entirely this is what orthogonal projection is many times is often accompanied of an isometric projection view of the object of interest.
	
	\pagebreak
	\subsection{Spherical projections}
	Spherical projections\index{spherical projections} is an important field in engineering topography and geography (and therefore navigation). As far as we know there is not far from more than $200$ various techniques of spherical projection. 
	
	Indeed, it is impossible to render paths on a sphere onto a flat surface in such a way that all distances remain the same. In drawing a map of a sphere, therefore, some compromises must be made. Most maps adopt one of two possible strategies that areas are preserved or that angles are preserved. Stereographic projection (ie. projection on a plane) is one way of making maps, and it preserves angles. It has been used since ancient times for this purpose, and its basic geometrical properties were known even then.

	We will focus in this book only on the most known spherical projections and also without detailing the problems of projecting images from a sphere to a plane when the number of points (pixels) is not infinite...
	
	For example here is a summary of some well known spherical projections\footnote{the choice of the projection used in school depends mainly on the country government as some projections give more importance to some countries than others... But anyway any student can now see a Earth Sphere in its class and this is the only $1:1$ existing projection!}:
	\begin{figure}[H]
		\centering
		\includegraphics[width=0.53\textwidth]{img/geometry/spherical_projections.jpg}
		\caption[Various spherical projections]{Various spherical projections like \\ Mercator, Gall-Peters, Mollweide, Robinson, etc. (author: Mike Bostock)}
	\end{figure}
	Or with a more detailed idea of each technique and its corresponding name:
	\begin{figure}[H]
		\centering
		\includegraphics[scale=0.45]{img/geometry/projections_list.jpg}
		\caption{Technical view of various spherical projections techniques}
	\end{figure}
	\begin{tcolorbox}[title=Remark,colframe=black,arc=10pt]
	An exhaustive list of spherical projections is given on the projection list web page of the Geocart software available here: 	
	\begin{center}
	\url{https://www.mapthematics.com/ProjectionsList.php}
	\end{center}
	\end{tcolorbox}
	
	\subsubsection{Stereographic projection}
	The "\NewTerm{stereographic projection}\index{stereographic projection}" (in the classical sense of the term), or more precisely the "\NewTerm{azimuthal normal aspect projection}\index{azimuthal normal aspect projection}", is one way of projecting the points that lie on a spherical surface onto a plane. Such projections are commonly used in Earth and space mapping where the geometry is often inherently spherical and needs to be displayed on a flat surface such as paper or a computer display. Any attempt to map a sphere onto a plane requires distortion, stereographic projections are no exception and indeed it is not an ideal approach if minimal distortion is desired. 

	A physical model of stereographic projections is to imagine a transparent sphere sitting on a plane. If we name the point at which  the sphere touches the plane the "south pole" then we place a light source at the "north pole". Each ray from the light passes through a point on the and then strikes the plane, this is the stereographic projection of the point on the sphere. 
	
	In order to derive the formula for the projection of a point $(x,y,z)$ lying on the sphere assume the sphere is centered at the origin and is of radius $r$. The plane is all the points $z = -r$, and the light source is at point $(0,0,r)$. The cross section of this arrangement is shown below in what is commonly named a "\NewTerm{Schlegel diagram}\index{Schlegel diagram}" that looks like below in pseudo-perspective:
	\begin{figure}[H]
		\centering
		\includegraphics{img/geometry/schlegal_diagram_pseudo_perspective.jpg}
		\caption{Schlegel diagram in pseuod-perspective}
	\end{figure}
	and in front view:
	\begin{figure}[H]
		\centering
		\includegraphics{img/geometry/schlegal_diagram_pseudo_front_view.jpg}
		\caption[]{Schlegel diagram in front view}
	\end{figure}
	As we can see above the projection of $(0,0,r)$ gives obviously $(0,0,-r)$ and using Thales's theorem (or "intercept theorem") as proved in the section of Euclidean Geometry we have immediately that the projection of $(-r,0,0)$ is $(-2r,0,r)$.
	
	Let us consider now the equation of the line from $P_1=(0,0,r)$ through a point $P_2=(x,y,z)$ on the sphere. Therefore we have for the projected point $P$:
	
	Solving this for $a$ for the $z$ component yields:
	
	or:
	
	Finally:
	
	Notice that:
	\begin{itemize}
		\item The South pole is at the center of the projected points

		\item Lines of latitude project to concentric circles about $(0,0,-r)$

		\item Lines of longitude project to rays from the point $(0,0,-r)$

		\item There is little distortion near the South pole

		\item The equation project to a circle of radius $2r$

		\item The distortion increases the closer one gets to the north pole finally becoming infinite at the north pole
		
		\item As we can see in the image below, angles are preserved (since latitude and longitude are still perpendicular on the projected plane)
	\end{itemize}
	\begin{figure}[H]
		\centering
		\includegraphics{img/geometry/earth_stereographic_projection.jpg}
		\caption[Earth's stereographic projection]{Earth's stereographic projection (source: ?)}
	\end{figure}
	Some author put the origin of the frame at the center of the sphere. Therefore the previous relations becomes:
	
	and take for radius $r=1$. Then we have:
	
	And as the denote the components project points by $a,b,c$ instead of $x,y,z$, we get:
	
	And as the last coordinate is always $0$ we only write:
	
	And when the projection plane is assimilated to the (complex) projection plane, then we get the relation that we can found most of time in books:
	
	\begin{theorem}
	The stereographic projection is a conformal projection\index{conformal projection} (ie. conserves the angles). That is to say:
	
	If we consider the stereographic projection $\pi:S^2 \setminus\{\vec{e}_3\}\in\mathbb{R}^3$ as we have proved it, defined by:
	
	with $x^2+y^2+z^2=1$ and $\vec{e}_3=(0,0,1)^T$.

	Given two paths $\varphi$ and $\gamma$ of type $\mathcal{C}^1$ on the sphere (that is to say two differentiable applications in which the derivative is continuous for recall...) such as (empirical choice):
	
	and such that $\varphi(0)=\gamma(0)\neq \vec{e}_3$, we want to check if (we use the bracket notation for the dot product to not confuse it with the round symbol of function composition):
	
	where $\varphi'(0)$ and $\gamma'(0)$ are the tangent vectors of $\varphi$ and $\gamma$ on point $0$ and $(\pi\circ\varphi)'(0)$ and $(\pi\circ\gamma)'(0)$ are there corresponding tangent vectors on the projection plane.
	\end{theorem}
	\begin{dem}
	To do the proof let start by developing the expressions $(\pi\circ\varphi)'(0)$ and $(\pi\circ\gamma)'(0)$. Let us write $\varphi=(\varphi_x,\varphi_y,\varphi_z)^T$. We have :
	
	Let us simplify the denominator by putting:
	
	Then we get:
	
	Hence:
	
	And same, by putting:
	
	we get:
	
	Let us now develop $\langle (\phi\circ\varphi)'(0),(\phi\circ\gamma)'(0)\rangle$:
	
	The change of variables take us immediately to:
	
	Therefore:
	
	As by construction:
	
	and that we also have:
	
	and identically:
	
	then the above dot product becomes:
	
	But:
	
	since by construction $\varphi_x^2+\varphi_y^2+\varphi_z^2=1$.
	
	So finally our numerator is given by:
	
	Let us now calculate the first term of the denominator $||(\pi\circ\varphi)'(0)||$:
	
	as for recall (see proof just before) $\langle u,u'\rangle=-{u'}_z$.
	
	But we also have:
	
	Therefore the previous expressions becomes:
	
	Similarly:
	
	To finish, we put:
	
	inside:
	
	and we get:
	
	\begin{flushright}
		$\blacksquare$  Q.E.D.
	\end{flushright}
	\end{dem}
	
	\pagebreak
	\subsubsection{Cylindrical projection}
	The "\NewTerm{cylindrical projection}\index{cylindrical projection}" (in the classical sense of the term) or more precisely named the "\NewTerm{cylindrical normal aspect projection}\index{cylindrical normal aspect projection}", is one where lines of latitude are projected to equally spaced parallel lines and lines of longitude are projected onto not necessarily equally spaced parallel lines. The figure below illustrates the basic projection, a line is projected from the center of the sphere through each point on the sphere until it intersects.
	\begin{figure}[H]
		\centering
		\includegraphics[scale=0.85]{img/geometry/cylindrical_projection_mathematical_aspect.jpg}
	\end{figure}
	As we can see from the above figure it is quite obvious first that:
	
	and for $x'$ we have obviously:
	
	with for recall:
	
	So finally:
	
	Schematically the steps of construction are the following:
	\begin{figure}[H]
		\centering
		\includegraphics{img/geometry/cylindrical_projection_cylinder_closed.jpg}
		\caption{Earth's cylindrical projection}
	\end{figure}
	which results in:
	\begin{figure}[H]
		\centering
		\includegraphics{img/geometry/cylindrical_projection_cylinder_open.jpg}
	\end{figure}
	It should be quite obvious (without proof) that this projection does not conserve the areas, but conserve the angles!
	
	\subsubsection{Mercator projection}
	A "\NewTerm{Mercator projection}\index{Mercator projection}" is similar in appearance to a cylindrical projection but has a different distortion in the spacing of the lines of longitude (the same idea applies for the so named "Gall stereographic projection" or "Gall–Peters projection" that we will not introduce in this book). 

	Like the cylindrical projection North and South are always vertical, we cannot therefore represent the poles because the mathematics have an infinity singularity there.

	The Mercator projection is the most common projection used in mapping the Earth onto a flat surface. 
	
	The result of such a projection is:
	\begin{figure}[H]
		\centering
		\includegraphics[scale=0.14]{img/geometry/mercator_earth_projection.jpg}
		\caption{Earth's Mercator projection}
	\end{figure}
	As we have seen previously, the cylindrical map projection is specified by formula linking the geographic coordinates of latitude $\beta$ and longitude $\alpha$ to Cartesian coordinates on the map with origin on the equator and $x$-axis along the equator. By construction, all points on the same meridian lie on the same generator of the cylinder at a constant value of $x$, but the distance $y$ along the generator (measured from the equator) is an arbitrary function of latitude, $y(\beta)$. In general this function does not describe the geometrical projection (as of light rays onto a screen) from the center of the globe to the cylinder, which is only one of an unlimited number of ways to conceptually project a cylindrical map.
	
	Since the cylinder is tangential to the globe of radius $R$ at the equator, the scale factor between globe and cylinder is obviously unity on the equator but nowhere else. In particular since the radius of a parallel, or circle of latitude, is $R\cos(\beta)$, the corresponding parallel on the map must have been stretched by a factor of:
	
	This scale factor on the parallel is conventionally denoted by $k$ and the corresponding scale factor on the meridian is denoted by $h$.
	
	The relations between $y(\beta)$ and properties of the projection, such as the transformation of angles and the variation in scale, follow from the geometry of corresponding small elements on the globe and map. The figure below shows a point $P$ at latitude $\beta$ and longitude $\alpha$ on the globe and a nearby point $Q$ at latitude $\phi + \delta \phi$ and longitude $\alpha + \delta \alpha$. The vertical segments $\wideparen{PK}$ and $\wideparen{MQ}$ are arcs of meridians of length $R\delta \beta$ (see the proof in the section Trigonometry). The horizontal segment $\wideparen{PM}$ and $\wideparen{KQ}$ are arcs of parallels of length $R\cos(\beta)\delta\alpha$ (still following from the same proof of the section Trigonometry). The corresponding points on the projection define a rectangle of width $\delta x$ and height $\delta y$:
	\begin{figure}[H]
		\centering
		\includegraphics[scale=0.8]{img/geometry/mercator_construction_principle.jpg}
		\caption{Mercator projection function construction}
	\end{figure}
	For small elements, the angle $\widehat{PKQ}$ is approximately a right angle and therefore
	
	The previously mentioned scaling factors from globe to cylinder are given by:
	\begin{itemize}
		\item Parallel scale factor: 
		 

		\item Meridian scale factor:
		 
	\end{itemize}
	Consider now that the projection plane we take has for $x$-coordinates the range $[-\pi R,+\pi R]$ (the whole being then equal to $2\pi R$). Then for every meridian:
	 
	Indeed, it is easy to see that if $(\alpha-\alpha_0)=2\pi$ we have well $x=2\pi r$ when we take time $\alpha_0=0$. But in the general case we have obviously:
	 
	Therefore we can write:
	
	Where we have considered that:
	
	The choice of the function $y(\beta)$ for the Mercator projection is determined by the demand that the projection be conformal, a condition which can be defined as the equality of angles as proved earlier above. The condition that a sailing course of constant azimuth $\lambda$ on the globe is mapped into a constant grid bearing $\gamma$ on the map. Setting $\lambda = \gamma$ in:
	
 	gives immediately after rearranging:
	
	with $y(0) = 0$, by using the primitives proved in the section of Differential and Integral Calculus we have immediately:
	
	So finally:
	
	In the first equation $\alpha_0$ is the longitude of an arbitrary central meridian usually, but not always, that of Greenwich (i.e., zero).
	
	\begin{figure}[H]
		\centering
		\includegraphics[scale=0.5]{img/geometry/mercator_1569.jpg}
		\caption[Mercator world map as seen in 1569]{Mercator world map as seen in 1569... (source: Wikipedia)}
	\end{figure}
	
	\subsubsection{Lambert's equivalent projection (Peters projection)}
	The idea behind the "\NewTerm{Lambert's equivalent projection}\index{Lambert's equivalent projection}", also named "\NewTerm{Lambert's cylindrical equal-area projection}\index{Lambert's cylindrical equal-area  projection}" or sometimes "\NewTerm{Peter's projection}\index{Peter's projection}" (must not be confused with "Gall's-Peter projection"!), is quite easy as illustrated in the figure below:
	\begin{figure}[H]
		\centering
		\includegraphics[scale=0.7]{img/geometry/lambert_spherical_projection.jpg}
		\caption{Lambert's equivalent projection (or Lambert's equal area projection)}
	\end{figure}
	Suppose we have a terrestrial globe as visible above. It is enough to wrap a sheet of paper around the globe as in the figure above and to project horizontally each point of the sphere on the cylinder. For that, if $A$ is a point of the sphere, we take the line passing through $A$ and perpendicular to the axis of the poles This line intersects the cylinder at $A'$: the point $A'$ represents the point $A$ on the map.
	
	Or more explicitly:
	\begin{figure}[H]
		\centering
		\includegraphics{img/geometry/lambert_projection_earth.jpg}
	\end{figure}
	Now using the following figure, let us prove that the Lambert's cylindrical projection conserve the areas:
	\begin{figure}[H]
		\centering
		\includegraphics[scale=0.7]{img/geometry/lambert_projection_equal_area.jpg}
	\end{figure}
	For this proof, we will look to a small area of the sphere delimited by the parallels of latitude $\theta_1$ and $\theta_2$ and of the meridians of longitude $\phi_1$ and $\phi_2$. We will denote by $R$ the radius of the sphere and we will assume that $\theta_2-\theta_1$ and $\phi_2-\phi_1$ are positive and small.

	According to this assumption, the area of this region is that of a small rectangle (see figure above). At the latitude $\theta_1$, the radius of the parallel is obviously $R\cos(\theta_1)$. Then by a simple rule of three, the length of the arc from angle $\phi_2-\phi_1$ is equal to $R(\phi_2-\phi_1)\cos(\theta_1)$, as the length of the area is the product or the radius by the angle in radian (\SeeChapter{see section Trigonometry page \pageref{arc length trigonometry}}).
	
	What is now the height of this rectangle? As the meridian is a circle of radius $R$ and that the height of the rectangle is given by angle $\theta_2-\theta_1$, then this height is given approximately by $R(\theta_2-\theta_1)$. Then the area of our small region is approximately equal to:
	
	Let us calculate now the area of its projection on the cylinder. This projection is now a real rectangle. The parallel has been projected on a circle of radius $R$. As the corresponding angle has remained the same, that is $\phi_2-\phi_1$, the width of the rectangle is $R(\phi_2-\phi_1)$. For the height, the reader must notice that the tangent to the meridian at the latitude $\theta_1$ make an angle of:
	
	with the horizontal. The projection on the vertical of a segment of length $R(\theta_2-\theta_1)$ is then equal to:
	
	The area on the map is then equal to:
	
	Thus, the same area than that of the small area on the sphere. This finish the proof that Lambert's cylindrical projection conserves the area.
	
	\subsubsection{Mollweide projection}
	The "\NewTerm{Mollweide projection}\index{Mollweide projection}\label{Mollweide projection}" is a very famous equal-area, pseudocylindrical map projection generally used for global maps of the world or night sky (it is also known as the "Babinet projection", "homalographic projection", "homolographic projection" and "elliptical projection"). The projection trades accuracy of angle and shape for accuracy of proportions in area, and as such is used where that property is needed, such as maps depicting global distributions. 
	
	The text that follows is entirely inspired from the paper \cite{lapaine2011mollweideova} where the method is derived in details using three different proofs. Here we will focus on what seems to us the simplest of the three derivations that make sense.
	
	Pseudocylindrical map projections have in common straight parallel lines of latitude and curved meridians. Until the 19th century the only pseudocylindrical projection with important properties was the sinusoidal or Sanson-Flamsteed. The sinusoidal has equally spaced parallels of latitude, true scale along parallels, and equivalency or equal-area. As a world map, it has disadvantage of high distortion at latitudes near the poles, especially those far from the central meridian:
	\begin{figure}[H]
		\centering
		\includegraphics[width=0.7\textwidth]{img/geometry/sanson_flamsteed.jpg}
		\caption{Sansons or Sanson-Flamsteed or Sinuoidal projection}
	\end{figure} 
	In 1805, Mollweide announced an equal-area world map projection that is aesthetically more pleasing than the sinusoidal because the world is placed in an ellipse with axes in a ratio and all the meridians are equally spaced semi-ellipses. The Mollweide projection was the only new pseudocylindrical projection of the nineteenth century to receive much more than academic interest:
	\begin{figure}[H]
		\centering
		\includegraphics[width=0.7\textwidth]{img/geometry/mollweide_projection.jpg}
		\caption{Mollweide projection}
	\end{figure} 
	O'Connor and Robertson stated that Mollweide produced the map projection to correct the distortions in the Mercator projection, first used by Gerardus Mercator
in 1569. While the Mercator projection is well adapted for sea charts, its very great exaggeration of land areas in high latitudes makes it unsuitable for most other purposes. In the Mercator projection the angles of intersection between
the parallels and meridians, and the general configuration of the land, are preserved but as a consequence areas and distances are increasingly exaggerated as one moves away from the equator. To correct these defects, Mollweide drew 
his elliptical projection; but in preserving the correct relation between the areas he was compelled to sacrifice configuration and angular measurement. The Mollweide projection lay relatively dormant until J. Babinet reintroduced it in 1857 under the name "homalographic projection".

	The well known equations of the Mollweide projections that we will prove further below read as follows:
	
	In these formulas $x$ and $y$ are rectangular coordinates in the plane of projection, $\varphi$ (latitude angle) and $\lambda$ (longitude angle) are geographic coordinates of the points on the sphere and $R$ is the radius of the sphere to be mapped. The angle $\beta$ is an auxiliary angle that is connected with the latitude $\varphi$ by the relation $2\beta+\sin(2\beta)=\pi\sin(\varphi)$. For given latitude $\varphi$, the equation $2\beta+\sin(2\beta)=\pi\sin(\varphi)$ is a transcendental equation in $\beta$. In the past, it was solved by using tables and interpolation method. In our days, it is usually solved by using some iterative numerical method, like bisection or Newton-Raphson method.
	\begin{dem}
	Given the earth's radius $R$, suppose the equatorial aspect of an equal-area projection with the following properties:
	\begin{itemize}
		\item A world map is bounded by an ellipse twice broader than tall
		\item Parallels map into parallel straight lines with uniform scale
		\item The central meridian is a part of straight line; all other ones are semi-elliptical arcs.
	\end{itemize}
	Suppose an earth-sized map; let us define two regions, $S_{1}$ on the map and $S_{2}$ on the earth, both bounded by the equator and a parallel (the figure depicts and 2D representation but the left geometry is an ellipse - not an ellipsoid ! - and the right one a sphere):
	\begin{figure}[H]
		\centering
		\includegraphics[width=0.7\textwidth]{img/geometry/derivation_mollweide_projection_equations.jpg}
	\end{figure}
	The equal-area property can be used to calculate $x$ for given $\varphi$. Given $x$ and $\lambda$ (longitude), $y$ can be calculated immediately from the ellipse equation, since horizontal scale is constant. Equation of ellipse centred in origin, with major axis on $y$-axis is as we already know:
	
	For $0 \leq x \leq a$:
	
	The area between $y$-axis and parallel mapped into $x=x_{1}$ is:
	
	Let us introduce the auxiliary angle:
	
	and:
	
	then:
	
	Since:
	
	Then:
	
	Therefore:
	
	for some $0 \leq \beta \leq \pi/2$, corresponding to $x_{1}=a \sin (\beta)$ and because of $a b \pi=4 R^2 \pi$.
	
	On a sphere, the area between the equator and parallel $\varphi$ is:
	
	But as $S_{1}=S_{2}$, then:
	
	Hence:
	
	Simplifying we fall back the third relation of the Mollweide projection:
	
	The auxiliary angle $\beta$, as already mentioned, must be found by interpolation or successive approximation. 
	
	Finally, since horizontal scale is uniform, from the two equalities (remember that the left $ab\pi$ is the surface of the elliptical projection 2D and the right $4R^2\pi$ is the surface of the projected sphere!):
	
	We get:
	
	and therefore we can fall back on the first relation of the Mollweide projection:
	
	Now, remember that $\lambda$ is the longitude that goes starting from $0$ [rad], to $-\pi$ on the West to $+\pi$ on the East. As it is symmetrical, from the meridian $0$ [rad] just having one of the value of the angle going to West or East in enough information to make the mapping on the ellipse (as it is symmetrical!). The clever idea now is to shrink the elliptical projection so that when the latitude is near zero we get a very small projection ellipse as visible in red below and when it is equal to $+\pi$  we get the whole projection ellipse in green below (remember we do that in the horizontal direction of $y$). So multiplying the elliptic function $y$ by the ratio $\lambda/\pi$ will the expected shrinkage:
	
	\begin{figure}[H]
		\centering
		\includegraphics[width=0.7\textwidth]{img/geometry/latitude_shrinkage_mollweide.jpg}
	\end{figure}
	Therefore:
	
	Then we fall back on the second relation of the Mollweide projection:
	
	We have all of the tree relations and this finish the proofs!
	\begin{flushright}
		$\blacksquare$  Q.E.D.
	\end{flushright}
	\end{dem}
	Although the applied condition was that a half of the sphere has to be mapped onto a disc, the final projection equations hold for the whole sphere and give its image situated into an ellipse.
	
	Let us consider the shape of the Mollweide projection of the whole sphere. From:
	
	is easy to get the equation of a meridian in the projection:
	
	It is obvious that for a given $\lambda(14)$ is the equation of an ellipse. It follows that the semi-axis $a$ is constant, while $b$ depends on the longitude $\lambda$. If we take $\lambda=\pi$, than $b=2 \sqrt{2} R$, and $a: b=1: 2$ and that is the ratio of semi-axes in the Mollweide projection. The question arises: is it possible to find out a pseudocylindrical equal-area projection that will give the whole word in an arbitrary ellipse satisfying any given ratio $a: b$ or $b: a$ ? The answer is yes and this leads to "Generalized Mollweide projection". But as the proof is of no interest in this book, the reader can refer to \cite{lapaine2011mollweideova} for a detailed proof of it.

	So the Mollweide is a pseudocylindrical projection in which the equator is represented as a straight horizontal line perpendicular to a central meridian one-half its length. The other parallels compress near the poles, while the other meridians are equally spaced at the equator. The meridians at 90 degrees east and west form a perfect circle, and the whole earth is depicted in a proportional 2:1 ellipse.
	
	Notice that the Mollweide and Hammer projections are occasionally confused, since they are both equal-area and share the elliptical boundary; however, the latter design has curved parallels and is not pseudocylindrical:
	\begin{figure}[H]
		\centering
		\includegraphics[width=0.7\textwidth]{img/geometry/hammer_projection.jpg}
		\caption{Hammer projection}
	\end{figure}
	
	\subsection{Other perspectives}
	We can also choose any other angle of perspective and that not keep anything as equal but that are just well adapted to the technology or to human perception. For example the game Diablo II or Age of Empire use an angle in degrees of $53.130102^\circ$ (rounded to $53$) as for old $4:3$\footnote{The origin of the $4:3$ is interesting and useful as the radio forms a Pythagorean triangle, meaning that the length of the diagonal (hypotenuse) is an exact integer value (of pixels...).} screens it gives as result that an object have a vanishing line on the upper left corner of the $4:3$ screen will also pass trough the bottom right corner of the screen. It is therefore just choice of comfort.
	\begin{figure}[H]
		\centering
		\includegraphics[scale=0.8]{img/geometry/diablo_II_perspective_grid.jpg}
		\caption[]{Diablo II perspective grid}
	\end{figure}
	Alternatively there are also a few games which use a trimetric projection where one axis is 1:2 and the other is 2:1 as below:
	\begin{figure}[H]
		\centering
		\includegraphics{img/geometry/trimetric_projection.jpg}
		\caption[]{Example of trimetric projection}
	\end{figure}
	To resume we have seen so far the following projections (the spherical projections are not listed in the figure below for the moment!):
	\begin{figure}[H]
		\centering
		\includegraphics{img/geometry/projections_summary.jpg}
		\caption{Non-exhaustive orgchart of various projections methods}
	\end{figure}
	
	\subsection{Homogeneous Coordinates (projection coordinates)}
	In mathematics, "\NewTerm{homogeneous coordinates}\index{homogeneous coordinates}\label{homogeneous coordinates}" also named "\NewTerm{projection coordinates}", introduced by August Ferdinand Möbius, make calculations possible in the projective space as do the Cartesian coordinates in Euclidean space. Homogeneous coordinates are used extensively in computer graphics and more particularly for the representation of three dimensional (3D) scenes  because they are adapted to projective geometry and allow to characterize the transformations of space under optimal algorithmic form. The matrix notation form is particularly used in 3D graphics programming  libraries such as OpenGL and Direct3D.

	We proved in our study of transformations in the plane and space (\SeeChapter{see section Euclidean Geometry page \pageref{geometric transformations}}) that the translation, the scaling, rotation or reflection were not possible to be translate represented in (real) matrix form without a trick that was to add a dummy extra dimension to the vector of coordinates and to the associated matrix transformation but we did it only for the translation. So let us see it now for all other classical transformation.
	
	\subsubsection{$\mathcal{P}^2$ Projective Space}
	Thus, we saw that the translation in $\mathbb{R}^2$ could be written in matrix form:
	
	So always in $\mathbb{R}^2$, a scaling of factor $k$ who was written:
	
	becomes in homogeneous coordinates:
	
	So always in,$\mathbb{R}^2$ a rotation of angle $\theta$ in the $x-y$:
	
	becomes (and as we proved it, this is indeed a rotation around the $z$-axis):
	
	So always in $\mathbb{R}^2$, the reflection that was written along the $x$-axis by:
	
	becomes:
	
	and so on for other transformations that we have proved and that were resumed in the following figure we have already see and where the homogeneous coordinates are in gray:
	\begin{figure}[H]
		\centering
		\includegraphics[scale=0.9]{img/geometry/affine_transformation_matrix_summary.jpg}
		\caption{Summary homogeneous coordinates transformations}
	\end{figure}
	Mathematicians say then that we when we use the above transformations of the composition of some of them, we put ourselves in the projective space $\mathcal{P}^2$.
	
	\subsubsection{$\mathcal{P}^3$ Projective Space}
	Verbatim, we can do the same with a point $P$ of $\mathbb{R}^3$ which will then in $\mathcal{P}^3$ be represented by a $4\times 1$ coordinate vector:
	
	Thus we have in space the following transformation matrices:
	\begin{itemize}
		\item For translations:
		
		
		\item For rotations it is enough for sure in 3D to have as we proved it:
		
		But if we want to work with projection coordinates, this becomes obviously:
		
		
		\item For homoteties (scaling) obviously:
		
		
		\item In space, one can push in two coordinate axis directions and keep the third one fixed. The following is the shear transformation in both $x-$ and $y-$directions with shearing factors $a$ and $b$, respectively, keeping the $z-$ coordinate the same (so its only one of the possible special cases):
		
		Thus, a point $(x, y, z)$ in space is transformed to $(x + az, y + bz, z)$. Therefore, the $z$-coordinate does not change, while $(x, y)$ is "pushed'' in the direction of $(a, b, 0)$ with a factor $z$.
	\end{itemize}
	Mathematicians say then that we when we use the above transformations of the composition of some of them, we put ourselves in the projective space $\mathcal{P}^3$.
	
	We have proved above that in the case of the conical project, we had the following homographic transformations:
	
	where $x',y',z'$ are the projection of the relation points $x,y,z$ on the table with a focal distance $\Delta$ for recall...

	What is traditionally written by putting $z:=z+\Delta$ and $f:=\Delta$:
	
	where we see that if the focal distance $f$ is infinite, the subject coincides with the $XY$ plane (or if the object is infinitely far away as it is equivalent...).
	
	Let us put:
	
	We can the write:
	
	Or better:
	
	We then use the following matrix for the conical projection:
	
	and then we normalize the coordinates by the ratio $z/f$. Indeed (a reader request the details):
	
	
	\begin{flushright}
	\begin{tabular}{l c}
	\circled{80} & \pbox{20cm}{\score{4}{5} \\ {\tiny 35 votes,  82.29\%}} 
	\end{tabular} 
	\end{flushright}
	
	%to make section start on odd page
	\newpage
	\thispagestyle{empty}
	\mbox{}		
	\section{Analytical Geometry}

	\lettrine[lines=4]{\color{BrickRed}T}he "\NewTerm{analytic geometry}\index{analytic geometry}" is the branch of geometry that deals with the study of geometric shapes and their properties using the advanced tools of Calculus such as functional analysis, vector calculus and linear algebra. Its border lies in the tools used and originated by the work of René Descartes in the early 17th century. The "vector geometry" is a subset of analytical geometry and we will also us it in this section as in this of Differential Geometry.

	\begin{tcolorbox}[title=Remark,colframe=black,arc=10pt]
	When we use for these same subjects that will follow the differential and integral calculus we then say that we do "differential geometry" (see section of the same name page \pageref{differential geometry}).
	\end{tcolorbox}
	
	Analytic geometry is a very broad area (like everything else in this book) then... we discussed here that the elements indispensable for the study of physics (especially astronomy and quantum physics particle) and engineering (spacial engineering). These elements are also often studied in small classes and are (cited in order of study at high-school): the conics, the equations of the line, of the plan, of the sphere, etc. their intersections, their tangent planes and many others.	
	
	The reader will notice that we begin this section with conics and not straight lines and this is simply because we will use much more conics properties than straight lines in this book. For analytical geometric properties of straight lines see the subsection after the conics.

	\subsection{Conics}\label{conics}
	It has been very difficult for us to choose whether to put the study of the conics in the chapter on Algebra or on Geometry. We finally decided to put this study in this chapter (hence Geometry...) which suggests that the reader that has made a linear reading of this book has already covered all chapters and sections presenting the mathematical tools necessary to study of conics. We hope that our choice will be the best one for the reader.
	
	The study of conic will be very useful in the section of Astronomy (we own to Kepler is  many results from on the study on conics) and also in the sections of Geometrical Optics, Statistics and Industrial Engineering. It is therefore appropriate to go in the details of this subject.

	\subsubsection{Algebraic approach}
	Consider $(\text{O},\vec{i},\vec{j})$ an orthonormal plan. The simplest algebraic curves that we can found after the lines whose general form equations are (reminder):
	
are the curves of the second degree, named "\NewTerm{analytical conics}\index{analytical conics}", that is to say by extension:
	
with $\alpha,\beta,\gamma$ all non-zero.

	The second degree corresponding curves are named "conics" (also named "quadrics\index{quadrics}\label{quadrics}" because of the presence of a quadratic term). The conic will be named "proper conic" if it is an ellipse, a parabola or a hyperbole as we shall see further below.

This last relation can also be written in matrix form (very important in the practice of Numerical Methods as we will see it in the section of the same name and in the classification of differential equations as discussed in the section of Differential and Integral Calculus):
	
notation for which there exists several variants...:
	

	gives for different values of $g$ the following paths:
	\begin{figure}[H]
		\centering
		\includegraphics[scale=0.75]{img/geometry/conics_plane_plots.eps}
		\caption{Conics with respectively $b=0.5, 1.5$ and $1$ and more values of $g$}
	\end{figure}
	Our first task will be to obtain, by translation and rotation of the coordinate system in which this relation is expressed, a reduced equation much simpler by eliminating the term $xy$. Indeed, let us choose a new referential being deduced by the previous one just by a rotation of angle equation. Let $x'$ and $y'$ be the coordinates of the new points. We have (\SeeChapter{see section Euclidean Geometric page \pageref{rotation matrix in the plane}}):
	
Thus:
	
The equation becomes:
	
So we want all the terms $x'y'$ to be grouped as:
	
Since (\SeeChapter{see section Trigonometry page \pageref{remarkable trigonometric identities}}):
	
substituting, we get:
	
To have only the terms in $x'y'$ that simplifies, we just have to choose the rotation angle $\theta$ such as:
	
Therefore we will consider now the equation:
	
	\begin{enumerate}
		\item If we write that $\alpha=0$ and  $\beta=0$. And we divide by $\beta$, we can take us back to an equation of the following type:
		
	Where:
		\begin{itemize}
			\item If $\gamma\neq 0$, we end up with an equation describing the figure of a "\NewTerm{parabola}\index{parabola}" of axis parallel to $\text{O}X$.
			\item If $\gamma = 0$, it is a degenerate case (a degenerate conic is a conic that is reduced to be two lines that may or may not be parallel or a single line).
		\end{itemize}
	\item If we write $\alpha \neq 0$ and $\beta = 0$ thus the situation is treated exactly as before.
	
	\item If $\alpha \neq 0$ and  $\beta \neq 0$, we can remove the terms $\delta x$ and $\varepsilon y$ terms as follows:
	
	
	And by a simple change of referential via translations, we arrive at an equation that looks like following:
		
		\begin{itemize}
			\item If $\gamma=0$ the previous relations simplifies to:
				
				And therefore it is obvious that if $\sgn(x)=\sgn(y)$ there is only the trivial solution $X=Y=0$ if we stay in $\mathbb{R}$. Instead if $\sgn(x) \neq \sgn(y)$ thus we have an infinity of solutions represented by a straight line.
			\item $\gamma \neq 0$ let us put that:
				
			and let us divide by $\mid \gamma \mid$ such that:
				
			If we put now:
			
			We get:
			
			So we therefore have several possible situations:
			
			Two terms above are impossible in $\mathbb{R}^2$, this is why we crossed them (the sum of two positive numbers can not be negative and vice versa).
			
			There are several cases of interesting figures:			
			\begin{enumerate}
				\item For\label{equation of a circle}:
					
				and $a=b=1$, we have a unit circle. The reader can have fun testing this with the following commands in Maple 4.00b:\\
				
				\texttt{>with(plots):}\\
				\texttt{>a:=1;b:=1;}\\
				\texttt{>implicitplot(x\string^2/a\string^2+y\string^2/b\string^2=1,x=-10..10,y=-10..10);}
				
				\item For:
					
				and $\forall a,b \in \mathbb{R}^{*}$ we have an ellipse\label{analytical expression ellipse} (which therefore contains the specific case of the circle) with half-axes $a$ and $b$ (for the visual representation of an ellipse see further below or otherwise see the section of Geometric Forms).\\
				
				The reader can have fun testing this with the following command in Maple 4.00b:\\
				
				\texttt{>with(plots):}\\
				\texttt{>a:=4;b:=8;}\\
				\texttt{>implicitplot(x\string^2/a\string^2+y\string^2/b\string^2=1,x=-10..10,y=-10..10);}
				
				\item For:
				
				and $\forall a,b \in \mathbb{R}^{*}$, we get hyperbolas whose symmetry axis is parallel to $\text{O}x$ or $\text{O}y$ (for visual representation see further below). We say that the hyperbole is an "\NewTerm{equilateral hyperbola}\index{hyperbola}\label{hyperbola}" when $a = b$.\\
				
				The reader can have fun testing this with the following commands in Maple 4.00b:\\
				
				\texttt{>with(plots):}\\
				\texttt{>a:=4;b:=4;}\\
				\texttt{>implicitplot(x\string^2/a\string^2-y\string^2/b\string^2=1,x=-10..10,y=-10..10);}	
			\end{enumerate}
		\end{itemize}
		The term "conical" comes from the fact that one of the first definitions of conicals consisted of the intersection of a cone and a plane.
		
		Indeed, given:		 
				
		the equation of a cone having an angle at the top of $\pi/4$ (or an angle of $\pi/2$ relatively to its surface and the $x$-axis). This is simple the application of Pythagorean theorem...
		
		Plus also given a plane of oriented with an angle $\theta$ relatively to the $Z$ axis:
		
		where the reader can see that if $\theta=0$ we get $z=h$ for $\forall x,y$ the plane is therefore parallel to $x\text{O}y$, if $\theta=\pi/2$ we get $y=h$ for $\forall z,y$ the plane is parallel to $x\text{O}z$. In both situation and for every $\theta$ the plane never intersect with the axis $X$ and we have a normal vector that is always in the plane $x\text{O}y$.
		
		This normal vector has almost obviously the following coordinates (we us cosine-director notation):
		
		Consider now the rotation matrix in the space $\mathbb{R}^3$ relative to the axis $\text{O}z$ (\SeeChapter{see section Euclidean Geometry page \pageref{3d rotation matrix around}}):
		
		So we have for expression of rotation of our plane:
		
		After simplification, $\forall \theta$:
		
		This just mean that when we apply the rotation matrix to our plane with the same angle $\theta$ we take it to is "initial" position.
		Identically, for the cone, a rotation according to the $Z$ axis (so it does not happen much):
		
		After development and simplification:
		
		Equation that gives a horizontal cone for $\theta=0$ and a vertical one for $\theta=\frac{\pi}{2}$.
		Thus, we have one possible general system:
		
		We see then that for:
		\begin{itemize}
			\item $\theta \in \left]\dfrac{\pi}{4},\dfrac{\pi}{2}\right]$ we get an intersection between the plane and the cone giving an ellipse
			\item $\theta=\dfrac{\pi}{4}$ we get a parabola
			\item $\theta \in \left[0,\dfrac{\pi}{4}\right[$ we get a hyperbola
		\end{itemize}
		\begin{figure}[H]
			\centering
			\includegraphics{img/geometry/cone_plane_intersections.jpg}
			\caption{Various intersections between a cone and a plane}
		\end{figure}
		and possible real world neighbourhood noise impact analysis:
		\begin{figure}[H]
			\centering
			\includegraphics[scale=0.4]{img/geometry/show_wave_cone_hyperbola.jpg}
			\caption[A shock wave intersecting the ground]{A shock wave intersecting the ground (source: OpenStax)}
		\end{figure}
		\begin{tcolorbox}[title=Remark,colframe=black,arc=10pt]
		We also give the curve of equation $xy=1$ the name hyperbole because, with a variable change:
		
		Which brings us to:
		
		which as we have seen, is the equation of a hyperbole.
		\end{tcolorbox}
		For those who have Maple 4.00b or later here are some non-trivial commands to have fun to do a parabola:
		
		\texttt{>restart: with(plots):}\\
		\texttt{>c:=sqrt(x\string^2+y\string^2):}\\
		\texttt{>p:=y+3:}\\
		\texttt{>Y:=solve(c=p,y);}\\
		\texttt{>intsect:=subs(y=Y,c);}\\
		\texttt{>intsect := (x +(1/6*x-3/2))}\\
		\texttt{>P1:=plot3d(c,x=-5..5,y=-5..5,axes=normal,color=red,numpoints=2000}
		\texttt{,view=[-5..5,-5..5,0..5],style=wireframe):}\\
		\texttt{>P2:=plot3d(p,x=-5..5,y=-5..5,axes=normal,color=yellow,}\\
		\texttt{numpoints=2000,view=[-5..5,-5..5,0..5],style=patchnogrid):}\\
		\texttt{>display(P1,P2,scaling=constrained, orientation=[-10,75]);}\\
		
		that gives:
		\begin{figure}[H]
			\centering
			\includegraphics{img/geometry/plot_maple_parabola.jpg}
			\caption{Maple 4.00b plot to obtain a parabola by the intersection of a cone and a plane}
		\end{figure}
		or to obtain a hyperbola:
		
		\texttt{>restart: with(plots):}\\
		\texttt{>c:=sqrt(x\string^2+y\string^2):}\\
		\texttt{>p:=4*y+5:}\\
		\texttt{>Y:=solve(c=p,y);}\\
		\texttt{>intsect:=subs(y=Y[1],c);}\\	\texttt{>P1:=plot3d(c,x=-5..5,y=-5..5,axes=normal,color=red,numpoints=2000}\\
		\texttt{,view=[-5..5,-5..5,0..5],style=wireframe):}\\	\texttt{>P2:=plot3d(p,x=-5..5,y=-5..5,axes=normal,color=yellow,numpoints=2000}\\
		\texttt{,view=[-5..5,-5..5,0..5],style=patchnogrid):}\\	\texttt{>P3:=spacecurve([x,Y[1],intsect],x=-5..5,color=black,thickness=3):}\\
	\texttt{P3:=spacecurve([x,Y[1],intsect],x=-5..5,color=black,thickness=3):}\\
		\texttt{>display(P1,P2,P3,scaling=constrained);}\\
		
		that gives:
		\begin{figure}[H]
			\centering
			\includegraphics{img/geometry/plot_maple_hyperbola.jpg}
			\caption{Maple 4.00b plot to obtain a hyperparabola by the intersection of a cone and a plane}
		\end{figure}
		or to obtain an ellipse:
		
		\texttt{>restart: with(plots):}\\
		\texttt{>c:=sqrt(x\string^2+y\string^2):}\\
		\texttt{>p:=y/3+3:}\\
		\texttt{>Y:=solve(c=p,y); }\\
		\texttt{>E1:=subs(y=Y[1],c); E2:=subs(y=Y[2],c);}\\
		\texttt{>P1:=plot3d(c,x=-5..5,y=-5..5,axes=normal,color=red,numpoints=2000,}\\
		\texttt{view=[-5..5,-5..5,0..5],style=wireframe):}\\
		\texttt{>P2:=plot3d(p,x=-5..5,y=-5..5,axes=normal,color=yellow,numpoints=2000,}\\
		\texttt{view=[-5..5,-5..5,0..5],style=patchnogrid):}\\
		\texttt{>P3:=spacecurve({[x,Y[1],E1],[x,Y[2],E2]},x=-5..5,color=black,thickness=3,numpoints=2000):}\\
		\texttt{>display(P1,P2,P3,scaling=constrained);}\\
		
		that gives:
		\begin{figure}[H]
			\centering
			\includegraphics{img/geometry/plot_maple_ellipse.jpg}
			\caption{Maple 4.00b plot to obtain an ellipse by the intersection of a cone and a plane}
		\end{figure}
		or finally to obtain a circle:
		
		\texttt{>restart: with(plots):}\\
		\texttt{>c:=sqrt(x\string^2+y\string^2):}\\
		\texttt{>p:=3:}\\
		\texttt{>Y:=solve(c=p,y);}\\
		\texttt{>circ1:=subs(y=Y[1],c); circ2:=subs(y=Y[2],c);}\\
		\texttt{>P1:=plot3d(c,x=-5..5,y=-5..5,axes=normal,color=red,numpoints=2000,}\\
		\texttt{view=[-5..5,-5..5,0..5],style=wireframe):}\\
		\texttt{>P2:=plot3d(p,x=-5..5,y=-5..5,axes=normal,color=yellow,numpoints=2000,}\\
		\texttt{view=[-5..5,-5..5,0..5],style=patchnogrid):}\\
		\texttt{>P3:=spacecurve({[x,Y[1],circ1],[x,Y[2],circ2]},
x=-5..5,color=black,thickness=3,numpoints=2000):}\\
		\texttt{>display(P1,P2,P3);}\\

		that gives:
		\begin{figure}[H]
			\centering
			\includegraphics{img/geometry/plot_maple_circle.jpg}
			\caption{Maple 4.00b plot to obtain a circle by the intersection of a cone and a plane}
		\end{figure}		
	\end{enumerate}
	However, the conicals also have a geometrical definition:
	
	\pagebreak
	\subsubsection{Geometric Approach}
	Let $F$ be a point in the plane, $D$ a line but not containing $F$ (but to a non-zero distance of it) and $e$ a positive real number. We are interested in the set of points $M$ defined by:
	
	where $F$ is named the "\NewTerm{focus}\index{focus}", $D$ the "\NewTerm{direction of the conical}\index{direction of the conical}" and $e$ the "\NewTerm{eccentricity}\index{eccentricity}\label{eccentricity}":
	\begin{figure}[H]
		\centering
		\includegraphics{img/geometry/focus_direction_excentricity.jpg}
		\caption{Definition of the focus, direction of the conical and eccentricity}
	\end{figure}
	We will arrange subsequently to have always $F$ as the origin of the basis for the conical, therefore $D$ will have by definition for equation:
	
	with $h>0$ so we therefore denote by $d (M, D)$ the distance between the point $M$ and $D$.
	
	Therefore:
	
	We well fall back on the equation of a conic since the latter is a special case of:
	
	We can now consider several specific cases:
	\begin{enumerate}
		\item Case where $e=1$:
		The equation:
		
		then reduces trivially to:
		
		It is therefore a parabola of axis orthogonal to $D$, whose top $\Omega$ is the mid-segment $\overline{FK}$, where $K$ is the projection of $F$ on $D$ (see figure below).
		
		If we rewrite the last relation in the form:
		
		and redefining the origin with respect to $\Omega$ by a translation of $h/2$, the generating focus of the parabola will be in $F=(h/2,0)$ and the latter equation is then reduced to:
		
		where $h$ is named "\NewTerm{parameter of the parabola}\index{parameter of the parabola}" and relatively to $\Omega$, the focus will be given by the coordinates $F=(h/2,0)$ and the direction by the equation $D:x=-h/2$. Written in a different way we fall back on the famous relation of the parabola in functional analysis (the reader can refer to a practical calculation example for an antenna in the section about Optics):
		
		That is:
		
		As shown in the figure below, the distance to the direction to $\Omega$ is imposed by the conditions of the model:
		\begin{figure}[H]
			\centering
			\includegraphics{img/geometry/conical_parabola_eccentricity_equal_one.jpg}
			\caption{Representation of the parameters of the conical for the parabola}
		\end{figure}
	
		\item Case where $e<1$:
		
		It is an ellipse. Indeed it is not easy to see it but by rearranging the terms of equation:
		
		we have (and this independently of whether $e<1$ or not):
		
		The last term of the ante-previous relation can recovered as following after development:
		
		Let us define:
		
		is the origin of the ellipse on $X$ (then the $x$ and $y$ must now be taken relative to this new origin!!!). The previous equation can therefore be simplified and becomes (this always regardless of whether $e<1$ or not):
		
		Therefore if $e<1$, the denominators are both positive and we fall back on the reduced equation of an ellipse since $a\neq b$.
		\begin{tcolorbox}[title=Remark,colframe=black,arc=10pt]
		You must notice that this definition can not include the circle among the ellipses otherwise there is a singularity because the denominators would be zero (the eccentricity being equal to zero for the circle).
		\end{tcolorbox}
		To get the semi-major axis of the ellipse we just have to put $y=0$. Thus, we have:
		
		then the semi-major axis is equal to:
		
		in the same way, we get for the semi-minor axis:
		
		putting $p=eh$ being the "\NewTerm{parameter of the ellipse}\index{parameter of the ellipse}\label{parameter of the ellipse}" or "\NewTerm{focal parameter of the ellipse}\index{focal parameter of the ellipse}\label{focal parameter of the ellipse}", we get:
		
		whose first relation will be very useful in the section of Astronomy and also in the section of General Relativity.
		
		Notice also by the way that we have then:
		
		
		It is customary to denote the distance from the center $\Omega$ of the ellipse to the foci point $F$ with the letter $c$ such that:
		
		We then have:
		
		There are therefore two foci to the ellipse at a distance equivalent but opposite of the center $\Omega$. We define therefore the eccentricity of an ellipse by the ratio:
		
		The reader will perhaps also notice that we have $x = 0$ in:
		
		we get for the ordinate the product $eh$ and it can be expressed by the ellipse parameters (even is this may seem odd at units level this is implicitly correct):
		
		We can then also prove a relation that is very common in formula compilation book about ellipses:
		
		That is to say:
		
		This property is then proved and that brings us to be able to write the eccentricity only thanks to the conventional parameters of the ellipse:
		
		\begin{figure}[H]
			\centering
			\includegraphics{img/geometry/ellipse_parameters.jpg}
			\caption{Representation of the parameters of the conical for the ellipse}
		\end{figure}
		Here are some well known elliptic eccentricities:
		\begin{table}[H]
			\centering
			\begin{tabular}{|l|c|}
			\hline
			\rowcolor[HTML]{9B9B9B} 
			\multicolumn{1}{|c|}{\cellcolor[HTML]{9B9B9B}\textbf{Celestial object orbit}} & \textbf{Orbit eccentricity $\pmb{e}$} \\ \hline
			Mercury & $0.206$ \\ \hline
			Venus & $0.007$ \\ \hline
			Earth & $0.017$ \\ \hline
			Mars & $0.093$ \\ \hline
			Jupiter & $0.048$ \\ \hline
			Saturn & $0.056$ \\ \hline
			Uranus & $0.047$ \\ \hline
			Neptun & $0.09$ \\ \hline
			Pluton & $0.250$ \\ \hline
			Hubble Telescope & $0.0002842$ \\ \hline
			International Space Station & $0.0005309$ \\ \hline
			Halley's comet & $0.970$ \\ \hline
			\end{tabular}
			\caption{Some elliptic eccentricity values}
		\end{table}
		
		We can also find out where are the director lines $D$ of the ellipse with respect to the edge of this latter by using the definition of eccentricity when $y=0$. Then we have:
		
		
		therefore the factor of $a$ is between $0$ and $+\infty$ and this has for meaning that the director lines will be situated (at least in the special case presented here where the eccentricity is positive and strictly less than the unit) always outside of the ellipse border and in the nearest case tangential to this same border (but in all cases they will be outside of the ellipse).
		
		A useful and obvious parametric representation of the ellipse is an extension of the trigonometric circle (\SeeChapter{see section Trigonometry page \pageref{trigonometric circle}}):
		
		Indeed, if we consider the Cartesian equation of the ellipse proved above:
		
		and putting $R=X/a$ and $S=Y/b$ we get:
		
		If we remember the unit circle in trigonometry, this equation admits as solutions $R=\cos(t)$ and $S=\sin(t)$. It comes then\label{parametric equation of an ellipse}:
		
		That's it! And obviously if $a=b\neq 0$ we fall back on the parametric equation of a circle!
		
		Now let us prove an important property of ellipses: The sum of the distance of the two foci to a point $M$ on the border of the ellipse is constant and equal to the major axis.
		
		\begin{dem}
		We consider the center is the midpoint of the foci of coordinate $\Omega(0,0)$. The foci lie on the major axis $y=0$ and same for the equation of the minor axis for which $x=0$.
		
		Now, if the length of the major, minor axes are $2a,2b$ respectively with eccentricity $e$ the equation of the ellipse is:
		
		Any point $M$ on the ellipse can be as we just see before:
		
		
		So, the distance between a point $M(a\cos(\theta),b\sin(\theta))$ and one of the foci of coordinate $(ae,0)$ is:
		
		Similarly (as always you can request the details if necessary), the distance between $(a\cos(\theta),b\sin(\theta)),(-ae,0)$ is equal to $a(1+e\cos(\theta))$.
		
		Therefore the sum gives:
		
		\begin{flushright}
			$\blacksquare$  Q.E.D.
		\end{flushright}
		\end{dem}
		It is this result that authorize us to say that an ellipse can be constructed in the famous following way:
		\begin{figure}[H]
			\centering
			\includegraphics[scale=0.7]{img/geometry/ellipse_hand_made_construction.jpg}
			\caption[Hand-made construction of an ellipse]{Hand-made construction of an ellipse (source: OpenStax)}
		\end{figure}
		It is interesting also to see the opposite reasoning! We suppose the construction above as a definition and we build from it the equation of the ellipse centered at the Origin!
		
		For this we begin with the foci $\left(-c,0\right)$ and $\left(c,0\right)$. The ellipse is then by definition the set of all points $\left(x,y\right)$ such that the sum of the distances from $\left(x,y\right)$ to the foci is constant, as shown in the figure above.
		
		If $\left(a,0\right)$ is a vertex of the ellipse, the distance from $\left(-c,0\right)$ to $\left(a,0\right)$ is:
		
		The distance from $\left(c,0\right)$ to $\left(a,0\right)$ is $a-c$. The sum of the distances from the foci to the vertex is:
		
		If $\left(x,y\right)$ is a point on the ellipse, then we can define the following variables:
		
		By the definition of an ellipse, the \underline{sum} of  distances ${d}_{1}$, ${d}_{2}$ of the foci to any point $(x,y)$ of the ellipse is equal to  and is constant! We know that the sum of these distances is $2a$ for the vertex $(a,0)$. It follows that:
		
		for any point on the ellipse. We will begin the derivation by applying the distance formula. The rest of the derivation is algebraic:
		
		Thus, the standard equation of an ellipse is:
		
		The equation defines an ellipse centered at the origin, with for recall:
		\begin{itemize}
			\item If $a>b$, the ellipse is stretched further in the horizontal direction
			\item If $b>a$, the ellipse is stretched further in the vertical direction
			\item The length of the major axis is $2a$
			\item The coordinates of the vertices are $(0,\pm a)$
			\item The length of the minor axis is $2$b
			\item The coordinates of the co-vertices are $(\pm b,0)$
			\item The coordinates of the foci are $(0,\pm c)$
			\item We have the identity $c^2=a^2-b^2$
		\end{itemize}
		\label{centered offset ellipse}The standard form of the of an ellipse with center $(h,k)$ is then obviously:
		
	
		However, there is another form of the equation of the ellipse, much more important, which is found in the sections of mechanics, astrophysics and quantum physics of this book.
		
		First let us recall that:
		
		In polar coordinates, this gives:
		
		Therefore:
		
		after identification:
		
		We obtain two different equations, but it is actually the same curve that describes the radius of the ellipse from one of its two foci. And we can see that:
		
		Since the $p=eh$ is defined as the parameter of the conical, the polar equation of the ellipse is given by:
		
		Notice the three special values\index{focal parameter}\index{pericenter}\index{apocenter}:
		
		and we get obviously:
		
		and since $e=\pm c/a$ we get also another famous relation for eccentricity:
		
		The pericenter is better known as the "\NewTerm{perigee}\index{perigee}" in astronomy as well as the apocenter which is better known as the "\NewTerm{apogee}\index{apogee}" and are used a lot in astronomy and astrodynamics.
		\begin{figure}[H]
			\centering
			\includegraphics[scale=0.8]{img/geometry/apogee_perigee.jpg}
			\caption{Apogee and Perigee}
		\end{figure}
		\begin{tcolorbox}[title=Remark,colframe=black,arc=10pt]
		In facts, "apogee" and "perigee" are reserved words when the Earth is on the foci. If it is the Sun that is the foci we then speak of "aphelion" (corresponding to beginning of Winter) and "perihelion" (corresponding to the beginning of summer).
		\end{tcolorbox}
		In the general case, $D$ may do any angle with the axis of polar angles, and the general equation is then (very important relation in astronomy and aerospace engineering!):
		
		
		\item Case where $e>1$:
		This is a hyperbola (same reasoning as the ellipse considering by starting with any other value than $e>1 $or not):
		
		Where we put again:
		
		is the origin of hyperbole. The above equation is simplified and becomes (so far, we end up with exactly the same expression as for the ellipse):
		
		But as the denominator $e>1$ the denominator of the second term will be negative and therefore we fall back on the reduced equation of a hyperbole!
		
		We have therefore for semi-major axis and semi-minor axis (same reasoning as for the ellipse):
		
		and:
		
		and the corresponding figure:
		\begin{figure}[H]
			\centering
			\includegraphics{img/geometry/hyperbole_parameters.jpg}
			\caption{Representation of the parameters of the conical for the hyperbola}
		\end{figure}
	\end{enumerate}
	where we presented the two asymptotes using a simple technique of passing to the limit:
	
	If the hyperbole is equilateral, then it is obvious that the two asymptotes are perpendiculars since their slope is then respectively $+1$ and $-1$. In the equilateral case, we also have the eccentricity which is immediately given by:
	
	We can also determine where are the director lines $D$ of the hyperbolas with respect to their border by using the definition of the eccentricity when $y=0$ by performing exactly in the same as for the ellipse. Then we have:
	
	therefore the factor of $a$ is between $0$ and tend to $1$ as $e$ is very large meaning that the director lines will be located (at least in the case here where this eccentricity is positive and strictly smaller than unity) is tangential to the hyperbola that is to say the closest to the $y$-axis (but in all cases they will therefore be between the two hyperbolas).
	
	There is another nice way to builder the equation of the hyperbola by assuming that he hyperbola is the set of all points $(x,y)$ such that the \underline{difference} of the distances $d_1$, $d_2$ from $(x,y)$ to the foci is constant.

	For this proof let us consider $(-c,0)$ and $(c,0)$ be the foci of a hyperbola centered at the origin.

	If $(a,0)$ is a vertex of the hyperbola, the distance from $(-c,0)$ to $(a,0)$ is:
	
	The distance from $(c,0)$ to $(a,0)$ is $c-a$. The sum of the distances from the foci to the vertex is:
	
	If $(x,y)$ is a point on the hyperbola, we can define the following variables:
	
	By assuming that for a hyperbola, $d_2-d_1$ is constant for any point $(x,y)$ on the hyperbola it follows for the vertex $(a,0)$ that:
	
	and therefore the same value should be consistent for the whole hyperbola.  As with the derivation of the equation of an ellipse seen just previously, we will begin by applying the distance formula. The rest of the derivation is algebraic. Compare this derivation with the one from the previous section for ellipses.
	
	This equation defines a hyperbola centered at the origin with vertices $(\pm a,0)$ and co-vertices $(0,\pm b)$.
	
	Therefore standard form of the equation of a hyperbola with center $(0,0)$ and transverse axis on the $x$-axis is:
	
	where:
	\begin{itemize}
		\item The length of the transverse axis is $2a$
		\item The coordinates of the vertices are $(\pm a,0)$
		\item The length of the conjugate axis is $2$b
		\item The coordinates of the co-vertices are $(0,\pm b)$
		\item The distances between the foci is $2c$
		\item We have the identity $c^2=a^2+b^2$
		\item The coordinates of the foci are $(\pm c,0)$
		\item The equations of the asymptotes are $y=\pm\displaystyle \dfrac{b}{a}$
	\end{itemize}
	And for the standard form equation of a hyperbola with center $(h,k)$ and transverse axis parallel to the $x$-axis we have:
	
	
	\pagebreak
	\subsubsection{Dandelin Theorem (Dandelin spheres)}
	In geometry, the "\NewTerm{Dandelin spheres}\index{Dandelin spheres}" are one or two spheres that are tangent both to a plane and to a cone that intersects the plane. The intersection of the cone and the plane is a conic section, and the point at which either sphere touches the plane is a focus of the conic section, so the Dandelin spheres are also sometimes called focal spheres.
	
	The Dandelin spheres can be used to prove at least two important theorems:
	\begin{enumerate}
		\item The first theorem is that a closed conic section (i.e. an ellipse) is the locus of points such that the sum of the distances to two fixed points (the foci) is constant. 
		
		As we already have proved this result previously in a way that we consider as the most simple and more pretty way we will not do it again using Dandelin spheres.

		\item The second theorem is that for any conic section, the distance from a fixed point (the focus) is proportional to the distance from a fixed line (the directrix), the constant of proportionality being named the "eccentricity".
		
		As we already have proved this result previously in a way that we consider as the most simple and more pretty way we will also not do it again using Dandelin spheres. 
	\end{enumerate}
	
	\tikzset{
	MyPersp/.style={scale=1.8,x={(-0.8cm,-0.4cm)},y={(0.8cm,-0.4cm)},
    z={(0cm,1cm)}},
%  MyPersp/.style={scale=1.5,x={(0cm,0cm)},y={(1cm,0cm)},
%    z={(0cm,1cm)}}, % uncomment the two lines to get a lateral view
	MyPoints/.style={fill=white,draw=black,thick}
	}
	
	\begin{figure}[H]
		\begin{center}
	\begin{tikzpicture}[MyPersp,font=\large, scale=0.8]
		% the base circle is the unit circle in plane Oxy
		\def\h{2.5}% Heigth of the ellipse center (on the axis of the cylinder)
		\def\a{35}% angle of the section plane with the horizontal
		\def\aa{35}% angle that defines position of generatrix PA--PB
		\pgfmathparse{\h/tan(\a)}
	  \let\b\pgfmathresult
		\pgfmathparse{sqrt(1/cos(\a)/cos(\a)-1)}
	  \let\c\pgfmathresult %Center Focus distance of the section ellipse.
		\pgfmathparse{\c/sin(\a)}
	  \let\p\pgfmathresult % Position of Dandelin spheres centers
	                       % on the Oz axis (\h +/- \p)
		\coordinate (A) at (2,\b,0);
		\coordinate (B) at (-2,\b,0);
		\coordinate (C) at (-2,-1.5,{(1.5+\b)*tan(\a)});
		\coordinate (D) at (2,-1.5,{(1.5+\b)*tan(\a)});
		\coordinate (E) at (2,-1.5,0);
		\coordinate (F) at (-2,-1.5,0);
		\coordinate (CLS) at (0,0,{\h-\p});
		\coordinate (CUS) at (0,0,{\h+\p});
		\coordinate (FA) at (0,{\c*cos(\a)},{-\c*sin(\a)+\h});% Focii
		\coordinate (FB) at (0,{-\c*cos(\a)},{\c*sin(\a)+\h});
		\coordinate (SA) at (0,1,{-tan(\a)+\h}); % Vertices of the
	                                           % great axes of the ellipse
		\coordinate (SB) at (0,-1,{tan(\a)+\h});
		\coordinate (PA) at ({sin(\aa},{cos(\aa)},{\h+\p});
		\coordinate (PB) at ({sin(\aa},{cos(\aa)},{\h-\p});
		\coordinate (P) at ({sin(\aa)},{cos(\aa)},{-tan(\a)*cos(\aa)+\h});
	     % Point on the ellipse on generatrix PA--PB
	
		\draw (A)--(B)--(C)--(D)--cycle;
		\draw (D)--(E)--(F)--(C);
		\draw (A)--(E) (B)--(F);
		\draw[blue,very thick] (SA)--(SB);
	
	%	\coordinate (O) at (0,0,0);
	%	\draw[->] (O)--(2.5,0,0)node[below left]{x};
	%	\draw[->] (O)--(0,3,0)node[right]{y};
	%	\draw[->] (O)--(0,0,6)node[left]{z};
	
		\foreach \t in {20,40,...,360}% generatrices
			\draw[magenta,dashed] ({cos(\t)},{sin(\t)},0)
	      --({cos(\t)},{sin(\t)},{2.0*\h});
		\draw[magenta,very thick] (1,0,0) % lower circle
			\foreach \t in {5,10,...,360}
				{--({cos(\t)},{sin(\t)},0)}--cycle;
		\draw[magenta,very thick] (1,0,{2*\h}) % upper circle
			\foreach \t in {10,20,...,360}
				{--({cos(\t)},{sin(\t)},{2*\h})}--cycle;
		\fill[blue!15,draw=blue,very thick,opacity=0.5]
	     (0,1,{\h-tan(\a)}) % elliptical section
			\foreach \t in {5,10,...,360}
				{--({sin(\t)},{cos(\t)},{-tan(\a)*cos(\t)+\h})}--cycle;
	
		\foreach \i in {-1,1}{%Spheres!
			\foreach \t in {0,15,...,165}% meridians
				{\draw[gray] ({cos(\t)},{sin(\t)},\h+\i*\p)
					\foreach \rho in {5,10,...,360}
						{--({cos(\t)*cos(\rho)},{sin(\t)*cos(\rho)},
	          {sin(\rho)+\h+\i*\p})}--cycle;
				}
			\foreach \t in {-75,-60,...,75}% parallels
				{\draw[gray] ({cos(\t)},0,{sin(\t)+\h+\i*\p})
					\foreach \rho in {5,10,...,360}
						{--({cos(\t)*cos(\rho)},{cos(\t)*sin(\rho)},
	          {sin(\t)+\h+\i*\p})}--cycle;
				}
						\draw[orange,very thick] (1,0,{\h+\i*\p})% Equators
			\foreach \t in {5,10,...,360}
				{--({cos(\t)},{sin(\t)},{\h+\i*\p})}--cycle;
			}
		\draw[red,very thick] (PA)--(PB);
		\draw[red,very thick] (FA)--(P)--(FB);
	%	\fill[MyPoints] (CLS) circle (1pt);% center of lower sphere
	%	\fill[MyPoints] (CUS) circle (1pt);% center of upper sphere
		\fill[MyPoints] (FA) circle (1pt)node[right]{$F_1$};
		\fill[MyPoints] (FB) circle (1pt)node[left]{$F_2$};
		\fill[MyPoints] (SA) circle (1pt);
		\fill[MyPoints] (SB) circle (1pt);
		\fill[MyPoints] (P) circle (1pt)node[below left]{$P$};
		\fill[MyPoints] (PA) circle (1pt)node[below right]{$P_1$};
		\fill[MyPoints] (PB) circle (1pt)node[above right]{$P_2$};
	\end{tikzpicture}
	\end{center}
		\caption[Dudlin Sphere]{Dudlin Sphere (source: Hugues Vermeiren)}
	\end{figure}
	Sorry for the reader that likes the Dudlin proof but we don't want to lose time by rewriting proofs just in another way and this especially with the \LaTeX{} language script... (even if the proofs are small). 
	
	\subsubsection{Classification of conical by the determinant}\label{classification of conical by the determinant}
	For the purposes of the classification of the partial differential equations which we shall study in the section of Differential and Integral Calculus, let us return to the general equation of conics (with a $2$ factor on some terms to simplify the developments that will follow):
	
	with $(a,b,c)\neq (0,0,0)$. 

	We will try to classify the conics by using a purely matrix property and drawing inspiration from what has been seen previously.

	As we know, any $\Gamma$ of $\mathbb{R}^2$ admitting an equation of the above form is named a "conic".

	We have also show that we can rewrite the equation above as:
	
	with:
	
	and let us recall that we have proved in the Linear Algebra chapter that if is symmetric, there exists an orthogonal matrix $S$ (thus satisfying $S^TS=\mathds{1}$) such that:
	
 	where $\lambda$ and $\mu$ are the eigenvalues of $A$.

	Before proceeding, let us notice that by using one of the properties of the determinant demonstrated in the section of Linear Algebra we have:
	
	because:
	
	Therefore:
	
	\begin{tcolorbox}[title=Remark,colframe=black,arc=10pt]
	If we would not have introduce the factor $2$ at the beginning this would have been written:
	
	This is why a lot of textbooks classify the conic by multiplying the determinant by $4$ and to make an analogy to the discriminant of polynomial (it's more easy for the student to remember) they multiply it by $-1$ to get therefore the "\NewTerm{discriminant of conic sections}\index{discriminant of conic section}":
	
	\end{tcolorbox}
 	Now let us observe what happens as a function of the value of this determinant and therefore in extenso as a function of the components of the matrix $A$.

	There are two cases to consider from which only three a very exclusive:
	\begin{itemize}
		\item Case of a non-null determinant $\det(A)\neq 0$:
		In the case where $\det(A)\neq 0$, then we have then that is $A$ invertible (\SeeChapter{see demonstration Linear Algebra}). In order to simplify the equation:
		
	 	to make the term $X$ disappear in order to obtain something well known to us, we are going to seek to translate the curve (the choice of starting with a translation being the most trivial one). By a small trial and error, we find that it is enough to translate each point $X$ of the curve from an $Y$ worth:
		
		We then have as $(\vec{x}+\vec{y})\in \Gamma$:
		
	 	After simplification we get by using the associativity properties of the multiplication, the distributivity of the matrix addition and by remembering that for a symmetric matrix (\SeeChapter{see section Linear Algebra page \pageref{symmetric matrix}}):
		
		and also remembering that (\SeeChapter{see section Linear Algebra page \pageref{transposed matrix}}):
		
		Then we have:
		
		To continue, remember that $\vec{x}$ and $B$ are vectors, then $\vec{x}^TB$ and $B^T\vec{x}$ are scalars that are equals and therefore cancel. It remains then to us:
		
		We can then say that the translated curve admits as equation:
	
	where obviously:
	
 	or what is explicitly equivalent to:
	
	We thus succeeded in simplifying the equation in the case where $A$ is invertible by removing the term in $\vec{x}$.

	The type of curve (ellipse, parabola, hyperbola, etc.) is obviously not changed by a translation, which is why we consider the previous equation to be the simplest equation.

	More generally, the type of curve is not changed by an orthogonal matrix (easy to see because the image of an orthonormal basis by an orthogonal matrix is still an orthonormal basis). Thus, by choosing to undergo an orthogonal transformation to $X$ by $S$ (which is an orthogonal matrix for information), we are then led to write:
	
	given that for recall that:
	
 	we deduce that:
	
	In conclusion, we can assert that the curve defined by the equation above is of the same type as the original curve $\Gamma$.

		\item Case of a strictly positive determinant $\det(A)>0$:
		
		In the case where $\det(A)>0$, that is to say $ac-b^2>0$ or equivalently as we have mention it the previous remark $b^2-4ac<0$, $\lambda$ and $\mu$ have the same sign as by construction and for recall:
		
		The relation:
		
		still remains valid in this situation.
	
		But, in function of the value of $C_1$ we have prove earlier above how to interpret this equation. That is for recall (don't forget that we are in the space of real numbers!):
		\begin{itemize}
			\item If $C_1$ has the same sign as $\lambda$ and $\mu$ the equation has no solution and $\Gamma=0$
	
			\item If $C_1=0$ we see that the equation is reduced ton a point
	
			\item If $C_1$ has a sign opposed to $\lambda$ and $\mu$ then we recognized the equation of an ellipse
		\end{itemize}

		\item Case of a strictly negative determinant $\det(A)<0$:
		
		In the case where $\det(A)<0$, that is to say $ac-b^2<0$ or equivalently as we have mention it the previous remark $b^2-4ac>0$, $\lambda$ and $\mu$ have opposite signs as by construction and for recall:
		
		The relation:
		
		still remains valid in this situation.

		But, in function of the value of $C_1$ we have prove earlier above how to interpret this equation. That is for recall (don't forget that we are in the space of real numbers!):
		\begin{itemize}
			\item If $C_1=0$ then $\Gamma$ is the union of the secant lines
	
			\item If $C_1\neq 0$ we recognize the equation of a hyperbola
		\end{itemize}

		\item Case of a null determinant $\det(A)=0$:
		
		In the case where $\det(A)=0$, that is to say $ac-b^2=0$ or equivalently as we have mention it the previous remark $b^2-4ac=0$, one of the eigenvalues is equal to zero, for example $\mu=0$ (only because we must have $A\neq 0$). As we know:
		
		and by the equation
		
		we then have:
		
		Therefore:
		
		By developing explicitly, we get:
		
		The latter equation can be rewritten as:
		
		with obviously:
		
		If $e=0$ we can have three situations:
		\begin{itemize}
			\item $C_2=0$, then $\Gamma$ is a straight line

			\item $C_2$ and $\lambda$ have the same signs the $\Gamma=\varnothing$

			\item $C_2$ and $\lambda$ are of opposite signs, then $\Gamma$ is the union of two parallel distinct straight lines
		\end{itemize}
		If $e\neq 0$, then $\Gamma$ is a parabola.
	\end{itemize}
	Let us make a summary of this classification\label{type of conics determinant}
	\begin{itemize}
		\item If $\det(A)=ac-b^2>0$ (or $b^2-4ac<0$), then $\Gamma$ is either empty, reduced to a point, or an ellipse

		\item If $\det(A)=ac-b^2<0$ (or $b^2-4ac>0$), then $\Gamma$ is either the intersection of two secant straight lines, or an hyperbola

			\item If $\det(A)=ac-b^2=0$ (or $b^2-4ac=0$), then $\Gamma$ is either empty, or the parallel straight lines, or a parabola
	\end{itemize}
	And we can now also introduce (considering $b=0$ just to simplify the conceptual representation for human brain)\label{type of conics matrix approach}
	\begin{itemize}
		\item "\NewTerm{Positive-definite matrices}\index{positive-definite matrix}" that are such that:
		
		that will look like ellipsoids. For this $a$ and $c$ must then be of the same sign and positive satisfying then $-4ac<0$.
	
		\item "\NewTerm{Positive-semidefinite matrices}\index{positive-semidefinite matrix}" that are such that:
		
		that look like a half-cylinders (see it as a surface generated by parabolas). For this we must have $a$ or $c$ equal to zero satisfying the $-4ac=0$. Obviously, we see that positive-definite matrices are then a special case of positive-semidefinite matrices (and we would also be able to define negative-semidefinite matrices).
	
		\item "\NewTerm{Indefinite matrices}\index{indefinite matrix}" that are such that $\vec{x}^TA\vec{x}$ are negative or positive depending of the values of $\vec{x}$ since $a\neq 0$, $c\neq 0$ and that $\mathrm{sgn}(a)\neq\mathrm{sng}(c)$ (ie must be of opposite sign and not null) and that look like a saddle (surface generated by hyperbolas).
	\end{itemize}
	\begin{figure}[H]
		\centering
		\includegraphics[width=\textwidth]{img/geometry/positive_postive_semidefinite_indefinite_matrices.jpg}
		\caption{Positive definite, positive-semidefinite and indefinite matrices visual representation}
	\end{figure}

	\pagebreak
	\subsection{Parametrizations}
	Parametrization  is the process of deciding and defining the parameters necessary for a complete or relevant specification of a model or geometric object.
	
	Parametrization is also the process of finding parametric equations of a curve, a surface, or, more generally, a manifold or a variety, defined by an implicit equation. The inverse process is called "\NewTerm{implicitization}\index{implicitization}".
	
	Most often, parametrization is a mathematical process involving the identification of a complete set of effective coordinates or degrees of freedom of the system, process or model, without regard to their utility in some design. Parametrization of a line, surface or volume, for example, implies identification of a set of coordinates that allows one to uniquely identify any point (on the line, surface, or volume) with an ordered list of numbers. Each of the coordinates can be defined parametrically in the form of a parametric curve (one-dimensional) or a parametric equation (2+ dimensions).
	
	Generally, the minimum number of parameters required to describe a model or geometric object is equal to its dimension, and the scope of the parameters—within their allowed ranges—is the parameter space.
	
	\begin{tcolorbox}[title=Remark,colframe=black,arc=10pt]
		Parametrizations are not unique. The ordinary three-dimensional object can be parametrized (or "coordinatized") equally efficiently with Cartesian coordinates, cylindrical polar coordinates, spherical coordinates or other coordinate systems (\SeeChapter{see section Vector Calculus page \pageref{system of coordinates}}).
	\end{tcolorbox}
	As notice above for some of the forms described below, it is possible to select a different coordinate than the Cartesian coordinates such as, for example cylindrical or spherical coordinates which are in some cases much simpler to implement. We will try as far as possible to present the most important one.
	
	\subsubsection{Equation of the Plane}\label{equation of the plane}
	Given a plane $P$ for which we know a normal and unit vector $\vec{n}(a,b,c)$ but not the equation and $A(x,y,z)$ a point of the plane $P$.
	
	\begin{tcolorbox}[title=Remark,colframe=black,arc=10pt]
	If we don't have $\vec{n}$ but instead three points of the plane $P_1(x_1,y_1,z_1),P_2(x_2,y_2,z_2),P_3(x_3,y_3,z_3)$ we easily get  $\vec{n}$ by calculating the cross product:
	
	\end{tcolorbox}
	
	For a point $M$ with coordinates $(x, y, z)$ belongs to the plane $P$ it is necessary and sufficient that the vectors $\overrightarrow{AM}$ and $\vec{n}$ are orthogonal.
	
	So given the point by the vector $\overrightarrow{AM}$ of coordinates:
	
	If $\overrightarrow{AM}$ is perpendicular to $\vec{n}$ then the dot product must be zero as:
	
	Which can also be written as:
	
	such that we get the general Cartesian equation of the plane:
	
	This equation where $(a,b,c)\cong (0,0,0)$ that checks that the coordinates of any point $M(x,y,z)$ belongs to the plan $P$ is therefore named "\NewTerm{Cartesian equation of the plane $P$}\index{Cartesian equation of a plane}".
	
	If we write the equation with the direction cosines of $\vec{n}$ (\SeeChapter{see section Calculus Vector page \pageref{cosines directions}}), we therefore have also:
	
	\begin{tcolorbox}[title=Remark,colframe=black,arc=10pt]
	To get a cube in space, with only need six planes delimited by conditions such as $x\leq...,y\leq...,z\leq...$.
	\end{tcolorbox}	
	It is relatively easy to go from case by case from the Cartesian equation of the plane to the parametric equation of the plane. We can (with the usual precautions...) resume the equation:
	
	and rewriting it as follows:
	
	and therefore, the parametric equation of the plane in the space of dimension $3$ will be written:
	
	\begin{tcolorbox}[colframe=black,colback=white,sharp corners]
	\textbf{{\Large \ding{45}}Example:}\\\\
	With Maple 4.00b:\\

	\texttt{>a:=3:b:=-2:c:=1:d:=5:\\
	>plot3d([x,y,(-d-b*y+a*x)/c],x=-2..2,y=-2..2, orientation=[-87,81],style=PATCH,
axes=NORMAL);
	}
	\begin{figure}[H]
		\centering
		\includegraphics{img/geometry/plane.jpg}
		\caption{Example of plane with Maple 4.00b}
	\end{figure}
	\end{tcolorbox}
	\begin{theorem}
	The vector $\vec{n}=(a,b,c)$ is normal\index{normal vector} to the plane of equation:
	
	\end{theorem}
	\begin{dem}
	Take two points $P_1=(x_1,y_1,z:1)$ and $P_2=(x_2,y_2,z_2)$. Then:
	
	As $\vec{P}_2-\vec{P}_1=\overrightarrow{P_2P_1}$ is in the plane $ax+by+cz=d$ and that by subtracting both equations we have:
	
	that is equivalent to:
	
	This proves indeed that\label{vector normal plane}:
	
	is perpendicular to the plane.
	\begin{flushright}
		$\blacksquare$  Q.E.D.
	\end{flushright}
	\end{dem}
	In Projective Geometry (see section of the same name page \pageref{projective geometry}) an important case to consider is the plane of perpendicular to the $\vec{n}=(1,1,1)$ that give the isometric perspective that keeps lengths of cube of equal size (this plane being assimilated to the table of the observer)\label{isometric plane}:
	\begin{figure}[H]
		\centering
		\includegraphics{img/geometry/isometric_perspective_plane.jpg}
	\end{figure}

	So using the previous relation we get:
	
	Now to calculate the angle between this plane and the $xy$-plane we calculate the dot product between the respective normal vectors and subtract $\pi$ such that:
	
	Thus:
	
	 Therefore:
	
	 Therefore the angle $\beta$ between the two planes is in degrees (as it is usage in perspective geometry to work in degrees):
	 
	 This is why perspective geometry angles are rounded to the famous $30^\circ$
	
	\pagebreak	
	\subsubsection{Equation of the Straight line}\label{equation of the straight line}
	As we saw it in the section of Functional Analysis, a straight line in the plane can be described by the function:
	
	and we have also proved how to determine the equation of the mediator of two points in the plane (we had mentioned also that the proof was more on the order of Analytical Geometry as Functional Analysis).
	
	The general Cartesian equation of the straight line is then simply given by:
	
	or sometimes:
	
	Indeed, simplifying we fall back on the "\NewTerm{reduced Cartesian equation}\index{reduced Cartesian equation of a line}":
	
	with by definition:
	
	We also saw in the section Calculus that straight lines with inequalities rather than strict equalities gives the possibility to define special areas on the plane:
	\begin{center}
	\begin{tikzpicture}

    \draw[gray!50, thin, step=0.5] (-1,-3) grid (5,4);
    \draw[very thick,->] (-1,0) -- (5.2,0) node[right] {$x_1$};
    \draw[very thick,->] (0,-3) -- (0,4.2) node[above] {$x_2$};

    \foreach \x in {-1,...,5} \draw (\x,0.05) -- (\x,-0.05) node[below] {\tiny\x};
    \foreach \y in {-3,...,4} \draw (-0.05,\y) -- (0.05,\y) node[right] {\tiny\y};

    \fill[blue!50!cyan,opacity=0.3] (8/3,1/3) -- (1,2) -- (13/3,11/3) -- cycle;

    \draw (-1,4) -- node[below,sloped] {\tiny$x_1+x_2\geq3$} (5,-2);
    \draw (1,-3) -- (3,1) -- node[below left,sloped] {\tiny$2x_1-x_2\leq5$} (4.5,4);
    \draw (-1,1) -- node[above,sloped] {\tiny$-x_1+2x_2\leq3$} (5,4);

	\end{tikzpicture}
	\end{center}
	\textbf{Definition (\#\mydef):} We name "\NewTerm{direction vector}\index{direction vector}" of a line $(D)$, all non-zero vector in the same direction as the straight line.
	
	Let us now prove two small friendly theorems:
	\begin{theorem}
	If a line has for equation $y=ax+b$ then the vector:
	
	is a direction vector for this line.
	\end{theorem}
	\begin{dem}
	Given $D:ax+b$ and $A, B$ two points of this line taken such as ${A(0,b),B(1,a+b)}$. Since $A, B$ are two points $D$ then $\overrightarrow{AB}$ is a direction vector of $D$ then:
	
	On the way a small interesting corollary that has an application in physics!:
	
	If a straight line $D_1$ has direction vector being:
	
	and another straight line $D_2$ with a direction vector being:
	
	therefore their scalar product (\SeeChapter{see section Vector Calculus page \pageref{dot product}}) is zero:
	
	that shows that two straight lines which multiplication of slopes (second coordinate of the direction vector) is equal to $-1$ are perpendicular!
	\begin{flushright}
		$\blacksquare$  Q.E.D.
	\end{flushright}
	\end{dem}
	\begin{theorem}
	If a line has for equation $ax+by+c=0$ then the vector:
	
	is a direction vector for this line.
	\end{theorem}
	\begin{dem}
	Given:
	
	therefore:
	
	then the vector:
	
	is a direction vector $D$ and any vector:
	
	with $\alpha\in \mathbb{R}$.
	Thus, there are infinite ways to define the same line, because the line is composed of an infinite number of points (all of which can serve as an anchor point) and there is an infinity of multiple direction vector!
	\begin{flushright}
		$\blacksquare$  Q.E.D.
	\end{flushright}
	\end{dem}
	
	\pagebreak
	\paragraph{Distance from a line to a point}\mbox{}\\\\
	Often we seek the distance between a line and a point external to it. Thus, consider the following figure:
	\begin{figure}[H]
		\centering
		\includegraphics{img/geometry/point_to_line.jpg}
		\caption{Representation of the search of the distance from a point to a line}
	\end{figure}
	With the point $H$ the orthogonal projection of $A$ on the straight line $d$, $P$ an arbitrary point of the straight line $d$ and $\vec{n}$ any orthogonal vector (normal) to $d$.
	
	We have (\SeeChapter{see section Vector Calculus page \pageref{dot product}}):
	
	As $\alpha=0$ or $\alpha=\pi$. Therefore:
	
	where $\delta$ is the distance (we can note denote the distance with the letter $d$ as we did at the beginning of this chapter, otherwise there would be confusion a with the $d$ chosen to represent the straight right in this development).
	
	So we get the relation:
	
	Consider now explicitly the point $A(x_0,y_0)$ and the straight line equation: $d:ax+by+c=0$.
	
	Let us choose a point $P\in d:P(0,-c/b)$ and a vector normal $\vec{n}$, normal to $d:\vec{n}\begin{pmatrix}a \\ b\end{pmatrix}$. Indeed, as we have proven just before that:
	
	is a direction vector of a straight line $D$ we have well:
	
	and thus $\vec{v}$ and $\vec{n}$ are perpendicular.
	
	Thus, by applying the previous relation, we have:
	
	and therefore:
	
	
	\paragraph{Line defined by the intersection of planes}\mbox{}\\\\
	If we now consider two non-parallel planes in space, their intersection is a straight line. Indeed, consider two planes of respective equations:
	
	and $D$ and their straight line of intersection.
	
	Obviously a point $M(x_0,y_0,z_0)$ in space is located on the straight line $D$ if and only if the point $M$ satisfies the system of equations:
	
	An example with images and numerical application is given in the section of Theoretical Computing.
	\begin{tcolorbox}[title=Remark,colframe=black,arc=10pt]
	While in $\mathbb{R}^2$ a straight line is fully characterized by an equation of the type $ax+by+c=0$, in space, a single equation of the form $ax+by+cz+d=0$ obviously characterizes a plan. To characterize a straight line outside of the planes of the axes, it is necessary (parametric equation apart) to have two equations of planes.
	\end{tcolorbox}	
	
	\paragraph{Parametric equation of a line in $\mathbb{R}^3$}\mbox{}\\\\
	It is relatively trivial (but we will still prove it) that the parametric equation in $\mathbb{R}^3$ of a straight line is a kind of system of equations:
	
	Thus, each component increases linearly with respect to the same variable to a constant and a factor. This can also be written in vector form (more traditional):
	
	The vector $\vec{v}=(a,b,c)$ is obviously named the "direction vector".
	\begin{dem}
	So we have the system of equations (two equations with three unknowns, and therefore one unknown will be indeterminate):
	
	Let us eliminate one of the variables (arbitrarily we start with $z$):
	
	where $\alpha=c/c'$ therefore:
	
	So (it's a little stupid to write but...):
	
	Similarly with $y$ such that $\beta=b/b'$ we have:
	
	Therefore:
	
	Finally we have:
	
	The direction vector and the vector or ordinates have all constant components. This allows us to write more generally:
	
	\begin{flushright}
		$\blacksquare$  Q.E.D.
	\end{flushright}
	\end{dem}
	\begin{tcolorbox}[title=Remarks,colframe=black,arc=10pt]
	\textbf{R1.} The equation of a straight line is almost one of the most important thing in the synthesis of 3 images because from this latter we can build polygons and assemble them to build more complex three-dimensional shapes.\\
	
	\textbf{R2.} As already seen before with an example, to determine if a line is perpendicular to a plane we must determine at least two intersecting lines in the same plane and perform the cross product of their direction vector and then calculate the dot product between the result of the vector product and the first straight light for which we seek the orthogonality. Indeed, a straight single of the plane does not permit to determine the orientation of the latter; wee need a least two straight lines.
	\end{tcolorbox}
	
	\subsubsection{Equation of a Square}
	In 1992, Manuel Hernandez Gusting introduced an algebraic equation for representing an intermediate shape between the circle and the square. His equation included a parameter $s$ that can be used to blend the circle and the square smoothly thanks to a quite simple equation named "\NewTerm{FG-squircle}\index{FG-squircle}\label{fg squircle}" and given by:
	
	The squareness parameter $s$ can have any value between $0$ and $1$. When $s = 0$, the equation produces a circle with radius $k$. When $s = 1$, the equation produces a square with a side length of $2k$. In between, the equation produces a smooth curve that interpolates between the two shapes.
	
	We will focus here to the case $k=s=1$ that gives therefore:
	
	and when $-1<x<1$ and $-1<y<1$ this corresponds to a square centred and the origin. Indeed with Maple:
	
	\texttt{>with(plots):\\
	>implicit plot(x\string^2+y\string^2-x\string^2*y\string^2= 1,x=-1..1,y=-1..1);
	}
	\begin{figure}[H]
		\centering
		\includegraphics{img/geometry/fg_squircle.jpg}
		\caption{FG-squircle representation in Maple 4.00}
	\end{figure}

	This implicit equation of the square will be useful to us to study the isomorphism of a circle and a square in the section of Topology.
	
	\subsubsection{Equation of a Cycloid}\label{cycloid curve}
	A cycloid is the curve traced by a point on the rim of a circular wheel as the wheel rolls along a straight line without slippage. It is an example of a roulette, a curve generated by a curve rolling on another curve.

	The cycloid is an important curve in physics! Indeed as we will prove it in the section of Classical Mechanics, the cycloid, with the cusps pointing upward, is the curve of fastest descent under constant gravity, and is also the form of a curve for which the period of an object in descent on the curve does not depend on the object's starting position. It is also a mathematical curve on which we will fall back during our detailed study of Friedmann-Lemaître-Robertson-Walker Universe Models in the section of Cosmology (based on the mathematics of General Relativity)!

	\begin{center}
	  \begin{tikzpicture}[scale=1.8]
	  \coordinate (O) at (0,0);
	  \coordinate (A) at (0,3);
	  \def\r{1} % radius
	  \def\c{1.4} % center
	  \coordinate (C) at (\c, \r);
	
	
	  \draw[-latex] (O) -- (A) node[anchor=south] {$y$};
	  \draw[-latex] (O) -- (2.6*pi,0) node[anchor=west] {$x$};
	  \draw[red,domain=-0.5*pi:2.5*pi,samples=50, line width=1] 
	       plot ({\x - sin(\x r)},{1 - cos(\x r)});
	  \draw[blue, line width=1] (C) circle (\r);
	  \draw[] (C) circle (\r);
	
	  % coordinate x 
	  \def\x{0.4} % coordinate x
	  \def\y{0.83} % coordinate y
	  \def\xa{0.3} % coordinate x for arc left
	  \def\ya{1.2} % coordinate y for arc left
	  \coordinate (X) at (\x, 0 );
	  \coordinate (Y) at (0, \y );
	  \coordinate (XY) at (\x, \y );
	
	  \node[anchor=north] at (X) {$x$} ;
	
	  % draw center of circle
	  \draw[fill=blue] (C) circle (1pt);
	
	  % draw radius of the circle
	  \draw[] (C) -- node[anchor=south] {\; $a$} (XY);
	
	  % bottom of circle, radius to the bottom
	  \coordinate (B) at (\c, 0);
	  \draw[] (C) -- (B) node[anchor=north] {$a \, \theta$};
	
	  % projections of point XY
	  \draw[dotted] (XY) -- (X);
	  \draw[dotted] (XY) -- (Y) node[anchor=east, xshift=1mm] {$\quad y$};
	
	  % arc theta
	  % start arc
	  \coordinate (S) at (\c, 0.4);
	  \draw[->] (S) arc (-90:-165:0.6);
	  \node[xshift=-2mm, yshift=-2mm] at (C) {\scriptsize $\theta$};
	
	  % arc above
	  \coordinate (AA) at (\xa, \ya);
	  \draw[-latex, rotate=25] (AA) arc (-220:-260:1.3);
	
	  % arc below
	  \def\xb{2.5} % coordinate x for arc bottom
	  \def\yb{0.8} % coordinate y for arc bottom
	  \coordinate (AB) at (\xb, \yb);
	  \draw[-latex, rotate=-10] (AB) arc (-5:-45:1.3);
	
	  % XY dot
	  \draw[fill=black] (XY) circle (1pt);
	
	  % top label
	  \coordinate (T) at (pi, 2);
	  \node[anchor=south] at (T)  {$(\pi a, 2 a )$} ;
	  \draw[fill=black] (T) circle (1pt);
	
	  % equations
	  \coordinate (E) at ( 4,1.2);
	  \coordinate (F) at ( 4,0.9);
	  \node[] at (E) { $x=a(\theta - \sin (\theta))$};
	  \node[] at (F) { $y=a(1 - \cos(\theta))$};
	
	  % label 2pi a
	  \coordinate (TPA) at (2*pi, 0);
	  \node[anchor=north] at (TPA) {$2 \pi a$};
	
	  \end{tikzpicture}
	\end{center}
	We see on the figure above that as an arc $\theta$ of the circle of radius $a$ has a length:
	
	This corresponds to the abscissa distance in the range $[0,2\pi a]$.
	
	But to get the $x$  value of the tracked point we must subtract the horizontal projection $a'$ of the radius $a$ on the abscissa and as:
	
	Therefore:
	
	And the same procedure applies for $y$.
	
	\begin{tcolorbox}[title=Remark,colframe=black,arc=10pt]
	Notice that it means if $\theta$ do a whole round ($\theta=2\pi$), then the horizontal travelled distance will be 
	\end{tcolorbox}

	So we finally get:
	
	Notice that the complete arc length $L$ of the cycloid is given by:
	
	
	\subsubsection{Equation of an Epicycloid}
	In geometry, an epicycloid is a plane curve produced by tracing the path of a chosen point on the circumference of a circle - named an "epicycle" - which rolls without slipping around a fixed circle. It is a particular kind of roulette.
	\begin{figure}[H]
		\centering
		\includegraphics{img/geometry/epicycloid_path.jpg}
		\caption[Epicycloid path]{Epicycloid path (source: Wikipedia, Sam Derbyshire)}
	\end{figure}
	As we will see in the section of Mechanical Engineering page \pageref{epicyclic gears}, the epicycloid is used to build efficient energy gearing geometry shapes!
	
	For the need of the section of Mechanical Engineering we need to determine the parametric equation of the epicycloid . For this consider the following figure:
	\begin{figure}[H]
		\centering
		\includegraphics[scale=0.6]{img/geometry/epicycloid_sketch_for_proof.jpg}
		\caption[Epicycloid sketch for proof]{Epicycloid sketch for proof (source: Wikipedia, Sam Derbyshire)}
	\end{figure}
	We assume that the position of $P$ is what we want to solve, {$\alpha$  is the radian from the tangential point to the moving point $P$, and $\theta$  is the radian from the starting point to the tangential point.

	Since there is no sliding between the two cycles, then we have that:
	
	By the definition of radian (which is the rate arc over radius), then we have that:
		
	From these two conditions, we get the identity:
	
	By rearranging, we get the relation between $\alpha$  and $\theta$, which is
	
	From the figure, we see the position of the point $P$ is clearly given by:
	
	
	\subsubsection{Equation of a Spiral}
	A "\NewTerm{spiral}\index{spiral}" is a curve in the plane or in the space, which runs around a center in a special way. It sometimes a useful shape in some industrial fields (robotics, watchmaking, texts designs, etc.).
	
	We can make a spiral by two motions of a point: There is a uniform motion in a fixed direction and a motion in a circle with constant speed. Both motions start at the same point. This construction is named the "\NewTerm{Archimedean Spiral}\index{Archimedean Spiral}"  (also known as the "arithmetic spiral").
	
	We get the Archimedean Spiral by analogy to the circle parametric equations.  Let us recall that the parametric equation of a circle is given by:
	
	Let us rewrite is a following:
	
	where $a$ is a chosen constant.
	
	Now for the Archimedean Spiral, the idea is to make $R$ also proportional of $t$. Therefore the parametric equation is:
	
	Below is an example of an Archimedean Spiral with in polar coordinate $r=a\theta$:
	\begin{figure}[H]
		\centering
		\begin{tikzpicture}
	    \begin{polaraxis}
	      [no marks,samples=201,smooth,domain=0:4]
	      \addplot+ (4*180*x,x);
	    \end{polaraxis}
		\end{tikzpicture}
		\caption{Archimedean Spiral}
	\end{figure}
	
	The distances between the spiral branches are the same. More exactly: The distances of intersection points along a line through the origin are the same. 
	
	\subsubsection{Equation of an Hypocycloid}
	In geometry, a hypocycloid is a special plane curve generated by the trace of a fixed point on a small circle that rolls within a larger circle. It is comparable to the cycloid or epicycloid but instead of the circle rolling along a line, it rolls within a circle.
	\begin{figure}[H]
		\centering
		\includegraphics{img/geometry/hipocycloid_path.jpg}
		\caption[Hypocycloid path]{Hypocycloid path (source: Wikipedia, Sam Derbyshire)}
	\end{figure}
	When we know the parametric equation of the epicycloid, that of the hypocycloid is immediate:
	
	Now for the section of Mechanical Engineering it will help if the reader compare side by side and epicycloid and a hypocycloid one:
	\begin{figure}[H]
		\centering
		\begin{subfigure}{.4\textwidth}
		  \centering
		  \includegraphics[scale=0.9]{img/geometry/epicycloid.jpg}
		  \caption[Epicycloid]{Epicycloid (source: Wikipedia)}
		\end{subfigure}
		\begin{subfigure}{.4\textwidth}
		  \centering
		  \includegraphics[scale=1]{img/geometry/hipocycloid.jpg}
		  \caption{Hipocycloid}
		\end{subfigure}
	\end{figure}
		
	
	
	\pagebreak
	\subsubsection{Surface of revolution}
	More generally many surfaces (and also some of which we've seen before as the sphere, tore and cylinder) can be described by revolving a primary form of smaller size and then by rotation.
	
	\textbf{Definition (\#\mydef):} A "\NewTerm{surface of revolution}\index{surface of revolution}" is a surface obtained by rotating a plane curve (e.g. $z=f(x)$), named the "\NewTerm{generatrix}\index{generatrix}", around the $z$-axis (for example!). So we pass from a plane of $\mathbb{R}^2$ to a basis in $\mathbb{R}^3$, the $x$-axis then generates a plane that has become the $y$O$z$ plane.
	\begin{figure}[H]
		\centering
		\includegraphics[scale=0.85]{img/geometry/revolution_surface.jpg}
	\end{figure}
	Let us see seven famous cases:
	
	\paragraph{Cone}\label{cone of revolution}\mbox{}\\\\
	Consider a circular cone of vertex O (origin) with a circle of center $(0,0, h)$ (where $h$ is a positive real number corresponding to the height of the) and radius $R$:
	\begin{figure}[H]
		\centering
		\includegraphics{img/geometry/cone.jpg}
		\caption{Representation of a cone}
	\end{figure}
	A parametric representation of the circle at the height $h$ is given as we already proved earlier by:
	
	where $t$ belongs to the interval $[-\pi,\pi]$.
	
	By extension, the parametric representation of a cone is:
	
	this can also be written in vector form (more traditional):
	
	
	In other words, the circle linearly propagates in all directions.
	\begin{tcolorbox}[colframe=black,colback=white,sharp corners]
	\textbf{{\Large \ding{45}}Example:}\\\\
	With Maple 4.00b:\\

	\texttt{>r:=1:h:=4:\\
	>plot3d([k*r*cos(t),k*r*sin(t),k*h],k=0..10,t=0..2*Pi,\\
	orientation=[50,60],style=PATCH,axes=NORMAL);
	}
	\begin{figure}[H]
		\centering
		\includegraphics{img/geometry/cone_maple.jpg}
		\caption{Parametric representation of a cone with Maple 4.00b}
	\end{figure}
	\end{tcolorbox}
	This suggests also the following cartesian equation of this cone:
	
	As we have $\tan(\alpha)=r/h$ we can write:
	
	Therefore:
	
	which is the Cartesian equation of a cone in space that we will see again in the section of Special Relativity in our study of light cones.
	
	\paragraph{Sphere}\label{sphere}\mbox{}\\\\
	Consider the orthonormal basis ${\vec{i},\vec{j},\vec{k}}$, Given the sphere $S^2$ of  center $\Omega(a,b,c)$ and of radius $r$:
	\begin{figure}[H]
		\centering
		\includegraphics{img/geometry/sphere.jpg}
		\caption{Representation of a sphere}
	\end{figure}
	$M(x,y,z)$ belongs to the sphere  $S^2$ of radius $r$ fi and only if:
	
	that is to say by applying Pythagoras theorem:
	
	Hence the "\NewTerm{Cartesian equation of the sphere}\index{Cartesian equation of the sphere}" in the basis ${\vec{i},\vec{j},\vec{k}}$:
	
	There is another way to describe the sphere using the parametric equation. Indeed, we saw in the section of Vector Calculus that the transformation of the Cartesian coordinates to spherical coordinates is given by the following curvilinear coordinates:
	
	Therefore we have well (by a change of variable this also represent a sphere not centered at the origin):
	
	So the parametric equation of the sphere is well:
	
	We fall back therefore on the Cartesian equation of a sphere to a given translation constant.
	\begin{tcolorbox}[colframe=black,colback=white,sharp corners]
	\textbf{{\Large \ding{45}}Example:}\\\\
	With Maple 4.00b for a sphere of unit radius:\\

	\texttt{>r:=1:h:=4:\\
>plot3d([sin(theta)*cos(phi),sin(theta)*sin(phi),cos(theta)],\\
theta=0..Pi,phi=-Pi..Pi,scaling=CONSTRAINED,orientation=[50,60],\\
style=PATCH,axes=NORMAL);
	}
	\begin{figure}[H]
		\centering
		\includegraphics{img/geometry/sphere_maple.jpg}
		\caption{Parametric representation of a sphere with Maple 4.00b}
	\end{figure}
	\end{tcolorbox}
	
	\subparagraph{Great circle of sphere}\mbox{}\\\\
	We will introduce here a concept that will be useful to us when we will study the geodesic of a sphere in the section of Analytical Mechanics.
	
	\textbf{Definition (\#\mydef):}  A "\NewTerm{great circle}\index{great circle}", also known as an "\NewTerm{orthodrome}\index{orthodrome}" or "\NewTerm{Riemannian circle}\index{Riemannian circle}", of a sphere is the intersection of the sphere and a plane that passes through the center point of the sphere. This partial case of a circle of a sphere is opposed to a "\NewTerm{small circle}\index{small circle}", the intersection of the sphere and a plane that does not pass through the center. Any diameter of any great circle coincides with a diameter of the sphere, and therefore all great circles have the same circumference as each other, and have the same center as the sphere. A great circle is the largest circle that can be drawn on any given sphere. Every circle in Euclidean 3-space is a great circle of exactly one sphere.
	\begin{figure}[H]
		\centering
		\includegraphics[scale=0.5]{img/geometry/great_circle.jpg}
		\caption{A great circle divides the sphere in two equal hemispheres}
	\end{figure}
	Let use choose now a unit normal vector to a plane passing through the origin using Euler angles be:
	
	where for rotated our such that the component $n_2$ is equal to $0$ (as it is always possible to do so whatever the plan we choose that pass through the origin!).
	
	If we choose a sphere of radius $r=1$ such that it's parametric equation is written as:
	
	And as the set of points that are at the intersection of the plane and the sphere are by construction such that:
	
	Therefore we get:
	
	After rearranging:
	
	Let us put:
	
	and using the definition of the cotangent:
	
	This is therefore the "\NewTerm{equation of a great circle}\index{equation of a great circle}".
	
	\subparagraph{$n$-Sphere volume}\mbox{}\\\\
	Before we deal with the subject, we want to thanks professor Howard Haber that has provided us the major \LaTeX source code of the text below that will be useful to us in the sections of Statistical Mechanics (especially to derive the Sackur-Tetrode equation) and of General Relativity!
	
	An $n$-dimensional hypersphere of radius $R$ consists of the locus of points such that the distance from the origin is less than or equal to $R$.
	
	A point in an $n$-dimensional Euclidean space is designated by $(x_1\,,\,x_2\,,\,\ldots\,,\,x_n)$.

	In equation form, the hypersphere corresponds then by definition to the set of points such that:
	
	To compute the volume of this hypersphere, we simply integrate the infinitesimal volume element $\mathrm{d}V=\mathrm{d}x_1 \mathrm{d}x_2\cdots \mathrm{d}x_n$ over the region of $n$-dimensional space indicated by \ref{eqsphere}.  We wish to compute this volume $V_n(R)$. Explicitly:
	
	The factor of $R^n$ is a consequence of dimensional analysis.
	
	The surface-area of the $n$-dimensional hypersphere defined by will be denoted in what follows by $S_{n-1}(R)$.
	
	The surface of the hypersphere corresponds to the locus of points such that:
	
	We can construct the volume $V_n(R)$ by adding infinitely thin spherical shells of radius $0\leq r\leq R$.  In equation form, this reads:
	
	And as we know:
	
	where we have used \ref{vn}.  Thus, the only remaining task is to compute $C_n$.
	
	In order to obtain a better intuition on the meaning of $C_n$, let us equate the two expressions we have for $V_n(R)$, namely \ref{vn} and \ref{vds}.
	
	In the latter, $S_{n-1}(r)$ is determined by \ref{sn}.  Thus:
	
	This last relation is simply the evaluation of an $n$-dimensional integral either in rectangular co-ordinates or hyper-spherical co-ordinates.  In $n$-dimensions,
	we would write
	
	where $\mathrm{d}\Omega_{n-1}$ contains all of the angular factors.  For example, for
	$n=2$, $\mathrm{d}\Omega_1=\mathrm{d}\theta$;  for $n=3$, $\mathrm{d}\Omega_2=\sin\theta\, \mathrm{d}\theta\, d\phi$, etc.
	
	One could explicitly define the $n-1$ angular variables in $n$-dimensions.
	
	However, if we are integrating over a spherically symmetric function, then
	we will simply integrate over $\mathrm{d}\Omega_{n-1}$. Comparing \ref{polar} and \ref{omega},
	we see that:
	
	
	We will present a trick for computing $C_n$ without ever explicitly parametrizing the angles of the hyper-spherical coordinate system.  For this purpose, consider the function:
	
	If we integrate this function over the full $n$-dimensional space in both rectangular and hyper-spherical coordinates, we obtain:
	
	Since the integrand on the right hand side depends only on $r$ (there is no angular dependence), we can immediately perform the integral over $\mathrm{d}\Omega_{n-1}$.   Using \ref{ncn}:
	
	All the integrals above can be evaluated exactly using Gauss integral (\SeeChapter{see section Statistics page \pageref{Gauss integral}}) and Euler-Gamma integral (\SeeChapter{see section Differential and Integral Calculus page \pageref{gamma euler function}}):
	
	Inserting these results into \ref{intrel}, we get :
	
	where we used the property of the Gamma function, $x\Gamma(x)=\Gamma(x+1)$, at the final
	step.  Solving for $C_n$, we get:
	
	
	Although we chose a particular function $f(x_1\,,\,x_2\,,\,\ldots\,,\,x_n)$ to get the final result for $C_n$, it is clear that the value of $C_n$ is independent of this function.  After all, the defining equation for $C_n$ [\ref{ncn}] makes no reference to any function.  As a check, let us evaluate $n\,C_n$ for $n=2$ and $n=3$:
	
	In the last computation, we used $\Gamma\left(\frac{5}{2}\right)=\frac{3}{2}
	\Gamma\left(\frac{3}{2}\right)=(\frac{3}{2})\,(\frac{1}{2})\,\Gamma\left(\frac{1}{2}\right)=
	\frac{3}{4}\sqrt{\pi}$.  Indeed, we have produced the correct values for the
	integration over the angular factors in two and three dimensions.
	
	We are finally ready to compute the volume and surface-area of an $n$-dimensional hypersphere.  Inserting \ref{cn} into \ref{vn} and \ref{sn}, we find:
	
	The expression for $S_n(R)$ can be slightly simplified by writing $\Gamma\left(1+\dfrac{n}{2}\right)=\dfrac{n}{2}\,\Gamma\left(\dfrac{n}{2}\right)$. This	yields:
	
	You should check that you get the expected results for $n=2$ and $n=3$.  For example,$V_3(R)=\frac{4}{3}\pi R^3$ and $S_2(R)=4\pi R^2$.
	
	Finally, it is amusing to note that $\lim_{n\to\infty} V_n(R)=0$ and $\lim_{n\to\infty} S_n(R)=0$.  
	
	To understand this peculiar behaviour, consider a hypercube in $n$-dimensional space measuring one unit on each side.  The total volume of this hypercube is $1$. We can fit a hypersphere of diameter 1 (or radius $\frac{1}{2}$) inside the hypercube such that the surface of the hypersphere just touches each of the walls of the hypercube. Then $1-V_n(\frac{1}{2})$ is the volume inside the cube but outside the hypersphere.  
	
	In particular, as $n$ becomes large, $1-V_n(\frac{1}{2})$ rapidly approaches $1$, which is consistent 	with the assertion that $\lim_{n\to\infty} V_n(R)=0$.  This simply means that as the number of dimensions becomes larger and larger, the amount of space outside the hypersphere (but inside the cube) is become relatively more and more important.  This is already happening as you go from 2 to 3 dimensions.  So you can check your intuition by inscribing a circle in a unit square and a sphere in a unit cube and computing the total volume in three dimensions (area in two dimensions) outside the sphere (circle) but inside the cube (square).  If you take the ratio of volumes (areas) of the sphere (circle) to that of the cube (square), this ratio actually \textit{decreases} as you go from 2 dimensions to 3 dimensions!
	
	This shows that the integer dimension with maximum integer coefficient is $n = 5$. 
	
	\begin{table}[H]
		\centering
		\begin{tabular}{|c|c|c|}
		\rowcolor[HTML]{C0C0C0} 
		\textbf{Dimension} & \textbf{Volume} & \textbf{Volume at $\pmb{r=1}$} \\
		$2$ & $\pi r^2$ & $3.14159$ \\ \hline
		$3$ & $4/3\pi r^3$ & $4.18879$ \\ \hline
		$4$ & $1/2 \pi^2 r^4$ & $4.93480$ \\ \hline
		$5$ & $8/15 \pi^2 r^5$ & $5.26379$ \\ \hline
		$6$ & $1/6 \pi^3 r^6$ & $5.16771$ \\ \hline
		$7$ & $16/105 \pi^3 r^7$ & $4.72477$ \\ \hline
		$8$ & $1/24 \pi^4 r^8$ & $4.05871$ \\ \hline
		$9$ & $32/945 \pi^4 r^9$ & $3.29851$ \\ \hline
		$10$ & $1/120 \pi^5 r^{10}$ & $2.55016$\\ \hline
		\end{tabular}
	\end{table}
	
	\pagebreak
	\paragraph{Ellipsoid (spheroid)}\mbox{}\\\\
	We saw in our study of conic that the cartesian equation of an ellipse in the plane was given by:
	
	with $a, b$ the two axes of the ellipse (small and large one).
	
	Thus, without rigorous proof, we can verify by hand or using computers that the Cartesian equation:
	
	is an ellipsoid:
	\begin{figure}[H]
		\centering
		\includegraphics{img/geometry/ellipsoid.jpg}
		\caption{Representation of an ellipsoid}
	\end{figure}
	However, there is another way of describing an ellipsoid using also curvilinear coordinates:
	
	So the parametric equation of an ellipsoid will be:
	
	We can see that this is simply the parametric equation of a sphere but with different radii along the axes of the chosen basis.
	\begin{tcolorbox}[colframe=black,colback=white,sharp corners]
	\textbf{{\Large \ding{45}}Example:}\\\\
	With Maple 4.00b:\\

	\texttt{>a:=100:b:=2:c:=20:\\
>plot3d([a*cos(theta)*cos(lambda),b*cos(theta)*sin(lambda),\\
c*sin(theta)],lambda=0..Pi,theta=-Pi..Pi,orientation=[50,60],\\
style=PATCH,axes=NORMAL);
	}
	\begin{figure}[H]
		\centering
		\includegraphics{img/geometry/ellipsoid_maple.jpg}
		\caption{Parametric representation of an ellipsoid with Maple 4.00b}
	\end{figure}
	\end{tcolorbox}
	So we have:
	
	hence:
	
	Finally:
	
	Is astronomy (\SeeChapter{see section Astronomy page \pageref{astronomy}}) the ellipsoid is mainly named "\NewTerm{oblate spheroid}\index{oblate spheroid}".
	\begin{figure}[H]
		\centering
		\includegraphics{img/geometry/spheroid.jpg}
		\caption[Left oblate spheroid, right prolate spheroid]{Left oblate spheroid, right prolate spheroid (source: Wikipedia)}
	\end{figure}
	and we have technically a oblate spheroid when $c<a$ and prolate one when $c>a$. Obviously the case $a = c$ reduces to a sphere.
	
	\begin{tcolorbox}[title=Remark,colframe=black,arc=10pt]
	Because of the combined effects of gravity and rotation, the Earth's shape is not quite a sphere but instead is slightly flattened in the direction of its axis of rotation. For that reason, in cartography the Earth is often approximated by an oblate spheroid instead of a sphere. The current World Geodetic System model uses a spheroid whose radius is $6,378.137$ [km] at the equator and $6,356.752$ [km] at the poles.
	\end{tcolorbox}
	The aspect ratio of an oblate spheroid/ellipse, $b : a$, is the ratio of the polar to equatorial lengths, while the flattening (also named "oblateness") $f$, is the ratio of the equatorial-polar length difference to the equatorial length:
	
	
	\pagebreak
	\paragraph{Cylinder}\label{cylinder}\mbox{}\\\\
	It goes without saying that the parametric equation of a cylinder of radius $r$ is given by:
	
	We see well that the components $x, y$ satisfy the Cartesian equation of a circle for every $z$ since:
	
	\begin{tcolorbox}[colframe=black,colback=white,sharp corners]
	\textbf{{\Large \ding{45}}Example:}\\\\
	With Maple 4.00b a cylinder of radius $r=1$:\\

	\texttt{>plot3d([cos(phi),sin(phi),z],phi=-Pi..Pi,z=0..2,orientation=[50,60],\\
	style=PATCH,axes=NORMAL);
	}
	\begin{figure}[H]
		\centering
		\includegraphics{img/geometry/cylinder.jpg}
		\caption{Parametric representation of a cylinder with Maple 4.00b}
	\end{figure}
	\end{tcolorbox}
	It is obvious that the parametric equation of an elliptical base cylinder is given by:
	
	which also satisfies the parametric equation of an ellipse in the plane:
	
	
	\pagebreak
	\paragraph{Paraboloid}\mbox{}\\\\
	Given the equation of the parabola around the $z$-axis:
	
	with for reminder:
	
	We have obviously (by cutting the paraboloid by a horizontal plane which gives therefore a circle of radius $r$) the relation:
	
	named "\NewTerm{cylinder equation}\index{cylinder equation}". But we also have by applying the Pythagorean theorem in the circle above:
	
	Therefore:
	
	Hence we get the "\NewTerm{Cartesian equation of the paraboloid (of revolution)}\index{Cartesian equation of the paraboloid }":
	
	We build the parametric equation of the paraboloid exactly in the same way as for the cone, at the difference that the evolution along the $z$-axis will not be linear relatively to a parameter $k$ but to the square of the latter. Therefore we have:
	
	\begin{tcolorbox}[colframe=black,colback=white,sharp corners]
	\textbf{{\Large \ding{45}}Example:}\\\\
	With Maple 4.00b a paraboloid with $h=1,p=1$:\\

	\texttt{>p:=1:h:=1:\\
	>plot3d([k*1/(2*p)*cos(t),k*1/(2*p)*sin(t),k\string^2*h],k=0..10,\\
	t=0..2*Pi,orientation=[50,60],style=PATCH,axes=NORMAL); 
	}
	\begin{figure}[H]
		\centering
		\includegraphics{img/geometry/paraboloid.jpg}
		\caption{Parametric representation of a paraboloid with Maple 4.00b}
	\end{figure}
	\end{tcolorbox}
	
	\pagebreak
	\paragraph{Hyperboloid}\label{hyperboloid}\mbox{}\\\\
	Given the linear function:
	
	that we turn around the $z$-axis. As the line passes trough the origin, the revolution around the $z$-axis is similar to two cones facing each other with their apex at the origin O. Therefore:
	
	which give us obviously:
	
	\textbf{Definition (\#\mydef):} Any surface generated by a line is a "\NewTerm{ruled surface}\index{ruled surface}". More explicitly a surface $S$ is ruled if through every point of $S$ there is a straight line that lies on $S$. A ruled surface $S$ is a curved surface which can be generated by the continuous motion of a straight line in space along a space curve named the "\NewTerm{directrix}\index{directrix}".
	
	Late us take an important example (nuclear central chimney shape, gears, etc.) that is the one sheet hyperboloid of equation:
	
	f we put $a=b=c=1$ we have therefore (rotated line around the $z$ axis but not going through the origin O):
	
	which can also be written as the product of the equation of two lines such that:
	
	That we can also rewrite with $k \in \mathbb{R}$:
	
	by identification:
	
	It is not trivial but these to two lines belong to the same surface and any point belonging to one of these two lines is contained therein. The figures below show well that fact, every point belongs to these two lines:
	\begin{figure}[H]
		\centering
		\includegraphics{img/geometry/hyperbolic_ruled_surface.jpg}
		\caption{Hyperbolic Rules Surface}
	\end{figure}
	We could also described this by circles such that (this is the traditional way on computers):
	
	\begin{figure}[H]
		\centering
		\includegraphics{img/geometry/hyperbolic_circles.jpg}
		\caption{Hyperbolic Circles Surface}
	\end{figure}
	Here is also a very interesting figure that highlight the difference between a one sheet and two sheets hyperboloids\label{two sheets hyperboloid}.
	\begin{figure}[H]
		\centering
		\includegraphics[scale=0.7]{img/geometry/hyperboloid_one_two_sheets.jpg}
		\caption{Two hyperboloids (of one sheet and two sheets) which are asymptotic to an identical elliptic cone}
	\end{figure}
	
	We would like to share now a very nice and interesting (instructive) summary of most well known conical surfaces that we have study so far and provided by OpenStax:
	\begin{figure}[H]
		\centering
		\includegraphics[scale=1]{img/geometry/common_quadric_surfaces_01.jpg}
	\end{figure}
	\begin{figure}[H]
		\centering
		\includegraphics[scale=1]{img/geometry/common_quadric_surfaces_02.jpg}
		\caption[Common conical surfaces]{Common conical surfaces (source: OpenStax)}
	\end{figure}
	
	
	
	\pagebreak
	\paragraph{Torus}\mbox{}\\\\
	\textbf{Definition (\#\mydef):} In geometry, a "\NewTerm{torus}\index{torus}\label{torus}" is a surface of revolution generated by revolving a circle in three-dimensional space about an axis coplanar with the circle. If the axis of revolution does not touch the circle, the surface has a ring shape and is named a "\NewTerm{torus of revolution}\index{torus of revolution}".
	
	We know that to generate a circle in the plane $x\text{O}y$, a possible parametric equation is:
	
	To offset this circle to the right (in the direction of the positive $x$), we'll just add a strictly positive constant term $x$:
	
	In space to draw a shifted circle in the $x\text{O}z$-plane we have therefore:
	
	Which traditional notation is in the context of the study of torus:
	
	If we want to generate a torus, we will rotate the circle around the $u$-axis by making it following a circle in the plane $x\text{O}y$. We then have the  "parametric equation of the torus of revolution":
	
	\begin{tcolorbox}[colframe=black,colback=white,sharp corners]
	\textbf{{\Large \ding{45}}Example:}\\\\
	With Maple 4.00b a torus with $r=1,R=4$:\\

	\texttt{>r:=1:R:=4\\
	>plot3d([(R+r*cos(phi))*cos(theta),(R+r*cos(phi))*sin(theta), 1*sin(phi)],\\
	theta=-Pi..Pi,phi=-Pi..Pi,scaling=CONSTRAINED,orientation=[50,60],\\
	style=PATCH,axes=NORMAL); 
	}
	\begin{figure}[H]
		\centering
		\includegraphics{img/geometry/torus_maple.jpg}
		\caption{Parametric representation of a torus with Maple 4.00b}
	\end{figure}
	\end{tcolorbox}
	
	\begin{flushright}
	\begin{tabular}{l c}
	\circled{80} & \pbox{20cm}{\score{4}{5} \\ {\tiny 14 votes,  67.14\%}} 
	\end{tabular} 
	\end{flushright}
	
	%to force start on odd page
	\newpage
	\thispagestyle{empty}
	\mbox{}
	\section{Differential Geometry}\label{differential geometry}
	\lettrine[lines=4]{\color{BrickRed}A}s we have already mentioned it in the section of Non-Euclidean geometry, differential geometry is the branch of geometry that aims to study the local and intrinsic  properties (in the neighbourhood of a point) of curves and non-Euclidean surfaces (as a generalization of Euclidean surfaces!).
	
	Differential geometry takes its name from the fact that it was born from the possibility of a cinematic interpretation that infinitesimal calculus brings to the study of curves. The points that we will discuss here will also serve well in the study of classical mechanics as complex analysis applied to many areas of study physical fields.
	
	\begin{tcolorbox}[title=Remark,colframe=black,arc=10pt]
	Before we tackle the very formal and abstract manner to address differential geometry with the tools of Topology (usual method used by mathematicians) we have chose in a first time to present the essential elements in the most simple and enjoyable as this is done in some engineering schools. Purists will therefore perhaps excuse us... or  wait for something better...
	\end{tcolorbox}	
	
	\subsection{Parametric Curves}\label{parametric curves}
	\textbf{Definition (\#\mydef):} We will assimilate the "physical space" to $\mathbb{R}^3$ and will suppose that it includes a referential $R=(\text{O},\vec{i},\vec{j},\vec{k})$ and we will denote by $B$ the base $(\vec{i},\vec{j},\vec{k})$.
	
	Let us consider a set $I \subset \mathbb{R}$ and a function $\Gamma=f(I)$ such that:
	
	that is a parametric equations that defines a group of quantities as functions of one or more independent "parameters".
	
	Parametric curves are commonly used in kinematics, where the trajectory of an object is represented by equations depending on time as the parameter. Because of this application, a single parameter is often labelled $t$l However, parameters can represent other physical quantities (such as geometric variables) or can be selected arbitrarily for convenience.
	\begin{tcolorbox}[title=Remarks,colframe=black,arc=10pt]
	\textbf{R1.} If $f$ is continuous, then $\Gamma$ is a curve in space named "curve in one piece".\\
	
	\textbf{R2.} A parabola, a sinusoid are curves named  "\NewTerm{flat curves}\index{flat curves}". An ellipse, a circle are named "\NewTerm{closed planar curves}\index{closed planar curves}". For these examples, all points of the considered curves are located in a same plane. Conversely, a curve is named "\NewTerm{left curve}\index{left curve}" if it is not so.
	\end{tcolorbox}
	\begin{figure}[H]
		\centering
		\includegraphics[scale=0.7]{img/geometry/parametric_curve_parametric_surface.jpg}
		\caption{Parametric curve and parametric surface}
	\end{figure}
	Let us choose $t_0 \in I$ and put $M_0=f(t_0)$ which we will denote by $M(t_0)$ then we can state the following definition: the pair $(f, I)$ where $f$ is a continuous function is named "\NewTerm{parametric arc}\index{parametric arc}". $\Gamma$ is named the "\NewTerm{support}\index{support}" of $(f, I)$ and $t_0$ is an "\NewTerm{origin}\index{origin}" of $(f, I)$.
	
	Note that parametric representations are generally non-unique, so the same quantities may be expressed by a number of different parametrizations.
	\begin{tcolorbox}[title=Remarks,colframe=black,arc=10pt]
	\textbf{R1.} Abusively, and more frequently, we also say that $(f, I)$ is a "parametrization" of $\Gamma$.\\
	
	\textbf{R2.} It is easy to define other parametrized arcs admitting also $\Gamma$ as support. To do this, it is sufficient to give us a bijective function $\varphi$ of $I$ to equation $J\subset \mathbb{R}$ and such that $f(t)=f(\varphi^{-1}(x))$.
	\end{tcolorbox}
	Before continuing, remember that in differential geometry, the "\NewTerm{curvilinear abscissa}\index{curvilinear abscissa}" or "\NewTerm{line-element}\index{line-element}" is a kind of algebraic variation of the length of an arc (this is therefore the analogous, on a curve, of the abscissa on a straight oriented line).
	
	Let us consider now the following curvilinear abscissa (\SeeChapter{see section Special Relativity page \pageref{interval invariant}}):
	
	we have already seen that in a canonical Euclidean space (in $\mathbb{R}^n$) the curvilinear abscissa is then written:
	
	with $i,j=1,2,3$ and as we have $\delta_{ij}=0,i\neq j$, it remains in $\mathbb{R}^3$:
	
	In the Cartesian system:
	
	so it comes:
	
	which is therefore the linear differential element of an Euclidean space (the shortest path or the "\NewTerm{geodesic}\index{geodesic}" or "\NewTerm{differential curvilinear abscissa}\index{differential curvilinear abscissa}") that we have already met many times in various section of this book. So this is nothing new or surprising!
	
	If we restrict ourselves to the plane, the differential curvilinear abscissa of a plane curve is then obviously:
	
	We already know how to use this equation (we used it in the section of Analytical Mechanics). But as a reminder never hurts, let us do examples with a straight line, a parabola and a half-circle (the choice is not innocent!).
	\begin{tcolorbox}[colframe=black,colback=white,sharp corners]
	\textbf{{\Large \ding{45}}Examples:}\\\\
	E1. Consider the general equation of a line in the plane (it's not a plane curve for reminder but a straight flat line!):
	
	It then comes immediately:
	
	Therefore:
	
	E2. Consider now the following equation of a parabola in the plane:
	
	It then comes immediately:
	
	Therefore:
	
	E3. Consider the equation of an ellipse (\SeeChapter{see section Analytical Geometry page \pageref{analytical expression ellipse}}) written as:
	
	after rearrangement:
	
	It then comes immediately:
	
	Hence:
	
	\end{tcolorbox}
	
	\pagebreak
	\begin{tcolorbox}[colframe=black,colback=white,sharp corners]
	Notice that by making a Maclaurin approximation (when $x$ is therefore zero, which corresponds to the study of the pole of the ellipse) we get (\SeeChapter{see section Sequences and Series page \pageref{usual maclaurin developments}}):
	
	Following the request of a reader here are the details of the development of the previous result. First recall the Taylor series (\SeeChapter{see section Sequences and Series page \pageref{taylor series}}):
	
	If we put $x_0=0$, we get the Maclaurin series:
	
	Therefore we get proceeding in two stages:
	
	\end{tcolorbox}
	\begin{tcolorbox}[colframe=black,colback=white,sharp corners]
	
	by putting $x_0=0$ we get:
	
	It then comes immediately:
	
	Therefore:
	
	\end{tcolorbox}
	
	\pagebreak
	\begin{tcolorbox}[colframe=black,colback=white,sharp corners]
	We see then that the curvilinear abscissa of an ellipse in the plane becomes that of a parabola when we make a Maclaurin series expansion of the equation of the ellipse at the pole.\\
	
	We could do the same with an hyperbole and end up with the same form of differential curvilinear abscissa, usually denoted by tradition:
	
	where $k_x$ is named the "\NewTerm{osculator parameter of the parabola}\index{osculator parameter of the parabola}".
	\end{tcolorbox}
	\label{curvature parameter} What is veeery important to notice with the developments above (if we did it also identically of the hyperbola that is just the elliptic case with a different sign that we can detail on readers request) is that we finally have (remember that the ellipse is a general case of the sphere!):
	
	Where the last three cases are generally denoted:
	
	We the understand better why many books on General Relativity or Differential Geometry say that:
	\begin{itemize}
		\item When $K=0$ all case reduce to a flat space
		
		\item When $K=+1$ (or more generally just positive) we are in a spherical space (in fact it's ellipsoidal)
		
		\item When $K=-1$ (or more generally just negative) we are in a hyperbolic space
	\end{itemize}
	These examples being closed, let us continue with the theory. We can obviously rewrite our differential curvilinear abscissa by dividing both sides of equality by $\mathrm{d}t$ as:
	
	\begin{tcolorbox}[colframe=black,colback=white,sharp corners]
	\textbf{{\Large \ding{45}}Example:}\\\\
	\label{curvilinear abscissa helix}Let us see an application of the parametrized differential curvilinear abscissa with an helix (the examples are pretty in differential geometry and therefore nice to see...) which is a typical left curve:\\
	
	Let $t,r,h\in \mathbb{R}^3,t>0,r>,h>0$ and the function:
	
	with $M(x,y,z)$ and parametric coordinates:
	
	We then have with Maple 4.00b by taking $r$ and $h$ as equal to $1$:
	\begin{figure}[H]
		\centering
		\includegraphics[scale=0.6]{img/geometry/helix.jpg}
		\caption{Parametric representation of an helix with Maple 4.00b}
	\end{figure}
	The function $f$ is a parametric arc whose support is named an "\NewTerm{helix}\index{helix}", $r$ is the radius and $h$ is the step. Taking $t_0=0$ as the origin, the curvilinear abscissa of this helix (one piece) is given by:
	
	Therefore:
	
	and hence by integration:
	
	\end{tcolorbox}
	
	\subsection{Isolines}\label{isoline}
	Let us see now  a very important point in mathematics but also in medical engineering, astrophysics, meteorology (among many other areas) that are isolines.
	
	Before addressing the subject in a purely mathematical form, we suggest the reader to open MATLAB™ 5.0.0.473 (we've done also pretty much the same example with Maple 4.00b in the section of Functional Analysis) and write in it:
	
	\texttt{>>[xx,yy,z]=peaks;\\
	>>figure(1);mesh(xx,yy,z);title('peaks')}
	
	We get:
	\begin{figure}[H]
		\centering
		\includegraphics{img/geometry/isoclines_plot_01.jpg}
		\caption[]{Initial plot in MATLAB™ 5.0.0.473}
	\end{figure}
	then for aesthetic reasons, to write:
	
	\texttt{>>figure(2);surf(xx,yy,z);title('surf')}
	
	We get:
	\begin{figure}[H]
		\centering
		\includegraphics{img/geometry/isoclines_plot_02.jpg}
		\caption[]{Plot rendering modification in MATLAB™ 5.0.0.473}
	\end{figure}
	
	Then we would like MATLAB™  to plot for us some level curves (the points where the value of the function $f (x, y)$ is constant), named by the mathematicians "\NewTerm{isolines}\index{isoline}" or "\NewTerm{iso-level curves}\index{iso-level curves}". For this we must write:
	
	\texttt{>>figure(3);contour3(xx,yy,z);title('contour')}
	
	We get:
	\begin{figure}[H]
		\centering
		\includegraphics{img/geometry/isoclines_plot_03.jpg}
		\caption{Plot of the isoclines in MATLAB™ 5.0.0.473}
	\end{figure}
	We will then ask the software to project the isoclines on the $X, Y$ plane. This is done with:
	
	\texttt{>>figure(3);contour(xx,yy,z);title('contour')}
	
	We get:
	\begin{figure}[H]
		\centering
		\includegraphics{img/geometry/isoclines_plot_04.jpg}
		\caption{Projection of the isoclines in MATLAB™ 5.0.0.473}
	\end{figure}
	And it is these curves that will interest us. We would like to determine the algebraic expression in the plane of the latter. But first let's have fun with MATLAB™  by writing again:
	
	\texttt{>>figure(3);contour(xx,yy,z);title('contour')}
	
	We get:
	\begin{figure}[H]
		\centering
		\includegraphics{img/geometry/isoclines_plot_05.jpg}
		\caption{Plane representation of isoclines with coloured gradients \\in MATLAB™ 5.0.0.473}
	\end{figure}
	
	but we can do even better by removing the grid with the command:
	
	\texttt{>>shading interp}
	
	We get:
	\begin{figure}[H]
		\centering
		\includegraphics{img/geometry/isoclines_plot_06.jpg}
		\caption[]{Plane representation of isoclines with coloured gradients \\and without grid in MATLAB™ 5.0.0.473}
	\end{figure}
	Then, without closing the above now add the line:
	
	
	\texttt{>>hold on; contour(xx,yy,z,'k')}
	
	We get:
	\begin{figure}[H]
		\centering
		\includegraphics[]{img/geometry/isoclines_plot_07.jpg}
		\caption{Isoclines association projected with color gradients \\in MATLAB™ 5.0.0.473}
	\end{figure}
	\textbf{Definition (\#\mydef):} In mathematics, a level set of a real-valued function $f$ of $n$ real variables is a set of the form:
	
	that is, a set where the function takes on a given constant value $c^{te}$.
	
	When the number of variables is two, a level set is generically a curve, named a "\NewTerm{level curve}\index{level curve}", "\NewTerm{contour line}\index{contour line}", or "\NewTerm{isoline}\index{isoline}". So a level curve is the set of all real-valued solutions of an equation in two variables $x_1$ and $x_2$. When $n = 3$, a level set is named a "\NewTerm{level surface}\index{level surface}" or "\NewTerm{isosurface}\index{isosurface}", and for higher values of $n$ the level set is a named a "\NewTerm{hypersurface}\index{hypersurface}". So a "\NewTerm{level surface}" is the set of all real-valued roots of an equation in three variables $x_1, x_2$ and $x_3$, and a "\NewTerm{level hypersurface}\index{level hypersurface}" is the set of all real-valued roots of an equation in $n\; (n > 3)$ variables.
	
	Let us now consider to determine the equation of the isolines the bivariate function $f(x,y)$ function and that we will impose as being $\mathbb{R}^2$-differentiable.
	
	The relation:
	
	defines a plan curve $\Gamma$ therefore named "isocline". It is a curve such that when $x$ varies, then $y$ will thus not vary such that $f$ remains constant.
	
	We saw in the section of Differential and Integral Calculus that the differential of $f$, for any infinitesimal variations of $x$ and $y$ is:
	
	Now, if we want when $x$ varies of $\mathrm{d}x$, the value of the function $f$ does not change, $\mathrm{d}y$ must not be anything but such that the variation of $\mathrm{d}f$ is zero. In other words:
	
	along $\Gamma$. But this equation is useless as such, but it fixed us the ratio of the derivative of the isoline in the plane such as:
	
	which gives us the slope of the tangent to $\Gamma$ and thus after by integration, the desired function itself!!!
	
	It goes without saying that the tangent vector to the curve $\Gamma$ is a vector parallel to the one whose having for components (by correspondence with the above relation):
	
	that we will denote by:
	
	Also recall that the gradient is given by (\SeeChapter{see section Vector Calculus page \pageref{gradient scalar field}}):
	
	We note that the latter two vectors are perpendicular (result which will be very useful in the section of Complex Analysis). Indeed, calculating the dot product (\SeeChapter{see section Vector Calculus page \pageref{dot product}}):
	
	In other words, the vector $\vec{\nabla}(f)$ defines the orthogonal lines to the curve $\Gamma$.
	\begin{tcolorbox}[colframe=black,colback=white,sharp corners]
	\textbf{{\Large \ding{45}}Example:}\\\\
	Consider the equation of a special parabola in $\mathbb{R}^3$:
	
	So we have the isoclines which are given by:
	
	hence their equation in the plane:
	
	That is to say circles in the plane whose radius is equal to the square root of the corresponding selected constant at the height $z$ height of $f$!
	
	Let us now calculate the slope of the tangent $\Gamma$ to these circles:
	
	This is consistent with the simple derivative of:
	
	We also have:
	
	We see that on $x=0$ this vector is equal to:
	
	which is intuitively consistent with the vector tangent to the circle that we have at this point of the $x$-axis.
	\end{tcolorbox}
	
	\pagebreak
	\subsection{Frenet Frame}\label{frenet frame}
	One of the most important tools used to analyse a curve is the Frenet frame, a moving frame that provides a coordinate system at each point of the curve that is "best adapted" to the curve near that point.
	
	The Frenet frame is a tool for the study of the local behaviour of the curves. More precisely, it is a local coordinate system associated with a point describing a curve $\Gamma$. Its construction method is different depending on the ambient space is $2$-dimensional (plane curve) or $3$-dimensional (left curve).
	
	The Frenet, and the Frenet formulas (giving the derivatives vectors of this frame), gives the opportunity to conduct in a systematic way calculations of curvature and twisting of left curves and to introduce interesting geometric concepts associated with curves.
	
	Let us first consider a curve $\Gamma$ with its curvilinear abscissa $s(t)$ and $M_0=s(t_0)$ its origin. We denote by definition:
	
	the tangent to the parametrized curve $\Gamma$ of parameter $t$ near a point $M$ with respect to a frame put on O with $\mathrm{d}s$ that is calculated as we have shown previously.
	
	It is interesting to note that if $t$ is interpreted as begin the time (don't forget that this book is mainly target for engineering application), then we get a tangential speed:
	
	and thus the  $\vec{T}$ vector is obviously directed in the direction of movement.
	
	Moreover, by construction and definition of the curvilinear abscissa, we always have:
	
	and therefore the tangent vector $\vec{T}$ to the point $M$ is unitary (and not zero!).
	
	Now, without knowing exactly for what it will be useful for us now, let us look closer to the vector:
	
	Knowing trivially from the foregoing that:
	
	Then we have:
	
	therefore first $\mathrm{d}\vec{T}/\mathrm{d}s$ is a posteriori not unitary and $\vec{T}$ is perpendicular to it (results that will serve us several times afterwards so it must be remember!).
	
	Therefore:
	
	
	Let us put:
	
	Given the previous result, $\vec{N}$ is the perpendicular vector, named unitary "\NewTerm{curvature vector}\index{curvature vector}" at $\vec{T}$ on $M$. We say that this couple of vector $\left(\vec{T},\mathrm{d}\vec{T}/\mathrm{d}s\right)$ is "\NewTerm{direct orthonormal}\index{direct orthonormal vectors}" and $C$ is by definition the "curvature".
	
	We can also approach the curvature $C$ in a more geometrically way rather than formal and abstract previous definition. Let us see how:
	
	We know at this step of our study that on a point $M_0$ of a curve $\Gamma$ (differentiable at least once on every point - that is to say of class $\mathcal{C}^1$), there is a non-zero vector $\vec{T}$ which is tangent.
	
	In any neighbouring point $M$ (of curvilinear abscissa $s$), the tangent vector $\vec{T}$ can be written in approximation:
	
	if the curve is locally in the same plane (for now we are here studying the curvature and not the twisting of a curve)!
	
	Two normal at $M$ and $M_0$ thus intersecting in a point $\Omega$, the following figure:
	\begin{figure}[H]
		\centering
		\includegraphics{img/geometry/curvature_tangent_vector_decomposition.jpg}
		\caption{Decomposition of the tangent vector of the curve path}
	\end{figure}
	shows that at the first order in $\mathrm{d}s$, the point $M$ can be considered locally as deduced from the point $M_0$ by a rotation of center $\Omega$.
	
	The circle thus defined of radius:
	
	is the one that best tangent the curve locally at the point $M_0$. Its radius is derived from the figure (two similar triangles at the limit):
	
	hence, since $\vec{T}$ is unitary, the definition of the value of the curvature for a plane curve:
	
	and that's it!
	
	It is possible to interpret the concept of curvature as the speed of rotation of the Frenet frame relatively to a fixed direction.
	
	The pair of vectors $(\vec{T}, \vec{N})$ is named a "\NewTerm{Frenet frame}\index{Frenet Frame}" and the basis vectors the "\NewTerm{Frenet vectors}\index{Frenet vectors}".
	
	The Frenet frame is a moving frame and the elements of this frame change depending on the point in question. In physics, we must not confuse this with the repository: as Frenet vectors move with the point!
	
	\begin{tcolorbox}[title=Remark,colframe=black,arc=10pt]
	The definition of $C$ as above is true in the context of an choice of positive curvature. It's taken in a mechanical point of view but not necessary in pure mathematics.
	\end{tcolorbox}	
	If $C\neq 0$, then as seen above we can now write:
	
	where $R$ is named the "\NewTerm{curvature radius}\index{curvature radius}\label{curvature radius}".
	
	About the relation:
	
	is named the "\NewTerm{first Frenet formula}\index{first Frenet formula}\label{first Frenet formula}" and shows obviously that $\vec{N}$ and $\mathrm{d}\vec{T}/\mathrm{d}s$ are collinear and hence their cross product is zero (result that will be used later).
	
	These relations are justified by the analogy with mechanics. Indeed, remember that we have shown above that:
	
	
	Let us now calculate the acceleration:
	
	we fall back on the result obtained in the section of Classical Mechanics in our study of the osculating plane.
	
	To give a more accurate geometric interpretation of the curvature, we define first through $\Omega$ the " \NewTerm{center of curvature}\index{center of curvature}\label{center of curvature}" of the "\NewTerm{osculating circle}\index{osculating circle}" (placed in the osculating plane) or "\NewTerm{curvature circle}\index{curvature circle}" of radius $R$ that tangent locally the best the curve $\Gamma$ in the Frenet frame:
	
	To clarify geometrically what is the osculating circle, take a curve and a point $M$ on that curve. Then draw the normal at this point of the locally plane curve equation and take a point $\Omega$ on the normal. Then, the center $\Omega$ of the circle passing through the point $M$ is tangent to the curve. But all the circles tangent to the curve are not tangent in the same way! Indeed, if $\Omega $is far from $M$, the circle will be located rather outside of the curve (blue circle in the figure below). If $\Omega$ is close to $M$, the circle will be located rather inside of the curve (pink circle in the figure below). The limit radius between being "inside the curve" and being "outside the curve" is conventionally the "radius of curvature" we defined above. The circle of that radius is then the famous "osculating circle"!
	\begin{figure}[H]
		\centering
		\includegraphics{img/geometry/osculating_circle.jpg}
		\caption{Representation of the osculating circle}
	\end{figure}
	In the particular case where $\vec{T}$ is a constant vector we have obviously:
	
	and therefore $C=0$ implying that $R$ is no longer define. We sometimes say that in this case that the radius of curvature  of $\Gamma$ is infinite (a straight line then has a zero curvature at any point).
	
	Let us now consider the vector perpendicular to the osculating plane defined and denoted by:
	
	We can already say, since $\vec{T}$ and $\vec{N}$ are of unit norm that $\vec{B}$ is also of unit norm (which will serve us later)!
	
	We can see that $\mathrm{d}\vec{B}/\mathrm{d}s$ is obviously orthogonal to $\vec{T}$. Indeed:
	
	where we took the particularly case $\mathrm{d}\vec{T}/\mathrm{d}s=\vec{0}$  (but in anyway in general $\mathrm{d}\vec{T}/\mathrm{d}s$ and $\vec{N}$ are collinear as we have prove it earlier and therefore the cross product of these two vectors is always zero).
	
	As $\mathrm{d}\vec{B}/\mathrm{d}s$ is perpendicular to $\vec{T}$ it is therefore collinear to $\vec{N}$.
	
	Therefore, as $\mathrm{d}\vec{T}/\mathrm{d}s$ is perpendicular to $\vec{T}$, $\mathrm{d}\vec{B}/\mathrm{d}s$ is perpendicular $\vec{B}$. This can be proven using exactly the same trick as when we have proved that $\mathrm{d}\vec{T}/\mathrm{d}s$ is perpendicular to $\vec{T}$.
	
	Therefore:
	
	
	Let us put:
	
	This relation is the "\NewTerm{second Frenet formula}\index{second Frenet formula}" where by definition, $\vec{B}$ is the "\NewTerm{bi-normal vector}\index{bi-normal vector}" of $\Gamma$ at the point $M$ and $\tau$ the "\NewTerm{twist}\index{twist}" and $R'$ the "\NewTerm{torsion radius}\index{torsion radius}".
	
	We can now establish the third Frenet formula. For this we start from:
	
	from which we get:
	
	But, by the properties of the cross product (\SeeChapter{see section Vector Calculus page \pageref{cross product}}):
	
	hence the "\NewTerm{third Frenet formula}\index{third Frenet formula}":
	
	We name "\NewTerm{Frenet trihedron}\index{Frenet trihedron}" or "\NewTerm{Frenet-Serret frame}\index{Frenet-Serret frame}" associated to the curve $\Gamma$ at point $M$, the natural orthonormal space frame $(M,\vec{T},\vec{N},\vec{B})$:
	\begin{figure}[H]
		\centering
		\includegraphics[scale=0.75]{img/geometry/frenet_frame.jpg}
		\caption{Representation of the trihedron}
	\end{figure}
	where, in mechanics, the vector $\vec{T}$ is collinear to the velocity and to tangential acceleration and $\vec{N}$ is collinear with the normal acceleration.
	\begin{tcolorbox}[title=Remark,colframe=black,arc=10pt]
	The radius of curvature $R$ is in the osculating plane (the plane formed by the tangent and normal vector  to the curve) which is the best plan in which is contained the curve. So, the radius of curvature gives in a point (locally) the best ("truest") radius of the curve. The twist gives us against the tendency of the curve of going out of the osculating plane (verbatim if the curve is contained in a plane, torsion is zero).
	\end{tcolorbox}
	\begin{tcolorbox}[colframe=black,colback=white,sharp corners]
	\textbf{{\Large \ding{45}}Examples:}\\\\
	E1. We consider the plane parametric equation of a parabola:
	
	Therefore we have:
	
	Hence:
	
	\end{tcolorbox}
	
	\begin{tcolorbox}[colframe=black,colback=white,sharp corners]
	Therefore we have:
	
	By the way you will notice that we have well:
	
	Thus, the curvature (inverse of the radius of curvature) is given by:
	
	And therefore:
	
	At $t=0$ the parabola has therefore a curvature $R(0)=0.5$.
	\begin{figure}[H]
		\centering
		\includegraphics{img/geometry/osculating_circle_parabola.jpg}
		\caption[]{Osculating circle to parabola at point $t=0$}
	\end{figure}
	
	E2. Let us seek the radius and the center of curvature for any $M$ of our helix defined above as practical example. Recall that its parametric function is given by:
	
	and that:
	
	\end{tcolorbox}
	
	\pagebreak
	\begin{tcolorbox}[colframe=black,colback=white,sharp corners]
	
	Therefore we have:
	
	On the way, the reader will have probably notices that we have:
	
	Thus, the curvature (the inverse of the radius of curvature) is given by:
	
	Therefore the radius of curvature is equal to:
	
	This is consistent with intuition because when the step $h$ of the helix is equal to zero, the radius or curvature is equal to $r$ (the radius of the corresponding circle) and when $h$ tends to infinity the radius of curvature tends to infinity and the curvature tends to zero. This example is a famous case in engineering applied to the exhaust chimney of smoke evacuation which are surrounded by a spiral of Archimedes and whose objective is to raise the air flows upward (the difficulty being to identify the radius $R$ of the metal plate to cut that will follow the desired curvature at best... at least locally by knowing the radius of the chimney and the height $h$ of the step of the helix):
	\begin{figure}[H]
		\centering
		\includegraphics[scale=0.7]{img/geometry/helix_curvature_cheminee_figure.jpg}
		\caption[]{Basic principle of an industrial chimney with a spiral (source: Frank Morgan, Riemmanian Geometry)}
	\end{figure}
	\end{tcolorbox}
	
	\pagebreak
	\begin{tcolorbox}[colframe=black,colback=white,sharp corners]
	Or the real corresponding version:
	\begin{figure}[H]
		\centering
		\includegraphics[scale=0.8]{img/geometry/helix_curvature_cheminee_photo.jpg}
		\caption[]{Industrial evacuation chimney with spiral}
	\end{figure}
	But in this engineering case, the height $h$ should be obtained by a complete rotation. Therefore the radius of curvature will be written:
	
	Anyway, to get back to our example and finish it, it comes with the first Frenet formula the following normal vector:
	
	and for which the point (extremity of the vector) are indistinctness of the $z$-axis (merged with it) of our helix regardless the height $h$! The coordinate of the component $z$ of this vector is zero since the normal is taken with respect to a point $M$ of the curve already at a given implicit height $h$.
	
	By the third Frenet formula with the binormal vector:
	
	and the torsion radius is given by the relation:
	
	we therefore have:
	\end{tcolorbox}
	
	\pagebreak
	\begin{tcolorbox}[colframe=black,colback=white,sharp corners]
	
	And as we got the three following relations:
	
	We deduce the torsion radius:
	
	
	E2. Let us now determine the important case of the explicit expression of the radius of curvature in Cartesian coordinates (result used in the section of Civil Engineering and useful in many other areas of physics!). Consider for this purpose the following figure:
	\begin{figure}[H]
		\centering
		\includegraphics{img/geometry/cartesian_curvature.jpg}
		\caption[]{Illustrated approach of the "1D" curvature expression}
	\end{figure}
	So we have the radius of curvature which is intuitively given by the following relation if we do not make use of vector analysis:
	
	We also have:
	
	and as:
	
	\end{tcolorbox}
	
	\pagebreak
	\begin{tcolorbox}[colframe=black,colback=white,sharp corners]
	therefore we have:
	
	and therefore:
	
	We have proved in the section of Differential and Integral Calculus the following derivative:	
	
	We then have immediately by the composed derivatives:
	
	Having done this we also need $\mathrm{d}x/\mathrm{d}s$. But, we have already proved with different approaches in the section of Analytical Mechanics and Geometric Shapes (among others) by simply using Pythagoras theorem that:
	
	By grouping all we finally have:
	
	So it comes that the radius of the local osculating circle of a Cartesian function in the plane (by taking the absolute value of the second derivative to avoid having a negative radius...) is given by\label{radius of curvature}:
	
	\end{tcolorbox}
	
	\pagebreak
	\subsection{Surface Patches}
	The study of surface patches is strongly related to the properties of parametric surfaces expression as seen at the end of the section of Analytical Geometry. We are therefore interested to calculate the tangent plane and the curvature of a surface on a given point. It is therefore an extension of the previous subjects. For example, the study of isolines of a surface that we have seen earlier can also be considered as belonging the field of Surface Patches study.
	 
	Now let us consider to begin $D \subset \mathbb{R}^2$:
	
	with:
	
	such that $h(I)\subset D$.

	We can define $g\circ h$:
	
	If we assume $h$ continuous, it is clear that $g\circ h$ is a parametric arc. Let us denote by $\Gamma$ its support, we have $\Sigma \subset \Sigma$ and we say that $\Sigma$ a "\NewTerm{plotted curve}\index{plotted curve}" or "\NewTerm{inscribed curve}\index{inscribed curve}" on $\Sigma$ defined by the "\NewTerm{Gauss Coordinates}\index{Gauss coordinates}" $u$ and $v$ (already seen in the section of Non-Euclidean Geometry).
	
	\begin{tcolorbox}[title=Remark,colframe=black,arc=10pt]
	We always assume now that $D=I\times J$.
	\end{tcolorbox}
	Given $M_0 \in \Sigma, M_0=g(u_0,v_0)$. Let us look at the two curves drawn on $\Sigma$ defined by the following parametrized arcs:
	
	$g_{u_0}$ and $g_{v_0}$ are the two functions named "\NewTerm{partial functions}\index{partial functions}" of $g$ on $(u_0,v_0)$.
	
	The support of $(g_{u_0},J)$ and $(g_{v_0},I)$ are named "\NewTerm{coordinate-curves}\index{coordinate-curve}" of $\Sigma$ on $M_0$ relatively to the parametrization $(g,D)$. We denote them  respectively $\Gamma_{u_0}$ and $\Gamma_{v_0}$ (see figure below). We also name $\Gamma_{u_0}$ "\NewTerm{first coordinate-curve}\index{first coordinate-curve}" and $\Gamma_{v_0}$ "\NewTerm{second coordinate-curve}\index{second coordinate-curve}".
	
	It is of course quite obvious that (\SeeChapter{see section Differential and Integral Calculus page \pageref{partial derivative}}) that:
	
	is tangent to $\Gamma_{u_0}$ on $M_0$ and that $\partial \overrightarrow{\text{O}M}/\partial u$ is tangent to $\Gamma_{v_0}$ on $M_0$.
	
	As we have already prove it in the section of Tensor Calculus, this expression is independent of the surface patch as the infinitesimal  element length $\mathrm{d}s$ is independent of the parametrization of $\Sigma$. This quadratic form is therefore an invariant that represents the metric of $\Sigma$. It is also denoted as follows by tradition:
	
	or even in more condensed form using tensor notation:
	
	
	\begin{figure}[H]
		\centering
		\includegraphics{img/geometry/surface_patch.jpg}
		\caption{Representation of a surface patch}
	\end{figure}
	
	\subsubsection{Metric of a Surface Patch}
	Given again:
	
	with:
	
	Let us write $\mathrm{d}g=(\mathrm{d}x,\mathrm{d}y,\mathrm{d}z)$ or in other words:
	
	We also have (\SeeChapter{see section Differential and Integral Calculus page \pageref{total exact differential}}):
	
	and we have proved at the beginning of this section that the curvilinear abscissa in a Cartesian space (in Riemann coordinates Riemann) was given by:
	
	We therefore have after substitution in Gaussian coordinates:
	
	Which is equivalent to write (take caution not to read there the square of a vector but it is the dot product of the vector with itself!):
	
	In a more traditional way with the notation:
	
	we get a relation named the "\NewTerm{first fundamental quadratic form}\index{first fundamental quadric form}" (we will note prove in this book the second one):
	
	also named "\NewTerm{first Gauss differential form}\index{first Gauss differential form}". It is interesting to write this last relation in the form:
	
	and we see for $\mathrm{d}s^2$ to be positive, $E$ and also $EG-F^2$ must be positive!
	
	The first fundamental form may be also represented as a symmetric matrix:
	
	The first fundamental form is often written in the modern notation of the metric tensor. The coefficients may then be written as $g_{ij}$:
	
	and we will prove in the section of Tensor Calculus during our study of Gram's determinant that this relation can be used to calculate the surface of any (regular) surface patch!!
	
	\pagebreak
	\paragraph{Regularity of a Surface}\mbox{}\\\\
	\textbf{Definition (\#\mydef):} A point $M$ belonging to the surface $\Sigma$ is said (this is quite relatively logical...) a "\NewTerm{regular point}\index{regular point}" if and only if:
	
	A surface patch is logically named "\NewTerm{smooth surface}\index{smooth surface}" if and only if all its points are regular (if the cross product is zero then there is somewhere a "fold" at $\pi/2$... and this is quite annoying for continuity purposes).

	Let us notice that:
	
	The angle $\alpha$ between the two coordinate-curves $\Gamma_{u_0}$ and $\Gamma_{v_0}$ to $\Sigma$ on $M_0$ is given by the product definition of the dot product:
	
		Hence the expression:
	
	Therefore a necessary and sufficient condition for the coordinates-curves $\Gamma_{u_0}$ and $\Gamma_{v_0}$ to be perpendicular to $\Sigma$ on $M_0$ is that $F$ is equal to zero. In this particular case, we say that the curvilinear coordinates $u$, $v$ on the surface are orthogonal coordinates!!!
	\begin{tcolorbox}[colframe=black,colback=white,sharp corners]
	\textbf{{\Large \ding{45}}Examples:}\\\\
	E1. Let us consider the parametrization of the cartesian plane. Then we have:
	
	hence:
	
	Therefore:
	
	We fall back on the same differential curvilinear abscissa than that seen in the section of Tensor Calculus and General Relativity with the diagonal metric of the flat space.
	
	We also have:
	
	Then the surface is indeed regular. \\

	We also have:
	
	So the both coordinate-curves are perpendicular!\\
	
	E2. Let us consider the parametrization of the cylinder surface. We then have (\SeeChapter{see section Vector Calculus page \pageref{cylindrical coordinates}}):
	
	hence:
	
	\end{tcolorbox}
	
	\begin{tcolorbox}[colframe=black,colback=white,sharp corners]
	Therefore:
	
	We fall back on the same differential curvilinear abscissa than that seen in the section of Tensor Calculus and General Relativity with the diagonal metric in polar coordinates.\\
	
	We also have:
	
	Therefore the surface is regular as long as $r$ is not zero. We also have:
	
	So the both coordinates-curves are perpendicular on the cylinder.\\
	
	E3. Let us consider the parametrization of the sphere of origin O and radius $r$ \label{metric two sphere}. Then we have (\SeeChapter{see section Vector Calculus page \pageref{spherical coordinates}}):
	
	hence:
	
	Therefore:
	
	We fall back on the same differential curvilinear abscissa than that seen in the section of Tensor Calculus and General Relativity with the diagonal metric in spherical coordinates.\\
	
	We also have:
	
	Therefore the surface is regular as long as $r$ is not zero. We also have:
	
	So the both coordinates-curves are perpendicular on the sphere.\\
	\end{tcolorbox}
	
	\begin{tcolorbox}[colframe=black,colback=white,sharp corners]
	E4. Let us consider the parametrization of the hyperboloid. Then we have (\SeeChapter{see section Analytical Geometry page \pageref{hyperboloid}}):
	
	hence:
	
	Therefore:
	
	Let us take $b$ as being equal to zero. Then we have:
	
	Therefore the surface is regular as long as $a$ not equal to zero. We also have:
	
	So the both coordinates-curves are perpendicular on the hyperboloid.\\
	\end{tcolorbox}
	
	\begin{flushright}
	\begin{tabular}{l c}
	\circled{80} & \pbox{20cm}{\score{4}{5} \\ {\tiny 35 votes,  74.86\%}} 
	\end{tabular} 
	\end{flushright}
	
	%to make section start on odd page
	\newpage
	\thispagestyle{empty}
	\mbox{}
	\section{Geometric Shapes}

\lettrine[lines=4]{\color{BrickRed}W}e have already defined at the beginning of the section on Euclidean Geometry the concepts of topological dimensions, what were a zero-dimensional point and a unit dimensional curve. We will not come back on these concepts we will focus here mostly on shapes of higher dimensions that we will use a lot in the chapters related to theoretical Physics.

The purpose of section is to categorize with proofs some remarkable mathematical properties of well-known geometric shapes and bodies (area, volume, center of mass, moment of inertia). Indeed, there are many mathematical books listing those properties without proofs but few or simply no books at all prove them all (at least we have never seen such a book at this day...). The list below is at this date far away to exhaustive (since there are an infinite number of geometric shapes), but it will be completed in time.

\begin{figure}[H]
\centering
\includegraphics[scale=0.75]{img/geometry/delucq_shoot_or_point.eps}
\end{figure}

The few shapes that we wanted to present in this section allow easily to find the remarkable properties of a large number of shapes not listed in this section by assembly or decomposition.

	\begin{tcolorbox}[title=Remarks,colframe=black,arc=10pt]
	\textbf{R1.} The trigonometric relations and remarkable integrals that will be used for the study of the geometric shapes below are not proven in this section. They all have already been proved in the sections dealing specifically with Trigonometry and Differential Calculus and Integral.\\
	
	\textbf{R2.} We understand by "center of gravity", the "barycentre" as discussed in the chapter of Euclidean Geometry.
	\end{tcolorbox}	
	
	\subsection{Usual Surfaces (Areas)}\label{known surfaces}
	There are several definitions of the concept of surface (area) due to Euclid and another modern due to the topology (see sections of the corresponding names).

\textbf{Definitions (\#\mydef):}
	\begin{enumerate}
		\item[D1.] A "\NewTerm{plane surface}\index{plane surface}" is a closed area that can be identify almost approximatively by a height and a width.
		\item[D2.] A "\NewTerm{surface}\index{surface}" is a topological variety of dimension 2.
	\end{enumerate}
	Depending on the authors a surface or area is denoted by the symbols: $A, S, \mathcal{A}, \mathcal{S}$.

	\begin{tcolorbox}[title=Remark,colframe=black,arc=10pt]
we will focus initially only on the properties (perimeter, area, center of gravity, etc.) of surfaces dived in Euclidean plane and space.
	\end{tcolorbox}	

	\subsubsection{Polygons}\label{polygon}
	\textbf{Definition (\#\mydef):} A "\NewTerm{polygon}\index{polygon}" is a plane figure bounded by straight line segments (i.e.: with a closed polyline).

\begin{figure}[H]
\centering
\includegraphics[scale=0.75]{img/geometry/polygone.eps}
\caption{Example of arbitrary plane polygon}
\end{figure}

\textbf{Definition (\#\mydef):} A "\NewTerm{quadrilateral}\index{quadrilateral}", "\NewTerm{pentagon}\index{pentagon}", "\NewTerm{hexagon}\index{hexagon}", "\NewTerm{heptagon}\index{heptagon}" are polygons with respectively four, five, six, seven... sides.

We distinguish three main families (but they are not the only ones!) of polygons: 
	\begin{enumerate}
		\item Crossed polygons
		\item Concave polygons
		\item Convex polygons
	\end{enumerate}
We will find these two last families in different sections of this book (but for details see below).

Before going any further we would like to clarify to the reader that there is, at least as far as we know, only mathematical relations to calculate the surface for simple polygons. Although in practice we almost always meet non simple polygons and we considered as useless to dwell on the determination of a relation that would reduce the calculation of the surface of any polygon to that of simple polygons or to use Monte Carlo methods (\SeeChapter{see section of Numerical Methods page \pageref{monte carlo simulations}}).

	\textbf{Definition (\#\mydef):} A polygon is named "\NewTerm{crossed polygon}\index{crossed polygon}" if at least two of its sides intersect, that is to say, if at least two of its sides cross each other. 

This is the case of the pentagon $ABCDE$ below:

\begin{figure}[H]
\centering
\includegraphics[scale=0.75]{img/geometry/crossed_polygon.eps}
\caption{Example of a crossed polygon}
\end{figure}

	\begin{tcolorbox}[title=Remark,colframe=black,arc=10pt]
The "\NewTerm{envelope}\index{envelope of a polygon}" of a polygon is the polygon obtained by following its outer contour. For example, the envelope of  the previous polygon is a decagon whose vertices are the five summits of the pentagon and the five intersections of its sides.
	\end{tcolorbox}	

	\textbf{Definition (\#\mydef):} A polygon is named "\NewTerm{concave polygon}\index{concave polygon}" if it is not crossed and if one or more of its diagonals are not entirely within the area bounded by the polygon.

This is the case of the pentagon $ABCDE$ below:

\begin{figure}[H]
\centering
\includegraphics[scale=0.75]{img/geometry/concave_polygon.eps}
\caption{Example of a concave polygon}
\end{figure}

	\textbf{Definition (\#\mydef):} A polygon is named "\NewTerm{convex polygon}\index{convex polygon}" if it is not crossed and if all its diagonals are entirely within the area bounded by the polygon. 

The hexagon $MNOPQR$ below is an example of convex polygon:
\begin{figure}[H]
\centering
\includegraphics[scale=0.75]{img/geometry/convex_polygon.eps}
\caption{Example of a convex polygon}
\end{figure}

With respect to the definitions given above where the diagonals were identified, let us see if there is a relation permitting to know the number of diagonals relatively to the number of edges of a polygon.

For this let us start from a  polygon of $n$ sides (also note that it has $n$ vertices):

\begin{figure}[H]
\centering
\includegraphics[scale=0.3]{img/geometry/pentagon.eps}
\caption{Example of a polygon (pentagon with $5$ edges and $5$ vertices)}
\end{figure}

We define now the total number segments $S$ as equal to the quantity of the sides (edges) $E$ more the quantity of diagonals $D$ such that:
	
With our example we can see that $S=10$:

\begin{figure}[H]
\centering
\includegraphics[scale=0.7]{img/geometry/pentagon_segments.eps}
\caption{Pentagon with all its $10$ segments}
\end{figure}

	Now take one first point of our pentagon. We see that we can reach all $n$ points (vertices), excepted the chosen one $(-1)$ therefore we can create $n-1$ segments as shown in the figure below:

	\begin{figure}[H]
	\centering
		\includegraphics[scale=0.6]{img/geometry/pentagon_first_step_segments.eps}
	\caption{Example of segments starting from a chosen point}
	\end{figure}

	With the second point, we can also reach all the points $n$, except the chosen point itself $(-1)$ and we must remove the first point already taken in the previous step $(-1)$ . We therefore created $n-2$ segments:

	\begin{figure}[H]
	\centering
		\includegraphics{img/geometry/pentagon_second_step_segments.jpg}
	\caption{Approach with the 2nd point}
	\end{figure}
	With the third one, we can also reach all the $n$ points, excepted the considerated point $(-1)$ and less the two points already seen $(-2)$ that is to say the formation of $n-3$ segments:
	\begin{figure}[H]
	\centering
		\includegraphics{img/geometry/pentagon_third_step_segments.jpg}
	\caption{Approach with the 3rd point}
	\end{figure}
	We continue with the other points: the 4th that gives $n - 4$ segments, the fifth that gives $n - 5$ segments ... Therefore we see that the $(n - 2)\text{th}$ item thus gives $n - (n - 2 )$ segments, etc.
	
	So we finally have for:
	
	By simplifying we find therefore:
	
	We find ourselves with two relations:
	
	Therefore it follows that:
	
	
	\textbf{Definition (\#\mydef):} A "\NewTerm{regular polygon}\index{regular polygon}\label{regular polygon}" is a convex-polygon in which every side has the same length and every angle is the same. For any natural number $n\geq 3$, we can draw a regular polygon with $n$ sides; sometimes we'll call this polygon a "\NewTerm{regular $n$-gon}\index{regular $n$-gon}" to emphasize that it has $n$ sides. Below are the regular $n$-gons for $n = 3, 4, ..., 12$. As $n$ gets larger, the regular $n$-gons look more like a perfect circle:
	\begin{figure}[H]
	\centering
		\includegraphics{img/geometry/regular_polygons.jpg}
	\caption{Regular polygons (subfamily of convex polygons)}
	\end{figure}
	
	\subsubsection{Rectangle}
	\textbf{Definition (\#\mydef):} The "\NewTerm{rectangle}\index{rectangle}" is a special case of the quadrangle in the sense that his sides $L$ and $H$ (notation according to length and height as we can see in figure below) are equal in pairs and with right angle (in other words, $L$ is not necessarily equal to $H$).
	
	Other possible definitions include that a rectangle is a parallelogram with a right angle or a quadrilateral with four right angles...
	
	\begin{tcolorbox}[title=Remark,colframe=black,arc=10pt]
	The rectangle can be seen as the composition of two (or more) triangles (see below for definition). To build a rectangle, it would be sufficient to have a single rectangle triangle and make him undergo a double reflection and rotation relative to a well-chosen axis (\SeeChapter{see section Euclidean Geometry page \pageref{geometric transformations}}).
	\end{tcolorbox}
	
	\begin{figure}[H]
		\centering
		\includegraphics{img/geometry/rectangle_example.jpg}
		\caption{Example of rectangle}
	\end{figure}
	By the Euclid's axioms (\SeeChapter{see section Euclidean Geometry page \pageref{euclid's postulates}}), the perimeter of a rectangle is given by:
	
	And by definition its surface by:
	
	and the length of its  diagonal (application of the Pythagorean theorem):
	
	The position of the rectangle's center of gravity, if we put the Cartesian coordinate system in the lower left corner of this geometry, is trivially given by:
	
	Finally, indicate that if we were living beings in a two-dimensional space, the rectangle is what we should perceive if a cuboid crossed our universe parallel to its faces.
	
	Finally, let us notice, and without proof, that the sum of the angles or a plane rectangle is equal to $4\cdot \pi/2=2\pi$.
	
	\subsubsection{Square}
	\textbf{Definition (\#\mydef):} The "\NewTerm{square}\index{square}" is a special case of the rectangle in the sense that its four sides are equal such that $L=H$.
	\begin{figure}[H]
		\centering
		\includegraphics{img/geometry/square_example.jpg}
		\caption{Example of square}
	\end{figure}
	By the Euclid's axioms (\SeeChapter{see section Euclidean Geometry  page \pageref{euclid's postulates}}), the perimeter of the square is given by:
	
	So it comes for the surface:
	
	and for its diagonal:
	
	The position of the square's center of gravity, if put the Cartesian coordinate system in the lower left corner of the form, is trivially given by:
	
	Finally, let us indicate that if we were living beings in a two-dimensional space, the square is what we should perceive if a cube crossed our universe parallel to its faces.
	
	Also notice, and without proof, that the sum of the angles or a plane square is equal to $4\cdot \pi/2=2\pi$.
	
	\pagebreak
	\subsubsection{Unspecified Triangle}
	\textbf{Definition (\#\mydef):} An "\NewTerm{unspecified triangle}\index{unspecified triangle}\label{unspecified triangle}" is a polygon with three sides and includes in particular cases the: isosceles, equilateral and rectangles triangles.  A triangle with vertices $A, B$, and $C$ is denoted $\triangle ABC$.
	\begin{figure}[H]
		\centering
		\includegraphics{img/geometry/unspecified_triangle_example.jpg}
		\caption{Example of unspecified triangle with inscribed circle and median}
	\end{figure}
	\begin{tcolorbox}[title=Remark,colframe=black,arc=10pt]
	For the details about the Circumcircles and Incircles of triangles we send the reader to the section of Euclidean Geometry.
	\end{tcolorbox}	
	By the Euclid's axioms (\SeeChapter{Euclidean Geometry}), the perimeter of an unspecified triangle is given by:
	
	An unspecified plane triangle can always be decomposed into two right angle triangles. Thus, the one of the above figure can be divided into two right angle triangles:
	\begin{figure}[H]
		\centering
		\includegraphics{img/geometry/unspecified_triangle_decomposition.jpg}
		\caption{Unspecified triangle divided into two right triangles}
	\end{figure}
	of respective basis $a_1$ and $a_2$ (defined by the orthogonal projection of the top on opposite to the segment $a$) such as:
	
	The surface of each of these two right angle triangles is as we have said already implicitly in our study of the rectangle, half the area of a rectangle of the same length and same height. Therefore:
	
	Therefore, the sum of these surfaces, gives us the surface of unspecified triangle:
	
	We can say from this latter equality that the surface of any one triangle is similar to the half of the area of a rectangle of length $L=a$ and height $H=h$.
	\begin{tcolorbox}[title=Remark,colframe=black,arc=10pt]
	Whatever the basis $a, b, c$ and the respective height $h_a,h_b,h_c$, the previous reasoning remains of course completely correct.
	\end{tcolorbox}
	That said ... it is nice and pretty but there are very many practical cases where we do not know the height but only the length of the three sides. So, what do we do? Well, we will use the cosine theorem (Al-Kashi's Theorem) proved in the section Trigonometry which gives us a reminder for any unspecified triangle of the type:
	\begin{figure}[H]
		\centering
		\includegraphics{img/geometry/cosine_theorem_triangle.jpg}
		\caption{Reminder of the construction for the cosine's theorem proof}
	\end{figure}
	
	and also the relation resulting in the triangle relative to the surface:
	
	Therefore we have:
	
	Therefore after rearrangement we get finally:
	
	Relation that is commonly named "\NewTerm{Heron's formula}\index{Heron's formula}\label{heron formula}" also sometimes named "\NewTerm{Hero's formula}\index{Hero's formula}" most know in the following form when we put $\Sigma:=\dfrac{1}{2}\left(a+b+c\right)$:
	
	
	The determination of the center of gravity (or centroid) $G$ (\SeeChapter{see section Euclidean Geometry page \pageref{barycenter}}) is a little less intuitive than in the case of the rectangle...
	
	\begin{theorem}
	Given a triangle $\triangle ABC$. We name $A'$ the "middle of the segment" $\overline{BC}$, $B'$ that of $\overline{AC}$ and $C'$ that of $\overline{AB}$:
	\begin{figure}[H]
		\centering
		\includegraphics{img/geometry/triangle_center_of_gravity.jpg}
		\caption{Starting point for determining the center of gravity}
	\end{figure}
	We will prove that only the point $G$ satisfying (\SeeChapter{section Euclidean Geometry}):
	
	is the point of intersection of the three medians of triangle $\triangle ABC$. This proof will be made two stages: two proposals (at the end, we can conclude by the theorem):
	\begin{enumerate}
		\item[P1.] If $\triangle ABC$ is a triangle then there exists one and only one center of gravity $G$ such that $\overrightarrow{GA}+\overrightarrow{GB}+\overrightarrow{GC}=\vec{0}$.
		\begin{dem}
		Let $G$ be a point in the plane such as $\overrightarrow{GA}+\overrightarrow{GB}+\overrightarrow{GC}=\vec{0}$. Then we can write that:
		
		Therefore:
		
		This vector equality ensures that the $G$ point is unique and we can even move it by moving the edges of the triangle!
		\begin{flushright}
			$\blacksquare$  Q.E.D.
		\end{flushright}
		\end{dem}
		\item[P2.] The three medians of a triangle are concurrent. Their intersection is the point $G$.
		\begin{dem}
		To prove that the three medians are concurrent, we will prove that $G$ belongs to each of the three medians.
		
		Just before we proved $G$ satisfies the equality:
		
		As $A'$ is the midpoint of $\overline{BC}$, then we can write that:
		
		Therefore we get:
		
		The vectors $\overrightarrow{AG}$ and $\overrightarrow{AA'}$ are collinear! Therefore the points $A, G, A'$ are aligned. Writing in an other way, the point $G$ is part of the median $\overline{AA'}$ of the triangle $\triangle ABC$. We can even say that is placed at the two thirds of the segment $\overline{AA'}$ from the top $A$.
		
		What we have just shown with the median $AA'$ is of course also true for the other two medians. Therefore:
		
		To resume, the point $G$ is therefore part of the three medians $\overline{AA'}$, $\overline{BB'}$ and $\overline{CC'}$. These three lines are concurrent and the point $G$ is the intersection point. This result will be useful later in our study of polyhedra.
		\begin{flushright}
			$\blacksquare$  Q.E.D.
		\end{flushright}
		\end{dem}
	\end{enumerate}
	\end{theorem}
	Finally, let us indicate that if we were living beings in a two-dimensional space, the triangle is what we should perceive if geometric shapes composed of at least three joined sides would cross our universe through one of its vertex.
	
	Also let us recall that the sum of the angles of any plane triangle is equal to $\pi$ ($180^.\circ$) as proved in the section of Euclidean Geometry page \pageref{angle sum theorem}.
	
	We will stop here this analogy with a two-dimensional space that can be generalized to each geometric shape that we present thereafter (circle and sphere, ellipse and ellipsoid, etc.). The idea was mainly to present the conception that volumes we experience in our daily lives can also be seen as 4-dimensional space shapes crossing our 3-dimensional space.
	
	\subsubsection{Isosceles Triangle}
	\textbf{Definition (\#\mydef):} An "\NewTerm{isosceles triangle}\index{isosceles triangle}" is a triangle that has two sides of equal length. Sometimes it is specified as having two and only two sides of equal length, and sometimes as having at least two sides of equal length, the latter version thus including the equilateral triangle as a special case.
	
	\begin{figure}[H]
		\centering
		\includegraphics{img/geometry/iscoles_triangle.jpg}
		\caption{Example of isosceles triangle}
	\end{figure}
	In an isosceles triangle, the two equal sides are named "\NewTerm{legs}\index{legs}", and the remaining side is named the "\NewTerm{base}\index{base (triangle)}". The opposite angle to the base is named the "\NewTerm{vertex angle}\index{vertex angle}", and the point associated with that angle is named the "\NewTerm{apex}\index{apex}". The two equal angles are named the "\NewTerm{isosceles angles}\index{isosceles angles}".
	
	The perimeter of such a triangle remains:
	
	but as it has two equal sides, then we can always simplify as follows:
	
	The surface as we have proved above in the general case remains:
	
	And if we don't know the height but we know the internal angles we can by apply elementary trigonometry if needed (\SeeChapter{see section Trigonometry page \pageref{circle trigonometry}}).
	
	And the center of gravity remains, as we have proved in the general case (unspecified triangle), at the position:
	
	A remarkable property of an isosceles triangle is that the bisector and the median of the third side, that is of not equal size than the two others, are not-distinguishable and therefore obviously of equal length (\SeeChapter{see section Euclidean Geometry page \pageref{triangles remarkable interior lines}}).
	
	A famous isosceles triangle is the right angle isosceles triangle because its angles are canonical and we get immediately $\theta=\pi/4$ and also we have immediately the length of the hypotenuse by applying the Pythagorean theorem:
	\begin{figure}[H]
		\centering
		\includegraphics{img/geometry/isosceles_right_angle_triangle.jpg}
		\caption{Right angle isosceles triangle}
	\end{figure}
	Also let us recall that the sum of the angles of any plane triangle is equal to $\pi$ ($180^.\circ$) as proved in the section of Euclidean Geometry page \pageref{angle sum theorem}.
	
	\pagebreak
	\subsubsection{Equilateral Triangle}\label{lateral triangle}
	\textbf{Definition (\#\mydef):} An "\NewTerm{equilateral triangle}\index{equilateral triangle}" is a triangle in which all three sides are equal. In the familiar Euclidean geometry, equilateral triangles are also "\NewTerm{equiangular}\index{equiangular}"; that is, all three internal angles are also congruent to each other and are each equal to $\pi/3$. They are regular polygons, and can therefore also be referred to as regular triangles.
	\begin{figure}[H]
		\centering
		\includegraphics{img/geometry/equilateral_triangle.jpg}
		\caption{Equilateral triangle}
	\end{figure}
	\begin{theorem}
	Each angle of an equilateral triangle is $\pi/3$ [rad] ($60^\circ$).
	\end{theorem}
	\begin{dem}
	First let us recall that by the angle sum theorem (page \pageref{angle sum theorem}) we have:
	
	We know also that $\Delta abc$ is equilateral, then:
	
	Then as $\angle\overline{ab}$ and $\angle\overline{ac}$ are congruent, we have:
	
	And proceeding identically:
	
	and:
	
	Hence we can write that:
	
	As using the angle sum theorem:
	
	We get:
	
	\begin{flushright}
		$\blacksquare$  Q.E.D.
	\end{flushright}
	\end{dem} 
	The perimeter of such a triangle remains:
	
	but as it has three equal sides, then we can always simplify as follows:
	
	The surface as we have proved above in the general case remains:
	
	Applying Pythagorean theorem we get if we don't know $h$:
	
	It can be shown that a triangle with given perimeter has the maximum possible area, if it is equilateral.
	\begin{dem}
	Remember the Heron's formula proved above ($\Sigma=\dfrac{1}{2}(a+b+c)$):
	
	To find the conditions for the maximum area, we want to compute derivatives with respect $a$, $b$ and $c$ but since they are not independent variables, we first substitute the semi-perimeter $\Sigma$ into Heron's equation to get rid of one degree of random:
	
	Therefore:
	
	Setting the derivative with respect to $a$ to zero:
	
	Since $b=\Sigma$ is not a solution $A\neq 0\Rightarrow b\neq \Sigma$.
	
	Similarly, setting the derivative with respect to $b$ to zero yields:
	
	Solving simultaneously gives
	
	and substituting back gives:
	
	So $a=b=c$ and the triangle is equilateral.
	\begin{flushright}
		$\blacksquare$  Q.E.D.
	\end{flushright}
	\end{dem}
	And the center of gravity remains, as we have proved in the general case (unspecified triangle), at the position:
	
	A remarkable property of an equilateral triangle is that all its bisectors and medians are non-distinguishable and obviously of the same length (\SeeChapter{see section Euclidean Geometry page \pageref{triangles remarkable interior lines}})!
	
	Let us now see the important little "\NewTerm{Viviani's theorem}\index{Viviani's theorem}" regularly used for graphical representation in the field of materials engineering (mixture), or statistics (mixture designs or distribution of three frequencies of data whose sum is always equal).
	
	Consider the following figure:
	\begin{figure}[H]
		\centering
		\includegraphics{img/geometry/viviani_theorem.jpg}
		\caption{Illustrative figure for Viviani's theorem}
	\end{figure}
	\begin{theorem}
	If we place a point in an equilateral triangle and that from this point we draw a line in the direction of each side, so that the lines are perpendicular to each side. No matter where we place the point, the sum of the perpendicular distances between the point and the sides is equal to the height of the triangle!
	\end{theorem}
	\begin{dem}
	For the proof, let us denote by $d_a,d_b,d_c$ the distances $M$ to the sides, the length of one side and the height $h$. Then we have:
	
	and we have therefore well:
	
	\begin{flushright}
		$\blacksquare$  Q.E.D.
	\end{flushright}
	\end{dem} 
	Also let us recall that the sum of the angles of any plane triangle is equal to $\pi$ ($180^.\circ$) as proved in the section of Euclidean Geometry page \pageref{angle sum theorem}.
	
	\subsubsection{Right Triangle}
	\textbf{Definition (\#\mydef):} A "\NewTerm{right triangle}\index{right triangle}" (American English) or "\NewTerm{right-angled triangle}\index{right-angled triangle}" (British English) is a triangle in which one angle is a right angle (that is, a $\pi/2$-degree angle). The relation between the sides and angles of a right triangle is the basis for trigonometry (see corresponding section).
	\begin{figure}[H]
		\centering
		\includegraphics{img/geometry/right_angle_triangle.jpg}
		\caption{Example of right angle triangle}
	\end{figure}
	The side opposite the right angle is named the "\NewTerm{hypotenuse}\index{hypotenuse}".
	
	The perimeter of such a triangle remains:
	
	The surface as we have proved above in the general case remains:
	
	
	And the center of gravity remains, as we have proved in the general case (unspecified triangle), at the position:
	
	Remarkable property of a right triangle: the triangle as this unique property that we can directly apply to it the Pythagorean theorem (\SeeChapter{see section Euclidean Geometry page \pageref{pythagorean theorem}}).
	
	Also let us recall that the sum of the angles of any plane triangle is equal to $\pi$ ($180^.\circ$) as proved in the section of Euclidean Geometry page \pageref{angle sum theorem}.
	
	\subsubsection{Trapezoid}
	\textbf{Definition (\#\mydef):} A convex quadrilateral with at least one pair of parallel sides is referred to as a "\NewTerm{trapezoid}\index{trapezoid}" in American and Canadian English but as a "\NewTerm{trapezium}\index{trapezium}" in English outside North America. The parallel sides are named the "\NewTerm{bases}\index{bases}" of the trapezoid and the other two sides are named the "\NewTerm{legs}\index{legs}" or the lateral sides (if they are not parallel; otherwise there are two pairs of bases). A "\NewTerm{scalene trapezoid}\index{scalene trapezoid}" is a trapezoid with no sides of equal measure in contrast to the special cases below.
	\begin{figure}[H]
		\centering
		\includegraphics{img/geometry/trapeze.jpg}
		\caption{Example of trapezoid}
	\end{figure}
	Calculating the perimeter of the trapezoid is obvious:
	
	Its surface is calculated using the decomposition:
	
	Finally:
	
	When both sides have the same length, we get the special cases of the square, the rectangle, of the rhombus or of the parallelogram (here we put the order from more specific to the more general, we could put the diamond rhombus as n\textdegree 2):
	\begin{figure}[H]
		\centering
		\includegraphics{img/geometry/trapezoid_classification.jpg}
		\caption{Trapezoid classification}
	\end{figure}
	Also a common use is to retain a more restrictive definition, in order to not take into account these particular figures. We add in this case that the lengths of two parallel sides are not equal (this allows students of high-school classes to avoid confusion arising from the existence of two names for the same object, such as rhombus and trapezius).
	
	\begin{tcolorbox}[title=Remark,colframe=black,arc=10pt]
	There is a particular case of the trapezoid, the "\NewTerm{isosceles trapezoid}\index{isosceles trapezoid}" as we can see on the previous figure, whose two non-parallel sides are the same length. (we can add: as both sides are not parallel, it is not a parallelogram).
	\end{tcolorbox}
	
	\subsubsection{Parallelogram}\label{parallelogram}
	\textbf{Definition (\#\mydef):} The "\NewTerm{parallelogram}\index{parallelogram}" is a special case of the quadrilateral and very important (in the context of the analysis of shapes in physics), where the sides are parallel in pairs and of opposite equal angles (opposite angles are congruent):
	\begin{figure}[H]
		\centering
		\includegraphics{img/geometry/parallelogram.jpg}
		\caption{Parallelogram example}
	\end{figure}
	The rhombus being a particular case of the parallelogram as we already know.
	
	The perimeter of the parallelogram is obviously given by:
	
	For the surface, as this is a special case of the trapezoid, it comes immediately (as we will use only this relation in the book we will not prove other method to calculate the surface):
	
	Given the above figure, it is clear that if we cut the parallelogram into two as below, we see that this is twice the same triangle:
	\begin{figure}[H]
		\centering
		\includegraphics{img/geometry/parallelogram_cutted.jpg}
		\caption{Cutter parallelogram}
	\end{figure}
	Thanks to this we will have a very useful property for the physical analysis of the static forces or phasors in the studies of the wave superposition (\SeeChapter{see section Wave Mechanics page \pageref{superposition wave principle}}).
	
	\begin{tcolorbox}[title=Remark,colframe=black,arc=10pt]
	As we already know parallelograms are therefore trapezoids like (see classification figure above for a refresh).
	\end{tcolorbox}	
	Also let us recall that as  the sum of the angles of any plane triangle is equal to $\pi$ ($180^.\circ$) as proved in the section of Euclidean Geometry page \pageref{angle sum theorem}, and that a parallelogram is made of two identical triangle, the sum of the angles of a parallelogram is then equal to $2\pi$.
	
	\subsubsection{Hexagon}	
	The hexagon is a quite interesting case. Indeed, is used for satellite signal Earth tiling and also because  the "Honeycomb conjecture" states that the hexagonal tiling is the best way to divide a surface into regions of equal area with the least total perimeter. 	
	
	This structure exists naturally in the form of graphite, where each sheet of graphene resembles chicken wire, with strong covalent carbon bonds. Tubular graphene sheets have been synthesised. These are known as carbon nanotubes. They have many potential applications, due to their high tensile strength and electrical properties. Silicene is similar.
	\begin{figure}[H]
		\centering
		\includegraphics[scale=0.3]{img/geometry/hexagons.jpg}
		\caption[A regular hexagonal grid]{A regular hexagonal grid (source: Wikipedia)}
	\end{figure}
	Some applications (the reader can also take a look about the European Extra Large Telescope):
	\begin{figure}[H]
		\centering
		\begin{minipage}{.5\textwidth}
		  \centering
		  \includegraphics[width=7cm,height=7cm]{img/geometry/james_watt_space_telescope}
		  \captionof{figure}{James Watt Space Telescope}
		\end{minipage}%
		\begin{minipage}{.5\textwidth}
		  \centering
		  \includegraphics[width=7cm,height=7cm]{img/geometry/reflector_antenna.jpg}
		  \captionof{figure}{TV broadcast reflector antenna}
		\end{minipage}
	\end{figure}
	
	\textbf{Definition (\#\mydef):} In geometry, a hexagon is a $6$-sided polygon or $6$-gon. The total of the internal angles of any hexagon is $4\pi$. A "\NewTerm{regular hexagon}\index{regular hexagon}" is defined as a hexagon that is both equilateral and equiangular.
	\begin{figure}[H]
		\centering
		\includegraphics{img/geometry/hexagon.jpg}
		\caption[Regular hexagon details]{Regular hexagon details (source: Wikipedia)}
	\end{figure}
	First the perimeter is obviously:
	
	The surface of a regular hexagon of side length $a$ is obviously given by
	
	Therefore:
	
	
	Now let us come back to our honeycomb conjecture!
	
	To begin our search, we had three ideas: the structure of the honeycomb cells should be strong, stable and economical.
	
	The cells are solid if they can be nested and leave no holes. We are therefore interested in tiling.
	
	It can be consider as intuitive that all cells must have the same form and that they will be stable if their form is regular. So we will study the regular polygons.
	
	Let us recall that regular polygon is a geometric figure that has all sides of the same length and all angles of equal measure.
	
	Of all the regular polygons, we have sought those who realize a tiling.
	
	This problem is like looking how it is possible to assemble identical regular polygons around its edges.
	
	For our study, we needed to know two things:
	\begin{enumerate}
		\item The sum of the measures of the three angles of a triangle is equal to $\pi$

		\item A round is equal to $2\pi$
	\end{enumerate}
	Let us try with different shapes:
	\begin{enumerate}
		\item Equilateral triangles ($\pi/3$)

		The three corners of an equilateral triangle are equal, and their measurement is $\pi/3=$. And as $6\cdot (\pi/2) = 2\pi$, we can assembled six equilateral triangles around the vertices and thus realize a tiling:
		\begin{figure}[H]
			\centering
			\includegraphics[scale=0.7]{img/geometry/tiling_equilateral_triangle.jpg}
			\caption[]{Equilateral Triangle Tiling}
		\end{figure}
		
		\item Squares ($2\pi/4$)

		The corners angles of a square measure $\pi/2$. And as $4\cdot (\pi/2)=2\pi$, we can assembled four square around the vertices and so achieve a tiling:
		\begin{figure}[H]
			\centering
			\includegraphics[scale=0.7]{img/geometry/tiling_equilateral_square.jpg}
			\caption[]{Square Tiling}
		\end{figure}
		
		\item Regular pentagons ($2\pi/5$)
		
		The internal angles of a regular pentagon is equal to $2\pi/5$ therefore the internal vertices angles are equal to $\pi-2\pi/5=3\pi/5$.
		
		If we assembled three of theme around a node this gives holes as $3\times (3\pi/5)<2\pi$, and obviously add one more pentagon will not solve the problem!
		\begin{figure}[H]
			\centering
			\includegraphics[scale=0.7]{img/geometry/tiling_equilateral_pentagon.jpg}
			\caption[]{Pentagon Tiling}
		\end{figure}
		We therefore can not provide a tiling with pentagons.
		
		\item Regular hexagons ($2\pi/6$)
		
		The internal angles of a regular pentagon is equal to $2\pi/6$ therefore the internal vertices angles are equal to $\pi-2\pi/6=2\pi/3$.
		
		And as $3\cdot (2\pi/3)=2\pi$, we can assembled three regular hexagons around the vertices and so achieve a tiling:
		\begin{figure}[H]
			\centering
			\includegraphics[scale=0.7]{img/geometry/tiling_equilateral_hexagon.jpg}
			\caption[]{Hexagon Tiling}
		\end{figure}
		
		\item Regular polygons with seven sides and more
		
		From polygon with seven sides and more, tiling is impossible because the internal angles are not integer dividers from $2\pi$.
	\end{enumerate}
	
	The general relation to get the measurement of the angles of a regular polygon with $n$ sides is:
	
	Therefore, the only regular polygons that realize a tiling are equilateral triangles, squares and regular hexagons!!
	
	Finally, we thought that bees must use the least amount of wax to make their cells, while having the maximum space inside.
	
	Therefore the question is: Among the equilateral triangle, the square and the regular hexagon, which he has the greatest area for the smallest perimeter?
	
	In other word we seek to maximize the ratio $A/P$ (or the smallest ratio $P/A$).
	
	So we have for each shape using our previous results:
	\begin{enumerate}
		\item Equilateral triangle:
			
		\item Square:
			

		\item Regular hexagon:
			
	\end{enumerate}
	Among the equilateral triangle, the square and the regular hexagon it is therefore the regular hexagon which gives for the maximum ratio "area on perimeter".
	
	So that the cell structure is solid, stable and economical, bees are incentive to build basic regular hexagons. We can see that this is what they do naturally (after a lot of trial and errors during their evolution)!
	
	\subsubsection{Rhombus}
	\textbf{Definition (\#\mydef):} A "\NewTerm{rhombus}\index{rhombus}" is a simple (non-self-intersecting) quadrilateral all of whose four sides have the same length. Another name is "\NewTerm{equilateral quadrilateral}\index{equilateral quadrilateral}", since equilateral means that all of its sides are equal in length. The rhombus is often named a "\NewTerm{diamond}\index{diamond}", after the diamonds suit in playing cards which resembles the projection of an octahedral diamond, though the former sometimes refers specifically to a rhombus with a $\pi/3$ angle and the latter sometimes refers specifically to a rhombus with a $\pi/4$ angle.
	
	\begin{figure}[H]
		\centering
		\includegraphics{img/geometry/rhombus.jpg}
		\caption{Example of rhombus}
	\end{figure}
	Calculating the perimeter and the surface of the rhombus immediately follow from those of the trapezium.
	
	\pagebreak
	\subsubsection{Circle}
	There are several possible definitions of the circle. Let's look at  least two.
	
	\textbf{Definitions (\#\mydef):}
	\begin{enumerate}
		\item[D1.] A "\NewTerm{circle}\index{circle}" is a special case of a polygon with an infinite number of sides.
		
		\item[D2.] A "\NewTerm{circle}\index{circle}" is a flat curve all of whose points are equidistant from a fixed point named "\NewTerm{center}\index{center}".
	\end{enumerate}
	
	\begin{tcolorbox}[title=Remark,colframe=black,arc=10pt]
	A circle may also be defined as a special ellipse in which the two foci are coincident and the eccentricity is $0$.
	\end{tcolorbox}
	
	\begin{figure}[H]
		\centering
		\includegraphics{img/geometry/circle.jpg}
		\caption{Example of circle}
	\end{figure}
	We will prove in the section of Theoretical Computing that the perimeter of a circle of radius $R$ and therefore diameter $\varnothing =2R$ is given by:
	
	The relation for the surface (of the corresponding "disc"\label{surface of a disc}) can be obtained in two common ways:
	\begin{enumerate}
		\item By the search of the primitive of perimeter $P$ which gives us:
		
		
		\item The second method is more aesthetic and uses the parametric equation of the circle, trivially given by the orthogonal projections of the Cartesian coordinates (\SeeChapter{see section Vector Calculus page \pageref{polar coordinates}}):
		
		We know that the area described by a function $f (x)$ is given by (\SeeChapter{see section Differential and Integral Calculus page \pageref{definite integral}}):
		
		We just have to substitute in this integral the parametrized variables:
		
		Therefore:
		
		The limits of integration are obviously $\theta=[0,2\pi]$ then we have:
		
		We therefore also by this method:
		
	\end{enumerate}
	The length $l$ of an opening angle range $\alpha$ of a circle of radius $R$ is obviously given by:
		
		and the surface $S_t$ of an opening angle $\alpha$ of a circle of radius $R$ identically by:
		
		Let us now determine the surface $S_d$ of cutted a part of disc that is to say:
		\begin{figure}[H]
			\centering
			\includegraphics{img/geometry/circle_cutted_part.jpg}
			\caption{Surface $S_d$ of a disc cutted part}
		\end{figure}
		We have:
		
		\begin{tcolorbox}[title=Remark,colframe=black,arc=10pt]
	A circle may also be defined as a special ellipse in which the two foci are coincident and the eccentricity is $0$.
		\end{tcolorbox}
	
	\subsubsection{Ellipse}\label{ellipse}
	\textbf{Definition (\#\mydef):} An "\NewTerm{ellipse}\index{ellipse}" is a closed curve where every point $P$ is such that the sum of its distances from two fixed points named "\NewTerm{foci $F1,F2$}\index{foci}" is constant (as we see it in details in the section of Analytical Geometry the ellipse can also be seen as an affine transformation of the circle).
	\begin{figure}[H]
		\centering
		\includegraphics{img/geometry/ellipse_foci.jpg}
		\caption{Example of ellipse with its visible foci and eccentricity $e$}
	\end{figure}
	The shape of an ellipse (how 'elongated' it is) is represented by its "\NewTerm{eccentricity $e$}\index{eccentricity}", which for an ellipse can be any number from $0$ (the limiting case of a circle) to arbitrarily close to but less than $1$.
	
	For the mathematical developments that will follow we will rather use the notation of the following figure:
	\begin{figure}[H]
		\centering
		\includegraphics{img/geometry/ellipse_technical.jpg}
	\end{figure}
	\begin{tcolorbox}[title=Remark,colframe=black,arc=10pt]
	The reader must therefore not forget that most important mathematical developments about the ellipse are in the section of Analytical Geometry.
	\end{tcolorbox}
	Let us Introduce to start a small text relative to the calculation of the perimeter of the ellipse!
	
	Given the parametric equation in Cartesian coordinates of an ellipse (\SeeChapter{see section Vector Calculus page \pageref{polar coordinates}}):
	
	The distance between the center of the ellipse and its perimeter is given by the Pythagorean Theorem:
	
	An arc element is then given by:
	
	The perimeter of the ellipse is the given by the integral:
	
	and then it starts to get tougher... This type integral can not easily be calculated using the usual primitives, integration by parts or change of variables. This is what we name a "\NewTerm{second-order elliptic integral $J$}\index{elliptic integral!second-order elliptic integral}\label{elliptic integral ellipse perimeter}" for $0<b<a$ (\SeeChapter{see section Differential and Integral Calculus page \pageref{elliptic integrals}}):
	
	Long developments which we present in a few years in the section of Differential and Integral Calculus give for calculating the perimeter after a limited series:
	
	The surface of the ellipse can be obtained in a very similar manner to that of the circle and the calculations are surprisingly much simpler than those of its perimeter. Remember that the parametric equation the ellipse is:
	
	We know that the area described by a function $f(x)$ is given by:
	
	We just have to substitute in this integral the parametrized variables:
	
	Therefore:
	
	The limits of integration being obviously $\theta=[0,-2\pi]$ we have:
	
	\begin{tcolorbox}[title=Remark,colframe=black,arc=10pt]
	Be careful with this kind of calculations with the order of integration bounds. Indeed, if we had taken the bounds from $[0,\pi]$ (instead of $[0,-\pi]$) one must imagine that the integrated function browse the perimeter in the negative direction of the $x$-axis. So the integral would be necessarily be negative.
	\end{tcolorbox}
	We therefore also have for this method:
	
	\begin{tcolorbox}[title=Remarks,colframe=black,arc=10pt]
	\textbf{R1.} We assume as obvious (and therefore without proof) that the ellipse's center of gravity coincides with the center of it.\\
	
	\textbf{R2.} We send the reader to the study of conical (\SeeChapter{see section Analytical Geometry page \pageref{conics}}) for the calculation of the area of an ellipse from its "ellipse parameter" and its "eccentricity" (everything is prove there).
	\end{tcolorbox}	
	
	\paragraph{Ellipse section surface}\mbox{}\\\\			
	Now let us focus on a common question on Internet Forums: What is the area of a section of an ellipse. To answer that question we start again from:
	
	but that we will write a in slightly different way:
	
	An element of surface in polar coordinates is given by:
	
	Therefore the area of a section has for expression:
	
	Thus:
	
	But using what we have seen in the section of Analytical Geometry we can write that latter:
	
	Therefore:
	
	Thus:
	
	To continue let us make a simple change of variables:
	
	Pulling out $a'^2\cos^2(\theta)$ we get:
	
	This gives us:
	Now we make a serious change of variable:
	
	Thus according to the usual derivatives (\SeeChapter{see section Differential and Integral Calculus page \pageref{usual derivatives}}):
	
	Then the integral becomes:
	
	We know according to usual integrals that (\SeeChapter{see section Differential and Integral Calculus page \pageref{usual primitives}}):
	
	So finally:
		
	
	\pagebreak
	\subsection{Usual Volumes}
	There are several definitions of the concept of volume (surface that limits a body). A definition due to Euclid and another due to the field of Topology (see the section of the same name).
	
	\textbf{Definitions (\#\mydef):}
	\begin{enumerate}
		\item[D1.] A "\NewTerm{volume}\index{volume}" is an entity that has a length, a width and a height.
		
		\item[D2.] A "\NewTerm{volume}\index{volume}" is a 3-manifold topology.
	\end{enumerate}
	The surfaces that surround a body may be planar or curved:
	\begin{figure}[H]
		\centering
		\includegraphics{img/geometry/volumes.jpg}
		\caption{Sample volumes bounded by a surface}
	\end{figure}
	On the left, the body is limited only by planar surfaces, in the middle by one and only one single curved surface, and on the right by a curved surface and two planar surfaces.
	\begin{tcolorbox}[title=Remark,colframe=black,arc=10pt]
	We will focus initially only to the properties (area, volume, center of gravity, moment of inertia ...) of volumes plunged into Euclidean geometries.
	\end{tcolorbox}	

	\pagebreak
	\subsubsection{Polyhedron}
	The study of polyhedra (especially Platonic polyhedra) is very important in physics (e.g. crystallography) and also in mathematics because it gives a nice application of finite groups (\SeeChapter{see section Set Algebra page \pageref{finite group}}). It is therefore appropriate to carefully read what will follow.
	
	Furthermore, the study of polyhedra is also a very educational and aesthetic way to see the implementation of several geometric, trigonometric and vector algebra theorems.
	
	It should be noted before all that the various polyhedra will deliberately not presented as equal in their study. Thus, we will focus only on given properties for some and not for others.
	
	\textbf{Definitions (\#\mydef):}
	\begin{enumerate}
		\item[D1.] A "\NewTerm{polyhedron}\index{polyhedron}" is a solid whose border is formed of plans or plan portions. The plan the portions, which include between them the polyhedron, are the faces; each face being limited by intersections (edges) with neighbouring faces, is a polygon. The sides of this polygon are the edges of the polyhedron. We name "\NewTerm{vertices}\index{vertices}" of a polyhedron any vertex of any of its faces.
		\begin{figure}[H]
			\centering
			\includegraphics{img/geometry/polyhedront_vocabulary.jpg}
			\caption{Polyhedron vocabulary}
		\end{figure}
		
		\item[D2.] A "\NewTerm{regular polygon}\index{regular polygon}" is a polygon which sides and all angles are equal (this definition will be useful for regular polyhedra further below).
	\end{enumerate}
	
	\paragraph{Parallelepiped}\mbox{}\\\\
	\textbf{Definition (\#\mydef):} In geometry, a "\NewTerm{parallelepiped}\index{parallelepiped}" is a three-dimensional figure formed by six parallelograms (the term rhomboid is also sometimes used with this meaning). By analogy, it relates to a parallelogram just as a cube relates to a square or as a cuboid to a rectangle.
	
	Three equivalent definitions of parallelepiped are
	\begin{itemize}
		\item A polyhedron with six faces (hexahedron), each of which is a parallelogram.
	
		\item A hexahedron with three pairs of parallel faces
		
		\item A prism of which the base is a parallelogram.
	\end{itemize}
	The rectangular cuboid (six rectangular faces), cube (six square faces), and the rhombohedron (six rhombus faces) are all specific cases of parallelepiped (see further below for details about these polyhedrons).
	
	\begin{figure}[H]
		\centering
		\includegraphics{img/geometry/parallelepiped.jpg}
		\caption{General parallelepiped}
	\end{figure}
	\textbf{Definitions (\#\mydef):}
	\begin{itemize}
		\item[D1.] The parallelepiped is a "\NewTerm{cube}\index{cube}" if and only if:
		
		\item[D2.] The parallelepiped is a "\NewTerm{cuboid}\index{cuboid}" (rectangular parallelepiped) if and only if:
		
		\item[D3.] The parallelepiped is a "\NewTerm{rhombohedron}\index{rhombohedron}"  if and only if:
		
	\end{itemize}
	\begin{figure}[H]
		\centering
		\includegraphics{img/geometry/cube_cuboid_rhomboid.jpg}
		\caption{Cube, Cuboid and Rhombohedron}
	\end{figure}
	
	\pagebreak
	The volume of the parallelepiped is simply obtained by using trigonometric properties. We can consider mainly two situations:
	\begin{itemize}
		\item We know $H,b,c,\alpha$ and the volume is obviously given by:
		
		Obviously we get therefore for a cuboid where $a=b=c$:
		
		\item  If we know only $\alpha,\beta,\gamma$ and $H,W,L$:
		
	\end{itemize}
	About its surface, it is simply the sum of the surfaces of each of its parallelogram without anything special (we have already prove earlier how to calculate the surface of a parallelogram) and therefore we need to know all deformation angles of each parallelogram.
	
	\subparagraph{Moment of Inertia of a rectangular plate}\label{moment of inertia of a rectangular plate}\mbox{}\\\\
	Now let us calculate the moment of inertia of a plate (rectangular parallelepiped) of cross-sectional area $S$ which axis of rotation is the $y$-axis:
	\begin{figure}[H]
		\centering
		\includegraphics{img/geometry/inertia_plate.jpg}
		\caption[]{Search for the moment of inertia of a plate along the section}
	\end{figure}
	A volume element of the rectangle (in gray) is given by:
	
	and (\SeeChapter{see section Classical Mechanics page \pageref{moment of inertia}}):
	
	Using Steiner's theorem (\SeeChapter{see section Classical Mechanics page \pageref{steiner theorem}})  we can easily get the moment of inertia of this plate when the rotation axis is collinear to the borders of the latter:
	
	and let us now focus on the moment of inertia of the plate relative to the $z$-axis (perpendicular to $x$ and $y$ therefore) and let us put axes so we have:
	\begin{figure}[H]
		\centering
		\includegraphics{img/geometry/inertia_plate_perpendicular.jpg}
		\caption[]{Search for the moment of inertia of a plate along perpendicular axis}
	\end{figure}
	We have:
	
	where $r$ is in the $x$ and $y$ plane.
	
	With:
	
	Therefore:
	
	If the plate is square of side $L$:
	
	
	\pagebreak
	\subparagraph{Moment of Inertia of a triangular plate}\mbox{}\\\\
	We will now prove that it possible to calculate the moment of inertia of an equilateral triangle plate and rectangle triangle plate for an axis passing still through the centroid from the latter.
	
	The method we propose here is direct and almost... brutal.... Indeed, we consider the moment of inertia always from the same axis, but for half the rectangle:
	\begin{figure}[H]
		\centering
		\includegraphics{img/geometry/triangle_inertia_rectangle_centroid.jpg}
	\end{figure}	
	 First we have obviously now:
	
	But we want the moment inertia of this equilateral triangle trough its centroid! That is to say:
	\begin{figure}[H]
		\centering
		\includegraphics{img/geometry/triangle_inertia_centroid.jpg}
	\end{figure}
	Therefore as the center of gravity is placed now on the third of the median starting at the square's center (see proof previously) of gravity and that we make use of Steiner's theorem (\SeeChapter{see section Classical Mechanics page \pageref{steiner theorem}}) we get:
	
	If $a=b=L$ we have the famous special case:
	
	If we have $a=b=L$ then the rectangle as has a height $h=L$ this is why we found most of time the previous relation as:
	
	which is the moment of inertia of an equilateral triangle plate for the special case of an axis passing through the center of mass (gravity).
	
	For sure as for other geometries we can choose any other axis and the possibilities are limitless... But to end we will calculate the moment of inertia of a rectangular triangle plate of dimensions $a,b$ whose rotation axis is along the $y$-axis but translated of $b/2$ (that is to say: the rotation axis is collinear to the basis of length $a$ of the rectangle triangle plate):
	\begin{figure}[H]
		\centering
		\includegraphics{img/geometry/triangle_inertia_vertical.jpg}
	\end{figure}
	Remember for this that we have first proved just above that for a rectangular plate in rotation along the a border was:
	
	Applying once again Steiner's theorem AND for half a plate (therefore for a rectangle triangle plate) we get immediately:
	
	And if we put as usual $L:=h$ then we get the another famous result:
	
	As we can see the inertia momentum in this configuration is the same as that one with a rectangular plate having the rotation axis trough the centroid following the $y$-axis.
	
	\paragraph{Pyramid}\mbox{}\\\\
	\textbf{Definition (\#\mydef):} The "\NewTerm{pyramid}\index{pyramid}" is a polyhedron whose base is a polygon and for side faces triangles connected on a single point made the "\NewTerm{apex}\index{apex}". The pyramid is not in the general case a regular polyhedron! A "\NewTerm{right pyramid}\index{right pyramid}" has its apex directly above the centroid of its base. Non-right pyramids are named "\NewTerm{oblique pyramids}\index{oblique pyramid}". A "\NewTerm{regular pyramid}" has a regular polygon base and is usually implied to be a right pyramid. When unspecified, a pyramid is usually assumed to be a "\NewTerm{regular square pyramid}\index{regular square pyramid}".
	\begin{figure}[H]
		\centering
		\includegraphics{img/geometry/pyramid.jpg}
		\caption{Example of pyramid (irregular basis) but non-oblique}
	\end{figure}
	Consider a surface $S(t)$ of the section of the pyramid with the plane of $z=t$ (that is to say for $t=0$ we have the surface of the basis of the pyramid), then the volume $V$ we are looking for will be equal to:
	
	We speak of plane equation when there is no defined basis for now. In fact, in the integral, $t$ varies between $0$ and $h$. This implies that we take a basis centered on $H$ (the projection of the apex of the pyramid on the basis), of axis therefore oriented towards $\overline{HO}$. The other two axes are chosen arbitrary in the plane of the basis of the pyramid.
	
	We must now clarify what is $S(t)$ depending on $t$:
	
	Let $S(0)=S$ be the area of the base of the pyramid. The section of the pyramid by the plane of equation $z=t$ is deduced by the scaling of center $O$ and ratio $t/h$. So the integral is written:
	
	where the fact that having taken the square of $t/h$ is due to the fact that each inner term of $S$ is the product of two terms (by the definition of a surface that is the multiplication of the length) each of homothetic ratio $t / h$.
	
	Thus we have:
	
	
	\subparagraph{Moment of Inertia of a regular square pyramid}\mbox{}\\\\
	We have along the axis of symmetry of the regular square pyramid of height $h$ and side $a$:
	
	As:
	
	We have for the mass of the pyramid:
	
	So:
	
	Therefore:
	
	
	\paragraph{Right Prism}\mbox{}\\\\
	\textbf{Definition (\#\mydef):} The "\NewTerm{(right) prism}\index{ right prism}\index{right prism}" is a polyhedron whose bases are two equals polygons with parallel sides (they have the same surface!), The side faces are parallelograms. Therefore, the prism is not a regular polyhedron! The two parallel sides and shape are named bases of the (right) prism.
	\begin{figure}[H]
		\centering
		\includegraphics{img/geometry/prism.jpg}
		\caption{Example of right prism}
	\end{figure}
	To calculate the volume $V$ of a right prism, we simply multiply the area of its base $B$ by its height $h$:
	
	As its base is a polygon, that is to say, it can be a triangle, a quadrilateral or a pentagon ... The reader must know how to calculate these areas to calculate the volume of the right prism.
	
	A right prism with a triangular basis is named a "triangular prism", a right prism with a rectangular basis is named a... "cuboid", a right prism with a square base is named a... "cube" and a right prism with a circular basis is named a... "cylinder"...
	
	\paragraph{Regular Polyhedron}\mbox{}\\\\
	\textbf{Definitions (\#\mydef):}
	\begin{enumerate}
		\item[D1.] A "\NewTerm{regular polyhedron}\index{regular polyhedron}" is made of all identical and regular surfaces. A common equivalent definition is that a regular polyhedron is such that faces are congruent regular polygons which are assembled in the same way around each vertex.
		
		\item[D2.] A "\NewTerm{convex polyhedron}\index{convex polyhedron}" is such that each point of a line segment joining any two points belongs to the (inside of) polyhedron.
	\end{enumerate}
	The regular polyhedra are at the number of nine, where five of them are convex and are known as the Platonic Solids. We sometimes named regular polyhedra only the Platonic solids and it is these that will interest us here.
	
	\begin{theorem}
	We will prove that there are only five convex regular polyhedra, which are therefore named the "\NewTerm{five Platonic solids}\index{five Platonic solids}" (the other columns of the table below will be proved and explained a little further), a regular polyhedron is identified by its "\NewTerm{Schläfli symbol}\index{Schläfli symbol}" of the form $(m,n)$, where $m$ is the number of sides of each face and $n$ the number of faces meeting at each vertex.
	
	For the proof  we will denote by $A$ the number of edges, $S$ the number of nodes (vertices) and $F$ the number of faces.
	\end{theorem}
	\begin{table}[H]
	\begin{center}
		\begin{tabular}{|>{\centering\arraybackslash}m{2.2cm} |>{\centering\arraybackslash}m{2.3cm} |>{\centering\arraybackslash}m{3.5cm} |>{\centering\arraybackslash}m{0.8cm} |>{\centering\arraybackslash}m{0.8cm} |>{\centering\arraybackslash}m{0.8cm} |>{\centering\arraybackslash}m{2.5cm} |}
			  \hline
			  \rowcolor[gray]{0.75}Name & Schläfli $(m,n)$ & Image & $S$ & $A$ & $F$ & $F-A+S$ \\ \hline
			  Tetrahedron & $(3, 3)$ & \includegraphics{img/geometry/platon_solid_tetrahedron.jpg} & $4$ & $6$ & $4$ & $2$\\ \hline
			  Cube & $(4, 3)$  & \includegraphics{img/geometry/platon_solid_cube.jpg} & $8$ & $12$ & $6$ & $2$ \\ \hline
			  Octahedron & $(3, 4)$ & \includegraphics{img/geometry/platon_solid_octahedron.jpg} & $6$ & $12$ & $8$ & $2$  \\ \hline
			  Dodecahedron & $(5, 3)$ & \includegraphics{img/geometry/platon_solid_dodecahedron.jpg} & $20$ & $30$ & $12$ & $2$  \\ \hline
			  Icosahedron & $(3, 5)$ & \includegraphics{img/geometry/platon_solid_icosahedron.jpg} & $12$ & $30$ & $20$ & $2$ \\\hline
		\end{tabular}
		\end{center}
		\caption{Five regular polyhedra (Platonic Solids)}
	\end{table}
	\begin{tcolorbox}[title=Remark,colframe=black,arc=10pt]
	A Platonic hydrocarbon is a hydrocarbon (molecule) whose structure matches one of the five Platonic solids, with carbon atoms replacing its vertices, carbon–carbon bonds replacing its edges, and hydrogen atoms as needed. As far as we know, not all Platonic solids have molecular hydrocarbon counterparts.
	\end{tcolorbox}
	\begin{dem}
	Let $m$ be the number of sides of each surface of a regular polyhedron, $n$ the number of edges that meet at each vertex. We have therefore that each angle of any face is given by:
	
	
	\begin{tcolorbox}[colback=red!5,borderline={1mm}{2mm}{red!5},arc=0mm,boxrule=0pt]
	\bcbombe Caution!!! It is the angle $2\beta$ that defines the angle of a face and not just $\beta$!
	\end{tcolorbox}
	
	
	Which is deduced from the following figure:
	\begin{figure}[H]
		\centering
		\includegraphics{img/geometry/polyhedron_edges.jpg}
		\caption{Angles between the edged of a polyhedron}
	\end{figure}
	where we have:
	
	and:
	
	But the sum of $n$ angles grouped around a vertex is smaller than the $n$ angles which intersect a plane into equal parts (we assume that intuitive by cutting)! Each of them is less than:
	
	therefore:
	
	hence:
	
	The numbers $m$ and $n$ are both at least equal to $3$ (the smallest polygon being a triangle). It follows that the only possible cases are:
		
	\begin{flushright}
		$\blacksquare$  Q.E.D.
	\end{flushright}
	\end{dem}

	Let us denote now by $F$ the number of faces, $A$ the number of edges and $S$ the number of vertices (nodes). Let recall that we have proved in the section of Graph Theory the "\NewTerm{Euler's formula}\index{Euler's formula}" (also named "\NewTerm{Descartes-Euler's theorem}\index{Descartes-Euler's theorem}" or "\NewTerm{polyhedral formula}\index{polyhedral formula}") such that:	
	
	and it is of course also valid for the flattening of a polyhedron in the plane (and therefore of a polyhedron in the space).
	\begin{tcolorbox}[title=Remark,colframe=black,arc=10pt]
	The representation as a flattening graph of a polyhedron is named a "\NewTerm{Schlegel diagram}\index{Schlegel diagram}" and is used in some fields of organic chemistry.
	\begin{center}
		\includegraphics{img/geometry/schlegel_diagram.jpg}
	\end{center}
	\end{tcolorbox}	
	In the case of regular polyhedra, each side (face) has $m$ edges so that $m\cdot F$ is the whole number of edges of all sides and as each edge as encounter exactly two faces, we have the equality (take an example to be convinced if necessary!):
	
	and as $n$ is the number of edges that meet at each vertex, and that each edge connects two vertices, we also have:
	
	Therefore:
	
	By injecting into Euler's formula, we have then:
	
	and we fall back on the inequality of the previous theorem. Let us take again our calculation:
	
	from which we get:
	
	We can now undertake the classification of regular polyhedra:
	\begin{enumerate}
		\item The tetrahedron $(m,n)=(3,3)$:
		
		
		\item The octahedron $(m,n)=(3,4)$:
		
		
		\item The hexahedron or cube $(m,n)=(4,3)$:
		
		
		\item The icosahedron $(m,n)=(3,5)$:
		
		
		\item The dodecahedron $(m,n)=(5,3)$:
		
	\end{enumerate}
	this completes our classification.
	
	Let us prove now that any polyhedron composed only of pentagons and hexagons has exactly mandatory  twelve pentagons (and therefore there is no constraint on the number of hexagons). It is not a priori not very intuitive!
	
	This proof will explain for example us why the soccer ball and the fullerene molecule have twelve pentagons:
	\begin{figure}[H]
		\centering
		\includegraphics{img/geometry/pentagone_rule.jpg}
		\caption{Known polyhedra made of $12$ pentagons}
	\end{figure}
	\begin{dem}
	For the proof, we start from the Euler's formula:
	
	Given $M$, the number of pentagons and $N$ the number of hexagons, then we have the number of faces that is equal to:
	
	In addition, each vertex is shared by three polygons in the case of the soccer ball and of the fullerene. Therefore, the number of vertices is equal to:
	
	Each edge is shared by two polygons. Therefore, the number of edges is:
	
	Then injecting Euler's formula:
	
	Therefore, to satisfy:
	
	$M$, the number of pentagons, must be equal to $12$.
	\begin{flushright}
		$\blacksquare$  Q.E.D.
	\end{flushright}
	\end{dem}
	
	\paragraph{Regular Tetrahedron}\mbox{}\\\\
	A tetrahedron is a polyhedron composed of $4$ triangular faces, $6$ straight edges, and $4$ vertex corners. The tetrahedron is the simplest of all the ordinary convex polyhedra and the only one that has fewer than $5$ faces.
	
	The tetrahedron is one kind of pyramid, which is a polyhedron with a flat polygon base and triangular faces connecting the base to a common point. In the case of a tetrahedron the base is a triangle (any of the four faces can be considered the base), so a tetrahedron is also known as a "\NewTerm{triangular pyramid}\index{triangular pyramid}".
	
	\textbf{Definition (\#\mydef):} A "\NewTerm{regular tetrahedron}\index{regular tetrahedron}" is one in which all $4$ faces are equilateral triangles. It is one of the five regular Platonic solids, which have been known since antiquity.
	
	We have shown that for the tetrahedron $F=4,m=3$ and it is relatively easy to guess that such a polyhedron is formed by $3$ identical equilateral triangles as shown in the figure below:
	\begin{figure}[H]
		\centering
		\includegraphics{img/geometry/regular_tetrahedron.jpg}
		\caption{Example of regular tetrahedron}
	\end{figure}
	For this purpose let us start by studying the following equilateral triangle:
	\begin{figure}[H]
		\centering
		\includegraphics{img/geometry/equilateral_triangle_for_tetrahedron_study.jpg}
		\caption[]{Equilateral triangle to start with to calculate the volume of the tetrahedron}
	\end{figure}
	In this equilateral triangle, $a$ is the side, $h$ is the height. The perpendicular bisectors are $h$, $h'$, $h''$ of the respective sides $\overline{BC}$, $\overline{AB}$, $\overline{AC}$.

	The segments $h$ and $h'$ intersect at a point $P$ (barycentre). By construction of the equilateral triangle, we have $\overline{PC}=\overline{PB}=\overline{PA}$ (just apply Pythagoras theorem to prove it if you want).
	
	Furthermore, we have already proved in our study of the triangle, that the barycentre of the latter is always at $2/3$ of the median height. As medians and bisectors are merged in the case of the equilateral triangle, then we have $\overline{AP}=2/3 \cdot h$.
	
	Now, if we draw a straight line passing through the point $P$ and perpendicular to the plane in which the triangle is located. Given $D$ a point on this line, as we have $\overline{PB}=\overline{PC}=\overline{PA}=2/3\cdot h$ we will have of course $\overline{BD}=\overline{AD}=\overline{CD}$ (again just apply Pythagoras theorem!).
	
	We only have to find a way such that $\overline{BD}=a$ and we will have the regular tetrahedron we wanted. We then calculate:
	
	and therefore:
	
	therefore
	
	\begin{figure}[H]
		\centering
		\includegraphics{img/geometry/tetrahedron_variables_summary.jpg}
		\caption[]{Tetrahedron variables summary}
	\end{figure}
	The bisector $\overline{BD}$ (on the right figure) passing through $M$ intersects $H$ at a point O, which is nothing more than the center of the sphere circumscribed to the tetrahedron. Indeed, by construction, we have $\overline{\text{O}B}=\overline{\text{O}C}=\overline{\text{O}A}$ and the bisector gives us $\text{O}B=\text{O}D$.
	
	The Thales theorem also gives us:
	
	and for the developments that will follows we will ask put: $H=\overline{PD}$, $\overline{\text{O}D}=R$.
	
	Now let us calculate the total area. It will necessarily be given by the surface of one side multiplied by the number of faces, and as we have proved how to calculate the area of a triangle above it comes immediately:
	
	For the volume, it's also just as simple as we have proved earlier above that is was equal to that of a pyramid. It then comes immediately:
	
	Therefore:
	
	and:
	
	
	\pagebreak
	\paragraph{Regular hexahedron (cube)}\mbox{}\\\\
	The cube is a regular polyhedron that is probably the most familiar to us, it has $6$ faces and its construction  does not require to be presented.
	
	The cube is also a square parallelepiped, an equilateral cuboid and a right rhombohedron. It is a regular square prism in three orientations, and a trigonal trapezohedron in four orientations.
	\begin{figure}[H]
		\centering
		\includegraphics{img/geometry/cube.jpg}
		\caption{Example of regular hexahedron (cube)}
	\end{figure}
	Since all sides are of length $a$, the surface is simply given by the multiplication of the $6$ faces surfaces. So:
	
	and the volume:
	
	
	\paragraph{Regular octahedron}\mbox{}\\\\
	We have shown that for the octahedron $F=8,m=4$ and it is relatively easy to guess that the regular octahedron is formed (by definition!!!) of $8$ identical equilateral triangles.
	
	 A regular octahedron is a square bi-pyramid in any of three orthogonal orientations. It is also a triangular anti-prism in any of four orientations.
	
	To build, and demonstrate that it is possible to build such a polyhedron, we put as before that his side is equal to $a$.
	\begin{figure}[H]
		\centering
		\includegraphics{img/geometry/regular_octahedron.jpg}
		\caption{Example of regular octahedron }
	\end{figure}
	Then we denote by O the point of intersection of the two diagonals. We have therefore by applying the Pythagorean theorem:
	
	and:
	
	On the perpendicular line to the plane containing our square, and passing through O, we add two vertices $E$, $F$ at a distance that we calculate as follows:
	
	from which we get:
	
	Therefore:
	
	Our polyhedron is the well formed by $8$ equilateral identical triangles. Each vertex has $4$ corners and $4$ faces, which allows us to affirm that it is regular and so ends our construction.
	
	The surface of the regular octahedron is:
	
	where $h$ is the height of the equilateral triangle of side $a$ that we have already calculated above. 

	For the volume, this is still based on that of the pyramid. So:
	
	And we will assume that it is obvious to the reader that our octahedron is inscribed in a sphere of radius $R$ whose center is the point $O$. For $R$, we have:
	
	Let us show already that now we can build the regular icosahedron from the octahedron and this latter exists and is buildable.

	For this purpose, we will first consider a reference frame of the octahedron placed at the origin O corresponding to its barycentre as visible in the figure below:
	\begin{figure}[H]
		\centering
		\includegraphics{img/geometry/octahedron.jpg}
		\caption{Example of the octahedron}
	\end{figure}
	We then have:
	
	After this, let us consider the following figure:
	\begin{figure}[H]
		\centering
		\includegraphics{img/geometry/tetrahedron_equilateral_triangle_search.jpg}
		\caption{Searches of the position of the equilateral triangle}
	\end{figure}
	In the figure above, $A'$ is a point that start from $A$ and that arrives in $B$, and given $B'$ a point which start from $C$ and arrives at $B$, and finally $E'$ a point that start from $B$ and arrives on $E$. These three points go together and moving at the same speed. If we follow these three points that form a triangle $A'B'E'$, we feel intuitively that there is a place such that $A'B'E'$ is an equilateral triangle.
	
	Let us determine this place:
	
	and therefore:
	
	and we want:
	
	Therefore:
	
	That is to say:
	
	Which simplifies in:
	
	and as $0\leq \mu \leq 1$, we get for the resolution of this polynomial of the second degree (\SeeChapter{see section Calculus page \pageref{second order polynomials}}) the only acceptable solution:
	
	where the reader may have perhaps noticed that this is the inverse of the golden ratio...
	
	According to the figure below, if we put:
	
	then we fall back on the same value for $\mu$ (the reader can check this himself, or we can on request add the detailed calculations here) and ditto for all other points:
	\begin{figure}[H]
		\centering
		\includegraphics{img/geometry/regular_icosahedron_construction.jpg}
		\caption{Construction of a regular icosahedron}
	\end{figure}
	Our new polyhedron therefore has one face by face of the octahedron and one face by edge of the octahedron. We have therefore $20$ faces composed of identical equilateral triangles. In addition, $5$ edges and $5$ faces meet at each vertices. We then get a regular icosahedron!!!
	
	We then have for the coordinates of each vertex (the reader must be observed that the vertices are opposed in pairs one component on the figure):
	
	
	\paragraph{Regular Icosahedron}\mbox{}\\\\
	We saw earlier how to build the regular icosahedron. There it exists!
	\begin{figure}[H]
		\centering
		\includegraphics{img/geometry/regular_icosahedron.jpg}
		\caption{Representation of the regular icosahedron}
	\end{figure}
	Knowing the coordinates of the different edges, let us now calculate the surface and volume of the regular icosahedron.
	
	The calculation of the surface is simple since it is $20$ equilateral triangles. We have then:
	
	therefore:
	
	Hence:
	
	Therefore:
	
	The calculation of the volume being quite more subtle! Let's go for it...
	
	The icosahedron is built around the pentagon and the golden section as we were able to notice in our previous study of the octahedron.
	
	If ever the reader is not convinced here is an additional figure where we see that each edge of the icosahedron is an edge of a pentagon ($AFECB$, $LGHJK$, $DAJKC$, $DEGHA$, $BJILC$, $FELIH$,...):
	\begin{figure}[H]
		\centering
		\includegraphics{img/geometry/pentagon_in_icosaedron.jpg}
		\caption{Pentagons in the regular icosahedron}
	\end{figure}
	Using the method of the pyramids, we have $20$ equilateral triangles that form the basis of a pyramid whose height goes up to the origin $O$ of the icosahedron (or the origin confused the with circumscribed or inscribed corresponding sphere).
	
	Take for example the base $ABD$ with the intersection of mediators located at the point $M$ as shown below:
	\begin{figure}[H]
		\centering
		\includegraphics{img/geometry/icosahedron_pyramid.jpg}
		\caption{Representation of one of the pyramids of the regular icosahedron}
	\end{figure}
	As we know, we proved that the volume of a pyramid is:
	
	The surface $b$ in our situation that of the equilateral triangle $\Delta ADB$ and the height $h$ is the segment $\overline{\text{O}M}$.

	If we denote by $a$ the triangle side, then the surface is given by:
	
	To find $h$, we know by construction of the point $M$ that the triangle OMA, OMB, OMD are right triangles.

	Let us work arbitrarily with the triangle $OMD$. First, let determine the length $\overline{DM}$. We have proved in our study of the bisector of length $H$ of the equilateral triangle (\SeeChapter{see section Euclidean Geometry page \pageref{equilateral triangle}}) that $\overline{DM}$ is then equal to:
	
	But:
	
	So finally:
	
	To find $h$ we must find the length $\overline{OD}=r$ in terms of length of the edges $a$ of icosahedron. For this, we must recognize one of the fundamental properties of the icosahedron.

	Before going further, let us show a property of the pentagon below with its diagonals $d$ and sides $c$:
	\begin{figure}[H]
		\centering
		\includegraphics{img/geometry/parallelogram_inside_pentagon.jpg}
		\caption{Parallelogram in a pentagon}
	\end{figure}
	$BSEA$ is a parallelogram. Indeed, the diagonal $\overline{BD}$ is parallel to the side $\overline{AE}$ (for example, because both are perpendicular to the axis of symmetry passing through $\overline{OC}$). Since $S$ is on $\overline{BD}$, it proves that $\overline{BS}$ and $\overline{AE}$ are parallel. We show in the same manner that $\overline{AB}$ and $\overline{SE}$ are parallel.
	
	We deduce that:
	
	and same for $\overline{CS}$:
	
	Let us continue ... we have the equality $\widehat{BAE}=\widehat{CSD}$. Furthermore, as $CD$ and $BE$ are parallel, the triangles $SCD$ and $ABE$ are similar. Therefore, the distances between their sides are kept (Thales):
	
	hence the relation:
	
	After some modifications:
	
	If $\mu$ now designates the ratio $d/c$, the above relation becomes:
	
	and $\mu$ being strictly positive, we have already seen during our study of the octahedron that the only positive root is the golden ratio:
	
	We have just finish to prove that a diagonal of a pentagon is equal to the golden ratio multiplied by the length of an edge of this same pentagon.
	
	Thus we have in the pentagons $AFECB$ and $LGHJK$ of our icosahedron:
	
	Let us notice also the rectangle $FBGK$ whose barycentre (centroid) coincides with that of the icosahedron. Moreover, $\overline{FK}$ and $\overline{BG}$ represent by construction the diameter of the sphere circumscribed to the icosahedron and therefore $\overline{OF}=\overline{OK}$ is the radius $r$ that we are looking for.

	We have:
	
	Therefore:
	
	hence:
	
	Now, we can calculate $h$:
	
	But:
	
	as the golden ratio is a root of $x^2=x+1$.

	Therefore:
	
	Finally:
	
	and:
	
	Therefore the volume of one pyramid of the icosahedron is:
	
	As the are $20$ pyramids:
	
	
	\paragraph{Regular Dodecahedron}\mbox{}\\\\
	After failing to find in the literature an aesthetic and simple way the of the proof of the construction feasibility of the dodecahedron, we will move in for now (it is possible to live without it...).
	
	Let us simply notice that the dodecahedron consists of $12$ pentagons and its volume is comparable to that of a cube on which we have placed on each face a sort of little roof which ultimately will give the pentagons (you can see below the cube in light grey and with its $6$ roofs):
	\begin{figure}[H]
		\centering
		\includegraphics{img/geometry/regular_dodecahedron.jpg}
		\caption{Regular dodecahedron}
	\end{figure}
	For our study of the dodecahedron, we will only focus to determine its surface and volume.
	
	For this, let us consider first the regular pentagon below:n
	\begin{figure}[H]
		\centering
		\includegraphics{img/geometry/regual_pentagone_for_study_of_dodecahedron.jpg}
		\caption{Regular pentagon for the study of the volume of the regular dodecahedron}
	\end{figure}
	We will first have to determine the length of $h$ and $b$.

	Let us first recall that we have already during our study of the icosahedron proved that the diagonal of a pentagon is connected to the length of its sides by the relation:
	
	where $\mu$ is the golden ratio. It remains to  then to determine $h$.

	It is obvious first that $\beta=2\pi/10=\pi/5=\alpha/2$ and that:
	
	But we have two missing information here: the angle and $c$. Begin by determining how much is the cosine without using the calculator (you will understand why ...).

	We have first by the identity (\SeeChapter{see section Trigonometry page \pageref{remarkable trigonometric identities}}):
	
	that:
	
	Which can also be written:
	
	But this can also be written always using the same trigonometric identity:
	
	Thus after simplification:
	
	By a change of variable and rearranging the different terms:
	
	We have $-1$ and $1/2$ that are two obvious roots so we get (\SeeChapter{see section Calculus page \pageref{double root}}):
	
	We just have to solve a simple equation of the second degree, the solution is trivial using the methods seen in the section Calculus and we get:
	
	Either by taking the only feasible solution, then we have:
	
	So we fall back on the golden ratio again! and this leads us directly to write:
	
	It remains to us to determine $c$. We have:
	
	and as $\mu^2-\mu-1=0$ we have:
	
	and therefore:
	
	hence:
	
	We have therefore for the calculation of the area of the dodecahedron, a surface composed of $12$ pentagons, each of which is composed of by $5$ triangle of base $a$ and  height $h$:
		
	Therefore:
	
	To calculate the volume, we're going to use the trick mentioned at the beginning. That is to say to cut in a first time the dodecahedron in a parallelepiped of side:
	
	since the side of the parallelepiped is a diagonal of the pentagon of side $s$ and of $6$ small roofs (which are clearly visible on the figure previously given of the dodecahedron).
	
	Every little roof following two different views have the following lengths (where we fall back obviously for some edges those of pentagons $s$ or the diagonals $c$ thereof): 
	\begin{figure}[H]
		\centering
		\includegraphics{img/geometry/dodecahedron_roofs.jpg}
		\caption[]{Dodecahedron roofs}
	\end{figure}
	Each small roof, we treat separately their ends by separating (cutting) them and merging them together. Finally, we have two pieces to study: the major visible part of the roof left in the figure below and the secondary part of the roof on the right in the figure below, which is nothing else other than the merging of the extremities of one small roof:
	\begin{figure}[H]
		\centering
		\includegraphics{img/geometry/dodecahedron_roofs_decomposition.jpg}
		\caption[]{Dodecahedron roofs decomposition for volume calculation}
	\end{figure}
	We must therefore determine $x$ and $l$ and $h$ since $c$ and $s$ are already known to us.

	First, we see trivially that:
	
	From the Pythagorean theorem, we then have:
	
	Therefore it comes:
	
	Hence:
	
	We can now calculate the volume $V_p$ of the $6$ small roofs:
	
	Therefore, the total volume of the dodecahedron is finally the volume of $6$ small roofs summed with the volume of the central parallelepiped:
	
	Finally:
	
	
	\subsubsection{Usual Solids of Revolution}
	Let us recall that as seen in the section of Analytical Geometry a "\NewTerm{solid of revolution}\index{solid of revolution}\label{solid of revolution}" is a solid figure obtained by rotating a plane curve around some straight line (the axis) that lies on the same plane. and the more formally a surface of revolution is a surface obtained by rotating a plane curve (e.g. $y=f(x)$), named the "\NewTerm{generatrix}\index{generatrix}", around the $y$-axis (for example!). So we pass from a plane of $\mathbb{R}^2$ to a basis in $\mathbb{R}^3$.
	
	Indeed many surfaces (and also some of which we've seen in the section of Analytical Geometry as the sphere, tore and cylinder) can be described by revolving a primary form of smaller size and then by rotation.
	\begin{figure}[H]
		\centering
		\includegraphics[scale=0.85]{img/geometry/revolution_surface.jpg}
	\end{figure}
	Let us see before going further a general method for determining the area of a body of revolution. That is to say the surface of the body generated by the rotation of a finite length curve about an axis:
	\begin{figure}[H]
		\centering
		\includegraphics{img/geometry/body_of_revolution_idea.jpg}
	\end{figure}
	For this, we notice that when the curve is given a function $y=f(x) $ and we notice by Pythagoras (see figure below) that the element of length $\mathrm{d}l$ satisfies (relation that we have already meet in other sections of this book):
	
	Therefore:
	
	\begin{figure}[H]
		\centering
		\includegraphics{img/geometry/body_of_revolution_formal_figure.jpg}
	\end{figure}
	Thus, the surface element generated by the rotation of the element of length $\mathrm{d}l$ is given by:
	
	The area of the surface of revolution generated by a function $f:[a,b]\mapsto \mathbb{R}$ continuously differentiable  is then given by the relation:
	
	\begin{tcolorbox}[title=Remark,colframe=black,arc=10pt]
	In the section Mechanics we will use another approach that can be used to calculate the surface (and volume) of bodies of revolution and that is named "Pappus's second centroid Theorem". In fact as we will see, it is more used in practice to determine the centroid position rather than the surface (or volume) of a body of revolution.
	\end{tcolorbox}
	
	\pagebreak
	\paragraph{Cylinder}\label{cylinder}\mbox{}\\\\
	\textbf{Definition (\#\mydef):} A "\NewTerm{cylinder}\index{cylinder}" is a surface generated by a line moving parallel to a fixed direction by meeting a fixed plane curve (circle), which plane cuts the given direction.
	\begin{figure}[H]
		\centering
		\includegraphics{img/geometry/cylinder_revolution.jpg}
		\caption{Example of a cylinder}
	\end{figure}
	The volume of an cylinder of revolution of height $x=r$ and height $h$ is calculated by the method of discs knowing that the surface of a circle (disc) is $\pi r^2$:
	
	Therefore:
	
	The surface of a cylinder is simply the sum of the surface of the both discs and of the surface of the folded rectangle of height $h$ and length of $2\pi r$:
	
	Now let us calculate the moment of inertia of a solid cylinder with respect to its vertical symmetry axis (revolution axis):
	\begin{figure}[H]
		\centering
		\includegraphics{img/geometry/cylinder_inertia.jpg}
	\end{figure}
	We have:
	
	Therefore:
	
	Given now $G$ the cylinder's center of gravity, $G_z$ coincides with the cylinder axis of the revolution. The axes $G_y$ and $G_x$ play identical roles. The moments of inertia $J_{G_x}$ and $J_{G_y}$ with respect to these axes are equal and are written:
	
	hence:
	
	hence:
	
	The first integral is in fact the moment of inertia of the cylinder relatively to the axis $G_z$ and we know it's equal to:
	
	The second integral is easily calculated by cutting the cylinder into slices  of thickness $\mathrm{d}z$ perpendicular to the axis $G_z$. The mass of the elementary slice is $\mathrm{d}m=\pi r^2\rho\mathrm{d}z$ thus:
	
	The moment of inertia of a cylinder relative to an axis perpendicular to its axis of revolution is thus written:
	
	In many high-school problems $R\ll h$ then we can found the latter relation under the form:
	
	Now let us found the moment of inertia of a cylinder rotation around an axis perpendicular to its rotation axis and at a distance $h$ of one of it ends. To calculate it, we consider:
	\begin{itemize}
		\item The cylinder is cut into infinitesimally many pieces of infinitesimally thin slices

		\item Each of these slices have a mass of $\mathrm{d}m$ and length of $\mathrm{d}x$

		\item A variable to sum. E.g. in this problem, we are summing from left of the axis to right of axis. The variable is $x$.
	\end{itemize}
	Now, we show our formula for the calculation for moment of inertia first:
	
	Recall that we are using $x$ to sum. Hence, we have to force a $\mathrm{d}x$ into the equation for moment of inertia. Now, lets find an expression for $\mathrm{d}m$. Since the rod is uniform, the mass varies linearly with distance such that:
	
	Using the relation for $mathrm{d}m$, we substitute it into the first relation. Hence, we have:
	
	Now we know obviously that:
	
	Substituting $\mathrm{d}J$, (write the appropriate limits):
	
	where the lower limit is $-h$ because the left side of the cylinder is $-h$ units away from the axis of rotation (we take right as positive). This part is in fact the tricky part...

	Solving the integration, we have the moment of inertia for a uniform rigid cylinder for any perpendicular rotation axis:
	
	Now in the special case $h=0$ (used in many high-school to study a falling chimney) we found the famous relation that can been found in many textbooks:
	
	If the rotation axis is through the center of mass of the rod. (remember rod is uniform, hence $h=L/2$) we fall back on:
	
	
	Now let us also calculate the moment of inertia of a tube or empty cylinder of non-zero thickness (always provided in formula summary books). 
	
	The moment of inertia of a tube relative to its axis of revolution is a great classical treatment of the cylinder inertia moment. Thus, let us consider a tube outer radius $r_e$ and inner radius $r_i$. As (\SeeChapter{see section Classical Mechanics page \pageref{sums of moments of inertia}}):
	
	Therefore the moment of inertia of a tube can be seen as the moment of inertia of cylinder of radius equal to the external radius of the tube decreased by the moment of inertia of cylinder of radius equal to the internal radius of the tube. So:
	
	and if $\bar{r}=r_e^2\cong r_i^2$, then we get the classical relation available in many physics textbooks:
	
	
	\pagebreak
	\paragraph{Cone}\mbox{}\\\\
	\textbf{Definition (\#\mydef):} A "\NewTerm{cone}\index{cone}" is a three-dimensional geometric shape that tapers smoothly from a flat base (frequently, though not necessarily, circular) to a point named the "\NewTerm{apex}\index{apex}" or "\NewTerm{vertex}\index{vertex}".
	\begin{figure}[H]
		\centering
		\includegraphics{img/geometry/cone_shape.jpg}
		\caption{Cone}
	\end{figure}
	The line passing through the points $(r,0)$ (end of the base of the cone/center of the disc) and $(0,h)$ (apex of the cone) is given obviously by:
	
	Indeed, when $x=0$, we have $y=h$ and when $x=r$ we have $y=0$.
	
	The rotation of this line relative to the $y$-axis gives the volume of the cone:
	
	Finally:
	
	To calculate the lateral surface of a cone, we will parametrized the straight line that go from the summit of the cone $(0.0)$ to $(r, h)$ so this is a different parametrization than for the volume (this gives the possibility to simplify some calculations). Then we have:
	
	and therefore:
	
	Therefore, the total surface of the cone (base + side surface) is:
	
	Finally:
	
	Let us now calculate the moment of inertia of a cone with respect to its axis of revolution:

	For these calculations we will use the moment of inertia of the cylinder $J_{z,\text{cyl}}$ and consider the cone as a stack of infinitesimal cylinders.
	
	Therefore:
	
	
	\pagebreak
	\paragraph{Sphere}\mbox{}\\\\
	\textbf{Definition (\#\mydef):} A "\NewTerm{sphere}\index{sphere}\label{sphere}" is a perfectly round geometrical object in three-dimensional space that is the surface of a completely round ball. Like a circle, which geometrically is a two-dimensional object, a sphere is defined mathematically as the set of points that are all at the same distance $r$ (radius) from a given point, but in three-dimensional space. The longest straight line through the ball, connecting two points of the sphere, passes through the center and its length is thus twice the radius (as for the circle); it is the diameter $\varnothing$ of the sphere.
	\begin{figure}[H]
		\centering
		\includegraphics{img/geometry/sphere_shape.jpg}
		\caption{Sphere}
	\end{figure}
	As the figure above is quite academic, here is a more friendly one if it can help:
	\begin{figure}[H]
		\centering
		\includegraphics[scale=0.3]{img/geometry/sphere_grid.jpg}
		\caption[Sphere Grid]{Sphere Grid (source: Wikipedia)}
	\end{figure}
	Obviously we can see a sphere of radius $R$, as a surface formed by rotating a semicircle around its major axis. The function describing a semicircle being (\SeeChapter{see section Analytical Geometry page \pageref{equation of a circle}}):
	
	The sphere can be dissected as a sum of disk of thickness $\Delta x$. The half-discs being perpendicular to the $x$-axis and of width $\Delta x$ at the position $x_k$  (see figure below).
	\begin{figure}[H]
		\centering
		\includegraphics{img/geometry/sphere_disc_decomposition.jpg}
		\caption{Calculation the volume of a sphere by a stack of discs}
	\end{figure}
	Then we have:
	
	The volume of a disk (cylinder) being given by (passing to the limit):
	
	and the radius $r_k$ being given by the function:
	
	then we have:
	
	By integrating between $x=[-R,R]$, we then have:
	
	We can also make the terminals as $x=[0,R]$ it gives the same to a factor $2$:
	
	Finally:
	
	The expression of the surface being given by differentiation with respect to the element generating the surface, we get (it's a bit limit to say this...)\label{surface of a sphere}:
	
	There is another way, more rigorous, to approach these both calculations (especially the last one...). Indeed, in the section of Differential and Integral Calculus we have introduced the concept of Jacobian that allows changing the coordinate system based on integration variables we are working on (for a detailed definition the reader should refer to section Differential and Integral Calculus page \pageref{jacobian}):
	
	and we have proved (always in the section of Differential and Integral Calculus page \pageref{jacobian spherical coordinates}) that the Jacobian in spherical coordinates was:
	
	So just as $\mathrm{d}x\mathrm{d}y\mathrm{d}z$ is a differential element of volume, we can convert this element into spherical coordinates and make verbatim appear a differential element of volume of the sphere of radius $r$. We then have just to integrate correctly for the size of the whole sphere.

	Therefore we have:
	
	and for the surface (for which the radius is constant):
	
	If necessary, we can find the element of surface geometrically rather than through the Jacobian because the latter is not very educational in small classes...
	
	Then, remembering that in the section of Trigonometry, we proved that the length of an element circular arc is given by:
	
	So, it becomes quite easy to complete the following figure:
	\begin{figure}[H]
		\centering
		\includegraphics{img/geometry/element_of_surface_of_sphere.jpg}
		\caption{Representation of a surface element in spherical coordinates}
	\end{figure}
	and then we see immediately that (most important surface element for us in this book so far for the sections treating of Physics and Engineering!)\label{infinitesimal element of a surface of a sphere}:
	
	what is obviously more funny...

	Let us now calculate the moment of inertia of a homogeneous solid ball of mass $M$ and density $\rho$. For this, the ball having a maximal symmetry, it is more convenient to first calculate the polar moment of inertia (\SeeChapter{see section Classical Mechanics page \pageref{polar moment of inertia}}), and then determine the axial moment of inertia using the first result:
	
	where we used the fact that:
	
	As $J_x,J_y,J_z$ are equally by the symmetry of the ball, it comes\label{inertia momentum ball}:
	
	
	\subparagraph{Sphere packing}\mbox{}\\\\
	In geometry, a sphere packing is an arrangement of non-overlapping spheres within a containing space. The spheres considered are usually all of identical size, and the space is usually three-dimensional Euclidean space. A typical sphere packing problem is to find an arrangement in which the spheres fill as much of the space as possible. The proportion of space filled by the spheres is named the density of the arrangement. 
	
	Many problems in the chemical and physical sciences can be related to sphere packing problems where more than one size of sphere is available. Here there is a choice between separating the spheres into regions of close-packed equal spheres, or combining the multiple sizes of spheres into a compound or interstitial packing (when many sizes of spheres - or a distribution - are available, the problem quickly becomes intractable).
	
	For the very special case of equal spheres in three dimensions, the densest packing uses approximately $74\%$ of the volume. 
	
	\begin{dem}
	Let us see for the two famous case:
	\begin{itemize}
		\item FCC (face cubic center) structure:
		\begin{figure}[H]
			\centering
			\includegraphics{img/geometry/sphere_packing_fcc.jpg}
		\end{figure}
		First we see obviously as the diagonal of the cube is equal to four times a sphere radius ($4R$) that using the Pythagoras theorem:
		
		
		Therefore, denoting $V_s$ the volume of sphere and $V_e$ the volume of the cube:
		
	
		\item HCP (hexagonal close packed):
		\begin{figure}[H]
			\centering
			\includegraphics{img/geometry/sphere_packing_hcp.jpg}
		\end{figure}
		First we see obviously that:
		
		Therefore:
		
	\end{itemize}
	\begin{flushright}
		$\blacksquare$  Q.E.D.
	\end{flushright}
	\end{dem}
	Carl Friedrich Gauss proved in 1831 in a non-trivial and non-formal way that these packings have the highest density amongst all possible lattice packings (we won't provide the proof here as it is quite non-elegant in our personal and subjective point of view).
	
	In 1611 Johannes Kepler conjectured that this is the maximum possible density amongst both regular and irregular arrangements—this became known as the Kepler conjecture. In 1998, Thomas Callister Hales, following the approach suggested by László Fejes Tóth in 1953, announced a proof of the Kepler conjecture (for an good presentation of the proof see \cite{hales2005proof}). Hales' proof is a proof by exhaustion involving checking of many individual cases using complex computer calculations. Referees said that they were "$99\%$ certain" of the correctness of Hales' proof. On 10 August 2014 Hales announced the completion of a formal proof using automated proof checking, removing any doubt.
	
	\begin{tcolorbox}[title=Remark,colframe=black,arc=10pt]
	Some other lattice packings are often found in physical systems. These include the cubic lattice with a density of $\frac{\pi}{6} \cong 0.5236$, the hexagonal lattice with a density of $\frac{\pi}{3 \sqrt{3}} \cong 0.6046$ and the tetrahedral lattice with a density of $\frac{\pi \sqrt{3}}{16} \cong 0.3401$, and loosest possible at a density of $0.0555$.
	\end{tcolorbox}
	
	Finally notice that many people like to compute how many times the Earth can be contained inside Jupiter or Inside the Sun. Dividing the volume for example of the Sun by that of Earth obviously gives a value ($\sim 1,300,000$) that assumes that the Earth can be cut into infinitesimal volumes. But if we consider that we can't cut the Earth into infinitesimal volumes (sphere packing problem), then the number is approximately $\sim 932,884$. Value to compare with $1,300,000\cdot 74.05\%=962,624$.
	
	\pagebreak
	\paragraph{Torus}\mbox{}\\\\
	\textbf{Definition (\#\mydef):} A "\NewTerm{torus}\index{torus}" is the surface generated by rotating a circle $c$ of "\NewTerm{minor radius}\index{torus!minor radius}" $r$ around a straight line at a distance $R$, named "\NewTerm{major radius}\index{torus!major radius}" from it center located in its plane, but not passing through its center. In other words,  it is a surface of revolution generated by revolving a circle in three-dimensional space about an axis coplanar with the circle and that does not touch the circle:
	\begin{figure}[H]
		\centering
		\includegraphics{img/geometry/torus.jpg}
		\caption{Torus}
	\end{figure}
	We see obviously in the figure above that: $a-b=2R$.
	
	To calculate the volume we could obviously use Pappus's centroid theorem as proved in the section of Classical Mechanic. But the latter method is not very elegant and quite "physics" oriented. We will look therefore for a more pure mathematical approach.
	
	Given the equation of a semicircle of center $(0, c)$ (\SeeChapter{see section Analytical Geometry page \pageref{centered offset ellipse}}):
	
	In order to write $y$ as a function of $x$, let us isolate $y$ in this equation:
	
	The circle is then made of the graphs of the following functions:
	\begin{itemize} 
		\item Top semicircle:
		

		\item Bottom semicircle:
		
	\end{itemize}
	Let us denote now by usage with the study of the torus $R:=c$.
	
	The requested volume is the difference between the volumes generated by the rotation of the surfaces (surfaces defined by the area between the function of the circle in question and the axis of abscissa $x=r,x=-r$) in the space around the $x$-axis .

	By applying the integration relation of solids of revolution we get:
	
	Let us calculate the latter integral by the classical substitution $x=r\sin(t)$  thus:
	
	if $x=-r$:
	
	if $x=r$:
	
	Therefore:
	
	Let us linearise that expression again using the trigonometric identities proved in the section Trigonometry (Carnot's formula):
	
	So the volume of a torus of a minor radius $r$ and major radius $R$ is given by:
	
	and the surface (by derivation of the generating surface element... but there are many other elementary way to get this result):
	
	The moment of inertia of the torus in relation to its axis of revolution is calculated as follows:

	First we start using the torus volume by change the notations a little bit:
	
	The volumetric density of the torus is given by (weight to volume):
	
	In cylindrical coordinates, we know that we have:
	
	Therefore:
	
	The moment of inertia is given by:
	
	Let us put $s=r-b$, then:
	\begin{enumerate}
		\item The limits of integration then become $-a$, $+ a$, as we bring all integration points at the origin by putting $s=r-b$.

		\item Trivially, since $r=s+b$ so we have $\mathrm{d}s=\mathrm{d}r$.
	\end{enumerate}
	Which gives:
	
	As we have proved in the section of Differential and Integral Calculus that the integral with two symmetrical terminals of an odd function (product of an even and odd function) is zero, the integrals of:
	
	are equal to zero.

	Then we have to calculate:
	
	Now let us put $s=a\sin(t)$ and therefore $\mathrm{d}s=a\cos(t)\mathrm{d}t$. Therefore it comes:
	
	But as:
	
	Therefore:
	
	Therefore (\SeeChapter{see section Differential and Integral Calculus page \pageref{usual primitives}}):
	
	So finally:
	
	
	\paragraph{Ellipsoid (spheroid)}\mbox{}\\\\
	\textbf{Definitions (\#\mydef):}
	\begin{enumerate}
		\item[D1.] An "\NewTerm{ellipsoid}\index{ellipsoid}" is a surface of the second degree of an three dimensional Euclidean. It is therefore part of quadrics (\SeeChapter{see section Analytical Geometry page \pageref{quadrics}}) given by the equation:
		
		\begin{figure}[H]
			\centering
			\includegraphics{img/geometry/ellipsoid.jpg}
			\caption{Ellipsoid}
		\end{figure}

		\item[D2.] A "\NewTerm{spheroid}\index{spheroid}", or "\NewTerm{ellipsoid of revolution}\index{ellipsoid of revolution}, is a quadric surface obtained by rotating an ellipse about one of its principal axes; in other words, an ellipsoid with two equal semi-diameters.
		
	The equation of a spheroid with $z$ as the symmetry axis is given by setting $a = b$:
	

	If the ellipse is rotated about its major axis, the result is a "\NewTerm{prolate (elongated) spheroid}\index{prolate (elongated) spheroid}", like an American football or rugby ball. If the ellipse is rotated about its minor axis, the result is an "\NewTerm{oblate (flattened) spheroid}\index{oblate (flattened) spheroid}", like a lentil:
		\begin{figure}[H]
			\centering
			\includegraphics{img/geometry/spheroid.jpg}
			\caption[Oblate spheroid on the left, prolate one on the right]{Oblate spheroid on the left, prolate one on the right (source: Wikipedia)}
		\end{figure}
	\end{enumerate}

	To calculate the volume delimited by an ellipsoid, we take the equation that we have determined during our study of conicals:
	
	The section by a plane parallel to the plane $YZ$ and lying at a distance $x$ of the latter, gives the ellipse:
	
	or:
	
	with for semi-axes:
	
	But as we proved it earlier, the surface of an ellipse is equal to $\pi b_1c_2$. Therefore:
	
	The volume of the ellipsoid is therefore equal to:
	
	Therefore:
	
	If $a=b=c=R$ we fall back on the volume of a sphere:
	
	
	Let us now calculate the surface and volume of an oblate spheroid. Remember that its Cartesian equation is:
	
	The ellipticity of an oblate spheroid is defined by:
	
	The surface area of an oblate spheroid can be computed as a surface of revolution about the $z$-axis,
	
	with radius function of $z$ given by:
	
	Therefore:
	
	Therefore:
	
	If we put:
	
	We can then write the integral as following:
	
	And as we have seen in the section of Differential and Integral Calculus and as $-\arcsin(\alpha)=\arcsin(-\alpha)$:
	
	Therefore:
	
	The calculation of the moment of inertia of an ellipsoid is very important in astrophysics since a large number of stars or planets in rotation on themselves by their deformation at the equator because of the centrifugal force are being distorted in a first approximation in this volume.
	
	For an ellipsoid, define $C$ as being the moment of inertia along the $c$-axis, $A$ is the moment of inertia along the $a$-axis, and $B$ the moment of inertia along the $b$-axis.

	To begin, let us consider the moment of inertia along the $c$-axis that we will assimilate to the $z$axis. Thus, in Cartesian coordinates, we have:
	
	By making the following substitution, we implicitly do that the previous integral is a normalization of an ellipsoid:
	
	what gives us for our integral (so we transform the volume $V$ of the ellipsoid in the volume $V'$ of a sphere):
	
	We can now move from Cartesian coordinates to spherical coordinates (\SeeChapter{see section Vector Calculus page \pageref{spherical coordinates}}) without forgetting to use the Jacobian (\SeeChapter{see section Differential and Integral Calculus page \pageref{jacobian}}) that we had proved as being in spherical coordinates given by:
	
	Therefore (we use again the common primitives proved in the section of Differential and Integral Calculus):
	
	By injecting for the ellipsoid:
	
	Then we get:
	
	and by symmetry, we get the following obvious results:
	
	The inertia matrix (\SeeChapter{see section Classical Mechanics page \pageref{inertia matrix}}) is therefore:
	
	
	\paragraph{Paraboloid}\label{paraboloid}\mbox{}\\\\
	\textbf{Definition (\#\mydef):} A "\NewTerm{paraboloid}\index{paraboloid}" is as we already know quadric surface of special kind. There are two kinds of paraboloids: elliptic and hyperbolic.
	
	The elliptic paraboloid is shaped like an oval cup and can have a maximum or minimum point. In a suitable coordinate system with three axes $x$, $y$, and $z$, it can be represented by the equation
	
	where $a$ and $b$ are constants that dictate the level of curvature in the $xz$ and $yz$ planes respectively. This is an elliptic paraboloid which opens upward for $c > 0$ and downward for $c < 0$.
	
	The latter relations is more commonly written as:
	
	
	The hyperbolic paraboloid - not to be confused with a hyperboloid (\SeeChapter{see section Analytical Geometry page \pageref{hyperboloid}}) - is a doubly ruled surface shaped like a saddle. In a suitable coordinate system, a hyperbolic paraboloid can be represented by the equation:
	
	\begin{figure}[H]
		\centering
		\begin{minipage}{.45\linewidth}
		  \includegraphics[width=7cm,height=7cm]{img/geometry/paraboloid_of_revolution.jpg}
		  \captionof{figure}{Paraboloid of revolution}
		  \label{img1}
		\end{minipage}
		\hspace{.05\linewidth}
		\begin{minipage}{.45\linewidth}
		  \includegraphics[width=7cm,height=7cm]{img/geometry/hyperbolic_paraboloid.jpg}
		  \captionof{figure}{Hyperbolic paraboloid}
		  \label{img2}
		\end{minipage}
	\end{figure}
	We will only calculate the volume of the paraboloid of revolution as it is the only result that we need in the other chapters of this book at this date. For this purpose let us consider the following figure:
	\begin{figure}[H]
		\centering
		\includegraphics{img/geometry/paraboloid_of_revolution_volume.jpg}
	\end{figure}
	Given by:
	
	Looking at the figure above we see that the parameters $a$ and $b$ depends  on the value of $z$. So as the surface of an ellipse is given by $\mathrm{d}S=\pi a b$, we have for every height of the paraboloid:
	
	So in the $xz$ plane as we have $y=0$ then:
	
	and also same for the plane $yz$:
	
	And we know that when $z=h$ we must have $a(h)=a$, and when $z=0$ we must have $a(h)=0$ (same with $b$). So we have to write:
	
	Therefore
	
	Hence:
	
	
	\paragraph{Gabriel's horn}\mbox{}\\\\
	"\NewTerm{Gabriel's horn}\index{Gabriel's horn}" (also named "\NewTerm{Torricelli's trumpet}\index{Torricelli's trumpet}") is a geometric figure which has infinite surface area but finite volume. The name refers to the Abrahamic tradition identifying the Archangel Gabriel as the angel who blows the horn to announce Judgement Day, associating the divine, or infinite, with the finite. The properties of this figure were first studied by Italian physicist and mathematician Evangelista Torricelli in the 17th century.
	
	Gabriel's horn is formed by taking the graph of:
	
	with the domain $x\geq 1$ and rotating it in three dimensions about the $x$-axis. With Maple 4.00b we can plot it using (notice that is has the same shape as a gravitational potential or electric potential):
	
	\texttt{>plot3d(1/x, theta=0..2*Pi, x = 1..20, coords=cylindrical}
	\begin{figure}[H]
		\centering
		\includegraphics{img/geometry/gabriel_horn.jpg}
		\caption{Gabriel's horn with Maple 4.00b}
	\end{figure}  	
	The discovery was made using Cavalieri's principle before the invention of calculus, but today calculus can be used to calculate the volume and surface area of the horn between $x = 1$ and $x = a$, where $a > 1$. Using integration (see Solid of revolution and Surface of revolution for details), it is possible to find the volume $V$ (using the method of discs again for the volume) and the surface area $S$:
	
	a can be as large as required, but it can be seen from the equation that the volume of the part of the horn between $x = 1$ and $x = a$ will never exceed $\pi$; however, it will get closer and closer to $\pi$ as a becomes larger. Mathematically, the volume approaches $\pi$ as a approaches infinity. Using the limit notation of calculus:
	
	he surface area formula above gives a lower bound for the area as $2\pi$ times the natural logarithm of $a$. There is no upper bound for the natural logarithm of $a$ as $a$ approaches infinity. That means, in this case, that the horn has an infinite surface area. That is to say:
	
	The apparent paradox formed part of a dispute over the nature of infinity involving many of the key thinkers of the time including Thomas Hobbes, John Wallis and Galileo Galilei.
	
	\begin{tcolorbox}[title=Remark,colframe=black,arc=10pt]
	Since the horn has finite volume but infinite surface area, there is an apparent paradox that the horn could be filled with a finite quantity of paint, and yet that paint would not be sufficient to coat its inner surface. The "paradox" is resolved by realizing that a finite amount of paint can in fact coat an infinite surface area - it simply needs to get thinner at a fast enough rate. 
	\end{tcolorbox}
	The converse phenomenon of Gabriel's horn – a surface of revolution that has a finite surface area but an infinite volume – cannot occur (there is a proof but it is not a very aesthetic one so we will not present it here).
	
	\paragraph{Wine Barrel with Circular Section}\mbox{}\\\\
	Now let us look for fun to a well known volume for winemakers (and not only!):
	\begin{figure}[H]
		\centering
		\includegraphics{img/geometry/wine_barrel.jpg}
		\caption[]{Wine Barrel with circular section}
	\end{figure}
	Let us consider that the lateral curvature of the barrel is a parabola:
	
	Let us put:
	
	given the way we have put the $x, y$-axis it is relatively easy to determine the coefficients of the polynomial. The coefficient $c$ is the simplest:
	
	We also have:
	
	Also:
	
	Therefore we have:
	
	The radius of a horizontal section of ordinate $x$ is $r=y$ and its surface is therefore:
	
	or:
	
	Let us develop this:
	
	The volume of liquid for a height $h$ will then be:
	
	To calculate the inner surface of the wine barrel, we consider the outer curve given by a parabolic arc as shown below:
	\begin{figure}[H]
		\centering
		\includegraphics{img/geometry/wine_barrel_vertical_section.jpg}
		\caption[]{Vertical section of the wine barell}
	\end{figure}
	To calculate the lateral area of the barrel, we must first determine the expression of the above parabola.
	
	Looking at the figure, we get:
	
	which is a system of three equations with the unknowns $a,b,c$.
	
	After resolution, we get:
	
	The side surface of the barrel including the surface of the two disks at the ends is then given by:
	
	By doing the change of variable $u=2ax+b$ we get:
	
	The latter integral can be calculated using the following relations (the second has been proved in the section of Differential and Integral Calculus the first one need to be detailed when we will have the time...):
	
	Where for recall (\SeeChapter{see section Trigonometry page \pageref{hyperbolic trigonometry}}):
	
	We will not go further because the resulting formula would be huge and not very interesting in our point of view (but on request we will do it).	
	
	Nevertheless, here is a numerical application. Suppose $r=20$ [cm], $R=30$ [cm], $H=60$ [cm].
	
	We calculate:
	
	Therefore:
	
	The first integral is equal to:
	
	The second integral is equal to:
	
	So finally we get:
	
	
	
	\begin{flushright}
	\begin{tabular}{l c}
	\circled{80} & \pbox{20cm}{\score{4}{5} \\ {\tiny 171 votes,  74.86\%}} 
	\end{tabular} 
	\end{flushright}
	
	\pagebreak
	%to force start on odd page
	\newpage
	\thispagestyle{empty}
	\mbox{}
	\section{Graph Theory}\label{graph theory}

\lettrine[lines=4]{\color{BrickRed}T}he history of graph theory (or also named "complex cellulars") may have start with the work of Euler in the 18th century and has its origins in the study of certain problems, such as the bridges of Königsberg (the inhabitants of Königsberg wondered if it was possible, starting from any part of the city, to cross all the bridges without passing two times by the same bridge and return to their starting point), the walk of the rider on the chessboard cards or the colouring problem and the shortest path between two points.

Graph theory was then developed in various disciplines such as chemistry (isomers), biology, social sciences (transport networks, social networks), project management (critical path method), IT (network topology, computational complexity, protocols transfers), quantum physics, spectral clustering, etc. Since the early 20th century, it is a full branch of mathematics, thanks to the works of König, Menger, Cayley and then Berge and Erdös. This branch of mathematics has get a great resurgence of interest following the emergence of Internet social networks whose connections between "friends" and "followers" are graphs whose topological and statistical analysis can teach us many things.

In general, a graph is used to represent the structure, the connections of a complex set by expressing the links between its components: communication network, road networks, interaction of various animal species, electrical networks, etc.

The graphs are therefore a thought method to model a wide variety of problems by reducing them to the study of vertices and edges.

Recent work in graph theory are often done by computer, because of the importance of the algorithmic aspect in their field (see the beginning of the section of Theoretical Computing some small examples).

Indeed, the purpose is essentially to modelize problems. We express the problem in terms of graphs so that it reports to a problem of graph theory that we know usually how to resolve because falling within a class of known problems.

The solutions to graph problems can be easy and effective (the time required to process computationally being reasonable because they depend polynomially on the number of vertices of the graph) or difficult (the processing time is then exponential) in which case we use a heuristic, that is to say a search process for a solution (not necessarily the best).

	If graph theory knows a quite big interests since the years 1980, maybe is it because it does not require in its elementary concepts considerable mathematical background. Actually, just going through the sections of Probability, Set Theory,  Linear Algebra and Topology presented in this book  to already feel comfortable with the different definitions.

	We will introduce the basic vocabulary of the graph theory. The terms used are those of the common language of Euclidean geometry (and unfortunately they are also in large numbers...).

	\pagebreak
	\subsection{Type of Graphs and Structures}
	\textbf{Definitions (\#\mydef):}
	\begin{enumerate}
	\item[D1.] A "\NewTerm{graph}\index{graph}" (or "\NewTerm{polygraph}\index{polygraph}") $G$ is a pair $G(X,E)$ consisting of a non-empty and finite set $X$ (the vertices/nodes), and a set $E$ (the edges/links) of $X$ of ordered pairs elements of $X$ connected by a line segment or said in another way (...) a part of the Cartesian product $X^2$ (\SeeChapter{see section Set Theory page \pageref{cartesian product}}).
	
	\begin{figure}[H]
		\centering
		\includegraphics[scale=0.75]{img/geometry/graph_vocabular.eps}
	\end{figure}
	
	\begin{tcolorbox}[title=Remark,colframe=black,arc=10pt]
	A graph is often denoted in English $G = (E, V)$ where $E$ is the first letter of the word "edges" and $V$ the first letter of the word "vertices".
	\end{tcolorbox}	
	
	\item[D2.] Let $(a_1, b_1)$ and $(a_2, b_2)$ be pairs. Then the characteristic (or defining) property of the "\NewTerm{ordered pair}\index{ordered pair}" is:
		
		 The order in which the objects appear in the pair is significant: the ordered pair $(a, b)$ is different from the ordered pair $(b, a)$ unless $a = b$. 
		 
		In graph theory it is useful to precise if a graph $G$ is made of ordered pairs in this way we can work if necessary with edges orientation to distinguish edges sharing the same pairs of vertices but having not the same orientation path.
	
	\item[D3.] A "\NewTerm{multigraph}\index{multigraph}" is a graph $G$ which is permitted to have edges sharing the same vertices and having edges looping on the same vertices.
	
	\begin{figure}[H]
	\centering
	\includegraphics[scale=0.75]{img/geometry/multigraph.eps}
	\caption{Example of a multigraph that we will be used much later}
	\end{figure}
	
	\begin{tcolorbox}[title=Remark,colframe=black,arc=10pt]
This a very common case in Markov chains analysis (see sections on Probabilities page \pageref{markov chains} or Games and Decision Theory page \pageref{markov decision process}) and as we will see further below the Königsberg bride problem is a multigraph!
	\end{tcolorbox}	 

	\item[D4.] The elements of $X$ are the "\NewTerm{vertices}\index{vertices}" of the graph $G$, those of $E$ are the "\NewTerm{edges}\index{edges}" of the graph $G$ (indeed an edge - link - is composed of two vertices joined by a line segment, hence the reference to pairs of elements in the previous definition). The set of vertices of a graph $G$ is denoted by $G (X)$ and the set of edges $G (E)$.
	
	\item[D5.] The plane where a graph is immersed is a "\NewTerm{face $F$}\index{face}" and each close surface (area) on this plane defined by a close path of edges is also a face $F$. 

	\item[D6.] A graph is named "\NewTerm{planar graph}\index{planar graph}" when we can represent it in a plane without intersecting edges.
	
	The following are some examples of (connected) planar graphs:
	\begin{figure}[H]
		\centering
		\includegraphics{img/geometry/planar_graphs.jpg}
		\caption{Example of planar graphs}
	\end{figure}

\begin{theorem}
	Now let us show that if $F$ is the number of faces of a planar graph, $A$ the number of edges and $S$ the number of nodes, then we have using a old traditional notation:
	
which is the relation known under the name "\NewTerm{Euler's formula}\index{Euler's formula}" or "\NewTerm{Descartes-Euler's theorem}\index{Descartes-Euler's theorem}" (proof after the example below) and which will be useful many times in this book (in this section and during our study of polyhedra in the section of Geometrical Shapes). So keep in mind that any graph that doesn't respect this relation cannot be a planar graph!

	\begin{tcolorbox}[colframe=black,colback=white,sharp corners]
	\textbf{{\Large \ding{45}}Example:}\\\\
	A graph with 2 faces (the light gray side is an infinite outside face), 4 vertices and 4 nodes:
	\begin{figure}[H]
		\centering
		\includegraphics[scale=0.4]{img/geometry/euler_theorem.eps}
		\caption{Graph with 2 faces, 4 vertices and 4 nodes}
	\end{figure}
	\end{tcolorbox}
\end{theorem}
\begin{dem}
We prove this theorem by performing a recurrence (\SeeChapter{see section Proof Theory page \pageref{proof by recurrence}}) on the difference $A - S$ (this is the trick!):

First it is obvious that the theorem is true for:
	
because in this case the graph is a tree so it only has one face (the outer infinite face), thus $F=1$.
\begin{figure}[H]
\centering
\includegraphics[scale=0.4]{img/geometry/tree_one_edge_two_vertices.eps}
\caption{Tree with one edge and two vertices}
\end{figure}
Therefore:
	
Then take a connected graph (see definition further below) containing at least one cycle $G$ (figure below is an example of a graph with $3$ cycles that is to say the two small insiders and the global one):
\begin{figure}[H]
\centering
\includegraphics[scale=0.5]{img/geometry/graph_with_at_least_one_cycle.eps}
\caption{Graph with at least one cycle}
\end{figure}

If we remove one edge $e$ to this cycle, then we should be able recursively to apply the same relation to the graph $G-\left\lbrace e \right\rbrace$ if it is right. Indeed, the graph without the edge $e$ will have $F$ faces, $S$ vertices and $A$ edges and thus we can use the relation prove above with the simple three (that had no cycles):
	
if we give put the edge back on place then we write:
	
	
	\begin{flushright}
		$\blacksquare$  Q.E.D.
	\end{flushright}
\end{dem}
	Regular polygons (\SeeChapter{see section Geometric Shapes page \pageref{regular polygon}}) with their diagonals drawn in can be viewed as graphs (with a vertex at each intersection):
	\begin{figure}[H]
		\centering
			\includegraphics{img/geometry/planar_graphs_regular_polygons.jpg}
		\caption{Regular polygons seen as graphs}
	\end{figure}
	 If we make every intersection point a vertex, then these pictures automatically become connected, planar graphs! If we know the number of intersection points and the number of segments, we can then use Euler's formula to find the number of regions (for example, drawing the two diagonals of a square results in $5$ intersections, $4$ regions, and $8$ segments. The $5$ diagonals of a regular pentagon make $10$ intersections, $11$ regions, and $20$ segments). 
	 
	 \begin{tcolorbox}[colframe=black,colback=white,sharp corners]
	\textbf{{\Large \ding{45}}Example:}\\\\
	Let us consider $K_5$:
	\begin{figure}[H]
		\centering
		\includegraphics[scale=0.4]{img/geometry/euler_formula_k5.jpg}
	\end{figure}
	We take the Euler formula:
	
	$K_5$ has $5$ vertices and $10$ edges, so we get:
	
	which says that if the graph is drawn without any edges crossing, there would be $S=7$ faces. And there is no way to find a way to do that with $K_5$.
	\end{tcolorbox}
	 
	\item[D7.] Given $e={x,y}$ an edge of the graph $G$, we say that the vertices $x, y$ which are the "ends" of the edge of $G$ are "\NewTerm{adjacent}\index{adjacent}" or "\NewTerm{neighbours}\index{neighbours}" in the graph $G$ and that the edge $e$ is "\NewTerm{incident}\index{incident}" to the vertices $x, y$.
	
	\item[D8.] If two edges $e$ and $e'$ have one end in common, we say they are "\NewTerm{incidental edges}\index{incidental edges}" otherwise that they are "\NewTerm{independent edges}\index{independent edges}".
	
	\begin{tcolorbox}[title=Remark,colframe=black,arc=10pt]
If $e$ is an edge of a graph $G$, we denote by $G-e$ the subgraph (see definition just below) $G'=(X,E-\left\lbrace e \right\rbrace)$. If $X'$ is a subset of $X$, we denote by $G-X'$ the graph $G$ without the vertices $X'$.
	\end{tcolorbox}
	\item[D9.] What we name "\NewTerm{order of a graph $\mathcal{O}$}\index{order of a graph}" is the number of its vertices (nodes).
	
	\item[D10.] A "\NewTerm{complete graph}\label{complete graph}" or "\NewTerm{strongly connex graph}\index{strongly connex graph}" is a simple graph in which every pair of distinct vertices is connected by a unique edge.
	\begin{figure}[H]
		\centering
		\begin{subfigure}{0.4\textwidth}
			\includegraphics[width=180pt,height=180pt]{img/geometry/euler_theorem.eps}
		\end{subfigure}
		\begin{subfigure}{0.4\textwidth}
			\includegraphics[width=180pt,height=180pt]{img/geometry/graph_with_at_least_one_cycle.eps}
		\end{subfigure}		
		\caption{Difference between a non-complete graph (left) and complete graph (right)}		
	\end{figure}
	
	\item[D11.] What we name "\NewTerm{size of a graph}\index{size of a graph}" of the graph is the number of its edges.
	
	Let $G$ be a complete graph (and not a multigraph!) of order $n$, the set $E$ of edges must be by definition chosen as a subset of all pairs of elements of the set $X$, therefore it's a set of cardinal:
	
	This is a relatively trivial result since each vertices is linked to all the other vertices except himself (hence the numerator) and we divide by two simply to not count the neighbours vertices twice (and they all are neighbours when we go through them all in a complete graph!).
	
	More explicitly (if needed) $n(n-1)/2$ comes from simple counting argument. Label the vertices $1,2,\ldots,n$. The first vertex is now joined to $n-1$ other vertices. The second vertex has already been joined to vertex $1$ and hence has to be joined to the remaining $n-2$ vertices and in general the $k$th vertex has already been joined to the previous $k-1$ vertices and hence has to be joined to the remaining $n-k$ vertices. So the total number of edges is given by:
	
	This result is illustrated by some examples further below.
	
	Consequently, there are (see section Probabilities on the arrangements  of $n$ indistinguishable elements in pairs of two page \pageref{simple arrangements with repetitions}):
	
	possible choices for $E$ and so many graphs admitting $X$ as set of vertices. 
	\begin{tcolorbox}[colframe=black,colback=white,sharp corners]
	\textbf{{\Large \ding{45}}Example:}\\\\
	We take a triangular graph - with three points - as example where we have $n(n-1)/2=3$ and therefore $2^{\frac{n(n-1)}{2}}=8$:
	\begin{figure}[H]
		\centering
		\includegraphics{img/geometry/subset_graphs.jpg}
		\caption{Example of graphs subsets}
	\end{figure}
	\end{tcolorbox}
	
	 Some of these graphs are, as we consider their vertices as indistinguishable, "\NewTerm{automorphic}\index{automorphic}" (see definition of this term a little bit late below in this section).
	
	This result means that there are about $2$ million of graphs for to $7$ vertices, graphs and almost $4\cdot 10^{105}$ for graphs with 27 vertices - a number to be compare with the fact that we estimate at $10^{100}$ the number of atoms in the universe...
	
	\item[D12.] The "\NewTerm{neighbourhood}\index{neighbourhood of a vertices}" of a vertices is the set $V$ of all its neighbours.
	
	\item[D13.]  We name "\NewTerm{degree}\index{degree of a vertices}" of a vertices and we denote by $D$ the number of its neighbours (the cardinal of $V$), which is also the number of edges that are incident to it (a zero degrees vertices being named an "\NewTerm{isolated vertices}\index{isolated vertices}" and a degree $1$ vertices is named "\NewTerm{pending vertices}\index{pending vertices}").
	
	Properties (without proof):
	\begin{enumerate}
		\item[P1.] The sum of the degrees of the vertices is equal to twice the number of edges.
		
		
		\item[P2.] In a graph such that  $\mathcal{O}>3$, the number of vertices with odd degrees is always even (handshaking lemma).
		
		Indeed, if the sum of all the degrees is equal to twice the number of edges. Since the sum of the degrees is even and the sum of the degrees of vertices with even degree is even, the sum of the degrees of vertices with odd degree must be even. If the sum of the degrees of vertices with odd degree is even, there must be an even number of those vertices.
	\end{enumerate}
	\begin{tcolorbox}[title=Remark,colframe=black,arc=10pt]
	A "\NewTerm{regular graph}\index{regular graph}" is a graph in which all vertices have same degree $k$. We say then that the graph is "\NewTerm{$k$-regular}\index{$k$-regular graph}".
	\end{tcolorbox}
	
	\item[D14.] We say that a graph $G'=(X',E')$ is a "\NewTerm{subgraph}\index{subgraph}" of a graph $G=(X,E)$ when $X' \subseteq X$ and $E' \subseteq E$. We say that $G'=(X',E')$ is an "\NewTerm{induced subgraph}\index{induced subgraph}" of a graph $G=(X,E)$ provided $X'\subseteq X$ and $E'$ contains all edges of $E$ which are subsets of $X'$.
	
	If $G'$ is a subgraph of $G$ and $u$ and $w$ are vertices of $G'$, then by the definition of a subgraph, $u$ and $w$ are also vertices of $G$. However, if $u$ and $w$ are adjacent in $G$ (i.e., there is an edge of $G$ joining them), the definition of subgraph does not require that the edge joining them in $G$ is also an edge of $G'$. If the subgraph $g'$ has the property that whenever two of its vertices are joined by an edge in $G$, this edge is also in $G'$, then we say that $G'$ is an induced subgraph. 
	
	Notice that every induced subgraph is also an ordinary subgraph, but not conversely. Think of a subgraph as the result of deleting some vertices and edges from the larger graph. For the subgraph to be an induced subgraph, we can still delete vertices, but now we only delete those edges that included the deleted vertices.
	
	\begin{tcolorbox}[colframe=black,colback=white,sharp corners]
	\textbf{{\Large \ding{45}}Example:}\\\\
	Consider the graphs:
	\begin{figure}[H]
		\centering
		\includegraphics[scale=0.7]{img/geometry/subset_graph_induced_graph.jpg}
		\caption{Example of subgraphs and induced graphs}
	\end{figure}
	Here both $G_2(X_2,E_2)$ and $G_3(X_2,E_3)$ are subgraphs of $G_1(X_1,E_1)$. But only $G_2$ is an induced subgraph. Every edge in $G_1$ that connects vertices in $G_2$ is also an edge in $G_2$. In $G_3$, the edge $\{a,b\}$ is in $E_1$ but not $E_3$, even though vertices $a$ and $b$ are in $X_3$.\\
	
	The graph $G_4(X_4,E_4)$ is not a subgraph of $G_1$, even though it looks like all we did is remove vertex $e$. The reason is that in $E_4$ we have the edge $\{c,f\}$ but this is not an element of $E_1$, so we don't have the required $E_4\subseteq E_1$.
	\end{tcolorbox}
	
	\item[D15.] A "\NewTerm{covering subgraph}\index{covering}" of a graph $G=(X,E)$ is a subgraph $G'=(X,E')$, that is to say a subgraph whose vertices are all vertices of $G$ and whose edges are in $E'$.
	
	\begin{figure}[H]
		\centering
		\includegraphics{img/geometry/covering_graph.jpg}
		\caption[]{In the following figure, the graph on the right is a covering graph of the graph on the left}
	\end{figure}
	
	\item[D16.] For a graph of order $n$, there are two extreme cases for all its edges: either the graph has no edges or all possible edges connecting the vertices by pairs are present. In the latter case the graph is said named a "\NewTerm{complete graph}\index{complete graph}".
	\begin{tcolorbox}[colframe=black,colback=white,sharp corners]
	\textbf{{\Large \ding{45}}Example:}\\\\
	Here are some complete graphs for which we have well:
	
	edges. We notice that the first four graphs are planar (indeed you can see how it is possible, by projecting a vertices in the plane, to transform the fourth $K_4$ in $K_4^{\prime}$ i a way so that there are no more intersections). The fifth graph $K_5$ is non-planar (we can not find movements avoiding crossings):
	\begin{figure}[H]
		\centering
		\includegraphics{img/geometry/complete_graphs.jpg}
		\caption{Example of complete graphs}
	\end{figure}
	A complete graph is therefore a graph where each vertex is connected to every other. The complete graph of order $n$ is denoted by $K_n$. In this type of graph each vertex is of degree $n-1$.
	\end{tcolorbox}
	
	\pagebreak
	\begin{tcolorbox}[colframe=black,colback=white,sharp corners]
	So a nice case to be processed is the "Star of David":
	\begin{figure}[H]
		\centering
		\includegraphics{img/geometry/complete_graph_david_star.jpg}
		\caption{Star of David}
	\end{figure}
	which is obviously a complete graph by definition if and only if we join all vertices between them (and we lose the geometry of the star but we get a graph $K_6$):
	\begin{figure}[H]
		\centering
		\includegraphics{img/geometry/complete_graph_david_star_corrected.jpg}
		\caption{Star of David completed}
	\end{figure}
	\end{tcolorbox}
	\begin{tcolorbox}[title=Remark,colframe=black,arc=10pt]
	This result is interesting in the field of management of the communication in business projects and IT security (see for this latter the section Cryptography). For example, if you manage a project with $10$ stakeholders (corresponding to $n$), so there are $n(n-1) / 2$ possible communication channels (email or phone) possible, that is to says $45$ (and in the field of IT security there would be $45$ encryption symmetric keys system to generate). Hence the importance in project management to establish clear communication rules (in the form of a graph) if one does not want to be flooded by emails or phones unnecessarily (and in the field of security to implement asymmetric systems). We will also encounter this results in the liquid drop model of the nucleus in the Nuclear Physics section of the chapter Atomistic.
	\end{tcolorbox}
	
	\item[D17.] A "\NewTerm{stable graph}\index{stable graph}" or "\NewTerm{independent set}\index{independent graph}" is subgraph without edges and a "\NewTerm{clique}\index{clique}" a complete subgraph.
	
	In other words an independent vertex set of a graph $G$ is a subset of the vertices such that no two vertices in the subset represent an edge of $G$. 
	\begin{figure}[H]
		\centering
		\includegraphics{img/geometry/stable_graph.jpg}
		\caption{Stable subgraph example (in blue)}
	\end{figure}
	\begin{tcolorbox}[title=Remark,colframe=black,arc=10pt]
	A "\NewTerm{maximum independent}\index{maximum independent}" set is an independent set of largest possible size for a given graph $G$. The problem of finding such a set is an NP-hard optimization problem (\SeeChapter{see section Numerical Methods page \pageref{np completude}}). As such, it is unlikely that there exists an efficient algorithm for finding a maximum independent set of a graph.
	\end{tcolorbox}

	\item[D18.] In a graph, it is natural to want to move from summit to summit following the edges. Such a walk through $n$ vertices is named a "\NewTerm{string $P_n$}\index{string (graph theory)}"  or "\NewTerm{path}\index{path}":
	
	A path is a list $P_k=(x_1,...,x_k)$ of vertices such that there exists an edge in the graph between every pair of successive vertices: $\forall i=1,...,k-1\; (x_i,x_{i+1})\in E$. The path length corresponds to the number of edges traversed: $k-1$.
	
	A path is named "\NewTerm{simple path}\index{simple path}" if each edge of the path is taken only once. Here for example a simple way with $5$ vertices:
	\begin{figure}[H]
		\centering
		\includegraphics{img/geometry/simple_path.jpg}
		\caption{Simple path example}
	\end{figure}
	Thus, we also define a "\NewTerm{cycle}\index{cycle}" as:
	
	as being a simple path ending at the starting point as $x_1=x_{k++}$. 
	
	\item[D19.] A "\NewTerm{simple cycle}\index{simple cycle}" is a cycle in which all the edges are different.
	
	\item[D20.] A "\NewTerm{directed graph}\index{directed graph}" or "\NewTerm{oriented graph}\index{oriented graph}\label{oriented graph}" is a graph whose edges have a direction and re therefore named "arcs" (thus opposite to an undirected graph).
	
	\begin{tcolorbox}[title=Remarks,colframe=black,arc=10pt]
	\textbf{R1.} The terms "path" and "circuit" are normally reserved to directed graphs. For undirected graphs we use mainly the words "string" or "cycle". However, the formal definition is exactly the same in both cases, only changes the structure (directed graph or not) on which they are defined.\\
	
	\textbf{R2.} An undirected graph is a symmetrical directed graph. Indeed, if an arc connects the vertice $a$ to the vertice $b$ top and another arc the vertice $b$ to the vertice $a$, then it is of usage to trace only a simple line between $a$ and $b$ which we call ... an edge.
	\end{tcolorbox}
	
	 \item[D21.] A path $P_k=(x_1,...,x_k)$ is named "\NewTerm{elementary path}\index{elementary path}" if each of the vertices of the path is visited only once: $\forall i,j=1,...,k \quad i\neq j,x_i \neq x_j$. An elementary path is therefore a simple path without cycle.
	 
	 In a graph of order $G$ we have the following properties:
	 \begin{enumerate}
	 	\item[P1.] All elementary path is of length at most $n-1$. Indeed, an elementary path visiting at most 1 time each vertex of the graph, its length (number of edges) may actually not exceed $n-1$.
	 	
	 	\item[P2.] The number of elementary paths in a graph is finite. Indeed, the number of paths of length $k(k=0,1,...,n-1)$ is at most the combinatorial of a sequence of $k + 1$ peaks distinguishable from $n$. There are therefore (\SeeChapter{see section Probabilities page \pageref{simple arrangements without repetitions}}):
	 	
		possible paths.
		
		The elementary paths are the natural restrictions we are looking at the concept of path. The question is whether we lose something by considering only the basic paths in a graph: can we always replace a path in a graph by an elementary path?
	 \end{enumerate}
	 \begin{lemma}
	 The "\NewTerm{König's lemma}\index{König's lemma}" answers this question affirmatively: from any path we can extract an elementary sub-path.
	 
	 In other words: If there is a path between two vertices $x$ and $y$, then there exists an elementary path between $x$ and $y$.
	 \end{lemma}
	 \begin{dem}
	 The idea of the proof is to choose a particular path between $x$ and $y$ and show that it is elementary. Which path to choose? If a path has a circuit, this circuit is a detour on the road of $x$ and $y$. A good candidate to be an elementary path appears to be a shortest path.
	 
	 Among all the paths connecting $x$ to $y$, let us choose a path:
	 
	 with the fewest possible edges. Suppose by contradiction that $P_k$ is not elementary. There exists in this path a vertices $z$ appearing at least two times along the path $P_k$.
	 
	 Given $i, j$ the first two indices such as $x_i=z$ and $x_j=z$:
	  
	 To get contradiction, we just delete the cycle between $x_i=z$ and $x_j=z$. So we have a new path:
	 
	 which is a path connecting $x$ to $y$. Its length strictly less than $P_k$: 
	  
	 which contradicts our initial choice as a shortest path.
	 \begin{flushright}
		$\blacksquare$  Q.E.D.
	\end{flushright}
	 \end{dem}
	 
	 \item[D22.] A graph $G$ is named "\NewTerm{connected graph}\index{connected graph}" or "\NewTerm{connex graph}\index{connex graph}" if and only if there is at least one simple path between each pair of vertices (the path being therefore not necessarily direct- can pass through one or more intermediate vertices).
	 
	 Therefore, we have that a "\NewTerm{connected component}\index{connect component}" $G'=(X',E')$ is a subgraph of $G$ in which any two vertices are connected to each other by paths, and which is connected to no additional vertices in the supergraph.
	 \begin{figure}[H]
		\centering
		\includegraphics[scale=0.75]{img/geometry/sug_connected_component.jpg}
		\caption[A graph with three connected components]{A graph with three connected components (source: Wikipedia)}
	\end{figure}
	\begin{tcolorbox}[title=Remarks,colframe=black,arc=10pt]
	\textbf{R1.} A graph with only one connected component is simply a... connected graph. \\
	
	\textbf{R2.} An isolated node (of degree $0$) always constitutes a connected component alone.\\\
	
	\textbf{R3.} The relation on the nodes "there is a path between ..." is an equivalence relation (reflexive, symmetric and transitive). The connected components of a graph correspond to equivalence classes of this relation.
	\end{tcolorbox}
	We will suppose intuitive that a graph $G$ of order $n$ is always associated with at least $n-1$ edges.
	
	\item[D23.] A "\NewTerm{tree}\index{tree}" or "\NewTerm{spanning tree}\index{spanning tree}" is a connected graph (undirected) with no cycles (acyclic). In a tree the number of edges is equal to the number of vertices $- 1$:
	 \begin{figure}[H]
		\centering
		\includegraphics{img/geometry/spanning_tree.jpg}
		\caption{Example of a spanning tree}
	\end{figure}
	 A minimum spanning tree with the minimum weight values of a valuated graph where all nodes are browsed is named a "\NewTerm{minimum spanning tree}\index{minimum spanning tree}":
	 \begin{figure}[H]
		\centering
		\includegraphics[scale=0.5]{img/geometry/minimum_spanning_tree.jpg}
		\caption[Example of a minimum spanning tree]{Example of a minimum spanning tree (source: Wikipedia)}
	\end{figure}
	If you imagine that every point is a city, and that a country has not enough money for now for the maintenance of all existing routes between several cities, the minimum spanning tree gives mathematically (without any other considerations) the road to clean and repair to minimize costs while connecting all the towns together.
	
	\item[D24.] A "\NewTerm{valuated tree}\index{valuated tree}" or "\NewTerm{weighted graph}\index{weighted graph}" is a tree (respectively a graph) where the edges have positive values (weights). The sum of all the values that are on the browsed edges is then named the "\NewTerm{cost tree Value}\index{cost tree value}" (respectively "\NewTerm{cost graph value}\index{cost graph value}").
	
	\begin{tcolorbox}[title=Remark,colframe=black,arc=10pt]
	The valued trees are used in many fields. These include computer networks in which we seek to optimize the number of interconnections between machines to avoid duplication of items of data packets or project management (see example below).
	\end{tcolorbox}	
	\begin{tcolorbox}[colframe=black,colback=white,sharp corners]
	\textbf{{\Large \ding{45}}Example:}\\\\
		An excellent practical example of a connected weighted and directed graph and directed (abbreviated to as "\NewTerm{digraph}\index{digraph}") is the one used in project management to calculate the critical path (of a project without constraints). This is a graph representing the dependencies between $n$ intermediate tasks required to complete a project, commonly named "\NewTerm{Gantt chart}\index{Gantt chart}\label{gantt chart}" or even "\NewTerm{scheduling graph}\index{scheduling graph}". The length (weight) of each task is the value of the externally incident weights to the correspondent node arcs. Edges represent the sequences of tasks. We are always adding an initial and final node (in the world of project management people rather talking about "milestone"...). The first node is connected by a zero value arc to all nodes without predecessors, and all nodes without successors are connected to the final node. The resulting graph should obviously be acyclic.\\
		
		A "\NewTerm{critical path}\index{critical path}\label{critical path}" is a path of maximum length between the two extreme nodes (milestones). It may have possibly be more than one of the same length. Any task located on a critical path can not be delayed without affecting the total duration of a project. In other words, it has zero "\NewTerm{total slack}\index{total slack}" (we then also say so that its early start finish date is strictly equal to late finish date). Furthermore, we also define in project management the "\NewTerm{free slack}\index{free slack}" that indicates the time over which a task can slide without moving/impacting the successor task. The free slack is calculated as the difference between the date of earliest start of a task summed of its duration and the late start of the successor task. By construction the free slack is always small or equal tot the total slack.\\
		
		Let us take for example a project that consists of the following tasks:
	\begin{table}[H]
	\begin{center}
		\begin{tabular}{|c|c|c|}
		  \hline
		  \rowcolor[gray]{0.75}Tasks & Predecessor Tasks & Duration \\ \hline
		  A & E & $3$ \\ \hline
		  B & K,C & $4$ \\ \hline
		  C & - & $3$ \\ \hline
		  E & E,J & $2$ \\ \hline
		  E & - & $2$ \\ \hline
		  F & G,L & $3$ \\ \hline
		  G & - & $4$ \\ \hline
		  H & A, M, R & $2$ \\ \hline
		  J & E & $2$ \\ \hline
		  K & C & $2$ \\ \hline
		  L & G & $5$ \\ \hline
		  M & C & $4$ \\ \hline
		  N & G & $3$ \\ \hline
		  R & J & $2$ \\ \hline
		\end{tabular}
	\end{center}
	\caption[]{Data for critical path example}
	\end{table}			
	The valued associated directed graph corresponding to this table once the definition of the critical path applied is the following using the start dates:	
	\end{tcolorbox}
	
	\pagebreak
	\begin{tcolorbox}[colframe=black,colback=white,sharp corners]
	\begin{figure}[H]
		\centering
		\includegraphics[scale=0.8]{img/geometry/pert_diagram.jpg}
		\caption{PERT (Program Evaluation Review Technique) diagram}
	\end{figure}
	We see well in this graph that the finite set of tasks $\left\lbrace \text{Start}, \text{G}, \text{L}, \text{Fin}\right\rbrace$ are critical.\\
	
	A great tool for using such graphs is (among others) Microsoft Project whose the corresponding Gantt diagram for the example above is:
	\begin{figure}[H]
		\centering
		\includegraphics[scale=0.8]{img/geometry/gantt_example.jpg}
		\caption{Same graph but seen in Microsoft Project 97}
	\end{figure}
	\end{tcolorbox}
	
	\item[D25.] A "\NewTerm{cycle}\index{cycle}" is a simple path looping on itself. A graph in which there is no cycles is known as "\NewTerm{acyclic}\index{acyclic}"\label{acyclic graph}.
	
	Non-connex acyclic graphs made of trees are an interesting class of graphs with remarkable properties and a name: "\NewTerm{forests}\index{forests}" (a term often used in computer networks).
	\begin{figure}[H]
		\centering
		\includegraphics{img/geometry/graph_connex_cyclic.jpg}
		\caption{Example of a connex graph containing a cycle}
	\end{figure}
	\begin{figure}[H]
		\centering
		\includegraphics{img/geometry/graph_connex_acyclic.jpg}
		\caption{Example of a connex graph containing non cycles}
	\end{figure}
	Below an example of forest (composed of trees, each tree being connex but the whole forming an acyclic and non-connex graph):
	\begin{figure}[H]
		\centering
		\includegraphics{img/geometry/graph_forest.jpg}
		\caption{Example of a forest (3 connex graphs)}
	\end{figure}
	Thus we see that a forest is a graph whose components are trees. The vertices of degree $1$ are named "\NewTerm{leaves}\index{leaves}" of the tree.
	
	Properties:
	\begin{enumerate}
		\item[P1.] If in a graph $G$ is any vertex is of degree greater than or equal to $2$, then $G$ has at least one cycle (trivial).
		\begin{tcolorbox}[title=Remark,colframe=black,arc=10pt]
		This simple property implies that a graph without cycle without graph has at least one vertex of degree $0$ or $1$.
	\end{tcolorbox}	
		
		\item[P2.] An acyclic graph $G$ with $n$ vertices has at most $n-1$ edges as we already know.
	\end{enumerate}

	\item[D26.] An "\NewTerm{Eulerian cycle}\index{Eulerian cycle}" is a cycle passing once and only once by each \underline{edge} of the graph and returning to the starting point (we will see further properties required by a graph for such a cycle exists on it) and a graph having an eulerian cycle is named "\NewTerm{Eulerian graph}\index{Eulerian graph}".
	
	\item[D27.] A "\NewTerm{Hamiltonian cycle}\index{Hamiltonian cycle}\label{hamiltonian cycle}" is a simple cycle through all the \underline{vertices} of the graph once and only once and a graph having an eulerian cycle is named "\NewTerm{Hamiltonian graph}\index{Hamiltonian graph}". For a Hamiltonian cycle, the graph must be connex and must not have any single node.
	
	To better understand the difference between the two previous definitions the reader can see in the figure below some graphs that are Eulerian and Hamiltonian, Eulerian but not Hamiltonian and so on... (the four possibilities):
	\begin{figure}[H]
		\centering
		\includegraphics[scale=0.7]{img/geometry/euler_vs_hamilton.jpg}
		\caption{Example of Hamiltonian or Non-Hamiltonian graphs}
	\end{figure}
	It is appropriate now to open a parenthesis (for straws ...) on one of the most famous problem in graph theory: Königsberg bridges.
	
	Leonhard Euler, liked to take a walk in his city of Königsberg. Following the legend he liked especially to walk through seven bridges that cross the river. Age coming (mathematical knowledge too ...), he wondered if without sacrificing his walk, he could shorten the length of his walk by crossing each bridge only once. This problem is probably one of the oldest known one in graph theory: that the existence of a chain passing once and only once by each edge (sadly today only four bridges remain standing).
	\begin{figure}[H]
		\centering
		\includegraphics{img/geometry/konigsberg.jpg}
		\caption{Antique map of the city of Königsberg}
	\end{figure}
	The river divides the city of Königsberg in four parts: $a, b, c, d$. Each bridge connects two such parties. Then we can represent our problem by a graph with four vertices, wherein each edge represents one of the seven bridges of Königsberg. In this example the graph is not a simple graph:
	\begin{figure}[H]
		\centering
		\includegraphics{img/geometry/konigsberg_graph.jpg}
		\caption{Königsberg equivalent graph}
	\end{figure}
	The question here is whether the graph is Eulerian or not? If our problem the obtained graph is Eulerian, we need to exhibit an Euler cycle, which does not seem easy. But if it is not? Euler gave a strong characterization of Eulerian graphs given by the following statement:
	\begin{theorem}[Euler Theorem]
	A graph is Eulerian if and only if it is connex and all vertices have even degree (so there is an even number of edges that arrive on each vertex of which half of them used to arrive at the vertices, the others to leave it) except for maximum two (these two exceptions being the vertice of departure and this of arrival).
	
	Specifically for a connex graph:
	\begin{itemize}
		\item The graph has no odd vertices, then it is Eulerian (and the chain is cyclic).
		
		\item The graph can have only one odd vertices from the property (already stated above) that in a graph, the vertices having odd order is always in even.
		
		\item If the graph has two odd vertices, then these vertices are the ends of the Eulerian cycle.
	\end{itemize}
	As corollary we have that a graph have more than two odd vertices never have an Eulerian cycle.
	\end{theorem}
	With this characterization (as we will prove it just after), the vertices $a,b,c,d$ being of odd order, we know immediately that it is impossible to walk cross all the Königsberg bridges in only one time during a walk.
	\begin{figure}[H]
		\centering
		\includegraphics{img/geometry/euler_konigsberg.jpg}
	\end{figure}
	
	\pagebreak
	\begin{dem}
	The proof will be done if three steps:
	\begin{enumerate}
		\item Let us suppose that a graph $G$ is Eulerian. Then there exists a chain $c$ browsing once and only once each edge (until then it's easy). Of course, in the case of a chain, the peaks being located at the ends of the chain are of odd order and are only two.
		
		\item Consider now a vertex $x$ and let us suppose this time not a Eulerian graph but an Euler cycle. During the course of this cycle, each time we go through the vertex $x$, we find ourselves at the starting point and to perform a new round two edged are available to us (indeed the path can be walked in two direction since the graph is undirected!). The vertex $x$ is the of even order and can be defined anywhere in the cycle, hence the fact that all the vertices are of even degree.
		
		\begin{itemize}
			\item If $G$ is reduced to a single isolated vertex, it is obviously Eulerian. Otherwise all the vertices of $G$ are of degree greater or equal to $2$. This implies that there is a cycle $\phi$ on $G$:
			
			\item Consider the partial graph $H$ consisting of the edges outside the cycle $\phi$. The vertices of $H$ are also of even degree, the cycle containing an even number of edges incident to each vertex. By induction each connex component $H_i$ of $H$ is an Eulerian graph, and so has an Eulerian cycle $\phi_i$. To rebuild an Euler cycle on $G$, we only need to merge the cycle $\phi$ the with different cycles $\phi$. For this, we travel the cycle $\phi$ from an arbitrary vertices; when we meet for the first time a vertices $x$ from $H_i$, we substitute it the cycle $\phi_i$. The resulting cycle is an Euler cycle in $G$, the cycle $\phi$ and the cycles $\phi_i$ form a partition of the edges.
			\begin{figure}[H]
				\centering
				\includegraphics{img/geometry/parties_eulerian_graph.jpg}
			\end{figure}
		\end{itemize}
	\end{enumerate}
	\begin{flushright}
		$\blacksquare$  Q.E.D.
	\end{flushright}
	\end{dem}
	\begin{tcolorbox}[title=Remark,colframe=black,arc=10pt]
	This principle of decomposing a graph into connex graphs and summing them permits to build a recursive algorithm for determining if a graph is Eulerian or not.
	\end{tcolorbox}
	
	\item[D28.] Two graphs $G$ and $G'$ are "\NewTerm{complementary graphs}\index{complementary graphs}" when they satisfy the following conditions:
	\begin{enumerate}
		\item They have the same set of vertices.
	
		\item Two vertices $x, y$ that are neighbours in $G$ are not neighbours in $G'$.
	\end{enumerate}
	\begin{figure}[H]
		\centering
		\includegraphics{img/geometry/complementary_graph.jpg}
		\caption{Example of complementary graph}
	\end{figure}
	A "\NewTerm{self-complementary}\index{self-complementary}" graph is a graph which is isomorphic to its complement. The simplest non-trivial self-complementary graphs are the 4-vertex path graph and the 5-vertex cycle graph.
	\begin{figure}[H]
		\centering
		\includegraphics{img/geometry/self_complementary.jpg}
		\caption[Example of self-complementary graph]{Example of self-complementary graph (source: Wikipedia)}
	\end{figure}
	
	\item[D29.] A "\NewTerm{bipartite graph}\index{bipartite graph}" is a graph $K_{p,q}$ such that we can partition the set of all its vertices into two classes cardinal of $p$ and $q$ respectively so that each edge has one end in each of the two classes.
		\begin{tcolorbox}[colframe=black,colback=white,sharp corners]
		\textbf{{\Large \ding{45}}Example:}\\\\
		Here is a representation of a classical bipartite graph $K_{3,3}$ . It is the famous problem of the supply of three houses from three supplier (water, electricity, gas) without the right of alignment of services (using same "pipelines"). We immediately see that this graph is non-planar.
		\begin{figure}[H]
			\centering
			\includegraphics{img/geometry/biparti_graph.jpg}
			\caption{Example of a $K_{3,3}$ bipartite graph}
		\end{figure}
		\end{tcolorbox}
		\begin{tcolorbox}[title=Remark,colframe=black,arc=10pt]
		The complete bipartite graph $K_{p,q}$ is a graph with $n=p+1$ vertices and configured in such a way that each vertex of a class is adjacent to all the vertices of the other and only thereto.
		\end{tcolorbox}	
		
		\item[D30.] Two graphs are "\NewTerm{isomorphic graphs}\index{isomorphic graphs}" if there exists a bijection of $f$ from $X$ into $X'$ such that for any vertices $x$ and $y$ of $G$:
		
		We also say that $f$ is an "\NewTerm{isomorphism of $G$ on $G'$}".
		
		More formally, an isomorphism of graphs $G_1$ and $G_2$ is a bijection $f:V(G_1)\mapsto V(G_2)$ that preserves adjacency. That is to say:
	
	
		For example on the figure below we have three isomorphic graphs:
		\begin{figure}[H]
			\centering
			\includegraphics{img/geometry/isomorphic_graphs.jpg}
			\caption{Isomorphic graphs}
		\end{figure}
		
		\begin{tcolorbox}[colback=red!5,borderline={1mm}{2mm}{red!5},arc=0mm,boxrule=0pt]
		\bcbombe Caution!!! Sometimes we talk about graphs equivalence "to a given isomorphism". This means more clearly, that at the exception of a single violation among the set of all edges, that the graphs are isomorphic.
		\end{tcolorbox}
		
		If there exists a bijection of $f$ of $X$ into itself such that:
		
		then we say that $f$ is an "\NewTerm{automorphism of graph}\index{automorphism of graph}" in $G$ (by permutation of the vertices there are many possible examples ...).
		
		\begin{tcolorbox}[colback=red!5,borderline={1mm}{2mm}{red!5},arc=0mm,boxrule=0pt]
		\bcbombe Caution!! Sometimes we talk about equivalent graphs "to a given isomorphism". This means, more clearly, that at the exception of a single violation and among the set of all edges, the graphs are isomorphic!
		\end{tcolorbox}
		
		As the isomorphism in the case of graphs go from one set to another of the same size $n$ ($n$ to $n$ mapping), the number of different possible bijections is (\SeeChapter{see section Probabilities page \pageref{distinguishable elements}}):
		
		This means there is a maximum of $n!$ graphs which can be grouped in a same equivalence class. Consequently, there exists at least (lower bound) the ratio of the total $n$ of vertices on the cardinal majorated by the greatest possible equivalence class (but not necessarily existing!):
			
		Using the gross majoration $n!<n^n$, we have:
		
		Therefore:
		
		Given:
		
		Thus, when $n$ approaches the infinity, $\log_2(N)$ admits a lower bound of order $n^2$.
	\end{enumerate}
	The reader should also keep in mind that some graphs are used more than other, and get special names:
	\begin{itemize}
		\item $K_n$: The complete graph on $n$ vertices
		\item $K_{m,n}$: The complete bipartite graph with sets of $m$ and $n$ vertices
		\item $C_n$: The cycle on $n$ vertices, just one big loop
		\item $P_n$: The path on $n$ vertices, just one long path
	\end{itemize}
	\begin{figure}[H]
		\centering
		\includegraphics[width=1.0\textwidth]{img/geometry/some_typical_graphs.jpg}
	\end{figure}
	
	\pagebreak
	\subsection{Graph Adjacency Matrix}\label{adjacency matrix}
	Formally, a graph is also a set on which we have defined a binary, anti-reflexive (no element is related to itself) and symmetric (if $x$ is related to $y$, then $y$ is related to $x$) relation. The graph structure may then seem particularly poor.
	
	But we can also associate to a graph $G$ vertices $x_1,\ldots, x_n$ a square matrix $M$ of dimensions $m\times n$ named the "\NewTerm{adjacency matrix}\index{adjacency matrix}" of the graph and whose elements are $0$ or $1$. Denoting by $m_{ij}$ the term located at the intersection of row number $i$ (representing the vertices $x_i$) and column $j$ (representing the vertices $x_j$), $M$ is defined as:
	
	Let us recall that such a matrix is said to be a "\NewTerm{symmetrical matrix}\index{symmetrical matrix}" (\SeeChapter{see section Linear Algebra page \pageref{symmetric matrix}}).
	\begin{tcolorbox}[title=Remark,colframe=black,arc=10pt]
	We also know that graphs can be represented by matrices in the context of the study of Markov chains in the field of probabilities (\SeeChapter{see section Probabilities page \pageref{markov chains}}).
	\end{tcolorbox}
	Let us see an example both abstract but also easily applicable to many fields of industry, sociology and biology (there is also application for the adjacency matrix in Text Mining as you can see in the R companion book). Consider the following directed graph and observe that it is not symmetrical nor antireflexive:
	\begin{figure}[H]
		\centering
		\includegraphics{img/geometry/markov_graph.jpg}
		\caption{Example of Markov chain (nor antireflexive or symmetric)}
	\end{figure}
	And as we just say we can represent this diagram (connected, and oriented graph) in the form of a table in which we denote by "$1$" a possible transition from  the state mentioned at the top of the column to the state mentioned at the beginning of the row and "$0$" otherwise:
	\begin{table}[H]
		\begin{center}
		\begin{tabular}{>{\columncolor[gray]{0.75}}c||c|c|c|c|c|c|c|}
	\hline
	\rowcolor[gray]{0.75}$\nearrow $ & E1 & E4 & E2 & E3 & E5 & E7 & E6 \\
	  \hline \hline
	  % after \\: \hline or \cline{col1-col2} \cline{col3-col4} ...
	E1. & $0$ & $0$ & $0$ & $0$ & $0$ & $0$ & $0$\\ \hline
	E4. & $0$ & $0$ & $1$ & $0$ & $0$ & $0$ & $0$\\ \hline
	E2. & $1$ & $1$ & $1$ & $1$ & $0$ & $0$ & $0$\\ \hline
	E3. & $0$ & $0$ & $1$ & $0$ & $0$ & $0$ & $0$\\ \hline
	E5. & $0$ & $0$ & $0$ & $1$ & $0$ & $0$ & $0$\\ \hline
	E7. & $0$ & $1$ & $1$ & $1$ & $0$ & $0$ & $0$\\ \hline
	E6. & $0$ & $1$ & $1$ & $1$ & $0$ & $0$ & $0$\\ \hline

		\end{tabular}
		\end{center}
		\caption[]{Adjacency matrix of the connected graph}
	\end{table}
	It is essential to fully understand the meaning of the values in this table! But at this level of the book that should not pose major difficulties to the reader.
	
	Now let us evaluate the number of ways to go:
	\begin{enumerate}
		\item From E2 to E2 in two stages
		\item From E3 to E4 in two stages
		\item Form E2 to E7 in two stages
	\end{enumerate}
	It is easy in the particular case above to count these possibilities. But in the case of a more complex graph it becomes difficult or even impossible for a human being within a reasonable time (and also for some computer at this time).
	
	We have to use then the following theorem:
	\begin{theorem}
	Given an orient graph with $x_1,x_2,\ldots, x_n$ vertices and of adjacency matrix:
	
	For any integer $k$, then the number of paths of length $k$ of vertices $x_1$ to a vertices $x_j$is given by:
	
	where the exponent of $M$, denotes the power $k$ of the adjacency matrix.
	\end{theorem}
	\begin{dem}
	Let us perform an induction on $k$:
	
	designates the number of paths going from $x_i$ to $x_j$ (where $i$ and $j$ can be equal). Let us suppose the result true for the integer $k-1$, with $k\geq 1$ as:
	
	we then have (by construction of matrix multiplication):
	
	By induction hypothesis, $m_{il}^{(k­1)}$ is the number of paths of length $k-1$ from $x_i$ to $x_l$ and $m_{lj}$ is, we know it (by construction!), the number of paths of length $1$ from $x_l$ to $x_j$ and in particular it is equal to $1$ if $(x_i,x_j)$ is an edge of the graph and equal to $0$ otherwise!
	
	So the product:
	
	gives for a given value of $l$ the number of paths of length $k$ from $x_i$ to $x_j$ whose last edge is $(x_l,x_j)$.

	The sum:
	
	therefore gives all the possibilities (paths) of length $k$ from $x_i$ to $x_j$ regardless of the starting point of the last edge!
	\begin{flushright}
		$\blacksquare$  Q.E.D.
	\end{flushright}
	\end{dem}
	Thus, in our example, the adjacency matrix $M$ is given by:
	
	and is neither symmetric or antireflexive as we have already mentioned.
	
	So if we carry this matrix at the power of $k$, each component $m_{ij}^{(k)}$ will give all possibilities (paths) of length $k$ going from $x_i$ to $x_j$ Thus we have for example with $k=2$:
	
	using for example the function \texttt{MMULT( )} available in most spreadsheet softwares.
	
	We then have the answer to all our three initial questions by reading the above matrix:
	\begin{enumerate}
		\item From E2 to E2 in two stages there are $3$ possibilities

		\item  From E3 to E4 in two stages there is $1$ possibility

		\item From E2 to E7 in two stages there are $3$ possibilities
	\end{enumerate}
	It is possible to gain some time in such calculations. If we denote $C_i$, the $i$-th column of the matrix $M$, then:
	
	So we always get the number of paths of length $k$ from a given starting point corresponding to the column $i$.
	
	This example has a very interesting approach in some areas studying the behaviour of individuals in different situations (shopping, tourisms, accidents, etc.).
	
	If instead of writing in the matrix $M$ the number of possible paths from one vertices to another, we write the probability (part) of the total number of individuals who go from one vertices to reach another one we have for example (in practice the values are imposed by experimental observation!) the following matrix:
	
	which is already the stochastic transpose  matrix of the visible graph below (according to the theory of Markov chains we saw in the section of Probabilities that we need to take the transpose of the stochastic matrix to calculate the probabilities of states).
	
	Considering that these probabilities do not change over time, then we have a homogeneous Markov chain (without cycles). We then see that:
	\begin{enumerate}
		\item The sum of the transition probabilities going from each vertices (state) must always logically be equal to $1$ (which we already mentioned in the section of Probabilities)

		\item Everyone start from the first vertices E1

		\item Some stagnate (stop) at some stage

		\item Those arriving at an end stage E5, E6 or E7 stay there and do not retrace their steps (absorbing states).
	\end{enumerate}
	The equivalent graph is therefore:
	\begin{figure}[H]
		\centering
		\includegraphics{img/geometry/markov_graph_probabilities.jpg}
		\caption[]{Markov probabilistic transition graph}
	\end{figure}
	By denoting $N$ (instead of $M$ to avoid confusion) the matrix constructed from the graph above we see that $C_i$ is if one of the columns of the matrix $M$ then for example:
	
	gives the sum of the transition probabilities that what goes from E1 arrives $2$ in respectively E1($0$), E2($0.662$), E3 ($0.218$) ... (the distribution of the initial vector may be any as long as the sum of column of values is equal to 1)!
	
	
	gives the sum of the transition probabilities that what goes from E1 arrives $3$ in respectively E1($0$), E2($0.691$), ... and so on. We can therefore know what is the probability that an individual arriving on E2 can arrive to one of the terminal nodes (E5, E6 or E7) on a number of finite step.

	\begin{tcolorbox}[title=Remark,colframe=black,arc=10pt]
	Remember that the sum of the probabilities of the resulting columns $T$ is always equal to $1$ for the transpose of the stochastic matrix.
	\end{tcolorbox}
	Continuing a long time as well ... we find that the equilibrium measurement $\pi$ that satisfies (\SeeChapter{see section Probabilities page \pageref{stationary measure}}):
	
	is:
	
	
	regardless of the distribution of the starting vector. Property named "\NewTerm{ergodic}\index{ergodic}" in the field of Markov chains.

	Which means $45\%$ probability of being in E5, $32\%$ probability of being in E7 and $23\%$ probability of being in E6. Another way to look at it is to say that if a cohort of $100$ individuals start at node E1 with constant probability in time between the various states transitions, the equilibrium we will have is $45$ people in E5, $32$ people in E7 and $23$ people in E6.

	\subsection{Categories}
	The introduction of categories through the "\NewTerm{category theory}\index{category theory}" by Eilenberg and MacLane in 1942 was to transform difficult problems of topology in more affordable algebra problems. Later, the theory of categories has grown significantly, both for itself and for its applications in the most diverse fields of mathematics (e.g. differential geometry). Although some of its autonomous development has sometimes been criticized, categories are now recognized as a powerful language to develop a universal semantics of mathematical structures. They are also used in logic and more recently in physics, and a fruitful collaboration seems to develop between IT and categoricians.
	
	\textbf{Definitions (\#\mydef):}
	\begin{enumerate}
		\item[D1.] Intuitively a "\NewTerm{category}\index{category}" is just a directed graph on which we have given a law to compose consecutive arrows, satisfying certain axioms.

		\item[D2.] An "\NewTerm{oriented graph}\index{oriented graph}" is formed by a set of nodes of graph, with links between them, represented by arrows from a node $A$ to a node $B$, what we denote by $f:A\mapsto B$. We say then that $A$ is the "source" of the arrow, and $B$ and its "goal". There may be several arrows having same source and the same goal (we say then that they are "parallel") and there can be "closed" arrows, which source and target coincide.

		\item[D3.] Two arrows $f$, $g$ are say to be "\NewTerm{consecutive arrow}\index{consecutive arrow}" if the target of the first is also the source of the second:
		
	\end{enumerate}
	then we say that they form a path of length $2$ from $A$ to $C$.

	A category is then a graph in which we define a composition of arrows, associating to any path $(f, g)$ of length $2$ from $A$ to $C$ an arrow of the graph from $A$ to $A$, named "\NewTerm{composed of the path}\index{composed of a path}", and denoted $fg$:
	\begin{figure}[H]
		\centering
		\includegraphics{img/geometry/graph_category.jpg}
		\caption{Example of graph of categories}
	\end{figure}
	This composition satisfies the following axioms:
	\begin{enumerate}
		\item[A1.] Associativity: If $fgh$ is a path of length $3$, the both compositions $f(gh)$ and $(fg)h$ that we deduce are associative. It follows that to any path of length $n$ is also associated only a single composition of vertices (invariance of the path).

		\item[A2.] Identities: At any node $A$ is associated a closed arrow from $A$ to $A$, obviously named "identity" of $A$ and denoted $\text{Id}_A$, from which the composition with an arrow of source or target $A$ is equal to this other arrow.
	\end{enumerate}
	More generally, a path (of length $n$) from $A$ to $A_n$ is a sequence of $n$ consecutive arrows:
	
	\begin{tcolorbox}[title=Remarks,colframe=black,arc=10pt]
	\textbf{R1.} The graph nodes are named "\NewTerm{objects}\index{objects}" of category and the arrows the "\NewTerm{morphisms}\index{morphisms}" (or simply "\NewTerm{links}\index{links}") within the framework of category theory.\\
	
	\textbf{R2.} An arrow $f$ is an isomorphism (\SeeChapter{see section Set Theory page \pageref{isomorphism}}) if there exists an arrow $g$ (named "\NewTerm{inverse}\index{inverse}") such that the compositions $fg$ and $gf$ are identities (this inverse is therefore unique).
	\end{tcolorbox}	
	Thus, a category is formed by objects (the nodes of the graph) and by the links between them (the arrows or morphisms), but the essential idea is to emphasize the links relatively to the objects. In fact, the success of the categories in the most varied fields is due to the quantity of information on objects that can be inferred from the mere consideration of links and operations on them, regardless of the nature and anatomy of these objects.
	
	In the next few lines, we will explain how to read the directed graphs we can sometimes encounter in math books. This will be a good example of category theory, because we have already encountered such graphs without describing them in the section Numbers and Cryptography for example.
	
	For simplicity we will explain these graphs when the basic objects are  sets (which is the most common case in this whole book anyway actually).
	Let us consider three sets $A$, $B$, and $C$ and three applications such that:
	
	and let us rewrite it as following:
	
	We can therefore consider the applications $f$, $g$ and $h$ as arrows that connect the objects (sets) $A$, $B$, and $C$ to form a triangle:
	\begin{figure}[H]
		\centering
		\includegraphics{img/geometry/commutative_graph.jpg}
		\caption{Example commutative diagram}
	\end{figure}
	Where obviously $g=h\circ f$ or more explicitly for those that forget the meaning of this notation that for any $a\in A$ we have $h(f(a))=g(a)$.
	
	\textbf{Definition (naive version \#\mydef):} We say that an oriented graph is a "\NewTerm{commutative diagram}\index{commutative diagram}" if all the paths we take to get from one object (set) to another one represent the same application.
	
	Or let us consider:
	
	and such that each application $f,g,h$ has its Kernel that is equal to the set of neutral elements of the starting set. Therefore:
	\begin{figure}[H]
		\centering
		\includegraphics[scale=0.8]{img/geometry/category_diagram.jpg}
	\end{figure}
	We can complicated almost without limits such diagrams considering more sets and arrows (applications) connecting them. For example:
	\begin{figure}[H]
		\centering
		\includegraphics{img/geometry/commutative_graph_more_complex.jpg}
	\end{figure}
	This diagram is commutative if and only if $h\circ f=r\circ g$.
	\begin{tcolorbox}[title=Remarks,colframe=black,arc=10pt]
	\textbf{R1.} Usually in the mathematical literature such diagrams are implicitly commutative.\\
	
	\textbf{R2.} As already mentioned, the objects of these diagrams can more generally be groups, rings, fields, topological space, etc. In these cases, the arrows are no longer any applications but group homomorphisms, or ring homomorphisms, continuous applications, etc.
	\end{tcolorbox}
	
	\begin{flushright}
	\begin{tabular}{l c}
	\circled{90} & \pbox{20cm}{\score{4}{5} \\ {\tiny 28 votes,  70.00\%}} 
	\end{tabular} 
	\end{flushright}

	 %to make section start on odd page
	\newpage
	\thispagestyle{empty}
	\mbox{}
	 \section{Knot Theory}
	 
	\lettrine[lines=4]{\color{BrickRed}R}ené Descartes had imagined a system of the World, when the universe was animated by vortices. At the end of last century, Tait and Kelvin have resurrected this theory, and interpreted the chemical bonds imagining knotted molecules that interlace.\\\\
	
	\begin{tcolorbox}[title=Remarks,colframe=black,arc=10pt]
	\textbf{R1.} The name "\NewTerm{Knots theory}\index{Knots theory}" that has emerged in the scientific community is quite unfortunate. Some (French)  mathematicians rightly use instead the name of "\NewTerm{theory of intertwining and braids}\index{theory of intertwining and braids}" which is more correct and general (since a node is interlaced with several components).\\
	
	\textbf{R2.} Most of the text and figures below is a reproduction, with agreement, of Professor Michael Eisermann course from the University of Stuttgart (\url{http://www.igt.uni-stuttgart.de/eiserm}).
	\end{tcolorbox}	
	
		The mathematical knot theory was pioneered by the work of Little and Kirkman who tried to give a basis for the physical ideas of Kelvin and Tait. The first nodes classifications using a wire image of the nodes, without thickness, and a planar projection, as the shadow cast on a screen. One immediately noticeable character concerns crossings where we distinguish the strand above and strand below. The cannages, or braiding strips are based on the consideration of the crossings, and the number of such intersections will be the first classifying element.
		
		\subsection{Braids Representation}
		An example of braid (set of strands) and the corresponding string (a string is a braid that was closed):
		\begin{figure}[H]
			\centering
			\includegraphics{img/geometry/strand_and_string.jpg}
			\caption{Braid and corresponding string (node diagram)}
		\end{figure}
		Let us focus on the braid to the left of the figure above to begin and let us ask ourselves what would be the best way of putting things in order to compare braids? Let us first attempt with the following representation of a particular braid:
		\begin{figure}[H]
			\centering
			\includegraphics{img/geometry/particular_braid_representation.jpg}
			\caption{Particular presentation of a braid}
		\end{figure}
		As the strands of a braid are flexible and can move, we notice immediately that this representation has a problem: all the braids are equal. Effectively:
		\begin{figure}[H]
			\centering
			\includegraphics{img/geometry/issue_of_braid_representation.jpg}
			\caption{Example of the problem of the braid representation}
		\end{figure}
		A better model then is perhaps to secure the ends of both sides:
		\begin{figure}[H]
			\centering
			\includegraphics{img/geometry/better_particular_braid_representation.jpg}
			\caption{Other choice of representation}
		\end{figure}
		Of course, with this model the middle strands may still move as below and the braid remains invariant:
		\begin{figure}[H]
			\centering
			\includegraphics{img/geometry/braid_first_invariant_example.jpg}
			\caption{First example of invariant braid}
		\end{figure}
		or otherwise as this:
		\begin{figure}[H]
			\centering
			\includegraphics{img/geometry/braid_second_invariant_example.jpg}
			\caption{Second example of invariant braid}
		\end{figure}
		We can also translate the crossings and the braid remains invariant:
		\begin{figure}[H]
			\centering
			\includegraphics{img/geometry/braid_tranlsation_invariance.jpg}
			\caption{Translation of crossing and preserving the trivial invariance}
		\end{figure}
		The length is not a variable in this model:
		\begin{figure}[H]
			\centering
			\includegraphics{img/geometry/braid_length_invariance.jpg}
			\caption{Length invariance}
		\end{figure}
		
		\subsubsection{Braids Group}
		If we define the multiplication of two braids as the operation that consists schematically to:
		\begin{figure}[H]
			\centering
			\includegraphics{img/geometry/braids_multiplication.jpg}
			\caption{Example of multiplication of two braids}
		\end{figure}
		We can ask ourselves if this is a type of commutative group?
		
		Remember before that we have defined the group structure in the section of Set Theory as follows: We refer to a set by the term "group" if its components meets the three conditions of what we call "internal law group ", defined below:
	
	If furthermore, the internal law is also commutative, then we say that the group is an "\NewTerm{abelian group}\index{abelian group}" or simply "\NewTerm{commutative group}\index{commutative group}".
	
	 \begin{enumerate}
	 	\item Let us start with the first check. Is this representation is associative?:
		\begin{figure}[H]
			\centering
			\includegraphics{img/geometry/braids_associative_check.jpg}
			\caption{Check of associativity of braids (by the multiplication)}
		\end{figure}
		Therefore the answer is NO beyond two strands (because with 2 it is commutative)!
		
		 \item Does it admits a neutral element?:
		 \begin{figure}[H]
			\centering
			\includegraphics{img/geometry/braid_neutral_check.jpg}
			\caption{Check of the existence of a neutral element for braids (by the multiplication)}
		\end{figure}
		The answer is therefore YES whatever the number of strands!
		
		\item Are there inverse elements (symmetrical)?:
		\begin{figure}[H]
			\centering
			\includegraphics{img/geometry/braids_symetrical_check.jpg}
			\caption{Check of the existence of an inverse element for braids (by the multiplication)}
		\end{figure}
		The answer is therefore YES whatever the number of strands!		
	 \end{enumerate}
	 
	 Therefore the braids with $n$ strands form a non-commutative group denoted by $(\mathcal{B}_n,\cdot)$ where the stylized $B$ represent the word "braids".
	 
	 Let us observe something interesting about the braids with two double strands. If we index their tendrils by integer numbers as below (imagine the braid from the middle of which you rotate the ends, it gives all other left and right tendrils):
	 \begin{figure}[H]
		\centering
		\includegraphics{img/geometry/braid_spin_tendrils.jpg}
		\caption{Indexing braids tendrils}
	\end{figure}
	with the convention of positive tendril $v$ and negative tendril $v$:
	\begin{figure}[H]
		\centering
		\includegraphics{img/geometry/braid_tendril_indexation_convention.jpg}
		\caption{Indexation convention}
	\end{figure}
	Beware to the following trap! (try to find it ... it is not always easy at first glance):
	\begin{figure}[H]
		\centering
		\includegraphics{img/geometry/braid_tendril_trap.jpg}
		\caption{Indexation tendril trap}
	\end{figure}
	Indeed, the right crossing above of the second braid is precisely not .... a cross! Then we say that the number of crossing is not an invariant. It is for this reason that what matters is the "tendril" because it does not change in the movements of the braid at the opposite of a simply cross!!!
	\begin{figure}[H]
		\centering
		\includegraphics{img/geometry/braid_difference_trindle_cross.jpg}
		\caption{Difference between tendril and cross}
	\end{figure}
	We then notice that regarding to the multiplication of braids (\SeeChapter{see section Set Theory page \pageref{homomorphism of group}}) and their indexing under the form of their number of tindrels, we have the following properties:
	
	So the multiplicative group of braid with 2 strands is a homomorphism of group with the addition law of the group of relative integers. Moreover, we see that the application $f$ is trivially bijective. We have therefore a isomorphism of group! Beyond 3 strands, we saw that the group was not more commutative!
	
	\subsection{Knot Representation}
	Let us return to the first figure presented above:
	\begin{figure}[H]
		\centering
		\includegraphics{img/geometry/strand_and_string.jpg}
		\caption{Braid and corresponding string (node diagram)}
	\end{figure}
	\begin{tcolorbox}[title=Remark,colframe=black,arc=10pt]
	Any node or any chain can be obtained from a braid (without proof).
	\end{tcolorbox}
	There exists many possible views, and different appearances of the same node. Two nodes of different projections are distinct (not isotopes)? In the first published table of node classification (that of Tait and Little) the minimum number of crossings is used as a classification principle (up to ten crossings) and it took almost a century to detect one duplication (see table further below): two identical nodes were taken for different.
	
	In practice a node rather looks like this:
	\begin{figure}[H]
		\centering
		\includegraphics{img/geometry/node_in_real_life.jpg}
		\caption{Example of a node in real life}
	\end{figure}
	But mathematicians have the usage to connect both ends of the string to get this representation of a node:
	\begin{figure}[H]
		\centering
		\includegraphics{img/geometry/node_usage_representation.jpg}
		\caption{Knot representation by mathematicians}
	\end{figure}
	therefore a node is always a loop that closes on itself. In other words, an open node is always equivalent to a closed node:
	\begin{figure}[H]
		\centering
		\includegraphics{img/geometry/equivalence_open_closed_knot.jpg}
		\caption{Equivalence between an open and a closed knot}
	\end{figure}
	A closed knot may look different depending on the angle from which we look at it. Thus, the two nodes below are two representations of the same type of knot named the "trefoil knot":
	\begin{figure}[H]
		\centering
		\includegraphics{img/geometry/clover_knot.jpg}
		\caption{2 different perspectives of the same trefoil knot}
	\end{figure}
	We can also ask ourselves if the left clover node and its mirror on the right are the same?:
	\begin{figure}[H]
		\centering
		\includegraphics{img/geometry/clover_knot_left_right_equality.jpg}
		\caption{Searching for equality between mirror symmetry of the clover node}
	\end{figure}
	or the same question but when the two knots are closed:
	\begin{figure}[H]
		\centering
		\includegraphics{img/geometry/clover_knot_closed_left_right_equality.jpg}
		\caption[]{Same situation but by closing the knot}
	\end{figure}
	More difficult, K. A. Perko has showed that the two knots below, named "\NewTerm{Perko par}\index{Perko par}", are in fact the same knot:
	\begin{figure}[H]
		\centering
		\includegraphics{img/geometry/knot_parko_par.jpg}
		\caption{Example of a Perko pair}
	\end{figure}
	It is therefore not trivial to know two physical objects represents basically the same knot. But, that's what interests mathematicians! Specifically, they want to classify the knots, that is to say determining all types of nodes that are fundamentally different, and not just in appearance (it's the same idea than in topology where mathematicians successfully proved that every volume could be reduced to three primary volumes).
	
	\subsubsection{Knots Group}
	Just as we did for braids, let us look if knots form a group?
	
	So we define the multiplication of two knots as the operation that is schematically:
	\begin{figure}[H]
		\centering
		\includegraphics{img/geometry/knot_multiplication_proposition.jpg}
		\caption{Refresh of the definition of the multiplication of two knots}
	\end{figure}
	We can ask ourselves if this is a commutative type group?
	
	\begin{enumerate}
		\item Let us start with the first check. Is this representation associative ?:
		\begin{figure}[H]
			\centering
			\includegraphics{img/geometry/knot_check_associative.jpg}
			\caption{Check of the commutativity of knots (by the multiplication)}
		\end{figure}
		The answer is YES whatever the number of knots!
		
		\item Second control: does it have a neutral element?:
		\begin{figure}[H]
			\centering
			\includegraphics{img/geometry/knot_check_neutral_element.jpg}
			\caption{Check of the existence of a neutral element for the knots (by the multiplication)}
		\end{figure}
		The answer is YES whatever the number of knots!
		
		\item Third check: is it commutative?:
		\begin{figure}[H]
			\centering
			\includegraphics{img/geometry/knot_check_commutativity.jpg}
			\caption{Check of the commutativity of knots (by the multiplication)}
		\end{figure}
		So unlike braids the answer is YES!
		
		\item Fourth and last check: Are there inverse elements (symmetrical)?:
		\begin{figure}[H]
			\centering
			\includegraphics{img/geometry/knot_check_inverse_element.jpg}
			\caption{Check of the existence of an inverse element for the knots (by the multiplication)}
		\end{figure}
		and that's the one million dollar question!	
	\end{enumerate}
	With time and the efforts of mathematicians, three essential mathematical perspectives  for studying knots emerged to solve different problems:
	\begin{enumerate}
		\item Topological, where the node is conceived as the union of a finite number of closed curves, define close to a given deformation.
		
		\item Algebraic, where there count the intersections for example, and where we associate the groups to the nodes.
		
		\item Geometrical, where we take into account the shape of the knot, measuring lengths or angles. In particular, the idea of the number of rotations appears as twisting and number of interlaces, and these concepts are fundamental to the study of DNA in molecular biology.
	\end{enumerate}
	
	The link between these views is tricky. There are evidences hard to prove rigorously. We think especially to the Jordan theorem asserting that a closed flat curve without crossing defines an interior and an exterior! It took almost two centuries to properly define the concept of curve and there are wild knots to eliminate before any classification. We must take care to properly define the deformation of a knot, that mathematician named "\NewTerm{isotopies}\index{isotopies (knot theory)}".
	
	The most effective mathematical idea for the study of the knots is probably that of "\NewTerm{knot invariant}\index{knot invariant}". An invariant of a node is a characteristic (integer number, real number, polynomial, group, etc.) which remains unchanged during deformation. If we have/found an invariant, we can say that two knots are really different when the invariant does not take the same value for both knots. But if two knots have the same invariant, we can not say they are the same type (deformable into each other). A typical example of two knots having the same invariant and for which it is not trivial to say whether they are the isotopes are the right and the left trefoil knots when the chosen invariant is the number of crossings denoted by $c(D)$. We should for this purpose have a complete system of invariants.
	
	Here, invariants are indispensable, and mathematicians have come up with dozens that distils different features of knots. But these invariants tend to have blind spots.
	
	Take, for example, an invariant named "\NewTerm{tricolorability}\index{tricolorability}". A knot diagram is tricolorable if there's a way to color its strands red, blue and green so that at every crossing, the three strands that meet are either all the same color or all different colors. Mathematicians have shown that even when you move the strands of a knot around, its tricolorability (or lack thereof) is unchanged. In other words, tricolorability is an innate feature of a knot.

	The trefoil is tricolorable:
	\begin{figure}[H]
		\centering
		\includegraphics{img/geometry/tricolorability.jpg}
		\caption[]{source: Quanta Magazine}
	\end{figure}
	But the "unknot" (a loop that has no actual knots, even if it appears tangled) is not tricolorable, providing an instant proof that the trefoil is not just the unknot in disguise. But while tricolorability enables us to distinguish some knots from the unknot, it's not a perfect tool for this purpose: Knots that are tricolorable are definitely knotted, but knots that aren't tricolorable aren't definitely unknotted. For instance, the figure-eight knot is not tricolorable, but it is genuinely knotted. This knot falls into tricolorability’s blind spot — it's as if the invariant is saying: \og The figure-eight knot is unknotted as far as I can tell\fg{}.
	
	The Conway knot, an i11-crossing knot discovered by John Horton Conway more than 50 years ago, is extraordinarily skilled at fooling knot invariants!
	\begin{figure}[H]
		\centering
		\includegraphics{img/geometry/conway_knot.jpg}
		\caption[]{source: Wikipedia, author: Saung Tadashi}
	\end{figure}
	The Russian mathematician V. Vasilyev introduced in 1990 a new class of invariants. It remains to make them explicitly computable and prove that they form a complete system.
	
	Henri Poincaré introduced in 1900 the concept of fundamental group of a space which describes the possibilities of paths with return to starting point. Applied to the outer space to a node, this provides the "\NewTerm{knot group}\index{knot group}" and the "\NewTerm{Alexander polynomial}\index{Alexander polynomial}" linked to it (see below the mathematical formalization).
	
	The same construction applied to what is named the "configuration space" provides an effective definition of "braid groups". These groups were introduced in an intuitive form, circa 1920, by the Viennese mathematician Emil Artin, one of the fathers of modern algebra. Since 1937, the Russian mathematician Markov connected knots and braids, and gave a theoretical method to define, with braids, knots invariants.
	
	At the beginning of years 1990, braid groups were a curiosity and their complexity repulsive. Then, suddenly, they have become a central theme of scientific research. Let us give an idea of the diversity of views that lead to braid groups:
	\begin{enumerate}
		\item In geometry, the Russian mathematician Vladimir Arnold classified under the name of "disasters" singularities of geometric configurations, which leads to the configuration spaces.
		
		\item In algebra, specifically in group theory, two avatars of braid groups play a central role: the Coxeter's groups and the Hecke's algebras.
		
		\item In statistical mechanics, the study of exactly soluble models is done through the use of Yang-Baxter relations and quantum groups. The link with the braid groups is profound.
		
		\item The two-dimensional physics has recently taken great importance and we expect from it a high temperature model of superconductivity. The usual classification of fermions and bosons particles is complicated in two dimensions. The new concept is that of anyone whose mathematical model is linked to braid groups.
		
		\item Quantum field theory is the mathematical model of elementary particles. Edward Witten has made the bridge between this theory and braid groups, via the Jones polynomial.
	\end{enumerate}
	The Knots Theory is therefore an active interface between physics and mathematics. Knots and braids now provide an effective modernization tool for the physic of polymers to the liquid crystal, through molecular biology. In the opposite direction, imported ideas of physics have sparked a revolution in mathematics: to somewhat marginal issue at the beginning of the years 1990, the Knots Theory at the beginning of the years 2000 one of the greatest mathematical subject.
	
	\subsection{Tait's Knot}
	The main aim of the knot theory is to clarify all knots and has its origin in physics in the early 19th century as we have already mentioned. Why?
	
	Recall that at that time atoms were a mystery: why they seemed indestructible and why they exist in so many varieties and can be combined to give countless other components?
	
	At that time the most beautiful equations of physics (which are often good candidates to explain things we do not understand...) were the Maxwell equations, then it was natural (or attempting) for physicists to try to explain the atomic mechanics in terms of electromagnetism, although we now know today that this path was destined to fail. Later in the mid 19th century, electromagnetic waves were largely conceptualized as vibrations of a medium named at the time "phosphor ether". The repository of this ether was then defined (or at least suspected) as an absolute reference... But later, the experimenters Michelson and Morley showed that the relative motion of our planet in this ether was undetectable (and worse... they measured the absolute constancy of the speed of light!). Their experiences also led Albert Einstein and Henri Poincaré to develop their famous theory of Special Relativity (\SeeChapter{see section Special Relativity page \pageref{special relativity}}):
	
	which is, we must admit, similar to fluid mechanics (see section of the same name page \pageref{fluid mechanics}) when we have an incompressible fluid without viscosity and without vortices:
	
	More generally, if the rotational $\vec{ \nabla}\times \vec{v}$ is not zero, Helmholtz proved in 1858 that the vortex field lines - defined by the lines of $\vec{ \nabla}\times \vec{v}$ - move in the direction of $\vec{v}$ as if they had an independent existence (and here this is the open breach ... of course!). These lines could have no end but could form loops.
	
	In 1867 the physicist Peter Guthrie Tait (assistant of Hamilton and a champion of quaternions) founded an ingenious way to prove this effect by cutting a circular hole in a box, filling it with smoke, and then expelling smoke afterwards through compression of the air in the box thereby forming smoke circles. It also showed this to his friend Kelvin that noted the analogy with electromagnetism and proposed a theory in which the atoms were vortex (knots) in ether! He hypothesized that the different types of atoms correspond to different types of knotted vortices (yes it's a bit farfetched but...)!
	
	Tait subsequently tried to classify (see table below) tied lines in agreement with:
	\begin{enumerate}
		\item The number of interlaces when they are projected on a plane
		
		\item By representing only the "\NewTerm{main nodes}\index{main nodes}"
	\end{enumerate}
	\textbf{Definition (\#\mydef):} A node can be complicated because it is the succession of simple knots:
	\begin{figure}[H]
		\centering
		\includegraphics{img/geometry/knot_open_succession_example.jpg}
		\caption{Example of a succession of knots on an open knot}
	\end{figure}
	These simple knots (with one strand) can be separated by cutting the string:
	\begin{figure}[H]
		\centering
		\includegraphics{img/geometry/knot_separation.jpg}
		\caption{Separation of the knot}
	\end{figure}
	A main knot is a knot that can not be separated into simpler knots: cut the string unties the knot.
	
	Classify knots this is trying to determine the building blocks: all main knots. The list Tait got initially was the following nodes (in the hope of obtaining the periodic table of elements but "ether" version...) and if necessary we recommend you to bring a string to ensure they are main one (sometimes it is mentally difficult):
	\begin{figure}[H]
		\centering
		\includegraphics{img/geometry/knot_tait_classification.jpg}
		\caption{Tait's classification}
	\end{figure}
	\begin{tcolorbox}[title=Remarks,colframe=black,arc=10pt]
	\textbf{R1.} This table is very important and we will very often make reference to it for application examples using the proposed nomenclature.\\
	
	\textbf{R2.} The corresponding values in the above table, named "\NewTerm{knot complexity measures}\index{knot complexity measures}", are noted $c(D)_{\text{Id}}^{c(B)}$ in generality where $c(D)$ is the number of crossings, $c(B)$ the number of strands, and $\text{Id}$ the number id of the node in the class $c(D)$.
	\end{tcolorbox}
	The beauty of this theory of "vortex atoms" was the fact that it was in agreement with the continuous appearance of the wonderful world of fluids, or to the Maxwell's equations by extension, to discretize the different types of atoms. A difficulty with this theory, however, was the remarkable stability of atoms. Kelvin admitted in 1905 after many years of failure in trying to prove that the movement Helmholtz circles was stable, the conclusion was that these circles were essentially unstable and should therefore dissipate. Curiously it is the extreme stability of the atoms that was one of the puzzle pieces to build the corpuscular quantum physics.
	
	Moreover, with the theory of Special Relativity, the concept of ether that was precious to Maxwell and his contemporaries, especially regarding the Knots Theory, became synonymous of a useless concept and with the addition of the theoretical quantum theory vortex atoms was completely forgotten. Knot theory resurfaced because of some conjecture that Tait was not able to proved (we will come back about this later) that were proved only in the 1980s in a hazardous turn in theoretical physics.
	
	What physicists abandoned intrigue mathematicians. The basic question remains the same: how can we say that two nodes are "isotopies" of each other (we will define further below what it is exactly). This question is closely related to the famous conjectures of Tait. To attack these conjectures and the basic problem of knots resemblance, the topologists developed invariants knots. An example of well known knot invariant and that have had  a great success are the Alexander polynomial discovered by J. W. Alexander in 1927 (see further below). Thus, if the polynomials of two nodes are different, these are then not isotopies. Unfortunately, there are some knots having polynomials Alexander that are equivalents which are not isotopies of each other...
	What physicists abandoned intrigue mathematicians. The basic question remains the same: how can we say that two nodes are "isotopies" of each other (we will define further below what it is exactly). This question is closely related to the famous conjectures of Tait. To attack these conjectures and the basic problem of knots resemblance, the topologists developed invariants knots. An example of well known knot invariant and that have had  a great success are the Alexander polynomial discovered by J. W. Alexander in 1927 (see further below). Thus, if the polynomials of two nodes are different, these are then not isotopies. Unfortunately, there are some knots having polynomials Alexander that are equivalents which are not isotopies of each other...
	
	The mathematical knot theory then developed during fifty years was a bit forgotten until the thunderclap of Jones in 1984 of a new knot invariant (the Jones polynomial). The discovery of Jones is pretty exemplar of the scientific point of view and can lead to meditation on the current organization of science. Jones was not at all an expert on knots. He was interested in the classification of factors in von Neumann algebra (functional analysis). He obtained matrices algebras whose commutation relations (Yang-Baxter equations) were close relations of the braid group. From braids to knots, there is only one step that Jones has taken with the help of Joan Birman which is a specialist in knots.
	
	Then we are see an explosion of discoveries: purely combinatorial version, new polynomial, etc. In 1989, Witten proved that the Jones polynomial can be obtained from quantum field theory by thanks to a Feynman integral, giving the first definition that does not use the flat projections of a knot. Somehow the Jones-Witten theory is a non-commutative extension of the work of Gauss. The Lie group (\SeeChapter{see section Set Algebra page \pageref{lie group}}) which acts in magnetism is $\text{U}(1)$, while the Witten invariant is a Feynman integral over a space of $\text{SU}(2)$-connections.
	
	\pagebreak
	\subsection{Mathematical Formalisation}
	A node is mathematically modelled by an injective application, differentiable and whose derivative does not vanish, of the circle in oriented three-dimensional space (trivial knot).
	
	The two central problems of the theory of knots is to decide in a calculable way if the knot is trivial (can be discard without cutting the string ..) or not (this problem is in this early 21th century not resolved as far as we know) and if two nodes are really equivalents.
	
	The first type of problem can be well represented on the fact whether the next knot is knotted or not...?:
	\begin{figure}[H]
		\centering
		\includegraphics[scale=0.7]{img/geometry/knotted_knot.jpg}
		\caption{Knotted knot?}
	\end{figure}
	The answer is no as shown in the figure below (read from left to right and top to bottom):
	\begin{figure}[H]
		\centering
		\includegraphics[scale=0.7]{img/geometry/knot_not_knotted.jpg}
		\caption[]{It is not...!}
	\end{figure}
	The goal for the second problem is to combine the knots computable mathematical objects (polynomials, numbers) named "\NewTerm{knot invariants}\index{knot invariants}" and that are insensitive to the deformation of a knot. If the invariant is not equal to that the trivial knot that is:
	\begin{figure}[H]
		\centering
		\includegraphics{img/geometry/knot_trivial.jpg}
		\caption{Trivial Knot}
	\end{figure}
	we are therefore sure that the knot is not trivial. And for example, the simplest non-trivial knot is the trefoil knot:
	\begin{figure}[H]
		\centering
		\includegraphics{img/geometry/knot_trefoil.jpg}
		\caption{Tefoil Knot}
	\end{figure}
	The problem is finding enough fine tuning invariants. We will describe two of them:
	\begin{enumerate}
		\item The number of interlacing (of two knots), idea of Gauss, and which occurs in electromagnetism
		
		\item The Jones polynomial (introduced in 1985 by Vaughan Jones, 1990 Fields Medal), which is subtle enough to distinguish for example right of the left trefoil knot.
	\end{enumerate}
	
	We will also describe a general class of invariants: finite type invariants or Vassiliev invariants. These invariants defined in a rather unconstructive way are maybe complete invariants, but we do not know for sure at this beginning of the 21st century. We will describe the number of interlacing of 2 knots as a combinatorial invariant calculable from a node diagram. We will then write the classical integral formula related to magnetism (Gauss) to calculate it. We will then do a little detour by the global differential geometry of curves in three-dimensional space to show the "White formula" that connects three geometric invariants associated with a ribbon. We will then describe the new polynomial invariant in a combinatorial point of view. The point of view of Feynman integrals will be discussed.
	
	Vassiliev has introduced a general family of invariants that contains most known invariants and we can say that they are of finite type. We will describe the principle.
	
	In biology the RNA and DNA strand and also amino acid filaments are wound in complex three-dimensional shapes (these are closed braids, as we know a more general case of knots). However, often, through the microscope, we see only a two-dimensional projection. Invariants give the possibility to trace back to three-dimensional information from 2D views we have:
	
	\begin{figure}[H]
		\centering
		\includegraphics{img/geometry/arn_or_adn.jpg}
		\caption{Example of RNA or DNA strand}
	\end{figure}
	
	On the other hand, biologists have observed knotted DNA molecules and found that the topological nature of the DNA molecule, that is to say the knot type formed by the molecule, affects its operational mechanism in the cells by conditioning some of its chemical properties. Some viruses attack cells by changing the long DNA molecules by interlacing them in different ways. Indeed, by the means of enzymes named topoisamérases, viruses cut and reattach different strands of the DNA molecule such that they take the form of a knot which can be very complex. It turns out that the type of knot obtained is in a way the virus signature. To fight effectively against the virus, it is imperative to recognize their signature by their action on DNA. Therefore to identify different viruses we must be able to recognize the different types of knots and this is where the theory of knots can help the biologist.
	
	\begin{tcolorbox}[title=Remark,colframe=black,arc=10pt]
	The identification of some knot type has been transformed by research centers in an online gaming which aims it to give a funny way to the people to contribute to find a solution. This seems to work since late 2011 some players seems to have made  relevant discovery regarding the analysis of proteins.
	\end{tcolorbox}
	
	Let us start now with some definitions. We have deliberately chosen to not to list a given number of definitions (as many in Graph Theory...) that the reader can easily find in literature or on the Internet. We allow ourselves to omit them in the sense that they will not be helpful in the application of knot theory in quantum field physics (and except for those who like make small drawings... they are useless in our point of view).
	
	\textbf{Definitions (\#\mydef):}
	\begin{enumerate}
		\item[D1.] A knot may be defined (as there are several possible definitions and some have small enough bothersome uncertainties...) by the image of the circle denoted by $S^1$ by an continuous application (the string is assumed as such), injective (this avoiding the string enters in itself):
		
		in other words, it is a curve without double points, drawn in Euclidean space of three dimensions.

		A knot is therefore represented by an injective application $f\in \mathcal{C}^1 ([0,1],\mathbb{R}^3)$ (impose $\mathcal{C}^1$ class knots avoids having too wild curves ...) verifying that $f(0)=f(1)$ (closed knot). The image of $f$ is sometimes named "\NewTerm{support of the knot $f$}\index{support of a knot}": this is the "physical" realization of the knot in space.
		\begin{tcolorbox}[title=Remark,colframe=black,arc=10pt]
		All knots will subsequently denoted by the letter $N$.
		\end{tcolorbox}
		
		To summarize crudely..., a knot is a string we for which we welded the ends.
		
		\item[D2.] We say that a knot is a "\NewTerm{trivial knot}\index{trivial knot}" if the application $f$ that defines it extends in a continuous application of the disc $D^2 \rightarrow \mathbb{R}^3$ and always injective: a trivial knot is therefore a knot that border an immersed disc of $\mathbb{R}^3$.
		
		\item[D3.] Two knots $\gamma_1, \gamma_2$ are equivalent if there exists a continuous application:
		
		such as $F(1)=\gamma_1$ and $F(2)=\gamma_2$. It remains to find $F$ that is a curve in the set of curves.
		
		\item[D4.] The $c(K)$ of a knot $K$ is the natural integer representing the minimum number of crossings for every diagram of a knot type (that is a natural measure of complexity).
		
		\begin{tcolorbox}[colframe=black,colback=white,sharp corners]
		\textbf{{\Large \ding{45}}Example:}\\\\
		The knot $0_1$ to has $c(K)=c(0_1)=0$ therefore null. It seems there exist no knots with $c(K)$ equal to the unit. A proof consist is to list all possible diagrams with one or two cross and see that they are in fact equivalent knots of the type $0_1$ or just interlaces. The trefoil knot (knot $3_1$) has a $c(K)=c(3_1)=3$.
		\end{tcolorbox}
		
		\item[D5.] The mirror image $\bar{K}$ of a knot $K$ is obtained by reflection on a plane in $\mathbb{R}^3$. The knot $K$ can be constructed by inversion of crossing of the knot diagram:
		\begin{figure}[H]
			\centering
			\includegraphics{img/geometry/knot_mirror_image.jpg}
			\caption{Mirror image by crosses inversion}
		\end{figure}
		
		\begin{tcolorbox}[colframe=black,colback=white,sharp corners]
		\textbf{{\Large \ding{45}}Example:}\\\\
		With the trefoil knot ("left trefoil" and "right trefoil"):
		\begin{figure}[H]
			\centering
			\includegraphics{img/geometry/knot_trefoil_mirror.jpg}
			\caption{Mirror of trefoil knot}
		\end{figure}
		\end{tcolorbox}
		
		\begin{tcolorbox}[title=Remark,colframe=black,arc=10pt]
		We also notice that in the Tait classification table, the mirrors knots are not represented!
		\end{tcolorbox}
		
		\item[D6.] Two knots are say to be "\NewTerm{isotopes knots}\index{isotope knots}" if we can switch from one to the other by continuous manipulation named "\NewTerm{isotopies}\index{isotopies (knots)}" (the left trefoil knot and right trefoil knot are for example not isotopes!) and as we know this is the problem number one in the knot theory: detect isotopes knots.
		
		In other words, an isotopie is a movement which does not deforms/change the knot.
		
		The isotopies give in the plane three particular types of movements named "\NewTerm{Reidemeister movements}\index{Reidemeister movements}":
		\begin{figure}[H]
			\centering
			\includegraphics{img/geometry/knot_reidemaster_movements.jpg}
			\caption{Reidemeister moves}
		\end{figure}
		It is therefore well three simple operations for changing a part of a knot without changing the nature of the knot itself.
		
		So two diagrams or knots are isotopes, if we can pass from one to the other by a finite sequence of Reidemeister movements. Thus, with the example below, we show that the initial knot is equivalent to the trivial knot:
		\begin{figure}[H]
			\centering
			\includegraphics{img/geometry/reidemeister_example.jpg}
			\caption{Simplistic application example of the Reidemeister movements}
		\end{figure}
		
		\item[D7.] Two knots are say to be "\NewTerm{equivalent knot}\index{equivalent knot}" or "\NewTerm{amphicheiral}\index{amphicheiral}". if they are isotopes OR if one is isotope to the image of the other in a mirror. According to the above definitions, each knot is necessarily equivalent to its own mirror image, but only reflective knots are isotopes to their mirror image. The eight knot $4_1$ is a good example of this kind of knots, which are almost rare:
		\begin{figure}[H]
			\centering
			\includegraphics{img/geometry/knot_isotopie.jpg}
			\caption{Knot isotopie}
		\end{figure}
		
		The Reidemeister movements are not always obvious to guess. Let us show you the details:
		\begin{figure}[H]
			\centering
			\includegraphics{img/geometry/knot_isotopie_details.jpg}
		\end{figure}
		
		\item[D8.] An "\NewTerm{interlaced}\index{interlaced}" is a subvariety(\SeeChapter{see section Topology \pageref{subvariety}}) compact (\SeeChapter{see section Topology page \pageref{compact}}) of class $\mathcal{C}^\infty$ and of dimension $1$.
		
		\item[D9.] The "\NewTerm{number of convex components}\index{number of convex components}" is denoted by $c(E)$. If $c(E)=1$ we say that $E$ is a knot.
		
		Most of time, interlaces are oriented (\SeeChapter{see section Graph Theory page \pageref{oriented graph}}) and we will identify the isotopes interlacing. So we represent the interlacing in the plane by projecting and specifying the type of crossing points.
		
		\begin{tcolorbox}[colframe=black,colback=white,sharp corners]
		\textbf{{\Large \ding{45}}Example:}\\\\
		Interlaces with three convex components (left) and trefoil knot (right):
		\begin{figure}[H]
			\centering
			\includegraphics{img/geometry/knot_interlaces.jpg}
			\caption{Interlace with three convex components}
		\end{figure}
		\end{tcolorbox}
		For sure two interlaces are isotopes if and only if we can move from one to the other by a finite sequence of Reidemeister movements.
		
		\item[D10.] Three interlaces $E_+,E_{-},E_0$ are named "\NewTerm{associates interlaces}\index{associates interlaces}" if they differ only in a crossing point and that at this point they are in one of the following configurations:
		\begin{figure}[H]
			\centering
			\includegraphics{img/geometry/interlaces_associated_example.jpg}
			\caption{Example of associated interlaces}
		\end{figure}
		We denote by $\mathcal{E}$ the set of all interlacing classes and we will now focus in functions of the type $P:\mathcal{E}\rightarrow A$ where $A$ is a commutative ring (the reader will must forget that we have seen that the coefficients of a polynomial are elements of a ring).
		
		\begin{tcolorbox}[title=Remark,colframe=black,arc=10pt]
		We must also be remembered that a knot is a curve and that any curve can be represented by a polynomial where the idea!
		\end{tcolorbox}
		
		\item[D11.] $P$ is say to be "\NewTerm{invariant by association}\index{invariant by association}" (by association interlaces...) if:
		
		and if there exist invertible $a_+,a_{-},a_0 \in A$ such that for any associated interlacing triplet $(E_+,E_{-},E_0)$ we have (!):
		
		We can already prove in a fairly basic way that if such a polynomial exists, then it is uniquely determined by the coefficients of the above equality. We summarize this in the following theorem:
		
		\begin{theorem}
			If $P$ is invariant by association, then it is uniquely determined by the coefficients  $a_+,a_{-},a_0$.
		\end{theorem}
		\begin{dem}
		Let us notice first that:
		
		where $\bigcirc^r$ describes the knot consisting of $r$ non-interlaced circles. Indeed, the relation $P(\bigcirc)=1$ and the invariance property of $P$ applied to the following interlaces:
		\begin{figure}[H]
			\centering
			\includegraphics{img/geometry/non_interlaces_notation_example.jpg}
			\caption{Example of non-interlaces notation}
		\end{figure}
		gives:
		
		Therefore wet:
		
		Thus, by induction on $r$ we get:
		
		\begin{flushright}
			$\blacksquare$  Q.E.D.
		\end{flushright}
		\end{dem}
		In other words, the function $P$ can be expressed only by its coefficients!!! So we can now try to see if a knot full of interlaces may always be reduced to $P(\bigcirc)$ recursively.
	\end{enumerate}
	
	\subsubsection{Planar Representation}
	We assume that the knot is in the oriented three dimensional Euclidean space. We look at projections of a knot on the plane $z=0$. It is intuitively clear that we can assume that node has no vertical tangent and that the crossing points are only double and transversal. Such screening will be named a "\NewTerm{good projection}\index{good projection}".
	
	We then indicate at each crossing point which is the strand which passes over the other. Such a drawing represents a knit in a non-unambiguous way: we say that we have a "\NewTerm{knot graph}\index{knot graph}". Of course, two equivalent knots have good projections that give different knots diagrams in general (herein lies also one of the problems as well). It is therefore important to read the equivalence of two knots directly on their charts.
	
	Here is an example with the trefoil knot:
	\begin{figure}[H]
		\centering
		\includegraphics{img/geometry/knot_diagram_representation.jpg}
		\caption{Trefoil knot and its diagram}
	\end{figure}
	\begin{tcolorbox}[title=Remark,colframe=black,arc=10pt]
	The trefoil knot is the knot having the minimum number of intersections, namely $3$; it actually exists in two amphiceiral (images of one another by reflection for refresh...). This is in fact in a simple knot the ends have been welded. The trefoil knot is the border of a Möbius with three half-twists and also the torus knot of order $(3,2)$ (3 windings around the core, on two revolutions), as well as the on of order $(2,3)$.
	\end{tcolorbox}
	
	As a drawing is not necessarily easy to transmit, or to draw on a computer, we can also give a coding diagram of the knot by a matrix with integer coefficients at three lines and whose number of columns equals the number of double points.
	
	We number the $2n$ crossing points on the closed curve (circle) in the order they arrive. We note that the pairs are all formed of an even number and an odd number. We build then the first two rows of the matrix by putting in each column both numbers giving the same projection: the odd on the first row, the odds on the second one. We then add to each column, an $\pm 1$ indicating the orientation of the two strands (oriented) ($+1$ if the bottom one passes through from right to left when we traverse the top one, $-1$ otherwise). For example, the matrix associated with the trefoil knot of the previous figure is:
	
	
	\begin{flushright}
	\begin{tabular}{l c}
	\circled{15} & \pbox{20cm}{\score{4}{5} \\ {\tiny 20 votes,  77.00\%}} 
	\end{tabular} 
	\end{flushright}
	
\chapter{Mechanics}

	\textit{\textbf{Mechanics is the branch of physics that relates to the study of forces and their actions in abstract form}}. (Larousse)
	\minitoc
	%to make section start on odd page
	\newpage
	\thispagestyle{empty}
	\mbox{}
	\section{Principia}
	\lettrine[lines=4]{\color{BrickRed}B}y introducing the Mechanics through the Principia of Physics we finally do, after a veeeery long and necessary travel into the beautiful world of mathematics, our first steps in the field of theoretical physics (in a simplified version) who does not seek only the precision but also most of times the understanding of phenomena (exact numerical simulations and calculations and real life experiments give anyway very similar results of the theories as we will see). Understanding a theory in physics is primarily equivalent understanding the underlying mathematics behind. The rest is just some imagination and if possible, observation.
	
	As we have already said, physics is a fundamental science that has a profound influence on all other sciences and also on human society and that aims to explain the \textit{how}, not the \textit{why} (see the Introduction of this book for details). Future physicists and engineers are thus not the only ones who should have understood its basic ideas, but anyone considering a career in science (including students majoring in biology, chemistry, mathematics and finance) should acquire the same comprehension.
	
	The primary purpose of this book, we insist..., is to give the reader a unified view of physics by presenting what we think are the fundamental ideas that constitute the body of knowledge (absolute minimum) of contemporary physics.
	
	Often, physics is taught as if it were a juxtaposition of several sciences, more or less connected, but without any real concern for unity and detail (which is why the majority of books and websites are without proofs). We rejected this mode of teaching (to have suffered of this during our studies) and opted for a unified and carefully detailed presentation by making when necessary reference to a section of the books that contains the proofs of mathematical tools used or another underlying physical theory. Obviously the law, relations and equations of physics can be compared to a computer program that possess, in addition to their immediate logical sense, most of time incomplete, a more global sense that can be revealed only sequentially according to the progress of mathematics and of our understanding of our Universe.
	
	We also insist that every reader should know the basics of logic, arithmetic, algebra, vector calculus, linear algebra, tensor calculus, differential and integral calculus and analytical geometry prior to the study of physical phenomena to work with all rigor and all the necessary understanding to the mathematical reasoning that will be introduced from now (mathematics is the foundation of the huge edifice of theoretical physics!). We will see in what follows that all tools or mathematical results presented so far will be used !!
	
	Let us recall that "\NewTerm{Theoretical Physics}\index{theoretical physics}" is the "exact/deductive science" that takes care of the best mathematical modeling natural phenomena, artificial, observable or non-observable. In shorter terms, we could talk about description of "reality" as to know if it is sensible or true reality...).
	
	When we want to predict or describe a specific physical phenomenon, we can usually go through an analytical model where different values given by unspecified (abstract values: variables) and the laws of physics by functions, when they are known (if so, we can make a hypothesis and test it). By equating a physical phenomenon, we translate reality (measurements) into a mathematical experience, virtual, according to certain rules. We conduct a simulation of reality on expressed quantities.
	
	The different "laws" are often historically developed on empirical facts first and are subsequently verified experimentally (see the hypothetical-deductive method in the section of Proof Theory). Assuming that these laws are valid in the context (time frame, basis frame, etc.), we can therefore expect that the mathematics are in line with the experimental facts (or vice versa). Of course, a virtual experience is not real and can not express the reality in all its subtlety. This is only a model! It is therefore clear that the prediction of a physical phenomenon may differ from actual experimental facts.
	
	\begin{tcolorbox}[title=Remarks,colframe=black,arc=10pt]
	\textbf{R1.} We would like to recall (because of an abuse or misunderstanding that we find too often on various Internet forums), that a "\NewTerm{scientific experiment}\index{scientific experiment}" is a practical work of the study of a phenomenon which is reproducible (by independent research groups) in similar conditions and which number of reproductions is high enough to ensure that the errors (standard deviations) on measures become negligible between the independent groups (there are also international ISO norms for this purpose!!!).\\
	
	\textbf{R2.} It should also be noted that most theoretical models that we will study now in this book making use of algebra or vector analysis and can be rewritten with the tools tensor analysis and based on the methods of Lagrangian formalism (see section Analytical/Lagrangian Mechanics page \pageref{lagrangian formalism} for know what it is...). However, these latter methods can not easily be used for a simple introduction to physics because they require additional efforts from reader and much more paper and time to get the same results (often at least...). However, and we will come back about this, these methods are now essential and of primary importance in the various fields of high-level modern physics such as fluid mechanics, General Relativity, quantum field physics, analysis of chaotic systems and some others.
	\end{tcolorbox}	
	Before we begin our study of physical phenomena, we must define the main concepts on which theoretical physics is based. Thus, we will see in the order:
	\begin{enumerate}
		\item Human beings have created a system of measurement units which are arbitrary  standard quantities to a given coefficient (this is why still some countries in the World don't have the same methods to count time or to weigth objects even if there are ISO norms), specific to identify simply and in a reproducible way every physical phenomenon.
		
		\item Some concepts interrelated to our vision of our environment lead us to make Assumptions/Hypothesis and Principles (to postulate something...) which relate to our sensory reality while being transposed to any other reality of this type.
		
		\item Theoretical physics leads us to consider the fundamentals of nature as abstract mathematical concepts. Thus our common observation gives us a concrete view of the Universe while theoretical physics gives us an abstract view of it. Bu theoretical physics let us sometimes inspects and understand (at least we suppose it) phenomenons that cannot be observed  experimentally (or some dozens or hundreds years later only after technologies developments).
	\end{enumerate}
	Then we can be led to ask however the following question: do the facts determine which theory/model is true?
	
	By observing nature, we can see facts: this is data that we do not create. Astronomers, for example, note the position of celestial objects. We understand a fact since it appears as the consequence of the order of things described by a theory. But theories still retain the status of assumptions: even when a theory is consistent with all observed facts, it does not prove it to be true (a famous example is the Theory of Evolution considered by many people as being absolutely true even if it does not respect the scientific protocol). Indeed, there is still an infinite number of possible theories that are all compatible with all the observed facts. Then we say that the facts "sub-determine" theories: facts impose constraints on theories in the sense that only the theories consistent with the observed facts, are acceptable. But these constraints will always be low enough to leave the choice among an infinite number of theories.
	
	Theoretical advances may consist in setting aside old, incorrect paradigms (e.g., aether theory of light propagation, caloric theory of heat, burning consisting of evolving phlogiston, or astronomical bodies revolving around the Earth) or may be an alternative model that provides answers that are more accurate or that can be more widely applied. 
	
	Theoretical physics consists of several different approaches. In this regard, theoretical particle physics forms a good example. For instance: "phenomenologists" might employ (semi-) empirical formulas to agree with experimental results, often without deep physical understanding. "Modelers" (also named "model-builders") often appear much like phenomenologists, but try to model speculative theories that have certain desirable features (rather than on experimental data), or apply the techniques of mathematical modeling to physics problems. Some attempt to create approximate theories, named "effective theories", because fully developed theories may be regarded as unsolvable or too complicated. Other theorists may try to unify, formalise, reinterpret or generalise extant theories, or create completely new ones altogether.[e] Sometimes the vision provided by pure mathematical systems can provide clues to how a physical system might be modeled. Theoretical problems that need computational investigation are often the concern of computational physics.
	
	Of course, scientists are actually considering a finite number of theories, depending on what seems easiest in a given conceptual framework. At the time of Kepler, for example, there were three main theories to explain the movements of the planets, all consistent with the observed facts. According to the Ptolemaic theory, planets orbits are circular or located on a epicycle around the Earth motionless at the center of the universe. In the Copernican theory, the Sun occupies the center, the orbits of the planets and the Earth being located on circles and epicycles. In the Keplerian theory, planets orbits are ellipses with the Sun at one focus and finally comes General Relativity that we will study later...
	
	\pagebreak
	\subsection{System of Units}
	The range of objects and phenomena studied in physics is immense. From the incredibly short lifetime of a nucleus to the age of Earth, from the tiny sizes of subnuclear particles to the vast distance to the edges of the known universe, from the force exerted by a jumping flea to the force between Earth and the Sun, there are enough factors of $10$ to challenge the imagination of even the most experienced scientist. Giving numerical values for physical quantities and equations for physical principles allows us to understand nature much more deeply than qualitative descriptions alone. To comprehend these vast ranges, we must also have accepted units in which to express them. We shall find that even in the potentially mundane discussion of meters, kilograms, and seconds, a profound simplicity of nature appears: all physical quantities can be expressed as combinations of only a few base physical quantities.

	\textbf{Definition (\#\mydef):} A "\NewTerm{measurement}\index{measurement}" or "\NewTerm{physical quantity}\index{physical quantity}" is the quantitative nomological expression of a property, of an effect or an abstract quantity defined by a model that represent the object or phenomenon under study. A fundamental measurement does not explain itself, it described itself with respect to a definition. 
	
	\textbf{Definition (\#\mydef):} A "\NewTerm{unit of measurement}\index{unit of measurement}" is a definite magnitude of a physical quantity, defined and adopted by convention or by law, that is used as a standard for measurement of the same physical quantity. Any other value of the physical quantity can be expressed as a simple multiple of the unit of measurement.
	
	In fact it's not enough to simply agree on units of measurement! These have to be consistent, reliable and standardized in order to enable international collaboration and interoperability. There is a science to measurement, and it's named "\NewTerm{metrology}\index{metrology}".
	
	\begin{tcolorbox}[colframe=black,colback=white,sharp corners]
	\textbf{{\Large \ding{45}}Example:}\\\\
	For example, length is a physical quantity. The "meter" is a unit of length that represents a definite predetermined length. When we say $10$ meters (or $10$ [m]), we actually mean $10$ times the definite predetermined length named "meter".
	\end{tcolorbox}
	
	We recognize two types of measurements:
	\begin{enumerate}
		\item The "\NewTerm{constants}\index{constants}": they have a concrete value expressible numerically and are supposed not to change during the studied phenomenon . They are "\NewTerm{passive measurements}\index{passive measurements}" (we come back later on this and will enumerate some of them).
		
		\item The "\NewTerm{variables}\index{variables}": they have a concrete value only in a particular state, but not when we observe the physical phenomenon as a whole (during all a range of time). These are "\NewTerm{active variables}\index{active variables}". The variables describing a physical phenomenon are often correlated to each other by means of functions. Then by definition we say that these variables have a "\NewTerm{functional relation}\index{functional relation}" between them.
	\end{enumerate}
	\begin{tcolorbox}[title=Remark,colframe=black,arc=10pt]
	One measurement only makes sense if it is "observable", measurement to which we associate a number, the result of the measurement performed by using a device.
	\end{tcolorbox}
	Measure a physical quantity is equivalent to compare it to a known quantity of the same kind (we also say "of the same dimension"), taken as an arbitrary standard (the rigorous approach of this related the abstract modern Measurement Theory that we have already study in the section of the same name). The measurement result is expressed thanks to two elements:
	\begin{enumerate}
		\item A number which is the ratio of the measured quantity to the standard of the measurement quantity.

		\item A name or acronym that identifies clearly the chosen standard.
	\end{enumerate}
	The notation of a measurement $A$ is therefore in abstract form:
	
	The "number" $\{A\}$ is in fact the measured value (magnitude) of the quantity $A$ as multiple $n$ of the corresponding standardized unit $\{S_A\}$ and $[A]$ is its "name" that we commonly call "\NewTerm{physical unit}\index{physical unit}" or simply "\NewTerm{unit}\index{unit}" (the quantitative expression of a variable or a function). These two elements are inseparable, the measured value is meaningful only if we show the same time the selected unit. It changes if we change the unit!!!
	\begin{tcolorbox}[title=Remark,colframe=black,arc=10pt]
	The transition from one unit to another to express the same quantity is named "\NewTerm{unit conversion}\index{unit conversion}". Conversion of units involves obviously the comparison of different standard physical values, either of a single physical quantity or of a physical quantity and a combination of other physical quantities.\\
	
	Starting with:
	\begin{gather*}
		A=n\{S_A\}\; [\text{A}]
	\end{gather*}
	just replace the original standard reference unit $\{S_A\}$ with it meaning in therms of the desired unit $\{S_B\}$, if:
	\begin{gather*}
		\{S_A\}=c_{A\rightarrow B}\;  \{S_B\}
	\end{gather*}
	then:
	\begin{gather*}
		A=n\{S_A\}\; [\text{A}]=nc_{A\rightarrow B} \{S_A\}\; [\text{B}]=n \{S_B\}\; [\text{B}]
	\end{gather*}
	Now $n$ and $c_{A\rightarrow B}$ are just the numerical values corresponding respectively to the multiplicand of the standard and the conversion factor, we just have to multiply them to finish the job.
	\end{tcolorbox}
	\begin{tcolorbox}[colframe=black,colback=white,sharp corners]
	\textbf{{\Large \ding{45}}Examples:}\\\\
	E1. One example of the importance of agreed units is the failure of the NASA Mars Climate Orbiter, which was accidentally destroyed on a mission to Mars in September 1999 instead of entering orbit due to miscommunications about the value of forces: different computer programs used different units of measurement (newton versus pound force). Considerable amounts of effort, time, and money were wasted.\\

	E2. On 15 April 1999, Korean Air cargo flight 6316 from Shanghai to Seoul was lost due to the crew confusing tower instructions (in meters) and altimeter readings (in feet). Three crew and five people on the ground were killed. Thirty-seven were injured. \\

	E3. In 1983, a Boeing 767 ran out of fuel in mid-flight because of two mistakes in figuring the fuel supply of Air Canada's first aircraft to use metric measurements. This accident was the result of both confusion due to the simultaneous use of metric and Imperial measures and confusion of mass and volume measures.
	\end{tcolorbox}
	\textbf{Definition (\#\mydef):} Some variables can, for the sake of simplification of writing, be expressed from other quantities. We then say that the new quantity "derives" from base units. We also say that two physical quantities are "\NewTerm{homogeneous quantities}\index{homogeneous quantities}", or "\NewTerm{dimensionally consistent}\index{dimensionally consistent}", if they are of the same physical nature or if we can express them both in the same basis unit(s).
	
	This also means that the equations in physics must obey the following rules:	
	\begin{itemize}
		\item Every term in an expression that is added or subtracted must have the same dimensions; it does not make sense to add or subtract quantities of differing dimension. In particular, the expressions on each side of the equality in an equation must have the same dimensions.

		\item The arguments of any of the standard mathematical functions such as trigonometric functions (such as sine and
cosine), logarithms, or exponential functions that appear in the equation must be dimensionless. These functions require pure numbers as inputs and give pure numbers as outputs.
	\end{itemize}
	
	So, after a long period of reflection the physical world seems to be reducible to the concepts of space, time and energy.
	
	Thus appears another possible definition of Physics:
	
	\textbf{Definition (\#\mydef):} "\NewTerm{Physics}\index{physics}" is the science of the properties and mutual relations in time of matter and energy to a given factor of charge.
	
	Our role is to give a description of these properties and relations under the form of physical equations or laws applied to observed phenomena in the context of a theory providing the predictible facts.
	
	Physical quantities are not all independent of each other but linked by certain laws or relations. It would be unreasonable, but possible, to choose a particular unit for each of the physical quantities regardless of their mutual relations.
	
	Build a coherent system of units consists therefore to determine a minimum number of units that establish the rules of construction of these mutual relations. Indeed, historically, a lot of units has been set for the measurement of a single quantity to meet specific needs of areas of practical life or science and technology. Their choices and their definitions were often empirical (or even based on religious beliefs sometimes!), and hence the conversions between units not always easy or clear.
	\begin{figure}[H]
		\centering
		\includegraphics[scale=0.7]{img/mechanics/based_units.jpg}
		\caption{Based SI units and derived units}
	\end{figure}
	In regards to the evolution of sciences and technologies, at the increased exchange of information, the need to unify the system appeared obvious, the aim being to keep a small as possible number of units. These are the "\NewTerm{fundamental units}\index{fundamental units}". From the physical laws and relations between the different fundamental units, we deduce the units of other values which then become the by sake of simplification of writing "\NewTerm{derived units}\index{derived units}".
	
	The choice and definition of the fundamental units set is a complex issue. Indeed, these units must be highly accurate and available in measurement laboratories of the entire planet.
	
	\begin{tcolorbox}[title=Remark,colframe=black,arc=10pt]
	The International System of Quantities (ISQ) is a system based on seven base quantities: length, mass, time, electric current, thermodynamic temperature, amount of substance, and luminous intensity. Other quantities such as area, pressure, and electrical resistance are derived from these base quantities by clear non-contradictory equations. The ISQ defines the quantities that are measured with the SI units. The ISQ is defined in the international standard ISO/IEC 80000, and was finalized in 2009 with the publication of ISO 80000-1. A small summary is given the figure above.
	\end{tcolorbox}
	The fours basic units are at the number of  $4$ (we will justify this later): the length (meters), weight (kg), time (seconds), the electric charge (coulombs). This system then constitutes the "\NewTerm{M.K.S.C. system}\index{M.K.S.C. system}" (the main redactor of this book assumes the choice of adding the Coulomb as fundamental unit).
	
	The units of the M.K.S.C system are within this book:
	\begin{enumerate}
		\item The meter [m], for the length $L$ (we have already defined the concept of length in the chapter Geometry but we will come back again on it later). Since 1983, the meter is now defined as the path length traveled by light in a vacuum in a period of $1 / 299,792,458$ second. So we can say that it derives from the time if we want and is therefore not a fundamental unit.

		\item The kilogram [kg] for the mass $M$ (we will return back later to the definition of the concept of mass) is since 1889 the mass of the platinum-iridium prototype deposited in the Internatioal Office of Weights and Measures at Sèvres (France).
		
		\item The second [s], for time and (time is not measurable in himself as it cannot be touched, but the time interval $\Delta t$ is an arbitrary concept quite valid - we also will also return back on the definition of this concept further below). The relative second is since of 1967 the duration of $9,192,631,770$ periods of the radiation corresponding to the transition between the two hyperfine levels of the ground state of the cesium 133 atom.

		\item The Coulomb [C] used as a fundamental unit of electric charge $q$ (not derived from any known at this date - we will also return to the definition of this concept further below).
	\end{enumerate}
	
	 \begin{tcolorbox}[title=Remarks,colframe=black,arc=10pt]
	\textbf{R1.} The concept of angle $\theta$ (in radians, degrees or steradians - see the texts on plane trigonometry, spherical trigonometry and plane geometry in the chapter Geometry) since it has no unit because by definition its a ratio of lengths (for the radian or degree of angle) or surface (for the steradian). It is therefore appropriate to assimilate it to a derived unit and not as fundamental unit. However, in physics, we have sometimes the habit to indicate its presence in the dimensional equations to help re-reading some of them and to know that their result is given with respect to an angle unit (otherwise it could risky generate misinterpretations for those using the equations without have read the proof...).\\
	
	\textbf{R2.} The development of science has led the General Conference of Weights and Measures to introduce some additional practices (but not necessary) units such as (among others as you can see it in the previous figure) the temperature in "Kelvin" [K] (derived from the average energy - Brownian motion), the amount of expressed in terms Moles, the current intensity in Ampere [A] and the light intensity expressed in Candelas [cd]. Thus, the actual International System (SI) with its seven fundamental base units (kilogram-meter-second, kelvin, candela, mole and ampere) and its seventeen-derived units does it suggests that seven units are needed to describe all physics? Actually no! As the Gauss analysis suggests it, among the seven fundamental units, four of them - Kelvin, Candela, the Mole and the Ampere - can be derived from the other three. The introduction of seven base units represents a pragmatic balance between experimentalists who require units adapted to their measures, and idealism theorists, whose goal is to reduce arbitrary redundancy to a minimum.\\
	
	\textbf{R3.} This is a huge chance to have a homogeneous system such as we have in the 21st century. Indeed, for the record, in 1522 alone in the Baden-Baden area (Germany) there were 112 units of various measures of length and 92 for surfaces .... that is to say the mess... a nightmare! In the early 21st century there is non-free ISO 80000 norm which aims to harmonize the notations, definitions and values of units in all fields of sciences.
	\end{tcolorbox}
	These clarifications donce, any physical quantity known at this date can be expressed using a unit which is expressed as the product at a given power (positive or negative power!) of the five fundemantal dimensional factors and an arbitrary scale factor $ $K (dimensionless number, which acts as a proportionality factor for taking from a scale, that of the fundamental unit, to that of the studied variable):
	
	where the numbers $m,l,t,\theta,c$ are named respectively "\NewTerm{mass order}", "\NewTerm{length order}", "\NewTerm{time order}", "\NewTerm{angle order}" and "\NewTerm{charge order}" are positive, negative or zero integers.
	
	The above expression will be written in the "\NewTerm{canonical form}\index{canonical form}" defined by the standard units:
	
	the angle having no units, we don't write it anymore (but can be still here implicitly depending on the context!).
	
	Any physical quantity $X$ is therefore expressed as:
	
	where $x$ is the magnitude of the physical quantity in the unit system associated with the scaling factor $K$. There are several possible pairs $(x, K)$, but we will always have:
	
	where the constant $x_0$ is the value of the physical magnitude when we choose to express it in the system M.K.S.C.
	
	So two physical quantities are homogeneous equation and equation if and only if the quadruplets:
	
	that are linked to them are equals:
	
	It follows from what we have said, that:
	\begin{itemize}
		\item The sum or difference of any number of quantities is meaningful only if these quantities are homogeneous and the result will have the same units as the operands.

		\item The product or division of multiple quantities has for units respectively the product or division of the operands units.
	\end{itemize}
	\begin{tcolorbox}[title=Remark,colframe=black,arc=10pt]
	 The units of the different quantities have a practical side but not infallible in theoretical physics: they nevertheless allow the physicist to verify whether a proven relation between two quantities is at least correct at the unit level. We name this kind of approach a "\NewTerm{dimensional analysis}\index{dimensional analysis}\label{dimensional analysis}" (we advise you to go see the proof of the Stokes law in the section of Continuum Mechanics for a very good example of application).
	\end{tcolorbox}
	
	\pagebreak
	For a little summary:
	
	\begin{tabular}[t]{||l|ll||}
	\multicolumn{3}{@{}l}{\large\bf Basic units}\\[1mm]
	\hline
	\bf Quantity    &\bf Unit&\bf Sym.\\
	\hline
	\hline
	Length           &meter   &m\\
	Mass             &kilogram&kg\\
	Time             &second  &s\\
	Therm. temp.     &kelvin  &K\\
	Electr. current  &ampere  &A\\
	Luminous intens. &candela &cd\\
	Amount of subst. &mol     &mol\\
	\hline
	\multicolumn{3}{l}{}\\
	\multicolumn{3}{@{}l}{\large\bf Extra units}\\[1mm]
	\hline
	Plane angle      &radian    &rad\\
	Solid angle      &sterradian&sr\\
	\hline
	\end{tabular}
	\hfill
	\begin{tabular}[t]{||l|lll||}
	\multicolumn{4}{@{}l}{\large\bf Derived units with special names}\\[1mm]
	\hline
	\bf Quantity       &\bf Unit&\bf Sym.&\bf Derivation\\
	\hline
	\hline
	Frequency          &hertz   &Hz  &$\rm s^{-1}$\rule{0pt}{11pt}\\
	Force              &newton  &N   &$\rm kg\cdot m\cdot s^{-2}$\\
	Pressure           &pascal  &Pa  &$\rm N\cdot m^{-2}$\\
	Energy             &joule   &J   &$\rm N\cdot m$\\
	Power              &watt    &W   &$\rm J\cdot s^{-1}$\\
	Charge             &coulomb &C   &$\rm A\cdot s$\\
	El.\ Potential     &volt    &V   &$\rm W\cdot A^{-1}$\\
	El.\ Capacitance   &farad   &F   &$\rm C\cdot V^{-1}$\\
	El.\ Resistance    &ohm &$\Omega$&$\rm V\cdot A^{-1}$\\
	El.\ Conductance   &siemens &S   &$\rm A\cdot V^{-1}$\\
	Mag.\ flux         &weber   &Wb  &$\rm V\cdot s$\\
	Mag.\ flux density &tesla   &T   &$\rm Wb\cdot m^{-2}$\\
	Inductance         &henry   &H   &$\rm Wb\cdot A^{-1}$\\
	Luminous flux      &lumen   &lm  &$\rm cd\cdot sr$\\
	Illuminance        &lux     &lx  &$\rm lm\cdot m^{-2}$\\
	Activity           &bequerel&Bq  &$\rm s^{-1}$\\
	Absorbed dose      &gray    &Gy  &$\rm J\cdot kg^{-1}$\\
	Dose equivalent    &sievert &Sv  &$\rm J\cdot kg^{-1}$\\
	\hline
	\end{tabular}

	
	\subsubsection{Dimensional Analysis}
	Dimensional analysis is an area of physics with that concerns the to units of the magnitudes. In particular, the fact that the units are relatively arbitrary makes that any valid equation of physics must be homogeneous: something that is measured in meters per second can not be equal to something that is measured in kilograms per meter. It is a very popular way and very effective way to check its own calculations (and that of others ...).
	
	This statement, which can seem paradoxical, effectively means: Do not start (if possible ...) in a complicated calculation without finding beforehand the qualitative form of the result with dimensional analysis.
	
	This qualitative form is traditionally named the "\NewTerm{dimensional equation}\index{dimensional equation}" and thus represents the formula that gives the possibility to determines the unit in which the result must be expressed in a research. This is an equation of variables (quantities/magnitudes), that is to say, in which the measured phenomena are represented by a unit symbol those we have seen in the previous paragraphs.
	
	\begin{tcolorbox}[colframe=black,colback=white,sharp corners]
	\textbf{{\Large \ding{45}}Example:}\\\\
	Let's see an example of a legend (?) often quoted in magazines and popular books:\\
	
	Dimensional analysis should have enabled G. I. Taylor in 1950 to estimate the energy released by the explosion of an atomic bomb (first detonation of a nuclear weapon in July 16, 1945: code name "Trinity", tested in the Jornada del Muerto desert), while this information was classified top secret, it would have been enough for him to watch a movie of the explosion, unwisely made public by the US military.\\
	
	The physicist G. I. Taylor suppose to arrive at this result that the process of expansion of the gas sphere depends at the minimum of the parameters of time $t$, the generated energy $E$ by the explosion and the density of air $\rho$.\\ 
	\begin{figure}[H]
		\centering
		\includegraphics[scale=0.67]{img/mechanics/atomic_bomb_dimensional_analysis.jpg}
		\caption[Supposed picture used by G. I. Taylor for the estimation]{Supposed picture used by G. I. Taylor for the estimation (source: ?)}
	\end{figure}
	Dimensional analysis then conducted him for the radius of the sphere of gas at time $t$ to:
	
	where $k$ is a dimensionless constant.\\
	
	And by trial and error we find relatively quickly $x=1/5,y=-1/5,z=2/5$  such as:
	
	Indeed:
	\end{tcolorbox}
	
	\begin{tcolorbox}[colframe=black,colback=white,sharp corners]
	
	Taylor then finds the time dilation law of the radius of the atomic mushroom is proportional to (there is no need to indicate other units, since only the temporal part interests us here!):
	
	If we know $r$ and $t$ from a film, and $k$ being supposed to be of the order of the unit and $\rho$ being known, we finally get:
	
	what remains a rough approximation. But getting such a result (of magnitude) with the heavy artillery of theoretical physics would require much more time and paper.
	\end{tcolorbox}
	
	\paragraph{Time}\label{time}\mbox{}\\\\
	\textbf{Definition (\#\mydef):} The "\NewTerm{time}\index{time}" is a state variable (not a "measurable") and therefore a notion impalpable but however rigorously defined. It also is a mathematical tool to equate the observation of physical phenomena (observable) and thus draw some information. This tool exists because there are living beings to observe (and measure) the Nature and its changes (Socratic principle) and matter and movement for there to be changes. Time is therefore the indefinite continued progression of existence and events that occur in apparently irreversible succession from the past through the present to the future. Time is a component quantity of various measurements used to sequence events, to compare the duration of events or the intervals between them, and to quantify rates of change of quantities in material reality or in the conscious experience. Time is often referred to as a fourth spatial dimension, along with the three spatial dimensions (\SeeChapter{see section Special Relativity page \pageref{special relativity}}), when multiplied by the speed of light.
	
	\begin{tcolorbox}[title=Remarks,colframe=black,arc=10pt]
	\textbf{R1.} Time (and time intervals) being an arbitrary concept, it is symmetrical, that is to say that every phenomenon observed can be in the imagination of a possible inverted time finding his initial conditions. We speak then of "\NewTerm{symmetry of time}\index{symmetry of time}" (for now it has never been proven as far as we know that time may suffer of a "\NewTerm{breaking symmetry}\index{breaking symmetry}"). In fact as we have seen in the section of Statistical Mechanics, the direction of time is defined by the concept of entropy and as the entropy can only increase (as far as we know), the question to go back in time is a non-sense (however in thermodynamic systems that are not closed, remember that entropy can decrease with time). We speak then of "\NewTerm{thermodynamic time}\index{thermodynamic time}".\\
	
	\textbf{R2.} Certain subatomic interactions involving the weak nuclear force violate the conservation of parity, but only very rarely. According to the CPT theorem, this means they should also be time irreversible, and so establish an arrow of time. This, however, is neither linked to the thermodynamic arrow of time, nor has anything to do with our daily experience of time irreversibility.\\
	
	\textbf{R3.} If an intensive variable is one that is unchanged with the thermodynamic operation of scaling of a system (\SeeChapter{see section Thermodynamics page \pageref{intensive quantities}}) the time is not an intensive variable if we think to General Relativity (mass influence on time).
	\end{tcolorbox}	
	We represent very often in physics time (within a interval) by a horizontal arrow (axis). Since time is a relative concept, we can then define each instant of time as being zero time denoted $t_0$. This concept is very often used in physics because the only thing that interests physicists is the time difference denoted $\Delta t$ (by the use of differential and integral calculus) or $\mathrm{d}t$ when the interval is very small.
	
	Let us prove now that the time reference is independent of the choice for an observer at rest. Given a time denoted by the letter $t$, then we have:
	
	where $t'$ is an arbitrary basis (not necessary) when we compare a temporal difference.
	
	The time interval is given by a standard measure that can only be at best a perfectly periodic movement (repeated over time). So, the first means of measuring time were day and night..., the positions of the Sun and Moon in the sky..., the pendulum, the relaxation of springs, the Cesium 137 degeneration period or oven binary systems binaries of massive stars. In short, as an observable system produces a stable periodic phenomenon and small enough for any physical action can be reduced, it can be used as a standard time interval!!!
	
	\textbf{Definitions (\#\mydef):}
	\begin{enumerate}
		\item[D1.] An "\NewTerm{event}\index{event (physics)}" consists to give meaning to a point in space-time.

		\item[D2.] Two events are say to be "\NewTerm{simultaneous events}\index{simultaneous events}" if they have the same value of the time coordinate.

		\item[D3.] We name "\NewTerm{coincidence}\index{coincidence}" the simultaneity of two events at the same point in space. The coincidence is an absolute fact, independent of the choice of the repository. This is actually a particular case of the causal conservation principle. Two events coinciding in a referential frame may be the cause and effect of each other (and vice versa), and that possibility is retained in the new marker.
	\end{enumerate}
	
	After Albert Einstein's Special Relativity (see section on the subject page \pageref{special relativity}), time can no longer be thought of being constant in all reference frames. Thus a watch that indicate one minute in one frame of reference will not be one minute in a reference frame that is traveling at a velocity relative to the first frame. This means time of an event is variable depending on the observer.
	
	\paragraph{Length}\mbox{}\\\\
	\textbf{Definition (\#\mydef):} The concept of "\NewTerm{length}\index{length}" $x$ is given by the information that gives the path taken by an object in a given time interval or simply be the distance between two points.
	
	Length may be distinguished from "height", which is vertical extent, and "width" or "breadth", which are the distance from side to side, measuring across the object at right angles to the length. Length is a measure of one dimension, whereas "area" is a measure of two dimensions (length squared) and "volume" is a measure of three dimensions (length cubed) as we will study this in details in the section of Euclidean Geometry. In most systems of measurement, the unit of length is a base unit, from which other units dimensions are defined.
	\begin{tcolorbox}[title=Remarks,colframe=black,arc=10pt]
	\textbf{R1.} If there was no matter in the Universe there would be no concept of movement and thus no path length and as we have already pointed it out, also no time (and again ... it is without considering some results of quantum physics that we will prove in the sections dedicated to it).\\
	
	\textbf{R2.} Length is obviously an extensive quantity (additive). Indeed, the total length of an itme is the sum of the length that composed it.
	\end{tcolorbox}
	As for time, there is no absolute origin of length measurement (there is no zero point in the Universe as postulated, among others, by the theory of relativity) and physicists are interested anyway especially only to differences of traveled path $\Delta x$ relative to an origin as they do for the time.
	
	Thus, identically to the time we have for an observer at rest observing a moving material point:
	
	where $x'$ is a mathematically arbitrary basis that is useless when comparing a difference of position in a time difference of a material point (as it cancels as you can see in the previous relation).
	
	If a material point is located in a three dimensional space (the most frequent case in classical mechanics, astronomy, astrophysics and particle physics) we arbitrarily chose an origin O, we note the position by of this body by $\vec{r}$, its distance in length by $x$, in width by $y$ and in height by $y$ (named "Cartesian coordinates" as we already know) by an imaginary arrow named "vector" (\SeeChapter{see section Vector Calculus page \pageref{vector}}) connecting the arbitrary origin of the reference frame to the interested point as follows:
	
	\begin{tcolorbox}[title=Remark,colframe=black,arc=10pt]
	The arrow above the $\vec{r}$ means of course that it is a vector!
	\end{tcolorbox}
	The notation:
	
	is a simplified notation used frequently in physics and who should prevail in small classes (take care! the fact that the numbers are in indices here, does not mean that they are covariant components - see section of Linear Algebra page \pageref{covariant components} - this is just a notation convention). However in this book we move from one to the other notations based on the needs and usage (it will be up to you to be careful to not confuse).
	
	The matrix:
	
	is the metric tensor of a canonical pre-Euclidean space with positive signature (\SeeChapter{see section of  Vector Calculus page \pageref{canonical basis} and General Relativity page \pageref{metric flat space}}). This is a particular case in theoretical physics but still a very common case study in classical mechanics (you must start with simple spaces before going further...).

	We will return in more detail on these concepts during our study punctual spaces further below.
	
	After Albert Einstein's Special Relativity (see section on the subject page \pageref{special relativity}), length can no longer be thought of being constant in all reference frames. Thus a ruler that is one meter long in one frame of reference will not be one meter long in a reference frame that is traveling at a velocity relative to the first frame. This means length of an object is variable depending on the observer.
	
	\paragraph{Mass}\mbox{}\\\\
	In general the "\NewTerm{mass}\index{mass}\label{mass}" $m$ of a body is in a closed system an amount which is preserved and that characterizes the amplitude with which a body interacts with other body through different (attractive) forces.
	
	\begin{tcolorbox}[title=Remark,colframe=black,arc=10pt]
	In an isolated system, they can not be spontaneous creation or destruction of mass. The mass emergence can only be due to an external action. Another way of saying the same thing is that the total mass in the Universe is constant!
	\end{tcolorbox}
	
	But in most books and classes we consider five type of masses.

	\textbf{Definitions (\#\mydef):}
	\begin{enumerate}
		\item[D1.] The "\NewTerm{invariant mass}\index{invariant mass}" or "\NewTerm{proper mass}\index{proper mass}" denoted by $m_0$ is scalar quantity, an intrinsic property of a body (preserved by Poincaré transformations, and with some clauses, in General Relativity). Is constant for subatomic particles (each species has its value of mass), as well as for atoms (and similar objects) on a specified energy level. Does not have a conservation law. Is not strictly constant for large bodies; for example, increases with heat.

		\item[D2.] The "\NewTerm{rest mass}\index{rest mass}" $m_0$ is the same as invariant mass if it is positive. For particles with zero invariant mass, strictly speaking, rest mass is undefined (although both terms might be synonymous in colloquial speech). A separate term is motivated by the fact that massive bodies, and only them, have their rest frames (may be also named "comoving frames").
	
		\item[D3.] The "\NewTerm{relativistic mass}\index{relativistic mass}" denoted $m$ Its the rest mass evaluated at a speed different form zero using Special Relativity equations (see corresponding section).
		
		\item[D4.] The "\NewTerm{inertial mass}\index{inertial mass}" is the "$m$" thing in Newtonian mechanic. Thought to be a conserved scalar quantity. For low speeds approximately equals to the rest mass, but generally an obsolete concept since relativistic dynamics. Depending on context, is superseded in relativity either with the rest mass or (along momentum vectors) with 4-momenta. One can’t consistently define inertial mass of a photon, for example.

		\item[D5.] The "\NewTerm{gravitational mass}\index{gravitational mass}" $m$ supposed as gravitational charge (for Newton's gravity law or some another theory). The interesting thing is that, physically, no difference has been found between gravitational and inertial mass. Many experiments have been performed to check the values and the experiments always agree to within the margin of error for the experiment. Albert Einstein used the fact that gravitational and inertial mass were equal to begin his theory of General Relativity in which he postulated that gravitational mass was the same as inertial mass.
		\begin{tcolorbox}[title=Remarks,colframe=black,arc=10pt]
		\textbf{R1.} Experiments have shown that the inertial and gravitational masses were proportional to the ten-billionth. This experimental identity named "\NewTerm{Galilean principle of equivalence}\index{Galilean principle of equivalence}" is the basis of the postulates of General Relativity (see section of the same name page \pageref{general relativity}).\\
		
		\textbf{R2.} Unlike electric charges (see further below the definition of "electric charge"), which characterize the amplitude of interaction by the electric force, it seems that they are only positive masses. Indeed, the electric charges can push or attract between them.
		\end{tcolorbox}
	\end{enumerate}
	In addition, the mass being an additive property (so "extensive" as we have already said) of matter: for a system of $n$ masses $m_i\;(i=1,\ldots,n)$, the total mass is:
	
	Similarly, for a continuous volume distribution, the total mass of a system of volume $V$ is given by:
	
	where $\rho_V(A)$ is the "\NewTerm{mass density}\index{mass density}" or "\NewTerm{volume density}\index{volume density}" of the system at point $A$ and where $\rho_V(\vec{r})$ is the density of the system at the point indicated by $\vec{r}=\overrightarrow{\text{O}A}$ (this is what is meant in the braces below the second triple integral).
	\begin{tcolorbox}[title=Remarks,colframe=black,arc=10pt]
	\textbf{R1.} The triple integral (on the three dimensions of space), can be reduced to a simple integral when exploiting the possible symmetries of the system.\\
	
	\textbf{R2.} The calculation of the mass in a non-continuous distribution (discrete) material should be done with the vector components calculated separately. Once done, then it is appropriate to take the norm.\\
	\end{tcolorbox}
	
	Therefore $\rho_V(A)$ is the mass density of a material elements, centered around $A$, of characteristic dimensions of those of the system, but large compared to the interatomic distances in the system defined by:
	
	If the system is homogeneous we have obviously:
	
	\begin{tcolorbox}[title=Remark,colframe=black,arc=10pt]
	The mass density is an intensive quantity. Indeed, the density of a physical system is not equal to the sum of these densities (this is common sense!). The reader will notice that this intensive quantity that is the density is the ratio of two extensive quantities.
	\end{tcolorbox}
	\textbf{Definitions (\#\mydef):}
	\begin{enumerate}
		\item[D1.] We say that a system is a "\NewTerm{massic homogeneous system}\index{massic homogeneous system}" if its volumic, surfacic, linear densities (see definition below) are constant on any small element of volume that constitute it.

		\item[D2.] We say that a system is an "\NewTerm{isotropic system}\index{isotropic system}", if its physical properties are identical in all points and directions.
	\end{enumerate}
	We also define sometimes the "\NewTerm{surfacic mass density}\index{surfacic mass density}" (or simply "\NewTerm{surfacic density}\index{surfacic density}") for systems with virtually no thickness and a "\NewTerm{linear mass density}\index{linear mass density}" (or  simply "\NewTerm{linear density}\index{linear density}" of mass) for systems with also negligible relatively to their length. We then have ($S$ being a surface and $s$ curvilinear abscissa):
	
	with in the general case:
	
	If the system is homogeneous we have again obviously with the traditional notation:
	
	\textbf{Definition (\#\mydef):} With the above, we can define the "\NewTerm{density}\index{density}" as the number of identical elements and all countable per unit volume, or unit surface or unit linear.
	
	The density of a material varies with temperature and pressure. This variation is typically small for solids and liquids but much greater for gases. Increasing the pressure on an object decreases the volume of the object and thus increases its density. Increasing the temperature of a substance (with a few exceptions) decreases its density by increasing its volume. In most materials, heating the bottom of a fluid results in convection of the heat from the bottom to the top, due to the decrease in the density of the heated fluid. This causes it to rise relative to more dense unheated material.
	
	\pagebreak
	\paragraph{Energy}\mbox{}\\\\
	We still do not know what exactly is energy (denoted in its general form with the letter $E$) but we know its effects. What we know however is that the energy can be classified can be transferred partially to other objects or converted into different forms. The main common forms being:
	\begin{itemize}
		\item "\NewTerm{Work Energy}\index{work energy}" which is the energy created by the application of a force on a body giving it a certain kinetic energy (temperature\footnote{The virial theorem, as proved in the section of Continuum Mechanics, indicates the absolute temperature as proportional to the average kinetic energy of the random microscopic motions of those of their constituent microscopic particles, such as electrons, atoms, and molecules, that move freely within the material.}, electricity) or potential energy (gravitational, electrostatic, magnetic potential energy) as studied in the sections of Classical Mechanics, Astronomy, Electrostatics, Electrodynamics, Thermodynamics and Continuum Mechanics.

		\item "\NewTerm{Heat Energy}\index{heat energy}" which is a form of energy determined by the number of microstates of a system as seen in the sections of Thermodynamics and Statistical Mechanics.

		\item "\NewTerm{Mass Energy}\index{mass energy}" is the energy contained in a certain amount of mass (nuclear energy, chemical energy) as seen in the sections of Special Relativity and Statistical Mechanics.
		
		\item "\NewTerm{Radiation energy}\index{radiation energy}" is the  energy held in light waves (electromagnetic wave of a particle without rest mass), allowing them to travel across space.
	\end{itemize} 
	\begin{figure}[H]
		\centering
		\includegraphics[scale=1]{img/mechanics/form_of_energy.jpg}
		\caption[Some form of energy]{Some form of energy (source: ?)}
	\end{figure}
	The total energy of a system can be subdivided and classified in various ways. For example, classical mechanics distinguishes between kinetic energy, which is determined by an object's movement through space, and potential energy, which is a function of the position of an object within a field. It may also be convenient to distinguish gravitational energy, thermal energy, several types of nuclear energy (which utilize potentials from the nuclear force and the weak force), electric energy (from the electric field), and magnetic energy (from the magnetic field), among others. Many of these classifications overlap; for instance, thermal energy usually consists partly of kinetic and partly of potential energy.

	Some types of energy are a varying mix of both potential and kinetic energy. An example is mechanical energy which is the sum of (usually macroscopic) kinetic and potential energy in a system. Elastic energy in materials is also dependent upon electrical potential energy (among atoms and molecules), as is chemical energy, which is stored and released from a reservoir of electrical potential energy between electrons, and the molecules or atomic nuclei that attract them..The list is also not necessarily complete. Whenever physical scientists discover that a certain phenomenon appears to violate the law of energy conservation, new forms are typically added that account for the discrepancy.

	Potential energies are often measured as positive or negative depending on whether they are greater or less than the energy of a specified base state or configuration such as two interacting bodies being infinitely far apart. Wave energies (such as radiant or sound energy), kinetic energy, and rest energy are each greater than or equal to zero because they are measured in comparison to a base state of zero energy: "no wave", "no motion", and "no inertia", respectively.
	
	We can still try to ask ourselves what is exactly energy?
	
	\textbf{Definition (\#\mydef):} The "\NewTerm{energy}\index{energy}" is the effect of a cause of a change or conservation of the properties of a system (many textbooks define energy as the variation of work but this is circular definition). This cause is not necessarily deterministic and being of zero mean and conservative in a closed system. The SI derived unit of energy is the Joule [J]. It is also common in Nuclear Physics, Plasma Physics and Particle Physics to use the electron-volt\index{electron-volt} that is by definition the energy gain by a charge of a single electron moving across an electric potential difference of one volt and equal to $1.6021766208(98)\cdot 10^{-19}$ [J] (\SeeChapter{see section Nuclear Physics page \pageref{electron volt}}).
	
	As we will see in the section of Analytical Mechanics during our study of Lagrangian formalism, the energy can be defined as the conserved quantity associated with time translation symmetry of the Lagrangian.
	
	\begin{tcolorbox}[title=Remarks,colframe=black,arc=10pt]
	\textbf{R1.} The mass and energy are equivalent as we will see during our study of Special Relativity (\SeeChapter{see section of Special Relativity page \pageref{special relativity}}), if we define a system of units such that the speed of light system is $c=1$ (widely used convention by physicists in advanced research) the mass is then just equal to the energy!\\
	
	\textbf{R2.} The energy, in the same way that mass, is an extensive quantity.\\
	
	\textbf{R3.} In an isolated system, they cannot be spontaneous creation or destruction of energy by definition. Spontaneous apparition of energy can only be due to an external action. Another way of saying the same thing is that the total energy in the Universe is constant if the latter is considered as isolated.\\
	
	\textbf{R4.} In certain study fields, such as plasma physics and particle physics, it is sometimes common to convert the electronvolt (energy) as a unit of temperature as it help for comparison analysis with an intuitive everyday life quantity (we will have such examples in this book). The conversion to the Kelvin scale is defined by using the Boltzmann constant $k$ (\SeeChapter{see section Continuum Mechanics page \pageref{boltzmann constant}}):
	
	For more detail and a criticism of the conversion, the reader must refer to the section of Nuclear Physics.
	\end{tcolorbox}	
	
	To finish, let us notice that as the mass and energy are equivalent to a given factor ($c$) , we can obviously also define an energy volumic density, energy surfacic density and energy linear density with also the same three remarks that apply.
	
	\paragraph{Electric Charge}\mbox{}\\\\
	It is difficult to say something about the electric charge (you can search a definition on Internet you'll see ...). However, if we refer to the approach of Hideki Yukawa we can try to give the following definition:
	
	\textbf{Definition (\#\mydef):} The "\NewTerm{electric charge}\index{electric charge}" $q$ seems to be a conservative physical property of matter that causes it to experience a force when placed in an electric or magnetic field. There are two types of electric charges: positive and negative. Like charges repel and unlike attract. An object (one not made of antimatter) is negatively charged if it has an excess of electrons, and is otherwise positively charged or uncharged. The SI derived unit of electric charge is the coulomb [C]. In electrical engineering, it is also common to use the ampere-hour (Ah), and, in chemistry, it is common to use the elementary charge ($e$) as a unit. The symbol $Q$ often denotes multiple electric charges. Early knowledge of how electrically charged substances interact is now named "classical electrodynamics" (\SeeChapter{see section Electrodynamics page \pageref{electrodynamics}}), and is still accurate for problems that don't require consideration of quantum effects (\SeeChapter{see section Quantum Field Theory page \pageref{quantum electrodynamics}}).
	
	More technically... (we refer to what we will see in the section of Quantum Field Theory) an another but still not accurate definition is to say that: An "electric charge" $q$ is a conservative property that has a particle lying in a spherically symmetric potential field to interact with the source of that field in the context of the exchange of a quantum of interaction (the photon is the quantum of this interaction created by the quantum fluctuations of vacuum in the presence of a mass) defining a vector field of the Coulomb type.
	
	Or another definition similar to that of the mass: The "electric charge" $q$ of a body in a closed system is a conservative quantity and that characterizes algebraically the amplitude with which the body interacts with other body through electrostatic and magnetic forces.

	The electric charge is a fundamental conserved property of some subatomic particles, which determines their electromagnetic interaction. Electrically charged matter is influenced by, and produces, electromagnetic fields. The interaction between a moving charge and an electromagnetic field is the source of the electromagnetic force, which is one of the four fundamental forces (See also: magnetic field).
	
	\begin{tcolorbox}[title=Remarks,colframe=black,arc=10pt]
	\textbf{R1.} In an isolated system, globally they cannot be  spontaneous creation or removal of electric charges. The emergence of electric charges can only be due to an external action as far as we know. Another way of saying the same thing - as for the mass - is that the total charge contained in the Universe seems to be constant.\\
		
	\textbf{R2.} The charge is an extensive quantity. Indeed, the total electric charge of a physical system is equal to the \underline{algebraic} sum of the charges which constitutes it.
	\end{tcolorbox}

	Twentieth-century experiments demonstrated that electric charge is quantized; that is, it comes in integer multiples of individual small units named the "elementary charge", $e$, approximately equal to $1.602\cdot 10^{-19}$ [C] (except for particles named quarks, which have charges that are integer multiples of $e/3$). The proton has a charge of $+e$, and the electron has a charge of $-e$. The study of charged particles, and how their interactions are mediated by photons, is named quantum electrodynamics (\SeeChapter{see section Elementary Particle Physics page \pageref{elementary particle physics}}).
	
	As already mentioned, contrary to the mass, there are positive and negative electric charges this is why we use the term "algebraic" interaction. The electric charge remains an additive property (extensive). Thus, for a system of $q_i$ electric charges, the total electric charge is therefore:
	
	The electric charge of a macroscopic object is the sum of the electric charges of the particles that make it up. This charge is often small, because matter is made of atoms, and atoms typically have equal numbers of protons and electrons, in which case - if the distribution of charges can be considered as homogeneous\footnote{For example the disposition of the quarks in the Neutron is not homogeneous this is why even if the Neutron seems... neutral under a strong magnetic field the neutron will have a precession movement and this is why even if the neutron has an electric charge that is to weak to be deviated by a magnetic field it is not small enough to hide the Larmor effect} - their charges cancel out, yielding a net charge of zero, thus making the atom neutral.
	
	Similarly as for the mass and the energy, for a total continuous volumic distribution of the electric charge of a system of volume $V$ (we denote the electric charge density in an identical way as that of the mass if the ambiguity is not possible):
	
	where $\rho_V(A)$ is the "\NewTerm{volumetric electric charge density}\index{volumetric electric charge density}" of the system at point $A$, that is to say the electric charge of a material element, centered around $A$, of characteristic dimensions in comparison to that of the system, but large in comparison of the interatomic distances in this system ($\rho_V(\vec{r})$ is the volumic density of electric charges of the system to the point indicated by $\vec{r}=\overrightarrow{\text{O}A}$) defined by:
	
	with also the same three remarks that apply.
	
	As for the mass, we can give the following definitions:
	
	\textbf{Definitions (\#\mydef):}
	\begin{enumerate}
		\item[D1.] We say that a system is an "\NewTerm{electric homogeneous system}\index{electric homogeneous system}" if its volumic, surfacic, linear charge densities (see definition below) are constant on any small element of volume that constitute it.

		\item[D2.] We say that a system is an "\NewTerm{electric isotropic system}\index{electric isotropic system}", if its electric properties are identical in all points and directions.
	\end{enumerate}
	We also define sometimes the "\NewTerm{electric charge surfacic density}\index{electric charge surfacic density}" (or simply "\NewTerm{surfacic density}\index{surfacic density}") for systems with virtually no thickness and a "\NewTerm{electric charge linear density}" (or  simply "\NewTerm{linear density}\index{linear density}" of mass) for systems with also negligible relatively to their length. We then have ($S$ being a surface and $s$ curvilinear abscissa):
	
	with in the general case:
	
	If the system is electrically homogeneous we have again obviously with the traditional notation:
	
	\textbf{Definition (\#\mydef):} With the above, we can define the "\NewTerm{electric density}\index{electric density}" as the number of identical elements and all countable per unit volume, or unit surface or unit linear.
	
	The electric charge density of a material varies, as for the mass, with temperature and pressure. This variation is typically small for solids and liquids but much greater for gases. Increasing the pressure on an object decreases the volume of the object and thus increases its density. Increasing the temperature of a substance (with a few exceptions) decreases its density by increasing its volume. In most materials, heating the bottom of a fluid results in convection of the heat from the bottom to the top, due to the decrease in the density of the heated fluid. This causes it to rise relative to more dense unheated material.

	\pagebreak
	\subsubsection{Scientific Notation and Metric Prefixes}
	Ok just now we have defined the concept of quantity/magnitude and associated main units as used by the physicists and engineers. But what about the scientific notation for big numbers (magnitudes) and associated metric prefixes??? Let us see this now!
	
	Scientific notation (also referred to as standard form or standard index form) is a way of expressing numbers that are too big or too small to be conveniently written in decimal form. It is commonly used by scientists, mathematicians and engineers. On scientific calculators it is known as "SCI" display mode.
	
	Indeed, it is common in physics that the manipulated variables are very large and heavy to write. For example, it is always a hassle to have magnitudes as $8,000,000,000$ or $0.000,000,000,1$.
	
	In scientific notation all numbers are written in the form:
	
	($m$ times ten raised to the power of $n$), where the exponent $n$ is an integer, and the coefficient $m$ is any real number, named the "\NewTerm{significand}\index{significand}" or "\NewTerm{mantissa}\index{mantissa}".
	
	\textbf{Definition (\#\mydef):} We say that $m\cdot 10^n$ is "\NewTerm{scientific notation}\index{scientific notation}" of a given number if $-10 < m <10$ (that is to say that $m$ is written with a single digit other than zero before the decimal point) and if $n$ is a signed integer.
	
	\begin{tcolorbox}[title=Remark,colframe=black,arc=10pt]
	"\NewTerm{Engineering notation}\index{engineering notation}" (often named "ENG" display mode on scientific calculators) differs from normalized scientific notation in that the exponent $n$ is restricted to multiples of $3$.
	\end{tcolorbox}
	
	\begin{tcolorbox}[colframe=black,colback=white,sharp corners]
	\textbf{{\Large \ding{45}}Examples:}\\\\
	E1. The number $8,000,000,000$ will be written in scientific and engineer notation as:
	
	That is to say $9$ zeros after the eight.\\
	
	E2. The number $0.000,000,000,1$ will be written in scientific notation:
	
	That is to say $10$ zeros before the one. Or in engineering notation:
	
	\end{tcolorbox}
	The advantage of the scientific notation is to give an order of magnitude of a number between $2$ consecutive powers of $10$ as:
	
	 
	\begin{tcolorbox}[title=Remarks,colframe=black,arc=10pt]
	\textbf{R1.} If we have a number like $154,434,347,786$, frequently and depending on the context, we allow ourselves to truncate it and then we write frequently with a precision to three decimal places and the latter number is becomes $154.434\cdot 10^9$ in engineering notation which is easier to write but dangerous to handle because of the error induced by the truncation. We refer the reader to the subject of Error Propagation in the section Statistics.\\
	
	\textbf{R2.} For the mathematicians the scientific notation is only one writing of a number among many and the choice of this writing is related to the context of the problem. Obviously these "results numbers " obtained may be simple numbers coming from abstract but also practical problems from experiences, measurement operations, etc. and here we will agree with the way of physicists.
	\end{tcolorbox}	
	A "\NewTerm{metric prefix}\index{metric prefix}" is a unit prefix that precedes a basic unit of measure to indicate a multiple or fraction of the unit.
	
	Decimal multiplicative prefixes have been a feature of all forms of the metric system with six dating back to the system's introduction in the 1790s.
	
	Each prefix has a unique symbol that is pretended to the unit symbol. The prefix kilo-, for example, may be added to gram to indicate multiplication by one thousand: one kilogram is equal to one thousand grams. The prefix milli-, likewise, may be added to meter to indicate division by one thousand; one millimeter is equal to one thousandth of a meter.
	
	The prefix and corresponding magnitudes are given in the table below:
	\begin{table}[H]
		\begin{center}
			\definecolor{gris}{gray}{0.85}
				\begin{tabular}{|c|c|c|}
					\hline
					\multicolumn{1}{c}{\cellcolor{black!30}\textbf{Prefix}} & 
	  \multicolumn{1}{c}{\cellcolor{black!30}\textbf{Factor}} & \multicolumn{1}{c}{\cellcolor{black!30}\textbf{Symbol}} \\ \hline
			yotta & $10^{24}$ & Y \\ \hline
			zetta & $10^{21}$ & Z \\ \hline
			exa & $10^{18}$ & E \\ \hline
			peta & $10^{15}$ & P \\ \hline
			tera & $10^{12}$ & T \\ \hline
			giga & $10^9$ & G \\ \hline
			mega & $10^6$ & M \\ \hline
			kilo & $10^3$ & k \\ \hline
			hecto & $10^2 $ & h \\ \hline
			deca & $10^1$ & da \\ \hline
			deci & $10^{-1}$ & d \\ \hline
			centi & $10^{-2}$ & c \\ \hline
			milli & $10^{-3}$ & m \\ \hline
			micro & $10^{-6}$ & $\mu$ \\ \hline
			nano & $10^{-9}$ & n \\ \hline
			pico & $10^{-12}$ & p \\ \hline
			femto & $10^{-15}$ & f \\ \hline
			atto & $10^{-18}$ & a \\ \hline
			zepto & $10^{-21}$ & z \\ \hline
			yocto & $10^{-24}$ & y \\ \hline
			\end{tabular}
		\end{center}
		\caption{Metric prefix}
	\end{table}
	For example, $10,000,000$ grams  conventionally denoted by:
	
	will be written in scientific notation:
	
	but with a physicist notation (according to the previous table):	
	
	
	\pagebreak
	\subsubsection{Scales of Measurements}
	As the main part of physics is understanding our surrounding it seemed important to us to give to the reader some scales of measurements of typical quantities. We own the tables below to William Nielsen Brandt.
	
	Some pressures values first:
	\begin{table}[H]
	\begin{center}
		\begin{tabular}{|p{5cm}|c|c|c|}
		  \hline
		  \rowcolor[gray]{0.75}\textbf{Place} & \textbf{Average pressure}& \textbf{Mean free path} & \textbf{Molecules by [cm${}^3$]} \\ \hline
		Rough neutron star pressure 
 & $1\cdot 10^{34}$ [Pa] &  &  \\ \hline		
		Rough white dwarf pressure
 & $1\cdot 10^{23}$ [Pa] &  &  \\ \hline
		Rough Sun central pressure 
 & $27,000$ [GPa] &  &  \\ \hline		
		Central pressure of the Earth
 & $100$ [GPa] &  &  \\ \hline	
		Pressure needed for natural diamonds crystallization  & $10$ [GPa] &  &  \\ \hline	
		Conventional high pressure laboratory press
 & $6$ [GPa] &  &  \\ \hline			
		Pressure at the bottom of the Marianas trench & $110$ [MPa] &  &  \\ \hline		
		Karate fist peak pressure & $90$ [MPa] &  &  \\ \hline						
		Typical scuba tank pressure & $20$ [MPa] &  &  \\ \hline		
		High pressure bicycle tire & $1.5$ [MPa] &  &  \\ \hline		
		Standard atmosphere & $\sim 100$ [kPa] & $\sim  66$ [nm] & $\sim  2.5\cdot 10^{19}$ \\ \hline
		Vaccum cleaner & $\sim 88$ [kPa] & $\sim 70$ [nm] & $\sim 10^{19}$ \\ \hline
		Steam turbine & $\sim 9$ [kPa] & &  \\ \hline
		Standing person & $15$ [Pa]& &  \\ \hline
		Human arterial blood & $10$ [Pa]& &  \\ \hline		
		Mars & $\sim 0.6$ [kPa]& &  \\ \hline
		Freeze Dryer & $\sim 50$ [Pa] & $\sim 500$ [$\mu$m] & $\sim 10^{16}$ \\ \hline
		Pressure of a sound  at the human threshold of pain & $30$ [Pa] &  &  \\ \hline
		Radiation pressure at the surface of the Sun & $0.2$ [Pa] &  &  \\ \hline
		Thermos bottle & $1$ to $0.01$ [Pa] & $\sim 1$ [m] & $\sim 10^{13}$ \\ \hline
		Vacuum tube & $0.1$ [$\mu$Pa] & $\sim 1,000$ [km] & $\sim 10^{7}$ \\ \hline
		Human threshold of hearing & $0.01$ [$\mu$Pa] &  &  \\ \hline
		Cryopump & $0.01$ [$\mu$Pa] & $\sim 10,000$ [km] & $\sim 10^{6}$ \\ \hline
		Solar radiation pressure & $0.005$ [$\mu$Pa] &  &  \\ \hline
		Moon & $ 0.1$ [nPa] & $\sim 10,000$ [km] & $\sim 10^{5}$ \\ \hline
		Interplanetary space & $0.0007$ [nPa] & & $\sim 11$ \\ \hline
		Best vacuum performed on Earth & $0.005$ [nPa] & &  \\ \hline
		Interstellar space & & & $\sim 1$ \\ \hline
		Intergalactic space & & & $\sim 0.000001$ \\ \hline
		\end{tabular}
	\end{center}
	\caption{Typical vacuum values and mean free paths}
	\end{table}	
	\begin{tcolorbox}[title=Remark,colframe=black,arc=10pt]
	It is common to define the unit of $1$ [bar] as the atmospheric pressure at sea level on Earth equal by definition to $100,000$ [Pa], but this unit is not approved as part of the International System of Units.
	\end{tcolorbox}
	
	
	And some time scales:
	\begin{table}[H]
	\begin{center}
		\begin{tabular}{|l|p{12cm}|}
		  \hline
		  \rowcolor[gray]{0.75}\textbf{Magntitude [s]} & \textbf{Description}  \\ \hline
		$5\cdot 10^{-44}$ & Planck-Wheeler time$\ = \bigl({G \hbar \over c^5}\bigr)^{1 \over 2}$\\ \hline
		$4\cdot 10^{-24}$ & Typical lifetime of strong interaction resonance$\ = {h\over m_{\rm p} c^2}$\\ \hline
		$1\cdot 10^{-13}$  & Typical period of vibration of an atom in a solid\\ \hline
		$1\cdot 10^{-13}$ & Typical X-ray line electric dipole radiative transition time\\ \hline
		$1.6\cdot 10^{-9}$ & Typical hydrogen 2p$\rightarrow$1s radiative transition time (electric dipole one photon process)\\ \hline
		$8\cdot 10^{-4}$ & Mass shedding minimum spin period for a neutron star\\ \hline
		$1.6\cdot 10^{-3}$  & Spin period of PSR 1957+20\\ \hline
		$0.12$   & Typical hydrogen 2s$\rightarrow$1s radiative transition time\\ \hline
		$10$   & Median duration of a classical $\gamma$-ray burst\\ \hline
		$887$  & Mean life of a neutron in free space\\ \hline
		$2,000$  & Sun dynamic time scale\\ \hline
		$8.6\cdot 10^{4}$  & Earth rotation time\\ \hline
		$3.2\cdot 10^{7}$   & Earth orbit time around the Sun\\ \hline
		$1.6\cdot 10^{9}$  & Typical time between Milky Way supernovae\\ \hline
		$1.9\cdot 10^{11}$  & Carbon-14 half-life\\ \hline
		$3\cdot 10^{13}$  & Rough time for evolution of a biological species\\ \hline
		$3\cdot 10^{14}$  & Rough Lyapunov time of the solar system\\ \hline
		$6.3\cdot 10^{14}$ & Sun thermal time scale\\ \hline
		$2\cdot 10^{15}$  & Timescale for Los Angeles to pass San Francisco via continental drift\\ \hline
		$7.3\cdot 10^{15}$  & Orbit time for sun around galaxy center\\ \hline
		$2\cdot 10^{16}$  & Rough supernova biological extinction time\\ \hline
		$6\cdot 10^{16}$   & Time for galaxy to cross a cluster\\ \hline
		$1.1\cdot 10^{17}$  & Primeval slime to man time\\ \hline
		$1.5\cdot 10^{17}$  & Age of Earth and Sun\\ \hline
		$1.5\cdot 10^{17}$  & Uranium-238 half-life\\ \hline
		$2.7\cdot 10^{17}$  & Look back time to $z=1$\\ \hline
		$3\cdot 10^{17}$ & Main sequence lifetime for a 1 M$_\odot$ star\\ \hline
		$3.3\cdot 10^{17}$  & Look back time to $z=2$\\ \hline
		$3.3\cdot 10^{17}$  & Sun nuclear time scale\\ \hline
		$3.7\cdot 10^{17}$  & Look back time to $z=4$\\ \hline
		$3.8\cdot 10^{17}$  & Rough age of the Milky Way\\ \hline
		$4.1\cdot 10^{17}$ & Age of the universe $= {2\over 3 H_0}$\\ \hline
		$1\cdot 10^{39}$   & Lower limit on the proton lifetime\\ \hline
		$4.7\cdot 10^{73}$  & 1 M$_\odot$ Hawking's Black Hole\\ \hline
		\end{tabular}
	\end{center}
	\caption{Typical time scales}
	\end{table}
	
	And some length scales:
	\begin{table}[H]
	\begin{center}
		\begin{tabular}{|l|p{12cm}|}
		  \hline
		  \rowcolor[gray]{0.75}\textbf{Magntitude [m]} & \textbf{Description}  \\ \hline
			$1.6\cdot 10^{-35}$ & Planck length\\ \hline
			$2\cdot 10^{-35}$  & Rough postulated superstring size\\ \hline
			$1.5\cdot 10^{-18}$  & Classical proton radius\\ \hline
			$1.6\cdot 10^{-17}$   & Rough weak force length\\ \hline
			$4\cdot 10^{-17}$  & LIGO 4 km gravity-wave detector needed sensitivity\\ \hline
			$1.44\cdot 10^{-15} A^{1\over 3}$ & Nuclear radius\\ \hline
			$2.8\cdot 10^{-15}$ & Classical electron radius\\ \hline
			$8.8\cdot 10^{-15}$ & Attractive strong force length\\ \hline
			$5.3\cdot 10^{-11}$ & Bohr radius (electron-proton distance)\\ \hline
			$2.6\cdot 10^{-10}$  & Copper atom spacing in solid copper\\ \hline
			$3.5\cdot 10^{-10}$  & H$_2$O molecular diameter\\ \hline
			$4\cdot 10^{-10}$ & ROSAT X-ray satellite mirror rms surface error\\ \hline
			$\sim 3\cdot 10^{-9}$ & Mean nucleon spacing on primordial nucleosynthesis, DNA turn length\\ \hline
			$1\cdot 10^{-7}$  & Typical size of a virus, Interstellar dust  size\\ \hline
			$4\cdot 10^{-6}$  & Typical size of a cell\\ \hline
			$2\cdot 10^{-4}$  & Small dust particle size\\ \hline
			$0.5-2.5$  & Human Adult tall range\\ \hline
			$3,700$  & Mean ocean depth\\ \hline
			$5,500$  & Rough radius of Halley's comet\\ \hline
			$8,847$   & Height of Mount Everest\\ \hline
			$10,000$   & Typical Neutron star, asteroid or comet radius \\ \hline
			$11,032$   & Mariana  Trench depth's, Troposphere height, Airliner cruising altitude\\ \hline
			$30,000$   & Typical thickness of the Earth's crust\\ \hline
			$3.2\cdot 10^{6}$   & Length of the Great Wall of China\\ \hline
			$6.3\cdot 10^{6}$   & Radius of the Earth\\ \hline
			$4.2\cdot 10^{7}$  & Geostationary satellite orbit height\\ \hline
			$7.1\cdot 10^{7}$ & Radius of Jupiter\\ \hline
			$9\cdot 10^{7}$  & Distance to the Earth's solar wind bow shock\\ \hline
			$3.8\cdot 10^{8}$  & Distance to the Moon\\ \hline
			$7.0\cdot 10^{8}$  & Radius of the Sun\\ \hline
			$1.50\cdot 10^{11}$  & Earth/Sun mean distance\\ \hline
			$5\cdot 10^{11}$  & Radius of the 20 M$_\odot$ red supergiant Betelgeuse at maximum light\\ \hline
			$5.91\cdot 10^{12}$  & Pluton/Sun mean distance\\ \hline
			$4\cdot 10^{14}$ & Rough stellar separation in a globular cluster\\ \hline
			$5\cdot 10^{17}$ & Typical interstellar medium cloud size\\ \hline
			$2\cdot 10^{18}$  & Rough radius of the local interstellar hot gas bubble\\ \hline
			$5\cdot 10^{18}$ & Rough height of Milky Way, Distance to  Betelgeuse supergiant\\ \hline
			$2\cdot 10^{19}$ & Characteristic height of the Milky Way main disk\\ \hline
			$2.4\cdot 10^{20}$ & Distance from Sun to galactic center\\ \hline
			$10^{22}$  & Typical active galaxy jet length and rough local group radius\\ \hline
			$7\cdot 10^{23}$ & Distance to the center of the Virgo cluster of galaxies\\ \hline
			$3\cdot 10^{25}$ & Schwarzschild radius of a singularity with Universe critical density\\ \hline
	\end{tabular}
	\end{center}
	\caption{Typical distances scales}
	\end{table}
	
	And some velocities scales:
	\begin{table}[H]
	\begin{center}
		\begin{tabular}{|l|p{12cm}|}
		  \hline
		  \rowcolor[gray]{0.75}\textbf{Magntitude [ms$^{-1}$]} & \textbf{Description}  \\ \hline
				$1\cdot 10^{-9}$  & Sea floor spreading rate\\ \hline
				$1.6\cdot 10^{-9}$ & Average slip rate of the San Andreas fault\\ \hline
				$1\cdot 10^{-8}$  & Typical rainfall rate in a semi-arid climate\\ \hline
				$2\cdot 10^{-8}$  & Grass growth rate\\ \hline
				$3\cdot 10^{-6}$  & Typical glacial advance rate\\ \hline
				$1\cdot 10^{-3}$  & Equivalent radial velocity resolution of pulsar pulse arrival time analysis\\ \hline
				$1.3$              & Human walking speed\\ \hline
				$3$                & Radial velocity accuracy of high precision Doppler spectroscopy\\ \hline
				$13$               & Speed of the reflex motion induced on the Sun by Jupiter\\ \hline
				$25$               & Car speed\\ \hline
				$100$              & Typical speed of an electric pulse in the nervous system\\ \hline
				$330$              & Sound speed in air\\ \hline
				$480$              & Earth's atmosphere molecular rms velocity\\ \hline
				$600$              & Fighter jet speed\\ \hline
				$2\, 380$          & Escape velocity from Moon's surface\\ \hline
				$10\, 000$         & Typical longitudinal seismic wave velocity in the Earth's mantle\\ \hline
				$11\, 000$         & Escape velocity from the Earth's surface\\ \hline
				$20\, 000$         & Globular cluster stellar velocity dispersion\\ \hline
				$29\, 000$         & Earth's motion around the Sun\\ \hline
				$40\, 000$         & Globular cluster stellar escape velocity\\ \hline
				$1\cdot 10^{5}$   & Typical Galactic pulsar vertical velocity component\\ \hline
				$1\cdot 10^{5}$   & Average speed of the initial stroke of a lightning flash\\ \hline
				$2.2\cdot 10^{5}$ & Rotational velocity of the Sun around the Milky Way's center\\ \hline
				$3\cdot 10^{5}$   & Orbital speed of PSR 1913+16\\ \hline
				$3\cdot 10^{5}$   & Rough velocity of Geminga's proper motion\\ \hline
				$3.65\cdot 10^{5}$ & Motion of the solar system barycenter relative to the cosmic microwave background\\ \hline
				$6.2\cdot 10^{5}$ & Escape velocity from the Sun's surface\\ \hline
				$6.2\cdot 10^{5}$ & Escape velocity from the Milky Way for objects in the solar neighborhood (ARAA 29, 429)\\ \hline
				$6.22\cdot 10^{5}$  & Motion of the Local Group relative to the cosmic microwave background\\ \hline
				$8\cdot 10^{5}$   & Typical galaxy cluster galaxy velocity dispersion\\ \hline
				$2\cdot 10^{6}$   & Speed of $n=1$ hydrogen electron = $Z\alpha c$\\ \hline
				$5\cdot 10^{6}$   & Young (months old) supernova ejecta\\ \hline
				$1.2\cdot 10^{7}$ & Velocity of the wind streaming out from H1413+117 ``Cloverleaf'' ({\it Nature\/} 371, 559)\\ \hline
				$7.8\cdot 10^{7}$ & SS433 jet speed\\ \hline
				$1.4\cdot 10^{8}$ & Keplerian orbital velocity at the surface of a neutron star\\ \hline
				$2\cdot 10^{8}$   & Escape velocity from neutron star surface\\ \hline
				$2.998\cdot 10^{8}$  & Light in a vacuum\\ \hline
				$2.7\cdot 10^{10}$  & Apparent superluminal motion of the jet from the $z=0.940$ BL Lac AO 0235+164 (may be lensed)\\ \hline
	\end{tabular}
	\end{center}
	\caption{Typical velocities scales}
	\end{table}
	
	And some masses scales:
	\begin{table}[H]
	\begin{center}
		\begin{tabular}{|l|p{12cm}|}
		  \hline
		  \rowcolor[gray]{0.75}\textbf{Magntitude [kg]} & \textbf{Description}  \\ \hline
				$4.2\cdot 10^{-36}$     & Mass equivalent of a green light photon\\ \hline
				$9.1\cdot 10^{-36}$     & Electron antineutrino upper mass limit\\ \hline
				$4.8\cdot 10^{-31}$     & Muon neutrino upper mass limit\\ \hline
				$9.11\cdot 10^{-31}$    & Electron mass\\ \hline
				$5.5\cdot 10^{-29}$     & Tau neutrino upper mass limit\\ \hline
				$1.67\cdot 10^{-27}$    & Proton mass\\ \hline
				$9\cdot 10^{-27}$       & Bottom quark mass\\ \hline
				$4.8\cdot 10^{-26}$     & Mean mass of atmosphere molecule\\ \hline
				$1.4\cdot 10^{-25}$     & $W^\pm$ mass ($80.22\pm 0.22$ GeV/$c^2$)\\ \hline
				$1.6\cdot 10^{-25}$     & $Z^0$ mass ($91.187\pm 0.007 $ GeV/$c^2$)\\ \hline
				$2\cdot 10^{-25}$       & Favored Higgs boson mass\\ \hline
				$3.1\cdot 10^{-25}$     & Top quark mass ($176\pm 13$ GeV/$c^2$)\\ \hline
				$4\cdot 10^{-25}$       & DNA nucleotide\\ \hline
				$1\cdot 10^{-22}$       & Typical protein molecule mass\\ \hline
				$5\cdot 10^{-21}$       & {\it E. Coli\/} ribosome\\ \hline
				$1\cdot 10^{-16}$       & Interstellar dust grain mass\\ \hline
				$8\cdot 10^{-15}$       & Rough mass of a human DNA molecule\\ \hline
				$7\cdot 10^{-13}$       & Typical mass of a cell\\ \hline
				$2.2\cdot 10^{-8}$      & Planck mass\\ \hline
				$0.02$                   & Typical goldfish mass\\ \hline
				$2-150$                     & Typical human mass\\ \hline
				$500-2,500$                & Typical car mass\\ \hline
				$10,000$               & Tyrannosaurus Rex\\ \hline
				$1\cdot 10^{13}$        & Typical comet/mountain mass\\ \hline
				$5.3\cdot 10^{18}$      & Total mass of Earth's atmosphere\\ \hline
				$3\cdot 10^{19}$        & Typical asteroid mass\\ \hline
				$1.4\cdot 10^{21}$      & Total mass of Earth's oceans\\ \hline
				$7.3\cdot 10^{22}$      & Mass of the Moon\\ \hline
				$5.98\cdot 10^{24}$     & Mass of the Earth\\ \hline
				$1.9\cdot 10^{27}$      & Mass of Jupiter\\ \hline
				$1.6\cdot 10^{29}$      & Minimum mass to fusion burn hydrogen\\ \hline
				$1.99\cdot 10^{30}$     & Mass of the Sun\\ \hline
				$2.8\cdot 10^{30}$      & Chandrasekhar mass (maximum mass for a white dwarf)\\ \hline
				$6.0\cdot 10^{30}$      & Oppenheimer-Volkoff mass (maximum mass for a neutron star)\\ \hline
				$4\cdot 10^{31}$        & Rough stellar mass above which the  endpoint is a black hole\\ \hline
				$1.2\cdot 10^{32}$      & Rough mass at which a star becomes unstable to pulsations\\ \hline
				$2\cdot 10^{33}$        & Typical interstellar cloud mass\\ \hline
				$1\cdot 10^{36}$        & Typical mass of a globular cluster\\ \hline
				$5\cdot 10^{36}$        & Approximate mass of the Milky Way central black hole\\ \hline
				$1.4\cdot 10^{49}$      & Rough total mass in spiral galaxies\\ \hline
				$3\cdot 10^{49}$        & Rough total mass in elliptical and spheroidal galaxies\\ \hline
				$8\cdot 10^{49}$        & Rough total mass of visible matter in the universe\\ \hline
				$2\cdot 10^{52}$        & Rough total mass of a critical density universe\\ \hline
	\end{tabular}
	\end{center}
	\caption{Typical masses scales}
	\end{table}
	And some densities scales:
	\begin{table}[H]
	\begin{center}
		\begin{tabular}{|l|p{12cm}|}
		  \hline
		  \rowcolor[gray]{0.75}\textbf{Magntitude [kg$\cdot$ m$^{-3}$]} & \textbf{Description}  \\ \hline
				$2\cdot 10^{-38}$    & Effective density of the 100-300 MHz radio background\\ \hline
				$1\cdot 10^{-35}$    & Effective density of the 1--10 MeV $\gamma$-ray background\\ \hline
				$8\cdot 10^{-35}$    & Effective density of the 2--100 keV X-ray background\\ \hline
				$1.1\cdot 10^{-33}$  & Upper limit to the effective density of the gravitational wave background\\ \hline
				$1\cdot 10^{-32}$    & Effective density of the starlight released in a Hubble time\\ \hline
				$4.6\cdot 10^{-31}$  & Effective density of the cosmic microwave background radiation\\ \hline
				$2\cdot 10^{-29}$    & Smoothed density of visible galactic material throughout universe\\ \hline
				$2\cdot 10^{-28}$    & Smoothed baryon density predicted by primordial nucleosynthesis\\ \hline
				$4.7\cdot 10^{-27}$  & Critical density of the universe $= {3 H_0^2\over 8 \pi G}$\\ \hline
				$2\cdot 10^{-24}$    & Typical gas in a cluster of galaxies\\ \hline
				$3\cdot 10^{-21}$    & Typical gas in the interstellar medium of the Milky Way\\ \hline
				$7\cdot 10^{-21}$    & Dynamically inferred Milky Way disk density\\ \hline
				$5\cdot 10^{-20}$    & Typical density of the gas in the central kiloparsec of an interacting or starburst galaxy\\ \hline
				$1\cdot 10^{-9}$     & Best room temperature vacuum achieved on Earth\\ \hline
				$1.7\cdot 10^{-4}$   & Mean density of Antares (19 M$_\odot$)\\ \hline
				$1.3$                 & Density of air\\ \hline
				$700$                 & Mean density of Saturn\\ \hline
				$1\, 000$             & Density of water\\ \hline
				$1\, 300$             & Mean density of Jupiter\\ \hline
				$1\, 400$             & Mean density of the Sun\\ \hline
				$3\, 300$             & Mean density of the Moon\\ \hline
				$5\, 500$             & Mean density of the Earth\\ \hline
				$7\, 860$             & Density of iron\\ \hline
				$19\, 300$            & Density of gold\\ \hline
				$5\cdot 10^{7}$      & Typical white dwarf mean density\\ \hline
				$3\cdot 10^{10}$     & Typical white dwarf central density\\ \hline
				$1.1\cdot 10^{12}$   & Inverse $\beta$ decay threshold\\ \hline
				$4.3\cdot 10^{14}$   & Neutron drip density\\ \hline
				$6\cdot 10^{17}$     & Nuclear density\\ \hline
				$1\cdot 10^{18}$     & Typical neutron star central density\\ \hline
				$5\cdot 10^{96}$     & Planck-Wheeler density, at which quantum gravitational effects become important$\ = {c^5\over G^2 \hbar}$\\ \hline
	\end{tabular}
	\end{center}
	\caption{Typical densities scales}
	\end{table}
	
	And some energies scales:
	\begin{table}[H]
	\begin{center}
		\begin{tabular}{|l|p{12cm}|}
		  \hline
		  \rowcolor[gray]{0.75}\textbf{Magntitude [J]} & \textbf{Description}  \\ \hline
				$4\cdot 10^{-21}$                        & $kT_{\rm room}$\ = translational kinetic energy of atmosphere gas molecule\\ \hline
				$7\cdot 10^{-21}$                        & Donor level/conduction band gap in a doped semiconductor\\ \hline
				$3\cdot 10^{-20}$                        & Molecular vibration transition\\ \hline
				$1.6\cdot 10^{-19}$                      & Valence band/conduction band gap in a semiconductor\\ \hline
				$3.8\cdot 10^{-19}$                      & Green light photon\\ \hline
				$9\cdot 10^{-19}$                        & Valence band/conduction band gap in an insulator\\ \hline
				$1.1\cdot 10^{-18}$                      & Fermi energy in copper = depth of Fermi sea\\ \hline
				$2.2\cdot 10^{-18}$                      & Hydrogen $n=1$ binding energy\\ \hline
				$1.6\cdot 10^{-16}$                      & 1 keV X-ray\\ \hline
				$1.5\cdot 10^{-15}$                      & Hydrogenic iron $n=1$ binding energy\\ \hline
				$8.18\cdot 10^{-14}$                     & Electron rest mass\\ \hline
				$1.6\cdot 10^{-13}$                      & 1 MeV $\gamma$-ray\\ \hline
				$1.6\cdot 10^{-13}\ (Z_1 \cdot Z_2)$    & Coulomb barrier height\\ \hline
				$1.3\cdot 10^{-12}$                      & Nucleon binding energy\\ \hline
				$4.3\cdot 10^{-12}$                      & Energy from $4\cdot(^1$H$)\rightarrow\ ^4$He\\ \hline
				$1.50\cdot 10^{-10}$                     & Proton rest mass\\ \hline
				$1.6\cdot 10^{-7}$                       & Particle kinetic energy in a 1 TeV accelerator\\ \hline
				$10$                                      & Well hit tennis ball\\ \hline
				$2\cdot 10^{5}$                          & Energy from a light bulb burning for 1 hour\\ \hline
				$2\cdot 10^{7}$                          & Rough energy from a 1 kilogram meal\\ \hline
				$4.2\cdot 10^{9}$                        & Explosion energy of 1 ton of TNT\\ \hline
				$2\cdot 10^{11}$                         & Rough human total energy output in a lifetime\\ \hline
				$1.5\cdot 10^{14}$                       & Typical atomic bomb explosion energy\\ \hline
				$6\cdot 10^{16}$                         & Tunguska 50 m diameter meteorite impact energy\\ \hline
				$1\cdot 10^{17}$                         & Powerful H-bomb explosion energy\\ \hline
				$3\cdot 10^{17}$                         & Elastic wave energy release from a large $(M=8.5)$ earthquake\\ \hline
				$9\cdot 10^{18}$                         & USA electricity usage in 1986\\ \hline
				$5\cdot 10^{19}$                         & Rough explosion energy of the Lake Toba eruption in Sumatra\\ \hline
				$2.5\cdot 10^{22}$                       & Shoemaker-Levy fragment G impact on Jupiter\\ \hline
				$4\cdot 10^{23}$                         & K-T 10 km diameter meteorite impact energy\\ \hline
				$1\cdot 10^{24}$                         & Power released by an evaporating black hole the last second of its life\\ \hline
				$2.1\cdot 10^{29}$                       & The Earth's rotational energy\\ \hline
				$3\cdot 10^{31}$                         & The Earth's total heat content\\ \hline
				$7\cdot 10^{34}$                         & Total rotational energy of the planets\\ \hline
				$2.5\cdot 10^{35}$                       & Rotational energy of the Sun\\ \hline
				$3\cdot 10^{36}$                         & Gravitational internal binding energy of Jupiter\\ \hline
				$3\cdot 10^{41}$                         & A 450 km s$^{-1}$ neutron star kick\\ \hline
				$1.3\cdot 10^{44}$                       & Total radiant energy from the Sun --- $({1\over 10})(0.007)$M$_\odot c^2$\\ \hline
				$3\cdot 10^{44}$                         & Energy in photons from a type II supernova explosion\\ \hline
				$1\cdot 10^{45}$                         & Rough total energy from a cosmological $\gamma$-ray burst\\ \hline  
				$3\cdot 10^{46}$                         & Energy in neutrinos from a type II supernova explosion\\ \hline
				$2\cdot 10^{53}$                         & Typical gravitational binding energy of a galaxy\\ \hline
				$2\cdot 10^{54}$                         & Typical magnetic and kinetic energy in a large radio lobe (Shu 311)\\ \hline
				$5.2\cdot 10^{54}$                       & Rotational energy of a $10^8$ M$_\odot$ maximal Kerr black hole $=0.29$M$_\bullet$ $c^2$\\ \hline
				$5\cdot 10^{57}$                         & Typical gravitational binding energy of a cluster of galaxies\\ \hline
	\end{tabular}
	\end{center}
	\caption{Typical energy scales}
	\end{table}
	
	And some powers scales:
	\begin{table}[H]
	\begin{center}
		\begin{tabular}{|l|p{12cm}|}
		  \hline
		  \rowcolor[gray]{0.75}\textbf{Magntitude [W]} & \textbf{Description}  \\ \hline
				$4\cdot 10^{-6}$     & Average energy output from 1 kilogram of the Milky Way\\ \hline
				$1\cdot 10^{-3}$     & Optical disc player laser\\ \hline
				$6$                   & Amateur short wave radio transmitter\\ \hline
				$60$                  & Light bulb\\ \hline
				$100$                 & Gravitational wave power from the Earth-Sun system\\ \hline
				$150$                 & Human being under normal conditions\\ \hline
				$750$                 & Maximum long duration horse output\\ \hline
				$1\, 500$             & Typical fireplace fire\\ \hline
				$20\, 000$            & Car\\ \hline
				$1\cdot 10^{5}$      & Running Tyrannosaurus Rex\\ \hline
				$3\cdot 10^{8}$      & Nuclear power reactor\\ \hline
				$3\cdot 10^{8}$      & Moderate thunderstorm electrical power generation rate\\ \hline
				$1.3\cdot 10^{9}$    & Hoover dam\\ \hline
				$3\cdot 10^{11}$     & USA average electricity usage rate in 1986\\ \hline
				$1\cdot 10^{13}$     & Solar radio luminosity\\ \hline
				$8\cdot 10^{13}$     & Powerful nanosecond pulse laser\\ \hline
				$1.7\cdot 10^{17}$   & Insolation of Earth\\ \hline
				$1\cdot 10^{20}$     & X-ray luminosity of the quiet Sun\\ \hline
				$1\cdot 10^{23}$     & Typical white dwarf luminosity\\ \hline
				$8\cdot 10^{24}$     & Gravitational wave radiation from PSR 1913+16\\ \hline
				$3.9\cdot 10^{26}$   & Solar luminosity\\ \hline
				$1\cdot 10^{28}$     & X-ray luminosity associated with Sgr A$^{\ast}$\\ \hline
				$3.5\cdot 10^{28}$   & Rotational energy loss rate from the Geminga pulsar\\ \hline
				$4.3\cdot 10^{28}$   & Megamaser in NGC 4258\\ \hline
				$2.4\cdot 10^{29}$   & 5 M$_\odot$ star on main sequence\\ \hline
				$4\cdot 10^{30}$     & Cygnus X-1 X-ray luminosity\\ \hline
				$5.5\cdot 10^{30}$   & Luminosity of the 20 M$_\odot$ red supergiant Betelgeuse at maximum light\\ \hline
				$1\cdot 10^{31}$     & Crab Nebula energy output\\ \hline
				$2\cdot 10^{31}$     & Eddington limit for a 1.4 M$_\odot$ neutron star\\ \hline
				$3\cdot 10^{31}$     & Typical luminosity of GRS 1915+105 in outburst ({\it Nature\/} 371, 46)\\ \hline
				$1\cdot 10^{33}$     & Rough luminosity of Eta Carinae in April 1843\\ \hline
				$5\cdot 10^{35}$     & Type II supernova peak photon luminosity\\ \hline
				$3\cdot 10^{36}$     & Milky Way\\ \hline
				$1.5\cdot 10^{38}$   & Coma cluster X-ray gas luminosity\\ \hline
				$1\cdot 10^{39}$     & $10^8$ M$_\odot$ black hole accreting at ${1\over 10}$ of the Eddington limit\\ \hline
				$1\cdot 10^{39}$     & Typical quasar luminosity\\ \hline
				$2\cdot 10^{41}$     & Rough luminosity of the $z=2.286$ ultraluminous IRAS galaxy IRAS F10214+4724\\ \hline
				$6\cdot 10^{43}$     & Rough luminosity of a cosmological $\gamma$-ray burst\\ \hline  
				$2\cdot 10^{46}$     & Rough luminosity of all the stars in the universe\\ \hline
				$1\cdot 10^{48}$     & Core collapse neutrino luminosity of a type II supernova (QJRAS 30, 423)\\ \hline
	\end{tabular}
	\end{center}
	\caption{Typical powers scales}
	\end{table}

	And some temperature scales:
	\begin{table}[H]
	\begin{center}
		\begin{tabular}{|l|p{12cm}|}
		  \hline
		  \rowcolor[gray]{0.75}\textbf{Magntitude [K]} & \textbf{Description}  \\ \hline
				$7\cdot 10^{-7}$     & Laser cooling of cesium atoms\\ \hline
				$1.3\cdot 10^{-5}$   & Cosmic microwave background quadrupole anisotropy\\ \hline
				$3.3\cdot 10^{-3}$   & Cosmic microwave background dipole anisotropy\\ \hline
				$0.01$                & Typical limit of liquid helium dilution cooling\\ \hline
				$0.3$                 & Typical limit of liquid helium evaporation cooling\\ \hline
				$2.17$                & Liquid $^4$He superfluid transition temperature\\ \hline
				$2.726$               & Cosmic microwave background temperature today\\ \hline
				$3.20$                & Liquid $^3$He boiling point\\ \hline
				$4$                   & Typical limit of Joule-Thomson effect cooling\\ \hline
				$4.18$                & Liquid $^4$He boiling point\\ \hline
				$6$                   & Typical noise temperature of a HEMT receiver at 30 GHz\\ \hline
				$12$                  & Lanthanum (under pressure) superconductivity critical temperature (highest for a pure element)\\ \hline
				$20$                  & Liquid H$_2$ boiling temperature\\ \hline
				$77$                  & Liquid N$_2$ boiling temperature\\ \hline
				$133$                 & Mercury-barium-calcium-copper oxide compound superconductivity critical temperature\\ \hline
				$273$                 & Water freezing temperature\\ \hline
				$311$                 & Human surface temperature\\ \hline
				$373$                 & Water boiling temperature\\ \hline
				$388$                 & Brimstone melting temperature - upper limit to the temperature of Hell\\ \hline
				$506$                 & Paper burning temperature\\ \hline
				$740$                 & Typical surface temperature of Venus\\ \hline
				$1811$                & Melting temperature of iron\\ \hline
				$3000$                & Cosmic microwave background temperature at decoupling\\ \hline
				$5770$                & Solar effective temperature\\ \hline
				$5\cdot 10^{5}$      & Surface temperature of the Geminga pulsar\\ \hline
				$3\cdot 10^{6}$      & Polar cap temperature of the Geminga pulsar\\ \hline
				$3\cdot 10^{6}$      & Typical fusion experiment\\ \hline
				$1.4\cdot 10^{7}$    & Center of the Sun\\ \hline
				$1.5\cdot 10^{7}$    & Changeover temperature from the proton-proton chain to the CNO cycle\\ \hline
				$2.7\cdot 10^{7}$    & Center of a 5 M$_\odot$ star\\ \hline
				$5\cdot 10^{7}$      & Typical gas temperature in a cluster of galaxies\\ \hline
				$4\cdot 10^{8}$      & Characteristic temperature for electron-positron pair production\\ \hline
				$4\cdot 10^{8}$      & Minimum primordial nucleosynthesis temperature\\ \hline
				$5\cdot 10^{8}$      & Inner accretion disc temperature of Cyg X-1\\ \hline
				$7\cdot 10^{8}$      & Thermal electrons become relativistic $(v_{\rm e}={c\over 2})$\\ \hline
				$1\cdot 10^{9}$      & Maximum primordial nucleosynthesis temperature\\ \hline
				$1\cdot 10^{9}$      & Rough superconductivity critical temperature in a neutron star\\ \hline
				$1\cdot 10^{10}$     & Rough plasma pair catastrophe temperature\\ \hline
				$3\cdot 10^{10}$     & Core collapse temperature of a supernova (QJRAS 30, 424)\\ \hline
				$3\cdot 10^{15}$     & Rough electroweak unification temperature\\ \hline
			\end{tabular}
	\end{center}
	\caption{Typical temperature scales}
	\end{table}
	
	And some magnetic intensities scales:
	\begin{table}[H]
	\begin{center}
		\begin{tabular}{|l|p{12cm}|}
		 \hline
		 \rowcolor[gray]{0.75}\textbf{Magntitude [T]} & \textbf{Description}  \\ \hline
		 	$1\cdot 10^{-15}$  & Smallest measured magnetic field (Schumann resonances)\\ \hline
			$1\cdot 10^{-13}$  & Spontaneous human brain activity\\ \hline
			$1\cdot 10^{-12}$  & Evoked human brain activity\\ \hline
			$1\cdot 10^{-12}$  & Typical magnetic field needed for good radio reception \\ \hline
			$1\cdot 10^{-11}$  & Typical magnetic field from a human heart\\ \hline
			$1\cdot 10^{-10}$  & Typical 50/60 Hz magnetic field inside a building\\ \hline
			$5\cdot 10^{-10}$  & Typical magnetic field strength in the local interstellar medium\\ \hline
			$1\cdot 10^{-9}$   & Typical magnetic field strength in a radio lobe\\ \hline
			$1\cdot 10^{-8}$   & Typical magnetic field strength in the central 300 parsecs of the Milky Way\\ \hline
			$5\cdot 10^{-8}$   & Magnetic field in the Crab Nebula\\ \hline
			$1\cdot 10^{-7}$   & Magnetic field in The Arc at the Milky Way center\\ \hline
			$1\cdot 10^{-6}$   & Typical magnetic field from a hand held cordless phone\\ \hline
			$3\cdot 10^{-5}$   & Magnetic field at Earth's surface\\ \hline
			$1\cdot 10^{-4}$   & Magnetic field near Sun's pole\\ \hline
			$4\cdot 10^{-4}$   & Magnetic field at Jupiter's cloud tops\\ \hline
			$1\cdot 10^{-4}$   & Magnetic field near mobile phone\\ \hline
			$0.1$               & Ap star magnetic field\\ \hline
			$0.2$               & Sunspot magnetic field\\ \hline
			$1$                 & Typical medical NMR magnetic field or Solar spots\\ \hline
			$2$                 & Magnetic field felt by the electron in an $n=1$ hydrogen atom\\ \hline
			$2$                 & RS CVn star spot magnetic field\\ \hline
			$12$                & Typical magnetic field used in high resolution NMR spectroscopy\\ \hline
			$25$                & Powerful superconducting/normal hybrid magnet\\ \hline
			$45$                & World record magnetic field in laboratory\\ \hline
			$1\cdot 10^{4}$    & Field of white darf\\ \hline
			$3\cdot 10^{4}$    & Fields in petawatt LASER pulses\\ \hline
			$1\cdot 10^{8}$    & Typical single pulsar dipole magnetic field strength\\ \hline
			
			$4.4\cdot 10^{9}$  & Magnetic field strength when the energy of the first electron Landau level is comparable to the electron rest mass (MNRAS 275, 257)\\ \hline
			$1\cdot 10^{11}$  & Highest field ever measured, on magnetar and soft gamma repeater SGR-1806-20\\ \hline
			$1\cdot 10^{12}$  & Estimated magnetic field near atomic nucleus\\ \hline
		\end{tabular}
	\end{center}
	\caption{Typical magnetic scales}
	\end{table}
	Without measurements, there are also no emotions! Indeed, our emotions are triggered by our senses. And all the impressions and all the information that our senses provide us are - among others - measurements!

	
	\pagebreak
	\subsection{Distributions}
	In physics and mechanics, mass distribution is the spatial distribution of mass within a solid body. In principle, it is relevant also for gases or liquids, but on earth their mass distribution is almost homogeneous.
	
	\textbf{Definitions (\#\mydef):}
	\begin{enumerate}
		\item[D1.]A mass an electric charge are named "\NewTerm{punctual}\index{punctual}" if they occupy a volume whose dimensions are much less than the distances of observations.
		\begin{tcolorbox}[title=Remark,colframe=black,arc=10pt]
	The elementary electric charge is a good approximation of a punctual charge given its small size which the classical radius of the order of the femtometer, which is of course very small compared to conventional observation dimensions (at least until the second millenary...).
	\end{tcolorbox}

		\item[D2.] We consider $N$ bodies of finite mass or electric charge magnitude in a finite volume $V$. If this volume is assumed large enough so that the average distance between the bodies is much greater than the size of these, we are dealing with a "\NewTerm{discontinuous distribution}\index{discontinuous distribution}" or "\NewTerm{discrete distributions}\index{discrete distributions}" of these bodies (we sometimes talk about non-uniform distribution).

		The calculations are most of time very difficult to do by starting with a discrete distribution because, in general, the number of bodies to be considered is very high when the volume has macroscopic dimensions. In this case, many theoretical models (but not all), try to introduce another type of distribution using some approximation tricks.
		
		\item[D3.] We consider $N$ bodies of finite mass or electric charge magnitude in a finite volume $V$. If the distribution of elements such that there is no "holes" between all of them (in other words, each element is clamped against a other) then we are dealing with a "\NewTerm{continuous distribution}". A continuous distribution can then be described by a function that shows how the elements are distributed in a volume, surface or line.
	\end{enumerate}
	\begin{tcolorbox}[title=Remarks,colframe=black,arc=10pt]
	\textbf{R1.}  We can specify sometimes, as we have already mentioned it during the definitions of mass or electric charge, that the distributions can be of volume, surface or linear type. If this is not stated is that the information is implicitly trivial in the context of the mathematical developments.\\
	
	\textbf{R2.} The term "continuous" in "continuous distribution" come from the fact that many times we need to integrate functions, hence the need of a continuous distribution.
	\end{tcolorbox}

	\subsection{Constants}
	The physical opposite of mathematics is an exact science in the sense that its verification and validity are based and are constantly tested by independent experimental facts and controls.

	As the human being had to arbitrarily choose a system of measures, some laws theoretically derived thanks to the properties of matter are often accurate only to a given multiplicative constant factor of normalization with respect to the measurement system. Then appear in the equations of physics constants whose existence is only due to this measurement system (but however this is not always the case), some constants although adapting the measurement system do will equal (at least it seems) never the unit.

	There are many constants in physics (an infinity in fact) but some have a special status in that they can not be deduce from other constants. 
	
	A constant can be seen also as out level of ignorance. Every constant that cannot be explained theoretically is the sign that perhaps a theory must be in-deep (until a given level). Actually the standard model and General Relativity need 21 fundamentals constants.
	
	We propose below a list and values (not exact values) that will frequently met again in our future mathematical our developments in this books.
	\begin{tcolorbox}[title=Remarks,colframe=black,arc=10pt]
	\textbf{R1.} The constants below are given for some for the moment at the reader read them (...) because according to some theories, values are for some of them not fixed at all (time dependent) or just their accuracy increase in function of our technologies.\\
	
	\textbf{R2.} The series of international norms ISO 80000 define the values of numerous scientific constants.
	\end{tcolorbox}
	
	\subsubsection{Mathematical Constants}
	We give here only common mathematical constants which origin (definition) was proven and usage is very common in this book (section of Functional Analysis, Euclidean Geometry, Differential and Integral Calculus). The list will be expanded in function of the evolution of the book itself!
	\begin{center}
		\begin{tabular}{||l|lll||}
		\hline
		{\textbf{Name}}&{\textbf{Symbol}}&{\textbf{Value}}&{\textbf{Unit}}\\
		\hline
		\hline
		Archimede's constant $\pi$  &$\pi$&3.14159265358979323846&\\
		Euler's number $e$          &e    &2.71828182845904523536&\\
		Euler's constant 			&$\gamma$ & 0.5772156629&\\
		Golden ratio	 			&$\varphi$ & $(1+\sqrt{5})/2$&\\
		Euler–Mascheroni constant  	&$\gamma$ & 0.57721&\\
		\hline
		\end{tabular}
	\end{center}
	Even this are mathematical constant they appears a lot in physics in so many fields that the best for the reader is just to read the whole remaining part of this book to see where...
	
	A mathematical constant is therefore a special number, usually a real number, that is "significantly interesting in some way". What it means for a constant to arise "naturally", and what makes a constant "interesting", is ultimately a matter of taste, and some mathematical constants are notable more for historical reasons than for their intrinsic mathematical interest. The more popular constants have been studied throughout the ages and computed to many decimal places.
	
	\begin{tcolorbox}[title=Remark,colframe=black,arc=10pt]
	The reader interested in the properties of chemical elements can download the periodic table of elements proposed in the download section of the companion website.
	\end{tcolorbox}

	\subsubsection{Universal Constants (fundamental constants)}
	A fundamental physical constant is a physical quantity that is generally believed to be both universal in nature and constant in time. It can be contrasted with a mathematical constant, which is a fixed numerical value, but does not directly involve any physical measurement.
	
	The values listed below are values which scientists noticed that they seemed (...) constant and independent of all parameters used, and that the theories presupposes really constant (at least in our Universe):	
	\begin{center}
	\begin{tabular}{||l|lll||}
	\hline
	{\textbf{Name}}&{\textbf{Symbol}}&{\textbf{Value}}&{\textbf{Unit}}\\
	\hline
	\hline
	Elementary charge            &$e$&$1.60217733\cdot10^{-19}$&C\rule{0pt}{13pt}\\
	Gravitational constant       &$G,\kappa$&$6.67408(31)\cdot10^{-11}$&m$^3$kg$^{-1}$s$^{-2}$\\
	Fine-structure constant      &$\alpha=e^2/2hc\varepsilon_0$&$\approx1/137$&\\
	Speed of light in vacuum     &$c=(\varepsilon_0\mu_0)^{-1}$&$2.99792458\cdot10^8$&m/s (def)\\
	Permittivity of the vacuum   &$\varepsilon_0$&$8.854187\cdot10^{-12}$&F/m\\
	Permeability of the vacuum   &$\mu_0$&$4\pi\cdot10^{-7}$&H/m\\
	Impedance of the vacuum   &$Z_0$&$376$&CVs$^{-1}$\\
	Coulomb's constant $(4\pi\varepsilon_0)^{-1}$   &&$8.9876\cdot10^9$&Nm$^2$C$^{-2}$\\
	\hline
	Planck's constant            &$h$&$6.6260755\cdot10^{-34}$&Js\rule{0pt}{13pt}\\
	Dirac's constant             &$\hbar=h/2\pi$&$1.0545727\cdot10^{-34}$&Js\\
	Bohr magneton                &$\mu_{\text{B}}=e\hbar/2m_{\text{e}}$&$9.2741\cdot10^{-24}$&Am$^2$\\
	Bohr radius                  &$a_0$&$0.52918$&\AA\\
	Rydberg's constant           &$R_y$&$10.973\cdot 10^{6}$& m$^{-1}$\\
	Electron Compton wavelength  &$\lambda_{\text{Ce}}=h/m_e c$&$2.2463\cdot 10^{-12}$&m\\
	Proton Compton wavelength    &$\lambda_{\text{Cp}}=h/m_{\text{p}}c$&$1.3214\cdot 10^{-15}$&m\\
	Reduced mass of the H-atom   &$\mu_{\text{H}}$&$9.1045755\cdot 10^{-31}$&kg\\
	Electron-Volt   &$1$[eV]&$1.60217744\cdot 10^{-19}$&J\\
	\hline
	Stefan constant  &$\sigma$&$5.67032\cdot 10^{-8}$&Wm$^{-2}$K$^{-4}$\rule{0pt}{13pt}\\
	Stefan-Boltzmann's constant  &$\sigma_{\text{S.B.}}$&$7.56\cdot 10^{-16}$&Wm$^{-2}$K$^{-4}$\rule{0pt}{13pt}\\
	Wien's constant              &$k_{\text{W}}$&$2.8978\cdot 10^{-3}$&mK\\
	Absolute Temperature         &$T_0$&$275.15$&K\\
	\hline
	Molar gas constant            &$R$&8.31441&J mol$^{-1}$K$^{-1}$\\
	Faraday's constant           &$F$&96,458&C mol$^{-1}$\\
	Avogadro's constant          &$N_{\text{A}}$&$6.0221367\cdot  10^{23}$&mol$^{-1}$\\
	Boltzmann's constant         &$k=R/N_{\text{A}}$&$1.380658\cdot10^{-23}$&JK$^{-1}$\\
	\hline
	Electron mass                &$m_{\text{e}}$&$9.1093897\cdot10^{-31}$&kg\rule{0pt}{13pt}\\
	Proton mass                  &$m_{\text{p}}$&$1.6726231\cdot10^{-27}$&kg\\
	Neutron mass                 &$m_{\text{n}}$&$1.674954\cdot10^{-27}$&kg\\
	Elementary mass unit         &$m_{\text{u}}=\frac{1}{12}m(^{12}_{~6}$C)&$1.6605656\cdot10^{-27}$&kg\\
	Nuclear magneton             &$\mu_{\text{N}}$&$5.0508\cdot10^{-27}$&J/T\\
	\hline
	\end{tabular}
	\end{center}
	Most of the constants above can be obtained by theoretical models (see next chapters of this book for the proofs). In fact, as far as we know today, only $e$, $\mu_0$, $\varepsilon_0$, $h$, $m_e$, $e$ cannot be obtain from other constants.
	
	In fact, the number of fundamental physical constants depends on the physical theory accepted as "fundamental". Currently, this is the theory of General Relativity for gravitation and the Standard Model for electromagnetic, weak and strong nuclear interactions and the matter fields. Between them, these theories account for a total of $19$ independent fundamental constants.
	
	\pagebreak
	\subsubsection{Astronomical/Astrophysical parameters and constants}
	Obviously an astronomical or astrophysic parameter is a physical "constant" used in astronomy theoretical models that in fact vary with time and can be obtained only by measurement.
	\begin{center}
	\begin{tabular}{||l|lll||}
	\hline
	{\textbf{Name}}&{\textbf{Symbol}}&{\textbf{Value}}&{\textbf{Unit}}\\
	\hline
	\hline
	Diameter of the Sun          &$D_\odot$&$1392\cdot10^6$&m\rule{0pt}{13pt}\\
	Earth's average gravitation	 &$g$&$9.81$&ms$^{-1}$\rule{0pt}{13pt}\\
	Mass of the Sun              &$M_\odot$&$1.989\cdot10^{30}$&kg\\
	Rotational period of the Sun &$T_\odot$&25.38&days\\
	Radius of Earth              &$R_{\text{A}}$&$6.378\cdot10^6$&m\\
	Mass of Earth                &$M_{\text{A}}$&$5.976\cdot10^{24}$&kg\\
	Rotational period of Earth   &$T_{\text{A}}$&23.96&hours\\
	Earth orbital period         &Tropical year&365.24219879&days\\
	Astronomical unit            &AU&$1.4959787066\cdot10^{11}$&m\\
	Light year                   &lj&$9.4605\cdot10^{15}$&m\\
	Parsec (Sun-Earth distance)                      &pc&$3.0857\cdot10^{16}$&m\\
	Hubble constant (non-SI)     &$H_0$&$67.8\pm0.7$&km$\cdot$s$^{-1}\cdot$Mpc$^{-1}$\\
	Hubble constant (SI units)  &$H_0$&$2.19725\cdot 10^{-18}$&s$^{-1}$\\
	Universe critical density	&$\rho_c$&$5\cdot 10^{-27}$&kg m$^{-3}$\\
	\hline
	\end{tabular}
	\end{center}
	\subsubsection{Chemical parameters}
	The physicochemical constants of physical constants that are especially found in all areas relating to chemistry and thermodynamics.
		\begin{center}
	\begin{tabular}{||l|lll||}
	\hline
	{\textbf{Name}}&{\textbf{Symbol}}&{\textbf{Value}}&{\textbf{Unit}}\\
	\hline
	\hline
	Standard Pression		&$P_0$&$101,325$&Pa\\
	Standard Temperature	&$T$&$293.15$&K\\
	Molar Volume         	&$V_M$&$22.4141$&l mol$^{3}$\\
	\hline
	\end{tabular}
	\end{center}
	\begin{tcolorbox}[title=Remark,colframe=black,arc=10pt]
	The reader interested in the properties of chemical elements can download the periodic table of elements proposed in the download section of the companion website.
	\end{tcolorbox}
	
	\pagebreak	
	\subsubsection{Material parameters}
	The material parameters are especially found in all areas relating to mechanics, electrodynamics, thermodynamics.
	\begin{center}
	\begin{tabular}{||l|lll||}
	\hline
	{\textbf{Name}}&{\textbf{Symbol}}&{\textbf{Value}}&{\textbf{Unit}}\\
	\hline
	\hline
	Refractive index of vacuum						&$n$&$1$&\\
	Refractive Vacuum/Air ($\lambda=589.29$ [nm])	&$n$&$1.000293$&\\
	Refractive Vacuum/Water ($\lambda=589.29$ [nm], $293$ [K])	&$n$&$1.330$&\\
	Refractive Vacuum/Diamond ($\lambda=589.29$ [nm])	&$n$&$2.419$&\\
	Refractive Vacuum/Ethanol (ethyl alcohol) ($\lambda=589.29$ [nm])	&$n$&$1.361$&\\\hline
	Resistivity copper (293 [K])	&$\rho$&$0.017$& $\Omega$mm$^2$m$^{-1}$\\
	Resistivity aluminium (293 [K])	&$\rho$&$0.0278$& $\Omega$mm$^2$m$^{-1}$\\
	Resistivity sast steel (293 [K])	&$\rho$&$0.13$& $\Omega$mm$^2$m$^{-1}$\\
	Resistivity silver (293 [K])	&$\rho$&$0.016$& $\Omega$mm$^2$m$^{-1}$\\
	Resistivity carbon (293 [K])	&$\rho$&$40$& $\Omega$mm$^2$m$^{-1}$\\
	Resistivity pure steel (293 [K])	&$\rho$&$0.1$& $\Omega$mm$^2$m$^{-1}$\\
	Resistivity quicksilver (293 [K])	&$\rho$&$0.941$& $\Omega$mm$^2$m$^{-1}$\\
	Resistivity gold (293 [K])	&$\rho$&$0.0222$& $\Omega$mm$^2$m$^{-1}$\\
	Resistivity lead (293 [K])	&$\rho$&$0.208$& $\Omega$mm$^2$m$^{-1}$\\
	Resistivity bakelite (293 [K])	&$\rho$&$10^{18}$& $\Omega$cm\\
	Resistivity pure water (293 [K])	&$\rho$&$10^7$& $\Omega$cm\\
	Resistivity marble (293 [K])	&$\rho$&$10^{10}$& $\Omega$cm\\
	Resistivity glas (293 [K])	&$\rho$&$10^{15}$& $\Omega$cm\\
	\hline
	\end{tabular}
	\end{center}
	
	\pagebreak
	\subsubsection{Planck's constants}\label{planck constants}
	The Planck's constants are mainly physical curiosities that arise from a specific system of units and whose values in the S.I. system are given in the table below.
	\begin{center}
	\begin{tabular}{||l|lll||}
	\hline
	{\textbf{Name}}&{\textbf{Symbol}}&{\textbf{Value}}&{\textbf{Unit}}\\
	\hline
	\hline
	Planck length &$l_P$&$1.616199(97)\cdot 10^{-35}$&m\rule{0pt}{13pt}\\
	Planck area		&$l_P^2$&$2.6121 \cdot 10^{-70}$&m$^2$\rule{0pt}{13pt}\\	
	Planck volume 		&$l_P^3$&$4.2217 \cdot 10^{-105}$&m$^3$\rule{0pt}{13pt}\\
	Planck mass	 	&$m_P$&$2.17651(13)\cdot 10^{-8}$&kg\rule{0pt}{13pt}\\
	Planck time	 	&$t_P$&$5.39106(32)\cdot 10^{-44}$&s\rule{0pt}{13pt}\\
	Planck electric charge &$q_P$&$1.875545956(41)\cdot 10^{-18}$&s\rule{0pt}{13pt}\\
	Planck temperature 	&$T_P$&$1.416833(85)\cdot 10^{32}$&K\rule{0pt}{13pt}\\
	Planck linear momentum 		&$p_P=m_p\cdot c$&$6.52485$&kg m s$^{-1}$\rule{0pt}{13pt}\\
	Planck energy 		&$E_P=m_Pc^2$&$1.9561\cdot 10^9 $&J\rule{0pt}{13pt}\\
	Planck force 		&$f_P=E_P\cdot l_P^{-1}$&$1.21027\cdot 10^{44} $&N\rule{0pt}{13pt}\\
	Planck power 		&$P_P=E_P\cdot t_P^{-1}$&$3.62831\cdot 10^{52}$&W\rule{0pt}{13pt}\\
	Planck density 		&$\rho_P=m_P\cdot l_P^{-3}$&$5.15500\cdot 10^{96}$&kg m$^{-3}$\rule{0pt}{13pt}\\
	Planck energy density 		&$u_P=E_P\cdot l_P^{-3}$&$4.63298\cdot 10^{113}$&J m$^{-3}$\rule{0pt}{13pt}\\
	Planck energy intensity 		&$I_P=u_P\cdot c$&$1.38893\cdot 10^{122}$&W m$^{-2}$\rule{0pt}{13pt}\\
	Planck angular frequency 		&$\omega_P=t_P^{-1}$&$1.85487\cdot 10^{43}$&s$^{-1}$\rule{0pt}{13pt}\\
	Planck pressure 		&$p_P=f_p\cdot l_P^2$&$4.63309\cdot 10^{113}$&N m$^{-2}$\rule{0pt}{13pt}\\
	Planck electric current 		&$I_P$&$3.4789\cdot 10^{25}$&A\rule{0pt}{13pt}\\
	Planck voltage		&$V_P$&$1.04295\cdot 10^{27}$&V\rule{0pt}{13pt}\\
	Planck impedance		&$Z_P=V_P\cdot I_P^{-1}$&$29.9792458$& Ohm\rule{0pt}{13pt}\\
	Planck acceleration		&$a_P=c\cdot t_P^{-1}$&$5.560815\cdot 10^{51}$&m s$^{-2}$\rule{0pt}{13pt}\\
	\hline
	\end{tabular}
	\end{center}
	We see above that some Planck units are suitable for measuring quantities that are familiar from daily experience. For example:
	\begin{itemize}
		\item $1$ Planck mass is about $22$ micrograms
		\item $1$ Planck momentum is about $6.5\;[\text{kg}\cdot\text{m}\cdot\text{s}^{-1}]$
		\item $1$ Planck energy is about $500\; [\text{kW}\cdot \text{h}]$
		\item $1$ Planck charge is close to $11.7$ elementary charges
		\item $1$ Planck impedance is close to $30\;[\Omega]$
	\end{itemize}

	Planck constants are based on the fact that as we will see further below that their definition comes only from properties of Nature and not from any human construct. Planck units are only one system of several systems of natural units, but Planck units are not based on properties of any prototype object or particle (that would be arbitrarily chosen), but rather on only the properties of free space.

	In this system of units the fundamentals constants $G$, $c$, $\hbar$, $(4\pi\varepsilon_0)^{-1}$ and $k$ (Boltzmann constant) are such that:
	
	Some physicists argue that communication with extraterrestrial intelligence would have to employ such a system of units in order to be understood as it is once again the Natural choice!!!
	
	The Planck constant are mainly physical curiosities and the reader must not see any special interpretation about them (at least as far as we know today).
	
	\begin{tcolorbox}[title=Remark,colframe=black,arc=10pt]
	The reader interested in the exact provenance of the various Planck constants  (Planck length, Planck mass, etc.) should go read the section of Wave Quantum Physics where these constants are determined with the necessary details.
	\end{tcolorbox}
	The table below shows how the use of Planck units simplifies many fundamental equations of physics that we will prove in details in the next chapters, because this gives each of the five fundamental constants, and products of them, a simple numeric value of $1$. In the SI form, the units should be accounted for. In the non-dimensionalized form, the units, which are now Planck units, need not be written if their use is understood.
	\begin{table}[H]
		\begin{center}
			\definecolor{gris}{gray}{0.85}
				\resizebox{\textwidth}{!}{\begin{tabular}{|l|c|c|}
					\hline
					\multicolumn{1}{c}{\cellcolor{black!30}\textbf{Model}} & 
	  \multicolumn{1}{c}{\cellcolor{black!30}\textbf{SI form}} & 
	  \multicolumn{1}{c}{\cellcolor{black!30}\textbf{Non-dimensionalized form}}  \\ \hline
					Newton's gravitational law & $ F = - G \frac{m_1 m_2}{r^2} $ &  $ F = - \frac{m_1 m_2}{r^2} $ \\ \hline
					Coulomb's law & $ F = \frac{1}{4 \pi \varepsilon_0} \frac{q_1 q_2}{r^2} $ &  $  F = \frac{q_1 q_2}{r^2}  $ \\ \hline
					Einstein field equations & ${ G_{\mu \nu} = 8 \pi {G \over c^4} T_{\mu \nu} } \ $ & ${ G_{\mu \nu} = 8 \pi T_{\mu \nu} } \ $ \\ \hline
					Mass–energy equivalence & ${ E = m c^2} \ $ & ${ E = m } \ $ \\ \hline
					Energy–momentum relation & $ E^2 = m^2 c^4 + p^2 c^2  \;$ & $ E^2 = m^2 + p^2  \;$ \\ \hline
					Thermal energy & ${ E = \tfrac12 k T} \ $ & ${ E = \tfrac12 T} \ $ \\ \hline
					Planck-Einstein relation & $E = \hbar \omega $ & $E = \omega $ \\ \hline
					Planck's law & $ I(\omega,T) = \frac{\hbar \omega^3 }{4 \pi^3 c^2}~\frac{1}{e^{\frac{\hbar \omega}{k_\text{B} T}}-1} $ & $ I(\omega,T) = \frac{\omega^3 }{4 \pi^3}~\frac{1}{e^{\omega/T}-1} $ \\ \hline
					Stefan–Boltzmann constant & $ \sigma =  \frac{\pi^2 k_\text{B}^4}{60 \hbar^3 c^2} $ & $\sigma = \frac{\pi^2}{60} $ \\ \hline
					Schrödinger's equation & $
- \frac{\hbar^2}{2m} \nabla^2\left(1 + V(\vec{r})\right) \Psi(\vec{r}, t) = i \hbar \dot{\Psi}(\vec{r}, t)$ & $
- \frac{1}{2m} \nabla^2 \left(1 + V(\vec{r})\right) \Psi(\vec{r}, t) = i \dot{\Psi}(\vec{r}, t)$ \\ \hline
					Dirac covariant equation & $\ ( i\hbar \gamma^\mu \partial_\mu - mc) \psi = 0$ & $\ ( i\gamma^\mu \partial_\mu - m) \psi = 0$  \\ \hline
					Maxwell Equations & \parbox{2.5cm}{$\vec{\nabla} \circ \vec{E} = \displaystyle\frac{1}{\varepsilon_0} \rho$ \\ $\vec{\nabla} \circ \vec{B} = 0 \ $ \\ $\vec{\nabla} \times \vec{E}=\dfrac{\partial \vec{B}}{\partial t}$ \\ $\vec{\nabla} \times \vec{B} = \displaystyle\frac{1}{c^2} \left(\dfrac{1}{\varepsilon_0}\vec{j}+\dfrac{\partial \vec{E}}{\partial t}\right)$} & \parbox{2.5cm}{$\vec{\nabla} \circ \vec{E} = 4\pi \rho$ \\ $\vec{\nabla} \circ \vec{B} = 0 \ $ \\ $\vec{\nabla} \times \vec{E}=\dfrac{\partial \vec{B}}{\partial t}$ \\ $\vec{\nabla} \times \vec{B} = 4\pi\vec{j}+\dfrac{\partial \vec{E}}{\partial t}$}  \\ \hline
			\end{tabular}}
		\end{center}
		\caption{Resume of main Combinatorial Analysis cases}
	\end{table}
	
	Despite the examples given how many  constant exists? Why play they a "central role" in physical theories? Do they all have the same importance or are some of them more fundamental than the others? Following what criteria? Can we then test whether they are really constant?
	
	To try to answer some of these questions, first of all we must notice that at each step of our theoretical constructions there remain some constant parameters that are not and can not be explained in terms of more fundamental quantities, simply because these latter do not exist in the state of our knowledge. When the theories are refined, it is possible that a constant is explained in terms of new parameters, more fundamental. Thus, the mass of the proton is a fundamental constant of nuclear physics, but must in principle be able to calculate, within the framework of quantum chromodynamics, according to the quark masses and energies of the electromagnetic and strong interactions. This change of status is associated with that of the proton, which is of elementary particle become a composite body.
	
	We will define constants modestly as all non-determined parameters in a given theoretical framework. This definition is equivalent to accept that we are unable to write an evolution equation for these constants and that thus revealed the limit of what the theories in which they appear are able to explain. However the constancy hypothesis is implicitly controlled by experimental validation of these theories. The results of experiments must be reproducible at various times and in various laboratories. If this is the case, within the limits of experimental details, then it is legitimate to consider that the constancy hypothesis is valid. This definition implies that there is no absolute constants list, because such a list depends on the theoretical frameworks chosen to describe the Nature and may change with the progress of knowledge.
	
	Now arises the question whether it is possible to characterize more precisely the concept of constant and determine whether, among all the constants, some are more fundamental than others. For this, we must first prove a relation between constants and units.

	Thus, Planck discovered in 1900 that it was possible to use the three fundamental physical constants:
	
	to define the three units of mass, time and length from the Planck mass, the Planck length and the Planck time (see again the chapter of Atomistic for the mathematical approach for determining their values).
		
		Planck named those units "\NewTerm{Natural System Units}\index{natural system units}\label{natural system units}" (SUN) because they are independent of a body or a material and [...] necessarily retain their significance for all times and all civilizations, even for aliens or non-human one. The significance of these units take long to emerge in the theoretical physicists community. They point the scales where quantum mechanics and gravitation become of the same amplitude. They are therefore very suited to the study of primordial cosmology and to that of black holes and relativistic quantum mechanics.
		
		The choice of Planck units as natural units is linked to the considerations justifying that $G$, $c$, $\hbar$ are the three most fundamental constants (known at this date). In these units, the numerical value of these three fundamental constants is $1$ as we have already noticed earlier above.

	The role of the constants in the structure of physic theories can be quite beautifully illustrated by the magic cube named the "\NewTerm{Okun cube}\index{Okun cube}\label{okun cube}" of physical theories visible in the figure below (whose validity remains to be check of course). The idea is to "turn on" or "turn off" one by one the three fundamental constants to see how physical theories are articulated with each other.
	\begin{tcolorbox}[title=Remark,colframe=black,arc=10pt]
	The reader will certainly better understand the explanations that follow when it will have studied and read the sections of General Relativity, Quantum Field Theory and therefore the Wave Quantum Physics, so if necessary he can skip the following explanations.
	\end{tcolorbox}
	\begin{figure}[H]
		\centering
		\includegraphics[scale=0.8]{img/mechanics/okun_cube.jpg}
		\caption[Okun Cube]{Okun Cube (source: Juergen Giesen)}
	\end{figure}
	When $G$ is set to $0$, this means removing all the gravitational forces and decoupling the matter of space and time. When $\hbar$ is set to $0$, we delete the quantum character of nature and we decouple the corpuscular and wave nature (consequence of De Broglie relation), when $1 / c$ is set to $0$, the speed of light is infinite and the time and space are uncoupled from each other (consequence of Lorentz transformation). To visualize this, we consider the cube above introduced by the Soviet physicist Mikhail Bronshtein, who gets an idea originally developed by Lev Landau, Dmitri Ivanenko and Georgi Gamow.
	
	Naturally, at the lowest level $(0,0,0)$, we find the Newtonian mechanics, which does not take into account relativistic, quantum or gravitational effects. At the top level where we consider the effect of one constant, we find the three of Special Relativity $(1,0,0)$, quantum mechanics $(0,1,0)$ and Newtonian gravitation $( 0,0,1)$, three theories tested with great precision in their field of validity.
	
	At an even higher level, the quantum field theory in $(1,1,0)$ takes into account the quantum and relativistic effects; the General Relativity in $(1,0,1)$ takes into account the gravitational and relativistic effects, and Newtonian quantum gravity in $(0,1,1)$ is supposed to provide a quantum and non-relativistic description of gravity. Only the first two theories are currently tested experimentally.
	
	The ultimate actual level is at $(1,1,1)$ the "theory of everything" (name very pretentious and too commercial), supposed to give a quantum and relativistic description of gravity. Its formulation is not yet known, even if String Theories (see section of the same name page \pageref{string theory}), intensively studied nowadays seem to be a serious candidates. These theories appear as limiting cases of a broader and deeper theory but not yet formulated: the M theory (M states for "Mother").
	
	\subsection{Principles of Physics}
	The advances of science in general and physics in particular were based two centuries before mainly on experimentation, that is to say that we reproduced in the laboratories given phenomena to analyze them systematically (the reproductibility with a given accuracy validating a hypothesis). This was equivalent as asking systematically specific questions to Nature and to describe and study the reactions caused thereby. The repetition at will of a phenomenon in an experiment would not be guaranteed if there was not a general principle of causality.
	
	\subsubsection{Principle of Causality}
	\textbf{Definition (\#\mydef):} We define the "\NewTerm{principle of causality}\index{principle of causality}" by the fact that in exactly the same conditions, the same causes always lead to the same effects. In other words, if certain initial conditions are well known, the phenomenon will take place in a determined manner, always the same (at least in classical physics, but not in quantum physics!!!).

	In fact, experience is not necessary if we consider the first order principles which are by definition "logical principles we can deduce by induction that we can experimentally verify with certainty".

	However, our modern society often give only a little time left to the great people of science to think about these first-order principles by imaginary experiences (very popular method used by Albert Einstein for the parenthesis...).

	It is in a trilemma proposed by the skepticist of antiquity named Agrippa, according to an argument reported by Sextus Empiricus, that the question of justification of knowledge has been put most explicitly:
	\begin{enumerate}
		\item[H1]. Knowledge is ultimately based on first principles but arbitrary one (axiomatic argument, which rests on accepted precepts)

		\item[H2.] Or we do not find such principles and we have an and infinite regress (regressive argument, in which each proof requires a further proof, ad infinitum)

		\item[H3.] Or justification is circular (circular argument, in which theory and proof support each other)
	\end{enumerate}
	The last two methods of reasoning are fundamentally weak, and because the Greek skeptics advocated deep questioning of all accepted values, they refused to accept proofs of the first sort. The trilemma, then, is the decision among the three equally unsatisfying options.
	
	This trilemma has also often, in contemporary philosophy, especially by Karl Popper, the name of "\NewTerm{Fries trilemma} \index{Fries trilemma}" or "\NewTerm{Munchausen trilemma}\index{Munchausen trilemma}" and we currently do not know in what scenario (H1, H2 or H3) we are.

	Let us state now first states three  fundamental principles (or assumptions):
	
	\subsubsection{Principle of Conservation of Energy}\label{principle of conservation of energy}
	The first principle of conservation of energy is stated as following (basically at least... see below ... for some remarks ...): The total energy, denoted $E$ or $E_\text{tot}$ of any isolated and inertial system does not vary with time relatively to a fixed reference frame if there is no transfer of energy (or mass/radiation) or heat from outside the system.
	
	This principle can be expressed by the formula:
	 
	where $\Delta E_\text{tot}$ is the total energy variation of the system, $\Delta U$ the variation of internal energy of the system, that is to says its own energy corresponding to the kinetic and potential microscopic energies of particles that constitute it.

	The term $\Delta E_c$, also sometimes denoted $\Delta E_\text{cin}$, is the change in kinetic energy at a macroscopic scale (motion of the system in a given reference framce) and the term $\Delta E_p$, also sometimes denoted $\Delta E_p$,  the variation of potential energy at a macroscopic scale, for the system interacting with gravitational or electromagnetic fields.

	In physics, a conservation law (nothing is lost, nothing is created) expresses a particular measurable property of an isolated physical system remains constant during the evolution of this system. For exemple for energy this means in the cas of a conservative system:
	 
	The list below lists somes conservation laws useful to the engineer and have never been put in default as far as we know untile this day and resulting for most of them of the conservation energy principle:
	\begin{itemize}
		\item Conservation of inertia
		\item Conservation of momentum
		\item Conservation of angular momentum
		\item Conservation of electric charge
		\item Conservation of the magnetic flux
   		\item Conservation mass
	\end{itemize}
	The Noether's theorem that we will see further below expresses the equivalence between conservation laws and invariance of physical laws regarding certain transformations (typically named "symmetries")!!!!!! 

	This theorem applies only to system describable by a Lagrangian (see next section of Analytical Mechanics). For example, the invariance by translation over time implies that energy is conserved (!!!), the translation invariance in space implies that the amount of movement is maintained (!!!), and the rotation invariance in space implies that the angular momentum is conserved (!!!).
	
	This equivalence can be proven and follows from the invariance in time of the laws of physics. This is the first Noether's principle (theorem) that we'll prove a little further below.

	The energy that the human being quantified with the unit "Joules" as we know however, can not be defined accurately as far as we know. Answering this question is equivalent to know what is a mass (Einstein equivalence relation) or if we know what is the mass thanks to the Higgs Bosons it is equivalent to know what is this one and thus meaning we know the fundamental element of the Universe (we have already mention this issue earlier above in this text).
	
	\begin{tcolorbox}[title=Remark,colframe=black,arc=10pt]
	\textbf{R1.} Energy can be found as we know in several forms (this does not to mean that there are several types of energies!!!) as heat, kinetic energy, potential, electrical, magnetic, etc ... as we already have made above.\\

	For example, in consumer applications, including in the field of nutrition, we often express the energy in "calories". The calorie being strictly speaking the energy we must provide to heat up one gram of pure water of one degree Kelvin (or one degree Celsius) as we will see more in details in the section of Thermodynamics. Even worst... nutritionists name for communication purposes "\NewTerm{calorie}\index{calorie}" what physicists name (properly) "\NewTerm{kilocalorie}\index{kilocalorie}".\\
	
	\textbf{R2.} The violation of this principle of conservation of energy in an isolated system has never been observed but its validity can not be proved as far as we know (hence the fact that it is a "first principle").\\
	
	\textbf{R3.} Some physicists are debating the fact that this first principle follows from Noether's theorem we shall see. But it is quite debatable as Noether's theorem considers the potential energy as constant over time hence.
	\end{tcolorbox}
	
	\pagebreak
	\subsubsection{Principle of Least Action}
	The "\NewTerm{principle of least action}\index{principle of least action}" (also named "\NewTerm{variational principle")}\index{variational principle}\label{variational principle}") is stated as following:

	All natural phenomena are consistent with the fact that Nature, in producing its effects, always acts by the simplest and most direct way.

	With this statement and the principle of conservation of energy, we can then establish a great powerful mathematical tools for the study of theoretical physics. But we can not introduce this tool at this level of our discussion since it requires mathematical tools that we will introduce later in details, in the section of Analytical Mechanics (during the study of the Lagrangian formalism to be more precise).

	Meanwhile here are the two main relations that summarize this principle:
	
	\begin{tcolorbox}[title=Remark,colframe=black,arc=10pt]
	The violation of this principle in an isolated system has never been observed as far as we know but its validity can also not be proved as far as we know (hence the fact that it is also a "first principle").
	\end{tcolorbox}
	
	\subsubsection{Noether's Principle (Noether's theorem)}\label{noether theorem}
	The Noether's principle, traditionally named "\NewTerm{Noether's theorem}\index{Noether's theorem}" as it can be proved, combines elegantly conserved physical quantities to the symmetries of the laws of nature!!!

	That is:
	\begin{itemize}
		\item Translational symmetry in the time (time invariant phenomenon) corresponds to the energy conservation
	
		\item Translation in the space corresponds to the conservation of momentum (quantity of motion)

		\item Rotation in space corresponds to the conservation of angular momentum
		
		\item Gauge transformations corresponds conservation of electric charge, weak and color change.
	\end{itemize}

	In other words, Noether's theorem states that physics is:
	\begin{itemize}
		\item Symmetrical (invariant) by translation in time (this having the consequence that there is no original time)

		\item Symmetrical (invariant) by translation in space (this having the consequence that there is no original in the space)

		\item Symmetrical (invariant) by rotation (this having the consequence that there is no preferred direction in space)
	\end{itemize}
	Noether's theorem is important, both because of the insight it gives into conservation laws, and also as a practical calculational tool. It allows investigators to determine the conserved quantities (invariants) from the observed symmetries of a physical system. Conversely, it allows researchers to consider whole classes of hypothetical Lagrangians with given invariants, to describe a physical system. As an illustration, suppose that a physical theory is proposed which conserves a quantity $X$. A researcher can calculate the types of Lagrangians that conserve $X$ through a continuous symmetry. Due to Noether's theorem, the properties of these Lagrangians provide further criteria to understand the implications and judge the fitness of the new theory.
	
	\begin{tcolorbox}[title=Remarks,colframe=black,arc=10pt]
	\textbf{R1.} This theorem implies that a Galilean reference frame (\SeeChapter{see section Classical Mechanics page \pageref{galilean reference frame}}) is homogeneous (no privileged time or space origin) and isotropic (no preferred direction).\\
	
	\textbf{R2.} We must not confuse between the invariance of the laws and the non-invariance of the theoretical solutions which lead to these laws! For example, the discharge of a capacitor (\SeeChapter{see section Electrical Engineering page \pageref{rc series circuit}}) is translation invariant in time but not its solution.
	\end{tcolorbox}
	This result established in 1915 by Emmy Noether just after arriving in Göttingen, was described by Albert Einstein as "monument of mathematical thinking". It is now one of the pillars of theoretical physics.

	Today it is still often presented only during course on quantum field theory. This makes it seem more complicated and mysterious than it is, and this is forget that it also applies to classical mechanics!!!
	\begin{tcolorbox}[title=Remark,colframe=black,arc=10pt]
	We recommend to most readers to read the proof of Noether's theorem in parallel of the section Analytical Mechanics and Classical Mechanics or read it now and come back read it again afterwards.
	\end{tcolorbox}
	Thus, symmetries play a major role in physics. They allow in one hand to simplify the problems and on the other hand to draw new laws. To illustrate the first application of symmetries let us simply evoke the mathematical form of the gravitational potential generated by a punctual mass located at the origin of reference frame (\SeeChapter{see section of Classical Mechanics page \pageref{gravitation potential}}). In Cartesian coordinates, the expression of the gravitational potential is relatively complex:
	
	while in spherical coordinates (coordinate system that takes advantage of the spherical symmetry of the potential) it takes a very simple form:
	
	The properties of symmetry of a problem are exploited here in order to simplify the mathematical treatment of physical laws. Although these mathematical considerations tell us about the physical properties of the system considered, they conserve however a purely technical nature.
	
	Symmetries also found another application whose physical significance is much deeper. The fact, not accidental, that a system possesses symmetries should certainly have physical implications. Intuitively, we understand that the presence of symmetries in a physical system translates the invariance of some of its physical properties under the application of space-time transformations or, more generally: under geometric transformations. The invariance of physical properties must necessarily induce relations of a new kind between the variables of the system. Such relations must in turn reveal deeper laws that combine the geometry of the system to the laws of Nature. This reasoning, even if it is intuitive, invites us to delve deeper into the relations that may exist between the physics laws and the geometrical properties of space-time.
	
	Let us consider a mechanical experience more or less complex observed simultaneously  by two physicists O and O' at rest or in uniform translation located in different places as each of them chooses a reference frame which he is the origin.

	They undertake to measure various physical quantities and obtain numerically results that generally differ. However, the physical laws they draw (same level of knowledge) are identical!! This conclusion is trivial because we all know that the laws of Nature must not depend not on the location of the observers in uniform translation or a rest.

	Mathematically, the difference between repositories O and O' following the chosen reference frame of the studied experiment is the change of basis from one to the other by a rotation $R(\phi)$ and / or translation $T(l)$ (\SeeChapter{see section Euclidean Geometry page \pageref{geometric transformations}}).

	The fact that the physical laws are independent of the position of the observer means that they do not vary after have applied to them a rotation and / or translation. We then say that they are "\NewTerm{invariant under rotation and translation}" or they are "\NewTerm{symmetrical in rotation and translation}".
	\begin{tcolorbox}[title=Remark,colframe=black,arc=10pt]
	In physics, the definition of symmetry is similar to that of mathematicians but applies to the laws of nature and not the geometric figures. Thus a physical symmetry is a transformation of system variables - which may be geometrical variables or more abstract variables - that does not change the formulation of physical laws.
	\end{tcolorbox}
	Let us give a more or less rigorous definition of a physical symmetry.

	Before let us consider a system $S$ whose state evolves over time. Let us denote the state of $S$ at time $t$ by $S(t)$. Then at the initial moment $t_0$, $S$ is therefore in the state $S(t_0)$. Let us consider a geometrical transformation $T$ (rotation or translation) that acts at each point of space and optionally of time . In a time $t$, the action of $T$ on the system $S$ has for effect of transforming it into a system $S'(t)=T[S(t)]$ such that at the moment $t_0$, the transformation by $T$ of $S(t_0)$ is $S'(t_0)$.
	
	\textbf{Definition (\#\mydef):} A transformation $T$ named a "\NewTerm{physical symmetry}\index{physical symmetry}" if the transformation by $T$ of a system $S$ (which gives $S'$) evolves in the same way as $S$, that is to say, if we apply the laws physics on $S'(t_0)$ to know its states $S''$ in a later time $t$ then $S''(t)=S'(t)=T[S(t)]$.
	
	\paragraph{Invariance by translation in space}\mbox{}\\\\
	Let us consider an isolated system consisting of $n$ interacting particles identified by the position vectors $\vec{r}_i$. The interaction of two particles $i, j$ derives from a potential $\varphi(\vec{r}_i,\vec{r}_j,t)$ (\SeeChapter{see section Classical Mechanics page \pageref{gravitational potential energy}}). Each particle is subjected to forces resulting from the interaction with other particles. For a given particle $i$ for example, the resultant of these forces is expressed according to Newton's second law (\SeeChapter{see section of Classical Mechanics page \pageref{newton second law}}):
	
	Let us apply to the system the following translation in space:
	
	where $\vec{l}$ is any space direction vector.

	Say that the translation of a system is a symmetry means that the acceleration and the force acting on each particle are unchanged after the transformation:
	
	Which implies:
	
	This equality must be true regardless of the position of the particles, so whatever are the $\vec{r}_i$ and the $\vec{r}_j$. It is clear that the only way to check the last equality in these conditions is to equalize the by pairs the potential between each particle $j$ with the particle $i$, that is to say:
	
	The potentials are then necessarily (and this is the power of Noether's theorem!) functions of the $\vec{r}_i-\vec{r}_j$ such that:
	
	Therefore, we deduce that:
	
	This immediately implies that the resultant of all the forces applied to the particles in the system is zero and therefore the total linear momentum is conserved:
	
	The translation invariance of Newton's law thus leads necessarily to:
	\begin{enumerate}
		\item The potential between the particles of an isolated system is a function of their relative distance (this will be confirmed in the section Astronomy during our study of the gravitational potential field, as well as in the section of Electrostatics regarding the electrostatic potential and also the Yukawa potential with spherical symmetry during our study of Quantum Field Theory).

		\item The law of equality between action and reaction.

		\item The conservation of total linear momentum of a system!!!
	\end{enumerate}
	As consequence of the point (3) the origin of the space is unobservable (since conservation of linear momentum is equivalent to translation invariance in space)!
	
	\paragraph{Invariance by rotation in space}\mbox{}\\\\
	Let us request now that rotations about a fixed point are symmetrics. This property must be true whatever the considered fixed point, in particular, if the fixed point is precisely the position of one of the particles in the system. It follows that the potential necessarily follows a spherical symmetry. The forces acting between the particles are then collinear to vectors that link them.

	The angular momentum of the system is the expressed as follows (\SeeChapter{see section of Classical Mechanics page \pageref{angular momentum}}):
	
	The derivative with respect to the time of the total angular momentum gives:

	
	But the last term of the vector product can be written:
	
	where the $\vec{F}_{ij}$ are the internal system forces of the particle $j$ acting on the particle $i$. 

	The prior-previous expression the becomes:
	
	We can group the terms $\vec{r}_j\times\vec{F}_{ji}$ and $\vec{r}_i\times\vec{F}_{ij}$ terms by pairs and by the property of the cross product we have necessarily:
	
	So we conclude that the angular momentum is conserved and that the conservation of angular momentum is thus equivalent to the rotational invariance.

	Consequence: there is no preferred direction in space!
	
	\paragraph{Invariance by translation in time}\mbox{}\\\\
	The total energy of a system is the sum of the kinetic energy of its particles and the potential energy resulting from the mutual interaction of this same particles, either in the form of classical mechanics:
	
	We will assume that the potential $V_{i,j}(\vec{r}_i,\vec{r}_j)$ does not vary with time. This assumption is justified empirically by the fact that potential observed in nature are independent of time in closed systems at equilibrium.

 Let us calculate the derivative of total energy with respect to time:
	
	But if the system is closed (no external mass input or external energy input), the term $\mathrm{d}m_i/\mathrm{d}t$ is zero (no relativistic mass change either because the speed of each particle or the entire system is constant or its variation is in average equal to zero or just because we are in a comoving reference frame). It is the same for the term $\mathrm{d}\dot{\vec{r}}_i/\mathrm{d}t$ where if the system is closed  (no external mass input or external energy input) the average acceleration of each particle or the entire system relatively the center of gravity will be zero. So:
	
	So we conclude that the total energy of the system is a constant!

	Ask the question of what is the mechanical quantity invariant by translation is equivalent to to ask what are the mathematical quantities which are unchanged when we apply to them a translation. There are two of them: the scalars and vectors.

	Intuitively, a scalar is treated as a real number (\SeeChapter{see section Numbers page \pageref{scalar}}). But, in mechanics, the real numbers that we can construct are made from vector quantities as the vectors position, velocity, etc. For such a real number to have the status of scalar it must be independent of space. Thus, a position vector can obviously not be be regarded as a scalar. The energy of the particle is a real number, but is not a scalar because as it depends explicitly, in its formulation, of the position vector.

	Similarly, a vector is not only a mathematical entity having its components in a base. To have the status of vector, a mathematical entity must be transformed in the same way as the basis vectors of the vector space. By this definition, the angular momentum is not a true vector as, being the composition by the cross product of two vectors, it does not change as the basis vectors. From a mathematical point of view it is a pseudo-vector (\SeeChapter{see section Vector Calculus page \pageref{pseudo vector}}).

	The only real vector that remains is the linear momentum as it is built using the derivative of the position vector which is, of course, a real vector. We conclude that the only quantity likely to be retained by translation is the total linear momentum of the system.

	By similar reasoning as the above one, it is possible to assume what quantity could be invariant under rotation. Knowing that only scalar and some pseudo-vectors are indeed invariant under rotation, we conclude that the only quantity that can be retained when rotations occurs is the total angular momentum of the system.

	Finally, still by the same type reasoning, the invariance of the laws of mechanics by time translation is equivalent to search the quantities conserved by a translation in time. These quantities are the real scalar and vector on the time-line. No mechanical quantity can be treated as a vector on the time-line. However, energy is a scalar indeed translationally invariant in time, since the potential energy is independent by hypothesis of the time. The invariance of the laws of mechanics by translation in time therefore suggests intuitively the conservation of energy.

	These reasoning can obviously not be used as a proof but they show a close relation between geometry and invariance properties of a system.
	
	\paragraph{Noether's theorem}\mbox{}\\\\
	We will now prove that Noether's (first) theorem states that every differentiable symmetry of the action of a physical system has a corresponding conservation law.
	
	For this purpose let us consider $L$ the Lagrangian (\SeeChapter{see section of Analytical Mechanics page \pageref{lagrangian formalism}}) of a system represented by the $2n$ generalized coordinates $q_i,\dot{q}_i$ (in a space of dimension $n$) in the configuration space. Let us suppose that this system is invariant under the following  infinitesimal transformation denoted $h_s$ :
	
	Where $s$ is a real number (rotation angle, translation distance, or translation time value) and continuous parameter and for which we have:
	
	The function $h_s(q)$ acts continuously on on the variational path following the intellectual approach that will be stated in the section of Analytical Mechanics.

	Let us suppose that the functions $q_i,\dot{q}_i$ are solutions of the Lagrange equations (which we will prove in the section of Analytical Mechanics). According to our assumptions the functions (defined):
	
	are therefore necessarily also solutions of the Lagrange equations, which is translated explicitly by (we omit the indication of the sum after to make the notation more light...!):
	
	Moreover, by hypothesis, the Lagrangian is invariant for the transformations of the type described by $h_s:q_i\rightarrow {q'}_i$. It follows that its derivative with respect to the parameter $s$ is necessarily zero:
	
	And we will prove in the section of Analytical Mechanics (under the form of an integral being equal to zero) the relation:
	
	which can finally be written:
	
	by dividing the both sides by $\mathrm{d}s$:
	
	but we also have from the Euler-Lagrange equation (\SeeChapter{see section Analytical Mechanics page \pageref{euler lagrange}}):
	
	We then get:
	
	So the quantity:
	
	is a constant of the movement!
	
	The Noether's theorem then is stated as follows:
	
	Given a system with a Lagrangian $L(q_i,\dot{q}_i,t)$ on which we apply an infinitesimal transformation $h_s:q_i\rightarrow {q'}_s$, where $s$ is a real and continuous parameter (angle rotation, distance translation, time value translation). Then there exist a constant of motion denoted $C({q'}_i,\dot{q'}_i,t)$ whose expression is given by:
	
	in other words:  every differentiable symmetry of the action of a physical system has a corresponding conservation law.
	
	Let us apply now the Noether's theorem to the cases previously studied. Let us fix an arbitrary Cartesian  reference frame of origin O. Let us write $(\vec{e}_1,\vec{e}_2,\vec{e}_3)$ the orthonormal basis of this reference frame. Let us consider a system of $n$ particles identified in O by their position vectors $\vec{r}_i$. The Lagrangian of this system is then $L(\vec{r}_{j,i},\dot{\vec{r}}_{j,i},t)$ where $j=x,y,z$ distinguishes the spatial components of the vectors $\vec{r}_i,\dot{\vec{r}}_i$.
	
	Let us suppose now that the system is invariant by translation of distance $s$ along the $x$-axis only. The translation along that axis will be written as follows:
	
	The constant of the motion given by application of Noether theorem is then (still following the $x$ axis):
	
	We will see in the section of Analytical Mechanics analytical that $\partial L/\partial \dot{r}_{x,j}$ is the conjugate momentum $p_{x,i}$. We conclude therefore that the conserved quantity following Noether's theorem is: $p_{x,i}$ that is to say, the total linear momentum of the system along the $x$ axis !!!
	
	If we did the same with the other axes, we would easily prove also the conservation of total momentum along the axes, allowing us to conclude that in the general case of an infinitesimal translation:
	
	the conserved quantity is then the total momentum of the system.

	Let us suppose now that the system is invariant under rotation of an infinitesimal angle $s$ around the $z$-axis. This rotation will be written (for the cross product imagine in your head what happen in the space... if you don't see let us know and we will add a figure):
	
	and it is therefore a vector.

	And we also get:
	
	Once again, knowing that:
	
	the conserved quantity obtained by application of Noether's theorem is therefore:
	
	and we have shown in the section of Vector Calculus page \pageref{differential operators identities} that:
	
	which brings us to write:
	
	We prove in the same way the invariance of the Lagrangian under rotations in the other axes which leads to the conservation of the components along these axes of the total angular momentum of the system.

	In conclusion, we have identified three well know different readings of the laws of physics (we omit the gauge transformations):
	\begin{table}[H]
		\begin{center}
		\definecolor{gris}{gray}{0.85}
		\begin{tabular}{|l|l|l|}
		\hline
		\multicolumn{1}{c}{\cellcolor{black!30}\textbf{Observation}} & 
		\multicolumn{1}{c}{\cellcolor{black!30}\textbf{Conservation Law}} & 
		\multicolumn{1}{c}{\cellcolor{black!30}\textbf{Physical interpretation}}  \\ \hline
		\parbox{4.5cm}{Invariance of physics laws\\ by translation} & Conservation of momentum &  \parbox{4.5cm}{Homogeneity of space: space has the same properties in all its points} \\ \hline
		\parbox{4.5cm}{Invariance of physics laws\\ by rotation} & Conservation of angular momentum &  \parbox{4.5cm}{Isotropy of space: space has the same properties in all directions} \\ \hline
		\parbox{4.5cm}{Invariance of physics laws\\ by translation in time} & Conservation of Energy & \parbox{4.5cm}{Homogeneity of the time: the laws of nature do not change over time} \\ \hline
		\end{tabular}
		\end{center}
		\caption{Conservation Laws}
	\end{table}
	In other words, the Universe seems to be:
	\begin{enumerate}
		\item[P1.] Homogeneous (no origin of time, or space, that is privileged) that is to say invariant under translation

		\item[P2.] Isotropic (no preferred direction) that is to say invariant under rotations
	\end{enumerate} 
	\begin{figure}[H]
		\centering
		\includegraphics[scale=0.8]{img/mechanics/isotropic_vs_homogeneous.jpg}
		\caption{Isotropic vs Homogeneous}
	\end{figure}
	
	\pagebreak
	\subsubsection{Curie's Principle}\label{curie principle}
	The Curie's principle (which we owe to Pierre Curie) arises somewhat intuitively from Noether's theorem and is stated as following: If a cause has a certain symmetry or invariance, then its effect will be the same symmetry (or the same invariance), or a higher symmetry, provided that the problem of the solution is unique.
	
	\begin{tcolorbox}[title=Remark,colframe=black,arc=10pt]
	Note that the elements of simmetries act on the directions of the vectorial quantities, while invariance act on variables that depend on these quantities.
	\end{tcolorbox}
	Thus, in a homogeneous and isotropic space, if we apply a geometric transformation to a physical system that is able to create certain effects (forces, fields), then these effects undergo the same transformation.

	In other words, if a physical system $\mathcal{S}$ has a certain degree of symmetry, then we can infer the effects created by the system at a point from effects at another point.

	Here are the $6$ properties of symmetries derived from the Curie's principle:
	
	\begin{enumerate}
		\item[P1.] Invariance by translation: if S is invariant throughout translation parallel to an axis, the effects are independent of the coordinate axis (interest is then work in Cartesian coordinates).

		\item[P2.] Axial symmetry: if $\mathcal{S}$ is invariant in rotation about a given axis, then its effects do not depend on the angle that defines the rotation (the interest beign then to work in cylindrical coordinates).

		\item[P3.] Cylindrical symmetry: if $\mathcal{S}$  is invariant by translation AND rotation, then its effects depend only on the distance from the axis of rotation (the interest being then to work in cylindrical coordinates).

		\item[P4.] Spherical symmetrical: if $\mathcal{S}$ invariant in rotation around a fixed point, then its effects depend only on the distance to the fixed point (the interest being then to in spherical coordinates).

		\item[P5.] Plane of symmetry: if $\mathcal{S}$ has a plane of symmetry, then at any point of this plane:
		\begin{itemize}
			\item an effect that has a vectorial origin is contained in the plane

			\item an effect that has pseudo-vector character  (see the section of Vector Calculus for the definition of a pseudo-vector) is perpendicular to the plane
		\end{itemize}

		\item[P6.] Plane of antisymmetry: if, by symmetry with respect to a plane,  $\mathcal{S}$ is converted to  $-\mathcal{S}$ then at any point of this plane:
		\begin{itemize}
			\item  an effect that has a vectorial origin is perpendicular to the plane

			\item an effect that has pseudo-vector character is contained in the plane  
		\end{itemize}
	\end{enumerate}
	
	\subsection{Point Spaces}\label{point spaces}
	We have hesitated quite long to write the text below as it already partially in the section of Vector Calculus and respectively of Tensor Calculus. But finally do a summary of the main concepts of these both fields only for Classical Mechanics seems to be perhaps a good choice or at least a good reminder!
	
	The study of physical phenomena use initially their representation in the space of classical Euclidean geometry with one time dimension and any number of spatial dimensions.

	The vectors we studied in the section of Vector Calculus (tensor of order 1) and the tensors (of any order) we studied also in the section of Tensor Calculus can as we have already mentioned, be used to connect each point in space-time to the origin of a given repository or reference frame, thereby forming vector and/or tensor fields. This mathematical fact, requires the mathematical definition of spaces formed by points also named "\NewTerm{point spaces}\index{point spaces}" or "\NewTerm{punctual spaces}\index{punctual spaces}".
	
	The precise definition of punctual vector space-time that we will give will be built from the concept of vector space that we saw in the section of Vector Calculus.

	Let us see first the particular example of the point space formed by triplets of numbers that is directly derived from the conventional three-dimensional euclidean geometrical space $\mathcal{E}^3$.

	So, let us take two triplets of numbers denoted by:
	
	etc. 

	Let us denote by $\mathcal{E}^{'3}$ the set of all the elements $\{\vec{A}, \vec{B}, ...\}$ of formed by triplets of numbers. At any pair $(\vec{A}, \vec{B})$ of two elements of two elements of $\mathcal{E}^{'3}$, taken in this order (because in physics the time has a direction!!! the vectors have a given order), we can make correspond a vector $\vec{x}$ belonging to a vector space $\mathcal{E}^3$, and denoted geometrically by $\overrightarrow{AB}$ and defining this latter space by a triplet of numbers such that (we use the index notation as studied in the section of Tensor Calculus):
		
	with $i=1,2,3$.
	
	We then have:
	
	
	Indeed, the choice of this expression comes once again from the fact that for example position vectors in physics are ordered by the time $t$:
	\begin{figure}[H]
		\centering
		\includegraphics[scale=0.9]{img/mechanics/punctual_spaces_ordered_vectors.jpg}
	\end{figure}
	But:
	\begin{figure}[H]
		\centering
		\includegraphics{img/mechanics/punctual_spaces_ordered_vectors_correct.jpg}
	\end{figure}
	If we define relatively to this element the addition and the multiplication by a scalar, we fall back on as we have already see it in the section of Set Theory with a vector space structure.

	The correspondence we thus build between any ordered pair of elements $(\vec{A}, \vec{B})$ of two elements of $\mathcal{E}^{'3}$ and a vector of a vector space of $\mathcal{E}^{3}$ , obviously satisfies the following properties:
	\begin{enumerate}
		\item[P1.] Antisymmetry: $\overrightarrow{AB}=-\overrightarrow{BA}$

		\item[P2.] Associativity with respect to addition: $\overrightarrow{AC}=\overrightarrow{AB}+\overrightarrow{BC}$

		\item[P3.] If O (an origin) is an arbitrary point chosen in $\mathcal{E}^{'3}$, to any vector $\vec{x}$ of $\mathcal{E}^{3}$, there is a point $M$ and a unique one, such that $\overrightarrow{\text{O}M}=\vec{x}$.
	\end{enumerate}
	Mathematicians say that a space can be say to be a "\NewTerm{point space}" if and only if the associated vector space respect the previous three relations (which seems quite obvious in fact...). The elements of $\mathcal{E}^3$ are then obviously named "\NewTerm{points}".
	
	When $\mathcal{E}^3$ is associated with a dot product (\SeeChapter{see section Vector Calculus page \pageref{dot product}}) the we know that $\mathcal{E}^3$ will be say to be a "\NewTerm{pre-Euclidean point space}\index{pre-Euclidean point space}".

	Let us consider any point O of a pre-Euclidean point space and base $(\vec{e}_i)$ associated with it.
	
	\textbf{Definitions (\#\mydef):}
	\begin{enumerate}
		\item[D1.] We name "\NewTerm{reference of the space}\index{reference of the space}" $\mathcal{E}^3$ the set consisting of the elements O (origin) and of the base $(\vec{e}_i)$. This kind of reference frame is denoted:
	

		\item[D2.] The "\NewTerm{components}" or "\NewTerm{coordinates}" of a point $M$ of a pre-Euclidean space point $\mathcal{E}^3$, relative to the reference $(\vec{e}_i)$, are the (when not specified: contravariant\footnote{As we have seen in the section of Tensor Calculus the contravariant and covariant components are the same in an orthonormal base.}) components $x^i$ of the vector $\vec{x}=\overrightarrow{\text{O}M}$ relatively to the base $(\vec{e}_i)$.
	\end{enumerate}
	Given two points $M$ and $M'$ of say $\mathcal{E}^3$ (as in fact we know that we can choose any number $n$ of dimensions) defined by respective (contravariant) coordinates $x^i$ and ${x'}^{i}$, we the have as we know for the section Tensor Calculus:
	
	Using the properties P1 and P2 given above:
	
	We conclude that the components of the vector $\overrightarrow{MM'}$, in the base $(\vec{e}_i)$ are the $3$ quantities $({x'}^i-x^i)¨$ ($n$ quantities in the case of a $n$ dimension space), differences of the contravariant coordinates of the points $M$ and $M'$.
	
	Given $(\vec{e}_i)$ and $(\vec{e'}_j)$ any two basis of $\mathcal{E}^n$ and linked by the general relations of change of basis (see sections of Tensor Calculus page \pageref{change of basis tensor calculus} and Linear Algebra page \pageref{change of basis} for the details)
	
	Let us seek the relations between the coordinates of a point $M$ of $\mathcal{E}^n$ relatively to these two bases. For this, let us express the vector $\overrightarrow{\text{O}\text{O}'}$ and $\overrightarrow{\text{O}'\text{O}}$ on each of these bases:
	
	and same for the vectors $\overrightarrow{\text{O}M}$ and $\overrightarrow{\text{O}'M}$, thus:
	
	On another part we have:
	
	Identifying the result relatively to the vector $\vec{e}_i$ in the expression of $\overrightarrow{OM}$, we have:
	
	And in an analogue way:
	
	These two relations are more than useful in physics where we often have to consider a base in a given reference frame (so we can express the position of a point from either one or the other using these relations).

	Let us now consider a pre-Euclidean point space as well as $M$ and $M'$ two points of this space. We have proved in our study of Topology (\SeeChapter{see section Topology page \pageref{distance}}) that the norm of the vector $\overrightarrow{MM'}$ is a possible measure of the distance between $M$ and $M'$ respecting some interesting and useful axioms. So we have if we denote by $y_i$ the covariant components of $\overrightarrow{MM'}$:
	
	If the both points $M$ and $M'$ have respectively for contravariatn coordinates $x^i$ and ${x'}^i$ with respect to a base $(\vec{e}_i)$, we know following our definition above in the case of ordered vectors that:
	
	The squared norm is given as we have seen in our study of Tensor Calculus (\SeeChapter{see section Tensor Calculus page \pageref{norm tensor notation}}) by the more general explicit relation:
	
	If we denote by $\mathrm{d}s$ the distance between the points $M$ and $M'$. The above relation provides the expression of the square of the distance between these two points in the form:
	
	Let us also recall (\SeeChapter{see section Tensor Calculus page \pageref{condensed flat metric space notation}}) that for a pre-Euclidean point space where the basis vectors are orthonormal, we have:
	
	and the expression of the distance becomes:
	
	We thus get wan expression that generalizes to $n$ dimensions, the square of the elementary distance, relatively to an orthogonal Cartesian coordinate system in the space of classical geometry (Euclidean).
	
	The vectors in physics are generally functions of one or more variables, they may be variables of space or time. When to a point $M$ of a point space $\mathcal{E}^n$, we attach a tensor, defined by its components relatively to the base $(\vec{e}_i)$, we say that we have given ourselves a "\NewTerm{tensor field}\index{tensor field}" (the tensor fields of order $1$ being the vector fields as we have seen it in the section of Tensor Calculus).
	
	For $n$-dimensional vectors, the concept of derivative from a three-dimensional vector is generalized and we get all the classic relations relatively to derivatives\footnote{A vector derivative - like the gradient, divergence, laplacian, etc. - is a derivative taken with respect to a vector field. Vector derivatives are extremely important in physics, where they arise throughout fluid mechanics, electricity and magnetism, elasticity, and many other areas of theoretical and applied physics}.

	Let us consider a vector $\vec{x}$ belonging to a pre-Euclidean space $\mathcal{E}^2$ whose components on a base $(\vec{e}_i)$ are functions of some parameter $\alpha$ (typically the time). We will denote this vector $vec{x}(\alpha)$ and we will have:
	
	By definition, the derivative from the vector $\vec{x}$ relatively to the component $\alpha$  is a vector denoted:
	
	following the notation of most mathematicians. Or:
	
	following the tradition of some physicists. Or even:
	
	that we see most of time with engineers. Or more explicitly if we respect the pure mathematical notations:
	
	In this book, we switch to a notation to another without prior notice depending on the desire and obviously to simplify the notation (the reader will have to do with it).

	Since we are doing more mathematics than physics, let us write just now explicitly:
	
	Recalling (\SeeChapter{see section of Differential and Integral Calculus page \pageref{differential}}) that the differential is given by:
	
	The different expressions of the derivatives of  three-dimensional vectors relatively to the sum of vectors,dot product of two vectors are easily transferable to the case of $n$-dimensional vectors.
	
	If a vector $\vec{x}$ of $\mathcal{E}^n$ depends on several independent parameters, $\alpha$, 
$\beta$, $\gamma$, the partial derivative of $\vec{x}(\alpha,\beta,\gamma)$ vector with respect to the variable $\alpha$, for example, is a vector denoted:
	
	whose components are the partial derivatives of the  components of $\vec{x}$, namely:
	
	The total differential of the vector $\vec{x}(\alpha,\beta,\gamma)$ being written (\SeeChapter{see section  of Differential and Integral Calculus page \pageref{total exact differential}}):
	
	In a bases $(\vec{e}_i)$ we will consider as obvious that if a vector $\vec{x}$ is dependent of a parameter $\alpha$, that its associated point $M$ such that $\vec{x}=\overrightarrow{\text{O}M}$ is also dependent of $\alpha$. So if $\vec{x}$ admits a derivative, it is also same of $\overrightarrow{\text{O}M}$!
	
	\begin{theorem}
	The derivative $\vec{x}'(\alpha)$ of a vector $\vec{x}$ does not depend on the (static) origin point O of the base but only of the considerated point $M$.
	\end{theorem}
	\begin{dem}
	Indeed, if O' is another point of origin, we have:
	
	and since the vector $\overrightarrow{\text{OO}'}$ is fixed and does not depend on $\alpha$, we have:
	
	hence:
	
	\begin{flushright}
		$\square$  Q.E.D.
	\end{flushright}
	\end{dem}
	We can write the derivative of the vector $\overrightarrow{\text{O}M}$ by only mentioning the point $M$ as it is the usage in physics and therefore write:
	
	The differential of $\overrightarrow{\text{O}M}$ is then written:
	
	If a point $M$ of $\mathcal{E}^n$ is associated, relatively to base $(\vec{e}_i)$, to a vector $\vec{x}(\alpha,\beta,\gamma)=\overrightarrow{OM}$, the partial derivatives of $\overrightarrow{OM}$ will depend only of the point $M$ and we will write, for example:
	
	To simplify the expressions of total partial derivatives of the functions dependent of $n$ independent variables we use, when the context permits, the index-based notations as seen in the section of Tensor Calculus. Therefore:
	
 	for a function $f$ dependent of $n$ independent variables $x^n$ (this includes the times in physics). There partial derivatives are written into the form:
	
	The second partial derivatives with respect to the variables $y^i$ and $y^k$ will be written:
	
	When $\vec{x}$ is a vector such that $\vec{x}=x^i\vec{e}_i$ whose components are dependent of $n$ independent variables $y^k$ such that:
	
	the partial derivatives of the vector will then be written, using the summation convention:
	
	The concept of space-time being now introduced, we can now proceed to study the Lagrangian formalism and determining the mathematical formulation of the principle of least action (see next section).

	\begin{flushright}
	\begin{tabular}{l c}
	\circled{95} & \pbox{20cm}{\score{3}{5} \\ {\tiny 71 votes,  67.32\%}} 
	\end{tabular} 
	\end{flushright}

	%to make section start on odd page
	\newpage
	\thispagestyle{empty}
	\mbox{}	
	\section{Analytical/Lagrangian Mechanics}\label{lagrangian mechanics}
	 \lettrine[lines=4]{\color{BrickRed}A}nalytical Mechanics is a reformulation of classical mechanics that we due the present form, also named "\NewTerm{Lagrangian mechanics}\index{Lagrangian mechanics}", from the work of Bernoulli brothers and especially Euler and Lagrange. It is indeed in 1696 that began the story of the true theoretical physics.
	 
	 In fact, the analytical mechanics starts from the event from the following observation (in the 17th century): Any conservative system seems to move from one state to another still using the simplest means and maintaining a constant magnitude between the two states.
	 
	\begin{tcolorbox}[title=Remarks,colframe=black,arc=10pt]
	\textbf{R1.} The above means can be: the shortest path, the fastest path (that is to say the spatio-temporal paths with lower amplitudes...).\\
	
	\textbf{R2.} According to the first fundamental principle of physics, the constant quantity is selected as being the total energy.
	\end{tcolorbox}
	This statement is named in the context of the Mechanics the "\NewTerm{principle of least action (of Maupertuis)}\index{principle of least action}" or as part of general physics "\NewTerm{variational principle}\index{variational principle}" or sometimes as part of optics "\NewTerm{principle of economy}\index{principle of economy}" or "\NewTerm{Fermat's principle}\index{Fermat's principle}". In the purely mathematical abstraction framework we speak of "\NewTerm{Hamilton's principle}\index{Hamilton's principle}".
	
	More technically, it is also formulated as follows: A system moves from one configuration to another so that the variation of the action (see below for details) between the natural path actually followed and any infinitely close and virtual trajectory having the same starts and ends in space and time is zero.
	
	In fact, although this statement may seem as consistent, it raises doubts ... but we will see that:
	\begin{enumerate}
		\item In classical mechanics we can prove the Newton's first law by admitting this principle as true and by superimposing the principle of conservation of energy and we can therefore also explain the wobbling motion of almost any simple solid.
		
		\item In electromagnetism, we will find back all the Maxwell equations (verbatim the Biot-Savart law, the Faraday law, the Lorentz force, the Laplace force, etc.) from the properties of the principle of least action and energy conservation.
		
		\item In optics, we will prove that the path followed by the light is always the shortest and this will allow us to prove Fermat's principle underlying any geometrical optics. Also in General Relativity this same assumption will take us to prove the deflection of light near massive bodies.
		
		\item In atomic physics, the properties of the principle of least action will allow us to determine some mathematical properties of atoms and other particles (fermions and bosons in quantum filed physics).
		
		\item The principle of least action will also enable us to prove that all bodies, with or without mass, is deflected by a field of acceleration and ... so will give us the opportunity to determine the Einstein Field Equation  that is the basis of the section on General Relativity.
		
		\item This principle also applies to get powerful results in geometry as we shall see a little further. Thus, the analytical mechanics techniques are very intimately related to pure mathematics.
		
		\item We can merge all of our current knowledge of electromagnetism, quantum physics and gravitation in a single formulation (obviously long ...) using the Lagrangian formalism.
	\end{enumerate}
	
	It needless to say that through this small examples the huge applications of this principle!!!

	Historically, it is interesting to know that this is Pierre-Louis Moreau de Maupertuis who first articulated the principle of least action in shortly scientific form. The intervention of Euler and Lagrange in this area has been to put this principle into a mathematical form and to prove that it follows from a simple mathematical property of optima of continuous functions. It is needless to say that this latter fact that allowed to re-prove all the laws of classical physics from only pure mathematical concepts has upset at this time some people... (and still do today!).
	
	This principle was (and always has) a significant impact and the problem was to apply the mathematical expression of the latter to all physical phenomena that had already been experimentally demonstrated and empirically at that time. Perform this proof wassimilar to explain why such phenomenon or such law was thus rather than otherwise. Imagine!
	
	For example given locality, Lorentz invariance (\SeeChapter{see section Special Relativity page \pageref{lorentz invariance}}) and know physical data since 1860 it can be shown that the Lagrangian describing all observed physical process (without gravity) can be written:
	\begin{figure}[H]
		\centering
		\includegraphics[scale=0.8]{img/mechanics/lagrangien_quantum_field_theory.jpg}
	\end{figure}
	
	Thus, as far as we know, it seems that the first to tackle the problem was the Bâlois (Switzerland) Leonhard Euler. But we have also kept the name of Lagrange (hence the expression "\NewTerm{Lagrangian formalism}\index{Lagrangian formalism}") to define the entire method and the mathematical formalism built around the principle of least action.
	
	Inf fact we will see in the section the differences between:
	\begin{enumerate}
		\item Newtonian Mechanics (or Newtonian Formalism)
		\item Lagrangian Mechanics (or Lagrangian Formalism)
		\item Hamiltonian Mechanics (or Hamiltonian Formalism)
	\end{enumerate}
	and we will see that $1$ is a special case of $3$ which is a special case of $2$. We will also prove with in some context (as quantum physics) the Hamiltonien is sometimes preferred to the Lagrangian an vice versa.
	
	\subsection{Lagrangian formalism}\label{lagrangian formalism}
	Classical mechanics can be formulated in different ways. The most common is the formulation of Newton, which uses the concept of force (\SeeChapter{see section Classical Mechanics page \pageref{newton laws}}). It is by far the easiest when it comes to consider a specific issue and that's why this is the one that is taught in small classes. But to treat more complex problems or more precisely and to be able to rigorous proofs the Newtom formulation is not the most practical one.
	
	Analytical mechanics, initiated in the 18th century, regroups therefore very different mathematized formulations of classical mechanics, including Hamiltonian and Lagrangian mechanis (all these formulations are equivalent!).
	
	This formulation is taught only a little bit in small schools because we must admit the Lagrangian and Hamiltonian formalism (therefore containing the principle of least action in mathematical form) uses an abstraction level a bit higher than the normal methods and despite being often of great help in the development of theories (basic physics, quantum physics, General Relativity, quantum field theory, superstring theory), it rarely follows new discovers from it (but rather useful and powerful reduction and validation method).
	
	Let's begin our work:
	
	\subsubsection{Generalized coordinates and frames}\label{general coordinates}
	A natural reflex usually leads to refer to the position of a point in space with the only knowledge of its three Cartesian coordinates $x, y, z$ (or polar, cylindrical, spherical, etc.). This attitude is also often justified by the simplicity of a lot of situations encountered in practice where it is not necessary to search for more elaborate methods or to go in other coordinate systems (\SeeChapter{see section Vector Calculus page \pageref{system of coordinates}}).
	
	To locate the position of a mobile (or a material point) in physics it is necessary at first to associate a frame to the repository. Thus, a "\NewTerm{frame}\index{frame}" is a system (concrete physical) of location (or "\NewTerm{reference system}"\index{reference system}) in the space associated with the repository.
	
	\begin{tcolorbox}[colframe=black,colback=white,sharp corners]
	\textbf{{\Large \ding{45}}Examples:}\\\\
	E1. In astronomy, for many problems, the Earths is taken as repository, but on the center of it we put the reference frame (most of time in spherical coordinates).\\
	
	E2. In mechanics a body is taken as a repository but it is on its inertial center is that we put the reference frame.
	\end{tcolorbox}
	
	Conventional frames in classical mechanics are mostly canonical oriented pre-Euclidean spaces basis (\SeeChapter{see section Vector Calculus page \pageref{pre euclidean vector space}}) and where each point or each vector of the space, can be represented algebraically by its affixes values (the value in the ordinate corresponding to the projection on the vertical axis) and its abscissas values (corresponding to the projection on the horizontal axis).
	
	Here are some trivial examples:
	\begin{figure}[H]
		\centering
		\includegraphics{img/mechanics/referential_frame.jpg}
		\caption[]{Movements in referential frames from to $1$, $2$ or $3$ dimensions}
	\end{figure}
	\textbf{Definitions (\#\mydef):}
	\begin{enumerate}
		\item[D1.] A repository, assimilated to a reference frame, is said to be a "\NewTerm{Galilean reference frame}\index{Galilean reference frame}\label{galilean reference frame}" (it's rare that we make explicit mention of in physics for lack of rigor) if:
		\begin{itemize}
			\item We can consider it as motionless during the study of the movement of the system or to be in an uniform rectilinear translation relative to another repository itself motionless (this is also named an "\NewTerm{inertial reference frame}\index{inertial reference frame}" because not accelerated).
			
			So if we neglect the rotational movement of the Sun around the center of the galaxy, then the heliocentric reference frame may be regarded as a Galilean reference frame. If we neglect the rotational movement of the Earth around the Sun, then the geocentric reference may be regarded as a Galilean reference frame. If we neglect the rotational motion of the Earth itself, then the terrestrial reference frame can be considered as Galilean. In many mechanical experiments on the surface of the Earth, we find that the terrestrial reference frame can be considered as Galilean with very good accuracy for short time experiments. Fortunately there are still a lot of phenomena which must take into account the rotation of the Earth (deviation to the east, Foucault pendulum, etc.)
			
			\item We can consider it as a system in which Newton's laws are satisfied (\SeeChapter{see section Classical Mechanics page \pageref{newton laws}}).
		\end{itemize}
		
		\item[D2.] A repository, assimilated to a reference frame, is named "\NewTerm{barycentric reference frame}\index{barycentric reference frame}" (\SeeChapter{see section Euclidean Geometry page \pageref{barycenter}}) if it has for origin the center of mass (\SeeChapter{see section of Classical Mechanics page \pageref{center of mass}}) of the body being studied.
		
		Therefore, the "\NewTerm{Copernicus reference frame}\index{Copernicus reference frame}" is assimilated to the center of gravity (inertia) of the solar system, the "\NewTerm{heliocentric reference frame}\index{heliocentric reference frame}" also named "\NewTerm{Kepler's reference frame}\index{Kepler's reference frame}" to the Sun center of mass.
		
		\item[D3.] A repository, assimilated to a reference frame, is named "\NewTerm{geocentric reference frame}\index{geocentric reference frame}" when we make reference to a system of axes placed at the center of mass of the Earth. The axes, parallel to those of a Copernicus reference frame, point in the direction of to three fixed stars. In this reference the Earth rotates on itself in $24$h.
		
		\item[D4.] A repository, assimilated to a reference frame, is named "\NewTerm{Terrestrial reference frame}\index{Terrestrial reference frame}" when we make reference to a system of axes placed at the center of mass of the Earth that has a uniform rotational movement corresponding to the speed of rotation of the Earth . Traditionally one of the axes is directed to the pole star. This is the reference framce that we refer to as in real life it is not strictly speaking a Galilean reference frame! This will induce specific effects on the movements in the atmosphere such as we experience them.
	\end{enumerate}
	\begin{tcolorbox}[title=Remarks,colframe=black,arc=10pt]
	\textbf{R1.} In the $2$-dimensions case, say that an orthonormal basis $(\text{O},\vec{i},\vec{j})$ is a "\NewTerm{direct reference frame}\index{direct reference frame}" means that the oriented angle of $(\vec{i},\vec{j})$ has for main value $\pi/2$ (clockwise). Say that an orthonormal basis  $(\text{O},\vec{i},\vec{j})$ is an "\NewTerm{indirect reference frame}\index{indirect reference frame}" means that the angle has for main value $-\pi/2$. In all what will follow, if we do not specify the orientation, this suggests that $(\text{O},\vec{i},\vec{j})$ is a direct reference frame.\\
	
	\textbf{R2.} Obviously, there are a number of arbitrary choices to make when we set up a reference frame: for example, where to place the origin of the spatial coordinates, how to orient the axes, and where to put the origin of time. But this freedom is not a real issue, indeed if we make a choice, there are mathematical techniques (\SeeChapter{see section Linear Algebra page \pageref{change of basis}}) that can be used to move from an old system (basis) to a new one.
	\end{tcolorbox}
	It is true that the three parameters $x$, $y$, $z$ are perfectly sufficient to identify a material point in ordinary space as we have already mentioned it in our study punctual spaces (\SeeChapter{see section Principles page \pageref{point spaces}}), but nevertheless it is sometimes inevitable, or just more advantageous to use a larger number of parameters than three. We can obviously consider all sorts of settings to achieve the coordinates of a point in space, so that in a more generalized manner we shall have to take into account relation of the type (we do not keep the same writing that the one we had in our study punctual spaces for consistency with the many existing references on the subject):
	
	The parameters $q^1,q^2,\ldots,q^n$ are named "\NewTerm{generalized coordinates}\index{generalized coordinates}" and are the parameters to which a problem will most often referred. Knowing their expression over time is a fundamental problem of dynamics (think for example a satellite wich coordinates $x,y,z$ are dependant on spherical components $\varphi,\theta,r$ that are themselves depending of the time $t$). This means that we will have reach a solution when we will have the independent relation:
	
	Geometrically these components can be lengths along straight lines, or arc lengths along curves, or angles; not necessarily Cartesian coordinates or other standard orthogonal coordinates. There is one for each degree of freedom, so the number of generalized coordinates equals the number of "\NewTerm{degrees of freedom}\index{degrees of freedom}", $n$. In physics a "degree of freedom" corresponds to a quantity that changes the configuration of the system, for example the angle of a pendulum, or the arc length traversed by a bead along a wire. etc-
	
	It is therefore important to notice that the number of parameters $q^i$ defining locating a point in space is at least equal to three in most common cases. It is ultimately the nature of the envisaged situations that suggest the choice of the number of parameters to be used (cartesian coordinates, cylindrical, spherical, ...).
	
	The vector:
	
	 is a point in the "\NewTerm{configuration space}\index{configuration space}" of the system.
	 
	 Indeed, is quite natural mathematically speaking to associate the manipulation of the $n$ generalized coordinates $q^i$ to a $n$-dimensional space, in which the $q^i$ would appear as the coordinates of a point $P$ representative of the configuration of any system. This is why we name this space the "configuration space" and is denoted $\mathcal{C}^n$.
	 
	 But the rigour of mathematical-physical, brings us to provide a more accurate description of the phenomena by adding the important variable that is the time, often seen as independent variable in the $q^i$. We thus arrive inevitably to use another configuration space $\mathcal{C}^{n+1}$ that we name "\NewTerm{event space}\index{event space}".
	 
	 This configuration space is of great interest to many problems of modern science and is particularly well suited to the relativistic problems. The independent variables constituting the spatial and temporal coordinates then form what we name the "\NewTerm{Euler variables}\index{Euler variables}" (as we will see later, they have this name because we can found them in the Euler-Lagrange equation).
	 
	 The components $x^1,x^2,\ldots,x^n$, that can be seen as functions as we have just see, will be assumed now defined, continuous, of class $\mathcal{C}^2$ (to work with the acceleration) relatively to the parameters $q^1,q^2, \ldots, q^n$ and will lead to a Jacobian different from zero (\SeeChapter{see section Differential and Integral Calculus page \pageref{jacobian}}).
	 
	\textbf{Definitions (\#\mydef):}
	\begin{enumerate}
		\item[D1.] A "\NewTerm{holonomic constraint}\index{holonomic constraint}" is a constraint equation of the form for particle:
		
		which connects all the 3 spatial coordinates of that particle together, so they are not independent and founding such a function can help to solve the study of a dynamic system.
		
		\item[D2.] A mechanical system can involve constraints on both the generalized coordinates and their derivatives. Constraints of this type are known as "\NewTerm{non-holonomic constraint}\index{non-holonomic constraint}" and are of the form:
		
	\end{enumerate}
	To study continuum medium (radically different concept from a material point), we will have however two different approaches:
	\begin{enumerate}
		\item The Lagrange method or Hilbert Method: we seek to characterize the movement of the medium described by a Lagrangian formulation consisting in characterizing it by a system of equations in the Newtonian sense. For derivatives, we the have the speed and acceleration of the medium.

		\item The Euler method: Instead of following the path of a point, we focus on the evolution of a physical characteristics at a given point as the speed, temperature acceleration, pressure or other. We speak so often of "\NewTerm{Eulerian system}\index{Eulerian system}".
	\end{enumerate}
	
	\subsubsection{Variational Principle}
	A "\NewTerm{variational principle}\index{variational principle}" is a scientific principle used within the calculus of variations, which develops general methods for finding functions which extremize the value of quantities that depend upon those functions. For example, to answer this question: "What is the shape of a chain suspended at both ends?" we can use the variational principle that the shape must minimize the gravitational potential energy.
	
	The "variational principle" is thus only the contemporary mathematical form of the principle of least action, which, as we have already mentioned, at the base of the Lagrangian formalism.
	
	Let us recall that according to the statement of the variational principle we must find in any physical phenomenon, an amount that is naturally optimized (minimized or maximized) and describing all the variables of the system studied and thus its outcome.
	
	Here are the steps that we will follow and once this approach presented, we will address its mathematical formalization.
	
	The proposals are the following:
	\begin{enumerate}
		\item[P1.] We assume the variational principle and the principle of conservation of energy as correct.

		\item[P2.] The total energy of a closed system is constant and consists only of the summation of the kinetic energy and potential energy.

		If we consider only kinetic energy, then the system is named "\NewTerm{free system}\index{free system}"; if both energies are considered, then we say that the system is a "\NewTerm{generalized system}\index{generalized system}".

		\item[P3.] We define a mathematical function (whose variables are the generalized coordinates) named "\NewTerm{Lagrangian}\index{Lagrangian}" which is given by the difference of the two previous mentjioned energies.

		\item[P4.] On the evolution of a system between two states, we seek for the properties of the function (Lagrangian) that minimize the variation of the difference of the two energies on temporal or metric evolution of the system.
	\end{enumerate}
	So to put it in a mathematically form, we first ask that there is a real function of $2n$ variables (the $n$ general coordinates and the corresponding $n$ derivatives to the implicit variable that will typically be in physics the time $t$):
	
	that we will name "\NewTerm{generalized Lagrangian}\index{generalized Lagrangian}" of the system, whose integral satisfies the following statement:

	In a natural movement starting from a point $A(q_1^A,\ldots,q_n^A)$ at the time $t_A$, arriving at the point $B(q_1^A,\ldots,q_n^A)$ at the time $t_B$ , the following integral (that is a pillar of modern physics!!!!!) is named "\NewTerm{action integral}\index{action integral}" or simply "\NewTerm{action}\index{action (analytical mechanics)}\label{action integral}":
	
	which may also be denoted more shortly:
	
	must be an extremal value (in fact, a "minimum" or "maximum"), since we could as well take $-L$ instead of $+L$ in the choice of the definition of the generalized Lagrangian).
	
	This "\NewTerm{action}\index{action}" or "\NewTerm{Lagrangian's action}\index{Lagrangian's action}\label{lagrangian action}" $S$ is what we commonly name in physics a "\NewTerm{functional}\index{functional}" and has units of energy multiplied by time since $L$ is energy.
	
	\subsubsection{Euler-Lagrange Equation}\label{euler lagrange}
	The principle of least action therefore states that the (integral) $S$ is extremal if:
	
	is the natural trajectory actually followed by the physical system.
	
	Let us consider then a very similar trajectory to the previous, that we will denote by:
	
	That is to say that for every $i$ we put:
	
	with:
	
	to ensure that we always start from the same point $A$ to reach the same point $B$. The perfect determination of the extreme points (characterized by variations of a path to another one identically zero) is a major assumption of the principle of least action!
	\begin{tcolorbox}[title=Remark,colframe=black,arc=10pt]
	Now we will omit writing the time arguments to facilitate the notations.
	\end{tcolorbox}
	If $S(q,\dot{q})$ is indeed the evolution of a system evolving following the principle of least action, then the action given by the variation:
	
	is zero for $\delta q$ and $\delta \dot{q}$ going toward zero (meaning that any physical system returns to its original state without external intervention).
	
	Which brings us to write:
	
	This allows us to justify the name of "\NewTerm{variational principle}\index{variational principle}" (also sometimes named the "\NewTerm{principle of stationarity of action}\index{principle of stationarity of action}"):
	
	This principle states that the path of a particle (or a more general system) is obtained by requiring a certain functional $S$ named "action" is stationary relative to a infinitesimal variation of the trajectory. In other words, if we make an infinitesimal variation of the trajectory, the change must be almost zero.
	
	\begin{figure}[H]
		\centering
		\includegraphics{img/mechanics/action_path.jpg}
		\caption{Variation idea of Least Action Principle}
	\end{figure}
	
	For a simple mechanical system, the action is therefore by evidence, by the principle of conservation of energy equal to the integral on the trajectory of (by definition of the Lagrangian) the difference between the kinetic energy and potential energy.
	
	Therefore, in a theory in which the forces derives from a potential $V$, we are naturally led to define the "\NewTerm{classic Lagrangian}\index{classic Lagrangian}" by (the reader must absolutely remember it!):
	
	where $T$ and $V$ are the traditional notation in the Lagrangian formalism for the kinetic energy and respectively potential energy given by (for a unit mass):
	
	The variational principle is then written in the conservative case (which facilitates the interpretation of the integral as an infinite sum of infinitesimal differences of the kinetic and potential energy between two instants and that must obviously be zero in a conservative system):
	
	Which implies that well known fact that in any conservative system any change of kinetic energy requires a contrary variation of potential energy (hence the fact that it is implicitly a negative sign):
	
	Caution! Strictly speaking, we do not always have the equality between the variational of the integral of the Lagrangian and the integral of the Lagrangian variational. Sometimes it can happen that:
	
	Indeed, while the integral on the right of inequality has a quite relatively easy interpretation since it is in physics since the infinite sum of infinitesimal differences between potential and kinetic energy in a conservative system is zero, the integral on the left is the inexact differential (\SeeChapter{see section Differential and Integral Calculus page \pageref{total inexact derivative}}) who will be zero if the Lagrangian does not depend on the way and therefore by extension does not explicitly depend on the variable $t$.
	\begin{tcolorbox}[title=Remark,colframe=black,arc=10pt]
	For the study of General Relativity, we are not looking for the variation of the difference of energy to be minimal as it is the case for mechanical systems, but the variation in the length $\mathrm{d}s$ of an arc (not time dependent unlike the previous example) in any space during a trajectory of a free system. Which bring us to simply write (remember it also because it will be very important!) the action:
	
	for a unit mass and taking the natural units.
	\end{tcolorbox}
	To return to our application of the variational principle in the case of the generalized Lagrangian, we can write the following exact total differential (\SeeChapter{see section Differential and Integral Calculus page \pageref{total inexact derivative}}) of $\delta L$ and then we get the relation:
	
	Let us integrate by parts (\SeeChapter{see section Differential and Integral Calculus page \pageref{integration by parts}}) the second term of the sum of the previous integral:
	
	The first term of the last equality is equal to zero:
	
	since we have already mentioned above that by construction we must have:
	
	The expression of the integral of least action can finally be written:
	
	But the as the $\delta q_i$ and $\dot{q}_i$ tend to $0$ in an infinite number of ways and we must have however $\delta S=0$. This means that each term in the sum of the integral can be taken independently and has to satisfy:
	
	But as the functions $\delta q_i$ and $\delta \dot{q}_i$ functions can always approach zero in many ways, and that this integral should be anyway equal to zero , we deduce that it is the integrands that are zero:
	
	These $n$ equations, satisfied by the generalized Lagrangian of the system to effectively track movement, are named "\NewTerm{Euler-Lagrange equations}\index{Euler-Lagrange equations}\label{equations of movement}", or more briefly (but rarely) "\NewTerm{Lagrange equations}\index{Lagrange equations}". These are, as we shall see, the equations of motion of the system: resolved, they give the effective evolution of the system over time:
	
	or more explicitly:
	
	And the reader must not forget that this in the case of a conservative system!!!!
	
	If the system is not conservative, by the analysis of units we have:
	
	where $F_i$ represents the generalized forces not arising from a potential (even if sometimes we consider a force arising from a potential as not arising from a potential as it is more easy to manipulate for students as it brings anyway to the same result when the force depends explicitly on the acceleration).
	
	The Lagrange equations should describe a particle motion. Given $L=T-V$ we should expect to recover Newton's second law for a particle moving under the influence of a force. We have (\SeeChapter{see section Classical Mechanics page \pageref{kinetic energy}}):
	
	It follows that in absence of a potential\label{free lagrangian}:
	
	Therefore:
	
	and:
	
	The non-conservative equation of motion is:
	
	where $q_1=x,q_2=y,q_3=z$ . The equation of motion then becomes:
	
	and identically for $y$ and $z$, meaning that we as expected revoered Newton's $2$nd law!
	
	\begin{tcolorbox}[colframe=black,colback=white,sharp corners]
	\textbf{{\Large \ding{45}}Example:}\\\\
	A common model of friction is:
	
	(see the study of trybology in the section of Classical Mechanics) where $v$ is the speed of the body in translational motion. Consider the corresponding "\NewTerm{Rayleigh dissipation function}\index{Rayleigh dissipation function}" (energy dissipation due to friction):
	
	If we use a Cartesian coordinate system, the generalized force due to friction, in $x$-direction for instance is given by:
	
	
	or as in Analytical Mechanics some physicists denoted $E_f$ by $\mathcal{F}$ we have also sometimes:
	
	Therefore through this example we see that we define the "\NewTerm{generalized force}\index{generalized force}" as:
	
	 So, the equations of motion in the case of a non-conservative force is written:
	
	If instead, the friction was proportional to the acceleration for some reason (as do a gravitational force for example but this latter is in the $V$ term as it arize from a potential), we could go one step further and define a new Lagrangian: 
	
	 and using Lagrange's equations on $L$ would give us the correct answer.
	\end{tcolorbox}
	Most of time when we use the Lagrangian formalism in quantum physics the studied system is in vacuum so the first boxed relation above applies but when the system if for example in a gravitational field then an external force $F$ is applied and then second boxed relation has to be applied (see the study of the gyroscope nutation in the section of Classical Mechanics for a detailed example).
	\begin{tcolorbox}[title=Remark,colframe=black,arc=10pt]
	This is while studying physics (the following sections of this book) that we can better understand the applications of this equation (obtained almost by purely mathematical developments !!!) and it becomes afterwards possible to understand its deep meaning. At our level of discourse, it is pointless to say anything. We must do physics, and physics again and again... to understand it well (learning by doing as we say!).
	\end{tcolorbox}

	So in the Lagrangian approach, we learn to reason from concepts of potential and kinetic energy instead of force concepts. The two approaches are obviously physically equivalent, but the energies are not vector quantities, they are conceptually easier to use in a wide range of situations. In quantum physics, for example, the notion of force has no meaning but energy concepts remains valid and same in General Relativity!!! This is one more reason to become familiar with their use. In addition, the force in the sense of Newton is instantenous distance action . In relativity, such a thing is impossible. The concept of force is therefore a purely classical macroscopic creation and contrary to our intuition, its interest is limited.

	Let us see a very first simple example of application of Euler-Lagrange's equation (other examples will be seen during our study of Newton's laws, of electrodynamics, special relativity, General Relativity, quantum physics fields, etc.):
	\begin{tcolorbox}[colframe=black,colback=white,sharp corners]
	\textbf{{\Large \ding{45}}Examples:}\\\\
	\label{euler lagrange example shortest path}E1. First, let us write in a conventional mathematical form the Euler-Lagrange equation  (the notation of the generalized coordinates is not the same in mathematics than in physics...):
	
	And now we take a simple practical mathematical example internationally known and very important (we will reuse the developments made below four our study of the Huygens pendulum later). \\

	The problem statement is: determine the shortest path between two points of a plane (we're guessing it's the straight line but must prove it!). In other words: to found the geodesic of a plane!\\
	
	This problem consist to find the shortest parametrized curve  $x(t),t\in[t_0,t_1]$that connects two points (caution with the variable $t$ has nothing to do with the time in this example!) such that:
	
	\end{tcolorbox}
	
	\begin{tcolorbox}[colframe=black,colback=white,sharp corners]
	Thus the infinitesimal length $\mathrm{d}L$ by applying Pythagoras is (differential curvilinear abcissa):
	
	Thus, the length $L$ of the parametric curve is given by\label{parametric curve length}:
	
	It is a relation we will often meet again in physics and math !!\\
	
	Thus, this problem, which the geometrical solution is very simple, is formulated as variational calculation problem as follows:
	
	Let us write the Euler-Lagrange of the solution to this problem, solution that if it exists, it has to satisfy.\\
	
	Therefore we have:
	
	The Euler-Lagrange equation therefore becomes in this :
	
	Therefore:
	
	where $C\in\mathbb{R}$ is a constant of integration (\SeeChapter{see section of Differential and Integral Calculus page \pageref{constant of integration}}). The latter equality implies that (we add an index for the constant that will avoid us to confused it with two other constants that will appear little further below):
	
	Returning to the notation used in the beginning of our problem statement:
	\end{tcolorbox}
	
	\begin{tcolorbox}[colframe=black,colback=white,sharp corners]
	
	Therefore:
	
	Therefore:
	
	and by integration it comes:
	
	This is the indeed the of a line named therefore the "\NewTerm{geodesic equation in cartesian coordinates}". This can be seen obviously by rewriting the constants as:
	
	Identical to the famous line equation:
	
	
	E2. We will now focus on an example frequently asked on Internet forums: what is the geodesic of an $\mathcal{S}^2$ sphere?\\
	
	For this, and without using Tensor Calculus (!), let us recall that in spherical coordinates (\SeeChapter{see section Vector Calculus page \pageref{spherical coordinates}}):
	
	Without loss of generality, we may take the sphere to be of unit constant radius. Therefore the prevous relations recude to:
	
	The length of a path from a point $A$ to $B$ on the sphere is the given by:
	
	So as before we the variational problem is equivalent to put that:
	
	Now applying the Euler-Lagrange equation:
	\end{tcolorbox}
	
	\begin{tcolorbox}[colframe=black,colback=white,sharp corners]
	
	in spherical coordinates:
	
	Explicitly:
	
	As:
	
	It remains:
	
	Therefore
	
	Hence:
	
	Therefore after squaring and rearranging we get:
	
	Hence:
	
	Finally:
	
	and the problem reduces to integrating this with respect to $\theta$ as $\phi'=\mathrm{d}\phi/\mathrm{d}\theta$:
	
	We substitute (\SeeChapter{see section Differential and Integral Calculus page \pageref{usual derivatives}}):
	\end{tcolorbox}
	
	\begin{tcolorbox}[colframe=black,colback=white,sharp corners]
	
	Then:
	
	But as:
	
	Therefore:
	
	Now we substitute:
	
	Hence (\SeeChapter{see section Differential and Integral Calculus page \pageref{usual primitives}}):
	
	where $C^{te}$ is a constant of integration that we will denote by $\phi_0$. Therefore:
	
	Since, for recall, $u=\cot(\theta)$, we have finally:
	
	Let us choose $\phi_0=\pi/2$. Then:
	
	Therefore:
	
	So this is the "\NewTerm{parametric equation of a geodesic on a $2$-sphere}" and we see that this is identical to relation of the "great circle" that we get in the section of Analytical Geometry during our study of the parametrization of a sphere in a three dimensional space.
	\end{tcolorbox}
	
	\pagebreak
	It seems to be personal preference, and all academic, whether you use the Lagrangian method $L=T-V$ or the $F = ma$ method. The two methods produce the same equations. However, in problems involving more than one variable (see typically the study of oscillating motions in the section of Classical Mechanics), it usually turns out to be much easier to write down $T$ and $V$, as opposed to writing down all the forces. This is because $T$ and $V$ are nice and simple scalars. The forces, on the other hand, are vectors, and it is easy to get confused if they point in various directions. The Lagrangian method has the advantage that once you've written down $L=T-V$, you don't have to think anymore. All you have to do is blindly take some derivatives.
	
	Therefore:
	
	and:
	
	
	\pagebreak
	\paragraph{Beltrami Identity}\mbox{}\\\\
	We will prove here a useful relation named "\NewTerm{Beltrami identity}\index{Beltrami identity}" which simplifies the application of the Euler-Lagrange equation in some particular situations!
	
	Let us first recall the Euler-Lagrange equation:
	
	Let us write the exact total differential (\SeeChapter{see section Differential and Integral Calculus page \pageref{total exact differential}}):
	
	What we will write under the form:
	
	Rearranging:
	
	Let us multiply before continuing the Euler-Lagrange by $\dot{q}_i$:
	
	and let us inject the prior previous relation in the previous one:
	
	After a small factorization we get:
	
	In the particular conditions (but relatively frequent in physics), which we will name in this book the "\NewTerm{Beltrami condition}\index{Beltrami condition}", where:
	
	We get the "\NewTerm{Beltrami identity}\index{Beltrami identity}":
	
	which bring us to have:
	
	that we will be very useful for example in the section of Classical Mechanics for our study of the brachistochrone page \pageref{brachistrochrone}.
	
	\paragraph{Theorem of Variational Calculus}\mbox{}\\\\
	The theorem of variational caluclus consist to prove that by considering a continuous function $f$ on the interval $[a,b]$ on $\mathbb{R}$ and $H$ the set of continuous functions on $[a,b]$ infinitely differentiable on $]a,b[$ and which vanish on $a$ and $b$, then for any function $h\in H$:
	
	$f$ is zero on $[a,b]$.

	Why focus on this theorem? Because we will meet it frequently during the application of the variational principle in similar configuration. Indeed, let us recall that the variational principle leads to have:
	
	and the integrated expression is rarely a simple function as the reader will notice in reading the different following chapters of this book. It is therefore important to know a property that sometimes simplifies the analysis of the problem. 
	
	\begin{tcolorbox}[title=Remark,colframe=black,arc=10pt]
	Some people may think that the case $f(x)=1$ with with $h(x)=\sin(x)$ and $x\in[0,2\pi]$ contradicts the statement of the theorem! In fact it's not really that ... this theorem must be valid for $\forall h\in H$ and not only just for the mentionned example. Hence the fact that $f$ will have to be equal zero as we will prove it just now.
	\end{tcolorbox}
	\begin{dem}
	To simplify we will take the case $a=-1,b=1$. With a little bit more technical details the proof by contradiction below can be adapted to the case of any $a$, $b$.
	
	Let us suppose that $f$ is not on $[-1,1]$. Then there exists $x_0\in ]-1,1[$ such that $f(x_0)\neq 0$. We can the suppose $f(x_0)>0$ (same reasoning if $f(x_0)<0$).
	
	By the initial assumption of continuity and of not nullity of $f$ there exists a small interval around $x_0$ on which $f$ is strictly positive. That is to say, there exists $\varepsilon >0$ such that $[x_0-\varepsilon,x_0+\varepsilon]\subset [-1,1]$ and $\forall y\in[x_0-\varepsilon,x_0+\varepsilon],f(y)>0$.
	
	Let us consider now the function $\pi:[-1,1]\mapsto \mathbb{R}$:
	
	We check easily that $\phi$ is continuous (positive) on $[-1,1]$ and infinitely differentiable on $[-1,1[$ (\SeeChapter{see section Differential and Integral Calculus page \pageref{smoothness}}).
	
	In addition, $\phi(-1)=\phi(1)=0$. And therefore, $\phi\in H$. Here is a graphical representation of $\phi$:
	\begin{figure}[H]
		\centering
		\includegraphics{img/mechanics/function_phi_theorem_variational.jpg}
		\caption[]{Sample Function to prove the variational calculus}
	\end{figure}
	Form $\phi$ we want get a continuous function on $[-1,1]$, infinitely differentiable on $]-1,1[$, positive on $[x_0-\varepsilon,x_0+\varepsilon]$ and zero outside $[x_0-\varepsilon,x_0+\varepsilon]$ to show the absurdity of the assumption of non nullity of $f$ so that the theorem is verified (remember that we are doing a proof by contradiction!).

	For this, it is engough to center $\phi$ on $x_0$ and contract it.

	The function $h:[-1,1]\mapsto \mathbb{R}$ define by:
	
	satisfy the required criteria. Moreover, $h(-1)=h(1)=0$ and therefore $h\in H$.

	Therefore, the function $f\cdot h$ will be continuous on $[-1,1]$ and positive on $[x_0-\varepsilon,x_0+\varepsilon]$ and zero everywhere else.

	We have:
		
	But, if a function $g:[a,b]\mapsto \mathbb{R}$ is continuous and positive and:
		
	this leads inevitably (we would suppose that as too intuitive to be proved) $g=0$ on $[a,b]$.
	
	Therefore $f(x)h(x)=0$ on $[x_0-\varepsilon,x_0+\varepsilon]$ or $f(x_0)h(x_0)\neq 0$ according to our initial absurd assumption, which is contradictory.

	The starting assumption is therefore false and $f$ must be zero on $[-1,1]$.
	\begin{flushright}
		$\square$  Q.E.D.
	\end{flushright}
	\end{dem}
	
	\subsection{Canonical Formalism}
	The canonical formalism does not introduce a new type of physics but offers a new range of tools to study physical phenomena. Its masterpiece, the "\NewTerm{Hamiltonian}\index{Hamiltonian}", plays an important role in quantum physics.

	As in the Lagrange formalism we work with quantities (scalars) such as energy, $T$ and $V$ rather than with vector quantities such as the Newton force.

	In the formalism of Lagrange, the description of an $n$ degrees of freedom mechanical system is as we know described by the general (unconstrained) independent coordinates $q_i$ that leads to $n$ Euler-Lagrange equations of the type:
	
	which are  second order differential equations (\SeeChapter{see section Differential and Integral Calculus page \pageref{second order differential equations}}).
	
	With Lagrange we compare mainly trajectories and therefore the $q_i$ and the $\dot{q}_i$ are independent. By Hamilton we must first learn to define the "\NewTerm{general momentum}\index{general momentum}\label{general momentum}" denoted $\vec{p}_i$, replacing the generalized coordinates $\dot{q}_i$ and $q_i$ which are also independent.
	
	\begin{tcolorbox}[title=Remark,colframe=black,arc=10pt]
	The origin of the conjugated momentum will be trivial when we will see further below a first concrete example.
	\end{tcolorbox}	
	
	\subsubsection{Legendre Transform}
	The "\NewTerm{Legendre transformation}\index{Legendre transformation}" is often used in thermodynamics (see section of the same name page \pageref{legendre transformation thermodynamics}) because it allows to interconnect the various thermodynamic potentials. In mechanics or in geometry it gives the possibility to define the Hamiltonian from the Lagrangian and vice versa. We will give here a simplified and sufficient description of it in the univariate case (that is to say: in one dimension).
	
	Given a function $f (u, v)$ where $u, v$ are the both independent variables of which $f$ depends.
	
	Let us define:
	
	The Legendre transformation gives the possibility to define a function $g(u,w)$ that can replace $f(u,v)$ in the sense that it rid of one of the explanatory variables (which is very useful in thermodynamics!):
	
	Given now the total differential of $f$ (\SeeChapter{see section of Differential and Integral Calculus page \pageref{total exact differential}}):
	
	From the definition of $g$ we get:
	
	and then we have:
	
	Obviously we end up just with a sign difference to the same result if we put since the beginning:
	
	As this is usage in Thermodynamics.
	
	\subsubsection{Hamiltonian}\label{hamiltonian mechanics}
	Given a $L(q_i,\dot{q}_i)$ that we will assimilate the function $f$ introduce just above with the $q_i$ serving playing the role of the $u$ and the $\dot{q}_i$ the role of th $v$. For $w$, we define the generalized moments also named "\NewTerm{canonical moments}\index{canonical moments}\label{canonical moments}".
	
	Then for each generalized velocity, there is one corresponding conjugate momentum, defined as:
	
	with $i,j\in\mathbb{N}^{*}$.
	
	In Cartesian coordinates, the generalized momenta are precisely the physical linear momenta. In circular polar coordinates, the generalized momentum corresponding to the angular velocity is the physical angular momentum. For an arbitrary choice of generalized coordinates, it may not be possible to obtain an intuitive interpretation of the conjugate momenta.
	
	Before continuing let's see what allows us to do this definition:
	
	So we define, in analogy with $g$, a function of the $q-i$ and the $p_i$ which we will denote by $H(q_i,p_i)$:
	
	\begin{center}
	\begin{tabular}{ccccc}
	$g(u,w)$ & $=$ & $vw$ & $-$ & $f$\\ 

	$\Downarrow$ & & $\Downarrow$ & & $\Downarrow$ \\ 
	$H(q_i,p_i)$ & $=$ & $\displaystyle\sum_{i=1}^n\dot{q}_ip_i$ & $-$ & $L(q_i,\dot{q}_i)$ \\ 
	\end{tabular} 
	\end{center}
	Caution! The relation obtained just above (that shows clearly that the Hamiltonian is the Legendre transform of the Lagrangian!):
	
	named "\NewTerm{Hamilton function}\index{Hamilton function}" or "\NewTerm{Hamiltonian}\index{Hamiltonian}" is more than important (like everything else anyway). We will see this relation again, among others, in the section of Relativistic Quantum Physics or Quantum Field Physics. Moreover, a beautiful example of what we have seen now is given in the section of Special Relativity where we calculate the Lagrangian and Hamiltonian of a free particle. The results are quite relevant and useful and their accuracy in electrodynamics are more than amazing!
	\begin{tcolorbox}[colframe=black,colback=white,sharp corners]
	\textbf{{\Large \ding{45}}Example:}\\\\
	Another important and well known application analytical mechanics is the calculation of minimal surfaces (for physics and architecture). If we focus on the determination of such a surface by requiring it to be a surface of revolution, we will see that we find a catenoid (the form taken by a soap film between two rings).\\
	
	We give us ourselves the radius $R_1$ and $R_2$ of two circles of distance between them of $l$. We seek a function $y$ of class $\mathcal{C}^1$ such that:
	
	and that the parametric surface of revolution of the type:
	
	has a minimal surface.\\
	
	We know that the surface of a volume of revolution can be written (\SeeChapter{see section Geometric Shapes page \pageref{solid of revolution}}):
	
	Thus by making the function vary:
	
	As $\delta (y')=(\delta y)'$ the integration by parts of the second term gives:
	
	As the limits of integration are fixed, the first term is zero. It then remains:
	
	and therefore:
	
	The minimum we sought corresponds to  $\delta S=0$ whatever $\delta y$, which imposes the condition:
	
	we fall back on the Euler-Lagrange equation.
	\end{tcolorbox}
	
	\begin{tcolorbox}[colframe=black,colback=white,sharp corners]
	This equation can also be written in another form. By introducing the canonical momentum to simplify:
	
	Then we get immediately:
	
	We then get:
	
	Thus, putting the analogy seen earlier above (Hamilton's method):
	
	We arrive at:
	
	Therefore, remembering that at the beginning we had:
	
	we arrive at:
	
	What we can also note (because the constant has an undetermined sign):
	
	Then we have:
	
	We have already integrated this type of differential equation in detail in the section of Civil Engineering during the study of the chain shape. The result is:
	
	the surface of revolution of this curve being a catenoid:
	\end{tcolorbox}
	
	\begin{tcolorbox}[colframe=black,colback=white,sharp corners]
	\begin{figure}[H]
		\centering
		\includegraphics[width=9cm,height=5cm]{img/mechanics/catenoid_soap.jpg}
		\includegraphics[width=9cm,height=5cm]{img/mechanics/catenoid_soap_maple.jpg}
		\caption[]{Catenoid made of soap between two circular rings and with Maple 4.00b}
	\end{figure}
	This figure can be obtained with Maple 4.00b as follows:\\
	
	\texttt{>y:=cosh(x);}\\
	\texttt{>y:=cosh(x);plot3d([x,y*cos(phi),y*sin(phi)],x=-2..2,phi=-2*Pi..2*Pi,}\\
	\texttt{numpoints = 50000,axes=normal,contours=10,style=patchcontour);}\\
	\end{tcolorbox}
	Now if $L$ depends on the time (which is still often the case ...) we have as total differential:
	
	we also calculate the total differential of $H(q_i,p_i)$ and substitute in it results previously obtained:
	
	which shows well that $H(q_i,p_i)$ is function of the $q_i$ and of the time $t$.

	We can also write for its total differential:
	
	and as the $q_i$ and $p_i$ are independent we identify, by comparing the two expressions, that:
	
	These relations are extremely important because we will see and use them again in the section of Magnetostatics in Relativistic Quantum Physics and Quantum Field Theory in a somewhat more barbaric form (but beautiful too...).
	
	The prior previous relation can be written obviously:
	
	
	Let us now consider the second term of the first member of the Euler-Lagrange equation. We have:
	
	and therefore we get the $2n$ equations below:
	\begin{gather*}
		\begin{rcases*}
		\dot{q}_i=\dfrac{\partial H}{\partial p_i} & i=1,2,\ldots,n \\
		\dot{p}_i=-\dfrac{\partial H}{\partial q_i} & i=1,2,\ldots,n
		\end{rcases*} 2n
	\end{gather*}
	These $2n$ equations are named "\NewTerm{canonical equations of motion}\index{canonical equations of motion}" and are differential equations of the first order.
	
	So now we can write the prior previous relation after rearrangement as:
	
	\begin{tcolorbox}[title=Remark,colframe=black,arc=10pt]
	The appearance of the minus sign "$-$" between the equations for the $q_i$ and those for their conjugate moments, is named a "\NewTerm{symplectic symmetry}\index{symplectic symmetry}".
	\end{tcolorbox}
	From:
	
	we can, on a trajectory that obeys the canonical equations, compute:
	
	\begin{tcolorbox}[title=Remark,colframe=black,arc=10pt]
	The appearance of the minus sign "$-$" between the equations for the $q_i$ and those for their conjugate moments, is named a "\NewTerm{symplectic symmetry}\index{symplectic symmetry}".
	\end{tcolorbox}
	If $H$ does not depend on time then we have $\partial H/\partial t=0$, then $H$ (and $L$) are a "\NewTerm{constant of movement}\index{constant of movement}".

	What must be understood so far is that in Newtonian mechanics, the time evolution is obtained by computing the total force being exerted on each particle of the system, and from Newton's second law, the time-evolutions of both position and velocity are computed. In contrast, in Hamiltonian mechanics, the time evolution is obtained by computing the Hamiltonian of the system in the generalized coordinates and inserting it in the Hamiltonian equations. This approach is equivalent to the one used in Lagrangian mechanics. In fact, the Hamiltonian is as we just seen the Legendre transform of the Lagrangian when holding $q$ and $t$ fixed and denoting $p$ as the dual variable, and thus both approaches give the same equations for the same generalized momentum. The main motivation to use Hamiltonian mechanics instead of Lagrangian mechanics comes from the symplectic structure of Hamiltonian systems.
	
	An example seems indispensable to us at this level of advancement of the study of the Lagrangian formalism. We will restrict ourselves to a particular case of a particle subjected to a force in one dimension. But even if this example and the related developments are simple we will find the results here in many other following sections of this book. It is therefore important for the reader to study and understand well what will follow (which unfortunately also requires that the contents of the section of Classical Mechanics is known by the reader).
	
	\begin{tcolorbox}[colframe=black,colback=white,sharp corners]
	\textbf{{\Large \ding{45}}Example:}\\\\
	Given a particle of mass $m$ traveling in one dimension (say $x$) and subjected to a force derived from a potential such that:
	
	We know that its Lagrangian is:
	
	We will have only one momentum (the linear momentum), denoted $p$ as always, conjugated to $x$ and defined as we also know by:
	
	equation that we can (that we must!) reverse (from the definition of the linear momentum):
	
	We may notice at this point that the momentum $p$ corresponds (hazard...!!) to the component $x$ of the elementary definition $p=mv$ (which is not always the case trivially).
	
	As defined by the Hamiltonian then it comes:
	
	That we often write in the form:
	
	where $T$ is therefore the kinetic energy expressed in function of the momentum.
	\end{tcolorbox}
	
	\subsubsection{Poisson bracket}\label{poisson bracket}
	The Poisson bracket is an important binary operation in Hamiltonian mechanics, playing a central role in Hamilton's equations of motion, which govern the time evolution of a Hamiltonian dynamical system. This formalism and definition will be quite useful to us in the section of Wave Quantum Mechanics.
	
	The Poisson bracket $\{A,B\}_{q,p}$ is the standard way of writing a certain operation that involves the quantities $A(q_i,p_i)$ and $B(q_i,p_i)$ and also all canonical variables $(q_i,p_i)$ defined by:
	
	that expresses the way to travel in a field (the bracket content is zero if the two paths are equal).
	
	We recognize here (!!!) the expression in brackets we introduced earlier above:
	
	Where as:
	
	
	From the definition of $\{A,B\}_{q,p}$ we can deduce some relatively trivial properties that will be useful to us in the sections of Wave Quantum Physics and Relativistic Quantum Physics:
	\begin{enumerate}
		\item[P1.] $\{A,B\}=-\{B,A\}$
		\begin{dem}
		
		\begin{flushright}
			$\square$  Q.E.D.
		\end{flushright}
		\end{dem}
		
		\item[P2.] $\{A,B+C\}=\{A,B\}+\{A,C\}$
		\begin{dem}
		
		\begin{flushright}
			$\square$  Q.E.D.
		\end{flushright}
		\end{dem}
		
		\item[P3.] $\{A,BC\}=B\{A,C\}+C\{A,B\}$
		\begin{dem}
		
		\begin{flushright}
			$\square$  Q.E.D.
		\end{flushright}
		\end{dem}	
		
		\item[P4.] $\{A,\{B,C\}\}+\{C,\{A,B\}\}+\{B,\{C,A\}\}=0$
		\begin{dem}
		 
		Well and here, just to not have an unreadable stuff, long and boring stuff we will prove the property for $n=1$ and assume (of course) that it is valid for all $n$:
		
		We have also (\SeeChapter{see section Differential and Integral Calculus page \pageref{Schwarz theorem}}) under certain conditions the property $\partial_{p,q}=\partial_{q,p}$. Therefore all the terms cancel (this is elementary algebra normally...) for finally have:
		
		expression named "\NewTerm{Jacobi identity}\index{Jacobi identity}".
		
		That is to say written in another way (perhaps more obvious) for  binary operation $\times$ on a set $S$ possessing a binary operation $+$ with an additive identity denoted by $0$ satisfies the Jacobi identity if:
		
		That is, the sum of all even permutations of $(a,(b,c))$ must be zero.
		
		\begin{flushright}
			$\square$  Q.E.D.
		\end{flushright}
		\end{dem}
	\end{enumerate}
	Now let us use the Poisson brackets with physical quantities! As by definition the coordinates and the linear momentum are not directly dependent, we have when we put $A=q_j$ and $B=q_k$:
	
	Therefore:
	
	hence:
	
	and identically:
	
	But:
	
	where for recall $\delta_{kj}$ is the Kronecker symbol defined par:
	
	We have also:
	
	It comes therefore:
	
	This implies a fairly general result that we see again in the section of Wave Quantum Physics:
	
	when $k\neq j$.
	
	\subsubsection{Canonical transformations}
	We say that of the $q_i,p_i$ that they  are "\NewTerm{generalized canonical variables}\index{generalized canonical variables}". It is not an understatement as there is virtually no limit to what they can represent physically.

	That being the case, there must be transformations between these different choices. We will denote by $Q_i,P_i$ the new canonical variables obtained after such a transformation.
	We are not surprised to see that these changes are subject to fairly strict conditions. Indeed, the $q_i,p_i$ are generalized and obey as we have just prove above:
	
	and the canonical equations:
	
	are form invariant.
	
	Thus, after a transformation of the $q_i,p_i$ to the $Q_i,P_i$ and defining an new hamiltonian that we will denote by $K(Q_i,P_i)$ we should have:
	
	and the canonical equations:
	
	Strictly speaking, the equations of transformation can be written:
	
	with $i,j=1,2,\ldots,n$ and  must be able to reverse as the physics remains independent of the variables we use to describe it, so we can write the inverse transformations:
	
	with $i,j=1,2,\ldots,n$.
	
	The $q_i,p_i,Q_i,P_i$ form $4n$ variables but it is obvious that only $2n$ of them are independent.
	 
	\begin{flushright}
	\begin{tabular}{l c}
	\circled{95} & \pbox{20cm}{\score{3}{5} \\ {\tiny 39 votes,  68.21\%}} 
	\end{tabular} 
	\end{flushright}
	
	%to make section start on odd page
	\newpage
	\thispagestyle{empty}
	\mbox{}
	\section{Classical Mechanics}
	
	\lettrine[lines=4]{\color{BrickRed}B}efore introducing the study of moving solid bodies as part of classical mechanics (at the opposite of relativistic mechanics) also named "rational mechanics" or "Newtonian mechanics", it may seem in the natural order of things to define and study first the properties relative to their static state.

	\begin{figure}[H]
		\centering
		\includegraphics[scale=0.5]{img/mechanics/tintin.jpg}
		\caption[]{All rights reserved ©Moulinsart}
	\end{figure}

\textbf{Definitions (\#\mydef):}

	\begin{enumerate}
		\item[D1.] A phenomenon is said to "\NewTerm{static}\index{static}" or in "\NewTerm{equilibrium}\index{equilibrium (static)}" when it suffers of no dynamics (no acceleration or verbatim: "no unbalanced forces"), at least in apparence. We can consider an equilibrium as a static state, although even if it is apparent because it can be the result of two opposing dynamics that cancel themselves! Thus, the variables which describe a static phenomenon are constants, the concrete values of these quantities can be calculated. Static is a major case study in Mechanical engineering and Civil Engineering (see sections of the same name page \pageref{mechanical engineering} and page \pageref{civil engineering}).
		
		In a more technical way this definition is raised to the rank of principle named the "\NewTerm{fundamental principle of static}\index{fundamental principle of static}\label{fundamental principle of static}" which states that for a system to be in equilibrium, it is necessary that the general resultant and the resultant moment of the external forces is equivalent to zero relative to its center of mass or gravity (the condition is sufficient for mechanical problems that deal with non-deformable solids).
		\item[D2.] The "Static" is the study of the conditions of equilibrium of a material point or body subjected to forces that are in equilibrium (canceling themselves).
		\item[D3.] Any cause that can accelerate (concept defined in detail further below) or distort a body is named a "\NewTerm{force}\index{force}" (a concept introduced by Newton and rigorously to which we return in detail later in the statement of three Newton's laws).
		
		An in-deep observation showed that a force is a macroscopic result of complex microscopic phenomena, namely interactions distance between particles. These interactions are in the number of 4 and we do not wish to talk about these $4$ interactions now because they make use of mathematical tools that are out of context in this section of the book.
		
	\begin{tcolorbox}[title=Remark,colframe=black,arc=10pt]
In classical mechanics we naturally do not ask ourself on the question of a transformation of time. The changes we focus on a related only to the measures "position" and its derivatives (velocity and acceleration). Indeed, in classical mechanics, we postulate the "\NewTerm{Newton's time}\index{Newton's time}": time flows in the same way from one referential to another.
	\end{tcolorbox}	
	
	\item[D4.] If the lines of action of all the forces acting on a body are in the same plane, the system of forces named "\NewTerm{coplanar system}\index{coplanar system}":
\begin{figure}[H]
\centering
\includegraphics[scale=0.75]{img/mechanics/coplanar_system.jpg}
\caption{Particular example of coplanar system}
\end{figure}
where the intersection of forces is named a "\NewTerm{forces node}\index{forces node}".

If we consider for example a particular set of coplanar forces whose intensities (norms) were measured using a dynamometer and angles relatively to an appropriately selected reference with a protractor. We will have a diagram of the following type named  "\NewTerm{dyname}\index{dyname}" (abbreviation of "dynamic diagram"):
\begin{figure}[H]
\centering
\includegraphics[scale=0.75]{img/mechanics/dyname.jpg}
\caption{Particular example of measured coplanar system (dyname)}
\end{figure}
To calculate the components of the resulting (and thus its norm), it may be easier sometimes to represent the three forces in the following form (after translation to a common point):
\begin{figure}[H]
\centering
\includegraphics[scale=0.75]{img/mechanics/dyname_translated.jpg}
\caption{Simplified representation of forces for treatment (translated dyname)}
\end{figure}
	Then, using elementary trigonometry, knowing the angles and the intensity of each of the three forces, it is possible to determine their respective components along the $X$ and $Y$  and their algebraic sum along each of the axes give the components of the resultant (today this type of reasoning sounds simple but it took still until the late 16th century that these reasonings on static forces emerges).

\begin{figure}[H]
\centering
\includegraphics[scale=0.75]{img/mechanics/coplanar_suspended_bridge.jpg}
\caption{Representation of the resultant of forces of a suspended bridge}
\end{figure}
	A small basic application case is the porting of a load:
	\begin{figure}[H]
		\centering
		\includegraphics{img/mechanics/load_port.jpg}
		\caption{Coplanar symmetrical load porting}
	\end{figure}
	By symmetry each of the left/right part of the cable is under the same tension (force). Applying elementary trigonometry (\SeeChapter{see section Trigonometry page \pageref{trigonometry}}), we have:
	
	Therefore:
	
	We notice that we have an interest in having:
	
	(therefore an angle greater than $\sim30^{\circ}$) otherwise we would have:
	
	As we have already mentioned it, a system of concurrent coplanar forces is in static equilibrium when the resultant of all forces (and moments) is zero. Graphically, the end of the last force on the dyname representation coincides with the origin of the first force. Mathematically this can be written:
	
	\item[D5.] A material system $S$ (set of material points $(P_i,m_i)$) is named "\NewTerm{rigid solid}\index{rigid solid}" or "\NewTerm{rigid body}\index{rigid body}", or simply "\NewTerm{solid}\index{solid}", if the mutual distances of the constituent material points do not vary over time. What we note technically as follows:
	
	\end{enumerate}
	
	\subsection{Newton's Laws}\label{newton laws}
	Newton's three laws are the foundation for classical mechanics. They describe the relation between a body and the forces acting upon it, and its motion in response to those forces. These laws are retrospectively unprovable and can not be formalized (proved) because they result from observations and thus derive from our daily experience.
	
	However, developments in modern physics and that are based on the consequences of these three laws are in such agreement with theoretical conditions that impose the principle of least action and relative experiences, that their validity could make any doubt (...).

	
	\subsubsection{Newton's First Law (Inertia Law)}
	
	\textbf{Definition (\#\mydef):} Any point or wide body perseveres in its shape (geometry) or its state of rest or of uniform motion (described by the center of mass), unless if "\NewTerm{applied forces}\index{applied forces}" force bring the body to change its shape or its movement.
	
	The punctual body is obviously dimensionless. It is a creation of the mind, a model representing a physical object that is animated as a translational motion (no rotation on itself because it is dimensionless!). We assume here that our physical space is three dimensional to which we add time that is not a spatial dimension in the context  of classical mechanics but an immutable and independent parameter!
	
	In other words: Every body at rest or in uniform motion is under pression of a given number of zero forces or of a sum of forces whose resultant is zero (that is the fundamental principle of statics also named "\NewTerm{principle of inertia}\index{principle of inertia}")!
	
	\begin{tcolorbox}[title=Remark,colframe=black,arc=10pt]
	We have proved this corollary in our study of Noether's theorem in the section of Analytical Mechanics.
	\end{tcolorbox}
	
	In this corollary (and also before), comes to us in full light the word "\NewTerm{force $\vec{F}$}". Let us questioning this word for which everyday language is full of different meanings: wrist force, soul's force, think force... Also, the force can be visible or not, measurable or not depending on the case... Anyway, it has the "power" to change the path (the movement) and shape (geometry) of things. Without ignoring this halo that surrounds the word that has puzzled more than a physicist before him, Newton gives the force a very precise meaning, that stands out from the intuitive idea of a physical effort.
	
	Let us give some intuitive properties of a force:
	\begin{enumerate}
		\item[P1.] The force is a vector quantity (\SeeChapter{see section Vector Calculus page \pageref{vector}}) with corresponding mathematical properties.
		
		\item[P2.] The effect of a force, does not change if we slide the force on his line of action.
		
		\item[P3.] The force is proportional to the mass.
	\end{enumerate}
	A force is therefore a physical quantity manifested by its effects:
	\begin{enumerate}
		\item[E1.] "\NewTerm{Dynamic effect}\index{dynamic effect}": a force is capable of producing a cause or alter the movement of a body.
		
		\item[E2.] "\NewTerm{Static effect}\index{static effect} a force is a cause capable of producing a deformation of the shape (geometry) of a body.
	\end{enumerate}
	Any force can be represented by a vector whose four properties are at least:
	\begin{enumerate}
		\item[P1.] "\NewTerm{Direction}\index{direction (force)}": the line in that the action takes place (this include the angle).
		
		\item[P2.] "\NewTerm{Orientation}\index{orientation (force)}": The way the line in which the action takes place is followed.
		
		\item[P3.] "\NewTerm{Application point}\index{application point (force)}": the point where the action is applied on the body.
		
		\item[P4.] "\NewTerm{Intensity}\index{intensity (force)}": the amplitude (norm) of the force
	\end{enumerate}
	It is possible to classify most of the forces by families such as:
	\begin{enumerate}
		\item[F1.] The "\NewTerm{forces of reaction}\index{forces of reaction}": each body applies a force on another body that is in contact with himself. For example, if an object is putted on a table, this table applied (most of time...) an equal and opposite force on the object (so that it does not sink into the table - that are quantum mechanisms that are responsible for this reaction force). This force is always vertical to the point of contact.
		
		\item[F2.] The "\NewTerm{frictional forces}\index{frictional forces}" the frictional force exists when two bodies are in contact. It always opposes to the movement. The friction force which opposes to the movement has not only has a negative effect, it is also essential to ensure contact between two surfaces (e.g. contact of tires with the road, brake, etc.).
		
		\item[F3.] The "\NewTerm{tension forces}\index{tension forces}": it is a force that pulls on an element of a body such as the tension exerted by a thread or  a spring (\SeeChapter{see section Mechanical Engineering page \pageref{spring tension}}).
		
		\item[F4.] The "\NewTerm{action-at-a-distance forces}\index{action-at-a-distance forces}": these are the forces that act through vector fields such as electric field, magnetic field or gravitational field (types of forces that result even when the two interacting objects are not in physical contact with each other, yet are able to exert a push or pull despite their physical separation). This last type of common force has the particularity, if the field is isotropic, to be reduced to the study of the body's center of gravity (we will prove it further below during our study of static forces).
	\end{enumerate}
	
	\begin{tcolorbox}[title=Remark,colframe=black,arc=10pt]
	As far as we know the are no official conventions for a finite number of properties for forces and for categories of forces. In some books the authors will tell you that there are only two type of forces: Contact Forces (frictional forces, tension forces, normal forces, air resistance forces, applied forces, spring forces, etc.) and Action-at-a-Distance Forces (gravitational force, electric force, magnetic force).
	\end{tcolorbox}
	
	\subsubsection{Newton's Second Law (Fundamental Principle of Dynamics)}\label{newton second law}
	\textbf{Definition (\#\mydef):} The change in motion is proportional to the "hard driving force", and is carried along the line through which this force is printed. Or more precisely: The acceleration of an object as produced by a net force is directly proportional to the magnitude of the net force, in the same direction as the net force, and inversely proportional to the mass of the object.
	
	A force, we know it, is in the language of Newton what causes the "change of the movement" and not something else...! But, supplement to the program, the words "change of motion" of the law hide a mathematical meaning, different from the intuition "variation in speed". For Newton, we saw that a body at rest was characterized by its quantity of matter, its mass. Inspired by some predecessors, Newton put that a body in motion "carries a certain amount", simply named at his time: the "\NewTerm{linear momentum}\index{linear momentum}". This is actually this amount which, under the simple word "movement", is contained in the statement of the second law. The amount of movement is the measure that we derive from both speed (concept which we will define later in our study of the kinematics) and quantity of matter, that is, by definition, the product of its mass by its speed:
	
	Using modern mathematical symbols, the first part of this second law can then be reformulated:
	
	The force is equal to the variation with time of the amount of movement or in a non-relativistic framework:
	
	This latter relation is valid until the speed is much lower than that of light as we will see in our study of Special Relativistic much later, as Newton supposed that the mass did not vary (or do not seem to vary .. .) as a function of speed. Thus, the "\NewTerm{fundamental relation of dynamics (FRD)}\index{fundamental relation of dynamics}" is also given by:
	
	and can be stated as follows: Given a constant mass $m$ of abody, the acceleration experienced by a body in an inertial frame is proportional to the net force it undergoes, and inversely proportional to its mass $m$ in a non relativistic framework.
	
	Remember that the "\NewTerm{mass}\index{mass}" is a measure for the quantity of material contained in a body (\SeeChapter{see section Principia page \pageref{mass}}). The mass is a quantity independent of where it is located (S.I. unit kilogram: [kg]). The "\NewTerm{weight}\index{weight}", corresponds to the force (S.I. unit Newton: [N]) that an object applied on another one through an action (or field). The latter depends on where we are located (see below the equation of the gravitational force of Newton).
	
	We will see (prove) that in the context of a body falling in a gravitational field with a spherical symmetry, we have:
	
	as part of our good old Earth, we have the habit to write:
	
	In an Eulerian system in Cartesian coordinates, a given quantity $\varphi$ of a continuous medium will have a distribution based on four independent variables $x, y, z, t$. For small changes $\mathrm{d}x$, $\mathrm{d}y$, $\mathrm{d}z$ and $\mathrm{d}t$, the total variation $\varphi$ being expressed by (\SeeChapter{see section Differential and Integral Calculus page page \pageref{total exact differential}}):
	
	By following a particle in its motion, we observe during a time $\mathrm{d}t$ displacements $\mathrm{d}x$, $\mathrm{d}y$, $\mathrm{d}z$. We can therefore express from the previous relation of the total variation of $\varphi$ for the time $\mathrm{d}t$. We obtain therefore the expression of a very important derivative in theoretical physics named the "\NewTerm{particular derivative}\index{particular derivative}":
	
	In mechanics we will particularly work with the Newtonian gravitational field. Therefore, the relation linking the force to the acceleration takes a more general form. Let's see how:
	Given the particuliar derivative of the velocity (for the three spatial coordinates):
	
	Which can also be written:
	
	Which can also be written in condensed form:
	
	The Newton's second law is written then in generalized form:
	
	where as we will see it during our study of fluid dynamics (\SeeChapter{see section Continous Mechanics page \pageref{advection term}}):
	
	is the "\NewTerm{advection term}\index{advection term}".
	
	This formulation of Newton's second law is of utmost importance in physics. It makes explicit account of the force experienced by a material point in a vector field depending on the speed and not only of the position. We will use this formulation in the section of Continuum Mechanics in our study of fluids and plasmas, in the section of Electromagnetism and in that of General Relativity!
	
	We can see that if $\vec{\nabla}(\vec{v})=\vec{0}$, that is to say if the gradient of the velocity field is zero (i.e. the velocity field is constant or equal to zero - the latter case being typical for any experience outside a force field like a gravitational one for example!), then we fall back on the special case of the simple second Newton law as:
	
	
	As we will prove during our study of Fluid Mechanics, the generalized Newton's second law can be written:
	
	In the case of a normal uniform gravitation field the last term vanish ("vorticity" as we will see it again during our study of Fluid Mechanics), then it remains:
	
	That is to say in the one dimensional case this reduce to:
	
	And as we will see later the last term between parenthesis in the Kinetic Energy. Therefore:
	
	 But as as variation of the Kinetic energy is in absolute value equal to that of the variation of potential energy, denoted $E_p$ and sometimes $U$, on the same distance as we will prove later  then we have:
	 
	That means once again that if we are far away from any force field (gravitational one for example in Classical Mechanics), the latter relation reduce to the simple Newton's second law again as $U=0$:
	
	so if we see a body that accelerates it can only be due to an impulse (a direct applied force on it). 

	If there is nobody or nothing entering in collisation on with our body or applying an impule on it but we see that our body accelerates, then it mean that we are in the presence of a force field and the generalized Newton's second law reduced to:
	
	That is implicitly... a force due to a gravitational field (we still omit the case of electric field at this level).

	But of sure we can have the both terms that are non-zero! That is when a body accelerates because of an impulse plus because of a force field like the gravitation field (typically a rocket entering into orbit in a far away planet).
	
	
	\subsubsection{Newton's Third Law (Law of Action and Reaction)}\label{newton third law}
	\textbf{Definition (\#\mydef):}  In every two-body static interaction, there is a pair of forces acting on the two interacting objects. The size of the forces on the first object equals the size of the force on the second object. The direction of the force on the first object is opposite to the direction of the force on the second object. Forces always come in pairs - equal and opposite action-reaction force pairs!
	
	This third law is must known under the name of "\NewTerm{action / reaction principle}\index{action / reaction principle}" and follows from the first Newton law by mathematical through the reasoning that we have already seen during our study of Noether's theorem in the section Principles.
	
	This law can be simply expressed in the following form:
	
	and in a little bit more general and explicit way as following:
	
	Thus, in nature, according to this principle, there is no isolated force, each force has its "opposite," they work in pairs!
	
	We can also say that two solid points or extended bodies in static contact exert on each other always opposing forces in direction but equal in intensity and direction.
	
	It follow and important property of the third Newton law: For a material point, subject to forces $\vec{F}_1,\vec{F}_2,\ldots, \vec{F}_n$ to be in static equilibrium the resultant of the forces must be equal to zero. That is the material point:
	
	satisfies the "\NewTerm{equilibrium condition}\index{equilibrium condition}".
	
	The previous relation, which thus defines any body in equilibrium, opens the study of huge field of practical applications names "\NewTerm{static forces}" and that we will develop further below after having introduced the concept of force momentum.
	
	\subsection{Center of Mass and Reduced Weight}\label{center of mass}
	The center of mass is a special case of the barycenter with all its properties that we have already largely developed in the section of Euclidean Geometry (therefore we strongly recommend to reader to refer to it) but applied to physics.
	
	We can make no distinction between "\NewTerm{center of mass}\index{center of mass}" and "\NewTerm{center of gravity}\index{center of gravity}" (also named "\NewTerm{centroid}\index{centroid}") if and only if the mass of the body is considered homogeneous.
	
	For those who want a simple but impressive example of the importance of understanding the concept of center of gravity in the context of engineering and business, I recommend searching the Internet for video on the keyword "Steadicam". This is a technique of moving the center of gravity of cameras for image stabilization that has revolutionized the way of filming in the movie industry (and not only!). Or playing with center of mass can also be an art:
	\begin{figure}[H]
		\centering
		\includegraphics[scale=0.4]{img/mechanics/stone_balancing.jpg}
	\end{figure}
	
	\textbf{Definition (\#\mydef):} Let us consider a solid formed of $n$ mass points $m_1,m_2,\ldots,m_n$ and identified by their respective position vectors $\vec{r}_1,\vec{r}_2,\ldots, \vec{r}_n$. We name "\NewTerm{center of mass}\index{center of mass}" (or "\NewTerm{center of inertia}\index{center of inertia}" if there is strict equality between gravitational mass and inertial mass) a point $G$ where we can connect all the mass of the system (and therefore its analysis !!) and such that, the origin being arbitrarily chosen it is given by:
	
	Relation that has to be compared by the reader with the one of the barycenter that we get in the section of Euclidean Geometry:
	
	In a identical way, we define the "\NewTerm{reduced mass}\index{reduced mass}" of the system by:
	
	The reduced mass is the "effective" inertial mass appearing in the two-body problem of Newtonian mechanics. It is a quantity which allows the two-body problem to be solved as if it were a one-body problem.
	
	\begin{theorem}
	Given two bodies, one with mass $m_1$ and the other with mass $m_2$, the equivalent one-body problem, with the position of one body with respect to the other as the unknown, is that of a single body of mass:
	
	\end{theorem}
	\begin{dem}
	This relation can be derived as follows:
	Using Newton's second law, the force exerted by body $2$ on body $1$ is:
	
	The force exerted by body $2$ on body $1$ is:
	
	According to Newton's third law, the force that body $2$ exerts on body $1$ is equal and opposite to the force that body $1$ exerts on body $2$:
	
	Therefore:
	
	and:
	
	The relative acceleration $\vec{a}_\text{rel}$ between the two bodies is given by:
	
	So we conclude that body $1$ moves with respect to the position of body $2$ as a body of mass equal to the reduced mass.
	\begin{flushright}
		$\square$  Q.E.D.
	\end{flushright}
	\end{dem}
	 
	If we consider the solid as continuous (true only at the microscopic scale in a first approximation) therefore it comes:
	
	Where the integrals are extended to whole solid volume.
	
	Furthermore, if the solid is homogeneous (special case), of volumetric mass $\rho$, then $\mathrm{d}m=\rho\mathrm{d}V$, $\mathrm{d}V$ being the volume element. The equation can then be written (the notation of the triple integral is reduced to only one integral by for clarity purpose):
	
	That is to say in components:
	
	Properties:
	\begin{enumerate}
		\item[P1.] If the solid has an axis of symmetry, then $G$ is on that axis.

		\item[P2.] If the solid has a plane of symmetry, then $G$ is on this plane.

		\item[P3.] If the solid has several axes of symmetry, then $G$ is at their intersection.
	\end{enumerate}
	\begin{tcolorbox}[title=Remarks,colframe=black,arc=10pt]
	The center of mass $M$ may be out of the solid (eg: a stool, a boomerang, etc.).
	\end{tcolorbox}
	It is not always obvious to calculate the center of mass of a relatively simple given body. It's not that the mathematical tools to handle are complex (simple integral, Pythagoras theorem and some multiplication and integration by parts) but we must approach the problem in an elegant way and if we do not immediately the right approach we will fail very quick. We advise the teachers that address this topic and the related exercises to do it with their students (ie in class) but by leaving them to discuss how the teacher must attack the problem on the blackboard (it works very well ).
	

	\subsubsection{Center of Mass Theorem}
	Under the action of external forces $\vec{F}_1,\ldots,\vec{F}_i,\ldots,\vec{F}_n$, acting at each point of a solid, each of these points takes an acceleration equal to the applied force $\vec{F}_i$. Using Newton's second law for each point and summing the effects we have (in a non-relativistic case):
	
	Under the action of external forces $\vec{F}_1,\ldots,\vec{F}_i,\ldots,\vec{F}_n$, acting at each point of a solid, each of these points takes an acceleration equal to the applied force $\vec{F}_i$. Using Newton's second law for each point and summing the effects we have (in a non-relativistic case):
	
	by the position of the center of mass given by the relation:
	
	In the special case of a binary system (used a lot in Astronomy and Corpuscular Quantum physics as we will see it) we have:
	\begin{figure}[H]
		\centering
		\includegraphics{img/atomistic/hydrogenoid_center_of_mass.jpg}
		\caption{Binary System Center of Mass (profile view)}
	\end{figure}
	Where:
	
	\begin{figure}[H]
		\centering
		\includegraphics{img/mechanics/center_of_mass_binary_system.jpg}
		\caption{Binary System Center of Mass when in motion}
	\end{figure}
	In the special case where $M=m$ we have obviously:
	
	Therefore the center of mass (CM) is indeed at the middle point!
	
	If we set the position of the center of mass at the origin ($\vec{r}_G=\vec{0}$). Then we have in the case where $M\neq m$:
	
	and then we have taking the norm:
	
	In other words, the distances $r_M$ and $r_M$ can change, but the ratio $r_m/r_M$ is fixed. Also, the center of mass always lies on the line connecting the two stars.
	
	To come back to the general case, it comes if the repository is placed on the center of mass:
	
	where:
	
	therefore:
	
	This is the theorem of the center of mass that we can state as follows:
	
	The center of mass of a solid moves as a material point of mass equal to that of the solid and on which would be applied the sum of external forces. A simple example is that of an explosive projectile in absence of gravity describing a curved path. If the projectile explodes and breaks, the shards of the center of mass continues to describe the curved path he had started.
	
	Or another famous example:
	\begin{figure}[H]
		\centering
		\includegraphics{img/mechanics/center_of_mass.jpg}
	\end{figure}
	\begin{tcolorbox}[title=Remark,colframe=black,arc=10pt]
	In the particular case of a solid (set of points) subjected to the field of gravity, $\vec{F}_\text{ext}$ is the weight of the solid and then $G$ is named the "\NewTerm{center of gravity}\index{center of gravity}" (hence the origin of this name).
	\end{tcolorbox}
	Let us come back on the relation:
	
	giving the position of center of mass. His speed is:
	
	By denoting:
	
	where $\vec{P}$ is the linear momentum of the system, it comes:
	
	This relation show that if the sum of the external forces is zero then:
	
	Thus the amount of movement of the entire system is conserved and movement of the system center of mass is unaltered. This justifies the remarks made during the study of the conservation of momentum.

	In the study of interactions between particles, it is often convenient to use a reference frame linked to the center of mass of all particles. This center of mass is at rest in that frame, its speed is zero and also the total linear momentum, as shown by the above equations. This property is the powerful benefit of this description.
	\begin{tcolorbox}[title=Remark,colframe=black,arc=10pt]
	In mechanics, the use of the center of mass (material point) is particularly easy because the system of forces is only governed by the Newton's law. With electric charged particles, it is quite different. Electromagnetic effects are dominant in their acceleration, which induces interactive wave phenomena significantly more complex. That is why we will never see a study in this book of "charge center" when we will get in the section of Electrostatic or Electrodynamics...
	\end{tcolorbox}
	\begin{tcolorbox}[colframe=black,colback=white,sharp corners]
	\textbf{{\Large \ding{45}}Example:}\\\\
	We want to determine how far the center of mass of the Earth-Moon system is from the center of Earth by ignoring the other objects in the solar system.

	We define the origin of the coordinate system as the center of Earth. Then, with just two objects, then the center of mass theorem will be written:
	
	we defined the center of Earth as the origin, so $r_{\text{Earth}}=0$ [m]. Inserting these into the equation for $r_G$ gives:
	
	The radius of Earth is $6.37\cdot 10^6$ [m] , so the center of mass of the Earth-Moon system is approximately $1,730$ [km] below the surface of Earth.
	\end{tcolorbox}
		
	\pagebreak 
	\subsubsection{Guldin's Theorem}
	The "\NewTerm{Guldin's Theorems}\index{Guldin's Theorems}" or also named "\NewTerm{Pappus's Centroid Theorem}\index{Pappus's Centroid Theorem}" gives the possibility in some cases to simplify the calculation of the center of mass of some bodies (or reciprocally the volume or surface when the coordinates of the center of mass is known).

	\begin{enumerate}
		\item First Theorem:
		\begin{theorem}
		The volume $V$ of a solid of revolution generated by rotating a plane figure $F$ about an external axis is equal to the product of the area $S$ of $F$ and the distance $d$ traveled by its geometric centroid.
		\end{theorem}
		\begin{dem}
		Given a flat plate, homogeneous, of constant thickness $e$, of volumic density $\rho$ placed in a Cartesian plane $x$O$y$. We know that we have compared to the $y$-axis:
		
		
		Therefore:
		
		
		Let us consider now the following figure:
		\begin{figure}[H]
			\centering
			\includegraphics{img/mechanics/guldin_schema.jpg}
		\end{figure}
		Let us consider a rotation of $2\pi y$ around the $x$ axis of small surface located on $y$ following the vertical axis. The element of volume described by a surface element $\mathrm{d}S$ during this rotation is obviously equal to:
		
		and therefore, the total volume described by the complete surface $S$ is using the previous relation:
		
		Thus, by doing the same for $x_G$, we finally get:
		
		\begin{flushright}
			$\square$  Q.E.D.
		\end{flushright}
		\end{dem}
		\begin{tcolorbox}[colframe=black,colback=white,sharp corners]
		\textbf{{\Large \ding{45}}Example:}\\\\
		For example, the volume of the torus (\SeeChapter{see section Geometric Shapes page \pageref{torus}}) with minor radius $r$ and major radius $R$ is:
		
		\end{tcolorbox}
		
		\item Second Theorem (or first depends on the point of view...):
		\begin{theorem}
		The surface area $S$ of a surface of revolution generated by rotating a plane curve $C$ about an axis external to $C$ and on the same plane is equal to the product of the arc length $L$ of $C$ and the distance $d$ traveled by its geometric centroid.
		\end{theorem}
		\begin{dem}
		Let us consider a curved path or string, homogeneous, of length $L$, of constant section and of linear mass density $\rho_l$. We know that we have compared to the $y$-axis:
		
		and therefore:
		
		
		Let us consider now the following figure:
		\begin{figure}[H]
			\centering
			\includegraphics{img/mechanics/guldin_schema_surface.jpg}
		\end{figure}
		Let us consider a rotation around the $x$ axis. The surface described by an element of length $\mathrm{d}l$ during this rotation is equal to:
		
		and therefore the total area described by the string of length $L$ is:
		
		Thus, by doing the same for $x_G$, we finally get:
		
		\begin{flushright}
			$\square$  Q.E.D.
		\end{flushright}
		\end{dem}
		
	
		\begin{tcolorbox}[colframe=black,colback=white,sharp corners]
		\textbf{{\Large \ding{45}}Example:}\\\\
		For example, the surface area of the torus (\SeeChapter{see section Geometric Shapes page \pageref{torus}}) with minor radius $r$ and major radius $R$ is:
		
		\end{tcolorbox}
	\end{enumerate}
	
	\pagebreak
	\subsection{Kinematics of Rectilinear Motion}\label{kinematics of rectilinear motion}
		If a phenomenon is evolving, by observing it, we see a shift from the concrete value of one or more characteristic variables. These values are obviously not constant but variable. An evolution implies that there is a beginning, an infinite number of intermediate states and an end. 

	\textbf{Definition (\#\mydef):} A "\NewTerm{state}\index{state (physics)}" is the description of a snapshot of an evolving phenomenon (not necessarily in the temporal sense).

	The functional relation between variables for a given state can be described by an equation. For an evolutionary phenomenon, there may be an infinite number of states that we can describe by as many equations. In this form, it has no interest. We are looking for a single equation that relates the different quantities veryfing all the states that the evolutionary phenomenon may admit. By this equation, we can then calculate any state of the evolutionary phenomenon studied: this is the "\NewTerm{equation of state}\index{equation of state}" (a concept taken from the section of Thermodynamics).

	Therefore kinematics is the branch of Classical Mechanics which describes the motion of points (alternatively "particles"), bodies (objects), and systems of bodies without consideration of the masses of those objects nor the forces that may have caused the motion.
	
	Kinematics begins with a description of the geometry of the system and the initial conditions of known values of the position, velocity and or acceleration of various points that are a part of the system, then from geometrical arguments it can determine the position, the velocity and the acceleration of any part of the system. The study of the influence of forces acting on masses falls within the purview of kinetics.
	
	Kinematics is obviously used in wide range of fields like astrophysics to describe the motion of celestial bodies and collections of such bodies. In mechanical engineering, robotics, and biomechanics kinematics is used to describe the motion of systems composed of joined parts (multi-link systems) such as an engine, a robotic arm or the skeleton of the human body.
	
	We consider mainly three sub-domains of kinematics:
	\begin{enumerate}
		\item "\NewTerm{Rectilinear kinematics}\index{rectilinear kinematics}" or "\NewTerm{Translation kinematics}\index{translation kinematics}" investigates laws of motion of objects along straight line without any reference to forces that cause the motion to change.

		\item "\NewTerm{Curvilinear kinematics}\index{curvilinear kinematics}" investigates lows of motion of objects in space in two and three directions without any reference to forces that cause the motion to change.

		\item "\NewTerm{Rotational kinematics}\index{rotational kinematics}" investigates lows of motion of objects along circular path without any reference to forces that cause the motion to change.
	\end{enumerate}
	
	\pagebreak
	\subsubsection{Position}
	\textbf{Definition (\#\mydef):} The "\NewTerm{position}\index{position}" of an object is defined by its position vector in the particular case of a three-dimensional space given by:
	
	obviously each coordinate of a moving object may vary in function of time as:
	
	Rather than this notation rather heavy with parentheses ... physicists frequently note the position vector (or vector space) in the form of a vector of $4$ dimensions (three spatial components and one time component) and the we write:
	
	or simply denoted by:
	
	and we then call it a "\NewTerm{space-time four-vector}\index{space-time four-vector}" whose components are the generalized coordinates of the system.
	
	\subsubsection{Velocity}
	\textbf{Definition (\#\mydef):} The "\NewTerm{scalar velocity}\index{scalar velocity}", also named "\NewTerm{speed}\index{speed}", denoted $v$, is defined as the distance traveled by an object during a given amount of time:
	
	The SI unit for acceleration is obviously meter per second: [ms$^{-1}$].
	
	When a body is in uniform motion, that is to say, it travels a given distance following a dimension $x_i$ with $i=\{1,2,3\}$ in a time always equal, the previous ratio is constant in time:
	
	The "\NewTerm{arithmetic average speed}\index{arithmetic average speed}" or "\NewTerm{average linear speed}\index{average linear speed}" is defined as the ratio of the distance between a given starting point $x_{i,0}$ at instant $t_0$ and an end point $x_{i,f}$ at a time $t_f$:
	
	\begin{tcolorbox}[title=Remark,colframe=black,arc=10pt]
	You must take care when calculation the average speeds of as there are several types of averages in mathematics... (\SeeChapter{see section Statistics page \pageref{averages}})! For example, it is common to use the harmonic mean speed as we prove it in the section Statistics.
	\end{tcolorbox}
	This represents an average (because we are not interested to know how the path between $x_{i,f}$ and $x_{i,0}$ has been traveled) but not the instantaneous speed of the vehicle at any given time.
	
	If we want to know the speed named "\NewTerm{instantaneous speed}\index{instantaneous speed}" of the vehicle at a point of its trajectory must take the limit of the delta time $\Delta t$ to a differential element $\mathrm{d}t$ (\SeeChapter{see section of Differential and Integral Calculus page \pageref{differential}}) as:
	
	with $\varepsilon\rightarrow 0$.

	Mathematically, we note this correctly as follows:
	
	Thus, during an infinitely small time difference, the distance diffrerence will also be infinitesimal. So we have:
	
	and finally:
	
	The junior engineer or scientist must be careful when calculating the average velocity. Indeed as we have prove it in the section of Statistics, the harmonic average has to be used and not the arithmetic average!!!
	
	If the studied body is not in linear motion in three-dimensional Cartesian coordinate then its position will be given by the vector $(x,y,z)$ and we will denote its speed by:
	
	\begin{tcolorbox}[title=Remark,colframe=black,arc=10pt]
	If all the parts of a body moving at the same speed and in the same direction, then we have a "\NewTerm{translation movement}\index{translation movement}". By cons, in a "\NewTerm{rotational movement}\index{rotational movement}", the speeds of the various body parts are not the same, in amplitude and in direction (we will see it further below in details), and may vary over time.
	\end{tcolorbox}
	Caution! A movement can only be described scientifically with respect to a documented fixed reference point: the absolute motion does not exist. Galileo had already understood that: "The movement is as nothing." The movement does not exist in itself, but in relation to something else.
	
	\pagebreak
	\subsubsection{Acceleration}
	\textbf{Definition (\#\mydef):} The "\NewTerm{scalar acceleration}\index{scalar acceleration}", also known simply "\NewTerm{acceleration}\index{acceleration}", denoted "$a$", is by definition, the variation of the scalar speed during a given amount of time as (we go straight to the limit otherwise we speak of "\NewTerm{average linear acceleration}\index{average linear acceleration}"):
	
	or in other words: the speed at which the speed change...
	
	The SI unit for acceleration is obviously meter per second squared: [ms$^{-2}$].
	
	For example, when a car starts from a standstill (zero relative velocity) and travels in a straight line at increasing speeds, it is accelerating in the direction of travel. If the car turns, there is an acceleration toward the new direction. In this example, we can name the forward acceleration of the car a "linear acceleration", which passengers in the car might experience as a force pushing them back into their seats. When changing direction, we might name this "\NewTerm{non-linear acceleration}\index{non-linear acceleration}", which passengers might experience as a sideways force. If the speed of the car decreases, this is an acceleration in the opposite direction from the direction of the vehicle, sometimes named "\NewTerm{deceleration}\index{deceleration}". Passengers may experience deceleration as a force lifting them forwards. Mathematically, there is no separate formula for deceleration: both are changes in velocity. Each of these accelerations (linear, non-linear, deceleration) might be felt by passengers until their velocity (speed and direction) matches that of the car.
	
	Again, if the body is not uniform motion we have:
	
	If the body is in uniform rectilinear motion (we can always generalize to non-rectilinear movement as we will see further below) then we have:
	
	The constant is to be determined depending on the initial conditions. If the initial position at time zero is zero constant will be zero. If not we write:
	
	which gives us the distance traveled by a body during a given period of time.
	
	If the body is in rectilinear motion and constantly accelerates then we have:
	
	The constant is to be determined depending on the initial conditions. If the initial velocity at time zero is zero, the constant will be zero. Otherwise we write:
	
	We see this relationmore frequently in the form:
	
	or predominantly in the form:
	
	but we have:
	
	and if we integrate this relation, we get:
	
	that we find in the schools most frequently in the form:
	
	where the acceleration $a$ can be obviously negative when the object decelerate.
	
	This relation gives the position of a mobile in rectilinear and uniformly accelerated motion (in the majority of physical problems we consider the acceleration as constant). Of the latter, we deduce a large quantity of other useful mathematical relations that are very interesting in physics as well considering ideals cases (frictionless) or real cases. Indeed, rearranging it (elementary algebra), it is possible to obtain acceleration when the time to reach a given speed is known. We can also calculate the distance it takes for a mobile (always in rectilinear movement) with a given acceleration to reach a certain speed.
	
	The first case that we will consider and the best known, is the constant speed acceleration of a falling body in an environment free from any friction (case further treated in our study of tribology).
	
	As we have already proved above, we have when the initial velocity is zero:
	
	The two combined relations give (according to the tradition of use in schools, we replaced $x$ by $h$ to indicate that the position is often assimilated in practice at a height):
	
	We can take out from this relation the escape velocity of body (useful relation when we will study Astrophysics - see corresponding section - and interesting for comparison when for when we will study General Relativity).

	Suppose we already know that two bodies attract each other with an acceleration following classical Newton's model (that we will prove later in the section of Astronomy):
	
	Placed in the relation $v=\sqrt{2ah}$ of a falling body, we get:
	
	so at the surface of the main attractor body we have the "\NewTerm{escape velocity}\index{escape velocity}":
	
	We can answer from this relation, the question of why some solar system planets have atmospheres and others do not (obviously we must take normally into account the molecular agitation ...) as we will discussed ii more in details in the section Astronomy.
	
	What is also interesting in this relation is that we can calculate what should be the radius $R$ of a body of mass $m$ so that to its escape velocity is equal to that of light (referring to Black Holes as we will study them in the section of General Relativity).
	
	Therefore we have:
	
	We will see in the section of General Relativity after relatively long calculations in an isotropic gravitational field (Schwarzschild metric) that we will fall back on that relation.
	
	However...! This relation is not directly associated to Black Holes but to a hypothetical "\NewTerm{dark star}\index{dark star}" that is a theoretical object compatible with Newtonian mechanics that, due to its large mass, has a surface escape velocity that equals or exceeds the speed of light. Whether light is affected by gravity under Newtonian mechanics is unclear but if it were accelerated the same way as projectiles, any light emitted at the surface of a dark star would be trapped by the star's gravity, rendering it dark, hence the name. Dark stars are only analogous to Black Holes in general relativity.
	
	\pagebreak
	\paragraph{Osculator Plane}\mbox{}\\\\
	For this subject the reader should first absolutely refer to the section of Differential Geometry of the chapter Geometry and especially the part concerning the Frenet Fram that is a prerequisite to understand what will follow.
	
	The vectors speed $\vec{e}$ and acceleration $\vec{a}$ associated with a point $P$ moving form, at every moment $t$ a plane named "\NewTerm{osculating plane}\index{osculating plane}" of the path (usually curvilinear otherwise the plane is reduced to a straight line in the case of a rectilinear motion).

	It is often useful to split the acceleration vector in the osculating plane respectively along the tangent and the normal to the trajectory:
	
	where the first term of the right part is a vector parallel to the velocity and the second term a vector perpendicular to the velocity and located on the concave side of the path.

	Let us express these two vectors explicitly (a more general example is given in the section of Differential Geometry as already mentioned!).

	We can write:
	
	where $\mathrm{d}s$ is an element of the curve (curvilinear abscissa) from the path $\Lambda$ and $\vec{T}$ a unit vector tangent to the path and linked to the point $P$.
	
	The acceleration is then written:
	
	The first term on the right of the equality is the tangential acceleration and for the second term, even if the speed is constant it appears in the expression of the acceleration to express the change of direction of the speed vector.
	
	Let us decompose the vector $\vec{T}$ in the orthonormal Euclidean base $\mathbb{R}^2$ generated by the family of vector $\vec{i},\vec{j}$:
	
	After by derivating in respect to time:
	
	By comparing with the initial expression of the vector o$\vec{T}$, we see that the terms in brackets above are the components of a new unit vector $\vec{n}$ perpendicular  to the vector $\vec{T}$, thus perpendicular to the path and directed toward the center of curvature (\SeeChapter{see section of Differential Geometry page \pageref{frenet frame}}).

	Moreover by the definition of radian (\SeeChapter{see section Trigonometry page \pageref{radian}}), we have:
	
	where $R$ is the instantaneous (local) radius of curvature of the trajectory.	
	
	The expression $\mathrm{d}\vec{T}/\mathrm{d}T$ the becomes:
	
	and the second therm of the general expression of the acceleration then becomes:
	
	So we have finally (relation proved with another approach in the section of Differential Geometry):
	
	where once again $\vec{a}_\text{tg}$ is the tangential acceleration that express the variation of speed on the trajectory of the body (or point) $P$ and where  $\vec{a}_\text{n}$ is a term that express the change of direction of the body (point) $P$ without that the latter see its speed change! Commonly the normal acceleration is assimilated, when multiplied by the body (point) mass, to the "\NewTerm{centrifugal force}\index{centrifugal force}" (centrifugal meaning: that go away from the center). We will come back more in the details on this force further below. 
	\begin{tcolorbox}[title=Remark,colframe=black,arc=10pt]
	The centrifugal force is considered in physics as a fictitious force because in fact this is not a force that tends to move the body (particle) away from a center of rotation but it is just that there is a force which is not sufficient (the friction force in the case of carousel or the gravity force in case of planets or satellites) to prevent us from following a straight course by simple inertia. This is why when we are ejected from a carousel we leave tangentially to the rotation and not perpendicularly thereto.
	\end{tcolorbox}
	We immediately see that if $\vec{a}_\text{n}=0$ the movement can be at least necessarily linear but surely uniform, accelerated or not, while if $\vec{a}_\text{n}\neq \vec{0}$ the trajectory is necessarily curved.
	
	Indeed, as we just say the acceleration is basically rate of change of velocity. In a rotational motion, there are two components of the net acceleration: one Normal(along the radius) and one Tangential (along the circumference). The Normal acceleration serves the purpose of causing the motion to be circular, whereas the tangential one can make the rotation faster or slower. Since a uniform motion itself defines to be a constant velocity motion (i.e. the rotation is going on at a constant speed), the tangential component of acceleration is zero. If the rotation is non-uniform, that means the speed of rotation is changing, thus inevitably leading to the conclusion that the tangential component of the net acceleration is non-zero.
	
	\subsubsection{Galilean Relativity Principle}\label{Galilean Relativity Principle}
	\textbf{Definition (\#\mydef):} As far as we know it is impossible for a lively observer in uniform motion to know if he moves relatively to its environment or conversely if the environment is moving relative to him (we can not distinguish the rest and motion at constant speed and direction). Therefore, there can be no absolute reference frame (or privileged reference frame) that can be considered fixed vis-à-vis all other Galileans referential frame which clearly means that all Galilean reference frame must enjoy the same status in Mechanics as they can not be distinguished from each other. This principle is named the "\NewTerm{principle of Galilean relativity}\index{principle of Galilean relativity}" or "\NewTerm{Galilean invariance}\index{Galilean invariance}".
	
	Specifically, the term "Galilean invariance" today usually refers to this principle as applied to Newtonian mechanics, that is, Newton's laws hold in all frames related to one another by a Galilean transformation. In other words, all frames related to one another by such a transformation is inertial (meaning, Newton's equation of motion is valid in this frame). In this context it is sometimes named "\NewTerm{Newtonian relativity}\index{Newtonian relativity}".
	
	This principle (not to be confused with the principle of Special Relativity because the assumptions differ a little bit ...) follows directly from the study of what we name the "\NewTerm{Galilean transformation}\index{Galilean transformation}" that we will study below for Kinematics and for which we will give a more in-deep example in the section of Special Relativity with the Galilean Transformation of a wave equation.
	
	\textbf{Definition (\#\mydef):} A "\NewTerm{Galilean transformation}\index{Galilean transformation}" is a sequence of mathematical operations on a physical law that determines the properties of one or more "observable" (speed, strength, momentum, etc.) when we move during the study of a physical phenomenon from one repository to another repository: one supposed to rest (relatively to the second), and the other in uniform motion (or vice versa).
	
	The question at the historical origin was to answer if it was more legitimate to study a phenomenon in a repository or another one. More exactly, we want to determine whether the form of physical laws keep the same algebraic form regardless of the repository in which we study them.

	Let us have a more closer look at this:

	Given two repositories in movement relative to each other at a constant speed $\vec{u}(u_x,u_y,u_z)$. For a given Cartesian reference frame $(\vec{x},\vec{y},\vec{z})$ at rest (or supposed such) we will put for the second  repository $(\vec{x}',\vec{y}',\vec{z}')$ so that it is aligned with the $x$-axis of the first repository to simplify calculations with $\vec{u}(u_x,0,0)$:
	\begin{figure}[H]
		\centering
		\includegraphics{img/mechanics/repository_in_relative_movments.jpg}
		\caption{Example repositories in uniform relative motion with respect to each other}
	\end{figure}
	We will also put in the second frame of reference, a material point $P$ of coordinates $\vec{P}(x',y',z')$.
	\begin{tcolorbox}[title=Remark,colframe=black,arc=10pt]
	We assume known the concept of "linear momentum" $p$ defined further below rigorously. Let us just recall here that the linear momentum of the animated point $P$ of mass $m'$, with a speed $v$ (norm) in $(\vec{x}',\vec{y}',\vec{z}')$ is given by:
	
	\end{tcolorbox}
	We then have by applying the conventional relations of rectilinear kinematics:
	
	\begin{tcolorbox}[title=Remark,colframe=black,arc=10pt]
	Note that the last equation expresses the assumption of a universal time and mass independent of the relative motion of different observers. This has to be compared with our future study of Special Relativity (see the section with the corresponding name page \pageref{special relativity}).
	\end{tcolorbox}
	We get:
	
	hence ${v'}_v=v_x-u_x$ and therefore we get for the force (we will study further below what links linear momentum and force together):
	
	The result is interesting since Newton's second law keeps exactly the same shape and the same value in the two repositories (we better understand the name of "Newton relativity"). The fact that we were moving or not has therefore no impact on our vision of the world that remains exactly the same (in this model because when we will study Special Relativity we will see that this is not satisfied anymore!!!).
	
	Consequence: Since the forces are the same, no mechanical experience can determine whether a Galilean is the absolute coordinate system (i.e. two observers in two different Galilean frames, cannot with mechanical experience determine which moves relative to the other one).
	
	
	So in classical mechanics, there is no absolute Galilean reference frame!
	
	Notice, however, that this is achieved by assuming that:
	
	that is to say, we require that the relative velocity is uniform (constant), the mass is also constant and especially that $t=t'$.
	
	But in fact, this transformation is fundamentally wrong as discussed in more detail in our study of Special Relativity (see the section with the corresponding name page \pageref{special relativity}). Indeed, the easiest way to see it that is more experimental than theoretical (we keep the theoretical one for the section of Special Relativity) is to consider an object moving along the axis with a speed $v$ measured in the apostrophed reference frame:
	
	then what will be its velocity $w$ in the apostrophed reference frame? If the Galilean transformation is fundamentally true, it would be enought to replace in the above relation $x'$ and $t'$ by their expressions in function of $t$:
	
	thus (velocity addition law):
	
	But there was a  problem ... an experiment involving light rays (electromagnetic waves) was conducted at the beginning of the 20th century, and showed that this addition law was false. This experiment named "\NewTerm{Michelson-Morley experiment}\index{Michelson-Morley experiment}" changer forever our vision of the world ... and brought Albert Einstein to develop the theory of Special Relativity by requiring that the speed of light to be constant regardless of the movement of the reference frame (\SeeChapter{see section Special Relativity page \pageref{special relativity}}):
	
	If we measure the speed and other vector quantities, we find that the results of measurements of the components $x '$, $y'$, $z '$, $t'$ are not identical to those obtained on $x$, $y$, $z$, $t$. In physics, the "\NewTerm{principle of covariance}\index{principle of covariance}" emphasizes the formulation of physical laws using only certain physical quantities such that their measurements in different frames of reference can be unambiguously correlated. Mathematically, the physical quantities must transform "\NewTerm{covariantly}\index{covariantly transformation}", that is, under a certain representation of the group of coordinate transformations between admissible frames of reference of the physical theory. The corresponding group is the referred to as the "\NewTerm{covariance group}\index{covariance group}". The covariance principle states that the laws of physics should transform from one frame to another covariantly, that is, according to a representation of the covariance group (Lorentz matrices in Special Relativity and in General Relativity the covariance group consists of all arbitrary invertible and differentiable coordinate transformations).
	
	As we know, laws (formulas) are relations between observable (terms), relations deduced from many observations. The research of laws is governed by what we might name the "\NewTerm{principle of simplicity}\index{principle of simplicity}": laws in the smallest possible number, expression the simple as possible between terms and in minimum quantity. But the characteristic of a good law is also the covariance during a change of reference frame. The Physics (in the sense of the theory that describes the sensitive reality) will no longer be linked to the observer nor his associated Galilean space-time.
	
	However, let us give an important counterexample in classical mechanics: the force between two stationary electric charges in a referential frame use in this reference frame only electrostatic theory. If the same system is observed from a moving reference  frame relative to the first, it will have to be described by the electromagnetism theory (as we will see in the section of Electrodynamic an electric charge in movment produce an electric field). This can be explained only thanks to the covariance group of Special Relativity\footnote{See our study of the Joules-Bernoulli equations in the section of Special Relativity}.

	By construction classical mechanics is being covariant by Galilean transformation (change of Galilean reference frame): the dynamic postulate (force) takes the same form in the different Galilean frames as we have just seen!
	
	\pagebreak
	\subsection{Angular Momentum}
	\textbf{Definition (\#\mydef):} The "\NewTerm{angular momentum}\index{angular momentum}\label{angular momentum}" (rarely "\NewTerm{moment of momentum}\index{moment of momentum}" or "\NewTerm{rotational momentum}\index{rotational momentum}") $\vec{b}$ with respect to a point O of a particle of mass $m$ moving at the speed $\vec{v}$ on $\vec{r}$ is defined by:	
	
	with $\vec{p}$ being the linear momentum (see definition during the Newton's second law statement) given by:
	
	By definition, the angular momentum is a vector perpendicular to the plane containing the vectors $\vec{r}$ and $\vec{p}$ and if the particle moves in a plane, the direction of $\vec{b}$ is constant but not necessarily in the same direction depending or the rotating sense of the particle.
	\begin{figure}[H]
		\centering
		\includegraphics{img/mechanics/angular_momentum.jpg}
	\end{figure}
	A special but important case in astronomy and mechanics is to calculate the angular momentum of a circular plane motion of radius $r$. In this situation, the "\NewTerm{radius vector}\index{radius vector}" $\vec{r}$ is then always perpendicular to the direction of the velocity vector $\vec{v}$ and therefore:
	
	We see here appear the definition of the "\NewTerm{rotation vector $\vec{\omega}$}\index{rotation vector}" also sometimes denoted by $\vec{\Omega}$.
	
	For a non rectilinear motion path (a conical for example), we introduce the normal and tangential velocity components (we will in-deep this further below):
	
	But as the velocity is the change in position over time the velocity is (by definition) tangent to the curve. So we get:
	
	
	If we take a differentiable vector function $\vec{v}(t)$ (a velocity vector). If its speed is such that $\vec{v}(t)=c^{te}$ then at any point which $\mathrm{d}\vec{v}/\mathrm{d}t$ is non zero, $\mathrm{d}\vec{v}/\mathrm{d}t$ is perpendicular to  $\vec{v}(t)$.
	
	The shortest way to prove this assertion is to notice remember that:
	
	After differentiating:
	
	and since it speed is constant:
	
	Since $\vec{v}\neq 0$ and $\mathrm{d}\vec{v}/\mathrm{d}t\neq 0$ then they have to be perpendicular!
	
	So, coming back to:
	
	in scalar form this gives:
	
	where $r$ is therefore named the "\NewTerm{radius of curvature}\index{radius of curvature}" of the trajectory as we already know (\SeeChapter{see section Differential Geometry page \pageref{curvature radius}}).

	Let us now study the derivative of the angular momentum:
	
	In the right-hand side, we have by the definition of the vector product:
	
	and also:
	
	Which gives finally:
	
	The derivative with respect to time of the angular momentum of a moving point is equal to what we define as the "\NewTerm{moment of force $\vec{M}$}\index{moment of force}" which will also be discussed further below and whose units are that of energy and is a vector perpendicular to the plane defined by $\vec{r}$ and by $\vec{F}$ (by construction of the cross product as seen in the section of Vector Calculus!).
	\begin{tcolorbox}[title=Remarks,colframe=black,arc=10pt]
	\textbf{R1.} This latter relation makes that we sometimes name the angular momentum as "\NewTerm{moment of linear momentum}\index{moment of linear momentum}"...\\ 
	
	\textbf{R2.} When two or more waves propagate in a medium, the wave function that results is the algebraic sum of each wave of wave functions. We say then that the waves "interfere" and we name the phenomenon "\NewTerm{superposition wave principle}\index{superposition wave principle}\label{superposition wave principle}" (see further below for mathematical details).
	\end{tcolorbox}
	What is highly impressive in this result (instantaneous change of angular momentum), is that any body having a non-zero angular momentum and subject to no moment of force, maintains the direction and norm of $\vec{b}$ in space and time.
	
	This result will allow us to study the dynamic of the gyroscope and all other bodies having similar properties (such as the Earth rotating on itself and that points to the North Star which is a major cause of origin of the seasons!). We will study further below the gyroscope and its properties, because their behavior is fascinating and the theoretical results arising find applications in astrophysics, atomic physics and even in philosophy. Indeed, the conservation of the orientation of the angular momentum vector leads to the conclusion that even if the space was completely empty of its contents in the whole Universe and contained an object with an angular momentum, the empty space has however a property which allows the object in question where it oriented it angular momentum... which is quite disturbing! So the empty space is not nothing otherwise relatively to what turns the object into space? It rotates relative to the space itself!
	
	We also have:
	
	where the integral is named "\NewTerm{angular impulsion}\index{angular impulsion}" and the previous relation sometimes is named "\NewTerm{theorem of angular momentum}\index{theorem of angular momentum}" (we'll see a generalization of this theorem in the proof of König's theorem later). It is stated as following: The angular impulsion provided by a moment of force between the moments $t_1$ and $t_2$ is equal to the variation of the angular momentum during this time interval.
	
	In the study of solid dynamic this theorem plays a fundamental role, similar to the dynamic Newton equation $\vec{F}=m\vec{a}$.
	
	The use of angular momentum makes it easy to prove the Kepler law of areas (Kepler's second law), which plays an important role in the understanding of planetary motion (\SeeChapter{see section Astronomy page \pageref{kepler laws}}) or to prove that in an isolated Earth-Moon system, the total angular momentum is conserved, so if the Earth slows its rotation and the Moon keep it constant, it forces the Moon to increase its distance from Earth!!!
	
	Let us see this last statement:
	
	Imagine a particle moving under the action of a force $\vec{F}$ constant and parallel to $\vec{r}$. We say that this force is a "\NewTerm{central force}\index{central force}\label{central force}" if its direction constantly goes through the same fixed point, named the "\NewTerm{center of force}\index{center of force}". The magnitude of the force  only depends on the distance $r$ of the object from the origin and is directed along the line joining them (in the case of a force field):
	
	Therefore:
	
 	as in central force problems $\vec{F}$ is parallel to 
$\vec{r}$.

	Therefore the angular momentum relative to the center of force is constant if the force is central! The converse is also true: if the angular momentum is constant, its derivative with respect to time is zero and the direction of the force is always collinear to $\vec{r}$ so the force is central.

	For example, in the motion of a planet around the Sun or an electron around the nucleus of the atom (in the Bohr model...) the momentum of this force relatively to the center because is obviously zero as no outside element acts on the system, that is to say based on the figure below:
	\begin{figure}[H]
		\centering
		\includegraphics{img/mechanics/illustration_of_momentum.jpg}
		\caption{Illustration of momentum}
	\end{figure}
	Then we have:
	
	Therefore:
	
	On the other hand, the element surface $\mathrm{d}\vec{S}$ described by the motion of the radius $\vec{r}$ is (according to the figure above and the properties of the vector product):
	therefore:
	
	Using the relation $\vec{b}=\vec{r}\times m\vec{v}=\vec{c}^{te}$ we get:
	
	Consequences:
	\begin{enumerate}
		\item The areal velocity is constant in a central force problem, that is to say that the areas scanned in equal times are equal. It is the Kepler areas law (\SeeChapter{see section Astronomy page \pageref{kepler second law}})!

		\item The plane $\vec{r},\vec{v}$ is fixed because $\vec{r}\times m\vec{v}=\vec{c}^{te}$. Thus the trajectory of a planet in a ideal case is viewed as moving in a plane relatively to the reference frame of the force center.
	\end{enumerate}
	We'll obviously come back on this relation in the section Astronomy to write it in a slightly more traditional form.
	
	We (almost) all know that the Earth, the other planets and their moons all revolve around the Sun in elliptical orbits. However, have you ever stopped to wonder why these celestial objects move around the Sun in such a way that the Solar System appears to be lying on a plane, rather than going every which way and the same observation can also be done for Saturn's rings, galaxies or black holes disk accretion?
	\begin{figure}[H]
		\centering
		\includegraphics[scale=0.55]{img/mechanics/cloud_formation.jpg}
	\end{figure}
	We can trace the origin of our solar system to a massive shapeless 'blob' of matter floating through space about 4.6 billion years ago.

	The particles in this blob gradually began to move closer due to gravity, and whenever there are multiple particles and powerful gravitational forces, there are also collisions and therefore heating and emissions of radiations outside of the cloud. These collisions and the subsequent trajectories of these particles are obviously random, making them  almost impossible to predict. Although these objects move randomly, the one thing that remains constant as we just saw is the angular Momentum as the latter does not irradiate anything outside the cloud and to be changed must be influence by an external force!!! The Solar System is an isolated system in itself. The galaxy is also an isolated system, since the gravitational effect of other cosmic objects is negligible.

	So an isolated system and therefore without interaction with external objects, the total Angular Momentum has to be conserved.

	This physical quantity is constant around a fixed axis. This axis is a point in $2$-dimensional space, and in our world, which exists in $3$ dimensions, this axis turns out to be a line. The system rotates along a plane that is perpendicular to the axis. Therefore, whenever particles collide, they may move in any direction, but all the up and down motion cancels out, always following the rule that the total spin in that plane must be constant. Over time and after countless collisions, these particles lose their freedom in everything except 2-dimensional space, thereby aligning themselves in a plane.
	
	\subsubsection{Moments}
	 A "\NewTerm{moment}\index{moment}" (not to be confused with Momentum!) or "\NewTerm{torque}\index{torque}" is an expression involving the product of a distance and a physical quantity, and in this way they account for how the physical quantity is located or arranged. Moments are usually defined with respect to a fixed reference point; they deal with physical quantities as measured at some distance from that reference point. For example, the moment of force acting on an object, often named "torque", is the product of the force and the distance from a reference point. In principle, any physical quantity can be multiplied by distance to produce a moment; commonly used quantities include forces, masses, and electric charge distributions.
	 
	 OK let's go! In fact we have seen just above that the "\NewTerm{moment of force}\index{moment of force}" was defined by (temporal variation of the angular momentum):
	
	where $\vec{M}$ is the moment (or "torque" also denoted sometimes by the letter $\tau$) of the force $\vec{F}$ relatively to the origin of the vector $\vec{r}$. It is important to note that the force momentum has for units "energy" ([J]) and is also perpendicular to $\vec{r}$ and $\vec{F}$ by construction!
	
	It must be also notice that increasing the radius of force application thereby giving the opportunity to decrease the force necessary to keep a constant force momentum in a mechanical system certainly reduces the effort (force) but in the end not the energy expended since the traveled distance is bigger.
	
	If we express the norm of the force moment $\vec{M}$, by the definition of the cross product, we get:
	
	Therefore appears a quantity:
	
	which by definition is the "\NewTerm{lever arm}\index{lever arm}" of the force $\vec{F}$ and whose location is given by the axis of rotation of the body due to the resulting moment of force (do not confuse this with the notation $\vec{b}$ of the angular momentum!).
	
	Warning! The principle of the lever arm is thus a great force multiplier but in no case it multiplies the work (energy)!
	\begin{figure}[H]
		\centering
		\includegraphics{img/mechanics/torque.jpg}
		\caption{Example of some everyday lever arm}
	\end{figure}
	\begin{tcolorbox}[title=Remark,colframe=black,arc=10pt]
	The case of application of changes of winter/summer of tourism cars is well known by motorists as it is recommended by most manufacturers to apply a torque of $120$ [Nm] for the tightening of the bolts.
	\end{tcolorbox}
	For an extended body, subject to forces $\vec{F}_1,\vec{F}_2,\ldots,\vec{F}_n$ to be in total equilibrium, it is not only necessary that the resultant of these forces is zero (no translation) but also the resulting force moments is zero too (no rotation) . That is to say:
	
	Thus explicitly:
	Thus more explicitly:
	
	When the components of a system that meets the two above relations are known, then we speak of sometimes of "\NewTerm{isostatic system}\index{isostatic system}".
	
	By definition, a "\NewTerm{couple of force}\index{couple of force}" or simply "\NewTerm{couple}\index{couple}" is defined as a set of two forces of equal magnitude but opposite direction, acting along two parallel lines on the same extended body. The resultant of course no forces indicates that the couple produces no translation effect. But the sum of the moments is not zero, the body undergoes a rotation such that:
	
	\begin{figure}[H]
		\centering
		\includegraphics{img/mechanics/couple_of_force.jpg}
		\caption{Example of couple of force}
	\end{figure}
	To avoid any confusion here is a small refresh:
	\begin{itemize}
		\item A Moment is equivalent to a force multiplied by the length of the line passing through the point of reaction and that is perpendicular to the force.

		\item Torque is a moment that is applied in such a way that it tends to rotate a body around its axis. The usual example being a torsional moment on a shaft.

		\item  A Couple is the moment that is the resultant of two forces of the same magnitude, acting in opposite direction at the same distance from the reaction point. Therefore, this force system is statically equivalent to a pure moment (no resultant force).
	\end{itemize}
	Let us also mention the composition of force moments with the following ultra-classic case of balanced moment:
	\begin{figure}[H]
		\centering
		\includegraphics{img/mechanics/balanced_moment.jpg}
		\caption{Balanced moment example}
	\end{figure}
	And also the analysis of the typical well known situation involving two moments of force:
	\begin{figure}[H]
		\centering
		\includegraphics{img/mechanics/composition_of_force_moments.jpg}
		\caption[]{Composition of force moments}
	\end{figure}
	for which it is quite obvious that the force of the resulting moment of force has magntitude the sum of the two elementary forces:
	
	To calculate the corresponding resultant distance, we will use the theorem of the center of mass, which as we have prove above and in the section of Euclidean Geometry is given in the general case by:
	
	and that in the present case is reduced to:
	
	We thus have resulting final moment which is:
	
	Thus, the resultant moment of force is the simple sum of the elementary moment of forces of forces.
	\begin{figure}[H]
		\centering
		\includegraphics{img/mechanics/resulting_moment_of_force.jpg}
		\caption[]{ResultingcComposition of force moments}
	\end{figure}
	Now that we have quite properly defined what was a force and moment of force, we can immediately begin the study of static forces:
	
	\pagebreak
	\subsubsection{Static Forces}
	The study field of static forces is an quite difficulty domain to generalize when the target students discover the concept of forces and momentum. Most books use many examples (like pulley systems, levers, balances, friction, truss, etc.) to get the reader to assimilate the method of analysis needed to resolve problems in this field of classical mechanics. Far from being against this method, we do not want to restrict ourselves or dwell (following viewpoints) to specific examples, but we have preferred to propose a method of analysis that would work without fail in any case.
	
	\textbf{Definitions (\#\mydef):}
	\begin{enumerate}
		\item[D1.] The "\NewTerm{static forces}\index{static forces}" is the field of physics that studies the effect of the resulting forces (or moments of force) that are constant over time, applied on a punctual or extended body.

		\item[D2.] When sum of all vector forces and moments of force is zero, there is no movement for a body initially not moving. We then speak of "\NewTerm{static equilibrium}\index{static equilibrium}" (but the forces still exist within the system) such that the forces and moments of forces cancel each other:
		
		More generally an object is in equilibrium if the linear momentum of its center of mass is constant and if its angular momentum about its center of mass is constant:
	\end{enumerate} 
	\begin{tcolorbox}[title=Remark,colframe=black,arc=10pt]
	The previous relations show us that it is not because a system is in static equilibrium that is not subject to any force (the vector sum of forces can cancel but the forces are nonzero )!
	\end{tcolorbox}
	We have then the following corollaries:
	\begin{enumerate}
		\item[C1.] When analyzing a static forces system, we must always (!!!) work with vector components of forces and moments of forces (by the first Newton's law!).

		\item[C2.] We must therefore impose a reference frame in which all forces components will be expressed:
		\begin{itemize}
			\item In the case of punctual body (particle point) on which forces are applied, we must assimilate the origin of the coordinate to the position of that point.

			\item If the extension lines of all forces applied on an extended body are all converging at a given point, the system can be considered as an isolated body reduced at this point of convergence.

			\item If the extended body is plunged into an isotropic, coplanar and constant over time field of forces (gravitational, electrostatic, magnetic ...) then all applied foces can be reported to the center of gravity.
		\end{itemize}
		\begin{dem}
		We have seen during our study of the section of Vector Calculus that the sum of vectors of a  same set, put extremity by extremity (at the pictorial representation level) or algebraically added constitute what we name the "\NewTerm{resultant}\index{resultant}" of the system of forces or momentum forces (torques):
		
		It is clear that a material point is therefore defined in a static state if the resultant of the competing forces is equal to zero. Thus, a punctual body is at rest (constant speed equal to zero) if the quantity $\vec{R}_{\vec{F}}$ is zero.
	
		This condition is however not sufficient for an extended body (non-punctual): the latter can not move (no translation movement), but can turn on itself by the application forces outside its barycenter (the forces are then momentum forces- torques - acting on points of the body in question).
	
		Now let us imagine a set of forces $\vec{F}_i$, each of them applied at a point of position-vector $\vec{r}_i$ of an extended mobile and all parallel to a given common direction, indicated by a unit vector $\vec{u}$. The resultant of these forces is then:
		
		The resulting norm is then obviously:
		
		Similarly, the vector sum of the parallel corresponding moment of forces (torques) are written:
		
			Let us now search the vector position $\vec{r}_C$ of a fictive point $C$, named the "\NewTerm{center of force}\index{center of force}" such as the moment of the resultant $\vec{R}_{\vec{F}}$ applied at point $C$ is equal to the total time equation. In other words, $\vec{r}_C$ must be the solution of the vector equation:
		
		If it is possible to find such a point $C$, so we do not longer need, in principle, to calculate the individual force momentum of each force and make the vector sum. Rather it is enough to determine the resultant $\vec{R}_{\vec{R}}$ and evaluate its resultant momentum force (torque) applied to the fictive point 
	$C$.
	
		Combining the above relations, we have:
		
		But we have show just before that:
		
		thus by equating:
		
		It follows:
		
		From which we finally draw:
		
		as $F_i=m_ia_i$ (Newton's second law) let us now assume (special case but well know in astronomy when we deal with gravitational force at almost constant distance) that $\forall i,j: a_i=a_j$ we can then write this very important result:
		
		We recognize in the last expression the center of mass the we have studied earlier above:
		
		\begin{flushright}
			$\square$  Q.E.D.
		\end{flushright}
		\end{dem}
		\begin{figure}[H]
			\centering
			\includegraphics[scale=0.5]{img/mechanics/equilibrium_moment_of_force_center_of_gravity_funny.jpg}
			\caption[]{Funny application example of what we have just proved}
		\end{figure}
		
		\item[C3.] By the third Newton's law, any rigid solid body in stable equilibrium in contact with a set of rigid solids also in stable equilibrium, suffer all of the same equal force at each point of contact (identically distributed) but opposed by these latest (and assimilated through their center of gravity when it is an isotropic and constant vector field that is causing the contact). Therefore:
		\begin{itemize}
			\item The application points of action/reaction forces must be placed on the various points of contact when it is a countable amount of forces that are behind and that are not parallel.
			
			\item The application points of action/reaction forces can be places at the center of mass or gravity if the forces causing the contact (verbatim: acceleration) are the source of a gravitational vector field, respectively electrostatic / magnetic (meaning the forces are in a first approximation all parallels).
		\end{itemize}
	\end{enumerate}
	
	The problem is to find the maximum weight the crane could carry as a function of distance before it would tip over (the supports at $D$ and $E$ aren't bolted to the ground).
	
	\begin{tikzpicture}[scale=0.35]
	\draw (-4,0) -- (40,0);
	\draw[double] (0,0) -- ++(1.5,0.5)--++(3,0)--++(1.5,-0.5);
	\draw[double] (0,0) -- ++(1.5,1.5) ++(3,0)--++(1.5,-1.5);
	\draw[double] (1.5,0.5) -- ++(0,10)--++(3,0)--++(0,-10);
	\draw[double,join=bevel] (1.5,0.5) -- ++(18.43:3.16) -- ++(161.57:3.16) -- ++(18.43:3.16) -- ++(161.57:3.16) -- ++(18.43:3.16) -- ++(161.57:3.16) -- ++(18.43:3.16) -- ++(161.57:3.16) -- ++(18.43:3.16) -- ++(161.57:3.16);
	\draw[double] (1.5,0.5) -- +(135:0.707);
	\draw[double] (4.5,0.5) -- +(45:0.707);
	\draw[fill=lightgray] (-1,8.5) rectangle (-3,14) node at +(1,-2.75) {$A$};
	\draw[double] (-1,10.5) -- ++(40,0) -- ++(0,1.732) -- ++(-40,0);
	\draw[double,join=bevel] (-1,10.5+1.732) -- ++(-60:2) -- ++(60:2) -- ++(-60:2) -- ++(60:2) -- ++(-60:2) -- ++(60:2) -- ++(-60:2) -- ++(60:2) -- ++(-60:2) -- ++(60:2) -- ++(-60:2) -- ++(60:2) -- ++(-60:2) -- ++(60:2) -- ++(-60:2) -- ++(60:2) -- ++(-60:2) -- ++(60:2) -- ++(-60:2) -- ++(60:2) -- ++(-60:2) -- ++(60:2) -- ++(-60:2) -- ++(60:2) -- ++(-60:2) -- ++(60:2) -- ++(-60:2) -- ++(60:2) -- ++(-60:2) -- ++(60:2) -- ++(-60:2) -- ++(60:2) -- ++(-60:2) -- ++(60:2) -- ++(-60:2) -- ++(60:2) -- ++(-60:2) -- ++(60:2) -- ++(-60:2) -- ++(60:2) node at +(0.75,-1.723/2) {$B$};
	\draw[fill=lightgray] (4.6,10.5) rectangle (39.1,9.5);
	\draw[double] (0,10.5) -- ++(0,-1.5) -- +(1.5,0);
	\draw[double] (0,9) -- +(45:2.121);
	\draw[fill=gray] (-0.6,13) rectangle (0.6,10.5+1.732);
	\draw[fill=lightgray, even odd rule] (18.5,5) circle (0.25) circle (0.125);
	\draw[double distance=0.4] (-0.5,13) -- ++(0,-4) ++(0.5,-0.5) -- ++(18,0) ++(0.5,-0.5) -- ++(0,-3);
	\draw (18,8.55) arc (90:0:0.55);
	\draw[fill=white] (18,8) circle (0.5);
	\draw[color=white] (4.6,9.75) -- (39.1,9.75);
	\draw[fill=white] (18.625,10.125) circle (0.375) +(-1.25,0) circle (0.375);
	\draw[rounded corners, fill=lightgray] (18,7.5) -- (19,10.375) -- ++(-2,0) -- cycle;
	\draw (18,8) circle (0.07);
	\draw (18.625,10.125) circle (0.07) +(-1.25,0) circle (0.07);
	\draw (-0.55,9) arc (180:270:0.55);
	\draw[fill=gray] (0,9) circle (0.5) circle (0.07) node at +(0,-1.25) {$C$};
	\draw[fill=gray] (0,13) circle (0.6) circle (0.07) node at +(1.25,0) {$M$};
	\draw[double] (18.5,5) -- +(3,-1) (18.5,5) -- +(-3,-1);
	\draw[fill=gray] (15.5,4.05) rectangle (21.5,3.755);
	\draw[fill=red] (17,4.06) rectangle +(0.5,0.3);
	\draw[fill=red] (17.5,4.06) rectangle +(0.5,0.3);
	\draw[fill=red] (18,4.06) rectangle +(0.5,0.3);
	\draw[fill=red] (18.5,4.06) rectangle +(0.5,0.3);
	\draw[fill=red] (19,4.06) rectangle +(0.5,0.3);
	\draw[fill=red] (19.5,4.06) rectangle +(0.5,0.3);
	\draw[fill=red] (17.5,4.37) rectangle +(0.5,0.3);
	\draw[fill=red] (18,4.37) rectangle +(0.5,0.3);
	\draw[fill=red] (18.5,4.37) rectangle +(0.5,0.3);
	\draw[fill=red] (19,4.37) rectangle +(0.5,0.3);
	%Dimensions
	\draw[semithick] (0,-0.25) -- +(0,-1) node at +(0,1.25) {$D$} ++(3,-3) -- +(0,3.25+10.5+1.723) ++(3,3) -- +(0,-1) node at +(0,1.25) {$E$} ++(33,-3) -- +(0,12.25);
	\draw[semithick] (3,10.75+1.732) -- (3,13.6+0.25+1+1.5) ++(-3,-1.5) -- +(0,-1);
	\draw[semithick] (-2,14.25) -- (-2,13.6+0.25+1+1.5);
	\draw[semithick] (9,6)--++(0,2.25) ++(0,0.5)--(9,9.95);
	\draw[semithick] (18.5,2) -- +(0,1);
	\fill (9,10.5) -- ++(0.3,0) arc (0:-90:0.3) -- ++(0,0.6) arc (90:180:0.3);
	\fill[white] (9,10.5) -- ++(0.3,0) arc (0:90:0.3) -- ++(0,-0.6) arc (270:180:0.3);
	\draw[very thin] (9,10.5) circle (0.3) node at +(0,2.5) {$G$};
	\draw[semithick, to-to] (0, -0.75) -- +(3,0) node[font=\footnotesize] at +(1.5,-0.5) {3 m};
	\draw[semithick, to-to] (3, -0.75) -- +(3,0) node[font=\footnotesize] at +(1.5,-0.5) {3 m};
	\draw[semithick, to-to] (3, -2.5) -- +(36,0) node[fill=white, font=\footnotesize] at +(18,0) {36 m};
	\draw[semithick, to-to] (3,13.6+0.25+0.5) -- +(-3,0) node[font=\footnotesize] at +(-1.5,0.5) {3 m};
	\draw[semithick, to-to] (3,13.6+0.25+2) -- +(-5,0) node[font=\footnotesize] at +(-2.5,0.5) {5 m};
	\draw[semithick,to-to] (3,6.5) -- +(6,0) node[fill=white, font=\footnotesize] at +(3,0) {6 m};
	\draw[semithick,to-to] (3,2.5) -- +(15.5,0) node[fill=white,font=\footnotesize] at +(7.75,0) {$x$};
	\end{tikzpicture}
	\begin{tikzpicture}[domain=0:36,x=100, y=20, scale=0.1]
	    \draw[very thin,color=gray] (0,0) grid[xstep=5,ystep=20] (36,140);
	    \foreach \x in {0,5,...,35}
	        \draw (\x,1) -- (\x,-1)
	            node[anchor=north] {\x};
	\foreach \y in {0,20,...,140}
	        \draw (0.5,\y) -- (-0.5,\y)
	            node[anchor=east] {\y};
	    \draw[->] (0,0) -- (36,0) node at +(-18,-12) {$x$ (m)}; 
	    \draw[->] (0,0) -- (0,140) node[rotate=90] at +(-3,-70) {Weight (kN)};
	    \draw[color=red,domain=4.5:36, smooth, semithick] plot (\x,{209/(\x-3)}) node[right] {$W_{max}$};
	\end{tikzpicture}
	
	\subsection{Ballistics }
	\textbf{Definition (\#\mydef):} "\NewTerm{Ballistics}\index{ballistics}" is the science or study of the motion path of a body having an initial velocity through a given medium in space immersed in a given (most of time: constant) field (gravitational, magnetic or electrostatic).
	
	\begin{tcolorbox}[title=Remark,colframe=black,arc=10pt]
	We have hesitate, and still do, to write this subject in the section of Spatial Engineering. So it is not impossible that in a near of or far future this subject moves in another section.
	\end{tcolorbox}
		
	We will prove now thanks to our knowledge in kinematics that the parabolic movement is the natural movement of a mobile body in the field of gravity in empty space animated (without friction or where friction can be neglected) by a travel speed $\vec{v}_0$  not necessarily parallel to the acceleration of gravity $\vec{g}$ (in general this is therefore a curvilinear motion ). For example, a projectile having an initial velocity $\vec{v}_0$ inclined from an angle $\beta$ from the horizontal:

	\begin{figure}[H]
		\centering
		\includegraphics{img/mechanics/ballistics.jpg}
		\caption{Ballistics configuration study}
	\end{figure}
	And with some vocabulary stuff associated:
	\begin{figure}[H]
		\centering
		\includegraphics{img/mechanics/ballistics_vocabulary.jpg}
		\caption[]{Typical Artillery vocabulary in ballistics}
	\end{figure}
	Obviously the reader can ask himself if the study of ballistics is useful outside artillery?! In fact even if does not apply to rocket launch as their velocity is not constant (boosters) and the application to golf is not quite interesting for engineers we have some practical situation where surch calculations are useful. A contemporary one is "ice ballistic problem" for the security radius around wind turbine to avoid people or houses to receive blocks of ice (in winter) that detach from the pales in rotation at a given rotation speed $\omega$ and for sure the trajectory if this ice is a ballistic trajectory!!! 
	
	In the absence of gravity and friction the mobile $P$ would follow the straight departure line indefinitely. The action of the gravity is to take it down, at time $t$, of the well known value to us $\dfrac{1}{2}gt^2$.

	We put the projection on the axes:
	
	combination of smooth movement along the $x$-axes and a falling motion with initial velocity $v_{0y}$ along the $y$-axes. Which corresponds to the following equations:
		
	eliminating the time between these two equations we get the path (equation of a parabola):
		
	which had already been obtained by Galileo in the early 17th century.

	We calculate also the target range $x_{\max}$ of the projectile by putting $y=0$ in the above equation and we get easily:
		
	the solution $x=0$ have no interest.

	The maximum height $y_{\max}$ can be calculated by canceling the derivative of the equation of the trajectory. So we get easily:
	
	We notice that for the maximum range for a given initial speed we have two typical cases to consider:
	\begin{enumerate}
		\item We give ourselves and unreachable target range $x_{\max}$ as it is not possible that $\sin(\beta)>1$.

		\item As we have $\sin(2\beta)$ in the expression of $x_{\max}$ it is obvious that it is $\beta=\pi/4$ ($45^\circ$) that gives the maximum range of a give initial speed.
	\end{enumerate}
	\begin{figure}[H]
		\centering
		\includegraphics[scale=0.8]{img/mechanics/ballistics_trajectories.jpg}
		\caption[]{Influence of the angle $\beta$ on the trajectory (source: Femtophysique)}
	\end{figure}
	The curve enveloping all parables, plotted for a given initial speed $v_0$ in all possible directions, is still a parable named "\NewTerm{safety parabola}\index{safety parabola}". Its rotation about the $y$-axis creates a paraboloid which circumscribes (contains) the region of the space accessible only to projectiles.
	\begin{figure}[H]
		\centering
		\includegraphics{img/mechanics/safety_parabola.jpg}
		\caption[]{Ballistic safety parabola}
	\end{figure}
	Thus, it is not too difficult to find the equation of this safety parabola :

	The vertically shot ($\beta=\pi/4$) is known to us and is given by:
	
	The maximal range is as we know given by:
	
	So when $x=x_{\max}\Rightarrow y=0$ such that:
	
	which is the equation of the safety parabola.
	
	The time of flight is the time it takes for the projectile to finish its trajectory. As we have for the vertical component:
	
	and the "\NewTerm{time of flight}\index{time of flight}" is given when $y=0$, it comes therefore:
	
	That is to say:
	
	The "\NewTerm{angle of reach}\index{angle of reach}" is the angle $\beta$ at which a projectile must be launched in order to go a distance $x_{\max}$, given the initial velocity $v_0$. So from:
	
	We get:
	
	Ok now much more interesting... To hit a target at range $x_T$ and altitude $y_T$ when fired from $(0,0)$ and with initial speed $v_0$ the required angle(s) of launch $\theta$ are:
	
	The two roots of the equation correspond to the two possible launch angles, so long as they aren't imaginary, in which case the initial speed is not great enough to reach the point $(x,y)$ selected.
	\begin{dem}
	First, let us recall that:
	
	Solving the first for $t$ and substituting this expression in the second gives (using the trigonometric identities proved in the section of Trigonometry):
	
	After a small rearrangement:
	
	Let us put $\Gamma=\tan(\beta)$:
	
	Solving this second order polynomial as we learned to do it in the section Calculus we have for discriminant:
	
	So we get after simplifications:
	
	Hence:
	
	If instead of a coordinate $(x,y)$ it is required to hit a target at distance $r$ and angle of elevation $\theta$  (polar coordinates), we use the relations $x_T=r_T\cos(\theta_T), y_T=r_T\sin(\theta_T)$  and substitute to get:
	

	\begin{flushright}
		$\square$  Q.E.D.
	\end{flushright}
	\end{dem}
	
	\pagebreak
	\subsection{Kinematics of Circular Motion}\label{kinematics of circural motion}
	The motion of objects as they translate -- move bodily from one place to another -- follows a simple set of rules. It turns out that a very similar set of rules describes the motion of objects as the rotate -- spin around in place.

	Most situations we have considered so far involve the type of motion named "translation" (pick it up and put it down somewhere else).

	But centrifuges, planets, space stations, satellites, molecules, motor engines exhibit a different sort of motion: rotation (spit in in place)!

	The circular movement, also named "\NewTerm{rotation movements}\index{rotation movements}", thus describe the rotation of an object around an axis (or a point to make simpler). The usage is to define it with the following information:

	\begin{itemize}
		\item The direction of the axis of rotation plane in space

		\item The direction of rotation on the constant radius circle around this axis

		\item The speed $v$ on the circular path.
	\end{itemize}
	All this can be resumed in the following figure:
	\begin{figure}[H]
		\centering
		\includegraphics{img/mechanics/circular_motion_principle.jpg}
	\end{figure}
	where for:
	
	we have used the relation prove in the section of Trigonometry (it is just logical by definition if the angle is measured in radians):
	
	We summarize these three directions by the data give by a vector "\NewTerm{instantaneous angular velocity}\index{instantaneous angular velocity}":
	
	The direction of rotation is said to be "positive" when the thumb drawn in the direction of the unit vector $\vec{n}$, we take the axis of the right hand and one sees the object rotate in the direction of the other four fingers ("\NewTerm{right hand rule}\index{right hand rule}"):
	\begin{figure}[H]
		\centering
		\includegraphics[scale=0.5]{img/mechanics/right_hand_rule.jpg}
		\caption[Right Hand Rule]{Right Hand Rule (source: Wikipedia)}
	\end{figure}
	The norm of the instantaneous angular speed, represents obviously the angle traveled per unit of time,by the object that moves in the plane perpendicular to the unit vector $\vec{n}$:
	
	
	Of course, it goes without saying that the angular velocity is given in radians per second and not in revolutions per minute or degrees per second!!! We must therefore be careful to always do the right conversion!
	
	\begin{tcolorbox}[title=Remarks,colframe=black,arc=10pt]
	\textbf{R1.} In the general case of the circular motion, angular velocity of the object studied varies over time: $\omega=\omega(t)$.\\
	
	\textbf{R2.} When the direction of the rotation axis changes, the components of unit vector $\vec{n}$ are functions of time. This is the case of a motorcycle wheel in a turn for example.
	\end{tcolorbox}
	If $\mathrm{d}t$ is the time required for this movement, the curvilinear velocity of the point is therefore:
	
	then we fall back on the result already given previously.
	
	Now let us do the same that during our study of the rectilinear uniform movement and let us determine the angular position over time. Then we have:
	
	Therefore:
	
	and if $t_0=0$, then we have:
	
	to be compared with the equivalent relation between position and speed obtained during our study of rectilinear kinematics.
	
	Now let us consider the definition of "\NewTerm{angular acceleration}\index{angular acceleration}" (the traditional notation is a bit unfortunate ...):
	
	Then we have:
	
	and therefore:
	
	 and if $t_0=0$, then we have:
	
	What we can write:
	
	Therefore it comes:
	
	Hence:
	
	and if $t_0=0$, then we have:
	
	compared with the equivalent relation between position, velocity and acceleration obtained in our study of the straight kinematics.

Let us now look to the vector of circular motion which will be extremely important one step further and also in many other chapters. So give us a Euclidean orthonormal such as:
	\begin{figure}[H]
		\centering
		\includegraphics{img/mechanics/kinematics_circular_motion_vector_analysis.jpg}
	\end{figure}
	We see well from the above figure that:
	
	So finally we have:
	
	We see then that we are dealing with a cross product (\SeeChapter{see section Vector Calculus page \pageref{cross product}}) such that:
	
	So we have:
	
	that we also write:
	
	The acceleration of the circular motion is therefore is formed in the general case, of two terms, the first being the "\NewTerm{tangential acceleration}\index{tangential acceleration}" always expressing the variation of the speed along the path and the second acceleration perpendicular along the named "\NewTerm{normal acceleration}\index{normal acceleration}" or "\NewTerm{centripetal acceleration}\index{centripetal acceleration}" (centripetal meaning: "that tends to bring the center"). This is why the normal acceleration is also sometimes denoted by  $\vec{a}_c$.
	
	\begin{tcolorbox}[title=Remark,colframe=black,arc=10pt]
	Following Newton's second law we can associate a force to each of this both acceleration. However there must be no confusion between: "\NewTerm{centrifugal force}\index{centrifugal force}" (Latin for "center fleeing") that describes the tendency of an object following a curved path to fly outwards, away from the center of the curve. It's not really a force; it results from inertia — the tendency of an object to resist any change in its state of rest or motion, and the "\NewTerm{centripetal force}\index{centripetal force}" that is a real force that counteracts the centrifugal force and prevents the object from "flying out," keeping it moving instead with a uniform speed along a circular path.
	\end{tcolorbox}
	\begin{tcolorbox}[colframe=black,colback=white,sharp corners]
	\textbf{{\Large \ding{45}}Example:}\\\\
	The centrifugal "force" on Earth's equator is equal to:
	
	To compare at the same place to Earth's gravitational acceleration that is of $a_g\cong 9.800\;[\text{m}\cdot \text{s}^{-2}]$. The resultant is then:
	
	Interesting result for flat Earther that "believe" that if the Earth was really rotating we should be put in orbit because of the centripetal force....
	\end{tcolorbox}
	The reader interested to vector expressions of the speed and acceleration in cartesian, polar, cylindrical and spherical coordinates should refer to the section of Vector Calculus where all the details ar given where we get for recall:

	\begin{itemize}
		\item In cartesian coordinates:
			\begin{itemize}
				\item Velocity:
								
	
				\item Acceleration:
				
			\end{itemize}

		\item In polar coordinates:
			\begin{itemize}
				\item Velocity: 
				
	
				\item Acceleration: 
				
			\end{itemize}
		
		\item In cylindrical coordinates:
			\begin{itemize}
				\item Velocity:
				
	
				\item Acceleration:
				
			\end{itemize}
			
		\item In spherical coordinates:
			\begin{itemize}
				\item Velocity:
				
	
				\item Acceleration:
				
			\end{itemize}
	\end{itemize}
	For the details of the analysis of the velocity and acceleration of a elliptical motion the reader should refer to the section of Astronomy.
	
	Also from the vis-viva equation proved in the section of Aerospace Engineering we get the speed of a body in any point of an ellipse:
	
	where as proved in the section of Analytical Geometry:
	
	where $a$ is the semi-major axis of the ellipse.
	
	For non-planar and not rectilinear motions (that is to say: the generalization of what we have seen so far). The reader should refer to the section of Differential Geometry.
	
	If we express the circular movement of a point $P$ from a system of axes lying in the plane of the trajectory, for simplicity, then the position of the point $P$ is given parametrically by (\SeeChapter{see section Vector Calculus page \pageref{vector parametric circle equation}}):
	
	This shows that the circular movement can be considered as the superposition of two sinusoidal movements phase-shifted of $\pi/2$. But if we write now more generally we could also have:
	
	Or even more general:
	
	or more general again:
	
	we get by focusing on on the coordinates:
	
	by varying the phase difference $\varphi$ and the ratio $\omega_x/\omega_y$ we get curves that we name "\NewTerm{Lissajous figures}\index{Lissajous figures}":
	\begin{figure}[H]
		\centering
		\includegraphics{img/mechanics/lissajous_figures.jpg}
		\caption{Lissajous figures}
	\end{figure}
	An important case to consider for the study of circular particles accelerators is the acceleration vector of a uniform circular movement. For such a movement, the position vector in time $t$ is obviously given by:	
	
	where $R$ is the radius, $\omega$ the (constant) angular velocity and $\vec{e}_x$,$\vec{e}_y$ the orthonormal unit vectors for $x$ and $y$.

	The acceleration vector being given by:
	
	Carry out the two derivations we find:
	
	Therefore we get for the "\NewTerm{centrifugal acceleration}\index{centrifugal acceleration}\label{centrifugal acceleration}":
	
	So in uniform circular motion the only acceleration vector always points opposite to the position vector, to the center of the trajectory.
	
	The reader will find very important practical applications of the kinematics of the circular motion to for the industry relatively to mechanics in the section of Mechanical Engineering (\SeeChapter{see chapter Engineering}) and also to Astronomy (\SeeChapter{see chapter Cosmology}) and to Corpuscular Quantum Physics (\SeeChapter{see chapter Atomistic}).
	
	\pagebreak
	\subsection{Energy, Work and Power}\label{energy work power}
	When a force acts upon an object to cause a displacement of the object, it is said that work was done upon the object. There are three key ingredients to work - force, displacement, and cause. In order for a force to qualify as having done work on an object, there must be a displacement and the force must cause the displacement. There are several good examples of work that can be observed in everyday life - a horse pulling a plow through the field, a father pushing a grocery cart down the aisle of a grocery store, a freshman lifting a backpack full of books upon her shoulder, a weightlifter lifting a barbell above his head, an Olympian launching the shot-put, etc. In each case described here there is a force exerted upon an object to cause that object to be displaced.
	
	If a mass point $m$ undergoes an elementary displacement $\mathrm{d}\vec{r}$ under the effect of a force $\vec{F}$, this force performs an elementary work $W$ given by definition by the "\NewTerm{work equation}\index{work equation}\label{work equation}":
	
	However the reader must be careful. It's now why a body accelerates (begin to move) that there is necessarily a force!!! Indeed, Einstein has proved that what we have consider for centuries as the "gravitational force" was in fact only a body following the space-time curvature. So never go too quick in conclusion we making science...
	
	On occasion, a force acts upon a moving object to hinder a displacement. Examples might include a car skidding to a stop on a roadway surface or a baseball runner sliding to a stop on the infield dirt. In such instances, the force acts in the direction opposite the objects motion in order to slow it down. The force doesn't cause the displacement but rather hinders it. The negative work refers obviously to the numerical value that results when values of $F$, $d$ and $\theta$ are substituted into the work equation. Since the force vector is directly opposite the displacement vector, theta is $\pi$ degrees. The $\cos(\pi)$ is $-1$ and so a negative value results for the amount of work done upon the object.
	
	\textbf{Definition (\#\mydef):} If the work $W$ is positive the work is named "\NewTerm{engine work}\index{engine work}". If negative we speak then of "\NewTerm{resistive work}\index{resistive work}".
	
	If the mass $m$ moves from a point $A$ to place $B$, the total work is:
	
	For the units we have the express either in "\NewTerm{Joules [J]}\index{Joules}" given by:
	
	The definition above we can also quickly deduce the work of a moment of force (or in other words, the work of a force in a rotational movement) because of an infinitesimal element of movement, we have:
	
	So in the case of circular motion the work of a moment of force will be given by:
	
		
	\subsubsection{Conservative vector field}
	If the force $\vec{F}$ is constant in magnitude and in direction (gravity near the Earth's surface for example), the integral calculation of $W$ takes a simpler form:
	
	This result shows that the work depends only the initial and final positions and not of the path traveled in this special condition. 
	
	But in fact there is a more general result that we have already proved proved in the section of Differential and Integral Calculus that is the "\NewTerm{fundamental theorem for line integrals}\index{fundamental theorem for line integrals}" or "\NewTerm{gradient theorem for line integrals}\index{gradient theorem for line integrals}" that gives us:
	
	where $\vec{\nabla}()$ is the gradient operator (\SeeChapter{see section Vector Calculus page \pageref{gradient scalar field}}).
	
	The force of gravity, of electrostatic force is a special case of this type but in fact in all situations as it derives from a potential (see details further below)!
	
	A "\NewTerm{conservative vector field}\index{conservative vector field}" (also named a "\NewTerm{path-independent vector field}\index{path-independent vector field}") is a vector field $\vec{F}$ whose line integral $\oint_\Gamma \vec{F}\circ\mathrm{d}\vec{r}$ over any curve $\Gamma$ depends only on the endpoints of $\Gamma$. Therefore a vector field for which we have:
	
	is by definition automatically a conservative vector field and then we say that there is "no circulation around a closed curve".
	
	We have also proved in the section of Vector Calculus the Stokes theorem that gives for recall a simply connected domain, a continuously differentiable vector field $\vec{F}$ :
	
	Therefore for a conservative vector field:
	
	So a vector field that is continuously differentiable on a simply connected domain such that:
	
	is also a possible criteria of identification of a conservative vector field.
	
	\begin{tcolorbox}[colframe=black,colback=white,sharp corners]
	\textbf{{\Large \ding{45}}Example:}\\\\
	Our goal is to determine if the vector vield
	
	is conservative or not.

	One condition for path independence is as we know the following: for a simply connected domain, a continuously differentiable vector field $\vec{f}$ is path-independent if and only if its curl is zero.
	
	Then we have:
	
	We calculate:
	
	Since these partial derivatives are equal, the curl is zero.\\

	Can we conclude $\vec{F}$ is conservative? The problem is that $\vec{F}$ is not defined at the origin $(0,0)$. Its domain of definition has a hole in it, which for two-dimensional regions, is enough to prevent it from being simply connected. The test does not apply, and we still don't know whether or not $\vec{F}$ is conservative.\\
	
	Let's try another test, this time a test for path-dependence. If we can find a closed curve along which the integral of $\vec{F}$ is nonzero, then we can conclude that $\vec{F}$ is path-dependent. If the curve does not go around the origin, then we can use Green's theorem (\SeeChapter{see section Vector Calculus page \pageref{green theorem}}) as we have only the third component of the curl that was non-null above:
	
	to show the integral of $\vec{F}$ is zero or not using path integral rather than directly the curl.\\
	
	We choose for closed path a circle and then we can use a reparametrization of the field using $x= \cos(t)$ and $y=\sin(t)$ with $0\leq t\leq 2\pi$. Then on the unit circle, $\vec{F}$ takes a simple form:
	\end{tcolorbox}
	
	\begin{tcolorbox}[colframe=black,colback=white,sharp corners]
	
	and we have obviously:
	
	Therefore:
	
	The hole in the domain at the origin did end up causing trouble. We found a curve $\Gamma$ where the circulation around $\Gamma$ is not zero. The vector field $\vec{F}$ is therefore path-dependent.
	\end{tcolorbox}
	The relation:
	
	is sometimes named "\NewTerm{theorem of kinetic energy}\index{theorem of kinetic energy}" or "\NewTerm{work-energy theorem}\index{work-energy theorem}".

	Note that in the case of a rectilinear movement, to which a mobile initially at rest travels a distance $d$ distance being subjected to a certain force, we (omitting course the change in potential energy and energy loss by friction):
	
	that we name "\NewTerm{working force}\index{working force}".
	
	\subsubsection{Kinetic Energy}\label{kinetic energy}
	Newton's second law $\vec{F}=m\vec{a}$ applies along a path $A$-$B$. By using the expression of the work we have:
	
	and by developing the dot product using the components we have:
	
	By definition we say that each term in the last equality is the (non-relativistic) "\NewTerm{kinetic energy}\index{kinetic energy}" denoted by:
	
	and it is measured in obviously also in Joules [J] (or other exotic derivate units whose theoretical physicists sometimes abuse a little bit too much...) and is always positive in mechanics or any other field of physics.

	The previous development can then be rewritten as:
	
	Therefore when a body moves (without rotation on itself) under the action of any force $\vec{F}$, the work of this acceleration force on any path $A$, $B$ is equal to the change in kinetic energy of the body.
	
	\paragraph{Moment of inertia}\mbox{}\\\\
	The "\NewTerm{moment of inertia}\index{moment of inertia}\label{moment of inertia}", otherwise known as the "\NewTerm{angular mass}\index{angular mass}" or "\NewTerm{rotational inertia}\index{rotational inertia}", of a rigid body determines the torque needed for a desired angular acceleration about a rotational axis. It depends on the body's mass distribution and the axis chosen, with larger moments requiring more torque to change the body's rotation. 

	For bodies constrained to rotate in a plane, we will see that is sufficient to consider their moment of inertia about an axis perpendicular to the plane. For bodies free to rotate in three dimensions, their moments must be described by a symmetric $3\times 3$ matrix; each body has a set of mutually perpendicular principal axes for which this matrix is diagonal and torques around the axes act independently of each other.
	
	To a rigid solid rotating about an axis at the angular velocity $\vec{\omega}$, the elementary kinetic energy of any point of mass $\mathrm{d}m$, located off-axis, is:
	
	since $\vec{\omega}$ and $\vec{r}$ are perpendicular. The total kinetic energy is then:
	
	We have become accustomed in physics to writh this last relation:
	
	with the "\NewTerm{moment of inertia}\index{moment of inertia}" defined by:
	
	\begin{tcolorbox}[colframe=black,colback=white,sharp corners]
	\textbf{{\Large \ding{45}}Examples:}\\\\
	E1. Let us calculate the final velocity of a rigid ball falling on an incline rigid plane with nonzero friction (therefore the ball is going to rotate) in a gravitational potential field where the friction heat can be neglected and in vacuum.\\

	The answersofthe beginner will often be obtained by considering only the kinetic energy but not the rotating speed of the ball. Now we must take this into account through its moment of inertia.\\

	So we have the total kinetic energy that the translation kinetic energy of the center of mass plus the rotational energy around the same center of mass:
	
	\begin{figure}[H]
		\centering
		\includegraphics{img/mechanics/rotating_ball_inclined_plane.jpg}
	\end{figure}
	equaling that value with that of the gravitational potential energy (see further below), assuming a zero initial velocity of fall, we have:
	
	Thus we get the velocity acquired at the bottom of the plane (rolling friction not included...):
	
	and we have proved in the section of Geometric Shapes that the moment of inertia of a solid ball around its center of mass is:
	\end{tcolorbox}
	
	\begin{tcolorbox}[colframe=black,colback=white,sharp corners]
	
	Therefore it comes:
	
	We see in the special case of the ball, that final speed of drop is (without air friction and without energy loss due to friction) independent of its mass and radius and of then angle of the plane (whether the ball is empty or full) which is relatively counter-intuitive.\\
	
	E2. A second famous example is the calculation of the rotational kinetic energy of a perfectly spherical planet of homogeneous mass and constant rotational period. Then we have:
	
	\end{tcolorbox}
	In a solid, the distribution of the material about an axis will obviously be different depending on the chosen axis. The moment of inertia corresponding will also be obviously different. It is therefore essential to specify the axis with respect to which we wish to determine the moment of inertia. We see that in practice that engineers often place the axis so that it passes through the center of mass. In the tables, we frequently find expressions of inertia moments of common forms (along a given axis) such as cylinder, cone, sphere, bar, tube.
	
	Here is a summary of some moment of inertia of plane figures among all those proved in detail in the section of Geometric Shapes:
	\begin{figure}[H]
		\centering
		\includegraphics{img/mechanics/moment_inertia_plane_shapes.jpg}
	\end{figure}
	and same for some body (volumic) shapes (proofs are also available in the same section):
	\begin{figure}[H]
		\centering
		\includegraphics{img/mechanics/moment_inertia_body_shapes.jpg}
	\end{figure}
	We saw in our study of angular momentum that:
	
	and just before that the moment of inertia is given by:
	
	So we have:
	
	hence:
	
	We finally get:
	
	it is the expression of the angular momentum of a rigid body rotating on itself (on one of its possible axes of rotation).
	
	As we showed during our study of the angular momentum:
	
	then it comes under the assumption that the mass and geometry of the solid remain constant ... that the moment of force is then:
	
	And obviously, if we study a system in which the angular momentum is conservative, it is obvious that:
	
	This conservation of angular momentum is applicable in a multitude of experiences as the well known which is to turn on a chair and spread out hands or legs which will decrease the speed (and vice versa) as illustrated in figure below:
	\begin{figure}[H]
		\centering
		\includegraphics{img/mechanics/angular_momentum_moment_of_inertia_conservation_1.jpg}
		\caption{Conservation of the angular momentum on variation of inertia momentum}
	\end{figure}
	Another curious experience (but mathematically correct) to take place ourselves on a turntable with a wheel in rotation held horizontally (vertical angular momentum is zero) and put it then vertically (same type of experiment can be done in the opposite direction). As the vertical momentum must remain zero to counteract this (or must remain constant in direction and in amplitude as it is a vectorial quantity), the plateau on which is put the experimenter will rotate in the opposite direction of rotation of the wheel.
	\begin{figure}[H]
		\centering
		\includegraphics{img/mechanics/angular_momentum_moment_of_inertia_conservation_2.jpg}
		\caption{Conservation of total angular momentum by variation of intern angular momentum}
	\end{figure}
	\begin{tcolorbox}[title=Remarks,colframe=black,arc=10pt]
	\textbf{R1.} The movement of large masses on the surface of the Earth (icebergs, river flooding, tectonic plates, etc.) cause changes in moment of inertia of the Earth. It follows therefore fluctuations of the angular velocity and therefore imperfection of astronomical time standard (a few thousandths per day).\\
	
	\textbf{R2.} A "reaction wheel" (or "momentum wheel") is a type of flywheel used primarily by spacecraft for attitude control without using fuel for rockets or other reaction devices. They are particularly useful when the spacecraft must be rotated by very small amounts, such as keeping a telescope pointed at a star. They may also reduce the mass fraction needed for fuel. This is accomplished by equipping the spacecraft with an electric motor attached to a flywheel which, when its rotation speed is changed, causes the spacecraft to begin to counter-rotate proportionately through conservation of angular momentum. Reaction wheels can rotate a spacecraft only around its center of mass; they are not capable of moving the spacecraft from one place to another.
	\begin{figure}[H]
		\centering
		\includegraphics[scale=0.2]{img/mechanics/kepler_reaction_wheels.jpg}
		\caption{Kepler satellite reaction wheels}
	\end{figure}
	\end{tcolorbox}
	\begin{tcolorbox}[colframe=black,arc=10pt]
	\textbf{R3.} A "control moment gyroscope" is a related but different type of attitude actuator When mounted to a rigid spacecraft, applying a constant torque to the wheel using one of the gimbal motors causes the spacecraft to develop a constant angular velocity about a perpendicular axis, thus allowing control of the spacecraft's pointing direction. CMGs are generally able to produce larger sustained torques than reaction wheel with less motor heating, and are preferentially used in larger and/or more-agile spacecraft, including Skylab, Mir, and the International Space Station (ISS).
	\begin{figure}[H]
		\centering
		\includegraphics[scale=0.2]{img/mechanics/control_moment_gyroscope.jpg}
		\caption[ISS Control Moment Gyroscope]{One of the 4, $122$ [cm] diameter, ISS Control Moment Gyroscope (source: NASA)}
	\end{figure}
	\end{tcolorbox}
	Now let us come back to the methods of calculation of the moments of inertia. The kinetic energy of a body is the sum of kinetic energy of each element of this body at the macroscopic level, then we have:
	
	Therefore in the context of a solid rigid body rotating around an axis, we have\label{sums of moments of inertia}
	
	Thus, for a body composed of a plurality of different geometries, the total moment of inertia is the sum of the moments of inertia with respect to the axis of rotation of each of the sub-geometries such that:
	
	When we calculate the moment of a body of inertia relative to a given axis $\Delta$ of rotation, it may be interesting to know what is the distance from the axis where we can place fictitiously the whole mass of this body concentrated on a given point for the same moment of inertia. By definition, this distance denoted $k$ and named the "\NewTerm{radius of gyration}\index{radius of gyration}\label{radius of gyration}" is trivially given by:
	
	where $J_\Delta$ is the known moment of inertia of the body of mass $M$ relatively to an axis $\Delta$.
	
	By definition, the "\NewTerm{polar moment of inertia}\index{polar moment of inertia}\label{polar moment of inertia}" (or also "\NewTerm{squared moment of inertia}\index{squared moment of inertia}") is the moment of inertia defined with respect to a point (a "pole") and not relatively to an axis and written:
	
	This quantity intervene only for free spins and is of interest, for rotations around a fixed axis, only because it sometimes facilitates the calculation of the axial moments of inertia under the follows relation (in Cartesian coordinates):
	
	\begin{dem}
	\begin{lemma}
	The moment of inertia compared to the $x$O$y$ plane is trivially given by:
	
	\begin{figure}[H]
		\centering
		\includegraphics{img/mechanics/inertia_polar_momentum_lemma1.jpg}
	\end{figure}
	\end{lemma}
	\begin{lemma}
	The moment of inertia with respect to an axis is given by:
	
	\begin{figure}[H]
		\centering
		\includegraphics{img/mechanics/inertia_polar_momentum_lemma2.jpg}
	\end{figure}
	\end{lemma}
	Summing these relations, we deduce:
	
	The polar moment of inertia is then given by:
	
	\begin{figure}[H]
		\centering
		\includegraphics{img/mechanics/inertia_polar_momentum_proof.jpg}
	\end{figure}
	Comparing with Lemma 2 we get:
	
	\begin{flushright}
		$\square$  Q.E.D.
	\end{flushright}
	\end{dem}
	If the body in question has a spherical symmetry, it comes immediately as $J_x=J_y=J_z:J$ that:
	
	An example is given with the ball (full sphere) in the section on Geometric Shapes in the chapter Geometry.
	
	If the body in question has a spherical symmetry, it comes immediately as $J_x=J_y=J_z:=J$ that:
	
	An example is given with the ball (full sphere) in the section on Geometric Shapes in the chapter Geometry.
	
	Suppose now that we know the moment of inertia $J_z$ of a rigid solid body compared to any other perpendicular axis (the chosen axis is not necessarily only assimilated to the common $z$ axis) passing through the center of mass $G$. Then let us calculate the moment of inertia $J_z$, relatively to another axis $z'$, parallel to  $z$ and remote from distance $d$, and let us make up appear the existing link between these two different moments of inertia.
	
	So consider the following figure representing our subject of interest:
	\begin{figure}[H]
		\centering
		\includegraphics[scale=0.6]{img/mechanics/parallel_axis_theorem_huyghens_steiner.jpg}
	\end{figure}
	 In a planar Cartesian reference frame, we have for any points $(x, y)$:
   
   and:
   
   Then we have:
   
   and:
   
   The therm:
    
   is zero because if the moment of inertia is calculated relatively to the center mass $G$ as we have assumed it from the beginning, then:
   
   Ultimately, we finally get the "\NewTerm{Huygens-Steiner theorem}\index{Huygens-Steiner theorem}\index{Steiner's theorem}\label{steiner theorem}" also named "\NewTerm{parallel axis theorem}\index{parallel axis theorem}":
   
    As we will see it in the section on Geometric Shapes in the chapter Geometry, it then becomes easy to calculate the moment of inertia of an equilateral triangle knowing that of a square plate and moving the axis inertia to the point where the triangle is the center of gravity (ie at the third of the median between the center of the rectangle and of its vertices).

   Since there are as many moments of inertia as axes of rotation and that they are often in practical study cases related to the principal axes of inertia (axis related to the axes of revolution or symmetry planes - see below), it may be useful to introduce a mathematical being useful in the context of representation of moments of inertia which is none other than the "\NewTerm{inertia matrix}\index{inertia matrix}\label{inertia matrix}" or also known as (modern formulation) "\NewTerm{inertia tensor}\index{inertia tensor}\label{inertia tensor}".
   
   The approach to rigorously determine the expression of this tensor is the following: 

   Given $A_i$ a given point of a solid, for which we seek to calculate the moment of inertia and $\Delta$ the axis of origina O and vector of unit vector $\vec{u}$ relative to which we wish to calculate the moment of inertia. Any point $A_i$ of the solid can be projected (orthogonal projection) on a point $H\in \Delta$ from the knowledge of the angle $\theta$ between $\Delta$ and $\overrightarrow{\text{O}A_i}$ such that:
    
   Therefore:
   Therefore:
   
   Based on the properties of the mixed product (\SeeChapter{see section Vector Calculus page \pageref{mixed product}}) and of the scalar product:
   
   we have:
   
   and therefore:
  
  As $\vec{u}$ is a vector of constant direction whatever the point of integration, we can take it out of the integral such that:
   
   We can check that if we replace $\vec{u}$ by $\vec{u}_1+\vec{u}_2$, we will get a result that also sum (of the type $\vec{v}=\vec{v_1}+\vec{v}_2$ that is the sum of two integrals) up by the property of linearity  of the cross product (\SeeChapter{see section Vector Calculus page \pageref{cross product linearity}}). 

   Therefore, the application that to $\vec{u}$ associated a given $\vec{v}$ is therefore a linear application that can be represented in a given basis $\mathcal{B}$ by a matrix:
   
   The matrix $[J_0]_{\mathcal{B}}$ is named "\NewTerm{inertia tensor}\index{inertia tensor}" of the system relatively to the point O, in the base $\mathcal{B}$.

  The moment of inertia of a system with respect to any axis $\Delta$ of unit vector $\vec{u}$ is given by:
  
  The problem now is to calculate the elements of the tensor $[J_0]$ for a given basis $\mathcal{B}$. Given a reference frame $(\text{O},\vec{i},\vec{j},\vec{k})$ such that $\text{O} \in \Delta$ we put:
   
   and:
    
   Using the fact that we know that a cross product can be represented by as an assymetric matrix operation (\SeeChapter{see section Vector Calculus page \pageref{cross product matrix form}}):
   
   and therefore:
   
	In the above expression of the inertia matrix, we recognize the diagonal elements: it is simply the moments of inertia of the system relatively to the different axes of the base. We name "\NewTerm{product of inertia}\index{product of inertia}" the non-diagonal elements of the matrix and we write them:
   
   Therefore we have:
   
    where obviously:
   \begin{itemize}
       \item $J_{\text{O}x}$ is the inertial momentum relatively to $(\text{O},\vec{x})$
        \item $J_{\text{O}y}$ is the inertial momentum relatively to $(\text{O},\vec{y})$
        \item $J_{\text{O}z}$ is the inertial momentum relatively to $(\text{O},\vec{z})$
        \item $J_{yz}$ is the inertial momentum relatively to $(\vec{y},\vec{z})$
        \item $J_{xz}$ is the inertial momentum relatively to $(\vec{x},\vec{z})$
        \item $J_{xy}$ is the inertial momentum relatively to $(\vec{x},\vec{y})$
   \end{itemize}
   Also sometimes written as (USA notation and mechanical engineers notation):
   
   If O is assimilated to the center of mass of the solid, we simply write:
   
   It follows if thickness of the study body is negligible compared to the other dimensions then $z=0$ and then we have immediately:
   
   We can also generalize Huygens theorem by making use this matrix. To do so, let us write $(x', y', z')$ the coordinates of any point $A$ in $R'$ and $(x, y, z)$ its coordinates in $R$. We write $(a, b, c)$ the coordinates of the O' of $R'$ in $R$:
	
   as:
   
   We then have:
   
   Now if O' coincides with the center of mass $G$, then according to the definition of the center of mass:
   
   We deduce then:
   
   and also:
   
   with:
   
   We fall back on the classic Huygens theorem since $a^2+b^2$ is none other than the squared distance between the axis O$z$ and $Gz$ and same for $b^2+c^2$ that is the squared distance between O$x$ and $Gx$ and $a^2+c^2$ which is the distance between O$y$ and G$y$.
   
   If we now focus on the products of inertia, we get:
   
   Therefore, if O' coincide with $G$:
   
   In summary, the "\NewTerm{Huygens generalized theorem}\index{Huygens generalized theorem}" is written:
   
   The inertia tensor being real and symmetric, we have proved in the section of Linear Algebra (spectral theorem) that it is always possible to find three perpendicular directions of vectors $\vec{e}_1,\vec{e}_2,\vec{e}_3$ such as the symmetrical  tensor (matrix) is diagonalisable:
   
   The trihedron formed by the vector $\vec{e}_1,\vec{e}_2,\vec{e}_3$ is named "\NewTerm{main trihedron of inertia}\index{main trihedron of inertia}" and its axes are named "\NewTerm{principal axes of inertia}\index{principal axes of inertia}". In this reference frame $[J_0]$ takes the name of "\NewTerm{principal inertia tensor}\index{principal inertia tensor}". If furthermore O is assimilated to $G$, we speak of "\NewTerm{central tensor of inertia}\index{central tensor of inertia}".
   
   In fact, to find the moments of inertia relatively to the principal axes it is almost never necessary to diagonalise the inertia tensor because it is often enough to be guided by the symmetry of the system. We'll see the following theorems that if there are axes or planes of symmetry for the mass distribution, the axes of inertia are easy to find. In addition, the system is generally simple enough (or sufficiently broken down into simple elements ...) for these axes to be obvious.
    \begin{theorem}
    If the system has a material plane of symmetry (ie: $\rho(A)=\rho(A')$ if $A$ is the symmetric of $A'$ relatively to the plane) then any axis perpendicular to this plane is a principal axis of inertia and therefore the inertia matrix reduce to:
    
    \end{theorem}
    \begin{dem}
    Let us choose a reference frame $x\text{O}y$ in the plane relatively to which the system has a mass distribution that is symmetrice and O$z$ an axis perpendicular to this plane. To calculate $\int xz\mathrm{d}m$ or $\int yz\mathrm{d}m$ let us group the points by pairs, symmetrical to $x\text{O}y$. that is to say such that $z_A=-z_{A'}$. We then have:
     
      and even:
      
      that is to say, as (material symmetry!):
      
     that all contributions of symmetrical paires of points are zero, which implies:
    
    that is to say, the $z$ axis is a main direction of inertia.
    \begin{flushright}
		$\square$  Q.E.D.
	\end{flushright}
    \end{dem}
    \begin{theorem}
    Let us Choose O$z$ as axis of symmetry. As above, we have:
    
    \end{theorem}
    \begin{dem}
    Indeed, because if we group the points by pairs $A$ and $A'$ symmetrical to O$z$, we have:
    
    but as $z_A=z_{A'}$ we have always:
    
     and even:
     
    \begin{flushright}
		$\square$  Q.E.D.
	\end{flushright} 
    \end{dem}
    \begin{tcolorbox}[title=Remark,colframe=black,arc=10pt]
	When we have identified two principal axes of inertia through previous symmetries, the third is simply the necessary one to complete an orthogonal (trihedron) axis system.
	\end{tcolorbox}
    \begin{theorem}
    If a system admits an axis of revolution for its mass distribution, then all orthogonal trihedron including the axis of revolution, is the principal trihedron of  inertia. The material system is then said to be a "\NewTerm{cylindrical system}\index{cylindrical system}" and inside the principal inertial trihedron the inertial tensor takes the form (assuming that the axis of rotation is the third axis of the coordinate system):
    
    \end{theorem}
    \begin{dem}
	If O$z$ is an axis of revolution, any plane including O$z$ is a plane of symmetry and line perpendicular to O$z$ is principal axis of inertia (first theorem). In addition, all these lines perpendicular to O$z$ are equivalent.
	\begin{flushright}
		$\square$  Q.E.D.
	\end{flushright} 
    \end{dem}
    
    \textbf{Definition (\#\mydef):} If the matrix of inertia O of a material system is of the type:
    
    then we say that the system is a "\NewTerm{spherical system}\index{spherical system}" (or a "\NewTerm{spherically symmetric system}\index{spherically symmetric system}").
    \begin{tcolorbox}[title=Remark,colframe=black,arc=10pt]
	The systematic choice of a main trihedron of inertia gives the possibility to reduce the inertia tensor of $6$ to $3$ components only, calculated once and for all. However, this choice involves the use of a base that will most often be in motion relative to the repository used, which may pose problems derivations with respect to time of the basis vectors. We can then, if it's easier, obtain the components of the tensor of symmetry in any base with a matrix of bases change between the main base and the base used for the calculation of the main trihedron of inertia.
	\end{tcolorbox}
	\begin{theorem}
    When the moments of inertia of a solid are known in the directions of all principal axes of inertia, we can easily determine the moment of inertia $J$ with respect to any other axis through the center of gravity using what we name an "\NewTerm{ellipsoid of inertia}\index{ellipsoid of inertia}" (not to be confused with the moment of inertia of an ellipsoid - proved in the section Geometric Shapes).
    \end{theorem}
    \begin{dem}
    Given three axes, centered on $G$, parallel to the main axes. In their directions, let us chose lengths proportional to:
    
    
    In this phase space of moments of inertia, any point $P=(x,y,z)$ denotes a moment of inertia $J$ such that:
    
    To determine $J$ based on the $J_x,J_y,J_z$, without having to calculate $x, y, z$, we identify the cosine directions (\SeeChapter{see section Vector Calculus page \pageref{cosines directions}}) of the axis of rotation to those of the straight line $\overline{GP}$:
    \begin{figure}[H]
		\centering
		\includegraphics{img/mechanics/ellipsoid_of_inertia.jpg}
		\caption{Illustration of the ellipsoid of inertia}
	\end{figure}
    Thus we have:
    
    Therefore:
    
    Therefore:
    
    We can now calculate the conditions of normalization of this relation. Thus, if $y=z=0$ and $x=a$, we have:
    
    Respectively we will have:
    
    As:
      
	\begin{flushright}
		$\square$  Q.E.D.
	\end{flushright}
    \end{dem}
    Which brings us to write:
    
     By substitution, we get:
     
    So finally:
    
    Thus, knowing the moments of inertia of a body with respect to its principal axis $J_x,J_y,J_z$, we can know its moment of inertia relatively to any axis having an angle $\alpha,\beta,\gamma$ relatively to the main axes.
    
     \paragraph{Gyroscope}\label{gyroscope}\mbox{}\\\\
     Ok now that we know quite well what is the angular momentum, the moment of inertia and the Lagrangian (\SeeChapter{see section Analytical Mechanics page \pageref{lagrangian formalism}}) we can now study one of the most fascinating object of macroscopic physics that combines all these concepts!!!
     
     \textbf{Definition (\#\mydef):} A solid (solid of revolution for simplicity...) which can move freely around a fixed point and quickly turning on itself by definition is a "gyroscope".

	Out of their recreational use... because they allow configurations considered as pedagogically exceptional... gyroscopes are an important part of inertial navigation systems (before the advent of GPS...) in aviation, aerospace, marine (boat stabilization), movie/television (stabilization of cameras), weapons (ballistic missiles),  tunnel mining and many more. The guide instruments through inertia of these systems consist of gyroscopes and accelerometers, which calculate at any time the exact speed and direction of the system in motion (depending on the movement of the gyroscope a voltage variation is caused) . The collected signals are transmitted to a computer that records and then corrects any aberrations of the path.

	\begin{tcolorbox}[title=Remark,colframe=black,arc=10pt]
	Gyroscopes based on other operating principles also exist, such as the electronic, microchip-packaged MEMS gyroscopes found in consumer electronics devices, solid-state ring lasers, fibre optic gyroscopes, and the extremely sensitive quantum gyroscope.
	\end{tcolorbox}
	\begin{figure}[H]
		\centering
		\includegraphics[scale=0.25]{img/mechanics/gyroscope.jpg}
		\caption{3D Gyroscope}
	\end{figure}
	The planets are another example of famous gyroscopes. The best known example being our Earth rotating relatively quickly around itself and being very massive it's angular momentum makes that its north pole is always (at least in the time scale of a human being...) facing the North star regardless of its position in its orbit.
	
	We will study the gyroscope in two steps:
	\begin{enumerate}
         \item The classical approach (without using Lagrangian mechanics) that is easy to understand but that gives only the possibility to explain the precession angular velocity $\Omega$ of the gyroscope.

         \item The Lagrangian approach that is more abstract but that opens the door to the understanding of the precession AND the nutation rotation of the gyroscope.
     \end{enumerate}

	\pagebreak
	\subparagraph{Classical Approach with precession only}\mbox{}\\\\
	The following figure is an example of gyroscope known in laboratories schools named "\NewTerm{symmetric weighing gyroscope}\index{symmetric weighing gyroscope}". This is obviously a special and simplified case but for understanding the basic principle of the gyroscope:
     \begin{figure}[H]
		\centering
		\includegraphics{img/mechanics/gyroscope_symmetric_weighing.jpg}
		\caption{Symmetric weighing gyroscope}
	\end{figure}
	Consisting of an electric motor whose rotor, the main wheel, form the main mass in rapid angular rotation (a gyroscope of a V2 rocket turn at over $10,000$ rpm). The motor stator is fixed to a rod on which is positioned a counterweight on the opposite side. The assembly is placed on a support on the end of which is a gimbal mounted on a horizontal bearing which allows the gyroscope orientation almost without limitations in all directions.
	
	In this figure we have $\omega$ which is the instantaneous angular velocity of the removable disk of radius $R$, $\Omega$ is the rate of precession of the gyroscope (rotation around the support), $F$ is the force of the additional mass $m$ attached to the counterweight and that imbalance the gyroscope, $r$ is the distance of the gyroscope to the counterweight and finally $\alpha$ is the tilt angle that takes the axis of the gyroscope when we do an imbalance by attaching an additional weight to the counterweight.
	
	To begin the theoretical study of this system, remember that have proved above that the angular momentum for a solid having a moment of inertia $J$ is expressed by the following relation:
     
      and we saw that all solid in rotation around any axis also has also a momentum that is then customary to write according to what we have seen above:
     
      We also proved above that the rotor, like any rapid mass  in rotation, when started or regulated then produced a moment of force given by:
     
     which is vectorially collinear to $\vec{\omega}$ and therefore passes through the axis of symmetry of the gyroscope. As we already know, it is this latter relation that puts the best evidence that the gyroscope always keeps the same direction in space even when we move its support.

    In other words a free gyroscope driven at a high speed of rotation has for fundamental property to maintain its axis of rotation in a fixed orientation with respect to absolute space. This is what we named the "\NewTerm{first gyroscopic law}\index{first gyroscopic law}" or "\NewTerm{law of fixity}\index{law of fixity}".

    Typically the "\NewTerm{Foucault gyroscope}\index{Foucault gyroscope}" shown below, is an excellent practical example of the fixity law, it maintains its orientation regardless of how we handle the support on which it is build:
     \begin{figure}[H]
		\centering
		\includegraphics{img/mechanics/gyroscope_foucault.jpg}
		\caption{Foucault gyroscope}
	\end{figure}
	If we put the gyroscope a whole day on a table with an engine that maintains the rotation of the central massive disk constant , then we observe the rotation of the Earth because the gyroscope then rotates very slowly on itself in 24 hours!

    To return to our mathematical considerations ... now let's look at the force momentum of the counterweight that imbalance our symmetrical gyroscope while the disc is rotating, and generates a general rotation of the gyroscope as experience reveals. We then have for the force momentum making rotating the gyroscope about its axis (support rod):
    
     Since the gyroscope does not precess when the system is balanced this mean that the moment of force of the extra weight that imbalance the gyroscope generates an angular momentum according to the relation proved earlier above such that:
     
     What schematically can be represented as follows (this is our gyroscope seen from above):
     \begin{figure}[H]
		\centering
		\includegraphics{img/mechanics/gyroscope_symmetric_weighing_from_above.jpg}
	\end{figure}
	We then have the gyroscope will rotate (in a circular motion):
     
     Taking the Taylor approximation (\SeeChapter{see section Sequences and Series page \pageref{taylor series}}) to the first order of the tangent for small angles we get:
    
    Let us assume, to simplify the study of the problem, that the variation of total angular momentum with respect to the rotation axis of the gyroscope (that is to say the support rod!) can be assimilated to the angular momentum of the rotor only if that latter is turning fast enough and that its mass is large enough. Which means:
    
    Then we have:
    
	and therefore because of this approximation by entire force momentum is assigned to the change in angular momentum of the rotor itself:
    
	It finally comes for the precession angular velocity:
	
	Therefore when the symmetric weighted gyroscope is balanced (when $M$ is zero in the numerator of the fraction or if you prefer: $\sin(\alpha)=0$), its angular momentum therefore maintains a fixed orientation regardless of the value of the denominator since its precession $\Omega$ will be always equal to zero.

	\textbf{Definition (\#\mydef):} The rotational movement resulting from a non-equilibrated gyroscope is named "\NewTerm{precession movement}\index{precession movement}" when it is caused intentionally, and "\NewTerm{drift}\index{drift}" when it is due to a disruptive element.

	Let us finally indicate the playful gyroscopes for small children as the spinner below:
	\begin{figure}[H]
		\centering
		\includegraphics{img/mechanics/gyroscope_spinner.jpg}
	\end{figure}
	that we can roughly represent as following (side view and respectively top view) to make still some classical mathematical analysis (do the test with your children to see if it interests them as much as the toy...):
	\begin{figure}[H]
		\centering
		\includegraphics{img/mechanics/gyroscope_spinner_schema.jpg}
		\caption{Gyroscope spinner}
	\end{figure}
	where we do the hypothesis that the end of the axis of the spinning top (other name for "gyroscope spinner") is placed on the ground without possibility of slippage and that it has a constant angular velocity $\omega$ and that the latter is big enough to not have its ange $\theta$ which varies in time (otherwise another effect appear named "nutation" that we can only introduce robustly using Lagrangian mechanics).

	Using the same technique as for the symmetric weighted gyroscope we have (well... we could have also used simply the relation $L=\alpha R$ proved in the section Trigonometry...):
	
	We also for the moment of force:
	
	But again, the angular momentum change! Indeed, so we in this particular case:
	
	it follows that under the same assumptions as the symmetric weighted gyroscope that:
	
	hence in vector form:
	
	and we know that this last relation (proved made earlier above) can be completed by writing:
	
	It comes then:
	
	Hence:
	
	We see that the difference with the symmetric weighted gyroscope is that the speed of precession is then independent of the angle!
	\begin{tcolorbox}[title=Remarks,colframe=black,arc=10pt]
	\textbf{R1.} A cyclist traveling in a straight line is stabilized (fixity law requirement!) by the angular momentum of its wheels perpendicular to the rolling direction (the wheels behaves as a gyroscope).\\
	
	\textbf{R2.} Probably without realizing it, we tip up while biking in a curve to produce a precession in the wheels and turn more easily. Indeed the precession movement rotates the bicycle wheel in the direction you look without the need to turn the handlebars... !
	\end{tcolorbox}
	
	\pagebreak
	\subparagraph{Lagrangian Approach with precession and nutation}\mbox{}\\\\
	Symmetric spinning top is a case studied by Lagrange in 1788. After 110 years, in 1897-1898 Klein and Sommerfeld published two volumes, whereas in 1897 Klein separately published a shorter monograph on the mathematical theory of the top. In the eastern world literature, one of the oldest papers is probably due to Appel’rot (1894). Explicit integration of motion equations, to give the nutation in terms of elliptic integrals was given at the end of nineteenth century. 

	The analysis of the symmetrical top (spinning top) is a fascinating topic in classical mechanics. It is a simple system that exhibits counter-intuitive balancing behavior. It's famous "gravity-defying" behavior to remain perched on its pointed is well known (a simple internet search reveals many references concerning the spinning tops and gyroscopes as if they were "magical" instruments that defy gravity). The following developments first explores the Lagrangian formulation of the symmetrical top in a uniform force-field (e.g. gravity). After establishing a few constraints, a dynamic model is presented. From this model, precession rates and nutation behaviour are deduced!
	\begin{figure}[H]
		\centering
		\includegraphics{img/mechanics/gyroscope_symmetrical_top.jpg}
		\caption[]{Symmetrical top gyroscope in equilibrium}
	\end{figure}
	The geometry of the symmetric gyroscopic top in a gravitational field consists of a "heavy" wheel on a narrow stem whose bottom point is fixed to the origin, but is free to rotate (as show in the figure below):
	\begin{figure}[H]
		\centering
		\includegraphics[scale=0.85]{img/mechanics/gyroscope_spinning_top_precession_only.jpg}
		\caption[Spinning top in precession only]{Spinning top in precession only (source: ?)}
	\end{figure}

	The $z'$ axis passes through the center of rotation of the top. The motion of the top can be expressed in terms of the so-named Eulerian angles $(\phi,\theta,\psi)$. The variable $\theta$ is the angle that the $z'$ axis makes with the stationary $z$ axis. The variable $\phi$ is the angle that is formed by the projection of the $z'$ axis in the $x-y$ plane and the $x$ axis. Counterclockwise rotation yields positive angles. The variable $\psi$  is the angular position around the $z'$ axis. Since our top is circularly symmetric, the choice of origin for is arbitrary. We will only be concerned with the rate of angular rotation of the wheel about $z'$, given by $\dot{\psi}$ ($\omega=\dot{\psi}$ on the above figure). Other dynamical variables of interest are the time derivatives of $\phi$ and $\theta$. These variables $\dot{\phi}$ and $\dot{\theta}$ describe the precession and nutation of the top. The length of the stem is $R_G$. We assume the mass of the stem is negligible with respect to that of the wheel $m$. The acceleration due to uniform gravity is $g$ and is assumed to point downwards along the $z$ axis:
	\begin{figure}[H]
		\centering
		\includegraphics{img/mechanics/gyroscope_spinning_top_schema.jpg}
		\caption[]{Schematic view of spinning top}
	\end{figure}
	The most convenient way to derive the equations of motion for this system is to use the Lagrangian of Hamiltonian formalism as we have already mention it (\SeeChapter{see section Analytical Mechanics page \pageref{lagrangian formalism}}). 
	
	First let us recall that in the section of Euclidean Geometry we have proved the following group of relations:
	
	named "\NewTerm{kinematic Euler equations}\index{kinematic Euler equations}" and is  useful to express the rotation speed (around any axis of rotation) of a body in its own reference frame (most of time its center of mass).

	Let us choose O as origin for our reference frame  and we will choose the principal axes of inertia as the body reference frame. The total kinetic energy of the object will then be:
	
	As we take $\overrightarrow{\text{O}\Omega}$ as instantaneous axis of rotation relatively to an observer in O we have $v_G=0$.
	
	If for some readers it is not obvious that: 
	
	just remember first that $E_p=mgh$ but as our comoving reference frame is oriented along $\overrightarrow{\text{O}\Omega}$ we must project the gravitational force on this axis. As you can see with the extreme trivial situations where $\theta=0$ (vertical spinning top) we have as expected:
	
	and for $\theta=\pi/2$ (spinning top rotation axis in the plane of the ground!) then:
	
	
	Then it remains:
	
	where the index $123$ corresponds to the notation of the comoving reference frame as seen in the section of Euclidean Geometry during our study of the  Euler angles.

	For $J_1,J_2,J_3$ following the proof we made for the inertial moment of a rotating cylinder around its main symmetric axis in the section Geometric Shapes and using also the parallel axis theorem we have by symmetry of our problem:
	
	But for the following developments we will not use the explicit relations above. In this way the reader can choose the spinning top geometry he wants and after just put the $J_1=J_2$, $J_3$ (so we not consider an asymmetric spinning top otherwise the general formula is rather complicated) corresponding values in the final formula!
	
	For what will follow we will put $I_\phi:=J_3$ and $I_0:=J_1=J_2$. Therefore:
	

	The Lagrangian (\SeeChapter{see section Analytical Mechanics page \pageref{lagrangian formalism}}) of the spinning top is deceptively simple as given then by:
	
	We see that the Lagrangian does not depend explicitly on the variables $\psi,\phi$ or $t$.. As a consequence, the angular momenta $p_\phi$ and $p_\psi$ and the total energy $E$ are invariant (i.e. conserved).
	
	Let us recall that in the section of Analytical Mechanics we have proved that for a non-conservative force (that is not derived from a potential) we have:
	
	Or equivalently:
	
	Therefore to simplify the development we consider the potential gravitational force as a non-conservative force (this doesn't change the result anyway but its more easy to introduce it in this way to students or most readers):
	
	And looking to our Lagrangian we see that it simplifies automatically to:
	
	So let us focus on what seem the easiest to handle, that is:
	
	It follows immediately:
	
	But as we have seen in the section of Analytical Mechanics:
	
	where $p_\phi$ and $p_\gamma$ are the canonical momentum.
	
	Thus explicitly:
	
	Something important to notice is that since:
	
	and as we have obviously $I_\phi=c^{te}$ we have immediately:
	
	And it is normally obvious that this corresponds therefore to the rotation speed $\Omega$ of the spinning top on itself such that:
	
	The values $p_\phi,p_\gamma$ are therefore constants of motions and, as we will see it below, are therefore indispensible quantities for deriving other aspect of the top's motion such as precesssion and nutation.

	Now we consider that the reader is familiar with the gyroscop's slow circling behaviour as int apparently defies gravity in its odd leaning without falling over. This is as we know "precession" trough the azimuthal angle $\gamma$. For the moment we will ignore nutation $\dot{\theta}=0$ and focus on the leaning angle $\theta$, assumed constant, and the rate of precession $\dot{\gamma}$.

	We shall use the energy method to derive the precession rate. This entails finding a point where the total system energy is stationary with respect to the angle $\theta$, i.e.:
	
	We can do this because $E_{\text{tot}}$ is a conserved quantity set by the initial conditions. The energy must be a "stationary" or extreme value equal to the total system energy. Any deviation in $\theta$ from the correct value represents an error in the solution that pushes away from stationary value. This provides us with a handy mathematical trick for finding important physical properties of the system.

	Since we ignore for now nutation (only steady precession is assumed), we eliminate the variables $\dot{\phi}$ and $\dot{\gamma}$ in $E_{\text{tot}}$ by putting $E_{\text{tot}}$ in terme of the invariants $p_\phi$ and $p_\gamma$. We get (don't forget that for now we put $\dot{\theta}=0$!):
	
	Indeed as for the firs term this is not obviosu let us check it:
	
	Taking the derivative of $E_\text{tot}$ with respect to the elevation angle $\theta$ yields:
	
	The solution of the above equation for $\theta$ yields the angle at which the gyroscope will lean given the speed of rotation of the wheel. The angle $\theta$ is found obviously by solving it for the roots in terms of the invariant quantities.

	The precession rate is found by reorganizing:
	
	Therefore:
	
	Hence:
	
	So now we can rewrite:
	
	
	We multiply the both sides by:
	
	So we get:
	
	After simplification and rearranging we get the following polynomial equation in $(p_\phi-p_\gamma\cos(\theta))$ to solve:
	
	Using the quadratic formula to solve for the roots of this equation, we have (\SeeChapter{see section Calculus page \pageref{second order polynomial roots}}):
	
	Using now:
	
	we get:
	
	Notice that if:
	
	that is to say:
	
	there is no well defined uniform precession. 
	
	The quadratic equation produces two solutions: the so-named "fast precession" and the "slow precession" solutions.

	If the rotation speed of the heavy wheel is large (i.e. $p_\gamma \gg\sqrt{4mgR_G\cos(\theta)}$, the fast precession rate is:
	
	Or if we take again the notations of our first approach with the symmetric weighing gyroscope:
	
	So we see that the gravitational force and the weight does not appear anymore so we are far from the result we get previously with the "classic approach".
	
	The slow precession rate is found by using the small argument approximation for $\sqrt{1-x}$ as proved in the section of Sequences and Series (Maclaurin series page \pageref{taylor series}):
	
	Hence we find that:
	
	Or if we take again the notations of our first approach with the symmetric weighing gyroscope:
	
	So we fall back on exactly the same result!!!!
	
	Once again it is interesting that the slow precession rate does not depend on the "\NewTerm{leaning angle $\theta$}\index{leaning angle}". The gyroscope will precess at a rate solely based on the wheel speed, stem length and the gravitational pull on the center of mass of the gyroscope. The slow precession rate is the one most commonly observed experimentally as we already know.
	
	Now that we have an expression for the precession rate, by using again:
	
	Thas is:
	
	Using trigonometric identities:
	
	After rearranging:
	
	Or:
	
	Using the "slow" precession rate, i.e. $\dot{\phi}=mgR_G/p_\gamma$ and the quadratic equation to solve for $\cos(\theta)$, we see that:
	
	Only the solution with the minus sign gives us a meaningful solution, because $\cos(\theta)$ can only take values between $-1$ and $1$. Therefore:
	
	Moreover, if $p_\gamma^2\gg I_0mgR_G$,we can use the small argument approximation for the square-root, which gives us:	
	
	Therefore:
	
	Hence:
	
	Now for the nutation who's typical movement is represented in the figure below:
	\begin{figure}[H]
		\centering
		\includegraphics{img/mechanics/gyroscope_spinning_top_precession_nutation.jpg}
		\caption[]{Spinning top with precession AND nutation mouvement}
	\end{figure}
	 don't forget that we have determined for $\dot{\theta}$:
	
	Re-including the kinetic energy of $\theta$-directed motion (nutation), we have:
	
	A "potential energy" function that provides a "restoring force" can be taken from the above relation as:
	
	Therefore:
	
	Hence:
	
	The extreme values of $\theta$ can be found by solving the above relation for the roots of $E_\text{tot}-V (\theta)$ in $\theta$ (these are the points in $\theta$ where $\dot{\theta}$ vanishes; i.e. the nutation angle extremes and therefore its variation $\Delta\theta=\theta_{\max}-\theta_{\min}$):
	
	That is:
	
	Or that we can rewrite:
	
	and also:
	
	Grouping, we get:
	
	Factorizing:
	
	Reorganizing:
	
	Factorizing:
	
	
	Setting $x=\cos(\theta)$ we have:
	
	By solving for $x$ , we find one or two roots between $-1$ and $1$ and another non-physical root ($|x| > 1$).
The two physically meaningful roots will yield the span of angles of the nutation. If the roots are equivalent, there
is no nutation; only smooth precession.

	We will now write the latter relation as in our study of third degree polynomial equations in the section Calculus:
	
	with:
	
	We use now the trick used in the section Calculus. First we know we have to put:
	
	So we get the following simplified polynomial (but the explicit expression of the coefficients are quite long...):
	
	 After, as we have proved it in the section Calculus we must calculate the following discriminant:
	
	and determine it's sign.
	
	
	\begin{figure}[H]
		\centering
		\includegraphics[scale=0.2]{img/mechanics/gyroscope_plane_mechanical_installation.jpg}
		\caption[]{Mechanical plane gyroscope installation principle}
	\end{figure}
	\begin{figure}[H]
		\centering
		\includegraphics{img/mechanics/gyroscope_earth.jpg}
		\caption[]{Earth nutation movment}
	\end{figure}
	
	\paragraph{König's kinetic and angular momentum theorems}\mbox{}\\\\
	In kinetics, "\NewTerm{König's theorems}\index{König's theorems}" or "\NewTerm{König's decompositions}\index{König's decompositions}" is a couple of mathematical relations derived by Johann Samuel König that assists with the calculation of kinetic energy and angular momentum of bodies and systems of particles.
	
	These theorems expresses the kinetic energy and angular momentum of a system of particles in terms of the velocities of the individual particles. Specifically, they states that the kinetic energy and angular moment of a system of particles is the sum of the kinetic energy/angular momentum associated to the movement of the center of mass and the kinetic energy/angular momentum associated to the movement of the particles relative to the center of mass.
	
	As this result is quite obvious many teachers and books don't speak about these theorems and therefore don't prove them. I have put these theorem here only as a tribute to one of the teacher of my Engineering School but for not other reason as we will never mention them directly in any other section or chapter of this book (again because it's simply obvious!).
	
	OK. Let us start! So don't forget we have seen so far how to calculate the angular momentum or kinetic energy of a dynamic system compared to a single reference frame (either Galilean or centroid)

	The König's theorems give respectively the angular momentum and the total kinetic energy of a dynamic system with respect to a Galilean reference frame $R_{\text{gal}}$ or centroid one $R_\text{cen}$.
	
	\subparagraph{First König's Theorem (König's angular momentum theorem)}\mbox{}\\\\
	Let use to prove this theorem the angular momentum of a mass $M$ (the example is still easily expandable to a discrete or continuous dynamic material system ).

	Let us express the angular momentum $\vec{b}$ of an element $m_i$ of the solid body with respect to the origin O of the Galilean $R_\text{gal}$ reference frame (denoted thereafter $/\text{O},R$):
	
	Let us express the angular momentum in $R_\text{cen}$ relatively to its center of mass $G$ (denoted thereafter $/G,R'$):
	
	The reference frame $R_{\text{cen}}$ being in translation relatively to $R_\text{gal}$, we have:
	
	Without forgetting that:
	
	that we insert in the expression of the angular momentum relatively to $R_\text{gal}$:
	
	From the property of distributivity over addition of the cross product, we have:
	
	Let us now study the value that take each of the four terms of the previous relation. We have proved by definition of the center of mass that (in a non-relativistic framework):
	
	hence:
	
	So it comes:
	
	Therefore we have finally:
	
	This theorem that relates to a fixed point allows an easier application of the theorem of angular momentum.
	
	\subparagraph{Second König's Theorem (König's kinetic energy theorem)}\mbox{}\\\\
	To prove this theorem let us use the kinetic energy of a mass $M$ (the example is still easily expandable to a discrete or continuous dynamic system of material).

	Let us express the kinetic energy of an element $m_i$ of the solid body with respect to the origin O of the Galilean reference frame $R_\text{gal}$ (denoted thereafter $/\text{O},R$):
	
	Let us express the kinetic energy in $R_\text{cent}$ relatively to to its center of mass $G$ (denoted thereafter $/G,R'$):
	
	With same as above:
	
	Therefore we have:
	
	Hence (the dot products of the vectors are implicit!!!):
	
	and as for the angular momentum, by the definition of the mass center, we have:
	
	Hence the second König's theorem:
	
	
	\pagebreak
	\subsubsection{Gravitational Potential Energy}\label{gravitational potential energy}
	Gravitational potential energy is energy an object possesses because of its position in a gravitational field. The most common use of gravitational potential energy is for an object near the surface of the Earth. The gravitational potential is also known as the "\NewTerm{Newtonian potential}\index{Newtonian potential}" and is fundamental in the study of potential theory.

	Since the zero of gravitational potential energy can be chosen at any point (like the choice of the zero of a coordinate system), the potential energy at a height $h$ above that point is equal to the work which would be required to lift the object to that height with no net change in kinetic energy. Since the force required to lift it is equal to its weight, it follows that the gravitational potential energy is equal to its weight $mg$ times the height $h$ to which it is lifted $mgh$.
	
	Let us first recall that If the work of the force $\vec{F}$ between points $A$ and $B$ is independent of the path taken, we say that this force derives from a potential energy or that force field is a "conservative field" (counter-example: in a motion with friction works necessarily depends on the chosen path). This independence from the path followed that if given two points $A$ and $B$ in space. There are several possible ways to join these two points. If we choose two random point we have:
	\begin{figure}[H]
		\centering
		\includegraphics{img/mechanics/path_independance.jpg}
	\end{figure} 
	\begin{itemize}
		\item On the first path: $\displaystyle\int_A^B \vec{F}\circ\mathrm{d}\vec{r}_1$

		\item On the second path: $\displaystyle\int_A^B \vec{F}\circ\mathrm{d}\vec{r}_2$
	\end{itemize}
	If the field is conservative we have:
	
	or in other words that the total work on a closed path (go and come back) is zero. We notice that (\SeeChapter{see section of Differential and Integral Calculus page \pageref{curvilinear integral}}):
	
	The work involved is therefore a function of the location only ($E_p$) that is to say, depending only on the starting point and ending point. Indeed, if the work depended on the path it would be possible to choose the most generous way when the system provides work and the most economical way when we bring him back to the initial state. So it would be a perpetual motion and the principle of conservation of energy forbids that (\SeeChapter{see section of Thermodynamics page \pageref{heat flow}})!
	
	Let us then attach to each point $P$ of the force field a value of the function $E_p$ (a real number) corresponding to the work done by the force field when the mobile moves from one point $P$ to $0$, $0$ being a reference point arbitrarily chosen:
	\begin{figure}[H]
		\centering
		\includegraphics{img/mechanics/gravitation_potential_energy_reference_point.jpg}
	\end{figure}
	So by definition:
	
	with $E_p(0)=0$.
	
	Generalizing this definition a little bit, we say that the work $W$ done by a conservative force when moving from $A$ to $B$:
	\begin{figure}[H]
		\centering
		\includegraphics{img/mechanics/gravitation_potential_energy_between_two_points.jpg}
	\end{figure}
	is equal to the reduction of potential energy between $A$ and $B$:
	
	By definition $E_p$ is the potential energy and is measured in Joules [J] as for the kinetic energy.

	The above relation is used very often in its differential form is:
	
	There is also for recall a relation between energy and the gradient of the given force that simply follows from the definition of work:
	
	The most significant application in Classical Mechanics is the work of gravity and the gravitational potential energy near the surface of the Earth (or another planet!). It is therefore a special case where the force is constant...
	\begin{figure}[H]
		\centering
		\includegraphics{img/mechanics/gravitation_potential_energy_between_two_points_application.jpg}
		\caption[]{Illustration of gravitational work between two point in constant force field}
	\end{figure}
	Following the figure above given a point of mass $m$ moving following any trajectory $AB$ in a constant gravitational field force $\vec{g}$. The weight $m\vec{g}$ do then the following work:
	
	The difference $z_A-z_B$ is the difference in elevation between points $A$ and $B$. We notice well that the work does not depend on the path followed but only on the points of departure and arrival. If, conversely, we want to move the massive point from $B$ to $A$, the work, then provided by an external element is:
	
	which shows well that the total work on a closed path is zero is such a special case:
	
	Comparing the relations:
	
	and identifying, we get:
	
	which is the gravitational potential energy, $z$ is the altitude of the mass $m$. We denote this simply this relation this relation under the form:
	
	\begin{tcolorbox}[title=Remark,colframe=black,arc=10pt]
	The choice of the zero level of potential energy is often arbitrary; we set it ourselves for convenience depending on the situation. Only the differences of potential energy are generally interesting as we will see just below.
	\end{tcolorbox}
	The previous relation is actually a useful expression near the Earth's surface (or anyl other planet). At a distance $r\gg R$ where $R$ is the radius of the Earth, the gravitational force weakens and the approximation is no longer valid (if $r\ll R$ also anyway...).
	
	To determine the correct relation, consider two masses $m_1,m_2$. The first is supposed at rest and fixed and the second is brought from infinity to a given distance from $m_1$ (the same reasoning applies to the electric field!!!!). The work $\mathrm{d}W$ of the gravitational force at any point being thus:
	
	and the potential energy of the system:
	
	Then:
	
	where just after integration (the potential energy in a point):
	
	Let us see if this is consistent with $E_p=mgz$...:
	\begin{dem}
	At zero height of the planet surface $x=R$, we have:
	
	where the choice of the sign "$-$" depends only on the selected reference frame that is in this case in line with what is customary to take in schools.

	We raise the object of $h\ll R$:
	
	We use now the rough approximation (\SeeChapter{see section Sequences and Series page \pageref{taylor series}}):
	
	valid when $x\ll 1$, hence:
	
	As at the surface of the Earth (or any other planet) we usually put in laboratories that $U_0=0$, we get well eventually:
	
	and we see that it is indeed a rough approximation.
	\begin{flushright}
		$\square$  Q.E.D.
	\end{flushright}
	\end{dem}
	\begin{tcolorbox}[title=Remark,colframe=black,arc=10pt]
	We could apply the same development in the study of the Coulomb force and the of the Electric field but until now we have never been put a laboratory to the surface of an electric charge... (sic!).
	\end{tcolorbox}
	
	\paragraph{Gravitational Potential Energy of a Material Sphere}\mbox{}\\\\
	We will calculate here the potential energy of a material sphere. This style of exercise will be very useful in the section Astrophysics to determine the internal temperature of the stars and in the section Cosmology toe start the study of the Friedmann model.

	The potential energy of the spherical ring of inner radius $r$ and thickness $\mathrm{d}r$ is calculated as follows:
	
	Given a sphere of mass $M$, of mass density $\rho$ and radius $r$ surrounded by a spherical ring of inner radius $r$, of the same mass density $\rho$ and thickness $\mathrm{d}r$:
	\begin{figure}[H]
		\centering
		\includegraphics{img/mechanics/gravitation_sphere.jpg}
	\end{figure}
	The potential energy of the spherical ring of inner radius $r$ and thickness $\mathrm{d}r$ is calculated as follows:

	The mass of the sphere of radius $r$ and mass density $\rho$ is equal as we know:
	
	The mass of the ring surrounding the sphere of radius $r$, of thickness $\mathrm{d}r$ and mass density $\rho$ is:
	
	By introducing the last two terms in the expression of the potential energy:
	
	By integrating the previous expression between $0$ and $R$, this is equivalent to add successively a range of rings of thickness $\mathrm{d}r$ to get the whole sphere of radius $R$ and therefore the potential energy of the whole sphere.
	
	Which can be written also:
	
	Thus finally:
	
	Exactly the same calculation can be done in electrostatics by replacing the gravitational constant $G$ by the Coulomb constant $k_e$  and the mass $M$ by the total Electric charge $Q$. It is also a result we will reuse in the model of the liquid nuclear core (\SeeChapter{see section Nuclear Physics page \pageref{liquid drop model}}).
	 
	\subsubsection{Conservation of Total Mechanical Energy}
	Let us compare the relations:
	
	since it is the same work!

	Which leads to:
	
	sum of the two forms of energy at each point or also, considerating any locations $A$ and $B$, by rewriting the equation in in a more general form:
	
	\begin{tcolorbox}[title=Remark,colframe=black,arc=10pt]
	We often name the total energy of a system "\NewTerm{Hamiltonian of the system}\index{Hamiltonian of the system}" as we have already study it in the section of Analytical Mechanics.
	\end{tcolorbox}
	In the absence of friction of external force in the case of mechanical energy, we also write the varsiation such that:
	
	An increase in kinetic energy therefore results in a decrease of potential energy (and vice versa) since the sum of the two remains constant.

	Against-example: If there is friction, heat is released and then total mechanical energy is not constant! Also if the studied object is a rocket propulsed in space.

	Furthermore let us come back on the relation:
	
	and therefore:
	
	On the other hand, $E_p$ being a scalar function dependent on spatial coordinates, let us write its total differential:
	
	comparing with the previous equation and identifying term by term, we have:
	
	hence the sentence stating that the force derived from a potential energy if the work involved is independent of the followed path. If we express the force $\vec{F}$ in terms of unitary vectors , we get:
	
	Finally, the claim (statement) that the force derived from a potential energy $E_p$ can be summarized as:
	
	In the case of gravity:
	
	which is also written with the nabla operator (\SeeChapter{see section Vector Calculus page \pageref{gradient scalar field}}):
	
	That is is common to write also: 
	
	That is to say in the one dimensional case:
	
	So what we have seen so far about the potential energy bring us to what we name the "\NewTerm{potentiel theory}\index{potentiel theory}" and resumed by:
	\begin{figure}[H]
		\centering
		\includegraphics{img/mechanics/potential_theory.jpg}
		\caption{Potential theory summary}
	\end{figure}
	
	\pagebreak
	\paragraph{Generalized Newton Law}\mbox{}\\\\
	Now that we know explicitly how to express a kinetic and potential energy, let us come back to our principle of least action that we talked about in the section of Analytical Mechanics to proved that we can use it that latter to fall back on the generalized Newton's law demonstrated earlier in this section!
	
	Let us take the case of an object launched into the air and let us choose two points of its motion path in any two different times. And infinite quantity of curves existe between these two points, but the Nature choose only one. So what distinguishes this curve - physical path - from all others? To answer this question we could quite rightly reply that this curve differs from others in that it is a solution of the differential equation of the path ... with appropriate initial conditions. But if we ignore the initial conditions or when the problem can not be reduced to a differential equation, by what method can we then distinguish the physical path of all possible paths?
	
	The principle of last action in this context is expressed by a minimum speed for a minimum of motion path distance.

	About speed, it is better in Mechanics to consider the quantity of linear momentum as the latter quantity is directly related to the inertial properties of the body. Pierre Louis Moreau de Maupertuis mathematically translated the principle of least action as follows for recall (\SeeChapter{see section Analytical Mechanics page \pageref{lagrangian mechanics}}): If we consider the motion of a body between two points $A$ in $t_1$ and $B$ in $t_2$ by, for a given total energy $E$, the path selected by Nature is one for which the following quantity $A_A^B$ is minimum:
	
	\begin{figure}[H]
		\centering
		\includegraphics{img/mechanics/action_path_generalized_newton_law.jpg}
	\end{figure}
	The physical path between two points $A$ and $B$ at times $t_A$ and $t_B$ is that for which the action is minimal!

	Knowing that:
	
	then we get:
	
	where $T$ is as we the kinetic energy $E_c$ of the body but as denoted in Analytical Mechanics...
	
	We see that the action takes a surprisingly simple form and is expressed directly in terms of kinetic energy $T$. A few years later, from an intuition similar to that of Maupertuis, Leonhard Euler came to a very similar statement of the action but starting from the premise that the moving body tend to adopt a state where the potential energy is minimal. The Euler's action was expressed in terms of the potential energy rather than kinetic energy. Who of Maupertuis or Euler was right?
	
	In fact, their personal definition Action were equivalent. We know that in a conservative field if we denote by $U$ the potential energy then the total energy $E_\text{tot}$ is equal to $T + U$ and the energy is a constant (in an isolated system). We derive imediately that $T = E_\text{tot} - U$ and therefore:
	
	Hence:
	
	This relation is true regardless of the initial total energy $E_\text{tot}$ of the path. We conclude that the value of the constant $E_\text{tot}$ does not discriminate between different paths and can thus be removed from the formulation of the action. The action of Maupertuis can then be reduced to a new quantity denoted $S$ (we fall back on the notations seen in the section of Analytical Mechanics):
	
	This new formulation of the action was given by Joseph-Louis Lagrange in 1788. $S$ is as we already know the "Lagrangian's action" (\SeeChapter{see section Analytical Mechanics page \pageref{lagrangian action}}) and the function:
	
	is named as we know the "\NewTerm{Mechanics Lagrangrian}\index{mechanics Lagrangrian}". Thus formulated, the principle of least action became one of the most powerful tools of mechanics and above.

	We have already seen how we express the principle of least action mathematically (\SeeChapter{see section Analytical Mechanics page \pageref{lagrangian mechanics}}). In our case, the action is not a function of analytical variables but of trajectories!
	
	Let us consider the simple case of a body of mass $m$ that is moving on a single dimension (which we represent by the $x$-axis ) of a point of abscissa $x_1$ at the moment $t_1$ to a point $x_2$ at a moment $t_2$. Let us suppose that the body is subject to a potential $U$ that does not vary with the time that is to say $U=U(x)$. The action on this body on any path $\Gamma$ going from $(x_1,t_1)$ to $(x_2,t_2)$ is then:
	
	Given $\Gamma_0$ the physical path and $S_0$ the action on this path. Let us denote by $\bar{x}$ the values of the position $x$ on the physical path. Let us consider now a path $\Gamma$ very near of $\Gamma_0$ such that the positions along the path $\Gamma$ have the values $x(t)=\bar{x}(t)+\varepsilon(t)$ that we will write, to facilitate the notations: $x=\bar{x}+\varepsilon$.
	
	Let us now calculate the action for this path:
	
	As $\varepsilon$ is infinitely small, it is possible to express the potential as a limited Taylor development (\SeeChapter{see section Sequences and Series page \pageref{taylor series}}):
	
	And for the first term we simply develop the square identy:
	
	As we consider only the variations of the first order (very small one), the last term above can be neglected, which gives for the action on the path $\Gamma$:
	
	Let us now put the variation $\delta$ of the action between the physical path $\Gamma_0$ and $\Gamma$ is zero as it seems to be observed in Classical Mechanics labs:
	
	Let us now put the variation $\delta$ of the action between the physical path $\Gamma_0$ and $\Gamma$ is zero as it seems to be observed in Classical Mechanics labs:
	
	Therefore:
	
	The first term in the last integral above can be integrated by parts as follows (\SeeChapter{see section Differential and Integral Calculus page \pageref{integration by parts}}):
	
	But, all paths starting from $x_1$ at time $t_1$ and arriving at $x_1$ at time $t_2$. This implies that in $t_1$ and $t_2$ the variation of $\varepsilon$ is zero, what we write
	
	so the first term of integration by parts is zero such that:
	
	So finally:
	
	As this integral must be zero for all of the paths very close of the physical path $\Gamma_0$, regardless of the value of $\varepsilon$, the for such a condition to be fulfilled it seems necessary that the term in front of $\varepsilon$ is zero, that is to say:
	
	But we know in fact that equation: the first term is nothing but $m\cdot a$ where $a$ is the (average) acceleration of the body, and the second - the opposite of the potential gradient - is the amount of the gravitational force in a given point. This can the be rewritten as (but when the force is zero):
	
	So we fall back on the generalized Newton's second law already developed at the beginning of the section!!!!
	
	The principle of least action thus implicitly contains Newtonian mechanics. Thus, it is possible to reconstruct the entire Newtonian mechanics with the only the principle of least action and some assumptions!!!

	This scaffolding of calculations may seem complicated to achieve only t a result that we already knew  but the whole point of the principle of least action is that it allows to draw the basic laws from the only knowledge of the Lagrangian of a system (\SeeChapter{see section Analytical Mechanics page \pageref{lagrangian formalism}}).

	The most recent theories such as quantum field theory, gauge theories or superstring theory all have to start the expression of the action of the system (see corresponding sections in this book). Physicists then extract fundamental laws governing the behavior of elementary particles.
	
	\pagebreak	
	\subsubsection{Conservation of Linear Momentum}\label{conservation of linear momentum}
	A moving object, during an interaction (collision) with another moving object point may transmit all or part of its motion (kinetic energy and / or potential). This is the case during an impact, for example (in fact the detailed calculation an impact  is extremely difficult to do without many simplifications). And the exchanged quantity is the linear momentum $\vec{p}$. It is equal by definition (as we have seen it earlier above):
	
	Obviously, we have:
	
	The quantity:
	
	is sometimes named "\NewTerm{impulse}\index{impulse}", and the above relation sometimes is named "\NewTerm{momentum theorem}\index{momentum theorem}".
	
	This theorem is stated as follows: The impulse supplied by a force between the moments $t_1$ and $t_2$ is equal to the variation of the amount of linear momentum during this time interval.

	But let us come back to our conservation of linear momentum (and thus energy ... and vice versa). The interest of the quantity of linear momentum  is due to the fact that it is preserved in interactions in Classical Mechanics (at least in first approximation...). Indeed, given two moving objects in collision, under the equality of action and reaction (3rd Newton's law) we have:
	
	and using the theorem of linear moment we can write:
	
	By adding member to member these two equations, we deduce:
	
	as $\vec{F}_{2,1}=-\vec{F}_{1,2}$ and therefore in a "perfect" collision:
	
	The total linear momentum is constant, it is therefore conserved!
	
	\textbf{Definition (\#\mydef):} An "\NewTerm{elastic collision}\index{elastic collision}" is an encounter between two bodies in which the total kinetic energy of the two bodies after the encounter is equal to their total kinetic energy before the encounter (the objects in question "bounce perfectly" like an ideal elastic). Elastic collisions occur only if there is no net conversion of kinetic energy into other forms (such as heat or noise):
	\begin{figure}[H]
		\centering
		\includegraphics{img/mechanics/ellastic_vs_nonellastic_collision.jpg}
	\end{figure}
	An "\NewTerm{inelastic collision}\index{inelastic collision}" is one where some of the of the total kinetic energy is transformed into other forms of energy, such as sound and heat. Any collision in which the shapes of the objects are permanently altered, some kinetic energy is always lost to this deformation, and the collision is not elastic. It is common to refer to a "\NewTerm{completely inelastic}\index{completely inelastic}" collision whenever the two objects remain stuck together, but this does not mean that all the kinetic energy is lost; if the objects are still moving, they will still have some kinetic energy.
	\begin{tcolorbox}[title=Remark,colframe=black,arc=10pt]
	We will study in details an example of two body (non-relativistic) aligned elastic collision during our study of Newton Pendulum further below as a special case of the example above.
	\end{tcolorbox}
	
	\paragraph{Elastic Collision in $1$-dimensions}\mbox{}\\\\
	Given two objects, $m_1$ and $m_2$, with initial velocities of $v_{1i}$ and $v_{2i}$, respectively, how fast will they be going after they undergo a completely elastic collision? We can derive some expressions for $v_{1i}$ and $v_{2i}$ by using the conservation of kinetic energy and the conservation of momentum, and a lot of high-school algebra.
	
	Begin by making the following conservation statements:
	\begin{itemize}
		\item Conservation of Kinetic Energy:
		
	
		\item Conservation of Momentum:
		
	\end{itemize}
	To solve for $v_{1f}$ and $v_{2f}$ (which is really two equations in two unknowns), we need some algebra tricks to simplify the substitutions. Take both equations and group them according to the masses: put all the $m_1$ on one side of the equation and all the $m_2$ on the other. We'll also cancel out all the $1/2$ at this point.
	
	Conservation of Kinetic Energy becomes:
	
	which can be simplified as:
	
	Conservation of momentum becomes:
	
	which can be simplified as:
	
	Now comes the algebra fun. We divide:
	
	by:
	
	
	After all the cancellations, we are left with:
	
	Solving for $v_{1f}$ we get:
	
	Now we take the previous equation and substitute back into one of our original equations to solve for $v_2f$. Since the momentum equation is easier, lets use that.
	
	Conservation of Momentum becomes:
	
	Now do some algebra...
	
	Until we get:
	
	Now we substitute this result back into equation:
	
	 do some algebra to solve for $v_{1f}$:
	
	Now do some algebra again...:
	
	Until we get:
	
	
	\paragraph{Elastic Collision in $2$-dimensions}\mbox{}\\\\
	Suppose that an object of mass $m_1$, moving with initial speed $v_{i1}$, strikes a second object, of mass $m_2$, which is initially at rest. Suppose, further, that the collision is not head-on, so that after the collision the first object moves off at an angle $\theta_1$ to its initial direction of motion, whereas the second object moves off at an angle $\theta_2$ to this direction:
	\begin{figure}[H]
		\centering
		\includegraphics{img/mechanics/collision_2dimensions_elastic.jpg}
	\end{figure}
	 Let the final speeds of the two objects be $v_{f1}$ and $v_{f2}$, respectively.

	We are again considering a system in which there is zero net external force (the forces associated with the collision are internal in nature). It follows that the total momentum of the system is a conserved quantity. However, unlike before, we must now treat the total momentum as a vector quantity, since we are no longer dealing with 1-dimensional motion. Note that if the collision takes place wholly within the $x$-$y$ plane, as indicated in the figure above, then it is sufficient to equate the $x$- and $y$-components of the total momentum before and after the collision.

	Consider the $x$-component of the system's total momentum. Before the collision, the total $x$-momentum is simply $m_1 v_{i1}$, since the second object is initially stationary, and the first object is initially moving along the $x$-axis with speed $v_{i1}$. After the collision, the $x$-momentum of the first object is  $m_1 v_{f1} \cos(\theta_1)$: i.e., $m_1$ times the $x$-component of the first object's final velocity. Likewise, the final $x$-momentum of the second object is  $m_2 v_{f2} \cos(\theta_2)$. Hence, momentum conservation in the $x$-direction yields 
	
	
	Consider the $y$-component of the system's total momentum. Before the collision, the total $y$-momentum is zero, since there is initially no motion along the $y$-axis. After the collision, the $y$-momentum of the first object is  $-m_1 v_{f1} \sin\theta_1$: i.e., $m_1$ times the $y$-component of the first object's final velocity. Likewise, the final $y$-momentum of the second object is  $m_2 v_{f2} \sin\theta_2$. Hence, momentum conservation in the $y$-direction yields 
	
	
	For the special case of an elastic collision, we can equate the total kinetic energies of the two objects before and after the collision. Hence, we obtain 
	
	
	Given the initial conditions (i.e., $m_1$, $m_2$, and $v_{i1}$), we have a system of $3$ equations:
	
	
	 and $4$ unknowns (i.e., $\theta_1$, $\theta_2$, $v_{f1}$, and $v_{f2}$). Clearly, we cannot uniquely solve such a system without being given additional information: e.g., the direction of motion or speed of one of the objects after the collision.
	
	
	\paragraph{Inelastic Collision in $2$-dimensions}\mbox{}\\\\
	
	The figure below shows a $2$-dimensional totally inelastic collision. 
	\begin{figure}[H]
		\centering
		\includegraphics{img/mechanics/collision_2dimensions_nonelastic.jpg}
	\end{figure}
	In this case, the first object, mass $m_1$, initially moves along the $x$-axis with speed $v_{i1}$. On the other hand, the second object, mass $m_2$, initially moves at an angle $\theta_i$ to the $x$-axis with speed $v_{i2}$. After the collision, the two objects stick together and move off at an angle $\theta_f$ to the $x$-axis with speed $v_f$. Momentum conservation along the $x$-axis yields 
	
	
	Likewise, momentum conservation along the $y$-axis gives 
	
	
	Given the initial conditions (i.e., $m_1$, $m_2$, $v_{i1}$, $v_{i2}$, and $\theta_i$), we have a system of two equations  and two unknowns (i.e., $v_f$ and $\theta_f$). Clearly, we should be able to find a unique solution for such a system.
	
	\subsubsection{Power}\label{power}
	\textbf{Definition (\#\mydef):} The "\NewTerm{power}\index{power}" is the instantaneous rate of change of work (energy in any form). So we have the "\NewTerm{instantaneous power}\index{instantaneous power}" that is given by:
	
	The unit of power is the "\NewTerm{Watt}\index{Watt}" (J$\cdot$s$^{-1}$) and is denoted [W] but in technique, some still use the "\NewTerm{horsepower}\index{horsepower}" [ch] defined as being equal to $736$ [W] (as an average horse could at that time lift $75$ kilograms or $1$ meter in $1$ second under the Earth's gravity).
	
	If the work is provided on a regular basis (constant), then we have the "\NewTerm{average power}\index{average power}":
	
	that should be denoted $\bar{P}$ but depending on the context we don't need to strictly respect the notations!
	
	With this definition, the reader might think that a vehicle traveling at a constant speed can therefore provides then a power equal to zero because its kinetic energy does not change. In reality it is not so, because the car has to constantly overcome the friction of the tires with the road (see further the study of tribology), the viscous friction with the air (\SeeChapter{see section Continuum Mechanics page \pageref{horizontal viscious friction}}), and loss of energy due of vibration and friction of its own components like axles, ball bearings, springs, heat exchange entropy, etc. Thus, a vehicle must every second supply the energy he lost in these frictions and heat transfers. We then have the "\NewTerm{power of force}\index{power of force}":	
	
	where we used the definition of work $W$ (force over a distance) and where $F_T$ is the sum of various forces.
	
	However, because we have often have to account for acceleration when we apply a force, we usually write the equation in terms of average power and average speed:
	
	But this also apply to a rocket pushing a satellite in space vacuum far from any gravitational field!!!
	
	Notice finally that we can also found in some books:
	
	
	\begin{tcolorbox}[title=Remarks,colframe=black,arc=10pt]
	\textbf{R1.} Expressing the work $W$ (energy) from the relation $W=Pt$ where as we have just seen the power is given in [kW] and the time in hours, it appears the unit of energy [kWh] (kilowatt hour), widely used in practice by power plant, transformators and various domestic devices.\\
	
	\textbf{R2.} The $2000$ [W] society is an environmental vision, first introduced in 1998 by the Swiss Federal Institute of Technology in Zürich (ETH Zurich), which pictures the average First World citizen reducing their overall average primary energy usage to no more than $2,000$ [W] ($48$ kilowatt-hours per day) by the year 2050 - and without lowering their standard of living.
	\end{tcolorbox}
	
	\paragraph{Power of a turning machine}\mbox{}\\\\
	The elementary work $\mathrm{d}W$ done by the for $\vec{F}$ turning a solid (a cylinder in the case presented below) around its axis of symmetry of an angle $\theta$ is equal to:
	
	The instantaneous power is then:
	
	Now, as we have defined it earlier the force momentum is is:
	
	The power of a couple (torque) is then given by:
	
	This is a very popular relation well know by people passionate by motor vehicles and mechanician. Indeed, knowing the engine torque (the moment of force) and the motor speed (that must be converted intor the correct units), we easily get an approximation of the power developed by the engineer. If we divide the result by $736$, the reader will get the horsepower.
	\begin{tcolorbox}[colframe=black,colback=white,sharp corners]
	\textbf{{\Large \ding{45}}Example:}\\\\
	A boat engine is operating at $9.0\cdot 10^4$ [W] with running at $300$ revolutions per minute. The torque is then given by:
	
	\end{tcolorbox}
	
	
	\subparagraph{Power yield}\mbox{}\\\\
	Because of friction and unperfect materials, the power output by a machine known as "\NewTerm{power output}\index{power output}" is always less than the "\NewTerm{input power}\index{input power}" in an isolated system. We take account of this fact by using the "\NewTerm{power yield}\index{power yield}" defined by:
	
	As it the numerator and denominator are measured during the same amount of time we also have the equivalent in terms of enery:
	
	 and named sometimes also "\NewTerm{energy returned on energy invested (EROEI or ERoEI)}\index{energy returned on energy invested}"  or "\NewTerm{energy return on investment (EROI)}\index{energy return on investment}".

	When the EROEI of a resource is less than or equal to one, that energy source becomes a net "energy sink", and can no longer be used as a source of energy, but depending on the system might be useful for energy storage (for example a battery). A related measure 

	We will return in much more detail on the above relations in our study of Thermodynamics (see section of the same name page \pageref{thermal efficiency}).
	
	\pagebreak
	\subsection{Relative Movements and Inertial Forces}\label{relative movements and inertial forces}
	Let us now see the developments that will allow us to bring a very important and useful element in fluid mechanics (\SeeChapter{see section Continuum Mechanics page \pageref{fluid mechanics}}) and meteorology (\SeeChapter{see section Marine \& Weather Engineering page \pageref{meteorology}}).

	Let us consider a fixed reference system $X, Y, Z$ and a mobile reference frame $x, y, z$. They are therefore in relative motion and we consider a possible rotation of the moving frame. So our purpose is to express the speed, the acceleration of a point $P$ ot the space thanks to the coordinates of the fixed reference frame (absolute coordinates) from those attached to the moving frame (relative coordinates) and to the translation movement of this same moving frame.

	We define for our study:
	\begin{itemize}
		\item $\vec{r}=(x(t),y(t),z(t))$ position vector of $P$ relatively to the mobile frame
	\item $\vec{r}_0=(x_0(t),y_0(t),z_0(t))$ position vector of $P$ relatively to the fixed frame
		\item $\vec{\rho}=(\ldots)$ position vector of O relatively to the origine of the fixed frame
		\item $\vec{v}_a=(\ldots)$ absolute velocity of $P$ relatively to the fixed frame (supposed as unknown)
		\item $\vec{a}_a=(\ldots)$ absolute acceleration of $P$ relatively to the fixed frame (supposed as unknown)
		\item $\vec{v}_r=(\ldots)$ relative velocity of the point $P$ relatively to the mobile frame (supposed as known)
		\item $\vec{a}_r=(\ldots)$ relative acceleration of the point $P$ relatively to the mobile frame (supposed as known)
		\item $\vec{v}_e=(\ldots)$ translation velocity at the point O of the mobile frame relatively to the fixed frame (supposed as known)
		\item $\vec{a}_e=(\ldots)$ translation acceleration at the point O of the mobile frame relatively to the fixed frame (supposed as known)
		\item $\vec{\omega}=(\ldots)$ angular speed of the mobile frame
	\end{itemize}
	\begin{figure}[H]
		\centering
		\includegraphics{img/mechanics/relative_movments_and_inertia_forces.jpg}
		\caption{Example of moving moving and rotating reference frame relatively to a fixed reference frame}
	\end{figure}
	The position of the point $P$ is therefore is given obviously by the "\NewTerm{relation of composition of positions}\index{relation of composition of positions}":
	
	The absolute speed is calculated as follows:
	
	The last term is the contribution due to the rotation of the moving frame. We have now to express the value of this contribution by considering rotations of  angle $\mathrm{d}\theta_i=\omega_i\mathrm{d}t$ around each axis, successively:
	\begin{figure}[H]
		\centering
		\includegraphics{img/mechanics/relative_movments_and_inertia_forces_rotation.jpg}
	\end{figure}
	\begin{gather*}
		\mathrm{d}\vec{i}=(\vec{j}\omega_z-\vec{k}\omega_y)\mathrm{d}t\quad \mathrm{d}\vec{j}=(\vec{k}\omega_x-\vec{i}\omega_z)\mathrm{d}t\quad \mathrm{d}\vec{k}=(\vec{i}\omega_y-\vec{j}\omega_x)\mathrm{d}t
	\end{gather*}
	We obtain thus the elementary vectors $\mathrm{d}\vec{i},\mathrm{d}\vec{j},\mathrm{d}\vec{k}$ representing the shifting of the ends of the unit-vectors $\vec{i},\vec{j},\vec{k}$. We introduce them in the following expression which becomes, after rearrangement of the terms:
	
	by definition of the vectorn cross product (\SeeChapter{see section Vector Calculus page \pageref{cross product}}). The absolute velocity of the point $P$ is therefore expressed as the "\NewTerm{composition law of velocities}\index{composition law of velocities}":
	
	we see that in the particular case where the moving reference frame only undergoes a translation such that $\vec{\omega}=\vec{0}$, we fall back on:
	
	that is the characteristic of the Galilean transformation and then we say that the reference frame are in "\NewTerm{relative translations}\index{relative translations}".
	\begin{tcolorbox}[title=Remark,colframe=black,arc=10pt]
	If we focus only on the terms $\vec{v}_e$ and $\vec{\omega}$ of the moving reference frame then we get what we name the "\NewTerm{Bour formula}\index{Bour formula}".
	\end{tcolorbox}
	Proceeding in the same way as the search for the absolute velocity it comes, by derivating the prior previous boxed relation:
	
	with:
	
	If we regroup the terms:
	
	If we regroup the terms:
	
	If we regroup the terms:
	
	and we have:
	
	Therefore:
	
	Finally:
	
	but (!) let us recall that:
	
	
	The absolute acceleration or "\NewTerm{composition law of accelerations}\index{composition law of accelerations}" or "\NewTerm{acceleration transformation formula}\index{acceleration transformation formula}" is then expressed as:
	
	The term:
	
	is named the "\NewTerm{Coriolis acceleration}\index{Coriolis acceleration}" ($\sim$1820) and the term $\vec{\omega}\times(\vec{\omega}\times\vec{r})$ is simply the expression of the centripetal acceleration in this particular case.
	
	The physical acceleration $\vec{a}_a$ due to what observers in the inertial frame name "real external forces" on the object is, therefore, not simply the acceleration $\vec{a}_e+\vec{a}_r$ seen by observers in the rotational frame, but has several additional geometric acceleration terms associated with the rotation of moving rotational frame.
	
	The second Newton's law $\vec{F}=m\vec{a}$ must have all the terms contained in the general equation above. For an observer in a stationary system, this law is then:
	
	If we rewrite it as the point of view of the observer at the origin of the rotating and moving reference frame, we get obviously:
	
	The set:
	
	is the fictitious force used by the observer in the moving rotating frame to get the correct behavior of the object from Newton's laws.
	
	Above, the first term is the "\NewTerm{Coriolis force}\index{Coriolis force}\label{coriolis force}", the second term is the "\NewTerm{Euler Force}\index{Euler force}", and the third term is the "\NewTerm{centrifugal force}\index{centrifugal force}\label{centrifugal force}".
	
	Indeed in scalar form we recognize in the third term:
	
	
	The Euler force is the fictitious tangential force that is felt in reaction to any angular acceleration. That reactive acceleration is the "\NewTerm{Euler acceleration}\index{Euler acceleration}" also known as "\NewTerm{azimuthal acceleration}\index{azimuthal acceleration}" or "\NewTerm{transverse acceleration}\index{transverse acceleration}".  In other words, it is an acceleration that appears when a non-uniformly rotating reference frame is used for analysis of motion and there is variation in the angular velocity of the reference frame's axes. The Euler force is typically felt by a person riding a merry-go-round. As the ride starts, the Euler force will be the apparent force pushing the person to the back of the horse, and as the ride comes to a stop, it will be the apparent force pushing the person towards the front of the horse. The Euler force is perpendicular to the centrifugal force and is in the plane of rotation.	
	
	If the point $P$ is rigidly connected to the moving reference frame, an observer in this system will see no movement, therefore no acceleration $\vec{a}_r$. So we are dealing with a system of forces in equilibrium. The dynamics problem is then reduced to a static problem. This is the "\NewTerm{d'Alembert's principle of inertial forces}\index{d'Alembert's principle of inertial forces}".
	\begin{tcolorbox}[colframe=black,colback=white,sharp corners]
	\textbf{{\Large \ding{45}}Example:}\\\\
	We want to study the movement of a frictionless ball or puck on a turntable.\\

	Taking into account Coriolis and centrifugal forces, the fictitious force in the rotating frame of the turntable is as we have proved above:
	
	The coordinate system is defined with respect to the center of the turntable, with the $z$-axis pointing out of its plane. Thus:
	
	and hence:
	
	\end{tcolorbox}
	
	\begin{tcolorbox}[colframe=black,colback=white,sharp corners]
	and:
	
	Therefore we find the fictitious acceleration:
	
	Splitting this up into components, we see that:
	
	Now consider the trick consisting to involve a fake complex coordinates $q=x+\mathrm{i}y$. We multiply:
	
	by $\mathrm{i}$ and add:
	
	So we get:
	
	The auxiliary complex variable $q$ the occurs naturally in the unique differential equation, as follows:
	
	We have proved in the section of Differential and Integral Calculus that the homogeneous solution of this differential equation was depending the discriminant of the polynomial. So here we have:
	
	Therefore the homogeneous solution is according to what we have proved in the section of Differential and Integral Calculus:
	
	 with:
	 
	Following initial conditions, as at time $t=0$ we have $x=0$ and $y=0$ it is same for $q$.
	\end{tcolorbox}
	
	\begin{tcolorbox}[colframe=black,colback=white,sharp corners]
	Then we see immediately that we must have:
	
	Then it remains:
	
	 and as $q=x+\mathrm{i}y$ we have by correspondance:
	 
	 From the initial conditions it is quite obvious that:
	 
	 And we have only a one component initial velocity $v_0=v_{0x}=v_{0y}$ as their is not privileged directions for our reference frame in the turntable of the axis can always be oriented in the direction of the initial velocity.\\
	 
	With Maple 4.00b we get:\\
	
	\texttt{>plot([t*cos(t),-t*sin(t),t=0..6]);}\\
	\begin{figure}[H]
		\centering
		\includegraphics[scale=0.65]{img/mechanics/coriolis_turntable.jpg}
		\caption[]{Coriolis force trajectory of moving item on a turntable}
	\end{figure}
	We we have put on evidence in green the part of the trajectory that most of people know or have seen (even if the remaining part not highlighted really also exists as the reader can see it on many videos on YouTube!!!).
	\end{tcolorbox}
	
	\subsubsection{Coriolis force and deflection magnitude}
	Since almost all our observations are made on Earth, that is to say in a moving frame in the Universe, the Coriolis force can be detected.

	The study of the motion of a body in relation to the Earth is one of the most interesting applications of the composition law proved above. The Earth has an angular velocity (assumed constant!) whose direction is that of the axis of rotation of the Earth. Let us denote by $\vec{g}_0$ the acceleration of gravity measured at a point $A$ on the surface of the Earth if the latter did not rotate. $\vec{g}_0$ corresponds then to $\vec{a}_a$. 

	As we know we have just proved it, we have:
	
	
	where we suppose for the Earth that (it's rotation speed does not change):
	
	Therefore it remains:
	
	We first consider the obvious case of a body initially at rest or moving very slowly so that the Coriolis term is zero or negligible compared to last term. The acceleration that we measure in this case is name the "\NewTerm{effective acceleration}\index{effective acceleration}" of gravity, and we denote the it by $\vec{g}$

	Then it follows:
	
	Assuming that the Earth is a sphere (actually the shape is slightly away as we will see in the section Astronomy) and that there is no local anomalies, we can estimate that $\vec{g}_0$ is directed towards the center of the Earth. The second term $-\vec{\omega}\times(\vec{\omega}\times\vec{r})$ being the centrifugal acceleration is directed outwardly.

	Since $\vec{g}$ is the sum of $\vec{g}_0$ and the centrifugal acceleration, the direction of $\vec{g}$, named sometimes the "vertical direction", deviates actually slightly from the radial direction; this is determined experimentally by a plumb line or also by a liquid recipient as they always maintained in equilibrium with their surface perpendicular to $\vec{g}$.
	
	The magnitude of the centrifugal acceleration is:
	
	where $r$ is Earth's radius.
	
	So when we see the value we understand why it can be neglected in most experiments! However explains most of the observed variations in the value of gravity with latitude!!
	
	It also obvious as the centrifugal force involves the cross product (and the therefore the sinus angle) that the latter is maximum at the equation and minimum at the poles. The centrifugal acceleration also decreases from the equator to the poles because the radius of the Earth is not constant (the Earth is flattened at the poles).
	
	\begin{tcolorbox}[title=Remark,colframe=black,arc=10pt]
	The gradient of the centrifugal acceleration has the effect of slightly moving the radial direction of a body that falls freely: the movement is oriented south in the northern hemisphere and to the north in the southern hemisphere.
	\end{tcolorbox}
	Let us consider nw the Coriolis term. In the case of a falling body, the velocity $\vec{r}_r$ is directed downward. Moreover, as $\vec{\omega}$ lies along the axis of the Earth, $\omega\times\vec{r}_r$ or oriented west. The term Coriolis term $-2m\vec{\omega}\times\vec{v}_r$ is then directed to the east; the falling body will be deflected in that direction as show the figure below.
	
	For a body falling in a parallel and tangential to the surface of the Earth plane, we have:
	\begin{figure}[H]
		\centering
		\includegraphics{img/mechanics/earths_coriolis_force.jpg}
		\caption{Illustration of the Earth's Coriolis force}
	\end{figure}
	It is exactly this phenomenon that is observed in the case of cyclones (we will return on this more in detail during our study of meteorology in the section of Weather and Marine Engineering). A low pressure air zone (relative low pressure) would give radial converging air currents (winds) towards depression if the Earth did not rotate around its axis:
	\begin{figure}[H]
		\centering
		\includegraphics{img/mechanics/coriolis_effect_on_depression.jpg}
		\caption{Generation of cyclones by the Coriolis force}
	\end{figure}
	The Coriolis force due to the rotation of the Earth therefore deflects the North-South winds towards the West and South-North winds East for an observer located at the North Pole. We see therefore the formation of cyclones turning in the counterclockwise direction in the Northern Hemisphere and vice versa in the Southern hemisphere (due to the direction of the vector $\vec{\omega}$ in this part of the hemisphere).
	
	As a second example,let  consider the oscillations of a pendulum (that we will study in detail further below anyway!). For small amplitude oscillations, we can assume that the pendulum is in a horizontal trajectory. If we oscillates the pendulum initially in the North-South direction, the Coriolis force will deflect the pendulum to the right to an observer at the North Pole. In other words, the pendulum turns in the clockwise direction  in the Northern Hemisphere and counterclockwise in the Southern Hemisphere. This observed effect is zero at the equator (perfect parallelism between $\vec{r}$ and $\vec{\omega}$) and maximum to to the Poles.
	
	As a third example let us talk of the swirls that can be observed in the bathtub or sink. It is a legend that the latter rotates differently in different hemispheres. Because the speed and the mass involved are too small to be observable in such objects (see the numerical argument proof further below). In fact, the direction of rotation is due to imperfections (bumps) of the siphon. By cons, if you go in Ecuador, there are students who are happy to show you effect exists with a little experiment using a match. By moving $10$ meters to the North or the South , they will show you the direction of rotation of the siphon changes depending on which hemisphere you are in but in fact this is fake a fake experiment\footnote{Many videos can be found on YouTube}!
	
	Ok now enough speak! Let us come back on pure calculations!!! So let us recall that
	
	In scalar form this gives obviously:
	
	If we assumes that an air parcel, water parcel or any moving object moves at a constant speed $v$ at given latitude $\phi$, then the relation above can be integrated twice with respect to time to get the deflection magnitude $x$, of the horizontal deflection due to Coriolis. A first integration gives:
	
	So finally:
	
	 The justification for the oceanographer's observation that "\textit{...something moving slow can be deflected more than something moving fast...}" can be deduced obviously for the latter relation.
	 
	 We also define:
	
	as being the "\NewTerm{Coriolis parameter}\index{Coriolis parameter}" or "\NewTerm{Coriolis frequency}\index{Coriolis frequency}".
	
	The "\NewTerm{Rossby number}\index{Rossby number}" is a dimensionless ratio of the total acceleration to the Coriolis acceleration as given by:
	
	The Rossby number is greater than $2.0$ in the lower subtropics.  In the middle latitudes the Rossby number is less than $0.1$ on average.  Values of the Rossby number in the middle latitudes are $<0.05$ in the mid troposphere. 
	
	And what about the corresponding force? We know we have:
	
	If we that for data that of 1 [L] of water falling vertically under gravity on a distance of $0.1$ [m] we the Coriolis effect is the stronger, we get:
	
	where for recall $\phi$ is the lattitude (so we take the pole!). Then the magnitude of the coriolis is in absolute value:
	
	So we see that even at the north pole this would not be enough to see water rotate only because of Coriolis force. So we let you imagine at the Equator where $\phi\cong 0$  what is the magnitude of the Coriolis force. Then the videos that we can see on Internet are just fake experimental pseudo proofs!
	
	\pagebreak
	\subsection{Oscillating Movements}
	Oscillation is the repetitive variation, typically in time, of some measure about a central value (often a point of equilibrium) or between two or more different states. The term "\NewTerm{vibration}\index{vibration}" is precisely used to describe mechanical oscillation. Familiar examples of oscillation include a swinging pendulum and alternating current power.
	
	All real-world oscillator systems are thermodynamically irreversible. This means there are dissipative processes such as friction or electrical resistance which continually convert some of the energy stored in the oscillator into heat in the environment. This is named "damping". Thus, oscillations tend to decay with time unless there is some net source of energy into the system.
	
	There is an incredible amount of physical phenomena of this kind. We will deal in this subsection only on the classics from which the developments on more complex phenomena are based in this book (especially for the sections of Wave Quantum Phyisics and Mechanical Engineering).

	For examples, oscillations occur not only in mechanical systems but also in dynamic systems in virtually every area of science: for example the beating human heart, business cycles in economics, predator–prey population cycles in ecology, geothermal geysers in geology, vibrating strings in musical instruments, periodic firing of nerve cells in the brain, and the periodic swelling of Cepheid variable stars in astronomy, molecular vibrations, quantum physics bonds.
	
	We will limit here our study of oscillating movements to pendulums. The others situations will come step by step in the others chapters and respective sections of this book.

	As far as we know there are $9$ well-known pendulum study at school that are (order in which we will study): Newton's cradle, simple pendulum, physical pendulum clock, elastic pendulum, conical pendulum, torsion pendulum, Foucault pendulum, Huygens pendulum.
	
	\subsubsection{Newton's cradle}
	We will not take too much time to describe Newton's cradle. A photo will suffice:
	\begin{figure}[H]
		\centering
		\includegraphics[scale=0.5]{img/mechanics/newton_cradle.jpg}
		\caption{Newton's cradle (pendulum)}
	\end{figure}
	The operating principle is as follows:

	If you throw a ball, at the end a single ball will move. That seems logical and coherent according to the conservation of momentum arising from the conservation of energy as we have already seen.

	A little bit more curious (funny): when we initially launch two balls, two balls moves at the other end!

	The proof is simple and the operation is based on a very simple condition that we are going to determine for the special case of two balls (it's always the same principle for  a higher number of balls):

	Given $p_{1i},p_{2i}$ the linear momentum of the two initial balls, and $p_{1f},p_{2f}$ those of the two ball at the other end, we have:
	
	We then have for the kinetic energy:
	
	after consolidation and simplification of the two previous relations:
	
	From the second relation above we have:
	
	Dividing by the first it remains:
	
	We draw from that:
	
	Let us inject the first of these two relations in:
	
	Then we have:
	
	This gives us after rearranging:
	
	At the end by doing the same for the other final speed, we deduce the expression of the two speeds after the elastic collision:
	
	Hypothesis: Suppose now that by taking one of the balls with $v_{2i}=0$ there are two who go away at the other such as:
	
	and in the latter situation consider the case where all of Newton pendulum balls have the same mass (cases corresponding to that we can found in science shops). So:
	
	We see that our initial assumption is false: if with equal masses, a single ball is launched then, at the other end, only one ball will leave by conservation of momentum (assuming the "elastic collisions")!

	By cons, if we launch two balls in a Newton's cradle composed of identical masses we have after simplification of equations:
	
	two balls which leave at the other end.

	We just need make the make similar development for $3, 4, 5, \ldots$ balls.
	
	\subsubsection{Simple Pendulum}
	In physics, the "\NewTerm{simple pendulum}\index{simple pendulum}" is a point mass attached to the end of a massless wire attached to a friction-less pivot, inextensible and oscillating freely under the effect of uniform gravity. When a pendulum is displaced sideways from its resting, equilibrium position, it is subject to a restoring force due to gravity that will accelerate it back toward the equilibrium position. When released, the restoring force combined with the pendulum's mass causes it to oscillate about the equilibrium position, swinging back and forth. 
	
	This is the simplest model of "\NewTerm{weighted pendulum}\index{eighted pendulum}". It is sometimes named "\NewTerm{ideal gravity pendulum}\index{ideal gravity pendulum}" when we consider the experience in vacuum.
	
	\begin{tcolorbox}[title=Remark,colframe=black,arc=10pt]
	From the first scientific investigations of the pendulum around 1602 by Galileo Galilei, the regular motion of pendulums was used for timekeeping, and was the world's most accurate timekeeping technology until the 1930s. The pendulum clock invented by Christian Huygens in 1658 became the world's standard timekeeper, used in homes and offices for 270 years, and achieved accuracy of about one second per year before it was superseded as a time standard by quartz clocks in the 1930s. Pendulums are also used in scientific instruments such as accelerometers and seismometers. Historically they were used as gravimeters to measure the acceleration of gravity in geophysical surveys, and even as a standard of length.
	\end{tcolorbox}

	Either $T$ the period of time required for a simple pendulum (see figure below) goes through a complete cycle and that we can write:
	
	which is the inverse of the "\NewTerm{natural frequency}\index{natural frequency}" $f$ of the system in the absence of friction.
	\begin{figure}[H]
		\centering
		\includegraphics{img/mechanics/simple_pendulum.jpg}
		\caption{Schematic ideal simple pendulum}
	\end{figure}
	Given $L$ the length of the pendulum rod/wire, and either:
	
	its angular velocity. The velocity of the mass (massive bob) is then:
	
	The sum of the kinetic energy of the pendulum and its gravitational potential energy, measured from the pendulum suspension point (so that when $\theta=0$ the potential energy is zero), is in the case without friction and if the pendulum is not forced:
	
	So the total energy being constant by assumption, its derivative with respect to time is zero. Then we have:
	
	Then we divide by $mL^2\dot{\theta}$ which excludes the solution when the massive pop is stopped. Then we have:
	
	and from the perspective of the units, it is customary to note this in the form of a non-linear second order differential equation of the second of the type:
	
	with therefore:
	
	Who has a physical interpretation in the sense that when $\theta=0$, we have:
	
	That is why$\omega_0$ is considered as the initial pulse.
	
	\begin{tcolorbox}[title=Remark,colframe=black,arc=10pt]
	If the initial pulse depends of the time, the problem is taken down to what is named an "adiabatic pendulum".
	\end{tcolorbox}
	For small oscillations, we know that by Taylor expansion of the sine (\SeeChapter{see section Sequences and Series page \pageref{usual maclaurin developments}}), the differential equation can be written:
	 
	Which has a trivial solution:
	
	And we also deduce from it the frequency and thus the oscillation period:
	
	So we get the famous relation (don't forget it is an approximation!):
	
	So the swing period is independent of the amplitude if the latter is small, which explains why the number of swings per minute of a simple pendulum is constant, whatever the force we put to the swinging ... We talk then of "\NewTerm{isochronism}\index{isochronism}"!
	
	If:
	
	where $l_0$ is the position of the center of mass of the object and $N$ the number of potential links of the chain (rather than a wire) that we would have taken for the length $L$ of the chain and $P$ is the pitch of the chain.

	Which finally gives obviously:
	
	Notice that our differential equation can also be reduced to a differential system of equations of the first order (transformation useful for many computer softwares such as Maple and MATLAB™ for example as they request such an input!):
	
	With Maple 4.00b we can plot thanks to this above relation the phase diagram $\theta-\omega$:

	\texttt{>with(DEtools):}\\
	\texttt{>omega:=2:}\\
	\texttt{>phaseportrait([diff(x(t),t)=y(t),diff(y(t),t)=-omega*sin(x(t))],}\\
	\texttt{[x(t),y(t)],t=-10..20,[[x(0)=-2,y(0)=0],[x(0)=3,y(0)=0],[x(0)=1,y(0)=4]],}\\
	\texttt{x=-4..7,y=-7..7,stepsize=0.1,linecolour=[blue,violet,green]);}
	
	\begin{figure}[H]
		\centering
		\includegraphics[scale=0.75]{img/mechanics/phase_space_simple_pendulum.jpg}
		\caption{Phase space $\theta-\omega$ of the simple pendulum with Maple 4.00b}
	\end{figure}
	We must then know how to read this result.
	\begin{enumerate}
		\item If the starting angle \texttt{x(0)} is small or large and the starting pulsation \texttt{y(0)} is small, the angle remains bounded and the pendulum then has a finite (non-revolving) balanced movement (blue and violet on the plot)

		\item  Whether the starting angle \texttt{x(0)} is small or large if the launching speed \texttt{y(0)} is positive and greater than a certain critical value, then the pendulum has an infinite revolving movement (green on the plot)
	\end{enumerate}
	
	For the people interested in the Lagrangian approach of the simple ideal pendulum we start as always with the Euler-Lagrange equation (\SeeChapter{see section Analytical Mechanics page \pageref{euler lagrange}}):
	
	Taking the potential $V$ to be zero when the pendulum is horizontal $\theta=\pi/2$ and expressing the speed in polar coordinates (\SeeChapter{see section Vector Calculus page \pageref{polar coordinates}}).

	Now we use our equation of motion:
	
	This gives us:
	
	As we have $r=L$, then:
	
	and we fall back on our ODE...
	
	Finally to close this subject on the simple pendulum, some textbooks (especially computer science textbooks????) give another approach based on the fact that the final relation must be given using an Elliptic Integral (it's a matter of choice). For this purpose let us rewrite the total energy at the angle $\alpha_0$ where the kinetic energy is equal to the potential energy (obviously the $\alpha_0$ where the potential energy is equal to the kinetic energy is a the half height):
	
	that can be rewritten as:
	
	therefore:
	
	Now if we integrate from $\alpha=0...\alpha_0$, as we need four times this angle to make a complete cycle (indeed! remember that we were are the half height!), we then have:
	
	Now let us use some trigonometric identities (\SeeChapter{see section Trigonometry page \pageref{remarkable trigonometric identities}}):
	
	By putting:
	
	Then we have:
	
	We now carry out a series of subtle transformations beginning with a change of variable:
	
	So when:
	
	and:
	
 	In addition:
	
 	It follows, since we have put for recall that:
	
	that we then have:
	
	and therefore:
 	
	Then we have:
	
	Either in the condensed form of the elliptic integrals of the first kind (\SeeChapter{see section Integral and Differential Calculus}):
	
	We notice first that if in the integral $p=0$ we fall back on the relation we get earlier:
	
	We speak then of the "harmonic limit" (small initial angle amplitude).
	
	We can make an approximation of $F(\pi/2,p^2)$ at the neighborhood of $p=0$. For this we use the Taylor development (\SeeChapter{see section Sequences and Series page \pageref{usual maclaurin developments}}):
	
	The we can approach the integral for small $x$:
	
	Which gives:
	
	Finally, if we neglect the terms of order $4$, we have $\sin^2(\alpha_0/2)\cong \alpha_0^2/2$ and then we get finally the "\NewTerm{Borda approximation}":
	
	The dependence of the period with the amplitude of the oscillations is therefore quadratic. This an anharmonic effect due to the non-parabolic profile of the potential well in which is trapped the pendulum. However, the Borda formula produces an error greater than $1\%$ as soon as ene exceeds $74^\circ$.
	
	There is - and this is less known - an approximate formula based on the arithmetico-geometric mean, which by far exceeds the performance of most of the other formulas found in the literature.
	
	\pagebreak
	\subsubsection{Physical Pendulum}
	We name "\NewTerm{physical pendulum}\index{physical pendulum}" any solid that can oscillate freely in gravity and in vacuum around a friction-less axis $A$, with a small amplitude ($|\theta_0| < 5^\circ$):
	
	First remember that we have proved during our study of inertia momentum that:	
	
	What we will write here in scalar form as:
	
	where for recall $M$ is the force momentum (torque) that take back the pendulum to its equilibrium position and $J_a$ the moment of inertia of the pendulum from its support axis $A$.
	\begin{figure}[H]
		\centering
		\includegraphics{img/mechanics/physical_pendulum.jpg}
		\caption{Schematic ideal physical pendulum}
	\end{figure}
	By analyzing the forced of our pendulum relatively to its centroid $G$ we get another relation for $M$:
	
	For $\theta-0\leq \theta\leq \theta_0$ and where $d$ is the distance from the friction-less axis of support of the pendulum to its center of mass $G$. The negative term appears here to express the fact that the period decreases with time (because the torque opposes the angular displacement from equilibrium). As the angles $\theta$ are small, we replaced $\sin(\theta)$ without not too serious error by the first term of its development in Taylor series as for the simple pendulum (\SeeChapter{see section Sequences and Series page \pageref{taylor series}}).
	
	Therefore we can write:
	
	Hence the following differential equation of movement:
	
	Or in a more condensed way:
	
	So we have exactly the same type of differential equation as the simple pendulum for which a trivial solution is:
	
	but now with:
	
	Hence:
	
	Expressing the moment of inertia $J_A$ using Steiner's theorem proven earlier during our study of inertial momentum :
	
	and further introducing the radius of gyration $k$:
	
	Hence:
	
	Given $x$ the position of the axis of rotation $A$ measured relatively to any one origin and $a$ the position of center of gravity $G$ with respect to the same origin we have:
	
	as visible in the figure above.
	Therefore:
	
	which gives us then another well know form of the period of the physical pendulum:
	
	As the root square is not comfortable we rise the whole at the square, which ultimately gives us
	
	As we know $x$ and $T$, this relation will allow us to draw a graph to determine the position of $G$ and $k$ if needed.

	Thus, by plotting $T^2$ in function of $x$:
	\begin{figure}[H]
		\centering
		\includegraphics{img/mechanics/physical_pendulum_characteristic_plot.jpg}
		\caption{Illustration of the vertical asymptote for determining the center of mass}
	\end{figure}
	The resulting curve has a vertical asymptote ($T=+\infty$) for $x=a$ and two minima.

	Differentiating $^2$ with respect to $x$ and canceling the derivatives, we find the position of the minima:
	
	
	\subsubsection{Elastic Pendulum (spring pendulum)}
	A "\NewTerm{spring pendulum}\index{spring pendulum}" (also named "\NewTerm{elastic pendulum}\index{elastic pendulum}" or swinging spring) is a physical system where a piece of mass is connected to a spring so that the resulting motion contains elements of a simple pendulum as well as a spring. The system is much more complex than a simple pendulum, as the properties of the spring add an extra dimension of freedom to the system. For example, when the spring compresses, the shorter radius causes the spring to move faster due to the conservation of angular momentum. It is also possible that the spring has a range that is overtaken by the motion of the pendulum, making it practically neutral to the motion of the pendulum.
	
	\paragraph{One degree of freedom elastic pendulum with/without friction}\mbox{}\\\\
	First let us consider the natural oscillations of a solid suspended in an elastic spring as it oscillates. After the separation of the solid from the equilibrium position, it will perform harmonic oscillations in the vertical direction, under the assumption that the elastic spring undergoes deformations proportional to the elongation (tension\label{spring tension}) of the spring.
	
	\begin{figure}[H]
		\centering
		\includegraphics{img/mechanics/elastic_pendulum.jpg}
		\caption{Schematic ideal physical one degree of freedom pendulum}
	\end{figure}
	We will often have to make in this book small movements around an equilibrium position. This type of movement characteristic of what we name a "\NewTerm{harmonic oscillator}\index{harmonic oscillator}\label{harmonic oscillator}" is very common . It generalizes to all kinds of physical situations, such as RLC circuits (\SeeChapter{see section Electrical Engineering page \pageref{rlc circuit}}), the particle and wave quantum model of the atom (\SeeChapter{see section Wave Quantum Physics page \pageref{wave quantum physics}}), quartz resonators or other vibrating structure slightly around its equilibrium point (\SeeChapter{see section Molecular Chemistry page \pageref{molecular vibrations}}).

	We know that the restoring force\label{restoring force classical mechanics} (Hooke's law) of a spring is proportional and opposite to the strain such that (\SeeChapter{see section Continuum Mechanics page \pageref{spring tension}}):
	
	The differential equation of the harmonic oscillator can then be written:
	
	We will take the simple approach that to try a solution, intuitively of the type... (exactly the same type as for the Simple Pendulum and Physical Pendulum):
	
	It is a solution, because indeed:
	
	provided that we take the natural frequency:
	
	So we get the famous period relation for the elastic pendulum:
	
	We also have the "\NewTerm{proper mode}\index{proper mode}":
	
	as a solution.
	A general solution is as we know (\SeeChapter{see section Integral and Differential Calculus page \pageref{first order lde with constant coefficients}}):
	
	To find $A$ and $B$, we must specify the initial conditions. Let us take for example at $t=0$:
	
	Then we have:
	
	Let us now calculate the work (energy) needed to deform the harmonic oscillator. We have\label{potential energy harmonic oscillator}:
	
	Thus, the elastic potential energy in a spring of material constant $k$, which have undergone a deformation $x$ from it's initial position is thus given by:
	
	For a more realistic description, a better modeling, we assume that the oscillator is subjected to an additional force representing friction. Often the approximation in which the friction force is proportional to the speed, and opposite to the speed, is a good approximation. It is not the only possible one and it is not always the best. We'll talk about friction forces (tribology) later.

	Thus we consider a frictional force of the form (not to be confused with the notation of the momentum $\vec{b}$ that has absolutely no relation):
	
	For our coordinate system:
	
	The Newton's second law imposes:
	
	To comply with a standard notation as part of the oscillator's study, we write:
	
	hence the differential equation:
	
	We take the trail function:
	
	Substituting, we find:
	
	As we seek nonzero solutions ($e^{\lambda t}$) we must have:
	
	hence:
	
	and the general solution is:
	
	where the two constants are determined by the initial conditions.

	We will see that they corresponds to a low damping. Indeed, we can write with real square roots:
	
	the general solution can then be written as:
	
	Using the complex (imaginary) properties of exponential and particularly the "Euler's formula" (\SeeChapter{see section Numbers page \pageref{euler formula}}):
	
	Let us choose $A_1=A_2$ and recall that $\sin(-\alpha)=-\sin(\alpha)$ (\SeeChapter{see section Trigonometry page \pageref{trigonometry}}). Therefore:
	
	and as we also $\cos(-\alpha)=\cos(\alpha)$. Then:
	
	Let us put $\omega_1=\sqrt{\omega_0^2-\gamma^2}$ and as the trigonometric function is periodic at $\phi=\pm 2k\pi$ with $k\in\mathbb{N}$ then:
	
	The general appearance of $x(t)$ normalized to the unit is a follows:
	\begin{figure}[H]
		\centering
		\includegraphics{img/mechanics/spring_pendulum_damping.jpg}
		\caption{Illustration of the amplitude damping of an elastic pendulum}
	\end{figure}
	When $\omega_0=\gamma$ we say that there is a "\NewTerm{critical damping}\index{critical damping}", when $\omega_0<\gamma$, we say that there is an "\NewTerm{over-critical damping}\index{over-critical damping}".

	The dimensionless ratio:
	
	meanwhile is named the "\NewTerm{quality factor}\index{quality factor}".
	
	\paragraph{Two degrees of freedom elastic pendulum without friction}\mbox{}\\\\
	Now we consider an elastic pendulum that oscillates in the direction of the spring axes but that also oscillates in a $x-y$ plane:
	\begin{figure}[H]
		\centering
		\includegraphics[scale=0.7]{img/mechanics/elastic_pendulum_oscillating.jpg}
		\caption{Two degrees of freedom ideal elastic pendulum}
	\end{figure}
	In practice the movement of such a pendulum seems chaotic but if we consider a friction-less pivot and that the experiment occurs in vacuum, it is still predictable!
	
	Solving the equation of motion for such a pendulum cannot be done formally as far as we know and get the system of differential equation using the traditional method is quite awful.

	So we will use the lagrangian approach. That is:
	
	and:
	
	The kinetic energy may be broken up into the radial and tangential parts, so we have (the $x$ component is along the axis of the spring relatively to the eqluilibrum length $l_0$ of the spring):
	
	The potential energy comes from both gravity and the spring, so we have:
	
	 The lagrangian is therefore:
	 
	
	\pagebreak
	\subsubsection{Conical Pendulum}
	A "\NewTerm{conical pendulum}\index{conical pendulum}" consists of a mass (or bob) fixed on the end of a string (or rod) suspended from a pivot. Its construction is similar to an ordinary pendulum; however, instead of swinging back and forth, the bob of a conical pendulum moves at a constant speed in a circle with the string (or rod) tracing out a cone. 
	\begin{figure}[H]
		\centering
		\includegraphics{img/mechanics/conical_pendulum.jpg}
		\caption{Ideal conical pendulum}
	\end{figure}
	As we can see above, the conical pendulum consist in taking a mass $m$ considered as punctual and suspended to a rope of length $\overline{\text{O}A}=L$ to a friction-less pivot O.

	The mass being taking to angle angle $\alpha$ of the vertical, the aim of this pendulum is frequently (because it is the simplest case) to determine the dependence between the angle and speed when you consider that trajectories are circular.
	
	The mass $m$ moves around the vertical $\overline{\text{O}C}$, describing a circle of radius:
	
	The following forces act on the mass $m$:
	
	From the figure, we see that:
	
	or, as:
	
	then:
	
	The angle $\alpha$ is even greater than the angular velocity $\omega$ is high, which is confirmed by the experience. For this reason, the conical pendulum was long used as cruise control on steam engines (it closes the steam supply when the speed exceeds a limit set in advance and opens when it falls below this value).

	We have also:
	
	hence after simplification:
	
	As we have obviously $v=2\pi R/t$ (where $t$ is the period to not confuse it with the $T$ in the figure above that represents the tension) it comes:
	
	That is:
	
	For small angles $\theta$, $\cos(\theta)\cong 1$, and the period $t$ of a conical pendulum is equal to the period of an ordinary pendulum of the same length. Also, the period for small angles is approximately independent of changes in the angle $\theta$. This means the period of rotation is approximately independent of the force applied to keep it rotating. This property, named "isochronism", is shared with ordinary pendulums and makes both types of pendulums useful for timekeeping.	
	
	\pagebreak
	\subsubsection{Torsion Pendulum}
	Consider a disk suspended from a torsion wire attached to its center (see figure below). This setup is known as a "\NewTerm{torsion pendulum}\index{torsion pendulum}". A torsion wire is essentially inextensible, but is free to twist about its axis. Of course, as the wire twists it also causes the disk attached to it to rotate in the horizontal plane. Let $\theta$ be the angle of rotation of the disk, and let $\theta=0$ correspond to the case in which the wire is untwisted.
	
	\begin{tcolorbox}[title=Remark,colframe=black,arc=10pt]
	The torsion pendulum is also a system that was used by Charles-Augustin Coulomb for measuring the elementary charge $q_e$ and Henry Cavendish for measuring the gravitational constant $G$.
	\end{tcolorbox}
	
	\begin{figure}[H]
		\centering
		\includegraphics{img/mechanics/torsion_pendulum.jpg}
		\caption{Ideal torsion pendulum}
	\end{figure}
	Any twisting of the wire is inevitably associated with mechanical deformation. The wire resists such deformation by developing a restoring torque, $M$, which acts to restore the wire to its untwisted state. For relatively small angles of twist, the magnitude of this torque is assumed directly proportional to the twist angle. Hence, we can write:
	
	where $k$ is the "\NewTerm{torsion constant}\index{torsion constant}" of that particular material wire (\SeeChapter{see section Mechanical Engineering page \pageref{torsion}}).

	So we have:
	
	Hence the differential equation:
	
	By analogy with the simple pendulum and physical pendulum where we had a similar differential equation to a given factor, we get:
	
	We conclude that when a torsion pendulum is perturbed from its equilibrium state (i.e., $\theta=0$), it executes torsional oscillations about this state at a fixed frequency, $\omega$, which depends only on the torque constant of the wire and the moment of inertia of the disk. Note, in particular, that the frequency is independent of the amplitude of the oscillation (provided $\theta$ remains small enough). 

	Torsion pendulums are often used for time-keeping purposes. For instance, the balance wheel in a mechanical wristwatch is a torsion pendulum in which the restoring torque is provided by a coiled spring.
	
	\subsubsection{Foucault's Pendulum}
	The Foucault pendulum is a great experience to account for the rotation of the Earth.  While it had long been known that the Earth rotates, the introduction of the Foucault pendulum in 1851 was the first simple proof of the rotation in an easy-to-see experiment. Today, Foucault pendulums are popular displays in science museums and universities.

	There are several mathematical methods to analyze the behavior of the Foucault pendulum. We have chosen to present the simplest one which requires little computational pages.

	But first let us give a little explanatory text may be relevant as this experience is important and so well known.

	Foucault's experiment aims to demonstrate that the Earth rotates on itself as we have just say. We push a pendulum (a mass after a wire). He moved back and forth regularly in the same direction. If we take it in a car and we do not turn too sharply, the pendulum don't care of the turns you take with the car: he continues to fight in the same direction. This is because a simple pendulum always oscillate in the same plane, despite the movements of its support.

	This is why the French physicist Léon Foucault had the idea of attaching a heavy pendulum to a $67$ meters long wire under the dome of the Pantheon, in the presence of Napoleon III and some scholars:
	\begin{figure}[H]
		\centering
		\includegraphics[scale=0.87]{img/mechanics/foucault_pendulum.jpg}
		\caption[Schematic view of Foucault's pendulum]{Schematic view of Foucault's pendulum (source: ?)}
	\end{figure}
	At each of its whereabouts, the pendulum came modify a pile of sand where it left a mark. But the track was never in the same place: there were $3$ to $4$ millimeters of difference between a swing and the next one, $16$ seconds later. The pendulum remained in the same plane, but the Pantheon, Paris, the Earth are revolving!
	
	To introduce the maths, consider the figure below:
	\begin{figure}[H]
		\centering
		\includegraphics[scale=0.87]{img/mechanics/foucault_pendulum_study_diagram.jpg}
		\caption[]{Study diagram of Foucault's pendulum}
	\end{figure}
	We consider that this is the view from a geocentric reference (Earth) seen in section on a plane containing the axis of rotation.

	The size of the pendulum is obviously exaggerated in the figure above. However, it still oscillates in a meridian plane, between $A$ and $B$ (a terrestrial observer sees the line $\overline{AB}$ rotate relative to the Earth's surface by the green circle, seen in perspective, in the reverse direction).
	
	Given $T$ the rotation period of $A$ (or $B$).  The speed of $A$ ($v_a$), on this circle is due to the fact that in the geocentric frame of reference, the point $M$, to the vertical of the point of suspension at the latitude $\lambda$, and the point on the Earth's surface coinciding with $A$ at a given instant, does not have the same speed in the geocentric frame of reference: the point $M$ being furthest from the axis of rotation of the earth it is faster than $A$ (same for the speed $B$ that is higher than the speed of $M$).

	Indeed, as $r_B>r_M>r_A$ and that $v=\omega R$, $v_B>v_M>v_A$.
	
	The difference is of these speeds can easily be calculated assuming that rotation Earth's reasoning is uniform over a period of one day (sidereal) $T_0$.

	We know that:
	
	From this it easily follows that:
	
	Since in the triangle $AHM$ we have:
	
	Then:
	
	But we know $\Delta v$ is just on green circle:
	
	Therefore by equating and simplifying we get:
	
	\begin{tcolorbox}[title=Remark,colframe=black,arc=10pt]
	The direction of rotation of the pendulum is clockwise for an observer  in the Northern Hemisphere; and counterclockwise for an observer in the Southern Hemisphere.
	\end{tcolorbox}
	\begin{tcolorbox}[colframe=black,colback=white,sharp corners]
	\textbf{{\Large \ding{45}}Example:}\\\\
	The period of the Pantheon pendulum is $16.5$ [s], the maximum amplitude of $6$ [m] and the damping time of $6$ [h]. We can observe a displacement of several millimeters at each return of the pendulum.
	\begin{figure}[H]
		\centering
		\includegraphics[scale=0.75]{img/mechanics/foucault_pendulum_photo_pendulum.jpg}
		\caption[]{Photo of Foucault's pendulum in Pantheon}
	\end{figure}
	At the poles (where the angle is $\pi/2$ [rad] and the sinus is equal to $1$), the period of the pendulum equals that of the Earth and is therefore $24$ [h]. At the equator (where the angle is equal to 0 [rad] and the sinus is equal to $0$), the period of rotation of the oscillation plane is infinite: the plane of oscillation is stationary relative to the Earth! A Paris, where the latitude angle is equal $48^{\circ} 52'$ we have:
	
	That is to say in degrees: $11^{\circ}\;[\text{h}^{-1}]$. And this corresponds to the value given on Internet relatively to the Foucault's pendulum at the Pantheon!
	\end{tcolorbox}
	However, the importance of Foucault's pendulum is elsewhere ...

	The pendulum oscillation plane is fixed in reality fixed and it is the rotation of the Earth on itself that result in an apparent rotation. But finally ... what is the reference system?

	Indeed, all motion are relative. If the Earth is rotating, it because it's compared to something else. We can not talk about a movement without defining a frame of reference. The question that therefore arises is the to know relatively to what reference frame the pendulum oscillates in a fixed plane???
	
	The first idea that comes to mind is to say that the pendulum plane is fixed relative to the Sun. But if Foucault had managed to build a pendulum capable of oscillating long enough, say for a month, he would notice that the plane of oscillation also derived from the position of the Sun. Our star is therefore not part of the reference frame we are looking for!

	Perhaps it is then necessary to consider the stars near the Sun? But again, if the experience could last long enough, it would show that the plane of oscillation clearly moves relative to the stars after a few years. What object choose then? The galactic center, the Andromeda galaxy, the local group, the local superclusters? Each of these objects give the illusion of being stationary relative to the plane of the oscillations, but would end, after an increasingly long time, by revealing a drift.

	Finally, as a last resort, we can consider the most distant objects, galaxies or quasars located billions of light-years away. With this reference system, and if the experience of Foucault was feasible, the plan of oscillation would finally be fixed and there would be no drift. So it is that in considering the most distant objects in fact the observable Universe as a whole, that we can get a reference frame in which the plane of oscillation stabilizes.
	
	Foucault's pendulum then don't care of the presence of the Earth, the Sun or the Galaxy. A part of its movement is directly dictated by the Universe as a whole. This experience highlights a kind of mysterious link between each point and the entire Universe. Until further notice, the nature of that relationship is unknown.

	A similar conclusion was drawn by the Austrian physicist Ernst Mach in the late nineteenth century (we will talk of the "Mach's principle" in the section of Special Relativity).

	According to Newtonian physics, the product of the mass of a body by its acceleration is equal to the force applied on it. Therefore, for a given force, the more massive an object, the more its acceleration is low. From this point of view, the mass is therefore a measure of the inertia of the body, that is to say its ability to resist to a force.
	
	Now suppose that all matter in the Universe disappears, except for that body. The latter is then completely isolated and no force is exerted on it. This means, according to Newtonian physics, the product of its mass by its acceleration is zero. But the acceleration can not be equal ot zero. Indeed, as all matter in the Universe is gone, there is no reference system against which to set the speed or acceleration. The latter is indefinite and not zero. From a mathematical point of view, there is only one possibility, that the mass of the body is zero.

	This reasoning shows that the mass and inertia of a body are not really the properties of the object itself, but rather the result of an interaction with the rest of the Universe. Just as the Foucault pendulum, Mach's principle shows us that there must be some sort of connection between the local properties of a body and the global properties of the Universe. As in the previous case, the nature of this mysterious connection need to be determined.
	
	\pagebreak
	\subsubsection{Huygens' Pendulum (and brachistochrone curve)}\label{brachistrochrone}
	We seek to build a pendulum whose period is independent of the amplitude (and not just by approximation as we have seen above!), that it is customary to name "\NewTerm{rigorous isochronism}\index{rigorous isochronism}". For this we have two cycloidal lamella at symmetrical positions and determined as shown in the figure below:
	\begin{figure}[H]
		\centering
		\includegraphics[scale=0.85]{img/mechanics/huyghens_pendulum.jpg}
		\caption{Schematic Huygens' pendulum}
	\end{figure}
	From this moment such pendulum applied to the measurement of time are known as "Coster clock",  and at present all of these clocks manufactured under Huygen's license are in museums (the oldest known Huygens-style pendulum clock is dated 1657 and can be seen at the Museum Boerhaave in Leiden) or in private collections.
	\begin{figure}[H]
		\centering
		\includegraphics[scale=0.8]{img/mechanics/huyghens_pendulum_coster_clock.jpg}
		\caption[Reproduction of a coster clock]{Reproduction of a coster clock (source: Lamazares Gil)}
	\end{figure}
	The choice of the cycloid\index{cycloid} (also named "orthocycloïde\index{orthocycloïde}") is because it is a "\NewTerm{brachistochrone curve}\index{brachistochrone curve}" (see definition below) and since the work of Christian Huygens in 1659, we also know that it is a "\NewTerm{tautochronous curve}\index{tautochronous curve}" or "\NewTerm{isochrone curve}\index{isochrone curve}" (some pendulums in modern watches have traditionally this form). That is to say that the bodies that fall into an inverted cycloid reach the lowest point at the same time, in any height they begin to fall (and for the pendulum independently of the starting amplitude they will reach the lowest point a the same time!).

	So contrary to popular belief, the fastest way for a falling bodyon a solid support is not the straight line!

	Indeed, one of the most famous problems of the history of mathematics is the "brachistochrone problem" which is to find the curve along which a particle would slide from one point to another in the shortest time being subject to a uniform field of gravity. This problem was posed by Johann Bernoulli in 1696 as a challenge for mathematicians of his day (and that was a really important challenge!!!). The solution was found by Johann Bernoulli himself and his brother Jacques Bernoulli, Isaac Newton, Gottfried Wilhelm  Leibniz and the Marquis de l'Hospital. The brachistochrone  problem is important in the development of mathematics and proves to be a major application of the method of variation calculus.

	To introduce the corresponding maths we consider the two points $A$ and $B$ in the gravity field a material point $m$ moving without friction on a curve of ends $A$ and $B$. As already mention we want to determine the curve, named "brachistochrone", wherein the travel time is minimal when the point $m$ starts from the point with a zero velocity.

	Let us consider the figure below:
	\begin{figure}[H]
		\centering
		\includegraphics{img/mechanics/brachystochrone_schema_study.jpg}
		\caption{Any way to attack the brachistochrone problem without bias}
	\end{figure}
	At the $x$-coordinate on the path plot above, the lost potential energy is $E_p=mgy$, equivalent to the kinetic energy gained by the material point from the start such as:
	
	Hence without too many surprises:
	
	The speed $v$ is measured along the curve so that we have to rewrite the expression in horizontal and vertical components:

	We will put that $s$ is the curvilinear abscissa, $\mathrm{d}s$ the increase of this distance along the curve (\SeeChapter{see section Differential Geometry page \pageref{parametric curves}}). The quantities $\mathrm{d}x$ and  $\mathrm{d}y$ are the horizontal and vertical components of $\mathrm{d}s$.

	Therefore:
	\begin{itemize}
		\item $\mathrm{d}s/\mathrm{d}t$ represents the speed along the curve

		\item  $\mathrm{d}x/\mathrm{d}t$ the $x$ component of the speed

		\item $\mathrm{d}x$ and $\mathrm{d}y$ are given by the Pythagorean theorem in exactly the same way as we did during our study of the Lagrangian formalism in the section of Analytical Mechanics:
		
	\end{itemize}
	By inserting the obtained equation from the principles of dynamics:
	
	A simple integration then gives us the expression $t$ to minimize:
	
	We end up the with a function similar to the one we had in our study of a practical case of the Lagrangian formalism.

	The challenge now is to find the minimum reached by $t$ among all possible functions $y(x)$ satisfying the constraints:
	
	The fundamental problem said to be of the "\NewTerm{calculus of variations}\index{calculus of variations}" is to search among all continuously differentiable functions $y=f(x)$ over a given interval $[A, B]$ for which the functions $(A)$ and $f(B)$ are given known values, which make maximum or minimum the previous integral.

	To apply this method, we start from the Euler-Lagrange equation (\SeeChapter{see section Analytical Mechanics page \pageref{euler lagrange}}):
	
	which gives the extremum of the integral.

	Similarly to an example that we saw during our study of the Lagrangian formalism (\SeeChapter{see section Analytical Mechanics page \pageref{euler lagrange example shortest path}}) we put in our case:
	
	Therefore:
	
	We wish to inject the last two relations in the Euler-Lagrange equation:
	
	but we anticipate relatively easily that the derivative with respect to $x$ will give us an indigestible monster to simplify (I tried ... but I have the excuse of not being good at math).

	We will then use the Beltrami's identity proved in the section of Analytical Mechanics that is written with the above selected notation:
	
	that we have the right to do as the Beltrami 's condition here is satisfied:
	
	Which gives us:
	
	Thus after rearranging:
	
	So finally:
	
	Or written in another way:
	
	where the constant $K$ is only $-D$, the maximal point reached by the mobile. More explicitly:
	
	We must solve this differential equation to find the function that provides the fastest path. Let's us focus on it in the following form:
	
	and let us recall that on the figure above we have:
	
	Then we have (\SeeChapter{see section Trigonometry page \pageref{remarkable trigonometric identities}}):
	
	Hence:
	
	Note that we have (still using trigonometric identities):
	
	Therefore:
	
	Which is easy to integrate (\SeeChapter{see section Differential and Integral Calculus page \pageref{usual primitives}}):
	
	So we have at this point (for recall):
	So we have at this point (for recall):
	
	Let us put to simplify the writing $\alpha=2\theta$, we have then:
	
	Let us recall now that we have the initial condition:
	
	and as $K$ is not zero, this requires that in the prior previous relation that:
	
	and therefore that:
	
	Therefore, we must have:
	
	and we deduce immediately that:
	
	Since then:
	
	To close that subject, let us do the traditional variable change:
	
	It comes then:
	
	So in the end:
	
	As the sign of the coefficient $a$ of $x$ that has for only effect a translation alogn the $X$ axis, it is customary to represent that pair of equations in the following form where we recognize the parametric equations of a cycloid curve (\SeeChapter{see section Analytical Geometry page \pageref{cycloid curve}}):
	
	By putting $a=1$ we have in Maple 4.00b:
	
	 \texttt{>plot([theta-sin(theta),1-cos(theta),theta=0..6*Pi]);}
	 \begin{figure}[H]
		\centering
		\includegraphics{img/mechanics/cycloid_huyghens.jpg}
	\end{figure}
	This has to be compared with the intuitive answers that humans being gives when they try to guess the fastest curve (in red below the brachistochrone curve and in blue... most human beings answers):
	\begin{figure}[H]
		\centering
		\includegraphics{img/mechanics/brachystochrone_fastest_curve.jpg}
		\caption[Comparison of the brachistochrone with other intuitive paths]{Comparison of the brachistochrone with other intuitive paths (source: Wikipedia)}
	\end{figure}
	Given the derivatives:
	
	Therefore:
	
	The conservation of energy:
	
	is then written as we know:
	
	hence:
	
	Therefore the time required to go from top to bottom of the cycloid described the Huygens' pendulum is:
	
	This term therefore only dependent on fixed parameters and not of the amplitude of the pendulum!
	
	The statement in 1696 of the brachistochrone the problem can be considered as the true birth of the calculus of variations, because it is this problem which prompted the search for general methods gradually developed in the context of a real scientific (and business...) competition.
	\begin{tcolorbox}[title=Remark,colframe=black,arc=10pt]
	The brachistochrone of a surface is a curve on which a material  punctual mass slide without friction when placed in a uniform gravitational field so that travel time is minimal among all curves joining two fixed points. In other words, the brachistochrone  is the shortest lines in time, while the geodesic (\SeeChapter{see section Tensor Calculus page \pageref{geodesic equation}}) are the curves of the shortest distance for a given fixed time interval!
	\end{tcolorbox}
	
	\subsubsection{Double Pendulum}
	A double pendulum is a pendulum with another pendulum attached to its end, and is a simple physical system that exhibits rich dynamic behavior with a strong sensitivity to initial conditions.

	The double pendulum problem is a classic application of the Lagrangian formalism and of chaos theory and so verbatim a nice school example of nonlinear physics.

	Here is a figure of how it is often described (the $x$ and $y$ are positive on both axes represented below!):
	\begin{figure}[H]
		\centering
		\includegraphics{img/mechanics/double_pendulum.jpg}
		\caption{Illustration of the ideal double pendulum}
	\end{figure}
	The position of the mass $m_1$ will be given by the coordinates $(x_1,y_1)$ and that of the mass $m_2$ - linked ot $m_1$ and therefore not independent - will be given by the coordinates $(x_2,y_2)$.

	The best is to reduce the study of the system through the generalized coordinates $(\theta_1,\theta_2)$ trough the transformations:
	
	and:
	
	So in Cartesian coordinates and using the traditional notation of the Lagrangian formalism, the kinetic energy is then:
	
	and as:
	
	and:
	
	Then we have:
	
	For the potential energy we have (the sign is negative because the masses are below the point $0$):
	
	The Lagrangian (\SeeChapter{see section Analytical Mechanics page \pageref{lagrangian mechanics}}) is therefore given by:
	
	We now use the Euler-Lagrange equation with the chosen generalized coordinates:
	
	We then have for $\theta_1$:
	
	So we have ultimately for the Euler-Lagrange equation with the generalized coordinate $\theta_1$:
	
	We have also for $\theta_2$:
	
	So we have ultimately for the Euler-Lagrange equation with the generalized coordinate $\theta_2$:
	
	This gives us the finally the following system of differential equations:
	
	
	With Maple 4.00b this gives for the representation of the coordinates of $m_2$ over time:
	
	\texttt{> with(plots):\\
	> with(plottools):\\
	> Eq1:=(m1+m2)*l1\string^2*diff(diff(theta1(t),t),t)+m2*l1*l2*cos(theta1(t)-theta2(t))\\
	*diff(diff(theta2(t),t),t)+m2*l1*l2*sin(theta1(t)-theta2(t))*diff(theta2(t),t)\string^2\\
	+(m1+m2)*g*l1*sin(theta1(t))=0;\\
	> Eq2:=m2*l1*l2*cos(theta1(t)-theta2(t))*diff(diff(theta1(t),t),t)+m2*l2\string^2\\
	*diff(diff(theta2(t),t),t)-m2*l1*l2*sin(theta1(t)-theta2(t))*diff(theta1(t),t)\string^2\\
	+m2*g*l2*sin(theta2(t))=0;\\
	> m1:=2;m2:=3;l1:=6;l2:=4;g:=9.81;\\
	> ff:=dsolve({Eq1,Eq2,theta1(0)=0.5,D(theta1)(0)=4,theta2(0)=1,D(theta2)(0)=\\
	-2},{theta1(t),theta2(t)},type=numeric,output=listprocedure);\\
	> Theta1:=subs(ff,theta1(t));Theta2:=subs(ff,theta2(t));\\
	> X1:=t->l1*sin(Theta1(t));\\
	> Y1:=t->l1*cos(Theta1(t));\\
	> X2:=t->l1*sin(Theta1(t))+l2*sin(Theta2(t));\\
	> Y2:=t->l1*cos(Theta1(t))+l2*cos(Theta2(t));\\
	> plot([Y2,X2,0..100],numpoints=100);}
	\begin{figure}[H]
		\centering
		\includegraphics{img/mechanics/double_pendulum_maple.jpg}
		\caption{Plot  with Maple 4.00b of the double pendulum external mass}
	\end{figure}
	
	And in real life:
	\begin{figure}[H]
		\centering
		\includegraphics{img/mechanics/double_pendulum_irl.jpg}
		\caption[Long exposure of double pendulum exhibiting chaotic motion tracked with an LED]{Long exposure of double pendulum exhibiting chaotic motion tracked with an LED (source: Wikipedia)}
	\end{figure}
	
	\pagebreak
	\subsubsection{Inverted Pendulum}
	An inverted pendulum is a pendulum that has its center of mass above its pivot point. It is often implemented with the pivot point mounted on a cart that can move horizontally as shown in the photo below:
	\begin{figure}[H]
		\centering
		\includegraphics[scale=0.3]{img/mechanics/segway.jpg}
		\caption{A famous business application (SEGWAY™) of mobile inverted pendulum}
	\end{figure}
	Most applications limit the pendulum to $1$ degree of freedom by affixing the pole to an axis of rotation. Whereas a normal pendulum is stable when hanging downwards, an inverted pendulum is inherently unstable, and must be actively balanced in order to remain upright; this can be done either by applying a torque at the pivot point, by moving the pivot point horizontally as part of a feedback system, changing the rate of rotation of a mass mounted on the pendulum on an axis parallel to the pivot axis and thereby generating a net torque on the pendulum, or by oscillating the pivot point vertically. A simple demonstration of moving the pivot point in a feedback system is achieved by balancing an upturned broomstick on the end of one's finger. The inverted pendulum is a classic problem in dynamics and control theory and is used as a benchmark for testing control strategies. Arguably the most prevalent example of an inverted pendulum is a human being. A person with an upright body needs to make adjustments constantly to maintain balance whether standing, walking, or running.
	
	The inverted pendulum is a superb academic example of the application of Lagrangian mechanics to another coupled system than the that of the double pendulum seen previously. Moreover, we can make a very good parallel with the watch systems as they are used in many R\&D departments of watch manufacturers.

	For the mathematical study let us consider the following experiment (The rod is considered massless!):
	\begin{figure}[H]
		\centering
		\includegraphics[scale=1]{img/mechanics/inverted_pendulum.jpg}
		\caption{A schematic drawing of the inverted pendulum on a cart.}
	\end{figure}
	The position of the mass base carriage $M$ is $x_1$, the angle of the rod of length $L$ is $\theta$, the scalar force applied on the base of carriage is $F$. The coordinates of the weight at the extremity of the rod will be $(x_2,y_2)$.

	We will deduce the system of differential equations which describes the inverse pendulum dynamics using the Lagrange equations (\SeeChapter{see section Analytical/Lagrangian Mechanics page \pageref{lagrangian mechanics}}).

	The kinetic energy of the carriage and the extreme weight of the rod are therefore respectively:
	
	With:
	
 	such that:
	
 	Thus the total kinetic energy is written:
	
	The potential energy will be considered taking into account only the mass at the extremity of the rod (the carriage being at the level of the ground and the mass of the rod being considered negligible):
	
	The classical Lagrangian is then:
	
 	We will take as generalized coordinates of the system: $(x_1,\theta)$.

which are the variables we have just seen as sufficient to describe the state of the Lagrangian.

	Therefore, the Lagrange equations that we have are (compared to the usual writing we have multiplied each line by $-1$):
	
	We inject the preceding results into respectively each of the equations to obtain by already directly calculating the partial derivative:
	
 	and therefore:
	
	Let us now write this system in matrix form:
	
 	It is a typical Lagrangian system with the square inertial matrix of dimension $2$ multiplied by the acceleration vector. The term $mL\theta\dot{\theta}^2\sin(\theta)$ is a centripetal force whereas the term $mgL\sin(\theta)$ is the term of gravity.

	We solve this system by inverting the square inertia matrix of dimension $2$ as we have demonstrated in the section of Linear Algebra and we get:
	
	
	We inject the preceding results into respectively each of the equations to obtain by already directly calculating the partial derivative:
	
 	and therefore:
	
	Let us now write this system in matrix form:
	
 	It is a typical Lagrangian system with the square inertial matrix of dimension $2$ multiplied by the acceleration vector. The term $mL\theta\dot{\theta}^2\sin(\theta)$ is a centripetal force whereas the term $mgL\sin(\theta)$ is the term of gravity.

	We solve this system by inverting the square inertia matrix of dimension $2$ as we have demonstrated in the section of Linear Algebra and we get:
	
	Now, the state of the system would be adequately described in the phase space by the following variables:
	
	We can obviously simplify this expression by making first-order Taylor developments for small angles. What should follow is only the application of iterative numerical methods.
	
	\pagebreak
	\subsection{Tribology}
	\textbf{Definitions (\#\mydef):} The "\NewTerm{Tribology}\index{tribology}" is the science of friction (very intuitive concept to everyone because we can feel its effects in everyday life) that occur when two contacting surfaces are moved relative to each other producing a force that opposes to the motion. This field has for purpose to study not only the conditions of friction between the bodies but also the various phenomena and physical and technical aspects in relative movement of the outer surfaces of the bodies. This field is very important in in  Manufacturing Technology and Aeronautics.

	Most of these phenomena related to friction can be understood as a first approximation based on the phenomenological laws of friction set in the 18th century by Guillaume Amontons and Charles August Coulomb (but already highlighted by Leonardo Da Vinci 200 years ago) from the concept of coefficient of friction.

	The analysis of a tribological system introduced $4$ fundamental components: 
	\begin{enumerate}
		\item The body subject to the study (this is usually a solid)
		
		\item The support body on which acts directly the studied body (it can also be a solid, liquid or gas)
	
		\item The intermediate material interposed between the two preceding bodies
	
		\item And finally the environment.
	\end{enumerate}
	Guillaume Amontons and Charles August Coulomb already observed two types of friction a priori distinct:
	\begin{enumerate}
		\item The "\NewTerm{static friction}\index{static friction}" is the one who shows a resistance to the movement when object put on a planar surface is about the limit to slip and when we imposed to it an tensile force $\vec{F}_t$ (tangent to the plane). This opposition to the tensile force is experimentally measured as being proportional to the weight $\vec{P}$ of the object.

		But comes a limit value of the tangential tensil force from which the object starts to slip. This is what we note:
		
		where $\vec{F}_{T,\text{lim}}$ is the tensile limit force to move the initially static object, $\mu_S$ is the "\NewTerm{static friction coefficient}\index{static friction coefficient}" that is dimensionless and expresses the proportionality of the limit  friction force with the weight $\vec{P}$ or of its normal shape relatively to the application plane. This is why this last relation is often denoted:
		
		\begin{tcolorbox}[title=Remark,colframe=black,arc=10pt]
		In practice, it is very easy to determine the coefficient with a simple dynamometer to determine the limit force and a balance to determine the weight of the object studied.
		\end{tcolorbox}
		We experimentally observe that contrary to the intuition, the tensile limit force is in first approximation independent of the contact surface between the object and the support (within current physical case obviously because the small the contact surface, the higher the pressure is for a same weight, then the contact surface may become plastic at high pressures...!).
		
		So we can now sate the both Amonton's law:
		\begin{itemize}
			\item "\NewTerm{Amontons' first law of friction}\index{Amontons' first law of friction}" also named "\NewTerm{Leonardo da Vinci's law of friction}\index{Leonardo da Vinci's law of friction}" state that the friction force experienced by two dry bodies whose planar surfaces are in contact is independent of the area of contact between the bodies.
			
			\item "\NewTerm{Amontons' second law of friction}\index{Amontons' second law of friction}" state that the friction force (limiting value for static friction, actual value for kinetic friction) is proportional to the normal force.
		\end{itemize}
		
		In other words, if one kilo of sugar is placed on a table. To move the object, of weight $\vec{P}$ (mass multiplied by the gravitational constant), it must applied a force $\vec{F}$ parallel to the table surface. But experience shows that this object will not move until the force $\vec{F}_T$ is less than a minimum force $\vec{F}_{T,\text{lim}}$. And Amontons and Coulomb showed that this minimum force is directly proportional via a coefficient of static friction to the weight.

		We can detail the approach above relation by imagining two surfaces having roughness in saw-tooth of an angle $\alpha$ and nested together:
		\begin{figure}[H]
			\centering
			\includegraphics{img/mechanics/sawtooth_friction_model.jpg}
			\caption{Amonotons-Coulomb friction saw-tooth model}
		\end{figure}
		If we apply a normal force corresponding to the weight $\vec{P}$ and a horizontal force $\vec{F}_T$ we have because of the saw-tooth in the limiting case the following situation:
		\begin{figure}[H]
			\centering
			\includegraphics{img/mechanics/sawtooth_friction_model_diagram_forces.jpg}
			\caption[]{Amonotons-Coulomb friction saw-tooth free-body diagram}
		\end{figure}
		then we see well that the movable piece will begin to move when there will be slip, that is when:
		
		As simplistic as it is, this approach allows to relate the (static) friction to the characteristics of roughness. Furthermore typical experimental values of the static coefficients friction of the order of $0.3$ correspond surface roughness slope of the order of $15$-$20$ degrees, which is quite compatible with the typical characteristics that can be measured for roughness of surfaces!
		
		However, this argument is based on an implicit assumption: the perfect interlocking between the roughness of both surfaces. We speak in this case of "\NewTerm{commensurable surfaces}\index{commensurable surfaces}". This is of course not the case in general in the Nature: even at the atomic scale, two ideal surfaces are slightly different in interatomic distance that prevent nesting or the presence of air avoid a perfect interlocking. A slight disparity is enough to make very irregular the distribution of contact points  between the two surfaces unlike the commensurable case. We speak then of "\NewTerm{incommensurable surfaces}\index{incommensurable surfaces}".
		
		\item  The "\NewTerm{dynamic friction}\index{dynamic friction}" is the one who resists when an object placed on a plane is already slipping. This opposition to traction force is also experimentally again proportional to the weight of the object such that:
		
		but with the "\NewTerm{dynamic friction coefficient}\index{dynamic friction coefficient}" $\mu_D$ (which exists in several subfamilies: rolling friction coefficient, sliding friction coefficient, ...) which is usually much smaller than the coefficient of static friction:
		
		So the above relation states that for two dry solid surfaces sliding against one another, the magnitude of the kinetic friction exerted through the surface is independent of the magnitude of the velocity (i.e., the speed) of the slipping of the surfaces against each other.
		
		So the friction is not the same before our object start to slip than after he start slipping. This corresponds very well with our daily experience of friction (when moving furniture in our homes for examples).

		Again, we experimentally observe that contrary to intuition, the pulling force is again independent of the contact area between the object and its support.

		Thus, if we a packet of one kilo of sugar on one of its face or on its edge, assuming that at the atomic level its properties are the same everywhere, the friction force is the same (if the surface quality is the same from all sides of the packet of sugar)!
		
		Another quite surprising fact relates to the typical values of the both coefficients of friction, which deviates relatively little $\mu_D\cong \mu_S\cong 0.3$, for very different surfaces from each other. The technologies, however, enables the design of surfaces with coefficients of friction that are much smaller ($\mu_D\cong \mu_S\cong 0.001$) or larger ($\mu_D\cong \mu_S\gg 1$).
	\end{enumerate}
	The two above relation relating pressure and pull force are name "\NewTerm{Coulomb's friction laws}\index{Coulomb's friction laws}", and are traditionally written:
	
	and with the following notation in many contemporary high-school textbooks:
	
	where $F_N$ is the normal force to the support, $P$ is the pressure (weight) and $A$ the contact surface.
	
	The naive origin of friction between two solids is therefore due to the fact that:
	\begin{itemize}
		\item Any solid is never smooth but has asperities which makes the contact surface rough (asperities overlap partially or not and cause more or less friction).

		\item Impurities between the two contact surfaces are often more important in terms of friction sources than the surface asperities imbrication.

		\item Friction is weakly dependent on the surface area because the roughness at the atomic scale is such that only a very small percentage of the total area of the two objects are actually in contact (actual contact area is therefore much smaller than the apparent contact surface) which explains that the tangential traction force is proportional to the weight because it forces the actual contact area to increase.
	\end{itemize}
	\begin{tcolorbox}[title=Remark,colframe=black,arc=10pt]
	When two metals (pure - and clean...) OF identical in nature are in contact trough a smooth surface in a favorable environment (at best: in vacuum) when we can sometimes observe a "\NewTerm{cold welding}\index{cold welding}" (in vacuum it's a systematic and not just "sometimes") which is a case of extreme friction because sometimes it becomes almost impossible (or simply just impossible!) to then move one of the elements relative to the other (the result also depends slightly on the initial pressure applied to develop cold welding on the two objects of the same type). Cold welding was recognized as a general physical phenomenon of the materials science in the 1940s. The reason of this unexpected behavior is that when the atoms in contact are all of the same nature, they have no way to "know "they are present in different metal parts. When there are other atoms in the oxides and greases and other kinds of surface contaminants, the atoms "know" that they do not belong to the same object (there are widely used industrial adhesives using this principle for emergency repairs...).
	\end{tcolorbox}
	
	\pagebreak
	Now to see a famous case of treatment of static and dynamic friction let us consider the banked curve situation!
	
    First, the question that appears quite quickly is: What provides the centripetal force on the car for a banked curve? It is the Normal force? Gravity?.... No!!! they are both vertical forces. They have no component in the horizontal direction where we need a centripetal force. 

     What is it? It's friction!!!! But which friction is it?  Static or Dynamic? Well maybe a bit of both and another friction, that of just
sinking into the road. In our previous, discussion of friction implicitly considered just ideal rigid surfaces. But real roads do compress a bit under car weight.

     In any case, friction is a reactive force that tries to prevent
motion between surfaces and only turns on to try to prevent that.
It's like the Normal force which only turns on if you press inward
on a surface.

     In the car-on-a-curve, you angle the car wheels and ideally
static friction will be push exactly opposite to the car's current direction.

   This static friction can be resolved into two orthogonal components.
One component of frictions acts opposite to the wheel direction and
the other acts perpendicular to that direction and toward a 
center of curvature.

     Let's analyze the forces and see what relations we get.
	\begin{figure}[H]
		\centering
		\includegraphics[scale=0.7]{img/mechanics/banked_curve.jpg}
		\caption[Banked curve analysis]{Banked curve analysis (source: OpenStax)}
	\end{figure}
      In this case, we chose the $x$ direction to be the horizontal direction with the $x$ axis toward the curve center. The $y$ is vertical with up positive.

      We assume a vehicle is making the turn and that we observe that it is not sliding. This is the key condition!

     In the $x$ direction, we have
	
	where $F_{n}$ is the Normal force (with implicit direction Normal and outward from the surface) and $\theta$ is the angle of the normal force from the vertical and also, by geometry, the angle of the incline or banking from the horizontal. Here the $x$ component of the normal force is providing the centripetal force.
      
     In the $y$ direction, we have no motion. The path is rigid, and so vehicle can't sink into the path and in any perturbation to lift off the path, gravity pulls the vehicle back. So we must have balanced forces. So we have:
	
	where $F_{n}\cos(\theta)$ upward component of the Normal force and $mg$ is the force of gravity.

    Physics has given us two equations. From these we can solve for two unknowns. The most interesting variables to take to be unknowns are the
normal force $F_{\rm nor}$ and the angle of the incline $\theta$
(which is also gives the direction of the normal force, of course.) 
   
    The normal force magnitude is obtained by adding the squares
of  equations (\ref{eq-banked-x}) and (\ref{eq-banked-y}). We get:
	
    that is obviously bigger than $mg$!
 
    To get the incline angle, we divide equation \ref{eq-banked-x} by equation \ref{eq-banked-y} and we get:
    
	and
	
	Equation (\ref{eq-banking-angle-formula}) is the "\NewTerm{banking angle formula}\index{banking angle formula}". For a given speed and radius, the formula gives the angle one should bank a road at in order for a car to go around the curve with only the Normal force and gravity combining to give the acceleration.

      Note the formula is independent of mass! This is because the Normal force is linearly dependent on mass, and so mass cancels out.

    The Normal force provides all the force needed for the motion.
A car at the right speed would make a properly banked curve ideally even a road were ideal frictionless ice!!!!!!!!!!!!

    Note also that if $v^{2}/r$ goes to infinity, $\theta$ goes to $90^{\circ}$ and the car plane would be perpendicular to the ground.


	The banking angle formula equation always seems a little mysterious. How does the Normal force know to be just strong enough? But the answer is that you have assumed that you are making the curve. Therefore the Normal force \underline{must} be strong enough by the given condition. It's actually a constraint force!

    If you did a banking analysis with the friction force turned on and solved for Normal force and friction force, then you would find that the friction force goes to zero for the banking angle given by
the banking angle formula.

      Why do civil engineers bank curves using the banking angle formula? Somewhat obviously so that the cars don't need to rely on friction to make the curve. Failure of the Normal force doesn't much happen, except maybe on mud roads. The friction force can fail even on dry roads. Water and ice and snow reduce friction substantially.      Another obvious reason is that turning is much easier on a properly banked road. If you go at the specified speed, there is little need to turn the wheels at all and there is no tendency to loss of control.

      Yet another reason it is easier on the road structure, the car structure, and the car contents including the passengers. The above analysis actually applies to every bit of the car and contents. If you go at the specified speed, anticompressional forces do all the acceleration. Antishear forces (of which friction is only one) in the car and the road  are never called on. Antishear forces are any force that resists layers sliding over one another including those bonded together inside solids. Calling on such forces repeatedly leads to rapid wear. It's also more comfortable for passengers to be just compressed into their seats making a curve and have no tendency to slide sideways. 

      Of course, if you drive at the wrong speed around a curve, antishear forces do get called on with all their discomfort, tendency to wear, and chance of failure. So you should always drive at the specified speed.

      Too fast and you tend to slide outward. Too slow and you tend to slide inward. It's better to err on the too slow side, of course, because failure is probably less catastrophic. In very icy conditions what are you doing on the road anyway?

      Now civil engineers can't use just one banking angle. They must go from straight, unbanked road to maximally banked road continuously.
And they can't ask drivers to change speed more than once to the
curve speed.  So they must vary the radius of curvature to allow a continuous change from unbanked to maximally banked to unbanked again.

      Now all of the analysis above is a bit idealized. There are probably many complicating effects that real engineers must include
in their designs. One thing, which is probably minor in this context is that cars are not really, point masses. Forces are needed not just to make the car center of mass go round curves, but also rotate the car.
	
	\pagebreak
	\subsubsection{Exponential Friction}
	It is customary to designate as "\NewTerm{exponential friction}\index{exponential friction}" the friction generated by non-toothed belts (very important case in practice) on a fixed support!

	To investigate this case friction let consider the figure below where $\beta$ is the winding angle of the belt around the pulley where we neglect the thickness and mass of the belt relative to the other elements and where we place ourselves at the limit of slipping and in uniform motion:
	\begin{figure}[H]
		\centering
		\includegraphics[scale=1]{img/mechanics/friction_rope_capstan.jpg}
		\caption{Forces between a rope and capstan}
	\end{figure}
	By the existence of friction there is obviously a force differential $\mathrm{d}T$ on a point of contact but this is due only to friction since we suppose a uniform rotation. Let's take a closer look to this by isolating a portion of the capstan and applying the fundamental principle of static (with subtlety...):
	\begin{figure}[H]
		\centering
		\includegraphics[scale=1]{img/mechanics/friction_rope_capstan_element.jpg}
		\caption[]{Zoom on a differential element of the rope and capstan}
	\end{figure}
	
	Thus after simplification:
	
	and as the angle is very small, we have:
	
	and for the sine we use the MacLaurin development (\SeeChapter{see section Sequences and Series page \pageref{usual maclaurin developments}}) for small angles:
	
	Then we have:
	
		Then after rearranging:
	
	where the first equation is already known to us as it corresponds to the usual static horizontal friction already seen earlier (Coulomb's law). We can now combine these two relations to get:
	
	By integreting it comes therefore:
	
	Then we have:
	
	and finally we get the "\NewTerm{capstan equation}\index{capstan equation}" or "\NewTerm{belt friction equation}\index{belt friction equation}", also known as "\NewTerm{Eytelwein's formula}\index{Eytelwein's formula}":
	
	which is very important! For example in practice with a pull force of $500$ [N], it produces an actual pull force across the fixed circular support only of $250$ [N] because the friction is exponential. This is why it is better to use pulleys as fixed circular supports to raise a mass!
	\begin{tcolorbox}[title=Remark,colframe=black,arc=10pt]
	A "holding capstan" is a ratchet device that can turn only in one direction; once a load is pulled into place in that direction, it can be held with a much smaller force. A powered capstan, also named a "winch", rotates so that the applied tension is multiplied by the friction between rope and capstan. On a tall ship a holding capstan and a powered capstan are used in tandem so that a small force can be used to raise a heavy sail and then the rope can be easily removed from the powered capstan and tied off.\\
	
	In rock climbing with so-named "top-roping", a lighter person can hold (belay) a heavier person due to this effect.
	\end{tcolorbox}
	Obviously because of the friction we will have:
	
	which obviously implies that:
	
	and therefore:
	
	For the rest, the reader is referred to go to the sectionofCivil Engineering where the cylindrical rollers are presented in details.
	
	\subsubsection{Horizontal Viscous Friction}\label{horizontal viscious friction}
	We have seen that in a first approximation, the frictional force in the case of slipping is proportional to the weight of a body by a friction coefficient which value depends on the nature and condition of the contact surface, but independent of the contact area. We have also already mention that experience shows that in typical cases the friction force is independent of the velocity communicated to the moving body.

	By lubricating by a viscous fluid tow contact surfaces, the frictional force is reduced and therefore depends on the speed (this is typically the case of car tires which are viscous plastics).
	\begin{tcolorbox}[title=Remark,colframe=black,arc=10pt]
	For recall, the term "viscous" does not necessarily mean that it flows and it drool. This means that the behavior law depends on the deformation speed (\SeeChapter{see section Continuum Mechanics page \pageref{viscosity}}).
	\end{tcolorbox}
	Let us now consider a mobile in contact with a plane support via a fluid or viscous material. We know there will be friction and we assume that it is proportional to speed:
	
	where $k$ is the "\NewTerm{coefficient of viscous friction}\index{coefficient of viscous friction}".

	We then by applying Newton's first law:
	
	Thus we have:
	
	By integrating it comes:
	
	Therefore:
	
	Taking the exponential:
	
	Thus, the speed decreases exponentially form an initial velocity to a zero asymptotic value under the assumption of proportionality of friction with speed.

	It is a relation very often used in computer animations to represent objects that seem to stop naturally. You simply choose to choose the value of $k$.

	We observe an interesting thing is that heavy moving body decelerates more slowly because of friction forces than a slight moving body!

	Let us now show how to calculate the power lost by friction. We know that:
	
	if the variation in the force is negligible compared to the speed variation we have then:
	
	and therefore in the case of friction (in absolute value):
	
	Thus, the power dissipated during movement is proportional to the square of speed in the case of Coulomb friction and proportional to the square of the speed in the case of viscous friction.
	
	\subsubsection{Vertical Viscous Friction}
	Given a non-deformable solid falling vertically in a uniform gravity field. We assume that air resistance force is proportional to the speed (viscous behavior at low speeds)
	
	and using the fundamental principle of dynamics:
	
	Or written differently (more traditional):
	
	The solution of this linear differential equation of the first order is (\SeeChapter{see section Integral and Differential Calculus page \pageref{first order lde with constant coefficients}}):
	
	we can detail if required (on request).

	By assuming that at time zero we had a given initial velocity it comes:
	
	Therefore:
	
	Thus we see that when time tends to infinity (long enough in other words...) then the speed approaches:
	
	therefore $k$ can be determined experimentally!

	It is a relation once again very often used in computer animations to represent objects that seem to slow down at constant speed naturally by falling.
	
	The position at any time may be found by integrating the equation for $v$. With $v = \mathrm{d}y/\mathrm{d}t$:
	
	The calculations that follows are obvious (but we can detail on request if needed).
	
	\subsubsection{Stokes' Vertical Viscous Friction}
	This is typically the case of the parachutist performing a free fall. We proved in the section of Continuum Mechanics that Stokes' viscous force was given by (viscous behavior at middle and high speeds):
	
	when speed is subsonic (modest in other words...).
	
	The differential equation is the same as before in the case of the presence of a uniform gravitational field with the difference that the speed is this time squared:
	
	That is:
	
	but we will not try to solve this differential equation. We want only determine the limit of speed and this speed limit is reached when it does not longer varies .... (yes obviously...). So at that moment:
	
	and the differential equation becomes:
	
	Therefore:
	
	Thus, it is possible to change the falling speed limit depending on the form factor of the falling object and its apparent exhibition surface and its mass (in the above case study we have neglected the Archimedes force that applies on the parachutist and that also slows down his fall).
	\begin{tcolorbox}[colframe=black,colback=white,sharp corners]
	\textbf{{\Large \ding{45}}Example:}\\\\
	Let us consider a sphere of radius $R$, of mass density $\rho$, released without initial speed a liquid of mass density $\rho_{\text{liq}}$, of viscosity $\mu$ (see the classification of viscosity in the norm ISO 3448). The sphere is subjected to its own weight, to a viscous frictional force and to Archimedes' force.	We therefore have according to Newton's first law:
	
	where the last two terms (viscous Stokes's force and Archimedes force) have been proved in the section of Continuum Mechanics.\\

	Rearranging, we have:
	
	or also:
	
	Therefore it remains:
	
	We put now:
	
	which will be treated as a time constant. Then we have:
	
	But when the falling speed will become constant, we will have:
	
	Which simplifies our previous relation and gives therefore immediately:
	
	\end{tcolorbox}
	
	\begin{tcolorbox}[colframe=black,colback=white,sharp corners]
	Let us also solve the previous differential equation, beginning with that without the second member (\SeeChapter{see section Differential and Integral Calculus page \pageref{first order lde with constant coefficients}}):
	
	We then saw in the chapter of Differential and Integral Calculus that the homogeneous solution was then given by:
	
	We can add the particular solution that is logically when $t$ tends to infinity:
	
	Then we have:
	
	It remains to us to determine $C$ which is obtained when $t$ tends to $0$ because we have then:
	
	Therefore:
	
	Then we have:
	
	Finally:
	
	\end{tcolorbox}
	
	\pagebreak
	\subsubsection{Stokes' Horizontal Viscous Friction}
	This is a first approximation where are interested for example at a distance of stop with constantly braking of a mobile without taking into account the coefficient of friction with the ground but only with the ambient air (subsonic speed always. ..).

	We then have according to Newton's first law:
	
	Suppose we want to know how much time $T$ a mobile which had an initial velocity $v_0$ will have decelerated to a given final speed $v_f$.

	Then we have:
	
	Therefore:
	
	where we observe already a first problem with this model is that when stopped, the final velocity being equal to zero, it will take an infinite time to get there... but go on, we shall return to this finding.

	The evolution law of the speed is determined analogously as:
	
	Therefore:
	
	Let us put the time constant:
	
	Therefore:
	
	hence:
	
	The distance traveled at the instant $t$, leaving the mobile slow down uniquely by the frictional forces is therefore:
	
	That is:
	
	The result is pretty but we realize that it's not correct because after an infinite time, the car will have traveled an infinite distance which is clearly unrealistic. This comes from the fact that the model is too simplistic, therefore let us improve it.

	The ideal, objectively speaking, would be to take into account the viscous friction tire/ground plus the air friction. We would then have:
	
	but the problem with this differential equation is that it will lead us to a singularity if we continue the calculations. It is therefore not usage...

	We then try with the following form:
	
	which expresses that there is a friction force proportional to the weight of the vehicle which is simply the form of the second Coulomb's law:
	
	and the second term being the viscous Stokes' friction from which we have take out the mass factor included in the density being implicitly in the constant $k$ of $kv^2$.

	We then have simplifying the mass terms:
	So we already see that in this model we will lose the effect of the mass that normally have for effect to extend the travel distance before stop (in reality!). But let us continue anyway ...

	So we have to integrate:
	
	We have proved in the section of Differential and Integral Calculus that:
	
	Slightly rearranged it gives:
	
	So now we fall on a finite time for stop... which is more reassuring result.

	Let us now seek for stop distance. We use the fact that:
	
	gives us the possibility to write:
	
	Therefore:
	
	Which brings us to:
	
	Which gives us first:
	
	So we have our final result. Better than the last result but however independent of the mass ... but in the meantime it's better than nothing...
	
	\begin{tcolorbox}[title=Remark,colframe=black,arc=10pt]
	With a dry braking the front axle of a small car (Mercedes Benz Class A) has a braking force typically of the order of $2.8$ [kN].
	\end{tcolorbox}
	
	\pagebreak
	\subsubsection{Friction's Heat Factor}
	The sliding friction intervening between two bodies in contact results in a loss of mechanical energy almost entirely converted into heat energy $Q$ (\SeeChapter{see section Thermodynamics page \pageref{heat energy}}). The problem of conversion of friction energy into heat is always very complex and requires detailed studies but is of major importance in the industry.

	For simplicity, let assume that the friction component remains constant, applied to the center of gravity of the friction surface, and the speed of the body is linear and constant. The work done by the friction force, for a finite displacement is then:
	
	Assuming this mechanical work completely transformed into thermal energy, the quantity of heat produced must be evacuated out of the friction surfaces. This is a fundamental problem in the mechanisms with high heat release such as brakes, clutches, gear transmissions and worm gear and especially plain bearings. The released heat output is then calculated by:
	
	We just have to divide by the total contact area to the heating power per unit area. Then, in the special case of Coulomb's dynamic friction, we have this last relation that will be written:
	
	The heat output is proportional to the friction coefficient $\mu_D$ and the product of the mean pressure $P$ by the sliding velocity $v$ and the contact area $A$. This control method of heating is obviously a rough first approach because the coefficient of friction, the pressure and the sliding speed are only part of the factors influencing the temperature of operation of a mechanism.
	
	\begin{flushright}
	\begin{tabular}{l c}
	\circled{80} & \pbox{20cm}{\score{4}{5} \\ {\tiny 69 votes,  73.33\%}} 
	\end{tabular} 
	\end{flushright}
	
	%to make section start on odd page
	\newpage
	\thispagestyle{empty}
	\mbox{}
	\section{Wave Mechanics}\label{wave mechanics}
	\lettrine[lines=4]{\color{BrickRed}H}ere we focus on the study of mathematical properties of vibrating strings that we can also by extension and in a wish of generality assimilate to the immaterial cases of the important concept of "\NewTerm{waves}". This study is very important because we need some of the results that will be proven here in the section of Thermodynamics, Quantum physics, Astrophysics, Electrodynamics, Acoustics and String theory (to mention only the most important one).
	
	\textbf{Definitions (\#\mydef):}
	\begin{enumerate}
		\item[D1.] A "\NewTerm{wave}\index{wave}" is a transport of energy without transporting material. It is embodied in the propagation of a disturbance of a medium, hence the name "\NewTerm{traveling wave}\index{traveling wave}". The speed with which the wave progresses depends obviously on the physical properties of the medium.
		
		\item[D2.] In the case where the disturbance of the medium (deformation of the wave) is perpendicular to the wave propagation direction, we speak about "\NewTerm{transverse wave}\index{transverse wave}" or "\NewTerm{transverse disturbance}\index{transverse disturbance}" (typical of waves in string for example).
		\item[D3.] In the case where the disturbance of the medium is parallel to the wave propagation direction, we speak about "\NewTerm{longitudinal wave}\index{longitudinal wave}" or "\NewTerm{longitudinal disturbance} \index{longitudinal disturbance}" (typical of springs).
	\end{enumerate}
	
	\subsubsection{Wave Function}
	Given a disturbance $y=f(x)$ in a given region of space. If we replace $x$ by $x-b$ we define, in the same region, a disturbance $f (x-b)$ identical to $f (x)$ but translated a distance $b$ in the direction of positive $X$ (to the right if we adopt the conventional representation seen in the section of Functional Analysis).
	
	If $t$ represent a time and if we put $b=vt$, then $v$ can denote the translational speed of the disturbance.
	
	So we name "\NewTerm{wave function}\index{wave function}", the mathematical relation:
	
	that describes the progress of a disturbance $y (x, t)$ in the space:
	\begin{itemize}
		\item $y(x,t)=f(x-vt)$ describing a wave that progresses in the direction of $+X$
		\item $y(x,t)=f(x+vt)$ describing a wave that progresses in the direction of $-X$
	\end{itemize}
	$v$ is by definition named "\NewTerm{wave phase velocity}\index{wave phase velocity}" and $vt$ the "\NewTerm{phase shift}\index{phase shift}". It is constant in a homogeneous medium. The  "\NewTerm{amplitude of the wave}\index{amplitude of a wave}" is the maximum value of the perturbation:
	
	In the absence of damping, it will retain the same value in each of the points $x$ where the wave passes.
	
	\subsubsection{Wave Equation}\label{wave equation}
	\begin{theorem}
	Without getting into too technical considerations, we will say that every function $f$ whose argument is $x \pm vt$ has the property:
	\end{theorem}
	\begin{dem}
		
		and therefore the equality follows immediately (this is simply the application of composed derivative as proved in the section of Differential and Integral Calculus).
		\begin{flushright}
			$\square$  Q.E.D.
		\end{flushright}
		Following a request from a reader who did not find this proof very clear, let us make an example. Consider the following function:
		
		and therefore:
		
	\end{dem}
	Returning to the general case, we derive a second time, we obtain another form of the wave equation that we also meet frequently in practice and in the books:
	
	Which brings us to write one of the most important relations in physics named the "\NewTerm{wave equation}\index{wave equation}", "\NewTerm{propagation equation}\index{propagation equation}" or "\NewTerm{d'Alembert's equation}\index{d'Alembert's equation}" and that we see again many times in other sections of this book (Electrodynamics, Wave Quantum Physics, General Relativity, Acoustic):
	
	\begin{tcolorbox}[title=Remarks,colframe=black,arc=10pt]
	\textbf{R1.} Don't forget (\SeeChapter{see section of Differential and Integral Calculus page \pageref{linear differential equations}}) that the sum of the solutions to a differential equation is also a solution of the differential equation! Thus, the general solution Alembert's equation is the superposition can be two arbitrary waves going in opposite directions.\\
	
	\textbf{R2.} When two or more waves propagate in a medium, the wave function that results is the algebraic sum of each wave of wave functions. We say then that the waves "interfere" and we name the phenomenon "\NewTerm{superposition wave principle}\index{superposition wave principle}" (see further below for mathematical details).
	\end{tcolorbox}
	
	Consider now a string of length $L$ attached by one of its ends to a fixed terminal. Now suppose that a disturbance propagate into the string. When the disturbance reaches the end (terminal), we observe that it changes in sign and that the speed is reversed: the wave is then subject to a reflection with inversion.
	
	To describe the phenomenon, we must impose at the terminal:
	
	But any wave function of the form $f(x\pm vt)$, which progresses to the termination, can not verify the above condition:
	
	for all time values $t$.
	
	The trick is to replace it with another wave function $y(x, t)$ whose shape is similar to $f$ at large distance from the source of the interference, and which is zero at the point of termination for all time values $t$. For this, we can imagine at the point of termination, a mirror which gives an image of the string of the same length in which we invent a virtual wave:
	
	symmetrical to $f(x-vt)$, but of opposite sign.
	
	We decide therefore:
	\begin{itemize}
		\item That the two waves progress toward each other to cancel at the endpoint
		\item Any part of the real wave that pass through the endpoint becomes virtual
		
		\item Any part of the virtual wave which enters the string  becomes real
	\end{itemize}
	At their intersection, the two waves achieve a destructive interference at the endpoint. The algebraic sum of these two wave functions is also a wave function:
	
	which has the desired property at $x = L$ for any $t$:
	
	If we now consider a free termination, without friction with its fixed point, we find ourselves in a similar case to the previous one except that the interference is constructive at the endpoint rather than destructive so that the wave function will be written:
	
	\begin{tcolorbox}[title=Remarks,colframe=black,arc=10pt]
	\textbf{R1.} When the wave arrives on a free or fixed termination, the energy carried is fully returned back.\\
	
	\textbf{R2.} When the termination is not exactly adapted, only a portion of the energy is absorbed by it, the rest is reflected.
	\end{tcolorbox}
	
	\subsection{Type of Waves}
	In theoretical physics (and in practice), we often restrict the study of certain phenomena to particular cases of waves. Primarily, we recognize three of them that we will briefly but carefully develop here:
	
	\subsubsection{Periodic Waves}
	If an event produces a wave does that carry only single disturbance produced at a given point, there are many disturbances which are capable to excite a medium in a repeatedly (periodic) way.
	
	The spatial point of the source is then subjected to the same periodic disturbance. The duration of a complete cycle is ma,ed identically to our study of pendulum, the "\NewTerm{period $T$}\index{period}".
	
	If the disturbance can propagate as a wave, at speed $v$, it is described by the wave function that we know:
	
	At each point of the disturbed system, the periodic wave imposes a "\NewTerm{temporal periodicity}\index{temporal periodicity}" of the disturbance which requires us to write:
	
	After several cycles (periods) of disturbances from the source, more disturbances are distributed in space. The distance between two successive disturbances is named the "\NewTerm{wavelength $\lambda$}\index{wavelength}".
	
	The "\NewTerm{spatial periodicity}\index{spatial periodicity}" also requires that:
	
	Therefore, in the case of a periode wave, $\lambda$ is the path traveled by the wave during the time $T$:
	
	\begin{theorem}
	If a wave function is periodic in time, it is also in the space, provided that the pulse does not deform as it progresses.
	\end{theorem}
	\begin{dem}
	
	By putting $n_2=-n_1=n$ we have:
	
	that becomes:
	
	\begin{flushright}
		$\square$  Q.E.D.
	\end{flushright}
	\end{dem}
	
	\subsubsection{Harmonic Waves}\label{harmonic waves}
	For these waves, the wave function solution of the d'Alembert's equation is a trigonometric function of the sine or cosine type (or sum of them):
	
	The presence of $k$, named "\NewTerm{wave number}\index{wave number}\label{wave number}" or "\NewTerm{propagation number}\index{propagation number}", is required for two reasons:
	\begin{itemize}
		\item $k$ is express in $[\text{m}^{-1}]$ 

		\item the value of $k$ must ensure the periodicity of the wave function:
		\begin{enumerate}
			\item Angular periodicity of the mathematical function:
			
	
			\item Spatial periodicity of the wave function:
			
		\end{enumerate}
	\end{itemize}
	By equating these two expressions:
	
	which implies:
	
	Let us introduce:
	
	in the expression of $k$\label{pulsation frequency period wave number}:
	
	hence another important relation, the explicit relation of the "\NewTerm{phase velocity}\index{phase velocity}":
	
	The wave function of the harmonic wave can thus be written as:
	
	
	\paragraph{Phase velocity and Group velocity}\mbox{}\\\\
	According to what we have just seen above, the phase velocity of a monochromatic wave is equal to the ratio between its pulsation and its wave number (norm of the wave vector).

	Let us consider now the simple case consisting of a wave of the superposition of two different pulsations and unit amplitude:
	
	Using trigonometric identities (Simpson's formula) we have:
	
	Thus, the wave in question consists of the product of two terms:
	\begin{itemize}
		\item The first term is a monochromatic wave whose velocity phase is:
		
	
		\item The second therm is another monochromatic wave whose velocity phase is:
		
		This term intervenes as amplitude modulator of the first term.
	\end{itemize}
	There is thus a beat phenomenon where a sine wave is modulated in amplitude modulated by a lower pulse sinusoid. The group velocity is the by definition:
		
		Finally:
		
		We can illustrate the both velocities with an animation in Maple 4.00b:
		
		\texttt{>restart:with(plots):\\
		>lambda[0]:=1; T[0]:=1; k[0]:=2*Pi/lambda[0]; w[0]:=2*Pi/T[0];\\
		>delta\_k:=k[0]/8: k[1]:=k[0]-delta\_k; k[2]:=k[0]+delta\_k;
	delta\_w:=w[0]/10: w[1]:=w[0]-delta\_w; w[2]:=w[0]+delta\_w;\\
		>P1:=animate(cos(k[1]*x-w[1]*t)+cos(k[2]*x-w[2]*t), x=0..1*2*Pi/delta\_k, t=0..2*Pi/delta\_w, numpoints=200, frames=15, color=red):\\
		>P2:=animate({2*cos(-1/2*k[1]*x+1/2*w[1]*t+1/2*k[2]*x-1/2*w[2]*t), -2*cos(-1/2*k[1]*x+1/2*w[1]*t+1/2*k[2]*x-1/2*w[2]*t)}, x=0..1*2*Pi/delta\_k, t=0..2*Pi/delta\_w, numpoints=100, frames=15, color=blue):\\
		>display(P1,P2);}
		
		Which gives visually:
		\begin{figure}[H]
			\centering
			\includegraphics[scale=0.8]{img/mechanics/group_phase_velocity.jpg}
			\caption{Group and Phase velocities difference}
		\end{figure}
	
	\subsubsection{Stationnary Waves}
	Let us imagine a string excited in a harmonic way. Rather than adapt its end pour extract the energy of the string, let us assume that this end is fixed. The wavec is therefore reflected!!!
	
	A new wave function must be define to describe the superposition of the incoming wave:
	
	and of the reflected wave (symetric and of opposite sign):
	
	by analogy with the result that w have found during our study of the ends of strings:
	
	The following trigonometric identity proved in the section of Trigonometry gives:
	
	gives us:
	
	This is not a progressive wave as $x$ and $t$ does combine anymore like $(x-vt)$. Some points of the string never move. They satisfy:
	
	and are located therefore on:
	
	For obvious reasons, we will keep only the values of $n$ for which $x<L$.
	
	Each of these points is named a "\NewTerm{vibration node}\index{vibration node}".
	
	We observe in such a system, vibration zones, of length $\lambda/2$, in which the string vibrates transversely in an height region of $4\hat{y}$ (two up, two down).
	\begin{tcolorbox}[title=Remark,colframe=black,arc=10pt]
	The points at which the amplitude of vibration is maximum are named sometimes the "\NewTerm{antinodes of the vibration}\index{antinodes of the vibration}".
	\end{tcolorbox}
	Since $\sin(k(x-L))=1$, the antinodes are apart of $\lambda/2$ and located at $\lambda/4$ of the nodes.
	
	If we now impose a fixed termination on both ends of a vibrating string, we end up with a configuration named "\NewTerm{resonance system}\index{resonance system}".
	
	Most often, we do not observe much until we find the excitation frequency vibrations that places nodes on the two fixed endpoints.
	
	Therefore for a string of length $L$:
	
	involved:
	
	The string is then the seat of a standing wave whose vibration amplitude is considerably greater than the excitation amplitude (four times)!
	
	We then say that the string came into "\NewTerm{resonance}\index{resonance}" with the exciter.
	
	The relation:
	
	shows that there are several possible wavelengths whose largest corresponding to $n = 1$ and is named the "\NewTerm{fundamental wavelength}" or "\NewTerm{fundamental harmonic}\index{fundamental harmonic}" and is obviously equal to $\lambda=2L$:
	\begin{figure}[H]
		\centering
		\includegraphics[scale=0.8]{img/mechanics/harmonics.jpg}
	\end{figure}
	
	\pagebreak
	\subsubsection{Vibration Modes in a Stretch String}
	We have seen how a wave can move in a string. Let us show now why this is possible and establish the relation $y(x, t)$, giving the shape of the string over time.

	Given a wire of diameter $\varnothing$, of length $L$ and mass $m$, the linear density of the string (assumed constant along it) is then:
	
	By a small shock, we create a small transversal perturbation (so we do deform the string and maintain its linear density constant ). Let us isolate, in the disturbed area, a string element of length $\Delta L$.
	
	The approximations are the following:
	\begin{enumerate}
		\item Each element of the string may be cut infinitesimally so as to be almost parallel to the $x$-axis. The angles $\theta_P,\theta_Q$ are considered small.

		\item The string is considered as deformable but non-extensible therefore the norm of the forces in the string is constant at any point regardless of the deformation.
	\end{enumerate}
	For the rest of the reasoning, we use the figure below:
	\begin{figure}[H]
		\centering
		\includegraphics[scale=0.8]{img/mechanics/stretch_string.jpg}
		\caption{Stretch String Scheme}
	\end{figure}
	If the angles are small, the balance of forces gives:
	
	which means that there is no displacement following the $x$-axis:
	
	If the angles are really small, we have the first term of the Taylor development of $\sin(x)$ and $\tan(x)$ (\SeeChapter{see section Sequences and Series}) which gives:
	
	Therefore:
	
	is the acceleration following the $y$-axis for small angles.

	Newton's law applied to the mass $\Delta m=\rho_l \Delta x$ gives (we consider that each point of the mass moves only according to the $y$-axis since there is no elongation and as we are in the assumption of a transverse excitation):
	
	The tangents are given by the partial derivatives of the function $y(x)$:
	
	Which is equalized with the prior-previous relation:
	
	and therefore:
	
	If $\Delta x\rightarrow 0$, the two tangents tend to the same value, but the fraction of the right-hand side tends to a finite value:
	
	It follows the differential equation:
	
	
	The latter relation is usually written as follows:
	
	and is named "\NewTerm{equation of vibrating strings}\index{equation of vibrating strings}\label{equation of vibrating strings}".
	
	\begin{tcolorbox}[title=Remark,colframe=black,arc=10pt]
	In some books, the linear density is denoted by $\mu_0$ and the tension force in the string by $T_0$ which gives:
	
	\end{tcolorbox}
	We see easily that the units of $F/\rho_l$  are those of the square of velocity $[\text{ms}^{-1}]$ as required by dimensional analysis. To simplify the notations, we put:
	
	and therefore:
	
	We will now consider a very interesting case in the context of musicology which is that of the tense string (most string instruments works as this).
	
	\paragraph{Dirichlet Conditions}\mbox{}\\\\
	The goal is in the framework of the previous differential equation (small deformations in the context of musical instruments) to find a function $y(x, t)$ solution of the latter with the follwoing initial conditions that are typical for a musical instrument :
	\begin{enumerate}
		\item[IC1.] $y(0,t)=y(L,t)=0$: the extremities $A$ and $B$ of the string are fixed. These are the "\NewTerm{Dirichlet boundary conditions}\index{Dirichlet boundary conditions}".

		\item[IC2.] $y(x,0)=f(x)$: the initial shape of the string at excitation.

		\item[IC3.] $\dfrac{\partial y(x,0)}{\partial t}=0$: initial velocity is zero on any point of the string.
	\end{enumerate}
	The last two conditions are named "\NewTerm{Cauchy boundary conditions}\index{Cauchy boundary conditions}".
	
	To solve this linear differential equation, we will make use of the variables separation method by writing (\SeeChapter{see section Differential and Integral Calculus page \pageref{separation vaiables method}}):
	
	The differential equation:
	
	Therefore becomes:
	
	So we have:
	
	By putting the second relation in the first, we get:
	
	The left term of the last relation does not contain the variable $t$ and the right does not contain $x$. The only way to equalize these two expressions is to consider each as a constant, which we will denote by $-\lambda2$ such that:
	
	Thus, we have two differential equations:
	
	These two equations are similar, we solve them in a general way (\SeeChapter{see section Differential and Integral Calculus page \pageref{second order differential equations}}):
	
	The characteristic equation is then:
	
	hence:
	
	We know that the general solution if the roots of the characteristic equation are complex ($r_{1,2}\in \mathbb{C}$) is the of the form:
	
	For our two differential equations, so we have by similitude:
		
	This gives for the solution of our wave equation:
	
	Let us determine the constants $A_i,B_i$ by taking into account the initial conditions:
	
	Then it only remains:
	
	Let us put:
	
	The initial condition $y(L,t)=0,\forall t$ brings us to put that:
	
	To take into account the zero initial speed, we derivate $y(x,t)$ with respect to time:
	
	It only remains:
	
	The constant $b$ represents the amplitude of the transverse displacement of the string. As this amplitude can not be the same everywhere at a given time and a given location for any kind of excitement satisfying the initial conditions, there must be as many values $b_n$ as we choose values of $n$ in $n\pi x/\lambda l$.
	
	The principle of superposition of the solutions of linear differential equations (\SeeChapter{see section Differential and Integral Calculus page \pageref{linear differential equations}}) permits us to write that the linear combination of all solution for the string is therefore:
	
	The $b_n$ must be choose to satisfy the initial conditions that gives the shape of the perturbation:
	
	This expression for $f(x)$ suggest to compare it at a Fourier series development (\SeeChapter{see section Sequences and Series page \pageref{fourier transform}}):
	
	In which $k=\pi/L$ and $a_n=0,\forall n$. Fourier's theorem requires the $b_n$ are given by (\SeeChapter{see section Sequences and Series page \pageref{fourier series}}):
	
	Let us now imagine a string of length $L$ fixed at its ends and tense. Let us choose the simplest possible excitation form: we scratch the string in its middle in a very quick way, to take it away from a very small distance $H$ from its equilibrium position.
	
	The initial disturbance $y (x, 0)$ is then:
	
	Let us now compute the Fourier coefficients:
	
	Therefore:
	
	The wave function becomes:
	
	Because of the $\sin\left(n\dfrac{\pi}{2}\right)$, the terms for which $n$ is even are all zero. Then it remains:
	
	If we retain only the term $n = 1$, we keep obviously only:
	
	We have:
	
	that is the wavenumber corresponding the wavelength:
	
	and:
	
	which is the first harmonic frequency vibration of the string.
	
	Thus, for any value $n$, it is easy to show that the $n$-th "\NewTerm{eigenmode}\index{eigenmode}\label{eigenmodes}" is given by:
	
	with:
	
	relations name "\NewTerm{Mersenne's laws}\index{Mersenne's laws}" (1644-1648) where the mode of lower frequency with $n$ being equal to $1$ is named the "\NewTerm{fundamental mode}\index{fundamental mode}" associated with its "\NewTerm{fundamental frequency}\index{fundamental frequency}".

	Thus, after being scratched quickly in the middle of its length $L$, a string held rigidly at both ends can oscillate in several modes. The fundamental mode $n=1$ (fundamental harmonic) is the lowest possible frequency. It corresponds to it the wavelength $\lambda=2L$.

	The $n$ higher order frequencies are named "\NewTerm{harmonic frequencies}\index{harmonic frequencies}". For the same initial displacement $H$, the maximum amplitude of the vibration decreases according to $n^{-2}$ as we see it in the expression of our function above.

	Another way to excite the string is make it oscillates sinusoidally, which means therefore that $y(x, t)$ is of the form:
	
	Substituting this relation in the string wave equation, we get:
	
	The solution is then reduce to:
	
	The value $n = 0$ can not be included because it gives a string without excitement. By this function in the previous wave equation and simplifying, we get trivially:
	
	These are the "\NewTerm{Dirichlet's oscillations frequencies}\index{Dirichlet's oscillations frequencies}" of a string. The strings of a violin, for example, are Dirichlet strings.
	
	Remember the image seen previously above but focusing this time only on two harmonics:
	\begin{figure}[H]
		\centering
		\includegraphics{img/mechanics/eigen_modes_harmonics.jpg}
		\caption{First two basic modes of a fixed string}
	\end{figure}
	The same analogies, reasoning and developments can be made with the Neumann conditions below.
	\begin{tcolorbox}[title=Remarks,colframe=black,arc=10pt]
	\textbf{R1.} As we have just seen the theory predicts that the vibration can be a linear combination of several modes. This phenomenon is named "\NewTerm{simultaneous vibration}\index{simultaneous vibration}". It occurs abundantly in a piano..\\
	
	\textbf{R2.} Many musical instruments are designed to emit sounds with conventional frequencies, being admit that the note recognized by the ear is defined by the fundamental frequency, for example: Do (264 Hertz), The (440 Hertz ). For more details see the section of Mathematical Music.\\
	
	\textbf{R3.} During the construction of musical string instrument, we decide the value of $\rho_l$ (by choosing the diameter and the nature of the string) and we determine its length $L$ by seeking the compromise between the loudness we want to emit and the mechanical strength that the instrument that must support the tension forces $F$.
	\end{tcolorbox}
	
	\pagebreak
	\paragraph{Neumann Conditions}\mbox{}\\\\
	Alternatively to Dirichlet's conditions where the ends are fixed and of equal height, the Neumann conditions assume that the ends are small loops allowed to slide along two bars without friction.
	
	For our string, the Neumann conditions specify the values $\partial y/\partial x$ at the ends. But as the loops are supposed massless and without friction, the derivative $\partial x/\partial x$ must cancel at the end $x=\{0,L\}$ (this is an assumption not easy to guess this is why it has a name...!). If this were not the case, then by the nullity of the mass of the end, the shift will be due to an infinite acceleration, which can not be allowed non-relativistic framework! This is what we require instead of the Dirichlet condition, the Neumann condition defined by:
	
	\begin{enumerate}
		\item[IC1.] $\dfrac{\partial y}{\partial x}(0,t)=\dfrac{\partial y}{\partial x}(L,t)=0$
	\end{enumerate}
	the conditions IC1, IC2 remaining the same.
	
	This change of status does not prevent the variables separation method of resolution to be the same as before and that we will fall identically on the following relation to which we will apply the new initial condition (let us know if you want we put the details again):
	
	on which we therefore apply the Neumann conditions:
	
		Then it remains:
	
	by putting $b=A_1A_2$,$a=A_1B_2$ the function simplifies in:
	
	The initial condition $\partial y/\partial x(L,t)=0,\forall t$ requires:
	
	The same developments for the IC2. we have made with the Dirichlet condition IC1 then apply equally (we can write the details on request):
	
	then, the analogy with Fourier series applies similarly but with the cosine in place of the sine.

	The Neumann frequencies of a string are the same as for the Dirichlet either:
	
	The particularity, however, is the value of spatial function that is trivially this time:
	
	Indeed, for $n = 0$ we have this time an identitical  amplitude $b_n$ which is transmitted along the string without that it does vibrates!!!

	Let us also notice that the function $y(x,t)=at+b$ also fully satisfies the three initial conditions including that of Neumann.

	Indeed, we have well:
	\begin{enumerate}
		\item[IC1.] $\dfrac{\partial y}{\partial x}(0,t)=\dfrac{\partial y}{\partial x}(L,t)=0$

		\item[IC2.] $y(x,0)=f(x)=b$

		\item[IC3.] $\dfrac{\partial y(x,0)}{\partial t}=0$
	\end{enumerate}
	and morever, $y(x,t)=at+b$ satisfies also the wave equation:
	
	
	\subsection{Non-relativistic Lagrangian of a String}
	We will now determine the non-relativistic Lagrangian of a string, calculation that will be us useful to us for our study of String Theory in the chapter of Cosmology.
	
	We keep our string having a constant linear density and tension whose ends are located at $x=0,x=L$ and the speed of for which the tranverse speed is not relativistic $v\ll c$.
	
	The kinetic energy is then simply the sum of the kinetic energies of each infinitesimal element of the string. We can then write in Lagrangian notation:
	
	The potential energy is involved in the elongation of the string from which an infinitesimal portion can be seen as varying from $(x, 0)$ to $(x+\mathrm{d}x,0)$ when the string is in equilibrium. When a string is momentarily stressed from $(x, y)$ to $(x+\mathrm{d},y+\mathrm{d}y)$ then the variation of the length $\mathrm{d}l$ of an infinitesimal element of the string is trivially given by:
	
	We used above for the approximation the Taylor expansion at the second order (\SeeChapter{see section Sequences And Series page \pageref{usual maclaurin developments}}), which gives us:
	
	The work done to stretch every infinitesimal element being $F\Delta l$, the total potential energy is then expressed as:
	
	The Lagrangian being defined by $L=T-V$ (\SeeChapter{see section Analytical Mechanics page \pageref{lagrangian mechanics}}), then we have:
	
	where $\mathcal{L}$ is defined, well correctly, as the "\NewTerm{Lagrangian density of a string}\index{Lagrangian density of a string}\label{lagrangian desnity of a string}":
	
	The action of the string is therefore:
	
	In this action, the path of action is the function $y (x, t)$. To find the equations of motion, we must consider the variation of the action when we vary:
	
	Which gives:
	
	As:
	
	and this identically for the second term.

	We should not have time derivatives acting on the variations. Then using the following trivial relation on the first term:
	
	and identically on the second, we can rewrite the action:
	
	As we have seen it in the section of Analytical Mechanics, the correct path is given by $\delta S=0$. Therefore, we must have:
	
	Thus our expression contains three terms. Each of these terms must vanish independently as we shall see:
	\begin{enumerate}
		\item The cancellation of the third term is done following a trivial condition that we are already well known (hopefully ...):
		
		and therefore:
		
		So we fall back on the differential equation of a transverse wave as we had proved earlier above. Our hypothesis about the third term can then only be correct as well as the expression of our action.

		\item The first term is determined by the configuration of the string at the times $t_i,t_f$:
		
		Now, if we apply the knowledge of these configurations in time, we have by definition:
		
		(total knowledge of the action path as knowledge of the initial conditions, so no variation). This validates once again the expression of our action and the zero value of this term as expected.
		
		\item The second term is a little more interesting:
		
		First, it is only because we know the positions of the ends of the string that we can know its vibration modes, we know that well! So we have to know how the ends behave. For that we will come back on things that are well known to us: the Dirichlet and Neumann conditions of a string.
		
		Suppose we assume the Dirichlet conditions (see above), the ends are then fixed and we will have necessarily at those ends:
		
		and then the second term vanish well (phew!).
	
		If, in contrast, we choose the ends to move freely, then the variations:
		
		are unconstrained and therefore, only the Neumann conditions:
		
		(see above for more details) we permits us to have the second term of the action equal to zero.
		
		To be fully aware of the importance of these initial conditions, let us consider the linear momentum $p_y$ carried by the string (there are no other motion of the as we have implicitly assumed a transverse-excitement from since the beginning only in this direction $y$).

		The linear momentum is simply the sum of the linear momentum of each infinitesimal element along the string:
		
		Let us check out by curiosity (it's an anticipated curiosity ...) if the momentum is well preserved:
		
		where we have used the transverse wave equation for the substitution.
		
		We see by the result of this little calculation that the momentum is trivially  conserved if we impose the Neumann conditions, whereas for the Dirichlet conditions, most of time conservation is not respected! Indeed, it is trivial (there is no need for calculations to realize it normally), when the ends of the string are attached to the wall, the wall constantly exerts a force on the string.
	\end{enumerate}
	
	\subsection{Vibrational modes of a circular membrane}\label{circular drum}
	A two-dimensional elastic membrane under tension can support transverse vibrations. The properties of an idealized drumhead can be modeled by the vibrations of a circular membrane of uniform thickness, attached to a rigid frame. Due to the phenomenon of resonance, at certain vibration frequencies, its resonant frequencies, the membrane can store vibrational energy, the surface moving in a characteristic pattern of standing waves. This is named a "normal mode". A membrane has an infinite number of these normal modes, starting with a lowest frequency one named the fundamental mode.
	
	There exist infinitely many ways in which a membrane can vibrate, each depending on the shape of the membrane at some initial time, and the transverse velocity of each point on the membrane at that time. The vibrations of the membrane are given by the solutions of the two-dimensional wave equation with Dirichlet boundary conditions which represent the constraint of the frame. It can be shown that any arbitrarily complex vibration of the membrane can be decomposed into a possibly infinite series of the membrane's normal modes. This is analogous to the decomposition of a time signal into a Fourier series.
	
	We derive the phenomenon of vibration of a circular membrane in tension (typically a drum) in the same manner that the transverse vibration of a string whose equation was for recall:
		
	But first if we consider a vertical slice of the drum we change $y$ by $z$ as we work in 3D space and that the $z$-axes is perpendicular to the dream membrane:
	
	By the central symmetry of the problem we can also write (as the general solution is the sum of special solutions and we will consider that the force $F$ takes into account the sum of the special cases):
	
	However, the linear density $\rho_l$ of the string must be replaced by the surface density $\rho_s$ of the membrane denoted $\sigma$ by tradition. And for the coherence of units we must replace replace the unidirectional tension force $F$ of the string by a tension force applied on the periphery of the membrane. This force is exerted in all directions of the plane and is described per unit length:
	

	So finally:
	
	Or in a more short way using the Laplace operator (\SeeChapter{see section Vector Calculus page \pageref{scalar laplacian}}):
	
	Or in a more conventional way\label{three dimensions wave equation}:
	
	with:
	
	We seek the particular solution of the Laplace equation which satisfies the following conditions:
	\begin{enumerate}
		\item[IC1.] The membrane is fixed on its circular periphery R (boundary conditions)

		\item[IC2.] The initial position and speed are given (initial conditions)
	\end{enumerate}
	The symmetry of the problem suggests using the Laplacian of a scalar field in polar coordinates (\SeeChapter{see section Vector Calculus page \pageref{scalar laplacian}}):
	
	And the fixed conditions:
	\begin{enumerate}
		\item[C1.] $\forall t\geq 0,\forall \theta\in[0,2\pi],z(r,\theta,t)=0$ (boundary conditions)

		\item[C2.] $\forall \theta \in [\theta,2\pi],\forall r\in[0,R] \begin{cases} z(r,\theta,0)=f(r,\theta) \\ \partial_t z(r,\theta,0)=g(r,\theta)  \end{cases}$
	\end{enumerate}
	where the functions $f,g$ representing the shape of the drum at $t=0$ and the speed of its membrane also at $t=0$ must be given.
	
	Again, to find the solution, we will use the method of separation of the variables such that:
	
	and as identically as for the rope:
	
	and identically that for the rope, we get for the temporal part $T$ a solution of the type:
	
	For $X(r,\theta)$ the method changes with respect to the case of the rope seen above because we now have a differential equation with two variables such that:
	
	To integrate this last equation, we will seek the solutions also in the following separated form:
	
	we get by putting this in the differential equation:
	
	because for recall we have in polar coordinates:
	
	Hence, by separating the variables:
	
	The left-hand side of the last relation does not contain the variable $r$ and the right-hand side does not contain the variable $\theta$. The only way to equal these two expressions is to consider them each as equal to as constant we will denote $\mu$ such that:
	
	The differential equations verified by $R$ and $\Theta$ are then:
	
	The function $\Theta$ must obviously be periodic of period $2\pi$, so there exists a natural number $n$ such that $\mu=-n^2$ and therefore identically to the rope, we get quite immediately:
	
	If we put $\mu=n^2$ in the first differential equation, we get:
	
	That we often found in the literature in the following form:
	
	To simplify, we perform the change of variable $x=\lambda r$. That is:
	
	The differential equation therefore becomes:
	
	After dividing by $\lambda^2$ and multiplying by $x^2$ the differential equation becomes:
	
	We recognize here the Bessel differential equation\index{Bessel differential equation} of order $n$ as we have introduce it with its special solution in the section of Sequences and Series. Therefore, a special solution is of the type (as the general solution is an infinite sum of Bessel functions by linearity!):
	
	What ultimately gives us:
	
	Among the solutions to this equation, let us look for those which satisfy the boundary conditions by putting $R_\text{membrane}=L$:
	
	Unless $\Theta$ or $T$ is the null function, which gives for solution the equilibrium position ... (which probably does not satisfy the initial conditions), we must have $R(L)=0$, that is to say:
	
	The Bessel function $J_n$ has an infinity of positive zeros that we will denote here by $(\xi_m^n)_{m\in\mathbb{N}}$ (just plot this function with a computer to see it such as with Maple 4.00b by writing the command: \texttt{>plot(BesselJ(2,x),x=0...100);} where you can change the order value $2$ by another value) that provide an infinity of suitable values of $\lambda$ such that:
	
	This finally corresponds to an infinity of solutions of the initial differential equation that we can write:
	
	by having modified the name of the integration constants and having put $\omega_m^n=\lambda_m^n v$ (which verifies the dimensional analysis). Now that this solution satisfies the boundary conditions, we must address the initial conditions.
	
	First for the same reasons as the rope, the general solution is the linear superposition of solutions such as:
	
	We will determine the coefficients $\alpha_m^n$, $\beta_m^n$, $\gamma_m^n$, $\delta_m^n$ so that the solution $y(r,\theta,t)$ given above also satisfies the initial conditions, that is to say:
	
	These two relations are similar, let us study the first. It can be written:
	
	Which is the Fourier series expansion of the function $f(r,\theta)$ (\SeeChapter{see section Sequences and Series page \pageref{fourier series}}). Therefore we have (\SeeChapter{see section Sequences and Series page \pageref{fourier series}}):
	
	Using the orthogonality of the Bessel functions (\SeeChapter{see section Functional Analysis page \pageref{orthogonality of bessel functions}}) we can deduce from these relations the coefficients $\alpha_m^0$, $\alpha_m^n$, $\gamma_m^n$ (and the same for the others).

	For this purpose, let us suppose $n$ fixed and let us put $Z_m(r)=J_n(\lambda_m^n r)$. Let us show that for $m\neq p\Rightarrow Z_m\circ Z_p$ where the dot product is defined for Bessel functions (\SeeChapter{see section Functional Analysis page \pageref{orthogonality of bessel functions}}):
	
	Since $Z_m$, $Z_p$ satisfy the differential equation on $R(r)$, we have:
	
	By combining these two relations we get:
	
	By integrating member to member between $0$ and $L$ and taking into account that:
	
	we get:
	
	Hence the result stated earlier since $m\neq p\Rightarrow \lambda_m^n\neq \lambda_p^n$.
	
	The relation:
	
	Can therefore be written:
	
	Using the orthogonality $Z_m$, $Z_p$ for $m\neq p$ we deduce:
	
	The coefficients $\alpha_n$ are therefore given by:
	
	What is not easy to calculate by hand... and this so far... is only for the $\alpha_p^n$ coefficients!!!!!
	
	So we will assume that the reader can image that the same process is valid for the other coefficients. To avoid frustration however we will use Maple 5.00 to go a little further:
	\pagebreak
	\texttt{> F:=f(r,theta,t);\\
	> wave\_eq:=linalg[laplacian](F,[r,theta,z],coords=cylindrical)-diff(F,t,t); \# we take the Laplacian in polar coordinates\\
	> eval(subs(F=A(r,theta)*T(t),wave\_eq)); \#we apply separation of variables\\
	> expand(eval(subs(F=A(r,theta)*T(t),wave\_eq))/A(r,theta)/T(t)); \\ \#we simplify the result\\
	> time\_part:=numer(select(has,expand(eval(subs(F=A(r,theta)*T(t),wave\_eq))/\\A(r,theta)/T(t)),t)-lambda\string^2); \#we take the temporal part\\
	> space\_part:=numer(select(has,expand(eval(subs(F=A(r,theta)*T(t),wave\_eq))/\\ A(r,theta)/T(t)),r)+lambda\string^2); \#we take now the spatial part\\
	> sol\_time:=dsolve(time\_part,T(t)); \#we solve the spatial part\\
	> eval(subs(A(r,theta)=phi(r)*U(theta),space\_part));\\ \#we split the spatial part in radial and polar part\\
	> expand(eval(subs(A(r,theta)=phi(r)*U(theta),space\_part))/phi(r)/U(theta));\\ \#we simplify\\
	> polar\_part:=numer(select(has,expand(eval(subs(A(r,theta)=phi(r)*U(theta),\\
	space\_part))/phi(r)/U(theta)),theta)+kappa\string^2);\\
	\#we take only the polar part\\
	> radius\_part:=numer(select(has,expand(eval(subs(A(r,theta)=phi(r)*U(theta),\\
	space\_part))/phi(r)/U(theta)),r)-kappa\string^2); \#we take only the radial part\\
	> dsolve(polar\_part,U(theta)); \#we solve the angular part\\
	> assume(kappa,integer); \#we impose kappa as being an integer number\\
	> dsolve(radius\_part,phi(r)); \#we solve the radial part that makes appear the Bessel functions\\
	> dsolve(radius\_part,phi(r),series); \#the same solution expressed in limited expansion\\
	> sol\_radius:=eval(subs(BesselY=0,dsolve(radius\_part,phi(r)))); \#we the a special solution\\
	> sol\_lambda:=BesselJZeros(kappa,nu); \#we know that the answers are of the type J Bessel functions therefore the take the zeros in the real set (requires a Maple version great than the 4.00b)\\
	> sol\_radius:=BesselJ(kappa,sol\_lambda*r);\\
	> plots[animate3d]([r,theta,subs(kappa=2,nu=3,sol\_radius*cos(sol\_lambda*t)*\\cos(kappa*theta))],
	r=0..1,theta=0..2*Pi,t=0..2*Pi/subs(kappa=2,nu=3,\\
	sol\_lambda),coords=cylindrical,frames=20);}
	
	For those who have a PDF reader compatible with Adobe Flash, here is the resulting animated plot:
	\begin{center}
		\centering
		\includemedia[activate=pageopen,width=360pt,height=360pt,
	]{}{swf/circular_membrane.swf}
	\end{center}
	and for those who cannot see the animation here is a static decomposed plot:
	\begin{figure}[H]
		\centering
		\includegraphics[scale=0.25]{img/mechanics/circular_membran_vibration.jpg}
		\caption{Some vibration modes of a circular membrane}
	\end{figure}
	
	\pagebreak
	\subsection{Phasors}\label{phasors}
	There are several ways of expressing the wave functions that we have seen previously. Physicists (as well as electrotechnicians and engineers) use a formulation, named a "\NewTerm{phasor}\index{phasor}" or "\NewTerm{Fresnel representation}\index{Fresnel representation}", which simplify of the writing of equation and thus greatly simplifies the study of complex (or simple) problems. Phasors use the properties of complex numbers to express trigonometric wave functions in a simplified form in all phenomena where oscillations occur.

	What we call "phasor" is a function $f$ whose value is complex and which, in a space with $1$ dimension, is written:
	
	In all physics applications, $t$ is the time variable.
	
	Since this function is in $\mathbb{C}$, it has a real part that we will denote by $g$ and an imaginary part that we will denote by $h$. Their identification is easy since, as we have already proved it in our study of complex numbers (\SeeChapter{see section Numbers page \pageref{complex numbers}}):
	
	Therefore the real and imaginary part are obviously given by:
	
	The module of $f$ is easily computed by calculating:
	
	The real and imaginary parts vary when the position or time varies. The modulus does not change, it is always equal to $1$. The change is manifested simply by the simple variation of angle that makes the vector representing $f$ in the complex plane. This is a sufficient reason to speak of a phaser, since the variation of $f$ can be visualized as a simple change of angle or phase.

	If we are in a physical space with more than one dimension, say $3$, then the expression for $f$ becomes:
	
	The situation is a little more difficult to visualize (...). It is the same as in one dimension, but here everything happens along a direction defined in 3D space by the wave vector $\vec{k}$. Specifically, we will have:
	
	The variation in time remains the same as in the one spatial dimension case but a displacement in space is a little more complicated than in the one dimension case. Here, any spatial displacement in a direction that is not orthonormal to $\vec{k}$ will cause the scalar product to change so that the argument $f$ will vary. Here it is not only the magnitude or the norm of the wave vector $\vec{k}$ which gives the rate of variation of $f$ under a spatial displacement but also the angle that this displacement makes with respect to the direction of $\vec{k}$  since we have an argument that varies as $\vec{k}\circ\vec{r}$ and therefore depends on this angle denoted here $\varphi$. Indeed, we have:
	
	The quantity $\vec{k}$ is often referred to as the "\NewTerm{wave vector}\index{wave vector}". Physically it is most often connected to the equivalent of the (linear) momentum of the wave for which it is obvious that the usual definition of the momentum ($p=mv$) no longer makes sense.
	
	An important part of the systems studied in physics can not be characterized by a point and therefore described by a trajectory. A wave, an elastic band that oscillate do not have a single defined position, they are "continuous medium" over a certain interval. The question we ask ourselves in our attempt to describe them is rather how to describe the displacement of this medium in space and time. For example, for a wave, if we freeze time, how does the amplitude $A$ of this wave vary from one place to another in space? We can also freeze space by looking at one place and ask how the amplitude varies with time?
	
	The space coordinates and time now play a similar role as independent parameters. We measure the amplitude of the phenomenon at any time and in any place. We will therefore seek to obtain an expression of the type:
	
	in one or three dimensions.
	
	The important point is that we seek to express $A$, whose correct name is a "\NewTerm{field}", as a function of time and space coordinates. For example, if the wave is very regular perhaps is it adequately described by $A\sim \cos(\omega t)$ if we look at a single place and by $A\sim\cos(kx)$ if we are able to fix the time?
	
	The quantity $\omega$ will characterize the frequency of the variation of the same phenomenon. The name "\NewTerm{angular frequency}\index{angular frequency}" is easy to understand since the cosine or sine function cycles if its argument changes of $2\pi$ over a time of one period $\Delta t=T$.

	Let us recall that:
	
	In the description of harmonic systems (in Wave Mechanics, Quantum Physics or Quantum Chemistry), the phasor notation can be very useful as we have already said. The equation the most often encountered is the wave equation (which we proved at the beginning of this section). In one dimension, it is written for recall as:
	
	We check by simple substitution that some particular solutions are of the type:
	
	or a linear combination of which the most general form is:
	
	A quick and effective way of writing all these relations in a condensed and usefully pictorial way is to write:
	
	In all three cases, the substitution makes it possible to verify it (the differential equation). Let us take, for example, the sinus solution. Therefore:
	
	The replacement in the equation gives:
	
	Which will be satisfied if and only if:
	
	The substitution of the phasor as a solution of the wave equation transforms this last differential equation into a simple algebraic equation  $\omega=\omega(k)$ named "\NewTerm{dispersion relation}\index{dispersion relation}". This relation is obviously characteristic of the equation that generates it. That which appears above is particularly simple and characterizes a free wave in a non-dispersive medium, as described by the wave equation that we have written.

	The phasor solution:
	
	therefore also satisfies the wave equation, with the same dispersion relation. Is it then also the description of a wave? The answer is YES, and even twice rather than one, as we will see below.

¨	A physical wave is obviously not complex (at least in Classical Mechanics). The phasor solution is complex and therefore has a real part and an imaginary part. We show here that each of the two parts can represent a general real wave. Let us use a more general starting point. Let us assume that the amplitude $A_0$ is itself a complex number! We can therefore write:
	
	Therefore we have:
	
	We first study the real part of this expression and using trigonometric identities (\SeeChapter{see section Trigonometry page \pageref{remarkable trigonometric identities}}):
	
	This real part is therefore of the most general (and real) form of the monochromatic solution for the wave equation, namely:
	
	Clearly, it is sufficient to identify:
	
	which connect two parameters to two others. The real part of the phasor is therefore sufficient to fully describe the monochromatic wave.

	We can do exactly the same thing with the imaginary part of the phasor and prove in an identical way that it is sufficient to describe the monochromatic wave entirely.

	The conclusions is that it is therefore possible to use the phasir to do all the mathematical manipulations required by the physical problem and in the end, to keep only the real or imaginary part, according to what was agreed from the beginning.

	As we have already said, the most simple general real form of the solution is:
	
	This function has the behavior of a sine (or a cosine) whose amplitude is given by:
	
	Moreover, the initial conditions adjust $A_2$ such that, at $x=0$ and $t=0$ (initially), the field $A(x,t)$ has the  value $A_2$ therefore. These two conditions set these two parameters. We could have used the equally general form:
	
	Here the known amplitude is $A_0$ and the initial conditions are such that at $x=0$ and $t=0$, the field has the value:
	
	Again, two conditions set two parameters!
	
	The amplitude is an obvious thing, the phase a little less. For a function of the type:
	
	has any value that one wants when its $\arg$ (argument) is, say zero, it suffices to adjust the value of the phase $\delta$. It is like dragging the sinusoidal function along the $x$-axis, so as to satisfy the initial physical conditions imposed by the system studied.
	
	Of two functions of the sine or cosine type that do not start at the same point in their cycle, we say that they are "\NewTerm{out of phase}". This becomes vital when there is more than one wave present. Let us imagine the simplest case of two waves of the same amplitude:
	
	Here the argument ($\theta$) is a variable and the phases $\delta_1$ and $\delta_2$ the constant parameters. We consider the physical result of the wave resulting from the addition of these two waves.
	
	If $\delta_2-\delta_1=2k\pi$ for $k\in\mathbb{Z}$ the resulting wave will be a sinusoidal wave of amplitude $2$ times $A_0$. However, if $\delta_2-\delta_1=k\pi$ $k\in\mathbb{Z}$ the resulting wave will be identically zero everywhere. The difference is therefore considerable and we can produce all intermediate situations! It is therefore important to keep in mind the phase at the origin of the wave or better, its relative phase in relation to other waves of our physical system!
	
	In the phasor, either the real part or the imaginary part, is sufficient to give a general description of the wave (always monochromatic up to now). They are respectively composed of a cosine and a sinus and therefore phase-shifted of $\pi/2$ with respect to each other!

	It is desirable to return to the "dispersion relation" that we had obtained. We have seen that for the free monochromatic wave, that which is a solution of the homogeneous wave equation, this relation is written so that the phase of the solution corresponds to this reality:
	
	Still in the free case, we can have a physical situation that corresponds to the addition of several free monochromatic waves. The result is not monochromatic and is obviously written as a sum:	
	
	We often name it "\NewTerm{wave packet}\index{wave packet}" for obvious reasons (we will use this concept again in the section of Wave Quantum Physics). Since everything is free, each component will satisfy:
	
	In a non-dispersive wave medium, waves can propagate without deformation. Electromagnetic waves in unbounded free space are non-dispersive as well as non-dissipative and thus can propagate over astronomical distances. Sound waves in air are also nearly non-dispersive even in the ultrasonic frequency range. If not, that is, if high frequency notes (e.g., piccolo) and low frequency notes (e.g., base) propagate at different velocities, they would reach our ears at different times, and music played by an orchestra would not be harmonious. Most waves in material media are dispersive, however, and wave forms originally set up are bound to change in a manner that the wave energy is more spatially spread out or dispersed.
	
	In practice we consider four situation (we give also the Maple 4.00b code that permits the user to see the animation by themselves as it is very important for a good a deep comprehension):

	\pagebreak
	\begin{itemize}
		\item Dispersive wavepacket\index{dispersive wavepacket} where the waves in the packet oscillates and the packet becomes larger and larger (we strongly recommend to the reader to change the maximum values of \texttt{x} and \texttt{t}):
		
		\texttt{>with(plots): \\
				>animate(sum(.07*(exp(-(.1*k-3)\string^2)+exp(-(0.1*k+3)\string^2))*cos(.1*k*x-.1*k/\\
				sqrt(1+.1*(.1*k)\string^2)*t),k=1..100),x=-4..60,t=0..100,frames=60,numpoints=200,\\
				color=red);}		
		\begin{figure}[H]
			\centering
			\includegraphics[scale=0.65]{img/mechanics/dispersive_wavepacket.jpg}
			\caption{Dispersive Wavepackaet with Maple 4.00b}
		\end{figure}
	
		\item Non-dispersive wavepacket\index{non-dispersive wavepacket} where the waves in the packet don't oscillates and the packet conserve the same width:
			
		\texttt{>with(plots): \\
				>animate(sum(.07*(exp(-(.1*k-3)\string^2)+exp(-(0.1*k+3)\string^2))*cos(.1*k*x-.1*k/\\
				sqrt(1+.0*(.1*k)\string^2)*t),k=1..100),x=-4..20,t=0..30,frames=60,numpoints=200,\\
				color=red);}
		\begin{figure}[H]
			\centering
			\includegraphics[scale=0.65]{img/mechanics/nondispersive_wavepacket.jpg}
			\caption{Non-dispersive Wavepackaet with Maple 4.00b}
		\end{figure}
				
		\item Dispersive sinusoidal wave\index{dispersive waves} where we have multiple wavepackets that oscillates in each of the wavepackets and their width doesn't change:
		
		\texttt{>with(plots): \\
				>animate(sin(x-t)+sin(1.2*x-1.1*t),x=0..50,t=0..63,numpoints=150,\\
				frames=100,color=red);}
		\begin{figure}[H]
			\centering
			\includegraphics[scale=0.65]{img/mechanics/dispersive_sinusoidal_wave.jpg}
			\caption{Dispersive sinusoidal waves with Maple 4.00b}
		\end{figure}

		\item Non-dispersive sinusoidal wave\index{non-dispersive waves} where we have multiple wavepackets where the waves in the packet don't oscillates and their width doesn't change:
		
		\texttt{>with(plots): \\
				>animate(sin(x-t)+sin(1.2*x-1.2*t),x=0..50,t=0..63,numpoints=150,\\frames=100,color=red);}
		\begin{figure}[H]
			\centering
			\includegraphics[scale=0.65]{img/mechanics/nondispersive_sinusoidal_wave.jpg}
			\caption{Non-dispersive sinusoidal waves with Maple 4.00b}
		\end{figure}
	\end{itemize} 
	Here we will notice two things for the free-wave equation:
	\begin{enumerate}
		\item First, even for free waves, it is quadratic, so we can change the sign of the wave vector and / or the pulsation without affecting the equation. We observe trivially that for one or more components of the positive wave vector the wave propagates towards the $x$-crescents for these positive components and that conversely for one or more negative components the wave propagates towards the $x$ decreasing. Similarly, a negative or positive pulsation means that time varies towards the past or the future.
	
		\item On the other hand, certain types of waves do not obey an equation as simple as the homogeneous above. This is the case for water waves, for example, both because of the nature of the liquid in which they propagate and because of the gravitational recall force. Sometimes, too, a wave that would be totally free, or nearly so, seeks to propagate itself in an environment where the conditions of propagation are seriously affected. For example a sound wave that tries to spread in the mastic or an electromagnetic wave that seeks to propagate in a conductor (a metal). In this case, a large part of the difference between free wave and modified wave by the medium can be described by a change of the dispersion.
	\end{enumerate}
	
	\pagebreak
	\subsection{Superposition of periodic waves}
	The superposition of periodic waves is a very common case in electrodynamics, electronics, acoustics and marine engineering. In order not to repeat this study in each of the previously mentioned corresponding sections of this book, it seemed more appropriate to us to put all information needed here below.
	
	We will therefore focus on a particular case in order to see the intellectual approach. However, with computers, these calculations by hand have become somewhat obsolete. They must nevertheless be part of the general culture of the engineer or physicist.
	
	Let us first consider two periodic (harmonic) signals generated at the same point of the type:
	
	For pedagogical reasons, we noticed that it was better to work without phasers. So we will try to make a minimal use of them but for the beginning it will be indispensable. Indeed, the two waves and their sum can be represented in the Gauss plane (\SeeChapter{see section Numbers page \pageref{gauss plane}}) in the following form:
	\begin{figure}[H]
		\centering
		\includegraphics[scale=1]{img/mechanics/harmonics_in_gauss_plane.jpg}
		\caption{Phase representation of the two harmonics in the Gauss plane}
	\end{figure}
	wither therefore:
	
	The amplitude of the sum of the two signals is easily obtained by applying the cosine theorem (\SeeChapter{see section Trigonometry page \pageref{law of cosines}}). So let us first observe that since we are dealing with a parallelogram, we have (\SeeChapter{see section Geometric Shapes page \pageref{parallelogram}}):
	
	Therefore it comes:
	
	hence:
	
	The term $2A_1A_2\cos(\delta)$ is named the "\NewTerm{correlation term}". As the phasors are rotating, we have in reality:
	
	The phase $\phi$ of the sum of the waves is obtained simply by doing in the elementary trigonometry in the previous figure:
	
	Now, let us consider the particular case where the two waves are in phase:
	
	It then comes (\SeeChapter{see section Trigonometry page \pageref{periodicity trigonometric functions}}):
	
	and the amplitude is then written:
	
	So the amplitude will have two extremums:
	
	Hence:
	
	We then speak in the field of electrodynamics, and of acoustics of "\NewTerm{amplitude modulation}\index{amplitude modulation}". The "\NewTerm{modulation frequency}\index{modulation frequency}", also named "\NewTerm{beat frequency}\index{beat frequency}" is then:
	
	Let's see an example with Maple 4.00b (superimposition of a Bemol Si and of a Do with an intensity ratio of $0.8$):
	
	\texttt{>A:=1;B:=0.8;f1:=466.16;f2:=523;w1:=2*Pi*f1;w2:=2*Pi*f2;\\
		>plot(A*sin(w1*t)+B*sin(w2*t),t=0..0.05);}
	\begin{figure}[H]
		\centering
		\includegraphics[scale=1]{img/mechanics/amplitude_modulation_maple.jpg}
		\caption[]{Representation of the of the amplitude modulation with Maple 4.00b}
	\end{figure}
	And if we add the envelope:
	
	\texttt{>plot([A*sin(w1*t)+B*sin(w2*t),sqrt(A\string^2+B\string^2+2*A*B*cos((w2-w1)*t))],t=0..0.05);}
	\begin{figure}[H]
		\centering
		\includegraphics[scale=1]{img/mechanics/amplitude_modulation_with_envelope_maple.jpg}
		\caption{Representation of the envelope of the amplitude modulation with Maple 4.00b}
	\end{figure}
	In the general case, it is difficult to simplify the expression of the resulting signal:
	
	If the amplitudes are identical $A_2=A_1$ then we have:
	
	And thus using one of the Simpson formulas proved in the section of Trigonometry, it comes:
	
	Moreover, if the two frequencies are close, we have:
	
	Now, let consider the case where the waves advance (because until now they were stationary\label{stationary wave}):
	
	If the two amplitudes are identical, we have:
	
	Using Simpson's formula again, it comes:
	
	Therefore the amplitude is modulated according to the term:
	
	The amplitude is thus described by a wave function. The corresponding wave represents the progression of a carrier wave packet, progressing at the speed:
	
	We also notice that if we have the two progressive waves of direction of opposite propagation such as:
	
	then:
	
	If in addition the two harmonics are synchronous (of the same frequency), then we have:
	
	It is therefore a stationary wave but whose amplitude is modulated as a function of time and position. Indeed the reader will notice that at several points $r$ the wave is always zero.
	
	Now let us consider a system of loudspeakers (TV, stereo, etc.) that emit acoustic waves synchronously with the same amplitude and frequency. Since the source of the two harmonics is not reduced to a single point, we have in function of our distance $r_1$ the first loudspeaker and the of the second $r_2$:
	
	Applying Simpson's formula again, it comes:
	
	If we are equal distance from the two loudspeaker, it comes:
	
	Thus the amplitude is there maximal ("\NewTerm{constructive interference}\index{constructive interference}")) and is twice the amplitude of each harmonic. In fact, we find maxima of amplitude at each point where:
	
	thus:
	
	where $n$ is a relative integer (possibly equal to zero!). So, exactly between the two loudspeakers, we find a maxima. For a same $n$, all the values of $r_1$, $r_2$ that satisfy this relation are named "\NewTerm{ventral lines}\index{ventral lines}".
	
	We have in extenso the amplitude which is zero ("\NewTerm{destructive interference}\index{destructive interference}") whenever:
	
	Thus:
	
	where $n$ is a non-zero relative odd integer. For the same $n$, all the values $r_1$, $r_2$ that satisfy this relation are named "\NewTerm{nodal lines}\index{nodal lines}".
	\begin{figure}[H]
		\centering
		\includegraphics[scale=0.8]{img/mechanics/waves_interferences.jpg}
		\caption[Waves interferences of two loudspeakers]{Waves interferences of two loudspeakers (source: OpenStax)}
	\end{figure}
	
	\subsubsection{AM/FM Broadcasting}
	In telecommunications and signal processing, frequency modulation (FM) is the encoding of information in a carrier wave by varying the instantaneous frequency of the wave. This contrasts with amplitude modulation, in which the amplitude of the carrier wave varies, while the frequency remains constant.
	
	Before a radio station can begin broadcasting, the operator must first determine the frequency that best fits the information it wants to send, and then construct a suitable transmitter and antenna to cover the desired area over which the receivers will be located.  Next, the station operator must determine how the information to be broadcast will be superimposed onto the chosen radio frequency.  The process of superimposing information onto a radio frequency is named "\NewTerm{modulation}\index{modulation}".  There are many different modulation techniques available but the two most popular are "Amplitude Modulation (AM) and Frequency Modulation (FM).

	With "\NewTerm{Amplitude Modulation}\index{amplitude modulation broadcasting}", changes in the amplitude of the information signal causes a proportional change in the amplitude of the radio frequency signal (also referred to as a "carrier signal" as we know):
	\begin{figure}[H]
		\centering
		\includegraphics[scale=0.8]{img/mechanics/am_modulation_principle.jpg}
		\caption[AM modulation principle]{AM modulation principle (source: netZener)}
	\end{figure}
	With "\NewTerm{Frequency Modulation}\index{frequency modulation broadcasting}", changes in the amplitude of the information signal causes a proportional shift in the frequency of the "carrier signal". The instantaneous frequency deviation, the difference between the frequency of the carrier and its center frequency, is proportional to the modulating signal.
	\begin{figure}[H]
		\centering
		\includegraphics[scale=0.8]{img/mechanics/fm_modulation_principle.jpg}
		\caption[Waves interferences of two loudspeakers]{Waves interferences of two loudspeakers (source: netZener)}
	\end{figure}
	\begin{tcolorbox}[title=Remark,colframe=black,arc=10pt]
	FM radio is inherently less subject to noise from stray radio sources than AM radio. The reason is that amplitudes of waves add. So an AM receiver would interpret noise added onto the amplitude of its carrier wave as part of the information. An FM receiver can be made to reject amplitudes other than that of the basic carrier wave and only look for variations in frequency. It is thus easier to reject noise from FM, since noise produces a variation in amplitude.
	\end{tcolorbox}
	We will study in the section of Electrical Engineering how is build a FM modulator!
	
	\begin{flushright}
	\begin{tabular}{l c}
	\circled{95} & \pbox{20cm}{\score{4}{5} \\ {\tiny 21 votes,  69.52\%}} 
	\end{tabular} 
	\end{flushright}

	%to make section start on odd page
	\newpage
	\thispagestyle{empty}
	\mbox{}
	\section{Statistical Mechanics}\label{statistical mechanics}
	\lettrine[lines=4]{\color{BrickRed}S}tatistical mechanics, also named sometimes "statistical thermodynamics" or more generally "Statistical Physics", aims to explain the behavior of macroscopic systems (consisting of many interacting objects) from their microscopic characteristics using statistical and probabilistic tools. It is in a much more general way that quantum physics describes the properties and evolution of physical systems at the microscopic scale. Statistical mechanics is therefore built on this quantum description as discussed on mathematical developments that will follow.
	
	The approach presented here is to address the basic statistical mechanics to then deduce the fundamentals of Thermodynamics. Statistical mechanics is indeed with quantum physics and relativity, a cornerstone of modern physics in explaining phenomena from their constituents! It is important to perceive it immediately as a fundamental theory, and not as a mere attempt to justify retrospectively thermodynamics. Thermodynamics itself gains back more just and deeper understanding of its principles and methods.
	
	\subsection{Statistical Information Theory}
	
	The word "\NewTerm{information}\index{information}" is used in very diverse contexts, in totally different sense following scientific disciplines: we can cite as an example thermodynamics with the concept of entropy, applied physics with the theory of signal, biology with the genome theory and quantum physics with the probability of getting information.
	\begin{figure}[H]
		\centering
		\includegraphics[scale=0.9]{img/mechanics/information_theory.jpg}
	\end{figure}
	This raises the question whether it is possible to construct a theory of information and if it is unique? Our approach here is not information as such, but the amount of information! When we talk about quantity and measurement, we think to the concept of content or value of information. The information science in purpose must feel concerned by this questioning. If we define the "\NewTerm{infometrics}\index{infometrics}" as the set of technical mathematical and statistical measures of information, we wish to have a sufficiently clear definition of the concept of quantity of information that can lead us to define a measure, that is to say a set of well-defined operations, bringing us to clear axioms and whose result is a number.
	
	We are therefore concerned here with the foundations of the statistical information theory also known as "\NewTerm{Shannon's theory}\index{Shannon's theory}". The Shannon formula that emerges is certainly one of the fundamental concepts of physics and statistics as it affects all the irreducible brick physics, finance and statistics: information!!
	
	We will show (further below) that an isolated physical system has for most probable state the one that contains the more internal states and is therefore a fortiori the most unpredictable. The most improbable state is the one that is more predictable (concerns about emergence of life...). Therefore, since the unpredictability appears as an essential attribute of the information, we identify the quantitative measurement information to its improbability.
	
	Thus, the amount of information denoted by $h(x)$ given by the realization of an event $x$ of probability $P(x)$ is an increasing function $f$ of its improbability $\dfrac{1}{P(x)}$ following the postulate that the most probable state is the one that is the less predictable:
	
	Moreover, the realization of two independent events $x$ and $y$ intuitively provides an amount of information which is the sum of their respective amounts of information. We have seen in the section of Probabilities that for two independent event we have:
	
	
	That is to say we are looking for a function $f$ such that :
	
	The logarithm function (\SeeChapter{see section Functional Analysis page \pageref{logarithms}}) is by its properties a natural candidate for $f$ such that:
	
	where $\lambda$ is obviously a positive number.
	
	\begin{tcolorbox}[title=Remark,colframe=black,arc=10pt]
The choice of the base of the logarithm defines the information unit which is completely arbitrary. Thereafter, and unless otherwise specified, "log" will designate the logarithm in base $\alpha$.
	\end{tcolorbox}
	
	Thus, the "\NewTerm{quantity of intrinsic information}\index{quantity of intrinsic information}" of a event $x$ could be therefore using the properties of the logarithm:
		
	And can be considered as a measure of uncertainty about the event, or as that of the information needed to resolve this uncertainty. Here is an example that motivates this definition.
	\begin{tcolorbox}[colframe=black,colback=white,sharp corners]
\textbf{{\Large \ding{45}}Example:}\\\\
	Agree that knowing the realization of an event that has for probability 1/2 give us 1 bit of information (think for example to the binary system). We now launch $n$ coins and take knowledge of the result of the experience: we acquire $n$ bits of information. The event in question had the probability $\left(\dfrac{1}{2}\right)^n$. Therefore the number of bits of information is if we chose $\alpha=2$ and $\lambda=1$:
	
	This is an example that motivate our choice.
	\end{tcolorbox}
	
	\textbf{Definitions (\#\mydef):}
	\begin{enumerate}
		\item[D1.] We define the "\NewTerm{intrinsic information by pair}\index{intrinsic information by pair}" of two events $x$ and $y$ of joint probability $P(x, y)$ (\SeeChapter{see section Probabilities page \pageref{joint probability}}) by:
		
		\item[D2.] We define also the "\NewTerm{conditional information}\index{conditional information}" of $x$ when $y$ is known by (\SeeChapter{see section Probabilities page \pageref{bayesian inference}}):
		
		which can therefore also be written (this is just a different way to note that we have already presented in the section of Probabilities):
		
		This is the amount of information remaining about $x$ once $y$ has been observed. 
		
		We have proved in the section of Probabilities that the Bayes' formula allows us to notice immediately that if $x$ and $y$ are independent :
		
		This is consistent with common sense.
		
		\item[D3.] We also need sometimes to measure the amount of information that a particular variable for example $y$ brings on another for example $x$. This is particularly the case when we identify $x$ as the selection of a signal applied to the input of a channel and $y$ the corresponding observed output signal (analysis used in code correction error as seen in the section of Error-Correcting code).
		
		Let $P(x)$ the probability that $x$ has been emitted and $P(x/y)$ the a priori probability that $x$ was emitted, given that  $y$ was received.
		
		A measurement of this quantity of information, named "\NewTerm{mutual information}\index{conditional information}" is denoted by:
		
		It is the logarithmic measure of the increase in the probability of $x$ (thus the decrease in its amount information) due to its conditioning $y$. We can check if it is coherent with two situations:
		\begin{enumerate}
			\item If the data of $y$ is equivalent to $x$ (case of a perfect channel), it is equal to the intrinsic information $h (x)$:
			
			\item It is null if, conversely, $x$ and $y$ are independent.
			
		\end{enumerate}
		Obviously we have:
		
		And using the formula of compound probabilities (\SeeChapter{see section of Probabilities page \pageref{compound probabilities}}):
		
		The latter equality justifying the term "mutual". While the intrinsic information is positive, mutual information can be negative. We will see that his average, much more important in practice, can not be negative.
	\end{enumerate}
	The individual events are in general less important than the mean effect, we will consider them later as coming from a random source, discrete, finite, stationary, and white (i.e. independent successive events). The events are interpreted as choosing a symbol in the alphabet of the source. Let $n$ be the size of the alphabet, and $x_1,...,x_n$ its symbols. The source is therefore described by the random variable X, which takes values in the alphabet, with respective probabilities $p_1,...p_n$, such as:
	
	\textbf{Definition (\#\mydef):} The average amount of information from this source is the expected mean of the intrinsic information of each symbol of the alphabet of the source (\SeeChapter{see section Statistics page \pageref{expected mean continuous variable}}). Will name it "\NewTerm{entropy}\index{entropy}" and denoted by $S(X)$ and given by the relation:
	
	named "\NewTerm{Shannon formula}\index{Shannon formula}\label{shannon entropy}" and it is indeed a mean because its general expression for a discrete random variable is as we have seen in the section Statistics given by:
	
	but with in this case:
	
	The reader can also notice that if $n$ the event are of equal probabilities $1/n$ we have the famous relation:
	
	\begin{tcolorbox}[title=Remark,colframe=black,arc=10pt]
	For a business real application case of the Shannon formula you can see the subsection on the Diversity Index in the section of Statistics.
	\end{tcolorbox}
	However, the relation:
	
	is an abuse of notation. Indeed, the expectation makes sense if $h (x)$ is a function of $x$. But in our case $h (x)$ does not depend on the values of $x$, but only to the associated probabilities. We will therefore sometimes more rigorously denote the entropy of a distribution by:
	
	The "\NewTerm{joint and conditional entropies}\index{joint and conditional entropies}" are defined in similar manner with the suitable notations:
	
	and (we adopt this time the symbol "|" instead of the "/" to indicate the conditional relation):
	
	The "\NewTerm{mean mutual information}\index{mean mutual information}", sometimes badly named "\NewTerm{mutual information}\index{mutual information}" is defined also directly:
	
	\begin{tcolorbox}[title=Remark,colframe=black,arc=10pt]
	The reader must notice that the definition of the amount of information, by a logarithmic measure may seem arbitrary, though reasonable, given the expected properties of such a measure. Shannon, and later Khintchine showed that taking into account certain properties putted as axioms laid the logarithmic function be the only one suitable choice.
	\end{tcolorbox}
	\pagebreak
	\begin{tcolorbox}[colframe=black,colback=white,sharp corners]
	\textbf{{\Large \ding{45}}Examples:}\\\\
	E1. Given a binary random variable which value is $1$ with probability $p$ (and therefore $0$ with probability $1-p$). Its entropy is given by:
	
	with obviously $0<p<1$ and with a logarithm in base $2$ such that for any event with two equiprobable states, the entropy of obtaining one of the two states is equal to unity. Indeed:
	
	Therefore we must have $\lambda=1$.\\
	
	The entropy is shown in the figure below, in "Shannon" units (unit corresponding to the use of the logarithm with base $2$). We remark its symmetry about the value $1/2$ value at which it reaches its maximum, equal to $1$ as expected intuitively for such an experience.
	\begin{figure}[H]
		\centering
		\includegraphics{img/mechanics/entropy_binary.jpg}
		\caption{Entropy of a binary variable with $\lambda=1$}
	\end{figure}
	E2. In the field of astronomy where it is often necessary to compress images we seek to calculate the "\NewTerm{data redundancy level}\index{data redundancy level}" to choose a compression ratio eventually applicable. This is done by calculating the information contained in an image using its entropy. Given an image whose pixels are coded on $N$ bits, thus $2^N$ possible values by pixels, and a proportion (probability) $p_i$ of  pixels that can take each color. The entropy of the image is given by (with the notation usage that is traditional in digital imaging):
	
	which is therefore expressed in bits per pixel and represents the number of bits that are necessary to encode the image without loss of information.
	\end{tcolorbox}
	We must now make the connection between the statistical information theory and statistical mechanics:
	
	\subsubsection{Kullback–Leibler divergence}\label{kullback-leibler divergence}
	In mathematical statistics, the Kullback–Leibler divergence (also called "\NewTerm{relative entropy}\index{relative entropy}") is a measure of how one probability distribution diverges from a second, expected probability distribution.[1][2] Applications include characterizing the relative (Shannon) entropy in information systems, randomness in continuous time-series, and information gain when comparing statistical models of inference. In contrast to variation of information, it is a distribution-wise asymmetric measure and thus does not qualify as a statistical metric of spread. In the simple case, a Kullback–Leibler divergence of $0$ indicates that we can expect similar, if not the same, behavior of two different distributions, while a Kullback–Leibler divergence of $1$ indicates that the two distributions behave in such a different manner that the expectation given the first distribution approaches zero. In simplified terms, it is a measure of surprise, with diverse applications such as in statistics and machine learning (\SeeChapter{see section Numerical Methods page \pageref{data mining}}).
	
	Given the Shannon entropy introduced earlier before:
	
	Kullback-Leibler Divergence is just a slight modification of our formula for entropy. Rather than just having our probability distribution $p_X$ we add in an approximating distribution $p_Y$. Then we look at the difference of the log values for each (and we put $\lambda=1$):
	
	Essentially, what we're looking at with the KL divergence is the expectation of the log difference between the probability of data in the original distribution with the approximating distribution. So the Kullback–Leibler divergence is measure of entropy increase due to the use of an approximation to the true distribution rather than the true distribution itself.
	
	The "\NewTerm{discrete Kullback-Leibler divergence}\index{discrete Kullback-Leibler divergence}" between two distributions $X$ and $Y$ is given by:
	
	also sometimes denoted:
	
	and the "\NewTerm{continuous Kullback-Leibler divergence}\index{continuous Kullback-Leibler divergence}" between two distributions $X$ and $Y$ is given by:
	
	The Kullback-Leibler divergence is a kind of distance between the probability distributions of $X$ and $Y$, although $D_\text{KL}(p_X,p_Y)\neq D_\text{KL}(p_Y,p_X)$ (but it is therefore not a true "distance" because it is asymmetric!!!!).
	
	Let us 
	
	As it is quite abstract let us do a companion example of two Normal distributions:
	 
	So now we want obviously to calculate:
	
	But we have:
	 
	Therefore:
	
	But we have:
	
	Therefore:
	
	So finally:
	
	The Kullback–Leibler divergence has the following properties:
	\begin{itemize}
		\item[P1.] It is positive definite such that $D_\text{KL}\geq 0$
		\begin{dem}
		We start from:
		
		As $-\log(z)$ is a convex function we can apply the Jensen's equality (\SeeChapter{see section Statistics page \pageref{jensen inequality}}) that is given for recall by:
		
		Therefore:
		
		Hence if we simplify:
		
		This is equal to write:
		
		Hence:
		
		Therefore:
		
		\begin{flushright}
			$\square$  Q.E.D.
		\end{flushright}
		\end{dem}
	
		\item[P2.] It is zero if both distribution are equal $D_\text{KL}(p_X,p_Y)=0$ if $p_X=p_Y$.
		\begin{dem}
		The proof is immediate:
		
		\begin{flushright}
			$\square$  Q.E.D.
		\end{flushright}
		\end{dem}
	\end{itemize}
	
	\subsection{Boltzmann Law}
	First we will demonstrate through a simple case that for any system closed and isolated system, the most likely state is the state is that of equilibrium!
	
	Let us consider an isolated system (a system is said to be "\NewTerm{isolated}\index{isolated system}" when it is impervious to any flow - heat (adiabatic), matter, fields, etc.) populated by $N$ distinguishable particles. This system is divided into $2$ compartments (or levels) identical and separated by an impermeable wall. Each compartment is assumed to contain a quantity $N_1,N_2$ number of particles.
	
	For a given configuration of the system, we speak of "\NewTerm{microstate}\index{microstate}" in the sense that it is possible for the thanks to the quantity of particles to measure a macroscopic known quantity such as energy, mass, pressure, etc.
	
	If we fix this particular system, it is of course possible for number $N$ of particles to imagine a given number of macro-states (don't forget that there are two compartments). Such as:
	\begin{itemize}
		\item One particle: 2 macrostate (2 possible configurations, therefore 1 configuration by macrostate)

		\item Two particles: 3 macrostate (4 possible configurations by permutations of the compartments

		\item Three particles: 4 macrostate (8 possible configurations by permutations of the compartments)

		\item Four particles: 5 macrostate (16 possible configurations by permutations of the compartments)

		\item $n$ particles: $n+1$ macrostate  ($2^n$ possible configurations by permutations of the compartments
	\end{itemize}
	\textbf{Definition (\#\mydef):} We name "\NewTerm{microstate}\index{microstate}\label{microstate}", a configuration of permutation of macrostate.
	
	The probability of finding a system in a given state depends upon the multiplicity of that state. That is to say, it is proportional to the number of ways you can produce that state. Here a "state" is defined by some measurable property which would allow you to distinguish it from other states. 
	\begin{tcolorbox}[colframe=black,colback=white,sharp corners]
	\textbf{{\Large \ding{45}}Example:}\\\\
	In throwing a pair of dice, that measurable property is the sum of the number of dots facing up. The multiplicity for two dots showing is just one, because there is only one arrangement of the dice which will give that state. The multiplicity for seven dots showing is six, because there are six arrangements of the dice which will show a total of seven dots:
	\begin{figure}[H]
		\centering
		\includegraphics{img/mechanics/micro_macro_states.jpg}
	\end{figure}
	\end{tcolorbox}
	Let us now determine using combinatorial analysis (\SeeChapter{see section Probabilities page \pageref{combinatorial analysis}}) the number of possible microstates for each macro-state named "\NewTerm{multiplicity}\index{multiplicity}" and denoted by $\Omega$. By analogy, this corresponds to imagine that the system is a stem on which are strung balls (particles) and the stem is separated by an imaginary boundary in one of its points (Chinese abacus). In such a situation, we have:
	
	This gives us all the possible arrangements of the "lefts particles" with "right particles" (of the border) for a given macro-state (the number of ways in which particles can be shared between the two compartments). But we also have in this particular case:
	
	But this corresponds to the combinatorial such that:
	
	and therefore:
	
	We finally have for all macrostates of a system of $N$ particles, a total of:
	
	possible microstates (configurations). But we have seen in the initial example that:
	
	Thus, the probability of existence of a given microstate is $2^{-N}$ and is equally likely!!
	
	We can now state the first postulate of statistical mechanics ("\NewTerm{Gibbs postulate}\index{Gibbs postulate}") all discernible and accessible microstates of an isolated system are equally likely.
	
	Let us go back to our original question about equilibrium:
	
	The notion of equilibrium associated to a macro-state is provided by classical thermodynamics. We say that a system is said to be in equilibrium when its state is characterized by the temporal independence of macroscopic quantities (mass, power, pressure, ...) and the constancy of the thermodynamic potentials (internal energy, enthalpy, Gibbs energy, ...).
	
	To find out why the equilibrium is the most probable state, we only need to find what is the pair $(N_1,N_2)$ that maximizes:
	
	since all microstates are equally probable anyway. It is easy to check that this maximum is given for:
	
	We can now announce the second postulate of statistical mechanics: the equilibrium state is the state which is the one with the largest number of configurations (microstates) and is the most likely state!!
	
	Or in other words: A system reaches its equilibrium when its entropy is maximum and the entropy can only grow (fundamental postulate in physics!)!!
	
	\begin{tcolorbox}[title=Remark,colframe=black,arc=10pt]
	Some physicists have supposed that this fundamental postulate of increasing entropy is even stronger that the principle of the Darwin's natural selection model. Indeed, if we imagine molecules dispersed in a primordial ocean, by the principle of entropy, they will self-organize naturally to maximize the dissipation of heat from the ocean. What is powerful is that principle provides a physical basis for Darwin's natural selection phenomenon and helps explain some evolutionary trends that mere natural selection is not able to explain to this day: why some organisms have a characteristic $X$ rather than $Y$ is not due to the fact that $X$ is better suited than $Y$ but because the surrounding physical constraints makes that $X$ is more likely to move towards maximum entropy than $Y$.
	\end{tcolorbox}
	
	\pagebreak
	Let us now consider the following system:
	\begin{figure}[H]
		\centering
		\includegraphics{img/mechanics/entropy_evolution.jpg}
		\caption{Example of change of the entropy of a system}
	\end{figure}
	The distribution function $P (x)$ which describes the position of the particle along the $x$ axis at the equilibrium $(t=0)$ will change to another distribution function corresponding to a new equilibrium at $t_3$. At equilibrium $P (x)$ is constant (uniformly distributed). But between the two equilibria, it evolves and becomes wider. So we lose the information on the position of particles. We can therefore re-state the second assumption, saying that a system that is out of equilibrium always operates in the sense of a loss of information (an enlargement of the corresponding probabilistic distribution function).
	
	Meanwhile, the second principle of classical thermodynamics tells us that natural evolution must correspond to an increase of entropy $(\mathrm{d}S>0)$. So there must be a close link between the information we have about the state of each particle and the entropy of the system.
	
	The case we just described above shows clearly  that the parameters or concepts: number of configurations, disorder, equilibrium, amount of information and entropy of an isolated system can be used to represent the state of a system. These parameters have the same function. Mathematical relations must therefore connect each other!
	
	Let us recall that we have prove that the statistical information metric entropy of a system is given by:
	
	If we apply this relation to the case of a physical system in equilibrium for which we wish to calculate the entropy, we have proved earlier that:
	
	We still need to know what to what corresponds this constant probability. We have previously shown that at equilibrium, we had:
	
	which is therefore the number of microstates at equilibrium. Thus, the probability of drawing a microstate among all of them is:
	
	that we simply write (a little dangerous ...) by tradition:
	
	We therefore have:
	
	As the probabilities of microstates are equally likely and that we sum over all of these, it comes:
	
	and therefore (without forgetting that in this special case the probabilities all have the same values!):
	
	Since the equilibrium is related to the maximum disorder, and the disorder is related to the missing information, it seems reasonable to link the statistical entropy of information with statistical entropy  of thermodynamics in physics. For this, we need that the constant allows us to get the right units and it comes naturally to choose this constant as it is equal to the Boltzmann constant $k$ has the same units as the thermodynamic entropy. So:
	
	We still have to choose the base of the logarithm. The experience shows that we must choose the natural logarithm that gives us the possibility to fall back on the results of classical mechanics after developments.

	So we finally get the "\NewTerm{Boltzmann law}\index{Boltzmann law}\label{boltzmann law}":
	
	which gives us the thermodynamic entropy of a system in equilibrium!
	
	By the mathematical properties of the mean (\SeeChapter{see section Statistics page \pageref{means and averages properties}}), especially the multiplication by a constant, we have for a set of $N$ subsystems:
	
	since as we pointed out earlier in our presentation of the Shannon formula, the entropy $S$ is an expected mean (without forgetting that in this case the probabilities are of equal value!).
	
	Before we continue we would like to quote John Wheeler because what he said in his book \textit{Black Holes} is quite representative of the general trend of the physics of the 21st century (and some high level theories going also in this direction are quite promising):
	
	\begin{fquote}[] I think of my lifetime in physics as divided into three periods. In the first period... I was in the grip of the idea that Everything is Particles... I call my second period Everything is Fields. ... Now I am in the grip of a new vision, that Everything is Information 
 	\end{fquote}
	
	\subsection{Statistical Physics Distributions}\label{statistical physics distributions}
	We will distinguish in this book $4$ different statistics that come or not from quantum effects that will lead us to $4$ well known distinct statistical distributions. These are the: Maxwell distribution, Maxwell-Boltzmann distribution, Fermi-Dirac distribution and Bose-Einstein distribution. They have numerous applications in physics as the blackbody radiation that will be study in the section of Thermodynamics but also in Electrokinetic.
	
	\subsubsection{Maxwell Distribution (velocity distribution)}\label{maxwell distribution}
	For a gas at equilibrium, let us ask the following question: What is the probability that a molecule has its velocity components between $v_x$ and $v_x+\mathrm{d}x$, and $v_y$ and $v_y+\mathrm{d}y$, and also $v_z$ and $v_z+\mathrm{d}z$ in a conventional cartesian coordinate system?
	
	This probability $\mathrm{d}^3P$ depends of $\vec{v}$ (that is to say of: $(v_x,v_y,v_z)$) and of $\mathrm{d}v_x,\mathrm{d}v_y,\mathrm{d}v_z$. It does not depend on the position of the molecule since the gas is assumed to equilibrium with respect to its center of mass!
	
	We postulate in a first time that $\mathrm{d}P^3$ is proportional to each of the intervals $\mathrm{d}v_x,\mathrm{d}v_y,\mathrm{d}v_z$ such as:
	
	and there is no preferred spatial directions. We can make a circular rotation of the Cartesian axes, the probability will be unchanged (isotropy of space).
	
	The isotropy brings that the function $f(\vec{v})$ does not depend on vector $\vec{v}$, at best it dependent on the norm of the speed, such as:
	
	Given $\mathrm{d}P_x$ (also often denoted $\mathrm{d}P(v_x)$ depending on the context) the probability for a molecule to have its component along the O$x$ axis between $v_x$ and $v_x+\mathrm{d}v_x$ then:
	
	also same with:
	
	The isotropy of space brings to write:
	
	The law of compound probabilities (\SeeChapter{see section Probabilities page \pageref{compound probabilities}}) bring us to write:
	
	
	Determining the functions $f$ and $\varphi$ reveals within the following mathematical methodology:
	
	Since:
	
	Which gives us finally:
	
	The left side of the last equality depends only on $v$, the right one only on $v_x$. The result can only be a constant that we will denote $x_x$. It follows that:
	
	By integrating:
	
	Therefore:
	
	and let us put $c^{te}=\ln(A)$. It comes therefore:
	
	Identically, we have:
	
	The law of compound probabilities involving:
	
	We finally have:
	
	Let us notice that $\alpha$ is necessarily negative otherwise the probability for a molecule to have an infinite velocity component would be infinite which would mean that all the molecules will have an infinite velocity and therefore an infinite energy!!!

	We put:
	
	with $B>0$. 

	Finally, it comes:
	
	It remains to us to normalize $A$. The probability of a molecule having a velocity component between speed $[-\infty,+\infty]$ or a velocity norm between $[0,+\infty]$ is equal to $1$ ($100\%$ chance). So using exactly the same method computation than that seen in the section Statistics for the Normal distribution, we have:
	
	We have so far spoken in terms of probabilities. An equivalent language is to seek in a box containing $N$ molecules, the number $\mathrm{d}N$ of molecules having given characteristics, that is to say for example, the number of molecules $\mathrm{d}N(v_x)$ having a velocity component between $v_x$ and $v_x+\mathrm{d}v_2$. This number being obviously equal in one dimension to:
	
	More generally:
	
	To get $\mathrm{d}P$ we put ourselves in the configuration space of velocities that is to say, in a Cartesian coordinate system of coordinates $(\text{O},v_x,v_y,v_z)$. The components $(v_x,v_y,v_z)$ are not independent as related by the relation:
	
	The extremity of the velocity vectors $\vec{v}$ having a velocity norm of $v$, that is to say of velocity components bounded as the relation above, is in velocity space of configuration on a sphere of radius $v$. It will be the same for velocity vectors of having a norm equal to $v+\mathrm{d}v$.

	In the velocity space configuration, we outline a portion of space between the sphere of radius $v$ and the sphere of radius  $v+\mathrm{d}v$, equal in volume to $4\pi v^2\mathrm{d}v^2$.
	
	The probability $\mathrm{d}P^3$ is proportional to $\mathrm{d}v_x\mathrm{d}v_y\mathrm{d}v_z$ that is to say to the  elementary volume in the velocity configuration space. To get $\mathrm{d}P(v)$, we must integrate $\mathrm{d}P^3$ at all possible speeds vectors, that is to say having their extremities between the two spheres. This integration is particularly simple because the norm of $\vec{v}$ in this configuration space (inter-volume) is constant. So we get:
	
	and therefore, in a box containing $N$ molcules, the number of molcules $\mathrm{d}N(v)$ have a norm between $v$ and $v+\mathrm{d}v$ is equal to:
	
	Let us recall now that according to what we have studied in the section Statistics, the average value $\langle G \rangle$ of a quantity $G$ is the product of $G$ by all the probabilities of $G$ integrated with all possible values of this quantity such that in one dimension we have (according to the $x$ axis):
	
	where the integral is calculated by making the change $v_x^2=x$ and where we have then used an integration by parts.	

	We have respectively generally :
	
	where the integral has been calculated in the same manner.
	
	Therefore:
	
	or:
	
	Thermodynamics or fluid mechanics (see Virial theorem) gives us for a perfect monatomic gas (see the sections of Continuum Mechanics and Thermodynamics) of constant heat capacity and volume:
	
	If we hypothesize that the part of the energy associated with the temperature is due to the kinetic motion of molecules, we can write:
	
	We then define the "\NewTerm{average thermal velocity}\index{average thermal velocity}" or "\NewTerm{root-mean-square (rms) speed}\index{root-mean-square (rms) speed}":
	
	A numerical application for the electron in the case of semiconductors (where we approximate some relations by the Maxwell-Boltzmann distribution) gives an average thermal velocity at room temperature of $120,000\;[\text{ms}^{-1}]$.
	
	But to return to our distribution .... So we have:
	
	Then we finally have:
	
	that is the velocity distribution in a monatomic gas which plot looks (the units in ordinate axes are arbitrary):
	\begin{figure}[H]
		\centering
		\includegraphics{img/mechanics/maxwell_distribution.jpg}
		\caption{Maxwell's velocity distribution in a monatomic gas}
	\end{figure}
	The previous relation thus gives the proportion of gas molecules having at a given time $t$ a velocity $v$. We see that the probability density extends to infinity and we find a nonzero probability (approaching zero) that the particle have extremely high speed. In fact, if we have a system of identical particles at the same temperature $T$, then we will find a small given number of them with these very high speeds. At the heart of the stars it is these rare and very fast particles that are able to overcome the Coulomb repulsion between the nuclei and make them approach one another to initiate nuclear fusion mechanism which is the source energy of the star.
	
	We then for a single spatial direction and using a common notation in the English literature (by adopting the preceding developments we fall relatively easily on this result) the following distribution function:
	
	and for the three spatial dimensions still using the notation common in English literature (some terms are simply placed under the root):
	
	So this distribution function give us the possibility to define the most likely velocity (the "modal" value as we say in statistics) sometimes denoted $v_m$, which is the maximum of the curve $f (v)$, or where the first derivative is equal to zero. Thus, for a single spatial dimension we have for the proportion of gas molecules:
	
	Therefore (you notice that the result would have been the same if we had taken all spatial dimensions ... which is consistent!):
	
	hence the most likely speed in one dimension (caution! this is not the maximum speed since we know that the distribution extends to the infinite speed):
	
	For the average speed (the mean) and using directly all spatial dimensions we have:
	
	We then have an integral of the type:
	
	The best is to decompose $x^3e^{-ax^2}$ in $x^2xe^{-ax^2}$ and then we integrate by parts. Then we have:
	
	The last integral is easily calculable. This is the same as the one we already solved in the section Statistics for our introduction of the Normal law. So:
	
	Therefore it comes:
	
	but this time the result is dependent on the number of spatial dimensions that we take!
	
	\subsubsection{Maxwell-Boltzmann Distribution}
	In corpuscular quantum physics, we learn that the energy of a particle is quantized, that is, the possible values for the energy form a discrete spectrum (\SeeChapter{see section Corpuscular Quantum Physics page \pageref{quantification}}). Even if, in a number of common situations, for a particle and even more for a system consisting of a large number of particles, the energy levels are so tight that we can handle, mathematically, this spectrum as being continuous ("continuum approximation"), the fact remains that they are rigorously quantified. This quantified approach of physical distributions at the corpuscular level of the material is often referred as "\NewTerm{quantum statistics}\index{quantum statistics}".
	
	A particle having a level of energy $E_i$, can be in different sub-states.

	We know (we did not use the Schrödinger equation in the chapter of Corpuscular Quantum Physics to provie this strictly) that, to describe an atom, we introduce four quantum numbers, namely:
	\begin{enumerate}
		\item the principal quantum number that quantifies energy $n=n_r+n_\theta$

		\item the azimuthal quantum number that quantifies the momentum $n_\theta$

		\item the quantum magnetic  number that quantifies the magnetic moment $m_l$

		\item the tree spin that quantifies the "proper rotation" (...) of the electrons $s$
	\end{enumerate}
	Thus for a same energy (for a particular value of the principal quantum number), an atom or an electron can have different values of the secondary, magnetic or spin quantum numbers.

	To qualify the possibility  of sub-states corresponding to a same energy, we use the term "\NewTerm{degeneration}\index{degeneration}" and we translate by the variable $g_i$ the number of degeneration corresponding to the same level of energy $E_i$.

	We will consider a system of $N$ particles that are placed on $K$ different energy levels $E_i$. We find $n_i$ particles on the energy level $E_i$. We then always have the following relations for the total energy (which we denote by the same letter as that used in thermodynamics) and for the number of particles:
	
	We will assume (this is important) these quantities as being constant and that the system is completely determined by the distribution $\{n_i\}$  of the particles on the $K$ different energy levels.
	There are many possible microscopic configurations $\Omega$ that are compatible with the distribution $\{n_i\}$. There are (see permutations with repetition in the section of Probabilities):
	
	But we have neglected the possible degeneration of the levels $i$. If there exist $g_i$ sublevels with an energy $E_i$, we have then:
	
	This can be checked with the following figure:
	\begin{figure}[H]
		\centering
		\includegraphics{img/mechanics/maxwell_boltzmann_states_representation.jpg}
		\caption{States in the Maxwell-Boltzmann distribution}
	\end{figure}
	\begin{tcolorbox}[title=Remark,colframe=black,arc=10pt]
	The number $g_i$ is then the degeneration of the energy state $E_i$, ie the number of states with this energy.
	\end{tcolorbox}
	Taking the logarithm, it comes:
	
	and using Stirling's formula (\SeeChapter{see section of Theoretical Computing page \pageref{stirling}}):
	
	keeping in mind that the latter is a rough approximation for smaller values of $n$ but becomes an acceptable approximation for $n$ greater than $1,000$.
	Then we can write:
	
	We now look for the most probable distribution, that is to say one that maximizes $\Omega$ (and thus that implicitly maximizes entropy as it can only increase as show us every day life experience so far for an isolated system). To find the extremum, we will differentiate this expression such that:
	
	so all other parameters other than $n_i$ are fixed.

	But, as:
	
	where $\alpha,\beta$ are constants that help to ensure consistency of dimensional analysis (in other words: that the units are coherent ...).
	
	It is therefore equivalent and necessary to take into account also these intrinsic parameters (the fact to in such a clever way null terms is named "method of Lagrange multipliers" as seen in the section of Theoretical Computing) to write:
	
	in a more complete form below:
	
	So after rearrangement:
	
	giving after a first simplification (elimination of the derivative of constants):
	
	Which brings us to write (since all the terms of the sum indexed on a different value $i$ from that of the partial derivative are equal zero):
	
	Which give us finally:
	
	which will also therefore have to be equal zero. Thus, we have cleverly show a result which is nonetheless very powerful. By imposing to maximize the entropy we revealed two terms (each including a constant that we will have to determine later!) and whose sum has to be equal zero to ensure the experimental validity of theoretical model when the entropy of the isolated system is maximal.
	
	But this latter relation is equivalent as to seek (which will be very useful later!):
	
	Indeed, let using again the clever method of Lagrange multipliers as see in the section of Theoretical Computing (adding zero terms assuming energy levels $E_i$ and the number degeneration $g_i$ as fixed) and remembering (\SeeChapter{see section Differential and Integral Calculus page \pageref{usual primitives}}) that :
	
	Therefore we have:
	
	Therefore:
	
	Let us recall that we have for the entropy of a system:
	
	and therefore:
	
	and as:
	
	Then we have:
	
	Thermodynamics leads us to (\SeeChapter{see section Thermodynamics page \pageref{thermodynamic identity}}):
	
	So we see appear the chemical potential $\mu$ which has an influences when the number of particles considered varies.

	If all the particles (or particle groups) are identical $\mu_i=c^{te}=\mu$ thus:
	
	Which finally bring us to:
	
	and bring is to identify:
	
	and therefore:
	
	We can then rewrite the relation:
	
	in the final, traditional form of "\NewTerm{Maxwell-Boltzmann function}\index{Maxwell-Boltzmann function}":
	
	That we we find most frequently in the literature as follows:
	
	and especially in this form:
	
	We will use this relation in our study of semiconductor theory (\SeeChapter{see section Electrokinetics page \pageref{semiconductors}}) as well as in our study of plasmas (\SeeChapter{see section Continuum Mechanics page \pageref{plasmas}}).

	As $N=\sum_{i=1}^K n_i$, we have:
	
	We can then calculate $n_i/N$ and we get the discrete formulation of the "\NewTerm{Maxwell-Boltzmann statistics}\index{Maxwell-Boltzmann statistics}":
	
	This relation therefore gives the ratio (or "proportion" or "probability") between the number of particles which by hypothesis does not interact with each other (otherwise the model is too complex) in a given energy state $E_i$ and the number of particles that do not interact with each other and can take different discrete energy states $E_i$. Thus, knowing $N$, it is possible using this relation to know the proportion of particles in a particular energy state.
	Obviously, if there is only one sub energy level, this relation reduces to (it is also in this form that we find it often in the literature):
	
	\begin{tcolorbox}[title=Remark,colframe=black,arc=10pt]
	The Maxwell-Boltzmann statistics applies in the absence of interaction between particles and is therefore valid for an ideal gas, but does not apply, for example, for a liquid. Moreover, it applies only to high temperatures when quantum effects are negligible. At low temperatures we use the Bose-Einstein statistics for bosons (referring to the name of Bose...) and the Fermi-Dirac statistics for fermions (referring to the name of Fermi...) that we will study and prove further below.
	\end{tcolorbox}
	We name the term in the denominator, the "\NewTerm{canonical partition function}\index{canonical partition function}". It is most often denoted $Z_c$ such that:
	
	The microcanonical, canonical and grand canonical sets correspond to systems subject to various constraints which are respectively:
	\begin{itemize}
		\item Microcanonical: fixed energy, fixed number of particles, fixed volume
		\item Canonincal: fixed mean energe = fixed temperature, fixed number of particles, fixed volume
		
		\item Grand canonical: fixed mean energy = temperature fixed,  average number of particles fixed = fixed chemical potential, fixed volume
	\end{itemize}
	This is true in quantum physics as in classical physics.
	
	\paragraph{Boltzmann Distribution}\mbox{}\\\\
	Let us now see a first interesting application of the Maxwell-Boltzmann distribution function.

	Consider the interesting case where :
	\begin{itemize}
		\item The volume and number of particles are fixed (microcanonical system)
		
		\item The fact that the number of particles is fixed, the chemical potential is zero
		
		\item A temperature high enough so that the energy states are in endless quantity
	\end{itemize}
	
	It is quite obvious that if the energy states are in infinite quantity, the degeneration factor $g_i$ will be equal to the unit (because there is no place for degeneration levels other than the level itself).

	We then have the Maxwell-Boltzmann distribution that becomes in theses condition the "\NewTerm{Boltzmann distribution}\index{Boltzmann distribution}":
	
	that we can find more of the in the literature under the form:
	
	If we consider that in a broadly neutral gas, energy is defined only by the kinetic energy (in the case of the ideal gas!). 

	So with the transition to the continuum, we have:
	
	Therefore, we can integrate the Boltzmann function above on all speeds and we should have:
	
	With the microcanonical partition function:
	
	for the normalization condition to be satisfied (and anyways it matches to the developments made earlier above!).

	we recognize here an integral that is familiar to and already proven in the section Statistics during our study of the Normal distribution. Then we have:
	
	Therefore, the probability density function can be written:
	
	Now, since the speed is treated as a random variable, let focus on the calculation of the average value of the  square of the speed which traditionally is denoted as follows (the notation has no relation to the ket-bra notation as in Quantum Physics or in Functional Analysis with the dot!):
	
	In statistics we would write this:
	
	Let us focus on:
	
	But if we put:
	
	Then we have:
	
	Thus with a simple change of variable:
	
	and this is exactly the integral of the Normal centered  law as proved in the section Statistics as being equal to:
	
	Therefore:
	
	and finally:
	
	for a particle to a spatial dimension. We find the fact the same result as the Maxwell distribution of speeds shown, with a more rigorous approach, but we reach the same conclusions.
	
	\paragraph{Fermi-Dirac Distribution}\label{fermi dirac distribution}\mbox{}\\\\
	The principle of indistinguishability can have very important consequences for the statistics. We distinguish two types of indistinguishable particles: bosons (like the photon) and fermions (like the electron).

	Let us make two recalls:
	\begin{enumerate}
		\item The bosons correspond to particles whose representative wave function is always symmetrical, while that of fermions is antisymmetric.

		\item The Pauli exclusion principle requires that two fermions can not be in the same quantum state. But the boosons can!
	\end{enumerate}
	Their respective properties have for important consequence that the minimum energy of a set of $N$ bosons is equal to $N$ times the minimum energy of each boson. While for a set of fermions, the minimum energy is equal to the sum of the $N$ lower energies.

	This both types of particles cause two kinds of statistics: Fermi-Dirac statistics for fermions (that we will prove first) and the Bose-Einstein statistics for the bosons (which follows). 

	So there is only one way to distribute $N$ fermions on the accessible energy states (instead of the $N!$ for distinguishable particles). It can not be more than $n_i$ particles in an energy level $E_i$ there are degenerations levels $g_i$. So:
	
	The number of possible combinations for a level $E_i$, degenerated $g_i$ times and having particle $n_i$ particle is then the combinatorial $C_{n_i}^{g_i}$. The total number of configurations is therefore:
	
	This can be checked with the following figure:
	\begin{figure}[H]
		\centering
		\includegraphics{img/mechanics/fermi_dirac_states.jpg}
		\caption{States in the Fermi-Dirac distribution}
	\end{figure}
	 The statistics is therefore very different from the classical Maxwell-Boltzmann case. By taking the logarithm of the number of micro-states and making use of the Stirling's formula approximation  as above, we get:
	 
	that we can already simplify a first time:
	
	and differentiating this expression to find the maximum, we get:
	
	Term by term:
	\begin{enumerate}
		\item $\mathrm{d}(g_i\ln(g_i))=0$

		\item $\mathrm{d}(n_i\ln(n_i))=\mathrm{d}n_i\ln(n_i)+n_i\mathrm{d}(\ln(n_i))=\mathrm{d}n_i\ln(n_i)+n_i\dfrac{1}{n_i}\mathrm{d}n_i=\mathrm{d}n_i\ln(n_i)+\mathrm{d}n_i$

		\item $\mathrm{d}((g_i-n_i)\ln(g_i-n_i))=\mathrm{d}(g_i\ln(g_i-n_i))-\mathrm{d}(n_i\ln(g_i-n_i))$\\
				$=g_i\left(\dfrac{1}{g_i-n_i}\right)-\mathrm{d}n_i\ln(g_i-n_i)+n_i\left(\dfrac{1}{n_i}\right)=-\mathrm{d}n_i\ln(g_i-n_i)+\dfrac{g_i+n_i}{g_i-n_i}$
	\end{enumerate}
	But we have a conservation and symmetry:
	
	So finally:
	
	To respect the constraints on energy and the number of particles, we still use once the Lagrange multiplier method:
	
	Which brings us to the Fermi-Dirac distribution:
	
	The parameters $\alpha$ and $\beta$ play the same role as in the Maxwell-Boltzmann distribution. Then we have:
	
	Then we have:
	
	Thus, in quantum physics, The Fermi-Dirac statistics designate the number of indistinguishable fermions that do not interact with each other (otherwise the model is too complex) on the energy states of a system in a thermodynamic degenerated equilibrium (therefore the number of fermions occupying the given energy level!).

	For macroscopic systems, the energy levels are so tight (or there are so many of the) that we can consider the energy spectrum as continuous (continuum approximation  ).

	We will reason in this continuum context, which allows us to write by norming to the numbers of particles involved (the only one thing we ask to the function is to tell us how the $N$ particles are distributed):
	
	or:
	
	for the Fermi-Dirac function with the corresponding plot further below.

	This relation is very important for example in the semiconductor theory (\SeeChapter{see section Electrokinetics page \pageref{semiconductors}}) which is at the base of the electronics of the 20th and 21st centuries.

	Numerically, we can simulate the evolution of the shape of the distribution (which is not a distribution in the mathematical sense!) in function of energy and temperature. For this, we assume for simplicity that the Boltzmann constant $k$ is equal to $1$ and the chemical potential is $\mu$ is equal to $2$ (we must however notice that it is strictly speaking a function of temperature...). Thus, we can write the following small program in MATLAB™ 5.0.0.473:
	\begin{lstlisting}[language=MATLAB]
		clear all;kb=1; % Boltzmann constant
		mu=2; % Chemical potential
		T=0.001:0.1:1; % Gradient of temperature for the script
		for j=1:length(T)
		   beta(j)=1/(kb*T(j));
		   epsilon=0.1:0.1:4; % Energy with the step
		   for i=1:length(epsilon)
		      Nf(i,j)=1/(exp(beta(j)*(epsilon(i)-mu))+1); % Nb(epsilon,beta) average of fermions at the level of enery epsilon
		   end
		   hold on
		   plot(epsilon,Nf(:,j));
		   pause(0)
		end
	\end{lstlisting}
	That gives:
	\begin{figure}[H]
		\centering
		\includegraphics{img/mechanics/fermi_dirac_plot_matlab.jpg}
		\caption{Plot of the Bose-Einstein distribution with MALTAB™ 5.0.0.473}
	\end{figure}
	At absolute zero $T=0$ we see that we have a step. At this temperature, the degenerate energy levels are completely occupied starting from the lowest energy level to a certain level representing the fall of step. We then say that the Fermi gas is "\NewTerm{completely degenerated}\index{completely degenerated}".

	The Fermi-Dirac function, at absolute zero, is therefore equal to $1$ if $E$ is less than $\mu$ and $0$ for higher values (the system chooses its minimum energy state where the $N$ particles occupy the lowest $N$ energy states).

	Obviously, in the high excitation energies (or high temperatures) the probability (ratio) of occupation of a state is very low and therefore it is not important to apply the Pauli principle (the probability that two electrons want to occupy the same energy level is very low). For these reasons, this boundary system between the classical and quantum behavior is sometimes named "\NewTerm{classical regime}\index{classical regime}" and the corresponding high temperature energy excitement is named "\NewTerm{Fermi temperature}\index{Fermi temperature}\label{fermi temperature}".

	The chemical potential $\mu$ is itself by definition the last occupied energy level at absolute zero energy level (the famous step!). As discussed in the section of Wave Quantum Physics, it is denoted by $E_F$ (and denoted by $\mu$ in chemistry...) and we name it "\NewTerm{Fermi level}\index{Fermi level}" or "\NewTerm{Fermi energy}\index{Fermi energy}\label{fermi energy level semi conductors}". We then see immediately that whatever the temperature:
	
	So the probability of occupation by the fermions is of  $0.5$ at this level. The definition of the Fermi level can then be given by:
	
	The main application to solides of this statistic is the  modeling of electron transport phenomena (electron gas): theory of metals, semiconductors, population of the energy levels and conduction properties (\SeeChapter{see section Electrokinetics page \pageref{electrokinetics}}).
	
	\paragraph{Bose-Einstein Distribution}\label{bose einstein distribution}\mbox{}\\\\
	Bosons are other quantum particles that can indifferently be placed on all energy levels. The Pauli principle does not apply to them! In this case, the number of objects to be swapped is $n_i+g_i-1$ (the $n_i$ particles and the $g_i-1$ intervals between the levels). Because the particles are indistinguishable and interchangeable and the levels and sublevels can be permuted, we must divide by $n_i!$ but also by $(g_i-1)!$. The number of configurations on a level $E_i$,  degenerated $g_i$ times, containing $n_i$ indistinguishable particles is thus equal to:
	
	This can be verified with the following figure:
	\begin{figure}[H]
		\centering
		\includegraphics{img/mechanics/bose_einstein_states.jpg}
		\caption{States in the Bose-Einstein distribution}
	\end{figure}
	Its logarithm with use of Stirling's formula and after simplification is given by:
	
	The maximum of $\ln(\Omega)$ corresponds to cancel $\mathrm{d}(\ln(\Omega))$, that is:
	
	To respect the constraints on the energy and the number of particles, we use once the Lagrange multiplier method:
	
	and therefore:
	
	which is the Bose-Einstein statistical distribution . This distribution diverge when:
	
	This is the "\NewTerm{Bose-Einstein condensate}\index{Bose-Einstein condensate}". In this state, all bosons can be found in the same state.
	
	The parameters $\beta$ and $\alpha$ play the same role as in the distribution Maxwell-Boltzmann and Fermi-Dirac. We have:
	
	The (average) number of photons is therefore given by:
	
	to compare with the (average) number of fermions at the same level of energy (Fermi-Dirac function):
	
	Thus, in quantum physics, Bose-Einstein statistics refers to the statistical distribution of indistinguishable bosons (all similar) and do not interact with each other (otherwise the model is too complex) on the energy states of a system at thermodynamic equilibrium.

	For macroscopic systems, the energy levels are so tight (or there are so many) that we can consider the energy spectrum as continuous (approximation of the continuum).

	We will reason in this context, allowing us to write by norming to the numbers of particles involved (the only one think we request is that the function tell us how the $N$ particles are distributed):
	
	for the Bose-Einstein function. It is therefore defined only for the energies higher than that of the chemical potential (if not what it is negative!).
	
	Numerically, we can simulate the evolution of the shape of the distribution (which is not a distribution in the mathematical sense!) in function of energy and temperature. For this, we assume for simplicity that the Boltzmann constant $k$ is equal to $1$ and the chemical potential is $\mu$ is equal to $2$ (we must however notice that it is strictly speaking a function of temperature...). Thus, we can write the following small program in MATLAB™ 5.0.0.473:

	\begin{lstlisting}[language=MATLAB]
		clear all;kb=1; % Boltzmann constant
		mu=2; % Chemical potential
		T=0.001:0.1:1; % Gradient of temperature for the script
		for j=1:length(T)
		   beta(j)=1/(kb*T(j));
		   epsilon=0.1:0.1:4; % Energy with the step
		   for i=1:length(epsilon)
		      Nf(i,j)=1/(exp(beta(j)*(epsilon(i)-mu))-1); % Nb(epsilon,beta) average of fermions at the level of enery epsilon
		   end
		   hold on
		   plot(epsilon,Nf(:,j));
		   pause(0)
		end
	\end{lstlisting}
	\begin{figure}[H]
		\centering
		\includegraphics{img/mechanics/bose_einstein_plot_matlab.jpg}
		\caption{Plot of the Bose-Einstein distribution with MATLAB™ 5.0.0.473}
	\end{figure}
	At high temperatures, when quantum effects can be neglected, the Bose-Einstein statistics, such as the Fermi-Dirac governing the fermions, tends to the Maxwell-Boltzmann statistics as show the figure below:
	\begin{figure}[H]
		\centering
		\includegraphics{img/mechanics/be_mb_fd_distributions.jpg}
		\caption{Convergence of Maxwell-Boltzmann, Fermi-Dirac and Bose-Einstein in high temperatures domain}
	\end{figure}
	At low temperatures, however, the Bose-Einstein and Fermi-Dirac statistics  differ from each other. Indeed, let us consider, for example, at zero temperature: in the first we expect that the level of lowest energy contains all bosons, while in the second, the lowest energy levels contain $g_i$ fermions.
	
	Furthermore, always at zero temperature ($-273.15$ [C]), the Bose-Einstein statistics obviously shows that all particles must occupy the same quantum state: the lowest energy. This phenomenon can be observed on a macroscopic scale and is name a "\NewTerm{Bose-Einstein condensate}\index{Bose-Einstein condensate}". It is considered as a fourth state of matter with liquid, solid and plasma.
	
	Bose-Einstein statistics is useful for the understanding of wave electromagnetic phenomena because photons are bosons (blackbody radiation, interaction material/radiation). It is also very much helpful to the study of vibrational phenomena in solids (phonons follow Bose-Einstein statistics). It has also been used to explain phase transitions in helium (phenomenon at very low temperatures).
	\begin{tcolorbox}[title=Remarks,colframe=black,arc=10pt]
	\textbf{R1.} The Bose-Einstein statistics was introduced by Satyendra Nath Bose in 1920 for photons and generalized to atoms by Einstein in 1924.\\
	
	\textbf{R2.} A mathematical result named "\NewTerm{spin-statistics theorem}\index{spin-statistics theorem}" connects the spin of a particle and the type of statistic that it follows. It stipulates that integer spin are bosons, while the half-integer spin are fermions. The proof of this theorem is not yet in this today.
	\end{tcolorbox}
	Finally, here is a simplified summary of the things that can possibly help for a better understanding of what we see so far:
	\begin{table}[H]
		\begin{center}
		 \begin{tabular}{|m{3cm}|m{3.5cm}|m{8cm}|}
		 \hline 
		 \centering\arraybackslash\ \cellcolor{black!30} \textbf{Particles} & \centering\arraybackslash\ \cellcolor{black!30} \textbf{Statistic} & \centering\arraybackslash\ \cellcolor{black!30} \textbf{Specificity} \\  \hline 
		 \pbox{5cm}{Bosons \\ \tiny (photons, gluons, ...)} & \centering\arraybackslash\  \pbox{5cm}{$n_i=\dfrac{g_i}{e^{\dfrac{E_i-\mu}{kT}}-1}$ \\ \tiny Bose-Einstein-Statistic} & \pbox{8cm}{Indistinguishable particles that do not interact. \\ No constraint on the number of particles per state} \\ 
		 \hline 
		 \pbox{5cm}{Fermions \\ \tiny (electrons, protons, neutrinos, ...)} & \centering\arraybackslash\ \pbox{5cm}{$n_i=\dfrac{g_i}{e^{\dfrac{E_i-\mu}{kT}}+1}$ \\ \tiny Fermi-Dirac Statistic} & \pbox{8cm}{Indistinguishable particles that do not interact. \\ No constraint on the number of particles per state} \\ 
		 \hline 
			\pbox{5cm}{Classic} & \centering\arraybackslash\ \pbox{5cm}{$n_i=g_ie^{\dfrac{\mu-E_i}{kT}}$ \\ \tiny Maxwell-Boltzman statistics} & \pbox{8cm}{Distinguishable particles that do not interact. \\ No constraint on the number of particles per state}\\ 
		 \hline
		 \end{tabular}
		\caption[]{Similarities of the different quantum distributions}
		\end{center}
	\end{table}
	\begin{figure}[H]
		\centering
		\includegraphics{img/mechanics/statistical_physics_distributions_states_summary.jpg}
		\caption{Summary of the Maxwell, Bose-Einstein and Fermi-Dirac distributions}
	\end{figure} 
	Or in a more picture way, in the absolute zero case and by representing the Fermi energy (see page \pageref{fermi energy level semi conductors}):
	\begin{figure}[H]
		\centering
		\includegraphics{img/mechanics/bose_fermi.jpg}
	\end{figure}
	
	\subsection{Brownian Motion (random walk)}\label{brownian motion}
	The Brownian motion is a mathematical description of random motion of a small organic or non-organic impurity immersed in a fluid and that is not subject to any other interaction that the shocks with with small molecules in the surrounding fluid. This results in a very irregular movement of the small impourity, which has been described for the first time in 1827 by Robert Brown botanist observing particle movement inside pollen grains:
	\begin{figure}[H]
		\centering
		\includegraphics{img/mechanics/brownian_motion.jpg}
		\caption{Brownian Motion in a two-dimensional Lattice}
	\end{figure}
	The origin of the movement due to the molecules was not at all evident in the early 19th century as:
	\begin{enumerate}
		\item  It was not yet commonly accepted that the matter was simply discontinuous and therefore made up of molecules.

		\item We did not understand even assuming the molecular aspect of the matter that some molecules are capable of moving impurities several million times larger than those of the molecules of the fluid.

		\item Even if we admit the molecular aspect, it was not clear that a very large amount of billions of molecules shocks do not cancel each other and the impurity is would finally not be still (we will see that this is due to that the material is not continuous).
	\end{enumerate}
	The mathematical treatment of the problem using molecular and statistical point of view of the matter was a claim for scientific proponents of atomic aspect of the matter and opened once again  the door to the new domain of Statistical Mechanics already slightly used at that time by James Clerk Maxwell in the context of gas studies.
	
	Before we go in the mathematical physics aspect of the Brownian motion let us see an interesting numercial example with with Maple 4.00b:
	\begin{tcolorbox}[colframe=black,colback=white,sharp corners]
	\textbf{{\Large \ding{45}}Example:}\\\\
	First our aim is to simulate a path of a particle with Brownian motion:\\

	\texttt{>restart;\\
    with(linalg):\\
    with(plots):}\\
    
	First, we create an array to save the variables in, and we define the initial point of the path as the origin of the grid:\\
	
	\texttt{>a:=array(1..2,1..5000);\\
    a[1,1]:=0;\\
    a[2,1]:=0;}\\
    
    To plot the path, we will have to create an Array with points as entries:\\
    
   \texttt{A:=array(1..5000);}\\
    
    So here comes the interesting part:\\

	The idea of Brownian Motion is, that the particle moves randomly to a neighbored position on the grid. To simulate this, we create a random variable $R$, which can be $\{1,2,3,4\}$. In the for-loop following, we set the next point in the path according to the value of $R$, where \texttt{a[i-1]} is the last position, and \texttt{a[i]} the actual.\\
	
	\texttt{>X:=rand(1..4);\\
    for i from 2 to 5000 do\\
      R:=X();\\
      if (R=1) then a[1,i]:=a[1,i-1]+1: a[2,i]:=a[2,i-1]\\
        elif (R=2) then a[2,i]:=a[2,i-1]+1: a[1,i]:=a[1,i-1]\\
        elif (R=3) then a[1,i]:=a[1,i-1]-1: a[2,i]:=a[2,i-1]\\
        elif (R=4) then a[2,i]:=a[2,i-1]-1: a[1,i]:=a[1,i-1]\\
     fi:\\
    od:}\\
    
    Recall: \texttt{A[i]} is the array where the points are saved in!\\
    
    \texttt{for i from 1 to 5000 do\\
                A[i]:=[a[1,i], a[2,i]]:\\
            od:}\\
           
    So we finally can plot the path:\\
    
    \texttt{>pathpl:=plot(A):\\
            pts1:=plot([A[1],A[5000]], style=POINT, color=blue):}\\
           
	\end{tcolorbox}
	
	\begin{tcolorbox}[colframe=black,colback=white,sharp corners]
	 \texttt{>display(pathpl, pts1);}\\
	\begin{figure}[H]
		\centering
		\includegraphics[scale=0.55]{img/mechanics/brownian_motion_maple_plot.jpg}
		\caption[]{Brownian motion plot with Maple 4.00b}
	\end{figure}
	After creating the path, the final points of paths with Brownian motion should be distributed randomly. Let's do the simulation!\\
	
	We restart to save memory:\\
	
	 \texttt{>restart;}\\
	 
	 In order to get the final points of paths with Brownian motion, we have to do a for-loop over our created Brownian paths \texttt{b[j]} and save the final point in a array we shall call \texttt{B[j]} (the array \texttt{B1[j], Bx[j] } and \texttt{By[j]} will be discussed later):\\
	 
	 \texttt{>B:=array(1..5000):\\
			B1:=array(1..5000):\\
			Bx:=array(1..5000):\\
			By:=array(1..5000):}\\
 
 	We don't have to save the path, this time, so we will just create a array with two entries to save the path in, and we will overwrite the last point with each step:\\
 	
 	\texttt{>b:=array(1..2);}\\
 	
 	But we have to reset the actual point to the origin each time we do the \texttt{i}-loop for the path (the  \texttt{j}-loop is for saving the final points):

	\begin{tcolorbox}[title=Remark,colframe=black,arc=10pt]
	For calculations purposes, the paths have only a length of $1'000$ steps in this loop Thermodynamics...
	\end{tcolorbox}
	\texttt{>mov:=rand(1..4);}\\
	\end{tcolorbox}
	
	\begin{tcolorbox}[colframe=black,colback=white,sharp corners]
	 \texttt{for j from 1 to 5000 do\\
    b:=[0,0]:\\
    for i from 2 to 1000 do\\
       X:=mov();\\
       if (X=1) then b[1]:=b[1]+1:\\
          elif (X=2) then b[2]:=b[2]+1:\\
          elif (X=3) then b[1]:=b[1]-1:\\
          elif (X=4) then b[2]:=b[2]-1:\\
       fi:\\
       if (i=999) then B1[j]:=b fi:\\
       od:\\
       B[j]:=b:\\
       Bx[j]:=b[1]:\\
       By[j]:=b[2]:\\
	 od:}\\
	 
	 So now we can plot these final points:\\
	 
	 \texttt{>plot(B, style=POINT);}
	 \begin{figure}[H]
		\centering
		\includegraphics{img/mechanics/brownian_motion_maple_final_position.jpg}
		\caption[]{Brownian Motion final $(x,y)$ positions plot with Maple 4.00b}
	\end{figure}
	So now we should check, whether the prediction, that the points are really randomly scattered, is really true: Therefore, it is useful to plot the number of paths ending at a certain $x$-value and the $y$-values, respectively, to the values themselves - we should see a Gaussian distribution. To do that, we had to create the arrays \texttt{Bx[j]} and \texttt{By[j]}, so we can now build up a number array for both the axis' and then plot it's distributions:
	\end{tcolorbox}
	
	\begin{tcolorbox}[colframe=black,colback=white,sharp corners]
	\texttt{>Nx:=array(1..200):\\
			Ny:=array(1..200):\\
			for j from 1 to 200 do\\
			   Nx[j]:=0:\\
			   Ny[j]:=0:\\
			od:}\\
			
	To count the number of paths ending at a certain point, we have to make a loop over the path and raise the corresponding entry:\\
	
	\texttt{>Nx:=array(1..200):\\
			Ny:=array(1..200):\\
			for j from 1 to 200 do\\
			   Nx[j]:=0:\\
			   Ny[j]:=0:\\
			od:}\\
	
	and:\\
	
	\texttt{>pNx:=array(1..200):\\
			pNy:=array(1..200):\\
			for j from 1 to 200 do\\
			   pNx[j]:=[j, Nx[j]]:\\
			   pNy[j]:=[j, Ny[j]]:\\
			od:}\\
			
	which gives visually:\\
	
	\texttt{>plot(pNx, style=POINT, color=red);\\
    		plot(pNy, style=POINT, color=green);}\\
	\begin{figure}[H]
		\centering
		\includegraphics[scale=0.85]{img/mechanics/brownian_motion_maple_final_position_distributions.jpg}
		\caption[]{Brownian Motion final $x$ and $y$ positions distributions plot with Maple 4.00b}
	\end{figure}
	\end{tcolorbox}
	There is also a very well know random walk named "\NewTerm{binomial tree}\index{binomial tree}" (in finance and radiation) that is given the rule that at each trial, a particle can only go up (U) or down (D):
	\begin{figure}[H]
		\centering
		\includegraphics[scale=0.85]{img/mechanics/random_walk_binomial_01.jpg}
	\end{figure}
	Then a UDUDDU walk will give:
	\begin{figure}[H]
		\centering
		\includegraphics[scale=0.85]{img/mechanics/random_walk_binomial_02.jpg}
	\end{figure}
	Represented on a grid of all possibilities this gives:
	\begin{figure}[H]
		\centering
		\includegraphics[scale=0.85]{img/mechanics/random_walk_binomial_03.jpg}
	\end{figure}
	Or with other different paths (DUDUUU and UDDDUD):
	\begin{figure}[H]
		\centering
		\includegraphics[scale=0.85]{img/mechanics/random_walk_binomial_04.jpg}
	\end{figure}
	If we run multiple paths on the same grid we notice that there are more paths converging near the center rather than in the extremities:
	\begin{figure}[H]
		\centering
		\includegraphics[scale=0.85]{img/mechanics/random_walk_binomial_05.jpg}
	\end{figure}
	The explanation is obvious! There is only on way (combination UUUUUU or DDDDDD) to arrive at the extremities:
	\begin{figure}[H]
		\centering
		\includegraphics[scale=0.85]{img/mechanics/random_walk_binomial_06.jpg}
	\end{figure}
	But there are multiple ways to arrive at the center:
	\begin{figure}[H]
		\centering
		\includegraphics[scale=0.85]{img/mechanics/random_walk_binomial_07.jpg}
	\end{figure}
	There are above 20 different paths to get to the middle in 6 trials. So there is  20 times more chance that the particle arrives in the middle than on one of the points of the extremities.
	
	This concentration towards the center will be accentuated if we increase the size of the tree (the user can reproduce this example using the code given in the R companion book):
	\begin{figure}[H]
		\centering
		\includegraphics[scale=0.85]{img/mechanics/random_walk_binomial_08.jpg}
	\end{figure}
	Of course the green or blue path are not completely unlikely:
	\begin{figure}[H]
		\centering
		\includegraphics[scale=0.85]{img/mechanics/random_walk_binomial_09.jpg}
	\end{figure}
	With $1,000$ steps the most likely trajectories look almost straight:
	\begin{figure}[H]
		\centering
		\includegraphics[scale=0.85]{img/mechanics/random_walk_binomial_10.jpg}
	\end{figure}
	With $1,000,000$ million steps, the question does not arise anymore:
	\begin{figure}[H]
		\centering
		\includegraphics[scale=0.85]{img/mechanics/random_walk_binomial_11.jpg}
	\end{figure}
	With $1,000,000$ million steps the green path below is highly unprobable (but not impossible!):
	\begin{figure}[H]
		\centering
		\includegraphics[scale=0.85]{img/mechanics/random_walk_binomial_12.jpg}
	\end{figure}
	But we can also do the same simulation with a grid centered on a disc with particles starting from the edge of the disc and towards the outside. Therefore, we observe that the particles have a high probability of traveling radially from the center (typically the photons emitted on the surface of the Sun)! If this were not the case it could be statistically that for a given duration, with a given probability, no solar radiation would reach us (or on the contrary ALL the radiations emitted from the Sun may go towards us). But this does not happen precisely because of this type of probabilistic behavior. This guarantees with a near certainty in probability that the Sun radiates homogeneously throughout the space. And we must keep in mind that all the equations of physics involving particle systems are statistical phenomena that converge in law towards a given behavior.
	
	So physics laws of particles systems are not absolutely true but statistically accurate!
	
	As always there are several mathematical models and as it is the tradition in this book we chose the easiest one ... it is not that of Albert Einstein but of Paul Langevin (conducted two years after for a simplification purpose).

	The starting point is the theorem of equipartition of kinetic energy (\SeeChapter{see section Continuum Mechanics page \pageref{equipartition theorem of energy}}):
	
	That is to say for only one particle:
	
	with for recall $v$ that is the average speed! If we simplify the analysis to a single possible axis of translation the relation becomes (we specify now that it is the average speed in order not make a confusion later with another quantity):
	
	By writing this we stipulate that a particle in suspension in a fluid in thermal equilibrium has, in the $x$ direction, for example, an average kinetic energy equal to that of a gas molecule of any kind in a given direction, at the same temperature. So this is again a strong identity between dilute solutions and ideal gases.
	
	A particle like the one we are considering, large compared to the molecules of the liquid and moving at speed $v$ with respect to the latter undergoes as we have proved in the section of Continuum Mechanics (Stokes law) a resistance given by:
	
	where $\xi$ will be named "\NewTerm{friction coefficient}\index{friction coefficient}" (corresponding therefore to the viscosity) and where $R$ is the radius of the sphere.

	Now we write according to the dynamics equation of Newton (we no longer specifies that the speed is along $x$ in the notation below):
	
	The supplementary rand force $F_{\text{rand}}$ introduced initially by Langevin, is random (stochastic). We know a priori little of it, except that it seems a priori either positive or negative, and that his magnitude is such that it maintains the agitation of the particle, which without it would end moving by stopping under the effect of viscous resistance.

	The above relation multiplied by $x$, can be also written:
	
	
	This equation is named "\NewTerm{Langevin equation}\index{Langevin equation}" or "\NewTerm{stochastic Langevin equation}\index{stochastic Langevin equation}".

	But, let us recall that (\SeeChapter{see section Differential and Integral Calculus page \pageref{usual derivatives}}):
	
	hence:
	
	By establishing the latte relation, we have also just proved that:
	
	We can then write:
	
	and also:
	
	If we take the average (we change the notation of the average using the over bar by angle brackets that seem more adapted to us:
	
	We put now that:
	
	in other words we can see that as the average work of the random force.

	Therefore it remains:
	
	Let us put:
	
	Then we have:
	
	And as:
	
	Therefore:
	
	Or written differently:
	
	This a differential equation with constant coefficient or order $1$. We have proved in the section of Differential and Integral Calculus that the homogeneous solution was:
	
	where the constant $C$ is not important for now as we will see that it will vanish anyway in the developments that will follow. The special solution can be determined by the constraint at time zer0:
	
	That is after derivation and have put $t=0$ gives:
	
	A simple trivial solution is then:
	
	Then we have:
	
	But after a short time the term in the exponential becomes almost zero. So we have in the steady state:
	
	Therefore:
	
	where $D$ is the "\NewTerm{diffusion coefficient}\index{diffusion coefficient}\label{diffusion coefficient}" (as in Thermodynamics). We notice therefore that the average position of the traveled distance $x$ is by extensions proportional to the square root of time:
	
	Albert Einstein had applied a numerical example with particles in suspension of a thousandth of a millimeter in water at $290$ [K] and thus found a average moving path distance of six thousandths of a millimeter in a minute.
	
	The relation (where it should not be any confusion between the radius $R$ of the particle in suspensions in the denominator with the $R$ of ideal gas constant in the numerator):
	
	is named "\NewTerm{Stokes-Einstein relation}\index{Stokes-Einstein relation}" or sometimes "\NewTerm{Sutherland-Einstein relation}\index{Sutherland-Einstein relation}" (as William Sutherland discovered it in Australia almost at the same time when Einstein wrote this same relation in his thesis) and that we can also found frequently in the literature in the following form (where $k$ if for recall the Boltzmann constant):
	
	Albert Einstein had thus found a relation that offered the possibility of calculating the radius $R$ of atoms with a thermometer, a microscope and a stopwatch for only instruments. The physicist Jean Perrin won the Nobel Prize for experimental verification of this result.

	The reader will notice that if we do the same developments as before but by dividing the left and right differential equation:
	
	by $m$ we would have get for result (rightly considered as more general):
	
	The Langevin method gives the same result as the first model proposed by Albert Einstein two years earlier and the second model two years later.

	One may wonder in what the Sutherland-Einstein relation proves the existence of molecules??? In other words, what would be the limit of the diffusion coefficient $D$ if Nature was continuous, that is to say, if Avogadro's number was infinite!

	We obviously see that then $D$ would vanish (asymptotically equal to $0$), and that the Brownian diffusion movement would then also simply vanish in this limit. In other words, the Brownian motion would cease immediately if Nature was continuous! This is a remarkable proven result for the discrete aspect of Nature and some given scale level!
	
	\begin{flushright}
	\begin{tabular}{l c}
	\circled{90} & \pbox{20cm}{\score{3}{5} \\ {\tiny 26 votes,  66.15\%}} 
	\end{tabular} 
	\end{flushright}
	

	%to make section start on odd page
	\newpage
	\thispagestyle{empty}
	\mbox{}
	\section{Thermodynamics}
	\lettrine[lines=4]{\color{BrickRed}T}hermodynamics is the part of physics (and chemistry) that deals with the relations that permits to formally determine the exchanges (variations) of energy in the form of mechanical work and heat or radiation through the study of the transformations of the $4$ states of matter (but mainly ideal gas at school) under the basis of simplifying assumptions between a system (isolated, open or closed) and its external environment.
	
	 Thermodynamics defines macroscopic variables, such as internal energy, entropy, and pressure, that partly describe a body of matter or radiation. It states that the behavior of those variables is subject to general constraints, that are common to all materials, beyond the peculiar properties of particular materials. These general constraints are expressed in the four laws of thermodynamics. 
	
	The main objectives of thermodynamics are:
	\begin{enumerate}
		\item With a minimum of variables to determine the state and energy exchanges of a system (closed or open) under predefined constraints and often considered ideal ... (there are roughly a $5\%$ difference between the theoretical values and those measured) and between states of equilibrium and thus without the time variable.
		
		\item To find the "\NewTerm{state variables}\index{state variables}" (defining the state of the system studied at thermodynamic equilibrium) such that these different information can be obtained in the ideal case by knowing only the final and initial state of a system.
		
		\item To find workarounds to always to bring the equations to a form highlighting variables (variations) easily measurable in practice.
	\end{enumerate}
	We will see later that almost any system can be energetically described by:
	\begin{enumerate}
		\item Its volume, mass, pressure, temperature, number of items, ...
		
		\item Its potential energy, kinetic energy, chemical potential, ...
		
		\item Its physical properties such as the ability to absorb heat, irradiating, ...
	\end{enumerate}
	Caution!! Although the predictive power of thermodynamics equations is exciting to put into practice, this theory is ultimately... unfortunately (and inevitably)... a jungle of equations because of the number of possible combinations of some parameters that we can choose at leisure as fixed or variable in practice. So we encourage the reader to practice mentally in every day life thermodynamics because as we say... there is nothing better than learning by doing!
	
	\pagebreak
	\begin{tcolorbox}[title=Remarks,colframe=black,arc=10pt]
	\textbf{R1.} In our point of view it is very difficult to present thermodynamics in a purely linear educational way unlike other areas of physics. The plan of this section is therefore currently a real mix of concepts that are sometimes defined previously and others that are only defined much later in the text. The reader will therefore understand that this section in its present form requires a major restructuring and addition of a lot of illustrations to be mature and this will be done in a more or less future time ...\\
	
	\textbf{R2.} This section contains only few classical real and school applications examples because the assumptions needed to apply the theoretical thermodynamic equations are so restrictive and even sometimes absurd that the example loses all its interest (and yet the majority of examples / exercises in schools engineers and universities are of this type in our point of view...).\\
	
	\textbf{R3.} Many authors make "proofs" (in their own words) only with figures or schema in the field Thermodynamics. We wanted to absolutely avoid this knowing that a schema or figure is in no way a proof but only an illustration of an idea and that it becomes useless afterwards in some application fields such as Quantum Thermodynamics...
	\end{tcolorbox}

	\subsection{Thermodynamic Variables}
	In thermodynamics, there are several major concepts very useful depending on the type of studied system maintained or left in free evolution and the measurement methods available.
	
	We wish here to give the most important definitions, and if possible we will prove their origin knowing that it is very difficult to have a purely linear presentation of  the subject of Thermodynamics (as already mentioned!).
	
	\textbf{Definitions (\#\mydef):}
	\begin{enumerate}
		\item[D1.] A "\NewTerm{system}\index{system}" is a body or a group of bodies undergoing or not evolutions and that constitutes a well-defined and delimited set in space and surrounded by the external environment (or isolated from it). The set system/external environment is named a "\NewTerm{Universe}\index{Universe}".
		
		\item[D2.] The "\NewTerm{work}\index{work}", denoted $W$, is the energy associated with the value or to the change of dynamics of a system through the application of various mechanical or electrical forces (hence the choice of the term "work" ...).
		
		\item[D3.] The "\NewTerm{heat}\index{heat}" or "\NewTerm{heat energy}\label{heat energy}", denoted $Q$, is the energy associated with the value or change of the dynamics of a system due to the average agitation of the molecules, atoms or particles (average kinetic energy).
		
		\item[D4.] The "\NewTerm{total energy}\index{heat}" of a system, denoted $E_\text{tot}$, is the sum of all the energies that specify the system with respect to its center of mass (moment of inertia, mass, internal kinetic energy, etc.) and also  (!) relative to an external reference (kinetic energy, potential energy, incoming radiation, etc.).
		
		\item[D5.] The "\NewTerm{external energy}\index{external energy}" which is related to the position of the system like the kinetic energy or potential energy (gravitational / electrostatic).
		
		\item[D6.] The "\NewTerm{internal energy}\index{internal energy}" of a system, denoted $U$, is the sum of all types of internal energies that only distinguish it from its center of mass such as the work $W$ exchanged with the outside, the heat $Q$ exchanged (internal average kinetic energy), the mass energy (relativistic, nuclear), the moment of inertia, etc.
		
		Many high-school books show that at this point in our study of physics, we have typically as total energy of a body and therefore considered as internal energy:
		
		But when talking about ideal gases, their are no chemical interaction, no complex phase structure (because of no interaction), (usually) no significant potential energy. And the kinetic energy of each of the atoms is just the same as thermal energy or temperature. So in this special case, the internal energy equals "thermal energy\index{thermal energy}" - but only for an ideal gas!
		
		\item[D7.] The "\NewTerm{exchanged energy}\index{exchanged energy}" which are therefore the energies that are exchanged with the outside. The most common are the mechanical/electrical work $W$ and heat $Q$ defined above.
		
		\item[D8.] The "\NewTerm{entropy}\index{entropy}", denoted $S$, quantifies the quality and direction of change of the energy of a system. We will prove later that the entropy of an isolated system can only increase.
		
		\item[D9.] The "\NewTerm{enthalpy}\index{enthalpy}", denoted $H$, is the quanity of heat received by a system that operates at constant pressure (isobaric). In chemistry this concept is very useful as a chemical transformation (or physical) part of the injected energy will just served to push the ambient atmosphere during the volume change in the transformation. Thus, the enthalpy adds to the internal energy $U$ a corrective term taking into account the energy  stored/lost by the surrounding pressure that compresses the system (if it is compressible...).
		
		\item[D10.] The "\NewTerm{free energy}\index{free energy}" or "\NewTerm{Helmholtz energy}\index{Helmholtz energy}", denoted $F$, characterizes the internal energy fraction used in the form of work. It is simply the difference between the internal energy and the dissipated heat energy because of entropy at a given temperature.
		
		\item[D11.] The "\NewTerm{free energy}\index{free energy}" or "\NewTerm{Gibbs free energy}\index{Gibbs free energy}", denoted $G$, characterizes the enthalpy fraction available as work. It is simply the difference between the enthalpy and the dissipated heat energy because of the entropy at a given temperature.
	\end{enumerate}
	
	The reader must keep in mind that if not explicitly indicated otherwise, thermodynamic variables are averages of all possible values of the system (otherwise we should deal all the time with statistical distribution and this would be a pain in the ass to do so...).
	
	For summary, and in order of appearance and importance we have the following types of energy:
	\begin{table}[H]
		\begin{center}
			\definecolor{gris}{gray}{0.85}
				\begin{tabular}{|c|c|}
					\hline
					\multicolumn{1}{c}{\cellcolor{black!30}\textbf{Symbol}} & 
	  \multicolumn{1}{c}{\cellcolor{black!30}\textbf{Description}} \\ \hline
					$W$ & Work \\ \hline
					$Q$ & Heat \\ \hline
					$P$ & Pressure \\ \hline
					$T$ & Absolute temperature \\ \hline
					$V$ & Volume \\ \hline
					$E_\text{tot} $ & Total Energy \\ \hline
					$U$ & Internal Energy \\ \hline
					$S$ & Entropy \\ \hline
					$H$ & Enthalpy \\ \hline
					$F$ & Free energy \\ \hline
					$G$ & Free enthalpy \\ \hline
			\end{tabular}
		\end{center}
		\caption{Resume of main Combinatorial Analysis cases}
	\end{table}
	Let us indicate that we will not return back on the concept of temperature $T$ that is, as we have already seen in the section of Continuum Mechanics (Virial theorem) and Statistical Mechanics (blackbody radiation), a parameter that can advantageously link the average movement of different bodies with their average kinetic energy (meaning their "excitement" or "disorder") or respectively the radiation of certain bodies with their emission energy. But let us still recall the origin of the zero Kelvin, because this is a redundant question on the web:
	
	We have proved in the section of Continuum Mechanics that at constant pressure $P$ ("isobaric" system), the volume of a fixed amount of ideal gas is proportional to the absolute temperature. This is the "\NewTerm{Gay-Lussac's law}\index{Gay-Lussac's law}" (in the case of perfect gases ...!):
	
	and we have then mentioned that the experimental measurements then give a straight right, which extrapolated (somewhat abruptly ...) in negative values of temperature in Celsius gives zero volume for an ideal gas (neglecting quantum aspects to this border ...) at a temperature systematically of $-273.15\; [^\circ \text{C}]$:
	\begin{figure}[H]
		\centering
		\includegraphics{img/mechanics/gay_lussac.jpg}
		\caption{Representation of the Gay-Lussac Law with pressure}
	\end{figure}
	hence the comfortable choice to put the $0$ [K] at this value and redefine a new temperature scale!
	
	Moreover, a difficulty is to talk about temperature change, as for example when we study the sensitivity of water to move from solid to liquid phase. We indeed observe experimentally that under ideal laboratory conditions when we move from $-0.1\; [^\circ \text{C}]$ to $+0.1\; [^\circ \text{C}]$ it is enough to observe the passage of qualitative change of water from liquid to solid state. Speak then of  sensibility of $\%$ of temperature is difficult with the traditional scale in this case. But if we talk in absolute scale then this corresponds to a temperature changes from $273.05$ to $273.25$ [K], that is to say a phase transition sensitivity to water of $0.1\%$ in temperature around its melting point.
	
	\subsection{Thermodynamics Systems}
	Most of time a thermodynamic system is the set of bodies located inside an imaginary closed surface defined by a "\NewTerm{border}\index{border}" and often considered extensible without loss of energy. 

	The border can also under indication of its characteristics be material!

	Let us notice that the border can be limited to an elementary surface $\mathrm{d}A$ (we take $A$ to avoir confusion with the entropy $S$) associated with its normal vector wrapping a fluid particle (see the section Marine and Weather Engineering for example!). We name it in this case "\NewTerm{particulate border}\index{particulate border}".

	It is often useful in thermodynamics, to the energy accounting that is transferred between the thermodynamic system and the external environment, that is to say, consider everything that crosses the border!

	The main type of transfers (but not the only one!) that may be made are:
	\begin{enumerate}
		\item The "\NewTerm{work transfer $W$}\index{work transfer}":  (Mechanical) Work that operate most of time at macroscopic level done by a force over a distance. When no work  transfer (energy) is operated on a macroscopic scale, the system is named "\NewTerm{system without work}\index{system without work}".
		
		\item The "\NewTerm{heat-transfer $Q$}\index{heat-transfer}": (agitation) Energy from the variation in the number of microstates at the microscopic scale (\SeeChapter{see section Mechanical Statistics page \pageref{microstate}}). When no heat-transfer (energy) is operated at the microscopic level, the system is named "\NewTerm{adiabatic system}\index{adiabatic system}\label{adiabatic system}", otherwise it is said "\NewTerm{diathermal}\index{diathermal}".
		
		\item The "\NewTerm{mass transfer $M$}\index{mass transfer}": mass injected into the system. When no mass transfer is made, the system is named a "\NewTerm{closed system}\index{closed system}".
	\end{enumerate}
	
	\textbf{Definitions (\#\mydef):}
	\begin{enumerate}
		\item[D1.] An "\NewTerm{isolated system}\index{isolated system}" can not exchange neither work nor heat nor mass with its surroundings.

		\item[D2.] An "\NewTerm{open system}\index{open system}" can exchange work, heat and mass with its  surroundings.

		\item[D3.] A "\NewTerm{closed system}\index{closed system}" can share work and heat but no matter with its  surroundings.
	\end{enumerate}
	\begin{tcolorbox}[title=Remarks,colframe=black,arc=10pt]
	\textbf{R1.} Some systems count a bilan of external actions applied to them that is equal to zero. They are then say to be "\NewTerm{pseudo-isolated}\index{pseudo-isolated}".\\
	
	\textbf{R2.} We speak of "\NewTerm{homogeneous system}\index{homogeneous system}" if the nature of its constituents is equal in every point, it is a "\NewTerm{uniform system}" or "\NewTerm{isotropic system}\index{isotropic system}" if its characteristics are equal in every point.
	\end{tcolorbox}
	
	\subsection{Thermodynamic Transformations}
	\textbf{Definition (\#\mydef):} A "\NewTerm{thermodynamic transformation}\index{thermodynamic transformation}" is the operation in which the system state changes passing from an initial state to a final state.

	We distinguish at least two main types:
	\begin{enumerate}
		\item The "\NewTerm{quasi-static thermodynamic transformation}\index{quasi-static thermodynamic transformation}" that brings a system from an initial state to a final state through a succession of states that are exclusively equilibrium states (assumed as very slow transformation).

		\item The transformations where all state variables change simultaneously are named "\NewTerm{polytropic transformations}\index{polytropic transformations}".
	\end{enumerate}
	It is possible that some variables remain constant during a thermodynamic transformation. In this case, we use very specific denominations:
	\begin{table}[H]
		\begin{center}
			\definecolor{gris}{gray}{0.85}
				\begin{tabular}{|c|c|c|}
					\hline
					\multicolumn{1}{c}{\cellcolor{black!30}\textbf{Symbol}} & 
	  \multicolumn{1}{c}{\cellcolor{black!30}\textbf{Description}} & 
	  \multicolumn{1}{c}{\cellcolor{black!30}\textbf{Denomination}}  \\ \hline
					$Q$ & Heat & Adiabatic \\ \hline
					$P$ & Pressure & Isobar \\ \hline
					$T$ & Absolute temperature & Isotherm \\ \hline
					$V$ & Volume & Isochore\\ \hline
					$U$ & Internal Energy & Isoenergetic \\ \hline
					$S$ & Entropy & Isentropic \\ \hline
					$H$ & Enthalpy & Isenthalpic \\ \hline
					$G$ & Free enthalpy & Extensive \\ \hline
			\end{tabular}
		\end{center}
		\caption{Denomination of common constant systems}
	\end{table}
	\begin{tcolorbox}[title=Remarks,colframe=black,arc=10pt]
	\textbf{R1.} The studies of isobares systems are of great practical interest since all systems in contact with the atmosphere are often, at equilibrium, naturally or forcibly at constant pressure (that is to say at the same pressure as the surrounding atmosphere!) in laboratories.\\
	
	\textbf{R2.} It is quite difficult to build useful isothermal transformations. Keep the temperature constant requires excellent thermal contact or isolation and a long reaction time to keep the temperature uniform. This type of transformation will be in the real world, rather slow. Caution!!! An adiabatic transformation may be reversible or irreversible, while an isothermal transformation is reversible for ideal gas.\\
	
	\textbf{R3.} Many real transformations are considered as adiabatic. It is enough for this that the container is well isolated or even the transformation is fast enough such that the heat exchanges are negligible.
	\end{tcolorbox}
	\textbf{Definition (\#\mydef):} A "\NewTerm{thermodynamic cycle}\index{thermodynamic cycle}" consists of a linked sequence of thermodynamic processes that involve transfer of heat and work into and out of the system, while varying pressure, temperature, and other state variables within the system, and that eventually returns the system to its initial state.
	
	We also distinguish two main thermodynamic transformations cycles that are:
	\begin{itemize}
		\item The "\NewTerm{closed thermodynamic cycle}\index{closed thermodynamic cycle}": the system described a series of transformations such as the final state and the initial state of the process are the same and the quantities and properties of the elements involved in the cycle are always the same.

		\item The "\NewTerm{open thermodynamic cycle}\index{open thermodynamic cycle}": the system describes a series of transformations such that the final state and the initial state of the processe are also the same and the quantity and properties of the elements involved in the cycle are not always the same.
	\end{itemize}
	Below a figure with two illustrative example (we will come back of technical aspects further below):
	\begin{figure}[H]
		\centering
		\includegraphics[scale=0.9]{img/mechanics/cycles_examples.jpg}
		\caption[Two typical Thermodynamics cycles]{Two typical Thermodynamics cycles (source: ?)}
	\end{figure}
	
	\subsection{State Variables}
	\textbf{Definition (\#\mydef):} The "\NewTerm{thermodynamic state}\index{thermodynamic state}" of a system is the set of properties that characterize it, regardless of the shape of its border. The variables that describe the state of the system by knowing only the final and initial state of the latter are named most of time "\NewTerm{state functions}\index{state functions}" or more frequently "\NewTerm{state variables}" and sometimes even "\NewTerm{state quantities}\index{state quantities}"...  Intuitively, the state of a system describes enough about the system to determine its future behaviour in the absence of any external forces affecting the system.
	
	Until now in fact we already worked with state variables. Indeed, in mechanical systems, the position coordinates and velocities of mechanical parts are typical state variables. Knowing these, it is possible to determine the future state of the objects in the system.

	In thermodynamics state variables include temperature, pressure, volume, internal energy, enthalpy, and entropy. In contrast heat and work are not state functions, but "\NewTerm{process functions}\index{process functions}".

	As we will see in the chapter of Electrodynamics, in electronic circuits, the voltages of the nodes and the currents through components in the circuit are usually the state variables. In Social Sciences the population sizes (or concentrations) of plants, animals and resources (nutrients, organic material) are typical state variables.
	\begin{tcolorbox}[title=Remark,colframe=black,arc=10pt]
	Some state variables play a special role in the definition of equilibrium states of a system. These are accessible quantities, at the macroscopic scale, directly or indirectly through measuring instruments. These special state functions (such as pressure, temperature, volume, etc.) are named sometimes "\NewTerm{state variables of thermodynamic system at equilibrium}\index{state variables of thermodynamic system at equilibrium}".
	\end{tcolorbox}
	The previous definition implicitly assumes the existence of a state and state variables, that is to say, the system characteristic parameters are defined (or theoretically accessible in the sense of measurement) at any time and at any point of the system. This is far from being easy if we consider the rapid changes of events such as explosions.
	
	This difficulty can be studied by hiding us behind the  "\NewTerm{hypothesis of the local sate}\index{hypothesis of the local sate}": We assume that at any time, the characteristic quantities have locally the same expressions as in a stationary configuration, which implies that the time necessary to state changes are negligible relatively to the characteristics periods of evolution.
	
	The choice of the state variables depends on the nature of the problem studied. Nevertheless, we can separate all of these state variables at least into two groups:
	\begin{enumerate}
		\item[D1.] State variables are say to be "\NewTerm{extensive quantities}\index{extensive quantities}", if their are proportion to the quantity of matter and therefore the to the number of atoms/molecules of the system that are used to define them (therefore there are additive). This is typically the case of the mass $M$, the volume $V$, the Entropy  $S$, the energy $E$, etc.
		
		This means the system could be divided into any number of subsystems, and the extensive property measured for each subsystem; the value of the property for the system would be the sum of the property for each subsystem.

		\item[D2.] State variables are say to be "\NewTerm{intensive quantities}\index{intensive quantities}\label{intensive quantities}", if their are independent to the quantity of matter and therefore the to the number of atoms/molecules of the system that are used to define them (therefore there are not additive). This is typically the case of the pressure $P$, the temperature $T$, the density  $\rho$, the color, etc.
		
		An intensive property is a therefore a bulk property, meaning that it is a physical property of a system that does not depend on the system size or the amount of material in the system
	\end{enumerate} 
	\begin{figure}[H]
		\centering
		\includegraphics[scale=0.6]{img/mechanics/thermo_bottles.jpg}
		\caption{Intensive vs Extensive variables}
	\end{figure}
	In general, extensive variable depends on the point of view considered in the studied  system while extensive variable are defined on the whole of the system.
	
	Let us now show with a particular example that the ratio of two extensive variables is an intensive variable. For this, let us recall the ideal gas law (\SeeChapter{see section Continuum Mechanics page \pageref{ideal gas law}}) which is an equation of state (we will come back on state equations further below):
	
	which therefore includes three thermodynamic variables $P$, $V$ and $T$ (which leaves to us two choices of independent control variables and one dependent variable).
	
	We can now calculate the concentration $C$ (not to be confused with the notation of the heat capacity we will see later!), Which is therefore an intensive magnitude, by the ratio of two extensive quantities $n$, $V$ such that:
	
	The state variables have also a special property: their variations are not dependent on the nature of the process that affects the system, but only the final and initial state of the system at equilibrium (which is very useful in practice ...!). This is the related to the concept of path integral that we have already discussed in detail in the section of Classical Mechanics and Differential and Integral Calculus.

	Let us list the most current extensive and intensive quantities in thermodynamics:
	\begin{table}[H]
		\begin{center}
			\definecolor{gris}{gray}{0.85}
				\begin{tabular}{|c|c|c|}
					\hline
					\multicolumn{1}{c}{\cellcolor{black!30}\textbf{Symbol}} & 
	  \multicolumn{1}{c}{\cellcolor{black!30}\textbf{Legend}} & 
	  \multicolumn{1}{c}{\cellcolor{black!30}\textbf{Quantity}}  \\ \hline
					$Q$ & Heat & Extensive \\ \hline
					$P$ & Pressure & Intensive \\ \hline
					$T$ & Absolute temperature & Intensive \\ \hline
					$V$ & Volume & Extensive\\ \hline
					$E$ & Total Energy & Extensive\\ \hline
					$U$ & Internal Energy & Extensive \\ \hline
					$S$ & Entropy & Extensive \\ \hline
					$H$ & Enthalpy & Extensive \\ \hline
					$F$ & Free Energy & Extensive \\ \hline
					$G$ & Free enthalpy & Extensive \\ \hline
			\end{tabular}
		\end{center}
		\caption{Common extensives and intensives state variables}
	\end{table}
	In contrast, the work $W$ and heat $Q$ are not a state variable because they depend on the nature of the transformation. However there are special cases where the heat and work are no longer dependent on the path followed when changes are made either at constant pressure or at constant volume (we will see this later).
	
	\subsubsection{Phases}
	\textbf{Definitions (\#\mydef):} 
	\begin{enumerate}
		\item[D1.] A system in which various intensive variables vary continuously constitutes a "\NewTerm{phase}\index{phase}\label{phases of matter}". In other words,  a phase is a region of space (a thermodynamic system), throughout which all physical properties of a material are essentially uniform
	
	We can therefore consider that any intensive magnitude depends only on the coordinates of the studied point: the system consists of a single phase if the intensive variable is continuous throughout the system. This is the case of gases, liquids and of some homogeneous solids.

		\item[D2.] A "\NewTerm{phase transition}\index{phase transition}" is the transformation of a thermodynamic system from one phase or state of matter to another one by heat transfer. The term is most commonly used to describe transitions between solid, liquid and gaseous states of matter, and, in rare cases, plasma. A phase of a thermodynamic system and the states of matter have uniform physical properties. During a phase transition of a given medium certain properties of the medium change, often discontinuously, as a result of the change of some external condition, such as temperature, pressure, or others. For example, a liquid may become gas upon heating to the boiling point, resulting in an abrupt change in volume. The measurement of the external conditions at which the transformation occurs is termed the phase transition. Phase transitions are common in nature and used today in many technologies.
		\begin{figure}[H]
			\centering
			\includegraphics{img/mechanics/phase_transitions.jpg}
			\caption[Commons phase transitions]{Commons phase transitions (source: Wikipedia)}
		\end{figure}

		\item[D3.] If the intensive quantity has one discontinuity (or more), the system is said to be "\NewTerm{multi-phase}\index{multi-phase}". However, if the intensive quantities have the same value at any point in the system, the phase is said to be an "\NewTerm{uniform  phase}\index{uniform  phase}": the system will have same temperature, pressure and composition in each of its points.
	\end{enumerate}
	For a homogeneous system, it may be convenient to bring the extensive variables to the unit of mass. We speak then of "\NewTerm{mass quantities}\index{mass quantities}" (or "\NewTerm{specific quantities}\index{specific quantities}"), usually denoted in lower case.

	We use if possible to avoid confusion the following scoring rules:
	\begin{enumerate}
		\item Any non-mass quantity is represented by an uppercase Latin letter $ V, H, U, S, G, F,\ldots$ and then named respectively "volume", "enthalpy", "internal energy", "entropy", "free enthalpy", "free energy", and so on...
		
		\item Any mass quantity is represented by a lowercase Latin letter $v, h, u, s, g, f,\ldots$ and then named respectively "specific volume", "specific enthalpy", "specific internal energy", "specific entropy", "specific free enthalpy", "specific free energy", and so on...
	\end{enumerate}
	Otherwise, if we do not follow ths notation, we will have to specify what type of variable we are dealing with.
	
	Notice by the way that all specific variables are intensive variables!
	
	\subsection{Equation of State}
	In physics, particularly in thermodynamics, the "\NewTerm{equation of state}\index{equation of state}" of a system at thermodynamic equilibrium is a relation between different physical parameters, so named "state variables" already defined previously that determine its status. This may be for example of a relation between temperature, pressure and volume. From the characteristic equation of state of a physical system, it is possible to determine all the thermodynamic quantities describing this system and hence predict its properties.

	The equations of state are generally restricted to a type of behavior or to given physical phenomena. One body can have several equations of state, for example the magnetic state or its thermodynamic state.
	
	For a body to be characterized by an equation of state at a given time, it is necessary that the state of the body depends only of the values of the known parameters at this time. The body having a hysteresis phenomenon can not therefore be characterized by an equation of state.
	
	\subsubsection{Ideal Gaz Law}
	We have proved in the section of Continuum Mechanics that the state equation of an ideal gas was (relation already introduced previously a little bit fortuitously earlier above):
		
	That is to say also:
	
	and the equation of state of real gases (also provedin the section of Continuum Mechanics):
	
	To determine the particular and idelaized equations of state of solids and liquids, it will first introduce several factors, named "\NewTerm{thermo-elastic coefficients}\index{thermo-elastic}\label{thermo-elastic}", used to characterize some global properties of these continuous media.

	These coefficients should be defined carefully, depending on how we measure them, that is to say, depending on what is actually measured.
	
	Let us take, for example, the concept of (volumic) compressibility of a sample. It is reasonable to say that the underlying idea is the measurement of volume change under a change in pressure. If the pressure increases, the volume will decrease, so it seems reasonable to set a compressibility factor by:
	
	where the sign "$-$" is used to ensure a positive $\chi$ in the case of compression.
	
	But this definition does not mean anything. We have not specified the conditions under which the measures are taken. Clearly, we increase (or diminish) the pressure measured using a manometer for example, and we measure with a ruler (!) the volume change that results. It sounds simple, but it is incomplete. If we increase the pressure fast enough, the volume will decrease but the temperature will increase. Do we measure the volume change immediately or we wait that the temperature is returned that it was by placing the sample in thermal contact with a thermostat that we used before measuring the volume change. In the latter case, the measurement has been made at constant temperature.

	In other words, do we measure the volume change of a receptacle in perfect thermal contact with the external environment (we leave it the time to go to equilibrium) or thermally isolated from this environment. It is very different!

	Furthermore, $V$ being an extensive variable, there will be an advantage in defining the compressibility as an intensive quantity, that is to say independent of the amount of gas in question (here isothermal), which makes easier the construction of tables of values. We then have intuitively the following definition of the "\NewTerm{(relative) isothermal coefficient of compressibility}\index{isothermal coefficient of compressibility}":
	
	on which will be come back later more in details.

	In the same intuitive we, we can define the "\NewTerm{(relative) isobaric coefficient of compressibility}\index{isobaric coefficient of compressibility}" (more rarely named "\NewTerm{(relative) volume isobaric coefficient of compressibility}\index{volume isobaric coefficient of compressibility}"):
	
	on which will be come back later more in details.
	
	For example, for a perfect gas we have:
	
	We will later another approach to fall back on this equality.
	
	\subsubsection{State equation of a Liquid}
	Now the (macroscopic) state variables of a liquid are quite trivially the volume $V$, the pressure $P$ and temperature $T$. Its equation of state can then be written in the form:
	
	Under differential form this give:
	
	By introducing the thermoelastic coefficients as we did above, we get:
	
	Experience shows that for liquids, the thermoelastic coefficients vary very little with temperature and pressure (yes ... the experience is still useful sometimes ...). We can therefore assume them as constant, provided that the variations of the temperature and pressure are moderate. As the changes in volume of a liquid are very low when the transformations are slow, we can do the same approximation as that proved in the section of Continuum Mechanics with the true longitudinal deformation. Which means:
	
	where $V_0$ is the constant reference volume. Taking into account these remarks, we get trivially by integration the general state equation of a liquid:
	
	Let us notice that the approximation of an incompressible liquid, commonly used in fluid dynamics, assume that:
	
	The state equation of the liquid the becomes obviously:
	
	Let us also notice that in small classes, solids are considered sometimes as incompressible liquids. We then have the famous relation known by many students under the name of "\NewTerm{volumic dilatation formula}\index{volumic dilatation formula}":
	
	These two formulations are exactly the same that the technical formulation of the Gay-Lussac law for gas (\SeeChapter{see section Continuum Mechanics page \pageref{gay lussac law}}).

	We can do again the same reasoning with an element of length but by defining another thermoelastic coefficient: a linear one this time! Therefore, it comes another famous relation known by many students under the name of "\NewTerm{length dilatation formula}\index{length dilatation formula}":
	
	We can then determine the relation between $\alpha_{L,P}$ and $\alpha_{V,P}$ (which are both much small than the unit as far as we know). Indeed, for a cube of sides $L$, we have:
	
	and therefore it comes:
	
	We have for example for water at room temperature:
	
	\begin{tcolorbox}[colframe=black,colback=white,sharp corners]
	\textbf{{\Large \ding{45}}Example:}\\\\
	A tank of volume $V$, with walls supposed non-deformable, is completely filled with water initially in thermodynamic equilibrium with the initial pressure $P_0$ of $1$ [bar] and at the initial temperature $T_0$ of $298$ [K]. After having received a certain amount of heat, the liquid reaches a new state of thermodynamic equilibrium, at the temperature of $398$ [K]. We then have since $V=V_0$, the liquid state equation that reduces to:
	
	Or after rearrangement:
	
	\end{tcolorbox}
	
	\subsubsection{State equation of Solids}
	Let us now consider the special case of a beam (or a string) composed of an elastic material. The typical state variables of the system are then its length $L$, its temperature $T$ and the tensile force $F$ (positive in the case of traction and negative in the case of compression by convention). The equation of state is thus put in the following differential form:
	
	By analogy with isobaric volumic expansion coefficient $\alpha_{V,P}$, we define the "\NewTerm{coefficient of linear expansion}\index{coefficient of linear expansion}":
	
	Furthermore, we proved in the section of Continuum Mechanics, that for an isothermal tensile test (without tangential component) performed on an elastic material we have (implicitly it is the Hooke's law for recall):
	
	But we can also use the true longitudinal deformation afterwards it is then obviously necessary to adapt the right term accordingly:
	
	Therefore it comes:
	
	The state equation of solids can then be written by injecting inside the latter relation:
	
	This relation gives us the variation of relative elongation corresponding to an infinitesimal change of the tensioning force and of the temperature, knowing the linear expansion coefficient .

	The above state equation is easily integrated assuming that the coefficient of linear expansion, Young's modulus and sectional area are constant (let us know as always if you want the details):
	
	If the deformations are very small, we have proved in the section of Continuum Mechanics that the relative variation of elongation could be written:
	
	
	So after integration:
	
	\begin{figure}[H]
		\centering
		\includegraphics[scale=0.1]{img/mechanics/bridge_joints.jpg}
		\caption[Thermal expansion joints like these allow bridges to
change length without buckling]{Thermal expansion joints like these allow bridges to change length without buckling (source: ?)}
	\end{figure}
	\begin{tcolorbox}[colframe=black,colback=white,sharp corners]
	\textbf{{\Large \ding{45}}Example:}\\\\
	A cylindrical steel beam of radius $15$ [mm] is initially in a state of thermodynamic equilibrium characterized by an initial length of $1$ [m] at the initial temperature of $300$ [K] and a zero initial tension force. The beam is heated to a new equilibrium in which the final temperature is $400$ [K]. To calculate the new length in the particular case that the beam is free to move and that the deformation is considered as small:
	
	Therefore:
	
	If the beam is embed at the both sides, we can then calculate the internal stress force generated ("\NewTerm{thermal stress}\index{thermal stress}") under the assumption of small deformations. We then use:
	
	Therefore:
	
	We see here a phenomenon well known in mechanical engineering: the thermal deformation of a solid is usually very small, but prevent it can generate tremendous efforts, that could destroy a mechanism.
	\end{tcolorbox}
	
	\pagebreak
	\subsection{Laws of Thermodynamics}
	The principles of thermodynamics are the bricks of thermal or mechanical energy . Each principle involves a lot of concepts that we will try to present and define as best as we can further below. This is the "sensitive" part of this field of physics.
	
	Thermodynamics is based on four basic laws (some of which are provable):
	\begin{enumerate}
		\item[P0.] The "\NewTerm{zeroth law of thermodynamics}\index{zeroth law of thermodynamics}" or "\NewTerm{principle of thermal equilibrium}\index{principle of thermal equilibrium}" is defined by the fact that if two thermodynamic systems $1$ and $2$ are in thermodynamic equilibrium with a third one $3$, they are themselves in thermodynamic equilibrium (it is an "assertion" in the language of Proof Theory).  This can be denoted formally by:
		
		
		So if two materials in contacts are in thermal equilibrium, they have the same temperature and thus the state remains stationary. Two materials at the same temperature in contact are in thermal equilibrium.
		
		\begin{tcolorbox}[title=Remark,colframe=black,arc=10pt]
		We used this principle implicitly (when we stated that the steady state was the most likely one) in the section of Statistical Mechanics to prove the Boltzmann law.
		\end{tcolorbox}

		\item[P1.] The "\NewTerm{first law of thermodynamics}\index{first law of thermodynamics}\label{first law of thermodynamics}" or "\NewTerm{conservation principle}\index{conservation principle}" refers to the conservative nature of energy and states that during any of a system transformation, the change in its total energy is equal to the sum of changes in all types of energy defining it:
		
		If the system is isolated then the change in total energy will obviously always be zero (in average...)!
		\begin{tcolorbox}[title=Remark,colframe=black,arc=10pt]
		Depending on the authors, the kinetic energy and potential energy are part of the internal energy $U$.
		\end{tcolorbox}
		Of course, when the system studied (the container and its contents) are in the same repository as the observer, it remains only the term of internal energy variation:
		
		or in differential terms:
		
		It remains to determine the exact expression of the change in internal energy $U$ and variables on which it depends (developments that we will do a little further below).
		\begin{tcolorbox}[title=Remark,colframe=black,arc=10pt]
		This principle is provable if we accept Noether's theorem (\SeeChapter{see section Principia page \pageref{noether theorem}}) on time invariance of physics laws of physics as a higher principle.
		\end{tcolorbox}

		\item[P2.] The "\NewTerm{second law of thermodynamics}\index{second law of thermodynamics}" also named "\NewTerm{Carnot-Clausius' principle}\index{Carnot-Clausius' principle}" or "\NewTerm{principle of evolution}\index{principle of evolution}" refers to the irreversible nature and is associated with the concept of entropy. This principle, the most famous one, states that heat can only pass itself from a warm body to a cooler body (or a body will inevitably be cooler with time) in a closed system. The reverse operation requiring the contribution of mechanical work took on the external system that is given by equation (proved a little further below):
		
		The left hand side of equality states that entropy can only increase and that the amount of heat exchange with the external system is always positive (numerator term right to equality). Therefore the body delivers heat to the external system and cooled down inexorably.
		
		The special notation $\delta Q$ is here to indicate that heat is an inaccurate total differential. This is a concept to which we will return further below.
		\begin{tcolorbox}[title=Remark,colframe=black,arc=10pt]
		We will prove roughly ... this relation further below by using the Boltzmann law on the access to the information in a system such as studied in the section of Statistical Mechanics.
		\end{tcolorbox}

		\item[P3.] The "\NewTerm{third law of thermodynamics}\index{third law of thermodynamics}" also named "\NewTerm{Nernst principle}\index{Nernst principle}" concern the properties of matter in the vicinity of absolute zero and states that at the limit of the absolute zero temperature that can not be reached (\SeeChapter{see section of Wave Quantum Physics page \pageref{zero point energy}}), the equilibrium entropy of a system approaches a constant independent of the other intensive parameters, constant that is taken as zero. Or in other words: The entropy of a perfect crystal at absolute zero is exactly equal to zero.
		\begin{dem}
		This statement is based on the proof of the "\NewTerm{Nernst theorem}\index{Nernst theorem}" which formal result is simply based on the second law giving the entropy (\SeeChapter{see section Statistical Mechanics page \pageref{boltzmann law}}):
		
		We see that if we consider a crystal at absolute zero temperature, the particles position against each other are perfectly uniquely defined. Therefore, there is only one possible complexion for crystal at absolute zero. The number of complexions $\Omega$ is equal to $1$. This gives us:
		
		\begin{flushright}
			$\square$  Q.E.D.
		\end{flushright}
		\end{dem}
		Let us recall that the "\NewTerm{absolute zero}\index{absolute zero}" is the lowest temperature theoretically available in physics. At this temperature a substance no longer contains, at a macroscopic scale, the thermal energy (or heat) required for the occupation of several microscopic energy levels (\SeeChapter{see section Statistical Mechanics page \pageref{microstate}}). The particles that make it up (atoms, molecules) are all in the same minimum energy state (ground state). This results in a null entropy due to the indistinguishability of the particles in the same fundamental energy level of entropy (following the third law of thermodynamics) and a total immobility in the conventional sense following the Virial Theorem.
		
		However, in the quantum point of view, we will see in the section of Wave Quantum Physics that particles always have a non-zero linear momentum following the Heisenberg uncertainty relations and that following the Fermi-Dirac and Bose-Einstein distributions (\SeeChapter{see section Statistical Mechanics page \pageref{fermi dirac distribution} and page \pageref{bose einstein distribution}}), there are still different energy levels when we take into account the Pauli exclusion principle. Hence the emergence of quantum thermodynamics.		
	\end{enumerate}
	\begin{tcolorbox}[title=Remark,colframe=black,arc=10pt]
	A common question that arise is why the zeroth law of thermodynamics is named "zeroth law", why didn't physicists start the counting from one as in almost all other fields of physics? Indeed laws of physics are numbered according to their fundamental nature, the more fundamental the law, the higher it is in the pecking order of the laws. E.g., Newton gave the three laws of motion in one go, but he numbered the First Law of motion as the "first law" because it was the most fundamental of all the three law. Without understanding the first law, you cannot go to the second and the third laws.\\

	As far as Thermodynamics is concerned, the First Law had been in currency and had been accepted for long. But when some scientist postulated the (now named) "Zeroth law", the scientific community came to the conclusion that this new law is more fundamental than the First Law because it provides the basis for defining an entity named "Temperature". So they decided to call it the Zeroth Law, i.e. more fundamental than the First Law.
	\end{tcolorbox}
	
	\subsection{Calorific Capacities (heat capacity)}
	In a given transformation, an intake  of heat $\Delta Q$ generally results in an raising $\Delta T$ of the temperature of the system.

	\textbf{Definition (\#\mydef):} 
	We name "\NewTerm{heat capacity}\index{heat capacity}" (or "\NewTerm{calorific capacity}\index{calorific capacity}", or "\NewTerm{thermal capacity}\index{thermal capacity}") $C$ of system, the quantity:
	
	The heat capacity is therefore by its units, the energy it takes to bring a body to increase its temperature by $1$ Kelvin. Under infinitesimal form, we have:
	
	We see that the educational benefit of having a notation using discrete variations rather than infinitesimal allows us not have to ask the question if we have an exact total differential or inaccurate one... (see below for the distinction).
	
	Heat capacity is obviously an extensive property of matter, meaning it is proportional to the size of the system. When expressing the same phenomenon as an intensive property, the heat capacity is divided by the amount of substance, mass, or volume, so that the quantity is independent of the size or extent of the sample:
	
	and then we speak of "\NewTerm{specific heat}\index{specific heat}".
	
	The molar heat capacity is the heat capacity per unit amount (SI unit: mole) of a pure substance (\SeeChapter{see section Analytical Chemistry page \pageref{analytical chemistry}}).
	
	In the case of an ideal gas, we assume that all of the exchanged heat (energy) is changed into kinetic energy of the atoms or molecules (thermal energy), so we can write:
	
	and we will prove later how to determine the exact value of this capability for such gas.
	
	
	and the number of related states was given by:
	
	However, the prior-previous relation is independent of the fact that in the field of electrokinetic we consider the electron or any other particle vector of some property (electric charge, heat or other). Thus in the context of the thermodynamic crystalline solids, a model consist to assume the heat vector as a particle having similar  wave properties to that of an electron that it is customary to name a "\NewTerm{phonon}\index{phonon}".
	
	Assuming that the phonons are identical in all three spatial directions and remembering the Planck-Einstein relation (\SeeChapter{see section Corpuscular Quantum Physics page \pageref{planck einstein relation}}) valid for any particle by the principle of wave-particle duality:
	
	and (\SeeChapter{see section of Statistical Mechanics page \pageref{bose einstein distribution}}) from the Bose-Einstein statistics for bosons (particles not subjected to the Pauli exclusion principle for recall ...):
	
	which we write in the present case as following (since the chemical potential is zero in our case):
	
	we have by combining these three big results:
	
	Then we have:
	
	From here there are two main models for predicting the heat capacity of crystals, the Debye model and the Einstein model, each making assumptions about the density of phonon modes.

	In the Einstein model a single frequency mode (respectively pulsation) is supposed possible, so that the vibration of the solid is represented by $N$ harmonic oscillators at this frequency. This particular pulse is often associated with what we name the "\NewTerm{Einstein temperature}\index{Einstein temperature}":
	
	The density of modes is then symbolized by $N$ Dirac distributions (\SeeChapter{see section Differential and Integral Calculus page \pageref{dirac function}}) around such that:
	
	This model provides a limit value of the heat capacity at high temperatures (we use a Taylor series approximation):
	
	It is customary when we presented this relation to consider a quantity of material corresponding to a mole to make appear the ideal gas constant:
	
	This value was originally given by Dulong and Petit and is in fairly good agreement with experimental results. We often talk about the "\NewTerm{Dulong-Petit law}\index{Dulong-Petit law}".

	In contrast at low temperatures, it is necessary to use the Debye model that consist to assume as linear dispersion relation with the speed of propagation of phonons (speed of sound in the crystal) supposed to be independent of temperature:
	
	The density of states is then written:
	
	We then have for the heat capacity:
	
	In the Debye model, only the acoustic phonon branch is considered, this is equivalent to schematize the crystal as a mono-atomic network formed of heavy atoms. Therefore, the resonance modes are limited to a certain frequency (pulse), with the corresponding volume, known as "\NewTerm{cutting frequency}\index{cutting frequency}" (we change the notation of the Boltzmann constant in $K_B$ so to to confuse it with the wave number $k$):
	
	where we have (we have written the two most common forms):
	
	We then have for the heat capacity:
	
	If the absolute temperature is small, then we have:
	
	What is quite difficult to calculate. So we prefer rather use the following approach:
	
	And this last integral is known to us. Indeed, we will prove it in details below using the Riemann zeta function during our proof of the Stefan-Boltzmann constant:
	
	Therefore:
	
	and experimental measurement correspond weel to that behavior!
	
	In thermodynamics, we must also always specify what is considered transformation as the $\Delta Q$ depends on the nature of the transformation. We distinguish in particular the heat capacity at constant volume (isochoric) and denoted:
	
	and heat capacity at constant pressure (isobaric) often used in chemistry and denoted:
	
	The difference between the specific heat at constant pressure $C_P$ and constant volume specific heat $C_V$ is obviously due to the work that must be provided to expand the body in the presence of external pressure. Both term also highlight a fundamental difference in the experimental setups measurement, that is to say physically different situations. In practice, to use the concept of heat capacity, we need to identify for our experience what $C$ we must use depending on how the installation / machine will have been builed.

	Then it is quite intuitive that the specific heat at constant pressure is always strictly greater than the specific heat at constant volume $C_P> C_V$. We will also prove further below that we specifically for a perfect gas:
	
	If instead of considering the whole system, we report measurements to a unit of mass (which is much more useful in practice!), Then we define the "\NewTerm{specific heat capacity}\index{specific heat capacity}" at constant volume (isochoric):
	
	and the specific heat capacity at constant pressure (isobaric):
	
	If instead of considering the whole system, we report the measurement to one mole of the considered medium (which is much more useful in chemistry!), We define the "\NewTerm{molar heat capacity}\index{molar heat capacity}" at constant volume (isochoric):
	
	and the molar heat capacity at constant pressure (isobaric):
	
	where $n$ is the number of moles of the system.

	So we have of course by extension, by going to the discrete case, the relation that is very useful in practice:
	
	where $c$ is the specific polytropic heat capacity (and not the speed of light...!).

	This is a very important relation in practice and whose everyday applications are huge because it allows in idealized cases to calculate the heat used or provided when a certain amount of material undergoes a change in temperature or calculate the final temperature of two bodies in contact.
	
	This is a very important relationship in practice and whose everyday applications are huge because it allows in idealized cases to calculate the heat used or provided when a certain amount of material undergoes a change in temperature or to calculate the final temperature of two two bodies in contact.

	Think about it when you take your shower or you heat a pot of water. You know then $\Delta T$ and the amount of water $M$ used for your shower/bath or pan. This will give you the total energy used $\Delta Q$ and knowing the cost of energy in your city you quickly deduce the cost of a shower, a bath or a pot of boiling water!

	Another nice application is to calculate the temperature increase (in an ideal case) of a dropped object from a certain height under the assumption that all its potential energy is converted into heat after touching the ground (the only problem being to find the value of the massic polytropic heat capacity in the tables...):
	
	where $c_j$ is the massic polytropic energy capacity (it is simply $c$ but given in joules per kilogram per kelvin rather than calories per kilogram per kelvin). So after rearrangement we have:
	
	The major case of the final equilibrium temperature of two bodies in contact is easy to obtain (if the system is completely isolated from the outside). Indeed, as the heat lost by one of the two bodies will be won by the other, we have by choosing properly $\Delta T$ for each of the two bodies so that $\Delta Q$ is the same sign:
	
	Or after an elementary rearrangement:
	
	\begin{tcolorbox}[colframe=black,colback=white,sharp corners]
	\textbf{{\Large \ding{45}}Example:}\\\\
	A $0.3$ [kg] steel bar has an initial temperature of $1,200$ [K] is immersed in a tank of water at $300$ [K] (the tank is closed, isolated from the outside and full) containing $9$ [kg] of water. Taking the specific polytropic heat capacity of water that is $4.2\;[\text{KJ}\cdot\text{kg}^{-1}\cdot \text{K}^{-1}]$  and that for steel $0.42\;[\text{KJ}\cdot\text{kg}^{-1}\cdot \text{K}^{-1}]$, the final equilibrium temperature (of the bar and water) is then:
	
	\end{tcolorbox}
	Let us now consider the case where we have a system without exchange of mechanical work but only of heat by the following relation:
	
	However, power is energy divided by time (\SeeChapter{see section Classical Mechanics page \pageref{power}}). It comes therefore (take care not confuse the notation of power with with that of pressure!):
	
	and if the heat capacity is given in the massic form at constant pressure:
	
	Which already allows to calculate the power to be supplied to heat a quantity of material at a given temperature in a given time without state change otherwise we must we must take into account the latent heat.

	The previous fraction can also be expressed as the flow rate of a fluid $\dot{V}\;[\text{m}^3\cdot\text{s}^{-1}]$ multiplied by the fluid density:
	
	which allows to know the power supplied by a water flow into a radiator at a certain temperature and leaving it with the same flow but with a different temperature. Those working with surfaces prefer to write this relation as follows:
	
	which is often named "\NewTerm{Newton's law of heat transfer by convection}\index{Newton's law of heat transfer by convection}" and where $h_C$ is the "\NewTerm{convection coefficient}\index{convection coefficient}".

	We will also find sometimes the prior-previous relation in the literature as follows:
	
	after having obviously multiplied both sides of the equality by the time.
	
	\begin{tcolorbox}[colframe=black,colback=white,sharp corners]
	\textbf{{\Large \ding{45}}Example:}\\\\
	A microprocessor surface of $25\; [\text{mm}^2]$ is powered by a power source of $0.224$ [W]. The ambient atmosphere is at $293$ [K] ($20\; ^\circ C$) and the convection coefficient is given at the standard conditions of temperature and pressure as being equal to $h_c=150\;[\text{W}\cdot\text{m}^{-2}]$. \\

	Considerating that all the heat transfer takes place between the air and the microprocessor, the temperature of the latter can then be calculated as follows:
	
	Therefore:
	
	\end{tcolorbox}
	In some transformations, we have a contribution or a heat withdrawal while the system temperature remains constant (typically in phase change transformations of a body). This comes from another phenomenon:

	\textbf{Definitions (\#\mydef):} The "\NewTerm{enthalpy change of state}\index{enthalpy change of state}" or "\NewTerm{latent heat}\index{latent heat}", denoted by $L$ and is given in Joules per kilogram, corresponds to the quantity of heat required for a unit material quantity or a unit of mass of a body to changes its state. This is a very important value in practice to perform calculations.

	So we of course by extension we have the very useful relation in practice:
	
	which thus gives the quantity of heat required to change phase  a certain quantity of mass $m$ (the sign of $Q$ is then negative then by convention). Conversely, if the amount of material returns to a lower energy level, the amount of latent heat will be given to the environment and not absorbed and the sign of $Q$ will be positive! 
	
	Water has the following values for latent heat:
	\begin{itemize}
		\item The latent heat of vaporization of water is $540$ calories per gram (changes $1$ [g] water at $100\;[^\circ \text{C}]$ into $1$ [g] steam at $100\;[^\circ\text{C}]$)

		\item The latent heat of fusion of water is $80$ calories per gram (changes $1$ [g] of ice at $0\;[^\circ \text{C}]$ into $1$ [g] of water at $0\;[^\circ \text{C}]$)
	\end{itemize}
	The process also works in reverse.  Above, heat was added to the substance, changing its phase from solid to liquid or liquid to gas.  What if we cool the substance instead?

	In this case, heat is released as the substance changes phase from gas to liquid and liquid to solid.  It makes sense that the amount of heat required to change a gram of substance from liquid to gas is the same amount that is released when the substance changes from a gas back into a liquid.  Because the values are the same, and the only difference is the direction of the process, we only use a single term for each phase change:
	\begin{itemize}
		\item "\NewTerm{Latent heat of vaporization}\index{latent heat of vaporization}" or "\NewTerm{enthalpy of vaporization}\index{enthalpy of vaporization}": phase changes from liquid to gas OR gas to liquid;
		\item "\NewTerm{Latent heat of fusion}\index{latent heat of fusion}"  or "\NewTerm{enthalpy of fusion}\index{enthalpy of fusion}": phase changes from solid to liquid OR liquid to solid
	\end{itemize}
	
	The following picture shows what happens to $1$ gram ($1$ ml) of water. The orange arrow on the top represents adding heat, from left to right, and the blue arrows represent releasing heat from right to left on the bottom:
	\begin{figure}[H]
		\centering
		\includegraphics{img/mechanics/latent_heat.jpg}
		\caption[Water latent heat process]{Water latent heat process (source: ?)}
	\end{figure}
	
	We will prove later that at constant pressure the latter relation is then written more explicitly as follows:
	
	The enthalpy exchanged during the change of state results from the modification (rupture or establishment) of interatomic or intermolecular bonds. As we know are three main physical states for any pure substance: solid, liquid and gas. The links are stronger in solid form in the liquid state and these links are virtually absent in the gaseous state. There are, as we already know a fourth state obtained at very high temperature where the material is in the form of a plasma of ions and electrons (\SeeChapter{see section Continuum Mechanics page \pageref{plasmas}}).
	
	We differentiate three main modes of heat transfer:
	\begin{itemize}
		\item "\NewTerm{convection}\index{convection}": fluid coming alternately from a hot body and a cold body (which is typical of the movements of the Earth's atmosphere).

		\item "\NewTerm{radiation}\index{radiation}": a body at a given given temperature emits electromagnetic radiation that can be absorbed by another body (the electromagnetic radiation being not necessarilty in the visible spectrum part!).

		\item "\NewTerm{conduction}\index{conduction}": energy transfer occurs from hot areas to cold areas by collision of the excited particles (hot zone) with neighboring particles less excited and so on...\label{convection diffusion radiation}
	\end{itemize}
	\begin{figure}[H]
		\centering
		\includegraphics{img/mechanics/heat_transfers.jpg}
		\caption[Heat transfers mechanisms]{Heat transfers mechanisms (source: ?)}
	\end{figure}
	\subsection{Internal Energy}
	In the section of Classical Mechanics, we have proved (second König's theorem) that the total energy of a body with respect to an external inertial frame $R'$ to the center of mass was given by the sum of kinetic and potential energy compared to that same intertial frame, added to the kinetic energy and potential energy of that body relatively the reference frame of the center of mass $R$ (barycentric repository).
	
	where $U$ is the energy quantity or "\NewTerm{internal energy}\index{internal energy}" (also sometimes name "\NewTerm{proper energy}\index{proper energy}" or of the object itself relatively to its own reference frame.

	In a more simplified form, the previous relation is traditionally written in thermodynamics for a closed isolated system:
	
	The first principle of thermodynamics is then written:
	
	Therefore the total energy is the sum of the macroscopic mechanical energy (kinetic, potential or rotational) and microscopic (internal energy).

	Generally in thermodynamic studied systems are generally at rest ($v_G=0$) with respect to the experimenter and therefore $\Delta E_c=0$. Similarly, we have generally a constant and isotropic potential field in the experimental room or area which implies equation $\Delta E_p=0$.

	As a result of these simplifications the law of conservation of energy, or first principle of thermodynamics, is reduced as we have already mentioned to the internal energy as:
	
	or more rigorously:
	
	\begin{tcolorbox}[colframe=black,colback=white,sharp corners]
	\textbf{{\Large \ding{45}}Example:}\\\\
	A piston contains $0.9$ [kg] of air at a temperature of $300$ [K] and a pressure of $1$ [bar]. This air is compressed to a pressure of $6$ [bar] to a temperature of $470$ [K]. During the compression phase $20$ [kJ] heat is exchanged between the external environment (heat conduction).\\

	Using the fact that the internal massic energy of air is at $300$ [K] given in the thermodynamic tables is equal to $u_1=214.07\;[\text{kJ}\cdot \text{kg}^{-1}]$ at at $470$ [K] as being equal to  $u_2=337.32\;[\text{kJ}\cdot \text{kg}^{-1}]$, then we have the work done for the compression knowing that:
	
	that was equal to:
	
	\end{tcolorbox}
	\begin{tcolorbox}[title=Remark,colframe=black,arc=10pt]
	The internal energy $U$ is often given to a given additive constant, that is why some it is sometimes rightly named the "\NewTerm{internal over-energy}\index{internal over-energy}". As we have proved it in the section of Classical Mechanics, the total energy of a system is the sum of all elementary energies of it. Thus, the internal energy is an extensive quantity.
	\end{tcolorbox}
	Some may ask what is exactly the internal massic energy?! It's quite easy in fact. For example when you touch something hot like a stove, you might think that there is heat in the stove but heat is only the energy being transfer. The energy in the stove is named "Internal Energy". It is made up of the energy of the fast moving atoms and the energy stored in the bonds that hold the atoms of the stove together. The hotter something is the more internal energy the object has. The diagram on the right shows an image of atoms held together with bonds, this is what a molecule of a solid or liquid would look like. When heat is transferred to the atoms they move very quickly, if enough heat is transferred then the bonds will break and the molecule will change state (melt or boil).
	\begin{figure}[H]
		\centering
		\includegraphics{img/mechanics/internal_energy_vibrational.jpg}
		\caption[Vibration bonds source of the internal energy typical in solid and liquids]{Vibration bonds source of the internal energy typical in solid and liquids (source: http://www.physics.ie)}
	\end{figure}
	This is why the internal energy is defined as: the energy associated with the random, disordered motion of molecules. 
	
	Typically the internal energy can be related to the following three internal energies with the corresponding energies proportions for the three common states of matter:
	\begin{figure}[H]
		\centering
		\includegraphics{img/mechanics/internal_energy_proportions.jpg}
		\caption[Proportion of internal energy type by common state of matter]{Proportion of internal energy type by common state of matter (source: Hyperphysics)}
	\end{figure}
	Let us wow consider a system described by the thermodynamic state variables:
	
	In an elementary transformation $\mathrm{d}x_1,\mathrm{d}x_2,\ldots,\mathrm{d}x_n$ the elementary work $\mathrm{d}W$ (if we interested only to that form of energy) will be in the form (\SeeChapter{see section Differential and Integral Calculus page \pageref{total exact differential}}) of an exact total differential:
	
	if we have the following relation, proved in the section of Differential Calculus and Integral, which is satisfied (which are the coefficients of the $\mathrm{d}x_i$ above):
	
	Otherwise, if this relation is not satisfied, which would imply:
	
	whereas in the latter case, we have an inexact total differential (the differential depends of the path!) which would bring us to write:
	
	Since the mechanical work $W$ is in one way or another a force over a distance, we have already proved in the section of Classical Mechanics that it depends on the path traveled and not only the starting and final point.

	In the same transformation, the quantity of heat $\mathrm{d}Q$ can be put in a similar form:
	
	if again the following relation holds for the state variables:
	
	or otherwise:
	
	The characteristic quantities of a transformation of the system but whose value depends not only on the initial and final states, but also t the path followed are named "\NewTerm{transfer quantities}\index{transfer quantities}".
	
	Let us recall that knowing what has already been studied in the section of Differential and Integral Calculus then we have for a thermodynamic cycle in an isolated system:
	
	and that is the fact that the internal energy is an exact total differential which logically requires that the heat $Q$ is an inaccurate total differential by the property of linearity of integral (otherwise it would not cancel by summing the variations of the $W$ a closed cycle). Indeed, according to the first law of thermodynamics, if the energy is conservative for a closed isolated system the we have for the internal energy (in discrete form):
	
	This means that regardless of the transformation, the variations of work and heat are given within this conservative system by (in discrete form):
	
	or more technically:
	
	Therefore regardless of the ways in which transformations occur inside the system, the final and initial state are identical in terms of energy in an isolated system. This assumes that the thermodynamic transformations are independent of the manner in which the phenomena take place inside the system. The term $\Delta U$ is then an exact total differential (\SeeChapter{see section Differential and Integral Calculus page \pageref{total exact differential}}) that we will write $\mathrm{d}U$.
	
	\subsubsection{Work (energy) of Mechanical Forces}\label{work of mechanical forces}
	Mechanical work is the amount of energy transferred by a force. Like energy, it is a scalar quantity, with SI units of joules. Heat conduction is not considered to be a form of work, since there is no macroscopically measurable force, only microscopic forces occurring in atomic collisions. In the 1830s, French mathematician Gaspard-Gustave Coriolis coined the term work for the product of force and distance.
	
	Let us recall now that we have seen in the section of Classical Mechanics that, by definition, the work $W$ is given by a force over a distance (whatever the origin of this force: mechanical, radiant, nuclear, etc.). It follows therefore a very important relation in thermodynamics relation:
	
	which expresses, at constant pressure (isobaric), the variation of the energy due to the work of the external pressure forces on a system (typically a thermodynamic gas..) whose volume has varied (without being restricted by rigid boundary!).
	
	Obviously, if the volume variation is zero or there is no pressure... the change in energy due to work pressure forces will be zero at equilibrium...

	At equilibrium, the pressure $P$ is taken as as the considered system (internal pressure) or that of the surrounding atmosphere as anyways the pressure will be equal if the system is in contact with the atmospherie (otherwise there would be no equilibrium at the opposite of what we have just assumed at the beginning of this paragraph...).

	Similarly, the change in volume is taken as being either that of the considered system (proper volume) or the variation of the surrounding atmosphere since in any case the volume change will be the same for both!

	Moreover, we see that the path is appears explicitly in the previous expression of the work and therefore that it is a quantity that depends on the path traveled (\SeeChapter{see section Differential and Integral Calculus page \pageref{curvilinear integral}}). This implies that we must write $\delta W$ instead of equation $\mathrm{d}W$. There are of course many other ways of expressing the work, but the definition of it itself, it will always be an inexact differential.

	This result has a direct implication on the expression of heat change for which the Virial theorem (\SeeChapter{see section of Continuum Mechanics page \pageref{virial theorem}}): that it is given by the thermal agitation. It should be recalled that this agitation is given by the average kinetic energy and kinetic energy exist by applying a force over a distance for each particle. Thus heat is also an inexact differential $\delta Q$!

	Finally, we have in the case of a fluid (liquid or gas) having an internal polytropic energy change :
	
	During its evolution, the volume $V$ of the system may vary. If we consider an infinitesimal change during which the volume varies of $\mathrm{d}V$ and if we denote  by $P_e$ the external pressure on the system, we will write in a more general context (the negative sign in front of the pressure is a convention!):
	
	where $-P_e\mathrm{d}V$ is the work named "\NewTerm{outside delivery}\index{outside delivery}" (if $\mathrm{d}V>0$, the increase of the volume of the system expresses the supply of work to the external environment, which is what the sign $-$ explains indicating that the internal energy of the system decreases) and where $\mathrm{d}T$ (not to be confused with temperature variation!!!) is another form of energy (if the system is reactive to the chemical level for example, it can provide a potential chemical energy too).

	Except in special systems (batteries, accumulators, ...) the only energy exchanged is the work of the pressure forces so that we keep only:
	
	and then we can write $\mathrm{d}W$ at leisure instead of using the inexact differential $\delta W$ since this work is dependent only of on state variable (the notation $\mathrm{d}W$ being often used in physics in this case)!

	In addition, most systems are studied by hypothesis, at equilibrium, at the same internal pressure as the external pressure (ambient atmosphere). This allows us in this case to write:
	
	where $P$ is the pressure inside and outside of the system (the system is then said "isobaric" as we already know it and rarely "monobaric").
	
	\begin{figure}[H]
		\centering
		\includegraphics{img/mechanics/external_internal_pressure_equilibrium.jpg}
		\caption[]{Example of a sphere at pressure equilibrium with the environment}
	\end{figure}
	Therefore, we have in this particular case:
	

	The heat quantity $Q_V$ into play in an isochoric transformation (at constant volume) then reduces obviously the change in internal energy such that (caution with the traditional notation $Q_V$ that can make us forget that we are dealing with a heat variation !):
	
	If we consider a system evolving from a state $1$ to a state $2$ in a surrounding isobaric atmosphere (constant pressure) and therefore that its internal pressure is in equilibrium with the atmosphere, then we have the "\NewTerm{mechanical forces at constant pressure}\index{mechanical forces at constant pressure}" relation:
	
	Or if the system evolves from a state $1$ to a state $2$ in an isotherm environment, then we have the "\NewTerm{mechanical forces at constant temperature}\index{mechanical forces at constant temperature}" relation for a perfect gas:
	
	If we want to produce a reasonable amount of work it seems a priori that the volume will have to reach unrealistic dimensions. So this is why we build thermal machines with a cyclic return.
	
	OK... with what we have seen so far we can already make the following summary:
	\begin{table}[H]
	\begin{center}
		\definecolor{gris}{gray}{0.85}
			\begin{tabular}{|l|l|l|}
				\hline
				\cellcolor{black!30}\textbf{Process} & 
\cellcolor{black!30}\textbf{What is constant} & \cellcolor{black!30}\textbf{The first law predicts}  \\ \hline
		Isothermal & $T=$ constant & $\Delta T=0$ makes $\Delta U=0$, so $\Delta Q=\Delta W$ \\ \hline
		Isobaric & $P=$ constant & $\Delta P=0$ makes $\Delta Q=\Delta U+W=\Delta U+P\Delta V$ \\ \hline
		Isochoric & $V=$ constant & $\Delta V=0$ makes $\Delta W=0$ so $\Delta Q=\Delta U$ \\ \hline
		Adiabatic & $Q=$constant & $\Delta Q=0$ makes $\Delta U=-\Delta W$ \\ \hline
	\end{tabular}
	\end{center}
	\caption{Summary of basic thermodynamics processes}
	\end{table}
	
	\subsubsection{Enthalpy}\label{enthalpy}
	Let us consider the isobaric case very common in chemistry labs since the beakers are open to the ambient atmosphere. If we consider such a situation, then we have for the variation of internal energy:
	
	so where the pressure $P$ and the volume $V$ are the internal state variables to the studied system! It is important to notice again that if the change in heat $\Delta Q$ is zero and that the internal volume increases the internal energy $U$ variation is then negative, which will be interpreted by the experimenter as an energy taken to the external system and so it is common to speak of "loss" or "endothermic reaction"!
	
	The heat quantity $Q_P$ into play in an isobaric transformation (the index of $Q_P$ is indicated in this purpose... can make us forget that we are dealing with a variation of heat) is equal to the variation of two terms in identical form:
	
	where we therefore define a new convenient function of state, named "\NewTerm{enthalpy $H$}\index{enthalpy}" (extensive quantity) in an isobaric transformation as given by:
	
	where $n$ is the number of moles internal to the ideal gas undergoing the volume change! We see further that if the surrounding pressure is zero, the enthalpy is equal to the internal energy. The fact that there are external pressures force (or internal) adds an energy to the system which therefore defines the concept of enthalpy!

	The prior previous relation expresses the fact that when a system evolves at constant pressure, the heat received (or exchanged by the system with the external environment) is equal to the enthalpy change.

	It also comes with this new definition a new possible writing of to the heat capacity at constant pressure frequently used in chemistry:
	
	So get the very important relation in practice that we had specified earlier above for the latent heat (enthalpy of state change):
	
	where we must not forget that strictly speaking the fixed pressure heat capacity is not constant, it is quite often considered in practice as constant for a given temperature range.

	More explicitly, we have also for the prior previous relation using the ideal gas law:
	
	where for recall (\SeeChapter{see section Continuum Mechanics page \pageref{virial theorem}}) the number of degrees of freedom $\mathrm{df}$ for a perfect monatomic gas is equal to $3$. The last term of this relation is obviously not valid for fluids!

	Sometimes we obviously also write the definition of enthalpy in the form of a variation such that:
	
	in the particular case of application to the field of validity of the ideal gas (\SeeChapter{see section Continuum Mechanics page \pageref{ideal gas law}}) regarding to the last term.

	It is important to remember that the term $PV$ in the previous relations represents the work of the pressure forces from the surrounding atmosphere on the system or by equivalence, respectively of the system on the surrounding atmosphere that surrounds it. But the variable $n$ is always the number of moles of an ideal gas of the system studied and not the surrounding atmosphere.

	The enthalpy is a concept used extensively in thermal chemistry (see chapter of the same name) and we will use it constantly in his study.

	In practice it is difficult (if not impossible) to know the internal energy. We then calculate rather the relation relatively to one mole ($n$ being then equal to $1$) of a perfect gas:
	
	That is a strictly positive value giving the energy (or rather the surplus of energy) available in the ideal gas because of the surrounding pressure (there is only to put the pressure as being equal to zero to see it! ).
	
	\begin{tcolorbox}[colframe=black,colback=white,sharp corners]
	\textbf{{\Large \ding{45}}Example:}\\\\
	Let us take a molar unit volume of ideal gas at standard temperature and pressure (whatever that perfect gas, the molar volume under these conditions will always $24$ liters according to the ideal gas law!).

	The work of the pressure forces that were necessary to bring this ideal gas to the above conditions is then:
	
	or:
	
	which is energy in the form of mechanical forces of pressure that we could recover from a molar volume of air in comparison to the vacuum state.

	The internal energy into the form of heat and mechanical work that we could recover in theory (the problem is how to...) of the molar volume is given for a monoatomic gas (see further below for the detaield proof), by :
	
	and we could also calculate the internal energy of the electrons binding, and that of the nuclei of atoms... and the equivalence mass and energy, but we would go out of common industrial case...\\

	For water, we can only use the first relation, because the second one is valid only for ideal gases, the molar volume is $1,000$ times smaller, we can only extract of it an energy $1,000$ times smaller (about $2$ [J]). This is why in the case of fluids, we consider that there is no difference between the enthalpy and internal energy.
	\end{tcolorbox}
	Let us now see other theoretical implications of some previously seen items that will be very useful both in acoustic (\SeeChapter{see section Music Mathematics page \pageref{acoustic}}) or in Continuum Mechanics (see section of the same name page \pageref{continuum mechanics}).
	
	\subsubsection{Laplace's Law}
	We have proved in the section of Continuum Mechanics with the Virial theorem that the internal energy (kinetic energy) of a perfect monatomic gas was given by:
	
	where we took the notation of the section of Contiuum Mechanics (an uppercase $N$ instead of an uppercase $n$).

	Therefore we have:
	
	If the transformation process is at constant volume, we will assume that there is no mechanical work (force over a distance) provided (inelastic collisions on the walls) and then:
	
	therefore where $\mathrm{d}W$ is zero!

	It comes then the possibility to calculate the heat capacity at constant volume:
	
	where we then observed, that for an ideal gas, the volumetric heat capacity is independent of the volume !! We have verbatim for a mole:
	
	so that we can write for an ideal monatomic gas at constant volume the internal energy:
	
	which allows to know in practice, and from table, the internal energy variation of a gas depending on its temperature variation since:
	
	which of course corresponds to a supply work or absorbed work. Under its differential form we name it the "\NewTerm{first Joule's law}\index{first Joule's law}" (internal energy of an ideal gas depends only on the temperature)
	
	An interesting case of the prior previous relation is to remember that as we are at constant volume we have then:
	
	This allows us to write:
	
	With the important part:
	
	So if there is decompression, the latter relation is negative. In this case the gas loses heat to the environment this is why when a container of a gas release a gas it cools down!

	If the process takes place at constant pressure (constant kinetic energy of the gas atoms) then we have (see Virial theorem in the section of Continuum Mechanics)
	
	(the collisions that push the wall of the volume make the system loose hence the sign "$-$").

	Therefore:
	
	Thus we can also calculate the heat capacity at constant pressure:
	
	where can therefore observed, for an ideal gas, that the heat capacity at constant pressure is also independent of the pressure !! We have verbatim for a mole:
	
	Of the two boxed previous results we get for an ideal mono-atomic gas the "\NewTerm{heat capacity ratio}\index{heat capacity ratio}" or also named "\NewTerm{adiabatic index}\index{adiabatic index}" or "\NewTerm{ratio of specific heats}\index{ratio of specific heats}" or "\NewTerm{Poisson constant}\index{Poisson constant}":
	
	with the "\NewTerm{Mayer relation}\index{Mayer relation}"
	
	which gives us the same time as $R$ is positive:
	
	We get also a relation that will be useful later:
	
	\label{adiabatic index diatomic gas}If the ideal gas is diatomic, there are $5$ degrees of freedom ($3$ for the position of the first atom, $3$ for the position of the second one, $-1$ for the constraint that the distance between them is fixed) and we then then:
	
	By doing the same developments, we get (value that will use in the section Music Mathematics but which is useful in many other fields of physics):
	
	When an isolated system of ideal gas undergoes an adiabatic transformation at constant pressure, the internal energy change in the system will be drawn by the variation of internal work. Which traditionally is written by a negative sign such that(using the result above):
	
	\begin{tcolorbox}[title=Remark,colframe=black,arc=10pt]
	Warning!!! Let us recall that the choice of the sign for the work $W$ is only a sign convention !! Thus, in this case study it is the tradition to put a "$-$" instead of a "$+$". But this does not change the results that will follow !!!
	\end{tcolorbox}
	Hence:
	
	Let us now take the ideal gas equation $PV=nRT$ (without collisions) and differentiate. We get:
	
	Therefore by eliminating $\mathrm{d}T$ between the last two relations, we get:
	
	Thus after simplification and rearrangement of the terms:
	
	What reported to moles quantities is written (as customary for chemists)
	
	anyway ... by remembering that:
	
	We have:
	
	Using the definition of the Laplace coefficient, also named "adiabatic coefficient\label{adiabatic coefficient}" already met earlier:
	
	we get the expression:
	
	Rearranging we get:
	
	We get by integration:
	
	therefore:
	
	which is equivalent using the properties of logarithms to:
	
	This is the famous "\NewTerm{Laplace's thermodynamcis law}\index{Laplace's thermodynamcis law}" or "\NewTerm{polytropic gas equation}\index{polytropic gas equation}\label{polytropic gas equation}" which gives the relation between pressure and volume in an adiabatic transformation of a gas (which does not mean that the temperature is constant for recall but only that exchange of heat with the outside system is zero or at least negligible!). We will reuse besides this relation in the section of Marine \& Weather Engineering.

	Thus, we also have the following information that may be useful in the industry:
	
	As well as:
	
	verbatim:
	
	So in the end we have the very useful relations in practice for the polytropic case:
	
	The problem in practice... is to have in the value of the adiabatic coefficient ...
	
	OK so far we can make the following small summary hoping it helps:
	\begin{table}[H]
		\begin{center}
		\definecolor{gris}{gray}{0.85}
		\renewcommand{\arraystretch}{2.6}
		\begin{tabular}{|l|c|c|c|c|}
		\hline
		\cellcolor{black!30} & 
\cellcolor{black!30}$\pmb{\Delta Q}$ & \cellcolor{black!30}$\pmb{\Delta W}$ &\cellcolor{black!30}$\pmb{\Delta U}$ & \cellcolor{black!30}$\pmb{\Delta H}$ \\ \hline
		\cellcolor{black!30}\textbf{Isothermal} & \parbox{3.7cm}{$\Delta Q=nRT\ln\left(\dfrac{V_f}{V_i}\right)\\=nRT\ln\left(\dfrac{P_f}{P_i}\right)$} & $-\Delta Q$ & $0$ & $0$ \\ \hline
		\cellcolor{black!30}\textbf{Isochoric} & $\Delta Q=nC_v\Delta T=\Delta U$ & $0$ & $\Delta U=nC_v\Delta T$ & $\Delta H=nC_P\Delta T$ \\ \hline
		\cellcolor{black!30}\textbf{Isobaric} & $\Delta Q=nC_P\Delta T=\Delta H$ & \parbox{2.6cm}{$\Delta W=-P\Delta V\\=-nR\Delta T$} & $\Delta U=nC_v\Delta T$ & $\Delta H=nC_P\Delta T$ \\ \hline
		\cellcolor{black!30}\textbf{Adiabatic} & $0$ & $\Delta W=nC_V\Delta T$ & $\Delta U=nC_v\Delta T$ & $\Delta H=nC_P\Delta T$ \\ \hline
		\end{tabular}
		\end{center}
		\caption{Main ideal gas processes (reversible processes)}
	\end{table}
	We will use the polytropic gas equation above for an adiabatic expansion to calculate the change $\Delta P$ in pressure when we change the volume of the gas in a cylinder by an amount $\Delta V$. Before we compress we have:
	
	where $P_0$ and $V_0$ are our original pressure and volume. 
	
	After the expansion, $V$ goes to $V_0+\Delta V$ and $P$ goes to $P_0+\Delta P$, where we know that $\Delta P$ is negative for an expansion. Thus after expansion we have:
	
	With $PV^\gamma=P_0V_0^\gamma=c^{te}$, we get:
	
	We can use the fact that $\Delta V$ is very small compared to $V_0$ to get:
	
	Using the binomial series (\SeeChapter{see section Calculus page \pageref{binomial coefficient development}}):
	
	for a small $\alpha$, we have:
	
	Therefore:
	
	Multiplying this out gives:
	
	The factors $P_0V_0^\gamma$ cancel, and we can neglect the second order term $\Delta P\Delta V$, giving:
	
	After canceling the $1$'s and multiplying through by $P_0$ we get for the pressure change $\Delta P$:
	
	
	\subsubsection{Saint-Venant Thermodynamic Equation}
	There are several equations that bear the name of "\NewTerm{Saint-Venant equation}\index{Saint-Venant equation}" in physics ... one that concerns us here is very useful in the field of aero-dynamics and is more "thermodynamics" oriented that the equation of the same name in the field of incompressible fluid mechanics.

	Let us see what it is!
	
	We will consider a compressible gas or fluid always using the first law of thermodynamics but by highlighting from the internal energy explicitly three macroscopic energies commonly used in the field of continuum mechanics:
	
	To introduce the Saint-Venant equation for thermodynamics as simple as possible, we will consider only the cases where the changes in potential energy, heat exchanges and work are negligible or zero (which is a strong hypothesis but not non-existent in some practical cases). Then we have:
	
	which seems sometimes be named "\NewTerm{Thomson equation}\index{Thomson equation}" or "\NewTerm{Zeuner formula}\index{Zeuner formula}" (we give names to anything sometimes...).
	
	Which brings us to write:
	
	Now we have prove earlier above that:
	
	Often the latter relation is expressed in massic form such that:
	
	or written differently:
	
	It comes therefore the "\NewTerm{Saint-Venant equation for thermodynamics}\index{Saint-Venant equation for thermodynamics}":
	
	If we rewrite this relation between two points but we choose one of the two points has zero speed (whatever which one) we can always bring us back to a relation of the form:
	
	But we have proved in the section of Music Mathematics that the speed of sound was given by:
	
	where $r$ is the ideal gas massic constant ...

	Then we can write:
	
	We then introduce the "\NewTerm{Mach number}\index{Mach number}":
	
	Therefore:
	
	But we also have:
	
	Hence after a small rearrangement:
	
	But using the relations proved above and obtained from the Laplace's law we get:
	
	
	\begin{tcolorbox}[colframe=black,colback=white,sharp corners]
	\textbf{{\Large \ding{45}}Example:}\\\\
	Compressed air contained in a large tank escapes outward through a hole at a Mach number equal to $M=0.77$. Relaxation occurs in the atmosphere where there is an atmospheric pressure of $1.014$ [bar]. The massic ratio of air at the NTP is given as being worth $\gamma=1.4$. We then the pressure inside the tank which is given by:
	
	We can also have fun by calculating the internal necessary pressure of the tank for the flow to become supersonic (and then do not forget that ... action = reaction... lol).
	\end{tcolorbox}
	
	\subsubsection{Thermoelastics coefficients}
	If we differentiate $V(P, T)$ we have (\SeeChapter{see Differential and Integral Calculus}):
	
	or writtent differently:
	
	We have introduced naturally in the section Mathematical Music a coefficient named "\NewTerm{coefficient of isothermal compressibility}\index{coefficient of isothermal compressibility}" that was for recall (the index $T$ indicates that the temperature is kept constant):
	
	that we can reintroduce here:
	
	Also it would be interesting to have a different coefficient for the first term that it would suffice to define by analogy (the index $V$ is there to indicate that this is a volumic coefficient and the $P$ to indicate that we are at constant pressure):
	
	named "\NewTerm{isobaric volume expansion coefficient}\index{isobaric volume expansion coefficient}".

	Thus we have:
	
	Therefore we have the mechanical work (the total differential is inaccurate because we have more than one state variable) by multiplying the pressure to have the correct units:
	
	For an isothermal transformation the first term is zero, and for an isobaric transformation, it is the second which is zero.

	The fact to know the thermoelastic coefficients (measured experimentally) must bring us to fall back on the ideal gas law by integration of $V (P, T)$, which is minimum we should expect, since $V$ is a state function. In the case of the ideal gas, for example, we can write intuitively by the dimensions of the constants (we will introduce further below during our study of the equations of state the reverse mathematical path for pedagogical reasons):
	
	Therefore we have:
	
	and therefore by multiplying by the pressure:
	
	Which leads to:
	
	Therefore:
	
	Giving immediately after integration:
	
	Therefore=
	
	That said, we have differentiated $V$ to get two coefficients such as:
	
	We could do the same for the pressure and temperature, and then we have a total of three relations:
	
	but among the $6$ factors that we see in these three relations, $4$ are already defined (some are the opposite of the coefficients defined above). By cons a definition is missing for a single coefficient for the both missing factors. We choose the one that in practice is most often used by analogy to the other factors:
	
	named "\NewTerm{coefficient of increase of isochoric pressure}\index{coefficient of increase of isochoric pressure}".

	We have therefore three very used coefficients in practice:
	
	respectively in the order:
	\begin{enumerate}
		\item Coefficient of isothermal compressibility $\chi_T$

		\item Isobaric expansion coefficient $\alpha_{V,P}$

		\item Isochoric pressure increasing coefficient $\beta$
	 \end{enumerate}
	
	
	We wee find these coefficients again in our study of convective motions in the section Weather \& Marine Engineering.
	
	\subsection{Heat}
	Before continuing, we have at all costs to eliminate one of the greatest difficulties in the understanding of thermodynamics (apart from that of differentiation between correct and incorrect total differential that has already been set in the section of Differential and Integral Calculus):
	\begin{itemize}
		\item The difference between heat and temperature

		\item The difference between the energy-work and energy-heat
	\end{itemize}
	As we saw in the section of Continuum Mechanics, the temperature characterize a state of thermodynamic equilibrium and translates the existence of a thermal motion (virial theorem) and can vary when the outside provides a work $W>0$. However, experience shows us that it is by "heating" a system that we increase the more easily its temperature. But then what is the "heat"?
	
	Let us consider a thermodynamic system in equilibrium, and let us write its total internal energy $E$ as the sum of the products of the energy $E$ of a microstate $i$ by the population size $P$ of the same microstate $i$:
	
	Its variation during an infinitesimal transformation (which we assume with a constant number of particles) is (total differential):
	
	where $\mathrm{d}E_i$ is the variation of the energy $E_i$ of the microstate $i$ caused by the transformation and $\mathrm{d}P_i$ the variation of the population $P_i$ of this same microstate $i$.

	We have seen earlier above that under certain conditions, if the system during the infinitesimal transformation undergoes a variation of work $\delta W$ and of heat $\delta Q$, its internal energy varies of
	
	We can now compare this last relation (expressing the first law of thermodynamics) with the one before (resulting from statistical mechanics) and identify the terms as follows:
	
	Let us examine the case of processing in which the system receives only heat:
	
	In this situation no external parameters to the system generally varies in the transformation so that $\mathrm{d}E_i=0$. We conclude:
	
	Thus, when a system receives only heat, its energy vary by the change of the populations of its microstates: If the amount of heat received is positive, the probability of the high energy microstates increases, at the expense of the microstates of lower energy. Finally, if we take into account that the system must be in equilibrium (the most probable state as we have proved it in the section of Statistical Mechanics) in the initial and final states of transformation, we find, as we should expect, that the system temperature varies: it has increased if the amount of heat received is positive (this can be proved by choosing a canonical distribution to describe the system of macroscopic equilibrium states as done in the section of Statistical Mechanics).

	This explains the frequent confusion between these two very different concepts that are temperature and heat. This confusion is compounded by the gap between everyday language and scientific terminology. In everyday language, when we talk about "heat" of a body, we say actually that its temperature is high. The confusion is unfortunate because the concept of heat is present in physics but its meaning is different.

	Thus, heat a system, is provide heat to it, is increasing its internal energy (the number of high-energy micro-states) by means that are not purely mechanical. The heat is thus a particular form of energy!
	
	\pagebreak
	\subsubsection{Entropy}\label{entropy}
	An isolated macroscopic system tends towards equilibrium. It reached it in a finite time (which can be extremely long). The steady state is unique: the exceptions to this claim are too special to merit a digression at least for now.

	The very existence of a state of equilibrium is fundamental to thermodynamics. However, the equilibrium walk process is not due to a dogma: there must be no dogma in physics as we know (and in science in general)! Like any other law, it is subject verified experimentally and must be analyzed. One question in particular arises: what is the microscopic counterpart of the equilibrium walk process, that is a macroscopic process also.

	We proved in the section Statistical Mechanics that by definition: the equilibrium state is the state which is have the largest number of configurations (microstates) and is the most probable state.

	What had brought us to the following relation:
	
	where $S$ represents the statistical expected mean of information on microstates and that we named for recall "\NewTerm{entropy}\index{entropy}". Clearly $S$ has the units of $\lambda$ that as we showed it is a constant.

	The question that arises in thermodynamics is: what is the constant that permits characterize for a gas, liquid or solid the expected number of microstates.

	Then it comes when looking at all the relation in thermodynamics one constant that always appears when it comes to characterize a thermodynamic state. This is the Boltzmann constant:
	
	So $S$ has units corresponding to the ration J / K that thus permits to measure the degree of disorder in a system at the microscopic level. Intuitively: the higher the entropy of the system, the less its elements are ordered and capable of producing mechanical effects, and the greater the part of energy unused or used inconsistently.
	
	The entropy is an extensive quantity. Indeed, we have proved that the choice of the logarithm in the Boltzmann law came for the fact that the entropy of a macrostate was the expected mean of all microstates:
	
	what had brought us to:
	
	and shows well that entropy is an extensive quantity because summable on micro-states (complexions).

	What remains difficult now is whether to know if the energy in the units of entropy comes from the work $W$ or from heat $Q$ or from both? In fact the answer is quite simple because in our development of the Boltzmann law, at any time the system (ideal) studied has provided (mechanical) work. So the only energy involved is that of heat.

	Therefore:
	
	However, entropy can not be given rigorously by:
	
	because that is the definition of specific heat! Furthermore, if the reader remembers our developments in the section of Statistical Mechanics, the study was done in an isolated system with two cavities. So if the transition from one cavity to the other is very slow (so that there is not an expansion of the gas) the temperature will remain constant (isothermal expansion). Which implies the definition (so we fall back on the discrete form of the second law of thermodynamics already introduced at the beginning of this section):
	
	To move to the differential form, it should be remembered that heat $Q$ is an inexact differential. So for a reversible transformation:
	
	Therefore the entropy is an exact total differential (heat depends on how the transformation takes place, but $S$ depends only on the final and initial state of the heat $Q$) and as:
	
	Finally we have:
	
	If the system is in an adiabatic transformation (without heat and work exchange with the outside), then the numerator is zero and the entropy is also zero verbatim (this is why an adiabatic process is necessarily isentropic!). Otherwise, the system takes the entropy to the Universe in a natural evolution.
	
	This means that the entropy (expected mean of the intrinsic information) in a system in contact with the outside can only increase or remain constant.

	What is important is that any processes (non-adiabatic) converting energy from one form to another in an isolated system necessarily loses some form of heat.

	Regarding the Universe ... the question is whether this is an isolated thermodynamic system or not ... But as hypothetically the Big Bang created time and space (since Special Relativity shows that space and time are inseparable), then nothing existed before him since time did not exist before ... and there was no space since that is the Big Bang that at the origin of space (comoving concept proofs it as seen in the section of Cosmology). So there is nothing around (hypothetically...).

	We then have the very useful relation in fluid mechanics (and cosmology):
	
	which is named "\NewTerm{thermodynamic identity}\index{thermodynamic identity}\label{thermodynamic identity}" relatively to the internal energy $U$ or "\NewTerm{characteristic function of a fluid at equilibrium}\index{characteristic function of a fluid at equilibrium}" or more often "\NewTerm{fundamental Gibbs' equation}\index{fundamental Gibbs' equation}" (we will notice that it is a sum of products of intensive and extensive quantities). If the amount of material is not constant (material inserted or extracted from the system), the latter relation is then written for a variation of moles $\mathrm{d}n$:
	
	where $\mu$ is a coefficient of proportionality named "\NewTerm{chemical potential}\index{chemical potential}" and can therefore be measured by the fact that the previous relation leads to the definition:
	
	which means that the chemical potential $\mu$ is the internal energy variation of the gas when we inject it in or withdraw from it $\mathrm{d}n$ moles of gas, while keeping constant the volume of the tank and prohibiting heat exchange with the environment.

	We can often found in the prior previous relations follows after a simple rearrangement in many text books (and assuming constant the quantity of material):
	
	The latter relation is often assimilated to the first law of thermodynamics for closed systems for whose the overall variation of potential and kinetic energy is constant.

	Let us notice that by using the relation between the internal variation of energy of an ideal monoatomic gas in function of its temperature at constant volume as proved earlier above (first Joule's law):
	
	and the ideal gas equation (\SeeChapter{see section Continuum Mechanics page \pageref{ideal gas law}}):
	
	We then have using the molar heat capacity at constant volume (remember that we can inject into this relation only for an ideal gas because it is then independent of the volume!):
	
	So integrating between two states, we get:
	
	Thus the entropy change of an ideal gas at constant pressure:
	
	And as we have also proved above that:
	
	It comes therefore by using the algebraic properties of the logarithm:
	
	And as in practice it is easier to measure the pressure than the volume, we use the ideal gas law where at constant temperature we have (\SeeChapter{see section Continuum Mechanics page \pageref{boyle mariotte law}}):
	
	to get the very important next result in practice considered the entropy change of an ideal gas at constant volume:
	
	The reader must not forget that if the calculation with numerical applications give negative values for the change in entropy then the interpretation is quite simple: that means that the direction of the arbitrary considered transformation is not the real (natural) direction in which would conduct the system left to itself in the Universe (ie it is then an artificial transformation).

	\pagebreak
	\paragraph{Heat Flow}\mbox{}\\\\\label{heat flow}
	Now that we have introduced the basics of thermodynamics let do on an example of very important application of the first and second laws of thermodynamics.

	Let us consider two closed systems $1$ and $2$ being in thermal contact and forming a total system that is assumed thermally isolated from the environment:
	\begin{figure}[H]
		\centering
		\includegraphics{img/mechanics/isolated_sytem_with_two_systems_in_contact.jpg}
	\end{figure}
	In both systems, there are no mechanical parts, such as pistons or other, that can work. Therefore:
	
	The conditions and assumptions being known, we use ingredients of thermodynamics to make our recipes. It then comes by applying the first principle as proved above in a first time for the whole:
	
	We deduce that:
	
	We also have for each sub-system:
		
	Hence we deduce from the latter system of equations by summing:
	
	and as we have proved just before:
	
	It comes:
	
	and therefore (obvious!):
	
	Let us now apply the second principle (principle of evolution). We have:
	
	But with the prior previous relation, we have:
	
	We now have two possible scenarios. For the first, we have:
	
	and when:
	
	Therefore:
	
	For the second scenario, we have:
	
	and when:
	
	Therefore:
	
	We conclude that whatever happens, entropy increases for an isolated system. This result is therefore consistent with what we communicated during our presentation of the second principle. So this result proves the impossibility for an isolated dithermal thermal machine to provide a perpetual work because the entropy can only increase in all cases!

	We also see, as we have already mentioned it, that entropy defines the arrow of time in the Universe (or vice versa).
	
	\subsubsection{Carnot Cycle}
	The Carnot cycle's main objective is to calculate the thermal efficiency of a thermodynamic cycle of a dithermal idealized machine ... (so the model is far from the practice!) Based on a heat source from which we take heat to turn it into work and then from which we use work to provide heat and so on in loop(think of a locomotive for example!). We speak then of "\NewTerm{reverse cycle}\index{reverse cycle}".

	The idea is pretty simple ... but as always someone had to think to it first! 

	For information we have chosen here a mathematical approach of the Carnot cycle which allows to avoid any integral and that is closer to the reasoning made at the time of Carnot by colas Léonard Sadi Carnot himself. This approach can therefore make bristle the hair to some readers ...

	From the perspective of the internal energy, on a thermodynamic cycle, there will be no variations (since at each cycle beginning we must found ourselves in the situation of the previous cycle). Therefore:
	
	This is so far simply a recall of the principle of conservation of energy (first law of thermodynamics).

	This means that on a cycle the heat balance $Q_e$ borrowed by the machine to produce a (mechanical) work $W_p$  and  of the heat $Q_i$ reinjected and that required a (mechanical)  work $W_i$ is zero on a cycle such that:
	
	where what is obviously borrowed will, as it is customary, have a positive sign and what is injected a negative sign !!!!! It is customary to notice this last relation in the following condensed form:
	
	The "\NewTerm{thermal efficiency}\index{thermal efficiency}\label{thermal efficiency}" $\eta$ is defined as the ratio of the work that machine can really provide (ie the difference between what it gaves $W_p$ and what we must give to it $W_i$ so that it works) on heat that was injected (with the working equation). So:
	
	which is always less than $1$ (and greater than $0$), since the amount of heat removed (positive sign by convention) is necessarily less than or equal to the amount of injected heat (negative sign by convention). 
	\begin{figure}[H]
		\centering
		\includegraphics{img/mechanics/heat_engine.jpg}
		\caption[A schematic representation of a heat engine]{A schematic representation of a heat engine (source: OpenStax)}
	\end{figure}
	However, we have proved that in the condition of isothermal expansion or isothermal compression (so the cycle is assumed to be very slow in the Cournot model), we had:
	
	But, in our case $Q_e$, $Q_i$ are in reality an abusive notation in thermodynamics as it is heat variation (we should notice them objectively $\Delta Q_e$,$\Delta Q_i$). We then have for the reversible  thermal efficiency for isothermal compression or expansion:
	
	The transformation being quasi-static, the relation obtained is obviously the theoretical maximum efficiency for an engine operating between these temperatures. It is never reached in a real cycle! We then say that the "\NewTerm{Carnot efficiency}\index{Carnot efficiency}" is a maximum (the entropy injected being in a real case always greater than the entropy extracted by construction!).

	This leads us to the "Kelvin-Planck statement" that says:
	\begin{itemize}
		\item First alternative: it (seems as far as we know) that it is impossible to design a thermal machine that describes a cycle and that would have for only effect to produce work and to exchange heat with a single thermal tank.

		\item Second alternative: it is impossible for a thermal machine making a cyclic process to convert entirely into work the heat it absorbs.
	\end{itemize}
	In the case of a steam engine, the maximum theoretical efficiency would be (remember that the temperature is in Kelvin, and always positive!):
	
	For example in the case of a central nuclear working with pressurized water, we find the famous performance often mentioned in the press:
	
	The real thermal efficiency of the conversion of fission heat to electricity fissions is in the order of only $33\%$ because the real thermal efficient is always less than the theoretical maximum thermal efficiency!

	The relation:
	
	has a very useful interest in practice. Indeed, if an engineer said to have built a machine that is able to convert for example $1000$ [kJ] in input in $410$ [kJ] of pure work with an efficiency of $41\%$, then other engineers can check if this thermal efficiency is overestimated by measuring the temperature of the input source and that of the output.
	\begin{tcolorbox}[title=Remark,colframe=black,arc=10pt]
	We may notice the interest of the condenser. If we let escape to the atmosphere the steam coming out of turbines, firstly we would lose purified water instead of recycling it (for reuse in the cycle), on the other part we would have a heat source of higher temperature: $100$ [$^\circ$C] instead of $100$ [$^\circ$C] we would lose about $12\%$ of the Carnot efficiency.
	\end{tcolorbox}
	Finally, we implicitly also find one of the historical statements of the second law of thermodynamics considering where equation. In this case, it is not effective and the engine does not therefore provide any work.

	Let us return briefly to the fact that in the ideal case of the cycle considered above which for recall is therefore an ideal thermodynamic cycle consisting of four reversible processes: isothermal expansion, adiabatic expansion (so isentropic because reversible), isothermal compression, and adiabatic compression ...:
	\begin{figure}[H]
		\centering
		\includegraphics{img/mechanics/carnot_cycle.jpg}
		\caption{Carnot cycle illustration}
	\end{figure}
		
	we were led to write:
	
	But the second law of thermodynamics says that the entropy change is always positive or zero, what we wrote for recall:
	
	but in the above reversible case, for the cyclic variation of entropy of the reversible process to be zero that mean that one of the both term violates must violate this principle (at least in the case of a cycle) and therefore has a negative entropy (so that the sum is zero) such that we are led to write:
	
	It is a violation of the reality that thermodynamicists name "\NewTerm{Clausius inequality}\index{Clausius inequality}". It is then customary to summarize the thermodynamic cycles as follows:
	\begin{itemize}
		\item $\displaystyle\oint\dfrac{\delta Q}{T}=0$: no irreversibility in the system
	
		\item $\displaystyle\oint\dfrac{\delta Q}{T}>0$: irreversibilities in the system (the reality of our Universe)

		\item $\displaystyle\oint\dfrac{\delta Q}{T}<0$: impossible (at least as far as we know...)
	\end{itemize}
	As the second relation listed above is not possible we then have for any system exchanging heat with external reservoirs and undergoing a cyclic process:
	
	that is also named "\NewTerm{Clausius theorem}\index{Clausius theorem}".

	This theorem can be proved but all the proofs I know so far are not aesthetic in my point of view. But it's quite obvious and this relation is just a special case of the second principle of thermodynamics.

	Especially the case if during a cycle no information is lost (as we have proved in the section of Statistical Mechanics, entropy is related to information!) is is quite obvious that:
	
	
	\subsection{Maxwell relations}
	Maxwell's relations are a set of equations in thermodynamics which are derivable from the symmetry of second derivatives and from the definitions of the thermodynamic potentials. These relations can be useful when we are not able to directly measure a given thermodynamics property but that we can workaround this issue by expressing it in terms of other thermodynamic variables.
	
	Let us return first to what we have already mentioned a little earlier above but by restricting ourselves to two variables. That is to say the exact total differential:
	
	When we assume that we are authorized to write the differential of a state variable in thermodynamics as above, and especially for a simple compressible system that is completely specified by two independent, intensive properties, then we say that we are using the "\NewTerm{state postulate}\index{state postulate}".
	\begin{tcolorbox}[title=Remark,colframe=black,arc=10pt]
	Two properties are considered independent if one can be varied while the other is held constant. For example, temperature and specific volume are always independent. However, temperature and pressure are independent only for a single-phase system; for a multiphase system (such as a mixture of gas and liquid) this is not the case. (e.g., boiling point (temperature) depends on elevation (ambient pressure)).
	\end{tcolorbox}
	So we have also:
	
	By inserting $\mathrm{d}y$ in $\mathrm{d}x$:
	
	or also:
	
	as the terms in parenthesis are functions and that $\mathrm{d}x$ and $\mathrm{d}z$ are by cons arbitrary, the only solution to this relation is:
	
	Multiplying the second relation by $(\partial z/\partial x)_y$:
	
	Then we get after simplification:
	
	Let us now turn to the facts. Let us recall the fundamental Gibbs' equation:
	
	relation very useful in fluids where the pressure is constant and where the heat change is made by that of entropy. And the relation defining the enthalpy:
	
	from which will modify the differential:
	
	by injecting in it (first principle):
	
	we have (notice that the transformation is isobaric and that the enthalpy change then represents only the amount of heat received by the closed system):
	
	Thus, having just added $PV$ to the internal energy $U$ we do have a state function ($H$) where we control the pressure and the entropy as independent variables, while for $U$ alone, we control the entropy and volume as independent variables.

	So we will use the following two relations that will be helpful:
	
	We now introduce a new quantity named "\NewTerm{free energy}\index{free energy}" (which is really available in the system) and will be given by the "\NewTerm{Gibbs-Helmholtz relation}\index{Gibbs-Helmholtz relation}":
	
	and giving simply the difference between the internal energy and the heat energy dissipated because of the entropy at a given temperature. The idea being this time to control the volume and temperature as independent variables (see just a little further below when we take the differential), which is obviously an important situation in chemistry!

	We also introduce another new quantity that we name "\NewTerm{free enthalpy}\index{free enthalpy}" or "\NewTerm{Gibbs energy}\index{Gibbs energy}" (which is that really available in the system) and will be given identically by:
	
	which is simply the difference between the enthalpy and the dissipated heat energy because of the entropy at a given temperature.

	So we have for the free energy the differential form:
	
	by injecting in it the first principle and the second principle:
	
	Thus we see that if the transformation is isothermal $\mathrm{d}F$ reduces to the work received by the closed system (since $\mathrm{d}T$ is zero).

	Similarly for the free enthalpy:
	
	by injecting:
	
	Therefore, in the case of an isobaric and isothermal transformation the free energy variation is equal zero.

	So we have four relations:
	
	named "\NewTerm{Gibbs equations}\index{Gibbs equations}".
	
	We notice that all these equations are of the form:
	
	But, let recall that according to the Schwarz theorem (\SeeChapter{see section Differential and Integral Calculus page \pageref{Schwarz theorem}}) if $\mathrm{d}z$ is an exact total differential, then we have:
	
	This gives us immediately the four most common "\NewTerm{Maxwell relations}\index{Maxwell relations}":
	
	Furthermore, by the definition of partial derivatives and the four relations:
	
	We get immediately:
	
	All these relationships are used to calculate the thermodynamic variables not directly measurable from the experimental data.

	To close the subject about the definitions of these thermodynamic variables, let us indicate that the fact of creating (defining):
	
	or in other words, that fact that for every pair $X$, $Y$ of conjugated thermodynamic variables and function $f$ we put:
	
	is named a "\NewTerm{Legendre transformation}\index{Legendre transformation}\label{legendre transformation thermodynamics}" (for the mathematical details of the transformation see the section of Analytical Mechanics). This type of transformation allows to change the set of independent variables to get a set of best suited variables to the problem considered.

	Now let see a relation that will be useful to use in the section of Weather and Marine!

	We know that the specific heat is given by definition at constant pressure by:
	
	But at constant pressure, the heat variation can be written by definition with the enthalpy:
	
	Now remember that the enthalpy is given by:
	
	as $\mathrm{d}S$ is an exact differential, we can write it as a function of the parameters of temperature and pressure only:
	
	Therefore we have:
	
	As by the way $\mathrm{d}H$ is an exact differential, we can also write it in function of the parameters of temperature and pressure only:
	
	We have then the two relations to identify:
	
	It comes therefore:
	
	
	\subsection{Continuity Equation}\label{continuity equation}
	A continuity equation in physics is an equation that describes the transport of some quantity. It is particularly simple and particularly powerful when applied to a conserved quantity, but it can be generalized to apply to any extensive quantity. Since mass, energy, momentum, electric charge and other natural quantities are conserved under their respective appropriate conditions, a variety of physical phenomena may be described using continuity equations.
	
	Continuity equations are a stronger, local form of conservation laws. For example, a weak version of the law of conservation of energy states that energy can neither be created nor destroyed -i i.e., the total amount of energy is fixed. This statement does not immediately rule out the possibility that energy could disappear from a field in Canada while simultaneously appearing in a room in Indonesia. A stronger statement is that energy is locally conserved: Energy can neither be created nor destroyed, nor can it "teleport" from one place to another - it can only move by a continuous flow. A continuity equation is the mathematical way to express this kind of statement. For example, the continuity equation for electric charge states that the amount of electric charge at any point can only change by the amount of electric current flowing into or out of that point.
	
	Continuity equations more generally can include "source" and "sink" terms, which allow them to describe quantities that are often but not always conserved, such as the density of a molecular species which can be created or destroyed by chemical reactions. In an everyday example, there is a continuity equation for the number of people alive; it has a "source term" to account for people being born, and a "sink term" to account for people dying.
Any continuity equation can be expressed in an "integral form" (in terms of a flux integral), which applies to any finite region, or in a "differential form" (in terms of the divergence operator) which applies at a point.
Continuity equations underlie more specific transport equations such as the convection–diffusion equation, Boltzmann transport equation, and Navier–Stokes equations.

	Now let us consider generally an open system, limited by any boundary $\Omega$ (deformable or not) and animated by any movement (moving or stationary) relative to a reference frame considered as fixed.

	This system is shown in the figure below, is capable of transferring energy (or mass, or electric charges, etc.) between itself and the outside. This system may be inertial or not.

	Given an extensive quantity $A$ (like mass or charge). The corresponding quantitative magnitude will be denoted by $a$ (it can express for example the isotropy or anisotropy of the system).
	\begin{figure}[H]
		\centering
		\includegraphics{img/mechanics/continuity_equation.jpg}
		\caption[]{Open system in translation or not in a repository and transferring energy}
	\end{figure}
	In general, the value of $A$ within the system is, at any instant:
	
	$\rho$ being the density of an extensive quantity $A$.

	The spatial variation rate of $A$ is given by the derivative $\mathrm{d}A / \mathrm{d}t$. The causes of variations of $A$ can be related to two different phenomena: the flows and the sources or sinks.

	Counting positively what enters the system, the flow of $A$ through the boundary $\Omega$ is given by the surface integral:

	in which we define:
	\begin{itemize}
		\item $\vec{\Phi}$ as the total surfacic flux vector (or the "current density vector") relatively to $A$

		\item $\mathrm{d}\vec{S}$ as the boundary element, expressed by a vector normal to the surface and directed towards the outside
	\end{itemize}
	
	Let us notice that, at the contrary to the usual meaning in physics, the concept of flow $\Phi$ already contains here the derivation with respect to time. Furthermore, to simplify the text, the expression "surfacic vector flow" is reduced here the term "flow" in all that follows.

	This flow can be split into multiple flows, according to the relation:
	
	where:
	\begin{itemize}
		\item The term $\vec{\Phi}_a$ is a flow by absolute translation, characterizing a flow from a fluid flow. We have then relation:
		
		where $\vec{v}_a$ the absolute velocity of a fluid particle with respect to the fixed reference frame.

		\item The term $\vec{\Phi}$ is an apparent displacement flow, involved only when the boundary $\Omega$ moves. We have the relation:
		
		where $\vec{v}_d$ is apparent speed of movement of a point on the boundary $\Omega$, relatively to the fixed reference frame.

		\item The term $\vec{\Phi}_c$ is the total flow by conduction, characterizing a flow linked to a transfer phenomenon at the microscopic level, without fluid movement (eg heat conduction, electrical conduction, mechanical work).
	\end{itemize}
	Under the principle of composition of velocities (absolute speed is the sum of the relative speed and its apparent speed), we have:
	
	and therefore
	
	The corresponding flow of $A$ is then given by (under the assumption that $a$ is given in unit of our quantity of interest by unit mass):
	
	where $S_f$ denotes the boundary portion through which the mass flow occurs.
	
	If we count positively the effect of a source, the rate of increase of $A$ is given by:
	
	where $\phi$ is the volume flow from a source of $A$.

	Taking into account both flows and sources, we have the spatial rate variation of $A$:
	
	The spatial balance of $A$ is finally expressed by the relation:
	In the particular case of a system in continuous operation (eg in the case of a flowing fluid or a solid in which we can observe a thermal conduction, an electrical conduction, a nuclear reaction, etc.), all local variables are constant at any point of the system. If, furthermore, we choose a boundary $\Omega$ dimensionally stable, it is possible to reason with respect to a reference frame linked to the considered system. We then have, at any fixed point of the system relative to this reference frame:
	
	It results for the whole system:
	
	So in the case of a steady system, with a dimensionally stable boundary $\Omega$, linked to the system, the spatial variation rate  of any extensive scalar quantity is zero.
	
	The spatial variation rate of $A$ starting (for recall) from:
	
	given by:
	
	The elementary variation in the volume $V$ is due to the displacement (in the sense of the deformation!) of the boundary $\Omega$, so that:
	
	Therefore we can write that:
	
	Hence:
	
	Considering the flow of sources and wells, we have for the first term:
	
	The Gauss-Ostrogradsky theorem (\SeeChapter{see section Vector Calculus page \pageref{gauss ostrogradsky theorem}}) will allow us to write the surface integral into a volume integral, and grouping all terms under the same integral sign, we get immediately:
	
	As the limits of integration (boundary $\Omega$) are arbitrary, the expression in brackets is identically zero such that we are led to write:
	
	Let us consider now the special case where the scalar extensive quantity is the mass $M$, then we have:
	
	with $a=1$.
	
	Since the mass is not likely to be transferred by a conduction phenomenon (in a classic case (non-quantum)) we have $\vec{\Phi}_c$ that is zero (in many text books even when there is conduction we consider $\vec{\Phi}_c\ll \vec{\Phi}_a $...). Since the mass is conservative, there is no source or well so that $\phi$ is also zero.

	We have therefore:
	
	Since the mass is not likely to be transferred by a conduction phenomenon (in a classic case (non-quantum)) we have $\vec{\Phi}_c$ that is zero (in many text books even when there is conduction we consider $\vec{\Phi}_c\ll \vec{\Phi}_a $...). Since the mass is conservative, there is no source or well so that $\phi$ is also zero.

	We have therefore:
	
	The relation:
	
	is named "\NewTerm{continuity equation}\index{continuity equation}" or "\NewTerm{(mass) conservation equation}\index{conservation equation}\label{mass conservation equation}".

	The "-" sign is here because we defined the incoming flow as positive. It is possible that in the literature as well as in this e-book, you find a "+" instead of that sign.
	
	There is another much more common form in which we can found the continuity equation. The reader will have noticed that the term $\rho\vec{v}_a$ has units of a massic surface current density  $[\text{kg}\cdot \text{m}^{-2}\cdot \text{s}^{-1}]$ which bring us in analogy with electronics (\SeeChapter{see section Electrokinetics page \pageref{current density}}) to write:
	
	Which leads to the following form of the continuity equation:
	
	
	\pagebreak
	\subsubsection{Heat Equation}\label{heat equation}
	Let us apply this result to the diffusion of heat that as important applications in many fields of study (wave mechanics, Schrödinger equation in quantum physics, thermodynamics, fluid dynamics, Black \& Scholes equation in finance, for resistance study in electrokinetics, electromagnetic wave propagation in a medium, in chemistry for some reactions, in nuclear neutronics, etc.) and there is considerable literature on the various solutions of this differential equation of the second order. In the section of Functional Analysis will prove how to solve this equation with the Fourier transform and the Laplace transform.

	As for the conservation of mass equation, we can write for the heat in the absence of sources:
	
	where  $q$ is the amount of heat per unit volume (do not forget it otherwise we would have taken a capital $Q$!) and $\vec{\Phi}_q$ is the heat flow whose incoming quantity was defined as negative.

	A temperature variation causing a variation in the quantity of heat is defined as a first approximation by the following physical law (this follows from the definition of the specific massic heat...):
	
	where $\rho$ is the density of matter and $c_P$ is the specific heat capacity. Or equivalently (since in Thermodynamics, as we have already mention it, lower case is reported to the mass):
	
	
	So the heat equation is a parabolic partial differential equation that describes the distribution of heat (or variation in temperature) in a given region over time. 
	
	The heat flux being trivially induced by spatial temperature difference, we obtain the "\NewTerm{Fourier law}\index{Fourier law}\label{fourier law}" which expresses the heat flow in proportion to the spatial temperature gradient:
	
	The "$-$" sign is simply due to the fact the heat flow goes from the warmer to the colder and $\kappa$ is the "\NewTerm{heat transmission coefficient}\index{heat transmission coefficient}" expressing the "\NewTerm{thermal conductivity}\index{thermal conductivity}" of the material dependent on its atomic and structural macroscopic properties. In the 21st century where climate change is quite a big issue, the research for good insulating material (low thermal conductivity materials) to keep houses and buildings warm in Winter (in the purpose to decrease energy loss by radiation) is an important area of research.
	
	We find more often in practice this relation in the following form in the one-dimensional case (we will do a small practical example further with this form of writing):
	
	 \begin{figure}[H]
		\centering
		\includegraphics[scale=1]{img/mechanics/thermal_conductivity.jpg}
		\caption[]{Heat conduction occurs through any material, represented here by a rectangular bar (source: OpenStax)}
	\end{figure}
	By inserting the two prior previous relations in the heat conservation equation, we have:
	
	after rearranging a little bit:
	
	More aesthetically and more generally (but still in one dimension), we find it in the condensed form of the "\NewTerm{heat diffusion equation}\index{heat diffusion equation}" or more simply named "\NewTerm{heat equation}\index{heat equation}":
	
	where the coefficient of proportionality is named in the context of heat study: "\NewTerm{thermal diffusivity}\index{thermal diffusivity}".
	
	
	It is possible to prove its microscopic origin as we will prove it further below.

	It must always pay attention to the units of $\kappa$ following that we work the with the massic specific heat capacity $c_P$  or the heat capacity $C_P$ in the denominator!

	So under completely explicit form, we have in one dimension:
	
	\begin{tcolorbox}[title=Remarks,colframe=black,arc=10pt]
	\textbf{R1.} We will see again this equation in the section of Theoretical Computing to introduce the reader to the concept of solving differential equations using the finite element method with a simple example of resolution using Microsoft Excel or in the MATLAB companion book.\\

	\textbf{R2.} It is by studying this equation that Fourier introduced the series and transformation that bear his name, and which became so important in the study of phenomena of propagation / diffusion.\\

	\textbf{R3.} Often practitioners write:
	
	and thus we can introduce the concept of "\NewTerm{thermal resistance $R_\text{th.}$}\index{thermal resistance}" in the denominator. In short nothing spectacular but of frequent use.
	\end{tcolorbox}
	Let us insist that all relations of the type:
	
	are named "\NewTerm{diffusion equations}\index{diffusion equations}" of physical parameter $D$ and is a parabolic partial differential equation (\SeeChapter{see section Differential and Integral Calculus page \pageref{parabolic partial differential equation}}). We will immediately see how to solve this differential partial equation using and this will lead us to understand why Fourier introduced the famous transformations that bear today his name. But let us remind the reader that we find this partial differential equation in multiple contexts.	
	\begin{figure}[H]
		\centering
		\includegraphics[scale=0.4]{img/mechanics/thermal_conductivity_materials.jpg}
		\caption[Thermal conductivity values for various materials]{Thermal conductivity values for various materials (source: Wikipedia)}
	\end{figure}
	
	\begin{tcolorbox}[colframe=black,colback=white,sharp corners]
	\textbf{{\Large \ding{45}}Example:}\\\\
	E1. A parallelepiped made of steel (having a thermal conductivity of $17\;[\text{W}\cdot \text{m}^{-1}\cdot \text{K}^{-1}$) has one side thereof maintained at $331.15\; [\text{K}]$ and the other at $297.15\; [\text{K}]$, both faces being separated by $1\; [\text{cm}]$ and having a surface $1\; [\text{m}^2]$:
	Assuming a situation at equilibrium the heat transfer is given by:
	
	
	E2. For a cylindrical tube carrying a heated fluid (very important case in practice) we want to determine the amount of heat exchanged per unit of time with the outside environment. Knowing that the surface in contact with the fluid (with inner radius $r_i$) is not the same that the one in contact with the outside environment (external radius $r_e$) we can not simply calculate ratio of the temperature differences with the distance between the two surfaces. Therefore we have to proceed as following:
	
	\end{tcolorbox}
	Let us solve the general form of the diffusion equation without sources:
	
	To solve this differential equation of the second order, we will use the method of separation of variables (\SeeChapter{see section Differential and Integral Calculus page \pageref{separation vaiables method}}) rather than attacking directly by Fourier transforms (which requires as prior the proof of the dominated convergence theorem which makes me horror).

	We assume therefore using the separation of variables method that:
	
	And where $T(t)$ by the principle of the second law of thermodynamics must decrease as $t$ increases.

	We get immediately:
	
	hence the diffusion equation:
	
	what rearranged and condensed is also written:
	
	Which can be written by assuming that each of the expressions on the left and right of equality are assimilated to functions:
	
	so that equality is true for all $t$ and $x$, the functions $G$ and $F$ functions must be constant. So we have the right to write:
	
	The fact to write the constant negative and squared is a simple anticipation of the result historically already known... But to understand why the constant is necessarily negative, just think that that $T'$ is necessarily negative (the temperature of a isolated system will not rise alone but naturally decrease based on the principle of entropy) and as $D$ and $T$ are positive then verbatim ...

	What gives us the system of two independent differential equations of the second order:
	
	That some text books write as following:
	
	We solve the second differential equation first:
	
	Therefore:
	
	And for the first differential equation:
	
	We then have the characteristic polynomial:
	
	Thus the roots:
	
	So as the discriminant is negative (\SeeChapter{see section Differential and Integral Calculus page \pageref{discriminant differential equation}}) we have:
	
	Therefore:
	
	If we move the exponential of the constants in factors such as:
	
	this simplifies the expression:
	
	whose coefficients will be determined case by case by the initial conditions of which the most famous school cases and subject to an extensive literature are:
	\begin{itemize}
		\item Dirichlet bouandaries conditions:
		

		\item Neumann bouandaries conditions:
		
	\end{itemize}
	Since for each possible value of $\lambda$ we get a solution, it appears that doing the sum of all these solutions, we get the general solution. We therefore finite Fourier series:
	
	because of the presence of the square $\lambda$ in the temporal exponential in factor (right) of the parenthesis, if we sum over all the $\lambda$ from $-\infty,+\infty$ then the sum of all terms of the parenthesis cancel for each pair of positive and negative $\lambda$ of the same value. Then we must summon only on the semi-positive infinite line (either left or right it does not matter) if we do not want the result to be zero. If we choose arbitrarily the positive infinite line, then we get:
	
	Nothing now prevents us from taking out a factor $\lambda$ of the constant ${C'}_1(\lambda)$, ${C'}_2(\lambda)$, the only effect it will have is to change their normalisation given by the initial conditions .. Then we have:
	
	As $\lambda$ is a continuous real parameter (and also positive) in by making it tend to zero the sum can be changed into an integral (doing this in the physicist way...) and then we get:
	
	And making a change of variable for the second integral, we have:
	
	By grouping the constants, we have:	
	
	with:
	
	We have now something interesting! At the time $t=0$, the previous integral is written:
	
	and changing the notation for the parameter $\lambda$ and the notation for $x$, we have:
	
	And so what???? are you perhaps thinking just now...? Well as we saw in our study of the Fourier transforms (\SeeChapter{see section Sequences and Series page \pageref{fourier transform}}) the latter relation is an inverse Fourier transform which is an infinite sum of real trigonometric functions! This means that any function describing the temperature distribution at time $t$ zero for an infinitely long bar and whose ends at the same time tend to zero (thus implicitly it is a function of infinite period) such that:
	
	Satisfies the equation of Heat as an infinite trigonometric series. Let us return to our initial notation:
	
	As a recall of the section Sequences and Series (with the notation adapted here) we have by applying the inverse Fourier transform\label{fourier transform heat equation}
	
	Therefore $F(\lambda)$ is the Fourier transform of the desired function such that:
	
	Which then gives by injecting the Fourier transform with a change of variable that imposes itself:
	
	Always keeping in mind that if we put $t$ as being null below, we then fall back well on:
	
	Since in the last equality the term $e^{-\lambda^2 Dt+\mathrm{i}\lambda x}$ is independent of the variable $v$, we can put it in the second integral such that:
	
	We then recognize a double integral of which we can change the order of integration by applying the Fubini theorem (\SeeChapter{see section of Differential and Integral Calculus page \pageref{fubini theorem}}):
	
	It then comes in the case that interests us by making a correspondence term by term:
	
	Therefore:
	
	Thus any function $f(v)$ injected in the above relation will satisfy the initial conditions we have imposed to ourselves:
	
	For information, many physicists name the following expression visible in the previous boxed relation:
	
	The "\NewTerm{heat kernel}\index{heat kernel}". Thus the final result is in reality only a convolution of $f (v)$ with this "kernel".
	
	In practice this integral is not analytically calculable since it is a convolution with a Gaussian (\SeeChapter{see section Statistics page \pageref{gauss distribution}}). This is the reasons for which the heat equation is almost always solved at the initial conditions mentioned above using numerical methods (\SeeChapter{see section on Numerical Methods page \pageref{one space dimension fdm}}).

	But let us solve some simplistic but real cases:
	\begin{tcolorbox}[colframe=black,colback=white,sharp corners]
	\textbf{{\Large \ding{45}}Example:}\\\\
	E1. When two ends of a system of size $L$ are maintained at two different temperatures $T_1$ and $T_2$, the solution of the heat equation is stationary (independent of time). We then have:
	
	The solution to this differential equation is very simple and does not require the use of the previous result (it is elementary integral calculus with known initial conditions):
	
	This is a situation we find in everyday life ...\\
	
	E2. Let us consider the case where we put in contact two bars of infinite length (since with the Fourier transform it is always necessary that one of the variables range the whole set of real number) and of opposite temperature in sign (on the Celsius scale therefore...) such as:
	
	We then have:
	
	And let us make the following change of variable:
	
	Hence:
	
	Then we have:	
	\end{tcolorbox}
	
	\begin{tcolorbox}[colframe=black,colback=white,sharp corners]
	
	Since the integrated function is an even function (\SeeChapter{see section Functional Analysis page \pageref{even function}}) we have:
	
	Therefore:
	
	What it is customary to write:
	
	Where "$\mathrm{erf}$" is named the "\NewTerm{Gauss error function}\index{Gauss error function}\index{error function}\index{Gauss error function}\index{erf}\label{error function}" and can not be analytically computed accurately (we must go through limited series developments). It can however be found in spreadsheets like Microsoft Excel under the name \texttt{ERF( )}.	
	\end{tcolorbox}
	
	\pagebreak
	\paragraph{Fick's laws of diffusion}\mbox{}\\\\
	Equations based on Fick's law have been commonly used to model transport processes in foods, neurons, biopolymers, pharmaceuticals, porous soils, population dynamics, nuclear materials, semiconductor doping process, etc. Theory of all voltammetric methods is based on solutions of Fick's equation. A large amount of experimental research in polymer science and food science has shown that a more general approach is required to describe transport of components in materials undergoing glass transition. In the vicinity of glass transition the flow behavior becomes "non-Fickian".

	When two miscible liquids are brought into contact, and diffusion takes place, the macroscopic (or average) concentration evolves following Fick's law. On a mesoscopic scale, that is, between the macroscopic scale described by Fick's law and molecular scale, where molecular random walks take place, fluctuations cannot be neglected. Such situations can be successfully modeled with Landau-Lifshitz fluctuating hydrodynamics. In this theoretical framework, diffusion is due to fluctuations whose dimensions range from the molecular scale to the macroscopic scale.

	Integrated circuit fabrication technologies, model processes like CVD, thermal oxidation, wet oxidation, doping, etc. use diffusion equations obtained from Fick's law.
	
	We have just seen the proof of the heat propagation equation proposed by Fourier in 1822 obtained from the equation of continuity. We had obtained (it is highly recommended to the reader to refer to it again only to read a second time the remarks relating to the proof):
	
	Based on the same hypotheses as Fourier,  Adolf Fick proposed in 1855 that a flow of particles could diffuse through a material according to a similar law, the "\NewTerm{Fick's second law}\index{Fick's second law}" or "\NewTerm{Fick's law of diffusion}\index{Fick's law of diffusion}", of the form:
	
	Where the constant of proportionality $D_p$ is the "\NewTerm{diffusion coefficient of matter}\index{diffusion coefficient of matter}" and $\rho$ the density of particles per unit volume (and not the mass density!).
	\begin{tcolorbox}[title=Remark,colframe=black,arc=10pt]
	In practice, diffusion plays an essential role in the manufacture of ceramics, semiconductors (doping), solar cells and in the solidification of metals (carbon and heat treatment). Because since two heated materials are taken into contact, their atoms diffuse into each other (also a little bit when not heated).\\

	It must be understood that everything diffuses in everything as we have already mention it in our study of Tribology in the section of Classical Mechanics! So think about pesticides on fruits and vegetables, pollution in ground water, PET in drinks, etc.
	\end{tcolorbox}
	Therefore, the relation of the heat surfacic flow of that we have used above to obtain the Fourier law and which was:
	
	can then also be written (we will prove it) in the case of the mass in the form of a flow of particles named the "\NewTerm{first Fick's law}\index{first Fick's law}":
	
	where $D$ is the transport coefficient of the material (to be determined ...). We find in the literature this last relation frequently in unidimensional form as following:
	
	Either in discrete form:
	
	If the variation of the distance is put in correspondence with a length $L$ traveled by  the diffusion, then we have the notation even more simplified:
	
	To fall back on the famous relation known by small classes of chemistry it is necessary to know that the chemist is accustomed to write the density by the letter $C$ to indicate that it is a \textbf{C}oncentration. Therefore, we have:
	
	But this is not the only change of notation made by chemists. Indeed, it must be remembered that the surfacic flow of particles (mass) is defined by the quantity of mass passing through a constant surface per unit time. We then have:
	
	Since the origin of time is often taken as zero and the initial mass is zero then we get finally after a small rearrangement the classical relation known by chemists and biologists (to a given sign since it is only a matter of convention) :	
	
	
	\pagebreak
	\begin{tcolorbox}[colframe=black,colback=white,sharp corners]
	\textbf{{\Large \ding{45}}Example:}\\\\
	Plants absorb water for photosynthesis. We know from experimental measurements that the diffusion coefficient of water in the air is $D\cong 2.4\cdot 10^{-5}\;[\text{m}^2\cdot \text{s}^{-1}]$. The pores of absorption of plants have a section of the order of $S=8.0\cdot 10^{-11}\;[\text{m}^2]$. The diffusion distance is of the order of $L\cong 2.5\cdot 10^{-5}\;[\text{m}]$. The water vapor density inside the plant is of the order of $C_2\cong 0.022\;[\text{kg}\cdot \text{m}^{-3}]$ and outside the plan of $C_1\cong 0.011\;[\text{kg}\cdot \text{m}^{-3}]$. \\
	
	Consequently, the mass of water absorbed in one hour is approximately:
	
	This result must be multiplied by the millions of pores that a plant has.
	\end{tcolorbox}
	\begin{tcolorbox}[title=Remark,colframe=black,arc=10pt]
	In the facts, Fick proved at first the first law and proceeding in exactly the same way to the heat equation he obtained the second law that bears his name (that is to say in the opposite order presented above).
	\end{tcolorbox}
	We can also estimate the flow of thermal energy transported by these same particles along the $x$-axis. Indeed, in each slice of fluid, $n$ particles carry each an energy $E$ corresponding to a given amount of heat $Q$ (according to Joule's law). We therefore have a surfacic flux of energy $\Phi_q$ whose first component is given by the same type of balance as the previous developments:
	
	We find in it immediately the definition of the calorific capacity (if we divide by mass we would have the massic calorific capacity according to what we saw earlier above). Thus, in the one-dimensional case:
	
	There is therefore a simple ratio of proportionality between $\kappa$ and $C$.
	\begin{table}[H]
		\begin{center}
			\definecolor{gris}{gray}{0.85}
				\begin{tabular}{|c|c|c|}
					\hline
					\multicolumn{1}{c}{\cellcolor{black!30}\textbf{Fourier's law}} & 
	  \multicolumn{1}{c}{\cellcolor{black!30}\textbf{Fick's law}} & 
	  \multicolumn{1}{c}{\cellcolor{black!30}\textbf{Ohm's law}}  \\ \hline
					Thermodynamics & Thermodynamics & Electrokinetics \\ \hline
					$\vec{j}_q=-\kappa \vec{\nabla}T$ & $\vec{j}=-D\vec{\nabla}\rho$ & $\vec{j}=\sigma\vec{E}=-\sigma\vec{\nabla}U$ \\ \hline
					Thermic current density & Particle current density & Electric current density \\ \hline
					$T$: Temperature & $\rho$: concentration & $U$: electric potential \\ \hline
					$\kappa$: Thermic conductivity & $D$: coefficient of diffusion & $\sigma$: electric conductivity \\ \hline
					\parbox{3cm}{\centering Thermic flux:\\ $\Phi_q=\iint_S \vec{j}_q\circ\mathrm{d}\vec{S}$} & \parbox{3cm}{\centering Particle flux:\\ $\Phi_p=\iint_S \vec{j}_p\circ\mathrm{d}\vec{S}$} & \parbox{3cm}{\centering Electric flux:\\ $I=\iint_S \vec{j}\circ\mathrm{d}\vec{S}$} \\ \hline
			\end{tabular}
		\end{center}
		\caption{Similarities of different diffusion laws in physics}
	\end{table}
	
	\subsection{Thermal radiation}
	Thermal radiation is electromagnetic radiation generated by the thermal motion of charged particles in matter. All matter with a temperature greater than absolute zero emits thermal radiation. When the temperature of a body is greater than absolute zero, inter-atomic collisions cause the kinetic energy of the atoms or molecules to change. This results in charge-acceleration and/or dipole oscillation which produces electromagnetic radiation, and the wide spectrum of radiation reflects the wide spectrum of energies and accelerations that occur even at a single temperature.
	
	Examples of thermal radiation include the visible light and infrared light emitted by an incandescent light bulb, the infrared radiation emitted by animals is detectable with an infrared camera, and the cosmic microwave background radiation. Thermal radiation is different from thermal convection and thermal conduction person near a raging bonfire feels radiant heating from the fire, even if the surrounding air is very cold.
	
	Sunlight is part of thermal radiation generated by the hot plasma of the Sun. The Earth also emits thermal radiation, but at a much lower intensity and different spectral distribution (infrared rather than visible) because it is cooler. The Earth's absorption of solar radiation, followed by its outgoing thermal radiation are the two most important processes that determine the temperature and climate of the Earth.
	
	If a radiation-emitting object meets the physical characteristics of a black body in thermodynamic equilibrium, the radiation is named "blackbody radiation". Planck's law describes the spectrum of blackbody radiation, which depends only on the object's temperature. Wien's displacement law determines the most likely frequency of the emitted radiation, and the Stefan–Boltzmann law gives the radiant intensity.
	
	Thermal radiation is one of the fundamental mechanisms of heat transfer.
	
	\subsubsection{Black Body radiation}\label{black body}
	Black-body radiation is the type of electromagnetic radiation within or surrounding a body in thermodynamic equilibrium with its environment, or emitted by a black body (an opaque and non-reflective body), assumed for the sake of calculations and theory to be held at constant, uniform temperature. The radiation has a specific spectrum and intensity that depends only on the temperature of the body.
	
	The thermal radiation spontaneously emitted by many ordinary objects can be approximated as blackbody radiation. A perfectly insulated enclosure that is in thermal equilibrium internally contains black-body radiation and will emit it through a hole made in its wall, provided the hole is small enough to have negligible effect upon the equilibrium.
	
	A black-body at room temperature appears black, as most of the energy it radiates is infra-red and cannot be perceived by the human eye. Because the human eye cannot perceive color at very low light intensities, a black body, viewed in the dark at the lowest just faintly visible temperature, subjectively appears grey (but only because the human eye is sensitive only to black and white at very low intensities - in reality, the frequency of the light in the visible range would still be red, although the intensity would be too low to discern as red), even though its objective physical spectrum peaks in the infrared range.[5] When it becomes a little hotter, it appears dull red. As its temperature increases further it eventually becomes blue-white.
	
	Although planets and stars are neither in thermal equilibrium with their surroundings nor perfect black bodies, black-body radiation is used as a first approximation for the energy they emit.[6] Black holes are near-perfect black bodies, in the sense that they absorb all the radiation that falls on them. It has been proposed that they emit black-body radiation (named "Hawking radiation"), with a temperature that depends on the mass of the black hole.
	
	The study of the black body is the basis of the famous theory of Wave Quantum Physics (see section of the same name page \pageref{corpuscular quantum physics}), one of the pillars of actual modern physics. Indeed, some experimental results could not be explained without the introduction of the famous Planck constant, the use of quantification of energy (quanta) and the acceptance of the atomic model and Boltzmann statistical theory (implicitly the second principle of thermodynamics of which Max Planck was a specialist).

	Before starting we will make an exceptionally (for this book) an historical hook which that we suppose to be very useful to understand why the study of thermal radiation is so important in physics.

	The fact that all heated objects emit a light of the same color at the same temperature is a curiosity that had been known for centuries in some fields, long before Gustav Krichhoff began his scientific and theoretical investigations on the nature of this strange correlation. To simplify his analysis, he elaborated the concept of a perfectly absorbing and perfectly emissive object. Thus, as a perfect transmitter, it would be anything but black if its temperature were high enough so that it could radiate in wavelengths of the visible part of the spectrum. Therefore, with respect to experience feedback, the radiation of this object would have an intensity and a spectral extent independent of the type of material of which it is composed. The objective was then to measure the spectral distribution of the radiated energy for each wave length and for each temperature plateau and to derive an equation and therefore dependent of two variables (wavelength and temperature) to reproduce this distribution and therefore this for any material placed under the conditions of a perfectly emessive and receptive body.
	
	\textbf{Definition (\#\mydef):} 
	A "\NewTerm{black body}\index{black body}" (or "\NewTerm{integral receiver}\index{integral receiver}") is defined as a body having an "\NewTerm{energy absorption coefficient $\alpha$}\index{energy absorption coefficient}" (\SeeChapter{see section Optical Geometry page \pageref{kirchhoff law of radiation}}) and an "\NewTerm{emissivity coefficient $\varepsilon$}\index{emissivity coefficient}" equal to the unity.
	
	\textbf{Definition (\#\mydef):}  The emissive power of an arbitrary opaque body of fixed size and shape at a definite temperature can be described by a dimensionless ratio, named "\NewTerm{emissivity}\index{emissivity}", the ratio of the emissive power of the body to the emissive power of a black body of the same size and shape at the same fixed temperature\footnote{With this definition, Kirchhoff's law states that for an arbitrary body emitting and absorbing thermal radiation in thermodynamic equilibrium, the emissivity is equal to the absorptivity.}.
	
	The first principle of thermodynamics establishes an equivalence between work and heat as modes of energy transfer between a system and its environment (and in fact the balance in terms of internal energy). We are interested here in the heat, which we can be define as "the energy one body communicates to another because of their temperature difference".

	Heat is, for recall, communicated from one place to another in three different ways, as we have already mentioned earlier above:
	\begin{enumerate}
		\item By conduction: it is a transfer of heat in a set of material points in contact which takes place without macroscopic movements, under the influence of a temperature gradient. Conduction is therefore the result of molecular collisions. We observe it mainly in solids: in metals, it involves the free electrons which make them good conductors of heat. On the other hand, in insulators, conduction is badly done. Hence the strong correspondence between the thermal and electrical properties of solids.

		\item By convection: convection involves the transport of heat by a part of a fluid that mixes with another fluid or material. It has its origin in a macroscopic transport of matter and therefore does not concern solids most of time.
	
		\item By radiation: conduction and convection assume the presence of matter. Radiation, on the other hand, allows a transfer of energy that can be made through the vacuum. This is electromagnetic radiation. It should be noted that radiation is not a mode of heat transfer but energy, which can be transformed into heat in contact with a body.
	\end{enumerate}
	The thermal radiation emitted by a body at a given temperature results from a conversion of the internal energy of the body into radiation (excited quantum state level back to ground level). Conversely, absorption is the transformation of the incident energy into internal energy.

	When a surface is subjected to absorbed radiation, we perform the energy balance according to the Kirchhoff law seen in our study of photometry (\SeeChapter{see section Optical Geometry page \pageref{kirchhoff law of radiation}}):
	
	where let us recall that $\alpha$ is the fraction of the absorbed radiation, $\rho$ is the reflected part (scattered) and $\tau$ the transmitted part (which crosses the surface). This balance schema results from the principle of energy conservation.

	We will now examine the mechanisms of absorption and emission and establish a link between heat and radiant energy before we to focus directly to the black body:
	
	\paragraph{Stefan–Boltzmann law}\label{stefan boltzmann law}\mbox{}\\\\
	In our study of photometry, we defined the concept of emittance (energy irradiated by a non-punctual body per unit area) for the entire spectrum.

	What we had omitted to specify, however, is that in order for a body to radiate (apart from the fact that it can itself be illuminated by another body) it must be heated (that we provide an excitation energy to the constituents of the body in question - that is to say to the electrons).

	So we should be able to establish a relation between the temperature of a body and its emittance.

	In 1879, the physicist Jožef Stefan was able to establish experimentally that the total emittance of the black body (or "\NewTerm{energetic excitance}\index{energetic excitance}" of the black body) at a temperature $T$ increased in proportion to the fourth power of the temperature such that:
	
	Where $M (T)$ is the integration on all wavelengths (or frequencies ... whatever) of $M(\lambda,T)$:
	
	with $M(\lambda,T)$ given by the Planck law which we prove further below.

	Let us also recall that (this will be shown during our proof of Planck's law):
	
	Is the "\NewTerm{Stefan constant}\index{Stefan constant}" (not to be confused with Stefan-Boltzmann's constant !!!).
	
	In 1884,  Ludwig Boltzmann indirectly proved Stefan's law on the basis of the study of the black body in thermal equilibrium (where we consider that the border of the black body wall define the terminations of electromagnetic waves) from the theory of electromagnetism and using a powerful thermodynamic reasoning.
	\begin{dem}
	In a first step, Boltzmann determined the radiation pressure in such an enclosure (or in such a body).

	Here are the developments that led him to determine the radiation pressure $P (T)$ at the thermodynamic equilibrium temperature $T$ for the corresponding internal energy density $\rho(T)$:

	Let us recall the expression of the "Einstein relation" that we have proved during our study of Special Relativity:
	
	Let us consider now a chamber of volume $V$ whose walls are reflective for the photons (case of the black body):
	\begin{figure}[H]
		\centering
		\includegraphics[scale=0.6]{img/mechanics/conceptual_black_body.jpg}
		\caption{Conceptual black body}
	\end{figure}	
	We study the variation of the momentum before and after the collision on an infinitely small surface $\mathrm{d}s$ (which makes it possible to consider the trajectories before and after the impact as rectilinear and symmetrical with respect to the oriented axis O$x$ perpendicular to the surface of the black body coinciding with the surface $\mathrm{d}s$).
	\begin{figure}[H]
		\centering
		\includegraphics[scale=1]{img/mechanics/blackbody_radiator.jpg}
		\caption{Black body laboratory radiator}
	\end{figure}
	Thus, we have before collision for the linear momentum:
	
	and after collision:
	
	If the collision is elastic (which is comforting relative to the photon ...):
	
	Then we have:
	
	The variation of the momentum is then:
	
	As:
	
	then we have:
	
	By considering only the norm of the expression and that it is a single photon:
	
	\begin{tcolorbox}[title=Remark,colframe=black,arc=10pt]
	We assume that after its rebound, the photon conserves its frequency which leads us to suppose that the black body has stationary waves at thermodynamic equilibrium (\SeeChapter{see section Wave Mechanics page \pageref{stationary wave}}).
	\end{tcolorbox}
	Until now, we have reasoned on a single photon, but the enclosure actually contains a gas of photons. The internal volume energy $u$ of the radiation contains a volumic density $n$ of photons of identical frequency $\nu$. Consequently, the quantity $n$ of photons per unit volume in the enclosure is:
	
	\begin{tcolorbox}[title=Remark,colframe=black,arc=10pt]
	We specify the units, because we noticed that the following development sometimes presented some problems of understanding to some readers if the units were not given explicitly.
	\end{tcolorbox}
	We consider that during a time interval $\mathrm{d}t$ the number of photons potentially striking the surface $\mathrm{d}s$ at an angle of incidence $\alpha$ is contained in a cylinder of generator $c\mathrm{d}t$ whose axis is necessarily inclined by an angle $\alpha$ and having as base area $\mathrm{d}s$. The volume of this cylinder is then by the projection of the base surface:
	The number of photons $\mathrm{d}n_c$ that potentially strike the wall $\mathrm{d}s$ per unit time is:
	
	In this last expression, we assumed that all the photons of $\mathrm{d}V$ had a linear momentum  in the direction subtended by $\alpha$. In fact, the photons actually arriving at $\mathrm{d}s$ are contained in a solid angle\label{solid angle black body} $\mathrm{d}\Omega$ between two cones of half-angle at the apex of $\alpha$ and $\alpha+\mathrm{d}\alpha$ (for reasons of the geometry of the blackbody's experiment which was, except error, spherical and by the way this spherical symmetry facilitates the calculations...).

	The relation between $\mathrm{d}\Omega$ and $\alpha$ is as we have proved in the section of Trigonometry:
	
	Knowing that in the whole volume (for recall), the solid angle is:
	
	the number $\mathrm{d}n$ included in the elementary solid angle $\mathrm{d}\Omega$ which reaches the surface $\mathrm{d}s$ at an angle of incidence between $\alpha$ and $\mathrm{d}\alpha$ is then:
	
	Using now the definition of the pressure $P$:
 	
	by substituting what is covenant:
	
	Which gives after simplification:
	
	The total pressure of radiation being given in this case by:
	
	Which is equivalent to write (a equilibrium temperature for a given temperature):
	
	Relation that is very important and will be useful to us in the section Cosmology during our study of some Friedman Universe models. This relation is also important for rocket science using propulsion by radiation!

	The total energy is the energy density multiplied by the considered volume:
	
	Let us suppose that the volume could vary. The work done by the radiation pressure during a dilatation $\mathrm{d}V$ is obviously given by (dimensional analysis):
	
	The variation of internal energy of the system following the first principle of thermodynamics is:
	
	But as $E=\rho(T)V$, we have:
	
	Hence:
	
	and following the second principle of thermodynamics (do not confuse here the notation of the entropy $S$ with that of a surface $S$...) we have for an ideal reversible system:
	
	We have then:
	
	Rearranged this can be written:
	
	As $\mathrm{d}S$ is a total exact differential, we have proved in the section of Differential and Integral Calculus that tan $S$ satisfy the Schwarz' theorem:
	
	We have in this case:
	
	Which bring us to write:
	
	By calculating the derivative of the right member:
	
	By simplifying:
	
	after rearranging:
	
	Thus:
	
	which become the equation:
	
	Which gives after a trivial integration:
	
	Finally:
	
	which is commonly written:
	
	with:
	
	Being the Stefan-Boltzmann constant with the value and the units as they were given at the time of Stefan and Boltzmann and initially experimentally (very far from the present known value known with different units than those in use today). We will prove however further below the theoretical value with the common units used in the 20th century.
	
	We see above the correspondence which exists between the relation which we had posed at the beginning and that which we have just obtained:
	
	Since we have not yet proved Planck's law at this point, we can make a daring reasoning but which we will later justify with a proof as support.
	\begin{tcolorbox}[title=Remark,colframe=black,arc=10pt]
	The last two relationships give us a fundamental information that all bodies that are not at zero kelvin (absolute zero) radiate!
	\end{tcolorbox}
	$M (T)$ and $\rho(T)$ are differentiated at the level of the units by the dimensions of a velocity. Now, intuitively and roughly (...), the speed which can immediately appear to us as trivial in this case of study is the speed of light $c$. Thus, we notice that:
	
	Which give us:
	
	Curious isn't it...? But we will prove it later because our philosophy in this book is to never (or as little as possible) leave room for intuition.
	\begin{flushright}
		$\square$  Q.E.D.
	\end{flushright}
	\end{dem}
	\begin{tcolorbox}[title=Remark,colframe=black,arc=10pt]
	When we will study Planck's law, the $\rho(T)$ will be denoted $R (T)$ to not to confuse then energy density with the radiance (because the notation can unfortunately be confusing).
	\end{tcolorbox}
	Let us consider now an isolated chamber or cavity (like a furnace) in thermal equilibrium at a certain temperature $T$. This cavity will surely be filled with electromagnetic radiations of different wavelengths. Let us suppose that there exists a distribution function $M (T)$ which depends only on the temperature.

	Logically, the total quantity of electromagnetic energy at all wavelengths absorbed by the walls of the cavity must be equal to that emitted by the walls otherwise the body forming the cavity would see its temperature change. Gustav Robert Kirchhoff reasoned that if the body forming the cavity is made of different materials (thus behaving in different ways with temperature), then the equilibrium between emitted radiation and absorbed radiation must apply for each wavelength (or domain of wave length).

	We thus see that $M(T)$ is a universal function, the same for all the cavities, regardless of their composition, their geometry or the color of their walls. Kirchhoff did not give this function, but he pointed out that a perfectly absorbing body, that is to say a body for which $\alpha=1$ will appear, how to say... : black.

	It comes then that the radiation stored in equilibrium in an isolated cavity in thermodynamic equilibrium is in all aspects the same as that emitted by a perfectly black body at the same temperature.

	Obviously, if the cavity is closed, we can not measure the current of energy that escapes. But let us create a very small hole in this cavity (small enough not to disturb the equilibrium of the electromagnetic radiation inside), then the electromagnetic energy escaping from this small hole is the same as that emitted by a perfectly body black.

	However, no object is actually a black body. Coal black has an absorption coefficient very close to $1$ but only for certain frequencies (including, of course, the visible frequencies). Its absorption coefficient is much smaller in the far infrared. However most objects approach the behavior of a black body in certain frequency ranges. The human body, for example, is almost a black body in the infrared (hence the military night glasses ...). To treat the various bodies, named "\NewTerm{gray body}\index{gray body}", we introduce a factor named "\NewTerm{total emissivity}\index{total emissivity $\varepsilon$}" that connects the emittance emitted by the body to that emitted by a perfect black body for which $\varepsilon=1$ (for steel the emittance is approximately equal to $0.06$). Then we have:
	
	Practitioners of thermodynamics often prefer the following writing (which is actually more coherent in the context of pure thermodynamics):
	
	It is a very sympathetic relation because in all cases by putting $\varepsilon=1$ the curious can know for a given surface what it is the energy (or heat) at the worst (or at best following the point of view ...) emitted by a body of a certain surface at a certain temperature.
	\begin{tcolorbox}[title=Remark,colframe=black,arc=10pt]
	The Stefan-Boltzmann relation gives us the power emitted by a body per unit of surface by expressing it proportionally to the fourth power of the temperature. This exponent gives us the reason why it becomes more and more difficult to increase the temperature of a body by heating it, the latter losing more and more rapidly the energy that we provide for its heating.
	\end{tcolorbox}
	
	In the figure below we have depicted the situation in which the heat bath of the previous figure is produced by an ideal gas of indefinite extent for which there would be no fringe effects:
	\begin{figure}[H]
		\centering
		\includegraphics[scale=0.9]{img/mechanics/conceptual_black_body_radiation_wall.jpg}
		\caption[]{Conceptual black body with outside gas bath}
	\end{figure}
	The temperature of the gas is maintained at the required value of the cavity wall. In an equilibrium situation characterized by the conservation of energy, total energy will be partitioned equally among all constituents of a gas. This includes not only particulate components of the gas, but also the thermalized photons of electromagnetic energy.

	In this figure, as in the prior previous figure, all photons are seen to have originated as emissions or reflections off the wall of the cavity, and are assumed to remain unchanged during transmission unless and until they interact with the cavity wall in a subsequent scattering event. However, for a "bath" comprised of the stationary state ideal gas shown in the figure above, the situation is similar to taking away the cavity wall altogether and allowing the gas to move freely throughout what was formerly the cavity as is now shown in the figure further below.
	
	Photons would then scatter off of particles in the gas rather than the wall of the cavity, but to similar effect.  However, in order that it indeed be to similar effect, the gas must be dense enough or its extent great enough that the same density of photons is realized within the conceptual cavity walls as shown.
	
	\pagebreak
	\paragraph{Planck's law}\mbox{}\\\\
	To establish Planck's law, we are not going to make the original development of Max Planck even if interesting of the intuitively point of view it was only a succession of tinkering (admitted by Max Planck himself). We will introduce this law with a modern and as simple as possible approach in the point of view of the mathematical formalism as done by Albert Einstein itself!

	For this we now consider the black body as an isolated system at thermal equilibrium, in which the radiation is in the stationary state and reflected totally by the walls. The photons can therefore be considered as particles not interacting with one another in a potential well with rectilinear walls.

	Thus, identically to what we have seen in the section of Wave Quantum Physics, the solution of the problem is that of a potential well with rectilinear walls for which we have obtained for wave function:
	
	function to which the boundary conditions should be applied.

	The conditions we imposed during our study of this case in the section of Wave Quantum Physics were too restrictive (this is why they are named "strict boundary conditions"). Indeed, atoms of the wall absorb and emit radiation irrespective of how the radiation incident. But equilibrium imposes at least that boundary conditions are periodic by the very definition of equilibrium. This is why we impose what we name the "\NewTerm{periodic boundary conditions}\index{periodic boundary conditions}":

	\begin{itemize}
		\item For $x=0$ and $x=L_x$, we have: $\Psi(0)=\Psi(L_x)=c^{te}$

		\item The wave function $\Psi$ must have an integer number of half wavelengths over the length $L_x$

		\item In the black body, $U=0$ therefore $E_p=0$

		\item If at the ends ($x=0$ and $x=L_x$) we have $\Psi=c^{te}$ the argument of the sine has the same value $n2\pi$ (to a given integer multiplicative factor) in $0$ and in $L_x$.
	\end{itemize}
	So we must have:
	
	and as $E_p=0$, after some elementary algebraic simplifications, we get:
	
	Where $E_{nx}$ is the total energy of the quantum level $n$ according to $x$ (\SeeChapter{see section Wave Quantum Physics page \pageref{corpuscular quantum physics}}).

	The total energy of the particle thus presents a discrete sequence of values, the only ones permitted. The value of $L_x$ is determined using the Bohr or Sommerfeld model, depending on the case.

	Since the corresponding wave functions in the well are such that $U=0\Leftrightarrow E_p=0$, we have therefore:
	
	Thus, the total energy can be written:
	
	Thus, given that the wave function is a conditional probability, we have in the form of a phasor:
	
	And the associated discrete energies are then:
	
	The vector $\vec{k}$ is therefore defined by:
	
	\begin{tcolorbox}[title=Remark,colframe=black,arc=10pt]
	We can easily see that the differences in energy between consecutive levels are all smaller as the dimensions of the black body (assimilated to a box) $L_x$,$L_y$,$L_z$ are larger. For macroscopic dimensions, these differences are totally inappreciable. This observation will allow us a little further to make a small approximation.
	\end{tcolorbox}
	Explanation: For an electron ($m\cong 10^{-30}$ [kg]) enclosed in a cubic box of side $L=1$ [m], the difference between two consecutive levels is:
	
	Therefore approximately $10^{-35}$ [J]...

	The vectors $\vec{k}$ that interest us (since each represent respectively a possible micro-stat ), immersed in the phase space of the wave numbers, have their end situated in one of the nodes of a three-dimensional network constituted by elementary meshes whose edges are parallel to the axes and which measure respectively $2\pi/L_x$, $2\pi/L_y$ and $2\pi/L_z$. We want to evaluate the number of vectors for which this end falls in the interval between the two spheres centered at the origin and radii of standard $k$ and $k+\mathrm{d}k$. The infinitesimal volume of the spherical shell between the two spheres is thus trivially given by:
	
	The number of elementary cells (microstates) included in this region of the space of $\vec{k}$ is roughly equal to the number of times its volume contains that of the elementary cell, which is:
	
	We thus obtain the number of micro-states in the volume $\mathrm{d}V$ (hence the density of micro-states):
	
	We must not forget the following relations (\SeeChapter{see sections of Wave Mechanics page \pageref{wave number}, Wave Quantum Physics page \pageref{de broglie associated wave} and Special Relativity page \pageref{wave number special relativity}}):
	
	Therefore as:
	
	and ( for recall):
	
	Therefore it comes:
	
	But when physicists had developed this theoretical model (apparently Bose would have been the first to do so in a modern way), they had noticed that the final result did not correspond to the experience with a factor of $2$. From then on, they multiplied empirically the density of micro-states by a factor $2$ such that:
	
	This factor $2$ would be explained today (I have never seen the proof of this personally ...) by the spin $1$ of the photon. This permits a priori three values for its projection: $-1$, $0$, $1$. The value $0$ would be forbidden by the quantum theoretical field (if someone has the proof of this statement, I am a taker!).

	In a black body in thermodynamic equilibrium, photons (which are bosons) form a gas whose constituents do not interact chemically with each other. This type of situation is typically described by the Bose-Einstein distribution that we have proved in the section of Statistical Mechanics. Thus, since $\mu=0$, we have in the case of a discrete spectrum of energy states (it is thus here that appears the concept of quantification that Planck had originally introduced by tinkering):
	
	and in a case that we consider as continuous:
	
	Before continuing it is important for the reader to realize that the analysis of the radiation of the black body thus comes for the two previous relations to be imposed:
	\begin{itemize}
		\item First a quantification of energy levels in packets $h\nu$ whereas at the time of the discovery of the phenomenon the Nature was considered continuous in all its phenomena.

		\item Secondly, consider Boltzmann's theory of entropy and the resulting statistical distribution of Bose-Einstein is a pillar of the study of physical systems.
	\end{itemize}
	In the black body, we have for internal energy:
	
	Hence:
	
	The radiation of a black body (its "\NewTerm{monochromatic brightness}" as some say ...) is thus given by the "\NewTerm{Planck's law}\index{Planck's law}\label{planck law}":
	
	The first equality is often written in the following form which allows an interpretation of the result:
	
	Where:
	\begin{itemize}
		\item  $8\pi\nu^2/c^3$ corresponds to the volumic density of modes of the radiation of the black body

		\item $h\nu$ to the mean energy per quantum of energy (!!!!!)

		\item $\dfrac{1}{e^{\dfrac{h\nu}{kT}}-1}$ represents the population of the modes
	\end{itemize}

	Finally, let us see another common way of writing Planck's law. Since:
	
	therefore:
	
	but, as $R(\lambda,T)\geq 0$, $R(\nu,T)\geq 0$, it is convenient to take the absolute value:
	
	Finally, we get another form of Planck's law which expresses the energy flux density for a given wavelength:
	
	\begin{tcolorbox}[title=Remark,colframe=black,arc=10pt]
	Max Planck proposed this law by a succession of theoretical trial and errors and analogies in 1900 without knowing the statistical distribution of Bose-Einstein which is remarkable experimentally speaking!
	\end{tcolorbox}
	If $h\nu\ll kT$ (hence in the domain of long wavelengths), the Taylor development of $e^x$ for small $x$ (\SeeChapter{see section Sequences and Series page \pageref{usual maclaurin developments}}) gives:
	
	Which gives us:
	
	and the Planck's law thus becomes the "\NewTerm{Rayleigh-Jeans law}\index{Rayleigh-Jeans law}" (which had been discovered before Planck's law):
	
	That can also sometimes be found in the specialized literature in the form:
	
	Conversely, if $h\nu \gg kT$ we have:
	
	And the Planck's law therefore becomes:
	
	Which is nothing but the "\NewTerm{Wien's first law}\index{Wien's first law}" (which had been discovered before the Rayleigh-Jeans law). This law effectively describes the presence of a maximum of radiation, but contrary to the Rayleigh-Jeans law, it provides false values for large wavelengths (ie small frequencies). Moreover, it implies that the intensity of radiation tends towards zero with the increase of the temperature, which also contradicts the experiment.

	Here is a diagram showing the differences between the three laws:
	\begin{figure}[H]
		\centering
		\includegraphics[scale=1]{img/mechanics/planck_rayleigh_wien_radiation.jpg}
		\caption[Plot of the three radiation laws (Planck's, Rayleigh-Jeans, Wien) in a log-log scale for comparison]{Plot of the three radiation laws (Planck's, Rayleigh-Jeans, Wien) in a log-log scale for comparison (source: Wikipedia)}
	\end{figure}
	We see above that the Rayleigh-Jeans was good at low frequencies whereas that of Wien was good at the highs. Historically it took numerous tests and quite a few years to the physicists to find the adequate model that finally led Max Planck to consider the quantification of the states of energy (which he did not realize himself only quite a long time after having elaborated his model).
	
	Another form of the previous diagram only showing the Rayleigh-Jean law and Planck's law in function of wavelength rather than in frequencies is most well known:
	\begin{figure}[H]
		\centering
		\includegraphics[scale=0.9]{img/mechanics/rayleigh_jean_planck_law.jpg}
		\caption[Plot Planck's and Rayleigh-Jeans law]{Plot Planck's and Rayleigh-Jeans law (source: ?)}
	\end{figure}
	as it highlights the "\NewTerm{ultraviolet catastrophe}\index{ultraviolet catastrophe}", also named the "\NewTerm{Rayleigh–Jeans catastrophe}\index{Rayleigh–Jeans catastrophe}". Indeed, it was the prediction of late 19th century/early 20th century Classical Physics , based on mathematical developments of Rayleigh and Jeans using the equipartition theorem (\SeeChapter{see section Continuum Mechanics page \pageref{equipartition theorem of energy}}), that an ideal black body at thermal equilibrium will emit radiation in all frequency ranges, emitting more energy as the frequency increases. By calculating the total amount of radiated energy, (i.e., the sum of emissions in all frequency ranges), it can be shown that a black body would release an infinite amount of energy, contradicting the principles of conservation of energy and indicating that a new model for the behavior of black bodies was needed.
	\begin{tcolorbox}[title=Remark,colframe=black,arc=10pt]
	Many popular histories of physics, as well as a number of physics textbooks, present an incorrect version of the history of the ultraviolet catastrophe. In that version, the "catastrophe" was first noticed by Planck, who developed his formula in response. In fact Planck never concerned himself with this aspect of the problem, because he did not believe that the equipartition theorem (\SeeChapter{see section Continuum Mechanics page \pageref{equipartition theorem of energy}}) was fundamental – his motivation for introducing "quanta" was entirely different. That Planck's proposal happened to provide a solution for it was realized only later, as stated above. Though the true sequence of events has been known to historians for many decades, the historically incorrect version persists, in part because Planck's actual motivations for the proposal of the quantum are complicated and difficult to summarize for a lay audience.
	\end{tcolorbox}
	We can also prove again the Stefan's law (we have already done this but with another approach) but this time by explaining the origin of the Stefan-Boltzmann constant $\sigma_\text{S.B.}$.

	Let us first recall that the energy flux (\SeeChapter{see section Geometric Optics page \pageref{radiance and luminance}}) is given by:
	
	Since the luminance depends on the frequency and therefore the temperature of the emitter body, we can add:
	
	The energy radiated through a given elementary surface $\mathrm{d}S\cos(\theta)$ is therefore:
	
	If the volume of emission is considered as an elementary volume assimilated to a cylinder of height $c\mathrm{d}t$ and of roof having for surface $\mathrm{d}S\cos(\theta)$ (\SeeChapter{see section Geometric Optics page \pageref{surface projection emittence}}) the energy density per unit of frequency and steradian is then given by:
	
	Given the isotropy of the black body at equilibrium, we have by integrating on the whole of the solid angle the energy density per unit of frequency only:
	
	Dimensional analysis gives us:
	
	Finally, it is useful to consider the total power emitted per unit area (ie emittance):
	
	If we integrate on the half surface of a sphere (relatively to the surface point of the emitter):
	
	Indeed for a sphere (\SeeChapter{see section Trigonometry page \pageref{solid angle}}):
	
	Since the luminance is independent of $\theta$ (isotropy of the radiation of the black body), integration is elementary, and we find:
	
	The total emittance is then given by:
	
	By putting $x=h\nu/kT$, we can simplify the integrande so that:
	
	Let us prove, because its far to be easy that:
	
	Let us write the integrand in the following form:
	
	For the term:
	
	Let us put $t=e^{-x}$, then we have:
	
	And we have proved in the section of Functional Analysis, using results of the section Sequences and Series, that under the condition $|t|<1$, the last fraction can be written:
	
	Since then:
	
	Hence:
	
	We can therefore replace our integral by a sum of defined integrals. We see that by making integrations by successive parts, we get:
	
	Therefore:
	
	However, we have proved in the section of Sequences and Séries in our study of the Riemann zeta function that:
	
	Therefore:
	
	Finally, the Stefan-Boltzmann constant\index{Stefan-Boltzmann constant} can be expressed analytically in the form:
	
	Honestly ... it was hard to guess ... It must be notice that since Planck's constant was not introduced until a couple decades after the Stefan-Boltzmann law was developed, one might more appropriately say that the Stefan-Boltzmann constant pins down Planck's constant.
	
	For summary, we have then:
	
	Let us determine now for what frequency we have the maximum energy density. In other words, this is equivalent to seek where the derivative:
	
	cancels. Therefore:
	
	Let us divide by $R(\nu,T)$:
	
	The last relation admits a single positive root that we can determine with Maple 4.00b using the command:

	\texttt{>evalf(solve(exp(-x)-1+1/3*x=0,x));}
	
	
	What gives us the "\NewTerm{Wien's second law}\index{Wien's second law}" or "\NewTerm{Wien's law of displacement}\index{Wien's law of displacement}\label{wien displacement law}" which says that as the temperature of a black body increases, the wavelength at which the intensity of the radiation Is the strongest becomes shorter and shorter:
	
	where $a$ is named the "\NewTerm{Wien's constant}\index{Wien's constant}". The figure below shows this result using the logarithmic scales:
	\begin{figure}[H]
		\centering
		\includegraphics[scale=1]{img/mechanics/wien_second_law.jpg}
		\caption{Representation of Planck's law (distribution) with Wien's displacement}
	\end{figure}
	Thus, not only does the increase in temperature lead to an increase in the total amount of radiated energy but the Wien displacement law tells us that the wavelength at which the maximum amount of radiation is emitted multiplied by the temperature of the Black body is always a constant (or otherwise seen: the maximum wavelength is inversely proportional to the temperature as shown in the green line in the figure above)! This is an extremely simple result that tells us two important things:
	\begin{enumerate}
		\item If the temperature doubles, then the maximum wavelength will be half the previous wavelength.

		\item If we know the constant (which is our case) then we can calculate the maximum wavelength for any temperature at which a black body can be found.
	\end{enumerate}
	It is easy to understand now why any material at low temperatures emits mainly long wavelength radiation in the infrared part of the spectrum and as the temperature rises, there is more radiated energy in each region of the spectrum and the wavelength peak decreases as it moves toward shorter wavelengths. As a result, the color of the emitted light changes from red to orange, then yellow to bluish-white (and further into the ultraviolet).
	\begin{tcolorbox}[title=Remark,colframe=black,arc=10pt]
	Obviously the second Wien's law is also sometimes given in the literature not in function of the frequency, but to in function of the wavelength ...
	\end{tcolorbox}
	Let us insist on the fact that the Planck's law is valid only in the cases where the radiation is in thermal equilibrium. This restriction is important in practice since the phenomena of emission or absorption of radiation by the material occur most often under conditions out of equilibrium: in the case, for example, of lighting by an electric lamp or electric heating by infrared radiation, the is an irreversible (and therefore out of equilibrium) transformation of electrical energy into radiation energy. Similarly, solar radiation is produced by the nuclear reactions which take place inside the Sun and which gradually consume its substance; At the microscopic level also, the emission of a photon by an excited atom is very often an irreversible return of the atom to its fundamental state (spontaneous emission out of equilibrium). In the case of the black body, on the contrary, the radiation is confined within a closed enclosure (we eventually leave a negligible fraction of this radiation to escape outside to be subjected to the measurements) and we can thus achieve thermal equilibrium with the walls.

	Planck's law, which we have demonstrated above, is perfectly verified by experience in the whole range of temperatures accessible at this date for a black body and remains a quite good approximation also for Stars and other objects that are not at equilibrium:
	\begin{figure}[H]
		\centering
		\includegraphics[scale=1]{img/mechanics/planck_law.jpg}
		\caption[Planck's law at various temperatures]{Planck's law at various temperatures (source: ?)}
	\end{figure}
	We notice from the graph above that a body heated between $5,000$ and $6,000$ [K] has an emission peak in the middle of the visible spectrum. In the field of colorimetry, we associate a temperature with a color by looking for the temperature of the black body for which the peak of radiation has its maximum in the wavelength of the given color.

	It should be notice that many light sources emit a luminous flux that does not follow the law of the black body (a bulb filament, for example) and that the Wien law does not apply to them. On the other hand, it remains true that they emit at a wavelength all the shorter as they are hot.

	It should also be keep in mind that the luminous flux coming from an object is not necessarily of a thermal nature! In other words its color does not always indicate its temperature. For example, the color of the sky comes from blue sunlight diffused by the air (Rayleigh scattering), and not from a hypothetical temperature of $15,000$ [K]. Similarly a tree is green, not because it is $8'000$ [K] hot, but because it reflects the green light that makes up the light of day...
	
	\begin{figure}[H]
		\centering
		\includegraphics[scale=0.9]{img/mechanics/black_body_vs_cmb_cobe.jpg}
		\caption[Comparison of theoretical black body and our Universe Cosmic Microwave background]{Comparison of theoretical black body and our Universe Cosmic Microwave background (source: Wikipedia)}
	\end{figure}
	
	\begin{flushright}
	\begin{tabular}{l c}
	\circled{90} & \pbox{20cm}{\score{3}{5} \\ {\tiny 71 votes,  66.76\%}} 
	\end{tabular} 
	\end{flushright}

	%to make section start on odd page
	\newpage
	\thispagestyle{empty}
	\mbox{}
	\section{Continuum Mechanics}\label{continuum mechanics}
	\lettrine[lines=4]{\color{BrickRed}I}n a strict sense, the continuum mechanics is the branch of mechanics that has for purpose the study of motion, deformation and stress fields inside continuum materials.\\\\
	
	Modeling an object as a continuum assumes that the substance of the object completely fills the space it occupies. Modeling objects in this way ignores the fact that matter is made of atoms, and so is not continuous; however, on length scales much greater than that of inter-atomic distances, such models are highly accurate. Fundamental physical laws such as the conservation of mass, the conservation of momentum, and the conservation of energy may be applied to such models to derive differential equations describing the behavior of such objects, and some information about the particular material studied is added through constitutive relations.

	Continuum mechanics deals with physical properties of solids and fluids which are independent of any particular coordinate system in which they are observed. These physical properties are then represented by tensors, which are mathematical objects that have the required property of being independent of coordinate system. These tensors can be expressed in coordinate systems for computational convenience.
	
	\textbf{Definitions (\#\mydef):}
	\begin{enumerate}
		\item[D1.] We name "\NewTerm{medium}\index{medium}", any fluid (solid, liquid, gas or plasma depending on what we have seen in the section of Thermodynamics), deformable or not, when we consider it a macroscopic point of view, as opposed to a corpuscular description.
		
		\item[D2.] We name "\NewTerm{continuous medium}\index{continuous medium}", a medium such that if $M$ and $M'$ belong to a medium and $M'$ belongs to the neighborhood of $M$, then whatever the deformation undergone by that environment, $\mathrm{d}M'$ will still belong to the neighborhood of $\mathrm{d}M$.
	\end{enumerate}
	This field is often seen as the engineering science for understanding and describing the material world around us and currents phenomena that take place: liquid movements, gas, flying airplanes, helicopters, rockets, satellites, sailing boats, solids deformations, internal structure of stars, etc. By its relations with  thermal mechanics (thermodynamics), it extends to the thermal analysis, energy analysis, acoustic and even General Relativity (the Universe being considered as an continuum material on long distances)!
	
	Taking into account the behavior of continuum, it encompasses hydrodynamics, gas dynamics, elasticity, acoustics, plasticity and other behaviors. It is the key to what we now name "modeling", which is nothing but the art of analyzing a physical phenomenon and describe it mathematically, allowing their study with rigour specific to this discipline.
	
	This section of the book is divided into four main parts: solid, liquid, gas and plasma (some concepts have been deliberately developed in the section of Mathematic Music). In each part, we introduce the specific mathematical tools to study a particular continuum with a complexity (all relative) growing. However, by choice and as always in this book, it was decided to present the theorems with the simplest possible mathematical tools but all arriving at the same results. For example, the proof of the Navier-Stokes equation that would take 150 pages of rigorous mathematical developments do takes no more than 27. There is therefore a significant advantage both for the writer (...) and for the reader do so.
	\begin{tcolorbox}[title=Remark,colframe=black,arc=10pt]
	Regarding the Navier-Stokes, we also give practical closed-forms examples of these in our study of Meteorology (\SeeChapter{see section Marine \& Weather Engineering page \pageref{meteorology}}) and when we will have time we will give an easy example of a liquid wave simulation in MATLAB™ in the section of Theoretical Computing.
	\end{tcolorbox}
	
	\subsection{Rigid Bodies}	
	Atoms of the same element or of different elements are assembled in specific structures. This conditions the strength of their electrical interactions, which define the final structure of the substance. Under normal conditions on our planet, matter exists in solid, liquid, gas or plasma. If the interatomic forces are sufficiently intense, the particle collection retains its shape and volume.
	
	This property of retaining the shape and volume, as well as elastic properties distinguish solids from other states of matter.
	
	\subsubsection{Pressures}\label{constraints}
	The concepts of "\NewTerm{compression}\index{compression}" and "\NewTerm{constraint}\index{constraint}" (that we can include improperly into the global term "\NewTerm{pressure}\index{pressure}\label{pressure}") are of prime importance in continuum mechanics. It is therefore necessary to define the different types of pressure with a minimum of rigor!
	
	\textbf{Definitions (\#\mydef):}
	\begin{enumerate}
		\item[D1.] We name "\NewTerm{compression pressure}\index{compression pressure}" traditionally denoted $P$, the ratio expressed by the force $F$ which is exerted (applied) on a surface element $S$ to the normal thereof. Thus, in scalar form:
		
		\begin{tcolorbox}[title=Remark,colframe=black,arc=10pt]
		If a force acts on a finished surface, then we also speak of "\NewTerm{distributed force}\index{distributed force}".
		\end{tcolorbox}
		
		\item[D2.] We name "\NewTerm{constraint pressure}\index{constraint pressure}" the ratio expressed by the force $F$ that pulls on a surface element $S$ not necessarily perpendicularly, traction force  which can then be divided into two respectively tangent and normal vectors. So as into vector form:
		
		\begin{figure}[H]
			\centering
			\includegraphics{img/mechanics/normal_tangential_constraints.jpg}
			\caption{Illustration of normal and tangential constraints (stresses)}
		\end{figure}
		where $\vec{\sigma}$ and $\vec{\tau}$ are respectively the "\NewTerm{normal constraint}\index{normal constraint}" and the "\NewTerm{tangential constraint}\index{tangential constraint}" (also sometimes denoted with an $S$ as indices to indicates that this is relatively to a surface).
	\end{enumerate}
	Both variables above are measured in $[\text{N}\cdot \text{m}^{-2}]$ or "\NewTerm{Pascals [Pa]}\index{Pascals [Pas]}" and by definition (as already mentioned during our study of Units), $100,000$ [Pa] is equal $1$ [bar].
	
	We could very well include the above two definitions in only one and work with the signs of the forces. But to be consistent with what is taught in schools, we will keep these two definitions that identify by definition the fact that the forces are opposed relative to a surface element $S$.
	\begin{tcolorbox}[title=Remark,colframe=black,arc=10pt]
	We will give the definition of hydrostatic pressure further below during our study of liquids.
	\end{tcolorbox}
	
	\subsubsection{Elasticity of Solids}
	In one way or another, a compressive or tension constraint can deform the triple: height, width, thickness of a body. Directly address to the study of a case that deform these three parameters is a bit long and will be discussed further below in the part on the determination of the expression of Young's shear modulus.
	
	But it could be useful, at least only from the point of view of the vocabulary to give an example from the most simplistic case in can be. If we imagine a one dimension elastic body (having neither height nor width, but just a length) under the application of two perfectly collinear constraints but antagonists forces, we can imagine that the considerated body is lengthen from a given factor.
	
	\textbf{Definition (\#\mydef):} The "\NewTerm{normal deformation}\index{normal deformation}" under axial and antagonists forces is given by the ratio between the change in body length to its original length (ie: the relative elongation) such that:
	
	This relation is an extremely simplified form of all kinds of deformations that may exist. Indeed, when we rigorously studies the deformation, we must take into account the accumulated distortions and then we speak of "\NewTerm{true longitudinal deformation}\index{true longitudinal deformation}", or even of "\NewTerm{longitudinal rational deformation}\index{longitudinal rational deformation}", by the change in length such that:
	
	Therefore after integration:
	
	We can consider in all generality that:
	
	Therefore it comes (the same reasoning applying for a surface or a volume)
	
	But we see above a well-known expression of usual Taylor series (\SeeChapter{see section Sequences and Series page \pageref{usual maclaurin developments}}) and then we can write for small deformations:
	
	and we fall back on our initial naive definition. We also see that this approximation is also equivalent to put from stat that:
	
	which is a writing (approximation) that we will see again in the section of Thermodynamics in our determination of the liquid state equation but with the volume $V$ instead of the length $L$.
	
	There is necessarily a relation between compression and tensile forces and the variation in size of a body. This relation is dependent on the atomic structure of the material and should rigorously use quantum physics to be determined (we will omit this approach in the section of this book). We observe, however, depending on the materials various characteristics that interest a the highest point engineers:
	\begin{figure}[H]
		\centering
		\includegraphics{img/mechanics/elasticity_behavior_various_materials.jpg}
		\caption{Behavior under stress/compression for some materials}
	\end{figure}
		The above figures represent the variation of the compressive stress versus strain stress for some materials (typically we represent these characteristics by inverting the axes). We can observe that:
	\begin{itemize}
		\item Ductile materials such as mild steel (a) ceases to be linear in the elasticity limit denoted $\sigma_r$ above.

		\item Under traction, rubber  polymers (b) first lengthen their molecules and pulling their chemical bonds (\SeeChapter{see section Quantum Chemistry page \pageref{molecular chemistry}}).

		\item Most organic materials (c) are under constraint, even when if they are not distorted. The skin, for example, is as a big rubber glove enveloping the body.

		\item Elastin (d) is usually enhanced with collagen in biological systems such as arteries. A tendon is primarily made of collagen.
	\end{itemize}
	In a more general case, the engineers used to define the points shown below in their material constraint testing measurements:
	\begin{figure}[H]
		\centering
		\includegraphics{img/mechanics/tensile_stress_strain_diagram.jpg}
		\caption{Tensile Stress-Strain Diagram}
	\end{figure}
	The above characteristic includes a linear portion as is the case of a certain class of materials. This means that the slope of the characteristic is a constant which reflects the elastic deformation of the material under the effect of increasing stress. This elastic stress by deformation unit defines the "\NewTerm{Young's modulus}\index{Young's modulus}\label{young modulus}" (there is no tangential component in this case study!):
	
	also denoted sometimes by the letter $Y$ toi avoid obviously any confusion with energy.  This relation is valid both compressive or in tension stresses . We will return to this relation in the following paragraphs.
	\begin{tcolorbox}[title=Remark,colframe=black,arc=10pt]
	\textbf{R1.} "\NewTerm{Rheology}\index{rheology}" is field of the mechanics that study plasticity, elasticity, viscosity and the fluidity characteristics of the deformable body. This is a very important branch of industrial engineering.\\
	
	\textbf{R2.} Be careful the calculations that will follow are relatively long and difficult, even if we tried to simplify them to the maximum as we alway do in this book. However, all the results will be infinitely useful either to determine the Navier-Stokes equation or for the study of material strength (\SeeChapter{see section of Mechanical Engineering page \pageref{mechanical engineering}})!
	\end{tcolorbox}
	
	\textbf{Definition (\#\mydef):}  The "\NewTerm{toughness}\index{toughness}" is the ability of a material to absorb energy and plastically deform without fracturing. One definition of material toughness is the amount of energy per unit volume that a material can absorb before rupturing. It is also defined as a material's resistance to fracture when stressed.
	
	Toughness can be determined by integrating the stress-strain curve as seen above. It is the energy of mechanical deformation per unit volume prior to fracture. The explicit mathematical description is:
	
	 
	where $\varepsilon$ is strain, $\varepsilon_f$ is the strain upon failure and $\sigma$ is stress.
	
	Typically in laboratories the toughness is multiplied by the length of the test specimen. This is why the toughness is given many times in Energy units by Surface units as in the figure below:
	\begin{figure}[H]
		\centering
		\includegraphics[scale=0.8]{img/mechanics/toughness.jpg}
		\caption[Toughness schema of common material]{Toughness schema of common material (source: ?)}
	\end{figure}
	For example, brittle materials (like ceramics) that are strong but with limited ductility are not tough; conversely, very ductile materials with low strengths are also not tough. To be tough, a material should withstand both high stresses and high strains. Generally speaking, strength indicates how much force the material can support, while toughness indicates how much energy a material can absorb before rupturing.
	
	For general knowledge here is a diagram of strength vs density for various materials (sadly the source in unknown):
	\begin{figure}[H]
		\centering
		\includegraphics[scale=0.8]{img/mechanics/strengthvsdensity.jpg}
		\caption[Strength VS Density of some materials]{Strength VS Density of some materials (source: ?)}
	\end{figure}
	
	\paragraph{Hooke's law}\label{hooke law}\mbox{}\\\\
	Hooke's law is a principle of physics that states that the force $F$ needed to extend or compress a spring by some distance $\Delta L$ is proportional to that distance. An elastic body or material for which this assumption can be assumed is said to be "\NewTerm{linear-elastic}\index{linear-elastic material}" or "\NewTerm{Hookean}\index{Hookean material}".

	On the other hand, Hooke's law is an accurate approximation for most solid bodies, as long as the forces and deformations are small enough. For this reason, Hooke's law is extensively used in all branches of science and engineering, and is the foundation of many disciplines such as seismology, molecular mechanics and acoustics. It is also the fundamental principle behind the spring scale, the manometer, and the balance wheel of the mechanical clock.

	The modern theory of elasticity generalizes Hooke's law to say that the strain (deformation) of an elastic object or material is proportional to the stress applied to it.
	
	Taking into consideration the definitions given above:
		
	we get the relation:
	
	which by definition is the "\NewTerm{linear Hooke's Law}\index{linear Hooke's Law}" in normal stress only!
	\begin{figure}[H]
		\centering
		\includegraphics{img/mechanics/hooks_law_normal.jpg}
		\caption{Illustration of the effect of a normal stress}
	\end{figure}
	\begin{tcolorbox}[colframe=black,colback=white,sharp corners]
	\textbf{{\Large \ding{45}}Example:}\\\\
	Consider a suspension cable used to carry gondolas at ski resorts and that the cable has an unsupported span of $3$ [km]. We assume that the cable has a radius of $r=2.3$ [cm] and its tension is $3.0\cdot 10^6$ [N]. The stretch of the steel cable is therefore equal to:
	
	\end{tcolorbox}
	It is quite intuitive to assume that the higher the bonding force of the atoms constituting the study material, the greater is the force to be applied to take them away, so to stretch the body. Solids, which have big bond strengths, have a high melting temperature (this is detailed in the section of Quantum Chemistry).
	
	If we denote:
	
	We end up with a relation that we already know:
	
	which is the restoring force of springs (\SeeChapter{see sections of Classical Mechanics page \pageref{restoring force classical mechanics} and Mechanical Engineering page \pageref{spring tension}}) and just a variant notation of Hooke's law.
	
	There is also another way to introduced Hooke's law that is quite interesting and used in many other fields of physics (especially in Solid State Physics) as it used the microscopic picture of a solid.

	Let us consider for this purpose a solid piece of some simple material. If we look at pure elemental solids – for
example metals – we often find that when they form a solid the atoms arrange themselves into a regular lattice that is "close packed" – arranged so that the atoms more or less touch each other with a minimum of wasted volume. There are a number of kinds of lattices that appear (determined by the subtleties of the quantum mechanical interactions between the atoms) but in many cases the lattice is a variant of the cubic lattice where there are atoms on the corners of a regular cartesian grid in three dimensions (and sometimes additional atoms in the center of the cubes or on the faces of the cubes).

	Not all materials are so regular. Materials made out of a mixture of atoms, out of molecules made of a mixture of atoms, out of a mixture of molecules, or even out of living cells made out of a mixture of molecules. The resulting materials can be ordered, structured (not exactly the same thing as ordered, especially in the case of solids formed by life processes such as bone or coral), or disordered (amorphous).

	As usual, we will deal with all of this complexity by ignoring most of it for now and considering an "ideal" case where a single kind of atoms lined up in a regular simple cubic lattice is sufficient to help us understand properties that will hold, with different values of course, even for amorphous or structured solids.

	This is illustrated in the figure below:
	\begin{figure}[H]
		\centering
		\includegraphics[scale=0.8]{img/mechanics/microscopic_picture_of_a_solid.jpg}
		\caption{Idealized simple cubic lattice of atoms separated by the "springs" of interatomic forces that hold them in equilibrium positions}
	\end{figure}
	which is basically a mental cartoon model for a generic solid – lots of atoms in a regular cubic lattice with a cube side $a$, where the interactomic forces that hold each atom in position is represented by a spring, a concept that is valid as long as we don't compress or stretch these interatomic bondy by too much.
	
	We need to quantify the numbers of atoms and bonds in a way that helps us understand how stretching or compressing forces are distributed among all of the bonds. Suppose we have $N_x$ atoms in the $x$-direction, so that the length of the solid is $L = N_xa$. Suppose also that we have $N_y$ and $N_z$ atoms in the $y$- and $z$-directions respectively, so that the cross-sectional surface is therefore:
	
	Now let us imagine applying a force (magnitude $F$) uniformly to all of the atoms on the left and right ends (see figure below) that stretches all of these bonds by a small amount $\Delta x$, presumed to be "small" in precisely the sense that leaves the interatomic bonds still behaving like springs. The force $F$ has to be distributed equally among all of the atoms on the end areas on both sides, so that the force applied to each chain of atoms in the $x$-direction end to end is:
	
	\begin{figure}[H]
		\centering
		\includegraphics[scale=0.75]{img/mechanics/microscopic_picture_of_a_solid_with_force.jpg}
		\caption[]{The same lattice stretched by an amount $\Delta L$ as a force $F$ is applied to both ends, spread out uniformly across the cross-sectional area of the faces $A$.}
	\end{figure}
	Each spring in the chain is stretched by this force between the atoms on the ends, so that:
	
	The negative sign just means that the springs are trying to go back to their equilibrium length and
hence oppose the applied force. We multiply this by $1$ in the form $(aN_x)/(aN_x)$ to get:
	
	giving as$L=aN_x$:
	
	Finally, we rearrange this by putting all of the extra terms together as follows (we see that we fall back on Hooke's law):
	
	In words, we define $\sigma=F/S$ to be the "\NewTerm{Stress}\index{stress}", $\varepsilon=\Delta L/L$ to be the "\NewTerm{Strain}\index{strain}", and $E = K/a$ to be Young's Modulus. With these definitions, the equation above states that: Compressive or extensive Stress applied to a solid equals Young's Modulus times the Strain.
	
	But there are several types of constraints with their respective modules. So here are the most important definitions in the linear part of their characteristic associated with the explanatory diagram:
	\begin{figure}[H]
		\centering
		\includegraphics{img/mechanics/shear_stress_illustration.jpg}
		\caption{Illustration of the effect of a shear stress}
	\end{figure}
	\textbf{Definitions (\#\mydef):}
	\begin{enumerate}
		\item[D1.] We define the "\NewTerm{shear modulus}\index{shear modulus}" or "\NewTerm{rigidity modulus}\index{rigidity modulus}" by the ratio of the normal component of the force (compression pressure) to the shear strain:
		
		where the numerator is named the "\NewTerm{shear stress}\index{shear stress}" and where $\gamma$ is the "\NewTerm{angle of deformation}\index{angle of deformation}". Generally this angle being small, we have the approximation:
		
		$S$ is the surface of the upper or lower face of the deformed body shown in the figure above.
		
		\item[D2.] We define the "\NewTerm{sliding elastic modulus}\index{sliding elastic modulus}", also named "\NewTerm{sliding modulus}\index{sliding modulus}" and most commonly "\NewTerm{shear modulus}\index{}shear modulus" or "\NewTerm{Coulomb modulus}\index{Coulomb modulus}" by the ratio of the tangential component of the force (shear stress) to the shear strain:
		
		where $\eta$ is the "\NewTerm{Poisson's coefficient}\index{Poisson's coefficient}" which we will prove later in detail where it comes from.
		
		Notice well that even if the numerator of the above definition is a force divided by an area, this is not a pressure because the force is tangential to the surface (hence the $T$ in index of $F$).
		
		This is because any force can be decomposed into a normal and tangential force (see the above definition of the compression pressure and constraint pressure) that we have two distinct definitions above. In the vast majority of laboratories situations, we arrange to have a purely tangential force (again... hence the $T$ in index of the $F$) or purely normal force (hence the $N$ in index of the $F$) to the surface $S$.
		
		In practice many times make usage of only the second definition and thus to such a level that the latter is often just named "\NewTerm{rigidity modulus}\index{rigidity modulus}" also...
		
		\begin{tcolorbox}[colframe=black,colback=white,sharp corners]
		\textbf{{\Large \ding{45}}Example:}\\\\
		An very interesting thing (for the parenthesis ...) if we consider that the tectonic plates are in a shearing situation between themselves we then have according to the sliding modulus:
		
		But for a tectonic plate inf friction of length $L_0$ over a height $H$:
		
		and since energy is a force multiplied by a distance, we get:
		
		which is typically the energy liberated by the shear of the friction of two tectonic plates whose contact surfaces have an average height $H$, an initial length $L_0$ and undergoing a deformation $\Delta L$.\\
	
		Typically for an earthquake of the type of Sumatra in 2009 ($9.1$ to $9.3$ on Richter scale, $28$ [km] underground, $200,000$ deaths), we had:
		
		Therefore it comes:
		
		in other words ... a thousand times the energy of the Hiroshima nuclear bomb in.\\
	
		If we denote by $M$ the empirical definition of magnitude of the Richter's scale magnitude, we get:
		
		while estimations give a range of 6.2 to 9.2 ... so we are not too bad in the theoretical approach.\\
	
		This was non business application example...
		\end{tcolorbox}
		
		\pagebreak
		\item[D3.] We define the "\NewTerm{omnidirectional modulus}\index{omnidirectional modulus}", as the ratio of the volume constraint named "\NewTerm{bulk stress}\index{bulk stress}" on the volumetric strain named "\NewTerm{bulk strain}\index{bulk strain}" (we will prove further below the mathematical developments that lead to the last term of this relation):
		
		In practice it is much more interesting to use:
		
		and to seek what material (depending on the context) maximize or minimize $\Delta V$.
		
		We could define many other modules such as the bending modulus, the pure bending modulus, the composed bending modulus, the twisting modulus... We will study some of them further below.
	\end{enumerate}

	For each of the different definitions of modules that we can imagine, we can define a Hooke's law that is suitable to it. However, all this may seem rather arbitrary, but in fact it is not! Because all definitions of modules that we saw earlier are a special case of a generalized mathematical relation we will be prove in this book in a near future.
	
	\paragraph{Shear Modulus}\mbox{}\\\\
	The necessary condition for a rigid solid to be in static equilibrium as we saw it in the section of Classical Mechanics, that the resultant forces that applied to the body is equal zero:
	
	However, when a solid is subjected to stresses and it can tolerate them, there can be a deformation which may be followed by a breaking or a similar modification. More precisely, there is "\NewTerm{deformation}\index{deformation}" of a body (not necessarily solid) when the distance between the elements it is made of have changed.
	
	When in the theoretical study of elasticity, we exclude modifications of the body studied such as breaks, we say that we restrict ourselves to the "\NewTerm{elastic deformations}\index{elastic deformations}".

	The geometry and the physics of deformations can be complex. Their description is derived from that of a number of elementary deformations we will specify further the characteristics.
	\begin{figure}[H]
		\centering
		\includegraphics{img/mechanics/cube_normal_stresses.jpg}
		\caption{Cube under normal stresses}
	\end{figure}
	The scalar tensile stresses forces $F_x,F_y,F_z$ generate on their faces "normal" tensions (that is to say perpendicular... for recall...):
	
	Assuming the only the force $F_x$ acts, the unit deformation is by definition:
	
	When a parallelepiped is subjected to a tensile stress $F_y$, there is intuitively contraction of dimensions along the $x$ direction. Contraction also observable also intuitively for $F_z$.

	We then have if only the force $F_y$ acts:
	
	where the sign "$-$" indicates a contraction and where $\eta$ is a coefficient named "\NewTerm{Poisson's coefficient}\index{Poisson's coefficient}".
	
	If the Poisson coefficient of the studied matter is almost zero, this means that when you "crush" (compress) the material, its height decreases, but its width remains almost constant ...

	One remark on the way: There are products that have a negative Poisson coefficient: when compressed, their width decreases! In general, it is porous media, inverted honeycomb style like  expanded graphite that has this original property.

	If only the force $F_z$ acts:
	
	By accepting the principle of superposition of forces, the effect of several forces acting simultaneously is equal to the sum of the effects produced by each of the superimposed forces acting separately (this is obviously a strong assumption!). Since then:
	
	This can be considered as acceptable, given the linearity of the equations linking the unit deformation  and the normal stresses. We then get:
	
	having proceed in the same way for the other two directions following the $y$-axis and $z$-axis.
	
	From the previous relations, it is easy to find the equations linking $\sigma$ to $\varepsilon$ by solving the following linear system (\SeeChapter{see section Linear Algebra page \pageref{linear systems}}):
	
	Given a material subject to various constraints. Inside of it, we operate in thought, to the extraction of a rectangular parallelepiped:
	\begin{figure}[H]
		\centering
		\includegraphics{img/mechanics/parallelepiped_for_theoretical_study.jpg}
		\caption[]{Rectangle parallelepiped for the theoretical study}
	\end{figure}
	The surfaces thereof are sollicitad by normal $\sigma$ and tangential  $\tau$ stresses (in the figure below the solid  is in static equilibrium). Here is the yellow face:
	\begin{figure}[H]
		\centering
		\includegraphics{img/mechanics/parallelepiped_face_for_theoretical_study.jpg}
		\caption[]{Generic illustration of a material under normal and tangential stresses}
	\end{figure}
	The normal $\sigma$ and tangential $\tau$ constraints represent the actions of the parallelepiped material extracted mentally on the faces of the examined item.
	
	It is interesting (in the sense that it facilitates the analysis) to seek the constraints that exist in a plane making an angle $\theta$ with the $x$-axis. To do this, we imagine a material triangle having a top angle $\theta$ extracted from matter mentally. We neglect the influence of gravity.
	
	Therefore:
	\begin{figure}[H]
		\centering
		\includegraphics{img/mechanics/parallelepiped_face_oblique_triangle_for_theoretical_study.jpg}
		\caption[]{Search of constraints expressions in an oblique plane}
	\end{figure}
	Let us put:
	
	and $\mathrm{d}z$ being the thickness of the solid (not shown in the diagram above).

	On the length $\mathrm{d}s$ constraints appear and are decomposed under normal $\sigma$ and tangential $\tau$ stresses (the latter being also named "shear stress" or "bending stress" as we already know).

	The problem is to establish the relations between $\sigma,\tau$ and $\sigma_x,\sigma_y$ and $\tau$.

	The sign conventions are:
	\begin{itemize}
		\item The constraints $\sigma$ applying a traction are positive while the constraints $\sigma$ exerting  compression are negative.
		
		\item Constraints $\tau$ that tends to rotate the cuboid in the clockwise direction are positive. Counterclockwise, they will be negative.
	\end{itemize}
	The projection equation of equilibrium following the direction $ON$ is then:
	
	Let us recall that:
	
	As $\mathrm{d}x=\mathrm{d}s\cos(\theta)$ and $\mathrm{d}y=\mathrm{d}s\sin(\theta)$ we have:
	
	As:
	
	then:
	
	Finally:
	
	As conclusion we can say that in function of $\sigma_x$, $\sigma_y$ and $\tau_{xy}$, it is possible to calculate the normal stress that exists on a plane surface of angle $\theta$.
	
	The projection equation of equilibrium is following the direction $OT$ is:
	
	as $\cos^2(\theta)-\sin^2(\theta)=\cos(2\theta)$ then finally:
	
	The conclusion is that in function of $\sigma_x$, $\sigma_y$ and $\tau_{xy}$ it is possible to calculate the tangential stress $\tau$ that exists on a plane surface of angle $\theta$.
	
	So finally withe to both relations:
	
	gives the values of the normal and tangential stresses in any cut in function of the normal and tangential stresses to the primitives axes.
	
	Given now the following situation:
	\begin{figure}[H]
		\centering
		\includegraphics{img/mechanics/stressess_three_dimensional_case.jpg}
		\caption[]{Situation for coming back to the study of the three dimensional case}
	\end{figure}
	On the figure above on the left the reader must imagine a block of material from which is virtually extracted a small square plane (in blue on the left figure) from which we will study in a first time only of the right triangles that compose it for afterwards study the whole.
	
	Before solicitation, we therefore consider the diamond $abcd$ that is in fact originally a square with an angle of $\pi/4$ in the $x$ direction. On the request of a reader here is perspective diagram of this situation:
	\begin{figure}[H]
		\centering
		\includegraphics{img/mechanics/block_material_perspective_initial_situation.jpg}
		\caption[]{Initial material block situation}
	\end{figure}
	During solicitation, this diamond $abcd$ is deformed under the action of tangential stresses decomposed in pure shear stresses and becomes the diamond $a'b'c'd'$. Here is also a schema of this situation as requested by a reader:
	\begin{figure}[H]
		\centering
		\includegraphics{img/mechanics/block_material_perspective_final_situation.jpg}
		\caption[]{Final material block situation}
	\end{figure}
	The diagonal $\overline{bd}$ is then extended and the diagonal $\overline{ac}$ is compressed. The angle $\hat{a}$ that had for value $\pi/2$ is after deformation equal to $\pi/2+\gamma$ (in $a'$). Similarly, the angle $b$ which value was $\pi/2$ is after deformation equal to $\pi/2-\gamma$.
	
	\begin{tcolorbox}[title=Remark,colframe=black,arc=10pt]
	The angle $\gamma$ is named "\NewTerm{slip angle}\index{slip angle}" and we consider it be be small.
	\end{tcolorbox}
	We need only look at the effect of the deformation by isolating the diamond and by applying on it a rotation of $\pi/4$. After deformation, we have the shape indicated by the dotted lines (see Fig. A. before the prior previous figure above).

	The slip angle being small, we have:
	
	Therefore $\gamma$ represents the slip of the side $ab$ with respect to $dc$ divided by the distance between the two planes $ab$ and $dc$. The analysis that has been done is still valid whatever the solid or liquid body considered.

	Given, now, the case of an elastic solid obeying Hooke's law. The problem will be to establish the relation between the slip angle $\gamma$ and the tangential  constraints $\tau$ acting on the sides of the diamond.

	Given the right triangle $\text{O}ab$. The lengthening of the side $\overline{\text{O}b}$ and shortening of side $\overline{\text{O}a}$ during the deformation are obtained from the following equations:
	
	As:
	
	We have:
	
	Therefore:
	
	Then the length $\overline{\text{O}a'}$ decrease if $\tau$ increase.

	We also have:
	
	Then the length $\overline{\text{O}b'}$ decrease if $\tau$ increase.

	For the angle of the right triangle $\text{O}a'b'$, we have:
	
	But using trigonometric identities (\SeeChapter{see section Trigonometry page \pageref{remarkable trigonometric identities}}):
	
	As $\tan\left(\dfrac{\gamma}{2}\right)\cong \dfrac{\gamma}{2}$ when $\gamma$ is small as proved in the section of sequences and series, we have:
	
	Therefore:
	
	Finally, we fall back on the relation giving the "\NewTerm{shear modulus}\index{shear modulus}" or "\NewTerm{Coulomb module}\index{Coulomb module}" we had given earlier without proof:
	
	
	\paragraph{Compressibility Modulus (bulk modulus)}\mbox{}\\\\
	We still have to see the origin of mathematical expression from another modulus as important as the shear modulus: the bulk modulus.
	
	Given the system of equations determined in the previous study:
	
	If the forces applied on the cube are equal in intensity, we have:
	
	Which gives us:
	
	Summing the terms according to the principle of linear superposition of forces:
	
	But:
	
	That is:
	
	Therefore:
	
	So:
	
	Finally, we find for the relative change in volume of a solid after deformation:
	
	That we also write sometimes
	
	Therefore:
	
	with $\kappa$ being by definition the  "\NewTerm{compression coefficient}\index{compression coefficient}".

	We then define the "\NewTerm{bulk modulus}\index{bulk modulus}\label{bulk modulus}" by:
	
	Let us notice that when the Poisson coefficient $\eta$ is about $0.33$ then $K$ is equal to $E$ (metallic materials are close to this case) and when the Poisson's ratio approaches $0.$5 then $K$ approaches infinity and thus the material is incompressible (elastomers approach such an incompressible behavior).
	
	\pagebreak
	\paragraph{Flexural Modulus (bending modulus)}\mbox{}\\\\
	The "\NewTerm{flexural modulus}\index{flexural modulus}" or "\NewTerm{bending modulus}\index{bending modulus}" is the ratio of stress to strain in flexural deformation, or the tendency for a material to bend. It is determined from the slope of a stress-strain curve produced by a flexural test, and uses units of force per area. It is an intensive property.
	
	To study of the flexural modulus we consider the following situation:
	\begin{figure}[H]
		\centering
		\includegraphics{img/mechanics/flexurus_modulus_typical_situation.jpg}
		\caption[]{Example of a bar under flexural deformation}
	\end{figure}
	The above left figure shows a material in a static state. The right figure shows the same material but subject to a couple force momentum $M$ (the height variation of the middle point is named the "\NewTerm{deflection}\index{deflection}").
	
	As the material undergoes on its surface a compression and at the opposite side a tension, so it should exist a border (a line or plane) where no constraint exists. This line or plane (it's rare that we are dealing with a material having only two dimensions ...) is named "\NewTerm{neutral plane}\index{neutral plane}". This neutral plane will serve us  as a reference for defining the bending stress.
	
	Now that this plan is defined, let us consider the following figures:
	\begin{figure}[H]
		\centering
		\includegraphics{img/mechanics/neutral_plane_illustration.jpg}
		\caption[]{Illustration of the neutral plane to determine the flexural modulus}
	\end{figure}
	Given $R$ be the bar (cylinder, plate, parallelepiped, ...) radius of curvature. The deformation on the segment $\overline{i_1j_1}$ defined by the relation:
	
	The lengths $\overline{mn}$ and $\overline{ij}$ are defined by:
	
	and the length $\overline{i_1j_1}$ by:
	
	thus expression of the deformation becomes:
	
	indicating that the deformation varies linearly with $y$.
	
	indicating that the deformation varies linearly with $y$.

	We can define the flexural modulus by:
	
	Let us consider the static state of the bar. The sum of traction and compression constraints are therefore zero. Indeed, we see it well if we look at the diagram below:
	\begin{figure}[H]
		\centering
		\includegraphics{img/mechanics/flexion_plane.jpg}
		\caption[]{Zoom on the flexion plane}
	\end{figure}
	Let us consider $\sigma\mathrm{d}S$ the force acting on a surface element $\mathrm{d}S$. We can consider the equilibrium of forces in the static state such that:
	
	Substituting the expression of the constraint obtained previously:
	
	by assuming linear the strain characteristic in a first approximation, therefore $E=c^{te}$.

	By simplifying a little bit:
	
	If we multiply the integral by $\sigma$ then the relation should be equal to the force momentum (torque) applied such that:
	
	By substituting by the expression of the stress obtained previously:
	
	Which brings us to define the term:
	
	that engineers named the "\NewTerm{moment of inertia of the bar relative to the neutral plane}" or "\NewTerm{static moment of inertia}\index{static moment of inertia}". This term is a measure of the rigidity of the cross section of the bar from a geometric point of view, without consideration of material properties.

	Substituting this relationship in the bending stress equation, we get the general "\NewTerm{flexural modulus}\index{flexural modulus}" relation:
	
	The challenge for the engineer is often to locate the plane mathematically neutral ... We will see how to calculate the latter in some special cases in the section of Mechanical Engineering.
	
	\pagebreak
	\paragraph{Tranverse Wave in Solids}\label{transverse wave in solids}\mbox{}\\\\
	The transverse sound waves or "\NewTerm{S-waves}\index{S-waves}" ("\NewTerm{shear waves}\index{shear waves}"), at the opposition of "P-waves" that are compression/dilation type wave\footnote{"S" stands for "secondary" and the "P" form "primary"}, for occur only in solids and typical of seismic events. The successive layers of the medium move laterally without volume, density or pressure, change:
	\begin{figure}[H]
		\centering
		\includegraphics[scale=0.9]{img/mechanics/p_and_s_waves.jpg}
		\caption{P-wave and S-Wave in solids}
	\end{figure}
	and focusing only on the S-waves we have for the front view schematically:
	\begin{figure}[H]
		\centering
		\includegraphics{img/mechanics/p_wave_front_view.jpg}
		\caption{P-Wave front view}
	\end{figure}
	The medium is deformed in the same way that you can distort a book or ream of paper laid both on a flat table by pushing up horizontally the cover or respectively the first paper page. Neither the book nor the train not change volume!

	The way to obtain the wave equation for transverse waves is almost the same as for a string (\SeeChapter{see section Wave Mechanics page \pageref{equation of vibrating strings}}). Let us take three contiguous thin planar layers of the medium (see figure below):
	\begin{figure}[H]
		\centering
		\includegraphics[]{img/mechanics/shear_wave_zoom_for_analysis.jpg}
		\caption{Zoom on three layers of a shear wave}
	\end{figure}
	The centers of the layers being on $x_a,x_b,x_c$ with:
	
	The transversely displacement of the three adjacent layers is $y_a,y_b,y_c$. The angles of deformation respectively between the layer $b$ and the layer $a$, and, between the layer $c$ and layer $b$ are in the first order Taylor approximation (\SeeChapter{see section Sequences and Series page \pageref{usual maclaurin developments}}):
	
	If we calculate the forces between the layers for a piece of layer of surface $S$, we get:
	
	where $G$ is as we know the shear modulus of the medium. The resultant force is then:
	
	The force of the layer $F_b$ will be equal at any time to the product of the mass of the piece of layer $b$, of thickness $\mathrm{d}s$, of surface $S$ and density $\rho$, multiplied by the acceleration of the layer given by:
	
	Then we have:
	
	Which gives:
	
	What we have have just deduce for any value $x_b$ is true for any coordinate:
	
	and the transverse wave velocity is therefore:
	
	Indeed, the ratio $G/\rho$ has the units of the square of a speed:
	
	It we finish therefore with a wave equation (recall) of the form of a Poisson's equation (more particularly it is a "d'Alembert's equation" as we know), ie:
	
	Transverse waves propagate only in solids and therefore we can not hear them unless we transform them into longitudinal waves (P waves) by mechanical or electrical means. The transverse waves can be transmitted along a bar or rod of any or even a metal wire, and this without the need that the latter is under stress. Even if the metal wire is under stress, the speed of shear waves is not dependent on the stress (obviously in a given limit...). This is the high shear modulus $G$ of steel that gives electric guitars that characteristic sound.
	
	Another remarkable case of transverse waves (shear waves) is the seismic wave has we have already mention it. There are seismic shear waves and also longitudinal waves or pressure as we have show it also earlier. Shear waves propagate in the Earth's crust at the speed of $3,600\; [\text{m}\cdot\text{s}^{-1}]$ and the pressure wave at the speed of $6,000\;[\text{m}\cdot \text{s}^{-1}]$. During an earthquake or an atomic explosion, the two types of waves will be produced, but as the waves travel at different speeds, they will not arrive at the same time at remote sensing stations. It is from this difference in arrival times that we can determined the distance at the epicenter. The direction is obtained from the direction of the oscillations. Only sufficiently distant stations to receive both types of wave separately can make the determination of the epicenter.
	
	To summarize, we have for longitudinal waves in a solid (\SeeChapter{see section Music Mathematics page \pageref{propagation of a longitudinal deformation}}):
	
	and for transverse waves:
	
	For the details of the mathematical developments in the gas and solid about propagation waves, the reader should read the section of Mathematical Music (Acoustic) page \pageref{acoustic}.
	
	\pagebreak	
	\subsection{Liquids (fluid mechanics)}\label{fluid mechanics}
	A liquid is a nearly incompressible fluid\footnote{In fact nothing as far as we know is absolutely incompressible but when in the study we do the relative variations of pressure and speed a small enough we consider the fluid as being incompressible.} that conforms to the shape of its container but retains a (nearly) constant volume independent of pressure. As such, it is one of the four fundamental states of matter (the others being solid, gas, and plasma), and is the only state with a definite volume but no fixed shape. A liquid is made up of tiny vibrating particles of matter, such as atoms, held together by intermolecular bonds. Water is, by far, the most common liquid on Earth. Like a gas, a liquid is able to flow and take the shape of a container. Most liquids resist compression, although others can be compressed. Unlike a gas, a liquid does not disperse to fill every space of a container, and maintains a fairly constant density. A distinctive property of the liquid state is surface tension, leading to wetting phenomena.
	\begin{figure}[H]
		\centering
		\includegraphics[scale=0.6]{img/mechanics/water_tap.jpg}
	\end{figure}
	The density of a liquid is usually close to that of a solid, and much higher than in a gas. Therefore, liquid and solid are both termed condensed matter. On the other hand, as liquids and gases share the ability to flow, they are both named "fluids". Although liquid water is abundant on Earth, this state of matter is actually the least common in the known universe, because liquids require a relatively narrow temperature/pressure range to exist. Most known matter in the universe is in gaseous form (with traces of detectable solid matter) as interstellar clouds or in plasma form within stars.
	
	The distinction between liquid and gas is subtle. However, we can say that the proper volume of liquid manifests the existence of cohesiveness due to a fairly high density (van der Waals bonds); this cohesion disappears with the proper volume of a gas.

	If we compare the fluids with solids, the first that comes concerns the isotropy (properties are the same in all spatial directions) of conventional fluids that is always satisfied (at least if we do not act on the fluid directly!).
	
	In what follows, we will address the fluid mechanics theory in increasing difficulty and redundancy by trying to keep it simple and stupid as possible. First, we will be prove that the properties of a static fluid are isotropic (Pascal's theorem). Using this result, it will be easier for us to understand Bernoulli's theorem that will allow us, among others, to define the concept of "hydrostatic pressure". Then we will build a very important model for fluid dynamics, known as the "Navier-Stokes equations", which has usages almost all possible existing areas of science (astrophysics, quantum mechanics, meteorology, ..). This fluid dynamics model is quite big in theoretical developments and experimental results and can be considered as difficult to approach. However, for ease of reading, we chose not to address it by use of tensor calculus. We have made sure that the tensor  variables appear  by themselves following the simple previous vector analysis results we will get. Once the Navier-Stokes equations determined and proved, we will see that we can find the expression of Bernoulli's theorem from these equations.
	
	The Navier-Stokes equations and finally used with Finite Element Methods to make numerical simulations of shapes to optimize the penetration or movement of objects in water or in the air:
	\begin{figure}[H]
		\centering
		\includegraphics[scale=0.9]{img/mechanics/bulbous_bow.jpg}
		\caption[]{Bulbous bow}
	\end{figure}
	On the picture above we see a bulbous bow that is a protruding bulb at the bow (or front) of a ship and using the concept of destructive interference of waves to cancel the wave produced at the back by the bow. Large ships with bulbous bows generally have a twelve to fifteen percent better fuel efficiency than similar vessels without them.
	
	Fluid dynamics, or "\NewTerm{hydrodynamic}\index{hydrodynamic}" is by far the field of classical mechanics the less easy to understand as regards to the descriptions and predictions. This is why the Bernoulli's theorem is often used, not to explain in detail the behavior of a fluid, but to make a qualitative description for it.
	
	\pagebreak
	\subsubsection{Pascal's Fluid Theorem}
	 \textbf{Definitions (\#\mydef):} The "\NewTerm{Pascal's law}\index{Pascal's law}" or the "\NewTerm{principle of transmission of fluid-pressure}\index{principle of transmission of fluid-pressure}" (also named "\NewTerm{Pascal's Principle}\index{Pascal's Principle}") is a principle in fluid mechanics that states that a pressure change occurring anywhere in a confined incompressible fluid is transmitted throughout the fluid such that the same change occurs everywhere.
	
	\begin{figure}[H]
		\centering
		\includegraphics{img/mechanics/pascal_theorem.jpg}
		\caption[]{Elementary tetrahedron to introduce Pascal's fluid theorem}
	\end{figure}
	If we consider the forces acting, in the absence of movement on an elementary tetrahedron O$ABC$ of volume $V$, it is always possible to choose a sufficiently small volume in order to have a uniform pressure on all the faces of the tetrahedron .

	Given $P_{11},P_{22},P_{33},P$, the fluid reaction pressures due to external constraints soliciting the respective faces O$BC$, O$AB$, O$AC$ and $ABC$ of respecting surfaces $S_{\text{O}BC},S_{\text{O}AB},S_{\text{O}AC},S$. Given also the cosine direction (\SeeChapter{see section Vector Calculus page \pageref{cosines directions}}) of the unit vector normal $\vec{n}$ to the surface $ABC$.

	The system being in equilibrium, the resultant $\vec{R}$ of the of the forces of reaction of the system is zero. So we have the following equations resulting from the projection along the three coordinate axes:
	
	By elementary simplification, it comes:
	
	We then get the following relation:
	
	The major conclusion is that in any point of a fluid pressure is independent of the direction of the normal $\vec{n}$ to the elementary surface on which it is exercised.

	In other words, by the Newton's principle of action and reaction, we are led to state the "\NewTerm{Pascal's theorem}\index{Pascal's theorem}" and "\NewTerm{Pascal's principle}\index{Pascal's principle}": The incompressible fluids (in equilibrium) transmit integrally and in all directions, the pressures applied to them.
	
	This theorem is fundamental both in fluid mechanics and gas  mechanics and the practical implications are huge. This theorem explains among others the famous "\NewTerm{hydrostatic paradox}\index{hydrostatic paradox}" as what the pressure of a liquid on the bottom of a container is independent of its shape, and also of the bottom surface but only depends on the water level in the recipient. This theorem is the basis for the design of many hydraulic machines!

	Let's look at a very clear illustration. Consider the following diagram (we made an example in U, although if the majority of hydraulic systems are linear):
	\begin{figure}[H]
		\centering
		\includegraphics{img/mechanics/u_tube_pascal_theorem.jpg}
		\caption[]{Typical hydraulic machine (elevator, hydraulic jack)}
	\end{figure}
	What is happening exactly? Well on the left, we have the force $F_1$ that creates the pressure:
	
	and by the Pascal's principle, we must find the same pressure on the other side such that:
	
	It follows immediately the very important relation in practice:
	
	So we have a force multiplier! Warning! The principle of the hydraulic machine is to have a fantastic force multiplier but in no case it multiplies the work $W$! So by the principle of conservation of energy, we have:
	
	and then it comes:
	
	Therefore, in counterpart of the force multiplication effect, the energy conservation, the primary force must travel a greater distance.
	
	\subsubsection{Viscosity}\label{viscosity}
	The viscosity of a fluid is a measure of its resistance to gradual deformation by shear stress or tensile stress. For liquids, it corresponds to the informal concept of "thickness"; for example, honey has a much higher viscosity than water.

	Viscosity is a property arising from collisions between neighboring particles in a fluid that are moving at different velocities. When the fluid is forced through a tube, the particles which compose the fluid generally move more quickly near the tube's axis and more slowly near its walls; therefore some stress (such as a pressure difference between the two ends of the tube) is needed to overcome the friction between particle layers to keep the fluid moving. For a given velocity pattern, the stress required is proportional to the fluid's viscosity.

	A fluid that has no resistance to shear stress is known as an ideal or inviscid fluid. Zero viscosity is observed only at very low temperatures in superfluids. Otherwise, all fluids have positive viscosity, and are technically said to be viscous or viscid. In common parlance, however, a liquid is said to be viscous if its viscosity is substantially greater than that of water, and may be described as mobile if the viscosity is noticeably less than water. A fluid with a relatively high viscosity, such as pitch, may appear to be a solid.
	
	In fluid mechanics, it is useful to consider several types of fluids with characteristics that differentiate them. This is especially useful for simulations while remaining consistent with the experimental observations (\SeeChapter{see section Marine \& Weather Engineering page \pageref{meteorology}}).
	
	We define the "\NewTerm{viscosity}\index{viscosity}" $\mu$ by the internal forces opposing the movement of the various layers composing the fluid. We distinguish two type of viscosity:
	\begin{enumerate}
		\item The "\NewTerm{dynamic viscosity $\mu_d$}\index{dynamic viscosity}":
		
		which unit is the Poiseuille: [PI] (equivalent to the pascal-second $[\text{Pa}\cdot\text{s}$, or $[\text{N}\cdot \text{s}\cdot \text{m}^{-2}]$, or $[\text{kg}\cdot\text{m}^{-1}\cdot\text{s}^-1]$), $\mathrm{d}F$ the variation of the frictional force between two infinitely neighboring layers, $\partial_z v$ being the variation in speed by the distance between two infinitely neighboring layers and in $[\text{s}^{-1}]$ and $\mathrm{d}S$ being the considered surface $[\text{m}^2]$.
		
		If a fluid is placed between two plates with a distance of $1$ [m], and one plate is pushed sideways with a shear stress of one [Pa], and it moves at $v$ meter per second, then it has a viscosity of $1/v\;[\text{Pa}\cdot\text{s}]$. 

		For example, water at $293$ [K] has a viscosity of $0.001\;[\text{Pa}\cdot\text{s}]$ while a typical motor oil could have a viscosity of about $0.25\;[\text{Pa}\cdot\text{s}]$.
		
		Conclusion: The Poiseuille is the viscosity of a fluid needing a force of $1$ Newton to slide at a speed of $1$ meter per second, two fluid layers of $1$ square meter by a distance of $1$ meter.
		
		A transformation of the definition of dynamic viscosity gives (this relations must be remembered for later use!!):
		
		That is:
		
 
		\item The "\NewTerm{cinematic viscosity $\mu_c$}\index{cinematic viscosity}":
		
	\end{enumerate}
	By definition fluids having the following characteristics:
	\begin{figure}[H]
		\centering
		\includegraphics{img/mechanics/fluid_viscosity_characteristics.jpg}
		\caption{Viscosity characteristics of different fluids}
	\end{figure}
	are named respectively
	\begin{enumerate}
		\item \NewTerm{Pseudoplastic fluids}\index{pseudoplastic fluids}
		\item \NewTerm{Newtonian fluids}\index{Newtonian fluids} (shear stress proportional to the velocity gradient)
		\item \NewTerm{Dilatant fluids}\index{dilatant fluids}
	\end{enumerate}
	There are still three other types of fluids not shown in the figure and whose viscosity is assumed to be zero. We then speak of "\NewTerm{Pascalian fluids}\index{Pascalian fluids}":
	\begin{itemize}
		\item "\NewTerm{Perfect fluids}\index{perfect fluids}":  A fluid that can be completely characterized by its rest frame mass density $\rho$ and isotropic pressure $p$. Specifically, perfect fluids have no shear stresses, viscosity, or heat conduction.
		
		\item "\NewTerm{Semi-perfect fluids}\index{semi-perfect fluids}": A fluid that eventually conduct heat and whose heat capacity is only dependent of temperature.
		
		\item "\NewTerm{Real fluids}\index{real fluids}": A fluid with all possible complications that we observe in real life...
	\end{itemize}
	\begin{tcolorbox}[title=Remarks,colframe=black,arc=10pt]
	\textbf{R1.} The behavior of a perfect fluid is very different from that of a real fluid as small as the viscosity can be for the latter. Indeed, the perfect fluid because it has no viscosity, never dissipate kinetic energy. While a real fluid with a small viscosity effectively dissipates it very efficiently through turbulences.\\
	
	\textbf{R2.} We will come back on the properties of the dynamic and kinematic viscosity during the proof of the Navier-Stokes-(Reynolds) equations.\\
	
	\textbf{R3.} The fluids that are not Newtonians are named very generally in the literature "\NewTerm{non-Newtonian fluids}\index{non-Newtonian fluids}" ... and we will not deal with Ferrofluids in this book because they are theoretically outside the range of this book.
	\end{tcolorbox}
	Non-Newtonian fluids thus have a deformation which depends on the force that we apply to them. The best example is the wet sand by the sea: when we hit the sand, it has the high viscosity of a solid, whereas when we quietly press on it, it behaves like a paste. Moreover, some non-Newtonian fluids have properties such that it is possible for an individual to run on it without sinking or sink staying in position...
	
	\pagebreak
	\paragraph{Poiseuille's Law}\mbox{}\\\\
	In 1835 a French doctor, Jean Leonard Marie Poiseuille, made a series of careful experiments to determine how a viscous fluid flows through a narrow pipe. His goal was to understand the dynamics of blood flow in humans. Blood plasma behaves like a Newtonian fluid, while blood does not. Almost half of the volume of the normal blood is made large enough cells to disrupt laminar flow, especially when they come in contact with the vessel walls, a phenomenon that becomes important in very narrow capillaries. However, the analysis of Poiseuille applies to flow in the veins and larger arteries and it has great value, although it is a little bit simplistic.
 	\begin{figure}[H]
		\centering
		\includegraphics[scale=0.8]{img/mechanics/laminar_turbulent_flow.jpg}	
		\caption[Laminar VS Turbulent flow ]{Laminar VS Turbulent flow (source: OpenStax)}
	\end{figure}
	The result of Poiseuille can be established by considering the fluid in a pipe as formed by concentric cylindrical layers oriented along an $x$-axis of radius $r$ moving at speeds which decrease from the center (circular symmetry assumed).

	Then the relation defining the viscosity is:
	
	What gives us the viscous force on the cylinder. The contact surface of each cylindrical layer of length $l$ is given obviously by $S=2\pi r l$ and therefore:
	
	The origin of the acceleration (i.e. the force) can be done only by a pressure difference such as:
	
	which brings us to write:
	
	By integrating term by term, we get
	
	So we get the famous "\NewTerm{law of viscous laminar flow}\index{law of viscous laminar flow}" discovered by the French physicist Jean-Louis Marie Poiseuille in 1840:
	
	widely used in medicine to calculate the blood flow speed in the veins, in hydraulic systems (to determine the pressure required to have a certain speed) and in wind tunnels (for the same reason as in hydraulic systems).
	
	The representing curve of the speed in function of $r$ is a parabola whose summit is on the cylinder center axis ($r=0$). So the speed is greater at the center than at the edges (that is quite intuitive). Differentiating this last relationship with respect to $r$, we immediately get the velocity gradient in function of $r$ (calculation useful in practice).

	The "\NewTerm{volume flow}\index{volume flow}" $J$ (sometimes denoted $D$ in small classes) carried by a cylindrical layer between $r$ and $r + \mathrm{d}r$ is:
	
	Thus, the total flow is:
	
	and we get the "\NewTerm{Poiseuille law}\index{Poiseuille law}" for the viscous laminar flow:
	
	Some books introduce the "\NewTerm{resistance to flow}\index{resistance to flow}" denoted $R_f$ such that the Poiseuille law is finally written:
	
	So we find the logical result that the flow rate increases with the pressure gradient $\Delta P/l$ and the radius of the tube, and decreases with the viscosity.

	We find also a relation analogous to Ohm's law (\SeeChapter{see section Electrokinetics page \pageref{ohm law}}) where the potential difference is given by the resistance multiplied by the current so that the difference in pressure is given by the viscous resistance multiplied by the flaw. We will also come back on this relationship when we deal with the losses pressure in the pipes.
	
	\begin{tcolorbox}[title=Remark,colframe=black,arc=10pt]
	In the literature, the flow is sometimes denoted $J$, $Q$ or $D$ or even $\dot{V}$ ... In short, there is no standard notation as far as we know and we must do with ...
	\end{tcolorbox}
	
	\begin{tcolorbox}[colframe=black,colback=white,sharp corners]
	\textbf{{\Large \ding{45}}Example:}\\\\
	An air conditioning system is being designed to supply air at a gauge pressure of $0.054$ [Pa] at a temperature of $293$ [K]. The air is sent through an insulated, round conduit with a diameter of $0.18$ [m]. The conduit is $20$ [m] long and is open to a room at atmospheric pressure $101.30$ [kPa]. The room has a length of $12$ [m], a width of $6$ [m], and a height of $3$ [m] and we assume a constant pressure difference and using the viscosity $\mu_d= 0.0181\;[\text{mPa}\cdot\text{s}]$.\\ 

	The volume flow rate through the pipe, assuming laminar flow is then given by:
	
	The length of time to completely replace the air in the room is then by assuming constant flow equal to:
	
	\end{tcolorbox}
	The previous relation:
	
	Is also denoted in some textbooks:
	
	and after rearranging we get:
	
	This is an expression well know to experimentally measure the dynamic viscosity of a fluid based on simple parameters.
	
	\subsubsection{Bernouilli's Theorem}
	When we discuss of the motion of a fluid, the continuity equation (\SeeChapter{see section Thermodynamics page \pageref{continuity equation}}), which expresses the conservation of mass (volume) of the fluid is an important concept.
	
	Let us consider this equation in the particular case that interest us here: a non-viscous fluid in laminar flow moving inside a tube following parallel streams lines (the fluid motion is irrotational type - see section Vector Calculus page \pageref{irrotational}), delimited by the surface $A_1$:
	\begin{figure}[H]
		\centering
		\includegraphics{img/mechanics/bernoulli_line_stream.jpg}
		\caption{Non-viscous fluid in laminar flow from in a tube of parallel lines of current lines converging}
	\end{figure}
	We consider that we are in steady state (the appearance of movement is independent of time) and the mass is not provided by a source or removed by a well within the region. The volume of fluid which crosses $A_1$ in the interval corresponds $\Delta t$ corresponds to a cylinder of base $A_1$, and length $v_1\Delta t$ and therefore volume of volume $A_1v_1\Delta t$. The mass of fluid which has passed through $A_1$ for the time $\Delta t$ is therefore:
	
	And also:
	
	is the mass of fluid which has passed through $A_2$ during the same time interval. With the our assumptions, the mass conservation equation requires that two masses are the same, or expressed differently that
	
	Hence:
	
	This is the form of the continuity equation that interest us in our context. In addition, if the fluid is incompressible, the density is everywhere the same and the above relation reduces to:
	
	Thus, the ratio of input/output speed  in a tubular pipe will be usually (not included frictions!):
	
	The continuity equation is indirectly known to all children and adults who play with or use a garden hose (thumb serves as a "convergent"):
	\begin{figure}[H]
		\centering
		\includegraphics{img/mechanics/continuity_equation_in_real_life.jpg}
		\caption{Application in everyday life of the continuity equation}
	\end{figure}
	Let us now consider a region in a fluid where there is a steady flow\footnote{That is, the pressure, velocity, density, etc. can change from place to place, but they must not change as a function of time at any particular place (in your frame of reference).} as shown in the figure below:
	\begin{figure}[H]
		\centering
		\includegraphics{img/mechanics/bernoulli_line_stream_zoom.jpg}
		\caption{Zoom of a fluid region for the Bernoulli's theorem}
	\end{figure}
	During a short time interval $\Delta t$, the fluid which initially crossed the section $A_1$ has moved to a surface ${A'}_1$ at a distance $\Delta x_1=v_1\Delta t$ while the fluid crossing the section $A_2$ is found on ${A'}_2$ at a distance $\Delta x_2=v_2\Delta t$. 

	The both small cylinders have the same volume as the fluid is assumed to be incompressible and that the fluid passing through $A_1$ pushes the volume that pass through $A_2$ and therefore the equation continuity remains valid.
	
	Given $F_1$ and $F_2$ the forces applied indirectly on the surface $A_1$ and $A_2$ due to the pressure prevailing in the fluid. Because of these forces, the fluid produce or receives work by moving the two volumes. In $A_1$, the surface is pushed by the fluid and the work performed on the fluid is $F_1\Delta x$ when on $A_2$ the fluid pushes the surface and the work done by the fluid is $F_2\Delta x$. The total work exerted on the fluid volume situated between $A_1$ and $A_2$ is therefore equal to:
		
	where $P_1$ and $P_2$ are obviously the respective pressures on $A_1$ and $A_2$ and by writing obviously:	
	
	according to the definition of the pressure. As:
	
	according to the continuity equation and the assumption of incompressibility, we can write that:
	
	The external work applied on the system changes its internal energy as established in the section Thermodynamics ($W=\Delta U$). For the considered volume of fluid, the internal energy of the volumes highlighted  in the figure above includes kinetic energy and gravitational energy. The fluid between $A_1$ and $A_2$ gains energy in the volume $A_2\Delta x_2$ . Suppose the two volumes have equal mass $m$, again because of the continuity equation. Then the net energy gain is:
	
	Since we have already assumed the fluid as incompressible, the density $\rho$ is the same everywhere and $m$ can be replaced by $\rho A_1\Delta x_1$ at both ends. Hence:
	
	Combining this relation with $W=(P_1-P_2)A_1\Delta x_1$ we get:
	
	or:
	
	As the above equation relates the magnitudes taken between two arbitrary points along a streamline, we can generalize and write:
	
	This result, known as the "\NewTerm{Bernoulli's theorem}\index{Bernoulli's theorem}", expresses the constancy of the pressure along a streamline in an incompressible fluid, irrotational and non-viscous and where external volumic forces are derived from a potential energy (we'll come back on this more in detail after having determined the Navier-Stokes equations). The above relation can also obviously be interpreted as saying the enthalpy of the parcel remains constant as it flows along a streamline.
	
	We will see further below during our study of gas the generalized version of the above expression involving the polytropic constant $\gamma$ (heat capacity ratio).
	\begin{figure}[H]
		\centering
		\includegraphics[scale=0.35]{img/mechanics/bernoulli_condensation.jpg}
		\caption[Condensation visible over the upper surface of an Airbus A340 wing]{Condensation visible over the upper surface of an Airbus A340 wing caused by the fall in temperature accompanying the fall in pressure, both due to acceleration of the air (source: Wikipedia)}
	\end{figure}
	
	The conservation of the left quantity expresses the conservation of energy along a stream line and we find respectively the volumic kinetic energy, the gravitational volumic potential energy and the pressure.
	\begin{tcolorbox}[title=Remark,colframe=black,arc=10pt]
	Form a stream line to the other, that is the value of the constant $c^{te}$ that change. In addition, use of Bernoulli's theorem requires to know (or at least to have an idea) of the shape of the stream line.
	\end{tcolorbox}
	
	The reader must keep in mind that the following assumptions must be met for this Bernoulli equation to apply:
	\begin{itemize}
		\item the flow must be steady, i.e. the fluid velocity at a point cannot change with time

		\item the flow must be incompressible - even though pressure varies, the density must remain constant along a streamline
		
		\item friction by viscous forces has to be negligible (no thermic dissipation)
	\end{itemize}
	Let us notice also an another elegant and easy way to find this relation. Indeed, the conservation of energy gives us along a stream line:
	
	with respectively the kinetic energy $E_c$, the potential energy $E_p$ and the pressure energy $E_P$ (that latter is not intuitive). Hence:
	
	and if we divide all by the volume, Then We get:
	
	that's it... Therefore Bernoulli equation is simply a statement of the principle of conservation of energy in fluids!
	
	Now consider now two important applications of Bernoulli's theorem.

	If the fluid moves in a horizontal plane, the gravitational potential energy remains constant and then the Bernoulli equation reduces then to (or also in absence of gravitational field):
	
	Thus, in a closed horizontal pipe (wind tunnel), the velocity increase as the pressure decrease and vice versa. We also use this effect to participate in the thrust of an airplane at least in the case of wind tunnel tests in a closed cylinder (caution! this parameter is minor in the case of a displacement in the atmosphere which is an open volume , because it is not what contributes the most to the flight of an airplane, it is the Magnus effect  whose prove is given in the Aerospace section).
	\begin{figure}[H]
		\centering
		\includegraphics[scale=0.7]{img/mechanics/bernoulli_theorem_wing_wind_tunnel.jpg}
		\caption{Illustration of the profile of a wing with the corresponding pressures and speeds in a wind tunnel}
	\end{figure}
	The profile of a wing is most of time constructed in such a way that the air has can in some given configuration of the wind greater velocity above the wing surface than under it, producing a higher pressure below that of the wing, Above in the case of an air flow in a closed cylinder (as was the case in the first wind tunnel tests in the early 1900s!). This results in a resultant upward force (but it's not the only force from which the lift origins!).

	In other words, a specialist in aerodynamics (for aircraft) or in hydrodynamics (for roll stabilizers of large boats) would say in the case of closed volume tests:
	\begin{itemize}
		\item At the "\NewTerm{extrados}\index{extrados}": By curvature effect, the air (water) particles will pass through a smaller transverse surface in the test cylinder and by Venturi effect will then accelerate. Their velocity will therefore first increase strongly and then diminish in order to find at the trailing edge the initial velocity of the flow. All the extrados is therefore the seat of a generalized local depression.

		\item In the "\NewTerm{intrados}\index{intradox}": By curvature, the air (water) particles will pass through a larger transverse surface in the test cylinder and by Venturi effect will then decelerate. Their velocity will therefore decrease strongly and then increase in order to recover the initial velocity of the flow. All the intrados is therefore the center of a generalized local overpressure.
	\end{itemize}
	The lift of a wing with regard to its geometry is due in part to the depression on the extrados, not to the overpressure on the intrados. The wing does not mainly "rest" on the air, but is sucked upward little bit by the latter (for more details see the Aerospace section)!
	\begin{figure}[H]
		\centering
		\begin{subfigure}{.45\textwidth}
		  \centering
		  \includegraphics[width=1\linewidth]{img/mechanics/wing_in_wind_tunnel_in_real_life_01.jpg}
		\end{subfigure}
		\begin{subfigure}{.45\textwidth}
		  \centering
		  \includegraphics[width=1\linewidth]{img/mechanics/wing_in_wind_tunnel_in_real_life_02.jpg}
		\end{subfigure}
	\end{figure}
	\begin{figure}[H]
		\centering
		\begin{subfigure}{.45\textwidth}
		  \centering
		  \includegraphics[width=1\linewidth]{img/mechanics/wing_in_wind_tunnel_in_real_life_03.jpg}
		\end{subfigure}
		\begin{subfigure}{.45\textwidth}
		  \centering
		  \includegraphics[width=1\linewidth]{img/mechanics/wing_in_wind_tunnel_in_real_life_04.jpg}
		\end{subfigure}
	\end{figure}

	\begin{figure}[H]
		\centering
		\begin{subfigure}{.45\textwidth}
		  \centering
		  \includegraphics[width=1\linewidth]{img/mechanics/wing_in_wind_tunnel_in_real_life_05.jpg}
		\end{subfigure}
		\begin{subfigure}{.45\textwidth}
		  \centering
		  \includegraphics[width=1\linewidth]{img/mechanics/wing_in_wind_tunnel_in_real_life_06.jpg}
		\end{subfigure}
	\end{figure}
	
	\begin{figure}[H]
		\centering
		\begin{subfigure}{.45\textwidth}
		  \centering
		  \includegraphics[width=1\linewidth]{img/mechanics/wing_in_wind_tunnel_in_real_life_07.jpg}
		\end{subfigure}
		\caption{Wing streamlines in wind tunnel in real life}
	\end{figure}
	
	Another thing, if the fluid is not in motion, we have then Bernoulli equation which is obviously written:
	
	and that is sometimes named the "\NewTerm{Laplace hydrostatic equation}\index{Laplace hydrostatic equation}" (used in communicating vessels study).
	
	\paragraph{Torricelli's law}\mbox{}\\\\
	The "\NewTerm{Torricelli's law}\index{Torricelli's law}", also known as "\NewTerm{Torricelli's theorem}\index{Torricelli's theorem}", is a theorem in fluid dynamics relating the speed of fluid flowing out of an opening to the height of fluid above the opening assuming no air resistance, viscosity, or other hindrance to the fluid flow. This is a classic case study in high schools.
	
	For this study let us consider a closed volume (recipient) containing a liquid of equal density $\rho$ and having small orifice of surface $S_2$ from which the liquid flows outwards. We want to determine the speed $v_2$ at which the liquid flow of this orifice:
	\begin{figure}[H]
		\centering
		\includegraphics[scale=1]{img/mechanics/toricelli_experiment.jpg}
		\caption{Toricelli experiments configuration}
	\end{figure}
	The volume is assumed to be big enough that neither the liquid level nor the pressure $P$ above its surface $S_1$ will appreciably vary during the flow. Since the liquid exhausting virtual tube runs from the region of the surface of the liquid to the open orifice in the open air, we have:
	
	A liquid flowing in the open air is at atmospheric pressure, $P_2=P_A$, because the liquid is surrounded by free air and nothing is present to maintain a pressure difference. According to Bernoulli's thereom, with:
	
	we have therefore found a current line (streamline) and we can write:
	
	hence:
	
	From the continuity equation principle (flow conservation):
	
	 we deduce that if $S_1\gg S_2$ then $v_2\gg v_1$ and $v_1^2$ is then negligible relatively to $v_2^2$. In the particular but frequent case where the recipient is open to the open air ($P=P_A$), the pressure energy density disappears. The fluid flows only under the effect of gravity, without being pushed by a difference in pressure. We then find (by multiplying this result by the surface of the orifice, we obtain the flow rate):
	
	which is independent of the sections! Obviously by using this last relation, in the case where the flow and the surface of the orifice are given, we can deduce the flow velocity in $\text{m}^2\cdot \text{s}^{-1}$.

	This relation constitutes the "Torricelli theorem". Curiously, we have already seen this relation in the section of Classical Mechanics for the rate of free fall of a body. Therefore it comes the observation made by Torricelli: if the jet is directed directly upwards, it almost reaches the level of the surface of the liquid in the volume. The reason why the jet would not actually reach this level is because some energy loss due to friction.
	
	From this result we can study an interesting case in practice and often also used in numerical modeling courses. It is the exericise consisting in determining the time for a tank (recipient) of the following type to empty:
	\begin{figure}[H]
		\centering
		\includegraphics[scale=1]{img/mechanics/tank_emptying.jpg}
		\caption{Tank emptying}
	\end{figure}
	We start from the theorem of Toricelli in the case where the orifice is much smaller than the diameter of the tank (we will see after that if not the speed is not the same!):
	
	And if we denoted the flow rate by $D$ then we have trivially:
	
	where the sign "$-$" is there to indicate that the tank is emptying. We also have the flow rate which is defined by:
	
	and the variation of volume $V$ in the tank be expressed as:
	
	We then have explicitly:
	
	In our particular case, the surface does not depend on the $z$. We have then:
	
	Either after rearrangement:
	
	Now we integrate on the left of $H$ to $0$ ($H$ being the initial height of the fluid) and on the right from the time $0$ to $T$. We then have:
	
	Either after rearrangement we deduce that the time $T$ of emptying is:
	
	This formula can be used to calibrate a water clock.

	And we notice that if the radius of the orifice of evacuation is equal to the radius of the cylinder we fall back on the expression of the free fall time of a point mass as proved in the chapter of Classical Mechanics:
	
	The figure below show set of Toricelli jets, vertically aligned, leaving the reservoir horizontally:
	\begin{figure}[H]
		\centering
		\includegraphics[scale=1]{img/mechanics/toricelli_jets_enveloppe.jpg}
		\caption[Toricelli jets envelope illustration]{Toricelli jets envelope illustration (source: Wikipedia, author: Matt Cook)}
	\end{figure}
	In this case, the jets have an envelope (a concept also due to Torricelli) which is a line descending at $45^\circ$ from the water's surface over the jets. Each jet reaches farther than any other jet at the point where it touches the envelope, which is at twice the depth of the jet's source. The depth at which two jets cross is the sum of their source depths. Every jet (even if not leaving horizontally) takes a parabolic path whose directrix is the surface of the water.
	
	\pagebreak
	\paragraph{Communicating vessels}\mbox{}\\\\
	Let's talk a little bit about the communicating vessels because like this it seems to be nothing quite interesting but... still in this beginning of the 21st century, most buildings have a supply of running water based on this principle (reserve lakes in height, Water castles and so on...) and many people have used the principle of communicating vessels to siphon a fluid from one tank / container to another at least once in their lives.

	We can first introduce the concept intuitively based on the experimental observation of the $4$ communicating vessels of the following figure with a non-viscous fluid:
	\begin{figure}[H]
		\centering
		\includegraphics[scale=0.6]{img/mechanics/communicating_vessels.jpg}
		\caption{Photo of the principle of communicating vessels}
	\end{figure}
	we see that the fluid stabilizes at the same height for the four vessels independently of heir shape!

	The equilibrium of vessel is explained be the development that follows!
	
	Let us recall that we have by definition:
	
	and therefore:
	
	Using the gradient theorem proved in the section of Vector Calculus, we have:
	
	In a volume of fluid in an homogeneous gravitational field we have:
	
	Therefore putting the two previous triple integrals: 
	
	together we identify:
	
	In one dimension that can be rewritten:
	
	That is:
	
	That is the "\NewTerm{fundamental theorem of hydrostatics}\index{fundamental theorem of hydrostatics}\label{fundamental theorem of hydrostatics}" or sometimes also named "\NewTerm{Stevin's law}\index{Stevin's law}": the pressure at a point in a liquid in statics equilibrium depends only on the depth at that point!
	\begin{tcolorbox}[title=Remark,colframe=black,arc=10pt]
	We will prove far far far further below that we fall back on this relation from the Navier-Stokes equation!
	\end{tcolorbox}
	The same theorems applies for a water tower (essential if it is to reduce the costs of supply and if the location does not allow to play with the topography of the places):
	\begin{figure}[H]
		\centering
		\includegraphics[scale=1]{img/mechanics/water_tower.jpg}
		\caption{Application of communicating vessels with water towers}
	\end{figure}
	So if we want for example on the $4$th floor of a building to have a minimum flow at a given tap, we will have to apply Torricelli's theorem to know how many meters $h$ above the tap level we must found the water level in a water tower.

	We had obtained higher:
	
	Therefore:
	
	Either by using the flow definition:
	
	Where $D$ is the flow rate to be supplied in cubic meters per second and $S$ is the cross-section through which we must ensure the same flow! Thus, a flow rate of $0.05$ liters per second (typical of a kitchen or bathroom tap), that is to say $5\cdot 10^{-5}\;[\text{m}^3\cdot \text{s}^{-1}]$, through a section having a radius of $0.5$ centimeters will give:
	
	This value that gives the minimum height difference between the water tower water level and the 4th floor where is our tap of interest obviously does not take into account the pressure drop in the pipe!
	
	\paragraph{Venturi effect}\mbox{}\\\\
	Some practical applications of fluid mechanics result from the interdependence of pressure and velocity and there is a category of situations in which the potential gravitational energy variation is negligible. The Bernoulli equation then connects the pressure difference to the kinetic energy difference and therefore the variation of the square of the velocity.

	To introduce the Venturi effect let us consider an incompressible (!) fluid, non-viscous and of density $\rho$. The fluid flows in steady state in a cylindrical pipe of radius $R_1$ and cross section $S_1$ followed by a cylindrical tube of radius $R_2$ and cross section $S_2$ and which then takes back the initial geometry of radius $R_1$ and of section $S_1$ (it is typically connection which it is customary to name "\NewTerm{Venturi tube}"). The connection is made by a long cylindrical pipe so that we stay in laminar regime:
	\begin{figure}[H]
		\centering
		\includegraphics[scale=1]{img/mechanics/venturi_tube.jpg}
		\caption{Venturi tube configuration study}
	\end{figure}
	We know (continuity equation) that:
	
	Which means, as we have seen it already, that a diminution of the section through which the fluid passes, results in an increase of its velocity.

	In any situation where the incoming flux is about the same level as the decreasing radius we have $Z_a\cong Z_B$ and then the Bernoulli's equation:
	
	becomes:
	
	Using the equation of continuity, to eliminate (arbitrarily) $v_1$, we get after rearranging:
	
	We can also write using the fundamental equation of hydrostatic proven just earlier above:
	
	As $S_1>S_2$ the second member of the relation is positive and therefore $P_1>P_2$: there is therefore a pressure drop in the tight part of the tube. On arriving at the divergent region again in $S_3=S_2$, the pressure of the fluid increases again and the speed takes back its initial value. This decrease in pressure which accompanies the increase in speed is named the "\NewTerm{Bernoulli effect}\index{Bernoulli effect}" or "\NewTerm{Venturi effect}\index{Venturi effect}".

	Thus, the velocity of the fluid increases in a bottleneck to satisfy the equation of continuity (conservation of the flux / mass) and the fact that it is incompressible (otherwise there would be a sort of plug).	
	\begin{tcolorbox}[title=Remark,colframe=black,arc=10pt]
	Paradoxically, the Venturi effect also occurs when atmospheric air cross a summit or crest  or also in the streets of cities. Indeed, the air that arrives on the mountain or the crest tends to "crash" on it. The cross-section of air flow at the top is therefore lower than at the base. There is therefore also a Venturi effect: the wind speed is higher on the summits and the ridges than at the bottom (the professionals of the glider know something ...). The Venturi effect is also a dangerous phenomenon to take into account when building maintenance works occurs in cities near tunnels. Indeed, as the speed of wind can increase dramatically near the exit of the tunnel it can damage or simply take down all maintenance installation (in such a configuration an accident happened in Fribourg in Switzerland a few years ago).
	\end{tcolorbox}
	The Venturi tube is used in aerospace to measure the relative speed in flight, but also in automotive carburettors and in airbrushes. The Venturi tube is also used in industrial sites to measure the flow of a fluid in a pipe.
	
	From the last relation, we derive trivially a very useful equality in practice:
	
	that it is sufficient to multiply by the surface $S_2$ to have the flow rate in the connection. 

	Obviously we have identically:
	
	Here is a typical Venturi tube to measures the $D_1$ flow rate thanks to pressure differential:
	\begin{figure}[H]
		\centering
		\includegraphics[scale=1]{img/mechanics/venturi_tube_real.jpg}
		\caption[Industrial Venturi tube]{Industrial Venturi tube (source: WIKA)}
	\end{figure}
	For (old) airplanes as already mentioned Venturi tube can be used to measure relative velocity of the plane:
	\begin{figure}[H]
		\centering
		\includegraphics[scale=0.6]{img/mechanics/venturi_tube_cesan_airplane.jpg}
		\caption{Venturi tube on a CESSNA}
	\end{figure}
	
	\paragraph{Pitot Tube}\mbox{}\\\\
	The Pitot tube is used to measure the flow velocity of a perfect subsonic fluid / gas (assumed to be in continuous flow). The Pitot tube most commonly used in aeronautics consists in forming in a tube a pressure tapping orifice at $A$ and at $B$ as visible in the figure below:
	\begin{figure}[H]
		\centering
		\begin{tikzpicture} [scale=1.5,decoration={markings,mark=at position 1cm with {\arrow[blue]{stealth};}}]
			\draw[pattern=north east lines,rounded corners=18pt] (0,-0.6) rectangle(6,0.6);
			\draw[blue,postaction={decorate}] (-2,0.1) ..controls +(4.5,0) and +(-8.5,0).. (5,0.8);
			\draw[blue,postaction={decorate}] (-2,0.3) ..controls +(3,0) and +(-7,0).. (5,0.9);
			\draw[blue,postaction={decorate}] (-2,0.5) ..controls +(1.5,0) and +(-5.5,0).. (5,1);
			\draw[blue,postaction={decorate}] (-2,0.7) ..controls +(1.5,0) and +(-4,0).. (5,1.1);
			\draw[blue,postaction={decorate}] (-2,-0.1) ..controls +(4.5,0) and +(-8.5,0).. (5,-0.8);
			\draw[blue,postaction={decorate}] (-2,-0.3) ..controls +(3,0) and +(-7,0).. (5,-0.9);
			\draw[blue,postaction={decorate}] (-2,-0.5) ..controls +(1.5,0) and +(-5.5,0).. (5,-1);
			\draw[blue,postaction={decorate}] (-2,-0.7) ..controls +(1.5,0) and +(-4,0).. (5,-1.1);
			\draw[blue,postaction={decorate}] (-2,0)--(0,0); ;
			\draw [double distance = 3pt] (0,0) --++ (6,0);
			\draw [double distance = 3pt] (4,0.6)|-++ (2,-0.4);
			\node[draw,fill=white,rectangle,minimum height=1cm] (P) at(6,0){\small Differential Manometer};
			\draw[](0,0) node{\scriptsize  $\bullet$}node[above left]{$A$};
			\draw[](4,0.6) node{\scriptsize $\bullet$}node[above=2pt]{$B$};
			\draw[](2,0.1) node[above]{\small Pitot tube};
		\end{tikzpicture}
		\caption[Horizontal Pitot tube]{Horizontal Pitot tube (source:  http://femto-physique.fr author: Jimmy Roussel)}
	\end{figure}
	Point $A$ is a stop point because the velocity is zero (there is no flow in the orifice, it is just a pressure tap). Far from the obstacle (the Pitot tube) the flow is assumed to in uniform speed v$ $and pressure $P_B$

	At $A$ (breakpoint), using the Bernoulli equation along the current line and considering the variation in height between $A$ and $B$ negligible, the pressure is equal to:
	
	Therefore have:
	
	So for aircraft from the difference of a measure of pressure and knowledge of gas density, it is possible to know the speed! By multiplying by the section of the conduit we also have the corresponding flow rate.
	\begin{figure}[H]
		\centering
		\includegraphics[scale=0.07]{img/mechanics/pitot_tube_in_real_life.jpg}
		\caption{Real Pitot tube}
	\end{figure}
	 Likely due to the aircraft's Pitot tubes being obstructed by ice crystals – caused the autopilot to disconnect, after which the crew reacted incorrectly and ultimately caused the aircraft to enter an aerodynamic stall from which it did not recover - the Air France Flight 447 (AF447/AFR447) from Rio de Janeiro to Paris was the deadliest in the history of Air France on 1 June 2009 with 228 deaths.
	\begin{tcolorbox}[title=Remark,colframe=black,arc=10pt]
	In aeronautics, the dynamic pressure is added to the static pressure (of ambiant atmosphere) to give the total pressure which can be measured at the point of zero velocity of the Pitot tube. By removing the static pressure, one finds the "\NewTerm{dynamic pressure}\index{dynamic pressure}".
	\end{tcolorbox}
	There is another variant of the Pitot tube more used in the industrial field of which here is a diagram:
	\begin{figure}[H]
		\centering
		\includegraphics[scale=1]{img/mechanics/pitot_tube_vertical.jpg}
		\caption{Vertical Pitot tube}
	\end{figure}
	We therefore also apply the Bernoulli theorem on the axis of the pipe. We have then in it (we detail one the developments a little more than in the previous example):
	
	but:
	
	 and therefore:
	
	But we also have using the fundamental theorem of hydrostatics:
	
	We have then:
	
	Therefore:
	
	
	\paragraph{Pressure drop (pressure loss)}\mbox{}\\\\
	Pressure drop is defined as the difference in pressure between two points of a fluid carrying network. Pressure drop occurs when frictional forces, caused by the resistance to flow, act on a fluid as it flows through the tube. The main determinants of resistance to fluid flow are fluid velocity through the pipe and fluid viscosity.

	When there is no machine (no pump or turbine) in the flow of a perfect fluid between points (1) and (2) of the same current line, the Bernoulli relation can be written In the following form:
	
	When the fluid passes through a hydraulic machine, it exchanges energy with this machine in  form of work for a given time. The power $P$ exchanged is then (\SeeChapter{see section Classical Mechanics page \pageref{power}}):
	
	Where by convention, if $P>0$ the energy is received by the fluid (pump) otherwise, if $P<0$ the energy is supplied by the fluid (turbine).

	If the volumic flow is $Q_V$ then Bernoulli's relation is written logically:
	
	where:
	
	A perfect fluid does not exist. During a flow in a pipe, the friction forces dissipate part of the kinetic and potential energy, which results in the existence of pressure losses which must be taken into account.
	
	Let us consider a stationary and incompressible horizontal cylindrical flow. If we apply the Bernoulli relation between input and output, we obtain:
	
	However, experimentally, we observe that we must impose a higher input pressure to maintain the steady state. Indeed, the viscosity forces resist to the flow. It is therefore necessary to impose an overpression $\Delta P_f$ which we name the "\NewTerm{pressure loss}\index{pressure loss}" and that is due to the existence of frictional forces (viscosity) or singular losses (geometry of the distribution circuits).

	The generalized Bernoulli equation is then written in this case of study which is part of the field of process engineering:
	
	This relationship is often used in the theoretical (...) study of pipeline problems.

	In practice, the pressure drop is calculated from the Poiseuille law which we have already demonstrated earlier above:
	
	where for recall $J$ is the traditional notation of the volumetric flow in this particular field.
	
	What is interesting to observe with the Poiseulle law in the case of the analysis of the pressure losses is that the volumic flow is therefore inversely proportional to the distance $l$ of the horizontal pipe. Thus, the loss of pressure and therefore the pressure at the end of a pipe is very easily calculated by rearranging the above relationship:
	
	So obviously the pressure at the end of a pipe is less than the pressure at its beginning.

	\pagebreak
	\subsubsection{Navier-Stokes Equations}\label{navier stokes equations}
	Let us consider an elementary parallelepiped extracted from a static fluid at equilibrium of dimensions $\mathrm{d}x$, $\mathrm{d}y$, $\mathrm{d}z$ represented in the figure below. The equilibrium material composing the parallelepiped is generally subjected to volumic forces in all directions (Pascal's theorem) whose components on the three orthogonal axes are represented in the figure below (these forces may be of Gravitational, Electromagnetic or Inertial origin...).
	\begin{figure}[H]
		\centering
		\includegraphics[scale=1]{img/mechanics/navier_stokes_elementary_parallelepiped.jpg}
		\caption{Elementary parallelepiped extracted from a static fluid at equilibrium}
	\end{figure}
	\begin{tcolorbox}[title=Remarks,colframe=black,arc=10pt]
	\textbf{R1.} It is important to notice that the components of all the vectors visible in the figure above are expressed in newtons per unit area or in other words per unit pressure (which is the unit for recall... of a constraint...).\\
	
	\textbf{R2.} It is important to be very attentive to what follows because some of the results we will get here will be reused in the section of General Relativity to understand the energy-momentum tensor!!
	\end{tcolorbox}
	We can, as we have represented it above, decompose and translate all the forces to which the parallelepiped is subjected to the centers of the faces of the latter. We obviously represent each of the constraints on each of the faces as the sum of the normal and tangential stresses as we did for the study of constrained solids (according to the three axes always, hence the sum of three components! ).

	In total, we find ourselves with $18$ components of normal and tangential stresses:
	
	We try to minimize the number of normal components in order to determine the sufficient constraints on each of the axes. Therefore we will put:
	
	Thus, three components are sufficient to know the normal stress forces on the surfaces along each axis.

	If we make the sum of the moments of forces with respect to the centers of gravity for each axis of symmetry of the parallelepiped $(XX', YY', ZZ')$, it is evident that of the $12$ tangential components $6$ are sufficient to describe the whole of system.
	
	Therefore for the $X$O$Y$ plane passing through the center of gravity we have:
	
	For the $X$O$Z$ plane:
	
	For the $Z$O$Y$ plane:
	
	We can obtain the same components of equilibrium  by considering this time a static regular elementary tetrahedron (extracted from the cube). The aim is to demonstrate that we find the $6$ components determined previously.
	\begin{figure}[H]
		\centering
		\includegraphics[scale=1]{img/mechanics/navier_stokes_elementary_tetrahedron.jpg}
		\caption{Regular elemental tetrahedron extracted from a fluid at equilibrium}
	\end{figure}
	\begin{tcolorbox}[title=Remark,colframe=black,arc=10pt]
	It is important to notice again that the components of all the vectors visible in the figure above are expressed in newtons per unit area or in other words per unit of pressure (which is the unit of constraints for recall...).
	\end{tcolorbox}
	To know the surface of the faces O$AC$, O$BC$, O$AB$, we multiply the surface $ABC$ (denoted hereafter: $S$) by the cosine of the angle that form the vectors $\vec{\sigma}_i$ and $\vec{N}$.

	Indeed, given the surfaces:
	
	However, let us try to express the $S_i$ according to $S$. The schema below (slice of the tetrahedron) should help to understand the reasoning:
	
	\begin{figure}[H]
		\centering
		\includegraphics[scale=1]{img/mechanics/navier_stokes_elementary_tetrahedron_slice.jpg}
	\end{figure}
	and therefore:
	
	Finally:
	
	The ratio:
	
	hence:
	
	The principle of analysis being the same for all other surfaces such as:
	
	We will write:
	
	such that:
	
	\begin{tcolorbox}[title=Remark,colframe=black,arc=10pt]
	We can easily know the values of the $\theta_i$ using vector analysis. Indeed, the plane $ABC$ being of equation (\SeeChapter{see section Analytical Geometry page \pageref{equation of the plane}}):
	
	by simplifying by $L_1$:
	
	The vector normal to the plane being indeed
	
	To know the cosines of the angle of the normal vector with the $\vec{\sigma}_i$, it is enough to assimilate these to the basis vectors $(\vec{i},\vec{j},\vec{k})$ such that (elementary trigonometry):
	
	and proceeding in the same way for all the other $\theta$.
	\end{tcolorbox}
	The equilibrium of forces gives us:
	
	After simplification:
	
	Following to the other axes:
	
	In summary:
	
	Using the matrix representation, we get:
	
	Either in index notation the normal and tangential constraints are given by the following traditional relation (where we do not distinguish what is tangential from what is normal so there is a loss of clarity that is typical habit of solid states physicists...):
	
	We see appear a mathematical quantity $\sigma_{ij}$ having $9$ components, whereas a vector in the same space $\mathbb{R}^3$ possesses $3$  components. We know this kind of mathematical being that we have already studied in the in the section of Tensor Calculus. The quantity $\sigma_{ij}$ is named "\NewTerm{second order stress tensor}\index{second order stress tensors}" or "\NewTerm{Cauchy stress tensor}\index{Cauchy stress tensor}\label{cauchy stress tensor}". Moreover, some components can be equal ($\sigma_{ij}=\sigma_{ji}$, if $i\neq j$), which would make it symmetrical. It then has only the $6$ distinct components, relative to the numbers of components sufficient to completely describe a system at equilibrium.

	To study the deformations of a continuous medium such as a fluid, we will first consider the case of very small deformations. The small displacements $\vec{D}$ of a point will be represented by $(u, v, w)$ parallel to the axes of a referential O$XYZ$. We assume that these components are very small quantities continuously varying in the volume of the considerated body.

	Let us consider a linear segment $\overline{\text{O}P}$ located in a solid before deformation. In an O$XYZ$ repository, we will denote by $(x_0,y_0,z_0)$ and $(x_p,y_p,z_p)$ the coordinates of O and $P$.

	During the deformation, the line $\overline{\text{O}P}$ becomes $\overline{\text{O}'P'}$ as shown below:
	\begin{figure}[H]
		\centering
		\includegraphics[scale=1]{img/mechanics/navier_stokes_deformation_line_before_after.jpg}
		\caption[]{Linear segment in a solid of fluid before and after deformation}
	\end{figure}
	Given $u_0$, $v_0$, $w_0$ the displacements of the point O parallel to the axes O$X$, O$Y$, O$Z$ and $u_p$, $v_p$, $w_p$ the displacements of the point $P$ parallel to the same axes.

	The coordinates of the points O' and $P'$ are then:
	
	Before deformation, given $L$ the length $\overline{\text{O}P}$:
	
	After deformation, we have a length $L'$ being equal to:
	
	If $\Delta L$ is the elongation of the element $\overline{\text{O}P}$ during the deformation, we have:
	
	By carrying out the following transformations:
	
	By developing:
	
	Therefore:
	
	By neglecting the terms of displacement of higher order and taking into account the relation:
	
	It comes that $L^2$ disappears with $(x_p-x_0)^2+(y_p-y_0)^2+(z_p-z_0)^2$ as well as the terms squared, we have then:
	
	But the Analytical Geometry (elementary trigonometry: ratio of the sides opposite and adjacent to the hypotenuse) gives the following relations:
	
	which are the direction cosines of line $L$.

	We can then write:
	
	The variation $u_p$ being a small displacement, we use a Taylor series development (\SeeChapter{see section Sequences and Series page \pageref{multivariate taylor series}}) whose upper order terms are neglected (linearization of equations):
	
	We also have:
	
	The difference gives:
	
	So we can now write:
	
	Finally:
	
	By grouping, we have:
	
	This expression allows, at any point, the computation of the deformation $\varepsilon$ in a direction having as cosine direction $l$, $m$, $n$ as a function of the displacements $u$, $v$, $w$ at this point.

	Let us consider the case where the line $L$ coincides with the axis O$X$, we have $l=1$, $m=n=0$ the preceding equation then becomes:
	
	We have, if $L$ coincides with the O$Y$ axis ($m=1$, $l=n=0$) or with the O$Z$ axis ($n=1$, $l=m=0$):
	
	The quantities $\varepsilon_x$, $\varepsilon_y$, $\varepsilon_z$  are named "\NewTerm{normal deformations}\index{normal deformations}" and have no dimensional units.

	For the interpretation of the terms $(\partial_z v+\partial_y w)$, $(\partial_z u+\partial_x w)$, $(\partial_y u+\partial_x v)$ , we refer to the following figure:
	\begin{figure}[H]
		\centering
		\includegraphics[scale=0.8]{img/mechanics/navier_stokes_shear_stresse_emergence.jpg}
		\caption[]{Situation that allows to make emerge shear stresses}
	\end{figure}
	Given two line segments $\overline{\text{O}R}$ and $\overline{\text{O}Q}$ located in the plane $X$O$Y$. Before deformation $\overline{\text{O}R}$ and $\overline{\text{O}Q}$ coincided with the orthonormal referential $Y$O$X$. After deformation, they can take the position $\overline{\text{O}'R'}$ and $\overline{\text{O}'Q'}$. The components of the displacement of O are $u$ and $v$.

	\begin{itemize}
		\item The displacement component of $R'$ is calculated as follows:
		
		with:
		
		as the angle is small.

		In general, as $u=f(x,y,z)$, we will write:
		

		\item The component of the displacement of $Q'$ is it:
		
	\end{itemize}
	As before deformation, the angle $\widehat{Q\text{O}R}$ is of $\pi/2$, after deformation, the right angle is reduced of a quantity $\alpha+\beta$. This reduction equation is named the "\NewTerm{shear deformation}\index{shear deformation}" or "\NewTerm{tangential deformation}\index{tangential deformation}" and is denoted by $\gamma_{xy}$.

	We will proceed in the same way for the other terms, hence:
	
	Seeing what precedes, it is customary to define the "\NewTerm{matrix of differential operators}\index{matrix of differential operators}":
	
	Given the quadruplet of groups of equations previously demonstrated in this section (see deformations of solids):
	
	We can summarize:
	
	At the view of the previous relations, it is customary to define the "\NewTerm{matrix of transformation of deformation stresses}":
	
	where it is possible to invert the matrix to thus obtain the "\NewTerm{matrix of transformation of the stresses strains}\index{matrix of transformation of the stresses strains}".

	Generally, we put to simplify the notations (the reader must however not believe that the shear deformation becomes a normal deformation!!!! it is only a convention of writing that must be remembered!):
	
	Similarly, we put:
	
	Thus finally:
	
	Taking into account that:
	
	We get the shear tensions as follows:
	
	Consider now, for example, a fluid flowing in the direction of O$Y$ with a velocity gradient in the direction of $x$:
	\begin{figure}[H]
		\centering
		\includegraphics[scale=1]{img/mechanics/navier_stokes_normal_stresse_emergence.jpg}
		\caption{Situation that allows the emergence of normal constraints}
	\end{figure}
	By placing ourselves at the level of $y$ and at point $1$ of the abscissa $x$, we have a velocity $v_y$ and at point $2$ of abscissa $x + \mathrm{d}x$, a speed:
	
	with $\partial_x v_y\neq 0$.
	
	In the direction of $x$, there is no velocity component therefore:
	
	with $\partial_y v_x=0$.
	
	We now assume that the shear tensions are proportional to $\partial_x v_y$ to a given factor such that:
	
	with for recall:
	
	It is therefore possible to consider displacements per unit of time by putting:
	
	By relating the previous relation of $\mu_d$ of:
	
	we can say that $G$ initially valid in a solid elastic medium considered by its displacements is the analogue of $\mu_d$ in the case of a viscous fluid considered by its displacements per unit of time. So we see that the units are conserved.

	Considering also the deformations per unit of time for the normal stresses (we will return to this in detail later), we then have the system of equations:
	
	Thus, we get a condensed notation:
	
	where $\delta_{ij}$ is the Kronecker's symbol:
	
	The tensor $P_{ij}$ thus describes in part the set of stresses of a viscous fluid in which we assumed, within the framework of the Newtonian fluid hypothesis, that there are linear relations between the tensions and the normal deformations!!!!

	We now put the sum of dynamic constraints in a general form that we will justify:
	
	where the term $p\delta_{ij}$ is justified by the fact that in the static case a constant dynamic pressure $p$ always exists at a point of a fluid, situation that we do not have in the case of a solid. To justify the negative sign, we observe that in the expression of $P_{ij}$, the first two terms of the right-hand side correspond, in the previous study, to extension constraints while the pressure $p$ corresponds to a compression of the fluid.
	
	We now have to determine the coefficient $\alpha$. Given $i=j$, then we have $\delta_{ij}=1$. It comes successively and by addition:
	
	This expression must satisfy also a fluid which is also in a static (at rest) situation such that:
	
	It comes then that in the static case:
	
	Since:
	
	We have then:	
	
	The general expression of the constraints is then written:
	
	Presently, we will introduce the operators of the vector analysis in order to have a more general expression. In this way, we can adapt the formulation to any coordinate system (Cartesian, cylindrical, spherical, etc.) which will facilitate the solving of practical problems.
	
	We have seen that for a solid we had:
	
	We will determine these equations in an index form by always considering displacements per unit of time (velocities).
	
	such that $i,j=\{1,2,3\}$ and that:
	
	For $i=j=1$, we have therefore:
	
	or:
	
	
	By doing the sum of the terms of the normal deformations rate we get:
	
	But, the tools of Vector Calculus allow us to write:
	
	For a fluid, we will then have by analogy:
	
	The general dynamic equationofstresses will then be written in the following form of a Newtonian fluid (under small deformations):
	
	This stress tensors is written by some teachers and textbooks authors as following (this notion is a little bit dangerous but it can be justified in the context of an in-deep study of fluids dynamics):
	
	or also to differentiate vector and tensor:
	
	If the normal constraints are negligible (incompressible fluid) the second term simplifies and we have then (relation that we will use in the section of Marine \& Weather Engineering:
	
	It is now useful to go back in a developed form for the preceding equation, remembering that (see above):
	
	Let us now write the system of Newton equations (sums of dynamic internal and external constraints to a volume element of a fluid) which is:
	
	where:
	\begin{itemize}
		\item $\dfrac{\partial}{\partial x_i}\displaystyle\sum_{j=1}^3 P_{ij}$ is the sum of external forces by unit volume
		
		\item $f_i$ is the traditional (unfortunate ...) notation of the mass acceleration in $[\text{m}\cdot \text{s}^{-2}]$

		\item $\rho$ is the fluid density
	\end{itemize}
	The previous relation can be rewritten in vector form:
	
	with obviously using tensor summation convention:
	
	Then we have:
	
	By introducing the expressions of $P_{ij}$ obtained in the above relation, we arrive at the equations:
	
	These are the "\NewTerm{Navier-Stokes equations of the dynamics of Newtonian fluids}\index{Navier-Stokes equations of the dynamics of Newtonian fluids}" for small deformations. There are two condensed forms which we shall now proceed to determine:

	By taking Navier-Stokes's first equation and expanding it, it comes:
	
	As:
	
	and that:
	
	We get:
	
	Simplifying, it comes therefore:
	
	By operating in the same way for the other two components, we can reduce the system of Navier-Stokes equations to a single vector equation:
	
	As (\SeeChapter{see section Vector Calculus page \pageref{laplacian of vector fields} and page \pageref{differential operators identities}}) the vector Laplacian is given by:
	
	We have:
	
	Thus finally:
	
	\begin{tcolorbox}[title=Remark,colframe=black,arc=10pt]
	We can also sometimes find in the literature an equation containing a second viscosity ${\mu'}_d$, whereas $\mu_d$ only manifests itself rigorously during pure shearing according to our hypotheses, ${\mu'}_d$ appears during an omnidirectional compression accompanied by a variation of density.
	\end{tcolorbox}
	The preceding equation is then written:
	
	This is the "\NewTerm{Navier-Stokes equation}\index{Navier-Stokes equation}" (still for Newtonian fluids in small deformations) or also named "\NewTerm{equation of motion for a Newtonian fluid}\index{equation of motion for a Newtonian fluid}".
	
	\paragraph{Incompressible flow}\mbox{}\\\\
	In fluid mechanics or more generally continuum mechanics, an "\NewTerm{incompressible flow}\index{incompressible flow}" or "\NewTerm{isochoric flow}\index{isochoric flow}" refers to a flow in which the material density is constant within a fluid parcel an infinitesimal volume that moves with the flow velocity. An equivalent statement that implies incompressibility is that the divergence of the flow velocity is zero (see the derivation below, which illustrates why these conditions are equivalent).

	Incompressible flow does not imply that the fluid itself is incompressible. It is shown in the derivation below that (under the right conditions) even compressible fluids can – to a good approximation – be modelled as an incompressible flow. Incompressible flow implies that the density remains constant within a parcel of fluid that moves with the flow velocity.
	
	In an incompressible fluid, we have by definition $\rho=c^{te}$. The conservation equation which is (\SeeChapter{see section  Thermodynamics page \pageref{mass conservation equation}}):
	
	is then written:
	
	Therefore the divergence of the velocity field is zero:
	
	The Navier-Stokes equation in the form:
	
	is then written:
	
	or differently:
	
	If, moreover the viscosity $\mu_d$ is negligible, we have for a perfect fluid:
	
	This equation is named the "\NewTerm{Euler equation of the 1st form}\index{Euler equation of the 1st form}" or "\NewTerm{local equation of the balance conservation of the momentum}\index{local equation of the balance conservation of the momentum}". We will reuse this relation as part of our study of gravity waves in the section of Weather \& Marine Engineering and also in the section of Cosmology to study the Einstein Universe models!
	\begin{tcolorbox}[title=Remark,colframe=black,arc=10pt]
	It was proved in 1996 by Vladimir Scheffer (but the demonstration is horribly complex and comprehensible only by some people on the planet) that the Euler equation in the plan allows a spontaneous creation of energy! So an energy creation from nothing! A year later another mathematician presents a new proof of the same result. A few years later, two young mathematicians proved that it was useless to impose a criterion to the solutions to avoid this paradox ...
	\end{tcolorbox}
	There exists a second form of the Euler equation in the framework of an incompressible fluid and of negligible viscosity which we will immediately determine (often used in industry):

	If $\vec{v}=f(x,y,z,t)$ we can write:
	
	This can also be written:
	
	What is also written:
	
	The first factor can be considered as the following scalar product:
	
	Thus:
	
	The "\NewTerm{material derivative}\index{material derivative}\label{material derivative}" (there are many other names for the material derivative\footnote{advective derivative, convective derivative, derivative following the motion,  hydrodynamic derivative, Lagrangian derivative, Lagrangian acceleration, particle derivative, substantial derivative, substantive derivative, Stokes derivative, total derivative, ...}) can then take the following condensed form (we thus find the same result as in our study of the second generalized  Newton force proved in the section of Classical Mechanics!!!):
	
	That is often written in text books:
	
	where:
	\begin{itemize}
		\item $\vec{\nabla}\circ\vec{v}$ is the "\NewTerm{advection term}\index{advection term}\label{advection term}". Advection is the transport of a scalar or vectorial quantity preserved by a vector field. An example is the transport of polluting matter by the flow of a river. In meteorology and oceanography, advection refers mainly to the horizontal transport of certain properties by the fluids considered, including transport by wind or currents: advection of water vapor, heat, salinity, etc.

		\item $\vec{v}\vec{\nabla}\circ\vec{v}$ is the "\NewTerm{convective acceleration}\index{convective acceleration}". Convective is the acceleration defined as the rate of change of velocity due to the change of position of fluid particles in a fluid flow.
	\end{itemize}
	\begin{tcolorbox}[title=Remark,colframe=black,arc=10pt]
	The $x$ component of the material derivative is therefore (we will use this again in the section of Marine \& Weather Engineering):
	
	what some experts in the field generally note for any component:
	
	\end{tcolorbox}
	The Euler equation of the first form:
	
	becomes taking into account the material derivative:
	
	Or (common form in the literature):
	
	We have proved in the section of Vector Calculus that:
	
	If we put $\vec{u}=\vec{v}$, we have:
	
	Therefore:
	
	Finally, we get a new equation named the "\NewTerm{Euler equation of the $2$nd form}\index{Euler equation of the $2$nd form}" which is written:
	
	Although the two Euler equations of are very important, there exists a variation form very useful in meteorology which we will now determine.

	We always rely on the flow of an incompressible and non-viscous fluid, but whose volumic forces are derived this time from a potential $\vec{f}=-\vec{\nabla}U$ where $U$ is typicall the gravitional potentiel for example.

	In this case, we use the Euler equation in its first form:
	
	Since the volumic forces $\vec{f}$ derive from a potential $U$, we have:
	
	Let us recall the relation (\SeeChapter{see section Vector Calculus page \pageref{differential operators identities}}):
	
	Given $\vec{\nabla}\times\vec{v}$ be a vector $\vec{A}$, it comes:
	
	Therefore:
	
	so we can also write:
	
	By taking up the material derivative:
	
	the equation:
	
	then becomes:
	
	and using:
	
	the latter becomes:
	
	and since:
	
	we can finally write:
	
	Let us generalize this last relation by making emerge possible rotations. For this we know that (\SeeChapter{see section Classical Mechancis page \pageref{kinematics of circural motion}}):
	
	therefore:
	
	By writing the vector product $\vec{\omega}\times\vec{r}$ in developed form, we have:
	
	Which gives:
	Which gives:
	
	Let us suppose that $\vec{\omega}$ is a constant angular velocity vector, we then have:
	
	\textbf{Definitions (\#\mydef):} We say that a "\NewTerm{vortex flow}" if the rotation of the velocity field is non-zero:
	
	everywhere or at certain points. We also define from the prior-previous relationship the "\NewTerm{vorticity}" by:
	
	Example of partially vortex flow simulation (at some points):
	\begin{figure}[H]
		\centering
		\includegraphics[scale=0.8]{img/mechanics/vortex_flow.jpg}
		\caption{Example of a simulated vortex flow}
	\end{figure}
	and in real life at large scale levels\footnote{This a rebuild pic of ocean surface currents between June 2005 to December 2007 from NASA satellites on our planet. The Pacific carry warm waters across thousands of kilometers at speeds greater than six kilometers per hour; coastal currents like the Agulhas in the Southern Hemisphere move equatorial waters toward Earth's poles; and how thousands of other ocean currents are confined to particular regions and form slow-moving, circular pools.} (Jupiter has also a famous one...):
	\begin{figure}[H]
		\centering
		\includegraphics[scale=0.8]{img/mechanics/nasa_flow_ocean_earth.jpg}
		\caption[Earth currents and vortex between June 2005 to December 2007]{Earth currents and vortex between June 2005 to December 2007 (source: NASA/SVS}
	\end{figure}
	
	The equation:
	
	is then written:
	
	We find in this equation, used in meteorology, the Coriolis  acceleration that we had determined in the section of Classical Mechanics!
	
	If the flow is at constant velocity then:
	
	and the fluid is then not rotational (non-turbulent):
	
	then the above equation reduces to:
	
	In classical dynamics of the rigid material point (\SeeChapter{see section Classical Mechanics page \pageref{gravitational potential energy}}), we have shown that in the case of a terrestrial gravitational potential:
	
	$z$ being the altitude of a point of the fluid with respect to an reference level $z_0$. If we take $z_0$ for the ground level, the prior-previous relation becomes then in the case of a flow then say to be "\NewTerm{potential flow}\index{potential flow}":
	
	The term in brackets to satisfy this relationship must be such that:
	
	We thus fall back well on Bernoulli's theorem, which confirms our model of Newtonian fluids according to the Navier-Stokes model.

	If the flow is inversely a vortex flow, such that:
	
	Now, as we have proved in the section of Vector Calculus, if the rotational of a vector variable is zero, the variable can be expressed as the gradient of a scalar potential! Consequently, in the case of the vortex, the velocity field then derives from a velocity potential:
	
	And if in addition the fluid is incompressible, as we saw above:
	
	Then by combining the last two relations it comes (divergence of the gradient):
	
	Thus explicitly:
	
	The potential of the velocities in the case of an uncompressible and non-vortex liquid thus verifies the Laplace's equation:
	
	
	\pagebreak
	\paragraph{Compressible flow}\mbox{}\\\\
	In this case $\rho_0$ is a function of the pressure $p$ (case of the "\NewTerm{barotropic fluids}\index{barotropic fluids}"). We also consider that the viscosity is negligible. He then comes:
	
	The equation:
	
	is then written:
	
	
	\paragraph{Static flow}\mbox{}\\\\
	The fluid on which the net force is zero is considered as a "\NewTerm{static flow}\index{static flow}". Means the bulk of fluid moving with constant velocity would be a static fluid. This also includes the stationary fluid. i.e. zero velocity. Non ideal fluids (with viscocity and turbulence present) would never have a flow with constant bulk velocity due to presence of non zero shear forces. But in some cases, like the fluid in a tank moving with constant acceleration is considered as static case with the assumption that the internal motion within the fluid is negligible.
	
	In the static case:
	
	and the equation:
	
	becomes simply:
	
	which is the "\NewTerm{equation of the static fluid}\index{equation of the static fluid}" or the "\NewTerm{fundamental law of the hydrostatic}\index{fundamental law of the hydrostatic}". In discrete and more classical (rearranged) form, we fall back on the relation that must textbooks provides and that we have also proved earlier above using another approach (gradient theorem):
	
	or by taking the initial height as being equal to zero:
	
	\begin{tcolorbox}[title=Remark,colframe=black,arc=10pt]
	As the viscosities disappear, the static fluids are the same for viscous or non-viscous fluids.
	\end{tcolorbox}
	
	\paragraph{Reynolds number}\mbox{}\\\\
	The "\NewTerm{Reynolds number (Re)}\index{Reynolds number}" is an important dimensionless quantity in fluid mechanics that is used to help predict flow patterns in different fluid flow situations. It is widely used in many applications ranging from liquid flow in a pipe to the passage of air over an aircraft wing. The Reynolds Number is valuable as a guide to the transition point from laminar to turbulent flow in a particular flow situation, and for the scaling of similar but different-sized flow situations, such as between an aircraft model in a wind tunnel and the full size version.
	
	Let us consider first, for simplicity, the incompressible case.

	The equation of continuity, or conservation of mass, (\SeeChapter{see section Thermodynamics page \pageref{mass conservation equation}}) that is for recall:
	
	is then written in this particular case:
	
	We now choose several quantities of dimensionless references denoted by an index $r$ such that:
	
	By these definitions, we have for example:
	
	Therefore the equation of the deformations per unit of time becomes:
	
	But we also have:
	
	Let us restrict ourselves to the study of only one component:
	
	By multiplying this last relation by the density $\rho$ and by definition of the speed we get:
	
	Let us now take one of the possible formulations of the Navier-Stokes equation proved earlier above:
	
	By not forgetting that for an incompressible fluid we have:
	
	the previous Navier-Stokes equation, reduces then to:
	
	But, for a fluid we had assumed above that the shear tensions were given by:
	
	The terms where the viscosity coefficients appear can be rewritten such that:
	
	Thus by correspondence:
	
	that we can write in even more condensed form by using a rather abusive tensor notation:
	
	By introducing the dimensionless variables:
	
	Let us now multiply this last relation by $\dfrac{L_r}{v^2}$ and let us divide it by $\rho_r$ on both sides of equality such that it becomes:
	
	At the dimensional level, let us notice that we have:
	
	Finally:
	
	This differential equation expressed in relative and dimensionless variables is named the "\NewTerm{dimensionless Navier-Stokes-Reynolds equation}\index{dimensionless Navier-Stokes-Reynolds equation}".

	The term Re, as we know is the Reynolds number, symbolically represented the ratio of the inertial forces to the viscous forces:
	
	where $\mu_{rc}$ is the "\NewTerm{relative kinematic viscosity}\index{relative kinematic viscosity}". The dynamic viscosity is therefore a term inversely proportional to the value of the Reynolds number.

	We also often find this last relation in the following form (term-to-term identification with the first equation of the previous expression where $2R$ is the diameter of the circular section of the tube that transports the fluid):
	
	If instead of the velocity it is the mass flow which is measured through a circular section, we then have using the following relation between the mass flow rate and fluid velocity in a cylindrical tube:
	
	Another writing for the prior-previous relation that we will often using in industrial practice:
	
	In practice we regularly calculate the time it would take to fill a given volume tank with a fluid having a known mass flow rate, a known  density, a known  viscosity and a known number of Reynolds in a cylindrical pipe whose radii are also known. Therefore it comes:
	
	
	\pagebreak
	\paragraph{Boussinesq approximation (buoyancy)}\mbox{}\\\\
	In fluid dynamics, the Boussinesq approximation is used in the field of buoyancy-driven flow (also known as natural convection). It ignores density differences except where they appear in terms multiplied by $g$, the acceleration due to gravity. The essence of the Boussinesq approximation is that the difference in inertia is negligible but gravity is sufficiently strong to make the specific weight appreciably different between the two fluids. Sound waves are impossible/neglected when the Boussinesq approximation is used since sound waves move via density variations.

	Boussinesq flows are common in nature (such as atmospheric fronts, oceanic circulation, katabatic winds), industry (dense gas dispersion, fume cupboard ventilation), and the built environment (natural ventilation, central heating). The approximation is extremely accurate for many such flows, and makes the mathematics and physics simpler.
	
	Given the relation already proved previously:
	
	Putting in it again the term containing the viscosity:
	
	without forgetting that at the notation level (we know ... it's a bit annoying):
	
	If the potential is a gravitational one, it goes without saying that:
	
	Therefore:
	
	
	If we can consider the context of the experiment such that the volume density is less than or equal to that of the water and the velocities are small, then we can eliminate the second degree terms such that the previous relation will be written:
	
	We place ourselves within the framework of a weakly turbulent fluid, in which the pressure and the density are written:
	
	where $p'$, $\rho'$ represents the turbulent increasing terms with respect to the static values of the fluid.

	We also neglect the friction on the bounds and therefore the viscosity, assuming that the effect of the turbulence quickly becomes preponderant on the value of the friction.

	So we have the system of equations:
	
	that can be written:
	
	and even:
	
	which can also written:
	
	But in the static case we know (we proved it many times) that we have:
	
	It then remains to us:
	
	Therefore it remains:
	
	By dividing the whole by $\rho$:
	
	but again we can write this:
	
	The Boussinesq approximation  consisting of supposing that the fluid is incompressible and that the system is at constant temperature and not very turbulent, we have:
	
	Which gives us:
	
	This equation is named the "\NewTerm{Boussinesq equation}\index{Boussinesq equation}" and will allow us to introduce the theory of chaos in the field of meteorology and fluids in the particular case of convection cells (\SeeChapter{see section Weather \& Marine Engineering page \pageref{convection cells}}).
	
	\paragraph{Stokes' law}\label{stoke law}\mbox{}\\\\
	In 1851, George Gabriel Stokes derived an expression, now known as Stokes' law, for the frictional force - also named drag force - exerted on spherical objects with very small Reynolds numbers (i.e. very small particles) in a viscous fluid. Stokes' law is derived by solving the Stokes flow limit for small Reynolds numbers of the Navier–Stokes equations.
	
	The complexity of hydrodynamics is a suitable framework for the application of dimensional analysis which we introduced at the very beginning of our study of Mechanics (\SeeChapter{see section Principia page \pageref{dimensional analysis})}. The example analyzed here clearly shows the possibilities, but also the limitations of this method.
	
	We consider a solid of any form immersed in an incompressible fluid with a uniform velocity at great distance (the problem is equivalent to that of a solid which moves at a constant velocity in a fluid at rest). We try to express the force $F$ that the fluid exerts on the obstacle, supposed to be immobile (and in particular devoid of any rotational motion).

	The analytical solution is too complex to waste time solving this kind of practical problem. Dimensional analysis should be used.

	The relevant parameters are in our study:
	\begin{itemize}
		\item $L$ the linear dimension of the obstacle

		\item $v$ the fluid velocity at far distances

		\item $\rho$ the fluid density

		\item $\mu$ the viscosity coefficient of the fluid
	\end{itemize}
	As a matter of fact, all these parameters are constants, even if the velocity varies in direction and norm in the vicinity of the obstacle: at long distances, it is uniform and its value $v$ is indeed a relevant parameter.

	We might ask ourselves if pressure should not be one of those parameters. This is not the case. The pressure is conditioned by the value of the velocity and by those of the constant parameters as we saw in Bernoulli's theorem. There is no need to add a redundant term.

	Without seeking the unique combination without dimension of the first four, we apply the systematic approach. We want to determine $A$, $B$, $C$, $D$ such that:
	
	As:
	
	It comes:
	
	The dimensionality system is then written:
	
	Therefore:
	
	Hence:
	
	and curiously we fall back here in the parenthesis on what we had seen in our development of the Reynolds number:
		
	Thus the force exerted by the fluid is then written:
		
	In the literature, we find the notation:
	
	Where $C$ depends on obviously on Re. We also find this relation in some textbooks in the following the condensed form:
	
	where appears the "\NewTerm{drag coefficient $C_D$}\index{drag coefficient}" which is determined in wind tunnels or experimentally based on the fact that an object in free fall in a perfect liquid or gas will follow the differential equation:
	
	Differential equation that we can solve ingeniously by writing:
	
	It then comes after rearrangement of the terms:
	
	Either after integration:
	
	Then it comes a relation which makes it possible to determine experimentally the drag coefficient for an object in free fall without too many difficulties:
	
	\begin{tcolorbox}[title=Remark,colframe=black,arc=10pt]
	 Obviously $C_D$ contains a mass unit, so nothing forbid us to extract a mass term and also a factor $1/2$, and also the surface of the object $S$ (obviously we can guess that the surface has an incidence on the drag force..) to introduce a new drag coefficient denoted $C$ such that:
	
	Then the "\NewTerm{terminal velocity}\index{terminal velocity}" (equilibrium between the gravitational force and drag force of a falling object in gas or liquid) is simply given by:
	
	Hence:
	
	\end{tcolorbox}
	\begin{tcolorbox}[colframe=black,colback=white,sharp corners]
	\textbf{{\Large \ding{45}}Example:}\\\\
	To keep an object in motion in the presence of drag (aerodynamic or otherwise) requires an ongoing input of energy. Work must be done over some time. Power must be used. Recall the following chain of reasoning that starts from the definition of power as the rate at which work is done:
	
	or more specifically, in the case of pressure drag:
	
	Therefore we have for maximum speed:
	
	\end{tcolorbox}
	\begin{tcolorbox}[colframe=black,colback=white,sharp corners]
	\begin{figure}[H]
		\centering
		\includegraphics[width=1.0\textwidth]{img/mechanics/koenigsegg_agera_rs.jpg}
		\caption[]{Koenigsegg Agera RS theoretical maximum speed (source: Real Engineering/YouTube)}
	\end{figure}
	\end{tcolorbox}
	The limits of the dimensional analytical method (and just analytic one...) appear when we confront this model with the experiment (obviously we could make numerical models of the Navier-Stokes-Reynolds equation for the computer and thus the honor would be safe):
	\begin{figure}[H]
		\centering
		\includegraphics[scale=0.9]{img/mechanics/drag_coefficient_sphere.jpg}
		\caption{Effect of Reynolds number on the drag coefficient $C_D$ of a smooth sphere of diameter $d$}
	\end{figure}
	or more explicitly:
	\begin{figure}[H]
		\centering
		\includegraphics[scale=0.8]{img/mechanics/drag_coefficient_sphere_and_regimes.jpg}
		\caption{Effect of Reynolds number on the drag coefficient $C_D$ of a smooth sphere with regime}
	\end{figure}
	Or comparing the drag coefficient of a sphere and a cylinder:
	\begin{figure}[H]
		\centering
		\includegraphics[scale=1]{img/mechanics/drag_coefficient_sphere_vs_cylinder.jpg}
		\caption{Drag coefficient comparison of a sphere and cylinder}
	\end{figure}
	The curves above have two remarkable characteristics:
	\begin{enumerate}
		\item It was obtained by independently modifying the values of the four parameters. We find that $C_D$ depends only on the Reynolds number Re: it is a success of the dimensional analysis!

		\item It seems useless to hope to find a simple analytic function which reproduces the experimental curve. We must therefore take a closer look at the various regimes corresponding to this complex curve.
	\end{enumerate}
	The figure below schematizes the flow of a viscous fluid around a cylinder for different values of the Reynolds number:
	\begin{figure}[H]
		\centering
		\includegraphics[scale=1]{img/mechanics/reynolds_number_cylinder.jpg}
		\caption{Various fluid regimes for cylinder with corresponding Reynolds number}
	\end{figure}
	The regime corresponding to figure (a) is name "\NewTerm{steady state}\index{steady state}" or "\NewTerm{steady regime}\index{steady regime}". We can speak of a "quasi-static" movement on each point of the fluid where the acceleration is negligible. We must therefore expect that the inertia of the fluid does not intervene in the expression of force. For this, it is necessary and sufficient that:
	
	where $C$ is independent of Re.

	Therefore we have:
	
	The parameter $C'$ without dimensions can depend only on the geometry of the obstacle. In the case where the obstacle is spherical (very important case in physics with $L = R$), $C'$ has been determined experimentally as being equall to $6\pi$ such that:
	
	known as the "\NewTerm{Stokes' Law}\index{Stokes' law}" or "\NewTerm{Stokes formula}\index{Stokes formula}\index{Stokes formula}". Attention ... this law only applies well for small speeds and small spheres.

	In the regime described by (b), two vortices takes place symmetrically behind the cylinder. When Re increases beyond $40$, we distinguish the alley of "\NewTerm{Kármán vortex street}\index{Kármán vortex street}".
	\begin{figure}[H]
		\centering
		\includegraphics[scale=1]{img/mechanics/vortex_street_cylinder.jpg}
		\caption[Visualisation of the vortex street behind a circular cylinder in air]{Visualisation of the vortex street behind a circular cylinder in air; the flow is made visible through release of oil vapor in the air near the cylinder (source: Wikipedia, author: Jürgen Wagner}
	\end{figure}
	And at the macroscopic level on Earth typically (there is quantity of such photos on the Internet):
	\begin{figure}[H]
		\centering
		\includegraphics[scale=1]{img/mechanics/vortex_street_guadalupe.jpg}
		\caption[Visualization of the vortex street around Guadalupe Island]{Visualization of the vortex street around Guadalupe Island (source: NASA/GSFC/JPL, MISR Team}
	\end{figure}
	And at a planetary level:
	\begin{figure}[H]
		\centering
		\includegraphics[scale=0.2]{img/mechanics/vortex_street_jupiter.jpg}
		\caption{Visualisation of the vortex street around Jupiter red spot}
	\end{figure}
	
	
	\subsubsection{Hydrostatic Pressure}
	We have previously demonstrated without difficulty that:
	
	If the fluid velocity is zero:
	
	This gives in differential form:
	
	If we measure the liquid pressure from its upper area $z_0$:
	
	If we take $z_0$ as a reference, we can put that:
	
	hence:
	
	If we find ourselves in the case of a receptacle filled with a fluid in contact with the atmosphere, in order to calculate the pressure in this fluid at a given height $h$, we should take in consideration the atmospheric pressure, which also is "put" on the fluid. Thus, the "\NewTerm{hydrostatic pressure}\index{hydrostatic pressure}" is given by:
	
	The consequence of this relation is that in a liquid at rest, homogeneous, the equipotential gravific lines are confused with isobaric surfaces. Otherwise, there would be transverse movements.
	
	\begin{tcolorbox}[title=Remark,colframe=black,arc=10pt]
	 In the technical field study of pumps, tires or compressed air tanks, we measure the pressure with "\NewTerm{manometers}\index{manometers}" whose zero corresponds to the atmospheric pressure.
	\end{tcolorbox}
	
	Finally here are some common drag coefficient values:
		\begin{table}[H]
		\begin{center}
		\definecolor{gris}{gray}{0.85}
		\begin{tabular}{|c|c|}
		\hline
		\cellcolor{black!30}\textbf{$C_d$} & 
	 	\cellcolor{black!30}\textbf{Item} \\ \hline
		$0.001$ & Laminar flat plate parallel to the flow (Re<$10^{6}$)\\ \hline
		$0.005$ & Turbulent flat plate parallel to the flow (Re>$10^{6}$)\\ \hline
		$0.021$ & F-4 Phantom II (subsonic) \\ \hline
		$0.022$ & Learjet 24 \\ \hline
		$0.024$ & Boeing 787 \\ \hline
		$0.0265$ & Airbus A380 \\ \hline
		$0.022$ & Cessna 172/182 \\ \hline
		$0.031$ & Boeing 747 \\ \hline
		$0.044$ & F-4 Phantom II (supersonic) \\ \hline
		$0.048$ & F-104 Starfighter \\ \hline
		$0.075$ & Pac-car\\ \hline
		$0.095$ & X-15 (not confirmed) \\ \hline
		$0.1$ & Smooth sphere (Re=$10^6$)\\ \hline
		$0.15$ & Schlörwagen Model 1939 \\ \hline
		$0.18$ & Mercedes-Benz T80 Model 1939 \\ \hline
		$0.19$ & General Motors EV1 Model 1996 \\ \hline
		$0.212$ & Tatra 77A Model 1935 \\ \hline
		$0.24$ & Tesla Model S \\ \hline
		$0.25$ & Toyota Prius (4th Generation) \\ \hline
		$0.26$ & BMW i8 \\ \hline
		$0.28$ &  Dodge Charger Daytona Model 1969 \\ \hline
		$0.29$ & Mazda3 Model 2007 \\ \hline
		$0.295$ & bullet (not ogive, at subsonic velocity) \\ \hline
		$0.3$ & Audi 100 C3 Model 1983 \\ \hline
		$0.324$ & Ford Focus Mk2/2.5 Model 2004-2011 \\ \hline
		$0.35$ & Maserati Quattroporte V Model 2003–2012 \\ \hline
		$0.36$ & Citroen CX Model 1974-1991 \\ \hline
		$0.48$ & Volkswagen Beetle \\ \hline
		$0.58$ & Jeep Wrangler TJ Model 1997-2005 \\ \hline
		$0.75$ & Typical model rocket \\ \hline
		$1$ & Coffee filter, face-up \\ \hline
		$1-1.3$ & Wires and cables or typical adult human \\ \hline
		$1.28$ & Flat plate perpendicular to flow  \\ \hline
		$1.3-1.5$ & Empire State Building  \\ \hline
		$1.8-2.0$ & Eiffel Tower  \\ \hline
		\end{tabular}
		\end{center}
		\caption[Drag coefficient $C_D$ examples]{Drag coefficient $C_D$ examples (source: Wikipedia)}
	\end{table}
	
	\pagebreak
	\subsubsection{Archimedes' principle}
	The Archimedes' principle, almost worldwide known phenomenon ..., is often rebellious to the first intuition. By the way, we have too much tendency in schools to present the Archimedes force as a "principle" and this wrongly because a simple mathematical analysis is enough to prove it.

	Indeed, if we isolate an arbitrary portion $\Delta V$ of a fluid in static equilibrium, the conditions of this equilibrium are necessarily written (otherwise the volume dissociates and is no longer in static equilibrium):
	
	where $P$ denotes the weight ($\vec{P}=m\vec{a}$ in a first approximation ...) of equation whereas the term equation describes the resultant of the forces of pressure exerted on the surface of equation.
	
	where $P$ denotes the weight ($\vec{P}=m\vec{a}$ in a first approximation ...) of $\Delta V$ whereas the term $\oint (\vec{n}p)\mathrm{d}S$ describes the resultant of the forces of pressure exerted on the surface of $\Delta V$

	Each surface element $\mathrm{d}S$ therefore undergoes a force:
	
	where $p$ is the pressure applied locally on $\mathrm{d}S$. And for $\vec{n}$, we know that it is a unit vector directed normally (at the perpendicular) of $\mathrm{d}S$ but this time towards the interior of $\Delta S$. The resultant of all these forces is denoted historically as follows:
	
	which therefore expresses, as you can guess, the famous "\NewTerm{Archimedes' force}\index{Archimedes' force}" that the rest of the fluid exerts on the element. The integral bears on the whole surface (this surface is closed, hence the corresponding curvilinear integral) of the element $\Delta V$.
	
	The equilibrium condition therefore requires that:
	
	We easily understand that $\vec{F}_A$ is directed upwards: under the effect of the gravitational field and therefore $p$ increases with depth.

	If we replace the fluid contained in the volume with any fluid or solid object but that occupies the same volume, the Archimedes' force is not altered. Because of the relation $\vec{F}_A=-\vec{P}$ we are accustomed to say that it is equivalent to the weight of the displaced fluid.
	\begin{figure}[H]
		\centering
		\includegraphics[scale=0.9]{img/mechanics/archimedes_principle.jpg}
		\caption{Archimedes's principle}
	\end{figure}
	In other words, for an object floating on a liquid surface (like a boat) or floating submerged in a fluid (like a submarine in water or dirigible in air) the weight of the displaced liquid equals the weight of the object. 
	
	In the case where the direction and the intensity in time of $\vec{g}$ are uniform and constant we can write:
	
	and we find again the relation of the "\NewTerm{Archimedes' law}\index{Archimedes' law}" well known to all schoolchildren:
	
	There is another possibility to arrive at this demonstration which requires fewer mathematical tools and is therefore more affordable:
	
	For this list us consider a cylinder of volume $V$ immersed in a liquid at the vertical. The horizontal components of the pressure forces vanish, but the vertical component $P_1$ at the top of the cylinder (close to the surface) is less in intensity (except external cause) than that at its base $P_2$ We can therefore write:
	
	It is a bit simpler and it keeps in a single line without integrals ...

	It should be remembered that the Archimedes principle is a force which applies to fluids and therefore also to gases. It is thus thanks to Archimedes' force that a hot air balloon or an airship can rise in the air (in both cases, a gas of lower density than air is used, eg. heated air or helium).
	\begin{figure}[H]
		\centering
		\includegraphics[width=\textwidth]{img/mechanics/balloon.jpg}
		\caption[]{The hot air inside these balloons is less dense than the surrounding cool air. This results in a buoyant force that causes the balloons to rise when their guy lines are untied (source: Anthony Quintano)}
	\end{figure}
	It is also funny, after the proof of the law of perfect gases (see below), to determine the pressure that our atmosphere should have to have the same density as water and so that a human can then float in the air...
	
	The weightings that are made on a balance will, of course, give the weight in air in most laboratories. When an object displaces its volume in air, it will be buoyed up by the weight of air displaced according to Achimedes's principle. The density of air is $0.0012$ [g] ($1.2$ [mg]) per milliliter. If the density of the weights and the density of the object being weighed are the same, then they will be buoyed up by the same amount, and the recorded weight will be equal to the weight in a vacuum, where there is no buoyancy. If the densities are markedly different, the differences in the buoyancies will lead to a small error in the weighing: One will be buoyed up more than the other, and an unbalance will result. Such a situation arises in the weighing of very dense objects [e.g., platinum vessels (density = $21.4$) or mercury (density=$13.6$)] or light, bulky objects [e.g., water (density $\cong 1$)]; and in very careful work, a correction should be made for this error.

	Note that in most cases, a correction is not necessary because the error resulting from the buoyancy will cancel out in percent composition calculations. The same error will occur in the numerator (as the concentration of a standard solution or weight of a gravimetric precipitate) and in the denominator (as the weight of the sample). Of course, all weightings must be made with the materials in the same type of container (same density) to keep the error constant. Furthermore for most objects weighed, buoyancy errors can be neglected as they are only about one part per thousand!
	
	\subsubsection{Speed of sound in a liquid}
	Let us take a moment to calculate the speed of sound in a liquid. We have proved in the our study of the longitudinal sound waves of the Mathematical Music section that:
	
	where, for recall, $\gamma$ is the "Laplace coefficient", also named "adiabatic coefficient\index{adiabatic coefficient}" defined by:
	
	and we had proved that the speed of the sound wave was given by:
	
	Combining it comes:
	
	The fraction:
	
	that is to say the ratio between a variation in pressure and the relative variation in volume which it entails is named the "\NewTerm{volumic modulus of elasticity}\index{volumic modulus of elasticity}". Notice that the sign $-$ is required for $B$ to be positive: when the pressure increases, the volume decreases!

	We have for example for water:
	
	The measured value being in real life in standard condition of temperature and pression: $1,441\;[\text{m}\cdot \text{s}^{-1}]$. It may seem surprising that the velocity of sound in a liquid, which is much more difficult to compress than a gas is only $5$ times greater than in a gas. The reason is that the density of a liquid is about a thousand times higher than that of a gas. In both, the two properties partially compensate.
	
	\pagebreak
	\subsection{Gas}
	The solids have a well defined shape and are difficult to compress. Liquids can flow freely and their flow is limited by self-formed surfaces. The gases dilate freely to occupy the volume of the receptacle containing them, and have a density about a thousand times lower than that of liquids and solids. They conduct heat and electricity only a little, unless we ionize them (forming then plasma). The molecules of a neutral gas move along rectilinear trajectories which change direction with each collision with another molecule. Unlike solids and liquids, interactions between molecules remain weak. The macroscopic properties of a gas are thus directly deduced from the properties of the molecules that compose it (or atoms in the case of a monoatomic gas).
	
	\subsubsection{Types of Gas}
	In gas theory\index{gas theory} (we often speak of "\NewTerm{kinetic theory of gases}\index{kinetic theory of gases}") we always consider two types of neutral gases:

	\paragraph{Perfect Gas}\mbox{}\\\\
	The "\NewTerm{perfec gas}\index{perfect gas}" is a model in which we neglect the molecular interactions of the gas, with the exception of collisions, and whose own volume is negligible relatively tothe volume of the container. We deduce from these considerations that the behavior of a real gas can be approached by a perfect gas if the volume it occupies is sufficiently large, which will occur at low pressure and at high temperature. In practice, the perfect gas model gives satisfactory results for many conventional gases taken at atmospheric pressure and at ambient temperature (nitrogen, oxygen, steam, etc.).

	When a gas is at low pressure, the interactions between its molecules are weak. Thus, the properties of a real gas at low pressure approximate those of a perfect gas. We can then describe the behavior of the gas by the "\NewTerm{equation of state of perfect gases}\index{equation of state of perfect gases}" that we will prove in detail further below during our study of the virial theorem:
	
	With $n$ the number of moles of gas, $P$ the gas pressure, $V$ the volume occupied by the $n$ moles and $T$ the absolute temperature of the gas. The constant $R$ being the "\NewTerm{constant of the perfect gases}\index{constant of the perfect gases}":
	
	This equation shows trivially that:
	\begin{itemize}
		\item At constant temperature $T$ ("isothermal" system), the volume of a fixed quantity of gas is inversely proportional to its pressure. This is the "\NewTerm{Boyle-Mariotte's law}\index{Boyle-Mariotte's law}\label{boyle mariotte law}":
		
		\begin{figure}[H]
			\centering
			\includegraphics[scale=1]{img/mechanics/boyle_martiotte_law.jpg}
			\caption{Boyle-Mariotte's law illustration}
		\end{figure}
		From the Boyle-Mariotte's law, we deduce immediately that at constant temperature (ie during an isothermal transformation):
		

		\item At constant pressure $P$ ("isobaric" system), the volume of a fixed quantity of gas is proportional to the absolute temperature. This is the "\NewTerm{Gay-Lussac's law}\index{Gay-Lussac's law}\label{gay lussac law}" (in the case of perfect gases ...):
		
		\begin{figure}[H]
			\centering
			\includegraphics[scale=1]{img/mechanics/gay_lussac_law.jpg}
			\caption{Gay-Lussac's law illustration}
		\end{figure}
		It is this relation which is often used in the labs of small classes to show that with an extrapolation of the measured line, the volume theoretically becomes zero at a temperature of about $-273.15$ [$^\circ$C]. No by the way seriously from a certain temperature, it is necessary to use quantum models and moreover this relation is valid only for the gases (thus, when the water vapor becomes liquid ... it is no longer valid ).
		
		Let also recall that we have proved a more elaborate expression of the Gay-Lussac's law in the section of Thermodynamics by making use of equations of state and thermoelastic coefficients ...:
		
		This show us that the Gay-Lussac's law then no longer really have the satut of "law" but rather that of a "relation" as in can be proved! We have also proved in the chapter of Thermodynamics that this law is only approached to the high pressures for the gases since they must be considered as incompressible like liquids to obtain the relation in question.
		
		\item At constant volume $V$ ("isochore" system), the pressure of a fixed quantity of gas is proportional to its absolute temperature. This is the "\NewTerm{Charles' law}\index{Charles' law}":
		
		\begin{figure}[H]
			\centering
			\includegraphics[scale=1]{img/mechanics/charles_law.jpg}
			\caption{Charles' law illustration}
		\end{figure}
	\end{itemize}
	Subsequently, Amedeo Avogadro, to whom we owe the word of "\NewTerm{molecule}\index{molecule}" affirms the molecular concept of gases and concludes that equal volumes of different gases, taken under the same conditions of temperature and pressure, contain the same number Of molecules (this observation now follows automatically from the equation of perfect gases which will be proved further below in details). This property was named the "\NewTerm{Avogadro and Ampere law}\index{Avogadro and Ampere law}".
	\begin{tcolorbox}[colframe=black,colback=white,sharp corners]
	\textbf{{\Large \ding{45}}Example:}\\\\
	The air in a diving cylinder is initially at the pressure $P_i$ in an volume $V_i$ corresponding to that of the diving cylinder that we assume to be $0.015\;[\text{m}^3\cdot \text{s}^{-1}]$. A diver never breathes air directly into the bottle because of the pressure that is about $200$ times higher than the atomospheric pressure $20.02\cdot 10^7$ [Pa] would blow his lungs. A valve makes it possible to adjust the pressure of the air sent to the plunger so that it is equal to the ambient pressure $P_t$ of the water depth. It is interesting to calculate the time $t$ that the diver can stay underwater at a depth of $10$ meters if it is assumed to consume $0.03$ cubic meters per minute.\\
	
	To answer this question we must therefore first know the pressure $P_f$ at $10$ meters depth. For this we will use the Hydrostatic theorem proved during our study of fluids:
	
	Afterwards the volume in the bottle should be set to the above pressure volume using the Boyle-Mariotte's law:
	
	We then have as equivalent volume:
	
	But the trap is that the usable volume is smaller because it still requires air to push the remaining air. We must then subtract the two volumes:
	
	And dividing by a standard volume consumption rate $C$ we have at the end:
	
	\end{tcolorbox}
	Now let us come back to the equation of state of perfect gases:
	

	\pagebreak
	\paragraph{Real Gas}\mbox{}\\\\
	The perfect gas equation is obviously approximate. For example, a perfect gas could neither liquefy nor solidify, regardless of the cooling and compression to which it is subjected. Real gases, especially under conditions of pressure and temperature close to the transition to the liquid state, can present considerable deviations with the law of perfect gases!

	It must therefore be adapted to real cases. For this we use the Van der Waals equation of state which is particularly useful and well known. It can be obtained qualitatively once the perfect gas equation is demonstrated and is then given by (see below how we prove it):
	
	for one mole, $a$ and $b$ being adaptable parameters determined by experimental measurements carried out on the concerned gas ($a$ is a measure of the average attraction between particles, $b$ the volume excluded by a mole of particles). These are parameters that vary from one gas to another.

	The Van der Waals equation can also be interpreted at the microscopic level. Molecules interact with each other. This interaction is strongly repulsive for the molecules close to one another, becomes slightly attractive for a mean distance and disappears when the distance is important. At high pressure, the perfect gas law must be rectified to take into account attractive or repulsive forces.
	
	\subsubsection{Virial Theorem}\label{virial theorem}
	We will here discuss a study of perfect gases via a particular method. It makes it possible to obtain an interesting result and particularly for astrophysics (\SeeChapter{see section Astrophysics page \pageref{astrophysics}}). The virial theorem\footnote{The word \textit{virial} derives from \textit{vis}, the Latin word for \textit{force} or \textit{energy}, and was given its technical definition by Rudolf Clausius in 1870.} also makes it possible to obtain other very interesting results but which pedagogically are somewhat difficult to access. The reader who would be interested in this second part of the results can directly go back a little further where the concepts of pressure and kinetic temperature are treated.
	By definition, the "\NewTerm{virial expression $V_\text{ir.}$}\index{virial expression}" of a material point is the scalar:
	
	By definition, the "\NewTerm{virial}\index{virial}" of a system composed of $N$ material points is:
	
	Subjected to a central force, the virial is written (by the properties of the scalar product):
	
	The "\NewTerm{virial theorem}\index{virial theorem}" states: For a system in equilibrium (!), the internal energy is equal to the opposite of its total half-virial when all the particles are located with respect to its center of mass.
	\begin{dem}
	Given the differentation (\SeeChapter{see section Differential and Integral Calculus page \pageref{differential}}):
	
	Its second derivative:
	
	By multiplying by by $m_i$ and summing on $i$:
	
	But:
	
	and:
	
	Therefore:
	
	This last expression is valid whatever the position of the adopted coordinate system. However, it is interesting to place its origin at the center of mass (centroid) of the system because we are no longer dependent on its movement.

	If the system is in equilibrium, the macroscopic quantities which characterize it are not dependent on time. We then conclude that the sum of any quantity attached to any material point of the system is in fact a quantity of that system.

	Thus, $\sum m_ir_i$ is a macroscopic quantity independent of time. That implies that:
	
	What is written again (we multiply by $1/2$ on both sides):
	
	So we have finally:
	
	This expression of kinetic energy is known as "\NewTerm{virial theorem}\index{virial theorem}" and the right-hand side is therefore named the "\NewTerm{virial of the system}\index{virial of the system}".
	\begin{flushright}
		$\square$  Q.E.D.
	\end{flushright}
	\end{dem}
	We write that:
	
	where $E_c$ is the total kinetic energy associated with all the material points of the system. We name it the "\NewTerm{internal energy of the system}\index{internal energy of a system}" and thermodynamicists often denote it with the letter $U$ as we know it...! The term $\langle E_c \rangle$ is the energy of any material point in the system.

	It is possible to fall back on the expression of the virial from a system of particles (cloud in accretion). Strictly speaking, equilibrium does not exist in such a case. Nevertheless, we can assume that if the gravitational contraction is sufficiently slow then its different phases can be considered as a succession of equilibrium states (typical thermodynamic approach that we already know).

	In the case of a central force and derived from a potential, we can write (\SeeChapter{see section Classical Mechanics page \pageref{gravitational potential energy}}):
	
	and therefore:
	
	If the potential energy is of the form $k / r$ (which is the case for the electrical and gravitational potential) then it comes:
	
	and it remains:
	
	sometimes denoted and rewritten (especially in Astrophysics):
	
	In summary, the virial theorem gives us a relation between total kinetic and potential energies. To be valid, the mobile must describe a trajectory around the center of central force and remain indefinitely in a finite volume (bound state). This type of reasoning can be applied to a very large number of phenomena, from the structure of certain galaxies to the release of energy in nuclear explosions through the study of the Sun and the behavior of real gases. This is the first result we were interested in.
	\begin{tcolorbox}[title=Remark,colframe=black,arc=10pt]
	Astrophysicists use (veeery roughly and stupidly as the assumptions are not satisfied!) the relation:
	
	for Galaxy clusters.... with:
	
	where $M_D$ is the cluster mass and:
	
	where $R$ is the estimated galaxy cluster radius. Therefore:
	
	and according to the bad astrophysicts that do this huge approximation, the observed $M_D$ is not compliant with the calculated one, so there must be something missing that we can't see and that they name the "dark matter\index{dark matter}"... It should be interesting to do the calculations without so huge approximations and with a model that have assumptions that are compatible with galaxies...
	\end{tcolorbox}
	In a gaseous system, the potential energy can be written as the sum of the energy of the forces acting from the outside and those which are internal even to the gas. Such as:
	
	But the internal forces can be written as:
	
	It is not necessary in this sum to take the force exerted by each of the particles upon itself. Such as:
	
	What give us:
	
	In the double sum, we can group the terms two by two and use the principle of action-reaction such as:
	
	To get by definition:
	
	What ultimately gives us:
	
	And:
	
	The first term on the right involves the internal forces (interactions) between the (pairs of) particles and the second term on the right involves external forces.

	Let us consider now a gas contained in a tank. Its molecules are subject to external forces only when they hit a wall and we imagine that on average this force is perpendicular to the wall (elastic shocks).
	\begin{figure}[H]
		\centering
		\includegraphics{img/mechanics/gas_in_tank.jpg}
		\caption{Gas contained in a tank}
	\end{figure}
	For all the faces contained in the planes defined by the axes, we always have:
	
	Since on average $\vec{F}_i$ is always perpendicular to $\vec{r}_i$.

	For the other faces ($BCFE$ for example) we have:
	
	and thus:
	
	where we denote by $a$ the coordinate according to O$y$ of the end of $\vec{r}_i$ (not to be confused with acceleration!). Therefore for each face:
	
	Indeed because the internal pressure of the system on the walls is defined by:
	
	By the virial theorem, by adding the non-zero contributions of the faces $BCFE$, $DEFG$ and $ABED$, it comes:
	
	If the average kinetic energy of a molecule is:
	
	The total mean kinetic energy for $N$ molecules is then (we will return to this relation later but with another approach):
	
	that thermodynamicists often write:
	
	where $U$ is as we know the internal energy of the gas and df the number of degrees of freedom of the constituents.

	The prior relation, named the "\NewTerm{equipartition theorem of energy}\index{equipartition theorem of energy}\label{equipartition theorem of energy}" (as it assumes that the total kinetic energy is equally distributed by each degree of freedom) is important because it allows:
	\begin{enumerate}
		\item A microscopic interpretation of temperature $T$ and to determine the internal energy of a monoatomic perfect gas (and by extension of other gases with other degrees of freedom).

		\item To notice that the temperature system in Celsius is not adapted to the physical reality. Indeed, in the system using the Celsius, at $0$ [$^\circ$C] everything should be immobile (zero kinetic energy) and it is obvious that this is not the case for all substances. Therefore, a new temperature must be introduced that matches the measured kinetic energy with the traditional temperature. It will be the "\NewTerm{absolute temperature}" measured in Kelvin [K] whose equivalence kinetic energy / temperature measured is such that $0$ [$^\circ$C] corresponds to $273.15$ [K].
	\end{enumerate}
	Internal energy is a contribution to energy that does not appear in Classical Mechanics as we already know. From the macroscopic point of view, a container at rest that contains a fluid does not possess kinetic energy, while its potential energy is constant. We can ignore it by giving the value zero to this constant.

	From the microscopic point of view, however, things change! Indeed, the atoms or molecules of the fluid are in motion and interact. It is necessary to associate to them an energy (the internal energy) which is the sum of the contributions relative to each atom.

	Since then:
	
	It is the "\NewTerm{general equation of state of a real gas}\index{general equation of state of a real gas}", that is, the equation of state which takes into account the interactions between molecules. It is interesting to notice that finally this relation can allow us to calculate the energy of the gas even if there are no walls!

	If the gas is perfect, there are no interactions between the $N$ molecules (by hypothesis) and then we have then fall back on the "\NewTerm{ideal gas law}\index{ideal gas law}\label{ideal gas law}" as follows:
	
	where $P$ is the absolute pressure of the gas, $N$ is the number of molecules in the given volume $V$ and $k$ (written $k_B$ if there is a risk of confusion) is the Boltzmann constant:
	
	That we find much more frequently in the form:
	
	if $n$ is expressed in moles with $R$ the constant of the perfect gases. Obviously for $n=1$,  $N$ is equal to the number of particles in one mole (Avogadro's number: $N_\text{Av}$).

	If the temperature is constant, we fall back on "\NewTerm{Boyle-Mariotte's law}\index{Boyle-Mariotte's law}":
	
	Before continuing we can also come back on Avogadro statement: "equal volumes of all gases, at the same temperature and pressure, have the same number of molecules". That is to say for a given mass of an ideal gas, the volume and amount (moles) of the gas are directly proportional if the temperature and pressure are constant which can be written as:
	
	or:
	
	This law describes how, under the same condition of temperature and pressure, equal volumes of all gases contain the same number of molecules. For comparing the same substance under two different sets of conditions, the law can be usefully expressed as follows:
	
	The most significant consequence of Avogadro's law is that the ideal gas constant has the same value for all gases. This means that:
	
	
	To arrive at the perfect gas equation, it is useful to remember that we have made implicitly three hypotheses:
	\begin{enumerate}
		\item[H1]. The molecules are assimilated to hard spheres whose diameter is negligible in comparison with the average distance between them. That is what we name the "\NewTerm{structural hypothesis}\index{structural hypthesis}".

		\item[H2]. Ultimately, and this is what we have learned, if we consider molecules as punctual, the possibility of interaction between the particles vanishes. The only remaining interactions will be shocks on the walls of the tank that contains the gas. These shocks are perfectly elastic so that we can apply the laws of conservation of the momentum of kinetic energy. This is what we name the "\NewTerm{interactive hypothesis limit}\index{interactive hypothesis limit}"

		\item[H3]. The gas is studied in a state of thermodynamic equilibrium, which results in the homogeneity of intensive and extensive variables. That is what we name the "\NewTerm{hypothesis of molecular chaos}\index{hypothesis of molecular chaos}".
	\end{enumerate}
	In a particular case, if the interactions are derived from a central potential:
	
	We can rewrite this as follows:
	
	If in addition the potential energy is of the type (be careful not to confuse the parameter $k$ with the Boltzmann constant denoted in the same way!):
	
	Then we have:
	
	And therefore:
	
	Where $E_p$ is then mean total energy of the system.
	
	In fact, care must be taken that we have not rigorously proved the ideal gas law. Indeed, when we had put earlier above:
	
	This implicitly assumed that the perfect gas equation was already known (...).

	This approach to gas kinetics is interesting because it is useful in astrophysics. However, this is by far not the simplest in a school context and pedagogically... We propose the reader to come back to these same results further below via the concepts of pressure and kinetic temperature once the Van der Waals equation is determined.

	In 1875 the Dutch scientist J.D. Van der Waals tried to replace the perfect gas equation by a relation which would take into account the intermolecular forces and the size of the molecules. The first and most obvious correction to the equation of perfect gases:
	
	is to subtract the volume of the gas molecules from volume $V$. We can do this logically by replacing $V$ by $V-Nb$ where $b$ is a very small constant, which obviously represents the average volume per molecule (tables exist for this). So:
	
	where the term $Nb$ is commonly referred to as the "\NewTerm{covolume}".

	To take account of the intermolecular forces that we have previously neglected, we can try an approximate approach by knowing already that the attraction force of each molecule will be on $N-1$ molecules. Consequently, the numerator of the attraction force will trivially  contain (by the sum of all terms) if the gas is isotropic and homogeneous a term of the type $N (N-1)$ for the influence of all the molecules together which if $N$ is very large can be approximated by $N^2$.
	
	We also know that in the numerator there will be a mass term for each particle. If we know the ratio $N / V$ then we only need to know the mass density of the gas (but this is not an extensive variable so we will avoid making it appear explicitly). Thus, the term $N^2$ can be directly written $(N/V)^2$.

	Following this reasoning, Van der Waals added to the right-hand side of the relation above a negative term proportional to the quantity $(N/V)^2$. The presence of this term results in a lowering of the pressure as the density of the gas increases. The modified relation is thus:
	
	where $a$ is a constant of proportionality. We can rewrite this relation in the form:
	
	which is named the "\NewTerm{Van der Waals state equation}\index{Van der Waals state equation}" or "\NewTerm{Van der Waals real gas equation}\index{Van der Waals real gas equation}" (we find in the literature several equivalent ways of writing this last relation). It is an excellent description of the equation of state in a wide range of the variables $P$, $V$, $T$, the values $a$ and $b$ being characteristic of each gas. The constants $a$ and $b$ are determined experimentally as we have already mentioned.

	We can represent the Van der Waals equation in a $P$-$V$ diagram (hence constant temperature). Here is an example with $\mathrm{CO}_2$ (carbon dioxide):
	\begin{figure}[H]
		\centering
		\includegraphics[scale=1]{img/mechanics/van_der_waal_p_v_diagram_co2.jpg}
		\caption{Van der Waals $P$-$V$ diagram for $\mathrm{CO}_2$}
	\end{figure}
	Which corresponds to:
	
	If we set $a = 0$ and $b = 0$ we see that we fall back on a famous relation...

	The bottom figure shows the general case. One can pass from state $F$ (vapor) to state $I$ (liquid) by following a path passing through states $A$ and $E$ (liquefaction stage), then a phase transition is observed. If we follow another path, for example by using the isotherm $\overline{GH}$ ($T> T_k$, red path), there is no liquefaction stage, we pass continuously from the gaseous state to the liquid state:	
	\begin{figure}[H]
		\centering
		\includegraphics[scale=1]{img/mechanics/van_der_waal_p_v_general_diagram.jpg}
		\caption{Van der Waals general $P$-$V$ diagram}
	\end{figure}
	The Van der Waals equation can be put in the form of a virial type development using a Taylor-Maclaurin development (\SeeChapter{see section Sequences and Series page \pageref{usual maclaurin developments}}):
	
	or by rearranging the terms:
	
	Let us notice that there exists a temperature for which the term $\left(b-\dfrac{a}{kT}\right)$ is zero. We name it the "\NewTerm{Boyle Temperature}\index{Boyle Temperature}" of the gas: this is the temperature at which the actual gas behaves like an ideal gas.
	
	Form information here is for comparison a $PVT$ diagram of the ideal gas law:
	\begin{figure}[H]
		\centering
		\includegraphics[scale=0.7]{img/mechanics/pvt_diagram.jpg}
		\caption{$PVT$ diagram of the ideal gas law}
	\end{figure}
	And on the next page a more realistic one where we can identify the Van der Waals law application domain:
	\begin{figure}[H]
		\centering
		\includegraphics[scale=0.9]{img/mechanics/3d_phase_diagram.jpg}
		\caption{$PVT$ Phase diagram for real gas}
	\end{figure}
	
	\pagebreak
	\subsubsection{Kinetic pressures (kinetic theory of gases)}
	The kinetic theory describe a gas as a large number of submicroscopic particles (atoms or molecules), all of which are in constant rapid motion that has randomness arising from their many collisions with each other and with the walls of the container.
	
	Kinetic theory explains macroscopic properties of gases, such as pressure, temperature, viscosity, thermal conductivity, and volume, by considering their molecular composition and motion. The theory posits that gas pressure is due to the impacts, on the walls of a container, of molecules or atoms moving at different velocities.
	
	Kinetic theory defines temperature in its own way, not identical with the thermodynamic definition.
	
	Let us seek the number of molecules of a perfect gas, all supposed to be animated by an equal speed $v$, which strike a surface $S$ for a duration $\mathrm{d}t$.

	If the gas studied is in a state of thermodynamic equilibrium, this will result in the homogeneity of the intensive variables (hypothesis of molecular chaos).

	It follows that the density of the particles is constant:
	
	If we admit that the molecules are in motion, we shall assume that there is isotropy of velocities. In other words, since the velocities can be described in a system of $3$ orthogonal axes (three spatial dimensions), there are $6$ possible primary directions in total ($2$ directions per axis: forward, backward ...) .

	This translates into equivalence between the different directions. There is:
	
	particles having a velocity $v$ along one of the primary directions.

	Thus, during a time $\mathrm{d}t$, the surface $S$ of the wall is percussed only by a part of the molecules contained in the volume $Sv\mathrm{d}t$ Indeed, only $1/6$ of the molecules contained in this volume are actually directed towards the surface $S$.
	\begin{figure}[H]
		\centering
		\includegraphics[scale=1]{img/mechanics/kinetic_theory_of_gas.jpg}
		\caption[]{One of the walls of the enclosure (tank)}
	\end{figure}
	The number of molecules which strike the wall during the duration $\mathrm{d}t$ is thus:
	
	Let us now study the dynamics of the impact of a particle on the wall!

	The particle of a mass $m$ arriving on the wall with the speed $\vec{v}_x$ goes away again with a speed $-\vec{v}_x$ if the shock is perfectly elastic.

	The variation of the momentum of the particle is thus:
	
	
	Following the conservation of the linear momentum, this linear momentum is transferred to the wall: the variation of the quantity of motion undergone by the wall is the opposite of that of the particle:
	
	The total variation of the linear momentum of all the particles which strike the wall during the period $\mathrm{d}t$ is then equal in norm to:
	
	Applying the fundamental principle of dynamics, we can pass to the force undergone by the wall during the striking of these molecules:
	
	therefore:
	
	Now by defining the pressure and taking the module:
	But by definition of the pressure and taking the module:
	
	we then have the "\NewTerm{kinetic pressure}\index{kinetic pressure}" given by:
	
	The kinetic pressure is therefore nothing else than the frequency of the shocks of the particles (molecules) on the wall. The more molecules (term in $N_v$) and the quicker they are (term in $v$), the more the number of shocks increases. This is consistent with experience and intuition.
	
	\pagebreak
	\subsubsection{Kinetic Temperature}
	Now comes the proof we are waiting from since the beginning of our study of gas: the proof that temperature is nothing more that a quantity related to the average velocity but with a conversion factor that is nothing else than the Boltzmann constant!

	For this purpose we start again from:
	
	By replacing $N_v$ by the quotient $N / V$, the preceding relation can be written as:
	
	If we identify it with the perfect gas state equation presented earlier above:
	
	it comes:
	
	But the quantity of moles $n$ is equal to the ratio of the number of particles to the number of Avogadro as we know:
	
	which makes it possible to define the "\NewTerm{kinetic temperature}" by:
	
	We then fall back on a way to introduce "\NewTerm{Boltzmann constant}\index{Boltzmann constant}\label{boltzmann constant}" and such that:
	
	The expression of kinetic temperature then takes the form:
	
	This relationship shows that the kinetic temperature is only the representation of the kinetic energy of the particles (whose temperature is in fact a measure of the average velocity of the molecules). Imaginatively, it is the image of the violence of shocks. We can also better understand the origin of the definition of absolute temperature (otherwise, the velocity should be a complex number) and the argument that everything is immobile at absolute zero!
	
	Thus the energy of a perfect gas is reduced to the sum of the kinetic energies of the particles which constitute it:
	
	The atoms of a monoatomic perfect gas can be assimilated to material points. Their kinetic energy is a kinetic energy of translation whose mean value, per atom, is written:
	
	In the velocity space, all directions are equivalent! There is isotropy of the velocity distribution. In Cartesian coordinates, it comes:
	
	and as a result of isotropy:
	
	hence:
	
	Thus, the average kinetic energy per degree of freedom of translation is equal to:
	
	The reader will of course notice that we obtain here the same results and conclusions as in our study of the virial theorem with the difference that the approach here is simpler and therefore more didactic. 
	
	\subsubsection{Amagat and Dalton's law}
	Let us consider a mixture of two perfect gases in a volume $V$ imposed constant at constant temperature $T$ also imposed. We can write for each of the gases:
	
	Where $P_1$, $P_2$ are named "\NewTerm{partial pressures}\index{partial pressures}". The whole mixture will be written:
	
	But:
	
	and therefore:
	
	Finally we conclude that:
	
	The pressure of a mixture of two perfect gases is the sum of the partial pressures, this is the "\NewTerm{Dalton's law}\index{Dalton's law}" also named "\NewTerm{Dalton's law of partial pressures}\index{Dalton's law of partial pressures}".
	\begin{figure}[H]
		\centering
		\includegraphics[scale=0.4]{img/mechanics/dalton_law.jpg}
		\caption{Dalton's law illustration for air at normal temperature}
	\end{figure}
	\begin{tcolorbox}[title=Remark,colframe=black,arc=10pt]
	We mentioned in the section Thermodynamics that pressure was an intensive variable and therefore not additive. But this is not the case in the situation of perfect gases when the volume is forced as being constant.
	\end{tcolorbox}
	Thus, the partial pressure $P_i$ of a gas in a mixture is equal to the product of the total pressure $P$ of the mixture by the percentage (concentration) $c_i$ of the gas considered in the mixture:
	
	\begin{figure}[H]
		\centering
		\includegraphics[width=\textwidth]{img/mechanics/dalton_law_numberical_application.jpg}
		\caption[Numerical application of Dalton's law]{Numerical application of Dalton's law (source: OpenStax)}
	\end{figure}
	We can make the same reasoning at constant pressure $P$ and constant temperature $T$. We then get:
	
	hence:
	
	The volume of a mixture of two perfect gases at constant pressure and constant temperature is the sum of the partial volumes, it is the "\NewTerm{Amagat's law}\index{Amagat's law}".
	
	\subsubsection{Mean free path (in kinetic theory)}\label{mean free path}
	The mean free path is the average distance traveled by a moving particle (such as an atom, a molecule, a photon) between successive impacts (collisions), which modify its direction or energy or other particle properties.
	
	We will now see a very interesting case study of gases which makes it possible to clarify a lot of misunderstandings in everyday life (smoke in restaurants, heat near a radiator, ...). However, the phenomena are in reality more complex, it is also necessary to take into account diffusion, convection, etc (\SeeChapter{see section Thermodynamics page \pageref{convection diffusion radiation}}).

	Consider a molecule, which moves at the average speed $\bar{v}$. Its sphere of influence $V_i$ then sweeps, during the unit of time of its displacement, a volume (cylinder) given by:
	
	\begin{tcolorbox}[title=Remark,colframe=black,arc=10pt]
	In the case of an atom or molecule as a sphere of influence, it is often referred to as the "\NewTerm{Van der Waals volume}\index{Van der Waals volume}" and the associated "\NewTerm{Van der Waals radius}\index{Van der Waals radius}".
	\end{tcolorbox}
	If the unit of volume contains $n$ molecules ($n$ has therefore the units of a volumic density), the number of shocks (collisions) during the unit of time in this same unit of volume will then be obviously:
	
	at least if the other spheres of influence were immobile ... So to take into account the movement of other spheres of influence, consider the figure below with three simplistic scenarios:
	\begin{figure}[H]
		\centering
		\includegraphics[scale=1]{img/mechanics/mean_free_path_collision_scenarios.jpg}
		\caption[]{Three collision scenarios}
	\end{figure}
	From left to right we have:
	\begin{enumerate}
		\item The relative speed of the molecules is $2\bar{v}$

		\item The relative speed of the molecules is zero

		\item The relative speed of the molecules equals (Pythagoras) $\sqrt{2}\bar{v}$
	\end{enumerate}
	The first two cases are somewhat extreme ... The third one will be considered as an average and can serve as a new basis for the previous calculation.

	Thus, using the relative speed of the last scenario, the number of collisions during the unit of time in the unit volume  becomes:
	
	Thus, between two collisions a molecule travels an average distance:
	
	It is therefore the expression of the mean free path as a function of the molecular (particle) density $n$ (and not the number of moles!!!) and the radius $r$ of the sphere of influence, the intrinsic parameter of the gas under consideration. 

	Given the ideal gas law:
	
	we have:
	
	and therefore:
	
	Where $n$ is still the molecular (particles) density (and not the number of moles !!!) and therefore we can write:
	
	A numerical application gives for the element at normal conditions of temperature and pressure a free mean path which for the majority of the current molecules is several thousands of times the diameter of a molecule of standard size (the mean free path is therefore of the micrometer\footnote{In the section Principia we have given a table with a lot of examples of vacuum levels and theire corresponding mean free path for recall...}).
	\begin{figure}[H]
		\centering
		\includegraphics[scale=1]{img/mechanics/mean_free_path.jpg}
		\caption[Simplified illustration of free mean path]{Simplified illustration of free mean path (source: ?)}
	\end{figure}
	Using the relation of the most probable mean velocity  under the hypothesis of a Maxwell distribution of the velocities (\SeeChapter{see section Statistical Mechanics page \pageref{maxwell distribution}}) we have:
	
	For a molecule with a molar mass of $30$ [g] at normal temperature. This gives a number of collisions for one mole of molecules:
	
	This high frequency of collisions explains, under normal conditions of temperature and pressure, the rapidity with which the statistical equilibrium is established within a gas and also the speed of sound propagation.

	At ambient temperature and for a vacuum of $7.5\cdot 10^{-10}$ [Pa], we get with the same values a mean free path of approximately $1$ [km].

	Since the dimensions of the containers containing the gases being almost always less than this order of magnitude, it appears that when the vacuum is produced in a close volume, the intermolecular collisions are negligible vis-à-vis the molecule-wall collisions.
	\begin{figure}[H]
		\centering
		\includegraphics[scale=1]{img/mechanics/tank_vacuum_implosion.jpg}
		\caption[Tank trailer Vacuum implosion to atmospheric pressure]{Tank trailer Vacuum implosion to atmospheric pressure (source: YouTube)}
	\end{figure}
	
	\pagebreak
	\subsubsection{Generalized Bernoulli equation}\label{generalized bernoulli equation}
	To determine the Bernoulli equation for gas (quite important for our study later of aerospace engineering) we consider the behavior of a particular parcel of fluid as it flows along a stream line. We assume the system is non-dissipative; that is, we assume that viscosity and thermal conductivity make a negligible contribution to the energy budget. For simplicity, we assume a steady pressure distribution.
	
	We start by invoking the law of conservation of energy. As always, this law says that any change in the energy within the parcel must be exactly balanced by the transfer of energy across the boundary of the parcel. That is:
	
	The energy inside the parcel can take various forms, including kinetic energy, gravitational energy, and the energy stored in the springiness of the fluid. In a gas such as air, the latter is sometimes named the "thermal energy", but of course the laws of thermodynamics apply to all forms of energy, not just the air-spring energy. The relation below restates what we have just said, expressing it in mathematical terms:
	
	This last relation could describe almost any parcel under almost any conditions. To make progress, we need to invoke the fact that our parcel is flowing along a streamline, in thermodynamic equilibrium with the surrounding fluid. We continue to assume that viscosity and thermal conductivity are not significant, so no entropy is being created and no entropy is flowing across the boundary of the parcel.
	
	In thermodynamic equilibrium, the total entropy of the parcel and its surroundings is at a maximum with respect to small changes in P, V, h, et cetera. In the present situation, where the environment exerts a pressure on the parcel, this is provably equivalent to saying that the enthalpy of the parcel not counting the surroundings is at a minimum. Therefore we are interested in (\SeeChapter{see section Thermodynamcis page \pageref{enthalpy}}):
	
	You may ask, how can we assert thermodynamical equilibrium, when the parcel is not even in mechanical equilibrium? That is, what about the fact that the parcel is accelerating and decelerating as it flows along? That's a good question, for which we have a good answer: Our derivation hinges on the fact that even though the parcel is exchanging energy with the surroundings, it does so in a way that does not exchange any entropy. The only way this can happen is if $\mathrm{d}H=0$. This depends on our assumption that viscosity and thermal conductivity are negligible.
	
	Therefore:
	
	We wish to integrate equation 6, but before we can integrate the V dP term, we need to know the equation of state. We assume a polytropic equation of state, namely (\SeeChapter{see section Thermodynamics page \pageref{polytropic gas equation}}):
	
	Rearranging a bit:
	
	Therefore:
	
	After integration:
	
	where we have absorbed a constant of integration into our definition of $H$.
	
	As:
	
	Then the prior-previous relation can be simplified as:
	
	We can make these equations more practical by eliminating the explicit dependence on V in favor of P in equation 10. We can do this by using:
	
	Then:
	
	But as:
	
	Therefore:
	
	That's it.
	
	Now as $H=E+PV$ and:
	
	We get:
	
	Any of the results in this section can be considered the general form of the Bernoulli equation.
	
	For air tables give $\gamma\cong 1.4$ at $293$ [K]. It is interesting to notice that for when $\gamma=2$ we fall back on the Bernoulli equation that we have determined during our study of liquids.
	
	\pagebreak
	\subsection{Plasmas}\label{plasmas}
	We define a "\NewTerm{plasma}\index{plasma}" as a state of matter in which certain electronic links have been broken, causing the emergence of free, negatively charged and positively charged ions. Low-ionized gases named "plasmas" by language abuse have the same mechanical properties (flows, acoustic waves, etc.) as neutral gases. On the other hand, their electromagnetic properties (electrical conductivity, refractive index) differ by the presence of free electrons within them.
	
	Plasma has become increasingly important as theiir application to Tokamak (nuclear fusion reactor) and Pulsed plasma thruster for space engine have become technically possible.
	\begin{figure}[H]
		\centering
		\includegraphics[scale=0.9]{img/mechanics/plasma.jpg}
		\caption{Some well known everyday life of plasma in the 20th and 21st century}
	\end{figure}
	\begin{tcolorbox}[title=Remark,colframe=black,arc=10pt]
	Plasma is also named as we already know (\SeeChapter{see section Thermodynamics page \pageref{phases of matter}}) the "\NewTerm{fourth (classical) state of matter}" (after solid, liquid and gaseous states).
	\end{tcolorbox}
	In their normal state, gases are electrical insulators. This is because they do not contain free charged particles, but only neutral molecules. However, if we apply a fairly strong electric fields to them, they become conductors. The complex phenomena that occur then are named "\NewTerm{discharges in gases}" and are due to the appearance of electrons and free ions.

	The result of a discharge in a gas is therefore the production of an ionized gas containing, for example, an average density of electrons $n_e$equations, positive ion $n_i$ and $n_0$ neutral itmes (atoms or molecules). In general, the gas is macroscopically neutral. We have then:
	
	Or otherwise expressed, neutrality is also written:
	
	This neutrality is the consequence of the very intense electrostatic forces that appear as soon as one has equation. The particle density is therefore the first fundamental quantity.
	
	The "\NewTerm{degree of ionization}" of a gas is defined by the ratio:
	
	where $n_0$ is the density of neutral particles and $n$ that of the electrons (or positive ions). The value of the degree of ionization in the various types of ionized gas varies in practice from very low values, from the order of $10^{-10}$, for example, to $1$.

	The second fundamental quantity is the temperature $T$. When heating a gas at a sufficiently high temperature (of the order of $10^4$ [T]), the mean kinetic energy (see virial theorem):
	
	of its molecules can become of the same order as their ionization energy $E_i$. Under these conditions, when two molecules collide, there can be ionization of one of them.
	
	If the gas is in thermodynamic equilibrium, the collision ionization is counterbalanced by recombination processes between electrons and ions and the result is that the three equation variables $\alpha$, $n$ and $T$ are not independent: ionization is determined by pressure and temperature, we say then the gas is in "\NewTerm{equilibrium state of thermal ionization}\index{equilibrium state of thermal ionization}".

	At higher temperatures, the gas atoms can be ionized several times. In many cases, ionization is due to an external electric field (+ an external magnetic fields sometimes to stabilize the process), and the gas is not in thermodynamic equilibrium. It will often reach a stationary state that can be characterized by the parameters $\alpha$, $n$, $T_e$ (temperature of the electrons), $T_i$ (temperature of the ions) and $T_0$ (temperature of the molecules).
	
	The three temperatures thus introduced are defined by the condition that $3/2 kT_a$ represents the average kinetic energy of the particles of species $a$, in a frame where they have a mean zero velocity. The difference between $T_e$, $T_i$ and $T_0$ can be important! for example, in a typical discharge tube, we can have: $T_0\cong T_i \cong 300$ [K] and $T_e\cong 3\cdot 10^4$ [K]. The high value of $T_e$ is due to the action of the electric field on the electrons, and the ionization is then produced by the collisions of these hot electrons on the neutral molecules of the gas.

	In conclusion, there are only two basic quantities to characterize a plasma: the density and the electronic temperature. We shall now turn to two other important but non-fundamental magnitudes in the sense that they are expressed on the basis of density and temperature.

	\subsubsection{Plasma Frequency}
	The "\NewTerm{plasma oscillations}\index{plasma oscillations}", also known as "\NewTerm{Langmuir waves}\index{Langmuir waves}"  (after Irving Langmuir) or "\NewTerm{plasma frequence}\index{plasma frequency}", are rapid oscillations of the electron density in conducting media such as plasma or metals. The oscillations can be described as an instability in the dielectric function of a free electron gas. The frequency only depends weakly on the wavelength of the oscillation. The quasiparticle resulting from the quantization of these oscillations is the "\NewTerm{plasmon}\index{plasmon}".

	If in an initially neutral plasma we produce a local perturbation in the form of an excess of positive or negative electrical charge, it will tend to return to the state of equilibrium of neutrality. However, we can easily see that the initial perturbation generates an undamped pendular oscillation of the plasma around its equilibrium state. Consider, for example, the situation shown in the figure below:
	\begin{figure}[H]
		\centering
		\includegraphics[scale=1]{img/mechanics/plasma_oscillation.jpg}
		\caption{Plasma electronic oscillation}
	\end{figure}
	At the initial moment the region at the center contains an electron deficit and the region all around an excess of electrons. This produces an electric field tending to create a movement of the electrons in the direction of the arrows. In this movement, these will acquire a certain kinetic energy and they will be able, after a given time, to go beyond the position of equilibrium. Too many electrons left the outer region, there will be a defect of electrons in this region and an electric field tending to bring them back towards it. After a certain time, the initial situation is reconstructed and the cycle begins again. The vibrations thus produced are the "electronic plasma oscillations" we have defined previously.

	\pagebreak
	We can quantitatively study this problem by posing the general equations of an electronic charge oscillation and with the following simplifying assumptions:
	\begin{enumerate}
		\item[H1.] The ions are assumed immobile since they are much heavier than electrons, and their quantity is equal to $n_{i0}$

		\item[H2.] Thermal agitation is negligible (assumption named "\NewTerm{cold electron plasma oscillation model}\index{cold electron plasma oscillation model}").

		\item[H3.] The collisions are negligible (we speak then of model of "\NewTerm{free transport}\index{free transport}")

		\item[H4.] The oscillations are of small amplitude

		\item[H5.] There is no electric or magnetic field imposed by external sources
	\end{enumerate}
	Now, let us recall that we have proved in the section of Electrodynamics that (equation of conservation of the electric charge):
	
	and in the section of Electrokinetics that:
	
	Therefore, by adopting the aforementioned notations, it comes locally:
	
	which constitutes the "\NewTerm{hydrodynamic equation of electrons}\index{hydrodynamic equation of electrons}".
	\begin{tcolorbox}[title=Remark,colframe=black,arc=10pt]
	A plasma is theoretically generally neutral, but we may theoretically have a non-neutral volume. It is this hypothesis which allows us to postulate that the divergence of the current is not zero!
	\end{tcolorbox}
	Let us now recall the Coulomb's force (\SeeChapter{see section Electrostatics page \pageref{coulomb force}}):
	
	where we have clearly neglected the kinetic pressure term  and the collision term (hypotheses H2 and H3) and neglected the magnetic field associated with the oscillation.

	We can simplify these equations using the hypothesis H4 in the form:
	
	where $n_\text{el}(\vec{r},t)$ is a small perturbation around the equilibrium.
	
	Let us suppose moreover that variable quantities vary at the pulsation $\omega$, we can therefore write:
	
	Consequently, the hydrodynamic equation of the electrons becomes:
	
	Thus finally:
	
	From the expression of Coulomb's force we also deduce:
	
	From where we get:
	
	But on the other hand we have the Gauss's law (\SeeChapter{see section Electrodynamics page \pageref{gauss law}}):
	
	indeed,  as taking into account the neutrality condition of the undisturbed plasma, we have:
	
	Therefore:
	
	From the relation proved previously:
	
	replacing in the following relation (also proved just earlier above):
	
	we get:
	
	Finally, replacing the latter expression in Gauss's law, we get:
	Finally, we get the "\NewTerm{plasma frequency for cold electrons}\index{plasma frequency for cold electrons}" or "\NewTerm{Langmuir frequency for cold electrons}\index{Langmuir frequency for cold electrons}" given by:
	
	In physics, the plasma frequency is thus the characteristic frequency of plasma waves, that is to say the oscillations of the electric charges present in conductive media, such as metal or plasmas. Like the electromagnetic wave which, quantified, is described by photons, this plasma wave is quantified in "plasmons" as we have already mention it before.

	The oscillations of the electric charges can be understood by the following reasoning: if the electrons of a zone of the plasma are moved, then the ions of this zone, having little moved due to their large mass, will exert on these electrons an attractive Coulomb force. So they're going back to their original position, and so on...
	
	Notice that $f_\text{ce}$ depends only on physical constants and concentration of electrons $n_{e0}$. The numeric expression for plasma ordinary frequency is then:
	
	
	\begin{flushright}
	\begin{tabular}{l c}
	\circled{90} & \pbox{20cm}{\score{3}{5} \\ {\tiny 27 votes,  80.74\%}} 
	\end{tabular} 
	\end{flushright}
	
\chapter{Electromagnetism}

	\textit{\textbf{Electrodynamic is the field of physics that study the dynamic action of electric currents and the propagation of electromagnetic waves.}}
	\minitoc
	\pagebreak
		%to force start on odd page
	\newpage
	\thispagestyle{empty}
	\mbox{}	
	\section{Electrostatics}
	\lettrine[lines=4]{\color{BrickRed}S}o far we have focus only on the gravitational interaction and the characteristic quantity of matter, named "mass" associated with it. We discussed the electromagnetic interaction, analyzing macroscopic phenomena such as friction, cohesion, elasticity, the forces of contact, etc. Now we look at electronic forces and the characteristic of matter named "\NewTerm{electric charge}\index{electric charge}" associated with them. The electromagnetic interaction binds matter in all its observable forms. It is this that holds the electrons to the nucleus in the atom, which holds together the atoms in molecules, molecules into objects and even your nose to your face...
	
	The "\NewTerm{electric charge}" generate the "\NewTerm{electric force}\index{electric force}" or "\NewTerm{Coulomb force}\index{Coulomb force}" and we are only beginning to understand this force thanks to quantum field theory (see further below in this book). The electric charge is a fundamental concept, which can not be described in terms of more simple and fundamentals concepts. We know it by its effects and unfortunately not by what it is (this was also the case for the mass before the discovery of the Higgs Boson).
	
	Experience has also shown that even if the electric charge is an additive quantity such as the mass, however, it also has negative value (and not exclusively positive as know nowadays for the mass). Thus, in the current language, and as confirmed by experience, two identical sign electric charges repel and two opposite sign electric charges attract (we will see a schematic figure of this further below).
	\begin{figure}[H]
		\centering
		\includegraphics[scale=0.9]{img/electromagnetism/electrostatic_cat.jpg}
		\caption[]{Electric charges exist all around us. They can cause objects to be repelled from each other or to be attracted to each
other. (credit: modification of work by Sean McGrath)}
	\end{figure}
	Let us now see the classic force that is associated with the electric charge:
	
	\subsection{Electric Force}
	It has experimentally been established by Coulomb that a reference point particle undergoes a force $\vec{F}$ of an intensity proportional to its charge $q$, when placed in the neighborhood of one or more electrical charges $Q_i$ in a medium of absolute electrical permittivity $\varepsilon$ (permittivity to the electric field of course...!) given by in vector notation and non-relativistic:
	
	where $\vec{r}_i$ is the vector position of the sample charge $Q_i$ and $\vec{r}$ of the reference point charge particle relatively to a same orthonormal vector basis.
	
	In other words, two point electric charged particles attract or repel each other in a force directly proportional to their electric charge and inversely proportional to the square of the distance that separates them.
	
	In the case of a system with two particles separated by a distance $r$, we have the same simplified relation and we fall back on the most common form of the electric force or "\NewTerm{Coulomb force}\index{Coulomb force}" as given in most books (as scalar and non-relativistic):
	
	Frequently, these relation is defined as the "\NewTerm{Coulomb's law}\index{Coulomb's law}" in most schools and admitted as unprovable. In fact, it is not! This relation can be proved as we will see in the study of quantum fields physics (\SeeChapter{see section Quantum Field Theory}) using the Klein-Gordon equation in the context of a potential field with spherical symmetry (proof performed by Yukawa).
	
	\begin{tcolorbox}[title=Remark,colframe=black,arc=10pt]
	Don't forget that as $2\pi$ appears frequently in many theorems of physics and mathematics (more than $\pi$ alone), the value $4\pi$ is frequently denoted by $2\tau$ as by definition $\tau=2\pi$.
	\end{tcolorbox}
	In either formulation, Coulomb's law is fully accurate only when the objects are stationary, and remains approximately correct only for slow movement. These conditions are collectively known as the electrostatic approximation. When movement takes place, magnetic fields that alter the force on the two objects are produced. The magnetic interaction between moving charges may be thought of as a manifestation of the force from the electrostatic field but with Einstein's theory of relativity taken into consideration.
	\begin{tcolorbox}[title=Remark,colframe=black,arc=10pt]
	For the relativistic form of Coulomb's law, the reader is referred to the section Special Relativity where it is proved that (vector form):
	
	\end{tcolorbox}
	The value of electric permittivity in vacuum is in turn given experimentally by the "\NewTerm{dielectric constant}\index{dielectric constant}" or simple "\NewTerm{electric constant}\index{electric constant}":
	
	and relatively to medium considered, we define a "\NewTerm{relative dielectric permittivity}\index{relative dielectric permittivity}" $\varepsilon$ that makes it easier to determine the properties of a material with respect to the electric field so that we have the "\NewTerm{absolute electrical permittivity}\index{absolute electrical permittivity}":
	
	It should be mentioned that some authors define the permittivity of vacuum from the speed of light and the magnetic permeability of vaccum (\SeeChapter{see section Magnetostatics}). Therefore, the value of the electrical permittivity of vacuum is obviously correct by definition. But this only makes sense once known the Maxwell's theory and this will be presented and proved later in the section Electrodynamics (we follow the steps in the historical scientific discoveries).
	
	 It also appears in the Coulomb force constant, as the "\NewTerm{Coulomb constant}\index{Coulomb constant}":
	  
	The factor into parentheses in:
	
	depends only on the distribution of charges $Q_i$ in the volume and the absolute electrical permittivity of the medium $\varepsilon$. Since its value varies from one place to another and depends on the position vector $\vec{r}$ of the reference electric charge, it forms a set of vectors, which property is this of a multitude of electrical field lines hence the use of term "\NewTerm{electric field}\index{electric field}".
	
	The set of these vectors $\vec{E}$ carries therefore the name "electric field", at the point $\vec{r}$ ,developed in the electric charge distribution $Q_i$:
	
	Engineers often use another notation that allows to characterize only the geometry of the field regardless the environment (medium) and for this purpose they introduce the concept of "\NewTerm{displacement field}\index{displacement field}":
	
	We will meet this vector again in the section Electrodynamics during our synthesis of Maxwell's equations.
	
	Coulomb force acting on the sample charge $q$, is then written in a conventional way:
	
	One configuration is of particular interest: two separated point charges of opposite charge. In the limit of vanishing separation, it is named "dipole". Its field fundamentally differs from that of just a single charge even though it is just the sum of the charge. The dipole as a concept is extremely important throughout electrodynamics. It is applied for example explaining the emission of electromagnetic radiation or as a model for molecules.
	
	Let us first consider the case of opposite electric charges. For the given problem we have $\vec{r}_1=-d/2\vec{e}_x$ and $\vec{r}_2=d/2\vec{e}_x$. So the charges lie on the $x$-axis with a separation $d$. Remember that $\vec{e}_x$ is the unit vector in $x$ direction (\SeeChapter{see section Vector Calculus}). The electric field is then given by:
	
	Imagine we are interested in the magnitude and the direction of the field only about the line of the $y$-axis. To figure it out both, we simply calculate:
	
	We see that the electric field has only a component in $x$-direction. Because of the symmetric choice of the coordinate system we could have guessed this in the first place.
	
	The magnitude is therefore given by the norm of the electric field:
	
	Let us now consider the case of equal charges (positive for example):
	
	The direction of the field is in this case always parallel to the $y$-axis but changing sign at $y=0$. Its magnitude is given by:
	
	We find that for equal charges the magnitude of the electric field decreases for large $y$.
	
	With Maple 4.00b we can easily plot these two magnitudes:
	
	\texttt{>plot({1/(0.5\string^2+y\string^2)\string^(3/2),1/(y\string^2)*1/((1/(2*y))\string^2+1)\string^(3/2)},y=-5..5);}
	\begin{figure}[H]
		\centering
		\includegraphics{img/electromagnetism/dipole_profile.jpg}
		\caption{Dipole profile for $q1=-q2$ and $q_1=q_2$ along $y$-axis}
	\end{figure}
	This is how looks like some lines of electric field of tow identical charges particles ($q_1=q_2$) with a nice perspective effect:
	\begin{figure}[H]
		\centering
		\includegraphics[scale=1]{img/electromagnetism/3d_dipole.jpg}
		\caption{3D dipole profile with $q_1=q_2$}
	\end{figure}
	
	\pagebreak
	\subsection{Electric Potential}
	 Given two points $A$ and $B$ in a region of space where there is an electric field $\vec{E}(x,y,z)$ and given $\Gamma$ a path connecting these two points. So, in the particular case where the source of a field $\vec{E}$ is a sphere or a punctual body and we put  a charge as its neighborhood, we have for the work done by the force to move the charge from point $A$ to point $B$:
	 
	Moreover, this work is as we shall see, equivalent to the potential energy. We thus define the "\NewTerm{potential difference}\index{potential difference}" or simply the "\NewTerm{potential}\index{potential}" by the relation:
	
	and therefore:
	
	\begin{tcolorbox}[title=Remarks,colframe=black,arc=10pt]
	\textbf{R1.} The potential is often named "\NewTerm{voltage}\index{voltage}" by electricians, electrical engineers and other engineers and the unit of measurement of the potential that is the "\NewTerm{Volt}\index{Volt}" denoted by [V].\\
	
	\textbf{R2.} The potential difference can either exists between two terminals of opposite charge $(+, -)$, or between two terminals of the type $(+, \text{neutral})$ or also $(-, \text{neutral})$. The latter two cases represents typically the configuration used by trains, trams, storms and almost all electromechanical appliances.
	\end{tcolorbox}
	\begin{theorem}
	We will now show in the more general framework that exists that the stationary vector field $\vec{E}$ derivates from a potential field:
	\end{theorem}
	\begin{dem}
	Given a charge $Q$ located relative to a reference frame by the vector $\vec{r}_Q$. Then in each point of space there exists a field $\vec{E}$ such as:
	
	let us develop this expression:
	
	If $\vec{E}$ is a stationary potential field then, there must be a potential $\Phi(x,y,z)$ of this field which satisfies:
	
	Let us look to the potential $\Phi$ exists for a Coulomb field. Then we must have for the field in $x$:
	
	therefore:
	
	and if we do the same development for each component, we also get the same result. Thus the electric potential is a scalar field and not a vector field (the electric field is it obviously a vector field)!
	\begin{flushright}
		$\square$  Q.E.D.
	\end{flushright}
	\end{dem}
	The potential $\Phi(x,y,z)$ is named in the case of a Coulomb field of "\NewTerm{Coulomb potential}\index{Coulomb potential}" and is conventionally chosen such that:
	
	As we can see it by the prior-previous expression of $\Phi(x,y,z)$, $c^{te}$ is an arbitrary constant, which imposes in the case of absence of charges that:
	
	Which finally gives us:
	
	Giving for all components:
	
	that we write in a more condensed way:
	
	\begin{tcolorbox}[title=Remark,colframe=black,arc=10pt]
	The same developments and results (and those that follow) are applicable regarding the gravitational potential field. However, they rarely are made in the literature or schools because the human being does not control (at least until now) the gravitational field with an ease and intensity equivalent to that of the electric field...
	\end{tcolorbox}
	It therefore follows that:
	
	and so for example for amspherical symmetric potential (case that we will find in many other sections of this book), it comes:
	
	
	\pagebreak
	\subsubsection{Path Independance}
	Let us prove now that the potential difference between two points $A$ and $B$ is independent of the path $\Gamma$ traveled such as we did for the gravitational potential field in the section of Classical Mechanics.

	Given $\Gamma$ a path between two points $A$ and $B$ and a field $\vec{E}$ and let us make so that we can express the field on $x$, $y$ and $z$ with respect to a single variable $t$ (which has nothing to do with time...) that would account for its magnitude change for  any movement between these two points:
	
	So with $A$ corresponding to the value $t_1$ of the parametrization and $B$ to the value $t_2$. 
	
	But, we know that (\SeeChapter{see section Differential and Integral Calculus}) that:
	
	Therefore it comes:
	
	Therefore:
	
	This last expression shows that $U$ is independent of the path $\Gamma$ regardless of the way we parametrize it.
	
	The Coulomb field is therefore a "\NewTerm{conservative vector field}\index{conservative vector field}" Indeed, if we consider a closed path $\Gamma$ and $A$ and $B$ are two confused points of the part then the potential difference will be zero!
	
	Notice that sometimes we also say that the gradient of the potential is conservative.

	\pagebreak
	\subsection{Equipotential and Field lines}
	Now we can from what we have built, define the "equipotentials" and "field lines".
	
	Given a Coulomb field defined relatively to a given repository (reference frame). Then at each point $(x, y, z)$ of space, we can associate an electric field vector $\vec{E}(x,y,z)$ and an electric potential.
	
	\textbf{Definition (\#\mydef):} The "\NewTerm{field lines}\index{field lines}" is a family of curves for which the vector $\vec{E}(x,y,z)$ is tangent and constant at each point and the "equipotentials" as the lines for which the potential $U (x, y, z)$ are also constant.
	
	\begin{theorem}
	In this case, and that is what we will show, all field lines are perpendicular to all equipotentiels if the vector field derived from a potential.
	\end{theorem}
	Let us use the property conservation of the Coulomb field for the proof:
	\begin{dem}
	
	As we are in the presence of an electric field, this therefore derives from a potential as we know. This implies that if the field is not null the potential there is also not. So in the line integral:
	
	one of the terms is zero! This is not the electric field $\vec{E}$ as we in presence of one, which discredits the potential $U$ and as the charge moves $\mathrm{d}\vec{r}$ is not null either. Then write the line integral in another way:
	
	Therefore:
	
	we can then conclude that the equipotential are perpendicular to the lines of the electric field and vice versa. This is what was to be proved.
	\begin{flushright}
		$\square$  Q.E.D.
	\end{flushright}
	\end{dem}
	Here are examples of level lines including field lines and equipotential lines obtained using Maple 4.00b (we will show in our study of differential equations how to get the mathematical functions of the field lines):
	\begin{figure}[H]
		\centering
		\includegraphics{img/electromagnetism/field_and_equpotential_lines_01.jpg}
		\caption{Left: a single charge - Right: two charges of the same sign}
		\includegraphics{img/electromagnetism/field_and_equpotential_lines_02.jpg}
		\caption{Left: two charges of opposite signs - Right: four charges of the same sign}
	\end{figure}
	\begin{tcolorbox}[title=Remark,colframe=black,arc=10pt]
	Apart from the opposite charges, we recall that the same results are applicable to the masses with the gravitational field!
	\end{tcolorbox}
	Two applications of these results are very important (for which we will limit ourselves to the study of the most important properties):
	\begin{enumerate}
		\item The determination of the field lines and equipotential lines for an infinite straight wire such as we can consider in approximation in the electric circuits or high voltage overhead lines (to determine the influence of fields from wires with their environment - this study is part of the branch of electro-engineering we name EMC for "\NewTerm{ElectroMagnetic Compatibility}\index{ElectroMagnetic Compatibility}"). The results can also be used to determine the "\NewTerm{voltage step}\index{voltage step}" for some rectilinear systems which determines for a given distance, the potential per meter for which a mammal may be killed by electroshock near such a wire. An extension (which I do not wish to deal even if the subject is exciting but very contreversial) is also the influence of this type of potential on the functioning of the human brain in the case of the use of portable phones (antennas transmitting a potential) or about houses near high voltage power lines...
		\begin{tcolorbox}[title=Remark,colframe=black,arc=10pt]
		We will determine in the section of Magnetostatics the Biot and Savart law giving the magnetic field for such a wire carrying a given current intensity.
		\end{tcolorbox}
		
		\item The determination of the field lines and equipotential of an electric dipole has a huge importance in chemistry. We will also see what is the dynamics of it when immersed in a uniform electric field and the energy of interaction between dipoles (as is often the case in chemistry).
	\end{enumerate}
	
	\subsubsection{Infinite straight wire}
	Given:
	
	We have:
	
	by making use of the concept of linear charge density as we have defined it in the section of Principles of Mechanics, we have:
	
	Let us consider an infinite line (wire) of negligible section, and carrying a linear continuous density charge $\gamma$. The goal is then to calculate the electric field and potential at any point $M$ of the space outside this line (wire) in order to know the influences of charges of this line (wire) on the environment by considering the influence of the electric field (if the charges were moving we should also take into account the influence of the magnetic field, which we will do in the section Magnetostatics).
	
	For this, the method is to cut the line (wire) into small elements of line (wire) $\mathrm{d}l$, each element carrying a charge load $\mathrm{d}q$.  The electric field created by the charge load on $P$ at point $M$ located at a distance $x$ and of orthogonal projection $H$ on the line is:
	
	The trick now is to take the symmetric $P'$ of $P$ with respect to $H$ (the orthogonal projection of $M$ on the wire)
	\begin{figure}[H]
		\centering
		\includegraphics{img/electromagnetism/infinite_straight_wire.jpg}
		\caption{Configuration for the analysis of the electrif field of an infinite straight wire}
	\end{figure}
	for which we have identically:
	
	The total field is therefore:
	
	But we have:
	
	Therefore:
	
	As we might expect, the latter relation shows that the field is perpendicular to the line (to the wire...).
	The norm of $\mathrm{d}\vec{E}(M)$ is:
	
	This relation has three dependent variables $r$, $\mathrm{d}l$, $x$. The norm of the total field on a point is equal to the sum of the norms of all the vectors $\mathrm{d}\vec{E}(M)$ of the entire length of the wire since all vectors have the same direction.
	
	For this calculation, we will make a change of variable, and put $r$, $\mathrm{d}l$, $x$ according to the angle $\alpha$ between the line (wire) and the vector $\overrightarrow{PM}$. In the rectangle triangle $HMP$:
	
	if we take the origin of $z$ in $H$. We also have:
	
	and:
	
	therefore:
	
	
	and therefore:
	
	The potential is easily deduced by taking the primitive of $E$ since:
	
	Then we have:
	
	The constant is undetermined since as $r$ approaches infinity, $U$ tends therefore to zero and leads to an infinite constant. This uncertainty is mainly due to the approximation of the infinite wire length.

	\subsubsection{Electric Rigid Dipole}
	An important and interesting disposition of electric charges is that forming an "\NewTerm{electric dipole}\index{electric dipole}" rigorously named "\NewTerm{rigid electric dipole}\index{rigid electric dipole}" or "\NewTerm{electrostatic dipole}\index{electrostatic dipole}". It consists of two equal and opposite electric charges $+ q, -q$ separated by a very small distance. We will seek to determine the potential and the electric field at a point $M$ of the dipole environment.
	
	To determine this let us consider an electric charge $q_i$ on a point $A_i$ and a very distant point $M$ from $A_i$. Let us take an galileen reference frame in O:
	\begin{figure}[H]
		\centering
		\includegraphics{img/electromagnetism/elecrostatic_dipole_study_configuration.jpg}
		\caption{Electric field at a distant point $M$ of the dipole}
	\end{figure}
	The electric potential created at the $M$ by the charge $q_i$ is:
	
	In the triangle $\text{O}A_i M$, the distance $\overline{A_i M}$ can be written with the cosine theorem:
	
	The potential becomes:
	
	At great distance, $r$ becomes much higher than $r_i$, the quantity:
	
	tends to zero. So we can make a Maclaurin development (\SeeChapter{see section Sequences And Series}) of $(1+u)^{-1/2}$ at the neighborhood of $u=0$. To avoid making heavy calculations, we will limited ourselves to the order two on $r$:
	
	therefore:
	
	Keeping only the terms of the second order in $r$:
	
	The potential becomes:
	
	We kept in the expression of the potential three terms. The term $U_i^0$ is the potential created by a charge that would be in O. In other words, at the zero order, the potential created by a charge located at a point near O is identical to the potential created by a charge that would be in O. The terms $U_i^1,U_i^2$ are correction terms, at the first order and second order respectively. We notice that these terms vary on $1/r^2$ and $1/r^3$, thus decreasing faster than the first. These two terms are thus more effective at smaller distances.
	We see that the terms $U_i^1,U_i^2$ involve the quantity $q_ir_i$. This quantity is what we define as the "\NewTerm{dipole moment}\index{dipole moment}" of the electric rigid dipole:
	
	\begin{tcolorbox}[title=Remark,colframe=black,arc=10pt]
	The dipole moment is expressed in Coulomb meter, but for convenience (...) it is expressed in Debye [D] by some engineers.
	\end{tcolorbox}
	The potential created at great distance by a discrete charge distribution is obtained by summing all individual contributions:
	
	Which can also be written:
	
	By definition, $U_0(M)$ is unipolar or monpolar term, $U_1(M)$ is the dipolar term, $U_2(M)$ is the quadrupolar term. If the electric charge distribution is a globally zero, as is the case of an ideal atom or of an ideal non-ionized molecule, it only remain the multipolar contributions.
	
	Let us return to the particular case of the dipole:
	
	The term monopolar term is zero, since the sum of the electric charges is zero. If we neglect the terms higher than the first order, there remain the dipole contribution.
	
	The angles $\theta_1$ and $\theta_2$ of the dipole are complementary, therefore $\cos(\theta_1)=-\cos(\theta_2)$. But as $q_1=-q_2=q$, the product $q_i\cos(\theta_i)$ is constant.
	
	If the two charges of the dipole are at a constant distance from each other and equidistant from the origin O. We will put that $r_1=r_2=d$.

	The potentiel is therefore reduce to:
	
	
	where $a$ is simply the constant distance between the two electric charges.
	It is customary in the case of the study of the electric dipole to write the above relation as:
	
	where $\vec{p}$ is the definition of the dipolar moment and:
	
	Let us recall now that we proved at the beginning of this section that:
	
	and as we saw in section of Vector Calculus, the gradient in spherical coordinates leads us to write:
	
	therefore:
	
	To determine the equation of the equipotential, remember that these lines (or "surfaces" when in space) are obtained through by the constraint:
	
	Therefore:
	
	with:
	
	The electric field must by definition be tangential to the field lines, thus parallel to the elementary displacement:
	
	Since $E_\phi=0$, we have:
	
	So finally there remains only:
	
	Which is a differential equation that can be easily integrate:
	
	Which is equivalent to write:
	The plot of the field and equipotential lines  then gives in spherical coordinates (remember that the vertical component is zero by symmetry):
	\begin{figure}[H]
		\centering
		\includegraphics{img/electromagnetism/dipole_electric_equipotentials.jpg}
		\caption{Polar plot of the field lines and equipotentials of an electric dipole}
	\end{figure}
	Even if in an electric dipole the two charges are equal and opposite, giving a zero net charge, the fact they are slightly displaced is sufficient to produce a non identically zero electric field. In atoms, the center of mass of the electrons coincides with the core, and therefore the average electric dipole moment of the atom is zero. But if an external field is applied, the movement of electrons is distorted and the electron mass center is displaced by a given distance from the core. The atom is then polarised and becomes an electric dipole of moment $\vec{p}$. This moment is proportional to the external applied field $\vec{E}$.

	As we did not found in Maple 17.00 how to do the polar plot above here is another Maple 17.00  code to have fun with a dipole:
	
	\texttt{>V:=1/sqrt((x-1)\string^2+y\string^2+z\string^2)-2/sqrt((x+1)\string^2+y\string^2+z\string^2):\\
	>with(LinearAlgebra); with(VectorCalculus); with(plots);\\
	>with(plottools);\\
	>Efield := Gradient(-V, [x, y, z]);\\
	>fieldplot3d([Efield[1], Efield[2], Efield[3]],x =-1.5..1.5, y =-1.5..1.5, z =-1.5..1.5);\\
	>NormEfield := Normalize(Efield,2);\\
	>p1:=sphere([1, 0, 0],0.75,color=red);\\
	>p2:=sphere([-1, 0, 0],1.5,color=blue);\\
	>p3:=fieldplot3d(NormEfield,x=-4.5..4.5,y=-4.5..4.5,z =-4.5..4.5,color=black);\\
	>display([p1, p2, p3], scaling = constrained);
	}

	Giving at first (obviously the dimensions are fictitious):
	\begin{figure}[H]
		\centering
		\includegraphics{img/electromagnetism/dipole_electric_field_maple.jpg}
		\caption{Electric field lines for two charges of opposite signs with Maple 17.00}
	\end{figure}
	Or to get the view of the potential weel with the equipotentials:

	\texttt{>z:= 0;
	>plot3d(V,x =-1.5..1.5, y=-1.5..1.5, style=patchcontour,contours=200);
	}

	\begin{figure}[H]
		\centering
		\includegraphics{img/electromagnetism/dipole_potential_well.jpg}
		\caption{Potential well and equipotentiels of the dipole with Maple 17.00}
	\end{figure}
	Or to get a 2D profile of the electric field and equipotentials:
	
	\texttt{>p4:=implicitplot({seq(V=(1/10)*b,b=-10..10)},x=-5..5,y=-4..4);\\
	>p5:=fieldplot([NormEfield[1],NormEfield[2]],x=-5 .. 5,y=-4..4); 
	\\>display([p4, p5], scaling = constrained);\\
	}
	\begin{figure}[H]
		\centering
		\includegraphics{img/electromagnetism/dipole_potential_field_profile_maple.jpg}
		\caption{2D profile of electric field and potential of the dipole with Maple 17.00}
	\end{figure}
	\begin{tcolorbox}[title=Remark,colframe=black,arc=10pt]
	Molecules also may have a permanent electric moment. Such molecules are named "\NewTerm{polar molecules}\index{polar molecules}" For example, in the HCl molecule the electron of the hydrogen atom spends more time to move around the chlorine that around the hydrogen atom. Also, the center of negative charges does he not coincide with that of the positive charge of the center and the molecule has a dipole moment. By cons, in the molecule $\text{CO}_2$, all the atoms are aligned, and the resulting electric dipole moment is zero for reasons of symmetry.
	\end{tcolorbox}
	When an electric dipole is positioned in an electric field, a force is exerted on each of the charges of the dipole. The resulting force is:
	
	Let us consider the particular case where the electric field is directed along the $x$-axis and where the dipole is oriented parallel to this field. If we only consider the quantities:
	
	with $a$ being the distance between the two charges, and therefore:
	dipole is oriented parallel to this field. If we only consider the quantities:
	
	This result show that an electric dipole oriented parallel to the field tends to move in the direction in which the field increases (as the gradient thereof). We note that if the electric field is uniform, the resultant force on the dipole is zero!!!
	
	The potential energy of the dipole is:
	
	If we use the equation:
	
	to describe the uniform electric field and if $\theta$ is the angle between the dipole and the electric field, the last factor $(U_+ - U_-)/a$ is just the component $E_a=E\cos(\theta)$ of the field $\vec{E}$ parallel to $a$. So:
	
	or:
	
	The potential energy is minimum for $\theta=0$, which shows that the dipole is in equilibrium when it is oriented parallel to the field.

	These configurations of a dipole placed in an electric field have very important applications. For example, the electric field of an ion in a solution polarized the solvent molecules that surround the ions and they are oriented as in the figure below:
	\begin{figure}[H]
		\centering
		\includegraphics{img/electromagnetism/dipole_solvant.jpg}
		\caption{Example of what happens in a solution with an ion}
	\end{figure}
	In a solvent with polar molecules such as water, ions of an electrolyte in solution are surrounded by a number of these molecules due to the dipole-charge interaction. This phenomenon is named "solvation" of the ion, specifically "hydration" if the solvent is water.
	
	For example with Salz (NaCl):
	\begin{figure}[H]
		\centering
		\includegraphics{img/electromagnetism/nacl_hydration.jpg}
		\caption{Hydration of Salz (source: ?)}
	\end{figure}
	These oriented molecules become more or less dependant of the ion, increasing its effective mass and decreasing its effective charge, which is partially obscured by the molecules (screening). The net effect is that the mobility of a ion in an external field is reduced. Similarly, when a gas or liquid, which  molecules are permanent dipoles is placed in an electric field, the molecules as a result of the force couple applied due to the electric field, tend to align with their dipoles parallel. We then say that the substance was "\NewTerm{polarized}\index{polarized}".

	It may therefore be interesting to determine the vector electric field produced by a dipole rather than its potential. The electrostatic field created at a point $M$ by the doublet is obtained by performing the vector sum of the fields created in this point by positive $P$ and negative $N$ charges, hence:
	
	The distribution of charges being invariant by rotation about the $z$ axis of the doublet, the topography is independent of the azimuthal angle $\pi$ of the spherical coordinates. We can then represent it in any meridian plane passing through the axis $NP$. The field $\vec{E}$ is the given by:
	
	Having:
	
	vectorially, we get:
	
	The dot product being the multiplication of components one by one, we have:
	
	Hence:
	
	Finally:
	
	So by a limited Maclaurin series development as we did at the beginning:
	
	By introducing:
	
	We can simplify the notations:
	
	It may also be relevant to calculate the energy of interaction between two electric dipoles. If we denote by $\vec{p}_1$ the dipole moment, we can write:
	
	If we denote by $\vec{p}_2$ the moment of the second dipole and if we use the relation:
	
	we find that the interaction energy between two dipoles is:
	
	We can draw out several important conclusions from this result. The energy of interaction $E_{P,1,2}$ is symmetric relatively to the two dipoles, because the permutation of $\vec{p}_1$ and of $\vec{p}_2$ leaves it unchanged. This is an expected result. The interaction between two dipoles is not central as it depends of the angles which the vector position or the unitary vector $\vec{u}_r$ do with $\vec{p}_1$ and $\vec{p}_2$.
	
	An atom, molecule or a ion, whose dipole moment is zero in the fundamental state, acquire a dipole moment under the action of the applied non-uniform electric field as we have seen since the opposing signs charges are sollicited in opposed directions. The centers of gravity of positive and negative charges do not coincide anymore, it appears an "\NewTerm{induced dipole moment}\index{induced dipole moment}". 

	In an experiment with linear approximation for weak excitatory electric fields, the induced dipole moment is proportional to the applied field $\vec{E}$, which we translate by  (it is in fact an approximation of the Langevin-Debye relation we will prove later):
	
	The quantity $\alpha$, which physical dimension is that of a volume is the "\NewTerm{polarizability}\index{polarizability}" of the structure. The dipole-dipole electrostatic interaction was introduced by J.D. Van der Waals in 1873, in the case of molecules, to interpret real deviations from the ideal gas.
	
	The Van der Waals forces are repulsive as the distance between molecules is very low because they oppose the interpenetration of electron clouds, what we express by introducing their volume (covolume).

	By cons, they are attractive when the distance is sufficient. We attribute this attraction to three types of interactions involving rigid or induced dipoles:
	\begin{enumerate}
		\item The forces between polar molecules (rigid dipoles), say to be "\NewTerm{Keesom forces}\index{Keesom forces}".

		\item The forces between a polar molecule (rigid dipole) and a polarizable molecule (induced dipole) named "\NewTerm{Debye forces}\index{Debye forces}".

		\item	The average forces between induced dipoles which appear even when the molecules are not polar, named "\NewTerm{London forces}\index{London forces}".
	\end{enumerate}
	In all these three cases, the electrostatic energy is negative (attraction) and varies as $r^{-6}$. To prove this statement, let us calculate the interaction energy between two rigid dipoles, of dipolar moments $\vec{p}_1$ and $\vec{p}_2$:
	
	with:
	
	and:
	
	Therefore:
	
	This is the van der Waals interaction potential between two dipoles atoms where $C_6$ is the "\NewTerm{van der Waals constant}\index{van der Waals constant}".
	
	Hence:
	
	Thus, the radial dependence of the force is indeed in $r^{-7}$. This very rapid decay of the Van der Waals force with distance helps explain its very short range and therefore its influence when the medium is sufficiently dense. However the decay is slower than the chemical bond that deacrease exponentially (\SeeChapter{see section Quantum Chemistry}).
	\begin{tcolorbox}[title=Remark,colframe=black,arc=10pt]
	The interaction between polar molecules, of the Keesom type, is made very high in the presence of hydrogen, because the latter, thanks to its small size, also interacts with the atoms of other molecules. It is that one which is at the origin of the "\NewTerm{hydrogen bonding}\index{hydrogen bonding}".
	\end{tcolorbox}
	
	\subsection{Electric Field Flow}
	Given $\vec{V}$ a vector field and $S$ a surface named "\NewTerm{Gauss surface}\index{Gauss surface}" in space. If we divide this area into a number $N$ of smaller surfaces $\mathrm{d}S$ each traversed by a field $\vec{V}_i$ and having a unit perpendicular vector $\vec{n}_i$ (special case) on their surface, then we can form the sum of:
	
	When $N$ tends to infinity and all $\mathrm{d}S$ to zero, we get for this sum:
	
	The value of this integral thus gives the flow $\Phi_V$ of the field $\vec{V}$ through the surface $S$ delimited by a domain $\Lambda$ and where:
	
	In the case of the electrostatic field, we write:
	
	This expression, define the "\NewTerm{electric flow}\index{electric flow}" or also named "\NewTerm{electric flux}\index{electric flux}".	
	
	The inevitable question that arises is then: what is its physical meaning? The flux of a fluid is the amount of fluid (especially the quantity of volume) passing through a surface by second. Then there is a flux (flow) of something. About the electrical flux (flow), from the classical point of view, nothing flows, the electric field is already established and is static, but is flow through the surface. The value of the electric field at any point in space is the field intensity at that point, while the flow can be considered the quantity of field which passes through the surface $S$. 

	There is a hundred years, physicists identified the flow with the number of electric field lines passing through the surface. But the least we can say is that the simplistic view that the field lines have a distinct reality and that we can count is misleading. We will see during our study of Quantum Field Theory, in the section of the same name, that the latter supports a virtual photons current that is the nature of electromagnetic interactions. Despite this, the physicists did not hurry to associate the flow of virtual photons of the 20th century to the continuous electric field lines of the 19th century. Whatever its nature, the concept of flow is powerful and of great practical use, both in electricity and magnetism.
	
	As will be prove it in the framework of Maxwell's equations (\SeeChapter{see section Electrodynamics}), solving this integral gives in the general case (this is the "Gauss law" also named "Gauss theorem"):
	
	In the case where the surface is not closed and reduce to a plane, the previous close surface integral is reduced to:
	
	
	\subsubsection{Capacitor}
	A capacitor (originally known as a condenser) is a passive two-terminal electrical component used to store electrical energy temporarily in an electric field. The forms of practical capacitors vary widely, but all contain at least two electrical conductors (plates) separated by a dielectric (i.e. an insulator that can store energy by becoming polarized). The conductors can be thin films, foils or sintered beads of metal or conductive electrolyte, etc. The nonconducting dielectric acts to increase the capacitor's charge capacity. Materials commonly used as dielectrics include glass, ceramic, plastic film, air, vacuum, paper, mica, and oxide layers. Capacitors are widely used as parts of electrical circuits in many common electrical devices. Unlike a resistor, an ideal capacitor does not dissipate energy. Instead, a capacitor stores energy in the form of an electrostatic field between its plates.

	When there is a potential difference across the conductors (e.g., when a capacitor is attached across a battery), an electric field develops across the dielectric, causing positive charge $+Q$ to collect on one plate and negative charge $-Q$ to collect on the other plate. If a battery has been attached to a capacitor for a sufficient amount of time, no current can flow through the capacitor. However, if a time-varying voltage is applied across the leads of the capacitor, a displacement current can flow.

An ideal capacitor is characterized by a single constant value, its capacitance. Capacitance is defined as the ratio of the electric charge $Q$ on each conductor to the potential difference $U$ between them. The SI unit of capacitance is the farad [F, which is equal to one coulomb per volt ($1$ [C$\cdot$V$^{-1}$]). Typical capacitance values range from about $1$ [pF] to about $1$ [mF].

	The larger the surface area of the "plates" (conductors) and the narrower the gap between them, the greater the capacitance is. In practice, the dielectric between the plates passes a small amount of leakage current and also has an electric field strength limit, known as the breakdown voltage. The conductors and leads introduce an undesired inductance and resistance.

	Capacitors are widely used in electronic circuits for blocking direct current while allowing alternating current to pass. In analog filter networks, they smooth the output of power supplies. In resonant circuits they tune radios to particular frequencies. In electric power transmission systems, they stabilize voltage and power flow.
	\begin{figure}[H]
		\centering
		\includegraphics[scale=0.25]{img/electromagnetism/capacitors.jpg}
		\caption{Some capacitive dipoles (source: Martin Bircher http://www.e-style.ch)}
	\end{figure}
	As a direct application of Gauss's theorem, very useful in electronics and for engineers, consider a large thin a flat sheet, wearing a homogeneous surfacic electronic charge density $\sigma$ and immersed in an environment of absolute electrical permittivity $\varepsilon$. In the area close to its center, the electric field resulting from all the fields of the fields is normal, uniform, constant and go away from the sheet. Let us consider a Gaussian surface with a cylinder shape limited by the bases $S_1=S_2=S$ and its tubular surface $S_3$ symmetrical with respect to the sheet. It therefore encloses an electric charge $\sigma S$. It follows that:
	
	and as $E=E_1=E_2$ and $E_3=0$, we find:
	
		Finally, the electric field of a large and thin loaded flat sheet is:
	
	If we put face to face two identical plates but with opposite charges, the algebraic sum will of course be:
	
	Excepted of the extremities where the side effect is important, the overall field is everywhere the vector sum of uniform fields from two opposing thin loaded layers. We name such a system a "\NewTerm{plane and parallel capacitor}\index{plane and parallel capacitor}":
	\begin{figure}[H]
		\centering
		\includegraphics{img/electromagnetism/plane_capacitor.jpg}
		\caption{Schematic diagram of plane and parallel capacitor}
	\end{figure}
	The result is remarkable because it is independent of the distance between the planes (in fact never forget it is an approximation for very small distances!). The calculation of electric potential is therefore simplified. Thus:
	
	Thus, the capacitance of the plane and parallel capacitor is consequently:
	
	Let us now recall that we have shown previously that:
	
	Since each plate of a parallel and plane capacitor contributes equally, the total electric field between the plates would be :
	
	The potential difference is:
	
	Solving for $Q$ yields:
	
	The plates are oppositely charged, so the attractive force $F_\text{att}$ between the two plates is equal to the electric field produced by one of the plates times the charge on the other:
	 
	This is therefore the force between the plates of a parallel (infinite) plate capacitor.
	
	The plates of a cylindrical capacitor are two infinite rolled cylinders (or very long relatively to their diameter), coaxial of respective radius $R_1$ and $R_2$. It is therefore the very important case of the coaxial cable (which dielectric is often polyethylene) that we can found in many laboratories (and not only!):
	\begin{figure}[H]
		\centering
		\includegraphics{img/electromagnetism/capacitors_cylindrical.jpg}
		\caption{Schematic diagram of cylindrical capacitor}
	\end{figure}
	Let us see a second academic example that is the "\NewTerm{cylindrical capacitor}\index{cylindrical capacitor}":

	By Gauss theorem, we know that:
	
	And since the field $\vec{E}$ is collinear at every point of the surface $\vec{S}$, it immediately comes by knowing the expression of the surface of the cylinder (\SeeChapter{see section Geometric Shapes}):
	
	But:
	
	Therefore:
	
	
	Let us also calculate the capacity of a spherical capacitor which corresponds in a first approximation to some of Van Der Graaf generators that we have in the labs of some schools, museums or even research centers...
	
	A "\NewTerm{spherical capacitor}\index{spherical capacitor}" consists of two concentric spheres of radius $R_1$ and $R_2$ with $R_1<R_2$.
	\begin{figure}[H]
		\centering
		\includegraphics{img/electromagnetism/capacitors_spherical.jpg}
		\caption{Schematic diagram of spherical capacitor}
	\end{figure}
	We now have immediately:
	By Gauss theorem, we know that:
	
	And therefore since the field is assumed to be colinear at any point on the surface it comes immediately knowing the expression of the surface of a sphere:
	
	But:
	
	Then we have:
	
	Then we have:
	
	And thats all for the classical examples...
	
	Well! We have just seen that the capacity was defined as:
	
	or in alternative current notation (\SeeChapter{see section Electrokinetics}):
	
	We then for the instantaneous power (\SeeChapter{see section Electrokinetics}):	
	
	Assuming an ideal capacitor (which does not dissipate energy by Joule effect) we get:
	
	
	This energy is always positive and is stored in electrostatic form in the capacitor.
	
	In the context of a sine wave alternative current, the average power will be equal to zero. We can generalize this by assuming that a perfect capacitor does notdissipate power no Joule effect.
	\begin{tcolorbox}[title=Remark,colframe=black,arc=10pt]
	Some scientific experiments requiring enormous energy use thousands of giant capacitors charged in the long term to accelerate particles or to make operate megajoules LASER. However we can not actually store the surplus of electric power of some power plants, which is why we use this surplus to pull back the water in the dams that can use their pool as a reserve of potential energy to produce a complement to electricity at the time of peak consumption (inverse transformation). We can also found huge capacitor that have for purpose to manage critical voltage collapse in a changing environment (this are the big cylinder in transformation stations).
	\end{tcolorbox}
	
	
	\paragraph{Dielectric strength}\mbox{}\\\\
	The term "\NewTerm{dielectric strength}\index{dielectric strength}" denoted $\mathcal{R}$ and that for SI units volts per meter [V$\cdot$m$^{-2}$] has the following meanings:
	\begin{itemize}
		\item Of an insulating material, the maximum electric field that a pure material can withstand under ideal conditions without breaking down (i.e., without experiencing failure of its insulating properties).
		
		\item For a specific configuration of dielectric material and electrodes, the minimum applied electric field (i.e., the applied voltage divided by electrode separation distance) that results in breakdown.
	\end{itemize}
	The theoretical dielectric strength of a material is an intrinsic property of the bulk material and is independent of the configuration of the material or the electrodes with which the field is applied. This "intrinsic dielectric strength" corresponds to what would be measured using pure materials under ideal laboratory conditions. At breakdown, the electric field frees bound electrons. If the applied electric field is sufficiently high, free electrons from background radiation may become accelerated to velocities that can liberate additional electrons during collisions with neutral atoms or molecules in a process named "avalanche breakdown". Breakdown occurs quite abruptly (typically in nanoseconds), resulting in the formation of an electrically conductive path and a disruptive discharge through the material. For solid materials, a breakdown event severely degrades, or even destroys, its insulating capability.

	The factors affecting apparent dielectric strength are typically:
	\begin{itemize}
		\item it decreases with increased sample thickness.
		\item it decreases with increased operating temperature.
		\item it decreases with increased frequency.
		\item for gases (e.g. nitrogen, sulfur hexafluoride) it normally decreases with increased humidity.
		\item for air, dielectric strength increases slightly as humidity increases
	\end{itemize}
	For a capacitor used in electronics, if we exceed the dilectric strenght value, we observe the destruction of the element. This maximum value of the voltage applied to the terminals, is named sometimes the "\NewTerm{breakdown voltage}\index{breakdown voltage}" of the capacitor and denoted $U_c$ . We can define the dilectric strength of the material (medium) as:
	
	\begin{tcolorbox}[colframe=black,colback=white,sharp corners]
	\textbf{{\Large \ding{45}}Example:}\\\\
	Two common dielectric strength values that have to be know by students are:
	
	\end{tcolorbox}
	When we talk of dielectric strength, we also speak of the dielectric is an insulator or a substance that does not conduct electricity and is polarizable by an electric field. In most cases, the dielectric properties are due to the polarization of the substance. When the dielectric (in this case, air is the dielectric) is placed in an electric field, the electrons and protons of its atoms are reoriented and, in some cases, at the molecular level, a polarisation is induced (as we have seen in our study of the dipoles above). This polarization produces a potential difference, or voltage, between both terminals of the dielectric; it then stores the energy which becomes available when the electric field is removed. The effectiveness of a dielectric is its relative ability to store energy compared to that of vacuum. It is expressed by the relative electric permittivity $\varepsilon_r$, determined relative to that of a vacuum $\varepsilon_0$. The dielectric strength is then ability of a dielectric to withstand electric fields without losing its insulating properties. An effective dielectric releases a big fraction of the energy it had stored when the electric field is reversed.
	
	\subsection{Electrostatic potential energy}
	Let us consider two electric charges $q_1,q_2$. The first is supposed to be at rest and fixed; the second is brought from infinity to a distance $a$ from $q_1$ (the same reasoning was applied to the gravitational field in the section of Classical Mechanics!). Suppose that the two charges are of the same sign. As $q_1,q_2$ tend to repel each other, we have to provide a potentielle energy $E_p$ to take $q_2$ (infinitely slowly) of $q_1$. The work $\mathrm{d}W$ done by the electrostatic force at any point is by definition as we know:
	
	The potential enery of the system is:
	
	as the force $F$ is resistive (hence the origin of the sign "$-$").

	Therefore:
	
	We then get the potential energy at a point ($x$ in the numerator is simplified with an $x$ in the denominator) to a given sign convention:
	
	
	If the path is not linear we have obviously as in Classical Mechanics with the electric field and using the notations seen at the begining of the section:
	
	Notice that:
	
	can also be written as:
	
	\begin{tcolorbox}[title=Remark,colframe=black,arc=10pt]
	Caution! When do physics, we have to be sure what potential energy we are about talking. This is a major problem! If you take for example the potential energy due to gravitational force, it can take any value depending on the reference point. If the reference is at sea level, a point below the sea level will have a negative potential energy, by cons if the reference is the center of the Earth, there will be only positive potential energies. That is why we are writing rather the potential energy in the form of height difference relative to a reference in mechanics. For the potential energy of the electron, we must know with what it interacts. If it is with a negative charge, the product of the electric charge values will be positive and therefore the potential energy of interaction will be positive, if it interacts with a positive charge, the product will be negative and the potential energy will be negative. In short, we must know what we are talking about. This is an example where words are important in physics too (same problems as the foundations of mathematics!).\\
	
	In general, if the potential energy decreases when the distance increase, the force is repulsive if it increases when the distance increse, the force is attractive.
	\end{tcolorbox}
	
	\begin{flushright}
	\begin{tabular}{l c}
	\circled{90} & \pbox{20cm}{\score{3}{5} \\ {\tiny 75 votes,  66.13\%}} 
	\end{tabular} 
	\end{flushright}

	%to force start on odd page
	\newpage
	\thispagestyle{empty}
	\mbox{}	
	\section{Magnetostatics}
	\lettrine[lines=4]{\color{BrickRed}M}agnets are known since ancient times (without that it was known at that the origin of their properties) under the name "magnetite": black stones found near the city of Magnesia (Turkey). This is from this stone that also comes the current name of the "magnetic field". The Chinese were the first to use the properties of the individual magnets more than 1,000 years before for compasses. They consisted of a magnetite needle resting on straw floating on the water contained in a graduated container.
	
	Magnetostatics is the study of magnetic fields in systems where the currents are steady (not changing with time). It is the magnetic analogue of electrostatics, where the charges are stationary. The magnetization need not be static; the equations of magnetostatics can be used to predict fast magnetic switching events that occur on time scales of nanoseconds or less. It is even a good approximation when the currents are not static - as long as the currents do not alternate rapidly.
	
	As for the electric field, a good/better understanding of the origin of this field can be done through modern theories such as wave quantum physics and quantum field theory. The beginner reader will have therefore to be patient, as for the study of the electric field, until to get in this book the knowledge to study these modern theories.
	
	The quantitative study of the interactions between magnets and currents was made by the physicists Biot and Savart in 1820 only (nobody knew before that the compass was influenced by the current inside the core of the Earth). They measured the amplitude of the oscillations of a magnetic needle according to its distance to a straight current. They found that the force acting on a pole is directed perpendicular to the direction connecting the center to this conductor wire and it varies inversely with the distance. This is the first case that we will study.
	
	The idea of the experiment was the following:
	\begin{figure}[H]
		\centering
		\includegraphics[scale=0.9]{img/electromagnetism/biot_savart.jpg}
	\end{figure}
	\pagebreak
	Or in real life with a much simple configuration:
	\begin{figure}[H]
		\centering
		\includegraphics{img/electromagnetism/biot_savart_simple.jpg}
	\end{figure}
	Given a movement of electric charges (i.e.: a current $I$) generating in the space a vector field whose effects are measurable and whose properties differ from those of the electrostatic field. We deduce the existence of a new vector field we name (temporarily) "\NewTerm{magnetic field}\index{magnetic field}" and which we denote by $\vec{B}$.
	
	The physical units of the magnetic field will naturally been deduced from the time we will be able to connect this magnetic field to something known as a Force for example . This is what we will see further below during our study of the "Laplace Force".
	
	The simplest case of study consisting of an indefinite straight conductor wire (example that we can also assimilate to a simple move of charges without having necessarily a wire to carry them) carrying a current $I$ (see the section Electrokinetics for the concept of "current") shows that the magnetic field lines are circles having for axis the conductor wire itself:
	\begin{figure}[H]
		\centering
		\includegraphics{img/electromagnetism/ampere_wire.jpg}
		\caption{Magnetic field around a straight infinite wire}
	\end{figure}
	The direction of $\vec{B}$ is usually defined by the intermittance of "\NewTerm{Ampere observer}\index{Ampere observer}", that is to say, an observer that would be positioned along the wire, so that the current goes to its feet towards its head and who would look at the point $M$ where we evaluate the magnetic field. The field $\vec{B}$ is directed from the right to the left of this observer. This is represented many times by the following had figure:
	\begin{figure}[H]
		\centering
		\includegraphics{img/electromagnetism/ampere_observer.jpg}
		\caption{Ampere observer hand rule}
	\end{figure}
	It is said that is was experimentally established by Biot and Savart in 1820 that the norm of the magnetic field $\vec{B}$ at the distance $r$ of the wire is proportional to the current $I$ running through it and inversely proportional to $r$ such that:
	
	This relation is traditionally considered as the pillar of the study of the magnetic field.
	
	The proportionality coefficient $k$ depends as always to selected unit system (as for other study fields). For the set of all its consequences, it is advantageous to write the above expression in a form that makes appear the perimeter of the circle of radius $r$. So we put:
	
	\begin{tcolorbox}[title=Remark,colframe=black,arc=10pt]
	Don't forget that as $2\pi$ appears frequently in many theorems of physics and mathematics (more than $\pi$ alone), this value is frequently denoted by the letter $\tau$.
	\end{tcolorbox}
	Thus we obtain the value of the magnetic field at a distance $r$ from a straight conductor wire through which a constant current $I$ pass through:
	
	where $\mu_0$ is a new constant which we name "\NewTerm{magnetic permeability of vacuum}\index{magnetic permeability of vacuum}" (again as for for the electrical permittivity, there is a "\NewTerm{relative magnetic permeability}\index{relative magnetic permeability}") and whose value is given as usual in this book with the other physical constants in the section Principles of Mechanics chapter.
	
	The units of this constant, even if given in the section Principias of the Mechanics chapter, will be deducted automatically as soon as we have managed to link the magnetic field with the concept already known of "Force" (see below). This is what we will see when we will study the "\NewTerm{Laplace Force}".
	
	\subsection{Ampere's theorem }
	In classical electromagnetism, the "\NewTerm{Ampère's theorem}\index{Ampère's theorem}" or "\NewTerm{Ampère's circuital law}\index{Ampère's circuital law}" or simply "\NewTerm{Ampère's law}\index{Ampère's law}" relates the integrated magnetic field around a closed loop to the electric current passing through the loop.
	
	It is interesting to calculate the "\NewTerm{circulation of the magnetic field}\index{circulation of the magnetic field}" $\vec{B}$ in vacuum (or not) along a contour $\Lambda$ that turns once in the positive direction (anti-clockwise) around the wire oriented in the direction of the current (Ampere observer):
	
	
	\begin{tcolorbox}[title=Remark,colframe=black,arc=10pt]
	The field is collinear along the path as we have seen previously hence the fact that the dot product can be written as a simple product of norms.
	\end{tcolorbox}
	We obtain then by definition the "\NewTerm{Ampère's Law}\index{Ampère's Law}" (or erroneously named "\NewTerm{Ampère's Theorem}\index{Ampère's Theorem}" because this result is not provable ... at least as far as we know...):
	
	where the current $I$ in a high symmetry system can be assimilated to a simple algebraic sum of currents snared by the path such that: we have
	
	Warning!!! This is not because the circulation of the magnetic field is zero in a region of space that the magnetic field is zero at any point!
	\begin{tcolorbox}[title=Remarks,colframe=black,arc=10pt]
	\textbf{R1.} The Ampere's law will give us the possibility to determine the fourth Maxwell equation that we will prove in the section of Electrodynamics.\\
	
	\textbf{R2.} The prior-previous relation is sometimes wrongly named "Ampere theorem" when in fact this result is not provable as already mentioned. Some physicists, however, use the fourth Maxwell equation to prove the prior-previous relation but then this is the snake biting its tail...
	\end{tcolorbox}
	The expression that we obtained can be simplified even more if we introduce a new physical being named "\NewTerm{magnetic field intensity}\index{magnetic field intensity}" or more commonly "\NewTerm{magnetic excitation}\index{magnetic excitation}" and which is denoted by the letter $\vec{H}$ (which is intrinsically independent of the propagation medium!).
	
	If we consider that we are always in the vacuum where there is no magnetic dipole then we define it in the vacuum by:
	
	Therefore, we are often taken to speak about "\NewTerm{magnetic induction}\index{magnetic induction}" for $\vec{B}$ and "\NewTerm{magnetic field}\index{magnetic field}" for $\vec{H}$. But both are happily confused following the authors/professors/teachers and especially depending on the contexts (as will be the case also in this book). When we deal with magnets that have intrinsic magnetization by the properties of the material that are made from, we denote distinctly the external magnetic field by:
	
	which is a more general form of the previous relation. It is therefore usual to define the "\NewTerm{magnetic susceptibility}\index{magnetic susceptibility}" as being the dimensionless ratio:
	
	Thus, the magnetic susceptibility indicates the amplitude with which a material responds magnetically to the presence of a magnetic excitation. We then get by this definition, the relation between relative magnetic permeability and magnetic susceptibility:
	
	Therefore:
	
	where $\mu$ is named the "\NewTerm{absolute magnetic permeability}\index{absolute magnetic permeability}".
	
	It is customary to name materials that have a positive magnetic susceptibility "\NewTerm{paramagnetic materials}\index{paramagnetic materials}" (contributing to increase the magnetic field) and those with a negative magnetic susceptibility "\NewTerm{diamagnetic materials}\index{diamagnetic materials}" (tend to oppose to the magnetic field ). We will see later the Langevin theoretical models to explain quantitatively with a relatively good approximation the two phenomena (in both cases the magnetic susceptibility has a value which is very low).
	
	So finally we can also write the Ampere's Law as following:
	
	The interest of the Ampere's law and of the concept of circulation of the magnetic field may appears like this more evident.
	
	This latter relation obviously is obviously very useful in theoretical physics because it will allow us to determine other important powerful results. Otherwise, in practice, the physicist or electrician/electrical engineer will often be faced with having to use electromagnets for small and medium experiences which he may wish to recalibrate the nominal values, or even solenoids.
	
	
	\subsubsection{Infinitely long solenoid }
	Also a particularly important application in electronics and electrical engineering is the calculation of the induction field in a coil of wire through which pass a current that we will consider as constant in a first time. This is nothing more than an induction coil more technically named an "\NewTerm{inductance}\index{inductance}". Let's see what it is:
	
	A solenoid is a coil of wire formed by a conductor wire wound helically through which pass a current of intensity $I$. In what follows, we assume that the induction field $\vec{B}$ of a solenoid is zero between the coils and parallel to the axis of the solenoid.
	
	Let us consider the following figure and let us look in approximation only to the inside part of the solenoid admitting that the external field is zero by the infinite length of it and that the joins of the coils are perfect (not letting dissipate any magnetic field)...:
	
	\begin{figure}[H]
		\centering
		\includegraphics{img/electromagnetism/infinite_solenoid.jpg}
		\caption{Infinite solenoid (coil of wire)}
	\end{figure}
	Let us apply the Ampere law to a rectangular path $abcd$. Therefore:
	
	The first integral of the right member gives $B\cdot h$ where $B$ is the norm of $\vec{B}$ inside the solenoid and $h$ the length of the segment $\overline{ab}$. We can notice that the segment $\overline{ab}$, even if it is parallel to the axis of the solenoid, don't need not coincide with it.
	
	The second and fourth integral is equal zero because for these two segments $\vec{B}$ and $\mathrm{d}\vec{l}$ are everywhere perpendicular: as $\vec{B}\circ\mathrm{d}\vec{l}$ is zero everywhere, the two integrals are equal to zero. The third integral is also equal to zero since the segment is calculated outside the solenoid where we assumed that the magnetic field is equal to zero as we consider the solenoid to be perfect.	
	
	Thus, the integral $\oint\vec{B}\circ\mathrm{d}\vec{l}$ for the entire rectangular path is equal to $B\cdot h$ such that:
	
	but the current $I$ is the sum of the currents $I_0$ passing through the $N$ turns contained in the path of integration. But in electronics, we used to work with the value $n$ (we choose the lowercase letter by analogy to Thermodynamics where lowercase letters represent densities) which is the number of turns per unit length:
	
	Therefore we have:
	
	That is:
	
	\begin{figure}[H]
		\centering
		\includegraphics{img/electromagnetism/solenoide_analogy_dipole_magnet.jpg}
		\caption{Linear solenoid analogy with (dipole) magnet (source: }
	\end{figure}
	Even if this relation has been established for an ideal infinite solenoid, it gives a pretty good magnitude (but not exact!) of magnetic induction field for interior points near the center of a real solenoid! This relation also shows that the magnetic field is independent in approximation of the solenoid diameter and that this latter is uniform across the section thereof!!! In laboratories, a solenoid is a convenient device for producing a uniform field induction in the same way that the plane capacitor is used to produce a uniform electric field.
	
	\pagebreak
	\subsubsection{Toroidal coils}
	The toroidal coil is another important example of the application of Ampere's Law. Indeed, we especially find this configuration in low power electronics (e.g. computers) where the inductors are mostly toroidal or in the energy production with the famous Tokomak that are (very ...) schematically reduced to toroidal coils.
	\begin{figure}[H]
		\centering
		\includegraphics[scale=0.5]{img/electromagnetism/toroidal_coil.jpg}
		\caption{Toroidal coil}
	\end{figure}
	For symmetry reasons, it is clear that the magnetic induction lines form concentric circles inside the coil. Let us apply Ampere's law to the radius $r$ of integration of the circular path:
	
	That is to day:
	
	If follows then:
	
	Thus, unlike $B$ inside a solenoid, $B$ is not constant within the toroidal coil.
	
	\pagebreak
	\subsubsection{Electromagnet}
	Let us determine for example (important and interesting one!) the magnetic field in the air gap of length $L_a$ and section $S_a$ of an electromagnet of length $L_{\text{Fe}}$ and section $S_{\text{Fe}}$ as shown below:
	\begin{figure}[H]
		\centering
		\includegraphics{img/electromagnetism/electromagnet_rectangular.jpg}
		\caption{Schematic rectangulare Electromagnet}
	\end{figure}
	The Ampere's law gives us in vacuum:
	
	in the case of the electromagnet, we can write that the circulation of the field is the sum of the circulation of the airgap field and the magnet field itself:
	
	where $N$ corresponds to the number of current loops surrounding the magnet and which allows the production of the magnetic field.
	
	Let us recall that we have by definition:
	
	Therefore:
	
	If the airgrap is not to big $L_\text{Fe}>>L_\text{airgap}$ we can write:
	
	Therefore:
	
	Finally:
	
	The relation is the same for an electromagnet having two coils! The reader will have perhaps notice on the way that this relation can also be used experimentally if we seek to determine the value of the relative magnetic permeability of the iron when all other parameters are known.
	
	\paragraph{Strength of a magnet or electromagnet}\mbox{}\\\\
	If we have know the norm of the magnetic field $B$ produced by a magnet at its surface, we can calculate with a certain approximation the force required to detach it from an iron surface.
	
	For this, we will denote by $F$ the necessary force to take off the magnet from a distance $d$ of an iron surface. We will assume the distance $d$ to be sufficiently small distance so that we can accept that the in the entire volume between the magnet and the iron, the magnetic field is constant.
	
	Thus, the work done by the force $F$ is (\SeeChapter{see section Classical Mechanics}):
	
	This work has been turned into magnetic field energy in the volume created between the magnet and iron. The volumetric energy density due to the magnetic field of air being (\SeeChapter{see section Electrodynamics}):
	
	The volume of the space created between the magnet and the iron being equal to $V=Sd$ where $S$ is the surface of the magnet that was bonded to iron. We then have the following dimensional equivalence:
	
	Hence we deduce the contact force for small values of $d$:
	
	where $B$ is the limit value of the magnetic field which causes our material to stick to the magnet (so that by raising the magnet, the associated material follows).
	
	If we look at a lifting electromagnet of radius $0.75$ [m] capable of lifting $200$ [kg]:
	\begin{figure}[H]
		\centering
		\includegraphics{img/electromagnetism/lifting_electromagnet.jpg}
		\caption{Lifting electromagnet}
	\end{figure}
	Then we have:
	
	We can also use roughly the same calculation to determine the magnetic field of the electromagnet of the famous worldwide following playful toy known by people passionate in physics:
	\begin{figure}[H]
		\centering
		\includegraphics{img/electromagnetism/playful_magnetic_toy.jpg}
		\caption{Playful lifting electromagnet}
	\end{figure}
	
	\pagebreak
	\subsection{Maxwell-Ampere Relation}
	Given $\vec{J}$ the current density at any point of the space in the case of a three-dimensional distribution and let $S$ be a closed surface which is based on any contour $\Gamma$. The current $I$ that pass through $\Gamma$ is of course given by:
	
	According to Ampere's law, the flow of the magnetic field along $\Gamma$ is equal to this integral. It may therefore take here, given the selected contour $\Gamma$, an infinite continuous variable values. On the other hand, the Stokes' theorem (\SeeChapter{see section Vector Calculus}) give us that:
	
	Therefore:
	
	and finally we conclude that:
	
	We can make a bold comparison of this result with the relation below (proven in the section Electrodynamics), by extension of the static and dynamic electric charge:
	
	which is nothing else than the first Maxwell equation (\SeeChapter{see section Electrodynamics}). Therefore, as we have seen in the secton Electrostatics, we have:
	
	By analogy, the idea is to put (this assumption is verified a little further below in the text by the remarkable results obtained):
	
	relation that we can write in a more elegant way by assuming the current not dependent on the observer's position in space and collinear with perpendicular vector perpendicular of the crossing surface:
	
	where $\Gamma$ represents the perimeter of the wire in which the current $I$ flows.
	
	\subsection{Biot-Savart law}
	Form the latest development, we get:
	
	Remember that at the last stage of our previous developments (we have specified it implicitly) the integration path is perpendicular to the electric current! But the magnetic field can not be zero at any point of the current line. Therefore, we are led to write what is hidden:
	
	The above relation allows us thus, by extension, to write in a more general form:
	 
	which is nothing else than the "\NewTerm{Biot-Savart law}\index{Biot-Savart law}" often presented first in high-school classes as starting point from the study of magnetism (originally it was experimentally determined in 1820 by Jean-Baptiste Biot and Félix Savart with the mathematical help of Simon de Laplace).
	
	This latter relation may as well be written (very important form):
	
	Therefore:
	
	Here we fall back on the non-relativistic approximation of the magnetic field as we have determined it in our study of Special Relativity (\SeeChapter{see section Special Relativity}), where we have prove that:
	
	Another important form of the expression for the magnetic field is:
	
	As the current density $\vec{J}$ is collinear $\mathrm{d}\vec{l}$, we can write:
	
	Therefore:
	
	An important remark is necessary to our level of our discussion in the context of pre-university academic studies, mathematical formulations of magnetic and electric fields are considered as unprovable laws from which we later deduce the Maxwell's equations (furthermore the developments are not of the most aesthetic and rigorous type). The experimental aspect of such important relations can give a negative image of theoretical physics to students. It is therefore clear that at university, we have an approach somewhat less pragmatic.
	
	Indeed, we postulate the Schrödinger equation (\SeeChapter{see section Wave Quantum Physics}) which we use to prove the non-relativistic formulation of Coulomb's law with the Yukawa theory (\SeeChapter{see sction of Quantum Field Theory}). Then, during the study of relativity (\SeeChapter{see section Special Relativity}), we determine the relativistic form of Coulomb's law. Then we admit the existence of the magnetic field whose expression is experimentally given by the Lorentz force (see further below in this section) and by the properties of Lorentz transformations and knowledge of the relativistic expression of the Coulomb's law, we determine the expression of relativistic magnetic field. Then, by non-relativistic approximation, we fall back on the Biot-Savart law. This approach is much more welcomed by students but not necessarily accessible to all levels.
	
	Now let us come back on the Biot-Savart law. An important example in astrophysics of the Biot-Savart law in the context of plasma jets accretion disks are the unique circular current loops (we also have to add to this the Lorentz force in the relativistic framework for understanding the dynamics of these jets).
	
	\subsubsection{Magnetic field for a current loop}
	The figure below is a good example of a current loop seen from profile:
	\begin{figure}[H]
		\centering
		\includegraphics{img/electromagnetism/current_loop.jpg}
		\caption{Magnetic field for a current loop}
	\end{figure}
	So we have a circular loop of radius $R$ traversed by an electric current of intensity $I$. The objective is to calculate $\vec{B}$ at a point $P$ on the axis of the loop.
	
	The vector $\mathrm{d}\vec{l}$ corresponding to an elementary electric current at the top of the loop exits perpendicularly from the plane of the page (screen). The angle $\theta$ between this vector and $\vec{r}$ is therefore $\pi/2$. The plane formed by $\mathrm{d}\vec{l}$ and $\vec{r}$ is normal to the figure. The vector $\mathrm{d}\vec{B}$ produced by this elementary electric current is normal to this plane by the form of the Biot-Savart law. It is therefore in the plane of the figure and at a right angle with the vector $\vec{r}$ as shown on the figure.
	
	Let us decompose $\mathrm{d}\vec{B}$ into two parts: the first $\mathrm{d}\vec{B}_{||}$ is along the axis of the loop and the second $\mathrm{d}\vec{B}_{\perp}$ is perpendicular to this axis. Only the component $\mathrm{d}\vec{B}_{||}$ contributes to the total magnetic induction at the point $P$. This is because the components $\mathrm{d}\vec{B}_{||}$ of all elementary currents are positioned on the axis and they are added directly. As for the components $\mathrm{d}\vec{B}_{\perp}$, they are directed in different directions perpendicular to that axis so that, by symmetry, their contribution is zero on this axis (be really careful with this special case!).
	
	 We get:
	
	It is a scalar integral performed on all elementary currents. We get by the Biot-Savart law:
	
	In addition, we have according to the figure above:
	
	Combining these relations, we get:
	
	The figure reveals that $r$ and $\alpha$ are not independent variables. We can express them in function of the new variable $x$, the distance between the center of the loop and the point $P$. The relations between these variables are:
	
	Substituting these values in the expression of $\mathrm{d}B_{||}$, we obtain:
	
	We notice that for all the elementary currents, $I$, $R$, $x$ have respectively the same values. Integrating this differential gives:
	
	An important point of this relation is therefore $x=0$ we get:
	
	Another important application of the Biot-Savart law consists to take the previous example, but for any continuous falt shape and considered as punctual and we would like to know the value of the field elsewhere than on the axis of symmetry . The results will be very useful when we will study the Corpuscular Quantum Physics and therefore the magnetic properties of metals.
	
	\subsubsection{Magnetic field for an infinite wire}
	Let us also prove (it is an interesting example!) that from the Biot-Savart law:
	
	we can get for an infinite straight wire the relation:
	
	that we get already earlier with the Ampere's theorem (which shows the equivalence between the two ways of calculating!).
	
	Let us choose for the infinite straight wire below $x$ as variable:
	\begin{figure}[H]
		\centering
		\includegraphics{img/electromagnetism/infinite_straight_wire_2.jpg}
		\caption{Infinite straight wire}
	\end{figure}
	We then have from the figure above:
	
	hence:
	
	By integrating:
	
	For the remaining part of the development, the trick is to use the following configuration:
	\begin{figure}[H]
		\centering
		\includegraphics{img/electromagnetism/infinite_straight_wire_3.jpg}
		\caption{Piece of the infinite wire}
	\end{figure}
	Therefore:
	
	Which gives us:
	
	After simplification:
	
	and so when the wire length tends is very big relatively to $l$ we get:
	
	
	\pagebreak
	\subsection{Magnetic dipole}
	The magnetic dipole has as its electrostatic counterpart enormous importance in the study of magnetic properties of materials for which it allows to develop good theoretical models.
	
	Before reading what follows, we strongly advise the reader (this even more than an advice) to read absolutely everything about the development of the rigid electric dipole moment in the section of Electrostatic. Indeed, most calculations that follow have the same reasoning, mathematical developments and approximations to some tiny nuances. We do not therefore wish to make the same intermediate calculations already present when calculating the electric dipole moment (however, if you really have trouble on reading what follow, we are ready to complete the missing as always in this book).
	
	A magnetic dipole is the limit of either a closed loop of electric current or a pair of poles as the dimensions of the source are reduced to zero while keeping the magnetic moment constant. It is a magnetic analogue of the electric dipole, but the analogy is not complete. In particular, a magnetic monopole, the magnetic analogue of an electric charge, has never been observed. Moreover, one form of magnetic dipole moment is associated with a fundamental quantum property: the spin of elementary particles.
	\begin{figure}[H]
	\centering
	\begin{subfigure}{.5\textwidth}
	  \centering
	  \includegraphics[scale=0.5]{img/electromagnetism/magnetic_dipole_as_a_current_loop.jpg}
	  \caption{Magnetic dipole as a current loop (source: Wikipedia)}
	\end{subfigure}
	\begin{subfigure}{.5\textwidth}
	  \centering
	  \includegraphics[scale=0.7]{img/electromagnetism/magnetic_dipole_as_a_magnet.jpg}
	  \caption{Magnetic dipole as a magnet}
	\end{subfigure}
	\caption{Microscopic dipoles point of views}
	\end{figure}
	Some planets as the Earth have a magnetic field similar to magnetic dipole (but quite not homogeneous and not static):
	\begin{figure}[H]
		\centering
		\includegraphics{img/electromagnetism/magnetic_dipole_as_earth.jpg}
		\caption{Earth view as a magnetic dipole}
	\end{figure}
	
	The magnetic dipole, however, has a significant difference in relation to the practical case we impose us as part of its study ... there are no two electric charges! Indeed, electric charges at rest emit in a first approximation (this is experimental and theoretical fact...) an intrinsic magnetic field far too low to be considered interesting in the context of the study of the magnetic properties of materials. However, we should clarify something interesting (very cool), the elementary Coulomb electric charges are modeled sometimes (wrongly!) by physicists as a sphere rotating on itself (the "spin") and are represented as a superposition of small circular loops (oh... a loop we already know something like this) that are infinitely small so an observer in the rest frame (center of the electric charge) can interpret the overall Coulomb electri charge as a current moving in the different current loops, thereby inducing an intrinsic magnetic field (not pretty!?).
	
	In short, let us consider a flat loop (hep... another loop again...), of any shape, center on O, through which travel a permanent and constant current $I$ and where one of the points of the loop is denoted by $P$. We want to calculate the magnetic field generated by this coil at any point $M$ in space, located at a distance $r$ of the loop (specifically, at large distances compared to the size of the loop).
	\begin{tcolorbox}[title=Remark,colframe=black,arc=10pt]
	Personally there are some steps in the calculation that I find... well... very far unconvincing... but... there are so many approximations anyway that these problematic steps can bee see as details only... hummm...
	\end{tcolorbox}
	We put:
	
	\begin{figure}[H]
		\centering
		\includegraphics{img/electromagnetism/dipole_configuration_study.jpg}
		\caption{Magnetic dipole study configuration}
	\end{figure}
	So we will use the Biot-Savart law:
	
	under the assumption that the point $M$ is located at great distance from the loop. What gives us the right to write:
	
	But as $\mathrm{d}\vec{l}=\mathrm{d}\vec{\rho}$ therefore:
	
	Let us evaluate the term $\vec{r}'/r^3$ for points $M$ located far away from the current loop (at the denominator we used the cosine theorem as during our study of therigid electric dipole moment in the section Electrostatics):
	
	where we did like the rigid electric dipole moment a Taylor limited development to order $1$. 
	\begin{tcolorbox}[title=Remark,colframe=black,arc=10pt]
	The last approximation is very rough in the sense that it is a clever choice of the terms to neglect to achieve a visually aesthetic result and to define the magnetic dipole moment (see further below) ...
	\end{tcolorbox}
	Putting this expression into the Biot-Savart law, we get:
	
	Let us evaluate each term involved in the parenthesis separately:
	\begin{enumerate}
		\item $\oint\mathrm{d}\vec{\rho}\times\vec{u}=\left(\oint\mathrm{d}\vec{\rho}\right)\times\vec{u}=\left(\vec{\rho}(P_0)-\vec{\rho}(P_0)\right)\times\vec{u}=\vec{0}\times\vec{u}=\vec{0}$

	since the vector $\vec{u}$ is independent of the point $P$ on the current loop and we do a curvilinear integration along the entire turn, returning to the starting point.

		\item $-\dfrac{1}{r}\oint \mathrm{d}\vec{\rho}\times\vec{\rho}$

		By the properties of the vector product:
		
		But since $\mathrm{d}\vec{\rho}$ and $\vec{\rho}$ are perpendicular and in a same plane, we have $\vec{\rho}\times\mathrm{d}\vec{\rho}$ which is the infinitesimal surface $\mathrm{d}S'$ of a rectangle and it represents nothing as the 
abscissa is curved relatively $O$. Indeed:
		\begin{figure}[H]
			\centering
			\includegraphics{img/electromagnetism/perpendicularite_vectors_magnetic_dipole.jpg}
		\end{figure}
		So we can write:
		
		where $\vec{n}$ is the vector normal to the plane of the loop (base vector of the $z$-axis). This result is generally valid regardless of the surface.

		Hence:
		
		
		\item $\oint\mathrm{d}\vec{\rho}\times\vec{u}(\vec{u}\circ\vec{\rho})=-\vec{u}\times\oint\mathrm{d}\vec{\rho}(\vec{u}\circ\vec{\rho})$ by the properties of the cross product (\SeeChapter{see section Vector Calculus}).
	\end{enumerate}
	We will use these relations to calculate the unknown integral of the start of our study. If we decompose the vector $\vec{\rho}$ and $\vec{u}$ in the base $(\vec{e}_1,\vec{e}_2)$ generating the plane of the loop, we get:
	
	as $\mathrm{d}\rho_3=0$ and $\rho_3=0$.

	We also have:
	
	Hence:
	
	Let us recall that:
	
	In the form of components (only the third term is non-zero since $\mathrm{d}\rho_3=0$ and $\rho_3=0$), we have:
	
	hence:
	
	Which brings us to write:
	
	Therefore:
	
	We notice that the latter relation is equal to:
	
	Then finally:
	
	By bringing together these results we obtain for the magnetic field:
	
	We see therefore an important quantity that appears because it fully describs the spire seen from a great distance, namely the "\NewTerm{magnetic local dipole moment}\index{magnetic dipole moment}" or simply "\NewTerm{magnetic dipole moment}":
	
	also often denoted by $\vec{\mathcal{M}}$ by some practitioners. The magnetic moment has then for units $[\text{A}\cdot\text{m}^2]=[\text{J}\cdot\text{T}^{-1}]=[\text{N}\cdot\text{m}\cdot \text{T}^{-1}]$. 
	
	We then have the prior-previous relation that can be written:
	
	By making use of the property of the double vector product (\SeeChapter{see section Vector Calculus}):
	
	We then get another form of expression of the approximate magnetic field created by a dipole:
	
	It is in the latter form that we found the most often the expression of the magnetic moment in the literature.
	
	This relation has to be compared (for fun) with the expression for the electric field for a rigid electric dipole:
	
	and thus we see that there is perfect correspondence.
	
	We still arrived to put this in a fairly identical and aesthetics and form after some approximations...

	We have also:
	
	Therefore:
	
	The origin of the magnetic field of any material must be microscopic. Using the Bohr model of the atom (\SeeChapter{see section Corpusculare Quantum Physics}) we can convince ourselves that atoms (at least some) have an intrinsic magnetic dipole moment. Indeed, the Bohr model of the Hydrogen atom consists of an electron of electri charge $q=-e$ in circular motion around a central nucleus (a proton with an electric charge $q=+e$) with a period $T=2\pi\omega$.
	
	If we look over long time scales compared to $T$, it is as if there was an electric current:
	
	So we have indeed a sort of circular loop, of mean radius equal to the average distance to the proton, ie the Bohr radius $r_0$ (\SeeChapter{see section Corpuscular Quantum Physics}). The Hydrogen atom therefore have an intrinsic magnetic moment of:
	
	where $\vec{b}$ is the momentum of the electron and $q / 2m$ the "\NewTerm{gyromagnetic factor}\index{gyromagnetic factor}" (this result is very important for the Langevin model of diamagnetism!). This reasoning can be generalized to other atoms. Indeed, a set of electric charges in rotation about an axis will produce a magnetic moment proportional to the total angular momentum. This happens even if the total electric charge is zero (material or neutral atom): what matters is the (scalar) existence of a current.
	
	So in fact, we can qualitatively explain the magnetic properties of materials based on the orientation of the magnetic moments of the atoms that compose them:
	\begin{itemize}
		\item "\NewTerm{Nonmagnetic materials}\index{Nonmagnetic materials}": These are materials where the magnetic moments are randomly distributed, there is no intrinsic magnetic field.

		\item "\NewTerm{Diamagnetic materials}\index{Diamagnetic materials}": these are the materials that subjected to a magnetic field, have their magnetic moments that opposed to it and are (marginally)  repelled by magnets. They therefore induce a magnetic moment opposite to the magnetic field direction.
		
		\item "\NewTerm{Paramagnetic materials}\index{Paramagnetic materials}": it is the materials for which moments can be oriented in the direction of an external magnetic field and can therefore be magnetized so (attracted) momentarily. They therefore induce a magnetic moment in the direction of the magnetic field.

		\item "\NewTerm{Ferromagnetic materials}\index{Ferromagnetic materials}": are materials whose moments are already oriented in a particular direction, permanently (natural magnets).
	\end{itemize}
	\begin{tcolorbox}[title=Remark,colframe=black,arc=10pt]
	Earth is known to have a dipolar magnetic field where the magnetic north pole corresponds to the geographic South Pole (at a given angle value). At the macroscopic level, the explanation for the existence of the magnetic field observed on the stars is still far from satisfactory. The theory of "dynamo effect" trying to explain the fields observed by the presence of currents, mainly azimuthal, in the core of stars. Several known facts remain partially unsolved:
	\begin{itemize}
		\item The magnetic cycles: the Sun has a large-scale magnetic field similar to that of Earth, approximately dipolar. However, there is a polarity inversion every $11$ years. For the Earth, geophysicsts were able to show that there had been a reversal about $700,000$ years before.
		
		\item Non-alignment with the angular momentum of the celestial body: it is of the order of ten degrees for the Earth, it is perpendicular for Neptune!
	\end{itemize}
	\end{tcolorbox}
	To see a pretty practical case of the magnetic dipole, remember that in the section Electrostatics, we had shown that in spherical coordinates we had for the electric dipole:
	
	And as we have shown there is perfect correspondence between the magnetic and electric dipole moment, then we can immediately write:
	
	We can then deduce the norm of the magnetic field vector:
	
	Therefore:
	
	
	\begin{tcolorbox}[colframe=black,colback=white,sharp corners]
	\textbf{{\Large \ding{45}}Example:}\\\\
	Thus, we can have fun to calculate the magnetic moment of the Earth at the geomagnetic equator. What gives, knowing that the Earth's geomagnetic field  at equator is in average equal to $32$ [$\mu$T]:
	
	\end{tcolorbox}
	
		\subsubsection{Magnetic torque}
	We have just introduced earlier the dipolar magnetic moment $\mu_l$. For a magnet, the magnetic moment (denoted $\mu$) is indirectly defined a quantity that determines the torque it will experience in an external magnetic field such that by definition:
	
	Thus, if a magnetic dipole (or a bar magnet) is placed in a uniform magnetic field in oblique orientation, it experiences no force but experiences a torque. This torque tends to align the dipole moment along the direction of magnetic field!
	
	We also have that a magnetic dipole moment may be defined as the torque acting on a magnetic dipole placed perpendicular to a uniform magnetic field of unit strength as in this case we have $\vec{\tau}=\vec{\mu}$.

	In the magnetic pole representation of a material, a magnetic material can be identified as two magnetic poles some distance apart. The magnetic moment of a material bar is then given by the product of the "\NewTerm{magnetic pole strength}\index{magnetic pole strength}" denoted $q_m$ and the distance between the poles $L$. Then for a magnetic bar where $\vec{n}$ is the unitary vector colinear to the magnet bar length, we have experimentally:
	
	That is in scalar form:
	
	This is the famous relation of the torque for a magnetic bar in a uniform constant magnetic field have two poles of magnetic pole strength $q_m$ and $-q_m$:
	\begin{figure}[H]
		\centering
		\includegraphics[scale=1]{img/electromagnetism/torque_bar_magnet.jpg}
		\caption{Torque on a bar magnet in a uniform magnetic field}
	\end{figure}
	The lines of action of both forces $\vec{F}_1$ and $\vec{F}_2$ are different, therefore, these two forces form a couple (or torque) which tends to rotate the magnet along the direction of magnetic field strength. 
	
	Thus, torque acting on the bar magnet is max, when it is placed perpendicular to the magnetic field.

	The work done in rotating the dipole through a small angle $\mathrm{d}\theta$:
	
	If the dipole is rotated from $\theta=0_1$ to $\theta=\theta_2$, total work done is given by:
	
	If $\theta_1=\pi/2$ and $\theta_2=\theta$, then:
	
	This work done is stored in the form of Potential Energy i.e:
	
	We will see again this relation further below.
	
	\pagebreak
	\subsection{Lorentz law (Lorentz force)}
	In electrostatics, we calculated the force exerted by one or a collection of electric charges at rest (...) on a stationary or moving electric charge. The electric force was the written as follows:
	
	In the most general case, where the acting electric charges are moving, the force they exert on a punctual electric charge $q$ placed in a given point in space is the sum of two terms: one that is independent of the relative speed $\vec{v}$ of this electric charge, the other that depends on it. Here's how is written this relation:
	
	that is the "\NewTerm{Lorentz law}\index{Lorentz law}" or "\NewTerm{Lorentz force}\index{Lorentz force}".
	
	To prove this relation, we will put two assumptions, but first it is important to inform the reader that this proof requires not necessarily obvious mathematical tools (most reader will have to read the section of Analytical Mechanics and Wave Quantum Physics to understand it well):
	\begin{enumerate}
		\item[H1.] Given a a non-relativistic point particle of mass $m$, of coordinates $x_i$ with $i=1,2,3$ and velocity $\dot{x}_i$ still with $i=1,2,3$. We assume that it is subjected to a force $\vec{F}$ that satisfies Newton's equations:
		
		with the following commutation relations (see sections of Analytical Mechanics and Wave Quantum Physics):
		
	 	The reader must realize that the last relation is an assumption (hypothesis) and it is not equivalent to the commutation rules that we saw in the section of Wave Quantum Physics between positions and linear momentum!

		\item[H2.] There exist fields $\vec{E}(\vec{x},t)$ and $\vec{B}(\vec{x},t)$ , not depending on the speed, such that:
		
		and which satisfy the Maxwell equations (\SeeChapter{see section Electrodynamics}):
		
	\end{enumerate}
	At a classical level, we express the commutation assumptions by using the Poisson correspondence commutation-brackets Fish-switches (\SeeChapter{see section Analytical Mechanics}), that is:
	
	with (recall):
	
	Now we define a vector potential $\vec{A}$ (\SeeChapter{see section Electrodynamics}) such that:
	
	then the hypothesis of commutation ($\{x_i,p_j\}=\delta_{ij}$)  can be written for $i\neq j$:
	
	So we can say that $\vec{A}$ depends only on $\vec{x}$ and $t$ and since it commutes identically to $p_j$.
	
	Moreover, we know that classical mechanics admits a Lagrangian formulation (equivalent to Newton's equations) for which the equations of mechanics become (\SeeChapter{see section of Analytical Mechanics}):
	
	where $L$ is the Lagrangian of the system. Therefore, with:
	
	we can integrate the relation:
	
	and we get:
	
	 Just as the electric scalar potential $\phi$ is a potential energy per unit charge, the magnetic vector potential $\vec{A}$ is potential energy per unit length element of current (but current has a direction, so the magnetic vector potential has to have three components, that's why it's a vector too...) that is [J$\cdot$m$^{-1}$A$^{-1}$]. It is an old tradition sometimes to use a derivate unit name the "\NewTerm{Weber}\index{Weber}" such that:
	 
	The "-" sign of the constant of integration of the vector potential is justified to be consistent with what we will in gauge theory (\SeeChapter{see section Electrodynamics}).

	The second Lagrange equation $p_j=m\dot{x}_j+qA_j(\vec{x},t)$ gives us then:
	
	By developping a little bit:
	
	and:
	
	For the set of all coordinates, this gives under condensed form and using the tools of vector analysis:
	
	So what must be absolutely noticed here is that it is the potential vector that appear in the Lagrangien and not the magnetic field!!! Therefore it's the potential vector that is at the origin of the variation of energy of a particle in an electromagnetic field and not the magnetic field!!!!!!!!
	
	Therefore:
	
	or written differently:
	
	We thus fall back well on the expression of the Lorentz force where $\vec{E}$ and $\vec{B}$ are given by:
	
	as we will see it in gauges theory. Certainly the proof is far from being obvious, but is possible.
	
	Many engineer have tested this result in school laboratory and know very well the following image (purple light is emitted along the electron path, due to the electrons colliding with gas molecules in the bulb and not because of bremsstrahlung that we will discuss later):
	\begin{figure}[H]
		\centering
		\includegraphics{img/electromagnetism/electron_lorentz_beam.jpg}
		\caption{Beam of electrons moving in a circle,\\ due to the presence of a magnetic field}
	\end{figure}
	You can also see this happening in photographs taken of charged particle tracks through a device known as the Wilson bubble chamber (we will come back on that one the section of Wave Quantum Physics as this detection procedure seems to be incompatible with Heisneberg's inequalities!): 
	
	\begin{figure}[H]
		\centering
		\includegraphics{img/electromagnetism/bubble_chamber.jpg}
		\caption{Bubble Chamber (source: CERN)}
	\end{figure}
	In principle, this is still how physicists frequently determine charge to mass ratios of particles produced in particle physics experiments.
	
	Spiralling tracks as visible in the figure above are a common feature of bubble chamber pictures, and they are caused by electrons $e^{-}$ (or positrons  $e^{+}$, which - key point - have the same mass).

	What a spiral tells us is that an electron loses energy at a considerable rate as it travels through a bubble chamber liquid. All other charged particles, unless they collide with a nucleus, very gradually slow down - get more curved - as they lose energy by ionisation (making bubbles in the bubble
chamber).

	Electrons are able to lose energy more quickly by another process known as "bremsstrahlung" (braking radiation). This process, which is a consequence of the fact that all accelerated charges radiate, is important for electrons because they have small masses. We will study mathematically this phenomenon in the section of Electrodynamics.
	
	\paragraph{Magnetic Vector Potential}\mbox{}\\\\
	To get some experience with the vector potential, let us look first at what it is for a uniform magnetic field $\vec{B}_0$. 

	Taking our $z$-axis in the direction of $\vec{B}_0$, we must have:
	
	By inspection, we see that one possible solution of these equations is:
	
	Or we could equally well take:
	
	Still another solution is a linear combination of the two:
	
	It is clear that for any particular field $\vec{B}$, the vector potential $\vec{A}$ is not unique; there are many possibilities.
	
	The third solution, more general, shows that $\vec{A}$ is perpendicular to the $z$-axis in this special examples. 
	
	So there are actually (infinitely) many vector fields $\vec{A}$ whose curl will equal an arbitrary magnetic flux density $\vec{B}$. In other words, given some vector field $\vec{B}$, the solution $\vec{A}$ to the differential equation $\vec{\nabla}\times\vec{A}=\vec{B}$ is not unique!

	To completely (i.e. uniquely) specify a vector field, we need to specify both its divergence and its curl and as we will see it in the section of Electrodynamics the divergence can satisfy the following gauge equation:
	
	Therefore finally we can write among other relations that we will also prove in the section of Electrodynamics that:
	
	Otherwise the vector potential for a uniform field can also be obtained in another way. The circulation of $A$ on any closed loop $\Gamma$ can be related to the surface integral of $\vec{\nabla}\times\vec{A}$ by Stokes' theorem (\SeeChapter{see section Vector Calculus}):
	
	But the integral on the right is equal to the flux of BB through the loop, so:
	
	So the circulation of $\vec{A}$ around any loop is equal to the flux of $\vec{B}$ through the loop. If we take a circular loop, of radius $r$ in a plane perpendicular to a uniform field $\vec{B}$, the flux is just:
	
	
	\paragraph{Work of Magnetic Field}\mbox{}\\\\
	Let us stop a moment on the expression of the Lorentz force. We see with this relation that a stationary (or not) electric charge in an electric field will experience a force that will give him the necessary impulsion to make vary its kinetic energy (zero or non-zero at the start). This fact is however not valid for the magnetic field. Indeed, when we put a stationary electric charge in a magnetic field, this latter will not be influenced by magnetic field strength and therefore will not see its kinetic energy vary. If the electric charge has a non-zero initial velocity, it follows that the magnetic field will change the components of the velocity vector but not its norm!!! So, we used to say that "the magnetic field does not work" (in the sense that the magnetic field is not make move an electric charge that is initially at rest or change the norm of its speed).
	Let's see how we can prove mathematically that the magnetic field does not work.
	\begin{dem}
	We know that for an electric charged particle immersed in a magnetic field, we have:
	
	hence:
	
	and let us express the temporal variation of the kinetic energy:
	
	and substituting in it the derivative of the speed by prior-prevous relation, we get:
	
	The kinetic energy of the particle therefore does indeed not change because the magnetic field is conservative in the special case of stationary magnetic (and hence electric) fields.
	
	This result also mean obvious that $\nabla{\vec{F}}=\vec{0}$. But if fact as we will prove it in the section of Electrodynamic this is a special case!! Indeed we will prove in the section of Electrokinetics the Faraday's induction law:
	
	Therefore it is immediate that $\nabla{\vec{F}}$ will not be null anymore (and hence $\vec{F}$) if $\vec{B}$ is not static.
	\begin{flushright}
		$\square$  Q.E.D.
	\end{flushright}
	\end{dem}
	Now, if we are interested only in the second term of this relation, we can get to prove the "Laplace's law"!

	So we have considerating only the second term of the Lorentz law:
	
	where $\rho$ is the volumic electric charge density. If $\vec{v}$ and $\mathrm{d}\vec{l}$ are supposed parallels we can write that:
	
	A current density allows us to calculate the intertial speed of the electric charge carriers in a conductor. The number of conduction electrons in a wire is equal to:
	
	where $n$ is the number of conduction electrons per unit volume and $Al$ the volume of the wire (and therefore $A$ is the area of the section of the conductor).
	
	A quantity of electric charges $q=\rho A l$ pass trough the surface section a wire in a time $t$ given by:
	
	The intensity of current $I$ being defined by:
	
	we get that:
	
	From:
	
	We can now draw that:
	
	Finally, we find that:
	
	which is the "\NewTerm{Laplace's law}\index{Laplace's law}" or "\NewTerm{Lorentz force}\index{Lorentz force}" and therefore is derived from the Lorentz law. We then deduce the in the units of the magnetic field:
	
	where T is a commonly tolerated unit in use named the "Tesla" in honor of Nikola Tesla one of the greatest spirit of all time with Albert Einstein. Therefore this unit implicitly contains the Coulomb unit which is then at the origin of the magnetic field! Knowing finally explicitly the units of the magnetic field, we can determine the units of the magnetic permeability constant by taking back the relation:
	
	He then comes for the units of the constant magnetic permeability:
	
	So we see that it has for units a force by square current intensity.

	Having done this, let us see some important cases of application of the Lorentz law:
	
	\pagebreak
	\subsubsection{Classical Hall effect}
	Previously, we have studied the action of a magnetic induction on a wire-like circuit having for purpose to find the expression of the magnetic forces applied to the same area of this circuit. Now let us turn our attention to the conductivity electrons themselves, placing us in the case of the figure below:
	\begin{figure}[H]
		\centering
		\includegraphics{img/electromagnetism/hall_effect.jpg}
		\caption{DC current through a metallic ribbon for the study of classical Hall effect}
	\end{figure}
	where a metallic ribbon is traversed by a continuous current $\vec{I}$. The vector of current density $\vec{J}$ is constant and parallel to the long sides $\overline{PQ}$ or $\overline{RS}$ of the ribbon.
	
	Let us imagine then that the ribbon is immersed in a uniform magnetic field perpendicular to the planes $PQ$ and $RS$ (following the $z$-axis). The mobile electric charges of volumic density $\rho$ contained in a volume element $\mathrm{d}V$ is therefore subject to the magnetic force:
	
	This force change the trajectories of mobile electrons and, during a transitional period, causing their accumulation on the front edge of the ribbon while an excess positive charge appears on the back edge.

	This phenomenon produces an additional electric field parallel to $\overline{RP}$ which exerts on the mobile charges of volume $\mathrm{d}V$ an electric force:
	
	The two forces are then opposing each other and the coulomb force tends to take back the electron paths in their initial position. A steady state is established gradually.
	\begin{tcolorbox}[title=Remark,colframe=black,arc=10pt]
	In fact, every time we talk about steady in physics, we lie a little bit. It is in fact just a stable equilibrium and in general, the system oscillates around its equilibrium position. After a while, a system like the conductor involved in our example shows negligible oscillations. Physics is sometimes also a matter of approximations...
	\end{tcolorbox}
	When this state is reached, the current density is again parallel to $\overline{PQ}$ and the electric and magnetic forces above are vectorially opposed. So we have:
	
	with:
	
	In some books, the croos product is made explicit in the form of its components such as:
	
	since the other components are zero (the current density is parallel to the ribbon and the magnetic field is perpendicular).

	Now, as we have proved it in the section Electrokinetics we have
	
	therefore:
	
	We then define the "\NewTerm{Hall coefficient}\index{Hall coefficient}" by:
	
	$R_H$ can be used at equilibrium for the measurement of $J_x$ if we suppose $v_x=c^{te}$ and that the potential $U$ (and therefore implicity $E_x$) and external field $B_z$ are known.  

	Notice that by construction we also have $R_H\propto \rho$ and this is used sometimes by engineer to determiner the density of electrical charge carriers in a new material sample.
	\begin{tcolorbox}[title=Remark,colframe=black,arc=10pt]
	We also tale of "\NewTerm{Hall's resistance}\index{Hall's resistance}". It is simply the ratio of the Hall voltage on the current through the sample. It must not be confuse with the Hall resistance $R_H$. 
	\end{tcolorbox}
	In a two-dimensional semiconductor , the Hall effect is also measurable. By cons, at sufficiently low temperature, we observe a series of trays for the Hall resistance as a function of the magnetic field. These plates appear to specific resistance values, and this, independently of the sample used. This is the subject of the "\NewTerm{quantum Hall effect}\index{quantum Hall effect}" that we will not discuss in this chapter.

	Under scalar form the relation of the "Hall effect", is written:
	
	We can also express it by expliciting the potential difference that corresponds by definition to the electric field.

	If $l$ is the width of the ribbon, we have:
	
	If $e$ is its thickness, the current $I$ that pass throug it is equal to:
	
	Given the relative positions of the various vectors, the relation expressing the Hall effect is equivalent to:
	
	More aesthetically and in traditional form, the voltage of the Hall effect is given by:
	
	with:
	
	which is the "\NewTerm{Hall constant}\index{Hall constant}". It is inversely proportional to the density of the free electric charge carriers and in the context of metals, it is negative.

	In other field of study such as the semiconductor, we write the Hall voltage in the following traditional form:
	
	where $q$ is the electron charge, and $n$ the traditional notation (sic!) of the carrier density within the study of semiconductors.

	We then in the latter field the Hall constant which is defined by:
	
	What has however made the reputation of the Hall effect, besides the fact that this result is enormously used to make magnetic fields sensors of all kinds (as Hall effect sensors operate without physical contact with the magnets), this is that for certain types of semiconductor this Hall constant is positive!!!! This would mean with the standard models we have available to us until now, there would be positive charges that would generate the current... and at the time of the establishment of this experience for semi-conductors this was inexplicable. At the time of  Edwin Herbert  Hall this experiment was used to check whether it was positive or negative charges that were moving and Hall concluded by testing it on conductive metals that only the negative electricity where moving in the conductors.
	\begin{tcolorbox}[title=Remark,colframe=black,arc=10pt]
	The most well know Hall probe by the population is the speedometer on bikes (odometer), which operates on the basis of attaching a small magnet on one of the spokes of a wheel and whose passage in front of the Hall sensor produces a signal processed by the electronic odometer.
	\end{tcolorbox}
	We will prove later that by using quantum theory in the context of semiconductors (\SeeChapter{see section Electrokinetics}) positive charges may yet appear under certain conditions and cause a current!
	
	\subsubsection{Larmor radius}
	A very interesting case of laboratory study is the movement of a charge in a uniform magnetic field. For this study, consider a particle of mass $m$ and electric charge $q$ placed in a uniform magnetic field with an initial velocity $\vec{v}_0$.

	We have following Lorentz' law:
	
	We will take advantage of the fact that the magnetic force is zero in the direction of the magnetic field.

	We'll break down the speed in two components, one parallel and one perpendicular to the magnetic field such as:
	
	The equation of motion is then:
	
	The trajectory remains rectilinear uniform in the direction of the magnetic field! In other words, if the speed of the charged particle was initially zero in the direction of the field then it will remain zero!
	
	Let us now consider a Cartesian coordinate system with the $Z$ axis is given by the direction of the magnetic field such that $\mathrm{d}v_z/\mathrm{d}t=0$. The equation of motion is therefore written only on two components, since:
	
	hence:
	
	A very simple special solution to these two differential equations in non relativistic framework is obviously:
	
	So where we chose an initial speed following the $x$-axis. By integrating, we get:
	
	where the integration constants were chosen as zero (arbitrary choice). The trajectory is thus a circle of radius:
	
	perpendicular to the magnetic field and named "\NewTerm{Larmor radius}\index{Larmor radius}", described with the pulsation:
	
	named "\NewTerm{gyro-synchrotron puslation}\index{gyro-synchrotron puslation}" This circle is traveled in the conventional positive direction for negative charges.
	\begin{tcolorbox}[title=Remark,colframe=black,arc=10pt]
	The movement is circular only if the particle, initially, therefore has no speed in the direction of the magnetic field. If it has one, it will keep it (the magnetic field has no action in that direction).
	\end{tcolorbox}
	The problem of such a configuration to build an accelerator is that if we increase the energy of the particle (by adding a synchronized electric field on the gyro-synchrotron pulsation and collinear to the movement), its speed increases and the radius Larmor also. But the "cyclotron" which is based on this system has a limited radius since it is difficult to maintain a constant magnetic field over a large area:
	\begin{figure}[H]
		\centering
		\includegraphics[scale=0.5]{img/electromagnetism/cyclotron.jpg}
		\caption{Cyclotron principle (source:  http://femto-physique.fr author: Jimmy Roussel)}
	\end{figure}

	Even more difficult, in the relativistic case, the pulse is then written with the Fitzgerald-Lorentz factor (\SeeChapter{see section Special Relativity}):
	
	We then see that we need to adjust the pulse of the electric field to the pulse of rotation when the sped increase: the accelerator is then now a "\NewTerm{synchrotron}\index{synchrotron}".
	
	To solve the problem of increasing radius, whereas we use a "Synchrotron" consisting of a single vacuum tube having straight sections containing the accelerating cavities and curved sections equipped with magnets creating at each instant the magnetic field adapted to the particle velocity. This technique, that it is easy to talk about but very difficult to put into practice, is the most used today. The CERN LHC is part of the family of synchrotrons.

	From this relation, it is easy to have the kinetic energy of the particle:
	
	It is based on this relationship that work the "\NewTerm{Dempster mass spectrometers}\index{Dempster mass spectrometers}". It is using this technique that researchers discovered in 1920 that atoms of the same chemical element does not necessarily have the same mass. The different varieties of atoms of the same chemical element, varieties that differ in mass, are the "isotopes" (\SeeChapter{see section Nuclear Physics}):
	\begin{figure}[H]
		\centering
		\includegraphics[scale=0.5]{img/electromagnetism/dempster_mass_spectrometers.jpg}
		\caption{Dempster mass spectrometers principle (source:  http://femto-physique.fr author: Jimmy Roussel)}
	\end{figure}
	The Larmor radius is the greatest distance that a particle can travel in the transverse direction before being deflected from its path. This corresponds to a kind of trapping distance. Unless it receive additional kinetic energy, a charged particle is thus trapped in a magnetic field.
	
	It is interesting to notice that more the kinetic energy of a particle is high (large mass or large transverse speed) and the the Larmor radius is big. Conversely, the higher the magnetic field and the smaller is the Larmor radius (hence the fact we need hug magnetic field at CERN to keep high speed particles into a circular trajectory).

	We come back on these concepts in the section Electrodynamics, where after studying the Maxwell equations, we will make some developments for the betatron type accelerator.
	\begin{tcolorbox}[title=Remark,colframe=black,arc=10pt]
	Confining the plasma in a tokamak is based on this property that charged particles have to describe a helical path around a magnetic field line. Hence the need to use a torus... in the simplest case.
	\end{tcolorbox}
	To close this subject, and following the comment of a reader, let us develop a little more in detail the solution of the both differential equations seen previously:
	
	simplifying the notation, they become:
	
	and we will take as initial conditions:
	
	The trick is to put:
	
	The system of differential equations the becomes (we drop the $Z$ component):
	
	and therefore:
	
	And by doing the same for $Y$, we get:
	
	The both differential being identical it is sufficient to solve one to have the other solutions. The roots of the characteristic equation are (\SeeChapter{see section Differential and Integral Calculus}):
	
	Since the discriminant is negative (the expression that is for recall under the root), then we have(\SeeChapter{see section Differential and Integral Calculus}) the homogeneous solution is then:
	
	Hence:
	
	And as there is no second member to our two differential equations, the homogeneous solution is then the general solution. Therefore:
	
	and same:
	
	Since we have put $X=\dot{x}$, $Y=\dot{y}$ and that we have $\ddot{x}=\omega\dot{y}$, then it comes:
	
	Thus explicitly:
	
	Since this equation must be valid for all time $t$, let us consider the case where $t = 0$, the equation is reduced to:
	
	By doing the same for $Y$ (always with $t = 0$), we get:
	
	Taking into account the initial conditions for speed, we get:
	
	So:
	
	\begin{tcolorbox}[title=Remark,colframe=black,arc=10pt]
	By putting $v_{0y}$ to be zero, we fall back on the simple special solution we proposed above with respect to speed.
	\end{tcolorbox}
	To continue, we take the primitive to find the position coordinates. It comes then:
	
	For the initial conditions:
	
	to be satisfied, we see pretty quickly that we must have:
	
	Thus finally:
	
	So if we take the $Y$ component of the velocity to be zero, we have:
	
	Which is not more the path of a circle because of the initial conditions chosen for this development. To fall back on the circular path it require the initial condition on $Y$ to be:
	
	Trapped particles in magnetic fields are found in the Van Allen radiation belts around Earth, which are part of Earth’s magnetic field. These belts were discovered by James Van Allen while trying to measure the flux of cosmic rays on Earth (high-energy particles that come from outside the solar system) to see whether this was similar to the flux measured on Earth. Van Allen found that due to the contribution of particles trapped in Earth’s magnetic field, the flux was much higher on Earth than in outer space. Aurorae, like the famous aurora borealis (northern lights) in the Northern Hemisphere, are beautiful displays of light emitted as ions recombine with electrons entering the atmosphere as they spiral along magnetic field lines. Aurorae have also been observed on other planets, such as Jupiter and Saturn.
	\begin{figure}[H]
		\centering
		\includegraphics[scale=1]{img/electromagnetism/van_allen_belt.jpg}
		\caption{Van Allen radiation belts around Earth (source: OpenStax)}
	\end{figure}
	
	\pagebreak
	\subsubsection{Energy of a magnetic dipole}
	Thanks to the fact that we now have the units of the magnetic field and those of the magnetic permeability constant and the Lorentz law, we will now be able to determine by dimensional analysis and intuition the total energy of a static (oriented!) magnetic dipole which will be very useful for us for the theory of paramagnetism developed further below.

	Let us consider for this a rigid magnet in the form of cylinder of length $L$ and of negligible radius that can be regarded as a north / south dipole (a "bar magnet") immersed in a constant and homogeneous magnetic field in the plane perpendicular to the axis of rotation of the dipole:
	\begin{figure}[H]
		\centering
		\includegraphics{img/electromagnetism/simple_static_magnetic_dipole.jpg}
		\caption[]{schematic diagram for the study of the static magnetic dipole}
	\end{figure}
	The experience shows that when the dipole is collinear with the magnetic field, it does not move. It follows that the force on one end depends in proportion to the sine of the angle with the magnetic field such that:
	
	At the dimension level we have so fare:
	
	We would have to get rid of Amps [A] by already making intervene at least obviously what characterizes a magnetic dipole... and we have already determined earlier above: its magnetic moment $\mu_l$ which units are for recall:
	
	It seems quite natural enough to write to go a little further:
	
	This now gives at the dimensional level:
	
	We then a unit of length in more. Then it seems quite natural to introduce the dipole length such that (the force should logically be equal at every point of the dipole so there is no reason to introduce here half of the dipole length!):
	
	Now to get back to the energy of the magnetic dipole, we consider that it is zero when the dipole is initially in a position perpendicular to the magnetic field. Using the common approximation as what infinitesimal movement of the two ends is given by for small angles:
	
	Thus, the elementary work (energy) to turn the dipole is (we multiply by two because we have to sum the two forces acting on each pole) by denoting by $B$ the norm of the magnetic field:
	
	where we see that the dipole length is no longer involved. In fact we must not forget that it is in the magnetic dipole moment $\mu_l$ that we have the equivalent area of the dipole.

	It comes by integration for a given final angle:
	
	Therefore:
	
	Let us see now a classic and academic case that we can get from this result and that we will see again during our study of the spin in the section of Wave Quantum Physics.

	So we have proved earlier above that an electric charged particle is deflected by a force given by the equation of the Lorentz' law:
	
	It follows that if the field has only a constant component in $z$ and the speed only one component in $x$, this will cause a helical motion in the plane perpendicular to the field as we have already proved above in our study of the Larmor radius.

	Let us consider now an electric charged particle launched at uniform speed along an $x$-axis between two poles of opposing magnets which generate a vertical and heterogeneous magnetic field and let us look only to the deflection along $z$ of the trajectory of the particle.
	
	From the perspective of the $z$-axis the particle can be considered in uniform acceleration (\SeeChapter{see section Classical Mechanics}):
	
	Since we assume that the initial position along $z$ of the particle is midway between the two poles and that its initial velocity along the $z$-axis such as:
	
	Then we have:
	
	Since the magnetic field does not work (see proof earlier above) and that implies that the kinetic energy remains constant, we can write that the time is simply the ratio of the horizontal distance traveled by the particle by the speed module:
	
	Let us recall that we have proved just earlier above that the potential energy of a magnetic dipole was in a constant and homogeneous field given by:
	
	Then in comes in our case:
	
	And as we can associate a force to a potential energy, it comes by assuming that the dipole magnetic moment is constant along the $z$-axis and the magnetic field is still somewhat a little bit inhomogeneous (happy mix of assumptions ... but we still do engineering physics in this book!):
	
	And finally:
	
	So we see in any case that $z$ can take a continuous spectrum of values that depends on the magnetic dipole moment of the particle. Now, as we shall see in the section of Wave Quantum Physics and Relativistic Quantum Physics during our study of quantum operators and especially that of the orbital angular momentum and of spin, an experiment named "\NewTerm{Stern-Gerlach experiment}\index{Stern-Gerlach experiment}" has shown that this was not the case for some particles or atoms for which the $z$ are clearly discrete, what classical physics seemed unable to explain!!!!
	\begin{figure}[H]
		\centering
		\includegraphics[scale=0.9]{img/electromagnetism/stern_gerlach_experiment.jpg}
		\caption[]{Stern-Gerlarch experiment principle and result (source: Wikipedia, source: JohnDC$^\dagger$)}
	\end{figure}
	Stern and Gerlach were astonished by what they saw. Not only were these "free" electrons deflected by the magnetic field, the pattern of their deflection was itself totally unexpected and mysterious. They found that the beam separated into two distinct parts. The basic experimental set-up is shown below.

	This pointed to two things. First, the "free" electrons must have some kind of built-in magnetic moment because they are interacting with the magnetic field. Second, this magnetic moment is no ordinary dipole moment - the electrons are not acting like tiny little magnetic spheres shooting through the magnetic field.
	
	\pagebreak
	\subsection{Langevin treatment of Diamagnetism and Paramagnetism}
	We have define earlier what were diamagnetic and paramagnetic materials. In 1905 Langevin proposed after thrial and error the theories of diamagnetism and paramagnetism (in 1906 Pierre Weiss proposed the theory of ferromagnetism) under some strong and simplified assumptions but that still work quite well to explain some matherial behaviors. We will see here what are the developements used to explained these behaviors.
	
	\subsubsection{Langevin model of diamagnetism}
	The purpose of this model is be able to explain the negative magnetism that is opposed to the magnetic excitation such that we can obseve it in practice. This model is rough relatively to the quantum model but it is interesting for two main reasons: the first is that it already gives enough concepts to the reader to start the study of this subject if not familiar with quantum theory, the second being that it is is a good training model (in the academic sense) because it shows how trials and successive approximations can lead to something relatively acceptable in practical perspective.
	\begin{tcolorbox}[title=Remark,colframe=black,arc=10pt]
	This model has its place in the section Magnetostatics because the excitation field used in the model is assumed to be constant!
	\end{tcolorbox}
	For this, we consider the classical model of Langevin (the quantum model giving the same result as we will see in the corresponding section), where the electron is regarded as traveling a circular orbit of radius $r$ and then can be assimilated to an electric current in a loop producing an electromotive force (\SeeChapter{see section of Electrokinetics}):
	
	that we can assimilated to an electric field such that (\SeeChapter{see section of Electrokinetics}):
	
	Therefore it comes:
	
	We then have by denoting $m_e$ the rest mass of the electron and $q_e$ its electric charge:
	
	hence:
	
	The application of an external magnetic excitation will have for effect to change the magnetic dipole moment $\mu_l$ of quantity $\Delta \mu_l$. But, we have prove earlier above that the magnetic dipole moment was given by:
	
	Then it comes for the electron:
	
	Then, the very clever and rough trick (in the approximate sense of the term relatively to the experience and to the quantum model developed a few years later) is to take into consideration the fact that the electron in classical point of view can be considered as a point object that can move in a sphere of radius $R$ given in the case of a mono-electronic atom and not just in a planar circular radius $r$ perpendicular to the direction of the magnetic excitation field.

	In this case, we then have of course:
	
	and we will consider that the three coordinates are independent and identically distributed random variables (so already the theoretical model is very rough but it's better than nothing...). Therefore, it follows that their expectation is equal such as:
	
	Using the property of linearity of the expected mean (\SeeChapter{see section Statistics}) and the notation used by the physicist, it is then written:
	
	and as the coordinates are considered like random variables identically distributed, we also have:
	
	It follows immediately that:
	
	And so if we are interested only to the average  disk radius containing all orbits perpendicular to the direction of the magnetic excitation field directed along the $z$-axis, then it comes:
	
	Therefore, finally, for an electron in all possible orbits of a limited sphere:
	
	where $\langle R^2 \rangle$ can be explicitly calculated with the quantum model (with wave functions to be precise!).

	For an atom containing $Z$ electrons, we will do the rough assumption that a simple sum of the effects is valid ...:
	
	From a macroscopic point of view, the number of atoms in a unit volume will be the ratio of the density of the material divided by the atomic weight of the element in question multiplied by Avogadro's number:
	
	It comes then by unit volume:
	
	and it is this result that is assmilated to the magnetic susceptibility as it's written in the form of the "\NewTerm{Langevin's diamagnetic susceptibility relation}\index{Langevin's diamagnetic susceptibility relation}":
	
	At unit level, we have well:
	
	So it is a coefficient without units as expected.... and it is because this coefficient is negative that we assimilate this model to diamagnetism (by definition). The reader will have perhaps noticed that if the magnetic excitation is zero, the susceptibility is also zero... which is an expected minimum behavior of the theoretical model. But by cons this latter does not depend on the temperature (the influence thereof is anyway almost negligible in standard laboratory temperature range).

	The agreement experiment / theory is excellent for spherically symmetric elements in the order of $\pm 10\%$ error. For non-spherical elements the error often reaches $50\%$ with respect to the experience.
	
	\pagebreak
	\subsubsection{Langevin model of paramagnetism}
	Langevin tried (with varying degrees of success here too) to explain the paramagnetism with the same underlying ideas, but however by opting for a completely different mathematical approach to ensure a positive outcome... (do it yourself physicist way of life...). What Langevin also knew it is that paramagnetim highly dependent on the temperature according to experimental studies of ferromagnetic materials, it was therefore necessary to choose an approach emphasizing the temperature and at the time there were no $10,000$ ways to do this! It follows that this model opens also the door to the theory of ferromagnetism!

	As at the starting point at that time we naturally assumed the (Maxwell-)Boltzmann distribution proved in the section of Statistical Mechanics that describes for recall the distribution of detectable particles that do not interact with any constraints on the number of particles state... (at the time of Langevin there were only this model available...):
	
	where as the reader notice it (\SeeChapter{see section Statistical Mechanics}) the potential energy $\mu_i$ is taken to be zero, so that all energy is in the form of kinetic energy.
	
	It is important to notice that the latter writing  of the distribution function, as we have seen in the section Statistical Mechanics, ignores the normalization constant to make it a true probability density function (but we will obviously calculate that constant a little further below).

	We have also proved just earlier that the magnetic potential energy of a dipole was given by:
	
	Now, let us recall that we have proved in the section of Geometric Shapes that the surface of a sphere element was given by:
	
	It comes therefore (the reader may refer to the figure of the section of Geometric Shapes) that for a crown of the sphere defined by two parallel planes, which middle contains the origin of the sphere is then given by (if needed and on readers request we can draw another figure) :
	
	Why are we talking about this? Well because a portion of the total number of magnetic dipoles included in an angle range $\mathrm{d}\theta$ is as we see above proportional to a surface element as:
	
	We then have that this number is given by (to a given unknown constant factor $K$):
	
	The corresponding proportion (so it is also a probability) over all angles, for a given angle, is then given after normalization obviously by:
	
	We saw during our study of the magnetic dipole moment that $\vec{\mu}_l$ contributes to the magnetic field. If we have a volumic density of $n$ magnetic dipoles, then we have a contribution of around $n\vec{\mu}_l$ if they are all oriented in the same direction. But if we project the vector of the magnetic moment on the direction of the magnetic field, the contribution of dipoles will then be written:
	
	but as there are many dipoles in different directions and that do many various angles with the field, then we have to take the average such that the contribution to the magnetic field is proportional to:
	
	And using simply the properties of density statistical functions (\SeeChapter{see section Statistics}), the expected mean contribution is given by:
	
	and as the denominator is just a normalization constant, we can get it out of the first integral:
	
	To integrate, let us do the little simplification of writing by putting:
	
	This gives us:
	
	and let us put:
	
	Then we have:
	
	It comes then:
	
	The primitive numerator is known to us because it is part of the usual primitive proved in detail in the section of Differential and Integral Calculus! The integral in the denominator is trivial:
	
	The function:
	
	is often named "\NewTerm{Langevin's function}\index{Langevin's function}" with for recall:
	
	\begin{figure}[H]
		\centering
		\includegraphics{img/electromagnetism/langevin_function_plot_maple.jpg}
		\caption[]{Plot of Langevin's function with Maple 4.00b}
	\end{figure}
	The Langevin function is equal $0$ when its parameter is $0$ and approaches $1$ when its argument tends to infinity. So the system eventually saturate when the magnetic field increases, which corresponds to the experimental behavior of paramagnetic materials. By cons the increase in temperature therefore decrease factor $\gamma$ and makes the Langevin's function tend to $0$ and has for effect to cancel the alignment of dipoles.

	For small values of the parameter $\gamma$, the Langevin's function can be considered as linear as we see on the plot above.
	To simplify the expression, we'll calculate the Taylor approximation of the hyperbolic cotangent using detailed proved presented in the section of Sequences and Series and when the hyperbolic cotangent argument is to recall strictly less than $1$ in absolute value:
	
	Then we have:
	
	We have therefore the magnetic field which is proportional to:
	
	We see that the factor:
	
	is dimensionless. Indeed:
	
	So we can consider it is the paramagnetic susceptibility and write the "\NewTerm{Langevin's relationship of the paramagnetic susceptibility}\index{Langevin's relationship of the paramagnetic susceptibility}":
	
	better known under the name "\NewTerm{Curie's law}\index{Curie's law}" and shows that shows that the magnetic susceptibility is inversely proportional to the temperature (good but obviously this law becomes false at low temperatures and we must then derive empirically the Curie-Weiss law).
	
	As a little summary can always be useful here is one:
	
	
	\begin{flushright}
	\begin{tabular}{l c}
	\circled{90} & \pbox{20cm}{\score{4}{5} \\ {\tiny 48 votes,  79.17\%}} 
	\end{tabular} 
	\end{flushright}

	%to force start on odd page
	\newpage
	\thispagestyle{empty}
	\mbox{}		
	\section{Electrodynamics}
	\lettrine[lines=4]{\color{BrickRed}C}lassical electrodynamics is a branch of theoretical physics that studies the interactions between electric charges and currents using an extension of the classical Newtonian model. The theory provides an excellent description of electromagnetic phenomena whenever the relevant length scales and field strengths are large enough that quantum mechanical effects are negligible. For small distances and low field strengths, such interactions are better described by quantum electrodynamics.
	
	In this section we will build a set of equations that can sum up themselves our entire knowledge about electrostatics and magnetostatics. These equations, at the number of four, are named "\NewTerm{Maxwell-Heaviside equations}\index{Maxwell-Heaviside equations}" (denomination that we will shorten as many other books under the name "\NewTerm{Maxwell's equations}\index{Maxwell's equations}") and will allow us to approach the branch of physics named "\NewTerm{electrodynamics}\index{electrodynamics}" and therefore electromagnetic waves (i.e.: light!).
	
	Electrodynamics is a pillar of the electronic revolution! Without this theory: no radio, no phones or cell phones, no computers, no satellites, no satellites, no electric motor, we would still at the technological state of the late 19th century without it!
	\begin{tcolorbox}[title=Remark,colframe=black,arc=10pt]
	It is very important to understand what will follow! Some developments will be reused in the section of Special Relativity, Quantum Field Theory, etc. Furthermore, the reader should also read in parallel the section of Special Relativity to better understand the ins and outs of certain results and the provenance of some mathematical tools.
	\end{tcolorbox}
	We assume before tackling mathematical models that everyone admits at the beginning of this third millennium than gamma rays, radio waves, microwaves, visible light (and not visible light) are electromagnetic waves (EM) of different frequencies:
	\begin{figure}[H]
		\centering
		\includegraphics{img/electromagnetism/spectrum.jpg}
	\end{figure}
	With a non-exhaustive summary of the current applications of the wave frequencies in this early 21st century:
	\begin{figure}[H]
		\centering
		\includegraphics{img/electromagnetism/spectrum_applications.jpg}
	\end{figure}
	Frequency allocation to the business economy, civil and military industry is the role of the International Telecommunication Union.
	
	\subsection{Maxwell Equations}
	As already mentioned Maxwell's equations are a set of partial differential equations that form the foundation of classical electrodynamics, classical optics, and electric circuits and encompass the knowledge of almost all 19th centure about the subject. These fields in turn underlie modern electrical and communications technologies. Maxwell's equations describe how electric and magnetic fields are generated and altered by each other and by charges and currents.
	
	If there is a God (concept that remains to be proved) many people say that probably God wrote the Maxwell Equations first and let there be light: and there was light (Genesis).
	
	\subsubsection{First Maxwell Equation (constant electric flow)}
	Let us define $\vec{\mathcal{E}}$ a vectors field in space. Consider a closed surface $S$ in this field. Then at each point $(x, y, z)$ belonging to the surface corresponds a vector of the field:
	
	In this case the Ostrogradski's theorem give us (\SeeChapter{see section Vector Calculus}):
	
	with $V$ being the volume delimited by the closed surface $S$ (so-named for recall "\NewTerm{Gauss surface}\index{Gauss surface}").
	\begin{tcolorbox}[title=Remark,colframe=black,arc=10pt]
	The Ostrogradski theorem is verified since there is no singularities in the volume $V$!
	\end{tcolorbox}
	Before continuing let us also recall that in the case of the Ostrogradsky theorem the normal vector $\vec{n}$ is conventionally directed outwardly from the surface.
	
	In the particular case of an electric field, we get some very interesting results. Indeed, given a charge $Q$ located relatively to a frame by the vector $\vec{r}_Q$. Then, we already saw in the section of Electrostatic that in every point in space, there is a field $\vec{E}$ such that:
	
	therefore:
	
	As we can see it, the field $\vec{E}$ has a singularity on $(r_{Q_x},r_{Q_y},r_{Q_z})$. Let us consider a Gauss surface such as the charge $Q$ lies outside of this surface! Within the volume $V$ delimited by the surface $S$ the field $\vec{E}$ then has no singularities. So we can calculate the divergence (\SeeChapter{see section Vector Calculus}) of $\vec{E}$:
	
	Therefore we get:
	
	So if we calculate the flow through this surface we find (see the section of Vector Calculus for a detailed description of the "Nabla" operator represented by the symbol "del"):
	
	The flow is equal to zero!
	
	In the case where the charge $Q$ is located inside the Gauss surface $\vec{\nabla}\circ\vec{E}$ is no longer defined on $(r_{Q_x},r_{Q_y},r_{Q_z})$ we have then:
	
	With $\Phi_B$ being the flow $\vec{E}$ on a small ball  $B$ around the punctual charge $Q$.
	
	In this case:
	
	as the divergence is defined everywhere on $V-B$. So it remains:
	
	But in the case of a sphere, it is quite easy to calculate:
	
	We have:
	
	hence the "\NewTerm{first Maxwell equation}\index{first Maxwell equation}" or "\NewTerm{Gaussian law}\index{Gaussian law}" for the electric field (or "\NewTerm{Gauss theorem}\index{Gauss theorem}") with a slightly condensed notation:
	
	where $\rho_Q$ is the charge density in Coulombs per $\text{m}^3$. On the left we have the integral form of the first Maxwell's (electro-magnetostatics) equation and its differential form on the right.
	
	This equation suggests that the flow of the electric field through a closed surface (hence the circle on the integral) is equal to a given dimensional factor, to the total charge enclosed in the latter surface.
	\begin{tcolorbox}[title=Remark,colframe=black,arc=10pt]
	The integral of the last relation is a line integral (thus evaluated on a curve). In the field electrodynamics, line integrals  apply very often on paths or closed surfaces hence the indication of a circle superimposed on the integral symbol and being named "\NewTerm{circulation of the vector field}\index{circulation of the vector field}".
	\end{tcolorbox}	
	If we now express this equation in terms of the electrical potential for which we have proved in the section Electrostatics that:
	
	We get:
	
	We can write the above relation more aesthetically using the scalar Laplacian (\SeeChapter{see section Vector Calculus}) such that we get the relation:
	
	named "\NewTerm{Maxwell-Poisson equation}\index{Maxwell-Poisson equation}".
	
	\subsubsection{Second Maxwell Equation (non-existence of magnetic monopomple)}
	In the particular case of a magnetic field, we also get some very interesting results.
	
	Indeed, given a current $I$ located relatively by a reference frame by the vector $\vec{r}_I$. Then at each point $\vec{r}$ in space, we proved in the section Magnetostatics that there is a field $\vec{B}$ such that:
	
	Therefore:
	
	As we can see it, the field $\vec{B}$ has a singularity on $(r_{I_x},r_{I_y},r_{I_z})$. Let us consider then a Gaussian surface such that the current $I$ is outside this surface!

	Within the volume $V$ delimited by the surface $S$ then field $\vec{B}$ has no singularities anymore. We can therefore calculate the divergence (\SeeChapter{see section Vector Calculus}) of $\vec{B}$:
	
	Therefore:
	
	If we calculate the flow through this surface, we then find:
	
	The flow is equal to zero!
	
	In the case where the current $I$ is located within the Gauss surface, $\vec{\nabla}\circ\vec{B}$ is no longer defined in $(r_{I_x},r_{I_y},r_{I_z})$ we have then:
	
	with $\Phi_{B'}$ being the flow of $\vec{B}$ on a small sphere $B'$ surrounding partially the straight conductor carrying the current $I$. In this case:
	
	as the divergence is defined everywhere on $V-B'$. Therefore it remains only that:
	
	But in the case of a sphere, it is easy to calculate:
	
	We then have the Gauss's law for the magnetic field:
	
	Indeed, in the case of the magnetic field, $\mathrm{d}\vec{S}$ and $\vec{B}$ are therefore perpendicular:
	
	\begin{tcolorbox}[title=Remark,colframe=black,arc=10pt]
	From which we can deduce that $\vec{B} \perp\vec{E}$ !
	\end{tcolorbox}
	Therefore, given a Gaussian surface in a magnetic field, then the magnetic field flow through this surface is equal to:
	
	relation which is the "\NewTerm{second Maxwell equation}\index{second Maxwell equation}". On the left we have the integral form of the fourth Maxwell's equation and its differential form on the right.
	
	This second equation is equivalent to say that there is no "\NewTerm{magnetic monopole}\index{magnetic monopole}" in nature, that is to say, at any positive pole, we have to find a negative one (from a magnet the field lines do not diverge). However, the second equation just add the idea (demonstrated by Dirac) that if it was possible to find a monopole in the Nature, it would be the point source of the magnetic field. We will see this a little further in detail below.
	
	\subsubsection{Third Maxwell Equation}
	We shall see in the section of Electrokinetics (because we need some concepts that we have not study yet), that the variation of the magnetic field flow in time through a conductive loop induces a voltage in the loop given by the "\NewTerm{Faraday's law}\index{Faraday's law}" or "\NewTerm{Lenz-Faraday's law}\index{Lenz-Faraday's law}":
	
	and we have already proved in the section of Electrostatics that:
	
	where the last equality is valid only in the special case where the pat is collinear to the electric field.
	\begin{figure}[H]
		\centering
		\includegraphics[scale=0.5]{img/electromagnetism/lorenz_law.jpg}
		\caption{Lorenz law illustration}
	\end{figure}
	\begin{tcolorbox}[title=Remark,colframe=black,arc=10pt]
	We will see in the section of Electrokinetics that it is not quite correct to denote the potential with $U$ as above because in fact, Faraday's law expresses the electromotive force EMF (electromotive potential) denoted by $e$ and this potential is not conservative unlike the electrostatic Coulomb potential (for which the integral over a closed path is zero as we have already proved it in the section of Electrostatics).
	\end{tcolorbox}
	For an element $\mathrm{d}\vec{l}$ of a circuit $\Gamma$ it comes:
	
	the sign change is here justified by the Lenz law that said that the induced current (and magnetic flow associated with it) has such a direction that it opposes to the change in the flow through the circuit (see figure above).
	
	If we develop this relation, using Stokes' theorem (\SeeChapter{see section Vector Calculus}) which is for recall:
	
	Then we have:
	
	Where, as we will see it in the section of Electrokinetics, the electric field above is not the simple Coulomb field but the sum of a Coulomb field and an electromotive field (implicitly generated by Biot-Savart force).
	
	We then have:
	
	And if the surface element does not move in space and only the magnetic field varies with time, then we have:
	
	Therefore:
	
	A trivial solution is then to say that:
	
	We then get finally:
	
	This is the "\NewTerm{third Maxwell equation}\index{third Maxwell equation}" or "\NewTerm{Faraday-Maxwell law}\index{Faraday-Maxwell law}" sometimes still named "\NewTerm{induction law}\index{induction law}". On the left we have therefore the integral form of the third Maxwell's equation and its differential form on the right.
	
	The third equation therefore affirms that a variation of the magnetic field produces an electric field in a conductor loop. We then say that the term with the partial derivative of the magnetic field is the "\NewTerm{magnetic coupling term}\index{magnetic coupling term}". This equation is based on the Faraday's theory.
	\begin{tcolorbox}[title=Remark,colframe=black,arc=10pt]
	Often in the scientific literature, the potential $U (t)$ is simply denoted by a tiny $u$.
	\end{tcolorbox}
	Faraday's law of induction is typically used by small portable devices such as below PEG (Personal Energy Generator) to charge portable electronic devices:
	\begin{figure}[H]
		\centering
		\includegraphics{img/electromagnetism/peg.jpg}
		\caption{Photo of a PEG (right) with a mobile phone (left)}
	\end{figure}

	\paragraph{Betatron}\mbox{}\\\\
	Among the many examples of application of the third law that we will see in other sections of this book, there is one that is particularly nice because it is reminiscent of modern physics on a large scale (even if in reality we are very far).
	
	Thanks to the Maxwell equations and the relations proved in the section Magnetostatics, we can make a small non-exhaustive theoretical study of the physical principle underlying one of the oldest nonlinear particles accelerator.
	
	One of the first non-linear methods that comes to mind is to accelerate a charged particle through magnetic induction. This type of accelerator is named a "\NewTerm{betatron}\index{betatron}" (the idea is that it accelerates the electrons as fast as the one that appears in the beta decay...) and was conceptualized in the 1930s.	
	
	The betatron is a particle accelerator that injects electrons into a vacuum torus (in white on the photo below) submitted to a magnetic field that will be considered here as homogeneous between the two magnets (in red in the photo below) to obtain intense X-rays or gamma rays useful in certain professional applications (medicine, analysis of structures, etc.). This accelerator is limited by the magnetic field that it can produce or support.
	\begin{figure}[H]
		\centering
		\includegraphics{img/electromagnetism/betatron.jpg}
		\caption{Photo of a Betatron}
	\end{figure}
	For this theoretical study, we will first use the result proved in the section Magnetostatics when we study the Larmor radius: an electron moving in a magnetic field will have a circular path which is perpendicular to the magnetic field.
	
	Then we will also need the third Maxwell equation in the integral form:
	
	which - for recall - said that a change in the magnetic field produces an electric field in a conductor loop (or in a charged particle movement that can be assimilated to a conductor loop!!!!).
	
	We have:
	
	and as the path is circular in the betatron as we have proved in our study of the Larmor radius in the section Magnetostatics, we have:
	
	Now, as the electric field is tangential to the circular path of the electrons and that they go in the opposite direction of that field (direction that is constant at any point of the path), we have since electrons travel in circles:
	
	But we also have:
	
	Then it comes:
	
	therefore:
	
	We would like to calculate the kinetic energy that the negatively charged particle acquires after several rounds. This latter is equal to the work done by the electric field to move the load on the circular path (remember that the magnetic field don't do "work"!!!).
	
	As we have proved in the section of Electrostatics, we have along an electric (constant) field line:
	
	Therefore it comes as the charge travels $N$ times the circumference of the betatron:
	
	\begin{tcolorbox}[colframe=black,colback=white,sharp corners]
	\textbf{{\Large \ding{45}}Example:}\\\\
	Let us consider a sinusoidal magnetic field of magntitude $B_0=1 [\text{T}]$ at a frequency of $f=50$ [Hz], therefore a period $T$ of $20$ [ms]. This means that in $5$ [ms] the magnetic field passes from a maximum to a zero value.\\

	Let us also consider that we have a betatron with a circular path of $1$ [m] and the electron can remain approximately $480,000$ rounds on this circular trajectory with this specific radius without deviating too much (the equivalent of about $3,000$ [km] traveled). The electron is injected with an energy of $2$ [MeV] in the vacuum torus (energy that is already very close to that of the speed of light!).\\
	
	Let us first calculate the initial radius for the trajectory according to the relativistic Larmor radius. For this, first we need the speed corresponding to the energy of $2$ [MeV]:
	
	After some elementary algebraic operations, we find:
	
	Therefore the initial Larmor radius is equal to:
	
	We then for during all the time of the acceleration that will take the electron to a Larmor radius of $1$ [m] a kinetic energy gain of:
	
	Which corresponds in electron volts to:
	
	Hence the energy that was measured experimentally at that time. This energy also corresponds to a speed which is very close to that of light. Thus, with the same calculation as above, we get:
	
	speed reached only in a hundredths of a second!
	\end{tcolorbox}
	\begin{tcolorbox}[title=Remark,colframe=black,arc=10pt]
	Therefore in reality, the centrifugal force increases gradually as the electron acquires kinetic energy (and therefore the speed). This force must be compensated by increasing the Lorentz force accordingly.
	\end{tcolorbox}
	
	\subsubsection{Fourth Maxwell Equation}
	The fourth Maxwell equation is in our poin of view the most important one. It is a generalization of the Ampere's law which has already been presented in the section Magnetostatics and for which we got the circulation $C_B$ of the magnetic field:
	
	Remember that the third Maxwell's equation tells us that the variation of a magnetic field gives generated to an electric field. We can therefore assume that the converse may be probably true (need experimentation to be checked!).
	
	A typical place where we can observe a variation of an electric field, for example, is the capacitor (\SeeChapter{see section Electrokinetics}).
	
	We know that:
	
	and that the electric field between two parallel planes, of surface $S$, bearing electric charges $Q_+=-Q_-$, uniformly distributed is given by (\SeeChapter{see section Electrostatics}):
	
	where $\sigma$ is the surfacic density.

	This result in independant of the distance $D$ between the planes. The first Maxwell equation give us:
	
	The capacity of a condensator is defined by (\SeeChapter{see section Electrostatics}):
	
	we obtained in the particular case of plane and parallel capacitor that the capacity was equal to:
	
	Therefore it comes:
	
	and using the fact that the electrostatic potential is the electric field multiplied by a distance that will be taken in this case as the distance $D$ between the two planes of the capacitor, we have:
	
	As the electric field may change with time, it is often customary to put a tiny $i$ for the electric current variable (this is a tradition that we will see again the section Electrokinetics) and as between the two capacitor planes there only vacuum, then we speak of "\NewTerm{displacement current}\index{displacement current}", which is why the latter relation is often denoted as follows:
	
	By expressing the above expression by using the surface current density, we get:
	
	If the electric field is not uniform in space and thus depends on the spatial coordinates, we'll use the partial derivatives such as:
	
	The displacement current generates a magnetic field calculable using Ampere's law:
	
	In all phenomena where we observe a charge displacement, we can assume that there is creation of a displacement current which is superimposed on the current flow because of capacitive effects in the material. We write therefore:
	
	where we have (recall of the section of Electrostatic and Magnetostatics):
	
	Furthermore, the Stokes theorem provides us that:
	
	therefore:
	
	and we get from this finally:
	
	This is the "\NewTerm{fourth Maxwell equation}\index{fourth Maxwell equation}" or "\NewTerm{Maxwell-Ampere equation}\index{Maxwell-Ampere equation}". On the left we have the integral form of the fourth Maxwell's equation and its differential form on the right.
	
	The fourth and last Maxwell equation combines the creation of a magnetic field to any variation of an electric field and / or the presence of an electric current (the presence of an electric current being a sufficient condition but not necessary in view of the second term). We then say that the term with the partial derivative of the electric field is the "\NewTerm{term of electrical coupling}\index{term of electrical coupling}".
	
	To summarize a bit we have therefore the following four Maxwell equations named "\NewTerm{local forms of Maxwell's equations}\index{local forms of Maxwell's equations}" in differential form (when the integrals are not indicated):
	
	In the case where $\mu_r,\varepsilon_r\neq =1$, that is to say if we do not work in vacuum but in the matter, we write the local Maxwell equations in the following form:
	
	where $\vec{D}$ is (for recall) named the "\NewTerm{displacement field}\index{displacement field}" or "\NewTerm{electric induction}" and (for recall) $\vec{H}$ is the "\NewTerm{magnetic excitation}\index{magnetic excitation}".
	\begin{tcolorbox}[title=Remark,colframe=black,arc=10pt]
	Caution!! $\vec{E}$ is a reaction of the vacuum to the field $\vec{D}$. This is due to the permittivity vacuum constant  set in the integral (at least that's one way of seeing the thing...).
	\end{tcolorbox}
	
	But in the vacuum and in the case where we consider an absence of charges, we get:
	
	This result is important because it expresses the possibility of the propagation of the electric and magnetic field even in the absence of sources!!! We will use these equations to determine the electromagnetic wave equations further below.
	\begin{tcolorbox}[title=Remark,colframe=black,arc=10pt]
	It is possible to express the Maxwell equations under relativistic form... but in reality, as we have already noticed, the equations are unchanged! Indeed, Maxwell's equations are already relativists. This is not surprising, because the vectors of electric and magnetic fields, the photons (\SeeChapter{see section of Quantum Field Theory}), travel at the speed of light. At this speed, relativity is the Queen and a correct theory could only be relativistic. However, we can express the latter equations using tensor mathematical notations (see below our demonstration of the tensor of the electromagnetic field). In this form the four equations become incredibly simple and compact (only one extremely short equation). Formulated in this way, the electric and magnetic fields are written as a single field named of course the "\NewTerm{electromagnetic field}\index{electromagnetic field}". It is a tensor field as we shall see later.
	\end{tcolorbox}
	

	\subsubsection{Magnetic Monopoles}
	Notice that opting for the natural measurement system where $c=1$, then we have to Maxwell's equations in vacuum:
	
	since as we prove it further below, in vacuum:
	
	Then the transformation:
	
	brings the second pair of preceding equations the first one!!!! This symmetry of Maxwell's equations is named "\NewTerm{duality}\index{duality}" and that's a clue that suggests that the electric and magnetic fields are only the unified parts of a whole.
	
	Moreover, if we introduce the following complex field:
	
	the duality (taking the real part only), then is written:
	
	the pair of Maxwell's equations indicated above is then reduced to (we use the property of linearity of the cross product) only one pair of equations which we must not forget to take only the real part:
	
	However, this symmetry does not extend to the Maxwell equations with sources expressed in the natural system by:
	
	but half of the time it does not work (make the substitution of $\vec{\mathcal{E}}$ you will see that you always get one of the equations on the pair that is consistent and the other not!). The trick then is to separate the two densities in their respective imaginary and real part (as we can see, generalize any physics theory to complex numbers always bring us to very interesting stuff as for maths!!!):
	
	We then get (always without forgetting to take the real parts and not forgetting that we are in natural units):
	
	then we simply have to put $\rho_m=\vec{j}_m=0$. These equations are certainly charming but their generalization brings nothing new because no magnetic charge expressed as:
	
	and named "\NewTerm{magnetic monopole}\index{magnetic monopole}" was observed to this date. In an experimental context, we say then that $\rho_Q,\vec{j}$ are real such that we have well:
	
	The previous system rewritten as following is named "\NewTerm{Symmetrized Dirac-Maxwell equations}\index{Symmetrized Dirac-Maxwell equations}" (in natural units):;
	
	To see if we have the right to write the relations above, remember first that if we take the divergence of the curl of the electric field, we know that is operation (\SeeChapter{see section Vector Calculus}) is always equal to zero regardless of the function considered:
	
	Then we have the following result:
	
	Hence:
	
	Therefore:
	
	Thus after rearrangement:
	
	So is the fact of falling back on a continuity equation (identical in the form to the thermodynamics continuity equation, fluid mechanics continuity equation, electric field continuity equation or even that of the quantum probability continuity equation, etc.) that would have brought Dirac to complete the four Maxwell equations we have written just above.
	
	Remember that current is the movement of charge. The continuity equation says that if charge is moving out of a differential volume (i.e. divergence of current density is positive) then the amount of charge within that volume is going to decrease, so the rate of change of charge density is negative. Therefore the continuity equation amounts to a conservation of charge.
	
	\subsection{Charge conservation equation}
	So we proved so far the four Maxwell's equations which are the foundations of classical electrodynamics.

	The Maxwell's equations can be divided into two groups:
	\begin{enumerate}
		\item Equations without sources:
		

		\item Equations with sources (in vaccuum below)
		
	\end{enumerate}
	Deriving the first equation with sources in respect to time:
	
	by taking the divergence of the second we get:
	
	The divergence of a curl (rotational) is always zero as we have prove it in the section of Vector Calculusand therefore the last expression is zero. But since a reader asked us, we detail this result more explicitly simplifying a bit:
	
	but, $\vec{\nabla}\circ\vec{B}=0$ and therefore:
	
	After simplification and using natural units (\SeeChapter{see section Principia}) we get:
	
	which is named the  "\NewTerm{charge conservation equation}\index{charge conservation equation}" or "\NewTerm{equation of continuity}\index{equation of continuity}" and said that in two near instants $t+\mathrm{d}t$, the variation $\mathrm{d}Q$ of the charge contained in a closed surface defining a system can only be exclusively  attributed to a charge exchange with the outside.
	
	This equation is very important because it involves in the study of relativity, the load is a translation invariant quantity.
	
	\subsection{Gauge Theory}
	Before you start reading this subsection it is of primary importance for the reader to go for a ride in the Algebra chapter of this book, in which there is a section Vector Calculus where we prove some differential vector operators that are of first importance in physics and especially for gauge theory and where we also study their main properties.
	
	What will follow is very important because besides the fact that we are going to naturally present a new vector field (the vector potential) which is essential in certain equations of relativistic quantum physics (see section of the same name) we will reuse this gauges approach in the section of Wave Quantum Physics where the consequences are much more important!
	
	Given the known relation:
	
	There exists by the properties of the divergence and curl (rotational) operators (\SeeChapter{see section Vector Calculus}) a "\NewTerm{vector potential $\vec{A}(\vec{r},t)$}\index{vector potential}" such that:
	
	thus satisfying then (the divergence of the curl - rotational - of a vector field is always zero because it is a mathematical property proven in the section of Vector Calculus):
	
	and in the context of magnetostatic named "\NewTerm{magnetic potential}\index{magnetic potential}". 
	\begin{tcolorbox}[title=Remark,colframe=black,arc=10pt]
	The vector potential is therefore a potential... and a vector! Similarly as we have define a potential $U$ which derivates from an electric field $\vec{E}$, we can thus define a potential $\vec{A}(\vec{r},t)$ for the field $\vec{B}$. But for technical reasons (coming from the expression of the curl - rotational - of $\vec{E}$ and of $\vec{B}$ in the Maxwell's equations), the potential vector $\vec{A}(\vec{r},t)$ is not that simple as $U$ and can not be expressed as a simple scalar, we must use a vector potential.
	\end{tcolorbox}
	\begin{figure}[H]
		\centering
		\includegraphics[scale=0.9]{img/electromagnetism/magnetic_potential.jpg}
		\caption{Vector Magnetic Potential $\vec{A}$}
	\end{figure}
	Thus, a depiction of the $\vec{A}$ field around a loop of $\vec{B}$ flow (as would be produced in a toroidal inductor) is qualitatively the same as the $\vec{B}$ field around a loop of current.
	
	The reader will have notice that at the opposite of the electric potential they can be no magnetic field but still a magnetic potential.
	
	If we take the relation $\vec{B}=\vec{\nabla}\times\vec{A}$ in the Maxwell equation $\vec{\nabla}\times\vec{E}=-\partial \vec{B}/\partial t$ we get:
	
	We put now (the notation $\vec{F}$ has no relation with the Newtonian force!):
	
	and we use the mathematical properties of the curl (rotational) and gradient operators to write a new relation (the "$-$" sign is an anticipation of what will follow):
	
	hence:
	
	where $\phi(\vec{r},t)$ is a "\NewTerm{scalar potential}\index{scalar potential}". 
	\begin{tcolorbox}[title=Remarks,colframe=black,arc=10pt]
	\textbf{R1.} The field $\vec{F}$ seems to obey the same properties as that of the gravitational field (Newton-Poisson Law), but it seem to be only a curiosity (units and other mathematical properties are not equivalent).\\
	
	\textbf{R2.} The reader probably easily see that if the vector potential is zero, then we fall back on (\SeeChapter{see section Electrostatic}):
	
	which reinforces the assumptions of previous developments (and that's not ...)
	\end{tcolorbox}
	In addition, the fields $\vec{E}$ and $\vec{B}$ remain unchanged if we do in the previous relations the following replacements (the terms cancel trivially):
	
	where $\psi$ is an arbitrary function of $\vec{r}$ and $t$.
	
	We name such a transformation a "\NewTerm{gauge change}\index{gauge change}". The freedom on the choice of potentials allow us to impose a constraint that we name "\NewTerm{gauge constraint}\index{gauge constraint}".
	
	There are several ways of forming this constraint among which we distinguish two very common in physics!

	Thus we will use either the "\NewTerm{Lorentz gauge}\index{Lorentz gauge}" by imposing:
	
	or the "\NewTerm{Coulomb gauge}\index{Coulomb gauge}" by imposing:
	
	holds. Thus, $\psi$ should satisfy:
	
	The relation:
	
	which is named "\NewTerm{Poisson equation of the potential vector}\index{Poisson equation of the potential vector}".
	
	Similarly, to show that it is always possible to impose the Lorenz condition, we simply need to find $\psi$ in the above equations:
	
	such that the relation (Lorenz gauge):
	
	is satisfied. Therefore, $\psi$ must satisfy:
	
	Or in other words and in a more condensed manner:
	
	where the operator:
	
	is by definition named the "\NewTerm{Alembertian}\index{Alembertian}" (we will often encounter this term from now as well in the section of Electrodynamics and Wave Quantum Physics) that is also invariant under Lorentz transformation as we will prove it in the section of Special Relativity.
	
	Reporting the equations:
	
	in the other two Maxwell equations in vacuum:
	
	we get, by making appear the Laplacian of a vector field $\Delta \vec{A}$ by on of the properties of the curl, gradient and divergence vector operators (\SeeChapter{see section Vector Calculus}):
	
	the following relations:
	
	the latter relation being named the "\NewTerm{arbitrary gauge}\index{arbitrary gauge}".
	
	For the Lorenz gauge, these last two equation simplify to (feel free to contact us if you do not see how):
	
	that we name "\NewTerm{wave equations for electromagnetic potentials}\index{wave equations for electromagnetic potentials}" in analogy with the wave equations of electric and magnetic fields that we will determine further below.
	For the Coulomb gauge, the same equations simplify to:
	
	knowing that $c^2=1/(\varepsilon_0\mu_0)$ we can write the two wave equations of electromagnetic potentials in the form:
	
	Let us now put $A^0=\phi/c$ (to homogenize the units) such that we define a "\NewTerm{four-vector potential}\index{four-vector potential}" which allows us to write vectorially the two above relations in unified manner:
	 
	\begin{tcolorbox}[title=Remark,colframe=black,arc=10pt]
	The fact that the Alembertian of the four-vector potential  is expressed from the four-vector current which is contravariant (\SeeChapter{see section Special Relativity}) bring us to put that the four-vector potential is itself contravariant!
	\end{tcolorbox}
	Relation that we will we denote is a more condense form as follows:
	
	where $j^\alpha$ is name "\NewTerm{four-vector current}\index{four-vector current}".
	\begin{tcolorbox}[title=Remark,colframe=black,arc=10pt]
	We will see again this four-vector during our determination of the electromagnetic field tensor further below (except that we will be in natural units but this does not change the idea...).
	\end{tcolorbox}
	The four-vector potential as defined above leads us to be able to write the (quadrivergence) Lorenz gauge by making use of tensor notation:
	
	Which ultimately allows you to write the Lorenz gauge in the covariant form below:
	
	It is therefore an equation of the form of the Klein-Gordon equation for a massless particle (\SeeChapter{see section Relativistic Quantum Physics}). So we can say in a sense that the invariance of the electromagnetic gauge is connected to the fact that the mass of the photon is zero!
	\begin{tcolorbox}[title=Remark,colframe=black,arc=10pt]
	It is useful to notice that the fact that writing $\dfrac{1}{c}\dfrac{\partial }{\partial t}=\partial_0$ (with or without natural units where $c=1$) is a notation that will also be adopted in our study of the Dirac equation (\SeeChapter{see section Relativistic Quantum Physics}) or also in quantum field theory (except that there will be an imaginary part).
	\end{tcolorbox}
	These notations finally lead us to be able to write:
	
	We obtain the continuity equation:
	
	that is tensor equivalent form of the following relation (see prof earlier above):
	
	To summarize roughly:
	
	A given number of physical effects are modelized, depending on the case, by fields that can be scalar, vector, tensor, or spinor fields that we therefore name "gauges". A given number of physical phenomena seems to comply with conditions of say of "symmetry", vis-à-vis these gauges. This symmetry is expressed by what we therefore call a "gauge invariance".

	For example, the field that modelize well the electromagnetic field is as we have seen it, a four-vectors field consisting of a scalar potential $\phi$ (whose gradient is the electric field $\vec{E}$) and a vector-potential $\vec{A}$ (whose curl is the magnetic field $\vec{B}$). It is this quadrivectoriel field that helps us to modelize the electromagnetic field that is nameda "gauge".

	It turns out that we were getting exactly the same physical effects on a system of electric charged particles if we replace this gauge by another gauge by adding it a gauge  constraint (typical example between the Lorenz gauge or Coulomb gauge as seen above) . The invariance of the laws of physics when switching from a gauge to another is a "gauge invariance". In the case of the electromagnetic field, this gauge invariance turns expressing the electric charge conservation (as we have just proved it).

	Mathematically, such gauges changes appear to be the result of the action of a symmetry group of infinite dimensions (transforming these into each other gauges) that we name the "gauge group" of the considered interaction  (here the electromagnetic interaction).

	For the gravitational field, for example (\SeeChapter{see section of General Relativity}), the gravitational interaction is modeled by a symmetric tensor field of rank $2$ with a given signature. This metric field is distributed on a 4D variety  modeling the space-time. This is the gauge of the gravitational interaction. According to General Relativity (the equivalence principle) we do not change anything to the gravitational interaction if we change the system of space-time coordinates in which we express the metric. The portion of an expression of the metric to another by changing coordinate system is also a gauge change. The invariance of General Gelativity gauge then expressed the possibility to change to a gauge to another without changing the geodesic followed by the testing particles falling freely in the gravitational field modeled by the metric field.

	The invariance of General Relativity gauge is what we named "diffeomorphism invariance" (bijective change of coordinate system with some degree of regularity) and the General Relativity gauge group is the "diffeomorphisms group" of $\mathbb{R}^2$ (named the "soft group").

	It should be noted also that the potential-vector $\vec{A}$ is perhaps not so virtual as it may seem. Indeed, it is possible to modify the trajectories of charged particles passing outside a cylindrical volume where there is a magnetic field $\vec{B}$ induced by an electric current (traveling in the winding of a solenoid where this field $\vec{B}$ is "trapped" ). It is therefore possible to influence the particle trajectory circulating in an area where the magnetic field $\vec{B}$ is zero but where its vector potential equation is not!!!

	Furthermore, we will use the results we just get here for our study of the Yang-Mills theory in our way to the electroweak unification (see the Standard Model in the section of Field Quantum Theory). 
	\begin{tcolorbox}[title=Remark,colframe=black,arc=10pt]
	The well-known experiment that involves the vector potential is the Aharonov-Bohm experiment (\SeeChapter{see section of Wave Quantum Physics}).
	\end{tcolorbox}
	
	\subsubsection{Electromagnetic field tensor}
	In electromagnetism, the "\NewTerm{electromagnetic tensor}\index{electromagnetic tensor}" or "\NewTerm{electromagnetic field tensor}\index{electromagnetic field tensor}" (sometimes named the ""\NewTerm{field strength tensor}\index{field strength tensor}", "\NewTerm{Faraday tensor}\index{Faraday tensor}" or "\NewTerm{Maxwell bivector}\index{Maxwell bivector}") is a mathematical object that describes the electromagnetic field in space-time of a physical system. The field tensor was first used after the $4$-dimensional tensor formulation of special relativity was introduced by Hermann Minkowski. The tensor allows some physical laws to be written in a very concise form.
	
	To determine the electromagnetic field tensor let us assume at first that the action (\SeeChapter{see section of Analytical Mechanics}) of an electric charged particle in an electromagnetic field is given by (a priori empirical choice... but you'll see a little further):
	
	\begin{tcolorbox}[title=Remark,colframe=black,arc=10pt]
	The notation $S_0$ is reserved to the action of a free particle (\SeeChapter{see section of Special Relativity}).
	\end{tcolorbox}
	The Lagrangian for an electric charged particle in an electromagnetic field is therefore the sum of the Lagrangian of the particle interacting with the electromagnetic field $L_1$ added to the Lagrangian of the free particle $L_0$ (\SeeChapter{see section Special Relativity}):
	
	\begin{tcolorbox}[title=Remark,colframe=black,arc=10pt]
	This is therefore the Lagrangian of the interaction of the particle with the field added to the Lagrangian of the mass of the particle. Thus we see that it still missing the Lagrangian of the electromagnetic field itself in the case of absence of electric charges (named: "Lagrangian of the free field") but we will see it further below.
	\end{tcolorbox}
	This is therefore (a priori) the Lagrangian of a charged particle in an electromagnetic field having for source a given quantity of electric charges in relative movement.
	
	We will now prove that this Lagrangian is correct (the previous relation could be stated as a theorem)!
	
	For this remember first that the general momentum is (see section of Analytical Mechanics and Special Relativity):
	
	To verify that we made the right choice of Lagrangian initially, we will obtain the equations of motion and ensure that they coincide with the Lorentz force. The Lagrange equations are in this case:
	
	But we have:
	
	and therefore:
	
	But we pointed out in the definition of the scalar potential that $\vec{E}=-\vec{\nabla}\cdot\phi$ hence:
	
	We should necessarily have by analogy with the Lorentz force:
	
	So we need before proceeding to check that:
	
	With:
	
	In components:
	
	Therefore:
	
	That is:
	
	Therefore, distributing the terms:
	
	and as:
	And as:
	
	Therefore we have indeed the equality (yes i know written like this it looks stupid but don't forget the relation above we start from):
	
	These developments thus confirm our initial hypothesis as what the action of the field can be written:
	
	and that it expresses the interaction of a charged particle with a a magnetic field (since it contains the Lorentz force!).

	So now we have proved that the "\NewTerm{Lagrangian of the current-field interaction}\index{Lagrangian of the current-field interaction}":
	
	which we assumed empirically the shape earlier and that is finally well correct!

	The action integral is then written:
	
	Let us introduce the velocity $\vec{v}$ of the particle in the form $\vec{v}=\mathrm{d}\vec{r}/\mathrm{d}t$ and the integral is then written:
	
	Let us recall that we have proved in the section of Special Relativity the invariant:
	
	and also:
	
	As the space-time intervals are invariants such as (\SeeChapter{see section Special Relativity}):
	
	
	Let us introduce the velocity $\vec{v}$ of the particle in the form $\vec{v}=\mathrm{d}\vec{r}/\mathrm{d}t$ and the integral is then written:
	
	Let us recall that we have proved in the section of Special Relativity the invariant:
	
	and also:
	
	As the space-time intervals are invariants such as (\SeeChapter{see section Special Relativity}):
	
	If the repository O' is not in movement ($\mathrm{d}x'=\mathrm{d}y'=\mathrm{d}z'=0$), we have:
	
	hence:
	
	which is also written as:
	
	Therefore:
	
	Now let us use of the contravariant four-vector potential (see above):
	
	and the contravariant four-vector displacement (\SeeChapter{see section Special Relativity}):
	
	The expression of the action of an electric charged particle in an electromagnetic field and in Minkowski metric $\eta_{ij}$ (\SeeChapter{see sections of Special Relativity and General Relativity}) is finally reduced to the condensed expression:
	
	with therefore:
	
	without forgetting that here we use the metric $+, -, -, -$ (\SeeChapter{see sections Special Relativity and General Relativity}).
	
	Let us notice that the integral action in the absence of magnetic and electric field is the written:
	
	which corresponds well to what we have obtained in the the section of Special Relativity for a free particle!

	According to the principle of least action, the action integral has zero variation to the effective motion of the particle, therefore:
	
	\begin{tcolorbox}[title=Remark,colframe=black,arc=10pt]
	By the equality with zero, we can eliminate the minus sign before the integral.
	\end{tcolorbox}
	Using the expression of the curvilinear abscissa (\SeeChapter{see section Tensor Calculus and General Relativity}):
	
	for the Minkowski metric, we can write (remember that in the Euclidean metric only the terms of the diagonal where $i=j$ are non-zero):
	
	Therefore:
	
	the preceding integral is then written:
	
	This gives using curvilinear components (\SeeChapter{see section Tensor Calculus}):
	
	Let us integrate by parts (\SeeChapter{see sectionD Differential and Integral Calculus}) the first integral:
	
	But as:
	
	Therefore:
	
	with:
	
	can be written:
	
	The quantities $\delta x_i$ being arbitrary, the expression in brackets is zero:
	
	Let us put:
	
	The contravariant quantities $F^{ik}$ form the contravariant components of what we named the "\NewTerm{tensor of the electromagnetic field}\index{tensor of the electromagnetic field}" or "\NewTerm{Faraday's tensor}\index{Faraday's tensor}" (hence the $F$...) or more commonly the "\NewTerm{Maxwell tensor}\index{Maxwell tensor}". We say then that $F^{ik}$ the "\NewTerm{curl of (the magnetic) potential}\index{curl of (the magnetic) potential}".

	The "\NewTerm{equations of motion of a particle in an electromagnetic field}\index{equations of motion of a particle in an electromagnetic field}" then take the form:
	
	that some physicists name the "\NewTerm{geodesic corrected by a Lorentz force}\index{geodesic corrected by a Lorentz force}".	
	\begin{tcolorbox}	[title=Remark,colframe=black,arc=10pt]
	The tensor of the electromagnetic field is invariant under the transformations:
	
	Indeed:
	
	\end{tcolorbox}
	In Minkowski metric $\eta_{ij}$ (we will need the tensor of the electromagnetic field in the section Special Relativity, hence the choice of this metric), we have however:
	
	Which gives:
	
	The term $\partial_jA^k-\partial^k A_j$ is often denoted $F^{ik}$ (even if it is not more fully contravariant).

	It remains to us to determine the contravariant components of the tensor $F^{ik}$ (tensor which has the property of being antisymmetric such that $F^{ik}=-F^{ki}$ aw we will see).

	Let's start with the simplest. We assume as obvious that:
	
	Then, remembering that $\vec{B}=\vec{\nabla}\times\vec{A}$:
	
	Hence (by choosing the Minkowski metric with signature $+, -, -, -$):
	
	Which give us so far:
	
	\begin{tcolorbox}[title=Remark,colframe=black,arc=10pt]
	Strictly speaking not to confuse the Faraday's tenso $F^{ik}=\partial^iA^k-\partial^kA^i$ with its matrix form, we should put the first term of the above equality between brackets $[F^{ik}]$ as we have already mention it in the section of Tensor Calculus!
	\end{tcolorbox}
	Now, being known that $x_0=ct$ and $A_0=\phi/c$ the other components of the tensor $F^{ki}$ are written taking into account that:
	
	and therefore:
	
	thus, with the partial contravariant  derivatives according to the Minkowski metric:
	
	Thus we have for the tensor of the electromagnetic field in contravariant components with and always with the Minkowski signature $+, -, -, -$:
	
	So that the equation of motion is finally:
	
	But as we will see it in the section of Special Relativity, the real tensor of the electromagnetic field is defined by (still in the metric $+, -, -, -$):	
	
	so that the Lorentz transformations are conform.

	The expression in tensor form of the electromagnetic field clearly shows the unity of the electromagnetic field while generally the electric and magnetic fields are considered separately in classical theory.

	But as in theoretical physics we often work in natural units (this is somewhat the "norm" ...), then we have in advanced graduate books:
	
	Hence the equation of motion in natural units:
	
	By denoting now the components of $1$ to $4$ instead of $0$ to $3$ (it is easier for students to find their way in the matrix) and without forgetting that the partial derivatives are covariant and adopting again the natural units such as $\mu=\varepsilon=01$ (verbatim $c=1$), the two Maxwell equations with source are written:
	
	Using the tensor of the electromagnetic field, it remarkably appears that these two equations can be written as in the condensed tensor equation:
	
	where $j^\nu$ is the "\NewTerm{four-vector current}\index{four-vector current}" defined by (in natural units!):
	
	Using the first definition of the Faraday's tensor (the one where the field components are divided by $c$) and taking as known (we will prove it later) that $c=1/\sqrt{\varepsilon_0\mu_0}$ we have in the SI system:
	
	with:
	
	As we shall immediately see, the temporal part of this equation gives the divergence of the electric field and the spatial part the rotational magnetic field.
	
	Indeed:
	
	This is latter equality is the equivalent of:
	
	And we also have:
	
	These three sets of equalities are the equivalent of:
	
	Also the both Maxwell equations:
	
	can be written in tensor form:
	
	Indeed we have first:
	
	that corresponds obviously to:
	
	and:
	
	These three sets of equalities are the equivalent of:
	
	Finally, all the Maxwell equations, adopting the natural units, are implicity contained into the both relations:
	
	We can also use a the pseudo antisymmetric tensor of rank $4$ that can be seen as a generalization of the Levi-Civita tensor (\SeeChapter{see section Tensor Calculus}) such that we can write:
	
	with:
	
	The Lagrangian we have determined earlier above, however, is not complete. Indeed, when we apply the variational principle, we have already seen many times in the various section of this book (Classical Mechanics, Wave  Mechanics, Magnetostatic, Special Relativity, General Relativity, etc.) that we could get the equations of motion ( trajectories) of the subjects (bodies) of interest. The equations obtained also contain the parameters that explained the source of this movement (material properties, speed, field, etc.) as it was the case before!

	Previously, we applied the variational principle on the Lagrangian of the charge-field interaction (electrostatic + magnetic) and got the equation of motion corrected by the Lorentz force.

	When we determined the equations of motion of the charged particle from the principle of least action, we fixed the electromagnetic field (the field is known) and we varied the trajectory. The variational principle, must also allow us to get the field equations from the opposite approach: we fix the path of the particle (known of course) and we vary the electromagnetic field (potential and tensor field).

	We should then obtain the Maxwell equations that, in the same way that we get what makes the motion of the particle when fixing the field in the variational principle, should gives us the information on what is the source of electric and magnetic field when the trajectory is fixed in the variational principle (I hope you have followed that explanation........).

	The desire is then great to just simply start again from the action obtained earlier:
	
	and apply to it a variation on the field.

	As we have:
	
	we can write:
	
	Let us consider electrical charges moving at the speed $v$ and let us write the following quantity (do not forget that we continue to work in natural units such as $\mathrm{d}x_0=c\mathrm{d}t=\mathrm{d}t$!):
	
	with in natural units: equation
	
	Therefore we have:
	
	If we apply the variational principle only on the field (constant in amplitude so that the source of the field is constant and therefore $\delta j_i=0$) and that we therefore consider the movement of electric charges known, it is immediate that the first term above is zero. Then we have:
	
	that this integral to be zero it would require that $j_i$ is zero ... which is pretty annoying if we wish to determine the characteristics of a source which then would not exist ... Therefore, we notice that something is missing in our Lagrangian!

	The idea is the following: we know a tensor equation which involves the current density which is $\partial_\mu F^{\nu\mu}=j^\mu$ and that implicitly contains the only two Maxwell equations that give information on the source of the respective electric and magnetic fields (the other two giving properties of the field, not sources!) that is (always in natural units) for recall:
	
	It is therefore sufficient to get these two equations (thus the corresponding tensor equation) following the variational principle to get the properties of the source of the field.
	
	This simply means that ideally, we should (and expect) to have:
	
	where the integral equation vanishes exactly when $\partial_mu F^{\nu\mu}=j^\nu$!

	It is then tempting to write something of the form (notice that we lowered the index of the potential $A$ and raised that of the current density $j$ in the second integral which does not change anything mathematically speaking to the result):
	
	We can use the following property of the Lagrangian quantities for to determine the missing expression "???": they all are invariant. In other words and for recall their pseudo-norm (scalar) is equal by Galilean basis change (\SeeChapter{see section Special Relativity}) such that:
	
	The first relationship is obvious, we have already proved it many times. The second is perhaps less obvious therefore let us give a small indication (not general) to check whether it's correct: $A_\nu j^\nu$ is the dot product of $j$ and $A$. If we apply the same (four) rotation to the two vectors, since Lorentz transformations are rotations (\SeeChapter{see section Special Relativity}), the angle between $j$ and $A$ remains unchanged and thus the dot product.

	But we must not forget that need to find the quantity "$???$" as a scalar invariant involving the Faraday tensor in one way or another.

	We can then try directly with the following quantity (knowing in advance thanks to our precursors that this is the right hypothesis):
	
	involving the covariant Faraday's tensor $F_{\mu\nu}$ and contravariant $F^{\mu \nu}$ because we know that:
	\begin{enumerate}
		\item It is a scalar invariant! Indeed, let us write $F_{\mu\nu}F^{\mu\nu}$ in terms of electric and magnetic fields to understand the physical meaning of that latter (still in natural units):
		
		\begin{tcolorbox}[title=Remark,colframe=black,arc=10pt]
		If we were not in natural units, the calculation result would be of the form:
		
		The quantity:
		
		(or $||\vec{B}||^2-||\vec{E}||^2$ in natural units) is therefore an invariant of the field.
		\end{tcolorbox}
		\begin{tcolorbox}[colframe=black,colback=white,sharp corners]
		\textbf{{\Large \ding{45}}Example:}\\\\
		In a repository O, let us consider a plane electromagnetic wave. The modules of the electric field and magnetic field are linked by the relation $E=cB$ (see below for proof). The invariant of the considered field is therefore zero. In another reference frame, with the same structure of the field, then we will also have $E'=cB'$.
		\end{tcolorbox}
		
		\item Because a variational on this term gives:
		
		where we guess ... by digging a little bit, that $\delta F_{\mu\nu}$ implicitly contains the term $\delta A^\nu$. We also see that a factor $2$ appears such that we will have to introduce a normalization constant $\alpha$,  that anyway has to be introduced for a matter of homogeneity of the units of the expression of the action.
	\end{enumerate}
	
	So eventually let try with something like:
	
	Now to find the equations of the electromagnetic field, we consider that the movement of electric charges are known and we use the principle of least action by varying only the components of the vector potential and those of the tensor of the electromagnetic field.

	It follows that the variation of the first integral is zero and that it remains:
	
	But we know that $F_{\mu\nu}$ is equal to $-F_{\nu\mu}$ since the Faraday tensor is antisymmetric:
	
	Nothing prevents us to swap the indices $\mu$,$\nu$ in the first terme at the right of the equality:
	
	But we know that $F_{\mu\nu}$ is equal to $-F_{\nu\mu}$ since the Faraday tensor is antisymmetric:
	
	Nothing prevents us to swap the indices $\mu$,$\nu$ in the first terme at the right of the equality:
	
	So finally:
	
	Let us look at the second integral:
	
	Applying the Fubini theorem (\SeeChapter{see section Differential and Integral Calculus}) that says that we can integrate in any order the integration variables (under certain conditions) then we can apply the integration by parts (\SeeChapter{see section Differential and Integral Calculus}) to write:
	
	where $\mathrm{d}S$ represents the boundary surface of the hyper-volume $\mathrm{d}\Omega$ which we were initially integrating and that omits the variable considered by the choice of the superscript $\nu$.

	Now according to the superscript $\nu$ concerned, the boundaries of the first term of the equality:
	
	will be on the components of time or components of space. If we focus on the temporal boundaries of integration, these are the initial and final moments of the action at which we apply this variational.
	
	But at the time ends, the variational of the vector potential $\delta A_\mu$ is zero (by definition), therefore the integral over the time component will be zero.

	Now on the spatial components, the (spatial) boundaries are those that permits to integrate the border surface of the hyper-volume at the final time. If this one is taken as being the infinity, the radius of the border area will be infinite and at every point of this surface, the energy carried by the field and the amplitude of the field components will be zero (see proof further below).

	Therefore the variational of action is finally written:
	\\
	The variations of the vector potential being arbitrary, the preceding integral will be zero if the integrand is also zero, hence the relations:
	
	which brings us to:
	
	\begin{tcolorbox}[title=Remark,colframe=black,arc=10pt]
	We return on this Lagrangian with another (very interesting) approach in the section of Quantum Field Theory.
	\end{tcolorbox}
	
	\pagebreak
	\subsection{Electromagnetic wave equation}
	Maxwell supposed that the electromagnetic wave was a combination of phenomena that explain the the third and fourth equations. If an electromagnetic wave is far from its source, then we can then neglect the surface density of current from the source as having no influence on the wave (we say then that it is the Maxwell equations without sources which we have already mentioned earlier above). So the third and fourth Maxwell equations are written:
	
	The excitation magnetic field $\vec{H}$ and electric field  $\vec{E}$ being perpendicular, let us place ourselves conveniently in a system of orthonormal and Euclidean axes $(\vec{i},\vec{j},\vec{k})$ belonging to $\mathbb{R}^3$ by choosing:
	
	\begin{tcolorbox}[title=Remark,colframe=black,arc=10pt]
	Warning! The reader must remember that in what follows, $H$ is the $z$ component of $\vec{H}$ and $E$ the $y$ component of $\vec{E}$.
	\end{tcolorbox}
	The (simple) calculations of $\vec{\nabla}\times\vec{E}$ and $¨\vec{\nabla}\times\vec{H}$ gives, after simplification:
	
	Before going further, a reader asked us to develop the details that lead to the left equality. So we start from:
	
	But:
	
	because the wave is a plane wave and the component of the electric field being in $y$, it does not vary following $z$. Then we have:
	
	This being done, if we continue, we have:
	
	Identifying similar terms, we get the "\NewTerm{equation of propagation of the electric field}\index{equation of propagation of the electric field}":
	
	and proceeding exactly in a smilar way:
	
	relationships that are both in the form of a wave equation (\SeeChapter{see section Wave Mechanics}) of the type (for recall) of a "Poisson equation" (specifically it is a "d'Alembert's equation"):
	
	where we have:
	
	The propagation speed of an electromagnetic wave in vacuum is:
	
	the units and numerical values agree...
	\begin{center}
		  \begin{tikzpicture}[scale=2,x={(-10:1cm)},y={(90:1cm)},z={(210:1cm)}]
		    % Axes
		    \draw (-1,0,0) node[above] {$x$} -- (5,0,0);
		    \draw (0,0,0) -- (0,2,0) node[above] {$y$};
		    \draw (0,0,0) -- (0,0,2) node[left] {$z$};
		    % Propagation
		    \draw[->,ultra thick] (5,0,0) -- node[above] {$c$} (6,0,0);
		    % Waves
		    \draw[thick] plot[domain=0:4.5,samples=200] (\x,{cos(deg(pi*\x))},0);
		    \draw[gray,thick] plot[domain=0:4.5,samples=200] (\x,0,{cos(deg(pi*\x))});
		    % Arrows
		    \foreach \x in {0.1,0.3,...,4.4} {
		      \draw[->,help lines] (\x,0,0) -- (\x,{cos(deg(pi*\x))},0);
		      \draw[->,help lines] (\x,0,0) -- (\x,0,{cos(deg(pi*\x))});
		    }
		    % Labels
		    \node[above right] at (0,1,0) {$\vec{E}$};
		    \node[below] at (0,0,1) {$\vec{B}$};
		  \end{tikzpicture}
		
		  \begin{minipage}{.5\linewidth}
		    \[
		      c = \frac{E}{B}
		    \]
		    \begin{tabular}{r@{${}={}$}p{.8\linewidth}}
		      $\vec{E}$ & electric field amplitude \\
		      $\vec{B}$ & magnetic field amplitude (instantaneous values) \\
		      $c$ & speed of light ($3\times10^8\mathrm{m/s}$) \\
		    \end{tabular}
		  \end{minipage}%
		  \begin{minipage}{.5\linewidth}
		    \[
		      c = \frac{1}{\sqrt{\mu_0 \varepsilon_0}}
		    \]
		    \begin{tabular}{r@{${}={}$}p{.8\linewidth}}
		      $\mu_0$ & magnetic permeability in a vacuum, $\mu_0 = 1.3\times10^{-6}\,\mathrm{N/A^2}$ \\
		      $\varepsilon_0$ & electric permeability in a vacuum, $\varepsilon_0 = 8.9\times10^{-12}\,\mathrm{C^2/N m^2}$ \\
		    \end{tabular}
		  \end{minipage}
	\end{center}

	The propagation velocity of the electromagnetic wave in the a material is then given by:
	
	We have $v<c$ because experience has show so far that we can not exceed the speed of light, which is one of the postulates of Special and General Relativity.

	So we can finally write:
	
	Therefore using the d'Alembertian in one dimension:
	
	As we failed to get a direct expression of $E(x, t)$ and $B (x, t)$, we have just obtained differential equations containing only one of these fields. We name these equations respectively "\NewTerm{wave equation for the electric field}\index{wave equation for the electric field}" and "\NewTerm{wave equation for the magnetic field of induction}\index{wave equation for the magnetic field of induction}".

	The both equations have the same form and admit a solution of the same type. An obvious and particular solution (we leave it to the reader to make this check) of these differential equations is the sine trigonometric function:
	
	by not forgetting the relation between the pulsation $\omega$, the propagation velocity $c$ and the wave number $k$ that we had proved in the section of Wave Mechanics!

	A more general solution is the sum of the trivial solutions (\SeeChapter{see section Differential and Integral Calculus}):
	
	But we have seen in our study of phasors (\SeeChapter{see section Wave Mechanics}) that this real solution is only a particular case of a more general solution that is in the imaginary number field. So finally we can write:
	
	Which constitutes the monochromatic plane wave which is the simplest type of wave to manipulate in physics.

	In three dimensions, the solution is by extension:
	
	where $\vec{k}$ is named the "\NewTerm{propagation vector}\index{propagation vector}".
	\begin{tcolorbox}[title=Remark,colframe=black,arc=10pt]
	The monochromatic wave can has no real signification as physical reality. Indeed, if we calculate the electrical energy associated with in all space, we obtain for it an infinite energy (as it has neither beginning nor end!) Which is not realistic. We will see in the section of Wave Quantum Physics that in fact the light is enclosed in a quantum of energy.
	\end{tcolorbox}
	But the wave equation is linear (solution is always the sum of other solutions). This implies that a superposition of waves of different frequencies (wave number and pulsation also!) is also a solution. Thus, by varying the wave vector $\vec{k}$ (and implicitly via its norm $||\vec{k}||$, pulsation $\omega$, frequency $f$ and period $T$) we also scan all the possible directions of propagation.

	Written mathematically this gives, for the electric field:
	
	And nothing prevents us from extracting a coefficient of the initial amplitude of the field such that:
	
	and we find here a relation very similar to that of an inverse Fourier transform (\SeeChapter{see section Sequences and Series}) which is a remarkable fact! So the trick is now to put $t=0$ because the previous relation is not just a simple analogy with the Fourier transform, it is a Fourier transform!!!!

	We can therefore relate the real field $\vec{E}(\vec{r},0)$ to the field $\vec{E}_0(\vec{k})$:
	
	These two relations are often condensed as following:
	
	The real field is thus at the initial instant the inverse Fourier transform of the field $\vec{E}_0(\vec{k})$. The term $\vec{E}_0(\vec{k})$ therefore represents the spectral component related to the particular wave vector $\vec{k}$ of the real field. This general solution of the wave equation is named "\NewTerm{wave packet}\index{wave packet}"
	\begin{tcolorbox}[title=Remarks,colframe=black,arc=10pt]
	\textbf{R1.} Identically to our study of Wave Mechanics the coefficients $\omega$ (pulsation) and $\vec{k}$ (wave number) coefficients are required to express the variation of the sine by radians and to give it a direction and a pulsation (\SeeChapter{see section Wave Mechanics}).\\

	\textbf{R2.} The periodicity of the sine function requires (\SeeChapter{see section Trigonometry}):
	
	hence the definition of the period of a wave:
	\\

	\textbf{R3.} The periodicity in space makes it possible to define identically the wavelength of the function as:
	
	We thus observe that the plane wave moves along $x$ by traveling a distance $\lambda$ in a time $T$. The velocity of the electromagnetic wave is then:
	
	\end{tcolorbox}
	By introducing:
	
	Into the one dimensional version of:
	
	we get:
	
	to finally obtain the famous result for the plane electromagnetic wave:
	
	
	\subsubsection{Helmholtz equation}
	Now let us examine in detail another solution of the form:
	
	where this time we make explicit mention of the coordinates in order to avoid any confusion.
	\begin{tcolorbox}[title=Remarks,colframe=black,arc=10pt]
	The particular solution with the cosine is more appreciated by many teachers than the one with the sine, as it allows, as we will see, a condensed writing using the phasors (\SeeChapter{see section Wave Mechanics}).
	\end{tcolorbox}
	If we use the concept of phasor, we can rewrite this solution in the form:
	
	Therefore:
	
	into the wave equation:
	
	we get:
	
	which is nothing else that one the "\NewTerm{one-dimensional Helmholtz equation for electrodynamics}\index{Helmholtz equation for electrodynamics}". It is just a special way to write the conventional Helmholtz that is traditional in this field of physics.

	More generally in undergraduate courses it is defined using the prior previous wave equation and rearranging it as following:
	
	and using the property established earlier above:
	
	But we can write it for the three dimension case also:
	
	And using the laplacian operator (\SeeChapter{see section Vector Calculus}):
	
	Factorized:
	
	And as it is also valid for the magnetif field we can generalize the notation a last time:
	
	to get the "\NewTerm{homogeneous wave equation}\index{homogeneous wave equation}" (as the second term is zero).

	This done, separation of variables (\SeeChapter{see section Differential and Integral Calculus} begins by assuming that the wave function $f(\vec{r}, t)$ is in fact separable:
	
	Substituting this form into the homogeneous wave equation, and then simplifying, we get immediately the following equation:
	
	Notice the expression on the left-hand side depends only on $\vec{r}$, whereas the right-hand expression depends only on $t$. As a result, this equation is valid in the general case if and only if both sides of the equation are equal to a constant value that we will denote by $C$. From this observation, we obtain two equations, one for $A(\vec{r})$, the other for $T(t)$:
	
	Rearranging the first equation, we get the famous "\NewTerm{general homogeneous Helmholtz equation}:
	
	and this is the common general "\NewTerm{Helmholtz equation}\index{Helmholtz equation}" partial differential equation.
	\begin{tcolorbox}[title=Remarks,colframe=black,arc=10pt]
	If in the Helmholtz equation we put $C=0$ we get:
	
	More commonly denoted:
	
	and named the "\NewTerm{Laplace's equation}index{Laplace's equation}". It is therefore a special case of the Poisson's equation but where the second member is equal to zero.
	\end{tcolorbox}
	
	
	\subsubsection{Energy flow transportation (Poynting vector)}
	It is relatively intuitive that any electromagnetic wave must carry energy. Let us express the value of this energy in a classical (non-probabilistic and non-quantic) point of view.

	Since the propagation direction of an electromagnetic wave is that of the vector $\vec{E}\times\vec{B}$ as proven earlier above, we can define the "\NewTerm{Poynting vector}\index{Poynting vector}" as:
	
	named sometimes the "\NewTerm{Abraham form}\index{Abraham form}" andhose value is expressed indeed in joules per second and per unit area (radiated power by surface unit) as $[\text{J}\cdot\text{s}^{-1}\cdot\text{m}^{-2}]$.

	The norm of the Poynting vector represents therefor the instantaneous power which is transported by the electromagnetic wave through a unitary perpendicular surface (we insist on the "perpendicular") to its direction of propagation. Therefore, we can also write the Poynting vector in the following form (be careful not to confuse the energy and the electric field which are represented by the same letter):
	
	where $\vec{n}$ is as usual the unit vector perpendicular to $\mathrm{d}S$ (this last relation will be useful to study a small property of the synchrotron radiation further below).

	For a plane electromagnetic wave, the norm of the Poynting vector is:
	
	This quantities varies according to time and place. At a given location, its average value is the mean value of the $sin^2(\ldots)$ for a period $T$:
	
	Let us recall that (\SeeChapter{see section Differential and Integral Calculus}):
	
	Therefore $\forall x$ we have:
	
	The average value of the Poynting vector of a plane electromagnetic wave is a constant ... which does not depend on position and time.
	
	\begin{tcolorbox}[title=Remark,colframe=black,arc=10pt]
	We can make a daring and fun analogy with electrokinetics by doing a dimensional analysis of the above product $\mu_0c$. We have:
	
	\end{tcolorbox}
	... to demonstrate the energy contained in a unit of volume pragmatic physicists would do a dimensional analysis. Let us avoid this and focus ourselves with the special case of the plane wave!

	For this purpose we base ourselves on the electrical energy of an ideal plane capacitor (\SeeChapter{see section Electrostatics}) producing plane electromagnetic waves with a yield of $100\%$:
	
	Therefore:
	
	and that is denoted in most textbooks by:
	
	from which we get:
	
	And the total energy transported by the electromagnetic wave in this particular case is thus:
	
	Therefore, the electrical energy density of an electromagnetic wave is equal to its magnetic energy density.

	From this result, we are led to define the "\NewTerm{mean intensity $I$ of an electromagnetic wave}" by the average value of its Pointing vector:
	
	It is therefore the average power that carries the wave per unit area. We have already demonstrated the mean expression of the Poynting vector, which leads us to write:
	
	Now, using the relation between energy and linear momentum (\SeeChapter{see section Wave Quantum Physics}):
	
	we get the "\NewTerm{linear momentum density}\index{linear moment density}" of the electromagnetic wave:
	
	But if the direction of $\vec{E}\times\vec{B} $ is perpendicular to the wave front and is therefore confounded with the propagation direction of the wave its module is:
	
	We therefore have for the linear momentum density:
	
	As the linear momentum density must have the direction of propagation, we can write it in vector form:
	
	If an electromagnetic wave has a linear momentum density, it also has a kinetic angular momentum. The angular momentum per unit volume is then:
	
	Thus, an electromagnetic wave carries both linear momentum density and angular momentum density as well as energy density !!!

	This result is not surprising. Electromagnetic interaction between two electrical charges involves an exchange of energy and momentum between loads. This is done through the electromagnetic field which carries a density of energy and momentum density that are exchanged.
		
	\pagebreak
	\subsubsection{Emissions}
	To predict the shape and properties of the radiation emitted by antennas or other sources, computers and numerical models corresponding to the problem to be studied should be rigorously used. Formally, solving Maxwell's equations in macroscopic systems is quite difficult and takes time. Moreover, this is rather the work of the engineer who seeks practical exploitation from fundamental theories. The theoretical physicist is interested in the foundations of the Universe and with isolated and perfect systems.

	However, we would like to expose the theory of diffraction and for this we must make a theoretical trip on an approximation of the properties of the radiation of a spherical point source in the vacuum.

	The wave in the case of a spherical point source propagates spherically in space (we speak then of "\NewTerm{spherical wave}\index{spherical wave}") and the Poynting vector of is therefore obviously radial to the source.
	\begin{figure}[H]
		\centering
		\includegraphics[scale=0.47]{img/electromagnetism/spherical_wave_front.jpg}
		\caption{Spherical wave front (source: ?)}
	\end{figure}
	The vector $\vec{E}(r,t)$ and $\vec{B}(r,t)$ and vectors are locally contained in the plane tangent to the sphere of radius $r$ (logical!) as shown in the figure below:
	\begin{figure}[H]
		\centering
		\includegraphics{img/electromagnetism/spherical_wave_poynting_vector.jpg}
		\caption[]{Representation of the propagation with respect to the plane tangent to the sphere}
	\end{figure}
	For the flow energy to be constant, the intensity of the wave must decrease with distance. Indeed, the conservation of the energy imposes that through a sphere of radius $r_1$ the energy radiated per unit of time (written with an right "E" so as not to be confused with the notation of the electric field) is equal to that which traverses the sphere of radius $r_2$:
	
	This naturally implies:
	
	But using the relation proved earlier above:
	
	and using the property of perpendicularity of the electric and magnetic field for a plane wave:
	
	Therefore:
	
	which implies:
	
	that is to say:
	
	and therefore the magnitude of the electric or magnetic of a electromagnetic plane wave field decrease inversely proportional to the distance $r$:
	
	Thus the intensity density $I$ of a spherical electromagnetic wave propagating in the vacuum decreases in $r^2$ since:
	
	Therefore:
	
	By extension (important information for mobile phones and radio communications), in view of the results demonstrated above, the energy transported thus decreases in also in $1/r^2$.
	
	It is now easy to understand now why physicists systematically use the frequency to characterize a electromagnetic wave because its amplitude is not constant in the vacuum whereas the frequency is a kind of signature of the transmitter that is not lost through a static empty space !!!
	
	\pagebreak
\	\subsection{Synchrotron radiation (bremsstrahlung)}
	Broadly speaking, "\NewTerm{Bremsstrahlung}\index{Bremsstrahlung}" or "\NewTerm{braking radiation}\index{braking radiation}" is any radiation produced due to the deceleration (negative acceleration) of a charged particle, which includes synchrotron radiation, cyclotron radiation, and the emission of electrons and positrons during beta decay. However, the term is frequently used in the more narrow sense of radiation from electrons (from whatever source) slowing in matter.
	
	A synchrotron light source is a source of electromagnetic radiation (EM) usually produced by a storage ring or more generally any electric charge particle in acceleration. First observed in synchrotrons, synchrotron light is now produced by storage rings and other specialized particle accelerators, typically accelerating electrons. Once the high-energy electron beam has been generated, it is directed into auxiliary components such as bending magnets and insertion devices (ondulators or wigglers) in storage rings and free electron lasers. These supply the strong magnetic fields perpendicular to the beam which are needed to convert high energy electrons into photons.
	\begin{figure}[H]
		\centering
		\includegraphics[scale=0.25]{img/electromagnetism/synchrotron.jpg}
		\caption{European Synchrotron (source: ESRF)}
	\end{figure}
	The major applications of synchrotron light are in condensed matter physics, materials science, biology and medicine. A large fraction of experiments using synchrotron light involve probing the structure of matter from the sub-nanometer level of electronic structure to the micrometer and millimeter level important in medical imaging. An example of a practical industrial application is the manufacturing of microstructures by the LIGA process.
	\begin{figure}[H]
		\centering
		\includegraphics[scale=1]{img/electromagnetism/synchrotron_source.jpg}
		\caption{Synchrotron source produces electromagnetic radiation, as evident from the visible glow (source: United States Department of Energy)}
	\end{figure}
	It is also an important subject of study because the Rutherford  Model (\SeeChapter{see section Corpuscular Quantum Physics}) was a not able to explain the fact that there was not Bremsstrahlung of the electron orbiting around the nucleus as at it's time the Bremsstrahlung was a well known effect!
	
	To begin, let us consider an electric charge in uniform rectilinear motion. The electrical and magnetic fields of such an electric charve have been studied in the previous sections. We have also shown above that the magnetic field is in this configuration, always perpendicular to the electric field. The first consequence is that the electric field is radial and the magnetic field transverse.

	So if we surround the moving particle of an imaginary closed spherical surface, we then have trivially (see the definition of the Poynting vector):
	
	since at any point on the surface, $\vec{E}$ is perpendicular to it and $\vec{B}$ tangent to it, hence $\vec{E}\times\vec{B}$ is also tangent to the surface and therefore the angle between $\vec{E}\times\vec{B}$ and $\vec{n}$ is equal to a right angle therefore the scalar product is zero.

	Thus, in conclusion, the total flux of radiated energy is zero for an electric charge in uniform rectilinear motion. In other words, a uniformly rectilinear moving electric charge does not radiate electromagnetic energy but carries with it the energy of the electromagnetic field (here we are reassured!). This is confirmed by the experimental observations.

	However, the situation is very different for an accelerated moving electric charge. The electric field of an accelerated charge is no longer radial and no longer possesses the symmetry with respect to the charge which it possesses when the motion is uniform (as will be prove it). Consequence: ... an accelerated electric charge radiates electromagnetic energy and therefore sees its kinetic energy that decrease!

	An important conclusion is that in order to maintain an electric charge in accelerated motion, it is necessary to provide energy to compensate for that lost by radiation. If the particle instead of being accelerated is decelerated (it is typically what we seek to do in radioprotection) again the particle will emit the same radiation in the same way (we will also prove it). This happens, for example, when an electric charge, such as an electron or a proton, strikes a target at high speed. A substantial fraction of its total energy goes in the form of a radiation.

	The equations that we are going to determine remain valid for any type of relativistic accelerated motion or not (but not from a quantic point of view!). For example, a charged particle moving in a circular orbit is subjected to centripetal acceleration and therefore emits radiation. Consequently, when an ion is accelerated in a cyclic accelerator such as a cyclotron, betatron or synchrotron, a fraction of the energy supplied to it is lost as electromagnetic radiation. Cyclic accelerators than in linear accelerators.
	
	When the electric charges reach very high energies, as it happens in synchrotrons where the acceleration is great (fortunately for us because it will allow us to make a very useful approximation ...), the losses due to radiation, named "synchrotron radiation" as we already know, become important and constitute a serious limitation in the construction of cyclic accelerators of very high energy but nevertheless remain infinitely useful to the advanced industry and scientific medicine and archeology.

	Another important consideration relates to the atomic structure. According to the atomic model of Rutherford (\SeeChapter{see section Corpuscular Quantum Physics}), we imagine the atom as formed of a positively charged central nucleus, the negatively charged electrons describing closed orbits around it. But this implies that the electrons move with accelerated motion and, if we apply the ideas developed so far, all atoms should continuously radiate energy (even in the absence of an external source of energy like the Sun). As a result of this loss of energy, electronic orbits should contract, resulting in a corresponding reduction in the size of all bodies. Fortunately for us, this is not observable (matter does not collapse on itself), but this leads us to suppose, within the framework of the Rutherford model, that the movements of electrons in atoms are governed by certain principles which were not considered at its time. This is what will lead us to create the Bohr model of the atom (\SeeChapter{see section Corpuscular Quantum Physics}) but which, as we shall see, will also have other defects.

	To determine the energy emitted by an electric charge in accelerated motion we will have to make use of mathematical tools that are no longer at the same level as those used previously. It is therefore recommended that the reader have a good mathematical background. Moreover, exceptionally we will make use of CAD softwares for certain points of the development.

	Let us consider first the following figure:
	\begin{figure}[H]
		\centering
		\includegraphics[scale=1]{img/electromagnetism/synchrotron_study_configuration.jpg}
		\caption[]{Scenario to consider for the study of synchrotron radiation}
	\end{figure}
	When the charge distribution $\rho(\vec{r},t')$ and the current distribution $\vec{J}(\vec{r},t')$ are at the point $P_2$, the point $M$ receives the electromagnetic wave emitted by the electric charges and the current when they were at the point $P_1$, ie at the instant $t'$ (because of the speed limit $c$ of the propagation of the field in space). The time delay is the propagation time from the point $P_1$ to the point $M$, ie:
	
	Therefore:
	
	Therefore:
		
	The scalar and vectorial potentials associated respectively with the electric field and the magnetic field at the point of vectorial coordinate $\vec{R}$ at time $t$ have, on the basis of the results obtained in the two preceding sections, the following expressions:
	
	where on the other hand we have to prove imediately in detail that the vector potential associated with the magnetic field is expressed as indicated above!
	\begin{tcolorbox}[title=Remark,colframe=black,arc=10pt]
	We will use these two relations of potential in our study of the radiated field because their similar mathematical form will allow us, at least we hope it..., to simplify the developments.
	\end{tcolorbox}
	These two relations are already partially familiar to us, the first one expressing the (delayed) electrical potential has been proved in the section of Electrostatics in the non-relativistic framework (so our calculations may not be correct if we come across a result that depends on the speed! We will see...).

	Concerning the second relation which expresses the delayed potential-vector, we have seen above that:
	
	was always correct to given gradient of an additive function for $\vec{A}$ (due to the properties of the differential vector operators) such that:
	
	and that $\vec{B}$ is in relativistic form or not, we have:
	
	Let us also recall that (\SeeChapter{see section Magnetostatics}) that:
	
	It follows that if we put:
	
	we fall back on the Biot-Savart law since if and only if $\vec{J}$ does not depend on $r$ then (trivial):
	
	We thus get indeed:
	
	Although this form of vector potential gives only the Biot-Savart's law in non-relativistic form, as still always satisfies:
	
	this is still valid in the relativistic framework because this Maxwell equation does not depend on the velocity. Moreover, if our results in the study of synchrotron radiation give us at the end an expression independent of velocity, we shall have once again confirmed this fact.

	\subsubsection{Liénard-Wiechert potentials}
	Liénard–Wiechert potentials describe the classical electromagnetic effect of a moving electric point charge in terms of a vector potential and a scalar potential in the Lorenz gauge. Built directly from Maxwell's equations, these potentials describe the complete, relativistically correct, time-varying electromagnetic field for a point charge in arbitrary motion, but are not corrected for quantum-mechanical effects. Electromagnetic radiation in the form of waves can be obtained from these potentials. These expressions were developed in part by Alfred-Marie Liénard in 1898 and independently by Emil Wiechert in 1900.
	
	To start this study, let us consider the case where a particle of mass $m$ and of electric charge $q$ travel a path $\Gamma$. Compared to an origin point O, its vector coordinate is $\overrightarrow{\text{O}P}=h(t)$, its velocity vector will be denoted:
	
	and its acceleration:
	
	If the point charge $q$ is at the origin O, we saw in the section of Differential and Integral Calculus that the Dirac function gives us:
	
	and if the punctual electric charge $q$ is at an abscissa $x_0$, we have:
	
	What has just been said for a one-dimensional space can also be applied to a three-dimensional space as we saw it and then we write:
	
	If we choose inverse cubic meters ($[\text{m}^{-3}]$) for units for the Dirac function, then we can write:
	
	where $q$ is then the total charge at the point $\vec{r}$.

	For the distribution of the current density, we also have also by choosing the same units as for the Dirac function:
	
	Therefore, at point $M$, the potentials at time $t$ have for expression:
	
	This is a very useful formulation (a workaround) that will enable us to solve our problem.

	For this purpose, when the electric charge is at the point $P_1$ at the time $t'$, we put:
	
	We will use a looooong artifice to solve the integral of the electrical potential (which is therefore an integral multiple in Cartesian coordinates)!

	This begins by multiplying the factor under the integrand $U(\vec{R},t)$ by:
	
	this does not modify the integral since:
	
	and that (\SeeChapter{see section Differential and Integral Calculus}):
	
	We then have the following expression in which the time $t'$ appears:
	
	what we have the right to write because the second integral does not depend explicitly on $t'$.
	
	Good now if we try to solve this integral, we will spend our lives on it... for nothing. We'll have to be more clever!

	Before looking for a solution of this integral, we must first deal with the more general case of the following integral:
	
	Or written in a more condensed form:
	
	which it is easy to approximate with the prior-previous integral:
	
	where we have deal such that $f_1,f_2,f_3,f_4$ depend respectively only (explicitly) on $x$, $y$, $z$ and $t'$ 

	We now want now to make the following change of variables:
	
	Let us recall that in changes of variables in multiple integrals (see the Jacobian in the section of Differential and Integral Calculus), we have, passing from Cartesian coordinates to curvilinear coordinates, the following relations:
	
	where for recall:
	
	and where:
	
	is not an absolute value but the determinant of a matrix!
	
	Now, in our case, let us recall that we have all the $f_i$ which are null and therefore:
	
	and in the case where during the developments one of the $f_i$ would no longer be zero for reasons not yet determined, we would have:
	
	The multiple integral then becomes:
	
	Where the term between braces is taken at $f_i=0$ by necessity of the construction of previous developments preparing the mathematical artifice!

	And let us recall once more (!!) the property of Dirac functions:
	
 	We then have immediately the simplification:
	
	where:
	
	is therefore the Jacobian of the transformation of the artifice ...

	It is evident that by construction of the Jacobian we have:
	
	Therefore it comes:
	
	For the integral $I$ we then have:
	
	Let us now calculate our Jacobian ...:
	
	Returning to the our treated case, $\vec{R}-\vec{h}(t')$ therefore has for components:
	
	Therefore, we have the calculation of the elements of the inverse of the Jacobian:
	
	Well... now that we have the components of the Jacobian matrix, we have only to calculate its determinant. So either we use the general relation of the calculus of determinant proved in the section of Linear Algebra, or we use Maple ... So to win a little bit time let's do it with Maple 4.00b:
	
	\texttt{>with(linalg):\\
	>A:=matrix(4,4,[1,0,0,a,0,1,0,b,0,0,1,c,d,e,f,1]);}
	
	where (for people having difficulty to read Maple notations):
	
	with:
	
	Let us continue with Maple 4.00b:

	\texttt{>Det (A);}
	
	Which gives as output:

	\begin{center}
		\texttt{1 - cf - eb - da = 1 - (fc + eb + da)}
	\end{center}
	The inverse of the Jacobian has the for expression:
	
	Where we used the dot product in the prior-previous equality in order to condense the expression.

	Therefore:
	
	The multiple integral:
	
	where for recall:
	
	or written differently:
	
	But as a result of our change of coordinate system we have for recall:	
	
	And let us recall once again that:
		
	Therefore we have to take $g$ on $f_1=f_2=f_3=f_4=0$! It comes:
	
	Which allows to write:
	
	It is same for:
	
	that can then be written as:
	
	Finally, the resolution of the integral $I$ is written:
	
	Finally, we get the expressions of the potentials.
	\begin{itemize}
		\item The scalar potential is then written:
		

		\item The vector potential is written:
		
		Considering that the integral which is almost the same as for the scalar potential except the term $\vec{v}'$, we arrive by making the same developments as before with to the expression (please... don't ask us for to put also the details for this one... plzzzzz!):
	
	\end{itemize}
	In summary, the potentials taken at the moment (time delay of propagation):
	
	have for expression in a very condensed form ($K_e$ in the Coulomb constant for recall as introduced in the section of Electrostatics!):
	
	These potentials are named "\NewTerm{Liénard-Wiechert retarted potentials}\index{Liénard-Wiechert retarted potentials}" with:
	
	
	\subsubsection{Retarded Electric and Magnetic fields}
	
	
	
	
	
	
	
	
	
	
	
	
	
	
	
	
	
	
	
	
	
	
	
	
	
	
	
	
	
	
	
	
	
	
	
	
	
	
	
	
	
	
	
	
	
	
	
	
	
	
	
	
	
	
	
	
	
	
	
	
	
	
	
	
	
	
	
	
	
	
	
	
	
	
	
	
	
	
	
	
	
	
	
	
	
	
	
	
	
	
	
	
	
	
	
	
	
	
	
	
	
	
	
	
	
	
	
	
	
	
	
	
	
	
	
	
	
	
	
	
	
	
	
	
	
	
	
	
	
	
	
	
	
	
	
	
	
	
	
	
	
	
	
	
	
	
	
	
	
	
	
	
	
	
	
	
	
	
	
	
	
	
	
	
	
	
	
	
	
	
	
	
	
	
	
	
	
	
	
	
	
	
	
	
	
	
	
	
	
	
	
	
	
	
	
	Anyway, that's a lot of words and not much to look at, so how about some soothing animations? 
	
	Here is a Macromedia Flash animation illustrating the effect (need Adobe Reader to be played) made with MATLAB™:
	\begin{figure}[H]
		\includemedia[activate=pageopen,width=\textwidth,height=500pt,
	]{}{swf/Synchrotron_radiation_gamma1.swf}
		\caption{Synchrotron radiation of a circle moving particle with $\gamma=1.2$ (author: Jason Cole)}
	\end{figure}
	The top left represents the Poynting flux $S$, or the radiation which would be observed. The retarded time is plotted beneath, and the two components of the electric field are plotted on the right. You can see how the retarded time is trailing the particle, and spreads out in a circle at the speed of light.

	We can turn up the energy a bit, now $\gamma = 5$:
	\begin{figure}[H]
		\includemedia[activate=pageopen,width=\textwidth,height=500pt,
	]{}{swf/Synchrotron_radiation_gamma5.swf}
		\caption{Synchrotron radiation of a circle moving particle with $\gamma=5$ (author: Jason Cole)}
	\end{figure}
	Here the electron is much closer to the speed of light, and is consequently chasing it's own lightcone – notice how there is a much larger jump in retarded time (from yellow to green). This causes the Poynting flux to be squeezed into a smaller time window, it's much more compressed, intense, and energetic. The temporal shortening of the radiation pulse corresponds to a broad bandwidth, and it is precisely this pulse of radiation that corresponds to broadband synchrotron radiation.
	
	What about something less extreme, an electron traveling at constant velocity in a straight line, again at $\gamma = 1.2$:
	\begin{figure}[H]
		\includemedia[activate=pageopen,width=\textwidth,height=500pt,
	]{}{swf/Synchrotron_radiation_gamma_linear.swf}
		\caption{Synchrotron radiation of a uniform moving particle with $\gamma=1.2$ (author: Jason Cole)}
	\end{figure}
	We now see that the field is being dragged along with the particle, but otherwise not radiating. Because the electron is traveling significantly slower than the speed of light, the field looks relatively undistorted. If you squint you'll notice that the transverse field $E_y$ is stronger than the longitudinal field, which is a manifestation of special relativity (Lorentz transforming a uniform field back into the lab frame where the charge is moving).

	Back to something more interesting, a simple dipole motion at $\gamma = 1.2$:
	\begin{figure}[H]
		\includemedia[activate=pageopen,width=\textwidth,height=500pt,
	]{}{swf/Synchrotron_radiation_dipole.swf}
		\caption{Synchrotron simple dipole motion with $\gamma=1.2$ (author: Jason Cole)}
	\end{figure}
	We see what we hoped for, oscillating radiation perpendicular to the dipole, and zero far-field radiation parallel to the dipole. In the near-field (close to the dipole), there are electric field components everywhere, but it is clear that these must correspond to non-radiative terms and they die off rapidly.

Finally, let’s turn to a use for this radiation: undulators/wigglers. These are machines which purposefully oscillate an electron bunch to force synchrotron radiation emission, where the electron travels with high velocity in one direction and oscillates in the other:
\begin{figure}[H]
		\includemedia[activate=pageopen,width=\textwidth,height=500pt,
	]{}{swf/Synchrotron_radiation_uniform_oscillating.swf}
		\caption{Synchrotron uniform oscillating moving particle (author: Jason Cole)}
	\end{figure}
	
	Notice that:
	
	as:
	
	can be written:
	
	The killer for CERN is that $E^6$ factor because if we double the energy of a particle, it radiates much more times the equivalent received energy. In the case of electrons, this very quickly becomes such a high power that the accelerator can't accelerate any more – the electrons radiate away their energy quicker than it can be pumped in. For protons of the same energy, because of the $m^6$ factor at the denominator the $E^6$ factor is $10^{19}$ times smaller (as $m_p/m_e\cong 1836$), so radiate much less. This is why the LHC is so big, and uses protons rather than electrons. There are plans to build an even bigger LHC, unimaginatively named the "Very Large Hadron Collider", or to accelerate electrons in a straight line so that they radiate much less energy at the proposed International Linear Collider. Both of these propositions are very expensive though, and very large, so there is much interest in smaller accelerators like plasma accelerators.
	
	We can alsom make appear the influence of the radius of the accelerator. Indeed, starting from:
	
	Remember that we have seen in the section of Classical mechanics that for a uniform circular motion:
	
	Therefore it follows:
	
	Hence:
	
	We therefore understand that the bigger is the radius, the small is the radiated power. This is why linear accelerator (considers as infinite curvature radius) are much more interesting than circular accelerator (but more difficult to build as wee need to found and large enough area to build them...).

	\begin{flushright}
	\begin{tabular}{l c}
	\circled{70} & \pbox{20cm}{\score{4}{5} \\ {\tiny 29 votes,  75.86\%}} 
	\end{tabular} 
	\end{flushright}

	%to force start on odd page
	\newpage
	\thispagestyle{empty}
	\mbox{}		
	\section{Electrokinetics}
	\lettrine[lines=4]{\color{BrickRed}T}he development of electrodynamics has enabled a part of the humanity to significantly change its life quality. We know almost all today what we have thanks to it: fridge, radio, TV, computers, scanners,  cars, trams, trains, planes, robots, smartphones, tablets, and other wonderful things and also sometimes less wonderful...
	
	Before you start studying electrokinetic (engineers speak of "electronics" or "electrotechnics"), that is to say the structures involving electric charges in movement, we will define the two fundamental laws (the term "law" is a misnomer, as the first one was proved in the section Electrostatics and the second one in the section of Electrodynamics but anyway...) of the study of electrokinetic and the basic terminology/jargon of electrical circuits or equipment (practical cases being studied in section Electrical Engineering). Even if some elements in the beginning of this section will perhaps not immediately be understood by the reader (especially industrial applications), they will become trivial as the progress of reading of the section.
	
	\textbf{Definitions (\#\mydef):}
	\begin{enumerate}
		\item[D1.] An electrical circuit is constituted by a set of devices named "\NewTerm{dipole}\index{dipole}", interconnected by a conductive wire.
		
		\item[D2.] A "\NewTerm{node}\index{node}" of a circuit is an interconnection where three or more conductor wire are connected together.
		
		\item[D3.] A "\NewTerm{branch}\index{branch}" is a circuit section between two nodes.
		
		\item[D4.] Finally, a "\NewTerm{mesh}\index{mesh}" is a set of branches forming a closed loop.
	\end{enumerate}
	\begin{figure}[H]
		\centering
		\includegraphics[scale=0.9]{img/electromagnetism/circuit_vocabulary.jpg}
		\caption{Circuit basic vocabulary}
	\end{figure}
	\begin{tcolorbox}[title=Remark,colframe=black,arc=10pt]
	It is very important to understand what will follow! Some developments will be reused in the section of Special Relativity, Quantum Field Theory, etc. Furthermore, the reader should also read in parallel the section of Special Relativity to better understand the ins and outs of certain results and the provenance of some mathematical tools.
	\end{tcolorbox}
	The dipole is characterized by the response in a current $I$ by (most of time) a difference of electric potential $U$ between its terminals. That is to say i.e. by the characteristic curve:
	
	Obviously some dipoles reacts on radiations difference, or on rotation difference, on pressure, on humidity and son on...
	
	We shall see that in any conductor, the presence of a resistivity (see below the concept) causes a voltage drop and, strictly speaking, it is the same for the wires (made of conductor material). But these latter being connected in series with other dipoles, we neglect usually in small circuits the resistance of wires relatively to those of the dipoles. Therefore, the wires located between two dipoles of a small circuit are supposed to be equipotentials (the potential is the same on the two terminals).
	
	\subsection{Kirchoff's laws}
	If you connect lots of passive or active elements together in a complicated network, then currents will flow through all the various elements so as to insure that charge is conserved, energy is conserved, and Ohm's Law is satisfied for each resistor.
	
	Simultaneously satisfying all these conditions will give you exactly one solution. The method for writing down equations to represent these conservation laws is named "\NewTerm{Kirchhoff's Laws}\index{Kirchhoff's Laws}" (not to be confused with those of the thermodynamics and optics bearing the same name!) and express the physical properties of the charge and the electric field and are at the number of $2$ (one law for each property).
	
	Briefly:
	\begin{enumerate}
		\item The total current flowing towards a node is equal to the total current flowing from that node
		
		\item In a closed circuit, the algebraic sum of the products of the current and the resistance of each part of the circuit is equal to the resultant electromotive force in the circuit. 
	\end{enumerate}
	
	They will enable us without using the heavy mathematical artillery implicitly hidden behind just to get highly relevant results.
	
	\subsubsection{Mesh law (Kirchhoff's Loop Law)}
	The mesh law (implicitly it is simply the conservation of energy) expresses the fact that when a charge browse a closed circuit (closed path), the energy it loses in by browsing a part of the circuit is equal to the energy it gains in the other. Thus, the algebraic sum of the potentials along a mesh is zero such that:
	
	For this, we must arbitrarily choose a direction of travel of the mesh and agree that the tensions whose arrow points in the direction of travel are counted as positive and others as negative.
	\begin{tcolorbox}[title=Remark,colframe=black,arc=10pt]
	This law simply expresses the fact that the electric field (Coulomb) is a conservative vector field as we have seen in the section of Electrostatic.
	\end{tcolorbox}
	
	\subsubsection{Nodes law (Kirchhoff's Point Law)}
	The nodes law (implicitly it is simply the current conservation law) expresses the conservation of charge, which means that the sum of currents leaving a node (a node can be seen as a separator of field lines - in extenso volumes connected by a same surface) is equal to the sum of currents entering the node. In other words, the algebraic sum of the currents is zero at every node of a circuit such that:
	
	For this, we must choose a sign for incoming currents and the opposite sign for outgoing currents (as we do in Thermodynamics with the mass).
	\begin{tcolorbox}[title=Remark,colframe=black,arc=10pt]
	This law simply expresses the charge conservation equation (or equivalently the law continuity of the charge) that we have proved in the section Electrodynamics.
	\end{tcolorbox}
	\begin{figure}[H]
		\centering
		\includegraphics[scale=0.5]{img/electromagnetism/apply_kirchhoffs_law.jpg}
		\caption[]{Try to apply Kirchoff's law...}
	\end{figure}
	
	\subsection{Drude model}
	The Drude model of electrical conduction will allow us to introduce the basic concepts of the electrokinetic. First, let us define inf what will follow the concept of "\NewTerm{Electric Current}\index{electric current}", "\NewTerm{electric current density}\index{electric current density}", and "\NewTerm{electric resistance}\index{electric resistance}".
	
	An electrical conductor (we are not talking about semiconductors and superconductors at this level of our discussion) can be seen in a very simplified way as a pipe section $\vec{S}$ containing an electron gas of $n$ elementary electric charges $q$ per unit volume.
	
	In the absence of electric field, each electron has a zero vector average speed because it remains in the vicinity of the atom. Under the action of a constant and homogeneous electric field $\vec{E}$ (the case of direct current therefore!), some electrons are moved in a particular direction, until they collide with another atom (traditional aspect... not quantume one obviously!!!) where they take again an average vectorial zero speed drift and so on.
	
	This is the oldest model and the most basic one of the electric current. The bases were laid by Paul Drude in 1902, shortly after the discovery of the electron by Joseph John Thomson (1897). Hence the name "\NewTerm{Drude model}\index{Drude model}".
	
	Insufficient to conceive and develop most of the active electrical components that exists since the late 20th century, the billiard balls model has nevertheless considerable interests:
	
	\begin{itemize}
		\item This is a useful tool to give to our limited mind a picture of phenomena we do not get any direct perception (at least at this day), since they take place in the infinitesimal world of atomes an elementary particles.

		\item The results, for the engineer, of more accurate modern theories, especially such as energy band theory, allow themselves to be formulated using the same concepts as those appearing in the "Billiard Balls" Drude model. Let us quote among them the "density number" and the "electron mobility" concepts (that we will introduce later rigorously further below in our study of energy band theory).

		\item Even if this approach is quite primitive, this model leads to a phenomenological interesting interpretation of the fundamental laws such as Ohm's law or the Joule's law. It binds together certain microscopic phenomena and observable quantities.
	\end{itemize}
	As it's friendly name suggests it, this model treats the electrons as tiny billiard balls. These particles are therefore considered as classical objects, simply governed by Newton's and Maxwell's laws proved in previous sections of this book. This corpuscular conception of the electron is also not totally opposed to the results of quantum physics (study in the next sections of this book), in which a wave packet can always be interpreted as a particle with its mass and speed (see the Ehrenfest's theorem in the section of Wave Quantum Physics).
	
	In a millimeter copper cube, we assume that the number of electrons is so high that it therefore does not matter then treat individually, which would also be irrelevant. It is the average behavior of electrons that should be studied! Two types of interactions that determine behavior are:
	
	\begin{itemize}
		\item The interaction of electrons with the material in which they operate, and to which they belong;
	
		\item The interaction of electrons with the electromagnetic field applied from the outside (all other interaction being neglected).
	\end{itemize}
	The distance $\lambda$ traveled by an electron distance is named "\NewTerm{electron mean free conduction path}\index{electron mean free conduction path}" and if $\tau$ is the time interval between two successive collisions then we have trivially:
	
	whe $v$ is obviously the mean electron velocity of the material.
	
	The collision time is a random variable. All physical parameters maintained as constant, this random variable is stationary, its average value is named "\NewTerm{mean collision time}\index{mean collision time}".
	
	We suppose that:
	
	the mean velocity, is therefore created by the acceleration of the electric field (\SeeChapter{see section Electrostatic}):
	
	We then get the "\NewTerm{mean drift velocity}\index{mean drift velocity}" or simply electrons "\NewTerm{drift velocity}\index{drift velocity}" given by:
	
	This relation is so named because their initial speed is maintained due to the thermal excitation of the external environment and corresponds to the thermal velocity for which we have determined the expression in our study of the Maxwell-Boltzmann distribution in the section of Statistical Mechanics (we will calculate their further below in this section).
	
	We admit, in the context of the billiard balls model, that electrons behave like atoms of an ideal gas. This is a gross approximation but enough satisfying for now!
	
	The mean velocity is assumed to be identical for all the free electrons when a constant homogoneous electric field is applied, stationary, and directed along a single axis. It gives the possibility to define the "\NewTerm{current intensity $I$}\index{current intensity}" of electric current in the conductor.
	
	\textbf{Definition (\#\mydef):} The "\NewTerm{electric current}\index{electric current}" or "\NewTerm{electric intensity}\index{electric intensity}", denoted by $I$,  measures the charge $\mathrm{d}Q=nq$ that passes through the cross section $S$ of a conductor per unit of time $\mathrm{d}t$ and is given according to what has been shown just before by:
	
	A slice of conductor, of volume $\mathrm{d}V=S\mathrm{d}L$ contains therefore the electric charge:
	
	It passes through the section $S$ in a time $\mathrm{d}t$, such as:
	
	The current is therefore written:
	
	\begin{tcolorbox}[title=Remark,colframe=black,arc=10pt]
	Attention to the usage of the latter relation in practice! If we consider a theoretical wire, the potential difference across the wire is zero. So there will be no gain / loss of kinetic energy of the electron and therefore no change in speed. If we now consider a real wire, resistive therefore, the potential difference across its terminals will be low but not zero, and the electron's potential drop will not be won in kinetic energy but dissipated as heat in the wire.
	\end{tcolorbox}
	If $I$ is seen as the flow of a "\NewTerm{current density $J$}\index{current density}" through the surface $S$, then we have:
	
	the current density being assumed constant at each point of the surface.
	
	We have therefore:
	
	and after simplification:
	
	which is therefore the expression of the "current density" in the conductor.

	As we know the expression of the velocity, we can write:
	
	And we define the "\NewTerm{conductivity}\index{conductivity}" by:
	
	where this time $n$ is not the number of electrons, but the number density of electrons! By definition, the "\NewTerm{resistance}\index{resistance}" is the inverse of the conductivity!
	
	We notice that the conductivity contains the product of the number density of electron by their mobility. Therefore it is necessary that at least one of these variables has a high value for a material has a high conductivity.
	
	The mobility is greater in the semiconductors than in metals. This characteristic however is completely masked by the ratio of the volumic numbers of electrons: $n$ is between $1,000,000$ to $100,000,000$ times lower in semiconductors than in metals, which explains the higher conductivity of these.
	
	Following the relation:
	
	proved just earlier above, the conductivity depends on the electric field, through the collision time. Indeed, the more the electric field grows, the more the speed of the electrons increases. The distance between the points of possible shocks remaining the same, the collision time, and hence the conductivity should decrease (and thus the resistance increase!).
	
	However, the independence of the conductivity (and respectively the resistance) with the electric field is an experimental fact established precisely with all usual conductors in normal civilian uses.	
	
	The origin of this contradiction lies in the large difference in the magnitudes of the thermal velocity given by the Maxwell-Boltzmann distribution (\SeeChapter{see section Statistical Mechanics}):
	
	and of the average speed drift seen above:
	
	with the mean free path time that will be obtained using the expression:
	
	
	We have proved in the section of Statistical Mechanics that for an electron at room temperature:
	
	And let us calculate the drift velocity for copper with for this particular metal the following values:
	
	This allows us to get the value:
	
	and therefore:
	
	Taking $E=0.32 \;[\text{Vm}{}^{-1}]$, which is considered as a high value since this field produces a current density of:
	
	we finally have:
	
	Therefore, even in a strong industrial electric field, the drift velocity is negligible compared to the thermal velocity.
	
	As the thermal velocity depends only a little bit of the electric field, it turns out that in practice the electron velocity is independent of the electric field. In other words, establishing a current, even intense, has only a negligible effect on the speed of electrons!
	
	\begin{tcolorbox}[title=Remark,colframe=black,arc=10pt]
	In the vast majority of cases, the conductor sizes are large, compared to the average distance that an electron travel between two consecutive shocks. The behavior of the surface of the conductive electrons then are of secondary importance. This is why the conductive medium is often implicitly considered as infinite. The FET and MOST transistors  are an important exception to this. The current circulates in a layer thin enough that the electron mobility is affected by electron scattering on surfaces defining this layer.
	\end{tcolorbox}
	However, an important point to notice is the calculation of the mean free path of electrons in the classical Drude model. We have indeed:
	
	which is much higher, at least an order of magnitude (factor of $10$), to the interatomic distances. It results from this that successive collisions with atoms of the network is not responsible for Ohm's law (which we will see now) contrary to one of the initial assumptions of the Drude model but that it are the impurities and defects of the material involved in it that generates the collisions! We will see a further below in the theoretical model of energy bands that the mean free path is in fact still much greater!
	
	Warning!!! This relation may suggest that since the mean free path is proportional to the thermal velocity and therefore proportional to the square root of the temperature, that the resistance decreases with temperature. But in fact it is not so! The Drude model is too simplistic because in reality it is the opposite that occurs for conductors (resistance increases with temperature because the time interval $\tau$ between collisions decreases faster than the speed increases). And then there is also an opposite problem ... almost at a temperature equal to the zero Kelvin (and over for some materials) the mean free path shuld be almost zero but superconductors show us that it is not the case! In short, without explicit relation depending on the temperature we are in total darkness!
	
	The only thing we know how to do is to admit that to a given constant factor $\alpha$ (positive or negative), a temperature change requires a relative change in resistance by (first order Taylor developement):
	
	hence:
	
	The we get the relation know in high-schools:
	
	Finally, let us notice that the fourth Maxwell equation (\SeeChapter{see section Electrodynamics}) can then be written by  the results obtained just previously:
	
	which then explicitly makes appear the conductivity coefficient.
	
	\pagebreak
	\subsection{Ohm's law}
	Ohm's law states that the current through a conductor between two points is directly proportional to the voltage across the two points. Introducing the constant of proportionality, the resistance.
	
	From the relation proved just above:
	
	and taking the definition of "\NewTerm{conductivity}\index{conductivity}" by:
	
	We have finally:
	
	which is the "\NewTerm{local Ohm's law}\index{local Ohm's law}". We have already see it in differential form in the section of Statistical Mechanics and we already know therefore that it belongs in fact to the family of diffusion laws!
	\begin{tcolorbox}[title=Remark,colframe=black,arc=10pt]
	Since the conductivity is necessarily a scalar, the vector notation of Ohm's law implies that the electrostatic field lines also indicate the path taken by the electrical charges. Moreover, as the conductivity is a scalar necessarily positive in the traditional model, this implies that the current has the same direction as the electric field.
	\end{tcolorbox}	
	If we multiply the previous equality under scalar left and right by a length $L$ we get:
	
	Then we have:
	
	or:
	
	We define the inverse of conductivity as the "\NewTerm{electric resistance}\index{electric resistance}" defined by:
	
	\begin{tcolorbox}[title=Remark,colframe=black,arc=10pt]
	It is important to notice that the electrical resistance is proportional to the length of the resistive element and inversely proportional to its sectional area. For example in high voltage cables, the resistance is given in ohms per kilometer, which permits then to calculates the power lost by kilometer and therefore the money lost by Joule loss.
	\end{tcolorbox}	
	Therefore, we can write Ohm's local law in its most commonly known form:
	
	whence (beware !!!) the potential $U$ is the potential difference over the length of the resistive element (also named "\NewTerm{resistive dipole}\index{resistive dipole}") as we see it in the previous developments and not the total outside potential!
	\begin{figure}[H]
		\centering
		\includegraphics[scale=0.6]{img/electromagnetism/resistors.jpg}
		\caption{Some resistive dipoles (source: Martin Bircher http://www.e-style.ch)}
	\end{figure}
	As for the capacitors seen in the section Electrostatics, resistors have also color codes that have as far as we know until know the same definitions:
	\begin{figure}[H]
		\centering
		\includegraphics[scale=0.8]{img/electromagnetism/resistors_color_code.jpg}
		\caption{Resistor color codes}
	\end{figure}
	\begin{tcolorbox}[title=Remark,colframe=black,arc=10pt]
	This relation is valid only for ideal conductors under normal conditions of temperature and pressure and for which the Drude model applies. So semiconductors and superconductors are excluded.
	\end{tcolorbox}
	Since $U$ is the potential of the resistive element, then we often do reference in the field of electrical engineering to the "\NewTerm{voltage drop}\index{voltage drop}" (indeed, beyond the resistive element the potential is not the same that at the point above this same resistive element).
	
	For copper cables in typical non-industrial use, there is a very useful American table in practice giving with a relatively good accuracy the resistivity according to the diameter and the maximum permissible current. Here is a sample of this table:
	
	where AWG stands for "\NewTerm{American Wire Gauge}\index{American Wire Gauge}" and corresponds to a small gauge that can be buy easily on Internet to determine the diameter of a cable without having to use a caliper:
	\begin{figure}[H]
		\centering
		\includegraphics{img/electromagnetism/awg_gauge.jpg}
		\caption{AWG Gauge  (source: Wikipedia)}
	\end{figure}
	
	\pagebreak
	\subsubsection{Equivalent Resistance}
	We can now study the entire length of a an electric field line collinear with a constant current $I$ supposed to be constant at any point (this is an approximation obviously...) to obtain total resistance if $n$ resistive elements are put  next to each other linearly:
	\begin{figure}[H]
		\centering
		\includegraphics{img/electromagnetism/resistor_series.jpg}
	\end{figure}
	
	The answer is relatively simple since if we denote by $U_{n-1}$ the potential in the first extremity of the resistive element and $U_n$ that of the other extremity. We then have (the reader will have notice that the use of the mesh law in the following relation logically without even necessarily be aware of it existence):
	
	that is to say, a result similar to that obtained by a single resistance whose value is approximately given by (if the electric current is constant throughout the wire) the "\NewTerm{equivalent resistance of resistors in series}\index{equivalent resistance of resistors in series}":
	
	which is the arithmetic sum of the individual resistances.

	Let us now consider $n$ resistance in parallel all at a given voltage $U$ (by law of mesh!) and alimented by an electric current $I$. The current then splits into $n$ streams such that:
	
	in each of the $n$ mesh. Applying the node law, we have:
	
	that is to say that the set of all the resistance put in parallel is analogous to an "\NewTerm{equivalent resistance of resistors in parallel}\index{equivalent resistance of resistors in parallel}":
	
	\begin{figure}[H]
		\centering
		\includegraphics{img/electromagnetism/resistor_parallel.jpg}
	\end{figure}
	Plugging devices in parallel allows to always have the same voltage across them (neglecting the voltage drops). This is the way that the electrical outlets are installed in a domestic installation!
	
	\subsubsection{Equivalent Capacities}
	Even if this has nothing to do with the Ohm's law we can apply the previous reasoning to capacitor.
	
	Let us recall that we defined in the section Electrostatics, the capacity given by:
	
	Let us consider, as well as resistors, $n$ capacitors of capacity  $C_i$ in series set behind each other:
	\begin{figure}[H]
		\centering
		\includegraphics{img/electromagnetism/capacitors_series.jpg}
	\end{figure}	
	 We put at potential $U_0$ and $U_n$ the two extremities of the chain and we bring the charge $Q$ on the whole system. The potential (voltage) across the total capacitor chain is then written simply:
	
	and therefore corresponds to that of a single capacitor $C_e$ that is the "\NewTerm{equivalent capacitance of capacitors in series}\index{equivalent capacitance of capacitors in series}":
	
	where we also find here a harmonic mean.
	
	Now let us consider $n$ capacitor of  capacity $C_i$ in parallel with the same potential $U$:
	\begin{figure}[H]
		\centering
		\includegraphics{img/electromagnetism/capacitors_parallel.jpg}
	\end{figure}	
	The electric charge of each is then imposed (by the mesh law) by the relation:
	
	The total electrical load is simply:
	
	which corresponds to an "\NewTerm{equivalent capacitance of capacitors in parallel}\index{equivalent capacitance of capacitors in parallel}":
	
	which is the arithmetic sum of the individual capacities.

	Finally to close this subject, let us recall that we have proved in the section of Electrostatics that:
	
	In the case where a capacity is alone in series with an AC sine generator (fairly typical case in the industrial world of the 19th and 20th century), then we have:
	
	And therefore we get:
	
	That we will write in analogy with Ohm's law in the form:
	
	hence:
		
	is named the "\NewTerm{capacitive reactance}\index{capacitive reactance}". We notice that in the continuous case where the pulsation is zero, the capacitive reactance becomes infinite and that we find the known situation where the capacity does not pass current (at least in the ideal case ...).
	
	Therefore, a capacitor connected to a circuit that changes over a given range of frequencies can be said to be "\NewTerm{Frequency dependent}\index{Frequency dependent capacitor}".
	
	\pagebreak
	\subsection{Electromotive Force}
	Given $AB$ a portion of an electric circuit traveled by a constant current $I$ from $A$ to $B$. The existence of this current implies that the potential on $A$ is greater (different) in absolute value than in $B$ (in absolute value) . This potential difference is reflected in the existence of the electrostatic field $\vec{E}$ producing a Coulomb force:
	
	capable of accelerating a charge $q$.

	Then, given:
	
	the power required to give a speed $v$ to a any particle of charge $q$. Knowing that this in the conductor there is $\rho_q$ electric charge per unit volume, the total power $P$ in the circuit $AB$ traveled by a current $I$ is:
	
	That is to say:
	
	where:
	
	This power is therefore the "\NewTerm{electric power}\index{electric power}" available between $A$ and $B$, simply because there flows a current $I$.

	If we consider in this electric circuit $\overline{AB}$ a resistive for which we measure a potential difference:
	
	then the power available inside thereof is given by the "\NewTerm{Joule power}\index{Joule power}"
	
	Thus, among this available power, a certain part is dissipated as heat (Joule effect) in a passive dipole such as a resistance. Obviously it is this power that invoice us our power company and to know the corresponding energy consumed, we simply multiply the power of the device which is used by the functioning duration.
	
	Now that we have the latter relation we can finally introduce the famous Ohm Circle that related the most important DC relations:
	\begin{figure}[H]
		\centering
		\includegraphics[scale=0.5]{img/electromagnetism/ohm_circle.jpg}
	\end{figure}
	However..., something is wrong in our previous developments if we look more closely. Indeed, if we apply the reasoning in a closed circuit, that is to say, if we look at the total power supplied between $A$ and $A$ by the Coulomb force, we get (obviously because the Coulomb electrostatic field is conservative):
	
	that is to say null power ?! Eh yes! This means they can not be a steady state current in a closed loop and when there is a current, then this implies that the Coulomb force is not responsible for the global movement of charge carriers in a conductor!!

	Therefore, the current in a conductor can be understood with the analogy of the river flowing in its riverbed. So that there is a flow, it is necessary that the water flows from a higher region to a lower region (a higher gravitational potential to another smaller one). Thus, the movement of water from a highest point towards a lowest point is indeed due to the simple force of gravity. But if we want to form a closed circuit, then we have to provide energy (by a pump) to bring the water to a greater height and the cycle starts again.
	
	This is exactly what happens in an electric circuit. If we want that a permanent current flows, there must be a force other than the electrostatic force that enables the electric charges to close the path (this is a purely mathematical reasoning)! It is for this reason that we must involve an "artificial" external energy source such as an "\NewTerm{electrical generator}\index{electrical generator}" which is then equivalent to the hydraulic pump for water.
	
	The generator involve then as physical property that only when its circuit is open (the current $I$ then being equal to zero) a "\NewTerm{potential difference}\index{potential difference}" is maintained between its terminals necessarily involving the presence of another force compensating the Coulomb attraction of the conductor. Thus, the total force acting on an electir charge $q$ is written thus:
	
	with being $\vec{E}_S$ the electrostatic field and $\vec{E}_M$ the "\NewTerm{electromotive field}\index{electromotive field}". In equilibrium and in the absence of current, we must have:
	
	This means that the difference of potential across an open generator is then:
	
	We name and denote by:
	
	(somewhat abusively) the proper "\NewTerm{electromotive force EMF}\index{electromotive force}" of the generator.
	
	Since, inside the generator, we have:
	
	at open circuit, this means that a generator is a non-conductive equipotential (or with "non-conservative field").

	At equilibrium, but in the presence of a current $I$ (generator set in a closed circuit), the charge carriers responsible for the current undergo additional force due to collisions occurring inside the conductor. For an ideal generator, these collisions are negligible and we get:
	
	However, for a non-ideal generator, such collisions occur and result in the existence of an internal resistance $r$ (very small for generators that just go out of the factory!). Thus, the true electromotive force is given by:
	
	The internal resistance of the generator introduces a voltage drop proportional to the current supplied, so it outputs a potential lower than that given by it electromotive force.
	
	The latter relation is sometimes noted as follows:
	
	and often with the following writing:
	
	What is measured with a voltmeter is however the generator electromotive force (GEF) since the generators have an internal resistance admit as infinite and therefore involve a current $I$ almost equal to zero:
	\begin{figure}[H]
		\centering
		\includegraphics[scale=0.8]{img/electromagnetism/electromotive_force.jpg}	
		\caption{Electromotive force schematic concept (source: OpenStax)}
	\end{figure}

	Generators vary depending on the energy source used and the method of conversion of the latter into electrical energy (ie, the nature of $\vec{E}_m$). We can then produce electrical energy from a battery (chemical energy), an electrostatic generator (mechanical energy), a dynamo (mechanical energy) a solar cell (radiation energy) or thermocouple (heat energy).
	
	The total power $P$ to be provided in steady state is therefore:
	
	where:
	
	is the total EMF of circuit. The integral being on the whole circuit, the total EMF is the sum of the EMF present along the circuit (if any). If they are located in dipoles, the expression becomes:
	
	where the $e_k$ are the algebraic values of the different EMF:
	\begin{itemize}
		\item $e_k>0$ corresponds to "\NewTerm{generators}\index{electric generators}" devices (production of electrical energy)

		\item $e_k<0$ corresponds to "\NewTerm{receptors}\index{receptors}" devices (consumption of electrical energy)
	\end{itemize}
	We also have for the electric power:
	
	and for the Joule (resistive) Power:
	
	A motor converts electrical energy into mechanical energy and therefore corresponds to a EMF receptor, we also say that it has a "counter-electromotive force" or CEMF.
	
	\subsubsection{Faraday's law of induction}
	Now that we have proved the necessity of the electromotive force, we will be able to prove the origin of the "\NewTerm{Faraday's law}\index{Faraday's law}" and also of the "\NewTerm{Lenz's law}\index{Lenz's law}" that we had used in the section of Electrodynamics to prove the third Maxwell equation. 

	The Determination of Faraday's law will also allow us to define the concept of inductance and study its properties.

	Let us do the same approach as did Faraday and ask ourselves the following question: How do we create an electric current?

	An electric current is a moving set of electric charges in a conductive material. These electric charges are moved thanks to a difference of potential that is maintained by an electromotive force. Thus, a battery by converting the chemical energy during a time $\mathrm{d}t$ provides a power $P$ modifying the kinetic energy of the $\mathrm{d}Q$ charge carriers producing then an electric current $I$.

	Given $P_q$ the power required to communicate a speed $\vec{v}$ to an electric charge particle $q$. Knowing that in a conductor, there is $n$ charge carriers per unit volume, the total power $P_q$ to be provided by the (ideal) generator is then (see above):
	
	We therefore put that the ideal electromotive force of a circuit is:
	
	However, the Coulomb force is unable to produce an electromative force as we have proved earlier. To create a direct (continuous) electric current in a closed circuit, we therefore need an electromotive field which circulation along the circuit is not zero. The Faraday's experiment thus shows that it is the existence of the magnetic field that allows the creation of a current (!!!!). This means that the Lorentz force must be responsible for the creation of an electromotive force, that is to say:
	
	Therefore:
	
	The properties of the vector product (\SeeChapter{see section Vector Calculus}) giving us:
	
	We can then write:
	
	Phenomenologically this can be summarized into the following figure:
	\begin{figure}[H]
		\centering
		\includegraphics[scale=1]{img/electromagnetism/faraday_induction_law.jpg}
	\end{figure}
	\begin{enumerate}
		\item[(a)] When a magnet is moved toward a loop of wire connected to a sensitive ammeter, the ammeter deflects as shown, indicating that a current is induced in the loop. 
		
		\item[(b)] When the magnet is held stationary, there is no induced current in the loop, even when the magnet is inside the loop
		
		\item[(c)] When the magnet is moved away from the loop, the ammeter deflects in the opposite direction, indicating that the induced current is opposite that shown in part (a). Changing the direction of the magnet's motion changes the direction of the current induced by that motion.
	\end{enumerate}
	
	\pagebreak
	\subsection{Skin effect}
	The "\NewTerm{skin effect}\index{skin effect}" or "\NewTerm{pellicular effect}\index{pellicular effect}" (or, more rarely, "\NewTerm{Kelvin effect}\index{Kelvin effect}") is an electromagnetic phenomenon which makes that at high frequency a current tends to circulate only at the surface of the conductors. This phenomenon of electromagnetic origin exists for all the conductors traversed by alternating currents. It causes the current density to decrease as one moves away from the periphery of the conductor. This results in an increase in the resistance of the conductor.

	This effect can be used to lighten the weight of high frequency transmission lines by using tubular conductors, or even pipes, without (too much) current loss. It is also used in the electromagnetic shielding of the coaxial wires by surrounding them with a thin metal case which keeps the currents induced by the high ambient frequencies on the outside of the cable.
	
	What we would now like to determine is the attenuation of the electric field (or an attenuation coefficient) in the material of a solid cylindrical conductive cable of conductivity $\sigma$ and permeability $\varepsilon$:
	\begin{figure}[H]
		\centering
		\includegraphics[scale=1]{img/electromagnetism/skin_effect.jpg}
		\caption{Configuration study for skin effect (source: Introductory Electromagnetics, Z. and B.D. Popovic}
	\end{figure}
	and assuming that the current density vector is parallel to the boundary surface, and that is has a single component, for example, $\vec{J}=J_z\vec{e}_z$, depending on the coordinate $y$ (the distance from the interface) only. We wish to determine the distributiono of current in the conducting half-space.
	
	To do this, we take the fourth Maxwell's equation in the form given previously:
	
	and we assume to work with a conductor which has no capacitive effect (ie no displacement current), contrary to the general case demonstrated in the Electrodynamics section, the latter relation then reduces to:
	
	and if we associate it with Maxwell's third equation (\SeeChapter{see section Electrodynamics}), and assuming a harmonic excitation, we have for recall:
	
	Therefore, from that latter:
	
	So we have so far:	
	
	We assumed the current density vector has only a $z$ component, which depends only on $y$. From the Biot-Savart law and symmetry it therefore follows that there is only an $x$ component of the vector $\vec{B}$ (the reader can request we put the details of the section of Magnetism here again if necessary). According to the expression for the curl in a rectangular coordinates system, the both previous relations become:
	
	where we use ordinary derivatives (not partial) because $j_z$ and $B_x$ depend only on $y$.
	From the both relations above we can eliminate $B_x$ to get an equation in $j_z$:
	
	Let us place ourselves in the important case of a harmonic regime
	
	and let us use temporarily the phasors notation:
	
	We then have:
	
	By injecting this into the previous differential equation and simplifying, we then get:
	
	Thus:
	
	and therefore:
	
	remembering (\SeeChapter{see section Numbers}) that:
	
	it comes:
	
	Therefore:
	
	Thus:
	
	We must reject for physical reasons (conservation of energy) the solution:
	
	Therefore it remains:
	
	that physicists write:
	
	because the units of $\delta$ are meters (parameter that is zero if the electric field is constant) and is assimilated to the "\NewTerm{attenuation coefficient}" that we had set ourselves to determine at the beginning:
	
	For a copper conductor, we have according to Wikipedia the following values:
	

	\pagebreak
	\subsection{Semiconductors}
	The main defect of the Drude model seen above is to consider the electron as a classical particle. A set of such particles is obviously not subject to quantum distributions and therefore to an explicit relationship of temperature.

	Moreover, if we observe our Drude model, it is difficult to say anything about resistivity as a function of temperature.

	In fact, we generally consider four dates at the source of the development of the semiconductor theory:
	\begin{itemize}
		\item In 1833, Michael Faraday reported the conductivity of a material that increases with temperature.

		\item In 1839, Antoine Becquerel discovers that under illumination an electrical tension appears at the junction of certain materials (and liquids). It is the photovoltaic effect, which will give birth much later (around 1950) to the solar cells.

		\item In 1873, Willoughby Smith shows that the conductivity of certain substances increases when illuminated. This is photoconductivity.

		\item Finally, in 1874, Karl Ferdinand Braun discovers the phenomenon of electrical straightening when a metallic tip is deposited on certain conductors, that is to say that the electric current passes in one direction when the electric potential applied to the point is positive but not when it is negative!
	\end{itemize}
	Although these discoveries were totally misunderstood and especially not recognized as the different expressions of the same physical phenomenon (semiconductivity), practical applications were immediate and led to the second industrial revolution, that of microelectronics!
	
	This type of difficulty (among many others...) largely disappears with the model of the free electron in a potential well, as imagined by Arnold Sommerfeld in 1928 (\SeeChapter{see section Wave Quantum Physics}). In this model the electrons, subjected to the Pauli exclusion principle, follow the Fermi-Dirac energy distribution (\SeeChapter{see section Statistical Mechanics}), whereas in the Drude model they followed the Maxwell-Boltzmann energy distribution.

	There are two important results:
	\begin{itemize}
		\item Only a fraction of the electrons is likely to see its energy vary under the effect of an external action (temperature, electric field, etc.)

		\item Even at absolute zero, the kinetic energy of the electrons is not zero.
	\end{itemize}
	The Sommerfeld model provides a basis for the construction of more specific theories and is the basis of the field of "\NewTerm{solid physics}\index{solid physics}" according to some sources. It is therefore not a completed model dealing with a specific problem such as electrical conduction or thermoelectronic emission. This base is the energy distribution of the electrons, obtained by the product of two functions: the density of the states and the Fermi-Dirac distribution.
	
	As the years go by, we will complete the developments that will follow to finally try to have the whole detailed approach. Until then, the reader will have to be patient or to go search the information in other sources on the Internet...

	We will make abstraction of the concepts that are not absolutely necessary for the introduction of the model to present here only the essential which is sufficient for the engineer in his daily work.

	To begin the mathematical part of the semiconductor study, we will consider a crystal subjected to a potential difference. A conduction electron of the crystal will therefore be subjected, on the one hand, to an internal force $\vec{F}_i$ resulting from the crystalline field and on the other hand to a force of external origin $\vec{F}_e$ resulting from the electric field applied to the crystal.
	
	The model assumptions (hypothesis) of the model are:
	\begin{enumerate}
		\item[H1.] There is a high enough potential barrier on the metal surface that prevents electrons from leaving the material.

		\item[H2.] Inside the material, electrons are subjected to constant potential!

		\item[H3.] The electrons are independent (no interactions between them).

		\item[H4.] The electrons obey the laws of Quantum Physics and Classical Mechanics.

		\item[H5.] The electrons obey to the Maxwell's electrodynamics laws.

		\item[H6.] The energy bands form a continuous spectrum of energy levels.
	\end{enumerate}
	The first hypothesis is based on the following observation: electrons traveling in a metal do not cross, at room temperature at least, the surfaces limiting the sample.

	The second hypothesis appears rather brutal. It banish from the model the notion of the "structure of matter". It will be replaced in the model of the energy bands by a periodic potential that accounts for the influence of the positively charged nuclei. This hypothesis reflects the fact that electrons are considered as free in the potential well defined by the sample.

	The potential barrier has a finite width but infinite height, that is to say that the passage of the potential inside the material to the potential outside it is done on some interatomic distances. However, since the dimensions of the sample are in practice always very great with respect to an interatomic distance, the potential barrier can be considered as infinitely abrupt, which simplifies the calculations.
	\begin{tcolorbox}[title=Remark,colframe=black,arc=10pt]
	In order to simplify the calculations, we will assume that the electrons move in only one direction (that of the electric field), thus avoid to work with vectors.
	\end{tcolorbox}
	The equation of the dynamics is then written naturally for this electron:
	
	We then writhe (nothing prohibits us to do so) that the electron in the crystal responds to the stress of the external force $\vec{F_e}$ as a quasi-particle of mass $m_e$ in a vacuum:
	
	It is the study of the latter term which will interest us. For this purpose let us recall that that in the detailed study of the propagation of the free electron in a vacuum, where we neglect the effects of its spin, we have shown that it must be described according to the Schrödinger equation by a wave packet (\SeeChapter{see section Wave Quantum Physics}) centered on a state $k_0$, otherwise its energy would be infinite.

	However, we can ask ourselves ... what leads us to consider it as free...? Well it is experience that shows that when we apply a certain threshold potential, a current begins to appear in the semiconductors.

	We have proved (always in the context of the propagation of the free spin-free particle in the section of Wave Quantum Physics) that the wave packet can then be seen in its mathematical solution as a (free) plane wave moving at phase speed:
	
	which we will write for as following in order to simplify the notations of the next developments:
	
	However, in the crystal lattice, the phase velocity can vary, depending on the location of the electron in the lattice due to the geometrical shape of the potential in the crystal. We must therefore use the instantaneous phase velocity:
	
	Let us recall that we also have in general the total energy given by:
	
	Therefore we get:
	
	The term:
	
	is by no means simple in the case of a crystal (it is even a nightmare ...).

	Obviously for a free particle (\SeeChapter{see section Wave Quantum Physics}), let us recall that it is equal to:
	
	But for a particle in a potential field having a complex geometry, the energy $E$ begins to have an expression dependent of $k$ as a function of the zones which can become very complex (see the examples of the section Wave Quantum Physics). Hence the justification for the use of the derivative.

	The acceleration in the classical sense of this electron is then given by:
	
	We also have (\SeeChapter{see section Classical Mechanics}):
	
	Therefore:
	
	Hence:
	
	The derivative of $\vec{F}$ with respect to $\vec{k}$ in the preceding relation will be canceled because the force is derived from the potential applied to the semiconductor only and not from the wave vector of the electron itself! We then have:
	
	Since here $E$ is only the total energy coming from the externally submitted potential, then the force $\vec{F}$ is the external force $\vec{F_e}$ generated by the application of this same potential. We then have:
	
	and:
	
	It then comes by equalization:
	
	Since the energy of the electron can have a complicated mathematical form according to the practical applications cases presented in the section of Wave Quantum Physics, let us express $E(\vec{k})$ in the form of a limited Taylor  development (\SeeChapter{see section Sequences and Series}) of a function of three variables at the second order by dropping the terms of interactions and by not taking the terms of first degree:
	
	In fact, this rough but still acceptable approximation in many practical cases is due to the fact that experience shows that the energy surfaces as a function of $k$ approximate a parabolic shape in certain semiconductor crystals.
	\begin{figure}[H]
		\centering
		\includegraphics[scale=0.7]{img/electromagnetism/band_conduction_parabolic_approximation.jpg}
		\caption[]{Band structure of Silicium following different plane in the crystal. The blue rectangle shows a region of one of the conduction bands which may be described by a parabola approximately}
	\end{figure}
	In the conductors, the approximation of the preceding relation is taken only at the first term.

 	Another way of seeing it is that for a free electron, for recall, in one dimension, the dispersion curve (\SeeChapter{see section Wave Quantum Physics}):
	
	which is indeed a parabola as a function of $k$. Indeed, if we take our Taylor development in one dimension it remains:
	
	and as we determined before that:
	
	It comes:
	
	If the electron is free, the dispersion curve imposes us to have (without the presence of a potential):
	
	which is then considered as the "\NewTerm{energy of the minimum}" $E_{\min}$. It then remains:
	
	and by taking $k_{x0}=0$ we fall back on the dispersion curve of a free particle (which therefore justifies the fact of having choosed $E(k_0)=0$ for a free electron):
	
	This shows that the approximation is not too false ... and justifies the fact that in some textbooks the previous relation (Taylor series) describes a so-named "\NewTerm{quasi-free particle}\index{quasi-free particle}".

	But let us come back to:
	
	And since the wave packet is centered around $k_0$, let us normalize it as being equal to $0$ (which is equivalent to centering the wave vector values). We then have:
	
	What is interesting with these developments is that we started from a free electron in the form of a wave packet and thanks to the Taylor development we find ourselves with an extremely simple expression of the energy of a quasi-free electron.

	It emerges that for a quasi-free electron, without interactions and without taking into account the effects of spin we have:
	
	We then notice a very sympathetic thing! This is that our quasi-free electron has a wave number which resembles in all points tot particle stuck in a potential well with rectilinear walls (see the proof in the section of Wave Quantum Physics).

	We now wish to calculate the density of states (in extenso of electrons) in the volume given by the corresponding rectangular well, by means of the expression of $k$ (not having directly that of $E$ as being too complex).

	We have proved in the section of Wave Quantum Physics that for the potential well with rectangular barriers that:
	
	if we imposed an integer half wavelength. If we impose an integer wavelength ("\NewTerm{Born-von Karman conditions}\index{Born-von Karman conditions}" so that after a translation of the periodic lattice of the crystal we fall back on the same properties) so that the solution is physically acceptable, then we have:
	
	Which obviously implies two times fewer states.

	By extension, for space, we then have in the three-dimensional case:
	
	with:
	
	and where $n_{x,y,z}=1,2,3,\ldots$.
	
	The result is very similar to that of the one-dimensional infinite rectangular potential well but now we have special boundary conditions in the purpose to have a correspondence with the experiment and three main quantum numbers instead of just one. Moreover, each combination of these three numbers corresponds to a different wave function (state). Moreover, these numbers are independent (no condition imposed).
	\begin{tcolorbox}[colframe=black,colback=white,sharp corners]
	\textbf{{\Large \ding{45}}Example:}\\\\
	Consider a solid metal cube of edge length $2$ [cm]. We want to know the value of the lowest energy level for an electron within the metal and what is the spacing between this level and the next energy level!\\
	
	The lowest energy level corresponds to the quantum numbers $n_x=n_y=n_z=1$. From the above relation, the energy of this level is:
	
	The next-higher energy level is reached by increasing any one of the three quantum numbers by $1$. Hence, there are actually three quantum states with the same energy. Suppose we increase $n_x$ by $1$. Then the energy becomes:
	\end{tcolorbox}
	\begin{tcolorbox}[colframe=black,colback=white,sharp corners]
	
	The energy spacing between the lowest energy state and the next-highest energy state is therefore:
	
	his is a very small energy difference. Compare this value to the average kinetic energy of a particle, as the product $kT$ is in comparison $1,000$ times greater than the previously calculated energy spacing.
	\end{tcolorbox}
	
	We then have the first level where all $n$ are unitary:
	
	If we accept to simplify that the well has edges of equal length (cubic crystal lattice semiconductor), we then have:
	
	Let us represent the space $k$ for such a cubic lattice and for different multiples of $n_x$, $n_y$, $n_z$:
	\begin{figure}[H]
		\centering
		\includegraphics{img/electromagnetism/cubic_crystal_lattice.jpg}
		\caption{$k$-space for a cubic crystal lattice}
	\end{figure}
	Therefore, all quantized states can take only values space from $2\pi/L$ in the space of $k$, which means that by elementary volume there is only one possible wave vector and therefore only one associated state. Indeed, the reader can draw a picture over the figure above if he wants and he will see (!) but do not rely on the big blackheads that are there only to show the ends of the elementary volumes and which do not all correspond to possible states!
	\begin{tcolorbox}[title=Remark,colframe=black,arc=10pt]
	The sphere of radius $k$ containing the levels with one electron is is sometimes referred to as the "\NewTerm{Fermi sphere}\index{Fermi sphere}". The radius value is then denoted $k_F$ and named the "\NewTerm{Fermi wave vector}\index{Fermi wave vector}". The surface of the Fermi sphere, which separates the occupied levels from those which are not occupied as we shall see later, is named the "\NewTerm{Fermi surface}".
	\end{tcolorbox}	
	Thus, in a spherical volume with radius $k$ of the $k$-space. We have a precise number (upper limit) of elementary volumes (states):
	
	where in literature it is customary (tradition) to retain only the form of the second equality in the developments. This relation has been a useful to us for recall in the section of Thermodynamics to determine the Debye-Einstein model of the constant-volume (isochoric) thermal capacity of crystalline solids!

	The density of modes in a volume $V$ will then be given by (relation used in the section of Thermodynamics to express the calorific capacity at constant volume of solids):
	
	In facts, due to the presence of potential bonds between the atomes of the crystal, the sphere is not perfect and has like giant pockmarks/photholes on what is otherwise a smooth surface:
	\begin{figure}[H]
		\centering
		\includegraphics[scale=0.8]{img/electromagnetism/fermi_surface_cupper.jpg}
		\caption{Fermi surface of Cupper}
	\end{figure}
	or for real, here is a (supposed) Fermi surface and electron momentum density of Copper in the reduced zone schema measured with 2D Angular Correlation of Electron Positron Annihilation Radiation method ("supposed" as i was not able to found the original article with the picture):
	\begin{figure}[H]
		\centering
		\includegraphics[scale=0.65]{img/electromagnetism/fermi_surface_cupper_acar.jpg}
		\caption{Fermi surface of Cupper with 2D ACAR method (source: Wikipedia)}
	\end{figure}
	Now considering the spin (yes indeed why not do it...?!) we multiply by $2$ since there are two spin states possible by state for the electron:
	
	(relation that will see again in the section of Nuclear Physics chapter during our study of the liquid drop nucleus model) and by injecting in it:
	
	we then have:
	
	The volume density of (quasi-free) states will be obtained by deriving this last relation by the volume:
	
	And if we want the density of (quasi-)free states (of vibrations) per unit of energy and volume, we will have to derive also with respect to the energy:
	
	which gives:
	
	This result does not depend on the volume, it is unchanged when the latter tends towards infinity! So it is valid for any point of the semiconductor crystal if this one is perfect and without bonds between atoms...
	
	What we also find sometimes in the following (somewhat unfortunate ...) forms in some textbooks:
	
	and there are also those who take into account the spin into account only much more later ... which gives a form identical to that of the previous three relations but divided by a factor $2$.and there are also those who take into account the spin into account only much more later ... which gives a form identical to that of the previous three relations but divided by a factor $2$.
	
	The difference $E-E_0$ as we will see futher below is denoted $E_F$, and for obvious reasons related to Statistical Mechanics that we will see further below named the "Fermi Energy". Therefore notice that we have using the previous results:
	
	After rearranging we get immediately:
	
	Therefore:
	
	Fermi energies for selected materials are listed in the following table:
	
	We can obviously associate the Fermi temperature (\SeeChapter{see section Statistical Mechanics}) with the Fermi Energy as $E=kT$. Therefore:
	
	\begin{tcolorbox}[colframe=black,colback=white,sharp corners]
	\textbf{{\Large \ding{45}}Example:}\\\\
	The Fermi temperature associated to the Fermi energy of silver is:
	
	which is much higher than room temperature and also the typical melting point ($\sim 10^3$ [K]) of a metal.
	\end{tcolorbox}
	
	\subsubsection{Non-degenerated statistic density of negative electric charge carriers}
	In short, however this relation has an issue (one more...)! Indeed, we have seen in the section of Statistical Mechanics in our study of Quantum Statistics that in a system where even the energy spectrum is considered continuous it is impossible not to take into account the degeneration of the different levels of energy. We have then demonstrated that for a population of fermions, at a given energy (or temperature) the percentage of degenerate levels occupied is given by the Fermi-Dirac function:
	
	and that function therefore returns a value between $0$ and $1$.

This function therefore gives for a fixed temperature T the probability that an electron occupies a state of energy $E$.

Hence our relation $D(E)$ overestimates the real density value of (quasi-)free occupied states for a given energy (or temperature). In order to obtain a better approximation, we logically write the volumic density of (quasi-free) states per unit of energy by:
	
	However, in practice, we will try to calculate the volume density of (quasi-free) states in a spectrum (interval) of energy. It then comes with the correction added previously:
	
	Thus:
	
	It then immediately follows that the volumic density of (quasi-free) states at a given temperature (normal temperature conditions for civilian applications) taking into account all possible states (continuous levels) of energy is then given by:
	
	Take $E_0$ as a lower boundary avoids us, as we shall see explicitly a further below, to find ourselves with a negative root ... which would be very a priori troublesome!

	Moreover, we can, without any appreciable error, postpone the limit of the integral to the infinity because $f_{\text{FD}}\rightarrow 0$ when $E$ is large.

	Unfortunately, this (improper) integral is generally not analytically soluble. We will have to use approximations.

	We will start by making the assumption that we are in the classical regime of electron gas. That is, we have:
	
	which implies:
	
	Therefore, we also have the approximation:
	
	In other words, the energy $E$ must be much higher than the chemical potential $\mu$ (assimilated often unfortunately to my knowledge wrongly in the literature on semiconductors to the Fermi level $E_F$). Physicists then note this energy $E_C$ to distinguish it and name it "\NewTerm{minimal energy of the conduction band}" (which corresponds to the minimum energy of a quasi-free electron to satisfy this condition).

	Therefore, we also change the notation for the charge density:
	
	\begin{tcolorbox}[title=Remark,colframe=black,arc=10pt]
	Unfortunately, as mentioned in the previous paragraph (!) in many quality textbooks on semiconductors, the chemical potential $\mu$, which is a purely thermodynamic notion implying a hypothesis of interactions, is replaced by the concept of Fermi energy $E_F$ and however this is not the same thing! The two energies coincide only in the case where the temperature $T$ is equal to zero!\\

	So we must consider the term "\NewTerm{Fermi level}\index{Fermi level}" as being nothing but a synonym for "\NewTerm{chemical potential}\index{chermical potential}" in the context of semiconductors.
	\end{tcolorbox}
	We then have:
	
	where $f_{\text{MB}}$ is the Maxwell-Boltzmann distribution (\SeeChapter{see section Statistical Mechanics}) given for recall by:
	
	and thus corresponds well to a non-quantum behavior (ie a non-degenerate electron gas!) because when:
	
	we have:
	
	and therefore the energy states of are far from being all occupied by electrons (there is therefore no degeneration).

	We are therefore well in a situation where Classical Physics predominates over Quantum Physics. This is why in this approximation (of Maxwell-Boltzmann type) we say that we are dealing with a "\NewTerm{non-degenerate semiconductor}\index{non-degenerate semiconductor}" because the electrons are not all stacked into the lowest available levels.

	To continue, we make a change of variable by putting:
	
	hence:
	
	Therefore it comes:
	
	We do an integration by parts:
	
	We then make a change of variable by putting:
	
	which gives:
	
	We have already calculated this integral in the Statistics section. It comes:
	
	We then have finally:
	
	Where, for recall, $m_e$ is the mass of the quasi-particle (and not the mass of the electron for recall!). So after integration everything happens as if all the electrons were concentrated on the level of energy $E_C$ with a number of available places corresponding to:
	
	The relation $\rho_C(E)$ is traditionally written (and in a somewhat unfortunate way ... because it is not easy to remember that it is a density):
	
	or also:
	
	where we have approximately at ambient temperature the following values of (quasi-)free states for Silicon:
	
	and for the Germanium:
	
	while there is a density of about $4.5\cdot 10^{22}$ atoms by cubic centimeters and about $10^{24}$ electron by cubic centimeter for these two elements.
	\begin{figure}[H]
		\centering
		\includegraphics[scale=0.6]{img/electromagnetism/silicum_vs_germanium.jpg}
		\caption{Silicium vs Germanium band and crystal structure (source: ?)}
	\end{figure}
	This means that there is thus a ratio of: $10^{19}/10^{24}=10^{-5}$ between the total electron density and the number of quasi-free electrons.

	We also notice that this theoretical model does not take into account the electronic structure (atomic number) of the studied material.

	Thus, we see that the variations of quasi-free electron densities as a function of temperature (in the temperature validity range ...) are essentially of increasing or decreasing exponential type.

	From the density of the free electrons (caution! it must be remembered that these are only the quasi-free electrons that are wandering in our mathematical equations so far...) in the semiconductor crystal, we can deduce the Fermi energy level (more rigorously it is the chemical potential!):
	
	hence:
	
	and since $N_{n,T}\ge N_n$ we always have because of the logarithm which is negative, the energy of Fermi (more rigorously it is the chemical potential!) which is less than or equal to the energy of the quasi-free electrons:
	
	or in other words, (quasi-)free electrons have an energy higher than the Fermi energy (chemical potential...), which is in line with the non-degenerate gas approximation made earlier above. This gives a condition of major importance for the negative carriers to be the generators of the conduction in the material.

	Thus, when we place ourselves at a temperature different from absolute zero, the electronic states are not all degenerate: there is spreading of occupied states in the neighborhood of what constitutes by definition the Fermi energy level (see sections of Statistical Mechanics And Quantum Quantum Physics), effect that is as increase as the temperature is high.
	
	\subsubsection{Non-degenerated statistic density of positive electric charge carriers}
	First of all, we must know that in the present state of our knowledge, "\NewTerm{holes}\index{holes}" do not emerge from equations, but are an empirical construction that makes it possible to match theory and experience (positive charges of the Hall effect for example). It is therefore an artifice to make a simple theory of a question that seems rigorously nowadays not manageable by Quantum Physics.

	Personally, I consider the holes in the same way as the Lagrange points in astronomy: Even if there are no bodies at these points of Lagrange this does not avoid a satellite from orbiting around them (possibility that we have not proved in the section of Astronomy) as if there was a mass even if the orbit is quasi-stable! Moreover, experiments would have shown in the early 2000s that Lagrange points appear at the level of the atom under certain ideal and simplified conditions!

	That said, it must be remembered that a hole is not a missing electron! It is an idiocy (in my opinion ...) that we see in some specialized works.

	At the risk of being repeated a little often, let us recall that for a fixed temperature $T$ the probability that an electron occupies a state of energy $E$ is given by:
	
	What bring us in order to get a better approximation, we logically had written the volume density of occupied states per unit of energy:
	
	This finally led to the following relation of the volumic density of negative charge states where the presence of a mass in the relation indicates that the occupied states are by quasi-particles such as:
	
	But what about the probability that an electron does not occupy for a fixed temperature $T$ a state of energy $E$ and trivially given by the difference:
	
	where the $n$ in index is there to indicate that the distribution concerns the "negative" carriers (distribution given as we have rightly proved earlier by the Maxwell-Boltzmann distribution which follows from an approximation of the Fermi-Dirac law).
	
	Well! We shall see that the equations lead us to the possibility to associate also to these unoccupied states a density of states with a given effective mass. We shall also see later that it will be possible to associate to these unoccupied states a positive and equal electric charge to that of the electron, hence the $p$ in index in the preceding relation and signifying "positive".

	We therefore have for these positive carriers:
	
	let us now make an approximation similar to that used for the negative carriers, that is to say:
	
	to impose a semi-classical regime and therefore the states of energy are by far not all occupied by the holes (there is therefore no degeneration).

	This restriction requires:
	
	Either written in the same way as for negative carriers:
	
	Unlike negative carriers, this imposes:
	
	in other words the energy must be much lower than the Fermi level (chemical potential). Physicists then write this energy $E_V$ to distinguish it and name it "\NewTerm{maximum energy of the valence band}" (which corresponds to the maximum energy of a quasi-free hole to satisfy this condition).
	\begin{tcolorbox}[title=Remark,colframe=black,arc=10pt]
	The reader may observe that the conditions mentioned earlier above also impose that $E$ is either very small in absolute value or negative. Which gives us already a track for the integration terminals to come...
	\end{tcolorbox}	
	We then have:
	
	Therefore, we also have the approximation:
	
	We are therefore well in a situation where classical physics predominates over quantum physics. This is why in this approximation we say that we are dealing with a "non-degenerate semiconductor" because the holes are not crammed into the highest levels available.

	We have then:
	
	where the reader will have observed that the integration terminals have been chosen according to the remark we had just made earlier and that the terms in the square root have been swapped so as not to have any negative value.

	To continue, we make a change of variable by putting:
	
	hence:
	
	Therefore it comes:
	
	We do an integration by parts:
	
	we then make a change of variable by putting:
	
	which gives:
	
	We have already calculated this integral in the section Statistics. It comes (since the function is even we use the property demonstrated in the section of Differential and Integral Calculus):
	
	We then have finally:
	
	Where, for recall, $m_e$ is the mass of the quasi-particle (and not the mass of the hole for recall!). So after integration everything happens as if all the holes were concentrated on the level of energy $E_V$ with a number of available places corresponding to:
	
	What we write traditionally (and in a somewhat unfortunate way ... because it is not easy to remember that it is a density):
	
	or:
	
	where we have approximately at ambient temperature the following values of (quasi-)free states for Silicon:
	
	and for Germanium:
	
	
	\subsubsection{Energy bands (conduction band, band gap, valence band)}
	The previous developments for negative and positive carriers have shown us that, in the approximation of a non-degenerate fermion gas, the energy of the negative carriers must be well above the Fermi level (chemical potential) and positive carrier energy well below.

	It is therefore as if there was a forbidden energy interval or neither electrons nor holes have the right to be located! This energy interval is traditionally referred to as "\NewTerm{forbidden energy band}\index{forbidden energy band}" or simply "\NewTerm{band gap}\index{band gap}" and abbreviated B.G. The forbidden energy interval is often named simple "\NewTerm{gap}\index{gap}" and is denoted $E_g$.

	Let us see this in a rough schematic form, taking care that this scheme is somewhat misleading because it gives the impression that the conduction or valence band occupies an entire block (furthermore "flat"...), whereas in reality the valence band is constituted by the last layer completely filled, the band of permitted energy which follows being named "\NewTerm{conduction band}\index{conduction band}".
	\begin{figure}[H]
		\centering
		\includegraphics[scale=1]{img/electromagnetism/band_structures_various_materials.jpg}
		\caption{Representation of band structures in different materials}
	\end{figure}
	Moreover, knowing that molecular chemistry makes it possible to demonstrate that structures are composed of multiple bands (as a function of the first and second quantum number), the following rigorous definitions are obtained:
	\begin{enumerate}
		\item[D1.] The "\NewTerm{conducting band}\index{conduction band}" (denoted sometimes "CB") of a solid structure is the lowest energy partially occupied or empty  band (other bands are above in energy terms but will only fill at high temperatures And exist only by a theoretical description when they are empty).

		\item[D2.] The "\NewTerm{valence band}\index{valence band}" (denoted sometimes "BV") of a solid structure is the band of highest energy that is saturated, that is to say where all the states of which are occupied (knowing that there may be below the valence band other multiple bands in energy terms and all saturated).
	\end{enumerate}
	We also have the traditional schematic association of the conduction and valence bands with the Fermi-Dirac function (which as already mentioned in all rigor should be the chemical potential at non-zero temperature!) represented in simplified form by:
	\begin{figure}[H]
		\centering
		\includegraphics[scale=1]{img/electromagnetism/band_structure_with_fermi_level.jpg}
		\caption{Association of band structure with Fermi-Dirac function}
	\end{figure}
	But in fact this representation, which we find almost everywhere in certain textbooks, is relatively erroneous ... since by making a semi-classical approximation by the Maxwell-Boltzmann distribution there is no longer question of rigorously representing the distribution in the form of a Fermi-Dirac distribution as the attentive reader will have noticed! This shows that we must always be useful with figures since the traditional representation of $f_{\text{FD}}$ in the semi-classical model would indicate that there would be occupied states in the forbidden band, whereas if we represented the Maxwell-Boltzmann distribution we see two distinct functions, above and below the forbidden band!
	
	And it must be remembered (!!!) that the above figure (even if it is quite false) represents conceptually a non-degenerate semiconductor following semi-classical approximations that we have made in our developments using the model of a non-degenerate gas (Maxwell-Boltzmann approximation) and that theoretically imposed:
	
	that many authors write (again it's unfortunate but it is so...):
	 
	So there is another possible definition of the non-degenerate semiconductor: this is where the Fermi level (the chemical potential!) lies in the forbidden band and this case corresponds to how the majority of the microelectronic components work.
	
	\begin{tcolorbox}[title=Remark,colframe=black,arc=10pt]
	Let us recall (\SeeChapter{see section Statistical Mechanics}) that the Maxwell-Boltzmann statistic was constructed on the assumption of the absence of interaction between the particles involved. Moreover, this statistic is constructed in the framework of Classical Mechanics and therefore applies only when the quantum effects are negligible, for example at high enough temperatures!
	\end{tcolorbox}
	Here are some experimental values for common semiconductors:
	
	and visually and in 3D here is another figure representing de various idealized bands:
	\begin{figure}[H]
		\centering
		\includegraphics[scale=0.9]{img/electromagnetism/metal_insulator_transition.jpg}
		\caption{Metal–insulator transitions (source: Carbon conductor corrupted, authors: Michael S. Fuhrer and Shaffique Adam)}
	\end{figure}
	We understand then on the basis of these figures why the diamond, with a quasi-identical crystalline and atomic structure, is insulating while the silicon becomes conductive!

	What is interesting for researchers is to combine materials in order to play with the width of the equation according to the needs!

	Moreover, we can also conclude hastily ... that what differentiates insulators and semiconductors is the width of their forbidden band.

	It should also be noted that the energy required for an electron to pass from the valence band to the conduction band can be supplied by radiation. In the case of light absorption, the energy equation of a photon may be sufficient for this as long as:
	
	At low temperatures, such a process is capable of making the material conductive (low-temperature space telescope technology). This property is named "\NewTerm{photoconductivity}\index{photoconductivity}".

	Finally, let us recall the two relations obtained above:
	
	The product of these two densities possesses a very interesting property. We can observe that it is independent of the position of the Fermi level and named "\NewTerm{intrinsic density}\index{intrinsic density}":
	
	For example, some values of the square root of the intrinsic density at $300$ [K] are given in the table below:
	
	We also notice that the critical density is strongly temperature dependent. These values of densities are obviously idealized, in reality these values are much lower due to imperfections (residual impurities, defects in crystallization, etc.) which locally disrupt the periodicity of the potential and, therefore, introduce energy levels which can be accessible to electrons. In contrast to the levels corresponding to the pure material, we will speak of "\NewTerm{extrinsic levels}\index{extrinsic levels}".
	
	\subsubsection{Ohm's law of Semiconductors}
	We have proved in the framework of the Drude model that the conductivity was given by:
	
	where $n$ is for recall the density of carriers in the material. We have also proved that the current is inversely proportional to the conductivity according to the relation:
	
	In the framework of the developments made above we have seen that the density $n$ of the carriers was given respectively by the following relations for a constant potential (model hypothesis):
	
	where the relative masses $m_e$ of the quasi-particle (negative carrier or positive carrier) are not necessarily equal! Thus, we have the resistance which can be approximated by a relation of the form:
	
	and we easily verify this dependence by graphically representing:
	
	that is $\ln(R)$ as a function of $1/T$ (the resistance therefore depends only on the theoretical temperature at constant voltage).
	
	The real complexity lies in the fact that many terms are dependent on temperature (Fermi level, mean free travel time, etc.) and the applied potential, which means that in reality the curves obtained are by no means according to the theory....!
	
	A numerical application shows that the carrier densities $N_n$ and $N_p$ therefore increase very quickly already from the ambient temperature! What is consistent with the experience with non-degenerate semiconductors because we will then have the conductivity which increases just as strongly which implies a rapid decrease in resistance!
	
	The high sensitivity of the conductivity of certain solids to temperature variations is the cause of many applications, both for conductive metals and semiconductors. This is what we name "\NewTerm{thermistors}\index{thermistors}".
	
	Finally, let us notice that in the case of Silicium, we have $E_g=1.12$ [eV] whereas the kinetic energy due to the thermal agitation ({see section Continuum Mechanics}) is given at ambient temperature by:
	
	But, we have already seen that only the electrons whose energy was close to that of the Fermi level could participate in the conduction. Their kinetic energy then being:
	
	where $v_F$ is the "\NewTerm{Fermi Velocity}\index{Fermi Velocity}" (velocity of electron-wave in a conductor).
	
	By equalizing the last two relations:
	
	There is therefore a ratio of a factor of $30$ between the two energies, either by taking the square root, a ratio of $5$ between the velocities. Therefore we have:
	
	 But we have already seen, in our study of the Drude model, that the thermal velocity led us to a superior average distance of one order of magnitude (factor $10$) of the interatomic distances. And here we have a factor $5$ more !!!! Thus more than $50$ interatomic distances! The mean free path of an electron of conduction is therefore much greater than that which we had determined from the classical Drude model. Thus, the mean free path does not seem to be due to collisions with the ions of the network but it is due to the network's imperfections: structural defects, foreign atoms ...

	A perfect (pure) semiconductor, without imperfections, as we have theoretically treated it so far, is named an "\NewTerm{intrinsic semiconductor}\index{intrinsic semiconductor}": it therefore has no impurity and its electrical behavior depends only on the Structure of the material. This behavior corresponds to a perfect semiconductor, that is to say without structural defects or chemical impurities. A real semiconductor is never perfectly intrinsic, but can sometimes be as close as pure monocrystalline silicon.

	In an intrinsic semiconductor, the charge carriers are created only by thermal excitation. The number of electrons in the conduction band is then equal to the number of holes in the valence band as shown by our theoretical model.

	It must be realized that these semiconductors actually do not, or very little, conduct the current, except if we carry them at high temperature.
	
	
	\pagebreak
	\subsection{Superconductivity}
	"\NewTerm{Superconductivity}" is a phenomenon of exactly zero electrical resistance and expulsion of magnetic flux fields occurring in certain materials, named "superconductors", when cooled below a characteristic critical temperature. It was discovered by Dutch physicist Heike Kamerlingh Onnes on April 8, 1911, in Leiden (he get the Nobel Price for this discovery on with the element Mercury $\mathrm{Hg}$). 
	
	Like ferromagnetism and atomic spectral lines, superconductivity is a quantum mechanical phenomenon. It is characterized by the Meissner effect, the complete ejection of magnetic field lines from the interior of the superconductor as it transitions into the superconducting state. The occurrence of the Meissner effect indicates that superconductivity cannot be understood simply as the idealization of perfect conductivity in classical physics.
	\begin{figure}[H]
		\centering
		\includegraphics[scale=1]{img/electromagnetism/superconductor_mercury.jpg}
	\end{figure}
	Superconductors are also able to maintain a current with no applied voltage whatsoever, a property exploited in superconducting electromagnets such as those found in MRI machines. Experiments have demonstrated that currents in superconducting coils can persist for years without any measurable degradation. Experimental evidence points to a current lifetime of at least 100,000 years. Theoretical estimates for the lifetime of a persistent current can exceed the estimated lifetime of the universe, depending on the wire geometry and the temperature.
	
	\subsubsection{Meissner effect (second London's equation)}
	The "\NewTerm{Meissner effect}\index{Meissner effect}" is the expulsion of a magnetic field from a superconductor during its transition to the superconducting state. The German physicists Walther Meissner and Robert Ochsenfeld discovered this phenomenon in 1933 by measuring the magnetic field distribution outside superconducting tin and lead samples. The samples, in the presence of an applied magnetic field, were cooled below their superconducting transition temperature. Below the transition temperature the samples canceled nearly all interior magnetic fields. They detected this effect only indirectly because the magnetic flux is conserved by a superconductor: when the interior field decreases, the exterior field increases. The experiment demonstrated for the first time that superconductors were more than just perfect conductors and provided a uniquely defining property of the superconductor state.
	\begin{figure}[H]
		\centering
		\includegraphics[scale=0.57]{img/electromagnetism/superconductor_cern.jpg}
		\caption{Electric cables for accelerators at CERN (source: Wikipedia)}
	\end{figure}
	A superconductor with little or no magnetic field within it is said to be in the "\NewTerm{Meissner state}". The Meissner state breaks down when the applied magnetic field is too large. Superconductors can be divided into two classes according to how this breakdown occurs:
	\begin{itemize}
		\item In "\NewTerm{Type I superconductors}\index{Type I superconductors}", superconductivity (perfect diamagnetism behavior) is abruptly destroyed when the strength of the applied field rises above a critical value $H_c$. Depending on the geometry of the sample, one may obtain an intermediate state consisting of a baroque pattern of regions of normal material carrying a magnetic field mixed with regions of superconducting material containing no field. 
		
		\item In "\NewTerm{Type II superconductors}\index{ype II superconductors}", raising the applied field past a critical value $H_{c1}$ leads to a mixed state (also known as the "\NewTerm{vortex state}") in which an increasing amount of magnetic flux penetrates the material, but there remains no resistance to the flow of electric current as long as the current is not too large. At a second critical field strength $H_{c2}$, superconductivity is destroyed. The mixed state is actually caused by vortices in the electronic superfluid, sometimes named "\NewTerm{fluxons}\index{fluxons}" because the flux carried by these vortices is quantized. Most pure elemental superconductors, except niobium and carbon nanotubes, are Type I, while almost all impure and compound superconductors are Type II.
	\end{itemize}
	
	We will now determine London's second equation. It relate the external magnetic field applied on a sample to the value of this same magnetic field inside the sample.
	
	We consider for this purpose that in the superconductor phase, the electric charge carrier of mass $m$, of electric charge $q$, of density $\rho$, animated of a speed $v$ such that their kinetic energy is written:
	
	where the integral is obviously extended to the whole sample volume of the superconductor. 
	
	We have proved earlier that we also:
	
	and in static regime, it is linked to the magnetic field $\vec{B}$ by the four Maxwell equation and if the conductor is perfect it has no capacity and we can omit the second term such that it remain:
	
	Combining these relations we have:
	
	It is important at this step to notice that we assume that in the superconductor that since $\vec{v}\neq 0$ there is like as permanent current! So if we observe the Meissner effect it means that implicitly that there is something like a current inside the superconductor sample and this without that we need to provide any external potential (voltage)!
	
	The term in parenthesis has the dimension of a length and is named "\NewTerm{London penetration depth}\index{London penetration depth}" and denoted $\lambda_L$ such that the latter equality is written:
	
	Furthermore, we have prove in the section of Electrodynamics that the energy density of a monochromatic wave was given by:
	
	Therefore we assume that to the external magnetic field energy is given by:
	
	The total energy is the written:
	
	The total energy should be, by the Maupertuis principle, minimal for the real magnetic field profile that we will denote $B^{*}(\vec{r})$ and the we are looking for.
	
	The variationnal $\delta E_\text{tot}$ is then equal to zero if the profile $B(r)$ departs only in an infinitesimal way from the real profile from a quantity $\delta B(\vec{r})$. Therefore:
	
	But we have introduced in the section of vector Calculus the following identity:
	
	Therefore with $\vec{x}=\text{rot}(\vec{B})$ and $\vec{y}=\delta\vec{B}$ we have:
	
	Therefore:
	
	Therefore our variational can be written:
	
	Now let us recall the Ostrogradsky theorem proved in the section of Vector Calculus:
	
	Therefore:
	
	The surface integral is always zero since, by hypothesis, $\vec{B}$ is fixed on the sample boundary ($\delta\vec{B}(r_\text{surface})=0$), therefore it remains:
	
	The real $B^{*}(\vec{r})$ we are looking for is such that:
	
	Therefore we can try the trivial solution:
	
	But we have seen in the section of Vector Calculus that:
	
	Therefore (don't forget that the divergence of the magnetic field is always equal to zero!):
	
	Therefore:
	
	Hence:
	
	Now it must be notice that we cannot have $B_x$ that of $x$ if the sample is put in a magnetic field that is static and uniform otherwise we would not have $\text{div}(B^{*}(\vec{r}))=0$. And as we need $\text{rot}(B^{*}(\vec{r}))=\mu\vec{j}$ then as simple solution is to suppose that each component $B_i$ depends only on a coordinate that is perpendicular to itself. Therefore we will choose $B_z$ as depending on $x$ such that the latter differential equation will be written for this component:
	
	A special simple solution is obviously:
	
	This relation is knows as the "\NewTerm{second London's equation}\index{second London's equation}". The fact that the magnetic field decrease exponential inside the material is interpreted (...) in many textbooks and by many physicists as the fact the surpraconductor rejects the magnetic field since in laboratories we have the following observation:
	\begin{figure}[H]
		\centering
		\includegraphics[scale=0.6]{img/electromagnetism/messnier_effect.jpg}
		\caption{Observation of a superconductor in levitation in magnetic field due to Meissner effect}
	\end{figure}
	This bring us to the following well know misleading figure below that we can see on many Internet sites and in many textbooks:
	\begin{figure}[H]
		\centering
		\includegraphics[scale=0.6]{img/electromagnetism/superconductor_criticaltemperature.jpg}
	\end{figure}
	
	The derivation above proposed by Pierre-Gilles de Gennes for undergraduate class that don't use Quantum Physics is however a coincidence (as there exists many others in physics)! Obviously we see that this derivation of the London equation is very not appropriate. In this derivation the fact that the metal is in superconducting state is not used at all for example... Therefore, if this derivation is correct, any metal should show the Meissner effect.

	\begin{flushright}
	\begin{tabular}{l c}
	\circled{40} & \pbox{20cm}{\score{4}{5} \\ {\tiny 23 votes,  66.09\%}} 
	\end{tabular} 
	\end{flushright}

	%to force start on odd page
	\newpage
	\thispagestyle{empty}
	\mbox{}	
	\section{Optics (ray optics)}
	\lettrine[lines=4]{\color{BrickRed}O}ptics is the study of the fraction of radiant energy sensitive to the retina, that is to say the "\NewTerm{light}\index{light}" or said more generally: the "\NewTerm{electromagnetic waves}\index{electromagnetic waves}" (\SeeChapter{see section Electrodynamics}) and over a wide frequency band which is not limited (depending on the case studies) to visible light!
	
	We chose in this book to split the study of "\NewTerm{geometrical optics}\index{geometrical optics}" or simply "\NewTerm{optics}\index{optics}" into three parts: the photometry (see below), geometrical optics (this section) and wave optics (next section).
	\begin{enumerate}
		\item The "\NewTerm{photometry}\index{photometry}" is responsible for the study of some of the variables definitions of energy properties of electromagnetic waves with respect to the visual sensitivity.
		
		\item The "\NewTerm{geometrical optics}\index{Geometrical optics}" is responsible for the study that describes the propagation of light in transparent media without involving the nature of light. This is a part of physics has the advantage not to require complicated mathematical tools, but a lot of good geometrical sense...
		
		\item The "\NewTerm{optical wave}\index{Optical wave}" is responsible for the study where the luminous phenomena are interpreted taking into account of the nature of light. This is considered as an electromagnetic wave of a given wavelength defining its color (subjective concept as we will discussed below). This use much more complicate tools as Fourier transforms and Green theorem for example.
	\end{enumerate}
	In some experiments, however, we must consider light as a corpuscular phenomenon (\SeeChapter{see section Wave Quantum Physics}). We assume then that it is formed by particles, "\NewTerm{photons}\index{photons}" whose energy is proportional to the light frequency according to the Planck's law (not the Planck's law of thermodynamics ... the other one!).
	
	For consistency reasons, as we have already mentioned, we have chosen to put photometry in the section of Geometric Optics (here so...).
	
	Before we start studying the mathematics of geometrical optics, it seemed good to us clarify some blurred areas of optics that are rarely clearly defined or not even treated at all in most physics books. Thus, we will first present what is a source of light or an absence of light and then how colors are seen and treated by humans (and most similar animals).
%	\begin{figure}[H]
%		\centering
%		\includegraphics[scale=0.09]{img/electromagnetism/sun_rays.jpg}
%	\end{figure}
	
	\pagebreak
	\subsection{Sources and Shadows}
	Experience teaches us that in a homogeneous and transparent medium light travels in a straight line (at least in a flat space) and that the latter always comes from "\NewTerm{light sources}\index{light sources}".
	
	Some objects are illuminated by themselves (Sun, Flames, etc.). Some other objects are not generally visible in the dark in the visible bandwidth (when there is absolutely no source of light) but if they are illuminated they send all or part of light in all directions (see sections of Electrodynamics and Quantum Particle Physics) and therefore behave like light sources but with less intensity as the original one.
	
	\textbf{Definitions (\#\mydef):}
	\begin{enumerate}
		\item[D1.] A "\NewTerm{punctual source}\index{punctual light source}" is a unique "\NewTerm{bright spot}\index{bright spot}".
		
		\item[D2.] An "\NewTerm{extended source}\index{extended source}" is a set being a sum of punctual sources.
		
		\item[D3.] A "\NewTerm{light ray}\index{light ray}" is any straight line (or "geodesic"...) that light will follow.
		
		\item[D4.] A "\NewTerm{reversible light ray}\index{reversible light ray}" is a light ray that will follow the exactly same path backwards if we reverse its direction of propagation.
		
		\item[D5.] A "\NewTerm{light beam}\index{light beam}" is a set of light rays.
		
		\item[D6.] A "\NewTerm{ray diagram}\index{ray diagram}" is used to determine the image location, size, orientation and type of image formed by objects when place at a given location from a mirror or lens. They proved useful information about object-image relations and focal-geometry properties.
		
		\item[D7.] An "\NewTerm{apparent diameter}\index{apparent diameter}" is the angle, usually small, under which we see  one of the dimensions of an object (angle in radians as seen in the section of Trigonometry).
	\end{enumerate}
	Light passes through empty space without undergoing alteration (on small distance at least...). Thus is how Sunlight, before reaching the limit of the earth's atmosphere, travel through huge empty spaces without undergoing major transformations.
	
	On Earth, between a luminous object and the seeing eye of  the object, the light passes through a certain thickness of air. The object remains visible in other gases, or through a sheet of glass, mica, cellophane ..., or even through a layer of water, alcohol, glycerin ... such bodies are "\NewTerm{transparent media}\index{transparent media}" (even if they alter the quality of the image).
	
	Most bodies do not let light pass through. Placed between the eye and a luminous object, they suppress the vision of the object: then we say that they are "\NewTerm{opaque bodies}\index{opaque bodies}".
	
	In fact, no substance is perfectly transparent and propagation in a transparent medium is always accompanied by a weakening. This phenomenon of absorption depends on the nature of the medium and increases with the thickness of material traversed. Thus the water, even very pure, is opaque at a thickness of a few hundred meters. Also the deep seabed never receive sunlight.
	
	Sometimes some body, named "\NewTerm{translucent bodies}\index{translucent bodies}" let light filter  without allowing the eye to identify the object that emits light. These are frosted glass, striped glass, thin porcelain, oiled paper...
	
	In a dark space, the eye located outside the path of light, sees this path thanks to fine solid particles (dust, tobacco smoke, fog, etc.) suspended in the air. These illuminated particles scatter the light they receive, becoming as many bright spots which materialize the volume traversed by the light. The familiar observation shows that these light volumes still seem limited by straight lines of light.
	
	We can therefore apply the theorem of Thales to some light phenomena. So imagine the following experiment:
	
	We create quite small dimensions light sources that we can consider as point sources (i.e. bright spots).
	
	Consider $S$ is such a point source of light. Consider the volume that the source $S$ illuminates through an aperture in a diaphragm located in the path of the light at a distance $d$. If we denote by $AB$ the circular diameter of the opening diaphragm $K$ and that we cut the light path by a screen $E$, parallel to $K$ and at a distance $D$ from the source, we would observe that the irradiated region is limited by a disc of diameter $A'B'$.
	
	\begin{figure}[H]
		\centering
		\includegraphics{img/electromagnetism/thales_theorem_punctual_source.jpg}
		\caption{Application of Thales' theorem on punctual sources}
	\end{figure}
	If we can measure the diameters $AB$ and $A'B'$ of the two discs and their distances $d$ and $D$ to the source, we would find that they satisfy the Thales Theorem and therefore that:
	
	It is also proof that the light volume is effectively limited by straight lines issues from $S$ and leaning on the edge of the aperture.
	
	These observational facts and basic experience suggest the following hypothesis: 
	
	In a homogeneous transparent medium (remember that a medium is homogeneous when all is infinitesimal volumes have the same physical properties) light coming from a light spot propagates along straight lines from this point. These lines are named as we already know "light rays".
	
	If we return to the previous above figure, all the light rays contained in the cone defined by the source $S$ and the diaphragm $K$ is also as we already know a "light beam".
	
	\textbf{Definitions (\#\mydef):}
	\begin{enumerate}
		\item[D1.] Light traveling here from $S$ we say then that the ray "diverges" or that the beam is a "\NewTerm{divergent beam}\index{divergent beam}" (at the opposite of the LASER for those that know what is a LASER).
		
		\item[D2.] When a point source is at infinity (as is almost a Star in the night sky for example), the rays that come from it are considered as parallel and the beams that they form are named "\NewTerm{parallel beam}\index{parallel beam}" or "\NewTerm{cylindrical beams}\index{cylindrical beams}".
		
		\item[D3.] Thanks to a convergent lens (magnifying glass for example), we will see that it is possible to change the direction of rays from a punctual source and make the concentrate on a point $S'$. Such a set of rays then constitutes a "\NewTerm{convergent beam}\index{convergent beam}".
	\end{enumerate}
	
	A very narrow light beam takes the name of "\NewTerm{light pen}\index{light pen}". For example, the rays from a light spot on the eye always form a very slender pencil of light, because the distance from the observed eye is necessarily large compared to the diameter of the pupil.
	
	If we return to our experience with the diaphragm: If we decrease the opening of the diaphragm that will limit  a light beam to a light pen, we observe (when the diameter is reduced to less than a few tenths of a millimeter) that the trace of the brush on a screen $E$, rather than diminish, expands (!) and this is an evidence that the light reaches now points outside of the cone $SA'B'$.
	
	It is as if the very small aperture $AB$ itself became a punctual source: we say that the light is "\NewTerm{diffracted}\index{diffracted}". We will return later to this property of light because it is a fairly elaborate mathematical study (\SeeChapter{see section Wave Optics}") and therefore complex to handle but still very very interesting (having huge implications because related to quantum physics!).
	
	Now consider a punctual source of light. Between the source and a screen $E$, let us interpose an opaque body of any shape. Under the assumption of rectilinear propagation, we observe a "\NewTerm{shadow cone}\index{shadow cone}" limited by the rays that are based on the outline of the interposed body.
	\begin{figure}[H]
		\centering
		\includegraphics{img/electromagnetism/shadow_cone.jpg}
		\caption{Shadow cone with proper shadow example}
	\end{figure}
	The non-illuminated region of the opaque body is the "\NewTerm{proper shadow}\index{proper shadow}" or "\NewTerm{umbra}\index{umbra}", one that matches the screen is the "\NewTerm{projected shadow}\index{projected shadow}".
	
	If the light source is extended, the proper shadow and projected shadow have no longer clearly defined contours. Their edges are surrounded by an intermediate area which is named the "\NewTerm{penumbra}\index{penumbra}". A well known example is given by the figure below:
	\begin{figure}[H]
		\centering
		\includegraphics{img/electromagnetism/shadow_penumbra.jpg}
		\caption{Shadow cone with Penumbra example}
	\end{figure}
	
	\subsection{Colors}
	\textbf{Definition (\#\mydef):} We name "\NewTerm{color}\index{color}" the perception of light excitation following neurophotochimical reaction process in the eye following of one or more light wave frequencies with one or more given amplitudes.
	
	\begin{tcolorbox}[title=Remark,colframe=black,arc=10pt]
	We will see that the fundamental constant characterizing quantum physics (like the speed of light characterizes relativity) is the Planck constant and deeper the fine structure constant.
	\end{tcolorbox}	
	
	It is important never to confuse "\NewTerm{color}\index{color}" that is a perceptive concept  (it is not a physical property!!!) and "\NewTerm{wavelength}\index{wavelength}" that is a physical concept. Thus, the human eye is most often unable to distinguish a theoretical yellow monochromatic (single wavelength) of a corresponding composition of green and red. This illusion will display yellow on our computer screens, and more generally any color.
	
	By the fact that the sensitive part of the retina of the human eye is made up of elements named "\NewTerm{cones}\index{cones}" each sensitive to a small bandwidth of wavelength corresponding respectively to Red (via the erythrolabe molecule sensible to wavelength $~700 [\text{nm}]$ and denoted by the letter $L$) , Green (via the chlorolabe molecule sensible to wavelength $~545 [\text{nm}]$ and denoted by the letter $M$) and Blue (via the molecule cyanolabe sensible to wavelength $~440 [\text{nm}]$ and denoted by the letter $S$): 
	\begin{figure}[H]
		\centering
		\includegraphics[scale=0.5]{img/electromagnetism/eye.jpg}
		\caption{Naive eye structure summary}
	\end{figure}
	We can create any color by adding these three basic colors named "\NewTerm{additive primary colors}\index{additive primary colors}" (or "\NewTerm{additive primary colors}\index{additive primary colors}"). This is named the "\NewTerm{additive synthesis of colors}\index{additive synthesis of colors}".
	
	\begin{tcolorbox}[title=Remark,colframe=black,arc=10pt]
	Some scientists believe (without proof and without being able to reproduce this fact) that the sensitivity to each of theses colors comes from: for the Red for the time where evolution was in water and animals were looking in infrared, for the Green at the time where animals were on earth with a luxurious and dense vegetation and finally for the Blue because of the blue sky!
	\end{tcolorbox}	
	In what follows, we will denote the red by ($R$), the green by ($G$), the blue by ($B$), the white by ($W$) and the black by ($N$):
	\begin{figure}[H]
		\centering
		\includegraphics{img/electromagnetism/visible_spectrum_summary.jpg}
		\caption{Visible spectrum summary}
	\end{figure}
	
	The French Association for Standardization (AFNOR) has defined in the 20th century the visual trivariance principle as follows:\textit{ A radiation of any color can be produced visually identically by algebraic mixture, in a uniquely defined proportions, of luminous beam of three rays that can be arbitrarily selected, provided that none of them can be reproduced by a combination of the other two}.
	
	Clearly when we see the frequencies of the visible spectrum it will not be tomorrow with the known material of this early 21st century that we are going to build antennas or parabola able to transmit at such frequencies! Already $120 [\text{GHz}]$ is a challenge by the year 2010 then $500 [\text{THz}]$ will not be for tomorrow...
	
	You should know that until the years 1800 it was not known if the colors were limited or not those visible to the human eye. It was with the advent of thermometers with mercury that were sufficiently sensitive and specific that the astronomer Herschel placed one in front of a light spectrum and found that moving it from a color strip to the other, from violet to red, the temperature raised. To his surprise, it continued to rise when he accidentally let the thermometer to on or two centimeters beyond the red light visible bandwidth. Herschel detected invisible light to the human eye, described later as being infrared radiation.
	\begin{figure}[H]
		\centering
		\includegraphics[scale=0.5]{img/electromagnetism/whole_spectrum_colors.jpg}
		\caption{Whole standard "color" spectrum (source: Wikipedia)}
	\end{figure}
	Or corresponding temperatures:
	\begin{figure}[H]
		\centering
		\includegraphics{img/electromagnetism/temperature_spectrum.jpg}
		\caption{Temperature Spectrum}
	\end{figure}
	A magistral and pedagogical example of what can be seen in the various spectra is the Crab Nebula that is a supernova remnant resulting from the explosion of a historical supernova (SN 1054) observed by several Chinese astronomers of the dynasty Song from July 1054 to April 1056:
	\begin{figure}[H]
		\centering
		\includegraphics{img/electromagnetism/SN1054.jpg}
		\caption{SN 1054 respectively (left to right) in radio waves, infrared, visible and X-ray}
	\end{figure}
	Pointing three light beams ($R$, $G$ and $B$) in the same place, we can get (in fact it would be more rigorous to say "perceived" because this is specific only to certain trichromats mammals): white light. We say then that white ($W$) (in the human sense) is the sum of the three additive primary colors (remember that white is rigorously the sum of all the colors of the spectrum - so that white is made of a continuous light spectrum). All imaginable colors are obtained by varying the intensity of each of the three beams. Black is obtained when we do send no light at all.
	
	For example, if we add (in the theoretical sense: with infinitely small and transparent colors component ...) just red ($R$) and green ($G$), we get the yellow ($Y$), if we add the red and blue we get the magenta ($M$), if we add green ($G$) and blue ($B$), we get the Cyan ($C$). So we can sum this up by the following equations:
	
	These three colors ($YMC$) obtained by adding two additive primary colors are named "\NewTerm{secondary additive colors}\index{secondary additive colors}". The notion that blue and yellow form green is only a perception in our eyes!!! If the reader shine blue and yellow light on a chromograph it will show blue and yellow independently. In fact light doesn't interfere with light in our regular experience set at the opposite of sound. And this because as our yeye only measure three things at the opposite of ears that measure every sound on the spectrum we can hear (as it do a indirectly a Fourier Transform).
	
	Diagram of additive synthesis:
	\begin{figure}[H]
		\centering
		\includegraphics{img/electromagnetism/rgb.jpg}
		\caption{Additive synthesis (RGB)}
	\end{figure}
	The existence of these three types of pigments in our photoreceptor cones serves as physiological basis to the "\NewTerm{trichromatic model}\index{trichromatic model}" or of "\NewTerm{visual trivariance}\index{visual trivariance}".
	
	\textbf{Definition (\#\mydef):} A color is named "\NewTerm{complementary color}\index{complementary color}" of another if they give white when you add them. For example, yellow ($Y$) is the complementary color of blue ($B$):
	
	In contrast to additive synthesis, there are the "\NewTerm{subtractive colors}\index{subtractive colors}": it is that we are talking about when we take away the color to a base color. This is for example the case of the ink or of color filters (in the sense that there is a base support where color must be treated).
	
	To understand what this is, let us put a red filter on an projector. The projected light will be red. We note that the filter has removed color to white light: $W$ became $R$ but as $W = RGB$, this means that the red color filter removed the $GB$ to the white light of the overhead projector. With the same reasoning, we understand that a $G$ filter subtracts the colors $RB$ and $B$ filter subtracts $RG$.
	
	If we overlay two filters of different primary colors: for example, $R$ and $G$ filter filters, we will not get anything at all, that is, to say $N$. Indeed, the $R$ filter let pass only red light and $G$ filter subtracts this color and nothing remains. This leaves no more color, i.e. $N$.
	
	We notice that filters $R, G$ and $B$ fail to synthesize different colors by subtraction as we get the black ($N$) as soon as we superimpose two different. What is very annoying when the concerned support is paper and the purpose is to print something colorful.
	
	It is therefore more useful to use the yellow, magenta and cyan filters ($Y$, $M$ and $C$) that is to say secondary additive colors. Indeed, a filter $Y$ let pass yellow, that is to say $RG$. It therefore subtracts only the blue $B$ to the original white $W$ light. By the same principle, a magenta $M$ filter subtracts green $G$ and a filter cyan $C$ subtrac red $R$.
	
	We note therefore that the superposition of two filters of these secondary colors gives a new color to an existing support. We can synthesize any color by varying the intensity of each of the three filters $Y$, $M$ and $C$ that we overlay (on the projector or paper for example). We name these three colors "\NewTerm{basic subtractive colors}\index{basic subtractive colors}".
	
	Scheme for subtractive synthesis:
	\begin{figure}[H]
		\centering
		\includegraphics{img/electromagnetism/cmy.jpg}
		\caption{Subtractive synthesis (CMY)}
	\end{figure}
	In relaity we do not get a perfect black $N$ this is why printers have a black color ink.
	
	\begin{tcolorbox}[colframe=black,colback=white,sharp corners]
	\textbf{{\Large \ding{45}}Examples:}\\\\
	E1. A television or computer screen works on the principle of additive color mixing. Indeed, looking at the screen through a magnifying glass, one can realize that it is filled with small groups (pixels) of three phosphors (bright area when excited by electricity) $R, G$ and $B$. These phosphors are so close that when they light up together, they seem to get confused (at least on screen have a good "definition": pixels per inch) and most human therefore only sees the additive synthesis of the three phosphors. For example, on a television screen entirely red, only red phosphors glow. By cons, if the screen turns yellow, this means that green phosphors glow along with the red one.\\
	
	E2. At the opposite to television or computer screens, we find the printing processes that operate in subtractive synthesis. Indeed, the sheet is white and you have to remove from it to get the color we are looking for. The technique is the same as that of filter: inks contain pigments which filter certain colors. Using $Y, M$ and $C$ inks we can get all colors of the visible spectrum. However, the pigments are not perfect and black $N$ is very difficult to obtain (ink and color overload rather dark brown). So we use black as a fourth color. This system is named "\NewTerm{quadrichromy printing}\index{quadrichromy printing}". It is used for example in most color printers and in rotary newspapers.
	\end{tcolorbox}
	It is interesting now to focus on the phenomena that overlap the two concepts (if we can say ...). Thus, a system that projects color following a $RGB$ concepts by additive or subtractive system itself may be illuminated by an equivalent system. This results in an effect overlay.
	
	So when we talk about the color of objects, we normally refer to the appearance they have when they are illuminated with white light!
	\begin{tcolorbox}[colframe=black,colback=white,sharp corners]
	\textbf{{\Large \ding{45}}Example:}\\\\
	A red $R$ tomato, absorbs some of the white light $W$ ($GB$ part) and distributes the rest (red $R$) back around. That is why it appears Red $R$ to us when under white light. A yellow lemon, appears yellow to us because it absorbs the Blue $B$ color distributes the rest around ($RG=R+G=Y$) .... But what about a tomato lit by a blue light? What looks like a yellow lemon if we illuminate it by red light?\\
	
	We can answer by reasoning as follows: as tomato absorbs $GB$ and therefore intrinsically Blue ($B$), there remains nothing. It then appears black $N$. For the lemon, as it absorbs the blue $B$ and diffuses back $R + V=Y$ light so if we only illuminate it with red $R$ it will only diffuse the red and therefore appear will appear Red $R$ to us.
	\end{tcolorbox}
	\begin{figure}[H]
		\centering
		\includegraphics[scale=0.6]{img/electromagnetism/additive_subtractive_colors.jpg}
		\caption{Additive and Subtractive experiment (source: physicsfun instagram)}
	\end{figure}
	
	\pagebreak
	\subsection{Radiometry/Photometry}
	Material are capable of emitting, transmitting and / or absorb electromagnetic energy. Several factors characterize ths radiation such as its spectral range, intensity, direction, and some intrinsic properties to the material. Photometry proposes to seek the magnitudes that are specific to the material and the laws governing them.
	
	We recognize two types of photometry: "\NewTerm{energetic photometry}\index{energetic photometry}" and "\NewTerm{visual photometry}\index{visual photometry}". In what follows, we will stick mainly to energetic photometry which generally relate to the energy carried by electromagnetic radiation, whatever its wavelength.
	
	Beforehand, we have to specify the conditions under which we will define the new variables/quantities. We will assume the following assumptions:
	\begin{enumerate}
		\item[H1.] The radiation propagates in a transparent medium for all intensities, wavelength and polarization.
		\item[H2.] The propagation takes place along the solid angles (\SeeChapter{see section Trigonometry}). We therefore omit the radiation in parallel rays.
		
		\item[H3.] The elementary surfaces $\mathrm{d}S$ that are study are sufficiently small so that the radiation of their points are identical but not too small to avoid phenomena such as diffraction.
	\end{enumerate}
	
	\subsubsection{Energy flow} 
	\textbf{Definition (\#\mydef):} The "\NewTerm{energy flow}\index{energy flow}" or "\NewTerm{radiant flow}\index{radiant flow}" of a radiation source is the power it radiates. The flow equation is measured in Watts [W] (or joules per second [J / s]), and it follows therefore that for a source that radiates energy (not necessarily constant), we have:
	
	In some professional fields the energy flow is expressed in photometric units as the "\NewTerm{Lumen}\index{Lumen}" denoted [lm] or in photonic units as a number of photons per second: $[\text{s}^{-1}]$. This is why, when buying lamps or displays in some stores, the units are not the same from one brand to another... (ISO norms are not respected...!).
	
	\paragraph{Beer–Lambert law}\mbox{}\\\\
	The "\NewTerm{Beer–Lambert law}\index{Beer–Lambert law}", also known as "\NewTerm{Beer's law}\index{Beer's law}", the "\NewTerm{Lambert–Beer law}\index{Lambert–Beer law}", or the "\NewTerm{Beer–Lambert–Bouguer law}\index{Beer–Lambert–Bouguer law}" relates the attenuation of light to the properties of the material through which the light is traveling. The law is commonly applied to chemical analysis measurements and used in understanding attenuation in physical optics, for photons, neutrons or rarefied gases.
	
	If the absorption and diffusion of a medium can be considered proportional to the thickness $\mathrm{d}z$ of the crossing matters, the energy flow variation can be written:
	
	in this expression $\Phi_0$ is the incident energy flow and $\mu\;[\text{m}^{-1}]$ is the "\NewTerm{linear attenuation coefficient}\index{linear attenuation coefficient}", which is a function of the radiation frequency depending of the material the radiation pass through.
	
	So we have a simple differential equation (\SeeChapter{see section Differential and Integral Calculus}):
	
	which is the "\NewTerm{Beer-Lambert law}\index{Beer-Lambert law}" (which can also be expressed from the light intensity that we will define further below).
	
	We have typically the following order of amplitude:
	\begin{gather*}
		\mu_{\text{yellow}}=10^{-5}\;[\text{m}^{-1}]\quad \mu_{\text{purple}}=4\cdot 10^{-5}\;[\text{m}^{-1}]\quad \mu_{\text{Glas (BKZ)},0.55 [\mu\text{m}]}=0.2\;[\text{m}^{-1}]
	\end{gather*}
	\begin{tcolorbox}[title=Remark,colframe=black,arc=10pt]
	The variation of the atmospheric absorption coefficient with the wavelength allows in particular to explain the blue color of the sky and because of water light reflection why our oceans looks like blue seen from space (yes don't forget that water don't have color... it's transparent!).
	\end{tcolorbox}
	There are many other formulations of the Beer-Lambert law with one fairly used in nuclear physics (see section of the same name in this book) in the framework of radiation protection. Let's see now what it is:
	
	Let us consider a flow $\Phi$ of particles striking perpendicularly the surface of a material of thickness $\mathrm{d}z$ and of atomic density $\rho_N$ (in [$\text{atoms}\cdot \text{m}^3$]). If we consider the particles striking a surface $S$, the latter can theoretically meet $\rho_NS\mathrm{d}z$ targets atomes in this layer. The number of interacting particles will be proportional to the intensity times this number, and we have:
	
	where $\sigma$ is a proportionality constant named "microscopic cross section." Its units are often expressed as "\NewTerm{barn}\index{barn}" ($1\;[\text{barn}]=10^{-26}\;[m]$).
	\begin{tcolorbox}[title=Remark,colframe=black,arc=10pt]
	The atomic density $\rho_N$ is equal to:
	\begin{gather*}
		\rho_N=\dfrac{\rho N_\text{Av.}}{M_\text{Mol.}}
	 \end{gather*}
	where  $\rho$ is the density in $\text{kg}\cdot\text{m}^3$, $N_\text{Av.}$ is the Avogadro's number $6.022\cdot 10^{23}\; [\text{atoms}\cdot \text{mole}^{-1}]$ and $M_\text{mol.}$ the molar mass of the target expressed in $[\text{kg}\cdot \text{mole}^{-1}]$.
	\end{tcolorbox}
	If we now assume that the scattering centers are the electrons and not the target atoms, then we must replace $\rho_N$ by $\rho_{N_e}$ where $N_e=\rho_NZ$ with $Z$ being the number of electrons interacting by target atom. Therefore:
	
	If we now admit that the scattering centers are the electrons, not the target atoms, then we must replace $N$ by $N_e$ where $N_e=NZ$ where $Z$  being the number of electrons interacting with the target atom. From where:
	
	By identifying with the first formulation of the Beer-Lambert law, we see that $\mu$ plays the same role as:
	
	And the hypothesis that the electron is a "\NewTerm{sphere of action}\index{sphere of action}" having a front surface $\pi r_e^2$, where $r_e$ is the radius of this sphere, then:
	
	and we have for the radius of the sphere of action of the electron:
	
	
	\subsubsection{Light Intensity (Radiant Intensity)}
	To describe the energy flow $\Phi$ of a source, you must first measure it. The used sensor (thermocouple, bolometer, photocell, eye or others) may receive only one part: that which happens in the solid angle  $\mathrm{d}\Sigma$ defined by its section.

	 \textbf{Definition (\#\mydef):} The "\NewTerm{light intensity}\index{light intensity}", "\NewTerm{energy intensity}\index{energy intensity}" or "\NewTerm{radiant intensity}\index{radiant intensity}" denoted $I$ of a punctual source is radiated flow $\Phi$ in the unit solid angle $\Omega$ centered around a direction $\Delta$ of emission:
	
	The light intensity is expressed in certain professional fields in the photometric units (see summary table further below) "\NewTerm{Candela}\index{Candela}" denoted [Cd] or in photonic units in $[\text{Ws}^{-1}]$ (remember that the steradians don't have any units as well as for radians). This is why, when you buy screens or lamps in some stores, the units are not the same from one brand to another.
	\begin{tcolorbox}[title=Remark,colframe=black,arc=10pt]
	A source is named an "\NewTerm{anisotropic source}\index{anisotropic source}" or "\NewTerm{directional source}\index{directional source}" if its intensity varies with the direction of observation.
	\end{tcolorbox}	
	By comparison (as it may help), a Candela unit is equivalent to the intensity of a source in a given direction, which emits a monochromatic radiation of frequency $540.1012$ [Hz] (which roughly corresponds to the frequency at which the eye is the most sensitive), and whose luminous flow(or intensity) in that direction of $1/683$ [W] per steradian.
	
	\subsubsection{Energy Emittance (Radiant Emittance)}
	\textbf{Definition (\#\mydef):} The "\NewTerm{emittance}\index{emittance}", "\NewTerm{exitence}\index{exitence}" or "\NewTerm{illumination}\index{illumination}" $M$ of a source is the radiated energy flow (power) per unit surface $\mathrm{d}S$ in [W / m${}^2$] in all directions of outer space the source and depends on the physico-chemical properties of the emitting surface:
	
	A common mistake is to make a confusion between the emittance and the energy!!!
	\begin{tcolorbox}[title=Remark,colframe=black,arc=10pt]
	"\NewTerm{Radiant emittance}\index{Radiant emittance}" is an old term for this quantity. Radiant exitance is often named "\NewTerm{intensity}\index{intensity}" in branches of physics other than radiometry, but in radiometry this usage leads to confusion with radiant intensity.
	\end{tcolorbox}	
	The emittence is often assimilated in the current vocabulary to the "luminosity" of a light source which sometimes leads to confusion with the concept of light intensity.
	
	The emittance is expressed by many professional of the field in photometric units named "\NewTerm{Lux}\index{Lux}" denoted [lx] or in photonic units $[\text{W}\text{m}^{-2}]$ or even worse ... in $[\text{lm}\cdot \text{m}^{-2}]$ or even more worse $[\text{cd}\cdot \text{sr} \text{m}^{-2}]$. For example when you buy a car, the headlights are indicated as being about $\sim 20$ [lx].
	
	If the source is punctual (or considered as!) and its radiation isotropic , its direction has not to be taken into consideration. In the case of the related sphere of radius $r$, the emittance is then expressed as:
	
	In the case of the sphere, one element $\mathrm{d}S$ of the spherical surface receives perpendicularly the radiation. Very generally, an elementary surface can be inclined relative to the direction of radiation with an angle $\theta$. So we have to project the surface perpendicular to the radiation using the basic reasoning of trigonometry (\SeeChapter{see section Trigonometry}):
	
	It is this projection that explains the seasons on Earth: the area swept by the emittance more or less constant and isotropic of the Sun (considered as a punctual source) is maximum at the equator (perpendicular surface) and implies a radiation higher compared to that receives at higher or lower latitude for which the perpendicular projection of the surface in question is smaller than at the equator for the same emittance.
	
	\begin{tcolorbox}[title=Remarks,colframe=black,arc=10pt]
	\textbf{R1.} The energy emittance is calculated only in the outer half space before (the point from which we look at the source), because only half of the energy exchanged by the points of the surface $\mathrm{d}$S is emitted as radiation. The other half is exchanged with the atoms located in the body.\\
	
	\textbf{R2.} The emittance is usually also sometimes denoted $F$ or even $E$. The practitioner will however take care not to confuse the emittance $M$ with the magnitude (denoted by the same letter) that we define in the section of Astrophysics.
	\end{tcolorbox}
	
	\subsubsection{Radiance and Luminance}
	Given a non-punctual source which emittance $M$ is known at any point. An element $\mathrm{d}S$ of the surface of this kind of source will by definition not necessarily by isotropic intensity and therefore brighter (more powerful) when we observe collinearly to the vector $\mathrm{d}\vec{S}$.
	
	The intensity $I$ that radiates in a direction forming an angle $\theta$ with the normal to the emission surface $S$ is always less than that radiated in the direction of the vector $\mathrm{d}\vec{S}$. So by simple application of trigonometry, we get the definition of "\NewTerm{luminance}\index{luminance}" (or "radiance"):
	
	expressed in certain areas, in "Nits"... photometric units  $[\text{Cd}\cdot\text{m}^{-2}]$ or photonic units $[\text{W}\text{m}^{-2}]$ (without specifying explicitly the steradian).
	
	Radiance is useful because it indicates how much of the power emitted, reflected, transmitted or received by a surface will be received by an optical system looking at that surface from some angle of view. In this case, the solid angle of interest is the solid angle subtended by the optical system's entrance pupil. Since the eye is an optical system, radiance and its cousin luminance are good indicators of how bright an object will appear. For this reason, radiance and luminance are both sometimes named "\NewTerm{brightness}\index{brightness}", even if this usage is now discouraged.
	\begin{tcolorbox}[title=Remark,colframe=black,arc=10pt]
	When we are concerned only with the human visible part of light, the luminance and radiance of a source is named by practitionners "brightness" (note this is not the case when we deal of shining as we will see it in the section of Astrophysics).
	\end{tcolorbox}	
	Luminance is a measure of how much power (energy over time) is emitted in a particular direction (basically how much light your TV will produce), Radiance broadly speaking, measures how much of the luminance you'll actually see. Finally brightness doesn't really measure anything it's broadly a catch-all term describing both luminance and radiance.
	
	We can obviously write by making some elementary algebra that:
	
	that gives us the energy intensity that radiates a source of luminance $L$ in a given direction.

	Johann Heinrich Lambert (1728-1777) has observed that the energy intensity of some anisotropic sources (among all imaginable types of sources ...) decreases as the cosine of the angle $\theta$, around the direction perpendicular to the surface source, such that:
 	
	and that we name "\NewTerm{Lambert's cosine law}\index{Lambert's cosine law}". So  the radiant intensity or luminous intensity observed from an ideal diffusely reflecting surface or ideal diffuse radiator is directly proportional to the cosine of the angle $\theta$ between the direction of the incident light and the surface normal:
	\begin{figure}[H]
		\centering
		\includegraphics{img/electromagnetism/lamberts_cosine_law.jpg}
		\caption{Lambertian Emittor (source: Wikipedia)}
	\end{figure}
	A surface which obeys Lambert's law is said to be "\NewTerm{Lambertian surface}\index{Lambertian surface}". The apparent brightness of a Lambertian surface to an observer is the same regardless of the observer's angle of view. More technically, the surface's luminance is isotropic, and the luminous intensity obeys Lambert's cosine law. Unfinished wood exhibits roughly Lambertian reflectance, but wood finished with a glossy coat of polyurethane does not, since the glossy coating creates specular highlights. Not all rough surfaces are Lambertian reflectors, but this is often a good approximation when the characteristics of the surface are unknown.
	\begin{tcolorbox}[title=Remark,colframe=black,arc=10pt]
	We speak of "luminance" of a source of "illumination" of an object (by a source).
	\end{tcolorbox}
	
	\pagebreak
	\paragraph{Lambert's Law}\mbox{}\\\\\
	A light source follow the Lambert's cosine law if its luminance (or radiance) is the same in all direction. Indeed, if we put first:
	
	we have:
	
	and we have:
	
	But we have proved in the section of trigonometry that:
	
	Which brings us to write:
	
	The energy emittance being equal to:
	
	This result is important for the study of black body radiation, since the luminance value measured by a sensor gives the possibility to deduce the emittance $M$, so the energy flow from the source is equal to: 
	

	Here is a small summary of what what we have seen so far as sometimes it can be quite confusing:
	
	
	\pagebreak
	\subsubsection{Kirchhoff's law of Radiation}
	Any body irradiated by an energy source sees the incident energy flow incident divided into three intuitive terms:
	
	where:
	\begin{itemize}
		\item $\Phi_r=\rho\Phi$: is the energy flow reflected or diffused.

		\item $\Phi_t=\tau\Phi$: is the energy that flow through the body without interactions (full transparency).

		\item $\Phi_a=\alpha\Phi$: is the energy flow transformed into other forms of energy.
	\end{itemize}
	The three coefficients respectively named "\NewTerm{reflectance factor $\rho$}\index{reflectance factor}"or more simple "\NewTerm{albedo}\index{albedo}", "\NewTerm{transmittance factor $\tau$}\index{transmittance factor}" and "\NewTerm{absorption factor $\alpha$}\index{absorption factor}" depend obviously on the wavelength $\lambda$ (and therefore the frequency) of the incident light and the of the temperature of receiver body.
	
	For each object we have obviously:
	
	which is the expression of the "\NewTerm{simple Kirchhoff law}\index{simple Kirchhoff law}" in photometry (at the opposite of the differential one).
	
	The albedo is an important concept in climatology as we will use it in the corresponding section of the book.
	\begin{figure}[H]
		\centering
		\includegraphics[scale=0.15]{img/electromagnetism/earth_albedo.jpg}
		\caption{2003–2004 mean annual clear-sky and total-sky albedo (source: Wikipedia)}
	\end{figure}
	
	\pagebreak
	A interesting (funny) but also very useful modern material for space telescopes imagery quality is Vantablack whose albedo is of  $0.00035$...! The result is quite mind blowing as you can see in the photo below:
	\begin{figure}[H]
		\centering
		\includegraphics{img/electromagnetism/ventablack.jpg}
		\caption{Vantablack painted face...}
	\end{figure}
	Looks like quite unreal isn't it? ;-)
	
	\subsubsection{Spectral Decomposition}
	From what has been said above, it follows that all the previously defined quantities can be related to their spectral decomposition in wavelength $\lambda$. This results comes from the superposition principle: any radiation can be treated as the superposition of monochromatic radiation (even if we will prove during our study of quantum physics later that a monochromatic radiation does not really exists).
	
	Thus, we define:
	
	Where we also write to simplify the notations:
	 
	\begin{tcolorbox}[title=Remark,colframe=black,arc=10pt]
	The units AND values of spectral decomposed energy flow, spectral intensity, spectral luminance/radiance, spectral emittance and also the spectral absorption factor, spectral reflectance factor and spectral transmission factor are of course not equivalent to their integrated expression.
	\end{tcolorbox}
	We will need for the density of the emittance in the study of black body in the section of Thermodynamics of the chapter Mechanics. Remember only that we have in S.I. units based on the spectral decomposition principle (and vice versa superposition):
	
	\begin{tcolorbox}[title=Remark,colframe=black,arc=10pt]
	 We have seen in the section of Thermodynamic that parameters defined above, being dependent on wavelength, they are also dependent on the temperature of the source which emits these waves.
	\end{tcolorbox}
	
	\subsection{Law of Refraction}
	The law of refraction, which is generally known as "\NewTerm{Snell's law}\index{Snell's law}", governs the behavior of light-rays as they propagate across a sharp interface between two transparent dielectric media.

	Pierre de Fermat proposed that light rays (electromagnetic waves) met a very general principle that the path taken by the light to travel from one point to another was the one for which the travel time was minimum (actually an extremum which may be a minimum or maximum). This proposal, known as "\NewTerm{Fermat's principle of least time}\index{Fermat's principle of least time}", a fundamental principle of geometrical optics is based on the principle of least action (principle we have already introduced in the section of Analytical Mechanics) as we prove it later below.

	Before starting the developments let us give some important definitions and introduce some concepts. some of which are related to the figure below:
	\begin{figure}[H]
		\centering
		\includegraphics{img/electromagnetism/vocabulary_optical_geometry.jpg}
		\caption{Vocabulary for the study of geometrical optics}
	\end{figure}
	
	\pagebreak
	\textbf{Definitions (\#\mydef):} 
	\begin{enumerate}
		\item[D1.] A "\NewTerm{refracting medium}\index{refracting medium}" is a medium that causes the deflection of an incident light ray.
		
		\item[D2.] The "\NewTerm{incident ray}\index{incident ray}" is the ray of light propagating in a medium $1$, pass wholly or partially in a refracting medium $2$, the rest being absorbed or partially reflected.
		
		\item[D3.] The "\NewTerm{angle of incidence}\index{angle of incidence}", sometimes denoted $i$ is the angle by which the incident beam enters the refracting medium.
		
		\item[D4.] The "\NewTerm{totally or reflected ray}\index{totally or reflected ray}" is the part of the light ray that having met the interface between the propagation medium and the refractive medium, continues his path in the propagation medium.
		
		\item[D5.] The "\NewTerm{reflection angle}\index{reflection angle}", sometimes denoted $r_x$ or simply $r$ if there is no possible confusion, is the angle by which the ray is reflected from the plane representing the interface between the propagation medium and itself. We will prove that the incident and reflected angles are equal in absolute values.
		
		\item[D6.] The "\NewTerm{partially or totally refracted ray}\index{partially or totally refracted ray}" is the portion of the light ray having met the interface between the propagation medium and the refractive medium, continues its path in the refracting medium.
		
		\item[D7.] The "\NewTerm{angle of refraction}\index{angle of refraction}", sometimes denoted $r_c$ or sometimes simply $r$ if there is no confusion, is the angle by which the ray is refracted by the plane representing the interface between the propagation medium and refracting medium. The incident and refracted angles are linked by a relation we prove further below.
	\end{enumerate}
	An illustration that could help to understand some of these definitions better is the following:
	\begin{figure}[H]
		\centering
		\includegraphics[scale=0.75]{img/electromagnetism/type_of_reflections.jpg}
		\caption{Type of reflections}
	\end{figure}
	For example with water this can give quite funny observations:
	\begin{figure}[H]
		\centering
		\includegraphics[scale=0.58]{img/electromagnetism/reflection_water_in_real_life.jpg}
		\caption{Funny water reflection effect}
	\end{figure}
	with its corresponding schematic explanation:
	\begin{figure}[H]
		\centering
		\includegraphics[scale=0.65]{img/electromagnetism/reflection_water.jpg}
		\caption{Water reflection ray diagram}
	\end{figure}
	Before we continuie, something important must be understand (or recall): "ideal refraction" must be discerned form "diffused refraction"! Since the light strikes different parts of the surface at different angles, it is reflected in many different directions, or diffused. So when we study refraction we in physics, we consider an "ideal refraction" with a "directional source of light"!
	\begin{figure}[H]
		\centering
		\includegraphics[scale=0.7]{img/electromagnetism/diffused_reflection.jpg}
		\caption{Light is diffused when it reflects from a rough surface (source: OpenStax)}
	\end{figure}
	Diffused light is what allows us to see a sheet of paper from any angle:
	\begin{figure}[H]
		\centering
		\includegraphics[scale=1]{img/electromagnetism/diffused_vs_ideal_reflection.jpg}
	\end{figure}
	
	\pagebreak
	\subsubsection{Refractive index}
	The "\NewTerm{absolute refractive index $n_\lambda$}\index{absolute refractive index}" of a medium at a given wavelength $\lambda$ (and hence at a given frequency $\nu$) measure the reduction factor of the phase velocity of light, denoted $v_n$ in Geometric Optics, in the considered medium relatively to the speed of light in vacuum  and is given quite generally by the "\NewTerm{Cauchy's Law}\index{Cauchy's Law}" (the only mathematical proof that I had in my hands of this law begins with the Maxwell equations and standing about $3$ A4 pages, but it was based on so many successive tricks and approximations that we will go ahead and admit that it can be established experimentally):
	
	where $A$ and $B$ are experimentally determined constants. We can notice through the Cauchy's law that the absolute refractive index decreases as the wavelength increases (verbatim when the frequency decrease).
	
	Anyway, the theory of light-matter interaction on which Cauchy based this equation was later found to be incorrect. The  "\NewTerm{Sellmeier equation}\index{Sellmeier equation}" is a later development of Cauchy's work that handles anomalously dispersive regions, and more accurately models a material's refractive index across the ultraviolet, visible, and infrared spectrum and that is given by:
	
	Following what we have seen in the section of Electrodynamics it comes immediately:
	
	Therefore:
	
	\begin{tcolorbox}[title=Remark,colframe=black,arc=10pt]
	Obviously in reality all that stuff is also a function of the temperature and... not only...
	\end{tcolorbox}	
	All pure materials have absolute refractive index of a positive value greater than $1$. More a medium is dense, more the phase velocity of light is slowed down, more the absolute refractive index is high.
	\begin{tcolorbox}[title=Remark,colframe=black,arc=10pt]
	There exist however material with smaller than $1$ refractive index and even negative-index metamaterial.
	\end{tcolorbox}	
	The process of describing light transport via the quantum mechanical description isn't trivial to explain the "slow down" of the phase velocity. The use of photons to explain such process involves the understanding of not just the properties of photons, but also the quantum mechanical properties of the material itself (something one learns in Solid State Physics courses).

	A common explanation that has been provided is that a photon moving through the material still moves at the speed of $c$, but when it encounters the atom of the material, it is absorbed by the atom via an atomic transition. After a very slight delay, a photon is then re-emitted. This explanation is incorrect and inconsistent with empirical observations. If this is what actually occurs, then the absorption spectrum will be discrete because atoms have only discrete energy states. Yet, in glass for example, we see almost the whole visible spectrum being transmitted with no discrete disruption in the measured speed. In fact, the index of refraction (which reflects the speed of light through that medium) varies continuously, rather than abruptly, with the frequency of light.

	Secondly, if that assertion is true, then the index of refraction would ONLY depend on the type of atom in the material, and nothing else, since the atom is responsible for the absorption of the photon. Again, if this is true, then we see a problem when we apply this to carbon, let's say. The index of refraction of graphite and diamond are different from each other. Yet, both are made up of carbon atoms. In fact, if we look at graphite alone, the index of refraction is different along different crystal directions. Obviously, materials with identical atoms can have different index of refraction. So it points to the evidence that it may have nothing to do with an "atomic transition".

	When atoms and molecules form a solid, they start to lose most of their individual identity and form a "collective behavior" with other atoms. It is as the result of this collective behavior that one obtains a metal, insulator, semiconductor, etc. Almost all of the properties of solids that we are familiar with are the results of the collective properties of the solid as a whole, not the properties of the individual atoms. The same applies to how a photon moves through a solid.

	A solid has a network of ions and electrons fixed in a "lattice". Think of this as a network of balls connected to each other by springs. Because of this, they have what is known as "collective vibrational modes", often named phonons as we will see it during our study of semi-conductors in the section Elelectrokinetics. These are quanta of lattice vibrations, similar to photons being the quanta of EM radiation. It is these vibrational modes that can absorb a photon. So when a photon encounters a solid, and it can interact with an available phonon mode (i.e. something similar to a resonance condition), this photon can be absorbed by the solid and then converted to heat (it is the energy of these vibrations or phonons that we commonly refer to as heat). The solid is then opaque to this particular photon (i.e. at that frequency). Now, unlike the atomic orbitals, the phonon spectrum can be broad and continuous over a large frequency range. That is why all materials have a "bandwidth" of transmission or absorption. The width here depends on how wide the phonon spectrum is.
	
	On the other hand, if a photon has an energy beyond the phonon spectrum, then while it can still cause a disturbance of the lattice ions, the solid cannot sustain this vibration, because the phonon mode isn’t available. This is similar to trying to oscillate something at a different frequency than the resonance frequency. So the lattice does not absorb this photon and it is re-emitted but with a very slight delay. This, naively, is the origin of the apparent slowdown of the light speed in the material. The emitted photon may encounter other lattice ions as it makes its way through the material and this accumulate the delay. 
	
	Moral of the story: the properties of a solid that we are familiar with have more to do with the "collective" behavior of a large number of atoms interacting with each other. In most cases, these do not reflect the properties of the individual, isolated atoms.
	
	Finally, the "\NewTerm{relative refractive index $n_{12}$}\index{relative refractive index}" specifically refers to comparing (ratio) of the absolute refractive index of an optically dense media to another background media such that:
	
	
	\pagebreak
	\subsubsection{Snell's law}
	Let us consider (see figure below) two respective medium $M_n$ and $M_m$ of refractive indices $n$ and $m$ (implicitly dependent on the wavelength as we will consider constant temperature) and whose contact surface is flat. Consider two points $A$ and $B$ respectively in the medium of index $n$ (point $A$) and in the medium of index $m$ (point $B$).

	Consider the path of light going from $A$ to $B$. Fermat's principle teaches us that the path taken by the light is such that the time taken to travel is minimum.

	Therefore we propose in a first time to apply a standard method to calculate the path of the light ray and secondly, we will show that Fermat's principle can be stated as a variational principle!!!
	
	Let us choose a reference frame that simplifies the problem: let us make coincide the $x$-axis with the contact plane of both medium and the $y$-axis through point $B$. In such a reference frame, points $A$ and $B$ have the following coordinates: $A(x_A,y_A)$, $B(0,y_B)$.
	
	Let us denote by $M(x,0)$, the point where the light beam passes through the contact surface between the two medium. The time $t$ taken by light to get from $A$ to $B$ is by applying simple kinematics relations:
	
	\begin{figure}[H]
		\centering
		\includegraphics{img/electromagnetism/refraction_law.jpg}
	\end{figure}
	where:
	
	are the phase velocity of light in the medium $M_n$ and $M_m$.
	
	We can observe on the figure above that the incident rays are refracted on the other side of the axis perpendicular to the interface. This is a typically characteristic of material having a positive refraction index. But it is physically possible to build since the 1990s artificial  composites  "metamaterials" having negative refraction  index:
	\begin{figure}[H]
		\centering
		\includegraphics{img/electromagnetism/positive_negative_refraction_index_material.jpg}
		\caption{Positive and Negative refraction index material behavior}
	\end{figure}
	
	The writing of the two previous relations:
	
	Developing the values of $\overline{AM}$ and $\overline{MB}$ we get the following dependence of $t$ as a function of the position $x$ of $M$:
	
	According to Fermat's principle, the path taken by the light is the one for which $t$ is minimum. The extremum of $t (x)$ is reached when the derivative with respect to $x$ is zero. Therefore:
	
	Notice that:
	
	The condition of an extremum time brought by the light is then expressed by:
	
	It is sufficient that the angles of incidence and refraction meet that condition so that the path traveled by light is actually the one that takes the least time.
	
	Hence we get the relation, known as the "Snell's law" (which is not anymore a law as we just proved it):
	

	We write more frequently the Snell-Descartes law in physics as following:
	
	\begin{tcolorbox}[title=Remarks,colframe=black,arc=10pt]
	\textbf{R1.} We will see during our study of wave optics that we can fall back (prove) the same relation but without the assumptions bases of geometrical optics. Therefore, the latter relation is named "Descartes-Snell relation " or just "Snell's law".\\
	
	\textbf{R2.} When we speak of the relative refractive index of a given medium without reference to another medium, the default environment is the vacuum.\\
	
	\textbf{R3.} Some materials do not have an absolute isotropic index of refraction : it then depends on the direction of propagation and the polarization state of light. This property is named "\NewTerm{birefringence}\index{birefringence}".
	\end{tcolorbox}
	Let us now study the relation between the relative refractive index and the phase velocity of light in the various medium through which it passes.

	A light ray connects two points $A_1$ and $A_2$ located on either side of an interface $S$ between the two mediums. This ligth ray is not shown in the figure below. Only are drawn the path located on either side of the light ray that satisfy the extremu  (we rely now on the study of the maximum path length). By hypothesis, they are extremely close, so the distance $\overline{PQ}$ is very small:
	
	We will admit that they correspond to the same travel time.
	
	\begin{figure}[H]
		\centering
		\includegraphics{img/electromagnetism/phase_speed_refraction_index.jpg}
		\caption{Figure allowing to connect phase velocity and refractive index}
	\end{figure}
	Since both paths are very close, we can assume the equality of the distances $\overline{A_1B_1}$ and $\overline{A_1Q}$ one one hand, and of $\overline{PA_2}$ and $\overline{B_2A_2}$ of the other one. Thus, by hypothesis:
	
	But, under the same hypothesis:
	
	such that:
	
	The "\NewTerm{refraction law}\index{refraction law}" is finally stated in general as following:
	
	And about the reflection angle, as we have already stated, it remains equal to the angle of incidence if the reflective surface is perfectly smooth and flat.
	
	If we consider the following writing:
	
	and the case where $n_1>n_2$ (for example passage from water to air). Then, for values close to $1$, that is to say for grazing angles of incidence (incident beam close to the surface), the Snell's law gives a value greater than $1$. We then get out of the validity domain of the law. This corresponds to situations where there is no refraction but only for reflection, then we speak of "\NewTerm{total reflection}\index{total reflection}":
	\begin{figure}[H]
		\centering
		\includegraphics[scale=0.5]{img/electromagnetism/critical_angle.jpg}
		\caption{Critical angle schematic illustration}
	\end{figure}
	To find the critical angle, we find the value for $\theta_i$ when $\theta_r$ is equal to $\pi/2$ and thus $\sin(\theta _r)=1$. The resulting value of $\theta _i$ is equal to the critical angle $\theta_c$.

	Now, we can solve for $\theta_i$, and we get the equation for the critical angle:
	
	In the case of a water-air boundary where $n_2 = 1.33$ and $n_1=1$ then the critical angle is $\theta_c = 48.7^\circ$.
	
	So in the real life we can get such things:
	\begin{figure}[H]
		\centering
		\includegraphics{img/electromagnetism/total_reflection_water_air.jpg}
		\caption[]{Air-Water total reflection example}
	\end{figure}
	\begin{tcolorbox}[title=Remark,colframe=black,arc=10pt]
	Total Internal Reflection is used to carry light in fiber optics.
	\end{tcolorbox}
	Fermat's principle therefore has obvious similarities with the principle of least action in that it consists of a minimum principle. Even if a rigorous description of light requires the introduction of Quantum Mechanics, it is however possible to apprehend it with the tools of Analytical Mechanics and to apply on it, under certain assumptions, the principle of least action. We will prove just now that we then fall back on the Fermat's principle.

	The calculations that we will present, introduce many hasardeous assumptions, but this process should be considered as an approximation. It must be known ot the reader that Fermat's principle also carries an approximation that we we can qualify of "classical limit".
	
	Now let us Imagine for the proof that light is composed of material "grains" . We must admit that these grains obey to rather unusual physical properties: its mass is zero since according to the classical description, the light rays are not deflected by the gravitational field. This lack of mass thus makes them insensitive to the Earth's gravitational field (be careful and don't forget: we are in a "classical" description of the light as it was done in 19th century and before!!!!).

	Let us write the action for one of these grains of light (\SeeChapter{see section Analytical Mechanics}):
	
	Now by assuming that the only existing potential field $V$ is that one that derives from the gravitational field and that we assume that light as it is insensitive to it (we know in General Relativity that this is wrong but we said just now that we would stay in classical point of view!!!), it follows that the action of light can be written:
	
	But no force applied to the light, so the kinetic energy $T$ is a constant of motion. Applying  the variational principle of least action we then have:
	
	Hence we get:
	
	This equation means that the time taken by the light along its path is minimum (or more generally, is an extremum). We then fall back indeed on Fermat's principle. So we have proved that the under the classical limit  Fermat's principle follows directly from the principle of least action.
	
	\pagebreak
	\subsubsection{Cherenkov radiation}
	Cherenkov radiation, also known as Vavilov–Cherenkov radiation, is electromagnetic radiation emitted when a charged particle (such as an electron) passes through a dielectric medium at a speed greater than the phase velocity of light in that medium. The characteristic blue glow of an underwater nuclear reactor is due to Cherenkov radiation. It is named after Soviet scientist Pavel Alekseyevich Cherenkov, the 1958 Nobel Prize winner who was the first to detect it experimentally. A theory of emitted corresponding spectrum was later developed within the framework of Einstein's special relativity theory by Igor Tamm and Ilya Frank, who also shared the Nobel Prize. Cherenkov radiation had been qualitatively predicted by the English polymath Oliver Heaviside in papers published in 1888–89.
	
	\begin{figure}[H]
		\centering
		\includegraphics[scale=0.19]{img/electromagnetism/cherenkov_radiation.jpg}
		\caption{Oak Ridge National Laboratory Nuclear Reactor}
	\end{figure}
	\begin{tcolorbox}[title=Remark,colframe=black,arc=10pt]
	Sometimes some wonder why charged particles can go faster than light in a medium other than a vacuum. It is simple to the point: even if the two particles meet roughly the same obstacles and difficulties to spread in a medium the photon cannot be accelerated by a pulse when a charged particle can be be accelerated by a given phenomenon in a given medium.
	\end{tcolorbox}
	We saw in the preceding paragraphs the hypothesis (relatively intuitive) than the phase propagation speed of light in an absolute refractive index medium $n$ was not equal to $c$ but still less by writing this:
	
	The Cherenkov effect is (basically) a phenomenon similar to that of an (acoustic) shock wave, but producing a flash of light instead of sound, which takes place on the path of a charged particle moving in a medium with a higher phase velocity that that of the speed of light in that medium (rigorous explanation is beyond the scope of study of this book because of its complexity treatment!).

	Indeed, let us first recall that we proved in the section Electrodynamics that any moving charged particle emits electromagnetic radiation. Then we have proved in the preceding paragraphs that the speed of light in a given medium depends on the absolute refractive index $n$ of the medium (hypothesis which is verified by the experimental accuracy of the theoretical developments arising therefrom).
	
	So we have two basic informations:
	\begin{enumerate}
		\item The speed of the charged particle that can be written as follows with the traditional relativistic notations:
		

		\item The phase velocity of light in a medium with absolute refractive index given:
		
	\end{enumerate}
	It is easy to see that to get $v>v_c$ we must have:
	
	Therefore:
	
	Some authors prefer to compare the distance traveled by the light in relation to the distance traveled by the particle. It comes that as:
	
	And therefore for the particle traverses distances equal to those of light at in the same time interval we must have that $\beta=1/n$. Beyond appears the Cherenkov effect.
	
	\pagebreak
	\subsection{Descartes' Formulas}
	We have previously discussed some phenomena that occur when a wavefront passes from one medium to another in which the propagation is different. Not only have we analyzed what becomes the wave front, but we have introduced the concept of "radius" which is particularly useful for geometric constructions. We now propose to deepen the phenomena of refraction and reflection of a geometric point of view using the concept of radius as the tool to describe the processes taking place in the discontinuity surfaces of the propogation. We also assume that the process is limited to reflections and refractions, no other changes affecting the wave surfaces.

	This geometry treatment is correct until surfaces and discontinuities encountered by the wave during its propagation are very large relatively to the wavelength. As long as this condition is met, the treatment applies both to light waves, sound (in particular ultrasound - very high frequency), seismic waves, etc.

	We begin by considering the wave reflection on a spherical surface. For this purpose we must first given some obvious definitions: The center of curvature $C$ (\SeeChapter{see section Differential Geometry}) is the center of the spherical surface of the figure below and the top O is the pole of the spherical cap.
	
	\textbf{Definition (\#\mydef):} The line passing through the pole cap O and the center of curvature $C$ is named the "\NewTerm{optical axis}\index{optical axis}".
	
	If we take O as coordinate origin, all measured quantities to the right of O will be taken as positive, all those on the left as negative !!!
	
	\begin{figure}[H]
		\centering
		\includegraphics{img/electromagnetism/descartes_formulas_optical_axes.jpg}
		\caption{Representation of the optical axis concept}
	\end{figure}
	Let us suppose that the point $P$ is a source of spherical waves. The radius $\overline{PA}$ gives by reflection the radius $\overline{AQ}$ and, as the angles of incidence and reflection are equal with respect to the perpendicular $\overline{AC}$ to the surface (as we have already observed it in our study of refraction), we see in the figure that:
	
	hence:
	
	Assuming that the angles $\alpha_1,\alpha_2$ and $\beta$ are vers small, that is to say, the rays are "\NewTerm{para-axial}\index{para-axial}" and the source is very far or that the sensor is very small compared to the source, we can write with a good approximation with a Maclaurin  development (\SeeChapter{see section Sequences And Series}) for small angles:
	
	Substituting these approximations of $\alpha_1,\alpha_2$ and $\beta$ into $\alpha_1+\alpha_2=2\beta$, we get:
	
	which is the "\NewTerm{Descartes formula for reflection on a concave spherical surface}". It involves, in the approximation used to establish it, that for all incident rays passing through $P$ will pass through $Q$ after reflection on the surface. We can then say that $Q$ is "\NewTerm{the image of the object}\index{image of an object}" $P$.
	
	In the special case where the incident beam is parallel to the optical axis, that is equivalent to place the object at a great distance of the lens, we have $p=+\infty$. The Descartes formula for reflection on a concave spherical surface becomes therefore:
	
	and the image is formed at the point $f$ named the  "\NewTerm{focal}\index{focal}" and its distance from the lens given by:
	
	and is named "\NewTerm{focal length}\index{focal length}". We also get the ratio $r/2$ if we make $q$ tend to infinity (that means the curvature radius is put to an infinite value).
	
	So the bigger is $r$, the bigger is $C$ (smaller curvature = wider angle photo) and therefore the bigger is $q$.

	The relation obtained above is also valid for a convex surface. Indeed, we simply need pull the lines representing the light rays beyond the concave surface to see that the object of study is the same to a given symmetry:
	\begin{figure}[H]
		\centering
		\includegraphics{img/electromagnetism/descartes_formulas_extension.jpg}
	\end{figure}
	The only difference between the concave and convex surface is that in the case of the convex surface, the reflected image of the object appears as if it is behind the surface (at the equivalent of to the point $P$). This leads us to define the following terminology:
	
	\textbf{Definitions (\#\mydef):}
	\begin{enumerate}
		\item[D1.] A "\NewTerm{virtual image}\index{virtual image}" is a term used in optics to refer to any image formed before the exit face of an optical instrument (in the direction of travel of light) and therefore will not be displayed on the projection screen. For a thin converging spherical lens an object placed between the object focus and the optical center of the lens will give a right (not inversed) virtual image. This is particularly the case of an optical system used as a magnifying glass, which provides a magnified image of the object observed through the lens.

		\item[D2.] A "\NewTerm{real picture}\index{real picture}" is a term used in optics to refer to any image formed after the exit face of an optical instrument (in the direction of travel of light). For a thin spherical converging lens an object placed against the object focus of the lens will give a real (inversed) image.
	\end{enumerate}
	\begin{figure}[H]
		\centering
		\includegraphics[scale=1]{img/electromagnetism/real_and_virtual_image.jpg}
		\caption{Why image are reverse when passing through a thin spherical lense (for simplification purposes the focal in the drawing above is supposed to be inside the lens...)}
	\end{figure}
	In the context of flat mirrors the meaning is quite different:
	\begin{figure}[H]
		\centering
		\includegraphics[scale=0.7]{img/electromagnetism/real_and_virtual_image_mirror.jpg}
	\end{figure}
	obviuosly for flat mirrors the distance $p$ of the real object to the mirror will have virtually the same distance $q$ that has the "virtual object" to this same mirror so that virtually we could writhe $p=q$. For flat mirror the image is obviously upright and it produces and apparent left-right reversal!
	
	\subsubsection{Stigmatism}
	If the opening (wide angle) of the mirror is large, so it receives steeply inclined rays, the above Descartes formula isn't anymore, we know it by construction, a good approximation. There is in this case no more well defined punctual image of a "point object" but an infinite number of them: consequently the image of a large object appears blurred since the images are superimposed . This effect is named "\NewTerm{spherical aberration}\index{spherical aberration}" and part of the optical axis which contains all the reflected images is then named the "\NewTerm{caustic reflection}\index{caustic reflection}". The spherical aberration can not be eliminated, but a suitable design of the surface allows to remove it for certain positions on the optical axis named "\NewTerm{stigmatic}\index{stigmatism}". For example, in our previous study case, it is clear (by geometrical construction) that if we put $P$ on $C$ then the point C becomes the stigmatic points. We say then that's the point "\NewTerm{strictly stigmatic}\index{strict stigmatism}".
	
	By cons, for the parabolic mirror all the rays converge on the focus of the mirror where is concentrated the light energy received by the mirror. Conversely, we place the filament of a lamp at the focus of a parabolic mirror to get far-reaching headlights (typically lighthouse having no Fresnel lens). We also give a parabolic shape to antennas needing to receive radio waves. For television broadcast by satellite as it works at the centimeter waves (frequency of several GHz) a focal distance of one meter is suitable for the antenna (ie. this applies to telescopes and radio telescopes).
	
	\begin{figure}[H]
		\centering
		\includegraphics{img/electromagnetism/stigmatism.jpg}
		\caption{Representation of the concept of stigmatism}
	\end{figure}
	The idea to prove mathematically that the focus of the parabola is the rigorous stigmatic point is this:

Let us recall the following figure we used in our study of conical in the section of Analytic Geometry:
	\begin{figure}[H]
		\centering
		\includegraphics{img/electromagnetism/stigmatism_figure_proof.jpg}
	\end{figure}
	We have added on the figure the point $\mu$ that is the orthogonal projection of the point $M$ (point of incidence of the light beam) and also the tangent to the parabola at the point $M$. If we can prove that the tangent to $M$ is the mediator of segment $\overline{F\mu}$, then we also prove that the angle of incidence and reflection are equal. Therefore as we already know that incidence and reflection angles are equal, the we prove indirecly that on any point $M$ of the parabola the incident ray comes on $F$.
	\begin{dem}
	Let us consider the equation of the parabola relatively to the focal as proved in the section of Analytical Geometry:
	
	We have also proved in the section of Analytical Geometry that relatively to $\Omega$ the Focal is at the position $F(h/2,0)$ and the direction had for equation:
	
	We get the equation of the tangent at $M(x_0,y_0)$ by the derivative at the same point (caution! ... remember the particular orientation of the parabola!):
	
	Which can also be written:
	
	and knowing that:
	
	So we get the equation of the tangent:
	
	On of the vector of the tangent is therefore as proved in the section of Analytical Geometry during our study of the parametric equation of a the line:
	
	where in our case of the parabola, $p$ is equal to $h$.
	
	On the other hand, we have (this is easily verified by taking $x_0=0$):
	
	So we have the scalar product:
	
	as the vectors $\overrightarrow{MF}$ and $\overrightarrow{M\mu}$ have the same norm by definition of the parabola in Analytical Geometry\footnote{Yes, for recall a "parabola" is the set of all points which are equidistant from a point, called the focus, and a line, named the "directrix"}, we deduce that the vector $\vec{v}$ (giving the direction of the tangent) leads the line bisecting the angle of the $\overrightarrow{MF}$ and $\overrightarrow{M\mu}$ and therefore by extension that the tangent $M$ is the mediator of $\overline{M\mu}$.
	\begin{flushright}
		$\square$  Q.E.D.
	\end{flushright}
	\end{dem} 
	\begin{figure}[H]
		\centering
		\includegraphics[scale=0.6]{img/electromagnetism/antenna.jpg}	
		\caption{Parabolic antenna for reception or emission (source: OpenStax)}
	\end{figure}
	Let us also calculate the focus position of a parabolic reflector whose equation is given by (\SeeChapter{see section Analytical Geometry}):
	
	We fix for example $x$ to $0$ then it remains:
	
	After rearranging we get:
	
	As:
	
	we get immediately:
	
	And we have prove in the section of Analytical Geometry that:
	
	A circular paraboloid is theoretically unlimited in size. Any practical reflector uses just a segment of it. Often, the segment includes the vertex of the paraboloid, where its curvature is greatest, and where the axis of symmetry intersects the paraboloid. However, if the reflector is used to focus incoming energy onto a receiver, the shadow of the receiver falls onto the vertex of the paraboloid, which is part of the reflector, so part of the reflector is wasted. This can be avoided by making the reflector from a segment of the paraboloid which is offset from the vertex and the axis of symmetry:
	\begin{figure}[H]
		\centering
		\includegraphics[scale=0.5]{img/electromagnetism/off_axis_satellite_dish_construction.jpg}	
		\caption[]{Off-axis receiver dish construction principle (source: Wikipedia)}
	\end{figure}
	The receiver is still placed at the focus of the paraboloid, but it does not cast a shadow onto the reflector. The whole reflector receives energy, which is then focused onto the receiver. This is frequently done, for example, in satellite-TV receiving dishes, and also in some types of astronomical telescope:
	\begin{figure}[H]
		\centering
		\includegraphics[scale=0.7]{img/electromagnetism/off_axis_satellite_dish.jpg}	
		\caption{Off-axis satellite dish}
	\end{figure}
	
	\pagebreak
	\subsubsection{Lenses}
	Let us now do a similar study to that carried out earlier, with the same properties of symmetry and defects, but the "\NewTerm{spherical dioptres boundaries}\index{spherical dioptres boundaries}" that is a spherical boundary between to mediums (interesting results regarding the study of the eye). The results will be helpful to us before tackling the study of spherical lens (the traditional magnifying glass).
	
	So we will now consider the refraction of the light during the passage through a spherical surface separating two medium with absolute refractive indices $n_1$ and $n_2$ (see figure below):
	\begin{figure}[H]
		\centering
		\includegraphics{img/electromagnetism/spherical_dioptre.jpg}
		\caption{A biconvex lens in real life... (source: Wikipedia)}
	\end{figure}
	where for recall the center of curvature $C$  (\SeeChapter{see section Differential Geometry}) is the center of the spherical surface of the figure below and the top O is the pole of the spherical cap.

	The basic geometric elements are the same as those specified for spherical surfaces. We therefore consider initially a concave refractive surface (concave dioptre) and observing that the "\NewTerm{object distance}\index{object distance}" is located at the opposite of the other points, we must adopt a sign convention to put this observation in evidence in the equations. Thus, $q$ will be defined as a negative value.

	An incident ray such as $\overline{PA}$ is refracted as following $\overline{AQ}$ and thus cross the optical axis $Q$. We see from the figure that:
	
	We have the Snell-Descartes' law:
	
	and we will admit as for spherical surfaces that the ray a only little bit inclined rays. Under these conditions the angles $\theta_i,\theta_r,\alpha_1,\alpha_2$ and $\beta$ are very small and we can write using the Maclaurin series expansion (\SeeChapter{see section Sequences and Series}):
	
	so that the Snell-Descartes' law is now written:
	
	From the figure, we can do the approximations:
	
	so that by substituting in the approximation of the Snell-Descartes' law we find elementary simplification:
	
	hence for a concave surface:
	
	The "\NewTerm{object focal $F_0$}\index{object focal}" also named "\NewTerm{first focal point}\index{first focal point}" of a refractive spherical surface is the position of a punctual object on the optical axis such that the refracted rays are parallel to the optical axis, which is equivalent to form the image of the point at infinity, where $q=\infty$.

	The distance of the object to the spherical surface is then named "\NewTerm{focal distance object}\index{focal distance object}" and we denote it by $f_0$. By putting $p=f_0$ and $q=\infty$, then we have for the concave  case:
	
	The focal length $f_0$ is positive and the system is the convergent when the object focus is real, is placed in front of the spherical surface. When the object focus is virtual, the focal length $f_0$ is negative and the system is then divergent.
	
	Similarly, if the incident rays are parallel to the optical axis, which is equivalent as to have a very distant object form the spherical surface $p=\infty$, the refracted rays pass through a point $F_i$ of the optical axis named "\NewTerm{image focus}\index{image focus}" or "\NewTerm{second focal point}\index{second focal point}" (again with the same problems of stigmatims).
	
	In this case the distance of the spherical surface to the image is named "\NewTerm{focal image distance}\index{focal image distance}" and we denote it by $f_i$. By putting $p=\infty$ and $q=f_i$ then we have for the concave case:
	
	\begin{figure}[H]
		\centering
		\includegraphics{img/electromagnetism/focals_schemes_dioptre.jpg}
		\caption{Position of first and second focal points in a dioptre}
	\end{figure}
	By mixing the two previous relationships, we have a useful result in practice:
	
	So if we know the relative refractive index and one of the focal lengths, we can infer the other. Or (and we can of course make also other combinations):
	
	Now let us take look at the type of reflective surfaces and refractions we were waiting for: the lenses!
	
	\textbf{Definition (\#\mydef):} A "\NewTerm{lens}\index{lens}" is a transmissive optical device that focuses or disperses a light beam by means of refraction. A simple lens consists of a single piece of transparent material (generally spherical), while a compound lens consists of several simple lenses (elements), usually arranged along a common axis. Lenses are made from materials such as glass or plastic, and are ground and polished or moulded to a desired shape. A lens can focus light to form an image, unlike a prism, which refracts light without focusing. Devices that similarly focus or disperse radiation other than visible light are also named "lenses", such as microwave lenses, electron lenses or acoustic lenses.
	\begin{figure}[H]
		\centering
		\includegraphics[scale=0.5]{img/electromagnetism/biconvex_lens.jpg}
		\caption{Concept of spherical dioptre}
	\end{figure}
	The study of optical lenses is a huge and subtil subject and our only purpose in this book is to give the mathematical basic properties of some elementary lenses shapes. As we will show it through some illustration, even traditional camera have already some huge complications to get qualitative results for the professional photographer as illustrated in the figure below:
	\begin{figure}[H]
		\centering
		\includegraphics[scale=0.55]{img/electromagnetism/zoom_nikkon_af_s_80_200_mm_f_2dot8D_if_ed.jpg}
		\caption{Zoom-Nikkor AF-S 80-200 mm f/2.8D IF-ED (credits: Pierre Toscani, \url{http://www.pierretoscani.com})}
	\end{figure}
	or even better (optical system also existing for smartphones!):
	\begin{figure}[H]
		\centering
		\includegraphics[scale=0.55]{img/electromagnetism/fish_eye_lens.jpg}
		\caption{Fisheye-Nikkon 6 mm f/2.8 simplified cut (credits: Pierre Toscani, \url{http://www.pierretoscani.com})}
	\end{figure}
	An incident ray thus undergoes two refractions at the crossing of the lens. Assume for simplicity that the media of both sides of the lens are identical and their absolute refraction index equal to $1$ (air or vacuum for example). We will also consider only thin lenses in this book, that is to say whose thickness is very small compared to the radii of curvature:
	\begin{figure}[H]
		\centering
		\includegraphics{img/electromagnetism/lens_technical_scheme.jpg}
		\caption{Technical representation of a lens}
	\end{figure}
	The optical axis is the line determined by the two centers $\overline{C_1C_2}$. We seek to establish a relation between the position of $P$ and $Q$ from easily measured physical parameters!

	We will consider for the analysis that the image formed after the refraction on the first surface is the object for the refraction on the second surface.

	Let us consider the incident ray $\overline{PA}$ through $P$. When passing the first surface, the incident ray is refracted following the radius $\overline{AB}$ and continues virtually to $Q'$ according to the behavior of a convex spherical surface. So we apply the relation prove just earlier above for the convex diopter:	
	
	In $B$ the ray undergoes a second refraction and becomes the radius $\overline{BQ}$ according to the behavior of a concave spherical surface. We can imagine that the incident beam in $B$ come from a virtual point $P'$ (not show in the figure above) that is on the optical axis and immersed in the material virtually extended to the right of the lens (that is the difficult side of this proof ... you have to imagine that!).

	So we have to apply the relation proved earlier above for the concave refractive surface (dioptre) but by being careful this time to the order of absolute refractive indices (subtle trap!) and by imagining that the point:
	
	As we consider the lens is surrounded by air (unitary refractive index), then we have:
	
	We will now assume that the thickness of the lens tends to zero. In other words his two radii tend to infinity. Then we have:
	
	By identifying term to term and remembering that what is left from the origin O is negative, then we have:
	
	while subtly ignoring the other two terms ... (we understand now easily why this proof is often omitted in the literature ...).

	Therefore, both prior previous relations become:
	
	Summing it comes finally:
	
	and that is often denoted in the following form, name "\NewTerm{Descartes's first formula for thin lenses}\index{Descartes's first formula for thin lenses}" or simply "\NewTerm{thin lens equation}\index{thin lens equation}":
	
	While being careful to rehabilitate the figure above such that it becomes:
	\begin{figure}[H]
		\centering
		\includegraphics{img/electromagnetism/descartes_first_formula_thin_lens_figure.jpg}
		\caption{Thin lens equation corresponding technical figure}
	\end{figure}
	By writing this formula it should be applied to $r_1,r_2$ the sign convention we have set, that is to say, the rays are positive for a concave surface and negative for a convex surface, as seen from the side on which the light hits the lens. Thus, if the two radii are the same, we have:
	
	The right term of the thin lens equation is only a constant depending on the physical characteristics of the lens and it customary to name it "\NewTerm{refractive power}\index{refractive power}" or "\NewTerm{dioptric power}\index{dioptric power}" and whose unity is the "diopter" and denoted:
	
	Converging lenses have positive optical power, while diverging lenses have negative power. When a lens is immersed in a refractive medium, its optical power and focal length change.
	
	The following figure can help to understand (play with the values of $q,p,r$ in the previous relations to understand it completely):
	\begin{figure}[H]
		\centering
		\includegraphics{img/electromagnetism/eye_adaptative_power_vision.jpg}
		\caption{Human eye adaptative refractive power mechanism}
	\end{figure}
	\begin{tcolorbox}[title=Remark,colframe=black,arc=10pt]
	An eye that has too much or too little refractive power to focus light onto the retina has a refractive error. A myopic eye has too much power so light is focused in front of the retina. Conversely, a hyperopic eye has too little power so when the eye is relaxed, light is focused behind the retina. An eye with a refractive power in one meridian that is different from the refractive power of the other meridians has astigmatism. Anisometropia is the condition in which one eye has a different refractive power than the other eye.
	\end{tcolorbox}
	The point O in the previous figure, is selected to coincide with the "optical center" of the lens. As we know, the optical center has the property of being such that any ray passing through it exit parallel to the direction of the incident ray !! This is an important property because any point of an object lying on one side of a lens (whatever which one because of the symmetry of the lens) will emit light from which some rays will pass through the optical center. Thus allowing to have similar triangles on the left and right of the symmetry axis of the lens and apply Thales' theorem (\SeeChapter{see section Euclidean Geometry}) to calculate the concept of "magnification" (see further below).

	To prove that such a point exists let us consider in the lens below (with horizontal and vertical symmetry):
	\begin{figure}[H]
		\centering
		\includegraphics{img/electromagnetism/optical_center_property.jpg}
		\caption[]{Figure representing the property of the optical center of a lens}
	\end{figure}
	 Let us consider two parallel curvatures radius:
	
	generators of the diopters (elements of the thin spherical lens for recall...) chosen such that the corresponding tangent planes $T_1$ and $T_2$ are also parallel.

	For the radius $\overline{R_1A_1}$, which direction is such that it is refracted following $\overline{A_1A_2}$, the emerging ray $\overline{A_2R_2}$ is parallel to $\overline{A_1R_1}$ by the horizontal symmetry of the lens. Thus the triangles $C_1A_1\text{O}$ and $C_2A_2\text{O}$ are similar regardless of the "generators radius", thus we see that the position of the optical center O is satisfied by the relation:
	
	and therefore exists independently of the generating radii.
	
	As in the case of a single diopter, the "\NewTerm{focal object}\index{focal object}" $F_0$, or "\NewTerm{first focal point of a lens}\index{first focal point of a lens}" is the position of the object for which the rays emerge parallel to the optical axis ($q=+\infty$) after have passed through the lens. The distance from the lens to the object focus is named the "\NewTerm{focal distance object}\index{focal distance object}" and we designate it in practice often by the simple letter $f$ for thin lenses.
	\begin{figure}[H]
		\centering
		\includegraphics{img/electromagnetism/thin_lense_first_focal_point.jpg}
		\caption{First focal point (source) is here on the left of the thin lens}
	\end{figure}
	Or for those who cannot see well the figure above, there is another more explicit illustration of what happen with the first focal point of a bi-convex lens:
	\begin{figure}[H]
		\centering
		\includegraphics[scale=1]{img/electromagnetism/first_focal_point_illustration.jpg}
		\caption{First focal point with bi-convex lens behavior (credits: Pierre Toscani, \url{http://www.pierretoscani.com})}
	\end{figure}
	Then by putting $p=f$ and $q=+\infty$ in the equation of spherical thin lenses we get:
	
	we then get the following focal distance object in the form named "\NewTerm{focal length of the lens}\index{focal length of a lens}":
	
	Similarly in the case of a simple diopter, the "\NewTerm{focal image}\index{focal iamge}" $F_i$, or "\NewTerm{second focal point of a lens}\index{second focal point of a lens}" is where converge the light rays after passing through the lens, but that were before the lens parallel between them and to the optical axis ($p=+\infty$). Thus, given the central spherical symmetry of thin lenses it is simple enough to reverse in our imagination the previous picture to get the concept:
	\begin{figure}[H]
		\centering
		\includegraphics{img/electromagnetism/thin_lense_second_focal_point.jpg}
		\caption{Second focal point (target) is here on the right of the thin lens}
	\end{figure}
	Or for those who cannot see well the figure above, there is another more explicit illustration of what happen with the second focal point of a bi-convex lens:
	\begin{figure}[H]
		\centering
		\includegraphics[scale=0.8]{img/electromagnetism/second_focal_point_illustration.jpg}
		\caption{Second focal point with bi-convex lens behavior (credits: Pierre Toscani, \url{http://www.pierretoscani.com})}
	\end{figure}
	The distance from the lens to the image focus is then named the "\NewTerm{focal image distance}\index{focal image distance}" and we denote it in practice with the same letter $f$ as by symmetry of the thin lens, by putting $q=f$ and $p=+\infty$ we have:
	
	Therefore, in a thin lens the two foci are located symmetrically on each side of the vertical axes of symmetry of the lens.

	So as in both the inverse of the focal length is equal to the refractive power nothing prevents us to write the equation of thin lenses as we often found in textbooks:
	
	In this form it is then named the "\NewTerm{equation of Opticians}\index{equation of Opticians}" or "\NewTerm{eyewear equation}\index{eyewear equation}"... In the following simple form that does not make appear the physical properties of the lens (which is often the relation shon in high-school classes and could be the reason why it seems that it has a different name):
	
	we name it "\NewTerm{conjugation equation}\index{conjugation equation}" or "\NewTerm{second Descartes formula for thin lenses}\index{second Descartes formula for thin lenses}" or "\NewTerm{mirror equation}\index{mirror equation}".

	Furthermore, if the focal distance is positive, and thus respectively the dioptric power as well, then the lens is say to be (obviously) a "\NewTerm{convergent lens}\index{convergent lens}":
	\begin{figure}[H]
		\centering
		\includegraphics{img/electromagnetism/thin_convergent_lenses_examples.jpg}
		\caption{Examples of converging lenses (bi-convex, plano-convex, convex meniscus)}
	\end{figure}
	if the focal length is negative, and thus the refractive power respectively also, the lens is say to be a "\NewTerm{diverging lens}\index{diverging lens}":
	\begin{figure}[H]
		\centering
		\includegraphics{img/electromagnetism/thin_divergent_lenses_examples.jpg}
		\caption{Examples of different lenses (bi-convex, plano-convex, concave meniscus)}
	\end{figure}
	Note also the following traditional symbolic representations of lenses (for example in the E-draw software):
	\begin{figure}[H]
		\centering
		\includegraphics{img/electromagnetism/thin_divergent_lenses_examples.jpg}
		\caption{Examples of different lenses (bi-convex, plano-convex, concave meniscus)}
	\end{figure}
	Notice also the following traditional symbolic representations of lenses (for example in the E-draw software):
	\begin{figure}[H]
		\centering
		\includegraphics{img/electromagnetism/lens_technical_softwares_symbols.jpg}
		\caption[]{Examples of different lenses (bi-convex, plano-convex, concave meniscus)}
	\end{figure}
	
	\paragraph{Optical Magnification}\mbox{}\\\\
	Magnification is the process of enlarging something only in appearance, not in physical size. This enlargement is quantified by a calculated number also named "\NewTerm{magnification}\index{magnification}". When this number is less than one, it refers to a reduction in size, sometimes named ""\NewTerm{minification}\index{minification}" or ""\NewTerm{de-magnification}\index{de-magnification}".
	\begin{figure}[H]
		\centering
		\includegraphics[scale=0.6]{img/electromagnetism/magnification_lens.jpg}
	\end{figure}
	Typically, magnification is related to scaling up visuals or images to be able to see more detail, increasing resolution, using microscope, printing techniques, or digital processing. In all cases, the magnification of the image does not change the perspective of the image.
	\begin{figure}[H]
		\centering
		\includegraphics[scale=0.3]{img/electromagnetism/magnifying_glass.jpg}
		\caption[]{The stamp appears larger with the use of a magnifying glass (source: Wikipedia)}
	\end{figure}
	Before studying the magnification of spherical convex lenses, let us focus on the general definition of a magnification. For this purpose, consider the following figure:
	\begin{figure}[H]
		\centering
		\includegraphics[scale=1]{img/electromagnetism/magnification_principle.jpg}
		\caption[]{Magnification basic principle}
	\end{figure}
	where for recall the center of curvature $C$ (\SeeChapter{see section Differential Geometry}) is the center of curvature of the spherical surface of the figure above and the top O is the pole of the spherical cap.

	Thus, the "\NewTerm{magnification $M$}\index{magnification}" of any optical system is defined as the ratio of the size of the image $\overline{ab}$ to that of the real object $\overline{AB}$, that is to say:
	
	We see from the figure above that:
	
	We therefore get, taking into account that $\theta_i={\theta'}_r$:
	
	Hence the "\NewTerm{magnification equation}\index{magnification equation}":
	
	So in the case of a thin symmetrical spherical lens whose rays pass through the optical center we obtain rays which are describe similar triangles on each side of the axis of symmetry of the spherical thin lens then applying Thales theorem we also get:
	
	That is the same result as that obtained already above.
	\begin{tcolorbox}[colframe=black,colback=white,sharp corners]
	\textbf{{\Large \ding{45}}Example:}\\\\
	Two sides of a convex lens have a radius of $3$ [cm]. The index of the lens material is $1.52$. An object of $1.80$ [m] height is set to $14$ [m] distance of the lens (no matter that it is left or right of the lens because it is assumed as biconvex and therefore symmetrical). Then the dioptric power of the lens is first:
	
	Which is indeed a positive value and thus gives a focal length (image or object focal regardless because of symmetry of the spherical thin lens!) of	$f\cong 28.84\;[\text{cm}]$ so it is better to have a camera with a telephoto lens in the present case... We notice by the value of the focal length (home), that the subject is obviously beyond the focus. The position of the image is given by:
	
	Therefore:
	
	So the picture is about $29.42$ centimeters beyond the focus and will by definition be named the "\NewTerm{real (reversed) image}\index{real (reversed) image}". The magnification value then be:
	
	The size of the inverted real image will be in $q$:
	
	\end{tcolorbox}
	Therefore we have:
	
	If we want a magnification (zoom), we must have:
	
	Therefore:
		
	Is trivially:
	
	Finally:
	
	So to do that there is magnification (zoom) it is necessary that the real object is between the center of curvature and the vertical axis of symmetry of the lens.
	\begin{tcolorbox}[title=Remark,colframe=black,arc=10pt]
	The study of this model will allow us to understand partially how the prism works and also of goniometer in astronomy for spectrum analysis as well as X-ray diffraction by a network of atoms (the importance of the latter being quite significant!).
	\end{tcolorbox}
	The concave mirror bends light inwards, towards ourselves. If we imagine the mirror as a part of a larger sphere, then if we view the mirror from a point closer than the center of this sphere, we see an enlarged (magnified) upright image. If look further away fro the center we will see reversed image as illustrated by the photo below:
	\begin{figure}[H]
		\centering
		\includegraphics[scale=0.947]{img/electromagnetism/concave_mirror.jpg}
		\caption{Concave mirror (source: ?)}
	\end{figure}
	Notice above now the image of the girl with the striped shirt is right side up and magnified. She is inside the focal point. The person in the plaid shirt is outside the focal point, so it is upside down and about the same size. But the image from the back of the room is both upside down and smaller than normal, because it is much further away from the focal point!
	
	The same rules applies with a famous case know by all young children:
	\begin{figure}[H]
		\centering
		\includegraphics[scale=0.90]{img/electromagnetism/spoon_mirror.jpg}
		\caption{Left concave spoon-mirror, Right convex spoon-mirror (source: ?)}
	\end{figure}
	\begin{tcolorbox}[title=Remark,colframe=black,arc=10pt]
	Depending on the manufacturer a makeup mirror is concave or convex as they have the same structure as a spoon but at the difference that they have a very polish surface. But most of time the visible part is a concave mirror build in such a way that a person at $25$ [cm] away from the surface see its image doubled.
	\end{tcolorbox}
	For practical purposes, let us indicate the following figure (left global view, right zoom on the curve part):
	\begin{figure}[H]
		\centering
		\includegraphics[scale=1]{img/electromagnetism/full_planed_curved_lens.jpg}
		\caption{Full plane-curved lens (source: ?)}
	\end{figure}
	and therefore a full plane-curved lens, as a parabolic mirror, has the property of making parallel the rays having for source its focal. It produced by refraction the effect that the parabolic mirror produced by reflection...!

	Fresnel invented a lens that is seen in many lighthouses and that achieves the same result with less material:	
	\begin{figure}[H]
		\centering
		\includegraphics[scale=1]{img/electromagnetism/fresnel_lense.jpg}
		\caption{Fresnel lens with lighthouse example in the left bottom corner (source: ?)}
	\end{figure}
	\pagebreak
	So far it could be important to make a summary of the sign convention for the equation of opticians (or "mirror equation") and the magnification equation:
	
	therefore:
	\begin{itemize}
		\item Focal length $f$:
			\begin{itemize}
				\item Positive for concave mirrors or concave lenses
				\item Negative for convex mirros or convex lenses
			\end{itemize}
		\item Magnification $M$:
			\begin{itemize}
				\item Positive for upright images
				\item Negative for inverted images
				\item Enlarged when $|M|>1$
				\item Reduced when $|M|<1$
			\end{itemize}
		\item Image distance $q$:
			\begin{itemize}
				\item Positive for real images
				\item Negative for virtual images
			\end{itemize}
	\end{itemize}
	
	\begin{tcolorbox}[colframe=black,colback=white,sharp corners]
	\textbf{{\Large \ding{45}}Example:}\\\\
	E1. Consider that the Sun is the object, so the object distance is essentially infinity: $q=+\infty$. The desired image distance
is $q=40.0$ [cm] . We use the mirror equation to find the focal length of the mirror:
	
	But we have proved earlier that in the approximation of small angles:
	
	Thus, the radius of the mirror if considered as spheric, will be equal to:
	
	\end{tcolorbox}
	
	\begin{figure}[H]
		\centering
		\includegraphics[scale=0.6]{img/electromagnetism/parabolic_heat_collector.jpg}
		\caption{Parabolic trough collectors are used to generate electricity in southern California (source: Wikipedia)}
	\end{figure}
	
	
	\pagebreak
	\paragraph{Human eyes}\mbox{}\\\\
	Let us do some human biology for close this topic about lenses...

	The lens of the eye which can be deformed under the effect of certain muscles, constitutes a variable focus lens to accommodate variable distance vision of objects. The distance from the optical center to the retina $r$ being fixed (see figure below), the only way to clearly see objects at different distances $d$ is to change the focal length $f$. In its normal state, the lens has a relatively flat configuration, with a large radius of curvature (he was then a long focal length).
	\begin{figure}[H]
		\centering
		\includegraphics[scale=1]{img/electromagnetism/eye_optics.jpg}
		\caption{Human eye optics (source: ?)}
	\end{figure}
	The human eye has the function of focusing light from an object at infinity on the retina. But all eyes are not doing this correctly and the far point (maximum distance of distinct vision without accommodation) or "\NewTerm{punctum remotum}\index{punctum remotum}" is sometimes at a finite distance, sometimes even less than $5$ meters (probably causing eyestrain) as practitioners consider $6$ meters already as infinity relatively to the retina modification over this distance being most of time not significant.
	
	If the object approaches the eye, the muscles of the retina contract, the lens swells and its focal length decreases so that the image is always formed on the retina. The nearest point which can be seen clearly with maximum accommodation is named the "near point" or "\NewTerm{punctum proximum}\index{punctum proximum}". This distance is changing significantly with age: it is ten centimeters for a ten years old child, a hundred centimeters for a person of sixty years old (that is "presbyopia").
	
	\subsubsection{Triangular Prism}
	 a prism is a transparent optical element with flat, polished surfaces that refract light. At least two of the flat surfaces must have an angle between them. The exact angles between the surfaces depend on the application. The traditional geometrical shape is that of a triangular prism with a triangular base and rectangular sides, and in colloquial use "prism" usually refers to this type. Some types of optical prism are not in fact in the shape of geometric prisms. Prisms can be made from any material that is transparent to the wavelengths for which they are designed. Typical materials include glass, plastic and fluorite.

	A dispersive prism can be used to break light up into its constituent spectral colors (the colors of the rainbow). Furthermore, prisms can be used to reflect light, or to split light into components with different polarizations.
	\begin{figure}[H]
		\centering
		\includegraphics[scale=1.5]{img/electromagnetism/prism_david_parker.jpg}
		\caption{Triangular prism dispersing white light ray (source: Wikipedia, David Parker)}
	\end{figure}
	In optics, the triangular prism is one of the most important components of optics. It can be found in chemistry, condensed matter physics, astrophysics, optoelectronics and still many other popular devices of everyday life (such as lentils). This is probably the first tool shaped by human to make "spectroscopy" (spectrum analysis) after the rainbow sky obviously... which is a natural phenomenon spectroscopy.
	
	We will in the following paragraphs identify the most important relations in relation to know about triangular prisms and useful to the engineer and physicist.

	We will focus on light rays entering through one side and exiting by another having undergone two refractions (we do not study the internal reflections).

	Here is the typical schematic representation of a triangular prism in optical geometry with the incident light ray $S$ and outgoing one $S'$ and the two normals $N$, $N'$, to the edges of the top of opening angle $\alpha$. More the different angles of incidence and refraction:
	\begin{figure}[H]
		\centering
		\includegraphics[scale=1]{img/electromagnetism/triangular_prism.jpg}
		\caption[]{Schematic triangular prism}
	\end{figure}
	We know that the sum of the angles of a quadrilateral (always decomposable into two triangles whose the sum of the angles is $\pi$) is equals $2\pi$ (\SeeChapter{see section Geometric Shapes}). So in the quadrilateral defined by the nodes $1234$. We have the sum of:
	
	Now that the situation has know let us focus on the optical part ...

	We have $4$ fundamental relations to prove for to the triangular prism.

	First, we have at the point of incidence $I$ and $I'$ the Descartes' law that allows us to write:
	
	As the absolute refractive index of air is $1$ we simply have in $I$:
	
	In the same idea in $I'$ we have:
	
	and therefore:
	
	We also have the relation:
	
	Therefore:
	
	The deflection angle $D$ is easy to determine. We just have to take the central quadrilateral:
	
	Therefore:
	
	So we have the $4$ fundamental relations of the triangular prism:
	
	Knowing $i$ and $i'$ and the relative refractive index $m$ then we can determine all parameters.

	The ideal would still be able to get rid of the experimental knowledge of $i'$.

	So we have:
	
	But as:
	
	Therefore it comes:
	
	Therefore:
	
	Since it is obvious that that the index $n$ of a material varies with the wavelength according to Cauchy law introduce at the beginning of this section, it is easy to understand that the triangular prism is capable of dispersing white light.

	Finally if $i$ is small:
	
	and if $i$ and $\alpha$ are small, we have to first order in Maclaurin development (\SeeChapter{see section Sequences and Series}):
	
	Therefore:
	
	either by explicitly introducing Cauchy's law as introduced earlier in this section:
	
	
	\pagebreak
	\subsubsection{Pentaprism}
	The pentaprism roof that is enclosed  in most viewfinder of single-lens reflex cameras is an adaptation of the pentaprism, or "\NewTerm{optical square}\index{optical square}", invented in 1864 by Charles Moÿse Goulier.
	\begin{figure}[H]
		\centering
		\includegraphics[scale=0.66]{img/electromagnetism/pentaprism.jpg}
		\caption{Pentaprism roof (credits: Pierre Toscani, \url{http://www.pierretoscani.com})}
	\end{figure}
	A pentaprism is a five-sided reflecting prism used to deviate a beam of light by a constant $90^\circ$, even if the entry beam is not at $90^\circ$ to the prism (see proof below). The beam reflects inside the prism twice, allowing the transmission of an image through a right angle without inverting it vertically (that is, without changing the image's handedness) as an ordinary right-angle prism or mirror would.
	
	
	The reflections inside the prism are not caused by total internal reflection, since the beams are incident at an angle less than the critical angle (the minimum angle for total internal reflection). Instead, the two faces are coated to provide mirror surfaces. The two opposite transmitting faces are often coated with an antireflection coating to reduce spurious reflections. The fifth face of the prism is not used optically but truncates what would otherwise be an awkward angle joining the two mirrored faces.
	
	As we have show it, the camera lens renders an image that is both vertically and laterally reversed, and the reflex mirror re-inverts it leaving an image laterally reversed. In this case, the image needs to be reflected left-to-right as the prism transmits the image formed on the camera's focusing screen. This lateral inversion is done by replacing one of the reflective faces of a normal pentaprism with a "roof" section, with two additional surfaces angled towards each other and meeting at $90^\circ$, which laterally reverses the image back to normal. 
	
	In a precision optical device, when the path of a light beam must be deflected at $90^\circ$, the simplest solution is to use a plane mirror oriented at $45^\circ$ relatively the direction of the incident beam (see figure below). The mirror must be accurately positioned and stable because any angular positioning error induces a double error in the direction of the reflected beam: the reflected beam rotates twice as fast as the mirror (this is obvious as when the mirror in oriented at $\alpha=45^\circ$ the output beam is at an output angle of $2\alpha=90^\circ$ and therefore in general any modification $\delta$ implies a deflection of $2\delta$):
	\begin{figure}[H]
		\centering
		\includegraphics[scale=0.6]{img/electromagnetism/mirror_deflection.jpg}
		\caption{Mirror double deflection property (credits: Pierre Toscani, \url{http://www.pierretoscani.com})}
	\end{figure}
	But for the pentaprism it is different... It behave like a set square:
	\begin{figure}[H]
		\centering
		\includegraphics[scale=0.6]{img/electromagnetism/pentaprism_set_square_property.jpg}
		\caption{Mirror double deflection property (credits: Pierre Toscani, \url{http://www.pierretoscani.com})}
	\end{figure}
	Before we deal with the maths, the reader should be aware that this pentaprism has the following properties:
	 \begin{itemize}
		\item Two transparent faces (input or output) forming an angle of $90^\circ$
		 \item Two reflecting faces forming an angle of $45^\circ$ , separated by a chamfer (fifth face, no function)
		\item The both reflective surfaces have to be metallized to provide good reflection coefficient.
	\end{itemize} 
	\begin{figure}[H]
		\centering
		\includegraphics[scale=0.75]{img/electromagnetism/pentaprism_detailed_design.jpg}
		\caption{Pentaprism detailed design (credits: Pierre Toscani, \url{http://www.pierretoscani.com})}
	\end{figure}
	Now to prove that property consider the following figure:
	\begin{figure}[H]
		\centering
		\includegraphics[scale=0.75]{img/electromagnetism/pentaprism_set_square_property_proof.jpg}
	\end{figure}
	We have then looking this figure:
	
	That we will rearrange and write:
	
	Now in the above triangle we have:
	
	Therefore:
	
	On the left triangle we have:
	
	Therefore:
	
	That is:
	
	Finally in the last triangle we have:
	
	And therefore:
	
	That is:
	
	Finally:
	
	And now at the exit of the pentaprism we have:
	
	But as $\alpha_5=\alpha_2$ we have:
	
	Therefore it follows that:
	
	And this finish our proof!
	
	\pagebreak
	\subsection{Rainbow}
	A rainbow is an optical and meteorological phenomenon that makes the spectrum of light visible  when the Sun shines through the rain and the observer looks at the sky in a direction opposite to that of the Sun. It is a colored arc with red on the outside and purple inside.
	\begin{figure}[H]
		\centering
		\includegraphics[scale=0.15]{img/electromagnetism/rainbow.jpg}
		\caption[]{Rainbow after the rain, Grodno, Belarus (source: Wikipedia)}
	\end{figure}
	The rainbow is caused by the scattering of sunlight by raindrops approximately spherical. The light is first refracted by penetrating the surface of the drop, then undergoes a partial reflection at the back of the drop and is refracted again on leaving (see figure below). The overall effect is that the incoming light is mainly refracted back at an angle of approximately $40-42^\circ$  (see proof below), regardless of the size of the drop. The precise value of the refraction angle depends on the wavelength (color) of light components as we know. 

	In the case of entry in a refracting medium, the angle of refraction of the blue light is lower than that of red light (phenomenon highlighted in the study of triangular prisms). Thus, after reflection at the water-air interface, the blue light comes out of a drop above the red light (see again figure below). The observer being fixed (at rest), he sees light from different drops of water with different angles with respect to the sunlight. Red therefore appears higher in the sky than the blue.
	
	Sometimes a less bright second rainbow can be seen above the primary rainbow. It is caused by a double reflection of sunlight inside the raindrops and appears in a $50-53^\circ$ angle still in the opposite direction to the Sun. Due to the additional reflection, the colors of the second arc are reversed relative to the primary arc, with the blue outside and red inside, and the arc is less bright. That is why it is more difficult to observe. 
	\begin{figure}[H]
		\centering
		\includegraphics[scale=0.24]{img/electromagnetism/double_rainbow.jpg}
		\caption[]{Double rainbow and supernumerary rainbows on the inside of the primary arc. The shadow of the photographer's head on the bottom marks the antisolar point (source: Wikipedia, Eric Rolph)}
	\end{figure}
	A third rainbow sky may be present near the second, and inverted with respect thereto (ie, identical to the first)...
	
	However, third rainbow is much less bright and observable only under exceptional conditions. In practice it is not very easy to distinguish supernumerary arcs associated with the secondary arc. It corresponds to the light rays having undergone five reflections in the water drops. 

	Two rainbows inverted with respect to each other can also be observed in the opposite direction, about $45^\circ$ from the Sun (therefore in the direction of it), but this is particularly difficult because of the proximity to the Sun . The few observations of these two arcs mention pieces of rainbows visible intermittently. These two rainbows correspond to light rays having undergone three and four reflections in the water drops. As they are located facing the Sun, it is not the same drops of water that contribute to it. In practice, favorable configurations to their observation are much fewer than those who favor the observation of the secondary arc, especially because of their proximity to the Sun.

	To study the phenomenon let first consider the spherical water drop below with an incident beam of light (one refraction, two reflections) we represented the red and purple component (the angles are approximate) and the refractive indices water and air:
	\begin{figure}[H]
		\centering
		\includegraphics[scale=1]{img/electromagnetism/spherical_wather_drop_generating_rainbow.jpg}
		\caption[]{Spherical water drop generating the rainbow sky}
	\end{figure}
	We seek to determine the angle between the incoming light ray (beam considered to contain all components of visible light) and the outgoing light beam (opposite the Sun: antisolar). Thus, the angle difference for two colors give us the angle by which we must change our view to observe two different colors in the rainbow sky.
	\begin{tcolorbox}[title=Remark,colframe=black,arc=10pt]
	There is no sense to me to calculate the angle we must look at relatively with the ground (assumed plane) to observe a rainbow as some books mention it. Indeed, anyway if we turn our eyes to a rainbow sky will seen anyway over a large angle from the ground. The only thing that really makes sense, is the angle difference between two "monochromatic" colors.
	\end{tcolorbox}
	For this study, we will consider the following approximate figure:
	\begin{figure}[H]
		\centering
		\includegraphics[scale=1]{img/electromagnetism/rainbow_technical_study.jpg}
		\caption[]{Path traveled by light in a spherical water drop}
	\end{figure}
	With the Snell-Descartes law we have initially:
	
	What interests us here is the angle of apparent reflection $2\delta$, which we will denote $D$ (careful! Some teachers choose the convention $D=\pi-2\delta$). To determine this, we start from the following relation of the triangle $ABE$:
	
	Therefore it comes:
	
	Hence:
	
	What we will write finally:
	
	and as:
	
	Therefore it comes:
	
	which is sometimes abusively written:
	
	If we make a practical application we have for red $750$ [nm] with for example an incidence angle of $30^\circ$ in the spherical water drop:
	
	and for purple $400$ [nm] with the same angle of incidence of $30^\circ$ in the spherical water drop:
	
	Thus an angle difference of about $2.4^\circ$.
	
	Finally, we could interest us in the angle $\alpha$ at which the angle $D$ is maximum (which corresponds to the angle of the most visible rainbow sky in term of size in reality). We start then of:
	
	and we seek the solutions of:
	
	with $\alpha\in[0,\pi/2]$ and $n\in]0,1[.$

	Let us recall that we have proved in the section of Differential and Integral Calculus that:
	
	Then we have:
	
	Hence:
	
	From there we get:
	
	that is to say:
	
	as we seek for solutions with $\alpha\in[0,\pi/2]$ and that the cosine is positive on this interval.

	Therefore:
	
	He then comes for $n\cong0.746$ (purple):
	
	This corresponds fairly well to the reality (the angle at which we raise our heads to see the most visible / larger part of the rainbow).
	
	The corresponding deviation is then:
	
	and is named "\NewTerm{rainbow angle}\index{rainbow angle}" for purple:
	\begin{figure}[H]
		\centering
		\includegraphics[scale=1]{img/electromagnetism/rainbow_angle.jpg}
		\caption[]{Rainbow angle (source: Wikipedia)}
	\end{figure}
	For the blue we get:
	
	So the average gives the famous $42^\circ$ that we can read in many books.
	
	Thus, the light rays received by the observer and in which the red (outer edge of the rainbow) dominates correspond to all the rays coming from the wall of rain and making angle of about $40^\circ$ with the direction of the sunlight (see figure below). The light rays forming each color of the rainbow sky then form the top of the cone observer's eyes and for axis the solar ray passing through the eyes of the observer:
	\begin{figure}[H]
		\centering
		\includegraphics[scale=1]{img/electromagnetism/rainbow_schematic_summary.jpg}
		\caption[]{Figure of the generation of the primary rainbow (source: Culture ENS-Sciences)}
	\end{figure}
	
	\begin{flushright}
	\begin{tabular}{l c}
	\circled{90} & \pbox{20cm}{\score{4}{5} \\ {\tiny 48 votes,  76.25\%}} 
	\end{tabular} 
	\end{flushright}

	%to make section start on odd page
	\newpage
	\thispagestyle{empty}
	\mbox{}	
	\section{Wave Optics}
	\lettrine[lines=4]{\color{BrickRed}I}n this section will study some elements that led to the development of quantum mechanics. Indeed, quantum mechanics was born, first, by a careful study of the nature of light. Although this new science was developed in the early 20th century, the considerations which have guided to it are undoubtedly the result of 25 centuries of maturation. Basically, it is a long history of the light full of controversies to which quantum mechanics in the 20th century finally brings a masterful conclusion.
	
	Electron microscopes can make images of individual atoms, but why will a visible-light microscope never be able to? Stereo speakers create the illusion of music that comes from a band arranged in your living room, but why doesn't the stereo illusion work with bass notes? Why are computer chip manufacturers investing billions of dollars in equipment to etch chips with X-rays instead of visible light?

	The answers to all of these questions have to do with the subject of wave optics. So far this book has discussed the interaction of light waves with matter, and its practical applications to optical devices like mirrors, but we have used the ray model of light almost exclusively. Hardly ever have we explicitly made use of the fact that light is an electromagnetic wave. We were able to get away with the simple ray model because the chunks of matter we were discussing, such as lenses and mirrors, were thousands of times larger than a wavelength of light. We now turn to phenomena and devices that can only be understood using the wave model of light.

	\subsection{Huygens' principle}
	Huygens visualized the propagation of light as a result of the process of a generation of spherical wavelets in each point reached by a wavefront, wavelet whose sum gave the propagation field. By drawing the tangent to the wave fronts of the wavelet at a given time, one obtained the wavefront of the total wave at the same moment.
	\begin{figure}[H]
		\centering
		\includegraphics{img/electromagnetism/huyghens_schema.jpg}
		\caption{Illus­tra­tion of Huy­gens' con­struc­tion of a prop­a­gat­ing wave}
	\end{figure}
	We recall that a wave or "\NewTerm{wave front surface}\index{wave front surface}" (\SeeChapter{see section Wave Mechanics}) is the localization of points of the medium reach by the wave motion at the same instant. The disturbance has therefore the same phase at any point of a wavefront. For a plane wave, for example, the disturbance is expressed by (we have prove this in the section of Wave Mechanics):
	
	or in a more general way:
	
	which therefore gives the expression of the propagation of the disturbance for which the "\NewTerm{wavefront}\index{wavefront}" is the locus of points where the phase $\vec{k}\circ\vec{r}-vt$ has the same value at a given instant. The wave surface is given accordingly by the equation:
	
	Huygens gave a pictorial representation method of the passage of a wave surface to another in the case where the wave is assumed to result from the movement of the particles constituting the material medium. Thus, if we consider the wave surface $S$ below:
	\begin{figure}[H]
		\centering
		\includegraphics{img/electromagnetism/huygens_wave_propagation.jpg}
		\caption{Representation of a wavefront according to Huygens}
	\end{figure}
	When the wave motion reaches this surface, each particle $a, b, c, ...$ of the surface becomes in turn a wave source, emitting secondary waves (indicated by the small half-circles in the figure above) that reach the next layer of particles of the medium. These particles are put in motion and form the new wave surface $S'$ and so on... So, Huygens had a wave conception of light, but he did not consider the periodic nature of the wave, which not allow him to introduce the concept of light of color (frequency); moreover, according to its principle, a wave propagating in the opposite direction to that of the incident wave should also occur, which is not the case in a homogeneous material...
	
	The intuition of Huygens is however close to reality as will show if Fresnel in its diffraction theory (see further below). However, it was Kirchhoff, which will introduce a tilt factor (obliquity) in the theory, to gen an explanation of the absence of wave propagating backward (when the time will come we will write related mathematical developments).
	
	As on the previous figure all the "corresponding points" $(a,b,c,\ldots),(a',b',c',\ldots)$ are  equidistant, by Huygens' principle, the time interval between corresponding points of two wavefronts is the same for any two corresponding points!
	
	The consequences are (refer simultaneously to the figure below):
	\begin{figure}[H]
		\centering
		\includegraphics{img/electromagnetism/huygens_time_interval_principle.jpg}
		\caption[]{Schematic diagram}
	\end{figure}
	If we denote the propagation velocities of the incident rays $R_1$, $R_2$ by $v_1$ and $v_2$ respectively, we have:
	
	\begin{itemize}
		\item When the wave propagates in a homogeneous medium, the light rays must be straight and the wave surfaces remain parallel.

		\item When the wave change of medium, the distance between two pairs of corresponding points vary from one medium to another, if the propagation velocities are different.
	\end{itemize}
	\begin{theorem}
	Let us prove that this principle allows to find the Descartes-Snellius law that we have already proven in the section of Geometric Optics, which ensures a priori that the Huygens principle is still valid in the context of Geometric Optics.
	\end{theorem}
	\begin{dem}
	Following the figure above, we have:
	
	by dividing each term by $\overline{a_1b_2}$, we get:
	
	so we fall back well on the Descartes-Snellius law as we had obtained it in the section of Geometric Optics:
	
	noting on the way that on the diagram, we also have $\theta_1={\theta'}_1$.
	\begin{flushright}
		$\square$  Q.E.D.
	\end{flushright}
	\end{dem}
		
	\pagebreak
	\subsection{Fraunhofer Diffraction}
	From the point of view of geometrical optics, a light beam is a cylinder of section $\Sigma$ which brings together a large number of parallel rays. It is therefore assumed to be straight when it is set in a homogeneous medium.
	
	The "\NewTerm{energy emittance}\index{energy emittance}" $M \; [\text{Wm}^{-2}]$ of the beam varies only if a lens (or another device) make vary its section $\Sigma$ or if the medium absorbs energy.
	
	The light beam "bursts" when an obstacle allows only a portion of its section $\Sigma$.
	
	Huygens' principle shows that it is the edges of the obstacle that generate this diffraction.
	
	The phenomenon is general but is well observed only if the ration $L/\Sigma$ is very large. $L$ being the length of the edges. This condition is necessary so that the intensity of the undiffracted portion of the beam does not mask the effect.
	
	\begin{enumerate}
		\item[D1.] We speak of "\NewTerm{Fraunhofer diffraction}\index{Fraunhofer diffraction}" when, as assumed above, the incident light rays are parallel and the phenomenon observed at relatively large distance from the screen.

		\item[D2.] We speak of "\NewTerm{Fresnel diffraction}\index{Fresnel diffraction}" when the incident rays form a divergent beam from a point source or if we observe the phenomenon at close range.
	\end{enumerate}
	
	Let us consider a generic case and in fact the most widespread case in the physics laboratories which is the diffraction by a narrow rectangular aperture!
	
	For this, we consider that the incident beam, perpendicular to the aperture, has a plane electromagnetic periodic wave front and is given by (\SeeChapter{see section Wave Mechanics}):
	
	where for recall, its wavelength is given by:
	
	
	\subsubsection{Case of a rectangular aperture}
	The span width $e$ of the aperture is oriented along the $y$ axis, the height $h$ (parameter that can not be represented in the figure because it is a view from above) is assumed as very big so we can neglected the border effects.

	Following the Huygens principle , the front of the plane wave, delimited by the aperture, is a multitude of sources $\mathrm{d}f(x,t)$, of width $\mathrm{d}y$, which emit in phase, spherical wavelet described by their associated field vector:
	
	Now consider an observation point $P$, at a distance $R$ from the source (assimilated to an aperture). We have proved during our study of the emission sources of spherical type  (\SeeChapter{see section Electrodynamics}) that their amplitude decreases inversely proportional to the distance as:
	
	However, the wavelets, each depending on the point of the aperture to which it is assimilated, will not all travel the same distance $R$ but a distance proper distance $r$. However, if $R$ is sufficiently distant from the aperture, we will allow ourselves to approximate:
	
	remains still the periodic term $\sin(kx-\omega t)$ where we put $x=r$. But we have for extremal values:
	
	These extremal values correspond respectively, to the  advance and the delay of the wave function describing the propagation of the wavelet at the borders of the aperture.
	
	Indeed, it is enough to see the figure below, considering $R \gg e$ and therefore:
	
	\begin{figure}[H]
		\centering
		\includegraphics{img/electromagnetism/fraunhofer_rectangular_slot.jpg}
	\end{figure}
	Thus, by placing the origin of the $y$ coordinate in the middle of the aperture, we have:
	
	So the different wavelets are out of phase and produce interferences.
	
	\textbf{Definition (\#\mydef):} In wave mechanics, we speak of "\NewTerm{interference}\index{interference}" to describe a phenomenon in which two waves superpose to form a resultant wave of greater, lower, or the same amplitude. Interference usually refers to the interaction of waves that are correlated or coherent with each other, either because they come from the same source or because they have the same or nearly the same frequency. Interference effects can be observed with all types of waves, for example, light, radio, acoustic, surface water waves or matter waves.
	
	The diffracted wave in the direction of $\theta$, is given by the sum of all contributions:
	
	Knowing that (trigonometric relation are explicited on request of a reader):
	
	We have therefore:
	
	We have proved in the section Electrodynamics that the energy (i.e. the intensity) of an electromagnetic wave was given (in vacuum) by the average scalar value of the Poynting vector:
	
	We therefore have by considering that the magnetic and electric field are proportional to the term:
	
	the following result:
	
	which is the luminous emittance emitted in the direction $\theta$ and where have put:
	
	If we introduce the sinus cardinal sinc that we have meet in our study of Fourier transforms in the section of Sequences and Series and that was defined in the section Trigonometry then we can write the previous relation in a more clearly condensed way:
	
	For which can have a generic form plotted with Maple 4.00b using:
	
	\texttt{>Gamma:=3;plot((sin(Gamma*x)/(Gamma*x))\string^2, x=-Pi..Pi);}
	
	So we can get the same result by taking the squared modulus of the Fourier transform of a monochromatic signal through a rectangular window. Thus, it seems possible to study the phenomena of diffraction by using the Fourier transform, and this area is named "\NewTerm{Fourier optics}\index{Fourier optics}" that we will study later.
	
	Here is a graphical representation of the relation $\bar{S}/\overline{S_0}$ for different values of the ratio $\Gamma=e/\lambda$:
	\begin{figure}[H]
		\centering
		\includegraphics{img/electromagnetism/fraunhofer_rectangular_slot_diffraction_plot.jpg}
		\caption{Representation of diffraction fringes}
	\end{figure}
	On either side of the central fringe, there are other fringes, more narrow and arranged symmetrically. Their intensity decreases very rapidly as the preponderant  term at the denominator:
	
	\begin{figure}[H]
		\centering
		\includegraphics{img/electromagnetism/fraunhofer_rectangular_slot_diffraction_plot_zoom.jpg}
	\end{figure}
	from which we have a real image here:
	\begin{figure}[H]
		\centering
		\includegraphics{img/electromagnetism/fraunhofer_rectangular_slot_real_photo.jpg}
		\caption{Photo of a real Fraunhofer diffraction fringe of a rectangular aperture}
	\end{figure}
	Between the fringes, there are dark areas that are the seat of destructive interference. Their position is given by the condition:
	
	excepted for $n=0\Rightarrow \theta=0$ where we observe a maximum!
	
	So we see black bands in the directions:
	
	Thus the angular width of the central band  is twice the angular value obtained for the first minimum:
	
	We get the width of the following peaks as follows:

	Two successive minima satisfy the conditions:
	
	Therefore:
	
	Put putting:
	
	Therefore it comes:
	
	Since the emittance decreases very rapidly, only the first bands (for which $\cos(\theta)\cong 1$) are observable. Therefore it remains:
	
	The positions of the maxima are then given by the condition:
	
	Let us put:
	
	The numerical resolution of:
	
	gives:
	
	The positions of the successive maxima are then:
	
	etc.
	We could have also easily get a decent approximation of this result, considering that the intensity is maximum when:
	
	Which bring us to write:
	
	with $n\in \mathbb{N}^{*}$.
	
	A remarkable result of the Fraunhofer experience is that it challenges the corpuscular view of light as we had in the 19th century.

	Indeed, many experiments such at the projection of the shadow of an object on a wall seemed to show that the light was such a particle that does not pass through materials and being net stopped by any obstacle either in its center or by its edges (you must draw your attention to the "edges" in particular).

	But, Fraunhofer experience and in particular that of Fresnel regarding the edges (we will see further because it is mathematically more difficult to study), show that the light appears to behave not as a single particle but as a wave (from the Huygens principle that we used for our developments) as we showed to us the earlier developments that perfectly explain the experimental results of Fraunhofer diffraction.

	But then why keep the corpuscular model of light? Simply because of other experimental and theoretical results of which the best known are the photoelectric effect or Compton scattering (\SeeChapter{see section of Nuclear Physics}), which can be explained theoretically quite well if it is not perfectly with a corpuscular  model light (and also with some other particle of different size, charge, spin, etc.).
	
	By the early 1800s, there were two theories about the nature of light. One of them, going back to Newton, is that light is a ray. But there was another idea – going back even farther (to Christiaan Huygens) – that light might also be a wave. And this gained a lot of support in 1799, when Thomas Young first passed light through two thin, nearby slits.  So if you're a good theorist, and you're interested in studying light, what do you do?
	
	Well, if you're famed French mathematician and physicist Simeon Poisson, you would think of the most ridiculous configuration you could imagine in the hopes of disproving the light-is-a-wave theory. And that's exactly what he did in 1818.

	He imagined that you took a wave source of light, and had it shine on and around a completely black, spherical obstacle, setting up a screen behind it. Obviously, he reasoned, you would see some light on the screen indicating the outside of the sphere, and darkness, or a shadow, on the inside.

	But, he calculated, if the wave theory of light were correct, you would get something completely absurd!

	Sure, you'd get light on the outside, and shadow on the inside, but what's that at the very center? Poisson predicted, using the wave theory of light, that you'd actually get a bright spot of light at the very center of this shadow! How absurd was that! And therefore, he reasoned, the wave theory of light was absolutely crazy, and had to be wrong.

	Shortly after Poisson's prediction, Francois Arago decided to put the theory to the test, and actually performed the experiment to look for the "theoretically absurd" spot.

	And what happens if, in fact, you perform this experiment yourself? :
	\begin{figure}[H]
		\centering
		\includegraphics{img/electromagnetism/arago_spot.jpg}
		\caption{Arago Spot (source: Thomas Bauer at Wellesley)}
	\end{figure}

	Amazingly, the spot is real! If your theory is any good, scientifically, this is exactly what it will do. It will not only explain what's already been observed, it will allow you to apply it to new situations, and make testable predictions about what you can expect to find. The crazier the prediction, and the more successful the experiment, the more compelling the theory becomes.
	
	\paragraph{Optical resolution}\mbox{}\\\\\
	According to the criterion of the English physicist Lord Rayleigh: the "\NewTerm{optical resolution}\index{optical resolution}" (or "\NewTerm{power angular resolution}\index{power angular resolution}") of a circular aperture, is the angle $\alpha_{\min}$ between two light rays of wavelength $\lambda$, emitted from two point sources $S_1$,$S_2$, far away, whose diffraction figures are separated such that the first zero of the diffraction pattern is found at the place of the maximum of the other as illustrated below:
	\begin{figure}[H]
		\centering
		\includegraphics{img/electromagnetism/optical_resolution.jpg}
		\caption{Optical resolution as defined by Lord Rayleigh}
	\end{figure}
	Or with a photo of experiment perhaps it's better:
	\begin{figure}[H]
		\centering
		\includegraphics{img/electromagnetism/optical_resolution_photo.jpg}
	\end{figure}
	This concept is used extensively in photography, astronomy, radio astronomy, etc. So the reader should pay a special attention to it!
	
	Now we have proved that the minima for a rectangular aperture were given by:
	
	And if we take the case where $n=1$, we fall back on the relation available in many books without proof:
	
	named "\NewTerm{Rayleigh criterion of a rectangular aperture}\index{Rayleigh criterion of a rectangular aperture}".
	
	Therefore the angular resolution of a rectangular aperture is proportional to the ratio of the wavelength $\lambda$ to the thickness $e$ of the slot. Obviously in practice the goal is to have either:
	
	\begin{enumerate}
		\item A large as possible value of the angle $\alpha_{\min}$ so that the objects (both source in this case) does not coincide and that their images are therefore distinct.

		\item A small as possible value  so that the sole observed object (only one source) do not have a blurry image. This is one reason why many popular books say you have to have a smaller wavelength than the observed object so that it is observable (so not too be blurred!).
	\end{enumerate}
	To increase the resolving power to deacrease the blur effect, so you have to be working with shorter wavelengths or increase the thickness of the slot of the instrument, as the wavelength is often imposed by the observed object, it is natural to make vary $e$.
	
	Therefore if the light that passes through a slot (rectangular or circular!) forms an image on a screen, and that this image is observed under a microscope, for example, it is impossible, whatever the magnification of the microscope ,to observe more details in the image than it is allowed by the resolving power of the slot. We must take these considerations into account when we design of optical instruments.
	
	\subsubsection{Case of a network of rectangular apertures}
	Let us now consider an array of $N$ apertures of width $e\cong \lambda$, of height $H \gg e$ and distant each other by $d$. A single incident beam illuminates all apertures.
	\begin{tcolorbox}[title=Remark,colframe=black,arc=10pt]
	The study of this model will allow us to understand partially how the prism works and also of goniometer in astronomy for spectrum analysis as well as X-ray diffraction by a network of atoms (the importance of the latter being quite significant!).
	\end{tcolorbox}
	Given the following figure:
	\begin{figure}[H]
		\centering
		\includegraphics{img/electromagnetism/network_of_rectangulare_apertures.jpg}
		\caption{Basic principle of the rectangular aperture network}
	\end{figure}
	We see in the diagram above that for some directions $\theta$, the distance $d\sin(\theta)$ is such that constructive or destructive interference are occurring.
	
	Let us consider that this network is placed in the $YZ$ plane and the beam direction is along the $X$ axis. We put ourselves in an observation point $P$ in the $XY$ plane. Following the properties of electromagnetic waves (\SeeChapter{see section Electrodynamics}), the electric field vector $\vec{E}_i$ of the wave emitted by the $i$-th slot is perpendicular to the direction of observation and can be expressed as:
	
	and we also proved earlier above that:
	
	where by analogy between these two relations, we recognize that:
	
	and therefore it comes:
	
	In any direction $\theta$, the waves issued from the two adjacent apertures are out of phase of $d\sin(\theta)$ (where $d$ is the step value between two apertures) and at the point $P$ of observation, the resulting electric field is given by the sum of the contributions of each aperture with its own $i$-th shift. Therefore:
	
	So we see that each wave is out of phase of:
	
	We can now represent $E(\theta,R,t)$ using phasors (\SeeChapter{see section Wave Mechanics}) in the phase space such that:
	
	Which gives graphically for the second term containing the summation variable $j$ for a fixed distance $R$:
	\begin{figure}[H]
		\centering
		\includegraphics{img/electromagnetism/phasor_fresnel_apertures.jpg}
		\caption[]{Representation of the summation of terms}
	\end{figure}
	We see that the $\vec{E}_i$ put together form a regular polygon inscribed in a circle of radius:
	
	The norm of the resulting electric field being equal to the chord defined by the angle:
	
 	we will have:
	
	The light energy (ie. intensity) emitted in the direction $\theta$ is proportional to the square of the electric field (\SeeChapter{see section Electrodynamics}), then we have for destructive or constructive interference:
	
	We now substitute $\overline{S}_i(\theta)$ by the expression found in our earlier study of diffraction by a single aperture::
	
	Thus we get to the addition of interference and diffraction effects:
	
	Although this relation seems complicated, its parameters do not have the same practical significance. Indeed, consider the function:
	
	The term $A$ has maxima when:
	
	and is equal to zero if:
	
	with $m\in \mathbb{N}$.
	
	Although the term $B$ makes the relation diverge for $x=m\pi$, the Hospital's rule (\SeeChapter{see section Differential and Integral Calculus}) provides that:
	
	It follows that for $x=m\pi$ and therefore for zero values of $A$ and $B$, the function $\psi$ presents enormous peak height:
	
	Since the huge amplitude, the main peaks are those we observe experimentally the more easy. Thus, the angular position of the maxima of the function $\psi$ is given by:
	
	with $m\in\mathbb{Z}$.

	The value $m$, designate the "\NewTerm{order number of the maximum's interference}\index{order number of the maximum's interference}".

	Let us apply these results to the interference relation:
	
	The peak of order $m$ is center on the equivalent value $\theta$ that cancel the numerator and the denominator of this fraction such that:
	
	with $m\in\mathbb{Z}$, hence:
	
	Thus an aperture network that we know the step value $d$ can be used to measure the wavelength $\lambda$ of an incident unknown light beam.
	
	However, if the incident light is polychromatic (typically for astronomical observations), the above relation gives us for a given wavelength given the position of the interference fringes. Thus, an astronomer passing polychromatic light of its telescope through a diffraction network can make a spectroscopic analysis of the light.

	This relation also gives us that for fixed values of $m$ and $d$ that the greater is $\lambda$ the greater is the angle $\theta$ is in a range $[0,2\pi]$. Thus, the spectral lines arising from the impact of a polychromatic beam show a spectrum from violet (shorter wavelength so small angle) to red (high angle so large wavelength).

	Using a goniometer, we measure the angles $\theta_m$ of the main peaks of order $m$ for the greatest possible number of $m$ values. We deduce from these measurement the slope $\lambda$ of the function plot:
	
	The peak bottom is located at an angle $\theta_m+\Delta\theta_m$ where the numerator:
	
	cancel for the first time after passage of the peak.
	
	Since the argument of this function increases of $m\pi$ between two successive peaks (among all main and secondary peaks), it is equal to $N(m\pi)$ at the place of the peak of order $m$ (main peak therefore) and has to travel $\pi$ radians further to reach the bottom of the peak.

	The numerator is then:
	
	The angular distance $\Delta \theta_m$ between the top and the bottom of the main peak is the given by:
	
	But from the first order, we have $\theta_m \ll \Delta\theta_m$. The difference of two sinus gives (\SeeChapter{see section Trigonometry}):
	
	A Maclaurin development (\SeeChapter{see section Sequences and Series}) gives when taking the first term of the development $\sin(\alpha)\cong \alpha$:
	
	But we also have the identity (\SeeChapter{see section Trigonometriy}):
	
	Hence the angular width of a peak of order $m$:
	
	But as:
	
	Therefore:
	
	It is clear that two superposed light lines will be seen as distinct if they are separated by an angular distance equal to their angular width. The expression:
	
	establishes that to two angular positions correspond two wavelengths. So we can give the separation of two light lines by $\Delta \lambda=\lambda_2-\lambda_1$ instead of $\Delta\theta=\theta_2-\theta_1$.
	
	Therefore from:
	
	we get:
	
	But:
	
	When $\Delta \theta_m$ and $\theta_m$ are small, we have:
	
	Which brings us to write by substitution:
	
	The resolving power $R$ of a network is its ability to separate two spectral lines of neighboring wavelengths $\lambda$ and $\lambda+\Delta\lambda$ such as:
	
	We see that the resolving power increases proportionaly with the diffraction order. This relation is very important in spectroscopy for astronomy. We can also calculate what should be the minimum number $N$ of lines that must have a network capable of separating two wavelengths in a spectrum of given order $m$.
	
	\subsubsection{Young's interference experiment}
	According to the wave-particle duality principle, light behaves both as a wave and as a particle (material particle). It is the solving of problems like those of the black body (\SeeChapter{see section Thermodynamics}), of the photoelectric effect (\SeeChapter{see section Nuclear Physics}) or that of the Compton effect (\SeeChapter{see section Nuclear Physics}) , which revealed the existence of this duality.

	But we will now study the most mind blowing proof the duality aspect of the matter at the atomic scale using the Yong's experiment. We will address this in a simplified manner as a special case of the rectangular slots network that has the advantage of being able to be easy to put in practice in an experiment showing the dual and probabilistic behavior of matter at the atomic scale.

	So to study this subject, let us consider a light source $S$, which radiates monochromatic wave $\Psi$ of wavelength $\lambda$ through two slots $F_1$ and $F_2$ perforated  in an opaque barrier to light, as shown in the figure below:
	\begin{figure}[H]
		\centering
		\includegraphics{img/electromagnetism/young_experiment.jpg}
		\caption{Implementation of Young's experience-slit}
	\end{figure}
	
	\begin{tcolorbox}[title=Remark,colframe=black,arc=10pt]
	The advantage of that device is that it can produce two sources of coherent light. That is to say two sources whose phase difference is constant throughout the experiment.
	\end{tcolorbox}
	We have on a viewing (projection) screen $E$ at a point $H$ such that the distance is:
	
	where $a$ would typically be of the order of the millimeters and $D$ of the meter.
	
	The wave $\Psi$ will after passing through the slots $F_1$ and $F_2$, as we have already seen it just previously, generates sub-waves $\Psi_1(r_1,t)$ and $\Psi_2(r_1,t)$ of same pulsation $\omega$ that will travel respectively the paths $r_1$ and $r_2$ and that will go interfere at the point $M$ of the screen $E$ that we are interested for.
	
	If the interference is constructive in $M$, this point will then be located on a bright fringe and if the interference is destructive $M$, it will be on a dark fringe. To observe this, let us first write the resulting wave at $M$:
	
	in which we have in terms of phasor (\SeeChapter{see section Wave Mechanics}):
	
	where $A$ is the amplitude, $k$ is the wave vector and $t$ is the time variable as we have already discussed in detail in the section of Wave Mechanics.

	Now let make a change of variable (just to not have to manage long exponential functions):
	
	\begin{tcolorbox}[title=Remark,colframe=black,arc=10pt]
	We will see later that in fact $D_1=r_1$ and $D_2=r_2$.
	\end{tcolorbox}
	To calculate the intensity at the point $M$, we will take the complex norm (module) of $\Psi_M$ that is thus written as the product of the complex variable and its conjugate (\SeeChapter{see section Numbers}):
	
	\begin{tcolorbox}[title=Remark,colframe=black,arc=10pt]
	This calculation is very important because the analogy with the Wave Quantum Physics is very strong at this level and similar to the calculation of the probability amplitude (\SeeChapter{see section Wave Quantum Physics}).
	\end{tcolorbox}
	Therefore:
	
	The intensity is then maximum if and only if:
	
	So that:
	
	with $n\in\mathbb{N}$. Which gives:
	
	\begin{tcolorbox}[title=Remark,colframe=black,arc=10pt]
	It is here that we see obviously that $D_1=r_1$ and $D_2=r_2$
	\end{tcolorbox}
	The intensity is zero if and only if:
	
	So that:
	
	with $n\in\mathbb{N}$. Which gives:
	
	Now we must calculate $r_1-r_2$ in function of $z$ to find out what we see on the projection screen $E$.
	
	For this let us consider the following figure:
	\begin{figure}[H]
		\centering
		\includegraphics{img/electromagnetism/magnification_and_special_case_of_young_experiment.jpg}
		\caption[]{Magnification and special case of young experiment}
	\end{figure}
	where $\overline{F_1P}$ and $\overline{F_2P}=r_2$.
	
	We have on our figure:
	
	But:
	
	So we have:
	
	As $z$ and $a$ are small relatively to $D$ and using the approximation:
	
	if $\varepsilon$ is small compared to $1$. Then we have:
	
	Also:
	
	So subtracting those two relations we get:
	
	So finally using the relation:
	
	It comes:
	
	Thus, the distance between two consecutive maxima is:
		
	and is named "\NewTerm{interfringe}\index{interfringe}".

	For fringes with zero intensity it comes immediately:
	
	This relation indicates that the intensity $I$ has minima (dark fringes) and maxima (bright fringes) distributed along the $z$ direction periodically. This does not surprise us more than that for now because it derives from the more general case studied above.
	\begin{figure}[H]
		\centering
		\includegraphics[angle=90,origin=c,scale=0.35]{img/electromagnetism/interference_fringe.jpg}
		\caption[]{Interfinge simplified schema for the Young experiment}
	\end{figure}
	It should be notice that the above calculations show that the intensity of the fringes is equal everywhere. But we observe experimentally (see figure above) that their intensity decreases with distance from the projection screen center. As we have already seen it, two are at the origin of this observation:
	\begin{enumerate}
		\item The slots have a given width, which implies a diffraction phenomenon. Indeed, a light sent on a small hole does not go out isotropically. This results in that the light is predominantly directed forward. This effect is reflected in the figure observed after Young's slots: the intensity of fringes decreases gradually as we moves away from the center of the projection screen.

		\item The fact that the waves emitted in $F_1$ and $F_2$ are spherical waves, that is to say, their amplitude decreases gradually as we go away from the screen projection center. Thus the amplitude of $F_1$ and $F_2$ will not be the same at the point $M$.
	\end{enumerate}
	So our calculations are remains approximate relatively to the study we made of the rectangular slots network but that is how Young's slots experiment is presented in most schools and this is enough to highlight the main result.
	
	The original experiment of Thomas Young can be interpreted using the simple Fresnel's laws as we have done with the network of rectangular of slots. Which highlights the wave nature of light. But this experience has subsequently be refined, particularly by ensuring that the source $S$ emits a quantum at a time. For example, it is possible in the third quarter of the 20th century to emit photons or electrons or atoms one by one. These are detected one by one placed on the screen after the double-slit. We see then that these impacts form gradually the interference pattern. By conventional laws concerning the trajectories of these particles, it is impossible to interpret this phenomenon!!! Hence the importance of the theoretical study of the Young's experiment!!!
	\begin{figure}[H]
		\centering
		\includegraphics[scale=0.55]{img/electromagnetism/young_experiment_perspective.jpg}
		\caption{Young’s interference experiment (source: ?)}
	\end{figure}
	From left to right, top to bottom, here are the patterns obtained by accumulating $10$, $300$, $2,000$ and $6,000$ electrons with a flow of $10$ electrons / second (the same experiment with the same result has been reproduced with neutrons and atoms!). The accumulation of electrons eventually form interference fringes which is quite confusing a priori!
	\begin{figure}[H]
		\centering
		\includegraphics{img/electromagnetism/young_real_experiment.jpg}
		\caption[]{Photos of patterns obtained in a real Young's experience (source: ?)}
	\end{figure}
	We will come back on this crucial phenomenon in the section of Wave Quantum Physics to say a little bit more about it.
	
	For summary here is another more detailed schema of what we haven seen so far:
	\begin{figure}[H]
		\centering
		\includegraphics[scale=0.6]{img/electromagnetism/young_real_experiment_detailed_summary.jpg}	
		\caption{Young experiment summary (source: OpenStax)}
	\end{figure}
		
	\pagebreak	
	\subsection{Light polarization}
	It was not before the 19th century that it was discovered the polarization of light (we will immediately explain what it is). However, at the time of Newton, we already knew a phenomenon due to polarization: the existence of crystals named "\NewTerm{birefringent crystals}\index{birefringent crystals}" (as the "Iceland spar") which have the property of refracting a single beam into two separate beams
	\begin{figure}[H]
		\centering
		\includegraphics[scale=0.7]{img/electromagnetism/icelan_spar.jpg}
		\caption{Icelan spar (crystalized calcite)}
	\end{figure}
	Now we know that the two rays refracted by such a crystal are polarize:
	\begin{figure}[H]
		\centering
		\includegraphics[scale=0.6]{img/electromagnetism/birefringent_crystal_polarization_principle.jpg}
	\end{figure}
	To understand the "polarization of light" let us come back to the case of a wave propagating on a string (\SeeChapter{see section Wave Mechanics}). Such a wave can do it in a vertical plane (right) as well as in a horizontal plane (left) or in all intermediate levels:
	\begin{figure}[H]
		\centering
		\includegraphics{img/electromagnetism/polarized_wave_illustration.jpg}
		\caption{Pictorial representation of the light polarization concept (source: ?)}
	\end{figure}
	In both cases, we say that the wave is "\NewTerm{linearly polarized}\index{linearly polarized}", which means that the oscillations are only and always in the same plane, named "\NewTerm{polarization plane}\index{polarization plane}". Such a wave can pass through a vertical slot if it is polarized vertically, a horizontally polarized wave can not.

	Let us recall that we have seen in the section of Electrodynamics that for electromagnetic waves, the electric $\vec{E}$ field oscillates (at least for the standard solution of Maxwell's equations) and is orthogonal to the direction of propagation.
	
	The electric field vector $\vec{E}$ of a wave can be decomposed into two perpendicular components to each other, $(\vec{E}_x,\vec{E}_y)$ if the wave propagates in the $z$ direction and each carrying half of the intensity of the wave. These two components change at any time when $\vec{E}$ varies. The result at any time is a total horizontal field and a total vertical field.
	\begin{figure}[H]
		\centering
		\includegraphics{img/electromagnetism/electric_field_wave_propragation_decomposition.jpg}
		\caption{Illustration of the decomposition of the electric field of a propagating wave (source: ?)}
	\end{figure}
	If $\vec{E}$ revolves around the propagation direction with its end describing circle, then we say that the wave is "\NewTerm{circularly polarized}\index{circularly polarized}":
	\begin{figure}[H]
		\centering
		\includegraphics{img/electromagnetism/circular_polarized_wave.jpg}
		\caption{Pictorial representation of a circular polarized wave (source: ?)}
	\end{figure}
	Then $\vec{E}$ remains of constant modulus but turns while moving, making one complete turn for each distance equal to one wavelength $\lambda$.
	\begin{tcolorbox}[title=Remark,colframe=black,arc=10pt]
	The light is not necessarily polarized! Every atom emits a wave train that lasts less than one hundred-millionth of a second (these wave trains are perfectly explained by the spread of a the free particle in quantum physics with the Fourier transforms as proved in the section of Wave Quantum Physics) and all these waves have no phase correlation or orientation. The resultant field in a given position in space, is the geometric sum of all these wave trains: it is constantly changing.
	\end{tcolorbox}
	Thus, the natural light is a random and quickly variable mixture of linearly polarized waves in all directions. Looking to the source we observe a resulting field $\vec{E}$ oscillating in a certain direction for a fraction of time and then suddenly jumps to a new random direction while remaining perpendicular to the direction of propagation:
	\begin{figure}[H]
		\centering
		\includegraphics{img/electromagnetism/natural_light.jpg}
		\caption{Pictorial representation of "natural" light wave (source: ?)}
	\end{figure}
	This introduction done, let us move now to something a bit more formal:

	We therefore saw in the section Electrodynamics that a hypothetical progressive monochromatic plane wave (even if physically it cannot exist ...) propagating in the vacuum consisted of an electric field $\vec{E}$ and a magnetic field $\vec{E}$ and was characterized by its pulsation $\omega$ (or corresponding wavelength $\lambda$), its electric field amplitude $\hat{E}$ and magnetic field amplitude $\hat{B}$ and its propagation direction given by a unit vector $\vec{u}_x$, $\vec{u}_y$, $\vec{u}_z$ at choice depending on the orientation of the selected reference frame.
	
	We have also proved that these waves have remarkable structural properties, in particular:
	\begin{itemize}
		\item $\vec{E}$ and $\vec{B}$ are transverse, that is to say, their direction is at any point and at any time orthogonal to the direction of propagation (Malus theorem). This, allow us to define a wave plane, plane generated by the two directions of $\vec{E}$ and $\vec{B}$.

		\item The norms of these two vectors are connected by the relations $\hat{B}=\hat{E}/c$ where $c$ is the speed of light in vacuum (this is this huge ratio between the magnetic field and electric field of an electromagnetic wave that makes developments presented further below are done preferably compared with the electric field component of the wave).

		\item Finally, these two vectors are orthogonal to each other, and the trihedron $(\vec{E},\vec{B},\vec{u}_i)$ is an orthogonal trihedron .
	\end{itemize}
	These three properties can be summarized as we have prove it by the relation:
	
	where we have chose the reference frame such as the wave propagates in the direction $\vec{u}_z$. Furthermore, we proved that the electric field is a trigonometric wave function given by an the arbitrary phase by:
	
	Let us now position ourselves in a base $(x, y, z)$. The most general expression of the electric field of a progressive monochromatic plane wave progressing in the $\vec{u}_z$ direction can be decomposed into two components:
	
	\begin{figure}[H]
		\centering
		\includegraphics[scale=0.8]{img/electromagnetism/wave_decomposition.jpg}
	\end{figure}
	The norm of the field is therefore given by:
	
	If $\varphi_x=\varphi_y$ (which is most often the case) we have then:
	
	By choosing another beginning of time, we are lead to write:
	
	with:
	
	\begin{tcolorbox}[title=Remark,colframe=black,arc=10pt]
	The choice of writing $-\varphi$ instead of $+\varphi$ will be useful later for the use of remarkable trigonometric relations and will allow us to find the equation of an ellipse (...patience the proof is not far anymore).
	\end{tcolorbox}
	Using the phasors (\SeeChapter{see section Wave Mechanics}) the last relations can be reduced to:
	
	However, to describe this field, and therefore the whole wave, it is convenient to put ourselves in the plane $z=0$ and describe the evolution of the vector in $\vec{E}$ in this plane. That's what we'll do next. This is equivalent as to choose a coordinate origin following the $z$-axis. In this case, we can write:
	
	But the most general polarization is described by a complex vector normalized to unity in a two-dimensional space of components:
	
	with:
	
	such that:
	
	
	
	\subsubsection{Linear polarization}
	\textbf{Definitions (\#\mydef):} We say that a wave is "\NewTerm{linearly polarized}\index{linearly polarized}" when $\varphi=0$ or $\varphi=\pi$.

	In the first case ($\varphi=0\Rightarrow \varphi_x=\varphi_y$), we have:
	
	
	Therefore, we have $\hat{E}_{0x}$,$\vec{E}_{0y}$ which have respectively values between:
	
	\begin{tcolorbox}[title=Remark,colframe=black,arc=10pt]
	We will see that the fundamental constant characterizing quantum physics (like the speed of light characterizes relativity) is the Planck constant and deeper the fine structure constant.
	\end{tcolorbox}	
	With respect to a diagram that we will see below it should be taken into account that when a component is positive the other is also positive, and vice versa.

	We then have at every instant:
	
	which means that the field keeps a fixed direction. Hence the fact that we speak of linearly polarized wave.

	If $\varphi=\pi$ we then have:
	
	and that can therefore be written:
	
	Therefore, we have $\hat{E}_{0x}$,$\vec{E}_{0y}$ which have respectively values between:
	
	We then have at every instant:
	
	which means that the field keeps a fixed direction. Hence the fact that we speak of linearly polarized wave.
	
	\subsubsection{Elliptical polarization}
	If $\varphi$ is arbitrary, and placing ourselves at $h=0$, we always have by starting from:
	
	The first relation:
	
	as well as:
	
	hence:
	
	Moreover, we can write:
	
	In squaring the two previous relations:
	
	and by summing, we eliminate the time and get:
	
	
	We notice that if $\varphi=0$, $\varphi=\pi$ we fall back on:
	
	By the way, in the prior previous relation we recognize the equation of an ellipse (\SeeChapter{see section Analytical Geometry}):
	
	similar in all respects to the general equation of the conical we have proved in the section of Analytical Geometry that was for recall:
	
	In this case, the end of the vector $\vec{E}$ thus describes an ellipse and we talk therefore naturally of "\NewTerm{elliptical polarization}\index{elliptical polarization}".

	According to the value of $\varphi$, the ellipse may be traversed in one direction or the other. To determine this direction, let us derivate the expression of the field:
	
	with respect to time and let us put ourselves at $t=0$ still in the same wave plane in $z=0$:
	
	Therefore:
	\begin{itemize}
		\item If $0<\varphi <\pi/2$ the ellipse is traveled in the forward direction (counter clock wise) as shown in the figure below. We say then that the polarization is "\NewTerm{elliptical direct left}\index{elliptical direct left}".

		\item If $\pi/2<\varphi <\pi$ the ellipse is traveled also in the forward direction (counter clock wise) as shown in the figure below. We say then that the polarization is "\NewTerm{elliptical direct right}\index{elliptical direct right}".

		\item If $\pi<\varphi<3\pi/2$ the ellipse is traveled clockwise. We say then that the polarization is "\NewTerm{elliptical indirect right}\index{elliptical indirect right}".

		\item If $3\pi/2<\varphi<2\pi$ the ellipse is traveled clockwise as shown in the figure below. We say when the polarization is "\NewTerm{elliptical indirect left}\index{elliptical indirect left}".
	\end{itemize}
	
	\subsubsection{Circular polarization}
	If:
	
	and:
	
	we then then the equation of the ellipse which reduces to
	
	which is the equation of a circle of radius $E_{0x,y}$, the sign being always given by the sign of the sine:
	\begin{itemize}
		\item If $\varphi=\pi/2$ it is a "\NewTerm{left circular polarization}\index{left circular polarization}"

		\item If $\varphi=3\pi/2$it is a "\NewTerm{right circular polarization}\index{right circular polarization}
	\end{itemize}
	....see figure below for a visual summary.
		
	\subsubsection{Natural polarization}
	We can consider the emission of a source as a succession of progressive theoretical monochromatic plane waves whose expression will be:
	
	These wave trains are in a particular polarization state. However, this states varies randomly from one wave train to another, and this in a very short time relative to the integration time of detectors. They therefore will see no particular bias and the field $\vec{E}$ will have no particular direction.

	We speak therefore of "\NewTerm{unpolarized light}\index{unpolarized light}". If we overlay this light to a polarized wave, we get what we name a "\NewTerm{partial polarization}\index{partial polarization}".
	
	Finally, we can summarize what we have seen so far in the following figure where we have:
	\begin{itemize}
		\item The linear polarization: $\varphi=0,\varphi=\pi,\varphi=2\pi$

		\item The linear partial polarization (not represented below)

		\item The direct elliptical polarization left $0<\varphi<\pi/2$ or right $\pi/2<\varphi<\pi$

		\item The indirect elliptical polarization left $\pi<\varphi<3\pi/2$ or $3\pi/2<\varphi <2\pi$

		\item The partial elliptical polarization (not represented below)

		\item The circular polarization left $\varphi=\pi/2$ or right $\varphi=3\pi/2$

		\item The partial circular polarization (not represented below)
	\end{itemize}
	\begin{figure}[H]
		\centering
		\includegraphics[scale=1]{img/electromagnetism/various_polarizations.jpg}
		\caption{Representations of different polarizations}
	\end{figure}
	
	We can represent this lively in with an animated plot in Maple 4.00b (we have not put the *.gif below as don't wanter to overload the PDF too much...) using the following commands:
	
	\texttt{> with (plots):\\
	> Ex:=1;Ey:=1;phi:=Pi/4;k:=1;omega:=1;\\
	> animate3d([x,a*Ex*cos(omega*t-k*x),a*Ey*cos(omega*t-k*x-phi)],\\
	a=0..1,x=-10..10,t=0..2*Pi,frames=15,grid=[35,35],style=patchnogrid,axes=boxed);}
	\begin{figure}[H]
		\centering
		\includegraphics[scale=1]{img/electromagnetism/wavefront_polarization_animation_maple.jpg}
		\caption{Animation of a polarized wave with Maple 4.00b}
	\end{figure}
	It is of course possible to change the parameters. For example, $\phi=\pi/2$ gives circular polarization, $\pi=\pi$ gives a linear polarization as we have prove above.

	We can also put the figure of Wikipedia which also summarizes very well the subject:
	\begin{figure}[H]
		\centering
		\includegraphics[scale=1]{img/electromagnetism/polarized_wave_illustration_wikipedia.jpg}
		\caption{Visual polarized wave principle summary as proposed by Wikipedia}
	\end{figure}
	
	\pagebreak
	\subsubsection{Malus' law}
	To polarize light, the physicist will use polarizers. We will not enter (because it is not part of the field of Wave Optics) in the details of atomic or molecular properties of matter that are the cause of the polarization of the transmitted light.

	For our purposes, we will restrict ourselves to a polarizer that polarizes incident light linearly along the $x$ axis (the component $E_y$ being therefore zero). We have therefore:
	
	But, we have proved in the secton of Electrodynamics during our study of Maxwell equations that:
	
	Therefore it comes for the maximum intensity (such that $e^{\mathrm{i}(\omega t-kz)}=1$):
	
	relationship that is the famous "\NewTerm{Malus' law}\index{Malus' law}".

	To investigate quantitatively the polarization, we will use a polarizer / analyzer set . We first let light pass through a polarizer whose axis makes an angle $\theta$ with the $x$-axis, then pass through a second polarizer, referred to as "analyzer", whose axis makes an angle $\alpha$ with the same axis (see figure below) with:
	
	whose norm is equal to unity!
	\begin{figure}[H]
		\centering
		\includegraphics[scale=0.6]{img/electromagnetism/polarizer_analyzer_set_principle.jpg}
		\caption{Simplified example of a polarizer / analyzer set}
	\end{figure}
	At the exit of the analyzer, the electric field $\vec{E}'$ is obtained by projecting the linear polarized light ($\varphi=0$) obtained at the exit of polarizer:
	
	with:
	
	on $\vec{n}$ (which means: projection = dot product, to get a vector we multiply for recall by the vector on which we do the projection):
	
	We the deduce the Malus law for the intensity:
	
	for the case of a linear polarizatioon of course. We will reuse that result for ou study of quantum cryptography (\SeeChapter{see section Cryptography}).
	
	
	\subsection{Coherence and interference}
	We will now see what are the conditions such that a plane waves interfere with each other. These developments help to explain  well much about the vision of the world around us through our eye (especially why all the waves received by our retinas do not mix and so the colors either!).

	Let us consider two plane waves $\Sigma_1$ and $\Sigma_2$ of pulsation $\omega_1$ and $\omega_2$ and of wave vectors $\vec{k}_1$ and $\vec{k}_2$ propagating both parallel to the $z$-axis.

	We denote by $\Psi_1$ and $\Psi_2$the complex amplitudes of the two waves and we are interested in the average intensity observed at a point O taken as the coordinate origin:
	\begin{figure}[H]
		\centering
		\includegraphics[scale=1]{img/electromagnetism/plane_wave_coherence_incoherence.jpg}
		\caption{Representation of plane waves}
	\end{figure}
	We put:
	
	and we will assume:
	
	At the point O the complex amplitudes are written:
	
	where $\theta_1$ and $\theta_2$ represent the phases of $\Psi_1$ and $\Psi_2$.
	
	Let us now calculate the instantaneous intensity at the point O will be denoted $J(t)$. As the average intensity $I$ is proportional to the square of the amplitude, we assume it will be the same for the instantaneous intensity. Which brings us to calculate the sum of the real parts of the amplitudes of the two waves:
	
	What is written keeping in mind that (\SeeChapter{see section Numbers}):
	
	Therefore:
	
	And then we have:
	
	It follows the sum of four terms:
	
	To calculate the average intensity, we will choose an experimental approach:

	The average intensity of the exposure time $\tau$ of the detector (electronic or biological) will be given by:
	
	$I$ is then the sum of the averages of the four terms intervening in $J(t)$ given just above. For visible light (in the case of our eye), the frequencies are of the order of $10^{15}$ [Hz] and the exposure time of the detectors vary between the millisecond and the second. Then $\tau$ typically contains $10^{12}$ periods of $\Psi_1$ and $\Psi_2$!!

	Let us examinate the effect on the average value of each of the term of $J (t)$ by first recalling the relation (\SeeChapter{see section Wave Mechanics}):
	
	\begin{enumerate}
		\item We have using the usual integral proved in the section of Differential and Integral Calculus, the following valuation of the integrand on an important number of periods:
		
		Calculation that is traditionally written and very abusively under the following condensed form:
		
		We can estimate that a large number of periods (opening time of the detector), it is this average that will be measured (in fact its really that latter!).
		
		\item  We have identically:
		
		with the same remarks as above regarding to the detector!
	
		\item For the third term it is a bit different:
		
		However, the average of a cosine and a sine on a period is zero. So if the detector makes a measure on an exposure time above $1/2\omega_0$, that is on a large number of periods, we will have:
		
		
		\item For the fourth term it is still different in the experimental approximation. Indeed:
		
		But, $\delta\omega \ll 2\omega_0$. Therefore the detector does not have time to measure the average intensity over a whole period in first approximation since:
		
		and that this value is much much greater in the visible spectrum that the opening / sampling time of the eye that it is $0.1$ [s].
	
		Thus, we will notice the average of the fourth term by:
		
	\end{enumerate}
	The average intensity is therefore in an experimental context:
	
	or:
	
	If the pulsations $\omega_1$, $\omega_2$ are equal (or almost equal), then it is the interference between two monochromatic plane waves. The average intensity is then written:
	
	When we know that the eye interprets the intensity to form perceptions of the objects we understand why two objects of two different colors are not a perception corresponding to a mixture of the two colors because even if in the visible spectrum, the pulses are almost equal, their phase is rarely at a given point in space zero such that:
	
	There is therefore no interference and we have in reality:
	
	\begin{tcolorbox}[title=Remark,colframe=black,arc=10pt]
	During the composition of several waves, we can always consider that there is interference. However, we name "\NewTerm{interference conditions}\index{interference conditions}" the conditions of observation of these interferences, ie conditions for the result of their composition to be stable enough to be observed. It is customary to speak of "visibility", which restricts to the single observation by the (human) eye.
	\end{tcolorbox}
	We have seen for the eye that the sampling time frequency is $10\;[\text{s}^{-1}]$. Knowing that the visible light has a frequency of $f\cong 10^{14}\;[s^{-1}]$, the frequency must then be stabilized by the source during:
	
	which is materially impossible except that the source is the same. We conclude that for interference to be visible to the eye, the sources must be synchronized at best at $10^{-13}$ which in practice leads to consider only sources absolutely synchronized on a single source.

	In the previous model, we also have neglected the fact that a real wave is limited in time. A photon is represented by a limited wave packet. Given $T$, it will have a length $l_c=cT$ in a vacuum or in the air that we name "\NewTerm{temporal coherence length}\index{interference conditions}".

	A given radiation is thus a superposition of of succession of wave trains whose average length is $l_c$, the successive wave trains have no phase relation between them: they can not interfere!
	
	\pagebreak
	\subsection{LASER}
	The conceptual contribution of Albert Einstein to light-matter interaction is essential. His approach, which was to introduce the concept of absorption and stimulated emission of radiation, is the origin of the LASER emission process (Light Amplification by Stimulated Emission of Radiation) used in many areas such as: read / write optical discs, satellite data transmission, conferences pointers, precision measuring devices (telemetry, anemometry), surgery (eye, hair removal, gyrometry, cutting or industrial welding, micromachining, priming nuclear reactions (megajoule  LASER) prototyping (3D printers), cleaning surfaces (ablation / LASER etching), structure analysis, cooling, etc.

	By analyzing the thermal equilibrium conditions in the interaction of electromagnetic radiation with matter, Albert Einstein understood that taking into account the spontaneous emission permits only to found the Wien's distribution law (\SeeChapter{see section Thermodynamics}). However Planck's law (\SeeChapter{see section Thermodynamics}) can not be obtained if we assume the existence of a stimulated emission process. This is where Einstein's work germ contain the development of coherent electromagnetic radiation sources.

	Indeed, the first proofs of a coherent emission of radiation have involved specialists in atomic physics to those of electromagnetism. This gave birth to the MASER (Microwave Amplification by Stimulated Emission of Radiation).

	It is the evolution of this multifaceted tool and the response that it has be able to bring to problems that contributed to the development of optoelectronics, optronics and biophotonics.
	
	\begin{tcolorbox}[title=Remark,colframe=black,arc=10pt]
	We will see that the fundamental constant characterizing quantum physics (like the speed of light characterizes relativity) is the Planck constant and deeper the fine structure constant.
	\end{tcolorbox}
	I have hesitated during a long time to put the basic developments of the LASER in the section of Statistical Mechanics, or Thermodynamics or Electrodynamics. As in my private circle the majority of people (not scientific) associated the LASER with optics, I thought it was more appropriate to present the below developments in this section of Optical Wave.
	
	We will, as did Albert Einstein, show by contradiction that consider only stimulated absorption and spontaneous emission are insufficient to fall back on the law Planck proved in the section Thermodynamics (law which describes the radiation density at equilibrium in a cavity of finite dimension):
	
	Let us represent schematically what are stimulated absorption and spontaneous emission disregarding the notations of energy levels as used in the sections of Corpuscular Quantum Physics, Wave Quantum Physics and Quantum Chemistry:
	\begin{figure}[H]
		\centering
		\includegraphics[scale=1]{img/electromagnetism/laser_absorption_emission.jpg}
	\end{figure}
	In our gedankenexperiment, $N_\text{tot}$ atoms are divided into the populations $N_1$ and $N_2$ respectively on two energy levels $E_1$ and $E_2$.

	We observe a resonant absorption of electromagnetic radiation when the frequency $\nu$ of the radiation is equal to the energy difference between the two considered levels. The energy emitted or absorbed is then linked to the relation:
	
	The absorption rate obviously depends on the spectral energy density of the incident electromagnetic field $R(\nu,T)$, of the population $N_1$ of the lower level and finally it is probably proportional to a factor $B_{12}$ that would give the the properties of the atomic system. Under these assumptions, we are led to write naturally:
	
	The phenomenon of stimulated resonant absorption is superimposed the fact that an excited system returns to its initial state with a characteristic time: the lifetime of the excited state. The de-excitation rate is proportional to the population of the upper level of the transition such as this brings us to write:
	
	The conservation of the total number of atoms:
	
	interacting with the radiation is reflected by the kinetic condition (it is just the derivative with respect to time of the preceding expression):
	
	By injecting previous relationships wisely, we get:
	
	Therefore (if we fall back on Planck's law, our assumptions for the gendankenexperiment will then be checked!):
	
	The population of levels at thermodynamic equilibrium following a Maxwell-Boltzmann distribution law (\SeeChapter{see section Statistical Mechancis}):
	
	We therefore have:
	
	So we see well that we will never be able to fall back on Planck's law like this:
	
	with our assumptions. Of course, by choosing the constant well, we fall back on the Wien's law (\SeeChapter{see section Thermodynamics}) but it has since the problems we already know.

	So to get the Planck's law we must have missed something or the approach is totally false! To keep it simple ... if we observe a long time the result we just get, we find that we could possibly reach our goal if the relation arising from the conservation of the number of atoms had one more term (actually, simply from the Planck's law and to developments in reverse: reverse engineering). This is the brilliant idea that had Albert Einstein and allowed the creation of the theoretical concept of LASER.

	Let us assume therefore that in addition to the spontaneous  emission and absorption, we have the concept of "\NewTerm{stimulated emission}\index{stimulated emission}" that is really not intuitive (but that appears when the developments are done in reverse):
	\begin{figure}[H]
		\centering
		\includegraphics[scale=1]{img/electromagnetism/laser_stimulated_emission.jpg}
	\end{figure}
	Therefore, in the spontaneous emission, the photon can be emitted in any direction. In the stimulated emission, we recognize (figure) two photons in the same direction, the incident photon causing the emission and the emitted photon. By filtering processes, it is then possible to obtain a monochromatic beam which can be used for power transmission, information or measurement.

	Therefore in the presence of radiation of the same frequency as that of the transition, the system has a certain probability of being desexcited to regain its fundamental or initial state:
		
	We therefore have the following kinetic equation:
	
	becomes:
	
	Therefore:
	
	Which brings us to:
	
	and as:
	
	then it comes:
	
	and then we see that to get the Planck's law:
	
	We just have to put:
	
	Bingo! The coefficients are named by tribute: "\NewTerm{Einstein coefficients}\index{Einstein coefficients}".
	
	For a system in thermal equilibrium, the lowest level of a transition is always more populated than the upper level. Accordingly, the medium behaves as an absorbent in the presence of incident radiation of the same frequency as that of the transition between the two states. However, if this equilibrium is changed so that the upper level is significantly more populated than the lower level, the system facilitates the stimulated emission process. We thus obtain an optical amplifying mechanism, which is associated with the population inversion produced through a technique named "\NewTerm{optical pumping}\index{optical pumping}".

	It must be indicate that the LASER may be made of gas mixtures, liquids doped with rare earths, semiconductor materials (laser diodes), crystals or glasses doped with active ions.

	Also let us mention that through this model, Albert Einstein was the first to introduce the probabilities in quantum physics in 1916. This led Max Born ten years later to advance a probabilistic interpretation of the Schrödinger's wave function (\SeeChapter{see section Wave Quantum Physics}).
	
	\begin{flushright}
	\begin{tabular}{l c}
	\circled{90} & \pbox{20cm}{\score{3}{5} \\ {\tiny 30 votes,  64.00\%}} 
	\end{tabular} 
	\end{flushright}

\chapter{Atomistic}

	\textit{\textbf{The atomic physics is the part of physics that deals with quantified energy states of corpuscular and wave particles and of exchange of energies within the atom}}. (Larousse)
	\minitoc
	\pagebreak
		%to make section start on odd page
	\newpage
	\thispagestyle{empty}
	\mbox{}
	\section{Corpuscular Quantum Physics}\label{corpuscular quantum physics}
	\lettrine[lines=4]{\color{BrickRed}N}ow it is time to plunge into the dark and impenetrable waters of atomic physics. It goes without saying that we will cover the theories of atomic physics only in the outline. In facts, we will limit ourselves only to the theoretical developments made between the years 1910 and about 1935 (beyond the complexity of theories requires too many pages for a general book like this one). We will also pass on many mathematical details that have already been proven and checked in other sections of this book.
	
	Atomic physics as you probably already know is the world of the infinitely small (points of zero dimension). This is a world, you'll see, almost special where fairly classical laws, those that govern our everyday macroscopic, does not apply in an intuitive way.

	Thus, in the early 20th century we knew only that atoms were formed by a simple core and electrons in orbit.

	The electron, the first subatomic particle (smaller than the atom) to be revealed, was showed to us by experiments on electrical currents in solids, liquids and gases. In the 19th century, physicists had no idea what was the charge, if it was continuous or particulate. Today we know that the charge seems to be a property of matter and that the total load in a system seems to be a multiple of an elementary charge corresponding to the charge of an electron (or proton).

	Michael Faraday suggested by electrolysis experiments that electricity consisted of particles of elementary charge $e$ and that a mole of such charges (see Chemistry section for the definition of the mole) was equivalent to a load of 1 Faraday that is to say $96,485\;[\text{C}]$. As Avogadro's number was not known at the time, it was not possible to determine e. However, one mole of a monovalent substance that can carry$ 1\;[\text{F}]$ charge, it had to follow only half a mole of the same substance would carry $1/2\;[\text{F}]$ and so on until the most smallest unit of charge $e$, which was to be carried by the smallest unit of mass $m$, corresponding to the mass of a single atom of the substance. In 1881, Helmholtz stated that if one accepts the hypothesis that the elementary substances were composed of atoms, we must logically conclude that electricity, both positive and negative, should be divided into finite portions that should behave like electricity atoms. George Stoney named this fundamental unit charge "\NewTerm{electron}\index{electron}". The elementary charge value is named prosaically today "\NewTerm{quantum charge}\index{quantum charge}".

	All subatomic charge known today whether positive or negative, carry a net charge that is an integer multiple of $e$. Quarks have a fractional charge but they do not appear as isolated entities. There are also fractional charges in the quantum Hall effect, but this is another story...

	Even today, some of the best physicists say that they don't really know what are an electron or even an atom. In fact, we still do not know what really what matter is (matter is energy but what is energy??)... The only thing that interest the scientist anyways is to have a mathematical tool that explains thinks quite well its field of interest and the interpretation in comparison with the sensible reality is not the main priority...

	Scientists have tried the development of several models to explain the experimental  observations of the microscopic world. Thus, there has been in the order for the must well know: the models of Dalton, Thomson, Rutherford, Bohr, Schrödinger, Sommerfeld (the latter including the major contributions of Heisenberg, de Broglie, Pauli, Dirac and Einstein for the most famous) and the standard model.

	We can situate the birth of the corpuscular quantum physics or simply "\NewTerm{quantum physics}\index{quantum physics}" ("quantum" meaning "fixed amount") in 1900, when Max Planck presented his famous paper on black body radiation (\SeeChapter{see section Thermodynamics page \pageref{black body}}) at a meeting of the German physical Society and the inability of classical physics (mechanics, thermodynamics, electromagnetism) to explain certain behavior of matter at the microscopic level, that is to say at the level of phenomena where particles of small masses located in very small regions of space are the main object for study.

	To arrive to a coherent interpretation of these experiments, it was necessary to introduce concepts radically different from those of classical physics. For example, we had to abandon the notion of classical trajectory, adopt the  quantification of energy (Planck's law) and consider that microscopic particles sometimes have a behavior similar to a wave. All these new concepts gave birth to a new physics, "quantum physics", which has developed rapidly since  in 1927, already, the foundations of the theory were completed. By abandonment of the key concepts of classical mechanics, we can say that quantum physics is a revolution (named also the "second revolution," the first being the theory of relativity) in the way of interpreting experimental measurements. With relativity introduced by Einstein, quantum physics is one of the pillars of the theoretical edifice of modern physics in the 21st century.

	Just as relativity contains classical mechanics as a limiting case (relativistic laws approaching classical laws when the velocity of a particle is sufficiently low compared to that of light), the new quantum physics contains also in the limit case the classical laws of statistical mechanics or even electromagnetism.

	\begin{tcolorbox}[title=Remark,colframe=black,arc=10pt]
	We will see that the fundamental constant characterizing quantum physics (like the speed of light characterizes relativity) is the Planck constant and deeper the fine structure constant.
	\end{tcolorbox}	

	\subsection{Dalton's model}

In 1803, John Dalton made the assumption that matter is composed of atoms of different masses and combine respecting simple mass proportions (though the idea of atom was really not new, it dates from thousands years before!). It is this theory that Dalton proposed that is the cornerstone of modern physical science. In 1808, Dalton's work entitled \textit{A new system of chemical philosophy} was published. In this book he listed the atomic weights of a number of known elements relative to the mass of hydrogen. His masses "AMU" (\SeeChapter{see chapter of Nuclear Physics}) were not entirely correct, but they are the basis for the modern periodic table of elements. Dalton arrived at his atomic theory through a study of the physical properties of atmospheric air and other gases.

Dalton assumed that the atom was a sphere:

\begin{figure}[H]
\centering
\includegraphics{img/atomistic/dalton_model.eps}
\caption{The ideal approach of Dalton}
\end{figure}

Thus, he could make a first estimate of the size of atoms:

Indeed, either $\rho$ the typical density, $m_A$ the atomic mass and $R$ the radius (unknown value) of an element which we seek to determine the size of the atom. We then have very simply:
	
Dalton knowing $\rho$ and  $m_A$ thanks to experimentation, he get:
	

\subsection{Thomson's model}

Thomson is at the origin of the discovery of the electron by his experiments on the flow of particles (electrons) created by cathode rays. Theoretician and experimenter, Thomson advanced in 1898 the "raisin bread theory" of atomic structure, in which the electrons are considered negative grapes pressed into a loaf of positive material. His model of the atom is represented by the figure below:

\begin{figure}[H]
\centering
\includegraphics{img/atomistic/thomsom_model.eps}
\caption{Thomson's Gourmet approach...}
\end{figure}

But, we know (the physicists of the 19th century also knew it) that no arrangement of static charges is stable if these charges are under the influence of the Coulomb force:
	
that we had studied in detail in the section of Electrostatic. Thus that this requires that the particles that make up the atom are in motion which leads us to develop another model: the following "Rutherford model".

\subsection{Rhuterfords's model}\label{rhuterford model}

Thus Rutherford assimilated intuitively by this theoretical fact, a few years after the discovery of Thomson, the atom to a planetary system whose center was occupied by a nucleus of positive charge, which contained almost the entire mass of the atom. The core was a hundred thousand times smaller than the atom and thus occupied a very small volume. The assumptions was more adapted to experimental results:
	\begin{figure}[H]
		\centering
		\includegraphics[scale=0.55]{img/atomistic/rutherford_thomson.jpg}	
		\caption[Rutherford experiment]{Rutherford experiment (source: OpenStax)}
	\end{figure}
Therefore here is a pictorial representation of how should looks like an atom in this model (the distances are not at scale for obvious reasons...):

\begin{figure}[H]
\centering
\includegraphics{img/atomistic/rutherford_model.eps}
\caption{The Rutherford's planetary approach...}
\end{figure}

Rutherford applied the results we get in astronomy (\SeeChapter{see section Astronomy page \pageref{conicity law}}) during our study of Keplerian orbits in the atom and therefore obtained conical trajectories for the rotation of the electron around the nucleus such as:
	
where $e$ is the eccentricity ($e=c/a<1$) and $p$ the focal parameter ($p=b^2/a$) of an ellipse (\SeeChapter{see section Analytic Geometry page \pageref{focal parameter of the ellipse}}), and where:
	

	\begin{tcolorbox}[title=Remark,colframe=black,arc=10pt]
	\textbf{R1.} You will have to remember that when we will discuss later on Bohr's model that in the Rutherford model, $r$ can take theoretically any value!\\

	\textbf{R2.} We will see in our study of the Rutherford scattering (\SeeChapter{see section of Nuclear Physics page \pageref{rutherford scattering}}) that Rutherford determined that the size of the gold atom as being worth $R\cong 3\cdot 10^{-14}\; [\text{m}]$. So we have a model with a factor 10,000 in comparison of the Dalton's model (that is to say...).
	\end{tcolorbox}	

But we have proven that during our study of electromagnetism that Maxwell displacement equations (third Maxwell equation) were given by (\SeeChapter{see section of Electrodynamics page \pageref{third maxwell equation}}):
	
and:
	
describe that an electron in motion (accelerated) emits energy in the form of electromagnetic radiation that we call in physics "Bremsstrahlung" explained by the Liénard-Wiechert's potentials (\SeeChapter{see section Electrodynamics page \pageref{linear wiechard potentials}}).

Rutherford and Thomson were therefore faced with the following dilemma:

If the electron emits energy in the form of electromagnetic radiation, it loses kinetic energy (speed) and thus necessarily sooner or later (except external intervention) will fall on the core of the atom (illustration of the phenomenon in the figure below). But we that material surrounding us is in fact stable.
\begin{figure}[H]
\centering
\includegraphics[scale=0.75]{img/atomistic/bremsstrahlung.eps}
\caption{Simplistic illustration of Bremsstrahlung}
\end{figure}
So they rejected their model and Bohr intervened then with a revolutionary hypothesis but that had to explain facts and not necessarily true reality... (Bohr was a genius experimentalist physicist  and probably the best debate partner of Einstein).

\subsection{Bohr's Model}\label{bohr model}

In 1913, Niels Bohr, who participated to the work of Rutherford on the diffusion $\alpha$ of particles (nuclei of two proton and two neutrons free of electrons), takes the model of Rutherford but includes in it three fundamental assumptions:

\subsubsection{Bohr's Postulates}\label{bohr postulates}

	\begin{enumerate}
		\item[P1.] The electron does not emit radiation when on certain orbits named "\NewTerm{stationary orbits}\index{stationary orbits}". This assertion is contrary to the theories of classical electrodynamics. So this implies that all the orbits are not allowed and this is a revolution in the approach of theoretical physics (the prohibited orbits are named "\NewTerm{non-stationary orbits}\index{non-stationary orbits}").
		
		\item[P2.] On any stable orbit the linear momentum  $p$ integrated on the path $r$ of the orbit is an integer multiple of the Planck's constant $h$ (assumption arising from the first one) according to the quantification of energy exchanges established by Planck's relation (\SeeChapter{see section Thermodynamics page \pageref{planck law}}). This assumption is sometimes named "\NewTerm{Planck's quantum hypothesis}\index{Planck's quantum hypothesis}".
		
		\item[P3.] The Planck's-Einstein relation (Planck's relation):
			
applies to the emission or absorption of radiation during the transition of an electron from one energy state $E_1$ to a state $E_2$ (which solidifies the first postulate).		
	\end{enumerate}
In fact, we find here a revolutionary and unprovable concept (today and to our knowledge) consisting in the quantification of certain physical properties.

Let us continue our analysis:

\subsubsection{Quantification}\label{quantification}

Let $M$ be the mass of the core of an atom with an electric charge $+e$ and $m$ the mass of the electron in "orbit" around the core. We assume that $M \gg m$ and that the central mass is motionless (which is obviously false in reality).

We assimilate the circular motion of the electron around the nucleus to that of a harmonic oscillator (mass connected to a spring exerting a force opposed to a proportional constant $k_r$ to retain the  electron linked).

If the oscillations occurs in the plane, its differential equation is (\SeeChapter{see section Classical Mechanics page \pageref{harmonic oscillator}}):
	
A (special) simple solution of this equation is (\SeeChapter{see section Differential and Integral Calculus page \pageref{second order differential equations}}):
	
The kinetic energy of the system is therefore given  by (\SeeChapter{see section Classical Mechanics page \pageref{kinetic energy}}):
	
and the potential energy of the system (\SeeChapter{see chapter Wave Mechanics page \pageref{potential energy harmonic oscillator}}):
	
If we denote by $\nu$ the frequency of oscillation of the oscillatory motion, we then have of course (\SeeChapter{see chapter Wave Mechanics page \pageref{pulsation frequency period wave number}}):
	
The total energy of the system is finally written after summation and simplification (elementary trigonometry):
	
We now assume that the bound edelectron can only occupy certain energy levels (first hypothsesis) according to the "\NewTerm{Planck-Einstein relation}\index{Planck-Einstein relation}\label{planck einstein relation}":
	
This gives us when we include the Planck-Einstein relation in the penultimate equation:
	
We note here that since the energy of the electron is quantified the amplitude of its movement is also quantified.

Consider now the following integral path also named "\NewTerm{action integral}\index{action integral}" (this is in fact angular momentum):
	
and considering the expression  for velocity previously obtained:
	
Over a period of revolution, we have:
	
Since (\SeeChapter{see section Trigonometry page \pageref{remarkable trigonometric identities}}):
	
The integration becomes:
	
Because $\omega=\dfrac{2\pi}{T}$ (\SeeChapter{see chapter Wave Mechanics page \pageref{pulsation frequency period wave number}}) we have:
	
So finally we get:
	
Given that $A^2=n\dfrac{h}{2\pi^2m\nu}$ and $\omega_0=2\pi\nu$ and also $\nu=\dfrac{1}{T}$ we get:
	
Finally:
	
This condition imposed by Bohr (second postulate) results from the quantization of energy exchanges (Planck-Einstein relation). This has the effect of imposing stationary energy levels that the electron around the nucleus can occupy.

For a circular orbit (remember though for now that we consider a circular orbit!) or radius $r$ the angular momentum (yes in fact the integral action is just the angular momentum) along the length of the orbital is:
	
or by using one of the traditional notation of angular momentum in quantum physics\label{quantized angular momentum}:
	
The angular momentum is quantized and non-zero in the Bohr's model (since $n$ is not zero). We will see that this is no longer the case in the wave model where the angular momentum can be zero.

	According to this result, and further results we will get, it is commonly accepted in science that action values or changes smaller than $\hbar\cong 1.06\cdot 10^{-32}\;[\text{J}\cdot \text{s}]$ are not observed.

	\subsubsection{Hydrogen Type Atoms Model without dragging}
	
	We define the study of "\NewTerm{hydrogen type atom without dragging}\index{hydrogen type atom without dragging}" as the fact of considering the atoms with a single electron of mass $m$ rotating around a central core with a charge of $Z \cdot e$ and mass $M$ such that $M \gg m$ (so the kernel is supposed fixed).
	
Let us now calculate the stationary orbits radius!

On its stationary orbit, the electron is in equilibrium because there is a true antagonism between the Coulomb force and centrifugal force. This should result in the following equality of strengths:
	
We will write starting from now (to reduce the notations):
	
This allows us to write the relation:
	
By using the quantization condition of Bohr and by squaring:
	
By dividing the last two relations to one another by:
	
we get:
		
given the expression of $k$.

The radius of the electron orbits allowed is then:
	
with $n \in \mathbb{N}^{*}$. This relation is commonly named the "\NewTerm{Bohr radius}\index{Bohr radius}\label{bohr radius}" for $n=Z=1$.

The orbits of an atom in this model therefore looks like:

\begin{figure}[H]
\centering
\fbox{\includegraphics[scale=0.8]{img/atomistic/bohr_plane_planetary_model.eps}}
\caption{Bohr planar planetary model}
\end{figure}
The energy of the hydrogenoide-like atom without dragging is given by classical mechanics (case of a central force), the sum of kinetic and potential electrostatic energy:
	
with:
	
we get:
	
By introducing the expression of quantified radius obtained previously:
	
	
We therefore find that the total energy of the atom is quantified and considered as negative (corresponding to stable states because it takes a supply of energy to undo the link) such that:
	
Between two levels, the transition of an electron of the level $n_2$ to the level $n_1$ (we will specify how during the study of the photoelectric effect later) results in the emission of a frequency line given by the expression of Planck's quantization hypothesis:
	
Thus, quantum physics explain the photon energy emission at a transition between two levels!

In fact, if we assume with Bohr that the energies of an electron on its orbit are given by the inverse of the square of the integer, the energy difference between two orbits characterized by high values of these integers approaches zero when the integers tend to infinity. We find then a semblance of continuous change for the energies exchanged by an atom with the electromagnetic field and the concept of trajectory of an electron takes then again sense!

By using the full expression of the total energy, we then find the appropriate frequency to the line emitted:
	
The wavelength emitted is thus easily deduced:
	
The constant $R_H$ (denoted as $R_y$ depending on the context) is named the "\NewTerm{Rydberg constant}\index{Rydberg constant}".

An electron occupying an orbit $n$ is in a "\NewTerm{steady state}\index{steady state}" if its energy does not change.

However, a direct transition $n_2 \rightarrow n_1 (n_1<n_2)$ is accompanied by the emission of a photon whose energy is given by the frequency as we proved it before.

The "\NewTerm{ionization energy}\index{ionization energy}" is the energy it takes to give away an electron to infinity of its orbit. Thus for the ground state of hydrogen, it would be necessary to write $n_1=1$ and $n_2=+\infty$.

The result obtained by Bohr for expressing the frequency depending on the electron energy levels is a great result (which surprised Bohr himself) because by involving only major fundamental constants he theoretically founded the spectrum law that hydrogen lines followed. Law that Johann Jakob Balmer had discovered experimentally in 1885 (28 years before).

Balmer had noticed that the spectral lines were extremely thin. This implied that the energy is not emitted by the atoms continuously but only at certain specific frequencies. Moreover, this fine stripes explains the precision with which he could determine the Rydberg constant.

Chemists had also seen that each element had its own atomic spectrum. It was therefore clear that any atomic theory should reflect these two characteristics and this is what made brilliantly the Bohr's model using the assumptions of energy levels.

We define, moreover, the following spectrum series of the hydrogen atom:

	\begin{itemize}
		\item For the series starting from $n_1=1$ and going to $n_2=2,3,4,...$  we get the result of measurements (spectrum) by Lyman in 1906 in the UV.
		
		\item For the series starting from $n_1=2$ and going to $n_2=3,4,5,6,\ldots$  we get the result of measurements (spectrum) by Balmer in 1885 in the visible spectrum (respectively red, bluegreen, violet, violet).
		
		\item For the series starting from $n_1=3$ and going to $n_2=4,5,6...$  we get the result of measurements (spectrum) by Paschen in 1908 in the infra-red.
		
		\item For the series starting from $n_1=4$ and going to $n_2=5,6,7...$  we get the result of measurements (spectrum) by Brackett in 1928 in the infra-red.
		
		\item For the series starting from $n_1=5$ and going to $n_2=6,7,8...$  we get the result of measurements (spectrum) by Pfund in 1924 in the infra-red.
	\end{itemize}
	Here is a picture representation resuming this:
	\begin{figure}[H]
		\centering
		\includegraphics[scale=0.75]{img/atomistic/hydrogen_spectrum_series.eps}
		\caption{Some spectral series of the hydrogen atom}
	\end{figure}
	or another representation for the three most known series:
	\begin{figure}[H]
		\centering
		\includegraphics[scale=0.75]{img/atomistic/hydrogen_spectrum_series_simplified.eps}
		\caption[Three main series of the hydrogen atom]{Three main series of the hydrogen atom (source: Wikipedia)}
	\end{figure}
	It follows of what we just see something very important in chemistry (to recognize easily with what element we are dealing with) and astrophysics (to analyze the elements in a cloud of gas surrounding a Star or another planet turning around a Star): the fact that materials (elements) all have a specific Line Spectra as show in the figure below:
	\begin{figure}[H]
		\centering
		\includegraphics[scale=0.6]{img/atomistic/line_spectra.jpg}
		\caption[Line spectra of various elements]{Line spectra of various elements (source: OpenStax)}
	\end{figure}

	\subsubsection{Hydrogen Type Atoms Model with dragging}
	
	The atomic nucleus has a mass $M$ that we have assumed stationary for simplification. In reality, the assembly core ($M$) and electron ($m$) is rotating about a common center of mass (obviously!).
	
	Hypothesis (assumptions):
	\begin{enumerate}
		\item[H1.] The hydrogen-like atom is considered an isolated system.
	
		\item[H2.] The nucleus and the electron orbit each in a circular orbit around a common center: the "center of mass CM" (\SeeChapter{see section Classical Mechanics page \pageref{center of mass}}).

		\item[H3.] They have the same angular velocity.
		\end{enumerate}
	The hydrogenoid atom being an isolated system, the movement of the center of mass is either in uniform rectilinear motion or at rest. It is therefore acceptable to place there an inertial reference system.
	\begin{figure}[H]
		\centering
		\includegraphics{img/atomistic/hydrogenoid_center_of_mass.jpg}
		\caption{Defining hydrogenoid atom configuration}
	\end{figure}
	The definition of the center of mass in a laboratory system is given by the theorem of the center of mass (\SeeChapter{see section of Classical Mechanics page \pageref{center of mass}}):
	
The present study will be made relatively to the center of mass! Then, the above relation becomes (\SeeChapter{see section of Classical Mechanics page \pageref{center of mass}}):
	
From the above relation, taking the norm and absolute value, it follows that:
	
The distance between the nucleus and the electron being constant (hypothesis) such that $r=r_M+r_m$ we write:
	
We conclude that trivially:
	
	Applying the law of dynamics, we write that the sum of forces (electrostatic and centrifugal) on the electron (only!) is balanced such that:
	
that we can rewrite by isolating $\omega^2$:
	
	We find again the well-knows expression of the reduced mass for a two-body system:
	
	The both relations:
	
	will also be useful to us in the sections of Quantum Chemistry and Molecular Chemistry!
	
	Let us now determine the total energy of the atom!

	The kinetic energy of the atom is the sum of the kinetic energies of the core ($N$) and the electron ($e$) such that:
	
As equation with the assumption that the angular velocity is identical for the nucleus and the electron:
	
Using the different radii determined previously:
	
The potential energy of the electron relatively to the center of mass is given by (\SeeChapter{see section Electrostatics page \pageref{electrostatic potential energy}}):
	
The total energy of the hydrogen-like atom is then:
	

Relatively to the center of mass, the total angular momentum is the sum of the angular momentum of the electron $b_m$ and the kernel $b_M$ (remember that the angular momentum is also often denoted by the letter $L$):
	
The parenthesis of the last equality has already been the subject of a calculation previously so we have:
	
It is here that Bohr introduced its quantization condition:
	
	But we know the detailed expression of the square of the pulsation:
	
	The  quantified radius has therefore for expression:
		
	The total energy of the atom finally becomes:
		
		Either with a condensed notation:
		
From this last relation, we can easily determine the expression (as we have already done) for wavelengths emitted by a de-excitation of the electron from one orbit $n_2$ to $n_1$. Let us calculate with the same way as we did for the free training model, the expression of the wavelength emitted during the transition from one level to another. We then identical developments:
		
It comes then:
		
The wavelength emitted is then easily deduced:
		

	\begin{tcolorbox}[title=Remark,colframe=black,arc=10pt]
It obviously to realize that this model is more accurate than the previous one.
	\end{tcolorbox}	
	
	\subsubsection{Neutron Assumption}

Spectroscopy results are known with high precision, therefore the Rydberg's constants also (because dependent on the mass of the atomic element studied).

The two blue stripes of the  Balmer series for the hydrogen denoted  by H ($\lambda_H\cong 4861.3 \left[ \mathring{A} \right]$ composed of one proton and one electron) and of deuterium $\mathrm{D}$ (isotope of hydrogen consisting in one neutron more) have a difference of wavelength of $\Delta \lambda \cong 1.32 \mathring{A}$.

The wavelength belonging to the Balmer series ($n_1=2,n_2=3,4,5,...$ is therefore expressed by (with the correction of the mass center as seen above):
	
	This last expression written successively for hydrogen and deuterium leads to:
	
	where we recall that the mass of the electron $m$ is known to us! 
	
	What is interesting here is that these two elements have identical chemical properties (hydrogen and deuterium) but different spectrum lines. Scientists of the this time wondered why and after that the Bohr hydrogen atom with dragging get available to them they were able to conclude that this difference came from the difference in the mass of the atomic nucleus.
	
	It was still necessary to determine the difference in mass and explain its origin!
	
	We therefore have:
	
	This simplifies to:
	
	Therefore:
	
	Which showed the scientists of that time that the deuterium nucleus consists of two particle mass equivalent to that of the proton. Thus by logical deduction, this core must be made of a proton (which is obviously knows!) and a neutral particle!!!
	
	This hypothesis is this of the "neutron", which was later discovered experimentally in 1932 by Chadwick.
	
	Today we know that:
	
	
	\pagebreak	
	\subsection{Wilson and Sommerfeld's Model}\label{wilson and sommerfled model}
	
	In developing their model, Sommerfeld and Wilson used classical dynamics to generalize Bohr's model to Keplerian orbits type (thus not only circular but elliptical in the general case!) because the Bohr's model did not explain the splitting of certain spectrum lines (without presence of any electric or magnetic field) that at the time was named "\NewTerm{fine structure}\index{fine structure}".
	
	As we have seen above, in the case of a system with two bodies solicited by a central force, the total energy of the system (we neglect gravitational potential energy...):
	
	To find the expression of the trajectory of the mass $m$, we will proceed in exactly the same manner as the one used in astronomy (\SeeChapter{see section Astronomy page \pageref{keplerian orbitals}}) to determine the Keplerian orbits.
	
	Thus, we have proved in the section Astronomy that:
	
	With:
	
	We don't need to insist that in our case, it's not anymore a gravitational potential but an electric potential. Which brings us to write to our problem as following:
	
	We still yet have to find the expression of $K$ in a quantified form (according to the Bohr's postulate).
	
	We fill focus first to determine the expression of the focal parameter $p$ of the trajectory:
	
	In our current problem, kinetic and potential energy expressed in polar coordinates give (\SeeChapter{see section Vector Calculus page \pageref{polar coordinates}}):
	
	The total energy of the atom is therefore given by:
	
	In an identical way of that of Bohr, Sommerfeld and Wilson applied the same form of quantification for the vector-radius and extended it to the quantification for the azimuthal angle.
	
	Given the angular momentum:
	
	Momentums are obtained by derivation of the Lagrangian relatively to the generalized coordinates (\SeeChapter{see section of Analytical Mechanics page \pageref{canonical moments})}:
	
	The quantification on the angle is immediate, since $p_\theta$ is a constant of the motion. Indeed, the Lagrangian $L$ being independent of $\theta$ (but no of $\dot{\theta}$), the invariance of the angular momentum results in the following Lagrange equation:
	
	This gives us:
	
	with $n_\theta \in \mathbb{N}^*$ being the "\NewTerm{azimuthal quantum number}\index{azimuthal quantum number}" (i.e. the second quantum number), to remember that it is related to the quantization of the polar angle.
	
	From this last relation we also get:
	
	Let us come back to:
	
	That give us:
	
	Let us focus now to determine the eccentricity $e$ of the path (notation not to be confused with the electrical charge if possible!).
	
	This give us:
	
	To determine the quantization of the angular momentum with respect to the radial variable, we will use a substitution:
	
	By simply denoting $r'$ the derivative $\mathrm{d}r/\mathrm{d}\theta$, the integral will be written:
	
	where we used $\dot{\theta}=p_\theta/(mr^2)$ as we have already prove it.
	
	Reporting it:
	
	in the integral of the radial momentum, we get (easy to obtain but you can ask for the details if needed as always):
	
	from which we deduce given $p_\theta=n_\theta\hbar$ that:
	
	which brings us to:
	
	and therefore:
	
	After some elementary algebraic simplifications we finally get:
	
	where $r$, also named "\NewTerm{radial quantum number}\index{radial quantum number}" can be null! Because this is the case when $\dot{r}=0$, that is to say, if the trajectory is a circle (special case of Bohr model!!!).
	
	We introduce then an integer $n$ so named "\NewTerm{principal quantum number}\index{principal quantum number}" as:
	
	with $n\in \mathbb{N}^*$.
	
	Sommerfeld and Wilson the proved that the orbital Bohr model must be determined by these two new quantum numbers:
	\begin{tcolorbox}[colframe=black,colback=white,sharp corners]
	\textbf{{\Large \ding{45}}Example:}\\\\
	For $n=2$ we have two possible sub-orbital:
	
	\end{tcolorbox}
	The value $n_\theta=0$ is impossible by definition because it would mean that the minor axis is zero (ellipse degenerated into a line) and the electron can not pass through the nucleus (in the classical model at least...). So the smallest possible integer value for $n_\theta$ is $1$.
	
	There is therefore $n$ orbits giving the same spectral term. In other words, there are $n$ times the same quantization energy. We also say that the energy level (total) $E_n$ is "\NewTerm{$n$ times degenerated}".
	
	The idea of Sommerfeld was to account the richness of the observed spectra. From this point of view, the results are disappointing: the quantification of all degrees of freedom clearly shows more states (now we need two quantum numbers to completely specify the state, while the Bohr model only considers one) but the additional degree only introduces an energy degeneration (sad!).
	
	To resume this model, so there is exactly the same number of energy levels and therefore the same number of transitions of possible energy states than the Bohr model. At the spectral point of view, the Sommerfeld-Wilson theory does nothing more than that of Bohr except that the orbits are elliptical and therefore does not explain the extent of the observed spectra.
	\begin{figure}[H]
		\centering
		\includegraphics[scale=0.75]{img/atomistic/sommerfeld_model.jpg}
		\caption{The eccentric... Sommerfeld approach...}
	\end{figure}
	In fact, the idea is going to be from now to take again the same model but by adding relativistic corrections. The work will necessarily be longer but oh so successful!
	
	\subsection{Relativistic Sommerfeld Model}\label{relativistic sommerfeld model}
	However, the Sommerfeld-Wilson's model can be considered incomplete if we do not take into account changes in parameters engendered by the results of the theory of Special Relativity (\SeeChapter{see section Special Relativity page \pageref{special relativity}}).
	
	\begin{tcolorbox}[title=Remark,colframe=black,arc=10pt]
	The section of Quantium Relativistic Quantum Physics is reserved solely to the study of probabilistic quantum physics based on the relativistic version of the Schrödinger equation (so strictly speaking the Relativistic Quantum Physics section should be named "Relativistic Wave Quantum Physics"). This is why it seemed to us more appropriate to put a corpuscular model, deterministic and relativistic (opposite of a wave model, probabilistic and relativistic) such as we will now study in the section of Corpuscual Quantum Physics.
	\end{tcolorbox}
	Indeed, as we have proved in the developement of the Bohr model, the kinetic energy of the electron is given by:
	
	That give us:
	
	For hydrogen and the level $n=1$, we find $v=2.19\cdot 10^{19}\; [\text{m}\text{s}^{-1}]$ and as Michelson-Morley factor (\SeeChapter{see section Special Relativity page \pageref{michelson morley factor}}):
	
	What the reader can quickly check with the then English version of Microsoft Excel:
	
	\texttt{=1/ROOT(1-(1*(1.60217656E-19)\string^2/(2*8.854187E-12*1*6.62068E-34))\string^2/(299792458)\string^2)}
	
	For sure the change is small but spectrometry values were so accurate that it was necessary to introduce Special Relativity to accommodate these tiny variations and thus validate the theory by experiment.
	\begin{tcolorbox}[title=Remark,colframe=black,arc=10pt]
	As we can easily see it, the relation shows that the particle is more away from the nucleus ($n$ large) more its speed is low. This result has been confirmed experimentally by replacing the electron artificially by a muon and scientists have found that the lifetime of the latter slightly increased with the value of $n$.
	\end{tcolorbox}
	Let us determine in the order of things, the expression of quantification conditions with the relativistic factors. Before we begin, it is important to understand that we consider the nucleus as fixed and as a repository of our system. Thus, compared to the referential, the mass of the electron undergoes a relativistic variation but not the electric potential (we should take into account the variation of the latter - as it is energy - if and only if the repository was the electron itself).
	
	In relativistic dynamics (\SeeChapter{see section Special Relativity page \pageref{mass energy equivalence}}), we proved that the kinetic energy (in the form of Lagrangian notation Lagrangian with "$T$" instead of $E_c$) is expressed in the form:
	
	The potential energy (in Lagrangian notation with "$V$" instead of $E_p$) not undergoing relativistic variation, we always have:
	
	The Lagrangian is therefore:
	
	By working in polar coordinates, where the velocity has for expression (\SeeChapter{see section Vector Calculus page \pageref{polar coordinates}}):
	
	Therefore:
	
	The Sommerfeld quantization conditions being:
	
	Now we must seek relativistic expressions for $b_r$ and $b_\theta$.
	
	Let us begin $p_r$:
	
	with:
	
	Therefore:
	
	This gives:
	
	As:
	
	We finally have:
	
	The first quantification condition gives therefore:
	
	For $p_\theta$:
	
	always with:
	
	Therefore:
	
	Which gives:
	
	As:
	
	we finally have:
	
	The second quantification condition\label{second quantification condition} is therefore written:
	
	In summary, the conditions of quantification for the relativistic atom of Sommerfeld are:
	
	We could, seeing the bot above results, conclude a little bit too quickly by thinking it would suffice to multiply the two conditions of quantification by the Michelson-Morley factor relative to the relativistic mass transformation. But, such a shortcut is completely wrong and all but rigorous! Indeed, if you apply this reasoning, it would be sufficient to take the expression of the total energy of the non-relativistic Sommerfeld-Wilson model and introduce everywhere where the mass appears the Michelson-Morley factor. However the result we get following this way has absolutely nothing in common with the result that we will get further below. So we must always be careful and work as a mathematician: without leapfrogging!
	
	The total relativistic energy of the atom (sum of kinetic energy, mass energy  and the potential energy of the electric field for the whole atom) is given by: 
	
	Indeed, using the notation of the section of Special Relativity:
	
	Now, as part of the study of particle quantum physics, it is (unfortunately) usage to write the rest mass with the symbol of the relativistic mass as:
	
	Therefore:
	
	We must express this total energy in function of the quantization conditions. There's a long mathematical work to do but necessary to achieve the result of our study.

	Given the calculation of the expression:
	
	with:
	
	Squaring:
	
	Therefore:
	
	We add of both sides of the equality $m^2c^2$ (in the idea to include the mass energy as you'll see some lines below), which gives:
	
	By multiplying both sides by $c^2$ we get:
	
	Extracting the square root:
	
	If we introduce this last relation in the expression of the total energy we get:
	
	Now it remains to us to determine the expressions of $N-r$ and $n_\theta$ according to $p_r$ and $p_\theta$.

	The integrale of quantification of the azimuthal angle is immediate:
	
	Therefore:
	
	The integral of quantification of the radius-vector requires a more significant development:
	
	Then come long and joyful mathematical developments...

	Reusing the expression of the total energy:
	
	We get:
	
	By squaring and by making a few changes:
	
	Therefore:
	
	Working on the bracketed term, we will put it as being equal to $A$ as:
	
	By adding and subtracting $2E_\text{tot}m$ and decomposing the term $-m^2c^2$ in $m^2c^2-2m^2c^2$ and then by regrouping:
	
	We will put for the simplification of the calculations (to reduce the number of terms to handle):
	
	Therefore we get:
	
	By taking $m^2c^2$ in evidence, we have:
	
	By adding and subtracting $1$ in the parenthesis:
	
	
	As $E_\text{tot}=E'+mc^2$ we have:
	
	By putting
	
	By putting also:
	
	as $p_\theta=n_\theta\hbar$

	Sommerfeld then introduced what he name the "\NewTerm{finestructure constant $\alpha$}\index{finestructure constant}" defined by the relation:
	
	We will meet this constant again in the section of Particle Physics.
	\begin{tcolorbox}[title=Remark,colframe=black,arc=10pt]
	The fine-structure constant is one of the most important physical constants. First, because it is dimensionless, and secondly because it is so far the best known (in terms of accuracy) of all constants and thirdly, because it depends only on terms that seem to be fundamental constants of the atomic world. So physicists and astrophysicists are looking to see if the value of this constant varies over time, which would imply immediately that at least one of the implicit constant is not timeless.
	\end{tcolorbox}
	Given the fine structure constant, we write:
	
	To resume:
	
	With:
	
	So we arrive at the following integral:
	
	The residue theorem (\SeeChapter{see section Complex Analysis page \pageref{residue theorem}}) applied to the preceding integral gives for expression:
	
	We see that there is a trivially a pole at origin $r=0$.

	We will calculate the residue at this point going to the limit $r=0$. We put for this:
	
	Passing to the limit built on the basis of the residue theorem:
	
	The residue corresponding to the pole $r=0$ is then:
	
	We also see that there is a second residue at infinity $r=+\infty$ and to calculate it, we make again a change of variable. We put (according to the method that we saw in the section of Complex  Analysis):
	
	The integral is then written:
	
	To find the residue, we will make a Laurent series expansion of:
	
	around this pole of zero value. To do this, we put:
	
	We know the Taylor expansion (\SeeChapter{see section Sequences and  Series page \pageref{usual maclaurin developments}}) of the resulting expression of this variable change:
	
	Applied to the radical, we get:
	
	It comes then automatically the Laurent series (nice!):
	
	we where see quite quick that the pole is of order $2$.
	
	The second residue is the coefficient on $\dfrac{1}{z}$:
	
	Indeed, we have simply applied the relation proved in the section of Complex Analysis:
	
	
	With:
	
	For the calculation of $\dfrac{B}{\sqrt{A}}$ we have:
	
		Therefore the curvilinear integral has for expression:
	
	After simplification:
	
	Finally:
	
	We raise squared and at each line below we simplify and rearrange:
	
	Therefore:
	
	hence:
	
	We put $n=n_r+n_\theta$. By working on the denominator:
	
	we can transform:
	
	By adding and subtracting $n_\theta$:
	
	Therefore:
	
	or also:
	
	or also:
	
	We consider in the term:
	
	the radical that can be written:
	
	Given the series development (\SeeChapter{see section Sequences and Series page \pageref{usual maclaurin developments}}) in analogy with the previous relation:
	
	then:
	
	Therefore:
	
	As $\dfrac{\alpha}{n_\theta} \ll 1$, we can neglect the terms above the order $2$ such that:
	
	Finally we can the write:
	
	Working on the term in the brackets we get taking into account our previous result:
	
	Given the following Taylor series development (\SeeChapter{see section Sequences and Series page \pageref{usual maclaurin developments}}):
	
	then:
	
	Neglecting terms over order $2$ we have finally:
	
	Therefore the term between braces can be written:
	
	We undertake the Taylor series development of the right term between braces using again:
	
	Neglecting terms over order $2$ we have finally (an approximation of an approximation that is itself an approximation...):
	
	By developing the square of the third term, we get:
	
	Therefore:
	
	The total energy of this atom the becomes:
	
	Finally we get of the expression of the energy:
	
	We can give another expression for the energy of the hydrogenoid atom as:
	
	The expression of the total energy of the hydrogen-like atom becomes then:
	
	Therefore:
	
	In the literature, we find other expressions for the total energy that are more interesting than the previous one (because more traditional). So, considering that $n=n_r+n_\theta$, we get:
	
	If we seek for an expression using to the Rydberg constant $R_H$ (see above) we get:
	
	So the most condensed expression of the total relativistic energy of the hydrogenoid  that we can find in the literature and we adopt in this books is:
	
	The above relation shows well the existence of a fine structure as the characteristics $n_r$ and $n_\theta$of the electron orbit appear separately in a ratio and not just as a sum as in the first model of Sommerfeld and Wilson.
	
	But strictly speaking, we should write because of the nucleus movement:
	
	or:
	
	In which the Rydberg constant ${R'}_H$ is defined as:
	
	However, as the mass of the core is $1,840$ times heavier than that of the electron, we can admit that as a first approximation in the case of the hydrogenous atom:
	
	
	\pagebreak
	\subsubsection{Magnetic dipole moment}\label{magnetic dipole moment}
	At the time of the development of the Sommerfeld model, some physicists strive to study another property of the atom. They observed that under the application of the magnetic field, the spectroscopic lines are doubled. To explain this, they had the brilliant and very simple idea  to explain this phenomenon by the magnetic moment of the electron.
	\begin{tcolorbox}[title=Remark,colframe=black,arc=10pt]
	We will see during our study of Wave Quantum Physics, that even in the absence of a magnetic field a very thing measuring of the rays shows that they are all double and this because of the "spin-orbit coupling". Therefore, a correct interpretation is that the magnetic field doubles the duplication of rays.
	\end{tcolorbox}
	Thus, given the expression of the norm of the magnetic dipole moment (\SeeChapter{see section Magnetostatics page \pageref{magnetic local dipole moment}}):
	
	The magnetic moment is equal, in the corpuscular point of view...., to the area enclosed by the orbit of the electron multiplied by the current of the electron (perpendicular to the unit vector of the surface) on its orbit path thus:
	
	where:
	
	is the period of the movement.
	
	We have seen that the total angular momentum being equal to:
     
      thus the ratio magnetic moment/angular momentum gives:
      
      The ratio $e/2m$ equation is named the "\NewTerm{orbital gyromagnetic ratio (for a classical rotating body)}\index{orbital gyromagnetic ratio}" equal to:
      
     and the quantity:
     
     is the "\NewTerm{Bohr Magneton}\index{Bohr magneton}".
    \begin{tcolorbox}[title=Remark,colframe=black,arc=10pt]
	It is important to remember the few developments and definitions just be made above for when we will develop the Pauli equation in the section of Relativistic Quantum Physics.
	\end{tcolorbox}
	Frequently we write the above relation as well:
    
     where $m_l$ is named "\NewTerm{magnetic quantum number}\index{magnetic quantum number}". 
     
     Knowing that the principal quantum number $n$ is divided by the radial $n_r$ and azimuthal $n_\theta$ quantum numbers, there are so many magnetic moments as there are different orbital geometries for a given value of the principal quantum number. In fact, there are even the double if we consider that the electron can rotate in clockwise or counterclockwise (the magnetic moment is a vector quantity!).
     
     Now let us take two examples using the previous proven rules:
    
     and:
     
    for which we put now:
    
     that is the number that we name "\NewTerm{quantum number of orbital angular momentum}\index{quantum number of orbital angular momentum}" and having values between (as shown in the both previous examples):
     
     What have we finally so far?
     \begin{enumerate}
         \item When $n=1$, we can only have $l=0$ and as $n=1$, there is only one sub-layer, so when applying a magnetic field we still always have one and only one visible line:
         \begin{figure}[H]
			\centering
			\includegraphics{img/atomistic/n_equal_0_orbital_decomposition.jpg}
			\caption{Decomposition of a very low orbital...}
		\end{figure}
			
		\item When $n=2$, we have $l=0$ and $l=1$ and as $n=2$ has two sub-layers. When no magnetic field is applied, the spectrum lines of the two sub-layers are superimposed so indistinguishable (we see only one). But when a magnetic field is applied to the two sub-layers are distinguished by their magnetic moment and therefore we-have two lines but in total there are 3 theoretically (without a field, and two with field):
		\begin{figure}[H]
			\centering
			\includegraphics{img/atomistic/n_equal_1_orbital_decomposition.jpg}
			\caption{Decomposition of an upper orbital...}
		\end{figure}
     \end{enumerate}
	Thus we have:
    
    where for recall\label{quantum number of orbital angular momentum interval}:
     
    The potential energy of a magnetic moment $\mu_l$ placed in a magnetic field $B$ is (\SeeChapter{see section Magnetostatics page \pageref{magnetic torque}}):
       
    So finally for each orbital electron subjected to a magnetic field, we have:
    
     always with :
     
     The observation of the spectrum of an atom in a magnetic field has the effect of adding spectrum lines by the potential energy of the magnetic moment. This is what we name the "\NewTerm{Zeeman effect}\index{Zeeman effect}" because it is the latter who measured these spectrum lines for the first time (before the theory that we will see later).
     
     \subsubsection{Spin}\label{spin}
     Various experimental facts led to attribute to the electron an intrisic magnetic moment and especially the experiment of the duplication of spectrum lines named "\NewTerm{abnormal Zeeman effect}\index{abnormal Zeeman effect}".

     It has indeed been experimentally measured that the magnetic moment was just equal to the value of the Bohr magneton. It is then tempting to attribute this to the electron magnetic moment and speculate that it might come from the fact that it turns on itself (intrinsic angular momentum), ie it would have a "spin" equal to the Bohr magneton and this latter cold take negative or positive values. We speak then of "\NewTerm{spin quantum number}\index{spin quantum number}" or historically "\NewTerm{Pauli quantum number}\index{Pauli quantum number}" and it gives the number of different values that can take the spin. Therefore the spin is one of two types of angular momentum in quantum mechanics, the other being orbital angular momentum

   However, this classic vision of a intrinsic rotation (intrinsic angular momentum) of the particle is actually too naive and at therefore erroneous.

    Indeed, at first, if the particle is punctual, the notion of intrinsic rotation around its axis is simply devoid of physical sense. Recall that since by definition, the axis of rotation of an object is the locus of points of the object which remain stationary, so if the particle is punctual, its own axis is on the particle, thus the latter is stationary...

   Secondly, if the particle is not punctual, then the rotation has a meaning, but it bring in this case to another difficulty. For example, suppose that the particle is an electron, modeled as a radius of a spherical body. We obtain an estimate of the radius by writing that the mass energy of the electron is of the order of magnitude of its electrostatic potential energy (\SeeChapter{see section Electrostatic page \pageref{electrostatic potential energy}}), that is:
   
   The numerical value of this "\NewTerm{classical electron radius}\index{classical electron radius}" is $r_c=10^{-15}$ [m] taking its mass at rest.

    If we attribute then this electron an angular momentum equal to $\hbar/2$ (which has units of angular momentum), we get for a point on the equator of the electron a speed $v$ satisfying:
    
    The numerical value of the speed is so then roughly $v\cong 6\cdot 10^{10}\;[\text{m}\cdot \text{s}^{-1}]$... so the speed is higher than the speed of light in vacuum..., which obviously causes problems with the theory of relativity (\SeeChapter{see section Special Relativity page \pageref{postulate of invariance}}).

   We can not with the mathematical tools of particle quantum physics rigorously formalize the notion of spin, but we will come back on this subject in the section of  Relativistic Quantum Physics (especially during our study of the Pauli equation) and show that the spin is in fact (sadly) something much more subtle than a simple rotation...

    But let us come back to our classical point of view. So when we see a duplication of spectrum lines (abnormal Zeeman effect), we assume that this is due to the electron spin $s$ which can take two different directions (vector direction!).

    It has been measured that the intrinsic magnetic moment of the electron is equal to the value of the Bohr magneton, that is:
    
    If we put $e=1,m=1$ (what physicists like to do...) then we have:
         
    (this just to get a similarity with $b=n\hbar$...)

     This value is constant but may be negative or positive depending on the intrinsic rotation direction of the electron relatively to the observer (the angular momentum having a vector direction). So:
    
    and therefore in practice, spin is given as a dimensionless spin quantum number 
     
    This result, of the utmost importance, also leads us to the conclusion that each magnetic quantum number is degenerate twice by the spin quantum number! Thus, as we shall see a little further into concrete examples (with supporting diagrams), each principal quantum number $n$ is say to be "\NewTerm{degenerated}" a number $2n^2$ of times.
    
    The skeptical reader should know that Spin has important theoretical implications and practical applications. Well-established direct applications of spin include (we will come back on some elements of this list in further section but in a very technical point of view):
   \begin{itemize}
      \item Nuclear magnetic resonance (NMR) spectroscopy in chemistry
      \item Electron spin resonance spectroscopy in chemistry and physics
      \item Magnetic resonance imaging (MRI) in medicine, a type of applied NMR, which relies on proton spin density;
      \item Giant magnetoresistive (GMR) drive head technology in modern hard disks.
   \end{itemize}
    Electron spin plays an important role in magnetism, with applications for instance in computer memories. The manipulation of nuclear spin by radiofrequency waves (nuclear magnetic resonance) is important in chemical spectroscopy and medical imaging.
     
    \pagebreak
    \subsubsection{Pauli exclusion principle}\label{pauli exclusion principle}
    The Pauli exclusion principle is the quantum mechanical principle that states that two identical fermions (particles with half-integer spin) cannot occupy the same quantum state simultaneously. In the case of electrons, it can be stated as follows: it is impossible for two electrons of a poly-electron atom to have the same values of the four quantum numbers: $n$, the principal quantum number, $l$ the angular momentum quantum number, $m_l$ the magnetic quantum number, and $m_s$ the spin quantum number. For example, if two electrons reside in the same orbital, and if their $n$, $l$, and $m_l$ values are the same, then their $m_s$ must be different, and thus the electrons must have opposite half-integer spins of $1/2$ and $-1/2$. This principle was formulated by Austrian physicist Wolfgang Pauli in 1925 for electrons, and later extended to all fermions with his spin-statistics theorem of 1940.
    
    Particles with an integer spin, or bosons, are not subject to the Pauli exclusion principle: any number of identical bosons can occupy the same quantum state, as with, for instance, photons produced by a laser and Bose–Einstein condensate (\SeeChapter{see section Statistical Mechanics page \pageref{bose einstein distribution}}).
    
    Therefore to resume, due to the fact that the state of an atomic electron can be characterized with at least the following $4$ quantum numbers (the first quantum number coming from Bohr, the two others from Sommerfeld and the last one by Pauli) the we have just proved the origin:
	
	or under the following extended form:
	
	Wolfgang Pauli has then postulate to explain some regularities in the atomic properties (specifically: chemistry) a principle of exclusion today named "\NewTerm{Pauli exclusion principle}\index{Pauli exclusion principle}" and stated as follows:

	In an atom, two electrons can't have the same ordered quadruplet $n,l,m_l,m_s$ of quantum numbers.
	\begin{figure}[H]
		\centering
		\includegraphics{img/atomistic/quantum_numbers_summary.jpg}
		\caption{Summary of quantum numbers}
	\end{figure}
	\begin{tcolorbox}[title=Remarks,colframe=black,arc=10pt]
	\textbf{R1.} We sometimes depending on the situations write $s$ rather than $m_s$ (this is not very important...).\\
	
	\textbf{R2.} Once again know from the wave quantum physics that the exclusion principle applies to particles that are "fermions". These are the particles (elementary or composite) which have half-integer spin, such as the proton, neutron and neutrino. This principle does not apply to the group of particles naed "bosons" which have zero or integer spin.
	\end{tcolorbox}
	It is possible from this principle, to establish a kind of catalog of atomic elements from the orbital filling possibilities, assumed layered, improving Mendeleev classification (see just further below the corresponding table).

	Students often see them for the first time in school when they study chemistry. They use it most of the time, without knowing what they really represent (and also sometimes the teachers...).
	
	\subsection{Electron configuration (atomic orbital)}\label{electron configuration}
	In the years 1920, Niels Bohr, Edmund Clifton Stoner and others designed a model of the electronic structure of atoms that gives the possibility to understand quite well the periodic table of elements. The work of Henry Moseley was used to determine the number of protons in the nucleus and, as the atom is globally neutral, then also the number of orbital electrons. It is not simple to determine the atomic structure and in this analysis, physicists have been helped by the experiments conducted by chemists (as for highly complexed atomes, electrons interacts together!).

	Thus, according to chemists, electrons occupy layers and sub-layers around the core by (average) increasing energy order based on rules associated with their quantum numbers we have previously determined. Thus, the "\NewTerm{electronic configuration}\index{electronic configuration}" is the arrangement of electrons in an atom, molecule or another body. Specifically, it is the position of the electrons in an atomic orbital, molecular or other forms of electronic orbitals.
	
	Knowledge of the electron configuration of different atoms is useful in understanding the structure of the periodic table of elements. The concept is also useful for describing the chemical bonds that hold atoms together. In bulk materials, this same idea helps explain the peculiar properties of lasers and semiconductors.
	
	\begin{tcolorbox}[title=Remark,colframe=black,arc=10pt]
		Rigorously the concept of "layer" as we can imagine it has absolutely no sense if we refer to the results of quantum mechanics (\SeeChapter{see section Wave Quantum Physics page \pageref{wave quantum physics}})!!! This is why the debate on how to fill the layers is sterile because it does not strictly exist without a rough approximation of the general rule
	\end{tcolorbox}
	Each layer corresponds to a specific value of the "\NewTerm{principal quantum number $n$}\index{principal quantum number}" and traditionally these layers are designated (this tradition should be abandoned ... but as all the traditions it has a long life...) by the capital letters:
	
	corresponding to the principal numbers $1$, $2$, $3$, $4$, $5$ ... that can take the principal quantum number $n$.
	The "\NewTerm{second/azimuthal quantum number}\index{second/azimuthal quantum number}" conventionally denoted by the letter $l$ corresponds to the degenerate states that can thake the layers (shells) for a given value of $n$ such that:

	For the layer $K$ ($n=1$) we have as we know a unique underlayer (sub-shell):
	
	and therefore for the layer $L$ ($n=2$) we have two sub-layers:
	
	and so for $M$ ($n=3$) we have three sub-layers:
	
	and so on...

	Chemists are accustomed to note the first sub-layers with the Latin letters:
	
	who are the alphabetical equivalent of the azimuthal quantum number $l$. To resume this in a table we have:
	\begin{table}[H]
		\begin{center}
		 \begin{tabular}{|m{4cm}|m{3.5cm}|m{3cm}|m{4.5cm}|}
		 \hline 
		 \centering\arraybackslash\ \cellcolor{black!30} \textbf{Main quantum} & \centering\arraybackslash\ \cellcolor{black!30} \textbf{Number of} & \centering\arraybackslash\ \cellcolor{black!30} \textbf{Notation of}  & \centering\arraybackslash\ \cellcolor{black!30}\textbf{Maximum number} \\ 
		 \centering\arraybackslash\ \cellcolor{black!30} \textbf{number $n$} & \centering  \cellcolor{black!30} \textbf{sublayers} & \centering  \cellcolor{black!30} \textbf{sublayers}  & \centering\arraybackslash \cellcolor{black!30}\textbf{of electrons $2n^2$} \\ 
		 \hline 
		 \centering\arraybackslash $1$ (K) &  \centering\arraybackslash $1$ & \centering\arraybackslash $1s$ &  \centering\arraybackslash $2$\\ \hline
		 \centering\arraybackslash $2$ (L) & \centering\arraybackslash $2$ & \centering\arraybackslash $2s2p$ & \centering\arraybackslash $4$\\
		 \centering\arraybackslash $3$ (M) &  \centering\arraybackslash $3$ & \centering\arraybackslash $3s3p3d$ &  \centering\arraybackslash $8$\\ \hline
		 \centering\arraybackslash $4$ (N) &  \centering\arraybackslash $4$ & \centering\arraybackslash $4s4p4d4f$ &  \centering\arraybackslash $18$\\ \hline
		 \centering\arraybackslash $5$ (O) &  \centering\arraybackslash $5$ & \centering\arraybackslash $5s5p5d5f5g$ &  \centering\arraybackslash $50$\\ \hline
		 \centering\arraybackslash $6$ (P) &  \centering\arraybackslash $6$ & \centering\arraybackslash $6s6p6d6f6g6h$ &  \centering\arraybackslash $72$\\ \hline
		 \centering\arraybackslash $7$ (Q) &  \centering\arraybackslash $7$ & \centering\arraybackslash $7s7p7d7f7g7h$ &  \centering\arraybackslash $98$\\ \hline
		 \end{tabular}
		\caption{Sub-levels of electrons in the main layers}
		\end{center}
	\end{table}
	\textbf{Definitions (\#\mydef):}
	 \begin{enumerate}
		\item[D1.] An "\NewTerm{electronic layer}\index{electronic layer}" is a group of states that have the same principal quantum number $n$.

		\item[D2.] A "\NewTerm{sub-layer}\index{sub-layer}" or "\NewTerm{sub-shell}\index{sub-shell}" is a smaller group of states which are characterized by the quantum numbers $n$ and $l$.

		\item[D3.] An "\NewTerm{orbital}\index{orbital (electronic)}" is specified by the three numbers quantum $n,l,m_l$ and it can contain two electrons one with spin up and the other with spin down.

		\item[D4.] A "\NewTerm{state}\index{state (electronic)}" is defined by the four quantum numbers $n,l,m_l,m_s$ and contains a single electron as required by the principle of exclusion.
		
		\item[D5.] In chemistry, a "\NewTerm{valence electron}\index{valence electron}" is an outer shell electron that is associated with an atom, and that can participate in the formation of a chemical bond if the outer shell is not 
	\end{enumerate}
	The number of electrons in an atom's outermost valence shell governs its bonding behavior. Therefore, elements whose atoms can have the same number of valence electrons are grouped together in the periodic table of the elements. Valence electrons are also responsible for the electrical conductivity of an element; as a result, an element may be classified as a metal, a nonmetal, or a semiconductor (or metalloid).
	
	Let us summarize in in the form of diagrams that we have seen so far only for the first two main layers beginning with $K$ ($n=1$):
	\begin{figure}[H]
		\centering
		\includegraphics{img/atomistic/electronic_lower_orbit_n1_orgchart_with_spin.jpg}
		\caption{Decomposition of a low orbit with spin}
	\end{figure}
	Thus, the Pauli exclusion principle provides that  $2$ electrons in the layer $K$ are authorized.

	And for $K$ ($n=2$):
	\begin{figure}[H]
		\centering
		\includegraphics{img/atomistic/electronic_lower_orbit_n2_orgchart_with_spin.jpg}
		\caption{Decomposition of a less lower orbit with spin}
	\end{figure}
	Thus, the Pauli exclusion principle provides that  $8$ electrons in the layer $L$ are authorized.
	
	And so on ... We notice that we have in adequation with what we have proved earlier that every layer can contain a number of electrons equal to $2n^2$.
	
	Under the form of a table this gives:
	\begin{figure}[H]
		\centering
		\includegraphics{img/atomistic/electron_configuration_summary_table.jpg}
		\caption[Modern quantum number notation]{Modern quantum number notation (source: Aco Z. Muradjan)}
	\end{figure}

	Under the form of Bohr diagram according to the naive Bohr model, this gives (the "groups" refers to columns of the periodic table of elements we will see further below):
	\begin{figure}[H]
		\centering
		\includegraphics{img/atomistic/bohr_diagram.jpg}
		\caption[Example of Bohr Diagram]{Example of Bohr Diagram (source: ?)}
	\end{figure}
	With a notation consistent with that of chemists, the basic configurations of some elements are then written:
	\begin{table}[H]
		\begin{center}
			\definecolor{gris}{gray}{0.85}
				\begin{tabular}{|l|c|c|c|}
					\hline
					\multicolumn{1}{c}{\cellcolor{black!30}\textbf{Element}} & 
	  \multicolumn{1}{c}{\cellcolor{black!30}\textbf{Symbol}} & 
	  \multicolumn{1}{c}{\cellcolor{black!30}\textbf{Atomic Number}}  & \multicolumn{1}{c}{\cellcolor{black!30}\textbf{Electronic configuration}} \\ \hline
				Hydrogen & H & $1$ & $1s$ \\ \hline
				Helium & He & $2$ & $1s^2$ \\ \hline
				Lithium & Li & $3$ & $1s^22s$ \\ \hline
				Beryllium & Be & $4$ & $1s^22s^2$ \\ \hline
				Boron & B & $5$ & $1s^22s^22p$ \\ \hline
				Carbon & C & $6$ & $1s^22s^22p^2$ \\ \hline
				Nitrogen & N & $7$ & $1s^22s^22p^3$ \\ \hline
				Oxygen & O & $8$ & $1s^22s^22p^4$ \\ \hline
				Fluorine & F & $9$ & $1s^22s^22p^5$ \\ \hline
				Neon & Ne & $10$ & $1s^22s^22p^6$ \\ \hline
				Sodium & Na & $11$ & $1s^22s^22p^63s$ \\ \hline
				Magnesium & Mg & $12$ & $1s^22s^22p^63s^2$ \\ \hline
				Aluminum & Al & $13$ & $1s^22s^22p^63s^23p$ \\ \hline
				Silicon & Si & $14$ & $1s^22s^22p^63s^23p^2$ \\ \hline
				Phosphorus & P & $15$ & $1s^22s^22p^63s^23p^3$ \\ \hline
				Sulfur & S & $16$ & $1s^22s^22p^63s^23p^4$ \\ \hline
				Chlorine & Cl & $17$ & $1s^22s^22p^63s^23p^5$ \\ \hline
				Argon & Ar & $18$ & $1s^22s^22p^63s^23p^6$ \\ \hline
				$\ldots$ & $\ldots$ & $\ldots$ & $\ldots$  \\ \hline
			\end{tabular}
		\end{center}
		\caption{Examples of some atomic elements electronic configuration}
	\end{table}
	Which is a condensed form of the following corresponding table:
	\begin{table}[H]
		\begin{center}
			\definecolor{gris}{gray}{0.85}
				\begin{tabular}{|l|lc|c|c|c|c|c|c|c|c|c|c|c|c|c|c|}
					\hline
					 \cellcolor{black!30} & \cellcolor{black!30} & \cellcolor{black!30} & \cellcolor{black!30}  & \multicolumn{1}{c}{\cellcolor{black!30}\textbf{K}} & \cellcolor{black!30} & \multicolumn{1}{c}{\cellcolor{black!30}\textbf{L}} & \cellcolor{black!30} & \cellcolor{black!30} & \multicolumn{1}{c}{\cellcolor{black!30}\textbf{M}} & \cellcolor{black!30} & \cellcolor{black!30} & \cellcolor{black!30} & \multicolumn{1}{c}{\cellcolor{black!30}\textbf{N}} & \cellcolor{black!30} & \cellcolor{black!30} & \cellcolor{black!30}\\ \hline
	  				\cellcolor{black!30} & \cellcolor{black!30} & \cellcolor{black!30} &  \multicolumn{1}{c}{\cellcolor{black!30}\textbf{$n$:}} & \multicolumn{1}{c}{\cellcolor{black!30}1} & \cellcolor{black!30} & \multicolumn{1}{c}{\cellcolor{black!30}2} & \cellcolor{black!30} & \cellcolor{black!30} & \multicolumn{1}{c}{\cellcolor{black!30}3} & \cellcolor{black!30} & \cellcolor{black!30} & \cellcolor{black!30} & \multicolumn{1}{c}{\cellcolor{black!30}4} & \cellcolor{black!30} & \cellcolor{black!30} & \cellcolor{black!30} \\ \hline
	  				\multicolumn{1}{c}{\cellcolor{black!30}\textbf{Z}} & 
	  \multicolumn{1}{c}{\cellcolor{black!30}\textbf{Element}} & \cellcolor{black!30} & 
	  \multicolumn{1}{c}{\cellcolor{black!30}\textbf{$l$:}} & \multicolumn{1}{c}{\cellcolor{black!30}$s$} & \cellcolor{black!30} & \multicolumn{1}{c}{\cellcolor{black!30}$s$} & \multicolumn{1}{c}{\cellcolor{black!30}$p$} & \cellcolor{black!30} & \multicolumn{1}{c}{\cellcolor{black!30}$s$} & \multicolumn{1}{c}{\cellcolor{black!30}$p$} & \multicolumn{1}{c}{\cellcolor{black!30}$d$} & \cellcolor{black!30} & \multicolumn{1}{c}{\cellcolor{black!30}$s$} & \multicolumn{1}{c}{\cellcolor{black!30}$p$} & \multicolumn{1}{c}{\cellcolor{black!30}$d$} & \multicolumn{1}{c}{\cellcolor{black!30}$f$} \\ \hline
				1 & Hydrogen & H & & $1$  \\ \hline
				2 & Helium & He & & $2$ \\ \hline
				3 & Lithium & Li & & $2$ & & $1$ \\ \hline
				4 & Beryllium & Be & &  $2$ & & $2$ \\ \hline
				5 & Boron & B & & $2$ & & $2$ & $1$ \\ \hline
				6 & Carbon & C & & $2$ & & $2$ & $2$ \\ \hline
				7 & Nitrogen & N & & $2$ & & $2$ & $3$ \\ \hline
				8 & Oxygen & O & & $2$ & & $2$ & $4$ \\ \hline
				9 & Fluorine & F & & $2$ & & $2$ & $5$ \\ \hline
				10 & Neon & Ne & & $2$ & & $2$ & $6$ \\ \hline
				11 & Sodium & Na & & $2$ & & $2$ & $6$ & & $1$ \\ \hline
				12 & Magnesium & Mg & & $2$ & & $2$ & $6$ & & $2$ \\ \hline
				13 & Aluminium & Al & & $2$ & & $2$ & $6$ & & $2$ & $1$ \\ \hline
				14 & Silicon & Si & & $2$ & & $2$ & $6$ & & $2$ & $2$ \\ \hline
				15 & Phosphorus & P & & $2$ & & $2$ & $6$ & & $2$ & $3$ \\ \hline
				16 & Sulfur & S & & $2$ & & $2$ & $6$ & & $2$ & $4$ \\ \hline
				17 & Chlorine & Cl & & $2$ & & $2$ & $6$ & & $2$ & $5$ \\ \hline
				18 & Argon & Ar & & $2$ & & $2$ & $6$ & & $2$ & $6$ \\ \hline
				19 & Potassium & K & & $2$ & & $2$ & $6$ & & $2$ & $6$ & $\cdot$ & & $1$ \\ \hline
				20 & Calcium & Ca & & $2$ & & $2$ & $6$ & & $2$ & $6$ & $\cdot$ & & $2$ \\ \hline
				21 & Scandium & Sc & & $2$ & & $2$ & $6$ & & $2$ & $6$ & $1$ & & $2$ \\ \hline
				22 & Titanium & Ti & & $2$ & & $2$ & $6$ & & $2$ & $6$ & $2$ & & $2$ \\ \hline
				23 & Vanadium & V & & $2$ & & $2$ & $6$ & & $2$ & $6$ & $3$ & & $2$ \\ \hline
				24 & Chromium & Cr & & $2$ & & $2$ & $6$ & & $2$ & $6$ & $4$ & & $1$ \\ \hline
				25 & Manganese & Mn & & $2$ & & $2$ & $6$ & & $2$ & $6$ & $5$ & & $2$ \\ \hline
				26 & Iron & Fe & & $2$ & & $2$ & $6$ & & $2$ & $6$ & $6$ & & $2$ \\ \hline
				27 & Cobalt & Co & & $2$ & & $2$ & $6$ & & $2$ & $6$ & $7$ & & $2$ \\ \hline
				28 & Nickel & Ni & & $2$ & & $2$ & $6$ & & $2$ & $6$ & $8$ & & $2$ \\ \hline
				29 & Copper & Cu & & $2$ & & $2$ & $6$ & & $2$ & $6$ & $10$ & & $1$ \\ \hline
				30 & Zinc & Zn & & $2$ & & $2$ & $6$ & & $2$ & $6$ & $10$ & & $2$ \\ \hline
				31 & Gallium & Ga & & $2$ & & $2$ & $6$ & & $2$ & $6$ & $10$ & & $2$ & $1$ \\ \hline
				32 & Germanium & Ge & & $2$ & & $2$ & $6$ & & $2$ & $6$ & $10$ & & $2$ & $2$ \\ \hline
				33 & Arsenic & As & & $2$ & & $2$ & $6$ & & $2$ & $6$ & $10$ & & $2$ & $3$ \\ \hline
				$\ldots$ & $\ldots$ & & & $\ldots$ & & $\ldots$ & & & $\ldots$ & & & & $\ldots$  \\ \hline
			\end{tabular}
		\end{center}
		\caption{Examples of some atomic elements electronic configuration (extended version)}
	\end{table}
	As we said earlier this is not an accurate way as the layers anywas overlap. This is why there exist another way to fill-in the layer (this other way works very well for the ground states of the atoms for the first $18$ elements, then decreasingly well for the following $100$ elements).

	\pagebreak
	\textbf{Definition (\#\mydef):} The "\NewTerm{Aufbau principle}\index{Aufbau principle}" (from the German Aufbau, "building up, construction") states that a maximum of $2$ electrons are put into orbitals in the order of increasing orbital energy: the lowest-energy orbitals are filled before electrons are placed in higher-energy orbitals such that:
	\begin{enumerate}
		\item Orbitals are filled in the order of increasing $n+l$

		\item Where two orbitals have the same value of $n+l$, they are filled in order of increasing $n$
	\end{enumerate}
	This gives the following order for filling the orbitals:
	
	and can be illustrated as following:
	\begin{figure}[H]
		\centering
		\includegraphics{img/atomistic/electronic_configuration_aufbau_principle.jpg}
		\caption{Aufbau Principle}
	\end{figure}
	In this list the orbitals in parentheses are not occupied in the ground state of the heaviest atom now known so far (Uuo, $Z = 118$).
	
	The Aufbau principle can be applied, in a modified form, to the protons and neutrons in the atomic nucleus, as in the shell model of nuclear physics and nuclear chemistry.
	
	And finally so far this is the only image of a nice periodic table that I have been able to found for this book but (there exist much more complete and technical one but their screen rendering quality is awful...):
	\begin{figure}[H]
		\centering
		\includegraphics[scale=0.75]{img/atomistic/periodic_table.jpg}
		\caption[Periodic (Mendeleev) table]{Periodic (Mendeleev) table (source: European Synchrotron)}
	\end{figure}
	Or a much more pedagogical one:
	\begin{figure}[H]
		\centering
		\includegraphics[width=1.0\textwidth]{img/atomistic/periodic_table_illustrated.pdf}
	\end{figure}
	\pagebreak
	The structure of the periodic table is closely related to the electron configuration of the atoms of the elements:
	\begin{figure}[H]
		\centering
		\includegraphics[scale=0.53]{img/atomistic/periodic_table_structure.jpg}
		\caption[Periodic (Mendeleev) table structure]{Periodic (Mendeleev) table structure (source: Wikipedia)}
	\end{figure}
	However, although the relativistic model of Sommerfeld is a amazing accurate and consistent model with experimental observations, it does not explain some important phenomena that we observe at the scale of the atom. Thus, this model is unable to explain the disintegration of the elements, the dual (complementary) behavior of matter between wave and particle, the annihilation of matter and antimatter, the split of spectral lines when in presence of a magnetic field and many more.

	These are much more complex developments and at the same time (hopefully!) consistent with what we saw which will be developed in the next section dealing with Wave Quantum Physics that gives the opportunity to explain in a very satisfactory way a large  number of unexplained phenomena that were unexplained at the atomic size (but sill not all as we will see after that we will need Relativistic Quantum Physics).
	
	\begin{flushright}
	\begin{tabular}{l c}
	\circled{95} & \pbox{20cm}{\score{5}{5} \\ {\tiny 81 votes,  100.00\%}} 
	\end{tabular} 
	\end{flushright}
	
	%to make section start on odd page
	\newpage
	\thispagestyle{empty}
	\mbox{}
	\section{Wave Quantum Physics}\label{wave quantum physics}
	\lettrine[lines=4]{\color{BrickRed}D}aughter of the old quantum theory (\SeeChapter{see section Corpuscular Quantum Physics page \pageref{corpuscular quantum physics}}), the wave quantum physics known also simply as "\NewTerm{quantum mechanics}\index{quantum mechanics}" is the mainstay of a set of physical theories that we group under the general heading of "\NewTerm{quantum physics}\index{quantum physics}".
	
	This denomination is opposed to that of Classical Physics, this latter failing in its description of the microscopic world (atoms and particles) and in that of certain properties of electromagnetic radiation (typically see the experiences of Young slits in the section of Wave Optics) or semiconductor (typically see the Hall Effect in the section of Electrokinetics).
	
	\begin{tcolorbox}[title=Remark,colframe=black,arc=10pt]
	The relativistic extension of Wave Quantum Physics is Relativistic Quantum Physics (see section of the same name page \pageref{relativistic quantum physics}).
	\end{tcolorbox}
	
	Quantum physics took over and developed the idea of wave-particle duality\index{wave-particle duality} introduced by de Broglie considerating all material particles (even the atoms) not only as point particles, but as waves, with some spatial extent. These two aspects wave/corpuscule of particles ("\NewTerm{quanton}\index{quanton}"), mutually exclusive, can not be observed simultaneously as far as we know. If we see a wave property, corpuscular appearance disappears and vice versa.
	\begin{figure}[H]
		\centering
		\includegraphics[scale=1]{img/atomistic/wave_particle_duality.jpg}
		\caption[Dual nature of light]{Dual nature of light (source: S. Tanzilli, CNRS)}
	\end{figure}
	To this date, no contradiction could be found between the predictions of quantum physics and associated experimental tests. This success has unfortunately a price: the theory is based on a rather abstract mathematical formalism, which makes it quite difficult at first. Hopefully until we not deal with the spin we can be dispensed Heisenberg's matrix formalism to focus on formalism of Schrödinger's wave equation that is much simpler and which to a certain level can get a mental picture of what is happening.
	
	What is even much more difficult is that it is very difficult if not impossible to introduce this field of physics in a linear pedagogical way ... This has the consequence that many books about this subject (including this text), addressed to specialists or not, see their explanations or texts subject to many interpretations criticism, proofreading and supplements.
	
	To find a plausible solution it is favorable to take as basis the "\NewTerm{principle of objectivity}\index{principle of objectivity}" due to Werner Heisenberg, which is the basis of "\NewTerm{standard quantum physics}\index{standard quantum physics}": exists only what is experimentally observable and reproducible. We could also align with the conviction of Max Born to rebuild corpuscular quantum physics that is objectively a potpourri of quantum and classical physics rules to give the place to a new coherent theory based only on y few postulates (Max Born was inspired by the elegance of Albert Einstein's theories that made use of a few well-defined assumptions). Wolfgang Pauli also felt it was essential to cease to sate arbitrary ad hoc hypotheses each time experiments produced data that were disagree with the theory.
	
	The duality principle is accepted by the majority of physicists, but not all. Is an electron in several places at the same time? For this to be admissible, we must have an experience that found them several places at the same time, which is not possible (problem of clock sync) then we are not required to answer the question! Say that it is in several places before we observe is not admissible in physics: principle of objectivity. In general, we'll also give up the notion of trajectory and movement, which will allow to remove the contradiction of radiation by breaking/Bremsstrahlung (\SeeChapter{see section of Electrodynamics page \pageref{bremsstrahlung}}): because there is no more movement in the classical sense. The concepts of speed and acceleration lose all sense at this scale!
	
	A minority of physicists denies this principle and founded a non-standard quantum physics with classical quantities this is why we can find especially in popular scientific papers presentations that deviate from the standard quantum physics (that of the most physicists). This non-standard version gives the same predictions for any feasible experience, so this is a possible model.
	
	Finally quantum physics is a theory considered by the current majority of physicists as unfinished and in which many points still remain rather obscure.
	
	Before we tackle the mathematical part, we would like to indicate to the reader that we will limit ourselves only to theoretical developments made between about 1910 and 1935 (beyond the complexity of theories requires too many pages to a general book as ours). Indeed, we very much believe that Wave Quantum Physics, as Relativistic Quantum Physics and Nuclear Physics are not finished "products", but rather a work in progress. They have developed historically, they continue to be simplified, clarified, expanded and applied through the hard work of physicists who see these theories from different angles. While we present in this book all these theories in a more or less linear fashion, we attempt to provide multiple viewpoints whenever possible.
	
	\pagebreak
	\subsection{Postulates}
	Unlike most books on the subject, we are pedagogically (not technically!) very unconvinced about the impact of the presentation of the postulates of quantum physics at the beginning of its study in classrooms. We allow ourselves to present our reasons (experiment done):
	
	\begin{enumerate}
		\item They can be deduced from simple mathematical and logical reasoning (elementary algebra and probabilities) based on the postulates of corpuscular quantum physics and the principle of complementarity and therefore can be deduce from an evolution of this latter. Even if the process is rigorously false at least it is pedagogical!
		\item These assumptions are indigestible or even incomprehensible if quantum physics (its formalism and vocabulary) was not initially apprehended by a given number of exercises or a regular use.
	\end{enumerate}
	We can then consider that the only non-provable elements theoretically (to our knowledge) that have their place as assumption would be: de Broglie complementarity principle (we'll talk about it later), the Planck-Einstein relation (already seen in a previous section) and the measurement of an observable.
	
	Nevertheless..., in the objective of the respect of tradition, and especially in respect of scientific methodology, we have chosen to present these assumptions at the beginning of this section but without overemphasize them. However, we strongly recommend the non-initiated reader, to read them without spending too much time to try to understand but just think about them a come back to them regularly for during the read of this section. Afterwards, everything will probably be clear and light will be...

	
	\begin{tcolorbox}[title=Remark,colframe=black,arc=10pt]
	We will see practical cases, in this section itself, of quantum theory for later use in the sections of quantum field theory and nuclear physics. However, we advise the reader to read at the same time the sections of Quantum Computing and of Quantum Chemistry and Molecular Chemistry which it seem help greatly in the understanding of some little too theoretical topics presented further below.
	\end{tcolorbox}
	
	\subsubsection{1st Postulate: Quantum State}\label{first postulate wave quantum physics}
	The state of a classical quantum system is specified by the generalized coordinates $(q_1,q_2,...)$ (\SeeChapter{see section Analytical Mechanics page \pageref{general coordinates}}) and is fully described by a finite function differentiable everywhere and denoted in all generality:
	
	named "\NewTerm{state function}\index{state function}" or "\NewTerm{wave function}\index{wave function}", whose squared modulus (multiplication of the function by its complex conjugate) must give the probability density to find instantly the system in the configuration $(q_1,q_2,...)$ at time $t$ (if the system is time dependent):
	
	that we will justify later!
	
	The above relation is named the "\NewTerm{Born rule}\index{Born rule}" or "\NewTerm{Born interpretation}\index{Born interpretation}". The Born rule is one of the key principles of quantum mechanics. There have been many attempts to derive the Born rule from the other assumptions of quantum mechanics, with so far inconclusive results...
	
	\begin{tcolorbox}[title=Remarks,colframe=black,arc=10pt]
	\textbf{R1.} The fact that we speak of "wave" instead of "particle" comes from the brilliant postulate and in fact quite logical of de Broglie that we name "complementarity principle" (which we will detail later too) and which associates with any material particle, a wave and vice versa.\\
	
	\textbf{R2.} The fact that we deal with probabilities and that they are proportional to the square of the wave function module comes from the Heisenberg uncertainty principle that we will prove further below later and mainly from the experience of Young slits experiment with electrons (\SeeChapter{see section Wave Optics page \pageref{young interference experiment}}).
	\end{tcolorbox}
	As a corollary, the particle being necessarily located somewhere in the entire space, we have the normalization condition that the integral over all space is:
	
	to a given phase factor. In other words $\Psi$ must be normalized, what we traditionally name the "\NewTerm{de Broglie normalization condition}\index{de Broglie normalization condition}\label{de broglie normalization condition}" (even if a posteriori the concept seems to come from Max Born).
	
	Indeed the Born rule was formulated by Born in a 1926 paper. In this paper, Born solves the Schrödinger equation for a scattering problem and, inspired by Einstein's work on the photoelectric effect,concluded, in a footnote, that the Born rule gives the only possible interpretation of the solution. In 1954, together with Walther Bothe, Born was awarded the Nobel Prize in Physics for this and other work.
	\begin{tcolorbox}[title=Remarks,colframe=black,arc=10pt]
	\textbf{R1.} Note that even normalized, $\Psi$ is determined to a given phase factor. In addition, it is preferable for $\Psi$ to be differentiable, because differential operators act on it to get the theoretical predictions on measurable properties, and also finished to be normalizable...\\
	
	\textbf{R2.} When the integral given above provides a finite amount, we say it is "\NewTerm{square integrable}\index{square integrable}". Otherwise, you have to normalize it so that the theoretical model corresponds to reality! We will also discussed it in more detail late (with proofs!). 
	\end{tcolorbox}
	
	Let us recall that a "\NewTerm{phase factor}\index{phase factor}" is a complex constant factor having a unitary module. We can write it (depending on what we have studied in the section on Numbers during our study of complex numbers) $e^{\mathrm{i}\delta}$, where $\delta$ is any angle, named the "\NewTerm{phase}\index{phase}" (\SeeChapter{see section Wave Mechanics page \pageref{phase shift}}). We will also proved further below rigorously why it has no influence.
	
	We can express this postulate in a little more formal way, because as we will see in several examples, the wave function is often a complex polynomial which can then be expressed in the Hilbert space of polynomials. This gives therefore in the language of Dirac bra-ket formalism (see further below for details) the following definition:
	
	\textbf{Definition (\#\mydef):} The state vector "\NewTerm{ket}\index{ket}" represented by $\Ket{\Psi(t)}$ belonging to the vector space $\mathcal{H}$ (Hilbert space) defines the state of the quantum system at time $t$. This state vector has all the mathematical properties required by quantum physics and especially the dot product of the vector $\Ket{\Psi(t)}$ by its dual vector (complex conjugate) "\NewTerm{bra}\index{kat}" $\Bra{\Psi(t)}$ must satisfy the functional scalar product:
	
	
	
	\begin{tcolorbox}[title=Remark,colframe=black,arc=10pt]
	The bra-ket notation was introduced by Paul Adrien Maurice Dirac Dirac to facilitate (at least it's supposed to...) the notation of quantum physics equations, but also to highlight the potential vector aspect of the object representing a state quantum. What is specific to quantum physics is that the vectors are not drawn with arrows but with ket and bra (not the bra you can think about...), but it is only a matter of notations and brings nothing mathematically new. Moreover, you should not imagine that we wrote explicitly in the calculations these vectors as column vectors (think to complex numbers.... it is rare that we write them in vector form in fact)!
	\end{tcolorbox}
	
	To summarize these last paragraphs, the two relations:
	
	and:
	
	are therefore equivalent!
	
	A generalization of the Born rule also exist (still as a postulate!). Indeed, if $|\varphi\rangle$ is a function (or vector) representing a state of a system and if $|\chi\rangle$ represents another state of this same system, then there exists and amplitude of probability\index{amplitude of probability}\label{amplitude of probability} $A(\varphi\rightarrow \chi)$ to found $|\varphi\rangle$ in the state $|\chi\rangle$, which is given by the dot product on $\mathcal{H}$:
	
	The probability $P(\varphi\rightarrow \chi)$ for the state $|\varphi\rangle$ to pass the test $|\chi\rangle$ is then obtained by taking the square module of this amplitude:
	
	
	\subsubsection{2nd postulate: Time evolution of a quantum state}
	If the system is not disturbed, the evolution (supposed non-relativistic!) of his state is governed by the Schrödinger's equation of evolution (so time dependent!):
	
	This relation simply means that it is the operator "total energy" or "Hamiltonian" $H$ of the system, which is responsible for the system to evolve over time. Indeed, the form of the equation shows that, in applying the Hamiltonian to the wave function $\Psi$ of the system, we obtain the derivative with respect to time, that is to say: how it varies over time!
	\begin{tcolorbox}[title=Remark,colframe=black,arc=10pt]
	We will prove further below this relation in details (it will not be trivial unfortunately... but it is possible (!) and so it eliminates the need to define it as a postulate).
	\end{tcolorbox}
	In the latter relation, $H$ is the Hamiltonian operator (total energy) of the system that we will prove further below as having for value in a particular and simple case\label{hamiltonian operator wave quantum physics}:
	
	In the case where the potential $\vec{V}(q_1,q_2,...q_n)$ is independent of time (corresponding to a conservative system in classical mechanics), there are (we will see it in various  examples later) a set of independent particular solutions time independent and satisfactory (relation which we will prove the origin later):
	
	where the $\psi_k(q_1,q_2,...)$ are the "\NewTerm{eigenfunctions}\index{eigenfunctions}" (in analogy with the eigenvectors seen in the section Linear Algebra) of the Hamiltonian / operator $H$ with eigenvalue / observable $E_k$.
	
	These particular solutions then describe special states named "\NewTerm{stationary states}\index{stationary states}" (as independent of time...), which we will show later the properties and origin of the name, and which form an orthogonal basis.
	
	The previous eigenvalue equation is often named "\NewTerm{time independent Schrödinger equation}\index{time independent Schrödinger equation}". It defines the stationary states and has a sense of course only that if the system is conservative.
	
	This is especially the time independent Schrödinger equation that concerns quantum chemistry and molecular chemistry (topics we cover in the Chemistry chapter of the book). Indeed, we seek in these fields to get the wave functions describing the stationary states, and especially the state of lowest energy, "\NewTerm{fundamental state}\index{fundamental state}" of atoms and molecules. The transitions observed in spectroscopy are done between these stationary states (as we will prove it further below), their determination is thus a prerequisite for the study of spectroscopy. However, we must remember that it is the Schrödinger evolution equation, which is (at first ...) the fundamental equation of nonrelativistic wave quantum physics: it plays the same role as the Newton equation in Classical Mechanics, or that of an equation of motion (see the proof of the  Ehrenfest's theorem further below).
	\begin{tcolorbox}[title=Remark,colframe=black,arc=10pt]
	In fact, we will prove (\SeeChapter{see section Relativistic Quantum Physics page \pageref{free Klein-Gordon equation}}) that the Schrödinger evolution equation is only a special case of what we name the "free Klein-Gordon equation" which itself is a special case of the "generalized Klein-Gordon equation", which itself is a limited model relatively to the "linearized Dirac equation" ... in short we have not finished to do maths....
	\end{tcolorbox}
	
	\subsubsection{3rd postulate: Observables and operators}\label{observables and operators}
	For each measurable physical property (an "observable") of a system denoted for example by:
	
	where $q_k$ are the generalized coordinates and $p_k$ the generalized momentum according to notations adopted in the section of Analytical Mechanics, corresponds a linear operator (so it can also be a matrix!) named "\NewTerm{Hermitian operator}\index{Hermitian operator}\label{hermitian operator}", frequently denoted with a hat such that for for the chosen example above it will be denoted by:
	
	which always intervene in the theoretical calculation of a physically measurable property (see the section Linear Algebra for a recall of what is an Hermitian matrix).
	
	In other words (but this is more related to the postulates that will follow): An observable value is represented by a quantum operator which operates on the wave function to predict the measurement value of the latter observable (the operator $\hat{\mathcal{O}}$ fix the mathematical representation of $\mathcal{O}$).
	
	To make ... a Hermitian operator in quantum physics is a mathematical expression as such that if we take its complex conjugate (or his adjoint matrix if the mathematical expression is a matrix!) then the theoretical calculation of the measurable value is always given by the same expression.
	
	Here are the best known examples for which we will prove in details the origin in this section and of this of Relativistic Quantum Physics:
	\begin{tcolorbox}[colframe=black,colback=white,sharp corners]
	\textbf{{\Large \ding{45}}Examples:}\\\\	
	E1.	Coordinates:
	
	for which we will see a practical example with the Ehrenfest theorem in this section.\\
	
	E2. Momentum:
	
	for which we will also see several practical examples (including one with the Ehrenfest theorem).\\
	
	These two examples are sometimes referred to under the name "\NewTerm{correspondence principle}\index{correspondence principle}".\\
	
	E3. Kinetic momentum:
	

 	E4. Pauli matrices (that as we will see later correspond to the spin operators):
	

	E5. The operator of evolution of energy of a quantum state:
	
	\end{tcolorbox}
	\begin{tcolorbox}[title=Remarks,colframe=black,arc=10pt]
	\textbf{R1.} This may seem to fall from the sky ... but we will see that it just comes naturally when we will later the development sof some very concrete examples or when reading the section of Quantum Computing.\\
	
	\textbf{R2.} As part of this boke, we write indifferently, the operators and observable without the hat symbol (that's the reader to know what we are working with... without being confused...).
	\end{tcolorbox}
	We will see also that some operators are not commutative and obey to what we name "\NewTerm{anticommutation relations}\index{anticommutation relations}" (which are the source of the Heisenberg' principles of uncertainty).
	
	\begin{tcolorbox}[colframe=black,colback=white,sharp corners]
	\textbf{{\Large \ding{45}}Example:}\\\\	
	Here is an example of such a relation that we will prove later:
	
	then we say that the two components of the two observable position and momentum are "\NewTerm{canonically conjugate}\index{canonically conjugate}".
	\end{tcolorbox}
	We will also see using a practical case that two observables $A$, $B$ whose respective operators commute such as:
	
	possess a common eigenvectors base. We then say that they are simultaneously measurable (the precise determination of one does not preclude the other's) with accuracy (otherwise we have an uncertainty... of Heisenberg). The two quantities $A$, $B$ can then be named "\NewTerm{compatible observable C.O.}\index{compatible observable}" This then means that the two operators should have the same specific eigenfunctions (so they are compatible only if the associated operators admit such common functions).
	
	The set of all attached C.O. to a physical system is a "\NewTerm{complete set of compatible observables C.S.C.O.}\index{complete set of compatible observables}".
	
	\subsubsection{4th postulate: Measure of a property}	
	The consequence of the previous postulate is that the measurement of $\mathcal{O}$ therefore always gives an eigenvalue of the associated Hermitian operator $\hat{\mathcal{O}}$. In other words, the only observable values of the property $\mathcal{O}$ are the eigenvalues (denoted for example: $o$) of the operator $\hat{\mathcal{O}}$!
	
	
	The eigenvectors and eigenvalues of an operator values have special meaning: the eigenvalues are the values that can result from an ideal measure of this property, the eigenvectors are the quantum states of the system during the measurement.
	
	It is because of this assumption that it is important to ensure that any physical property is represented by a Hermitian operator. 
	\begin{theorem}
	In other words, the hermiticity of $\hat{\mathcal{O}}$ ensures that its eigenvalues (so denoted by example: $o$) are real.
	
	Let's us first do the proof with the usual algebraic notation and then with another approach using the Dirac notation.
	\end{theorem}
	\begin{dem}
	Since an operator is associated with a specific eigenfunction and an eigenvalue by (\SeeChapter{see section Linear Algebra page \pageref{eigenvector}}):
	
	Using the property proved during in our study of complex numbers in the section Numbers like what the complex conjugate product of two complex numbers is the product of two conjugate numbers we have:
		
	If we multiply the first relation by the complex conjugate of $\varphi_k$ and integrate over all space, we have:
	
	and same with the prior previous relation:
	
	As by definition $\hat{\mathcal{O}}$ is Hermitian (therefore equal to its own conjugate complex), both left sides of the previous two relations are equal. Then we have:
	
	Therefore it comes:
	
	and as the integral is physically of square integrable and therefore not zero, this requires that:
	
	and therefore:
	
	and this is only possible if the eigenvalue is real.
	
	Now, with the Dirac notation and with an approach a little bit different:
	
	Since an operator is associated with a specific eigenfunction and eigenvalue by ((\SeeChapter{see section Linear Algebra page \pageref{eigenvector}}):
	
	We then have if the eigenfunction is normalized in Dirac notation:
	
	And if the operator $hat{\mathcal{O}}$ is a Hermitian operator (therefore equal to its complex conjugate in the case of a function and equal to the transposed complex in the case of a matrix), we have:
	
	So if the operator is indeed Hermitian we have:
	
	and this can only be satisfy if the eigenvalues are real.
	\begin{flushright}
		$\square$  Q.E.D.
	\end{flushright}
	\end{dem}
	Thus, own hermetic operator values are always real numbers (hopefully ...).
	
	\subsubsection{5th postulate: Average of a property}\label{fifth postulate of wave quantum physics}
	This postulate is the less intuitive and more difficult to prove (we will prove with an example when studying the Ehrenfest theorem). His statement is as follows:

	The mean value (expected) of a physical property $\mathcal{O}$, when the system is in the state described by the normalized function $\Psi$ is given by (do not confuse the notation of the average of an operator with that of the complex conjugate of an eigenvalue !!!):
	
	An equivalent expression quite difficult to readis the following: the probability of finding the eigenvalue $o_k$ (of the operator $\hat{\mathcal{O}}$ predominantly Hermitian), when measuring the property $\mathcal{O}$ at time $t$ on a quantum system prepared in state described by the function $\Psi$ is given by the square of module of the projection of the function $\Psi$ on the eigenfunction $\varphi_k$ associated to the eigenvalue $o_k$ (and its operator):
	
	where the "\NewTerm{projection}\index{projection}" (or "\NewTerm{representative}\index{representative}") is defined by:
	
	the index $k$ being here to indicate that there may be some for some operators several eigenvalues and eigenvectors.
	\begin{tcolorbox}[title=Remark,colframe=black,arc=10pt]
	We come back on this formalism and relations further below. They are also several practical examples provided in the section of Quantum Computing and a single and very nice example at the end of the section of Quantum Chemistry (for the mean radius and the mean angular momentum).
	\end{tcolorbox}
	For example, in one dimension and for a system that is time dependent, we use the operators presented in the previous postulate (but on which we will also return in detail):
	
	For which we will see practical examples in the sections already mentioned!
	
	\pagebreak
	\subsection{Classical principles of uncertainty}
	Before attacking directly quantum physics in front and the corresponding mathematical tools in-deep (and the pseudo-proofs of the five postulates), we must first introduce a simple classic example in which appears a special type of phenomena: the intrinsic presence of uncertainty in any measurement (as it is intrinsic we don't speak of statistical uncertainty as we have already study in the section Statistics!).
	
	This study in classical form is fundamentally not very rigorous, will normally help the average reader to better understand the quantum uncertainty (at least we hope so...) that we will study and determine further below and that will need no previous experimental considerations! However it say that it way also really one of the approaches used by Heisenberg himself!
	
	Imagine that we would like to measure through a microscope the abscissa $x$ of a particle and the components of its momentum $\vec{p}$. For the measurement of $x$ to be possible, there must be a monochromatic light beam (to simplify...) parallel to the $x$-axis comes illuminate the particle, and that at least one photon collides the particle and reaches the eye of the observer: 
	\begin{figure}[H]
		\centering
		\includegraphics{img/atomistic/classical_incertitude_experiment.jpg}
		\caption{Heisenberg's microscope}
	\end{figure}
	\begin{tcolorbox}[title=Remark,colframe=black,arc=10pt]
	Heisenberg's microscope exists only as a thought experiment, one that was proposed by Werner Heisenberg, criticized by his mentor Niels Bohr, and subsequently served as the nucleus of some commonly held ideas, and misunderstandings, about Quantum Mechanics. In particular, it provided an argument for the uncertainty principle on the basis of the principles of classical optics. While the act of measurement does lead to uncertainty, the loss of precision is less than that predicted by Heisenberg's argument when measured at the level of an individual state. The formal mathematical result remains valid.
	\end{tcolorbox}
	Once $x$ measured, we can imagine any method for measuring the linear momentum.

	Let us put $\alpha$ as the angle that do the direction of the photon after collision with the $z$-axis. Let us suppose to simplify the calculations that the particle has a relatively high mass so that we can neglect the energy change of the photon. We see that after the collision, the components of the linear momentum of the scattered photon following respectively the $x$-axis and $z$-axis are:
	
	Indeed, let us recall that the relations between the electromagnetic waves, the mass-energy equivalence and the linear momentum (\SeeChapter{see section Special Relativity page \pageref{mass energy equivalence}}) are:
	
	It follows that the particle will see its amount of linear momentum altered. The components of the variation are then (remember that initially it was zero following the $z$-axes) those of the variation of the photon following:
	
	between it initial and final linear momentum.

	The only information we have about the angle $\alpha$ is that it is, in absolute value less than or equal to the opening angle $u$ of the microscope objective (technical restrictions).

	This implies that:
	
	
	\subsubsection{First classical uncertainty relation}
	When we will have measured the linear momentum $\vec{p}$ at the end of the experiment, we just saw that we will have to make the corrections:
	
	of the linear momentum of the photon to know the true value of $\vec{p}$ of the particle just before the start of measurement.

	In these corrections, there is an unknown portion corresponding to measurement errors on $p_x$ and $p_z$ of the photon. It is possible to establish that the maximum error $\Delta p_x$ and $\Delta p_z$ on the initial linear momentum is given by the $x$ component of the "\NewTerm{first classical Heisenberg's uncertainty relation}\index{first classical Heisenberg's uncertainty relation}":
	
	as we have $|\sin(\alpha)|<\sin(u)$. So it is like a worst error!
	
	\subsubsection{Second classical uncertainty relation}
	Let us now see what we can say about the measurement of the position of the particle.

	Let us recall now that (\SeeChapter{see section Wave Optics page \pageref{destructive interference pattern}}) for a rectangular aperture we have by putting $n=1$:
	
	where $\theta$ (in Wave Optics) is the angle to clearly distinguish two diffraction minima (and therefore clearly an object emitting radiation between the same two points). Conversely, from the viewpoint of diffraction, the opening width $e$ is given by:
	
	The value of $e$ may also be seen as the vision area (orthogonal projection of the rectangular aperture on the $x$ axis) of width $x=e$ of the particle. Therefore:
	
	Just as the maximum error of the linear momentum is given by the condition $|\sin(\alpha)|<\sin(u)$, we can also write  $|\sin(\theta)|<\sin(u)$, which leads us to write:
	
	If we multiply the following relation we just proved before:
	
	with the prior-previous one, we get:
	
	After simplification we get the "second classical Heisenberg's uncertainty relation" also named "\NewTerm{classical spatial uncertainty}\index{classical spatial uncertainty}":
	
	which thus represents the maximum experimental error of a small rectangular aperture opening of a microscope. Many quantum physics books show that we fall back on exactly the same expression in many situations.
	
	In his celebrated 1927 paper, "\textit{Über den anschaulichen Inhalt der quantentheoretischen Kinematik und Mechanik}"), Werner Heisenberg established this expression as the minimum amount of unavoidable momentum disturbance caused by any position measurement, but he did not give a precise definition for the uncertainties $\Delta x$ and $\Delta p$. Instead, he gave some plausible estimates in each case separately.
	\begin{tcolorbox}[title=Remark,colframe=black,arc=10pt]
	The reader will easily verify that this relation applied to for a macroscopic object (size of the order of the centimeter), which position would be measurable with a precision of the micrometer gives a ridiculously low uncertainty of linear momentum and thus for the speed. But (!) the same relation applied to the mass of a particle such as that of an electron with a position measurement accuracy assumed of a tenth of a nanometer give an uncertainty about the speed of about $1,000\;[\text{m}\cdot\text{s}^{-1}]$!!\\

	Thus, if we try to locate a particle with an accuracy always bigger, its momentum reaches extreme values. At some point, the magnitude of the linear momentum can be so big that the corresponding energy is sufficient to produce a particle-antiparticle pair. In other words, if we try to confine a particle in an increasingly small box on the one hand, we know less and less its linear momentum and from a certain threshold, we do not even know not how many particles there are in the box!
	\end{tcolorbox}
	However (!), We will during our study of commutator applied to the quantum physics theory that the true uncertainty relation (whose value differs from the one above) only appears naturally from mathematical properties and the definition of the linear momentum.

	More generally, for a particle in a volume  of dimensions $x, y, z$, a vector state is characterized by six quantities $(x,y,z,p_x,p_y,p_z)$ in the phase space (phase space which is therefore of $6$ dimensions) and the quantum state occupies the "cube" of volume:
	
	What is remarkable in this simplistic approach is that the Planck constant naturally emerges as the unit of minimum universal measurement of uncertainty of experimental physics through the wave-particle dualism! Heisenberg wrote that this result establishes the ultimate failure of causality in quantum physics.
	
	\subsubsection{Third classical uncertainty relation}
	In special relativity, we saw that $(x, y, z, ct)$ are the components of a space-time four-vector and also $(p_x,p_y,p_z,Ec^{-1})$ those of a vector of a four energy-momentum vector.

	It is therefore natural to complete the three spatial relation of the type $\Delta x \Delta p_x=h$ by extension:
	
	We obtain thus roughly the "\NewTerm{third classical uncertainty relation}\index{third classical uncertainty relation}" also named "\NewTerm{classic temporal uncertainty}\index{classic temporal uncertainty}":
	
	However (!), we will see during our study of commutators applied to the theory of quantum physics, that this uncertainty relation (whose value differs from the one above) also appears naturally from only the mathematical properties and the definition of the linear momentum.
	\begin{tcolorbox}[title=Remark,colframe=black,arc=10pt]
	We will come back later on the implications of this temporal uncertainty whose implications are the basis of quantum cosmology (and the creation of our Universe... only this!) and also of quantum field theory and in particular regarding to the Yukawa potential (\SeeChapter{see section of Quantum Field Theory page \pageref{yukawa potential}}).
	\end{tcolorbox}
	The established classical uncertainties relation will allow us to better understand the "real" uncertainties relations in their modern quantum form. For this, among others, we will have to make use of the necessary mathematical artillery. However, for the sake of clarity, we wanted to present the wave quantum physics in the simplest and least formal way possible as always in this book. This presentation may perhaps take the reader to many misinterpretations, he must therefore remain cautious as long as it has not seen the rigorous proof!
	
	Let us now come back as promise in the section of Magnetostatics on the Wilson bubble chamber that obviously seems to be incompatible with Heisenberg' inequalities: 
	\begin{figure}[H]
		\centering
		\includegraphics{img/electromagnetism/bubble_chamber.jpg}
		\caption[Wilson Bubble Chamber]{Wilson Bubble Chamber (source: CERN)}
	\end{figure}
	The issue of having the possibility to track a particle with such precision and on such a long distance that seems to violate both: Heisenberg's inertitude principle and Wave-Particle duality is named in general the "\NewTerm{Mott problem}\footnote{So a Mott problem is a paradox that illustrates some of the difficulties of understanding the nature of wave function collapse and measurement in quantum mechanics}\index{Mott problem}".

	The problem was first formulated in 1929 by Sir Nevill Francis Mott and Werner Heisenberg, illustrating the paradox of the collapse of a spherically symmetric wave function into the linear tracks seen in a bubble chamber.

	In practice, virtually all high energy physics experiments, such as those conducted at particle colliders, involve wave functions which are inherently spherical. Yet, when the results of a particle collision are detected, they are invariably in the form of linear tracks (see, for example, the illustrations accompanying the article on bubble chambers). It is somewhat strange to think that a spherically symmetric wave function should be observed as a straight track, and yet, this occurs on a daily basis in all particle collider experiments.

	A related variant formulation was given in 1953 by Mauritius Renninger, and is now known as "\NewTerm{Renninger's negative-result gedanken experiment}\index{Renninger's negative-result gedanken experiment}". In this formulation, it is noted that the absence of a particle detection can also constitute a quantum measurement; namely, that a measurement can be performed even if no particle whatsoever is detected.

	In the original 1929 formulation by Mott and Heisenberg, the spherical wave function of an alpha ray emitted from the decay of a radioactive atomic nucleus was considered. It was noted that the result of such a decay is always observed as linear tracks seen in Wilson's cloud chamber. Intuitively, one might think that such a wave function should randomly ionize atoms throughout the cloud chamber, but this is not the case. Mott demonstrated that by considering the interaction in configuration space, where all of the atoms of the cloud chamber play a role, it is overwhelmingly probable that all of the condensed droplets in the cloud chamber will lie close to the same straight line. What is uncertain is which straight line the wave packet will reduce to; the probability distribution of straight tracks is spherically symmetric. 
	\begin{tcolorbox}[title=Remark,colframe=black,arc=10pt]
	Rigorously the "Heisenberg's incertitude principles" should be named "Heisenberg's indetermination theorem" as they can be proven (therefore it is not a principle!) and they are not a limit of the instrumentation of a limit of precision of Nature itself (hence the "indetermination" in place of the "incertitude").\\
	
	But this is a detail as many many statements in physics are provable  and keep the name "principle" because at the time they were discovered they were not provable!
	\end{tcolorbox}

	\pagebreak
	\subsection{Quantum algebra}
	Under uncommon and unofficial term of "quantum algebra" (so do not to abuse it!) we wish to introduce and recall to the reader some mathematical tools  that will be very useful to solve certain equations of Quantum Physics. It is therefore of prime importance to understand (or to have understand  regarding ot the reminders) at best what will follow!
	
	\begin{tcolorbox}[title=Remark,colframe=black,arc=10pt]
	Purists may climb the walls when reading what follows ...
	\end{tcolorbox}
	
	\subsubsection{Linear functional operators}
	\textbf{Definition (\#\mydef):} The "\NewTerm{linear operators}\index{linear operators}" are mathematical entities acting on functions (\SeeChapter{see section Functional Analysis page \pageref{functions}}) or vectors (\SeeChapter{see section Vector Calculus page \pageref{vector}}).

	The functions on which can operate on these operators can be functions of a single variable $x$, such as $f (x)$, or of three coordinates of a point $x, y, z$ such as $f (x, y, z)$ or written briefly $f(\vec{x})$.

	We will have to write integrals of these functions, which are often extended to the entire space. In the case of a function of the three-dimensional coordinates of a point, we will adopt the following notation that we already know:
	
	These notations, that are quite essential to simplify the expressions that we will meet in quantum physics being established, we can go back to our operators.

	Starting from a function $f$, if we associate to it a function $g$ of the same kind, that is to say dependent on the same variables, we can say that $g$ is the result of the action of an operator $\alpha$ on $f$ and write this symbolically as a single product:
	
	So $\alpha$ multiply, or better "acts on", $f$.
	
	But we introduce immediately a fundamental restriction: only interest us linear operators (as as in Linear Algebra...), that is to say, such as:
	
	whatever the coefficients $\lambda_1$ and $\lambda_2$

	A very simple category of operator consists simply of the numbers (real or complex scalars). Thus, in the relation:
	
	$\chi$ is a function that depends linearly of $f$, trough a linear operator that we write $\lambda$, $\lambda$ being a number (typically the position operator in quantum physics).

	There are two important special cases to consider:
	\begin{enumerate}
		\item Zero operator: $\lambda=0$ where $\chi=0\cdot f=0$ will be obviously a function equal to zero everywhere...

		\item Unit operator (or identity operator): $\lambda=1$ where $\chi=1\cdot f=1$ (which is just as simple...)
	\end{enumerate}
	\begin{tcolorbox}[title=Remark,colframe=black,arc=10pt]
	The "Nabla" operator $\nabla$ is also a functional linear operator (we will see that a bit later) that in quantum physics is found in the energy operator.
	\end{tcolorbox}
	We also check easily for functional operators that they are (see the sections of Set Theory and Linear Algebra if necessary):
	\begin{itemize}
		\item Commutative with respect to the addition
	
		\item Associations in relation to the addition and multiplication
		
		\item Bistributive over addition left and right
	\end{itemize}
	So far, nothing distinguishes operators algebra of that of numbers. But there are however two properties that we must always keep in mind to not to make mistakes when we make calculation with operators:
	\begin{enumerate}
		\item Two operators do not commute in general compared to the multiplication (as in linear algebra ...), that is to say that in general given two functional operators $\alpha$ and $\beta$:
		

		\item If we encounter an expression such as $\alpha_1\beta+\beta\alpha_2$, so we do not have the right to do in general a factorization (there is therefore a particular structure group that is non-commutative)!
	\end{enumerate}
	\begin{tcolorbox}[colframe=black,colback=white,sharp corners]
	\textbf{{\Large \ding{45}}Example:}\\\\
	A simple and important example, as useful for what will follow (very close to a practical case that we will see later), of two operators that do not commute with a function of a single variable is as follows (where $f$ is any) . 

	Let us consider the operator $\mathrm{d}/\mathrm{d}x$ acting on $xf(x)$:
	
	simplifying by $f$:
	
	So we have above an example of two operators that do not commute since:
	
	\end{tcolorbox}
	\begin{tcolorbox}[title=Remarks,colframe=black,arc=10pt]
	\textbf{R1.} If an operator can switch anyhow with another operator is that it is a number (it joined with the concept of measurement which we have mentioned in the postulates).\\
	
	\textbf{R2.} When a state (a mathematical function in the formal sense) is unchanged by an operator, the sate is then named "\NewTerm{eigenstate}\index{eigenstate}" or "\NewTerm{eigenvector}\index{eigenvector}\index{eigenvector}" of the system (we will see practical examples below). The state is then perfectly measurable and is assimilated to the classical observable.
	\end{tcolorbox}
	Before going to the next subject, to close this introduction on linear functional operators, consider that we know the three-dimensional Schrödinger equation (which we will prove further below):
	
	or written differently (that is more aesthetic ...) with the Laplacian of $\vec{\nabla}^2=\Delta$ of a scalar field (\SeeChapter{see section Vector Calculus page \pageref{scalar laplacian}}):
	
	or after rearranging:
	
	or after simplification:
	
	Then the total energy operator  (the Hamiltonian $H$ in other words ...) is expressed as:
	
	or in Hamiltonian notation:
	
	\begin{tcolorbox}[title=Remark,colframe=black,arc=10pt]
	Here we fall back naturally on the second expression given in the second postulate but the notation $V$ for the potential energy can be sometimes confusing with the electrical potential for beginners that have never deal too much with Analytical Mechanics.
	\end{tcolorbox}
	On the other hand, we know that:
	
	The last two expressions must be identical. The only way to meet these conditions is to put:
	
	which are the "\NewTerm{Hermitian operators of linear momentum}\index{Hermitian operators of linear momentum}" in wave quantum physics (expressed here in Cartesian coordinates) and that need to be remembered throughout this section of the book!
	\begin{tcolorbox}[title=Remark,colframe=black,arc=10pt]
	Here we fall back naturally one one of the operators listed in the third postulate.
	\end{tcolorbox}
	We can check the legitimacy of these operators by re-injecting them into the expression of the kinetic energy:
	
	Moreover, it is easy to verify that this development is remain correct if we take the complex conjugate of the linear momentum operator!!!

	Thus, the total energy operator $H$ (Hamiltonian) is, also, well Hermitean! This result is very important to check such calculations using the orthogonality property of eigenfunctions that we will see further below.
	
	\paragraph{Hermitian and Self-adjoint operators}\mbox{}\\\\\
	Caution!!!! Reading the following lines could be quite abstract... However, if you do not understand much this is not very important because often everything will become evident during the study and the developments of concrete examples that will be give later, after which you will be able to reread the following and understand it at the same time.
	
	Let us consider the two following integrals extended to the whole space (inside the integral it is a multiplication of functions and operators) without trying to understand their utility for the moment:
	
	where let us we recall that the notation $\bar{f}$ is the complex conjugate of $f$. It should be noted that in these two integrals, $\alpha$ and $\beta$ represent operators.
	
	We observe in the developments of quantum physics that these two integrals are equal (and for $Q$ and $\bar{Q}$ to be equal we must therefore have $Q\in\mathbb{R}$) and that there is a one-to-one correspondence between the operators $\alpha$ and $\beta$, we say then that $\beta$ is the "\NewTerm{adjoint operator}\index{adjoint operator}" of $\alpha$ (the transpose of the conjugate) or that it is the "\NewTerm{hermitian}" of $\alpha$ (both terms are customary) and we write:
	
	if the two previous integrals are satisfied.

	From this definition, we deduce the following important identity:
	
	\begin{tcolorbox}[colframe=black,colback=white,sharp corners]
	\textbf{{\Large \ding{45}}Example:}\\\\
	Let us consider the operator:
	
	Then by integration by parts:
	
	\end{tcolorbox}
	\begin{tcolorbox}[title=Remark,colframe=black,arc=10pt]
	We will prove the relation above in a concrete but particular example in our study of Quantum Field Theory (see the next section) and we will return to this in a more rigorous way in our presentation of the Dirac formalism in the section Relativistic Quantum Physics.
	\end{tcolorbox}
	
	The adjoint operator has several properties, the only ones that will interest us in this section are:
	\begin{enumerate}
		\item[P1.] $(\alpha^\dagger)^\dagger$ which it is unnecessary to proved, since this relation comes from the very definition of the adjoint operator.
	
		\item[P2.] $\alpha$ being considered as a complex number (special case of operator such as the linear momentum one that we have seen before) then we have $\alpha^\dagger=\bar{\alpha}$.
	\end{enumerate}
	An extremely important category of operators is thus constituted by the "\NewTerm{self-adjoint Hermitian operators}\index{self-adjoint Hermitian operators}", or simply "\NewTerm{self-adjoint operators}" equal by definition to their adjoint:
	
	It must be known to the reader that sadly some teachers and physicists say	that "self-adjoint" and "hermitian" are equivalent concepts. This is an abusive language used by physicists that you will never found by mathematicians. 
	\begin{tcolorbox}[title=Remark,colframe=black,arc=10pt]
	Remember that these operators can also be matrices!
	\end{tcolorbox}
	We also notice that if we take a Hermitian operator (such as that of the linear momentum to make a simple example) and that we multiply that latter by the unitary imaginary number $\mathrm{i}$ then it becomes non-Hermitian\index{non-hermitian operator}\label{non-hermitian operator} (being not in $\mathbb{C}$ anymore) but at the same time it becomes self-adjoint as it will be a real operator (belonging to $\mathbb{R}$).
	\begin{theorem}
	Any operator, that we will denote by $\xi$, can be decomposed in a unique way into Hermitian (self-ajdoint) $\alpha$ and non-Hermitian (non-self-adhoint) $\mathrm{i}\beta$ parts, that is, we can write:
	
	where $\alpha$ and $\beta$ are hermitians (self-adjoints).
	\end{theorem}

	\begin{dem}
	If:
	
	then:
	
	as it is a simple complex number, then:
	
	The sum or the difference of the operator and his adjoint is therefore a self-adjoint operator (the sum or subtraction between self-adjoints operators, hence remains self-adjoints).
	\begin{flushright}
		$\square$  Q.E.D.
	\end{flushright}
	\end{dem}
	In general, it is quite obvious that the product of two self-adjoints operators $\alpha$ and $\beta$  is not necessarily itself self-adjoint operator, that is:
	
 	since we can verify that the condition for which the product of two self-adjoint operators is itself self-adjoint is that the two operators "commute" (see below):
 	\begin{theorem}
	The product of two self-adjoint operators is self-adjoint if and only if the two operators commute.
	\end{theorem}

	\begin{dem}
	First remember that by definition and explicitly:
	
	Therefore (\SeeChapter{see section Linear Algebra page \pageref{transposed matrix}}):
	
	Therefore we see that:
	
	if and only if $\alpha^\dagger$ and $\beta^\dagger$ commute or equivalently, if and only if
	
	That is to say if and only if:
	
	\begin{flushright}
		$\square$  Q.E.D.
	\end{flushright}
	\end{dem}
	
	\pagebreak
	\paragraph{Commutators and Anticommutators}\label{commutators and anticommutators}\mbox{}\\\\\
	\textbf{Definitions (\#\mydef):}
	\begin{itemize}
		\item[D1.] The "\NewTerm{commutator}\index{commutator}" of two operators $\alpha$ and $\beta$, is written:
		
		\item[D2.] The "\NewTerm{anticommutator}\index{anticommutator}" of two operators $\alpha$ and $\beta$, is written:
		
		Which is generally written in the following form:
		
	\end{itemize}
	\begin{tcolorbox}[title=Remarks,colframe=black,arc=10pt]
	\textbf{R1.} As the commutator is much more common in developments than the anti-commutator, if there is no possible confusion, we denote it simple $[\alpha,\beta]$.\\
	
	\textbf{R2.} Concrete and trivial examples of these commutators in our study of Qantum Physics will be presented in the text that follows.
	\end{tcolorbox}
	Let us mention some obvious properties of commutators (those that we will use the most in this book):
	\begin{enumerate}
		\item[P1.] Anticommutativity:
		

		\item[P2.] Left-linearity:
		

		\item[P3.] Right-linearity:
		
	\end{enumerate}
	where $\lambda_1$, $\lambda_2$ are any number (proofs are made - if necessary - during the development of practical examples further below).
	
	Let us look for the adjoint of $[\alpha,\beta]$:
	
	Hence a very simple result:
	
	The following relation is very useful in practice (trivial, but as usual if necessary we can add the detailed proof on request):
	
	We also have:
	
	We will prove later with a concrete case that if two operators do not commute, then it is impossible to have a state having a precise and unique value for both operators at the same time (in quantum physics there is an experimental configuration where the first operator represents the linear momentum and the second the spatial coordinate), this is why the operators are often referred to as "observables".
	
	Let us take a moment to look at a concrete example of commutators and whose one of the results is absolutely fundamental!

	We have proved earlier above the relations:
	
	Consider the relation (simple usual mathematical differential):
	
	If we divide by $\psi$ on both sides of the equality and then multiply by $\hbar/\mathrm{i}$, then we get:
	
	what give us:
	
	therefore it comes that the commutator of $x$ and $p_x$ is equal to $\mathrm{i}\hbar$ and therefore that the quantities do not commute. We therefore have the following non-commuting relation\label{second quantum uncertainty relation}:
	
	that we often find in literature in the form (with the symbol of Kronecker):
	
	Thus (in the context of the second postulate), the two observables $x$ and $p_x$, whose operators do not commute, do not have a common eigenvector basis. They are therefore not simultaneously measurable with precision and therefore constitute an Heisenberg uncertainty (see proof after the remark)!
	\begin{tcolorbox}[title=Remarks,colframe=black,arc=10pt]
	\textbf{R1.} The abbreviation (cycl.) Mmeans that the letters ($x$, $y$, $z$) can be circularly changed and the result remains the same.\\
	
	\textbf{R2.} Although this result may seem surprising, it is none the less extremely correct as it results from mathematical reasoning. We can not make it more simpler and rigorous.\\
	\end{tcolorbox}
	\begin{theorem}
	Operators that do not commute, do not have a common eigenvector (eigenfunction) basis or conversely operators that do commute, have a common eigenvector (eigenfunction) basis.
	
	In other words, given $A$ and $B$ two hermitian operators (ie matrices $M_{nn}(\mathbb{C})$) such that $[A,B]=0$. We can the found a basis $\mathcal{H}$ made by common eigenvectors of $A$ and $B$.
	\end{theorem}

	\begin{dem}
	Let us consider two operators $A$ and $B$ which represent observables and are therefore Hermitian. If there is a $\psi$ such that:
	
	then $\psi$ is a simultaneous (common) eigenfunction (eigenvector) of $A$ and $B$, belonging to eigenvalues $a$ and $b$, respectively. Hence:
	
	and therefore:
	
	\begin{flushright}
		$\square$  Q.E.D.
	\end{flushright}
	\end{dem}
	Let us now also consider the relation:
	
	and proceeding in the same way as before, we get:
	
	The two relations:
	
	can be summarized as follows:
	
	using the generalized coordinates and generalized moments (\SeeChapter{see section Analytical Mechanics page \pageref{general coordinates} and page \pageref{general momentum}}) and are remarkable from several point of views:
	\begin{enumerate}
		\item First, because from purely theoretical and mathematical considerations we fall back in Quantum Physics an equivalent (but not equal) uncertainty to that obtained during our study of the Heisenberg principles of uncertainty seen earlier above (which we recall had been obtained From a classical practical case).

		Indeed, if we take the module of the left commutator, then we get the "\NewTerm{spatial uncertainty relation of Heisenberg}\index{spatial uncertainty relation of Heisenberg}":
		
		which for recall, can also be written in the form:
		
		The Planck constant being extremely small, this explains why this effect is impossible to detect on our macroscopic scale. By cons, the mass of the electrons being extremely small too, the above fraction becomes noticeable for an electron and the effect of this uncertainty is then important!
		
		Finally, by the commutation of the components of the linear momentum quadrivector (\SeeChapter{see section Special Relativity page \pageref{relativistic linear momentum}}), we get the "\NewTerm{temporal uncertainty relation of Heisenberg}\index{temporal uncertainty relation of Heisenberg}":
		
		\begin{tcolorbox}[title=Remark,colframe=black,arc=10pt]
		With the used of advanced mathematics we can prove that in fact a factor $1/2$ is missing in front of the $\hbar$.
		\end{tcolorbox}
		There is a quite fantastic consequence from the uncertainty about time and energy and relativity. Imagine the most complete vacuum (quantum vacuum) and suppose that we look at what happens at a given point in space for a very short time. Then the principle of temporal uncertainty tells us that the energy of this state (the vacuum!) is very imprecise. But Special Relativity says that energy is also mass (and therefore also a field), and therefore particles. So, during this very short time particles can spontaneously appear from the vacuum! We name them "\NewTerm{virtual particles}\index{virtual particles}" because they disappear very quickly and are generated by the "\NewTerm{vacuum quantum fluctuations}\index{vacuum quantum fluctuations}". We will come back on this later in the section Quantum Field Theory.
		\begin{figure}[H]
			\centering
			\includegraphics[scale=0.4]{img/atomistic/quantum_fluctuation.jpg}
			\caption[Conceptual representation of vacuum quantum fluctuations]{Conceptual representation of vacuum quantum fluctuations (source: ?)}
		\end{figure}
		However (!!!) the above "\NewTerm{time-energy uncertainty relation}\index{time-energy uncertainty relation}" (and other time-"observable" uncertainty relations that can be constructed) is (considered) not to have same meaning as previous "\NewTerm{canonical uncertainty relations}". Meaning uncertainty relations constructed from canonical dynamical variables/observables (in the Hamiltonian sense), like position and momentum, since time parameter is not an observable and also not an operator in Quantum Mechanics or Quantum Field Theory formalisms.
	
		In fact, there are various approaches and interpretations of time-energy uncertainty. For example:
		\begin{itemize}
			\item Energy-dispersion ($\Delta E$) of a state and lifetime ($\Delta t$) of the state itself
			\item Energy exchange ($\Delta E$) and time-frame ($\Delta t$) during which this can happen
			\item Energy measurement ($\Delta E$) and time ($\Delta t$) it needs for accuracy (although this is rigorously disputed)
		\end{itemize}
		..other similar or specialized formulations of the above.
		
		\item Secondly, these relations are remarkable because uncertainty is a complex value. This leads us to consider that the set $\mathbb{C}$ of complexes number is inherent to the real structure of our Univers (space-time) at the level of the quantum world (as confirm it the Casimir effect where the sum to infinity of integers prolongation with the Riemann Zeta function is equal to $-1/12$ as seen in the section of Sequences and Series). The quantum world is therefore a world of complex uncertainty. And this probability does not seem to be a consequence of our vagueness or ignorance but seems to be an intrinsic property of Nature!
	\end{enumerate}
	
	It also interesting to notice that time plays an unusual and subtle role in quantum physics. Unlike position, time is not usually treated as an operator; rather it is a simple parameter!
	\begin{tcolorbox}[title=Remark,colframe=black,arc=10pt]
	The relations and properties of commutators and anti-commutators will be indispensable to develop the quantified theory of angular momentum and spin further below.
	\end{tcolorbox}
	
	\paragraph{Representatives}\mbox{}\\\\\
	Let us now introduce contemporary quantum notations, which we will consider for now as abbreviations of integrals relating to wave functions, we will write (for the future purpose of calculating probability densities):
	
	since it is a complex functional dot product (\SeeChapter{see section Functional Analysis page \pageref{functional dot product} and Vector Calculus page \pageref{dot product}}).

	With this notation, the relation that we presented earlier in our study of operators:
	
	becomes (it is lighter already ... but less pedagogical):
	
	That said, the set $E$ of the functions $\psi$ that interest us in Wave Quantum Physics constitutes a linear functional space. Indeed, in Quantum Physics, the differential equations that we have to solve (Schrödinger's equation) to describe the behavior of a particle are such that the general solution can be very often decomposed into the sum of the particular solutions (we will prove this later!) . In mathematics, we say that "the states are linear", that is, every combination of states is still a state!
	
	Thus, the state of a particle is, as will be proved later, represented by a "\NewTerm{quantum state}\index{quantum state}" or a "\NewTerm{state vector}\index{state vector}" denoted $|\Psi\rangle$ which also corresponds to a mathematical function describing it completely.

	For example, if $|\Psi_1\rangle$ and $|\Psi_1\rangle$ are two possible states, then:
	
	is also a possible state for the system (due to the property of linear functional  spaces).

	Let us now return to our linear functional  space (or "\NewTerm{linear space of states}\index{linear space of states}"). The fact that the set $E$ of the functions $\psi$ that interest us constitutes a "\NewTerm{linear functional space}\index{linear functional space}\label{linear functional space}" means that if $\psi\in E$, $\chi\in E$ we have:
	
	regardless of the coefficients $\lambda$ and $\mu$, and also that:
	
	where for recall, $\delta_{ij}$ is the Kronecker symbol (\SeeChapter{see section Tensor Calculus page \pageref{kronecker symbol}}).
	
	\textbf{Definition (\#\mydef):}
	The base is name a "\NewTerm{complete base}\index{complete}" if of course any $\psi\in E$ can be develop in series of eigenfuctions such as:
	
	where $c_i$ is an arbitrary number (it is in part here that we have to return back to the $4$ and $5$ postulates of Wave Quantum Physics).

Let us now calculate the functional scalar product (\SeeChapter{see section Functional Analysis page \pageref{functional dot product}}):
	
	This last relation shows that we have identically (we change the notation of the indices):
	
	Thus, in a complete orthonormal basis $(\varphi_i$, an function $\psi$ will be well described by the data of the coefficients $c_i$. It will often be useful to put them in the format of the representative matrix of $\psi$ in the basis $(\varphi_i)$:
	
	Let us consider now an operator $\alpha$ such that:
	
	But we can also write (notice the apostrophe in the relation!):
	
	Let us multiply this last relation by $\bar{\varphi}_j$ and calculate the functional scalar product:
	
	To be compared with (obtained above):
	
	By noting $A_{ij}$, the "\NewTerm{representative matrix}" of $\alpha$ in the basis $\varphi_i$, we can thanks to the relation:
	
	write finally:
	
	
	\paragraph{Eigenvalues and Eigenfunctions}\mbox{}\\\\\
	Given an operator $\alpha$ (hermitian or not). The number $a$ is named the "\NewTerm{eigenvalue of the operator}" of $\alpha$, if there exists a function $\varphi$ not identically zero such that (for a reminder of similar notions see the section of Linear Algebra):

$\varphi $is then say to be the "\NewTerm{eigenfunction}" (in analogy with the "eigenvectors") of $\alpha$, associated with the eigenvalue of $a$. Notice that $a$ may very well be equal to zero (you will understand this better when we move on to the study of concrete cases).
	
	In more physical terms, this means that when a state (a mathematical function in the formal sense such as $\varphi$) is unchanged by an operator, then the state is named an "\NewTerm{eigenstate}\index{eigenstate}" or "\NewTerm{eigenvector}" of the system.

	Let us consider the set $E_a$ of eigenfunctions associated to $a$ and $a$ a functional linear space which we will name "\NewTerm{sub-eigenspace}" associated to $a$. The number of dimensions of $E_a$ is named the "\NewTerm{multiplicity}" or "\NewTerm{degeneracy order}\index{degeneracy order}" of the eigenvalue $a$, and we denote it $g$.

	Let us consider now $a$ a being a simple eigenvalue, or non degenerated: $g=1$. This means that there is only one eigenfunction associated with $a$, with a non-zero multiplicative coefficient.

	If $g=2$ (double eigenvalue), we can find two non-proportional eigenfunctions (unbounded) associated with $a$, etc.
	
	\begin{tcolorbox}[colframe=black,colback=white,sharp corners]
	\textbf{{\Large \ding{45}}Example:}\\\\
	Let us see a particular example of an eigenfunction with an eigenvalue other than the classical case using the Energy.\\
	
	Given:
	
	with $\alpha=-\mathrm{i}\hbar\dfrac{\partial}{\partial x}$ (operator that we have already seen previously) and $a$ an eigenvalue.\\

	The equation becomes:
	
	which is easily verified if:
	
	which is indeed an eigenfunction of the previously mentioned operator and which will be most useful to us in the following developements.
	\end{tcolorbox}
	
	\subparagraph{Orthogonality of eigenfunctions}\mbox{}\\\\\
	Two eigenfunctions (eigenvectors) $\varphi$ and $\varphi'$ associated with two different eigenvalues of the same Hermitian operator are orthogonal, that is to say:
	
	\begin{dem}
	Let us start this time first with the Dirac notation with two eigenfunctions and two associated eigenvalues:
	
	with $a\neq a'$.
	
	We multiply the two previous relations by $\bar{\varphi}'$ and $\bar{\varphi}$ and we integrate to get the functional scalar product:
	
	Let us recall to continue that we have proved earlier above that:
	
	So if the operator $\alpha$ is self-adjoint (which is the case of the Hamiltonian as we have shown it), that is to say that $\alpha=\alpha^\dagger$, we have:
	
	Hence, by subtracting from the relation $($\ref{orthgonalyeigen}$a)$ the conjugate complex of relation $($\ref{orthgonalyeigen}$b)$, the eigenvalue $a$ being assumed real (or an integer...), we have:
	
	which proves indeed that:
	
	since $a\neq a'$.
	
	Let us now make the same demonstration proof with the classical algebraic writing and a different one approach. We start from:
	
	and as we have proved earlier above that:
	
	So if the operator $\alpha$ is self-adjoint (which is the case of the Hamiltonian as we have shown it), that is to say that $\alpha=\alpha^\dagger$, we have:
	
	It follows that the two relations $($\ref{orthgonalyeigenclassic}$a)$ and $($\ref{orthgonalyeigenclassic}$b)$ are equal (do not forget that the eigenvalue is assumed to be real!). As we can write:
	
	We can then write:
	
	So if the eigenvalues are not equal it is forced that it is the integral that is zero (the functional scalar product) and therefore that the two eigenfunctions are orthogonal!
	\begin{flushright}
		$\square$  Q.E.D.
	\end{flushright}
	\end{dem}
	The same proof, but with the traditional and more pedagogical notation gives:
	
	If we multiply the first equation on the left by $\bar{\Psi}_2$, and the second equation by $\bar{\Psi}_1$, and that we integrate over the totality of space, we get the following two expressions (corresponding to mean values):
	
	If we take the case of real functions, we can write:
	
	Since the operator $H$ is Hermitian (self-adjoint) as we have proved it above, we have:
	
	and as $E_2$, $\Psi_1$, $\Psi_2$ are admitted as real functions, we also have:
	
	Therefore:
	
	can be written:
	
	Therefore it comes:
	
	which shows well that $\Psi_1$, $\Psi_2$ are orthogonal according to the definition of the functional scalar product.
	
	\paragraph{Dirac formalism}\label{dirac formalism}\mbox{}\\\\\
	Dirac has conceived a very practical general formalism, world-wide used by physicists, of which we will give the essential elements. The notations used have already been partially introduced in the preceding paragraphs.

	We will use the Dirac formalism for two points, the first being to better understand what has been seen so far when introducing to functional operators, the second being to introduce a notation and a method of solving that we can found in some textbooks. Moreover, in this book by simplification of writing, we will sometimes use this formalism.

	\subparagraph{Kets and Bras}\mbox{}\\\\\
	We consider a vector space $E_n$ with $n$ dimensions where $n$ can very well be infinite (Hilbert space). A vector is defined by $n$ components $x_i$ with $i=1,2,\ldots,n$ which we can store in column to form a column-matrix:
	
	We will say that this matrix describes the "\NewTerm{right vector}" or the "\NewTerm{ket $| x\rangle$}\index{ket}" (this must remind the "representatives"). It is possible to associate with the column-matrix the adjoint matrix (conjugated transpose for recall...):
	
	where the $\bar{x}_i$ are the conjugate complexes of the $x_i$. We will say that the adjoint line matrix describes the "\NewTerm{left vector}" or the "\NewTerm{bra $\langle x|$}\index{bra}" (this must also remind you the "representatives").
	
		The addition and multiplication by an number $\lambda$ are evident. Notice that if.
	
	we have trivially:
	
	
	With two vectors of components $x_i$ and $y_i$, we can then form the following quantity, named "\NewTerm{Hermitian scalar product}\index{Hermitian scalar product}" or "\NewTerm{Hermitian inner product}\index{Hermitian inner product}":
	
	We agree to write it $\langle x | y \rangle$. Notice that:
	
	the Hermitian scalar product is therefore not simply commutative!
	
	The product $\langle x|y \rangle$ depends linearly on $\langle x |$ and $|y\rangle$. Conversely, if a number $Q$ depends linearly on a ket $|x\rangle$, there exists a bra $\langle a |$ such that:
	
	In quantum physics, $\langle x|y\rangle$ is associated to the "amplitude" of being in the state $x$ if the system is in the state $y$. This Hermitian scalar product will be interpreted as the probability that the physical system is projected into the state $x$ if it is in the state $y$.

	An orthonormal basis of the studied space is constituted by $n$ vectors $|i\rangle$ such that (\SeeChapter{see section Vector Calculus page \pageref{orthogonal basis}}):
	
	where for recall, $\delta_{ij}$ is the Kronecker symbol (\SeeChapter{see section Tensor Calculus page \pageref{kronecker symbol}}).

Any vector (or function if we generalize vector space to function space) $|x\rangle$ of $E_n$ can be developed on this basis according to (\SeeChapter{see section Vector Calculus page \pageref{vector linear combination}}):
	
	where the $x_i$ are the components of $|x\rangle$ in the chosen base. We can easily verify that (already seen many times in the section of Vector Calculus):
	
	If a ket $|y\rangle$ depends linearly on a ket $|x\rangle$, we write symbolically:
	
	where $\alpha$ is a linear operator. 

	Given a linear operator defined by the preceding relation and a bra $\langle u|$, the Hermitian scalar product:
	
	is a number $Q$ which depends linearly on $|x\rangle$. From what has been seen above, there is a bra $\langle v|$ such as $Q=\langle v|x\rangle$. Then $\langle v|$ obviously depends of $\langle u|$ linearly. We agree to put:
	
	Using this convention we can write:
	
	If $|y\rangle=\alpha|x\rangle$, depends linearly on $\langle x|$. By definition, we will write:
	
	where for recall $\alpha^\dagger$ is the adjoint of $\alpha$.
	
	Let us form with a bra $\langle u|$ the Hermitian scalar product:
	
	and we can write (as we have proved it previously):
	
	Hence the relation of the first importance which we have already met several times without really explaining its origin:
	
	We simply recall with this relation that a Hermitian (self-adjoint) operator is an operator equal to his adjoint.

	Thanks to Dirac's formalism, what was before abstract definitions has now become evident evidence.

	To summarize:
	\setlength\extrarowheight{10pt}
	\begin{table}[H]
		\centering
		\begin{tabular}{|c|c|}
		\hline
		\rowcolor[HTML]{9B9B9B} 
		\textbf{Classic Formalism} & \textbf{Dirac Formalism} \\ \hline
		$\displaystyle\int \bar{\varphi}_i\varphi_j\mathrm{d}V$ &  $\langle \varphi_i | \varphi_j\rangle$\\ \hline
		$\displaystyle\int \bar{\varphi}_i\alpha\varphi_j\mathrm{d}V$ & $\langle \varphi_i | \alpha | \varphi_j\rangle$ \\ \hline
		$\overline{\displaystyle\int \bar{\varphi}_i\varphi_j\mathrm{d}V}=\displaystyle\int \bar{\varphi}_j\varphi_i\mathrm{d}V$ &  $\overline{\langle \varphi_i|\varphi_j\rangle}=\langle \varphi_j|\varphi_i\rangle$\\ \hline
		$\overline{\displaystyle\int \bar{\varphi}_i\alpha\varphi_j\mathrm{d}V}=\displaystyle\int \bar{\varphi}_j\alpha^\dagger\varphi_i\mathrm{d}V=\displaystyle\int\bar{\varphi}_j\alpha\varphi_i\mathrm{d}V$ &  $\overline{\langle \varphi_i|\alpha|\varphi_j\rangle}=\langle \varphi_j|\alpha|\varphi_i\rangle$\\ \hline
		$\Psi=\displaystyle\sum_i c_i\varphi_i$ & $|\Psi\rangle=\sum_i c_i|\varphi_i\rangle$\\ \hline
		$\displaystyle\sum_i c_i^{'}\displaystyle\int\bar{\varphi}_j\alpha\varphi_i\mathrm{d}\vec{x}$ & $c_i=\langle \varphi_i|\psi\rangle$ \\ \hline
		$c_i=\displaystyle\int\bar{\varphi}_j\psi\mathrm{d}V$ & $c_i=\langle \varphi_i |\psi\rangle$ \\ \hline
		\end{tabular}
		\caption{Comparison between Classic and Dirac formalism}
	\end{table}
	\setlength\extrarowheight{0pt}
	So the reader has probably noticed that Dirac bra-kets and their algebra are very much like the Linear Algebra.

	A ket is like a vector, a bra is like the conjugate transpose of a vector, a bra-ket is like a complex inner product, a ket-bra is like an outer product, and operator is like a matrix, and operator acting on a state is like matrix product. Even we speak about eigenstates which are very much like eigenvectors.

	So how is it happened, that we invented a new notation for Linear Algebra instead of sticking to the usual notation of vectors and matrices as usual?

	So in fact it is not that it's "like Linear Algebra" but it is exactly Linear Algebra!!!! At the time of Dirac Linear Algebra was not taught to all undergraduates\footnote{In the first decade (when they were all students) of 20th century matrix multiplication was not taught to students on a regular basis in the best European universities.}. So he invented his own notation. Physicists found them convenient for the questions they consider. The clear evidence of this is that Heisenberg had to invent matrix multiplication when he created quantum mechanics... Actually there is probably a causal relation between the invention of quantum mechanics and the spread of linear algebra as a part of the standard curriculum.
	
	\begin{tcolorbox}[title=Remark,colframe=black,arc=10pt]
	Again, an excellent practical example of the application of the Dirac formalism is proposed in the section on Quantum Computing.
	\end{tcolorbox}
	
	\subsection{Schrödinger Model}
	Experiments (Compton effect, photoelectric effect, Young's slits, geometric / wave optics, etc.) have shown that waves can be treated as corpuscles (and vice versa). It is these observations which led Niels Bohr to state his "\NewTerm{principle of complementarity}\index{principle of complementarity}" which says that, according to the experiments carried out, we must consider matter either as a wave or as corpuscles. These two aspects complement each other.
	
	\subsubsection{de Broglie associated wave}\label{de broglie associated wave}
	In 1924, the French physicist Louis Victor de Broglie suggested that particles (electrons, protons, and others, and even atoms and molecules) could also, in some cases, exhibit wave properties in the same way as the light! De Broglie then expressed the idea that there existed between the associated fictitious wavelength of a particle of matter and its momentum a relation similar to that of a photon, that is to say ($\nu$ is the notation for the frequency for recall...):
	
	so we can write using the relation established in the section of Wave Mechanics:
	
	Where:
	
	is named the "\NewTerm{de Broglie relation}\index{de Broglie relation}\label{de Broglie relation}".
	
	Hence the ratio:
	
	De Broglie then made the following hypothesis: For a corpuscle of mass $m$ and velocity $v$ we have:
	
	where $\lambda$ is named the "\NewTerm{associated de Broglie wavelength}\index{associated de Broglie wavelength}".
	
	The moving matter would therefore have an associated wavelength!? It is an extremely small wavelength for masses of the order of a kilogram. If the speed is for example of the order of $1\;[\text{m}\cdot\text{s}^{-1}]$ then $\lambda\cong 10^{-34}$ [m].

	As we have seen it, the phenomena of interference and diffraction are important only when the size of the objects or slits is not much greater than the wavelength. It is therefore impossible to detect the wave properties of everyday objects. It is not the same for elementary particles, the electrons in particular.
	
	The electrons can therefore have wavelengths of the order of $10^{-10}$ [m] which corresponds to the spacing of the atoms of a crystal. C. J. Davisson and L. H. Germer carried out a crucial experiment: they scattered electrons on the surface of a crystal and in early 1927 observed that the ejected electrons were distributed in regular peaks. When they interpreted these peaks as diffraction peaks, they found that the wavelength of the diffracted electron was exactly that predicted by de Broglie!
	
	But then what is an electron?? The illustrations that show an electron as a minuscule negatively charged sphere are only convenient but inaccurate images. In fact, we have to use the corpuscular or wave model, the one that works best according to the situation so that we can understand what is happening. But we must not conclude that an electron is a wave or a particle. Rather, we should say that an electron is "all of its measurable properties". Some physicists still use the term "\NewTerm{quanton}\index{quanton}" to describe any system behaving either as a wave or as a particle.
	
	De Broglie was then able to suggest that each quantified electron orbit (according to Bohr's quantization postulate) is then a stationary wave and that an electron can only occupy orbits that can accept an integer number of wavelengths of its associated fictitious wave (today we assimilated this rater to the fact that the electron is trapped in a potential wells). If there was no exact coincidence, there could be no stationary wave, and therefore no stationary orbit either.
	
	Consequently, as for the resonant modes of a string (standing waves), only the waves whose circumference of the circular orbit contains an integer number of $\lambda$ exist, or (the amplitude will be calculated by particular techniques which will be seen further):
	
	with $n\in\mathbb{N}^{*}$.
	\begin{figure}[H]
		\centering
		\includegraphics[scale=1]{img/atomistic/de_broglie_associated_wave.jpg}
		\caption[Approach to the wave aspect of orbits by de Broglie]{Approach to the wave aspect of orbits by de Broglie (source: ?)}
	\end{figure}
	By replacing $\lambda$ by $h/mv$, we get:
	
	That is indeed the quantum condition proposed by Bohr (\SeeChapter{see section Corpuscular Quantum Physics page \pageref{bohr postulates}}). The orbits and the quantized energy states of the Bohr model are due to the wave nature of the electron and to the fact that only resonant stationary waves persist. This supposes that the wave-particle duality is the basis of the structure of the atom.

	The wave's notion of the particle then allowed the physicist Erwin Schrödinger to develop a "wave equation" to describe the wave properties of particles.

	Let us do a small sympathetic interlude... since the associated wave of de Broglie known and given the result seen during our study of the virial theorem in the section of Continuum Mechanics, we can put relate together:
	
	Thus, for a fluid (liquid), we can get the value of the associated "\NewTerm{de Broglie thermal wave}\index{de Broglie thermal wave}". Which gives us:
	
	We will come back to this relation during our study of superfluids in the section of Continuum Mechanics.
	
	\pagebreak
	\subsubsection{Classical Schrödinger Wave Equation}\label{schrodinger wave equation}
	The physicist Peter Debye found the de Broglie model a bit far fetched. He argued that the physics of waves, from the sound waves to the electromagnetic waves, even the waves propagating on a string requires an equation that describes them. There was no wave equation for de Broglie's atomic model (for the simple reason that the latter had never tried and Albert Einstein either). The physicist Erwin Schrödinger then took it upon himself to find the missing equation and did it brilliantly.

	Let us recall the one-dimensional form of the wave equation (\SeeChapter{see section Wave Mechanics page \pageref{wave equation}}):
	
	To simplify, let us look for a particular solution of the form (see the section on Wave mechanics or the section on Electrodynamics for the analogy):
	
	$\Psi(x)$ is the amplitude of the field associated with the particle. It is important to notice that the periodic part does not contain any displacement parameter $k$ (as is the case in Electrodynamics, for example) because the function must describe "static" solutions (be careful not to take this literally).
	
	For historical reasons this amplitude is as we know commonly named "wave function" although this name is misleading. It might be better to name it simply "amplitude field associated with matter".
	
	It is the search for the expression of this function that will lead us during the study of a special case (see much further below in text) to the well-known expression of the ionisation energy of an electron of given main quantum number $n$ and for its atom of given atomic number $N$.
	
	If we introduce \ref{univariatewaveequationparticularsolution} into \ref{univariatewaveequation}, we get:
	
	We have also:
	
	
	Hence:
	
	if we introduce \ref{variablechangeschrodinger} into \ref{wavesolutioninwaveequation} then we get the "\NewTerm{classical one-dimensional Schrödinger equation}\index{classical one-dimensional Schrödinger equation}\label{classical one dimensional schrodinger equation}" (in the absence of magnetic field ...):
	
	\begin{tcolorbox}[title=Remark,colframe=black,arc=10pt]
	The potential energy could be gravitational as well as electric or both combined (therefore of any kind). But gravity is assumed to be so small at such scales compared to electrostatic forces that it is neglected.
	\end{tcolorbox}
	We can rewrite the preceding equation by generalizing it to a three-dimensional system. What ultimately gives us:
	
	where for recall $\Delta$ is the Laplacian of a scalar field (\SeeChapter{see section Vector Calculus page \pageref{scalar laplacian}}):
	
	\begin{tcolorbox}[title=Remarks,colframe=black,arc=10pt]
	\textbf{R1.} This equation is not a Lorentz invariant given that it was established from the classical expression of energy (and not the relativistic one).\\
	
	\textbf{R2.} The plane wave function that we have taken at the beginning does not have a physical significance since it carries an infinite energy. A better solution is to consider a wave packet. However, the wave packets generally employed consist of a superposition of plane waves (see proof further below). Hence, by studying its effects on one of the plane waves, we can still accept the physical conclusions which we can deduce from them...
	\end{tcolorbox}
	However as simple as it may seem, it took all the skill and experience of Schrödinger to be the first to write this wave equation and it was the foundation upon which he built the mathematical edifice of Wave Quantum Physics in the following months. But first it had to prove that it was indeed the wave equation sought by applying it to the hydrogen atom for which some results were well known thanks to Sommerfeld results (\SeeChapter{see section Corpuscular Quantum Physics page \pageref{relativistic sommerfeld model}}). His model eliminated all the successive and empirical tinkering of Corpuscular Quantum Physics.

	Some physicists judged Schrödinger's model to be pure madness (in particular Sommerfeld), but changed his mind shortly afterwards... so much so that the model was effective and replaced advantageously the horrible abstract matrix model of Heisenberg on which even Pauli had failed for the experimental results of the hydrogen atom. Soon after Max Born described the wave model as the deepest form of quantum laws, which obviously did not please his close friend Heisenberg...
	
	However, the Heisenberg model described particles as Schrodinger describes waves. The advantage of having two different formalisms - but equivalent - in quantum physics quickly became evident. For most of the problems physicists encounter, wave mechanics offers the easiest way to the solution. However, for others, such as those involving spin, Heisenberg's matrix approach proves (or rather "defines") its value (\SeeChapter{see section Relativistic Quantum Physics page \pageref{emerging electron spin value}}).
	
	\paragraph{Schrödinger Hamiltonian}\label{schrödinger hamiltonian}\mbox{}\\\\\
	The Schrödinger equation can also be written as (after a few small elementary factorizations) as follows:
	
	We write this in quantum physics in the form:
	
	Where $H$ is therefore the Hamiltonian of the system (or total energy) and constitutes a functional operator and where the total energy $E_\text{tot}$ is its eigenvalue.
	
	The Schrödinger equation is therefore an equation with partial derivatives of the second order, homogeneous linear. Whatever the total energy, it admits solutions (phew!), but we show that in general these solutions grow very rapidly (exponential growth) when we go to infinity in certain directions and are therefore physically unacceptable. There are only particular values of the total energy which give rise to physically acceptable solutions and in general all of these values include discrete values (trigonometric functions at the source with integer parameters) which are the "\NewTerm{linked levels}" of the system (because their eigenfunction decreases rapidly to infinity) and a continuum of values which are the "\NewTerm{unbound levels}" (their eigenfunction remaining finite at infinity). More precisely, if $W$ is the lower limit of the potential energy values at infinity, the bounded levels lie below $W$, while the values greater than $W$ constitute the continuum of the unbound levels.

	For example, in the study of the harmonic oscillator (one of the most difficult practical cases in terms of formalism) that we will do later, we will prove that we have:
	
	with $W=+\infty$. So there are only linked levels.

	For the hydrogen atom:
	
	with $W=0$. Linked levels are negative, therefore all positive energy values will be unbound levels.
	
	This having been said, let us also consider as an example (very important) how to determine the Hamiltonian $H$ of the Schrödinger equation of a non-relativistic charged particle in an electromagnetic field!

	We have seen in the section of Analytical Mechanics that the (classical) Lagrangian was defined by the subtraction of kinetic and potential energy according to the relation:
	
	We have proved in the section of Electrodynamics that the Lagrangian of the relativistic field-current interaction was given by:
	
	where for recall $\phi$ is the vector potential of the electric field (whose gradient is the electric field $\vec{E}$) and $\vec{A}$ is the vector potential of the magnetic field (whose curl is the magnetic field $\vec{B}$).

	If we add an electric field (and therefore an electrostatic potential $U$) in addition to the electromagnetic field, the Lagrangian is then written (since the potential is subtracted according to the Lagrangian definition):
	
	In the classical (non-relativistic) approximation we know that we have $v/c \ll 1$ and using the Maclaurin series of (\SeeChapter{see section Sequences and Series page \pageref{usual maclaurin developments}}):
	
	we can then write:
	
	As we restrict ourselves to the non-relativistic case, we can eliminate the constant term of energy of the mass at rest such that:
	
	Still in the section of Analytical Mechanics, we showed that the Hamiltonian was given by:
	
	Therefore we have
	
	Moreover, we have seen in the section of Analytical Mechanics that:
	
	It therefore comes that:
	
	Finally:
	
	Either after simplification:
	
	$H$ therefore contains the kinetic energy and total potential energy. There is no magnetic term because the Laplace force, as we proved it in the section of Magnetostatic, does not work (there are some who are lucky...). $H$ is indeed the total energy of the classical system, but the preceding relation is not really adapted to Hamilton's formalism because the conjugate moments do not appear. But it is very simple to introduce them from the result obtained previously which was:
	
	Therefore:
	
	If we go into Quantum Physics, we have to replace the $p_i$ by their respective operators:
	
	of which we have already proved the origin earlier above. Thus, we have that:
	
	which must be written in the general case (as we do not know if the vector potential commutes with the linear momentum operator, we will assume that it does not commute):
	
	What is traditionally written in the form (sic!)
	
	This last relation is very often found in the following simplified form in textbooks in the absence of potential $U$ and by explicating $V$:
	
	\begin{tcolorbox}[title=Remark,colframe=black,arc=10pt]
	In the section of Relativistic Quantum Physics we will prove the relativistic form of this Hamiltonian associated with the generalized Klein-Gordon equation or that of Dirac which includes the $1/2$ spin!
	\end{tcolorbox}
	Fortunately, we will not discuss examples where we will have to find solutions to the Schrödinger equation with such a Hamiltonian in this book...
	
	In the academic case, however, when application premises are addressed ..., we cancel either the vector potential (ie the particle is not immersed in a magnetic field) or the scalar potential (ie the particle is not immersed in an electric field). Moreover, when one of the fields is chosen as being non-zero, we take the case where it is constant and in a single dimension... Thus, if we want a constant electric field along a single axis (for example the $x$-axis), we take as scalar potential:
	
	since its gradient will give a constant according to $x$. In the case of a constant magnetic field along a single axis (for example the $z$-axis), we will arrange to take the vector potential:
	
	whose rotational (curl) gives a constant magnetic field according to $z$.
	
	\paragraph{De Broglie normalization condition}\mbox{}\\\\\
	In general, in a given dynamic state, the particle (if it is a one-particle system) described by the resolution of the Schrödinger equation for well-defined parameters is badly localized because $x$, $y$ and $z$ are not well defined by Heisenberg's principle of uncertainty. It is therefore necessary to define a probability $\mathrm{d}P$ of finding the particle in the volume element $\mathrm{d}x\mathrm{d}y\mathrm{d}z$ surrounding a point $(x, y, z)$, hence the existence of a distribution function of the coordinates such that:
	
	where $\rho(x,y,z)$ is therefore an essentially positive or zero quantity (probabilities obliged!) which must be expressed by means of the Schrödinger equation function $\Psi(x,y,z)$. We have, moreover, such very detailed examples in this section and at the end of that of Quantum Chemistry.
	
	Analogies with Classical WavePhysics, more precisely with Electrodynamics, have led us to admit that since the density of energy of an electromagnetic wave is proportional to the square of its amplitude (\SeeChapter{see section Electrodynamics page \pageref{poynting vector}}), the volumic density probability must be proportional to the square of the intensity of the associated field such that:
	
	where we use the Schrödinger function module as an amplitude analogy and where the constant is a real number. In the framework of Quantum Physics, it is much more common to find this last relation in the following obvious form:
	
	which highlights the necessary normalization of the Schrödinger equation.
	
	Finally, it is important to know that during mathematical developments, the huge majority of physicists have the habit of keeping the same notation for the non-normalized Schrödinger function as for the normalized one (which can lead to some confusion) such as:
	
	where $\lvert\Psi(x,y,z)\rvert^2$ then represents the density probability of finding the particle at a certain point in space.
	
	It is evident then that with this way of writing things we have then over all space:
	
	We can now consider the physical significance which may be attached to the intensity of the field associated with matter. Since this field describes the motion of a particle, we can say that the regions of space in which the particle is most likely to be found are those in which the $\lvert\Psi(x,y,z)\rvert^2$ is maximum.
	\begin{figure}[H]
		\centering
		\includegraphics[scale=0.8]{img/atomistic/atom_evolution.jpg}
		\caption[Atom model evolution]{Atom model evolution (source: ?)}
	\end{figure}
	Let us also inform you that the prior-previous relation is written using the notation Dirac ket-bra in the very refined (and very common...) way:
	
	or the following (it is simply the square root of the previous one):
	
	where the module in the denominator disappears, since, for recall, the integral is a real number. It should never be forgotten that physicists, for the vast majority, note in a similar way the non-normalized and normalized Schrödinger function as this last relation reminds us.
	
	As we have already said, we will see many detailed examples of this normalization in this section with one-dimensional spaces and in the context of volumes in the section on Quantum Chemistry.
	 \begin{tcolorbox}[colframe=black,colback=white,sharp corners]
	\textbf{{\Large \ding{45}}Example:}\\\\
	A ball is constrained to move along a line inside a tube of length $L$. The ball is equally likely to be found anywhere in the tube at some time $t$. What is the probability of finding the ball in the left half of the tube at that time? (the answer is $50\%$, of course, but how do we get this answer by using the probabilistic interpretation of the quantum mechanical wave function?)\\
	
	The first step to the answer is to write down the wave function. The ball is equally like to be found anywhere in the box, so one way to describe the ball with a constant wave function:
	\begin{figure}[H]
		\centering
		\includegraphics{img/atomistic/ball_in_a_tube.jpg}	
		\caption[Wave function for a ball in a tube of length $L$]{Wave function for a ball in a tube of length $L$ (source: OpenStax)}
	\end{figure}
	The normalization condition can be used to find the value of the function and a simple integration over half of the box yields the final answer.\\
	
	The wave function of the ball can be written as:
	
	where $c^{te}$ is a constant, and  $\Psi(x,t)=0$ otherwise. We can determine the constant $C^{te}$ by applying the normalization condition (we set  $t=0$  to simplify the notation):
	
	This integral can be broken into three parts: (1) negative infinity to zero, (2) zero to$ $L, and (3) $L$ to infinity. The particle is constrained to be in the tube, so $C^{te}=0$ outside the tube and the first and last integrations are zero. The above equation can therefore be written:
	
	\end{tcolorbox}
	
	\begin{tcolorbox}[colframe=black,colback=white,sharp corners]
	The constant $C^{te}$ does not depend on $x$ and can be taken out of the integral, so we obtain:
	
	Integration gives obviously:
	
	To determine the probability of finding the ball in the first half of the box $( 0\le x\le L)$, we have:
	
	The probability of finding the ball in the first half of the tube is $50\%$, as expected. Two observations are noteworthy. First, this result corresponds to the area under the constant function from $x=0$ to $L/2$ (the area of a square left of $L/2$). Second, this calculation requires an integration of the square of the wave function. A common mistake in performing such calculations is to forget to square the wave function before integration.
	\end{tcolorbox}
	
	Let us say one more thing about normalization. If you observe the Hamiltonian expression of the Schrödinger equations seen so far, then if $\lambda$ is a real or complex constant we always have:
	
	If we assume that $\Psi$ is a solution of the Schrödinger equation, we see that $\lambda\Psi$ is also a solution of the equation. Indeed, we get:
	
	Taking into account the fact that the function $\lambda\Psi$ is normalized, we then get:
	
	hence (some books restrict themselves to these solutions for pedagogical reasons):
	
	Or rigorously we have more generally:
	
	where $\theta$ is a real number. This is what we name the "\NewTerm{phase arbitrariness}" which we had already mentioned at the beginning of this section without proof. We will also come back on this in the section of Quantum Field Theory for our study of Gauge theories applied to Quantum Physics.

	These solutions are normalized and correspond to the same energy value $E$ as the same probability density. This shows that it is not useful to look for the meaning of a negative or complex value of $\Psi$ (if we take the particular pedagogical case of $\lambda=-1$), because $|\Psi|^2$ is real and is not negative (but nothing prevent us that the fact we require a result in $\mathbb{R}$ is wrong...). Only the square of a wave function, which corresponds to the probability density, seems to be interesting from a physical point of view.
	
	\paragraph{Bound and unbound states}\mbox{}\\\\\
	Let us suppose that $\Psi(x,y,z)$ decreases rapidly to infinity, so that the integral:
	
	converge. It is then possible to take advantage of the arbitrariness prevailing on the wave function (the fact that $\Psi$ and $\lambda\Psi$ describe the same state) to make this integral equal to unity. We then say that $\Psi$ is a "\NewTerm{standardized field state function}":
	
	We notice that there is still an arbitrary on $\Psi$ by a complex number of module $1$, $e^{\mathrm{i}\theta}$, without the condition of normalization being altered. We know that this is the "phase arbitrariness" and we have already introduced it just before!
	
	Such a dynamic state is named "\NewTerm{bound state}\index{bound state}" or "\NewTerm{bound level}", because the particle manifests itself in a limited region of space because of a potential. When, for example, the hydrogen atom is located on a fundamental level, it is in a bound state. We know that there is no chance of finding the electron at more than a few angstroms of the proton, treated as infinitely heavy and originally placed as we have seen in the study of the Bohr model. Here is a good schematic view of the thing (bound state):
	\begin{figure}[H]
		\centering
		\includegraphics{img/atomistic/bound_state.jpg}	
		\caption[Pictorial representation of a bound state]{Pictorial representation of a bound state (source: Pour la science)}
	\end{figure}
	An example of an "\NewTerm{unbounded (default) state}\index{unbounded state}" is the free particle that can propagate indefinitely in all directions of space (by the way, for the latter example it is a little more complicated ... but we will deal with it later).
	\begin{tcolorbox}[title=Remark,colframe=black,arc=10pt]
	It is perhaps good to know that these bound state concepts have classical analogues. Thus, the bound levels of the hydrogen atom correspond to the elliptic orbits, the unbound (positive energy) levels correspond to the hyperbolic orbits.
	\end{tcolorbox}
	
	\subsubsection{Classical Shrödinger equation of evolution}
	We know that in Classical Mechanics the dynamic state of a system evolves, in general, in time. This means that the position and the linear momentum (for example) are a function of time. For a given Hamiltonian system, knowledge of the initial dynamic state makes it possible to predict exactly the future evolution of this system due to the well-known properties of Hamilton's equations.
	
	In Quantum Physics, dynamic states will generally evolve over time. The wave function describing a dynamic state will then not only depend on the coordinates of the particles constituting the system, but it will also depend on time and be written:
	
	It is quite natural to assume, by analogy with Classical Mechanics, that for a given Hamiltonian system the knowledge of the initial dynamic state at the instant $t_0$ allows us to predict what will be the dynamic state of the system at a later time $t>t_0$.
	
	On the way let us notice that this is equivalent to say that an initially "pure set" remains a pure set in the during the subsequent evolution of the systems which constitute it without external action. It would therefore cease to be true if all the systems of the whole had not exactly the same Hamiltonian.
	
	Let us indicate that there are at least two simple possible approaches to determine the time-dependent functions:
	\begin{itemize}
		\item The first one, common in many fields of application of Quantum Physics, consists in using an "evolution operator" and makes it possible to make explicit the Schrödinger equation of evolution. We will begin with this one even if it is the most complicated or abstract approach in our point of view.

		\item The second, which is widely used for educational purposes, allows the time-dependent functions to be obtained by means of the technique of separation of the variables of differential equations (\SeeChapter{see section Differential and Integral Calculus page \pageref{separation vaiables method}}) but requires to admit the Schrödinger equation of evolution as a postulate.
	\end{itemize}
	
	\paragraph{Operator of evolution}\mbox{}\\\\\
	Given $\Psi(t)$ the normalized wave function describing the dynamic state of the system at time $t$ (we do not write the other variables on which equation depends for simplicity, ie the spatial coordinates of the system's particles!). From the above, if $\Psi(t_0)$ is known, $\Psi(t)$ is also known. Then we have the correspondence:
	
	and we will admit that this dependence is linear! There is therefore an operator $T(t,t_0)$, named "\NewTerm{evolution operator}\index{evolution operator}", such that:
	
	The function $\Psi(t+\mathrm{d}t)$ depends also linearly on $\Psi(t)$. Then the same applies to:
	
	Therefore we assume that it exist a linear operator $K$, such that:
	
	where the pure imaginary complex number $i$ simply comes from the fact that we intuitively guess that the result will be a complex wave function. What also led the physicists to write the latter equality thus were the known results of the wave equation describing a dynamic state according to the idea of de Broglie. We will then show why writing this equality is justified.
	
	We have to determine $K$. Since the knowledge of the Hamiltonian $H$ controls the evolution of the system, so $K$ must probably depend on $H$. To define the law that binds $K$ to $H$, we will examine a particular case, that of the free particle (for which e will make a detailed study later). In this case, $H$ is identifies with kinetic energy only.
	
	According to de Broglie's ideas it is natural to assume that the wave function describing a dynamic state in which the linear momentum is well-defined, ie $p=\hbar k$ (relation proved during the study of the free particle), and where the total energy is thus also well defined, ie equal $E=\hbar\omega$, is a plane wave of the classical form:
	
	where as we already know, and for recall, $\vec{k}$ is the wave vector of the wave and $\vec{r}=(x,y,z)$ its spatial coordinates.
	
	It is common in Quantum Physics to take a wave that propagates in the $+X$ direction. Therefore (\SeeChapter{see section Wave Mechanics \pageref{wave equation}}):
	
	But we have the relation between the Hamiltonian operator and the corresponding eigenvalue:
	
	The two preceding equations lead us to write:
	
	By comparing this last relation with:
	
	We are led to put:
	
	Physicists assume that this relation between $K$ and $H$ is general. Then the equation:
	
	In which $K$ is replaced by its expression:
	
	Therefore becomes:
	
	This equation constitutes the "\NewTerm{Schrödinger's classical evolution equation}\index{Schrödinger's classical evolution equation}" or  "\NewTerm{time-dependant Schrödinger's equation}\index{time-dependant Schrödinger's equation}" which allows to study systems evolving over time, in particular the processes involving the absorption and the emission of radiation and the diffusion of the radiation by the atoms.
	
	In particular, for a spinless particle subjected to a potential energy $E_p$, by always maintaining that the relation between $K$ and $H$ is general, the equation of evolution is then written:
	
	where the terms in parentheses correspond to the expression of the Hamiltonian.

	It is now necessary to solve the Schrödinger differential equation of evolution. For this, we will use the condition of normalization of de Broglie.

	Let us recall that this condition is written:
	
	and let us generalize this condition to a multidimensional and temporal study such that (according to the properties of complex numbers):
	
	Let us now perform the derivative with respect to time of the previous integral. We therefore necessarily have:
	
	and let use the Schrödinger equation of evolution:
	
	What gives us for our integral after substitution:
	
	Let us now prove that we can write:
	
	This would then be equivalent to the proof that $H$ can act identically "backwards" such that:
	
	where $H$ can be (or "contain" if you prefer) an operator (differential one for example).
	
	This equality can be proved if and only if $\Psi$ is a decreasing function  towards infinity and whose derivative tends towards zero towards the infinite (typical of bounded states)!
	
	Let us prove this on a particular case (but frequent in physics) and to see how this can be done, let us consider in $H$, a particular term of the following form (we don't care of the constant corresponding the potential energy as it doesn't change anything in the result below):
	
	Which leads us to write:
	
	By integration by parts (\SeeChapter{see section Differential and Integral Calculus page \pageref{integration by parts}}) on the differential operator term $\partial_q (g\Psi(t,q))$, we get:
	
	But, since $\Psi$  is a function decreasing towards infinity by hypothesis (physically necessary for bound states!), we will have the first term which will always be zero. It remains to us therefore:
	
	So it makes no difference to consider that the operator differentiates everything that is on the right or whatever what is on the left, so far as it must be accepted that the latter case involves a change of sign and that the considerate states are bounded. It is customary to name this result sometimes the "\NewTerm{condition of hermicity}" ....
	
	So we are indeed authorized to write:
	
	Which also leads us to write (this highlight why the constant related to the potential energy was ignored in the proof above as it is ubvious that the constants cancels each other):
	
	This can only be satisfied if:
	
	And in the mathematical field dealing with operators we saw that we should note this equality:
	
	Which brings us to write:
	
	Or using the representative (ket-bra) notation:
	
	To return to the resolution of:
	
	It is obvious that a possible solution is then:
	
	which is therefore a made of purely spatial (independent of time) component and a time-dependent complex exponential. Let us check this:
	
	This is what had to be proved (...).

	Let us also notice that once the purely spatial solutions are determined, the time- and space-dependent solutions are easily obtained.
	
	Similarly, thanks to the relation $H^\dagger=H$ that we have demonstrated before, we can write:
	
	Finally, the relation:
	
	becomes:
	
	with the "\NewTerm{Heisenberg operator (of evolution)}\index{Heisenberg operator (of evolution)}" defined by:
	
	\begin{tcolorbox}[title=Remark,colframe=black,arc=10pt]
	It may very well happen that $X$ is sometimes a simple constant (we will see an example further below).
	\end{tcolorbox}
	 One way is to think of the operator $X$ as time-independent and to consider all of the time dependence of its expectation value as coming from the state vector. This point of view is known as the "\NewTerm{Schrödinger picture}\index{Schrödinger picture}". Alternatively, we may think of the operator as evolving in time, while the state vector stays constant. This is known as the "\NewTerm{Heisenberg picture}\index{Heisenberg picture}". 
	 
	 To sum up, the time evolution of a state vector following Schrödinger picture is given by:
	
	or written more generally:
	
	or in representative notation:
	
	and that of an operator in the Heisenberg picture is:
	
	The Heisenberg picture is useful because we can see a closer connection to classical physics than with the Schrödinger picture. In classical physics, we describe the evolution of a system in terms of the time evolution of the observables, such as position or angular momentum, as dictated by the classical equations of motion. Classical mechanics does not include the concept of state vectors, as quantum mechanics does.
	
	\paragraph{Conserved Quantities and Commutators}\mbox{}\\\\\
	Let us recall the time-dependent Schrödinger equation obtained just earlier:
	
	We saw earlier that the expectation value of an observable physical quantity $\mathcal{O}$ is given by:
	
	where for recall $\hat{\mathbb{O}}$ is the operator representing the quantity (observable) $\mathcal{O}$. For $\mathcal{O}$ to be conserved, its value $\langle \mathcal{O} \rangle$ must not change in time, so the question is, when is  $\langle \mathcal{O} \rangle$ independent of time? To answer that question, we assume that  $\langle \hat{\mathcal{O}} \rangle$ is independent of time and compute $\mathrm{d}\langle \mathcal{O} \rangle/\mathrm{d}t$ as follows:
	
	The complex conjugate Schrödinger equation is:
	
	Using the fact that $H$ is real or hermitian, this can be written (in the case where the Hamiltonian is a matrix we use here the property $(AB)^T=B^TA^T$ proved in the section of Linear Algebra page \pageref{transposed matrix}):
	
	Therefore combining the two previous time-dependent Schrödinger equation into the previous integral we get:
	
	So if the observable $\mathcal{O}$ is conserved then we must have:
	
	That is:
	
	Hence a conserved quantity in quantum mechanics can be defined as one for which the probabilities of measuring the various eigenvalues for that quantity are independent of time in all states. Mathematically conserved quantities correspond to Hermitian operators $\hat{\mathcal{O}}$ satisfying:
	
	This is a very important result for our study of Wave Quantum Physics and also for the next section of Relativistic Quantum Physics!
	
	\paragraph{Schrödinger picture by separation of variables}\mbox{}\\\\\
	Let us also see an interesting mathematical manipulation and somewhat similar to the previous one of Schrödinger's equation of evolution. This manipulation will allow us to see that the separation of variables works very well with the equation of evolution and that it will allow us to fall back on a result obtained previously (it is always pedagogically to see several approaches).

We have therefore in a particular case:
	
	Rewritten in traditional form (according to the literature) and in one dimension, for a constant potential in time, this relation is then written:
	
	Let us suppose now that the wave function can separate into two functions of which it is the product such that:
	
	We would then have:
	
	What injected into the one-dimensional evolution equation yields:
	
	Which gives after simplification:
	
	The left term depends only on $t$, the right one on $x$. Since they are equal, they are necessarily also equal to a constant which has the dimension of an energy ($U(x)$ is a potential energy for recall).

	So for the left term:
	
	then:
	
	And for the right term:
	
	Which can be written:
	
	After factorization:
	
	Either with the notations of this book (after having rearranged a bit):
	
	we thus fall back on the classical one-dimensional Schrödinger  equation which is not bad at all as result!

	Now, since we have put:
	
		then we have finally:
	
	Which we can write under the notations of the preceding paragraphs:
	
	We find also this last relation in several different forms in the literature of which few samples:
	
	
	\paragraph{Linear combination of states (quantum superposition)}\label{quantum superposition}\mbox{}\\\\\
	We must notice before we move on to another subject something very important that we had just mentioned in the second postulate!

	Indeed, any equation of the following form seen previously:
	
	Is therefore a solution of Schrödinger's evolutionary equation and as in quantum systems the Hamiltonian can take (or be associated with) several discrete eigenvalues traditionally denoted by $E_n$, we then have, as mentioned at the beginning of this section, by the principle of linear combination of differential equations (\SeeChapter{see section Differential and Integral Calculus page \pageref{linear differential equations}}) the following more general solution:
	
	of which we shall have several practical examples (of the discretization of the states of energy and that these are in infinite number) in this section and in that of Quantum Chemistry.

	If we write the normalization constant of $\Psi_n(x)$ of the previous relation, then we have:
	
	This last relation would be written in the following traditional ket-bra form:
	
	where the constant coefficient $c_n$ is obviously assimilated to $A_n$ (admit that it notation is much simpler isn't it?).
	
	We then say that the state $|\Psi\rangle$ is a linear combination of elementary states $|\psi\rangle$. Therefore $|\Psi\rangle$ also represents a wave particle as being simultaneously in several different sub-states.
	\begin{tcolorbox}[colframe=black,colback=white,sharp corners]
	\textbf{{\Large \ding{45}}Example:}\\\\
	Given $|\varphi\rangle$ and $|\chi\rangle$ that are vectors of $\mathcal{H}$ representing physical states, the unitary vector of $\mathcal{H}$:
	
	where $\lambda,\mu\in\mathbb{C}$, also represent a physical state.
	\end{tcolorbox}
	It is interesting to notice that each solution:
	
	describes a "\NewTerm{steady state}" or "\NewTerm{stationnary state}". Let us see (at last!) rigorously what it is.

	Indeed, we have:
	
	which is therefore independent of the time from which the origin of the name "\NewTerm{stationary state}" (we promised to define its origin at the beginning of section... so that is done!).

	The functions being normalized we therefore have:
	
	The calculations above have showed us (we had proved it in two different ways) that the eigenfunctions have the following properties:
	
	When $k=n$ and:
	
	when $k\neq n$. That is to say with a more general notation:
	
	It is this property that has led us in the third postulate to speak of "\NewTerm{orthogonal basis of the stationary eigenfunctions}".
	
	Let us continue our calculation which can be written using the Kronecker symbol (\SeeChapter{see section Tensor Calculus page \pageref{kronecker symbol}}):
	
	We can then interpret the term $|A_k|^2$ as the weight of the eigenfunction $\Psi_k$ in the quantum state $k$, the probability of actually being in the eigenstate $\Psi_k$ is then equal to $|A_k|^2$ the normalization then imposes:
	
	We must therefore remember that any quantum state can always (as far as we know) be interpreted as a linear combination of eigenstates. The coefficient $A_k$ of an eignenfunction/eigenstate $\Psi_k$ is then associated with a probability $|A_k|^2$.
	
	It is this mathematical result, very important, which is at the origin of the paradox of "\NewTerm{Schrödinger's cat}\index{Schrödinger's cat}" (among others ...) and of many debates.

	To close this small subject, let us notice one thing:

	If the coefficients $A_n$ are not already normalized, then the physicists denote their normalization as follows:
	
	because very often they use the same notation for the normalized coefficient and the non normalized one in their developments ...

The writing of the last relation is easily justified because let us recall that we must have:
	
	and we have indeed after rearranging:
	
	Let us notice that with the traditional ket-bra notation, the relation:
	
	is often written in some specialized textbooks:
	
	which therefore always gives the probability of finding the state $n$ at the position $x$.
	
	\pagebreak
	\paragraph{Schrödinger continuity equation}\mbox{}\\\\\
	Let us consider now the important example of the equation of evolution for a free particle, that is to say with $E_p=0$. Therefore we have:
	
	The probability of finding the particle in a volume $V$ is, as we have seen, given by:
	
	Hence:
	
	Taking into account the equation of evolution of the free particle, the second term of equality is written:
	
	Where we have put:
	
	According to the Ostrogradsky theorem (\SeeChapter{see section Vector Calculus page \pageref{gauss ostrogradsky theorem}}), it therefore comes:
	
	where the integral of the right is carried out on the surface $S$ which limits the volume $V$. The preceding relation therefore expresses indeed that the variation per unit of time of the probability of finding the particle in $V$ is equal to the flux passing through the surface $S$ and the vector $\vec{j}$ can be interpreted as a probability current density which satisfies the continuity equation as we have determined it in the section of Thermodynamics:
	
	Hence:
	
	In Quantum physics, it seems therefore that there is conservation of the particles flow: there is no creation, neither destruction of particle, but in the Nature (experimental observations) we observe such phenomenons of destruction and creation (but in fact the particle must be in a non-zero potential field)... In other words it means that a particle can not appear or disappear in a given volume $V$, there must be a particle flow in the walls of $V$ for particles entering or leaving $V$.
	
	That latter relation is more often written:
	
	Intuitively, the above quantities indicate this represents the flow of probability. The chance of finding the particle at some position $r$ and time $t$ flows like a fluid; hence the term probability current, a vector field. The particle itself does not flow deterministically in this vector field.
	
	\subsubsection{Implications and Applications}
	The different definitions and tools that have been seen before, sometimes very abstract, will allow us to study some fundamental cases that lead to quite splendid results.

	In a first step, we will see how to treat the case of the free particle (unbound state) and what are the problems of this simple configuration.

	Then we solve the Schrödinger equation with a spin-free particle in a potential well with rectilinear infinite walls and show that with the formalism of quantum physics we will find the same results as the Bohr model (even more generalized!).

	After this, we will introduce the study of the harmonic oscillator by revisiting briefly the resolution of the Schrödinger equation of a free particle. This example constitutes a form of introduction to the theoretical study of quantum atomic systems. It is in this example that we will use all the power of functional linear operators. It will therefore be important not to read it in diagonal (if possible and if the reader is still motivated).

	We will also have to study another famous phenomenon, the "tunnel effect"! Obviously, we decided to introduce a particular case so that the reader can see the reasoning that led to the discovery of this astonishing (but logical) phenomenon. Again, this example will support the validity of quantum theory and demonstrate the value of the decay constants of many nuclear isotopes!

	With respect to relativistic cases, with or without spin, we refer the reader to the sections of Relativistic Quantum Physics, and as regards the simple atomic model, we refer it to the section on Quantum Chemistry.

	Enjoy!
	
	\paragraph{Free particles (zero potential)}\label{free particles}\mbox{}\\\\\
	Curiously, the resolution of the Schrödinger equation for a free particle (where the potential is zero) is the simple ... most complex case ... mathematically speaking because the integration bounds of normalization are infinite. Let us see this!

Let us recall first that we have proved in a simplified way in the section of Sequences and Series that the Fourier transform of a function $f$ and its inverse were given by:
	
	Either in a one-dimensional form:
	
	Let us now proceed to the change of variable which connects the wave number $k$ to the linear momentum (relation introduced at the beginning of this section):
	
	Which gives us:
	
	Let us now return to the Schrödinger's equation of evolution:
	
	If the particle is free, there is no potential and at one dimension we then have:
	
	This differential equation admits monochromatic plane wave solutions of the type (\SeeChapter{see section Electrodynamics page \pageref{monochromatic plane wave}}):
	
	with obviously the small nuance that we have to use the relation (otherwise it does not work by cons!):
	
	without forgetting that (this will be useful afterwards):
	
	The curve of the energy $E$ as a function of the wave vector $k$ is sometimes named a "\NewTerm{dispersion curve}\index{dispersion curve}\label{dispersion curve}" and it is a parabola (since $k$ is squared) for a free particle!

	Obviously, the probability density of this solution is equal to:
	
	but this can not correspond to reality because! Indeed, a monochromatic planar wave of constant norm will have to carry an infinite energy and this is not possible!

	In fact the solution comes from the fact that the true solution uses the principle of superposition of all the monochromatic waves of all the frequencies such as:
	
	and we fall back here a relation very similar to an inverse Fourier transform (\SeeChapter{see section Sequences and Series page \pageref{fourier transform}}). Such a superposition of plane waves is named "\NewTerm{one-dimensional wave packet}\index{one-dimensional wave packet}\label{wave packet}".
	
	What we can rewrite:
	
	But, we see immediately that we will not be able to normalized following:
	
	Therefore there is no longer a general solution. It is necessary to give a carrier envelope to the waves imposing a possible normalization. This carrier envelope can be a Dirac or a Gaussian or other functions of more or less complex distributions. Then physicists must use a property of the Fourier transforms which naturally reveal the true Heisenberg incertitudes. Thus, the latter are a condition for the normalization of free particles using Fourier transforms.

	To date, as already mention during our study of classical Heisenberg incertitudes, we have no pedagogical and simple proof to propose on this last point. It will come in a near of far future.

	On the other hand, we can take as a trivial solution the eigenmodes of the particle such that:
	
	Indeed:
	
	This is what we will use as a relation for our study of the harmonic oscillator further below.

	Before studying the particular case of the quasi-monochromatic wave packet, we will recall some results concerning the sum of two plane waves.

	Let us start by summing two monochromatic plane waves of neighboring frequencies:
	
	with:
	
	and:
	
	Notice that we therefore impose:
	
	The resultant wave is the expressed as:
	
	Either by using the remarkable trigonometric relations (\SeeChapter{see section Trigonometry page \pageref{remarkable trigonometric identities}}):
	
	which is a plane wave propagating along the positive $x$ (\SeeChapter{see section Wave Mechanics page \pageref{wave equation}}) with the pulsation $\omega_0$ and the mean wave vector $k_0$, and therefore at the phase velocity:
	
	The cosine term is then interpreted as the slowly variable amplitude of this plane wave ("amplitude modulation" as we have seen in the section of Wave Mechanics).

	Notice a rather important point !: The phase velocity is not consistent with the velocity we get by using the kinetic energy of a free particle. Indeed:
	
	Therefore the phase velocity does not represent the velocity in the usual conventional sense but that of the wave traveling at the group velocity\label{wave velocity group traveling wave}
	
	where we thus fall back on the classic formulation of the velocity from kinetic energy (not bad...)!

	We can easily represent all this with Maple 4.00b:
	
	\pagebreak
	\texttt{>restart:with(plots):\\
	>lambda[0]:=1; T[0]:=1; k[0]:=2*Pi/lambda[0]; w[0]:=2*Pi/T[0];\\
	>delta\_k:=k[0]/8: k[1]:=k[0]-delta\_k; k[2]:=k[0]+delta\_k;\\
delta\_w:=w[0]/10: w[1]:=w[0]-delta\_w; w[2]:=w[0]+delta\_w;\\
	>P1:=animate(cos(k[1]*x-w[1]*t)+cos(k[2]*x-w[2]*t), x=0..1*2*Pi/delta\_k, \\
	t=0..2*Pi/delta\_w, numpoints=200, frames=15, color=red):\\
	>P2:=animate({2*cos(-1/2*k[1]*x+1/2*w[1]*t+1/2*k[2]*x-1/2*w[2]*t), \\
	-2*cos(-1/2*k[1]*x+1/2*w[1]*t+1/2*k[2]*x-1/2*w[2]*t)}, x=0..1*2*Pi/delta\_k,\\
	 t=0..2*Pi/delta\_w, numpoints=100, frames=15, color=blue):\\
	>display(P1,P2);}
	
	Which gives:
	\begin{figure}[H]
		\centering
		\includegraphics{img/atomistic/phase_velocity_vs_group_velocity.jpg}	
		\caption{Representation of the concept of phase velocity and group velocity}
	\end{figure}
	Unlike the harmonic plane wave, this wave does not have a constant norm: its norm is zero in certain zones. On the other hand, it always extends over an infinite distance, and therefore has a infinite norm (the sum of the probability over all space). It therefore has no physical meaning (at least as far as we know).

	The preceding study can be extended by summing an increasing number $N$ of plane waves in the neighborhood of $\omega_0$ and $k_0$. Such a superposition leads to an increasingly localized function in certain areas of space (in particular near $x=0$, for example, for $t=0$), the distance between these zones increasing proportionally with $N$. At the limit $N\rightarrow +\infty$, then only the zone near $x=0$ remains, the others being rejected to infinity. The transition to this limit  $N\rightarrow +\infty$ is done by replacing the discrete sum on plane waves by a continuous summation ie by an integral of the form:
	
	with:
	
	with therefore:
	
	Such a packet is therefore named a "\NewTerm{quasi-monochromatic wave packet}\index{quasi-monochromatic wave packet}".

	This expression can be rewritten:
	
	It is important to understand that $\omega$ is a function of $k$ given by the dispersion equation. We will compute this expression using the fact that $\delta k/k_0\ll 1$. This condition implies that $\delta \omega/\omega_0\ll 1$. It is possible to carry out a limited Talyor development in the neighborhood of $\omega_0$:
	
	Therefore:
	
	Let us put $y=k-k_0$:
	
	Let us calculate the integral:
	
	with:
	
	Therefore:
	
	The last term is again interpreted as a plane wave moving at the phase velocity:
	
	The amplitude of this plane wave is given by a sinus cardinal function. At $t=0$, this $\mathrm{sinc}$ function has important values only in the zone where:
	
	It is therefore a well-localized function. Consequently, $\Psi$ is a summable square function. The calculation gives:
	
	The function can therefore be normalized by putting:
	
	We have therefore succeeded in obtaining a function satisfying both the Schrödinger equation and the normalization condition, by using an infinite sum of harmonic waves. The example we have dealt with is only one particular case. Other types of wave packets can be obtained by taking other distributions for the amplitudes of the plane waves that make up the packet (we assumed here that they all had the same amplitude). Consequently, the group velocity is conventionally associated with the velocity of the particle of mass $m$ and linear momentum $p$.

	Thus, the wave packet moves globally at the group velocity, which is identified with the velocity given by Classical Mechanics.

	Uncertainty relationships have already been introduced at the beginning of this section in two different ways. But in the example of the wave packet studied in the previous paragraph, we have seen that the function is located in a zone of range (half-height width):
	
	We therefore have the relation:
	
	We fall back here an expression of the uncertainty. The numerical coefficient could be slightly different according to the definition chosen for $\delta x$ and $\delta k$, or the type of packet. In particular, it could be significantly larger in some cases. We thus have in fact an inequality of the type:
	
	In quantum physics, these inequalities are expressed as a function of the linear momentum $p$, connected to $k$ as we have prove earlier by $p=\hbar k$. We therefore have:
	
	Thus, the more precisely the linear momentum of a wave packet is defined (implicitly the wavelength), the less it has components and the more it is spread, which increases the uncertainty relative to its position and respectively the more its position is known, the less its wavelength will be precise.

	It is therefore not a question of uncertainties in the sense of measurement, and which would be limited by the measuring devices, but of an intrinsic fundamental property, linked to the quantum representation of a particle according to the proposed mathematical model. The model of the Bohr atom have therefore to be rejected for the energy levels which are close to this equality.
	
	\pagebreak
	\paragraph{Infinite (rectangular) potential well}\label{quantum potential well}\mbox{}\\\\\
	Let us take as a first example, very important for the Nuclear Physics section and for semiconductor specialists, the resolution in the classical form of the infinite potential well with straight walls, also named "infinite rectangular well" (this example is really very important, take your time to understand it and to master it at best) or "particle in a box problem".

	This is the simplest example of a potential function $E_p(x)$, zero within the well and infinitely big on the walls, distant by a length $L$:
	\begin{figure}[H]
		\centering
		\includegraphics{img/atomistic/potential_well_infinite_rectangular.jpg}	
		\caption{Infinite potential well illustration}
	\end{figure} 
	\begin{tcolorbox}[title=Remark,colframe=black,arc=10pt]
	When $U=+\infty$ we say sometimes that the walls are "perfectly reflective".
	\end{tcolorbox}
	We assume a particle trapped in this well. It can not escape because the walls (ie the potential $U$) have an infinite height. But inside, it is free to move without interacting with the walls.

	This configuration is reflected by the boundary conditions where the potential electrostatic energy is denoted $U$:
	
	and:
	
	There exist at least two ways to approach this problem. Let us consider the two types of treatments because the first one allows to have a simplistic approach whereas the second one allows to have a more general approach which will be useful to us later on in our study of the Tunnel effect:
	
	\paragraph{First approach}\mbox{}\\\\\
	The (classical) Schrödinger equation :
	
	has therefore a simple particular solution respecting the initial conditions in one dimension, of the type:
	
	whose second derivative is:
	
	Introduced in the Schrödinger equation, we get, after some elementary algebra simplifications:
	
	So finally the solution is written:
	
	to which the boundary conditions must be applied (the cosine solution is in every aspect similar).

	If we want to be able, later, to make a parallel with one or more electron(s) trapped in the potential well of an atom (which is him not rectangular and not infinite), we are taken to the following considerations:

	The stability of the atoms suggests the existence of an electronic standing wave in the well. In addition, experimental observation shows that only certain energy levels appear to be allowed in the latter.

	If we make a similarity with the vibrating strings, the wave function of the electron must be such that:
	\begin{enumerate}
		\item For $x=0$ and $L=0$ there must be a vibration node, ie $\Psi(0)=\Psi(L)=0$

		\item The wave function $\Psi$ must have an integer value of half wave length along the length $L$

		\item In the box $U=0$ therefore $E_p=0$

		\item If at the ends ($x=0$ and $x=L$) $\Psi=0$ then the argument of the sine equals $n\pi\; \forall n\in\mathbb{N}^{*}$
	\end{enumerate}
	Therefore we must have:
	
	hence since the potential energy is zero in the well:
	
	The total energy of the particle thus ranges over a discrete sequence of values, the only ones permitted. The value of $L$ is determined using the Bohr or Sommerfeld model as depending on the situations (\SeeChapter{see section Corpuscular Quantum Physics page \pageref{bohr model} and \pageref{relativistic sommerfeld model}}).

	The total energy of the above particle is composed of the eigenvalues of the energy in the potential well.

	Thus the Schrödinger's equation allows us to ignore Bohr's third postulate in the sense that it directly explains the notion of quantification of the levels by by integer (discrete) values of the boundary conditions of a potential well considered as perfect.

	The corresponding wave functions in the well where $U=0 \equiv E_p=0$ are therefore:
	
	Either after simplification:
	
	This is the expression of one of the solutions of the differential equation for the ideal rectangular potential well. Thus, there exists a discrete sequence of solutions wave functions. These are the obviously the eigenfunctions of the particle.

	The constant $\Psi_0$ in this expression is determined by the de Broglie normalization  condition (about which we spoke at the beginning of this section), that is to say by the condition:
	
	We then find (normally integration calculus that should be obvious to the reader but on request we can as always give the details):
	
	and the final expression of the wave function associated with the eigenvalue $E_n$ thus reads:
	
	Some physicists typically write this in a complex form by obviously considering only the real part of the following expression (we use the "Euler formula" seen when introducing the complex numbers in the section Numbers):
	
	with:
	
	We then say that we have "\NewTerm{quantification conditions}" on $k$ imposed by the boundary conditions.

	This notation is sometimes useful and we will use it when studying the tunnel effect in the section of Nuclear Physics.

	We can deduce from the expression obtained the main properties of the wave functions describing the stationary states of the particle in a box:
	\begin{enumerate}
		\item[P1.] The figure below shows the $\Psi_n$ (left) and the probability density functions $|\Psi_n|^2$ (right) for the first energy levels $E_n$:
		\begin{figure}[H]
			\centering
			\includegraphics{img/atomistic/infinite_potential_well_functions_plots.jpg}	
			\caption[]{Representation of the wave functions and density for some energy levels}
		\end{figure}
		We notice that (obviously we could analyze this analytically and not graphically if we wanted to), in addition to the points $x=0$ and $x=L$, $\Psi_n$ has $n-1$ zeros located in:
		
		These points, where the wave function and the probability density are zero, are named "\NewTerm{nodal points}" or simply "\NewTerm{nodes}" of the wave function. The number of nodes increases when $n$ increases, that is to say when one moves to more and more excited states. 

		The wave function $\Psi_1$ of the ground state with $n = 1$ and therefore with:
		
		has no node, that of the first excited state $\Psi_2$ of energy:
		
		has one nodal point, that of the second excited state $\Psi_3$ has two nodal points, etc.
		
		The variation of the nodal properties of the wave functions when $n$ varies translates the orthogonality of the stationary states of different energy. Indeed, we easily check that $\langle \Psi_n|\Psi_m\rangle$ is zero when $m\neq n$:
		
		where we used one of the remarkable trigonometric identity demonstrated in the section of Trigonometry.
		
		\item[P2.] As we can see it in the previous figure, the probability density associated with any stationary state of the particle is symmetric with respect to the midpoint $x=L/2$.

		We therefore expect that the mean value of $x$ will be exactly equal to $L/2$ in such a state. Indeed, we have seen in the section of Statistics that the expected mean (average) of a probability event $P(x)$ is defined by:
		
		where $x$, $\text{E}(x)$ and $P(x)$ have no units (caution! as we will do a dimensional analysis!).

		Now, in quantum physics $\text{E}(x)$ and $x$ are identical dimensional quantities. This means that the dimensions of $P(x)$ must cancel those of $\mathrm{d}x$. Thus, we guess following the study of the de Broglie normalization conditions that:
		
		is a linear probability of the presence of the particle in [m$ ^{-1}$].
		
		The integration domain being $[0, L]$ we have finally:
		
		
		\item[P3.] Also we have for the linear momentum:
		
		
		\item[P4.] We can also verify without too much difficulty that what we saw in the statement of the second postulate of Quantum Physics is well verified  in this example. That is to say, the eigenfunctions of the wave are connected to the Hamiltonian operator via the eigenvalues of the energy:
		
		Indeed, in our example, this gives:
		
		
		\item[P5.] Our analysis of the quantum particle in a box would not be complete without discussing "\NewTerm{Bohr's correspondence principle}\index{Bohr's correspondence principle}". This principle states that for large quantum numbers, the laws of quantum physics must give identical results as the laws of classical physics. To illustrate how this principle works for a quantum particle in a box, we see that in the plot given earlier above of $|\Psi|^2$ when $n\rightarrow +\infty$ that the an number of "peaks" will increase to infinite and therefore result in a uniform (constant) probability to found the particle between $0$ and $L$.
	\end{enumerate}
	That's it... for the first approach of the problem. Let us now look at the second:
	
	
	\paragraph{Second approach}\mbox{}\\\\\
	We thus have the Schrödinger equation in the one-dimensional case:
	
	In regions outside the box where the potential is infinite, we have:
	
	Therefore:
	
	which gives:
	
	Let us consider now the case of the well where, since the electrostatic potential is zero, the Schrödinger equation reduces to:
	
	It is therefore a linear differential equation of order $2$ with constant coefficients, equation which is relatively easy to solve in the general case (\SeeChapter{see section Differential and Integral Calculus page \pageref{second order differential equations}}). Given the equation:
	
	Using the results we get during the treatment of the particular solution, let us suppose that the function $y$ that satisfy this differential equation is of the form $y=e^{Kx}$. We have then:
	
	provided, of course, that $e^{Kx}\neq 0$. This last relation is therefore the auxiliary quadratic equation of the differential equation (characteristic polynomial). It has two solutions / roots (it is a simple resolution of a polynomial of the second degree) that we will denote in the general case $K_1$, $K_2$. Which means that:
	
	is satisfied for both roots. If we make the sum since the both are equal to the same constant:
	
	Thus, it is immediate that the general solution of $y$ is of the type:
	
	where the reader should normally be able to verify that the addition of the constants $A$ and $B$ does not change the developments of the preceding paragraphs.

	In the present case that interest us:
	
	The quadratic equation is:
	
	thus:
	
	So finally the general solution is of the form:
	
	Let us put now:
	
	We then have:
	
	with:
	
	It is now necessary to determine $A'$ and $B'$ using the boundary conditions. Thus, at $x = 0$ and $x = L$ we should have $\Psi=0$ and we have for $x = 0$:
	
	The coefficient $A'$ must therefore be equal to zero. And in $x = L$ we should have:
	
	But in this case, $B'$ must be different from zero. Indeed, if it were zero, the wave function would be zero throughout the whole well, which is contrary to the physical reality of the problem. It is therefore necessary that the sinus to be zero, or that its argument be equal to a multiple of a non-zero integer number of angle $\pi$ such that:
	
	Therefore:
	
	We thus find exactly the same result as the preceding method!

	It remains to determine $B$ and the method is exactly identical to the first method of resolution which we have seen above. Thus, we have:
	
	What is especially important in this method is to remember for a moment the general form of the solution:
	
	
	\pagebreak
	\paragraph{Fermi Energy}\mbox{}\\\\\
	The "\NewTerm{Fermi energy}\index{Fermi energy}\label{fermi energy}", $E_F$ is a concept in Quantum Physics that designates the energy of the highest quantum state occupied in an idealized system in which all the layers are filled successively and without discontinuities, that is to say in practice when a system is at absolute zero: $0$ [K] (\SeeChapter{see section Statistical Mechanics page \pageref{fermi dirac distribution}}). It is an important concept in Solid states physics as we have seen it in our study of semiconductors in the section of Electrokinetics.
	
	The rectilinear infinite potential well is an excellent teaching tool to introduce practically the Fermi energy level and then to extend it to other particular cases.

	Let us recall that we have just obtained in two different ways:
	
	Then the total energy of an entire system composed of N particles can all have the same fundamental state $n = 1$ and thus that violate the principle of Pauli will be:
	
	But if we apply the Pauli's exclusion principle of electrons (fermions), each level can then take only two states (spin oppositions) in this model (which does not contain sublayers or other subtleties). Henceforth, each fundamental level can be occupied by only two states, and so it is for each level. The total energy is then in reality:
	
	Using the sum of the squares demonstrated in the section of Sequences and Series and writing as it is usual:
	
	we have then:
	
	The average energy per particle is then:
	
	By definition, the energy of the last level to be occupied is that of Fermi and therefore given by:
	
	Therefore:
	
	We also notice that we have whatever the value of $N$:
	
	The three-dimensional isotropic case is known as the "Fermi sphere" as already introduced in the section of Electrokinetics during our study of semiconductors.
	
	Let us now consider a three-dimensional cubical box that has a side length $L$. This turns out to be a very good approximation for describing electrons in a metal. The states are now labeled by three quantum numbers $n_x$, $n_y$, and $n_z$. The single particle energies are (where $m$ is the mass of fermion (electron in this case)) as we have seen in the section of Electrokinetics during our study of semiconductor:
	
	where $n_x$, $n_y$, and $n_z$ are positive integers. There are multiple states with the same energy, for example $E_{211}=E_{121}=E_{112}$. Now let's put $N$ non-interacting fermions of spin $1/2$ into this box. To calculate the Fermi energy, we look at the case where $N$ is large.

	If we introduce a vector $\vec{n}=\{n_{x},n_{y},n_{z}\}$ then each quantum state corresponds to a point in "$n$-space" with energy:
	
	with $||\vec{n}||^2$  denoting the square of the usual Euclidean length (norm). The number of states with energy less than  $E_{\vec{n}}$, defined as the "Fermi energy", is  is equal to the number of states that lie within a sphere of radius $R=||\vec{n}_F||^2$ in the region of $n$-space where $n_x$, $n_y$ and $n_z$ are positive. In the ground state this number equals the number of fermions in the system:
	
	the factor $2$ is once again because there are two spin states, the factor of $1/8$ is because only $1/8$ of the sphere lies in the region where all $n$ are positive. Therefore we get:
	
	so the Fermi energy is given by:
	
	Which results in a relationship between the Fermi energy and the number of particles per volume (when we replace $L^2$ with $V^{2/3}$):
	
	
	Confusingly, the term "Fermi energy" is often used to refer to a different but closely related concept, the "Fermi level" (also named "electrochemical potential\index{electrochemical potential}"). There are a few key differences between the Fermi level and Fermi energy, at least as they are used in this book:
	\begin{itemize}
		\item The Fermi energy is only defined at absolute zero, while the Fermi level is defined for any temperature.

		\item The Fermi energy is an energy difference (usually corresponding to a kinetic energy), whereas the Fermi level is a total energy level including kinetic energy and potential energy.

		\item The Fermi energy can only be defined for non-interacting fermions (where the potential energy or band edge is a static, well defined quantity), whereas the Fermi level (the electrochemical potential of an electron) remains well defined even in complex interacting systems, at thermodynamic equilibrium.
	\end{itemize}
	Since the Fermi level in a metal at absolute zero is the energy of the highest occupied single particle state, then the Fermi energy in a metal is the energy difference between the Fermi level and lowest occupied single-particle state, at zero-temperature.
	
	
	\paragraph{1D-Harmonic oscillator}\mbox{}\\\\\
	The study of the harmonic oscillator corresponding to that of a wave function stuck in a well of parabolic potential. This is roughly equivalent to the atoms where the walls of the potential well are naturally not rectangular and infinite ... The study that follows is therefore what is closest to what is available in Nature at the atomic level and this is why it is used a lot in Quantum Chemistry and also in Nuclear Physics.

	In the case of a free particle in rectilinear displacement, we have seen that the potential energy is zero ($E_p=0$) and the Schrödinger equation then becomes:
	
	However, for a free particle (in the absence of a potential field) the total energy is therefore equal to the kinetic energy:
	
	But we have:
	
	The ratio:
	
	being the associated de Broglie wavelength. By introducing the wave number $k=2\pi/\lambda$ (\SeeChapter{see section Wave Mechanics page \pageref{pulsation frequency period wave number}}), we have:
	
	named "\NewTerm{de Broglie's relation}\index{de Broglie's relation}". Finally:
	
	Therefore, Schrödinger's equation can be written:
	
	We see by direct substitution that this differential equation admits for solutions the wave functions:
	
	These two different solutions represent the displacement of the same particle once in the $+x$ direction and the other in $-x$. If $A=B=1$ we have:
	
	The fact that this result is equal to unity means that the probability of finding the particle is the same at all points as we don't need any special normalization. In other words, $\Psi(x)=e^{\pm\mathrm{i}kx}$ describes a situation in which the uncertainty on the position is total. This result is in agreement with the principle of uncertainty since $\Psi(x)=e^{\pm\mathrm{i}kx}$  describes a particle of which we know with precision the linear momentum $p=\hbar k$: that is to say that $\Delta p=0$, which implies $\Delta x\rightarrow +\infty$.

	In the section of Differential and Integral Calculus we showed that the most general solution of a differential equation is the sum of these solutions. In other words, in our example:
	
	with:
	
	By the way, we can notice that if $E_p\neq 0$ then the result is the same with the difference that we will have:
	
	When the particle of interest is in a well of potential described by the function (parabola):
	
	we then speak of "\NewTerm{1D-harmonic oscillator}\index{harmonic oscillator (1D)}".

	This system is very important because the Hamiltonian of the equation intervenes in all the problems involving oscillations such as molecular and crystalline vibrations (\SeeChapter{see section Quantum Chemistry page \pageref{molecular vibrations}}).

	Let us first take as an example the classic harmonic oscillator which consists of a body subjected to move along an axis and subjected to a return force proportional to the distance to a point situated on this axis.

	The equation of this body is governed by the equation of dynamics:
	
	We have seen in the section of Classical Mechanics that the general solution of this equation is:
	
	with for pulsation:
	
	The total energy of the system being the classical Hamiltonian, we write:
	
	It follows from the expression of the potential energy that the constant we had in the initial relation which defined the well of parabolic potential:
	
	is simply $k$ and therefore:
	
	but which we will write for the next development and by tradition in the following form:
	
	Now let us come back to our quantum framework. From this point of view we have for Hamiltonian (or total energy):
	
	where for recall, following what we have just seen:
	
	Using what we define as a "reduced writing", we write:
	
	where the "reduced operators" of linear momentum and positions are therefore respectively:
	
	where by a stupid tradition wee keep the constant instead of replacing by by $\omega_0^2m$.
	
	It is more or less easy to obtain the relation of commutation:
	
	\begin{dem}
	Remember the relations below that we saw in our study of functional linear operators at the beginning of this section:
	
	Let us consider the properties of the commutators with the linear momentum. We have also proved above earlier above the relation:
	
	By multiplying the latter by $\hbar/\mathrm{i}$, it comes:
	
	That we can also write:
	
	If you remember the definition of the of commutators ($[\alpha,\beta]=\alpha\beta-\beta\alpha$), we have:
	
	We therefore have for our oscillator:
	
	Let's write the definition of the commutator:
	
	Therefore:
	
	that's was what has to be proved...
	\begin{flushright}
		$\square$  Q.E.D.
	\end{flushright}
	\end{dem}
	We now have an interest for solving the differential equation to use the non-Hermitian operators equation defined by\footnote{This approach of solving this problem is sometimes named "Ladder operators\index{ladder operators} for the quantum harmonic oscillator"} (this is a definition so do not try to understand!):
	
	What thus defines us the operators (by temporarily putting $m=1$):
	
	which are linear combinations of the position and linear momentum operators and are trivially not Hermitian (not equal to its own conjugate for recall!).
	
	We see these two operators very frequently in Quantum Physics, especially in Quantum Field physics, and physicists then speak of "\NewTerm{creation operator $a^\dagger$}\index{creation operator}" or "\NewTerm{raising operator $a_{+}$}\index{raising operator}" and of the "\NewTerm{destruction operator $a$}\index{destruction operator}" or "\NewTerm{lowering operator $a_{-}$}\index{lowering operator}" (also named "\NewTerm{annihilation operator}\index{annihilation operator}"). The reasons of these names will be more clear later (it seems that it is Dirac himself that introduce them for the first time in the framework of the harmonic oscillator)!

	Given the commutators relation, we check (we need that for the next developments):
	
	\begin{dem}
	First:
	
	and:
	
	\begin{flushright}
		$\square$  Q.E.D.
	\end{flushright}
	\end{dem}
	And on the other hand we also need:
	
	\begin{dem}
	
	and therefore by dividing by $2$ on both sides of the equality, we get:
	
	\begin{flushright}
		$\square$  Q.E.D.
	\end{flushright}
	\end{dem}
	Now let us come back to:
	
	and using:
	
	where it appears that $N$ can be seen as an operator, named "\NewTerm{counter operator}\index{counter operator}" (hermitic) since:
	
	The same reasoning and assumption bring us to:
	
	with:
	
	It is then sufficient to quantify the counter operator to know the eigenvalues and the eigenstates of $H$ since these two elements are then linked by:
	
	Therefore in general using the other notation:
	
	therefore the Schrödinger equation can be written:
	
	We now do the assumption that $\Psi$ is an eigenfunction of $N$ associated with the eigenvalue $n$ such that:
	
	This assumption is very important because we will use it as an induction principle to find all the eigenfunctions starting from the fundamental one!

	Let us now establish commutator relations between $N$ and $N'$ and the operators $a$ or $a^\dagger$. For this, let us first multiply $aa^\dagger-a^\dagger a=1$ by $a^\dagger$, then we get:
	
	Similarly, by multiplying the $aa^\dagger-a^\dagger a=1$ by $a$, we get:
	
	Since according to our assumption $\Psi$ and $n$ are respectively eigenfunction and eigenvalue of $N$, we can write:
	
	But as we have:
	
	which multiplied on both side of the equality by the wave function $\Psi$ gives the relation:
	
	This equation has the following consequences:
	\begin{itemize}
		\item We have $a\Psi=0$ such that $Na\Psi=na\Psi$
		\item Or we have $a\Psi$ that is an eigenfunction of $N$ for the eigenvalue $(n-1)$
	\end{itemize}
	The same reasoning would establish that $a^\dagger\Psi$ is an eigenfunction of $N'$ for the eigenvalue $n' + 1$, if it is not zero (we shall see later that $a^\dagger\Psi$ is never zero):
	
	With the notations above it was still not clear for one of our read with the operator were named creation and annihilation operators. Let us see this with another approach before we continue! The following development was more explicit to explain their name (we keep the other notation of these operators for obvious reasons):
	
	By the same argument, we can show that $a_{-}\Psi$ is a wave function with energy $E-\hbar\omega_0$. For this reason $a_{+}$ and $a_{-}$ are called the raising and lowering operators, respectively:
	
	Now, since the harmonic oscillator potential is parabolic, we would expect no upper limit to the energy levels, but we would expect a lower limit, so that applying the lowering operator to the ground state should give us zero.
	
	Therefore we know that there exists a lower eigenvalue $n_0$ smaller all the others corresponding to the fundamental level (according to the Bohr-Sommerfeld model this eigenvalue always exists).
	
	Necessarily, its eigenfunction equation obeys the relation (the reader will be able to check with the results further):
	
	as they can not be , as we have already mention it, a level lower that the lower limit by applying the lowering operator!
	
	Otherwise $n_0-1$ would a eigenvalue and there would a physical contradiction.

	By multiplying this last relation by $a^\dagger$ we get:
	
	Which shows that the minimum eigenvalue $n_0$ is zero. We therefore know the fundamental level of the oscillator:
	
	named "\NewTerm{zero-point energy}\index{zero-point energy}\label{zero point energy}".
	\begin{tcolorbox}[title=Remark,colframe=black,arc=10pt]
	It should be noticed that the oscillator is never in a state of rest (put $n = 0$ in the expression of the energy) which also means that the absolute zero can not be accessible since the temperature "fix" the atomic agitation and that a rest state does not exist!
	\end{tcolorbox}
	To get the corresponding eigenfunction, we need the explicit expression of $a$. According to:
	
	Then we have:
	
	which gives us:
	
	Hence:
	
	But according to $a\Psi_0=0$:
	
	hence:
	
	thus (resolution of a simple differential equation):
	
	where $A$ is the constant of integration. Thus, explicitly:
	
	We can normalize this function using the standard Gauss integral (\SeeChapter{see section Statistics page \pageref{Gauss integral}}):
	
	so we get:
	 
	Therefore:
	
	So finally:
	
	It remains to build the other eigenfunctions and to normalize them. Indeed, if $\Psi_n$ is a normalized function associated with the level $E_n$, we have seen above that $a^\dagger\Psi_n$ is an eigenfunction associated with the level $n+1$ (the operator of creation $a^\dagger$ increase the energy as we have show it...), but there is no reason to normalize it again since it is precisely associated with an eigenfunction already normalized!

	Therefore we can write:
	
	Where $\alpha_n$ is a coefficient to be determined. Let us express the fact that $\Psi_n$ is already normed:
	
	Considering the relation $aa^\dagger-a^\dagger a=1$ we have:
	
	Let us recall that $N(a^\dagger\Psi)=(n+1)(a^\dagger\Psi)$ therefore:
	
	We have just verified on the way that $a^\dagger\Psi_0$ is never zero (fact that we had supposed above).

	All the functions $\Psi_n$ (except the already $\Psi_0$ already fixed) have an arbitrary phase factor (concept that we have seen during the definition of bounded and unbound states), independently of each other, the argument $\alpha_n$ therefore remains available to us and we will choose $\alpha_n$ as being a fixed positive real number. This fixes all the $\Psi_n$ according to the previous development:
	
	By iterating this relation on the wave function, we obtain easily (elementary algebra but that we can detail on request):
	
	by taking into account the following relations (which we have already proved just earlier above):
	
	We then have:
	
	This relation takes a simpler form, relying on the relation:
	
	Let us check this:
	
	hence, in the language of operators:
	
	Therefore:
	
	We thus obtain the expression of $\Psi_n$:
	
	Moreover, in the mathematical theory of families of orthogonal polynomials with the weight (with respect to the weight function\footnote{If you don't remember what is the weight function, you can refer of the section of Function Analysis} $w(x)=e^{-x}/2$), we encounter the "\NewTerm{Hermite polynomials}\index{Hermite polynomials}" defined by (physicist versions):
	
	They are polynomials of degree $n$, even or odd ($H_0=1$, $H_1=2Q$, $H_2=4Q^2-2$):
	\begin{figure}[H]
		\centering
		\includegraphics[scale=0.7]{img/atomistic/hermite_physicists_polynomials.jpg}	
		\caption[First five (physicists) Hermite polynomials]{First five (physicists) Hermite polynomials (source: Wikipedia)}
	\end{figure}
	By using these polynomials, the notation of:
	
	is simplified and becomes:
	
	Thus fully explicitly:
	
	These polynomials therefore constitute an orthonormal basis of the global quantum state and thus appear naturally in the general expression of the functions / eigenstates.

	Finally we have:
	\begin{table}[H]
		\centering
		\begin{tabular}{|
		>{\columncolor[HTML]{9B9B9B}}l |l|l|}
		\hline
		$\pmb{n}$ & \cellcolor[HTML]{9B9B9B}$\pmb{E_n}$ & \cellcolor[HTML]{9B9B9B}$\pmb{\Psi_n}$ \\ \hline
		$0$ & $\dfrac{1}{2}\hbar\omega_0$ & $\left(\displaystyle\frac{m\omega_0}{\pi\hbar}\right)^{-1/8}e^{-\dfrac{\omega_0m}{\hbar}\dfrac{x^2}{2}}$ \\ \hline
		$1$ & $\dfrac{3}{2}\hbar\omega_0$ & $-\left(\left(\displaystyle\frac{m\omega_0}{\pi\hbar}\right)^{1/4}2\right)^{-1/2}2\sqrt{\dfrac{\omega_0m}{\hbar}}xe^{-\dfrac{\omega_0m}{\hbar}\dfrac{x^2}{2}}$ \\ \hline
		$2$ & $\dfrac{5}{2}\hbar\omega_0$ & $\left(\left(\displaystyle\frac{m\omega_0}{\pi\hbar}\right)^{1/4}8\right)^{-1/2}\left(4\dfrac{\omega_0m}{\hbar}-2\right)e^{-\dfrac{\omega_0m}{\hbar}\dfrac{x^2}{2}}$ \\ \hline
		$3$ & $\dfrac{7}{2}\hbar\omega_0$ & $-\left(\left(\displaystyle\frac{m\omega_0}{\pi\hbar}\right)^{1/4}48\right)^{-1/2}\left(8\left(\dfrac{\omega_0m}{\hbar}\right)^{2/3}x^3-12\sqrt{\dfrac{\omega_0m}{\hbar}}x\right)e^{-\dfrac{\omega_0m}{\hbar}\dfrac{x^2}{2}}$ \\ \hline
		$\ldots$ & $\ldots$ & $\ldots$ \\ \hline
		\end{tabular}
		\caption{First eigenfunctions and eigenvalues for the $1$D harmonic oscillator}
	\end{table}
	With the no less famous graphic representation with on the left the eigenfunctions associated $\Psi_n$ and on the right the probability of presence $|\Psi_n|^2$:
	\begin{figure}[H]
		\centering
		\includegraphics{img/atomistic/harmonic_oscillator_plot_first_levels_energy_density_probability.jpg}	
		\caption{Plot of first eigenfunctions and density functions of some energy levels of the $1$D harmonic oscillator}
	\end{figure}
	We can easily get the left part with Maple 4.00b (the example below is not normalized from hence the abscissa which is a bit special compared to the figure above!):
	
	\texttt{>m:=1;omega:=1;h:=1;\\
>plot([(m*omega/(Pi*h))\string^(-1/8)*exp(-m*omega*x\string^2/(2*h)),-((m*omega/(Pi*h))\\
\string^(1/4)*2)\string^(-1/2)*2*sqrt(m*omega/h)*x*exp(-m*omega*x\string^2/(2*h)),((m*omega/(Pi*h))\\
\string^(1/4)*8)\string^(-1/2)*(4*m*omega/h*x\string^2-2)*exp(-m*omega*x\string^2/(2*h)),-((m*omega\\
/(Pi*h))\string^(1/4)*48)\string^(-1/2)*(8*(m*omega/h)\string^(2/3)*x\string^3-12*sqrt(m*omega/h)*x)\\
*exp(-m*omega*x\string^2/(2*h))],x=-6..6);}
	\begin{figure}[H]
		\centering
		\includegraphics[scale=0.5]{img/atomistic/harmonic_oscillator_plot_maple.jpg}	
		\caption{Plot of first eigenfunctions of the $1$D harmonic oscillator with Maple 4.00b}
	\end{figure}
	By analyzing these wave functions, we fall back on many classical results: the particle in the potential well has a wider probability of presence if it has a higher energy (a ball at the bottom of a well will rise higher on The edges if it has more energy), the particle is more likely to find itself on the positions distant from the center of the well when its energy is high.

	For all calculations where particles are in a potential well, the harmonic approximation is very interesting. For example, if we wish to study a two-dimensional "harmonic trap\index{harmonic trap}", ie 2D  Bose-Einstein condensate (\SeeChapter{see section Statistical Mechanics page \pageref{bose einstein distribution}}), we can set the following Hamiltonian to begin the study (in analogy with the one at $1$ dimension used above):
	
	
	
	\paragraph{Tunnel Effect (barrier potential)}\mbox{}\\\\\
	The "\NewTerm{tunnel effect}\index{tunnel effect}" or "\NewTerm{quantum tunneling}\index{quantum tunneling}" refers to the property of a quantum object to cross a potential barrier, which can not be explained by Classical Mechanics. Generally, the wave function of a particle, whose square of the module represents the amplitude of its probability of presence, does not cancel at the level of the barrier but attenuates within the barrier (see the previous plot of the harmonic oscillator!), practically exponentially for a fairly wide barrier as we will prove it soon. If at the output of the potential barrier the particle possesses a non-zero probability of presence, it can thus cross this barrier.

	The theoretical study of this phenomenon is of crucial importance in the theory of semiconductors (\SeeChapter{see section Electrokinetics}) and disintegration in nuclear physics (\SeeChapter{see section Nuclear Physics page \pageref{alpha disintegration}}). It is therefore necessary to pay special attention to it!

	The quantum barrier of width $L$ separates in simple cases the space in three (in the $1$ dimensional case), the left and right parts of which are considered to have constant potentials up to infinity. The intermediate part constitutes the barrier, which may be complicated, revealing a soft profile, or on the contrary formed by rectangular or possibly other type of barriers/wells.

	Let us now consider the case of systems where the potential energy $E_p$ (implicitly the relative potential) tends to finite boundaries, not necessarily equal when $x\rightarrow \pm \infty$. It is therefore a problem of unbounded states.

	First, we define a region \texttt{I} far to the left where $E_p(x)$ will be denoted:
	
	A region \texttt{III} far to the right where $E_p(x)$ will be denoted:
	
	By limiting ourselves to the simplest situations, there are three possibilities relative to the relations given above: 
	\begin{itemize}
		\item potential well (a)
		\item potential step (b)
		\item potential barrier (c)
	\end{itemize}
	as represented in the same order in the figure below:
	\begin{figure}[H]
		\centering
		\includegraphics{img/atomistic/type_potentials_tunnel_effect_study.jpg}	
		\caption{Classical types of potential variation for Tunnel Effect study}
	\end{figure}
	Now, let us write Schrödinger's equation:
	
	In the regions \texttt{I} and \texttt{III} of the potential barrier, the idea is that $E_{\text{tot}}-E_p$ is constant and positive so the differential equation can be written in one dimension:
	
	We thus get very simply the analytic expression of $\Psi$ in these regions in general form:
	
	We fall back on these two expressions that are identical to those of our study of the potential well with rectangular walls, with the difference that we have written more general solutions of the differential equation (\SeeChapter{see section Differential and Integral Calculus page \pageref{second order differential equations}}) without having determine the coefficients (because we are interested here in a generalization).

	Thus, in our study of the well with rectangular walls earlier above we had already determined that:
	
	\begin{tcolorbox}[title=Remarks,colframe=black,arc=10pt]
	\textbf{R1.} We see that the wave numbers $k$ are therefore proportional to the root of the kinetic energy. And since the kinetic energy is proportional to the square velocity of the particles, then the velocity is proportional to the wave number (and vice versa)!\\

	\textbf{R2.} In some textbooks, in order to simplify the notations, the potential in regions \texttt{I} and \texttt{III} is assumed as a reference and therefore equalized to $0$. The term $E_p$ therefore disappears from the two preceding expressions and this has the effect of equalizing the both wave numbers $k_{\texttt{I}}$, $k_{\texttt{III}}$ which are then simply denoted $k$.
	\end{tcolorbox}
	In region \texttt{II}, the idea is that $E_{\text{tot}}-E_p$ is negative and constant so the differential equation can be written in one dimension:
	
	and as we have seen it in our study of the infinite rectangular potential well according to the second approach, the solution is then of the form:
	
	with:
	
	\begin{tcolorbox}[title=Remark,colframe=black,arc=10pt]
	The parenthesis under the root of the preceding relation must therefore be positive. This would mean that the kinetic energy of the particle is negative. To overcome this problem within the framework of this simplified model, it is said that the particle has no right to exist in the barrier and that the particle borrow of energy from vacuum. But there are other more complex models that do not require this kind of "fantasies".
	\end{tcolorbox}
	We thus get very simply the analytic expression of $\Psi$ in the three regions in general form:
	
	Let us suppose now that we have at $-\infty$ (region \texttt{I}), a source of particles (which sends them to the right), with a kinetic energy obviously equal to $E_{\text{tot}}-E_{p}^{\texttt{I}}$.
	
	Thus, these particles have an energy $E_\text{tot}$ and the wave function that describes them obeys the Schrödinger equation. In Region \texttt{III}, it will be assumed that there can be only particles going to the right (no source at $+\infty$ by hypothesis).

	The region \texttt{III}, as well as region \texttt{I}, is of infinite extension, so the principle of uncertainty allows us to speak in theory of a perfectly determined linear momentum which we denote by $p'$.

	We know that (this is Classical Mechanics!) in the region \texttt{III} we then have:
	
	If $E_\text{tot}>E_p^{\texttt{III}}$ then $p'$ is positive, so thanks to the preceding relation and to the de Broglie relation we have:
	
	Therefore:
	
	Since the wave numbers are now known formally, let us return to the interpretation of solution \texttt{III}:
	
	The assumption that the particles come from the left imposes us that $B'=0$ so that the solution only describes particles that go to the right. Then, it is possible, for those coming from the left, to take $A'=1$. The region \texttt{III} is therefore relatively simple to analyze...
	
	\begin{tcolorbox}[title=Remark,colframe=black,arc=10pt]
	The conditions and assumptions used previously are sometimes referred to as "scattering conditions\index{scattering conditions}".
	\end{tcolorbox}
	The constants $A$ and $B$ of the region \texttt{I} will be completely determined by carrying out the connection of the solutions from one region to the other.

	Let us now turn to the interpretation of the equation in Region \texttt{I}:
	
	It is evident that $Ae^{\mathrm{i}k_{\texttt{I}}}$ describes particles which, in region \texttt{I} are directed to the right, when $Be^{-\mathrm{i}k_{\texttt{I}}}$ describes particles which in this same region are directed to the left. As we know, the first are the incident particles, the second are the reflected particles.
	
	What we ask of Quantum Physics now appears in a clear way: a particle arriving from the left (incident) can either:
	\begin{enumerate}
		\item Continue to the right, that is, cross the region \texttt{II} and become a transmitted particle

		\item Go back to the left and become a reflected particle.
	\end{enumerate}
	We are led to define a "\NewTerm{transmission coefficient $T$}\index{transmission coefficient}"  assimilated to the probability that the incident particle has to cross the region \texttt{II} and a "\NewTerm{reflection coefficient $R$}\index{reflection coefficient}", the probability that the incident particle will be reflected. We must obviously have:
	
	In the case of a potential barrier, $T$ is also named the "\NewTerm{barrier transparency}\index{barrier transparency}".

	To calculate $R$ and $T$, we will define the current fluxes of the various categories of particles (incident, transmitted, reflected).
	
	For example, since the incident particles are described by $A^{\mathrm{i}k_\texttt{I}x}$, the average number of these particles per unit length in region \texttt{I} must certainly be proportional to a given factor with $|A|^2$.
	
	Let $v_{\texttt{I}}$ be their velocity, we see that the flow of the incident particles denoted $j_i$, is then proportional to a given factor to $|A|^2v_{\texttt{I}}$ (through dimensional analysis). Thus, the coefficient of proportionality being of the same nature for the three categories of particles (incident $i$, reflected $j$, transmitted $t$) and from the fact that $v_{\texttt{I}}$ and $v_{\texttt{III}}$ are proportional to $k_{\texttt{I}}$ and $k_{\texttt{III}}$, it follows that $j_i$, $j_r$ (incident flow and reflected flow) and $j_t$ (transmitted flow) are respectively proportional (hence always to a given dimensional factor) ti $k_{\texttt{I}}|A|^2$, $k_{\texttt{I}}|B|^2$, and $k_{\texttt{III}}$ (since we recall that for region \texttt{III} we have found $A'=1$ and $B'=0$).
	
	We deduce from here simply, by a simple ratio, the expressions of the coefficients of reflection $R$ and transmission $T$:
	
	and as in our special case $k_{\texttt{I}}=kk_{\texttt{III}}$ it comes:
	
	Another way of writing things is to say that since the incident wave is summarized to:
	
	and the transmitted wave to:
	
	then:
	
	In all these situations, quantum theory leads in general to small but not zero values of $R$ and $T$!
	
	\begin{tcolorbox}[colframe=black,colback=white,sharp corners]
	\textbf{{\Large \ding{45}}Example:}\\\\
	Let us determine the explicit expression of transparency for the example of our rectangular barrier.\\ 

	For this we know that we must impose the continuity of $\Psi$ in $x=0$ and $x=L$, as well as the continuity of $\mathrm{d}\Psi/\mathrm{d}x$ in $x=0$ and $x=L$.\\

	Let us first recall that we have the three relations (by putting the reference of the potential at $0$):
	
	wither therefore:
	
	We then have for the continuity of $\Psi$ in $x=0$ and $x=L$:
	
	as well as the continuity of $\mathrm{d}\Psi/\mathrm{d}x$ in $x=0$ and $x=L$:
	
	Since $B'$ is zero we have a system of $4$ equations with $5$ unknowns:
	
	We will choose to express all the constants from $A$. To do this we multiply the first line by $\mathrm{i}l$ and we sum it to the second line. We then get:
	
	and then we multiply the third line by $-\mathrm{i}k$ and the sum it to the fourth line. We have then:
	\end{tcolorbox}
	
	\begin{tcolorbox}[colframe=black,colback=white,sharp corners]
	
	Therefore we have the two relations:
	
	or by putting $\alpha=k/K$:
	
	From the second relation, it comes:
	
	And injected into the first one:
	
	Therefore:
	
	Then we have:
	
	and if we put:
	
	Therefore it comes:
	
	Similarly, re-starting from:
	
	From the second relation, it comes:
	\end{tcolorbox}
	
	\begin{tcolorbox}[colframe=black,colback=white,sharp corners]
	
	And injected into the first one:
	
	Therefore:
	
	We then get:
	
	and let us still put:
	
	Therefore it comes:
	
	Notice that we also have:
	
	\end{tcolorbox}
	
	
	\begin{tcolorbox}[colframe=black,colback=white,sharp corners]
	We can now express the constants $A'$ and $B$ as a function of $A$ with the help of the preceding relations. Let us first begin with $B$:
	
	and:
	
	Therefore:
	
	So finally we have:
	
	and therefore:
	\end{tcolorbox}
	
	
	\begin{tcolorbox}[colframe=black,colback=white,sharp corners]
	
	using the properties of the complex module (\SeeChapter{see section Numbers page \pageref{complex module}}):
	
	It remains for us to calculate:
	
	Therefore:
	
	\end{tcolorbox}
	
	\begin{tcolorbox}[colframe=black,colback=white,sharp corners]
	
	We have therefore:
	
	But as (\SeeChapter{see section Trigonometry page \pageref{hyperbolic trigonometry}}):
	
	If $KL\gg 1$ (so at the atomic scale it is rather $K$ which is huge relatively to $L$) we have:
	
	Therefore:
	
	relation that we can found in many textbooks (but without detailed proof as above). Below we have plotted $T$:
	\begin{figure}[H]
		\centering
		\includegraphics[scale=0.7]{img/atomistic/transmission_coefficient_tunneling_matlab.jpg}	
		\caption{Plot of the transmission $T$ coefficient with Matlab 5.0.0.473 for tunneling effect}
	\end{figure}
	\end{tcolorbox}
	
	\begin{tcolorbox}[colframe=black,colback=white,sharp corners]
	following the relation:
	
	\end{tcolorbox}
	
	We thus find in the (special) above example that the coefficient $T$ is very sensitive (exponentially) to a small variation of the width of the barrier, $a$, when the potential of this barrier is small. We can thus visualize atomic sites, for example in Silicium, using a point very close to the material (or "\NewTerm{scanning tunneling microscope (STM)}\index{scanning tunneling microscope}") to be observed. This is the principle of the tunneling microscope where, by approaching a very finely cut conductive point (a few atoms) to a proximity of about $5$ Angstroms of a conductive surface, and imposing a potential difference of a few [mV], we measures a current of a few nanoamperes. The number of electrons passing through the potential barrier (here the vacuum between the two conductive electrodes) decreases exponentially with the width of the barrier.
	\begin{figure}[H]
		\centering
		\includegraphics[scale=1]{img/atomistic/scanning_tunneling_microscope.jpg}	
		\caption[Scanning tunneling microscope concept]{Scanning tunneling microscope concept (source: OpenStax)}
	\end{figure}
	By analyzing the signal of  the current passing through the circuit, one can access a very precise mapping of the measured surface of the order of $0.1$ Angstroms in vertical.
	
	We also notice according to the relation obtained that the not "heavy" particles like the electrons have a greater probability of making a tunnel effect than heavier particles because of the term of mass involved.
	
	STM images are in therefore gray scale, and coloring is added to bring up details to the human eye:
	\begin{figure}[H]
		\centering
		\includegraphics[scale=1]{img/atomistic/stm_nanotube.jpg}	
		\caption[An STM image of a carbon nanotube]{An STM image of a carbon nanotube (source: OpenStax)}
	\end{figure}

	Using the relation obtained previously, one can quite simply calculate the probability that a human being of mass $m$ crosses a wall with a height $h$ (it is then easy to calculate the potential energy) and a thickness $a$. The probability is of the order of $10^{-4\cdot 10^{30}}$....

	However, the most famous example of Quantum tunnelling that can be treated is that of the emission of $\alpha$ particles by radioactive heavy nuclei, the explanation of which was given by the Russian physicist G. Gamov in 1928.

	The proof is relatively simple, but since it constitutes a particular practical case, we have not detailed it in this section, but in that of Nuclear Physics. However, to solve this problem, we need to use an approximation method known as the "W.K.B. method" Named after the physicists Wentzel, Kramers and Brillouin.
	
	The results therefore give a transmission factor $T$ for the $\alpha$ "particle" of:
	
	For the Uranium atom $_{92}^{238}\mathrm{U}$. Moreover, in the semiclassical approximation, the particle $\alpha$ has, in the well, a velocity of the order of $10^7\;[\text{m}\cdot\text{s}^{-1}]$, and it goes back and forth in a nucleus whose radius is of the order of $10^{-14}$ [m]. It thus performs approximately $10^{21}$ oscillations per second where each time it has a probability $T$ to cross the potential barrier. This probability per unit time is thus determined by:
	
	Experimentally, we find:
	
	The model that we will see in detail in the section on Nuclear Physics gives therefore quite satisfactory results.
	
	And here is how looks like  a "typical" STM in this beginning of the 21st century:
	\begin{figure}[H]
		\centering
		\includegraphics[scale=0.7]{img/atomistic/stm_omicron.jpg}	
		\caption[Omicron Scanning tunneling microscope]{Omicron Scanning tunneling microscope (source: Omicron)}
	\end{figure}
	Besides these technical examples, we encounter the phenomenon of quantum tunneling in a much more accessible and very pedagogical case: Thus, when under the condition of total reflection of a beam of light, we approach another prism (on the face of the prism where no ray of light comes out or re-enters) so as to produce a sufficiently thin air space, a small transmitted beam is observed. This is sometimes named "\NewTerm{optical tunneling}\index{optical tunneling}".
	
	In the table we summarize some of the properties of the systems studied in this so far. The table gives an abbreviated name for each idealized system, and an example of a physical system whose potential and total energies are approximated by the idealization. It also gives sketches of the forms of the potential and total energies, and corresponding probability density functions, for each system. If the particle is not bound, it is incident from the left. We have chosen one significant feature of each system to list in the table, but there are many other significant features that we have discussed, which are not listed. In fact, in this chapter we have obtained most of the important predictions of quantum mechanics for systems involving one particle moving in a one-dimensional potential. In the following chapters we shall obtain predictions from the theory for systems involving three dimensions and several particles.
	
	\begin{figure}[H]
		\centering
		\includegraphics[scale=1]{img/atomistic/barrier_potentials_types.jpg}	
		\caption[]{A summary of the systems studied so far (source: \cite{eisberg1985quantum})}
	\end{figure}
	
	\pagebreak
	\subsubsection{Superposition principle}
	At the beginning of the section we have already introduce the basics of quantum superposition during our study of the linear combination of quantum states. We would like now to deepen the subject a bit more to get some very important results.

	Let us recall that the notion of the dynamic state of a classical system plays a crucial role in Classical Mechanics.

	Is it possible to fall back on this concept when we are dealing with a quantum system, that is to say a system such as an atom, a nucleus or a molecule, in short a microphysics system?
	
	At first glance no! Because we know that one defines the dynamic state of a classical system by the data of the generalized coordinates $q_i$ and conjugate moments $p_i$ at a given instant (\SeeChapter{see section Analytical Mechanics}). However, the principle of uncertainty is opposed to this procedure as soon as we are in the field of microphysics, given the impossibility of accurately measuring the $q_i$ and $p_i$. This is particularly clear when the system is reduced to a single particle which we describe by its cartesian coordinates $q_i(x,y,z)$ and the components of its linear momentum $\vec{p}(p_x,p_y,p_z)$.

	Fortunately, there is another definition of the dynamic state of a system which applies indifferently to Classical and Quantum systems and which, in the case of the former, is identified with the usual definition. We will give this definition on the basis of a brief theory of the sets of identical systems.
	
	If we have a set $(E)$ of a very large number of identical systems, we will make a statistical survey to characterize this set as follows: we take a system of the set, we measure a dynamic variable (coordinate, linear momentum, kinetic energy, etc.) and we reject the system (which disturbed by the measure, must not be reincorporated to the initial set). We thus draw up a balance sheet which is translated by distribution functions of all the possible dynamic variables. This makes it possible to define unambiguously the notion of "identity":
	
	\textbf{Definition (\#\mydef):} Two physical sets are say to be "\NewTerm{identical physical systems}", if the results of the measurements are the same for the both.
	
	Let us now consider a unique set $(E)$. Is it possible to realize it by juxtaposition of two sets non-identical sets $(E_1)$ and $(E_2)$? This would allow us to write:
	\\
	If Yes, we will say that $(E)$ is a "\NewTerm{mixture of physical systems}". Conversely, by suitable sampling, a mixture can be broken down into two different sub-systems. If not, we will say that $(E)$ is a "\NewTerm{pure set}" or "\NewTerm{pure physical system}". Any sampling will decompose the pure set into two subsets identical to each other and necessarily also identical to $(E)$! We then agree to say that all systems of a pure set are in the same dynamic state and that two different pure sets give rise to different dynamic states. It goes without saying that the systems constituting a mixture will be in different dynamic states.
	
	Suppose now that the studied systems obey the laws of Classical Mechanics. If the systems of a set have different pairs $(q_i,p_i)$, we sort them by grouping them by systems all having a same pairs $(q_i,p_i)$. We check then that the new definition of the dynamic state coincides with the usual definition. Let us notice this obvious but important fact (as opposed to quantum systems): in a pure set of classical systems, that is to say for a given dynamic state, every dynamic variable is well determined. Indeed, in Classical Analytical Mechanics, such a variable is a function of the $q_i$ and $p_i$ and, therefore, has a unique value.
	
	Let us pass to quantum systems. It is now possible to define for them a dynamic state, but immediately we see a fundamental distinction with Classical Mechanics. Indeed, in a pure set of quantum systems, that is to say for a given dynamic state, a dynamic variable is not, in general, well determined as already mentioned. When we measure it on systems extracted from the pure set, we usually do not find a single value, but a distribution of values as a result.
	
	The incertitude which reigns over the value of a dynamic variable in a given dynamic state is therefore purely quantum in nature and it is well to distinguish it from the indeterminacy of statistical origin manifested in a mixture, whatever they are classical or quantum systems.
	
	The formalism of quantum physics can only be edified if we know how to describe mathematically dynamic states and dynamic variables. We have seen that we can not expect from this formalism a precise prediction as in Classical Mechanics, but simply the probabilities of obtaining this or that value when we measure a dynamic variable on a system whose dynamic state is given.
	
	The whole theory we have seen so far allows us to conclude that the dynamic states of a system of a spin-free particle are described by complex, wave functions non-zero everywhere.

	If we apply this condition to dynamic systems, then we have the following assumption:
	
	$\Psi$ is then a wave function describing a possible dynamic state of the system. What is often written  with the Dirac ket-bra notation in the following form:
	
	This postulate seems quite natural because of the wave aspect of the physics of microsystems. Indeed, in the wave phenomena of classical physics, wave equations are, in most cases, linear homogeneous, and it follows that the waves can be superimposed (\SeeChapter{see section Wave Mechanics page \pageref{superposition wave principle}}). Now, the great interest of this postulate is that it contains in germ the explanation of this fundamental fact which is quantum indeterminacy (also sometimes named "\NewTerm{quantum coherence}\index{quantum coherence}").
	
	Let us see it on a very simple case where we assume that a dynamic variable $A$ has a well-defined value $a_1$ in the dynamic state $\Psi_1$ and a well-defined value $a_2$ in the dynamic state $\Psi_2$ with $a_1\neq a_2$. This means that if we repeat the measurement of $A$ on systems all in the dynamic state described by $\Psi_1$, we will find each time $a_1$, same for $\Psi_2$ with $a_2$.
	
	A question comes naturally to mind: if we measure $A$ on systems all in the dynamic state $\Psi$ what are we going to get? A naive idea would be to believe that $A$ will take a well defined intermediate value between $a_1$ and $a_2$.

	These two assumptions are false and we know that! First, $A$ is not well defined in quantum physics (uncertainty) and second it is not necessarily mathematically located between $a_1$ and $a_2$. The correct interpretation is as follows:
	
	If we measure $A$ on the system in the dynamic state $\Psi$, we find as a result of measurement, sometimes $a_1$, with a probability $p_1$, sometimes $a_2$, with a probability $p_2=1-p_1$. Of course, $p_1$ and $p_2$ and will have to be calculated according to $\lambda_1$ and $\lambda_2$.
	
	\begin{tcolorbox}[title=Remark,colframe=black,arc=10pt]
	The reader must especially not confuse the pure set of systems described by $\Psi$ with the mixture that we would obtain by juxtaposing two pure sets of systems respectively $\Psi_1$ and $\Psi_2$.
	\end{tcolorbox}
	
	\begin{tcolorbox}[colframe=black,colback=white,sharp corners]
	\textbf{{\Large \ding{45}}Example:}\\\\
	Let us consider the following superposed academic case:
	
	We can already easily verify that the system is normalized:
	
	Since the superposition is normalized, then it comes the probability of finding the system in for example in the state $\Psi_1$ which is then under the assumption of the orthonormality $\langle \Psi_i|\Psi_j \rangle=\delta_{ij}$:
	
	and if we do the same for each of the other two states the sum of the probabilities will always be equal to $1$. It should be pointed out that this probability is also the proportion of states that will be measured in the state $\Psi_1$ if the system is constituted of $N$ identical components.
	\end{tcolorbox}
	In fact, the interpretation given by de Broglie's theory (associate a wave function to a particle) with the principles of uncertainty is the most striking and well known example of superposition principle of states in quantum physics (Schrödinger's Cat case putted aside)! To see why, let us consider a de Broglie wave propagating in the direction of the $x$-axis, but limited to an interval $(-a,+a)$ at a given instant ($t=0$ if we want). Hence at $t=0$ the wave is written, dropping the multiplicative constant:
	
	If we measure the coordinate of the particle, we must find it there necessarily where $\Psi$ is not zero (otherwise we could not measure anything). We can say that $x=0$ with obviously an uncertainty $\Delta x=a$ (the interval where we are sure to find the particle with respect to the origin divided by two: $d((-a,+a))/2$.
	
	If we measure $p$, what do we find? We must not find $\hbar k_0$ (a relation which we have already proved earlier above), as this would be true for an indefinite plane wave, which is not the case here. Then we will decompose the wave into plane waves by means of the Fourier transform (\SeeChapter{see section Sequences and Series page \pageref{fourier transform}}):
	
	How to interpret this relation? One of the elementary plane waves (which we can also interpret as a state):
	
	whose sum gives back $\Psi(x)$, leads to a value $p=\hbar k$ of the linear momentum. But, the values of $k$ form a continuum. We are led to say that the possible values of $p$ then form also continuum and that there is therefore an uncertainty about the value of $p$. To go further, one has to evaluate $a(k)$ (which must be considered as variable of the probability of presence of each plane wave coming from the decomposition of $\Psi(x)$) by means of the relation (according to the properties of the Fourier transform proved in the section Sequences ans Series):
	
	which gives (reduced to) here:
	
	Let us put $k_0-k=\Delta k$, then the integral becomes:
	
	Let us recall the plot of $\sin(u)/u$ (that is the "cardinal sine" as seen in the section of Trigonometry) that show that $a(k)$ takes values that can be considered negligible for $|u|>\pi$:
	\begin{figure}[H]
		\centering
		\includegraphics[scale=1]{img/geometry/sinus_cardinal_2d.jpg}
		\caption{2D plot of the sinc function with Maple 4.00b}
	\end{figure}
	It follows that in the integral:
	
	It is the $k$ at the neighborhood of $k_0$ that are effective, and more precisely the $k$ such that:
	
	since:
	
	It follows that the values to be retained of $p$ are those close to $\hbar k_0$, more precisely we have if we multiply the priore previous relation by $\hbar$:
	
	Thus:
	
	That is:
	
	Therefore:
	
	Finally (it is usage to replace $a$ by the letter $x$):
	
	So we fall back on a well know result already introduced at the beginning of this section but now with an inequality "$<$".
	
	Similarly, if we propose to determine the $x$-coordinate of an electron by passing it through a slit of width $2b$ pierced in a screen:
	\begin{figure}[H]
		\centering
		\includegraphics[scale=1]{img/atomistic/electron_through_slot.jpg}
		\caption[]{Study configuration of the electron passing through a rectangular slot}
	\end{figure}
	The precision with which we know the position of this electron is limited by the size of the slit, ie $\Delta x=b$. On the other hand, the slit disturbs the associated wave. The result is a modification of the electron's motion, which is reflected by the diffraction pattern of the wave (which is in fact a representation of the linear superposition of its intrinsic states).

	The uncertainty on the dynamic component $p_x$ of the electron linear momentum is determined by the angle $\theta$ corresponding to the central maximum of the diffraction pattern. According to the theory of diffraction (\SeeChapter{see section Optical Wave page \pageref{fraunhofer diffraction}}) produced by a rectangular slot, we have $\sin(\theta)=\lambda/2b$ since the intensity $I(\theta)$ is written:
	
	Thus $p_x$ is comprised between $p\sin(\theta)$ and $-p\sin(\theta)$, $p$ being the linear momentum of the incident electron. Thus the uncertainty $\Delta p_x$ is:
	
	This simple result is quite extraordinary if we put it in relation, \underline{in order of magnitude}, with the result we had obtained just above and that was for recall:
	
	We can draw several conclusions of the first importance:
	\begin{enumerate}
		\item The associated de Broglie wave  is closely related to the principle of uncertainty and quantum physics must take simultaneously in account these two properties.

		\item If we take into account that the distribution of the intensity is obtained from the counting of the electrons (or particles) as a function of the angle and that we get the same distribution regardless of the intensity of the monokinetic electron beam that arrives on the slot, even if the electrons are sent one by one. We observe then that the motion of particles is no longer deterministic but probabilistic. Thus, the wave equation of the electron can be considered as a linear superposition of the states defined each as we have done previously, that is to say by its possible spectral decomposition by the Fourier transform.
	\end{enumerate}
	The observation of interference patterns in double-slit experiments with massive particles is generally regarded as the ultimate demonstration of the quantum nature of these objects (\SeeChapter{see section Wave Optics page \pageref{young interference experiment}}). Such matter–wave interference has been observed for electrons, neutrons, atoms, and molecules quite huge molecules (phthalocyanine) and, in contrast to classical physics, quantum interference can be observed when single particles arrive at the detector one by one. The build-up of such patterns in experiments with electrons has been described as the most beautiful experiment in physics!
	\begin{figure}[H]
		\centering
		\includegraphics[scale=0.9]{img/atomistic/molecules_quantum-superposition.jpg}
		\caption[Comparison of interference patterns]{Comparison of interference patterns for $\mathrm{PcH}_2$ and $\mathrm{F}_{24}\mathrm{PcH}_2$ (source: \cite{juffmann2012real})}
	\end{figure}
	What can we conclude from all that we have seen so far:
	\begin{enumerate}
		\item The equations of Wave Quantum Physics give us a probability density to find a particle in a certain volume of space-time.

		\item The linear superposition of states can be interpreted as the fact that it is possible to find a particle at several points in space-time at a given instant, and with for each of these points a certain probability of finding it there (by possible decomposition of the wave equation).
	\end{enumerate}
	If point (1) has been widely studied so far on this site, point (2) is new and arises from a simple mathematical operation of decomposition or superposition.
	
	But what happens if we try to measure the energy of an atom that is in a superposition of states of energy? We will never detect this superposition, but only one of the energies that constitute it, the action of measuring eliminates the superposition of the states in favor of a single one - we speak then of "\NewTerm{quantum decoherence}\index{quantum decoherence}" (this is the Copenhagen interpretation which we mentioned implicitly at the very beginning of this section and to which we shall return later). But which one? Quantum physics can not simply answer this question for now.... It seems that this choice is made randomly! On the other hand, in the absence of predicting the precise state which will be measured among all those who constitute the superposition, quantum theory can give the probability that one has to measure each state (what we have already repeated many times, right here). If many measurements are made, we finally find the proportions predicted by the theory (even if each measure is unpredictable).
	
	Erwin Schrödinger, had emphasized the absurdity (according to him) of these superpositions by resorting to a thought experiment become famous: Imagine a cat enclosed in an airtight box. In the box is also a radioactive atom and a device capable of spreading poison. When the radioactive atom disintegrates, it triggers the deadly device: the poison spreads into the box and the cat dies.
	
	The absurdity of this experience (that we will in-deep in the section of Quantum Computing) is obvious ... but difficult to prove, at least until we understand what distinguishes a Cat from a Particle. Always the problem of the quantum-classical boundary ...

	It was not until the 1980s that the situation finally progressed, both on the front of experience and on that of theory. In 1982, Wojciech Zurek, a researcher at the Los Alamos National Laboratory in New Mexico, takes up a very simple but brilliant idea: in a measurement, what produces decoherence, is the interaction of the system with its environment. More generally, quantum objects are never completely isolated from their environment - we mean everything that interacts with the system: an apparatus, air molecules, light photons, etc. So that in reality the quantum laws must be applied to the whole consisting of the object and all that surrounds it! However, Zurek shows that the multiple interactions with the environment lead to a very rapid destruction of the quantum coherence of the superpositions of states (also called "\NewTerm{quantum interference}\index{quantum interference}" since mathematically one deals with wave functions). By destroying interferences, the environment suppresses state superpositions and quantum behavior, so that only simple states remain and classic behavior returns.
	
	In a macroscopic object - a cat for example ... - each of the atoms is surrounded by many other atoms that interact with it. All these interactions spontaneously cause a jamming of the quantum interferences which disappear very quickly. This is why quantum physics does not apply to our scale (the probability that it happens is too small): systems are never isolated!

	The speed of the decoherence increases with the size of the system: a cat that has $10^{27}$ particles, "decoheres" in $10^{-23}$ seconds, which explains why we have never seen zombie living-dead cat until today...!
	
	Quantum physics is therefore a theory:
	\begin{itemize}
		\item Non-deterministic (probabilistic) hence the fact that it is considered as a theory of information

		\item Non-local: quantum objects may simultaneously have several positions

		\item non-separable: several quantum objects can be superimposed so that they can not be considered separately.
	\end{itemize}
	Another excellent example of the linear superposition of states is a remarkable application to the principle of least action!

	To see this let us consider a quantum particle going from point $M_1$ at a moment $t_1$ to a point $M_2$ at a moment $t_2$. We know that the probability of finding a particle at a given point and at a given instant is related to the square of the module of the wave function associated with it. Let us put in the simplest case where the wave function of the particle is a undimensional plane wave $\Psi(x,t)$ given by the solution function of the Schrödinger equation of evolution:
	
	where $\lambda$ and $\nu$ are respectively the wavelength and frequency of the wave associated with the particle.

	The particle can take an infinite number of paths to get from $(M_1,t_1)$ to $(M_2,t_2)$. Let us choose any one of these paths which we shall denote by the letter $C$. We can cut the path $C$ into an integer number of lengths $\mathrm{d}t$.
	\begin{figure}[H]
		\centering
		\includegraphics{img/atomistic/path_decomposition.jpg}	
	\end{figure}
	After having traveled the first part of the path, the wave function has the following value:
	
	From this we get that:
	
	But, Planck and de Broglie have established the following relations as we have showed it earlier above:
	
	hence, by replacing $\lambda$ and $\nu$ in the preceding relation, we get:
	
	Applying the same technique for the following part we get:
	
	Proceeding like this from part to part along the path $C$ then we get the value of the wave function on $(M_2,t_2)$ for the particle coming from $(M_1,t_1)$ by following the path $C$:
	
	Now, let the take the limit of the duration $\Delta t$ of each trajectory part to zero such that $\Delta t\rightarrow \mathrm{d}t$. The quantity $\Delta x/\Delta t$ then tends to the instantaneous velocity $\mathrm{d}x/\mathrm{d}t$ of the particle that we will denote by $\dot{x}$. The preceding relation then becomes:
	
	In the chapter of Analytical Mechanics, we have shown that the quantity $p\dot{x}-E$ is equal to the Lagrangian $L$. By substituting the Lagrangian in the preceding relation, we get:
	
	where $S_C$ is the action of the particle that has traveled the path $C$.
	Notice (without proof but we can detail on demand) that the module $\Psi_C(x_2,t_2)$ takes the same value for:
	
	for all $n\in\mathbb{N}$. The Planck constant then finds a physical significance directly related to the action of the particle!

	Let us recall the de Broglie condition of normalization:
	
	which therefore gives the probability that the particle, starting from $x_1$ at the instant $t_1$, is in $x_2$ at the instant $t_2$, having traveled the path $C$.

	The total probability is thus:
	
	To calculate the probability that the particle going from $x_1$ at the instant $t_1$ to $x_2$ at the instant $t_2$ requires to makes the sum of the contributions of each path either (by applying the principle of linear superposition since we perform a sum of the wave functions):
	
	That latter equality is most of written as:
	
	where $\mathcal{D}x$ denotes integration over all paths.
	
	This integral was discovered by Richard Feynman and is named "\NewTerm{Feynman path integral}\index{Feynman path integral}" or sometimes "\NewTerm{sum over histories method}\index{sum over histories method}". In the first analysis it seems to diverge as there exist an infinity of possible paths between two points. Let's take a closer look at what's going on. Let us place ourselves in the case where the trajectory is macroscopic. The value of the action $S_c$ is then much greater than $\hbar$ and varies greatly from one path to another, except for the paths close to the classical physical path for which the variation is almost zero (application of the variational statement of the principle Of lesser action).

	Since the actions of paths intervene as a phase in the path integral above, their contributions are destructive and therefore tend to cancel out, except in the case of paths close to the classical physical path where contributions are added. It follows that the integral of path takes the value of the classical action, indicating that quantum physics makes it possible to find the laws of classical mechanics on a macroscopic scale.
	\begin{figure}[H]
		\centering
		\includegraphics{img/atomistic/feynman_paths.jpg}	
		\caption{Representation of the phases according to the type of path}
	\end{figure}
	The situation becomes very different at the quantum scale, that is, for values of the action $S_C$ whose order of magnitude is that of the constant $\hbar$. An infinity of paths then brings non-destructive contributions. Feynman was able to prove that the path integral converged but on the other hand, it is no longer possible to predict which path the particle will take to the point that the concept of path itself vanishes. Thus on the quantum scale the particle seems to seek its way among all those that are possible but on a macroscopic scale, this quantum trial and error seems to have allowed the particle to find the "right path".

	The path integral formalism is a very original way of approaching and interpreting quantum physics that has been added to those developed by Schrödinger.	 
	\begin{tcolorbox}[title=Remark,colframe=black,arc=10pt]
	A common question on Internet is: \textit{Are quantum entanglement and superposition principle the same}? In fact, both are different concepts. Superposition of two state means a quantum system is in two state at a time. But, entanglement says the correlation of two or more system in a ensemble. Which means even if two two system are spatially separated the measurement of any observable will be effected by the other.
	\end{tcolorbox}
	
	\pagebreak
	\subsubsection{Ehrenfest theorem}
	This theorem makes it possible to connect Newton's Classical Mechanics to Quantum Physics by establishing similar relations with respect to the linear moment and the force.

	For this, we start from the special case of a massive particle moving at a non-relativistic velocity in a potential. We then have the one-dimensional Schrödinger equation of evolution:
	
	from which we get (useful for further below):
	
	If we take the complex conjugate on both sides of the equality and multiply the two members by $\hbar$:
	
	from which we get (useful also for further below):
	
	Let us consider the temporal variation of the mean position of the particle (5th postulate):
	
	We have:
	
	hence:
	
	Let us use this last relation:
	
	Let us now use the relation:
	
	and let us inject it into the prior-previous relation:
	
	The first term to the right of the equality is easy to integrate ... (since there is no need to integrate it...):
	
	and since the wave function must be $0$ at $x=\pm \infty$  (otherwise the energy is infinite) then this last relation is equal zero. We then have:
	
	Thus:
	
	and finally:
	
	Which is the equivalent in Classical Mechanics of:
	
	and which reconfirms the existence of the mathematical being:
	
	as being the linear momentum operator, and which we determined earlier by falling back on Newton's second law.

	But we can do a little better at the level of the classic/quantum analogy by derivating:
	
	Which gives:
	
	hence:
	
	Using:
	
	It comes:
	
	Let us focus on:
	
	Let us integrate the first term twice according to the relation proved in the chapter of Differential and Integral Calculus:
	
	We then have (always considering $\Psi$ as a decreasing function to infinity):
	
	and once again:
	
	So finally:
	
	We then have:
	
	However, we have demonstrated in the section of Classical Mechanics that:
	
	It therefore comes that:
	
	This extraordinarily simple result constitutes the "\NewTerm{Ehrenfest theorem}\index{Ehrenfest theorem}". We thus find the fundamental law of Classical Dynamics in the sense of mean values of position and force, calculated using the probability of presence!
	
	\pagebreak
	\subsection{Angular momentum and Spin}\label{angular momentum and spin}
	In classical physics, angular momentum is commonly divided into two types: orbital and rotational. In the case of the Earth, for example, the orbital angular momentum is associated with the revolution of the Earth about the Sun, while rotational angular momentum is associated with the Earth rotating on its axis. Rotational angular momentum is often known as spin, for fairly obvious reasons.

	In classical physics, of course, there is really only one type of angular momentum, since both the orbital and rotational types are associated with the rotation of masses about some axis. The division into the two types is purely a computational convenience.
	
	In quantum physics, the situation is quite different is quite different as we will see (it is important that the reader refers to the Corpuscular Quantum Physics section where we have proved the it is seems not acceptable ton consider the electron spinning on itself).

	Like the harmonic oscillator, the notion of angular momentum is of main importance in quantum theory and has many applications in all fields of physics: atomic and molecular physics, nuclear and subnuclear physics, physics of condensed state, etc. Thus, it plays an essential role in the study of the motion of a particle in a potential with spherical symmetry, as we shall see in the section of Quantum Chemistry (which is an excellent practical example). The angular momentum is also at the base of the group of rotations (\SeeChapter{see section Set Algebra page \pageref{unitary linear group}}) which satisfies the algebra of the angular momentum operators. Thus, it not only allows to construct the wave function of a given quantum system of symmetry, but also to predict if an optical transition is allowed and to determine its intensity (for example, during the study optical transitions between impurity states (in solid state), molecular states (quantum chemistry), nuclear physics, etc.).
	
	Finally, we will see that the algebraic method applied to the study of angular momentum will allow us to introduce quite naturally the notion of intrinsic angular momentum of a particle, the "spin", which has in fact no classical equivalent (so it's not an angular momentum!).

	The following developments may seem rather disconcerting in the sense that it is no longer necessary to rely on intuition but only on the properties and results of mathematics. As usual, if the reader need additional information, he must not hesitate to contact us.
	\begin{tcolorbox}[title=Remark,colframe=black,arc=10pt]
	"\NewTerm{Spintronics}\index{spintronics}" (a portmanteau meaning "spin transport electronics"), also known as "spinelectronics" or "fluxtronics", is the study of the intrinsic spin of the electron and its associated magnetic moment, in addition to its fundamental electronic charge, in solid-state devices or in Quantum computers. The simplest method of generating a spin-polarised current in a metal is to pass the current through a ferromagnetic material. The most common applications of this effect involve giant magnetoresistance.  (GMR) devices. 
	\end{tcolorbox}
	To begin, let us recall that the angular momentum of a particle relatively to their origin is given by (\SeeChapter{see section Classical Mechanics page \pageref{angular momentum}}):
	
	Since the linear momentum is quantified (it is an eigenvalue related to energy in one way or another), the angular momentum is necessarily also quantified (the angular momentum is therefore an eigenvalue) and the experience has supported this result (Stern-Gerlach experiment).

	Given $z$ the component of the resulting vector product:
	
	This relationship being cyclic, we can change the indices to get the other coordinates.

	Since $x$ and $y$ commute (in the sense that their commutator is equal to zero) and that we have proved earlier above that:
	
	We then have:
	
	Which gives:
	
	Using the gradient (we will see again this relation in the section of Relativistic Quantum Physics during our study of the Pauli equation!!):
	
	and putting for the "\NewTerm{operator of the orbital angular momentum}\index{operator of the orbital angular momentum}":
	
	Which lead us to write:
	
	with:
	
	\begin{tcolorbox}[title=Remark,colframe=black,arc=10pt]
	Most often in the literature the orbital angular momentum is denoted $\vec{L}$ (we have already made this remark in the section of Classical Mechanics) but we avoided this notation here in order to differentiate the orbital angular momentum and the \underline{total} orbital angular momentum.
	\end{tcolorbox}
	We will establish some commutation relations concerning $\vec{l}$ which will play an essential role in our study of the spin. Using the following commutation relations (proved during our study of the Heisenberg classical uncertainty principles):
	
	and:
	
	We have the relation (it is traditional to do the analysis on the component $z$ of the projection of $\vec{l}$):
	
	Therefore:
	
	and proceeding in the same way:
	
	\begin{tcolorbox}[title=Remark,colframe=black,arc=10pt]
	We find analogous relations with the linear momentum:
	
	\end{tcolorbox}
	Let ow now evaluate the quantity (following the request of a reader, we put all the details):
	
	Either after simplification (it is rather annoying for the experience that this does not commute):
	
	By the way, at this stage, the reader that has already covered the Spinor Calculus section beforehand (if not already done we strongly recommend the read to go read it!), he will notice that the Pauli matrices satisfy the preceding relations if we put ourselves in natural units (the reduced Planck constant then being equal to $1$):
	
	This observation will be useful for our study of Relativistic Quantum Physics (see section of the same name page \pageref{relativistic quantum physics}). Indeed, we know from our study of Spinor Calculus that the $2\times 2$ matrices in $\mathbb{C}$ of determinant $1$ form the group of rotations in the space $\text{SU}(2)$, whose Pauli matrices are the generators. Basically, the origin of the spin comes from the link between $\text{SU}(2)$ and the group of rotations of our ordinary space, $\text{SO}(3)$ (\SeeChapter{see section Set Algebra page \pageref{special real group orthogonal}}).
	Now, let us consider the norm:
	
	where we must consider the square of one of these operators in the following form:
	
	Let us study its commutation with a component (without having to explicit the thing!):
	
	Using the cyclic relation $[l_x,l_y]=\mathrm{i}l_z$ it comes:
	
	Therefore the norm of the orbital angular momentum commutes with its components:
	
	Conclusions of the results obtained so far: Since the commutator is equal to zero (the quantities commutes), it is therefore possible to simultaneously measure with precision a component as well as the square of the angular momentum (its squared norm), but it is impossible to do the same for two components!
	
	Let us also notice that finally the fact that we have explicitly:
	
	we can write (its quite an interesting relation):
	
	If we have a system of particles numbered by an the index $k$, each has an individual kinetic moment $\hbar\vec{l}^{(k)}$ and the total orbital angular momentum of the system $\hbar\vec{L}$ (not to confuse the notation of the $L$ with that of the Lagrangian that latter being anyway as scalar!!!) is then obviously given by:
	
	and if we simplify by $\hbar$ (or if we work in natural units):
	
	But $\vec{L}$ is not really the total angular momentum of the system! Indeed, a particle may possess an intrinsic angular momentum, or "\NewTerm{spin}\index{spin}". We can give a simple picture of the spin by saying that it translates an infinitesimal rotation of the particle on itself (caution! it is only an image because in fact the particle does not turn on itself!). As we saw in the section of Spinor Calculus, this corresponds mathematically to the limited development of the matrix of rotations in the vicinity of the identity matrix.
	
	Therefore we see that the usual approach to quantum spin is just to postulate that elementary particles have an intrinsic spin, and that this spin isn't due to any physical motion of the particle (again refer to the section of Corpuscular Quantum Physics); it just \textbf{\textit{is}}, in much the same way that particles have mass, charge and (in the case of the more esoteric particles) several other quantities such as strangeness. This is not terribly satisfactory from an intellectual point of view, but since I'm not qualified to write about the details of particle physics, we have to start somewhere.
	
	We will denote by $\vec{s}^{(k)}$ the intrinsic angular momentum of the $k$-th particle (in natural units) and the relation:
	
	will be the "\NewTerm{total spin}\index{total spin}\label{total spin}" and finally (still in natural units or after simplification of $\hbar$):
	
	will be the "\NewTerm{total angular momentum}\index{total angular momentum}" of the system (do not confuse the notation $J$ with the orbital angular momentum or the current density....!) and we will prove during our study of the spin-orbit coupling that this angular momentum is a constant of movement in the presence of this coupling.
	 \begin{tcolorbox}[title=Remark,colframe=black,arc=10pt]
	In the weak field case this vector model ($\vec{J}=\vec{L}+\vec{S}$) at left implies that the coupling of the orbital angular momentum $\vec{L}$ to the spin angular momentum $\vec{S}$ is stronger than their coupling to the external field. In this case where spin-orbit coupling is dominant, they can be visualized as combining to form a total angular momentum $\vec{J}$ which then precesses about the magnetic field direction (see further below for an illustration).\\
	
	In the strong-field case, $\vec{S}$ and $\vec{L}$ couple more strongly to the external magnetic field than to each other, and can be visualized as independently precessing about the external field direction such that $\vec{J}\neq \vec{L}+\vec{S}$. This is what we name the \NewTerm{Paschen-Back effect}\index{Paschen-Back effect}". The fields required to create Paschen-Back conditions for sodium are unrealistically high. Lithium, on the other hand, has a spin-orbit splitting of only $0.00004$ [eV] compared to $0.0021$ [eV] for sodium. The Paschen-Back conditions are met in some lithium spectra observed on the Sun, so this effect does have astronomical significance.
	\end{tcolorbox}
	We will assume (but this is relatively easy to prove once, among others, spinors are know\footnote{As always we can detail on reader request}) that each $\vec{s}^{(k)}$ and $\vec{l}^{(k)}$ also obeys the commutation rules seen previously:
	
	Therefore to sum up we have:
	
	where for recall $\varepsilon_{ijk}$ is the Levi-Civita symbol (\SeeChapter{see section Tensor Calculus page \pageref{levi civita symbol}}).
	
	We notice that in the matrix representation of Heisenberg there are also other matrices than the $2\times 2$ Pauli matrices that satisfy these relations (we can give the detail of the proof on reader request). For example, the following zero trace Hermitian matrices (of which the conjugate transpose is equal to itself for recall ....):
	
	\begin{tcolorbox}[title=Remark,colframe=black,arc=10pt]
	We will see in the section of Relativistic Quantum Physis that in fact the three matrices above are the spin operator of the massless spin-$1$ particles (photons).
	\end{tcolorbox}
	The prior-previous relations implies (also) in the same way as for the orbital angular momentum:
	
	with obviously the relation:
	
	named by the mathematicians "\NewTerm{Casimir element}\index{Casimir element}" or "\NewTerm{Casimir operator}\index{Casimir operator}" (a simple development perfectly similar to that obtained above is sufficient to prove it and we can write it on reader request).
	
	Let us now define in a purely formal way the two non-Hermitian operators named "\NewTerm{scale operators}\index{scale operators}" (Pauli's matrices and not only (!) still satisfy these relations!):
	
	where respectively $J_+$ is named "\NewTerm{elevator operator}" and $J_-$ "\NewTerm{step-down operator}".

	The $J_{\pm}$ commutes with $||\vec{J}||^2$, since it commutes with $J_x$ and $J_y$. This allows us to write the product:
	
	Furthermore:
	
	Therefore:
	
	Identically (we can detail on reader request):
	
	Finally, let us evaluate the products $J_+J_-$ and $J_-J_+$:
	
	Identically (we can also detail on reader request):
	
	Since the two Hermitian operators $||\vec{J}||^2$ and $J_z$ commute they therefore have common states and eigenvalues and, more precisely, they have a common complete common basis. When observables commuted and have their own common base, let us recall that we usually speak of a "CSCO" (Complete Set of Commuting Operators).
	
	To study their eigenvalues let us put:
	
	System which is sometimes denoted in the following form in the specialized literature (...):
	
	as it highlights the fact that the associated eigenstates will be defined at least in part by the parameters $K$, $m$ themselves.
	
	Many times in textbooks the last couple of relations is written:
	
	or even worst...:
	
	To begin, we know that the eigenvalues $K$ and $m$ are not independent since we have:
	
	The mean being denoted by the brackets $\langle \rangle$, we have by linearity of the mean (\SeeChapter{see section Statistics page \pageref{means and averages properties}}):
	
	What can be written:
	
	We see that the left-hand side of the above relation is therefore equal by definition to:
	
	Since the square of the total orbital angular momentum is anyway Hermitian (it has no complex component in $\mathbb{C}$), we then construct by the postulates of quantum physics:
	
	It comes then that:
	
	The latter relation therefore implies that:
	
	This brings so far to the following informations:
	
	From $|\Psi\rangle$, we build the state $J_+|\Psi\rangle$, we will show that if this state is not identically zero, it is an eigenvalue of $||\vec{J}||^2$ and of $J_z$. From the relation:
	
	already introduced previously, we put:
	
	The $J_{\pm}$ commutate with $||\vec{J}||^2$, since it commutes with $J_x$ and $J_y$. This gives us that the previous relation is zero such that:
	
	From the relation $J_zJ_+-J_+J_z=J_+$ we put in and identical way (we multiply both side of equality left by $|\Psi\rangle$) and rearrange a bit:
	
	Still without forgetting that:
	
	We finally have the following package of relations so far:
	
	Therefore in the corresponding order, we have that:
	\begin{enumerate}
		\item $J_+|\Psi\rangle$ and $J_-|\Psi\rangle$ are identically zero
		\item $J_+|\Psi\rangle$ and $J_-|\Psi\rangle$ are eigenstates of the operator $||\vec{J}||^2$ for the eigenvalue $K$
		
		\item $J_\pm|\Psi\rangle$ are eigenstates for the operator $J_z$ and the eigenvalue $m\pm 1$
	\end{enumerate} 
	Since the angular momentum is quantified as we know from our study of Corpuscular Quantum Physice and also from experiments, its eigenvalues must therefore have a minimum and a maximum with for each of these values and associated eigenfunction.
	
	Let us put for what will follow that $m'$ and $|\Psi'\rangle$ are the maxmimum eigenvalue and eigenstate and $m''$ and $|\Psi''\rangle$ the minimum eigenvalue and eigenstate.
	
	Given the following three relations proved earlier above:
	
	We write using also the properties of eigenvalues (\SeeChapter{see section Linear Algebra page \pageref{eigenvector}}):
	
	What intuitively is not obvious to put but which mathematically is quite justifiable.

	From the last two relations above, we can write by subtracting the first one to the second one (use of the superposition of sate principle), that gives us immediately for the eigenvalue that:
	
	thus:
	
	As we have $m'$ being the maximum, $m''$ being the minimum of the same set, we then have:
	
	What gives us after simplification of the second parenthesis that is therefore not zero (because if it could be zero we would not be able to simplify this parenthesis):
	
	Let us denote by $J$ the value $m'$ (which corresponds to the maximum eigenvalue of the quantity $J_z$) since $m''=-m'$ we have:
	
	(where often in the literature we find a lowercase $j$ to avoid confusion with the associated operator) therefore:
	
	Since the difference on the left of the equality is necessarily an integer number (based on the known results that we get in the section of Corpuscular Quantum Physics), this imposes that $2J$ is a positive integer or zero, but it also implies directly that $J$ can only be an integer (!), half-integer or zero such that:
	
	Now as for recall: $m'=J$ and $m''=-J$ it follows that if we denote by $m$ the values that can take $m''$ \underline{and} $m'$ respectively, then for each value of $J$ we have obviously:
	
	Finally, as:
	
	Then we have:
	
	And finally this gives us the eigenvalue:
	
	And since we have put that $m'$ is equal to $J$ and that we have the relation:
	
	therefore:
	
	In a more explicit and less confused form (be careful not to confuse the eigenvalues with the operators!):
	
	So we have finally:
	
	As we have have introduced earlier above the spin-orbit interaction:
	
	As we introduced earlier above:
	
	(component by component of their respective vector/matrix) and if the particle has no spin (S=0) then we have the eigenvalue of the total orbital angular momentum which reduces to the eigenvalue of the angular momentum:
	
	where we no longer indicate the indices of the components or of the matrix indices (useless!).
	
	If we have only one particle then:
	
	Therefore, the orbital angular momentum is written by remembering  that $l$ is also quantified (\SeeChapter{see section Corpuscular Quantum Physics page \pageref{magnetic dipole moment}}):
	
	If we have $l\gg 1$, then in this case:
	
	We thus fall back on the result obtained at the beginning of our study of the angular momentum.
	
	Roughly, if we now put $l\cong n$, we thus that we fall back on the angular momentum relation postulated by Bohr as seen in the section of Corpuscular Quantum Physics. This is why it is customary to take only the integer values of $l$!
	
	\begin{tcolorbox}[title=Remark,colframe=black,arc=10pt]
	Let us recall that in reality $0\le l \le n-1$ and therefore that unlike the Bohr corpuscular model the angular momentum can be zero wit the above approach...! Another way of accepting the values taken by $n$, putting apart the fact to refer to the Bohr model seen in the section of Corpuscular Quantum Physics, is to look at the values that can be taken by $l$ in the quantum model of the hydrogen atom of the section of Quantum Chemistry and that if the latter is equal to $0$ then the associated Legendre polynomials are no longer defined!
	\end{tcolorbox}
	This finding now justifies physically the use of the quantum number $l$ in the periodic table of elements as we saw and defined it  (without any real justification) in the previous section of Corpuscular Quantum Physics.

	Finally, let us mention that exactly the same reasoning and development leads to the following possible values of the pseudo angular momentum of spin:
	
	where experiments show us (to list only the most known examples) that spin $S=0$ is a characteristic of the Higgs boson or of certain atoms, spin $S=1/2$ is a characteristic of the electron / positron, spin $S=1$ is a characteristic of the photon, spin $S=2$ would be a still theoretical characteristic of the graviton. At the time of writing, no particles of spin $S=3/2$ or $S=5/2$ are known (some supposed it could be dark energy particles spin).
	\begin{figure}[H]
		\centering
		\includegraphics[scale=0.57]{img/atomistic/particles_spin_classification.jpg}	
		\caption{Some particles classification by spin (Author: James Childs)}
	\end{figure}
	We will see another diagram of the same type including the Higgs particle in the section of Particle Physics.
	
	We see therefore that the integer or semi-integral value of the spin seems to determine a crucial property of the particle: if its spin is integer, it is a "\NewTerm{boson}\index{boson}", if its spin is half-integer, it is a "\NewTerm{fermion}\index{fermion}" (this can be proven with the "spin-statistics theorem" that as already mention we will perhaps in the future give the proof in this book).
	
	The total angular momentum is thus approximately given by:
	
	By analogy (it is really a dubious analogy ...), we write for $J$ sufficiently large...:
	
	But since the spin of the electron may have only two possible orientations\footnote{Electrons spin in all directions, but in the Stern-Gerlach experiment they deflect only in the $z$ direction because there was only a field in the $z$ direction (\SeeChapter{see section Relativistic Quantum Physics page \pageref{pauli equation}}). This is why we speak of two "privileged" possible orientations.}, the values of $j$ will be in the case of a spin particle $1/2$:
	
	Hence a possible classification of the atomic electrons taking into account their spin\label{types of orbital and spin}:
	\begin{table}[H]
		\centering
		\renewcommand{\arraystretch}{2.6}
		\begin{tabular}{|c|c|c|c|c|c|}
		\hline
		\rowcolor[HTML]{9B9B9B} 
		\textbf{Orbital type} & \textbf{$\pmb{s}$} & \textbf{$\pmb{p}$} & \textbf{$\pmb{d}$} & \textbf{$\pmb{f}$} & \textbf{$\pmb{\ldots}$} \\ \hline
		\cellcolor[HTML]{9B9B9B}\textbf{$\pmb{l}$} & $0$ & $1$ & $2$ & $3$ & $\ldots$ \\ \hline
		\cellcolor[HTML]{9B9B9B}\textbf{$\pmb{j}$} & $\dfrac{1}{2}$ & $\dfrac{1}{2},\dfrac{3}{2}$ & $\dfrac{3}{2},\dfrac{5}{2}$ & $\dfrac{5}{2},\dfrac{7}{2}$ & $\ldots$ \\ \hline
		\multicolumn{1}{|l|}{\cellcolor[HTML]{9B9B9B}\textbf{Notation:}} & $s_{1/2}$ & $p_{1/2},p_{3/2}$ & $d_{3/2},d_{5/2}$ & $f_{5/2},f_{7/2}$ & $\ldots$ \\ \hline
		\end{tabular}
		\caption{Types of orbital and spin for electrons}
	\end{table}
	We see above that for a multiplicity of $2l+1$ states. Indeed, for $l=0$ we have $1$ states, for $l=1$ we have a total of $1+2=3$ states, for $l=2$ we have a total of $1+2+2=5$ states, and so on....
	
	Either in schematic form with the corresponding energy levels:
	\begin{figure}[H]
		\centering
		\includegraphics{img/atomistic/energy_levels_spin.jpg}	
		\caption{Schematic form of associated orbitals and spin energy levels}
	\end{figure}
	This table leads us to conclude that we finally:
	
	Also we therefore introduce a common nomenclature for naming energy terms when we deal with the details of spin-orbit coupling is\footnote{the spin term is sometimes omitted because it's the same for all the levels}:
	
	where for recall $n$ is the principal quantum number, $s$ is the spin therm, $l$ the orbital angular momentum and $j$ the total angular momentum.
	
	To return to more practical considerations ... we finally obtained for the norm of total angular momentum (in the case of a single particle and without spin):
	
	where $l$ is an integer (at least in the case where we don't do the approximation done earlier above). We also know from the section of Corpuscular Quantum Physics that the magnetic moment is given by:
	
	and that the secondary quantum number $l$ and the magnetic quantum number $m_l$ are in some way indissociable.

	In the same way we get:
	
	where can take for a particle like the electron only the values (don't forget that is doesn't mean it can take only two directions! It just means it can take two states and the famous "two directions" legend is due to the Stern-Gerlach experiment type configuration):
	
	Which simply correspond (in natural units) to the two eigenvalues of the eigenmatrices:
	
	which links the spin operator to the Pauli matrices for the electron  by the Dirac equation as we will prove it in the section of Relativistic Quantum Physics:
	
	\begin{tcolorbox}[title=Remark,colframe=black,arc=10pt]
	It is more accurate to say that in reality that the eigenvalues $\pm 1/2\hbar$ are linked to eigenmatrices $\frac{\hbar}{2}\sigma_i$ but as the $\hbar$ simplifies...
	\end{tcolorbox}
	Now, what we know of our results obtained in the section of Corpuscular Quantum Physics is that when $l$ is equal to $1$ we have the magnetic moment which can take three different values according to whether a magnetic field is applied or not:
	
	At this time, although the norm of the total angular momentum remains constant (as it is conservative quantity), its components must necessarily change. Since we know only one of the components of the angular momentum by knowing its norm (commutating operators), we choose focus for pedagogical conventions on $J_z$.
	
	We choose a reference frame such that one of the spatial components is always equal to zero (this is always possible as we know it!). It is then sufficient, for example, in the chosen planar referential $X$, $Z$ (ie the component $Y$ will be zero) to have the norm of $J$ that is equal to $l=1$:
	
	and idem with $S$ by imposing that the norm for $s = 1/2$ is equal to:
	
	There is then three possibilities to arrive at the same result by simply applying the Euclidean norm if one of the components is always imposed as zero! It is because we have:
	
	What we write so (because in fact there is an infinity of possibility) because we want to found a way to introduce the quantum number of orbital projection (which therefore quantifies the projection of the orbital angular momentum along $Z$ and is in multiplicity $2l + 1$):
	
	What some physicists like to represent in a very simplified way by the following diagram:
	\begin{figure}[H]
		\centering
		\includegraphics{img/atomistic/total_angular_momentum_quantification.jpg}	
		\caption{Biased schematic representation of the quantification of total angular momentum in $2$D}
	\end{figure}
	But who in reality (by the square of the components of the norm) should draw in the following form in $2$D (as in $3$D this is represented by cones as we will see below):
	\begin{figure}[H]
		\centering
		\includegraphics{img/atomistic/total_angular_momentum_quantification_improved.jpg}	
		\caption{Improved schematic representation of the quantification of total angular momentum in $2$D}
	\end{figure}
	This allows us to observe on the way that as we have mention earlier above, we can always rotate the system of axes such that one of the components is zero and earlier we did the choice:
	
	Notice that with elementary trigonometry we get:
	
	and therefore the angles take the following values:
	
	Finally, let us indicate that we then have in the particular case of the figure above where $l=1$, $m_l=\pm 1$, that the (quite abstract) system introduced earlier above:
	
	that will be written as:
	
	Which is often written in condensed form as following:
	
	In the same way with the spin $1/2$ of electron we have by introducing the quantum number of spin (which therefore quantifies the projection of the spin moment of spin along $Z$ and can take as many values as there are between $-s$ and $+s$ but by increments of $1$ as required by the experimental results, for which reason there is no zero $Z$ component below):
	
	What some physicists like to represent in a very simplified way by the following diagram:
	\begin{figure}[H]
		\centering
		\includegraphics{img/atomistic/spin_momentum_quantification.jpg}	
		\caption{Biased schematic representation of the quantification of spin in $2$D}
	\end{figure}
	with for angle:
	
	As before, we have in the particular case where $s=\dfrac{1}{2}$ and $m_s=\pm\dfrac{1}{2}$:
	
	This can be written in a more condensed form as following:
	
	We therefore have the only experimentally measurable variables that are:
	
	Which are therefore discrete observables (bivalent for the spin).

	With an artist's view of the concept for the pleasure of the eyes:
	\begin{figure}[H]
		\centering
		\includegraphics[scale=0.25]{img/atomistic/spin_sculpture.jpg}	
		\caption{Schematic representation of various quantifications of angular momentum by the physicist and sculptor Julian Voss-Andreae}
	\end{figure}
	Therefore, by applying a constant uniform magnetic field, the Pauli Hamiltonian (\SeeChapter{see section Relativistic Quantum Physics page \pageref{pauli hamiltonian for a constant direction magnetic field}}) will make jumps equivalent to the relation:
	
	This result means that the energy levels for a given energy ($n$-layer) are separated into several levels distant of $\dfrac{q\hbar}{2m_0}B$ when the atom is placed in a magnetic field. This result is the Zeeman effect which we have discussed several times.
	
	All this makes it possible to better understand the mathematical origin of the $4$ quantum numbers (principal quantum number, secondary or azimuthal quantum number, magnetic quantum number, spin):
	
	also written (since in the particular case of the particles studied in this book the magnetic quantum number of spin projection has the same value as the spin since we are dealing mainly with the electron):
	
	With to sum up a bit all this ...:
	
	The reader must not think that what we have seen so far was developed in one day or even in one year by physicist. Furthermore this abstract approach was highly speculative and needed to be check with experiments. So far it works very good to explain and predict experimental observations but still in this beginning of the 21st century the spin matrices theory is confronted and tested by new experimentation!
	
	\pagebreak
	\subsubsection{Spin–orbit interaction (LS coupling)}\label{ls coupling}
	In quantum physics, the "\NewTerm{spin–orbit interaction}\index{spin–orbit interaction}" (also named "\NewTerm{spin–orbit effect}\index{spin–orbit effect}"  or "\NewTerm{spin–orbit coupling}\index{spin–orbit coupling}") is an interaction of a particle's spin with its motion. The first and best known example of this is that spin–orbit interaction causes shifts in an electron's atomic energy levels due to electromagnetic interaction between the electron's spin and the magnetic field generated by the electron's orbit around the nucleus. This is detectable as a splitting of spectral lines, which can be thought of as a Zeeman Effect due to the internal field. A similar effect, due to the relationship between angular momentum and the strong nuclear force, occurs for protons and neutrons moving inside the nucleus, leading to a shift in their energy levels in the nucleus shell model. In the field of spintronics, spin–orbit effects for electrons in semiconductors and other materials are explored for technological applications. The spin–orbit interaction is one cause of magnetocrystalline anisotropy.
	
	Let us now recall that we have pointed out in the section of Corpuscular Quantum Physics that when we analyze the spectral lines of hydrogen in the absence of any external field at high resolution, we see that they are in fact made up of very close doublets, separated of $0.016$ [nm] in terms of wavelength. This phenomenon is due to a so-called spin-orbit coupling. It is now time to see where it comes from. So let us also recall that we previously obtained:
	
	Hence, the squared norm (what is measured) leads us to write:
	
	Which gives us after grouping:
	
	The term:
	
	is named "\NewTerm{spin-orbit coupling}\index{spin-orbit coupling}". It is the term that during the very precise measurements of spectral lines reveals a splitting of the lines due to the coupling between the electron spin and the orbital angular momentum (this is not an $||\vec{S}||^2$ because this term is always positive).
	\begin{tcolorbox}[title=Remark,colframe=black,arc=10pt]
	Let us recall that when we have two bodies in interaction the total angular momentum is a constant of motion as we have proved it in the section of Classical Mechanics. There may thus be a transfer of angular momentum between these two bodies (this is the spin-orbit coupling!). One loses angular momentum and the other gain angular momentum.
	\end{tcolorbox}
	The measured deviation is thus attributed to the interaction of the electron spin with its orbital angular momentum. The electron revolves around the nucleus, but if we place ourselves on the electron, we see the nucleus turning (on Earth the Sun revolves around the Earth!). Everything happens as if the nucleus created a magnetic field at the level of the electron, and this field interacts with the magnetic moment of the electron, the spin, and this differs according to whether the spin is in the direction of the field or opposite, It is this difference that adds or subtracts some energy from the level.
	
	Here is a first simplified diagram that summarizes the whole:
	\begin{figure}[H]
		\centering
		\includegraphics{img/atomistic/spin_orbit_coupling_simplified.jpg}	
		\caption{Pictorial representation of the spin-orbit interaction}
	\end{figure}
	or in $3$D perspective and that is a bit more accurate:
	\begin{figure}[H]
		\centering
		\includegraphics{img/atomistic/spin_orbit_coupling_3d_simplified.jpg}
	\end{figure}
	Let us prove that $\vec{J}$ as defined is a constant of motion. We have (needless to say that by squaring, these are the components of the vector that we put square and not the vector itself!):
	
	Hence:
	
	Let's make the development for one  component:
	
	But, by definition (of notation) $J_x=L_x+S_x$ therefore:
	
	Now we know that $[L_x,L_x^2]=0$ (because an operator always commutes with itself by construction) and concerning $[S_x,S_x^2$, we have mentioned it in the section of Spinor Calculus and we will prove it in the framework of the study of the free classical Dirac equation (\SeeChapter{see section Relativistic Quantum Physics page \pageref{classical free dirac equation}}), that the electron spin is fully described by the Pauli matrices which are linear operators. Let us then write to a constant factor:
	
	and we will see that this is in conformity with the Pauli equation that we introduce and study in the section of Relativistic Quantum Physics (and vice versa) !!!

	Therefore, disregarding the multiplicative constant:
	
	Which was in any case $100\%$ predictable since in any case, once again, a same operator always commutes with himself.

	So finally:	
	
	Therefore:
	
	Hence finally:
	
	$\vec{J}$ is the indeed total angular momentum which, even in the presence of spin-orbit interaction, is a constant of motion (an obligation for an isolated system!).
	\begin{tcolorbox}[title=Remark,colframe=black,arc=10pt]
	Another way of reading that thing is to say that the measurement on one of the elements of the preceding commutator adapts the other one immediately so that their commutation is zero therefore by extension the total angular momentum is a constant of the motion.
	\end{tcolorbox}
	One notable atomic spectral line of sodium vapor is the so-called D-line, which may be observed directly as the sodium flame-test line and also the major light output of low-pressure sodium lamps (these produce pressure sodium lamps (these produce an unnatural yellow. The D-line is one of the classified Fraunhofer lines Sodium vapor in the upper layers of lines. Sodium vapor in the upper layers of the sun creates a dark line in the emitted spectrum of electromagnetic radiation by absorbing visible light in a band of wavelengths around $589.5$ [nm]. This wavelength corresponds to transitions in atomic sodium in which valence-electron transitions from $3s$ to $3p$ electronic state.
	
	The splitting of the sodium doublet in the presence of an external magnetic field was observed by Pieter Zeeman in 1896, and the effect was subsequently named the "Zeeman effect". It is remarkable that so much detailed spectroscopy was done long before the Bohr theory, and perhaps even more remarkable that Zeeman's first study of the sodium Zeeman splitting was done the year before J. J.Thomson's discovery of the electron in 1897.
	\begin{figure}[H]
		\centering
		\includegraphics[scale=0.8]{img/atomistic/sodium_sl_coupling.jpg}
		\caption[]{Focus on $3P_{3/2}$ Sodium splitting (source: Hyperphysics)}
	\end{figure}
	Or more detailed:
	\begin{figure}[H]
		\centering
		\includegraphics[scale=1]{img/atomistic/sodium_splitting_general.jpg}
		\caption[General overview of Sodium splitting]{General overview of Sodium splitting (source: Hyperphysics)}
	\end{figure}
	After Thomson's work, Zeeman and Lorentz did further study of the influence of magnetic fields on the spectral emissions from atoms. By analysis of the splitting of the sodium doublet, they were able to demonstrate that the charge to mass ratio of the charge responsible for the splitting was the same as Thomson's electron. This was the first direct demonstration that electrons were involved in the production of the spectral line emissions.
	
	\subsubsection{Spin operator for spin $1/2$ particles}
	Let us now return to the relation demonstrated earlier above for $1/2$ spin particles:
	
	We have for sure obtained the eigenvalues, but it would be wise to determine the expression of the corresponding Spin operators $\hat{S}^2$ and $\hat{S}_z$. We have seen in the section of Spinor Calculus that:
	
	and we proved that for the eigenvalue $+1$ the associated eigenvectors were:
	
	We then have:
	
	By multiplying to the left and right by a familiar term:
	
	And by analogy of the fact that the matrix of Pauli are special case of rotations matrices $\text{SU}(2)$ (\SeeChapter{see section Set Algebra page \pageref{special unitary group}}), we put:
	
	And of course we deduce:
	
	Therefore for spin-$1/2$ particles we have:
	
	and we know that it is not the most explicit form as we should write the one we have determine in the section of Spinor Calculus for the Pauli matrices.
	
	\pagebreak
	\subsection{Planck dimensions}
	It is time now to open a small parenthesis to finish about the Planck's constant that we promised in the section Principles of the chapter Mechanics to indeep during our study of Wave Quantum Physics (because many books mention what we will see without the precautions of rigor in our point of view). We have just seen that the measurement of objects depends on the Heisenberg incertitude principles. This precision affects both the time measurements and the particle trajectory or the energy density of the Universe. Let's see that by extension ... there are other possible implications.

	We have proved previously at the beginning of this section that one of the uncertainties relations is given, by taking the module, by (from the order of Planck's constant to one given factor):
	
	So, we can say roughly that at a fluctuation $\Delta x$ of space, we can associate the linear momentum:
	
	To that latter, according to our results of the section of Special Relativity, corresponds the relation of the energy $E=Mc^2=pc$, or the equivalent mass (by dividing by $c^2$) $p/c$. By denoting by $M$ this mass associated with the perturbation $\Delta x$, we thus have:
	
	The gravitation due to this mass is characterized by a length $R$ which we will determine in order of magnitude by writing that the potential energy associated with it (this assumes that Classical and Quantum gravitation are governed by the same laws...) $GM^2/R$ (\SeeChapter{see section Astronomy page \pageref{newton gravitational law}}), is equal to the mass-energy $Mc^2$. This gives:
	
	or, by replacing $M$ by its previous expression:
	
	So that there is no self-amplification (and thus divergence) of the phenomenon of vacuum quantum fluctuation, we must preferably have $R=\Delta x$. By writing the equality between these two quantities, we arrive at a quantity which represents the minimum dimension (in order of magnitude) that physics can conceive according to the interpretation of some physicists. This is the famous "\NewTerm{Planck's length}\index{Planck's length}" where we replace $R$ by $l_p$:
	
	For which it corresponds the period or "\NewTerm{Planck time}\index{Planck time}" $t_P=l_p/c$ hence:
	
	We can now return to another more interesting expression of the fluctuating mass. Since:
	
	We have therefore after rearranging the "\NewTerm{Planck mass}\index{Planck's mass}":
	
	Another way to get the Planck's mass but that gives a slightly different result is first to suppose that the main three universal constant are:
	
	and that wee seek to write a mass by combining $h$, $c$ and $G$. So this is a dimensional analysis as:
	
	where we need to found the values of $\alpha$, $\beta$ and $\gamma$ to get a mass...
	
	We see quickly that to for the mass (kg) we must have:
	
	for the distance (m), we must have:
	
	and for the time (s):
	
	So we just need to solve this linear system:
	
	So first iteration:
	
	Second iteration:
	
	Third iteration:
	
	so finally:
	
	Therefore:
	
	to a given multiplicative dimensionless constant... This missing multiplicative constant is known to us as we have determined the Planck's mass earlier before. So with dimensional analysis we should consider $\hbar$ as being more fundamental than $h$.
	
	Dimensional analysis gives to a given constant factor and according to the virial theorem (\SeeChapter{see section Continuum Mechanics page \pageref{virial theorem}}):
	
	and therefore:
	
	Hence the "\NewTerm{Planck's temperature}\index{Planck's temperature}":
	
	and "\NewTerm{Planck's energy}\index{Planck's energy}":
	
	After all this, we easily get the "\NewTerm{Planck's density}\index{Planck's density}":
	
	That is:
	
	We can have fun getting other Planck values, but that have no physical experimental justification excepted for most of them invalid empirical and subjective interpretations (and we could go on for a long time with so many other magnitudes as we have give a quite exhaustive list in the section Principles of the Mechanics chapter). For example:
	\begin{itemize}
		\item The "\NewTerm{Planck's force}\index{Planck's force}":
		
		Therefore:
		
	
		\item The "\NewTerm{Planck's power}\index{Planck's power}":
		
	
		\item The "\NewTerm{Planck's pulsation}\index{Planck's pulsation}":
		
		
		\item By proceeding with the same initial reasoning that the one done with mass, but using the potential electrostatic energy instead of gravitational potential energy we can get the "\NewTerm{Planck's electric charge}\index{Planck's electric charge}":
		

		\item Then we can calculate a "\NewTerm{Planck's current}\index{Planck's current}":
		
		
		\item And also the "\NewTerm{Planck's voltage}\index{Planck's voltage}":
		
		
		\item ...and the "\NewTerm{Planck's impedance}\index{Planck's impedance}" (...):
		
	
		\item and so on\footnote{see section Principia of the chapter Mechanics page \pageref{planck constants} to see a most exhaustive and structured list}...
	\end{itemize}
	\begin{tcolorbox}[title=Remark,colframe=black,arc=10pt]
	Some physicists have used (and still use) the above results for wacky and dangerous reasonings that are only hazardous interpretions! It is therefore advisable to take with care all the information concerning the Planck dimensions  that you could find in textbooks and Interent (even if these seem very sympathetic ). The best known example is given by the Compton wavelength $\lambda_C$ (\SeeChapter{see section Nuclear Physics page \pageref{compton wavelength}}) which depends on the mass-energy of the photon. If this wavelength is equal to the classical Schwarzschild radius for the same mass-energy (\SeeChapter{see section General Relativity page \pageref{schwarzschild radius}}), then in this case its value is that of the Planck's length and its mass is equal to the Planck's mass. It is then tempting to say that the particle then forms a Black Hole. But this is an analogy because in this case, nothing tells us that the expression of the Schwarzschild radius applies to quantum physics... !
	\end{tcolorbox}
	
	\pagebreak
	\subsection{Wave Quantum Physics Interpretation}
	Ok now that we have some mathematical basics on Quantum Physics, before we continue with Relativistic Quantum Physics and Quantum Field theory we think that it is the good time to speak about the main existing interpretations of the results and mechanisms of the different models and results we get so far\footnote{For some physicists, interpret mathematics is a non-sense ans anyway human brain is limited. But as many people like to interpret stuff we still dedicate a subsection about this}. Indeed, we have seen that Quantum mechanics cannot, for most people, easily be reconciled with everyday language and observation. Its interpretation has often seemed counter-intuitive to physicists, including some of its inventors.
	
	\subsubsection{Copenhagen interpretation}
	In 1930, the probabilistic interpretation of the wave amplitude of a particle and the Heisenberg uncertainty principle constitute the elements of the "non-deterministic" standard interpretation of quantum physics as we have already mention it at the beginning of this section. This interpretation is often referred to as the "\NewTerm{Copenhagen interpretation}\index{Copenhagen interpretation}", as Niels Bohr who contributed largely to it, headed a renowned institute of physics at that time in this city. However many physicists, such as Albert Einstein and Erwin Schrödinger, who accepted the mathematical formulation of quantum physics, were not at ease with the Copenhagen interpretation of and criticized it. And up to the present day (still in year 2017), the question of the correct interpretation of mathematical formulation remains a problem.
	
	Indeed, we can ask ourselves the following question: Where is the reality? Is there a reality? Niels Bohr answers "\textit{No}": there is nothing at the quantum level, reality exists or appears only during a measurement. This view shared by most physicists (Copenhagen interpretation) implies that the measure "creates" the position of the electron. In other words: No elementary phenomenon is a real phenomenon before being an observed phenomenon!
	
	Albert Einstein thought that quantum physics, although very effective and very impressive, is not complete and gives only an imperfect picture of the quantum world. For him, there would be something else beyond that which would clarify and refine our present vision (in the same way as the theory of gases for which we had to wait for statistical models, Albert Einstein thought that hidden variables had yet to be discovered).
	
	Thus, in the Copenhagen interpretation of quantum mechanics, the principle of uncertainty means that, at an elementary level, the physical universe no longer exists in a deterministic way, but rather as a series of probabilities or potentials. For example, the pattern produced by millions of photons passing through a diffraction slot can be calculated using quantum mechanics, but the path of each photon can not be predicted by any known method following the Copenhagen interpretation. It is this interpretation that Albert Einstein doubted when he said: "\textit{I can not believe that God is playing dice with the Universe}". From a physical as well as a philosophical point of view, the principle of uncertainty implies the refutation of the universal determinism defended by Laplace at the beginning of the 19th century.
	
	An instantaneous reduction of all the possible states occurs from the observation of the system according to the Copenhagen  interpretation (wave function collapse). This random decision of the observed state respects the probabilities, corresponding to the square of the amplitudes of the states as we have seen it. In addition, the Copenhagen interpretation states that during a measurement, a reduction process, originating from the macroscopic object, eliminates superposition of quantum states.
	
	The Copenhagen interpretation therefore leads to the problem of measurement, since the Schrödinger cat's thought experiment states that when we measure a quantity, such as position or momentum, we intervene in the process of measurement by causing a radical change of the quantum state of the wave function. We change the quantities measured in an unpredictable manner and this (transition) state can not be described by the Schrödinger equation. Physicists and philosophers have reacted in several ways to this interpretation:
	\begin{itemize}
		\item Either we consider as Niels Bohr and Werner Heisenberg that this principle is law and that it is better not to seek the ultimate interpretation as our brain capacity is anyway limited in comparison to the powerful investigation tool that are mathematics. This attitude is accepted by most physicists.
		
		\item Either we consider quantum physics to be an incomplete theory and some, such as Albert Einstein, Eugene Wigner or David Bohm (de Broglie-Bohm pilot-wave interpretation that we will introduce very briefly further below), have not hesitated to seek other solutions, that seems sterile until now.
		
		\item Finally, Hugh Everett and many others take Schrödinger's equation very seriously, considering it a representation of reality. They consider that the  Copenhagen interpretation really represents the evolution of the wave function. The different terms of the equation would correspond to the different levels of energy in which the system is located. The reduction of the wave packet would be interpreted as a total division of the object and the measuring instrument into parallel universes (many-worlds interpretation).
		\begin{figure}[H]
			\centering
			\includegraphics[scale=0.75]{img/atomistic/many_world_interpretation.jpg}
			\caption[]{Schrödinger's cat paradox is resolved by assuming that different branches of the Universe\\ are created with each possible outcome (source: Wikipedia, author: Christian Schirm}
		\end{figure}
	\end{itemize}
	
	Today the debate remains open, but several experiments carried out since the 1930s allow us, step by step, to dispel the thick fog that covers the bottom of reality and answer a few questions. However, all these experiments confirm that the period of certainties is long gone. The most famous experiment remaining the "\NewTerm{EPR paradox}\index{v}" following the publication of an article by Einstein, Podolsky and Rosen whose sole purpose is to undermine the Copenhagen interpretation.
	
	\paragraph{EPR paradox}\mbox{}\\\\\
	The original article being a bit difficult (in a future version of that book we will perhaps write the underlying mathemtics in details), we will take the simplified version of use in small schools but that is used in laboratories, originally proposed by David Bohm. Whereas originally the paradox was presented with the pair (position, linear momentum), Bohm proposed to use the fermio spin which is a purely binary quantum property much more easy to deal with.
	
	As an electron can have only two states of spin "top" $|+\rangle$ or "bottom" $|-\rangle$ (in the famous $z$ orient magnetic field experiment), so the EPR experiment proposed by David Bohm consists in taking a null spin particle which disintegrates, thus producing two electrons $A$ and $B$ each being in the superposition of states $|-+\rangle$ or $|+-\rangle$. Since their combined spin must remain zero, one of the electrons must have its spin at the top and the other at the bottom when their are measured. The electrons are the driven in opposite directions until the distance separating them is large enough to eliminate any physical interaction between them, and the spin of each electron is measured exactly at the same instant using a spin detector.
	
	According to Niels Bohr, as long as no measurements have been made, neither electron $A$ nor electron $B$ possess a spin pre-existing in any direction. Instead, before being observed, electrons exist in a superposition of states, so that they are up and down at the same time. Since the two electrons are entangled, the information concerning the state of their spin is given by a wave function of the type:
	
	The electron $A$ has no determined spin component before a measurement is performed to determine it collapses the wave function of the system $A$ and $B$, after which it is either "top" or "bottom". At that moment, its partner $B$ acquires the opposite spin in the same direction, even though it is at the other side of the Universe violating therefore Special Relativity postulates (no information can go faster than the speed of light $c$). The Copenhagen interpretation is then named "\NewTerm{non-local interpretation}\index{non-local interpretation}" whereas Albert Einstein believed in local realism: that is, a particle can not be \underline{instantly} influenced by a distant event, and its properties exist independently of any measure.
	
	However, Bohm's approach has a flaw that Albert Einstein would probably have used as an argument: correlations could be explained by arguing that the two electrons each have spin values defined on each of the three axes $x$, $y$, $z$. Whether measured or not. So again, according to Albert Einstein, the fact that the pre-existing spin states of the electron couple can not be taken into account by wave quantum physics would still have been a proof of its incompleteness (hidden variables).
	
	A physicist, John Bell, however, had the idea of an experimental and theoretical means to break the impasse of the EPR paradox by changing the relative orientation of the two spin detectors.

	Thus, if the detectors measuring the electron spin $A$ and $B$ are aligned so as to be parallel, then there is a $100\%$ correlation between the two sets of measurements each time the spin is measured at "top" by the first detector, the "bottom" spin is measured by the other detector, and vice versa. If one of the detectors is turned slightly, they are no longer aligned. Now, if we measure the spin state of many pairs of entangled electrons, when we find "top" for electron $A$, the corresponding measure for $B$ will sometimes give "top" too. Increasing the angle between the axes of the two detectors therefore leads to a reduction in the degree of correlation. If the detectors are at right angles to one another and the experiment is repeated a large number of times, it is only in half of the cases that a spin is detected at the "bottom" for $B$ when we detect a "top" spin for $A$ following the $z$-axis. If the detectors are oriented $180$ degrees apart, the electron couple will be completely anticorrelated. If the measurement gives "top" for the spin state of $A$, then the spin state of $B$ will be "bottom".
	
	Although it is an imaginary experiment, it is possible to calculate the exact degree of spin correlation for a given orientation of the detectors, as predicted by quantum theory. However, it is not possible to perform a similar calculation by using an archetypal hidden variable theory and conserving the locality. The only thing that such a theory could predict would be an imperfect coupling between the spin states of $A$ and $B$. However, rigorously, it is insufficient to choose between quantum theory and a local theory with hidden variables. John Bell made an astonishing discovery. It was possible to decide between the predictions of quantum mechanics and those of any hidden-variable theory by measuring the correlations of electron pairs for a given configuration of the detectors and then repeating the experiment with a different orientation. This allowed John Bell to calculate the total correlation for the two orientation configurations in terms of individual results predicted by any local theory with hidden variables. Since, in any such theory, the result of a measurement performed by a detector can not be affected by what is measured with the other, it is possible to distinguish between hidden variables and quantum mechanics.
	
	John Bell succeeded in calculating the limits of the degree of spin correlation between pairs of entangled electrons in a Bohm-modified EPR experiment (if we have the time we will provide the mathematical details in the future). He found that in the ethereal kingdom of quanta there is a greater degree of correlation if quantum mechanics reigns as an absolute rule than in any universe that depends on hidden variables and locality. Bell's theorem stated that no local theory with hidden variables could reproduce the same set of correlations as quantum mechanics. Any local theory with hidden variables would lead to spin correlations generating numbers, name obviously "correlation coefficients", between $-2$ and $+2$. However, for some orientations of spin detectors, quantum mechanics predicts correlation coefficients that lie outside the range, named "\NewTerm{Bell inequalities}\index{Bell inequalities}", ranging from $-2$ to $+2$.

	The Bell theorem thus makes it possible to test face to face the local reality advocated by Albert Einstein with Niel Bohr Copenhagen interpretation, namely that the quantum universe exists independently of observation and that physical effects can not be transmitted to a speed higher than that of light. Bell had carried the Einstein-Bohr debate into a new arena: experimental philosophy. If Bell's inequality resisted, then Einstein's assertion that quantum mechanics was incomplete would be accurate. If this inequality were to be violated, Bohr would triumph. No more thought experiences! It would now be Einstein vs. Bohr in the lab.
	
	The first experiment that tested the Bell's inequalities used pairs of photons instead of pairs of electrons. This change was possible because the photons possess the polarization property, which for the needs of the test played the role of the quantum spin (moreover, the photons are easier to manipulate). This is certainly a simplification, but a photon can be considered to be polarized as "top" or "bottom". Like the spin of the electron, if the polarization of one of the photons on the $z$-axis is measured as "top", then the measurement of the other will give "bottom", since the combined polarizations Of the two photons must result in a zero spin (polarization state).
	
	The results violated Bell's inequalities, which was in favor of Bohr's non-local Copenhagen  interpretation and against the local reality supported by Albert Einstein.
	
	Bell derived this inequality from two assumptions (hypothesis):
	\begin{enumerate}
		\item[H1.] First, there is a reality independent of the observer. This is reflected in the fact that a particle has a well-defined property such as spin before it is measured. 

		\item[H2.] Second, the locality is conserved. There is no supra-luminous influence, so that what happens here can not instantly affect what happens elsewhere. 
	\end{enumerate}
	The experimental results mean that one of these two assumptions must be abandoned, but which one? Bell was willing to abandon the locality.
	
	\subsubsection{De Broglie-Bohm (pilot-wave) interpretation}
	The "\NewTerm{de Broglie–Bohm theory}\index{de Broglie–Bohm theory}", also known as the "\NewTerm{pilot-wave theory}\index{pilot-wave theory}" or "\NewTerm{Bohm's interpretation}\index{Bohm's interpretation}" is another interpretation of quantum theory. The theory is deterministic[1] and explicitly nonlocal: the velocity of any one particle depends on the value of the guiding equation, which depends on the configuration of the system given by its wavefunction; the latter depends on the boundary conditions of the system, which in principle may be the entire universe...

	The theory was historically developed in the 1920s by de Broglie, who in 1927 was persuaded to abandon it in favour of the then-mainstream Copenhagen interpretation. David Bohm, dissatisfied with the prevailing orthodoxy, rediscovered de Broglie's pilot-wave theory in 1952. Bohm's suggestions were not widely received then, partly due to reasons unrelated to their content, connected to Bohm's youthful communist affiliations. De Broglie–Bohm theory was widely deemed unacceptable by mainstream theorists, mostly because of its explicit non-locality. Bell's theorem (1964) was inspired by Bell's discovery of the work of David Bohm and his subsequent wondering whether the obvious nonlocality of the theory could be eliminated. Since the 1990s, there has been renewed interest in formulating extensions to de Broglie–Bohm theory, attempting to reconcile it with special relativity and quantum field theory, besides other features such as spin or curved spatial geometries.
	\begin{figure}[H]
		\centering
		\includegraphics[scale=0.46]{img/atomistic/pilot_wave.jpg}
	\end{figure}
	We strongly recommend the reader to go see on Internet videos about the "walking droplets\footnote{hydrodynamic quantum analog involving bouncing fluid droplets over a vibrating fluid bath}" also named "walker" to understand better the idea behind the concept of pilot wave. 
	
	 In the below gallery (read from left to right and up to bottom) thanks to a little layer of air, droplets of silicone oil can continuously bounce on the surface of the pool beneath them without being absorbed. But even though the droplets don't directly connect with the pool beneath them, they still make waves, and then interact with those waves, resulting in a forward momentum.
	 \begin{figure}[H]
		\centering
		\includegraphics[scale=0.46]{img/atomistic/bouncing_droplet_forward.jpg}
		\caption[]{Thirty successive pictures of a walker}
	\end{figure}
	All observed phenomena are related to the particle-wave interaction. The droplet moves in a medium modified by previously generated waves.As the droplet moves, the points of the interface it visits keep emitting waves.The wavefield has thus a complex structure that contains a "memory"of the path.
	
	While this experiment isn't on the quantum scale, it does help to demonstrate the way quantum-scale particles may operate according to the pilot wave theory. And for any lay people who have struggled with grasping why things are so strange on the quantum scale according to the standard interpretation, this pilot wave theory provides a far more palatable framework for understanding quantum mechanics.
	
	And what about walkers facing a double split? Here you are (still read from left to right and up to bottom):
	\begin{figure}[H]
		\centering
		\includegraphics[scale=0.46]{img/atomistic/bouncing_droplet_forward_double_split.jpg}
		\caption[]{Double split experiment with two walkers}
	\end{figure}
	The diffraction pattern obtained with the walker experiment is highly similar tot that of the diffraction pattern of photon or electron with the same experimental configuration.
	
	If we have the time in the future we will perhaps write a formal introduction (ie mathematical introduction) to this theory.
	
	
	
	\begin{flushright}
	\begin{tabular}{l c}
	\circled{90} & \pbox{20cm}{\score{4}{5} \\ {\tiny 61 votes,  74.43\%}} 
	\end{tabular} 
	\end{flushright}

	%to make section start on odd page
	\newpage
	\thispagestyle{empty}
	\mbox{}
   	\section{Relativistic Quantum Physics}\label{relativistic quantum physics}
   	\lettrine[lines=4]{\color{BrickRed}T}he attentive reader will have noticed that quantum mechanics (wave quantum physics) is a non-relativistic theory: it does not incorporate the principles of Einstein's Special Relativity (\SeeChapter{see section Relativity page \pageref{special relativity assumptions and principles}}). We will strive to fill this gap by studying now the relativistic version of the wave quantum physics.
   	
   	 Relativistic quantum mechanics (RQM) is any Poincaré covariant formulation of quantum mechanics. This theory is applicable to massive particles propagating at all velocities up to those comparable to the speed of light $c$, and can accommodate massless particles. The theory has application in high energy physics,particle physics and accelerator physics, as well as atomic physics, chemistry and condensed matter physics. Non-relativistic quantum mechanics refers to the mathematical formulation of quantum mechanics applied in the context of Galilean relativity, more specifically quantizing the equations of classical mechanics by replacing dynamical variables by operators. Relativistic quantum mechanics is quantum mechanics applied with Special Relativity, but not General Relativity. An attempt to incorporate General Relativity into quantum theory is the subject of quantum gravity, yet an unsolved problem in physics, although some theories, such as the Kaluza-Klein, have been proposed but are unfounded and without proof. Although the earlier formulations, like the Schrödinger picture and Heisenberg picture were originally formulated in a non-relativistic background, these pictures of quantum mechanics also apply with Special Relativity.

	The relativistic formulation is more successful than the original quantum mechanics in some contexts, in particular: the prediction of antimatter, electron spin, spin magnetic moments of elementary spin-$1/2$ particles, fine structure, and quantum dynamics of charged particles in electromagnetic fields. The key result is the Dirac equation, from which these predictions emerge automatically. By contrast, in quantum mechanics, terms have to be introduced artificially into the Hamiltonian operator to achieve agreement with experimental observations.

	Nevertheless, RQM is only an approximation to a fully self-consistent relativistic theory of known particle interactions because it does not describe cases where the number of particles changes; for example in matter creation and annihilation. By yet another theoretical advance, a more accurate theory that allows for these occurrences and other predictions is relativistic quantum field theory in which particles are interpreted as field quanta (see article for details).
   	
   	Before we tackle the mathematical part, we emphasize and reiterate that we will limit ourselves only to theoretical developments made between 1910 and about 1935 (beyond the complexity of theories requires too many pages to a general book like this one).
   
   	
   	\pagebreak
	\subsection{Relativistic Schrödinger evolution equation}
	Particle physics can not be properly and completely described in the context of quantum mechanics. As the energies are generally superior to the masses of the particles, it is necessary, in addition, to work in the context of the theory of relativity. Let's see how to include it in a first basic approach.
	
	The energy-momentum of a free particle of mass $m$, satisfied as we have proved in the section of Special Relativity the relation:
	
	We seek to quantify this equation. For this, we will return to the relations we have proved in the study of functional linear operators and of the evolutionary Schrödinger equation.
	
	Recall that the momentum is described by the relation (using the divergence operator):
	
	and the total energy by:
	
	Both relations have been proved in details in the section of Wave Quantum Physics!
	
	The substitutions of the two previous relations applied to the equality:
	
	leads to:
	
	Applying the squares:
	
	and rearranging and using the notation $\vec{\nabla}^2= \Delta$:
	
	And multiplied by the wave function $\Psi$ on both sides of equality lead to development:
	
	Hence:
	
	And rearranged:
	
	Using the d'Alembertian (\SeeChapter{see section Electrodynamics page \pageref{alembertian}}), we can write this latter relation in a more condensed form named "\NewTerm{relativistic Schrödinger evolution equation}\index{relativistic Schrödinger evolution equation}" or more frequently "\NewTerm{free Klein-Gordon equation}\index{free Klein-Gordon equation}\label{free Klein-Gordon equation}\index{free Klein-Gordon equation}" (in the absence of a magnetic field!):
	
	with the signature $(+, -, -, -)$ for the partial derivative:
	
	Some authors prefer the signature  $(-, +, +, +)$ therefore we have (this is also the choices we will make in the section of Quantum Field Theory as it is the tradition...):
	
	with:
	
	\begin{tcolorbox}[title=Remark,colframe=black,arc=10pt]
	In particle physics, this equation is named "\NewTerm{bosons relativistic covariant equation}\index{bosons relativistic covariant equation}" whatever the chosen signature.
	\end{tcolorbox}
	The free Klein-Gordon equation is often given as follows (more aesthetic):
	
	It is important to note that the Klein-Gordon equation involves scalar and therefore characterizes zero spin particles.
	\begin{tcolorbox}[title=Remarks,colframe=black,arc=10pt]
	\textbf{R1.} We can verify that the plane waves of the form:
	
	are solutions of the free Klein-Gordon equation (we'll go more into details in the section Elementary Particles Physics).\\
	
	\textbf{R2.} We will come back during our study of the Dirac equation and fermions spins on the free Klein-Gordon equation (to generalize it).
	\end{tcolorbox}
	
	\subsubsection{Antimatter}
	When we have proved the free Klein-Gordon equation, we purposely left aside a very interesting case of development that we have made.
	
	Maybe you did not notice it, but the equation:
	
	can take two values for a given pulsation:
	
	one positive and the other negative. The value of energy could therefore take all values in $[-\infty,+\infty]$...!!!
	
	So far we have implicitly assumed in Classical Mechanics that negative solutions were not natural and should therefore simply be discarded. This cannot be done in quantified theory without lead to serious inconsistencies. Rather than ignore these negative energy solutions, we should find them a physical interpretation.
	
	We notice first that all the negative energies are authorized by the above equation (as well as for positive energy). We say then that negative energy states are occupied but not observable; the electrons are named "\NewTerm{virtual electrons}\index{virtual electrons}".
	   	\begin{theorem}
	Let us imagine a wave packet formed by a superposition of plane waves at a narrow interval in pulse. This packet moves in space. In the one-dimensional case, it propagates at the speed:
	
	\end{theorem}
	\begin{dem}
	Always based on the assumption that the potential field is zero, we have:
	
	and:
	
	So to resume:
	
	\begin{flushright}
		$\square$  Q.E.D.
	\end{flushright}
	\end{dem}
	Let us consider first a particle of positive energy $E^+$. Its position in function of time is given by:
	
	A particle of negative energy $E^-$ moves following:
	
	In other words, and this will be our first conclusion, we can say that a particle of negative energy $E^-<0$ is equivalent to a particle of positive energy $|E^-|>0$ moving backwards in time and this is what we name an "\NewTerm{antiparticle}\index{antiparticle}".
	
	Now we have to see what is the interpretation of a particle moving backwards in time:
	
	To simplify, we consider a non-relativistic particle of electric charge ($-q$) immersed in a static electric field $\vec{E}$ and a static magnetic field $\vec{B}$. It satisfies to the equation of motion:
	
	We have studied in the section of Electrodynamics that the fields $\vec{E}$ and $\vec{B}$ could be constructed from a quadripotential $\vec{A}$. So we can rewrite the above equation from the two relation proven in the section of Electrodynamics:
	
	However, it is always possible to impose the following gauge (we leave it to do reader to do the check by usingexactly the same methodology as that used in the section of Electrodynamics):
	
	The equation of motion becomes:
	
	or also:
	
	Comparing the last two equations, we come to our second conclusion: a particle of charge $q$ moving backwards in time obeys to the same equations of  motion that an oppositely charged particle $-q$ moving forward in the time. The physical interpretation of the second particle is obvious ("\NewTerm{Feynman-Stückelberg interpretation}\index{Feynman-Stückelberg interpretation}").

	Relativistic quantum physics implies the existence of anti particles, which are actually observed in laboratories (accleratoris and not only...!).

	All this to get where exactly?
	\begin{enumerate}
		\item First, the theoretical discovery of antimatter provides a possible explanation for the existence of the Universe which previously violated the principle of conservation of energy. The theory we have just seen, therefore predicted that our Universe should contain as much matter as antimatter. Scientists are still looking for the presence of this antimatter.

		\item Secondly, if we consider in vacuum a photon of energy $h\nu>2m_ec^2$, it is capable of carrying a virtual electron to a state of positive energy, where it becomes real. It appears then locally a gap, or "hole" in the region of negative energies. According to the principle of charge conservation, one sees a positive electron, or positron, symmetrical and antimatter equivalent particle of the electron.
	\end{enumerate}
	Thus, the photon materializes in the form of a pair $e^+,e^{-}$, with:
	
	\begin{tcolorbox}[title=Remark,colframe=black,arc=10pt]
	Some experimental results suggest that antiparticles are not the perfect mirrors of the particles we know. Indeed, the right/left and time symmetries does not appear to be respected (there is symmetry breaking). We do not have anything written about in this book at this date but we will do so as soon as we can.
	\end{tcolorbox}
	
	\subsection{Generalized Klein-Gordon Equation}
	The free Klein-Gordon equation that we initially presented and proved earlier does not take into account the influence of magnetic field on the observation of the splitting of spectral lines of atoms (experimental observation!). That's why Klein and Gordon have integrated it into their equation the magnetic field. However, they did not take into account the spin of the electron. Only after their work Pauli developed his equation (named "Pauli equation") which then led to the Dirac equation (see further below).

	To determine the expression of the Klein-Gordon generalized equation of a charged particle in a magnetic field and an electrostatic potential, we will use the power of the Lagrangian formalism!

	The classical equation of motion accepted (\SeeChapter{see section Analytical Mechanics page \pageref{equations of movement}}), as valid also in Special Relativity, is given as we proved it by (Euler-Lagrange equation):
	
	In the section of Special Relativity, we have proved that the Lagrangian of a free particle is expressed as:
	
	with for recall:
	
	and in the section of Electrodynamics that the total Lagrangian was given by:
	
	For future needs, start by calculating:
	Let us calcualte the first term:
	
	As the potential does not dependent on the speed, the term:
	
	is equal zero.

	The vector potential does not depend on the velocity of the particle also, then:
	
	Therefore it comes:
	
	The classical Hamiltonian is written (\SeeChapter{see sectioin Analytical Mechanics}):
	
	So we have previously proved that:
	
	We can write the Hamiltonian with this relation in the form:
	
	The dot product $\vec{v}\circ\vec{p}$ has for expressed (since $\vec{p},\vec{v}$ are collinear)
	
	The Hamiltonian is therefore written:
	
	By working on the two first terms:
	
	But:
	
	Therefore:
	
	Finally, we get (for a conservative system):
	
	Always in the case of a particle moving in an electromagnetic field, the relation between the energy and impulsion (which is different from that of momentum by the presence of a term containing the vector potential) is calculated as follows:

	As (we have just prove it):
	
	and as we have prove it in the section of Wave Quantum Physics:
	
	where it is necessary to see the last term as an abusive notation for the square of the norm.
	
	Then by substituting into:
	
	we get (we change of notation for the Hamiltonian):
	
	If we rewrite this relation by making use of the corresponding operators (\SeeChapter{see section of Wave Quantum Physics page \pageref{observables and operators}}) of energy and momentum (canonical quantization):
	
	then finally we can write in analogy with the free Klein-Gordon equation (in the absence of field) the "\NewTerm{generalized Klein-Gordon equation}\index{generalized Klein-Gordon equation}":
	
	This equation is that of Klein-Gordon that applies to a particle of charge $q$ without  spin moving in an electromagnetic field.

	If $U=0,\vec{A}=\vec{0}$ the previous relation simplifies in multiple stages:
	
	We thus fall back on the Klein-Gordon equation for a free particle without spin!
	
	\subsubsection{Generalized Klein-Gordon Equation continuity equation}
	It would be interesting now to look at the expression of the continuity equation (let us recall that it expresses the conservation of energy!!) with the inclusion of the magnetic field (because in fact the continuity equation still makes problem... and even a very big problem!!!). For this, consider the case of a free particle moving with a momentum $\vec{p}$ and having an energy $E$. We saw that we could associate to it a plane wave of the form:
	
	Given the free Klein-Gordon equation and its complex conjugate (we work with naturellesunits):
	\begin{subequations}
	\label{equations}
	\begin{align}
	  \label{eq:a}
	  &-\partial_t^2\Psi+\Delta \Psi-m_0^2\Psi=0 \\
	  \label{eq:b}
	  &-\partial_t^2\bar{\Psi}+\Delta \bar{\Psi}-m_0^2\bar{\Psi}=0
	\end{align}
	\end{subequations}
	We multiply (a) by $-\mathrm{i}\bar{\Psi}$ and (b) by $-\mathrm{i}\Psi$:
	\begin{subequations}
	\label{equations}
	\begin{align}
	  \label{eq:a}
	  &-\mathrm{i}\bar{\Psi}\left(-\partial_t^2\Psi\right)-\mathrm{i}\bar{\Psi}\Delta \Psi+\mathrm{i}\bar{\Psi}m_0^2\Psi=0 \\
	  \label{eq:b}
	  &-\mathrm{i}\Psi\left(-\partial_t^2\bar{\Psi}\right)-\mathrm{i}\Psi\Delta \bar{\Psi}+\mathrm{i}\Psi m_0^2\bar{\Psi}=0
	\end{align}
	\end{subequations}
	Therefore:
	\begin{subequations}
	\label{equations}
	\begin{align}
	  \label{eq:a}
	  &\mathrm{i}\bar{\Psi}\partial_t^2\Psi-\mathrm{i}\bar{\Psi}\Delta \Psi+\mathrm{i}\bar{\Psi}m_0^2\Psi=0 \\
	  \label{eq:b}
	  &\mathrm{i}\Psi\partial_t^2\bar{\Psi}-\mathrm{i}\Psi\Delta \bar{\Psi}+\mathrm{i}\Psi m_0^2\bar{\Psi}=0
	\end{align}
	\end{subequations}
	By difference (a)-(b):
	
	Computing the derivatives with respect to $t$ of the following functions:
	\begin{subequations}
	\label{equations}
	\begin{align}
	  \label{eq:a}
	  &\partial_t(\bar{\Psi}\partial_t\Psi)=\partial_t\bar{\Psi}\partial_t\Psi+\bar{\Psi}\partial_t^2\Psi \\
	  \label{eq:b}
	  &\partial_t(\Psi\partial_t\bar{\Psi})=\partial_t\Psi\partial_t\bar{\Psi}+\Psi\partial_t^2\bar{\Psi}
	\end{align}
	\end{subequations}
	By difference (a)-(b):
	
	Which gives us finally:
	
	Let $f$ be a scalar field and $\vec{g}$ and a vector field. The vector analysis gives:
	
	Let us put:
	
	Therefore:
	
	Let us put now:
	
	Therefore:
	
	So we have finally two relations:
	\begin{subequations}
	\label{equations}
	\begin{align}
	  \label{eq:a}
	  &\vec{\nabla}\circ(f\cdot \vec{g})=\vec{\nabla}\circ(f\cdot \vec{g})-(\vec{\nabla} f)\circ \vec{g}\Rightarrow \Psi(\vec{\nabla}\circ\vec{\nabla}\bar{\Psi})=\vec{\nabla}\circ(\Psi\vec{\nabla}\bar{\Psi})-(\vec{\nabla}\Psi)\circ\vec{\nabla}\bar{\Psi}\\
	  \label{eq:b}
	  &f(\vec{\nabla}\circ\vec{g})=\vec{\nabla}\circ(f\cdot \vec{g})-(\vec{\nabla})\circ\vec{g}\Rightarrow \bar{\Psi}(\vec{\nabla}\circ\vec{\nabla}\Psi)=\vec{\nabla}\circ(\bar{\Psi}\vec{\nabla}\Psi)-(\vec{\nabla}\bar{\Psi})\circ\vec{\nabla}\Psi
	\end{align}
	\end{subequations}
	We subtract (a)-(b) to get:
	
	As $\vec{\nabla}\circ\vec{\nabla}=\vec{\nabla}^2=\Delta$:
	
	By changing the signs:
	
	This latter relation and:
	
	gives together:
	
	Again, let us compare this relation with the continuity equation:
	
	Let us recall that during our first study of the Klein-Gordon equation we have seen that in quantum physics equivalent is given by the same equation but with the following meanings: $\rho$ is the probability density, $\vec{j}$ is the flux density of particles.
	We then have:
	
	If the associated wave function $\Psi$ and its conjugate complex $\bar{\Psi}$ are given in a special case by:
	
	The derivatives with respect to time of these functions are obviously:
	
	And their gradients are calculated as follows:
	
	By taking again the expression of the probability density and given previous differential, we get:
	
	The density of probability has therefore for expression:
	
	Taking again the expression of current density and considering the differential, we get:
	
	The current density has therefore for expression:
	
	Putting ourselves in the situation of that time of knowledge of quantum physics, the Klein-Gordon equation has several pathologies and disadvantages:
	\begin{itemize}
		\item The probability density $\rho=2EA^2$ may become negative (since as we have seen, the energy can also be negative), which is inexplicable. Such a situation does not exist with the Schrödinger equation.
		
		\item The Klein-Gordon equation has the disadvantage of being of the second order in $t$ (the Schrödinger equation is of the first order). The temporal evolution requires therefore the knowledge of not only of $\Psi(t_0)$ but also of its derivative $(\partial_t \Psi)_{t=t}$

		\item If we applied this equation to the hydrogen atom, we would not find the same fine structure\footnote{"Fine-structure" refers to splitting of levels with the same $l$ but different $j$} energy levels as observed.		
	\end{itemize}
	All this led at the time preceding the works of Dirac, to reject this equation that, in addition, did not include the spin.

	\pagebreak	
	\subsection{Classical free Dirac equation}\label{classical free dirac equation}
	So far, any particle was regarded as punctual and without any structure or internal degree freedom. In this context, all the information on the system state at time $t$ is then reputated as entirerly contained in the wave function $\Psi(x,y,z,t)$.

	However, such a description is not sufficient, as we shall see. This insufficience comes from the experimental evidence demonstrating that a particle such as an electron or also neutron has a magnetic intrinsec moment, independently of any rotation in space around a center. The existence of this magnetic moment in turn leads to the existence of an intrinsic angular momentum which was named "\NewTerm{spin}\index{spin}" because it was believed at first that this degree of freedom was linked to a rotation of the particle on itself. This degree of freedom is "internal" - although the electron continues to be regarded as a point particle; that is, along with the electric charge or mass, an intrinsic attribute, given once and for all. It seems so far impossible to physicist to give to the spin a classic analogy! Imagine the electron as a small non-zero radius ball that turns on itself leads to absurdities (for example we found that a point located on the surface of the electron has a much higher speed than $c$). It remains that the spin of a massic particle is its angular momentum in the repository where it is at rest. It is commonly accepted that hypothesis of the electron spin was formulated first by George Uhlenbeck and Samuel Goudsmit in 1925.

	The spin of a particle is always half full or unitary, it is a fact of experience until now. The half or unitary characteristic nature of the spin defines two broad families of particles:
	\begin{itemize}
		\item bosons (integer spin)
		
		\item fermions (half-integer spin)
	\end{itemize}  	
	obeying very different statistics such as those we have discussed and study in the section os Statistical Mechanics (hence the existence of a relation named "spin-statistics theorem").

	Let us return to the case of the electron. The two possible values revealed by measuring $S$ (the $\mu_l$ we had in the section of Corpuscular Quantum Physics page \pageref{magnetic dipole moment}) are the eigenvalues $\pm\hbar/2$ (\SeeChapter{see section Wave Quantum Physics page \pageref{angular momentum and spin}}) associated with the two possible values of a quantum number $m_s=\pm 1/2$ itself associated at the free state ($\vec{L}=\vec{0}$) to the angular momentum:
	
	Therefore:
	
	A complete description of the state of the electron therefore necessarily contains a wave function as usual giving the probability density of presence, but also taking into account the degree of freedom of the spin, hence the notation $\Psi(x,y,z,m_s,t)$. If the coordinates of space are continuous real values, however the spin variable is essentially discrete!
	
	Maintaining the usual interpretation, the quantity $|\Psi(x,y,z,m_s,t)|^2\mathrm{d}r^2$ is the probability of presence around the point selected with the value $m_s\hbar$ for the spin. The condition of normalization of the probabilities introduced as always a sum, which covers not only the orbital degrees (continuous summation, that is to say: integration) but also the degrees of spin (discrete summation):
	
	expressing in particular the fact that we exhaust all the possibilities of the spin by summing on the two possible values. In any case, the electron no longer has one but two wave functions, one for each value of $m_s$.
	
	The previous notation is not necessarily the best for spin-free particles greater than $1/2$ as we have seen in our study of angular momentum. About the fact of having  a variable taking discrete values, it is just as legitimate to put $m_s$ in index of the $\Psi$ and to put: $\Phi_{m_S}(\vec{r},t)$. Finally, it is convenient to use a matrix notation, storing in columns the various functions corresponding to the possible values of the discrete variable $m_s$. Thus, for the electron, we will now assume that all information in the sense of Wave Quantum Physics is contained in a two-line column vector named "\NewTerm{spinor}\index{spinor}\label{spinor relativistic quantum physics}" (\SeeChapter{see section Spinor Calculus page \pageref{spinors}}) and denoted:
	
	Let us now come back to the free Klein-Gordon equation (more general than the Schrödinger equation, of course, but less than the one including the magnetic field):
	
	This equation is as we know unfortunately incomplete because it contains no information on the spin of the electron (if we focus still only on this particle).

	We can, however, in order to find a solution to this problem, make a parallel with the electromagnetic field. That latter includes also a spin, residing in the polarization of the field (\SeeChapter{see section Electrodynamics page \pageref{light polarization}}). This polarization is closely related to the vector nature of the electromagnetic field and is reflected in the Maxwell equations, which are as we know first-order derivatives. However, by combining Maxwell's equations, we saw in the Electrodynamics section that we could obtain the wave equations:
	
	which are (very relevant coincidence!) a special case of the Klein-Gordon equation when $m_0=0$:
	
	However, the wave equations contain less information than the original Maxwell equations: they do not explicitly contain any relation between the various components of the fields $\vec{E}$ and $\vec{B}$, such as the fact that in an electromagnetic wave of wave vector $\vec{k}$ the fields $\vec{E}$ and $\vec{B}$ are mutually perpendicular and both are perpendicular to the wave vector $\vec{k}$ (ie they are bot perpendicular to the direction of propagation of the wave). To establish these constraints, we must return to Maxwell's equations and thus to equations with first-order derivatives.

	The same is true for fermions (electrons are part of it). Klein-Gordon's equation, even it is not false, is incomplete. We must try here to establish a first-order equation in derivatives which describes well the spin $1/2$ of the electrons of the fermions. This last condition means that this equation must therefore involve the two components of a spinor (in analogy with the one we have determined above):
	
	We will write this equation which we seek as:
	
	where $D$ is an $2\times 2$ matrix involving first order derivatives (a first-order differential operator) that we have to determine (and that we will!!!).

	To give an example before going further, let us look at how Klein-Gordon's equation can be expressed in such a form.

	We have first (free Klein-Gordon equation):
	
	or (generalized Klein-Gordon equation):
	
	What can also be written for the free Klein-Gordon equation:
	
	or for the generalized Klein-Gordon equation:
	
	Let us now restrict ourselves to the case of the free Klein-Gordon equation (the reasoning being similar but ... longer for the generalized version).

	The last expression of the free Klein-Gordon equation suggests introducing the two combinations found after (it seems...) numerous trial and errors by our predecessors:
	
	from which results:
	
	Therefore:
	
	Can be written in two ways after respective substitutions:
	
	Either in matrix form:
	
	or even:
	
	What we can write:
	
	Therefore, relatively to our initial idea of having a relation in the form:
	
	We can make the similarity (correspondence term by term) with the prior-previous equation:
	
	where $D$ is indeed a $2\times 2$ matrix.
	
	where $A$ is a vector (but represented by tradition as a scalar...), $\vec{B}$ is a vector (and denoted as such and has nothing to do with the notation of the magnetic field) and $\sigma$ a vector composed of $2\times 2$ symmetric matrices (by reading the rest you will see that posing this makes it possible to find what we are looking for...).

	Let us recall that the multiplication between $\vec{B}$ and $\vec{\sigma}$ constitutes a dot product such as that defined in our study of the Spinor Calculus section.
	\begin{tcolorbox}[title=Remark,colframe=black,arc=10pt]
	We reader must be very cautious in the developments that follow, because the traditional notations in the field make it very difficult to distinguish between products, dot products, and products of vector components forming a vector (element-wise multiplication).
	\end{tcolorbox}
	Let us put (by the way our predecessors also made numerous trials and errors before putting this ...):
	
	What we were initially seeking, that is, $\mathrm{D}\xi=0$, then becomes:
	
	Thus $A=\partial/\partial t$, $\vec{B}=\lambda\vec{\nabla}$ and for $\vec{\sigma}$ that latter remains (let us imagine...) unknown. We must also determine $\lambda$.
	
	Always by analogy with the example above, let us try to find the wave equation to determine the constant $\lambda$:
	
	So that we fall back on the wave equation, we must therefore have:
	\begin{enumerate}
		\item First that $\left(\vec{\sigma}\circ\vec{\nabla}\right)\left(\vec{\sigma}\circ\vec{\nabla}\right)=\vec{\nabla}^2$. Indeed (caution! the reader must have ideally at least studied the section of Spinor Calculus to know in detail how we get at this development!):
		
	
		\item And that $\lambda=c$:
		
	\end{enumerate}
	There are therefore, always in analogy with the previous developements, two possibilities which can be applied to different fields that we will be denoted $\chi^{(\pm)}$ and which are named "\NewTerm{Weyl spinors}\index{Weyl spinors}". We thus have a sort of double spinor (as in the parenthesis we have a two components vector formed by $2\times 2$ matrices) such that:
	
	These equations are named "\NewTerm{Weyl equations}" as the above relation represents in reality $4$ equations (don't forget that the parenthesis is a $2$ components vector made of $2\times 2$ matrices and therefore $4$ rows). Let us recall that a two dimensional spinor is a $2$ components vector that are both in $\mathbb{C}$.

	We now have to generalize the Weyl equations to the case of a half-integer spin fermion with mass. This new equation must respect the following constraints:
	\begin{enumerate}
		\item[C1.] It must be reduced to the Weyl equations when the mass tends to zero

		\item[C2.] It must lead to the free Klein-Gordon equation

		\item[C3.] It must describe particles with a spin
	\end{enumerate}
	The very astute solution (extremely difficult to guess) then consists in coupling the two Weyl equations by using different factor such as:
	
	To verify that these factors have been correctly chosen, we apply:
	
	to the first equation and substitute the second equation. We thus find:
	
	or developing the left extreme previous term and keeping the right extreme previous term and dividing by $\mathrm{i}$:
	
	to compare with:
	
	that is the free Klein-Gordon equation (we prove the same correspondence for the component $\chi^{-}$) and thus reinforces the validity of the assumptions and developments made so far.

	It is usual to combine the two spinors in a single spinor (this becomes a "\NewTerm{bispinor}\index{bispinor}") of $4$ components (a spinor with $4$ components, two of which are in fact associated with the particles and two with the antiparticles as we shall see):
	
	and to define the following two matrices (commonly named "\NewTerm{Gamma matrices}\index{Gamma matrices}") in a so-named "\NewTerm{standard form}":
	
	where $\mathds{1}$ is the $2\times 2$ unit matrix traditionally defined by:
	
	and:
	
	where the $\sigma_i$ are the "\NewTerm{Pauli matrices}\index{Pauli matrices}\label{pauli matrices}" given by (\SeeChapter{see section Spinor Calculus page \pageref{pauli matrices origin}}):
	
	which must satisfy for recall (proved earlier above):
	
	which is for recall $2\times 2$ matrix. Pauli's matrices are therefore good candidates to solve our problem!
	
	We can also found in the literature the following notation based on the tensor product (\SeeChapter{see section Tensor Calculus page page \pageref{kronecker product}}):
	
	and:
	
	\begin{tcolorbox}[title=Remarks,colframe=black,arc=10pt]
	\textbf{R1.} As we saw in the section of Spinor Calculus (Algebra Chapter), $\sigma_0$ is not really a Pauli matrix in itself. However, in some books, it is indicated as being one (this is also our choice here).\\
	
	\textbf{R2.} As we have also seen in it in the section of Spinor Calculus, let us recall that the Pauli matrices implicitly represent infinitesimal spatial rotations of a spinor.\\
	
	\textbf{R3.} Caution! In the gamma matrices, the writing of the $0$ means in fact that it is $2\times 2$ matrices  which all the components are zero.
	\end{tcolorbox}	
	This allows us, finally, to combine the equations:
	
	in only one (do not forget the association of the operators $E\rightarrow \mathrm{i}\hbar\partial/\partial t$,$p\rightarrow -\mathrm{i}\hbar\vec{\nabla}$):
	
	Using the traditional notation in Tensor Calculus and choosing the natural units ($c=1,\hbar=1$) we have:
	
	which is the usual form of the "\NewTerm{Dirac equation}\index{Dirac equation}\label{dirac equation}" (implicitly a system of four coupled differential equations) or "\NewTerm{relativistic equation of the electron}\index{relativistic equation of the electron}" with the "\NewTerm{covariant derivative}\index{covariant derivative}":
	
	and where the null vector is (for recall...) a vector with four components all equal to zero!
	\begin{tcolorbox}[title=Remark,colframe=black,arc=10pt]
	In elementary particle physics, the antecedent relation is named the "\NewTerm{covariant relativistic equation of fermions}\index{covariant relativistic equation of fermions}" because it describes particles with spin $1/2$.
	\end{tcolorbox}
	The matrices $\gamma^\mu$ are named commonly "\NewTerm{Dirac matrices}\index{Dirac matrices}". In an even more condensed form (using the "Feynman slash") the Dirac equation is sometimes written:
	
	or more explicitly in natural units (rare form but still correct):
	
	Or with a minimum of abbreviations (all terms explicited):
	
	
	We thus have, as in analogy with Maxwell's equations, a differential equations of the first order which have for properties:
	\begin{enumerate}
		\item[P1.] To allow to fall back on the Klein-Gordon equation, in extenso on Wave equation (as for the Maxwell's equations)

		\item[P2.] To take into account (explicitly) describe the spinor character of wave functions as we shall see by looking more closely at the Pauli matrices.
	\end{enumerate}
	\begin{tcolorbox}[title=Remark,colframe=black,arc=10pt]
	Since the Dirac equation applies to the spin $1/2$ particles, it also applies to neutrinos whose mass at rest is zero (thus the resolution of the Dirac equation is largely simplified).
	\end{tcolorbox}
	
	\subsubsection{Gamma, alpha, beta matrices}
	In the purpose now of interpreting the physical content of the Dirac equation, we will use a different representation of the gamma matrices. We have introduced earlier the representation by the following matrices $4\times 4$:
	
	that named "\NewTerm{standard representation}\index{standard representation}" and denoted rigorously:
	
	with for recall:
	
	But in fact these matrices are by far not the only $4\times 4$ matrices that satisfy the following set of equalities that we will justify later during our study of the linearized Dirac equation (page \pageref{linearized dirac equation}):
	
	For example, we will use many times the "\NewTerm{Dirac representation}\index{Dirac representation}" defined by:
	
	Or more explicitly:
	
	With the corresponding Dirac alpha and beta matrices:
	
	This representation (also not unique!) is particularly useful because it highlights the spinor character (due to the half-integer spin) of the electron wave function and separates the positive and negative energy components.
	
	With Maple 4.00b it can easily be checked that they satisfy the set of three equalities introduced just earlier (see the Maple companion book):\\
	
	\texttt{>with(linalg):\\
	>beta:=array([[1,0,0,0],[0,1,0,0],[0,0,-1,0],[0,0,0,-1]]);\\
	>alpha1:=array([[0,0,0,1],[0,0,1,0],[0,1,0,0],[1,0,0,0]]);\\
	>alpha2:=array([[0,0,0,-I],[0,0,I,0],[0,-I,0,0],[I,0,0,0]]);\\
	>alpha3:=array([[0,0,1,0],[0,0,0,-1],[1,0,0,0],[0,-1,0,0]]);\\
	>evalm(beta\string^2);\\
	>evalm(alpha1\string^2);\\
	>evalm(alpha2\string^2);\\
	>evalm(alpha3\string^2);\\
	>readlib(commutat):\\
	>evalm(c(alpha1,alpha2));\\
	>evalm(c(alpha1,alpha3));\\
	>evalf(evalm(c(alpha2,alpha3)));\\
	>evalf(evalm(c(alpha1,beta)));\\
	>evalf(evalm(c(alpha2,beta)));\\
	>evalf(evalm(c(alpha3,beta)));\\
	}
	
	We can also easily check (elementary linear algebra) that the Dirac representation is obtained from the standard representation by the transformation:
	
	where:
	
	Let us recall that $U^\dagger$ is the adjoint matrix (the conjugate of the transposed matrix) of $U$ (\SeeChapter{see section Linear Algebra page \pageref{adjoint matrix}}). Now, when all the elements are real as it is the case above and the matrix is a square one symmetric relatively to all its off-diagonal elements then we have obviously that $U=U^\dagger$.
	\begin{dem}
	
	and:
	
	\begin{flushright}
		$\square$  Q.E.D.
	\end{flushright}
	\end{dem}
	But notice that there is also another possible choice, named the "\NewTerm{chiral representation}\index{chiral representation}" defined by:
	
	Or more explicitly:
	
	With the corresponding chiral alpha and beta matrices:
	
	Thus more explicitly:
	
	Its advantage is that both spinors transform independently under rotations and translations. It is particularly useful for massless particles, the equations being considerably simplified. It has been used for the neutrino although neutrino oscillations show that their mass is non-zero.
	
	But notice that there is again also another possible choice, named the "\NewTerm{Weyl representation}\index{Weyl representation}" defined by:
	
	Or more explicitly:
	
	With the corresponding Weyl alpha and beta matrices:
	
	Thus more explicitly:
	
	This representation is useful when one seeks to derive the Dirac equation using the irreducible representations of the Lorentz group.
	
	But notice that there is again also another possible choice, named the "\NewTerm{Majorana representation}\index{Majorana representation}" defined by:
	
	Majorana matrices in the representation above obey also the following mandatory gamma anticommutation relations (as so far we don't know the $\beta$ and $\alpha_i$ corresponding Majorana matrices) as seen at page \pageref{gamma anticommutation relations} (and that can easily be verified using the same type of Maple code as given just earlier!):
	
	Thus, for $i\neq j$:
	
	and it is also immediate using this anticommutation relation that:
	
	A further, purely imaginary representation of the Majorana matrices spread in the literature is given by:
	
	That latter Majorana representation can be obtained from the original representation by the unitary transformation:
	
	There is alsoooooo another Majorana representation (sorry...) also denoted identically as previously (sic!) and defined by:
	
	That is explicitly:
	
	This representation has the interesting property that all matrices $\gamma^\mu$ are pure imaginary what make the Dirac fields of the Dirac equation a non-complex field where the particles are therefore their own antiparticles.
	\begin{tcolorbox}[title=Remarks,colframe=black,arc=10pt]
	\textbf{R1.} All these possibilities explains why physicists speaks sometimes of Dirac, Majorana or Weyl fermions.\\
	
	\textbf{R2.} Sadly it seems that there is no international convention on the definition and names of the above representations. It is a bit a mess and depending on the textbook you will not get the same definitions. They're also the leading causes of stress disorders amongst physicists.
	\end{tcolorbox}
	
	\subsubsection{Free particle solution}
	Let us now look for solutions specific to the Dirac equation in the form of a free particle\footnote{If the developments may bee to hard for the reader, he can jump straight to the case of the free particle at rest solution at page \pageref{free particle solution at rest}}
	
	By substituting in the Dirac equation and after simplification by $e^{\mathrm{i}\vec{p}\circ\vec{r}-Et}$ we find easily:
	
	Indeed in natural units:
	
	With the representation of Dirac, we get after development (trivial calculation that we can detail on request as always):
	
	Or more explicitly:
	
	Indeed, with the Einstein summation convention (\SeeChapter{see section Tensor Calculus page \pageref{einstein summation convention}}):
	
	In order for this matrix equation to have nonzero solutions, as usual, the determinant of the matrix must be zero (\SeeChapter{see section Linear Algebra page \pageref{determinant matrix inverse}}). We easily check that:
	
	Which implies (do not forget that we are in natural units!):
	
	This is therefore a system of equations very funny (...) to solve....... (do not forget that each term of the above relation is implicitly a $2\times 2$ matrix).
	
	With Chirale representation we would have obtained (always by adopting the traditional notation of this book for the scalar product):
	
	and we would not have come fall back on such an aesthetic and physical condition for there to be solutions!

Since the mass seems always positive, the Dirac equation thus has four linearly independent solutions, two of which have positive energy:
	
	and two with negative energy:
	
	It is therefore indeed the antiparticles that we had determined during our study of the free Klein-Gordon equation but with the spin in addition, hence the doubling of the additional solutions (two possible spin orientations per particle and by antiparticle). With the Chirale representation, we would not have fallen back on this result. Hence the necessity of the use of the representation of Dirac of the Pauli matrices.

We therefore know that there are solutions to the Dirac equation. Let us now determine these. Let us put:
	
	where $\phi_a$, $\phi_b$ are for recall the two double components of the spinor. We thus write the system of equations (we use, as Dirac did, the simple symbol of multiplication in place of the symbol of the scalar product):
	
	which gives us (do not forget that the term to the denominator is actually a $2\times 2$ matrix...):
	
	Thus, we have:
	
	We know that solutions exist and Quantum Physics imposes that these solutions are linearly independent. Thus, let us choose the solutions for $\phi_a$, $\phi_b$ as being proportional to:
	
	and as (\SeeChapter{see section Spinor Calculus page \pageref{spinor dot product}}):
	
	We then have the following possibilities:
	
	The question is now ... should we use:
 	
	Well, for $(1)$ and $(2)$ above we must use $E=+\sqrt{\ldots}$ otherwise $1/(E+m_0)$ becomes a singularity for $\vec{p}=\vec{0}$. For $(3)$ and $(4)$ we must use $E=-\sqrt{\ldots}$ otherwise $1/(E-m_0)$ becomes a singularity for $\vec{p}=\vec{0}$.
	\begin{tcolorbox}[title=Remark,colframe=black,arc=10pt]
	The term $E=+\sqrt{\ldots}$ is often referred to as the "\NewTerm{particle solution}\index{particle solution}" in the literature and $E=-\sqrt{\ldots}$ as the "\NewTerm{antiparticle solution}\index{antiparticle solution}". The interpretation of this result as an anti-particle would sometimes be named as we already mentioned the "\NewTerm{Feynman-Stückelberg inter-operation}\index{Feynman-Stückelberg inter-operation}".
	\end{tcolorbox}
	Going back to:
	
	and noting the spinors by (we change the notation to come back to the original one):
	
	\label{free particle dirac equation involving factor N} We finally have using $(1)$ and $(2)$ and by denoting $N()$ the part of the solution we should normalize (see page \pageref{normalization dirac free particle solution}), the following possible solutions and which are independent:
	
	
	with $E=+\sqrt{\ldots}$ and also:
	
	
	with $E=-\sqrt{\ldots}$. It is from usage to conventionally associate the two above functions to the "\NewTerm{positron}\index{positron}".

	All this can be abbreviated globally:
	
	Let us say indicate that the term $\vec{p}\circ\vec{\sigma}$ which is therefore the projection of the momentum (momentum) on a mathematical entity directly related to the spin is named "\NewTerm{helicity}\index{helicity}" and therefore fined by:
	
	
	\subsubsection{Free particle solution at rest}\label{free particle solution at rest}
	We start again from:
		
	Let us consider the derivatives of the free particle solution above:
	
	Substituting these into the Dirac equation gives:
	
	which can be written:
	
	This is the Dirac equation in "momentum" – note it contains no derivatives.
	
	For a particle at rest $\vec{p}=\vec{0}$ then:
	
	Hence the Dirac equation reduces to:
	
	Explicitly:
	
	We see quite immediately that this equation has for independent orthogonal solutions:
	
	The two first solutions leads us to:
	
	And the two last to:
	
	So we still have negative energy solutions (antiparticles)!
	
	Including the time dependence from $\Psi=\Psi_0(E,\vec{0})e^{-\mathrm{i}Et}$ gives:
	
	We therefore have two spin states with $E>0$ and two spin states with $E<0$. More explicitly, $\Phi_1$ then describes an $S=\frac{1}{2}$ (in Planck units) fermion of mass $m$ with spin $\uparrow$, $\Phi_2$ describes an $S=\frac{1}{2}$ (in Planck units) fermion of mass $m$ with spin $\downarrow$, $\Phi_3$ describes an $S=\frac{1}{2}$ (in Planck units) antifermion of mass $m$ with spin $\uparrow$, $\Phi_4$ describes an $S=\frac{1}{2}$ (in Planck units) antifermion of mass $m$ with spin $\downarrow$.
	
	In quantum physics we can't just discard the $E<0$ solutions as unphysical
as we require a complete set of states, i.e. four solutions!
	
	We can also see something important! The antiparticle solution is the complex conjugate of the particle solution! Keep this in mind for when we will study the Majorana equation later!
	
	For information the reader can check that if we take back the relations proved earlier:
	
	
	
	
	and we put $\vec{p}=\vec{0}$ we fall back on the same type of spinors as we just get above!
	
	\subsubsection{Majorana equation}
	An electrically charged particle is different from its antiparticles as it has the opposite electric charge, and electric charge is a measurable, stable property. It is possible, however, for an electrically neutral particle to be its own antiparticle. Photons, which have spin $1$ (in units of the rationalized Planck's constant $\hbar$) are a familiar case; neutral ions (spin $0$) are a further example, and gravitons (spin $2$) another. Particles that are their own antiparticles must be created by field $\Phi$ that obey:
	
	that is... real fields (!), because the complex-conjugate field $\Psi^\dagger$ create their antiparticles. 
	
	The equations for particles with spin $0$, spin $1$ and spin $2$ - the Klein-Gordon, Maxwell (electromagnetism) and Einstein Field equations (General Relativity) respectively - readily accommodate real fields, as these equations are formulated using real numbers.
	
	Indeed, when in 1928, Paul Dirac discovered the theoretical framework for describing spin-$1/2$ particles, it seemed that complex numbers were unavoidable. Diracc's original equation contained both real and imaginary numbers, and therefore it can only pertain to complex fields. For Dirac, who was concerned with describing  electrons, this feature posed no problem, and even came to seem an advantage because it explained why positrons, the antiparticles of electrons, exist.
	
	Enter Ettore Majorana. In his 1937 paper, Majoran posed, and answered, the question of whether equations for spin-$1/2$ fields must necessarily, like Dirac's original equation, involve complex numbers.
	
	Considerations of mathematical elegance and symmetry both motivated and guided his investigation. Majorana discovered that, to the contrary, there is a simple, clever modification of Dirac's equation that involves only real numbers. With this discovery, Majorana made the idea that spin-$1/2$ particles could be their own antiparticles theoretically respectable, that is, consistent with the general principles of relativity and quantum theory.
	
	Majoran inquired whether it might be possible for a spin-$1/2$ particle to be its own antiparticle, by attempting to find the equation that such an object would satisfy. To get an equation of Dirac's type (that is, suitable for spin-$1/2$) but capable of governing a real field, required $\gamma$m matrices that are purely imaginary.
	
	These matrices that we have already introduced earlier and named the "\NewTerm{Majorana representation}\index{Majorana representation}" is defined by:
	
	And we know, the Majorana matrices in the representation above obey also the following mandatory gamma anticommutation relations (as so far we don't know the $\beta$ and $\alpha_i$ corresponding Majorana matrices for recall!) as seen at page \pageref{gamma anticommutation relations}:
	
	Thus, for $i\neq j$:
	
	and it is also immediate using this anticommutation relation that:
	
	A further, purely imaginary representation of the Majorana matrices spread in the literature is also a we know given by:
	
	That latter Majorana representation can be obtained from the original representation by the unitary transformation:
	
	There is alsoooooo again as we already know another Majorana representation also denoted identically as previously (sic!) and defined by:
	
	That is explicitly:
	
	This representation has the interesting property that all matrices $\gamma^\mu$ are pure imaginary what make the Dirac fields of the Dirac equation a non-complex field where the particles are therefore their own antiparticles.
	
	Indeed, the "\NewTerm{Majaorana's equation}\index{Majaorana's equation}", then, is simply:
	
	Because the $\tilde{\gamma}^\mu$ matrices are purely imaginary, the matrices $\mathrm{i}\tilde{\gamma}^\mu$ are real, and consequently this equation can govern a real field $\tilde{\Psi}$.
	
	Particles corresponding to Majorana spinors are known as Majorana particles, due to the above self-conjugacy constraint. All the fermions included in the Standard Model have been excluded as Majorana fermions (since they have non-zero electric charge they cannot be antiparticles of themselves) with the exception of the neutrino (which is neutral).

	Theoretically, the neutrino is a possible exception to this pattern. If so, neutrinoless double-beta decay, as well as a range of lepton-number violating meson and charged lepton decays, are possible. A number of experiments probing whether the neutrino is a Majorana particle are currently (year 2017) underway.
	
	\pagebreak
	\subsection{Linearized Dirac Equation}\label{linearized dirac equation}
	We saw at the beginning of our study of Wave Quantum Physics that the classic Schrödinger equation of evolution was:
	
	ie a differential equation of the first order with respect to time and the second with respect to the spatial coordinates.

	We had then determined the relativistic Schrödinger equation of evolution (free Klein-Gordon equation) given by:
	
	We notice that by passing to a relativistic form, we now have a differential equation of the second order in time AND space.

	Then through the generalized Klein-Gordon equation which also contained a second-order differential equation in time and space:
	
	and with the free Dirac equation we have just proved that we get in the same way a matrix differential equation of first order in time and second order in space:
	
	These changes of order of the differentials of a relativistic model or not impose of course in the case of a first order to know the initial conditions in time and space of the wave equation, which is feasible. However, when a second order appears, it is necessary to know more about the initial conditions of the derivatives of the wave functions (\SeeChapter{see section Differential and Integral Calculus page \pageref{initial conditions}}). Moreover, even if mathematically the (relative) rigor naturally led us to the different orders obtained, it is strange that changing to a relativist model that we change order of the differentials. Why?: For the quite simple reason that by approximating the relativistic equations we are not able because of the presence of the Planck constant factor to make approximations (development in series of $v/c$) that would bring us back to the first order. The relativistic and nonrelativistic equations seems then a priori incompatible within the non relativistic limits!
	
	The Dirac method to solve this problem will have been the following:

	The orders of the Klein-Gordon differential equation coming from the relation  of the total energy (see the beginnings of our developments of the free Klein-Gordon equation) in the absence of any field:
	
	It seem that Dirac would have had the brilliant idea of linearizing this Hamiltonian by putting (sadly we don't know how he get this idea):
	
	which we will have to determine the parameters $\beta$, $\alpha(\alpha_1,\alpha_2,\alpha_3)$ which can be scalars, vectors or matrices (wait a little ... the answer will come). It follows that the Hamiltonian is also a scalar, a vector, or a matrix.

	Thus, the simplest relativistic evolution wave equation that we can construct will be:
	
	In a much more common form in the literature:
	
	Or also:
	
	Or in natural units:
	
	As here:
	
	we find then the prior-previous relation also in the form:
	
	If the linear momentum were to be zero, we would thus fall back on the energy at rest for the Hamiltonian:
	
	where as we shall see later $\beta=\pm 1$.
	
	The validity of this linearization must be verified by falling back on the results obtained during our previous study of the Dirac equation.

	Let us now raise the operator to the square:
	
	and let us put:
	
	At this point, it is important to notice that we may be working with operators (typically matrices) that might not commute because the $\alpha$, $\beta$ are unknown. Therefore, the squaring will be done as follows:
	
	thus simply developed but without simplifying the sum $AB + BA$ into a $2AB$ or a $2BA$ since we are not sure for now if there will be commutativity or not.

	We therefore develop the Dirac Hamiltonian:
	
	By making the products of the terms in brackets and respecting the order of the operators, it comes:
	
	By grouping certain terms:
	
	To be consistent with our linearization assumptions, we must have:
	
	and we will see immediately that to satisfy these conditions the $\alpha$, $\beta$ will have to be $4\times 4$ matrices.

	Written in the form of commutators, we therefore have the following three conditions to satisfy:
	
	We observe the following:
	\begin{itemize}
		\item The square of each operator $\beta$ and $\alpha_j$ is equal to $1$ (or to the unit matrix if it is a matrix...).

		\item $\left[\alpha_i,\alpha_j\right]_{+}=0,\forall i,j=\{1,2,3\},i\neq j$ is an anticommutator

		\item $\left[\alpha_i,\beta\right]_{+}=0$ is an anticommutator
	\end{itemize}
	These three relations can be summarized as follows:
	
	At this point, we need to look for mathematical objects that meet the three conditions above. We could show that a square matrix of dimension $2$ or $3$ does not satisfy the three conditions and a scalar even less!

	Dirac then adopted by analogy with the previous developments, square matrices of dimension $4$ including Pauli matrices (so lucky guy...) and admitted for $\beta$ a unit matrix (this choice made by Dirac is particular, there are possible choices hence the necessity to check the theory with experimental observations!).

	So what we denoted "$1$" before is actually a square unit matrix of dimension $4$! And so the wave functions must have $4$ components and mathematical beings that have specificity are the spinors!

	The matrices considered by Dirac (which are a particular choice!) are therefore for $\alpha_i$, $\beta$:
	
	and for example in certain domains of quantum physics we use Weyl's choice:
	
	In the choice considered by Dirac or Weyl, we have the following Pauli matrices and unit matrix:
	
	This leads to the matrices $\alpha_i$, $\beta$ (we recognize that the $\alpha_i$ are in fact the gamma matrices $\gamma_i$ in their Chiral representation as introduced earlier above at the difference of a $-$ sign on one component but that anyways leads to exactly the same results!!!!!):
	
	\begin{tcolorbox}[title=Remark,colframe=black,arc=10pt]
	Notice that all matrices above are Hermitians!
	\end{tcolorbox}
	We can check that the linearization conditions are satisfied by the preceding matrices:
	\begin{itemize}
		\item First condition:
		
		Similarly for the $\alpha_j$:
		
		The first condition is therefore fulfilled!
	
		\item Second condition (caution with notations that sadly slip a bit by tradition between matrices and scalars!):
		
		and:
		
		Therefore (caution! this is in fact a $4\times 4$ matrix!!):
		
		The second condition is well fulfilled!
		\begin{tcolorbox}[title=Remark,colframe=black,arc=10pt]
		\label{gamma anticommutation relations}The last relation should be written:
		
		and if we take the gamma matrices with the Dirac or \underline{any other} representation introduce earlier above, the same is satisfied but generalized if we include $\gamma_0$ to:
		
		where $\eta_{ij}$ the Minkowski metric with signature $(+, -, -, -)$. This relation is important because it is sometimes used as the definition of the gamma matrices (!!!) and it makes appears the metric!!!\\
		
		From the abvoe relation we have immediately for $i\neq j$:
		
		and it is also immediate using the latter anticommutation relation that:
		
		The reader should also know that physicists have derived from the above relation a small dozens of remarkable identities involving the $\gamma$-matrices and that also can be used to very that the chosen matrices can have indeed the status of $\gamma$-matrices. We will not present and prove these identities actually in this book as they are useless relatively to the actual content!
		\end{tcolorbox}
		
		\item Third condition:
		
		The third and last condition is thus fulfilled.
	\end{itemize}
	Referring to the beginning equation written with the Dirac formalism:
	
	With:
	
	Which gives finally:
	
	We find ourselves in front of a state function having $4$ components in which:
	
	are spinners and the set:
	
	is therefore "\NewTerm{Dirac bispinor}\index{Dirac bispinor}" and we denote by:
	
	the "\NewTerm{Dirac state function}". The reader will have notice that we fall back one the same concepts as in our study of the free nonlinearized Dirac equation. This is quite a good sign!
	
	By developing, it comes:
	
	For a free electron, we know that the solution is:
	
	With Dirac bispinor we have:
	
	with:
	
	where $b_1$ to $b_2$ are the components of the Dirac bispinor.
	
	We will write:
	
	By calculating their derivatives with respect to $t$:
	
	With \eqref{diraclin02} and \eqref{diraclin03} in \eqref{diraclin01}, it comes:
	
	Hence a system of equations whose unknowns are $b_1,b_2,b_3,b_4$:
	
	We have solutions that are not all zero if and only if the determinant of the coefficients is zero (to know the reasons, refer to the section of Linear Algebra) and therefore an infinite number of solutions (for the components of the Dirac spinor). Therefore:
	
	By simplifying by $c$:
	
	The division in the preceding determinant allows the calculation of the partial determinants (\SeeChapter{see section of Linear Algebra page \pageref{partial determinant}}):
	
	In solving the preceding determinant, it comes:
	
	Hence the following relation:
	
	The values of the energy given by the Dirac equation are thus (we recognize here a famous relation determined in the section of Special Relativity!):
	
	Therefore:
	
	If we adopt for $E_{+}$ two constant values for $b_1$ and $b_2$ we have two relations to calculate $b_3$ and $b_4$, thus:
	\begin{itemize}
		\item With $($\ref{diracdeterminant} $c)$:
		
		Thus:
		
		
		\item With $($\ref{diracdeterminant} $d)$:
		
		Thus:
		
		By adopting $b_1=1$, $b_2=0$, it comes:
		
		Taking the natural units:
		
	\end{itemize}
	If we adopt for $E_{-}$ two constant values for $b_3$, $b_4$ we have two relations to calculate $b_1$, $b_2$, thus:
	\begin{itemize}
		\item With $($\ref{diracdeterminant} $a)$:
		
		Thus:
		
	
		\item With $($\ref{diracdeterminant} $b)$:
		
		Thus:
		
		Notice that by adopting $b_3=1$, $b_4=0$ it comes:
		
		With the natural units:
		
		By adopting $b_3=0$, $b_4=1$ it comes:
		
		Either with the natural units:
		
	\end{itemize}
	Even if the method is different, we fall back on the coefficients of the spinors obtained in our study of the classical free Dirac equation. This confort us in the assumptions made at the beginning of this linearization and validates these results. Moreover, the preceding relations also indicate a degeneracy of order two of the energy for each value of the pulsation. In the absence of an external field, the free electron is therefore not influenced by the orientation of its spin. We thus find the same results for either the classical or linearized free Dirac equation.

	However, Dirac's explanation for the positive and negative energies is that his equation applies not only to the state of a positive energy particle (in this case the electron) but also to the state of a particle with negative energy (its antiparticle is the "\NewTerm{positron}\index{positron}"). The absolute value of these two energies being strictly equal.

	The presence of the negative sign affecting the energy posed a problem at the time for its interpretation (in the context where we omit the time variable since we had seen when studying the free Klein-Gordon equation, a particle with negative energy can be seen as a particle that goes back in time).

	If we reason in the case where the term $p^2c^2$ is small compared to $m_0c^2$, we ask ourselves: how and what are the consequences of a transition between a state of energy $m_0c^2$ and that of the state of energy $-m_0c^2$ with a gap of $-m_0c^2$ (we will find again this value in our study of materialization in the section of Nuclear Physics).

	Dirac uses the image of a sea of negative energy named "\NewTerm{Dirac sea}\index{Dirac sea}" (since, remember, the number of solutions to our matrix system is infinite, hence the analogy with a sea than a discrete context) in which all negative energy states are occupied by the electrons and the positive energy states would be empty. If an electron is subjected to a transition (via, for example, a photon of energy greater than $2m_0c^2$), it leaves this sea leaving behind a gap (the famous "hole" of positive charge to which the electronics sometimes refers to...). This gap becomes a positive charge, of energy $E_{+}$. The appearance of this gap is assimilated to the appearance of a particle having a positive charge. Obviously, we can imagine the opposite case, it is only a matter of conventions.
	\begin{figure}[H]
		\centering
		\includegraphics{img/atomistic/dirac_sea.jpg}	
		\caption[Dirac Sea]{Dirac Sea (source: ?)}
	\end{figure}
	\subsubsection{Linearized Dirac Equation continuity equation}
	Now consider probability density/current – this is where the perceived problems with the Klein-Gordon equation arose. For this, identically as we did for the Generlized Klein-Gordon equation, we start with the linearized Dirac equation (in natural units to simplify):
	
	or:
	
	Where we know now obviously that:
	\begin{enumerate}
		\item $\vec{\alpha}$ are Pauli matrices
		
		\item $\vec{\alpha}$ and $\vec{\beta}$ are Hermitian
	\end{enumerate}
	Writing the above relation inf full gives:
	
	and its Hermitian conjugate:
	
	Now, exactly as we did the the Generalized Klein-Gordon equation we write take the first above expansion that we multiply by $\Phi^\dagger$ and we substract here be the second Hermitian expansion that we multiply by $\Phi$:
	
	Remembering that $\alpha$ and $\beta$ are Hermitian, we can write:
	
	The right hand side can easily be simplified by chain derivation rule:
	
	Se see also that the terms $m_0\beta\Psi$ cancels each other, then it remains:
	
	We multiply both sides by $\mathrm{i}$ and rearrange a bit:
	
	Now using the identity:
	
	Gives the linearized continuity Dirac equation:
	
	The probability density and current  can be identified as:
	
	where:
	
	Then, unlike the generalized Klein-Gordon equation, the Dirac equation has probability densities which are always positive!
	
	In regard of what we have seen earlier, we can introduce the "\NewTerm{four-vector fermion current}\index{four-vector fermion current}":
	
	and $\rho$ is the probability density as we know:
	
	
	\label{normalization dirac free particle solution}Now let us come back to the normalization of $N$ that appeared during our study of the free Dirac particle solution page \pageref{free particle dirac equation involving factor N}. If we proceed by analogy (...), remember first that we get with the Klein-Gordon equation that:
	
	And just now we get that:
	
	So now knowing this, let us consider the following solution that we have obtained:
	
	That to simplify we will put a being $\Psi$. Then (still in natural units):
	
	Now to get the same result as for the Klein-Gordon equation, we must have:
	
	Hence:
	
	Or more differently written:
	
	
	\subsubsection{Linearized Dirac Equation Angular Momentum + Helicity Conservation}
	For the Dirac linearized equation model is orbital angular momentum a good quantum number? I.e. does:
	
	commute with the Hamiltonian:
	
	where for recall the corresponding Dirac alpha and beta matrices are:
	
	To check this, let us consider the $x$ component of $\vec{L}$ (we implicitly assume that we work with the Dirac alpha and beta matrices so we don't denote the lower index $D$ anymore):
	
	So we may think that:
	
	even if the shortcut is nice and may look correct... this is a wrong shortcut in fact especially when we deal with differential operators as both arguments of the commutator (typical well know trap in maths when we do things too quickly)!!!
	\begin{tcolorbox}[title=Remark,colframe=black,arc=10pt]
	The term $\beta m_0$ disappears as it is not a differential operator (only a real matrix with a constant factor $m_0$). By cons, the first term has the $\vec{p}$ in it, and this latter is a vector of differential operators!
	\end{tcolorbox}
	
	We already know that for the linear momentum, the only non-zero contributions come form:
	
	Hence (we use a lot here the Poisson brackets properties seen at page \pageref{poisson bracket}):
	
	Two techniques were employed along the way: between the $4$th and the $5$th line of the above development, we used (see page \pageref{poisson bracket}):
	
	and between the $7$th and $8$th we used:
	
	The same strategy can be adopted to solve for $[H,L_y]$ and $[H,L_z]$:
	
	Thus $[H,\vec{L}]$ can be determined by:
	
	Therefore:
	
	Hence the angular momentum does not commute with the Hamiltonian and is not a constant of motion!
	
	Introduce a new $4\times 4$ operator:
	
	where as we know, the $\vec{\sigma}$ are the Pauli spin matrices i.e.:
	
	Now consider the commutator:
	
	here (obvious!):
	
	and hence:
	
	Consider the $x$ component:
	
	Taking each of the commutators in turn:
	
	Hence:
	
	The same strategy can be used to solve for $[H,\Sigma_2]$ and $[H,\Sigma_3]$:
	
	Therefore:
	
	Hence the observable corresponding to the operator $\vec{\Sigma}$ is also not a constant of motion. However, referring back to our relation:
	
	We have:
	
	Therefore:
	
	Because:
	
	Thus, Dirac equation proved a description of "intrinsic" angular moment (ie spin).
	
	How let us show that helicity $\hat{h}$ (projection of the spin along the direction of motion for recall), does commute with the Hamiltonian, which means that helicity is a conserved quantity. Let us recall the definition of helicity:
	
	So we want to calculate:
	
	We see here a problem... Indeed, on the left term with have $4\times 4$ matrices, and on the right term a $2\times 2$ matrices. So we try by moving to then generalized total spin matrix introduced earlier above:
	
	Therefore if we focus on a only a component given that the direction is only along the $z$-axis, we can write:
	
	And, for recall, as:
	
	Hence:
	
	If:
	
	then helicity commutes with the Hamiltonian, and therefore helicity is a conserved quantity. Let us verify the previous equality:
	
	and:
	
	Therefore:
	
	The same strategy can be adopted to solve for $[H,h_1]$ and $[H,h_2]$:
	
	Therefore:
	
	Helicity is therefore a conserved quantity!
	
	\pagebreak
	\subsection{Generalized Dirac Equation}
	In the case of the free electron, we have thus often seen and proved that the Hamiltonian has as expression:
	
	In the case of an electron moving in an electromagnetic field, we have also proved in our study the Klein-Gordon equation at the beginning of this section that:
	
	Therefore:
	
	OK finish for the recall!
	
	If we now take back the Dirac Hamiltonian for the free electron proved earlier above (which for recall is a $4\times 4$ matrix):
	
	and from the fact that we must add to the Hamiltonian the term of potential electrostatic energy via the potential vector, we get:
	
	where the potential $U$ will also have to be expressed in the form of a diagonal $4\times 4$ matrix.

	We then get the generalized Hamiltonian of Dirac in the following traditional form:
	
	Therefore we have in another well known form:
	
	Or in natural units:
	
	And as we know $\vec{p}=\mathrm{i}\vec{\nabla}=\mathrm{i}\partial_\mu$ and $\vec{\alpha}=\vec{\gamma}$, hence the well know form (using $q=e$ for the electron charge):
	
	
	\pagebreak
	\subsection{Pauli Equation}\label{pauli equation}
	In quantum mechanics, the Pauli equation or Schrödinger–Pauli equation is the formulation of the Schrödinger equation for spin-$1/2$ particles, which takes into account the interaction of the particle's spin with an external electromagnetic field. It is the non-relativistic limit of the Dirac equation and can be used where particles are moving at speeds much less than the speed of light, so that relativistic effects can be neglected. It was formulated by Wolfgang Pauli in 1927.
	
	Let us now consider a two-component representation of the spinor:
	
	and let us recall that:
	
	Therefore it comes:
	
	Hence:
	
	What after simplification gives:
	
	Before continuing, let us open an important parenthesis, otherwise we will not be able to find a solution to these two equations.

	Let us recall that one of the spinors solution of the free Dirac equation was given by (as we have proved it earlier above):
	
	Either in International Standard units:
	
	In order to simplify the calculation of the prior-previous equations we will reduce the situation to a non-relativistic case, that is to say when the mass energy is much greater than the kinetic energy. Therefore the previous solution becomes (we forget the second which would cause problems ...):
	
	The idea is simple but you had to think about it! It should be notice that depending on the authors and professor it is the following term which is named "\NewTerm{helicity}\index{helicity}":
	
	as it is the projection (as we will see later) of the spin operator on the linear momentum.
	
	The idea is then to find a solution to:
	
	such that we make a non-relativistic approximation and we cancel the magnetic field (in extenso the vector potential), we fall back on:
	
	After many trial and errors (yes Quantum Physics was not made in one day... not even if one year...) we find that a particular solution satisfying our previous idea is:
	
	Indeed:
	
	We finally have $2$ equations (yes yes... we know that in fact as spinor have two components this if $4$ equations...):
	
	Now let us consider only the second equation:
	
	Assuming (for free, after which it will be necessary to compare that assumption to experimental results) that the term $|\partial_t \phi_b|$ is much smaller than $|2m_0c^2\phi_{b0}|$ we can write:
	
	By making the same assumption with $|qU\phi_{b0}|$ we have:
	
	We then have (do not forget that the denominator is a diagonal matrix in reality ...):
	
	Now, we see that if the magnetic field (in extenso the vector potential) vanishes, we fall back on well our idea of departure! So the bet is good!

	Because of all these downward approximations, the $\phi_b$ component is often taken as the "small" component of the wave function $\Psi$, relatively to the large component $\phi_a$.

	The first equation:
	
	can now be simplified easily by taking the previous solution such that:
	
	Therefore:
	
	Using the remarkable identity proved in the section of Spinor Calculus:
	
	we get:
	
	Let us detail the cross product by remembering that it will act as an operator on $\phi_{a0}$:
	
	But, we have (\SeeChapter{see section Spinor Calculus page \pageref{spinor dot product}}):
	
	Let us just look at the component in the upper left corner (if not the calculations are too long and boring to do) of this sum of matrices. We must not forget that this component of the matrix will act on the first component as an operator on $\phi_{a0}$ (written with the same notation as the spinor itself to avoid to many indices):
	
	But:
	
	Therefore:
	 
	But, we recognize here the third component of a cross product that does not act as an operator. Finally, it comes:
	
	Therefore:
	
	Thus, the relation of the main component:
	
	becomes:
	
	The reader will notice that the notations are not the most enjoyable (between vectors, matrices and constants it is necessary to follow well to know what is in there ...).

	After rearrangement:
	
	which constitutes the "\NewTerm{Pauli equation}\index{Pauli equation}", or \NewTerm{Schrödinger-Pauli equation}\index{Schrödinger-Pauli equation}", and thus describes in a relativistic way the two components of $\phi_{a0}$ of the spin of the electron (this is really a two-equation system).

	The expression:
	
	is named the "\NewTerm{Stern-Gerlach term}\index{Stern-Gerlach term}" and represents the interaction energy of the magnetic field with the intrinsic moment of the electron (this is an $2\times 2$ matrix for recall). And the therm:
	
	is named the "\NewTerm{Pauli Hamiltonian}\index{Pauli Hamiltonian}" and denoted $H_P$.

	The Pauli equation, and therefore that of Dirac (since the latter is more general), give the correct gyromagnetic factor for a free electron of $g=2$. To verify this, let us take as it has been done experimentally, a constant magnetic field.

	We can easily verify that the choice of a vector potential corresponding to a constant magnetic field is then:
	
	This choice will have the effect of make disappear the vector potential in favor of the magnetic field in the Pauli equation, which will reveal the interaction between the orbital angular momentum and the magnetic field as we shall see:

	Indeed, we have:
	
	As requested by a reader, here is the proof of that latter relation:
	
	We then have in Pauli equation:
	
	But, let us recall that we have seen in the section of Vector Calculus page \pageref{cross product} that:
	
	This gives us therefore:
	
	where:
	
	denoted also $\vec{b}$ (\SeeChapter{see sections of Classical Mechanics page \pageref{angular momentum} and Corpuscular Quantum Physics page \pageref{quantized angular momentum}}), is therefore an operator representing the angular momentum.

	Therefore we have:
	
	By proceeding as Dirac did, that is, by defining the spin operator as the matrix (oh! we fall back on something known and seen in the section of Wave Quantum Physics!! it's beautiful no!?)\label{emerging electron spin value}:
	
	This relationship will be very useful to us in the section Quantum Computing. Let us indicate the following notation (quite logic) in the literature for the intrinsic magnetic moment:
	
	The Pauli equation is then written:
	
	or after factorization (the Hamiltonian will then be named "\NewTerm{Pauli Hamiltonian for a constant direction magnetic field}\label{pauli hamiltonian for a constant direction magnetic field}"):
	
	or even more condensed by being careful to differentiate what is an operator, what is a vector and what is a simple multiplication from an inner product and what is a function of a spinor ... (only happiness...) and by putting:
	
	as being the "\NewTerm{Landé factor}\index{Landé factor}" or "\NewTerm{gyromagnetic factor}\index{gyromagnetic factor}\label{gyromagnetic factor}", we have:
	
	with $\vec{\mu}_L$ being the "\NewTerm{orbital magnetic moment}\index{orbital magnetic moment}" (by comparing with the expression of the magnetic moment proved in the Magnetostatic section you will see that it actually has the same form and also the same units), $\vec{\mu}_S$ the "\NewTerm{magnetic moment of spin}\index{magnetic moment of spin}".

	Sometimes we find the expression in parentheses in the above-mentioned relationship as follows:
	
	With all this the Stern-Gerlach term (magnetic moment) becomes explicitly:
	
	Recalling that:
	
	is the Bohr magneton which we had introduced rigorously in the section of Corpuscular Quantum Physics.
	
	Interestingly ... the neutron has (experimental observation!) a spin magnetic moment that is non-zero yet its charge is zero ... so it must be composed of charged particles (if we remain within the framework of a model of the explanation of nature by particles ...). It is interesting to know that in practice we have for the electron, the proton and the neutron:
	
	and the following values were measured for spin $1/2$ particles such as the electron, proton and neutron (caution! the sign may change depending on how the Dirac equation is written):
	
	Therefore the Dirac theory in the non-relativistic framework predicts in a good approximation that the elementary particles of spin $1/2$ have a gyromagnetic factor of $2$, and this prediction consistent with the experiment for the electron is the greatest triumph of the Dirac equation. The deviations of the theoretical value (consequent deviations!) for the proton and the neutron are perfectly explained in the framework of quantum electrodynamics. These deviations show that the structure of the proton and that of the neutron are more complex (subparticle composition) than a spin $1/2$ point particle whereas in the case of the electron, it would seem that has no underlying substructure.
	\begin{tcolorbox}[title=Remark,colframe=black,arc=10pt]
	The gyromagnetic factor is sometimes taken as negative but it is only a matter of convention.
	\end{tcolorbox}
	It is by the way the interaction term between the magnetic field and the addition of the orbital angular momentum and intrinsic spin:
	
	of the Pauli Hamiltonian which gives the values measured by the Zeeman effect! Reason why this expression is sometimes named "\NewTerm{Zeeman energy}\index{Zeeman energy}" and sometimes written in the form of a Hamiltonian operator:
	
	\begin{tcolorbox}[title=Remark,colframe=black,arc=10pt]
	The above treatment of the Zeeman effect describes the phenomenon when the magnetic fields are small enough that the orbital and spin angular momenta can be considered to be coupled (\SeeChapter{see section Wave Quantum Physics page \pageref{ls coupling}}). For extremely strong magnetic fields this coupling is broken and another approach must be taken. The strong field effect is called the "\NewTerm{Paschen-Back effect}\index{Paschen-Back effect}".
	\end{tcolorbox}
	We know that this last relation can also be written for any single particle (\SeeChapter{see section Wave Quantum Physics page \pageref{angular momentum and spin}}) immersed in a magnetic field collinear to the $z$-axis and zero in the other directions:
	
	where $\gamma$ is the "\NewTerm{gyromagnetic ratio}\index{gyromagnetic ratio}" and the $\mu_B$ the "\NewTerm{Bohr magnetron}\index{Bohr magnetron}" (and we saw in the proof of this relation that the Bohr magnetron is only a special case with the Landé $g$ factor equal to that of the electron).
	
	Let us now focus only on the smallest variation in energy between two states of the orbital angular momentum (since the smallest variation in the variation is $1$ by the quantification of $l$). It will then always be by equal spin to the form:
	
	\begin{tcolorbox}[title=Remark,colframe=black,arc=10pt]
	As $\gamma\hbar\cong 5.8\cdot 10^{-5}\;[\text{eV}\cdot\text{T}^{-1}]$ and that largest field which can regularly be produced in a lab is around $10$ [T]. Thus, $\Delta E_{p,l}\cong 10^{-3}$ [eV]. This is a very small energy. Thus magnetic effects are wiped out by thermal fluctuation at about $10$ [K]! Thus experiments which measure the magnetism of non-interacting systems must be carried out at low temperatures. These experiments typically measure the susceptibility
of the system with a Faraday balance or a magnetometer (SQUID).
	\end{tcolorbox}
	This variation of energy will be restored in the form of electromagnetic waves corresponding to:
	
	hence:
	
	which is the "\NewTerm{Larmor relation}" (not to be confused with the "\NewTerm{Larmor radius}" seen in the Magnetostatic section). But in practice we mainly use the relation:
	
	which gives what we name the "\NewTerm{resonance frequency}" of the orbital angular momentum.
	
	Similarly, for the intrinsic kinetic moment, we get for a constant orbital kinetic moment (during the spin transition from $+1/2$ to $-1/2$ or vice versa):
	
	and putting again it comes:
	
	hence:
	
	and obviously:
	
	This last study of energy variation due to the application of a magnetic field and the resulting energy emission frequencies is the basis of "\NewTerm{nuclear magnetic resonance (NMR)}\index{nuclear magnetic resonance}", which therefore works only for the particles having the magnetic moment of spin $\mu_S$ (by construction of the Pauli Hamiltonian).
	
	NMR consists in modifying the nuclear magnetic moment, in other words, to pass the nucleus from one level of energy to another by applying magnetic fields to the sample to be studied. When the energy of the photons constituting these magnetic fields corresponds to the energy of transition from one level of energy to the other, these photons can be absorbed by the nucleus: we then say that there is "\NewTerm{nuclear resonance}\index{nuclear resonance}".
	
	We can characterize the transition energy of the nuclear magnetic spin moment by giving the frequency of the electromagnetic wave that allows resonance. For the usual fields (of the order of the Tesla), the proton resonance takes place in the range of radio waves ($100$ [MHz] approximately): $42$ [MHz] in a field of $1.0$ [T] and $63$ [MHz] In a field of $1.5$ [T].
	
	Let us point out that on the way that in some textbooks, since we have:
	
	Then for a particle with two possible states of spin the energy of each of the level is then obviously half the above difference, which is often denoted by:
	
	
	\subsubsection{Landé g factor}
	We have seen previously that the Landé g-factor is a multiplicative term appearing in the expression for the energy levels of an atom in a weak magnetic field. We seek now for a more detailed expression and u will see that all the abstract stuff we have introduced during our study of orbital-spin coupling in the section of Wave Quantum Physics will be useful to us.
	
	In the weak field limit, we assume that the magnetic dipole moment of the atom is proportional to the total angular momentum $\vec{J}$:
	
	where $m_B$ is still the Bohr magneton and $g_J$ is the Landé g factor for $\vec{J}$. To find a value of $g_J$, we relate $\vec{J}$ to the know values of $g_L$ and $g_S$ and we use (\SeeChapter{see section Wave Quantum Physics page \pageref{total spin}}):
	
	and therefore:
	
	with the assumption that:
	
	Hence:
	
	Therefore after simplification:
	
	We then take the dot product with $\vec{J}$ and get:
	
	Therefore:
	
	That is often written for simplification purposes:
	
	Now let us recall that in the section of Wave Quantum Physics during the study of spin-orbit coupling we have proved that (still simplifying the notation of the norm of a vector):
	
	Therefore:
	
	Now we introduce the eigenvalues of $J^2$, $L^2$ and $J^2$ as we have determined the during our study of Angular Momentum and Spin and we simplify by $\hbar^2$ directly and that gives:
	
	Therefore:
	
	If we put $g_L=1$ and $g_S\cong 2$ (implying that the spin angular momentum is twice as effective in producing a magnetic moment) then the above expression simplified to:
	
	Now following the same reasoning, let us determine in a most robust was the relation:
	
	determined earlier above.
	
	The problem with evaluating the scalar product above is that $\vec{L}$ and $\vec{S}$ continually change in direction as shown in the vector model. The strategy for dealing with this problem is to use the direction of the total angular momentum $\vec{J}$ as a coordinate axis and obtain the projection of each of the vectors in that direction. This is done by taking the scalar product of each vector with a unit vector in the $\vec{J}$ direction:
	
	Using the fact that:
	
	we get:
	
	These vector relationships must be evaluated and expressed in terms of quantum numbers in order to evaluate the energy shifts. Carrying out the scalar products above leads to:
	
	And we have also proved in the section of Wave Quantum Physics that:
	
	Rearranging gives the scalar product term needed:
	
	Substitution into the energy expression gives:
	
	which can be reduced to:
	
	where using eigenvalues we have:
	
	so we fall back on the approximation of $g_J$!
	
	So finally:
	
	
	\begin{tcolorbox}[colframe=black,colback=white,sharp corners]
	\textbf{{\Large \ding{45}}Example:}\\\\
	Let us give a word about the fine structure of sodium atoms.\\
	
	As we have seen in the sectionofCorpuscular Quantum Physics and following the aufbau principle (also introduced in the same section) for the atom of Sodium $\mathrm{Na}$ we have seen that its electronic configuration was:
	
	Explicitly we then say (following orbital-spin coupling notation), that is ground state is then:
	
	The first excited state is obtained (still following the aufbau principle) by promoting the $3s$ electron to the $3p$ orbital, to obtain the:	
	\end{tcolorbox}
	
	\begin{tcolorbox}[colframe=black,colback=white,sharp corners]
	
	configuration. In a sodium-vapor lamp for example, sodium atoms are excited to the $3p$ level by an electrical discharge, and return to the ground state by emitting yellow light of wavelength $589$ [nm].\\
	
	Now let us recall the following table (\SeeChapter{see section Wave Quantum Physics page \pageref{types of orbital and spin}}):
	\begin{table}[H]
		\centering
		\renewcommand{\arraystretch}{2.6}
		\begin{tabular}{|c|c|c|c|c|c|}
		\hline
		\rowcolor[HTML]{9B9B9B} 
		\textbf{Orbital type} & \textbf{$\pmb{s}$} & \textbf{$\pmb{p}$} & \textbf{$\pmb{d}$} & \textbf{$\pmb{f}$} & \textbf{$\pmb{\ldots}$} \\ \hline
		\cellcolor[HTML]{9B9B9B}\textbf{$\pmb{l}$} & $0$ & $1$ & $2$ & $3$ & $\ldots$ \\ \hline
		\cellcolor[HTML]{9B9B9B}\textbf{$\pmb{j}$} & $\dfrac{1}{2}$ & $\dfrac{1}{2},\dfrac{3}{2}$ & $\dfrac{3}{2},\dfrac{5}{2}$ & $\dfrac{5}{2},\dfrac{7}{2}$ & $\ldots$ \\ \hline
		\multicolumn{1}{|l|}{\cellcolor[HTML]{9B9B9B}\textbf{Notation:}} & $s_{1/2}$ & $p_{1/2},p_{3/2}$ & $d_{3/2},d_{5/2}$ & $f_{5/2},f_{7/2}$ & $\ldots$ \\ \hline
		\end{tabular}
	\end{table}
	Therefore for the ground state $3^2s_{1/2}$ of the Sodium we get $l=0$ and $j=1/2$. And in the absence of magnetic field $B=0$ we have:
	
	so there is no fine structure of the Sodium spectrum.\\
	
	But when the electron move to $3p$ under the influence of a weak magnetic field, we have (still look to the table above) that $l=1$ with $j=1/2$ and $j=3/2$. \\
	
	Therefore a fine structure appears for $3p$ and we get that the transition $3s\rightarrow 3p$ is split into a doublet that we name for Sodium the "D-lines" and respectively each line is named $\text{D}_1$ and $\text{D}_2$ lines. 
	\begin{figure}[H]
		\centering
		\includegraphics{img/atomistic/sodium_zeeman_effect.jpg}	
		\caption[]{Source: Hyperphysics}
	\end{figure}
	\end{tcolorbox}
	
	\begin{tcolorbox}[colframe=black,colback=white,sharp corners]
	The fact that there is a doublet shows the smaller dependence of the atomic energy levels on the total angular momentum. This effect is called the "\NewTerm{spin-orbit effect}\index{spin-orbit effect}".\\
	
	In fact this doublet are also visible without the presence of an additional externally applied magnetic field, but with a magnetic fields these levels are further split by the magnetic interaction (and therefore more easily measurable), showing dependence of the energies on the $z$-component of the total angular momentum. This splitting gives the Zeeman effect for sodium.\\
	
	Now let us roughly approximate the magnetic field necessary to create this doublet by taking:
	
	Assuming\footnote{In fact even if we we caculate precisely $m_j$ for each orbit we get almost the same result with less than $10\%$ error} same $g_L$ and same $m_j$ we get $B=18$ [T]. This is a very large magnetic field by laboratory standards. Large magnets with dimensions over a meter, used for NMR and ESR experiments, have magnetic fields on the order of a Tesla.\\

	Experimentally (therefore in reality) it was the opposite, the field was applied and the $g_L$ were determined as we get:
	\begin{table}[H]
		\centering
		\begin{tabular}{|l|c|c|c|c|c|}
		\hline
		\rowcolor[HTML]{9B9B9B} 
		\textbf{Term} & \textbf{$\pmb{J}$} & \textbf{$\pmb{L}$} & \textbf{$\pmb{S}$} & Experimental \textbf{$g_L$} & Calculated \textbf{$g_L$}\\ \hline
		\cellcolor[HTML]{9B9B9B}\textbf{$\pmb{3p_{3/2}}$} & $3/2$ & $1$ & $1/2$ & $4/3$ & $4/3$ \\ \hline
		\cellcolor[HTML]{9B9B9B}\textbf{$\pmb{3p_{1/2}}$} & $1/2$ & $1$ & $1/2$ & $2/3$ & $2/3$\\ \hline
		\cellcolor[HTML]{9B9B9B}\textbf{$\pmb{3s_{1/2}}$} & $1/2$ & $0$ & $1/2$ & $2$ & $2$ \\ \hline
		\end{tabular}
		\caption{$g_L$ Lande' $g$-factor for Sodium $\mathrm{Na}$}
	\end{table}
	where $g_L$ was calculated using:
	
	\end{tcolorbox}
	
	\subsubsection{Lamb Shift}
	But this is not the end with fine structure...
	
	Indeed, if we apply what we have seen so far to for Hydrogen when the electron goes to $n=1$ to $n=2$ (without magnetic field but using ad hoc excitation!), we know that for the layers $s$ and $p$:
	
	and therefore we get with the Bohr model, and the Dirac Model and in laboratories:
	\begin{figure}[H]
		\centering
		\includegraphics{img/atomistic/introducing_qed.jpg}
	\end{figure}
	Or in more technical and traditional view:
	\begin{figure}[H]
		\centering
		\includegraphics{img/atomistic/lamb_shift_hydrogen.jpg}
	\end{figure}
	So here there is an issue between theory and experiment because in labs it is clear that $2^2s_{1/2}$ and $2^2p_{1/2}$ are not degenerated (that means in other words, for recall, they don't have the same energy level) when theory give us clearly that they should be degenerated (ie have the same energy levels)!

	Indeed, let us recall that for $j=1/2$ we have:
	
	So all the possible authorized pairs that lead to this result are:
	
	Therefore we have indeed that $s_{1/2}$ and $p_{1/2}$ that should be degenerated but that are not experimentally whatever the level $n$. 

	Indeed, the (energy) degeneration doesn't come in this case from the term $(\vec{L}+2\vec{S})\circ\vec{B}$ as $\vec{B}\cong \vec{0}$ it the Lamb experiment. We must just apply $\vec{J}=\vec{L}+\vec{S}$ it this case!
	
	The (experimentally) measured positions show the $2s^2p_{1/2}$ level is shifted upwards relative to the position $(n=2,j=1/2)$ and is therefore not degenerate with the $2p^2p_{1/2}$ level. Such a shift occurs for all the $s$-electrons in laboratories (but the energy shift decreases with increasing $n$). 
	
	The most right splitting can only be explained by Quantum Electrodynamics (QED) and is know as the "\NewTerm{Lamb shift}\index{Lamb shift}" of the $s$ level and denoted "\NewTerm{Ly-$\alpha$}" in the special case of hydrogen. Indeed, according to the Dirac equation, these both sates should have the same energy. This shift was first observed in 1947-1952 (Lamb–Retherford experiment\footnote{They carried out the experiment using microwave techniques to stimulate radio-frequency transitions between the both level of interest. The energy difference Lamb and Retherford found was a rise of about $1000$ [MHz]}) and was crucial for the development of QED where the electromagnetic field is quantised and has a zero point energy (this means that even in the case of number of photons equal to zero, there is a fluctuating electric field). In 1948, Theodore A. Welton succeeded (reasonably after a lot of assumptions and approximations...) in accounting for the Lamb shift between the two above levels in terms of the perturbation of the electronic orbit brought about by vacuum fluctuations.

	Due to this fluctuating field the electron performs small oscillations and its charge is effectively smeared out over a range of about $0.1$ fermi. This causes the electron spin g-factor to be slightly different from $2$. There is also a slight weakening of the force on the electron when it is very close to the nucleus, causing the $2s_{1/2}$ electron (which has penetration all the way to the nucleus) to be slightly higher in energy than the $2p_{1/2}$ electron.
		
	When the Lamb shift was experimentally determined, it provided a high precision verification of theoretical calculations made with the quantum theory of electrodynamics. These calculations predicted that electrons continually exchanged photons, this being the mechanism by which the electromagnetic force acted. The effect of the continuous emission and absorption of photons on the electron g-factor could be calculated with great precision.

	The tiny Lamb shift, measured with great precision, agreed to many decimal places with the calculated result from quantum electrodynamics. The measured precision gives us the electron spin $g$-factor as:
	
	 The Lamb shift has since played a significant role through vacuum energy fluctuations in theoretical prediction of Hawking radiation from Black Holes.
	
	
	\begin{flushright}
	\begin{tabular}{l c}
	\circled{90} & \pbox{20cm}{\score{4}{5} \\ {\tiny 4 votes,  74.39\%}} 
	\end{tabular} 
	\end{flushright}

	%to make section start on odd page
	\newpage
	\thispagestyle{empty}
	\mbox{}
	\section{Nuclear Physics}\label{nuclear physics}
	\lettrine[lines=4]{\color{BrickRed}N}uclear physics is the field of physics that studies the constituents and interactions of atomic nuclei. The most commonly known applications of nuclear physics are nuclear power generation but the research has provided application in many fields, including those in nuclear medicine and magnetic resonance imaging, nuclear weapons, nuclear batteries as betavoltaic devices (we will come back on this subject later), ion implantation in materials engineering, and radiocarbon dating in geology and archaeology.
	
	\subsection{Nuclear Weapon}
	Without wishing to make any confusion, we consider as essential, at the time when nuclear weapon serves as a dissuasion vector and thus as element of global stability (but also threat of global destruction ...), for the general culture of the engineer to know some basic properties of the atomic fission bomb. We'll exceptionally in this small subsection without mathematics (mathematical developments of nuclear weapons and nuclear power plants will be presented during our study of neutronic further below) speak a little bit of this mass destruction weapon that often fascinates the young students.
	
	For sure, we will study later how theoretically cause a divergent chain reaction in a given closed constrained volume. However, the reader must not expect to gain the knowledge necessary to manufacture such a weapon since it did not appeal only to knowledge of physics, but also electronics, mechanics, chemistry, mathematics, etc.
	
	About an explosion, usage has immediately propagate to compare the energy of a nuclear weapon than a well know explosive material: the Trinitrotoluene (TNT). This TN provides $4,200,000$ joules per kilo, but the energies of nuclear weapons are such high that it is more meaningful to evaluate them in thousands of tonnes - or kilotons of TNT (subsequently thermonuclear weapons represented a new leap in explosive energy: the practical unity of comparison is one million tonnes - megaton of TNT).
	
	The fission of $56$ grams of Uranium 235 or Plutonium 239 gives the equivalent of $1$ kiloton with the fission of $145^{21}$ atoms (this is not even an integer value of Avogadro's number !!).
	
	The first know nuclear test explosion at Alamogordo in July 16, 1945 - was estimated at $20$ kt, with good accuracy because it had been possible to set up multiple measuring devices.
	
	Those of August 6, 1945 on Hiroshima (Uranium 235) and August 9 on Nagasaki (Plutonium 239) were first estimated as $20$ kt. Subsequently, and after a detailed study on the effects of the blast, their energies were respectively reduced to about $17$ to $15$ kt respectively. This, however, did not represented less of the equivalent of a load of TNT of a convoy of around $6,000$ trucks from the US Army.
	
	Here is a very good diagram (sad I found it only in French) both interesting and persuasive on the effects of an atomic bomb (for information from a speed of $220$ [km/h] an average human being is lifted from the ground):
	\begin{figure}[H]
		\begin{center}
		\includegraphics{img/atomistic/nuclear_bomb_range.jpg}
		\end{center}	
		\caption[Effects of fission weapon of $1$ Mt as a function of the distance]{Effects of fission weapon of $1$ Mt as a function of the distance (source: Pour la science)}
	\end{figure}
	So in other words here is a summary and approximately the effects of a 1 Mt fission weapon exploding at 2,450 meters of altitude (knowing that today the Americans and the Russians have nuclear fusion weapons with firepower exceeding the 50 megatonnes...):
	
	Up to $1.3$ [km], everything is down, even reinforced buildings. Up to $4.8$ [km], most factories and commercial buildings are destroyed; houses made of brick and wood are blow up, and debris scattered. Up to $7$ [km], commercial sets of light structure and private residences are demolished. Heavier buildings are seriously damaged. Up to $9.5$ [km], the walls of small vessels are reversed, private residences severely deteriorated. The breath (or pressure) is still powerful enough to kill the people who are outside (explosion of the lungs). Up to $18.6$ [km], different buildings are damaged, the streets are littered with debris and glass tiles. Between $10$ to $20$ minutes after the explosion, the debris drawn into the depression of the stem of the mushroom, fall to the ground. Follow after $1$ to $2$ hours later, the debris being in the head of the mushroom.
	
	Most of the effects shown in the diagram above are not proportional to the energy. There is therefore no need to make a simple multiplication for a weapon with multiple power that used for the above figure!
	
	For people interested in mushroom size (...) here is a figure whose source in unknown:
	\begin{figure}[H]
		\begin{center}
		\includegraphics[scale=0.75]{img/atomistic/nuclear_bomb_mushroom.jpg}
		\end{center}	
		\caption{Mushroom size for Little Boy Hiroshima (1) and Castle Bravo (2)}
	\end{figure}
	\begin{tcolorbox}[title=Remark,colframe=black,arc=10pt]
	For a nice little calculation on nuclear bombs using dimensional analysis the reader can refer to section Principles of Mechanics where we give the expression of the energy of a bomb according to the radius of the fireball.
	\end{tcolorbox}
	
	Before we start the mathematical part, we emphasize and reiterate that we will limit ourselves only to theoretical developments made between 1890 and about the years 1935 (beyond the complexity of theories requires too many pages to a general book like this one).
	
	\pagebreak
	\subsection{Radioactivity}
	When we experimentally analyze radioactivity, we first perceive that the nucleus does not emits its constituents. Then we discover new forces, that struggles and dominate in turn. Finally, new particulate matter, and even antimatter particles appear. Deciphering these puzzles provided a coherent picture of the infinitely small world that radioactivity has revealed, a world where physical laws escape to the common intuition of daily practice of our macroscopic world.
	
	From the start, the radioactivity has surprised. From the years 1900 it was known that radiation from uranium and his descendants had three components, named "alpha" $(\alpha)$, "beta" $(\beta)$, and "gamma" $(\gamma)$ separable by the action of a magnetic field as shown symbolically in the image below:
	\begin{figure}[H]
		\begin{center}
		\includegraphics[scale=0.85]{img/atomistic/radition_separation.jpg}
		\end{center}	
		\caption[Separation of radiation using a magnetic field]{Separation of radiation using a magnetic field (source: Pour la Science)}
	\end{figure}
	Later, it was shown that the alpha radioactivity was the emission of a helium nuclei (2 protons and two neutrons), beta radiation the emission of electrons. From these observations, it was logical to assume that the nucleus was formed of these three types of particles (protons, neutrons and electrons), which is not the case: the  components of nucleus seems to have been discovered by J. Chadwick in 1932.
	\begin{figure}[H]
		\begin{center}
		\includegraphics[scale=0.7]{img/atomistic/radiation_penetration.jpg}
		\end{center}	
		\caption{Typical radiation penetration}
	\end{figure}
	
	So why do radioactive nuclei not emit protons or neutrons? How do they eject nuclei other than their constituents? These issues must be preceded by another, perhaps more fundamental, why some nuclei are radioactive? The response is the same for all spontaneous physical phenomena. The apple falling from the tree, for example: it is because the system can reach a more stable state by losing potential energy, the excess energy escaping in the form of kinetic energy, i.e. in the form of movement.
	
	This reason also explains why the isotopes emit no protons or neutrons alone because often at the quantum structure of the nucleus, it is more favorable at the energy level to emit a small nucleus or to change a proton into a neutron (the quantum study of the nucleus exceeds the mathematical framework of the topics covered in this book actually).
	
	\textbf{Definitions (\#\mydef):}
	\begin{enumerate}
		\item[D1.] Any chemical element (\SeeChapter{see chapter Chemistry}) is characterized in the chemical point of view by its number of protons $Z$ named "\NewTerm{atomic number}\index{atomic number}".
		
		\item[D2.] The "\NewTerm{mass number}\index{mass number}" $A$ is by definition the number of protons $Z$ summoned by the number of neutrons $N$ for the given chemical element. Thus, the latter is completely characterized if we know its name or atomic number $Z$ and it  number $N$ of neutrons or its mass number $A$. We usually denote any element in the form:
		
		The chemical elements of the same species (same $Z$) may have different numbers of neutrons $N$, that is to say therefore different mass numbers $A$, then we speak of "\NewTerm{isotopes}\index{isotope}\label{isotope}" or "\NewTerm{nuclides}\index{nuclide}" (today this seems obvious as a definition but it took many years of research to get to this concept and we own this particularly to the works of Niels Bohr). The chemical elements which have the same $A$ are named "\NewTerm{isobars}\index{isobars}".
		\begin{figure}[H]
		\centering
		\includegraphics[width=\textwidth]{img/atomistic/chart_of_the_nuclides.jpg}	
		\caption[Table of isotopes]{Table of isotopes (source: ?)}
	\end{figure}
		Obviously, the nuclear energy (of the nucleus) associated with the same chemical element differs depending on the mass number $A$ and we will prove that there is number $A$ for which energy this minimal. Isotopes for which the energy is not a minimum may, for some of them and spontaneously, release the excess energy by doing a form of desintegration.
		
		\item[D3.] The property of certain atoms to spontaneously change the structure of their nuclei to reach a lower energy level, more fundamental, is named "\NewTerm{radioactivity}\index{radioactivity}". We then talk about "\NewTerm{radio-isotopes}\index{radio-isotopes}" for the atoms involved.
		
		The chemical properties of an atom depend (\SeeChapter{see section Corpuscular Quantum Physics page \pageref{electron configuration}}) on the number and arrangement of electrons in the electronic cloud (corresponding to the number of proton $Z$ for non-ionized atoms). Thus all isotopes of the same element have the same chemical properties (it is this chemical characteristic which is the basis for nuclear medicine). These are sort of "brothers" atoms. However, the small difference in mass of the nucleus makes that their physical properties differ somewhat.
		
		\item[D4.] The "\NewTerm{isotones}\index{isotones}" are the isotopes of various chemical elements (different $Z$) with the same number of neutrons $N$.
		
		 Rather than speak of "chemicals elements" when we actually are implicitly referring to the nucleus, the term "\NewTerm{nuclide}\index{nuclide}" is preferred.
		 
		 The smallness of atoms is an obvious problem of mass measurement. That is why it was preferred by physicists and chemists to develop an atomic mass system which is a system of numbers proportional to the real mass of the atoms.
		 
		 Since there are infinitely many possible systems, one was chosen judiciously as a reference and is the number $12$ for the carbon-$12$ isotope (for more details see the section of Analytical Chemistry):
		 
		where "\NewTerm{amu}" is the abbreviation of "\NewTerm{atomic mass units}\index{atomic mass units}".
		
		This choice has for interesting consequence to give to the proton and neutron an atomic weight near to the unit.
		
		We can therefore connect the I.S. system (\SeeChapter{see chapter Principles}) with the system of atomic mass units (amu).
		
		\item[D5.] The "\NewTerm{atomic mass unit}\label{atomic mass unit}" is defined as the mass of the $1/12$ of the atom of carbon ${}^{12}\mathrm{C}$, we have (the mass of the electrons is neglected as very small compared to that of nucleons):
		
		Therefore the uma mass of the proton is equal to:
		
		 for the value allowed the reader is referred to the literature of the ad hoc international norms (because it varies depending on the new versions of international norms).
		 
		However the reader must be careful, though the molar mass of a different isotope than ${}^{12}\mathrm{C}$ by adding the masses of the nucleons (protons and neutrons) that make up its nucleus because we must take into account the mass defect (a concept that we will see further below).
	
		The masses can also be expressed in energy units because there is a mass-energy equivalence as we have prove in the section of Special Relativity from the relation $E=mc^2$. The energy units used in nuclear physics is often the "\NewTerm{electron volt}\index{electron volt}".
	
	  \item[D6.] An "\NewTerm{electron-volt}\index{electron-volt}\label{electron volt}" denoted [eV] is the energy acquired by an elementary charge subject to a potential difference of $1$ [V].
	  
	  Thus, according to the relation between energy and the electrostatic potential $E=qU$ (\SeeChapter{see section Electrostatics page \pageref{electrostatic potential energy}}), we have:
	
	We get from this since the speed of light in vacuum is almost equal to $c=299,792,458\;[\text{ms}^{-1}]$:
	
	  
	 By mass–energy equivalence, the electron-volt is also a unit of mass. It is common in particle physics, where units of mass and energy are often interchanged, to express mass in units of $\text{eV}/c^2$. It is common to simply express mass in terms of "eV" as a unit of mass, effectively using a system of natural units with $c$ set to $1$ (\SeeChapter{see section Principia page \pageref{natural system units}}). The mass equivalent of $1$ $\text{eV}/c^2$ is:
	
	Again, for the accepted value in practice the reader has to refer to the literature of the ad hoc international norms (because it varies depending on the new versions of these norms).
	\begin{tcolorbox}[title=Remark,colframe=black,arc=10pt]
	Practicing astrophysicists and plasma physicists routinely refer to temperatures in units of [eV] or [keV], even thought this is wrong, because temperature is not dimensionally equivalent to energy. Nevertheless, they still do it, with the Boltzmann constant being implicitly included in the conversion. Here are formulas for temperature in Kelvin to [keV] conversion:
	
	because $1$ [eV] is by definition the energy required to move a charge $e$ through a potential difference of $1$ [V] and is equal to $1.6\cdot 10^{-19}$ [J]. Therefore:
	
	\end{tcolorbox}
	\end{enumerate}
	
	\pagebreak
	\subsubsection{Disintegration}
	Some nucleus therefore have the property of spontaneously change their internal structure to achieve a basic level of energy. This transformation is accompanied by the emission of particles and / or electromagnetic radiation. The residual core can be radioactive too and process to other transformations afterwards or be stable.

	The radioactive decay of an isotope is a random phenomenon and we can never tell when a nucleus will disintegrate (probability no memory effect).
	
	\begin{tcolorbox}[title=Remark,colframe=black,arc=10pt]
	To prove this assertion, the reader may refer to the section of Quantitative Management in the subsection of Queuing Theory, in particular modeling arrivals. Indeed, the development is identical in all point but only the object of study changes (that are no longer phone calls but disintegration). Thus, we proved that under certain assumptions the phenomenon follows a Poisson law and we we prove a little bit afterwards that it has no memory!
	\end{tcolorbox}
	We can only give the probability of disintegration (decay) per unit time. This probability is given by the "\NewTerm{disintegration constant}\index{isintegration constant}" and has for unit the inverse of time: $\lambda\; [\text{s}^{-1}]$. This constant can be calculated as we have already prove it during our the study of the tunnel effect in the section of Wave Quantum Physics.

	The radioactive constant varies for all known isotopes:
	
	Given now $N (t)$ the initial stock of atoms of a radioactive isotope at time $t$. The number of atoms disintegrating during the infinitesimal time $\mathrm{d}t$ is equal to:
	
	leading to a decrease in the stock obviously equal to:
	
	The decay differential equation (\SeeChapter{see section of Differential and Integral Calculus page \pageref{first order lde with constant coefficients}}) is thus written:
	
	or:
	
	which has the simple solution (\SeeChapter{see section of Differential and Integral Calculus page \pageref{first order lde with constant coefficients}}):
	
	with $N_0=N(t_0)$ being the initial stock of nucleus at time $t_0$. 
	
	\begin{tcolorbox}[title=Remark,colframe=black,arc=10pt]
	$N(t)$ does not represent the number of atoms remaining at time $t$ but the most likely number of radioactive atoms remaining at time $t$ !!
	\end{tcolorbox}
	
	In the practice, the measurement of the decay constant is done thanks the decrease of the activity (see below) of the individual isotope.
	
	\paragraph{Half-life isotope}\mbox{}\\\\\
	\textbf{Definition (\#\mydef):} The "\NewTerm{half-life period}\index{half-life period}" or just "\NewTerm{half-life}\index{half-life}" $T_{1/2}$ of an isotope is the average time that it takes for $50\%$ of the radioactive nuclei stock of a given isotope to disintegrate:
	
	Thus we have a very important relation between the half-life period and the decay constant!

	If the radioisotope has the choice to disintegrate in two separate path of disintegrations characterized by two distinct radioactive periods $T_{1/2}^1$ and $T_{1/2}^2$, the half-life of this nuclide is defined as the mean:
	
	and we calculate the number of remaining nuclides by the relation:
	
	
	\pagebreak
	\subsubsection{Activity}
	\textbf{Definition (\#\mydef):} The "\NewTerm{activity}\index{activity (nuclear)}" of a radioactive source is the number of disintegrations per time unit.
	\begin{tcolorbox}[title=Remark,colframe=black,arc=10pt]
	Its unit of measurement is the "\NewTerm{Becquerel}\index{Becquerel}" denoted [A]$=$[Bq]. One Becquerel thus corresponding to one disintegration per second.
	\end{tcolorbox}
	The old unit of measurement of radioactivity was the "\NewTerm{Curie}\index{Curie}" denoted [Ci]. The Curie was defined initially as the activity of about one gram of radium, natural element that we find in the soil with uranium. This unit is much larger than the previous one because, by definition $1$ [Ci] is equal to 37 billion disintegrations per second:
	
	The activity is obtained by the time derivative of the stock atoms of a given sample:
	
	The relation, named "\NewTerm{equation of activity}\index{equation of activity}":
	
	thus shows that the activity of a given number of $N$ atoms of a radioactive isotope is proportional to that number and inversely proportional to the half-life of the isotope (by the relation seen above between the radioactive constant and the half-life period).
	\begin{tcolorbox}[colframe=black,colback=white,sharp corners]
	\textbf{{\Large \ding{45}}Example:}\\\\
	One gram of ${^{225}\mathrm{Ra}}$ contains (the reader must regularly refer to the common international standards for the values of the constants used!):
	
	Therefore the activity of this gram is, knowing experimentally that $T_{1/2}=1,600\;[\text{years}]$, equal to:
	
	\end{tcolorbox}
	By the same reasoning (that we can detailed on reader request as always in this book), we show that the activity over time follows the same exponential law of the number of nuclides:
	
	with:
	
	Experimentally to determine the half-life period of an isotope we proceed as following:
	\begin{enumerate}
		\item We choose a pure as possible sample of an isotope which we wish to determine the half-life period.

		\item 	At the time $t_0$ we measure thanks to a detector for a fixed time interval time $\Delta t$ the number of disintegrations. We then have the number of decays during a time interval at the start of experiment (the unit of measurement is then decays and not the number of disintegrations per second).

		\item Then, for each consecutive $\Delta t$ (time interval is fixed) we measure the number of disintegrations during this time interval. This gives us a series of measures the number of decays observed for $\Delta t_1, \Delta t_2, \Delta t_3,\ldots$.
		
		\item To the set of all decay measurements, we remove the laboratory background noise (as the whole environment is radioactive).
	\end{enumerate}
	As:
	
	By taking the neperian logarithm we get:
	
	Thus:
	
	So this is the equation of a staright line of slope $-\lambda$ and intercept $\ln(A(t_0)\Delta t)$. Thus, the decay constant is immediately measured and we quickly deduced the period of half-life using the relation proved earlier above:
	
	
	\pagebreak
	\paragraph{Carbon-14 dating (radiocarbon dating)}\mbox{}\\\\\
	Radiocarbon dating (also referred to as carbon dating or carbon-14 dating) is a method for determining the age of an object containing organic material by using the properties of radiocarbon $^{14}\mathrm{C}$, a radioactive isotope of carbon.
	
	The method was developed by Willard Libby in the late 1940s and soon became a standard tool for archaeologists. Libby received the Nobel Prize for his work in 1960. The radiocarbon dating method is based on the fact that radiocarbon is constantly being created in the atmosphere by the interaction of cosmic rays with atmospheric nitrogen. The resulting radiocarbon combines with atmospheric oxygen to form radioactive carbon dioxide, which is incorporated into plants by photosynthesis; animals then acquire $^{14}\mathrm{C}$ by eating the plants. When the animal or plant dies, it stops exchanging carbon with its environment, and from that point onwards the amount of $^{14}\mathrm{C}$ it contains begins to decrease as the $^{14}\mathrm{C}$ undergoes radioactive decay. Measuring the amount of $^{14}\mathrm{C}$ in a sample from a dead plant or animal such as piece of wood or a fragment of bone provides information that can be used to calculate when the animal or plant died. The older a sample is, the less $^{14}\mathrm{C}$ there is to be detected, and because the half-life of $^{14}\mathrm{C}$ (the period of time after which half of a given sample will have decayed) is about $5,730$ years, the oldest dates that can be reliably measured by radiocarbon dating are around $50,000$ years ago, although special preparation methods occasionally permit dating of older samples.

	The idea behind radiocarbon dating is straightforward, but years of work were required to develop the technique to the point where accurate dates could be obtained. Research has been ongoing since the 1960s to determine what the proportion of $^{14}\mathrm{C}$ in the atmosphere has been over the past fifty thousand years. The resulting data, in the form of a calibration curve, is now used to convert a given measurement of radiocarbon in a sample into an estimate of the sample's calendar age. Other corrections must be made to account for the proportion of $^{14}\mathrm{C}$ in different types of organisms (fractionation), and the varying levels of $^{14}\mathrm{C}$ throughout the biosphere (reservoir effects). Additional complications come from the burning of fossil fuels such as coal and oil, and from the above-ground nuclear tests done in the 1950s and 1960s. Because the time it takes to convert biological materials to fossil fuels is substantially longer than the time it takes for its $^{14}\mathrm{C}$ to decay below detectable levels, they contain almost no $^{14}\mathrm{C}$, and as a result there was a noticeable drop in the proportion of $^{14}\mathrm{C}$ in the atmosphere beginning in the late 19th century. Conversely, nuclear testing increased the amount of $^{14}\mathrm{C}$ in the atmosphere, which attained a maximum in 1963 of almost twice what it had been before the testing began.
	
	Natural carbon has two stable isotopes: the $^{12}\mathrm{C}$ ($98.892\%$) and $^{13}\mathrm{C}$ ($1.108\%$). So there is no $^{14}C$ in the natural carbon. The latter is produced in the upper atmosphere through the action of cosmic neutrons on the $^{13}\mathrm{C}$ . We speak then of "\NewTerm{neutron capture}\index{neutron capture}" (see further below) or "\NewTerm{activation $^{13}\mathrm{C}(n,\gamma) ^{14}\mathrm{C}$}" following:
	
	where $n$ represents a neutron and $p$ represents a proton. 
	
	Thus, continuously $^{14} \mathrm{C}$ is produced in the upper atmosphere and decays naturally with a period of $5,730$ years. We easily imagine that the concentration in $^{14}\mathrm{C}$ is balanced when the production rate is equal to the disappearing rate due to radioactive decay process (otherwise they will no longer be any $^{14}\mathrm{C}$ everywhere today) following:
	
	By emitting a beta particle (an electron, $e^{-}$) and an electron antineutrino ($\nu_e$), one of the neutrons in the $^{14}\mathrm{C}$ nucleus changes to a proton and the $^{14}\mathrm{C}$  nucleus reverts to the stable (non-radioactive) isotope $^{14}\mathrm{N}$.
	
	\begin{figure}[H]
		\begin{center}
		\includegraphics[scale=0.6]{img/atomistic/carbon_dating.jpg}
		\end{center}	
		\caption[Idea behind Carbon dating]{Idea behind Carbon dating (source: ?)}
	\end{figure}
	
	About $2.5$ atoms of carbon $^{14}\mathrm{C}$ are forme every second by square centimeter on the Earth's surface (this value is however dependent on many factors but with negligible variations over the very long term - you can find entire books on this subject), the positive contribution to the number of atoms of  $^{14}\mathrm{C}$ is about:
	
	$R$ being the Earth's radius.
	
	Or in mass flow it represents:
	
	The desintegration rate is supposed equal to the rate of radioactive production, that is to say:
	
	As the global quantity:
	
	As the disintegration rate is equal to:
	
	We conclude that there are $N\cong 3.32\cdot 10^{30}$ atoms of $^{14}\mathrm{C}$. The reader can calculate by multiplying by the mass value of the  $^{14}\mathrm{C}$ atom ($2.34\cdot 10^{-26}$ [kg]) that it makes quite a significant mass in the atmosphere.
	
	As we already said it, this radioisotope is found in the chemical form $\mathrm{CO}_2$ and penetrates through photosynthesis and metabolism in plants and animals. On the death of the plant or animal, the quantity of $^{14}\mathrm{C}$ freezes and starts to decrease through radioactive decay over the ages:
	
	Radiocarbon dating is therefore a simple comparison between the concentration in  $^{14}\mathrm{C}$ of living matter and dead matter of two same objects/living entity. In fact, we determine the specific activities:
	
	where for objects "living entity" can be translate in "nowadays comparable element".
	
	That is to say the "\NewTerm{radiocarbon age}\index{radiocarbon age}" is given by:
	
	with for $^{14}\mathrm{C}$, $\dfrac{T_{1/2}}{\ln(2)}\cong 8267$.
	
	Therefore the latter relation is often written:
	
	and even more shortly:
	
	where $F_m$ is the mass fraction notation.
	
	\begin{tcolorbox}[title=Remark,colframe=black,arc=10pt]
	Measurement of radiocarbon was originally done by beta-counting devices, which counted the amount of beta radiation emitted by decaying $^14C$ atoms in a sample. More recently, accelerator mass spectrometry has become the method of choice; it counts all the $^{14}\mathrm{C}$ atoms in the sample and not just the few that happen to decay during the measurements; it can therefore be used with much smaller samples (as small as individual plant seeds), and gives results much more quickly.
	\end{tcolorbox}
	The reader must keep in mind that two hypothesis (assumptions) have been implicitly done for the calculations above:
	\begin{enumerate}
		\item[H1.] The amounts of parent and daughter isotopes at the beginning, when the rock formed, must know (the initial conditions) or their incertituded impacted on the resulting calculations results.

		\item[H2.] All daughter atoms measured today must only have been derived by in situ radioactive decay of parent atoms (a closed system) in the opposite case the external sources must be taken into account in the incertitude value of the resulting calculations results.
		
		\item[H3.] The radioisotope decay rates are constant and if not the variations must be taken into account in the incertitude value of the resulting calculations results.
	\end{enumerate}
	\begin{tcolorbox}[title=Remark,colframe=black,arc=10pt]
	The oldest dated rocks on Earth, as an aggregate of minerals that have not been subsequently broken down by erosion or melted, are more than $4$ billion years old, formed during the Hadean Eon of Earth's geological history. Such rocks are exposed on the Earth's surface in very few places. (Some meteorites, such as the ALH84001 Mars meteorite found in the Allan Hills of Antarctica, are older but they were not formed on Earth). Some of the oldest surface rock can be found in the Canadian Shield, Australia, Africa and in a few other old regions around the world. The ages of these felsic rocks are generally between $2.5$ and $3.8$ billion years. The approximate ages have a margin of error of millions of years. In 1999, the oldest known rock on Earth was dated to $4.031 \pm 0.003$ billion years, and is part of the Acasta Gneiss of the Slave craton in northwestern Canada.
	\end{tcolorbox}	
	
	\pagebreak
	\paragraph{Radioactive Decay chain}\mbox{}\\\\\
	 \textbf{Definition (\#\mydef):} A "\NewTerm{decay chain}\index{decay chain}" refers to the radioactive decay of different discrete radioactive decay products as a chained series of transformations. They are also known as "\NewTerm{radioactive cascades}\index{radioactive cascades}". Most radioisotopes do not decay directly to a stable state, but rather undergo a series of decays until eventually a stable isotope is reached.

	Decay stages are referred to by their relationship to previous or subsequent stages. A "\NewTerm{parent isotope} \index{parent isotope}" is one that undergoes decay to form a "\NewTerm{daughter isotope}\index{daughter isotope}". One example of this is uranium (atomic number 92) decaying into thorium (atomic number 90). The daughter isotope may be stable or it may decay to form a daughter isotope of its own. The daughter of a daughter isotope is sometimes named a "granddaughter isotope".
	
	\begin{figure}[H]
		\begin{center}
		\includegraphics{img/atomistic/uranium_decay_series.jpg}
		\end{center}	
		\caption[Uranium decay series]{Uranium decay series (source: Wikipedia)}
	\end{figure}
	
	In general we denote a radioactive cascade by:
	
	where the small symbol $^{*}$ denotes a given radioactive isotope, $\mathrm{X}_n$ being the final stable isotope of the radioactive of the mother element $\mathrm{X}_1^{*}$ (the elements between all being unstable nuclides).
	
	\subsubsection{Two level radioactive cascade}
	Let us consider the two nuclide situation (we will not focus on higher order level excepted on request of many readers). 

	Let us suppose that at the beginning of time, the first descendant exists only negligible quantity such that we have for initial conditions (IC) at $t=0$:
	
	The variation of the parent nuclide (1) is given by a negative contribution: the disintegration of (1).

	We have:
	
	having for solution taking into account the initial conditions:
	
	
	The variation of the descendant member (2), that is to say the daughter of (1), is given by a positive contribution (the atoms of (1) disintegrated) and one negative, the disintegration of (2). Then:
	
	So we have to solve this differential equation (\SeeChapter{see section of Differential and Integral Calculus page \pageref{first order lde with constant coefficients}}.
	
	We have for homogeneous solution (characteristic equation):
	
	We derive from the homogeneous equation the special (homogenous) solution:
	
	with the letter $h$ to signify that this is the "homogeneous" solution.

	Let us now determine the particular solution of:
	
	The idea to solve this is to put:
	
	with the letter $p$ for "particular" (implicitly: "special solution"). Substituting we find:
	
	Because if we had $\lambda_p=\lambda_2$ we would have a zero equality which is absurd and we have therefore:
	
	from which we derive that:
	
	Finally the general solution is the sum of the homogeneous solution and the particular, therefore:
	
	Let us apply the initial conditions:
	
	Finally we have:
	
	We will leave to the reader to plot the functions:
	
	to see how it looks like if he feels the need.
	
	$N_2(t)$ being zero for $t=0$ and for $t=+\infty$, it must pass, as the for activity $A_2(t)$, through a maximum. Given $t^{*}$ the time when the maximum is observed, we have:
	
	from where:
	
	The knowledge of $t^{*}$ is important especially in nuclear medicine where we want to administer the product $1$ for radio-diagnostic purposes  and minimize adverse effects of the daughter(s) element(s) of $1$. We then choose elements such that the time $t^{*}$ is greater than the biological elimination time (body's natural elimination pathways) of its daughter(s9 elements(s). We will return on this subject in a few paragraphs after studying three typical scenarios of radioactive filiation:
	
	\paragraph{Secular equilibrium}\mbox{}\\\\\
	This type of relation between mother and daughter activities occurs when the mother nucleus half-life is much greater than that of the daughter nucleus half-life. In other words, the radioactive decay of the mother nucleus is negligible and the activity of the descendent tends toward that of the parent.

	We then have in this particular case:
	
	So we have for the activity using the previously proved relation:
	
	Therefore:
	
	We also see that in the case where $\lambda_1\ll\lambda_2$ and $t\rightarrow +\infty$, we have:
	
	in other words, the mother and daughter activities, become equivalent after a large enough time. For example, after a period of a half-life of the daughter isotope, we already have the daughter activity that is $50\%$ that of the parent activity:
	
	If we have the case where the following approximation is acceptable:
	
	we will have:
	
	
	\pagebreak
	\paragraph{Transient equilibrium}\mbox{}\\\\\
	This type of relation between mother and daughter activity occurs when the half-life of the mother nucleus is greater than that of the daughter nucleus (but not much much greater in contrast to the secular equilibrium). In other words, the decay radioactivity of the mother nucleus and the activity of the descendants are equal to a constant factor (in other words, their radioactive decay curves are parallel after a sufficiently long time).

	We then in this particular case:
	
	So we have for the activity using the previously prove relation:
	
	After a long enough time, we get:
	
	where we see that the factor:
	
	is greater than unity. So after a sufficiently long time, not only the activity of the daughter isotope is parallel to that of the mother but in addition it is superior to that latter.
	
	\paragraph{Nonequilibrium}\mbox{}\\\\\
	Here the half life time of the daughter element is greater than that of the parent element. In other words we have the hypothesis:
	
	This implies for recall that the probability of decay of the parent element $1$ is greater than that of the daughter element $2$ which is logical here.

	The activity of the daughter isotope in the sample then increases according to the relation proved just above:
	
	Finally, after a sufficiently long time, only the activity of the daughter element will remain, since the activity of the parent element disappears at a higher rate following:
	
	After a time $t_{\max}$, the activity of the daugther element reach a maximum value for:
	
	Therefore:
	
	This simplifies to:
	
	It then comes immediately the result already proved in the example earlier above:
	
	Finally, let us consider the extreme case of the situation of non-equilibrium consisting to consider the case where:
	
	In other words the daughter element is not radioactive. We then fall back on the classic case:
	
	
	\subsubsection{Radioactive phenomena}
	When we "weigh" a nucleus, we experimentally observe a very important fact !: its mass is less than the sum of the masses of its constituents. This difference is named the "\NewTerm{mass defect}\index{mass defect}" and is relatively well determined with simplistic theoretical models.

	The mass defect is given by construction by:
	
	with $m_\text{nucleus}$ being the mass of the nucleus in its fundamental state, $m_p$ the proton mass and $m_n$  the neutron mass.

	The mass of a set of linked nucleons is less than the sum of the masses of individual nucleons (far enough of each other to not interact). We get from Special Relativity (see section of the same name page \pageref{mass energy equivalence}):
	
	where $E_\text{bound}^\text{nucleons}$ is the binding energy of the nucleons that compose the nucleus ($> 0$).
	
	$\Delta m$ is therefore positive for all the elements (emission of energy and therefore of mass to the outer system). If this were not the case, the nucleons have no reason to come together naturally to form stable nuclei (or more stable ...).
	
	Given $\bar{B}=f(A,Z)$ the average energy by nucleon of a given atom. We have then:
	
	which is by convention a positive value!

	Notice that the mass of the nucleus is linked to the mass of the atom by:
	
	Similarly, the nucleus mass added to the weight of its isolated electrons is higher than that of the nucleus surrounded by its linked electrons. Let us notice that the electronic binding energy can often neglected relatively to that of the nucleus and it is a rule we will adopt throughout this section.

	The energy released during the nuclear fusion, that is to say during the constitution of the atom from its constituents, also named "\NewTerm{binding energy}\index{binding energy}" (name that often makes  interpretations problems to young students) because it is that energy must be given if we want at the opposite, separate the elements. We must never forget that behind the term "binding energy", so there is the energy variation between individual elements and combined elements of an atomic element.
	
	The general practical expression of the average energy per nucleon of a given atom expressed in atomic mass units is:
	
	A numerical application for Helium ($\alpha$ particle) gives ($Z=N=2$, $A=4$) we get $\bar{B}\cong 7.07$ [MeV/nucleon].
	
	The principles of nuclear energy production from fission or fusion result of the shape of the average energy per nucleon according to $A$.

	We have in reality the following curve linking the average energy per nucleon (that is to say, the average energy variation between the nucleon alone and accompanied...) and the number of nucleons named "\NewTerm{Aston curve}\index{Aston curve}":
	\begin{figure}[H]
		\centering
		\includegraphics[scale=1]{img/atomistic/aston_curve.jpg}
		\caption{Aston Curve}
	\end{figure}
	where we see that from the Iron (element which is therefore the most "glued" and the most stable in energy terms because having the highest average bonding energy) the average energy decreases again. This decrease being due to the fact that from about $70$ nucleons it seems that appear an electrostatic force inside the nucleus that begins to take over another force that rules in the very small nuclei (this force will be named later the "\NewTerm{strong force}\index{stron force}" or "\NewTerm{strong interaction}\index{strong interaction}").

	By the way, what is really important to notice in the chart above is that there is a flexion point and it is this one that makes it possible to obtain energy both with the fusion, only with nuclear fission! We also see that the variation is much greater on the left than on the right, hence the fact that fusion releases much more energies.

	Of all radioactivity phenomena, we can distinguish $8$ of them, some of them are named "secondary" because they are only the possible side effects of the first $6$. Some of these phenomena are caused by man, others are natural and others are purely probabilistic:
	\begin{figure}[H]
		\centering
		\includegraphics[scale=0.19]{img/atomistic/radio_decay_types_2.jpg}
		\caption[Summary of the type, nuclear equation, representation, and any changes in the mass or atomic numbers for various types of decay]{This table summarizes the type, nuclear equation, representation, and any changes in the mass or atomic numbers for various types of decay (source: chemwiki.ucdavis.edu)}
	\end{figure}
	Here is a diagram representing at the top the "\NewTerm{valley of stability}\index{valley of stability}\label{valley of stability}" of atoms and isotopes and down the same valley but highlighting the type of disintegration:
	\begin{figure}[H]
		\centering
		\includegraphics[scale=0.75]{img/atomistic/stability_valley.jpg}
		\caption[Nuclear Stability Valley]{Nuclear Stability Valley (source: ?)}
	\end{figure}
	Let us see then the types of disintegration or modifications of the structure of the atom / nucleus which are possible in this details:
	
	\paragraph{Nuclear Fusion (1)}\label{nuclear fusion}\mbox{}\\\\\
	If we assemble two light nuclei $_{Z_X}^{A_x}\mathrm{X}$ and $_{Z_Y}^{A_Y}\mathrm{Y}$ (with $A_x \le A_y$) to form a "heavy" atom $_{Z_Z}^{A_Z}\mathrm{Z}$, then in accordance with the left part of the Aston curve seen above, we increase the mass defect since the average energy per nucleon increases. Indeed:
	\begin{itemize}
		\item The energy of $\mathrm{X}$ is equal to:
		
	
		\item The energy of $\mathrm{Y}$ is equal to:
		
		
		\item Then energy of $\mathrm{Z}$ is equal to:
		
	\end{itemize}
	As:
	
	then:
	
	is strictly positive.
	
	Nuclear fusion is almost exclusively caused by man (on Earth at least... because the stars do it alone with the help of gravity). The probability of observing a natural nuclear fusion under normal conditions of temperature and pressure is so low that it is unnecessary to speak of it at least to this day...
	
	The three most common method for achieving fusion are so far:
	\begin{itemize}
		\item Thermonuclear fusion (heated gas under the form of a plasma in analogy with what happens in Stars)
		\item Inertial confinement fusion (compression of a fuel target with LASER)
		\item Inertial electrostatic confinement (use of electrif field to heat ions to fusion conditions)
	\end{itemize}
	For more information about plasma the reader can reefer to the corresponding section in this books.
	\begin{figure}[H]
		\centering
		\includegraphics[width=1.0\textwidth]{img/atomistic/jet.jpg}
		\caption[Joint European Torus (JET) tokamak fusion detector]{Joint European Torus (JET) tokamak fusion detector (source: EUROfusion)}
	\end{figure}
	
	\paragraph{Nuclear Fission (2)}\mbox{}\\\\\
	Similarly, if we break with adequate tools (often with neutrons because to approach the nucleus and overcome its electrostatic repulsion it is the appropriate way... and this is the method used by nuclear power plant and nuclear bombs) an heavy atom $_{Z_X}^{A_X}\mathrm{X}$ in two light atoms $_{Z_Y}^{A_Y}\mathrm{Y}$ and $_{Z_Z}^{A_Z}\mathrm{Z}$ (with $A_Y\le A_Z$) we also increase the mass defect and the energy gained is therefore equal to:
	
	That it is in the case of fission or fusion, the energy released is then divided into the kinetic energy of the fission products, the neutrons emitted and finally the various radiations.
	\begin{tcolorbox}[title=Remark,colframe=black,arc=10pt]
	An atom is said to be "\NewTerm{fissible}\index{fissible}" (or sometimes "fissionable") when fast neutrons are needed to produce fission and "\NewTerm{fissile}\index{fissile}" when it is enough to have slow neutrons for fission (which is more rare).
	\end{tcolorbox}
	Nuclear energy is by far one of the most concentrated form of energy, since $1$ kilogram of natural Uranium supplies a heat quantity of $100,000$ [kWh] in a current power plant, while $1$ kilogram of coal supplies by burning only $8$ [kWh]. For this reason, only a relatively small amount of nuclear fuel is used for electricity production: a nuclear power plant with an electrical capacity of $1,000$ [MW] consumes $27$ tons of enriched Uranium annually, one quarter of its load, whereas a thermal power plant of the same power consumes $1,500,000$ tons of oil by year... For comparison in the Sun, $1$ kilogram of hydrogen produced, by nuclear reactions transforming it into helium, $180$ million kWh!!!! But beware, industrially we know how to extract only a small part of the nuclear energy stored in the material. Of the $27$ tons of enriched Uranium consumed in one year by a power plant, only a small amount of nucleus was really consumed (hence the economic need to reprocess the Uranium after use).

	We quickly realize that the calorific value of fission is gigantic compared to that of fossil energies. An estimate gives a ratio of energy released per atom of $50,000$ million !!! On the other hand, the risk ratio in terms of exploitation and management of waste and impact on the environment is of the same order but in the opposite direction. That mankind finds either another way of producing electricity, or changes its habits of consumption.

	For information in Switzerland, there are only $5$ at this nuclear power plants (at the beginning of the 21st century) for a population of almost $7$ million individuals in year 2000:
	 \begin{figure}[H]
		\centering
		\includegraphics[scale=0.75]{img/atomistic/swiss_nuclear_plants.jpg}
		\caption[Nuclear power plants in Switzerland at the end of the 20th century]{Nuclear power plants in Switzerland at the end of the 20th century (source: Swiss confederation)}
	\end{figure}
	This little introduction being made, let us return to the purely mathematical part!

	In the case of spontaneous (or natural) fission we have the emission of two fission products and of $w$ neutrons. What we we denote by:
	
	\begin{tcolorbox}[colframe=black,colback=white,sharp corners]
	\textbf{{\Large \ding{45}}Example:}\\\\
	The isotope of Carbon 15 by spontaneous fission gives an isotope of Nitrogen with the emission of an electron and an antineutrino:
	
	\end{tcolorbox}
	\begin{figure}[H]
		\centering
		\includegraphics[width=1.0\textwidth]{img/atomistic/sandia_core_reactor.jpg}
		\caption[]{Sandia National Laboratories' Annular Core Research Reactor (we can the bubbles of the evaporated water on the picture)}
	\end{figure}
	
	\paragraph{Alpha Disintegration (3)}\label{alpha disintegration}\mbox{}\\\\\
	\textbf{Definition (\#\mydef):} When a heavy nucleus contains too many protons and neutrons (like Uranium 238 for example), it will empty its nucleon overflow by emitting an alpha particle (Helium nucleus composed of $2$ protons and $2$ neutrons) and the final system which will be a new nucleus will have a smaller and possibly stable mass. This mode of disintegration is named "\NewTerm{alpha radioactivity}\index{alpha radioactivity}".
	
	The probability of disintegration is governed by the mechanism of penetration barrier ("\NewTerm{Tunnel effect}\index{tunnel effect}") as we will prove it a little further after a small introduction.

	The radioactive decay according to the alpha radioactivity can be schematized as (with some small variations depending on the case):
	
	where:
	
	\begin{tcolorbox}[colframe=black,colback=white,sharp corners]
	\textbf{{\Large \ding{45}}Example:}\\\\
	Radon is found in significant concentrations in underground mines and sometimes in some caves (in some mining or granitic areas):
	
	where Polonium is one of the main factors of induction of so-named "radon-induced lung cancers".
	\end{tcolorbox}
	The energy released during the transmutation is calculated thanks to the mass default:
	
	with $M_X$ being the mass of the initial nucleus, $M_Y$ the mass of the final nucleus and $M_\text{He}$ the mass of the Helium nucleus.

	By neglecting the binding energy of electrons, we have:
	
	Finally:
	
	This expression shows that the energy of $\alpha$ particle is well defined for given initial and final nuclei. In fact, we actually observe a discrete energy spectrum. We conclude that these emissions lead the nucleus to intermediate energy levels corresponding to excited states of the final nucleus. We can explain these observations by a layered nuclear structure!!! The de-excitation of the final is done by emission of $\gamma$ photons.

	The conservation of energy requires that the energy of the $\alpha$ disintegration is distributed between the kinetic energy of the two residual products:
	
	The conservation of the linear momentum gives us:
	
	and therefore:
	
	which we replace in the energy conservation equation:
	
	and we derive that the energy of the $\alpha$ particle is equal to:
	
	since the masses of the nucleus and the $\alpha$ particle are approximately proportional to their mass number, thus $A$ and $4$ respectively.
	
	Let us see now the details of the mechanism of $\alpha$ disintegration with an academic approach, simplified to the extreme and therefore approximative (but sufficient nevertheless). For this approach, we will use the developments on the tunnel effect that we have carried out in the section of Wave Quantum Physics.

	For nuclei having a nucleon number becoming too large, the Coulomb repulsion between protons takes significant values in comparison to the strong interaction which ensures the cohesion of the nucleus. We then observe a phenomenon of saturation, which gives rise to the $\alpha$ disintegration which is a special case of a spontaneous fission.

	George Gamow proposed a theoretical explanation for this phenomenon in 1928. He assumes that the particle $\alpha$ preexists in the nucleus and bumps on the potential walls. It then has a non-zero probability of crossing the potential barrier of the nucleus by tunneling.

	If by thought we disconnect Coulomb interactions, such an $\alpha$ particle is linked to the rest of the nucleus by a nuclear potential of short range equation and depth corresponding to a potential energy that we will determine.

	Schematically in the case of Uranium 238 the situation is considered as follows:
	\begin{figure}[H]
		\centering
		\includegraphics[scale=1]{img/atomistic/tunelling_effect.jpg}
		\caption[Gamow's idea of Tunelling effect]{Gamow's idea of Tunelling effect (source: Pour la Science)}
	\end{figure}
	In Classical Physics we would represent the $\alpha$ emission as the leak of the nucleus from the nucleus. This representation is not valid because it implies that the $\alpha$ particle is undergoing the electrostatic repulsion of the residual core of Thorium 234 would only be removed with an energy of about $25$ [MeV]. However, the low experimentally observed value (of only $4.2$ [MeV]) is found only with the tool of Quantum Physics.
	
	Let's move on to the mathematical part:
	
	Let us connect the Coulomb repulsion between the $\alpha$ particle of electric charge $+2e$ (two protons and two neutrons) and the rest of the nucleus, then of charge $+(Z-2)e$ outside the nuclear potential well.

	We then get the expression of the potential energy (\SeeChapter{see section Electrostatics page \pageref{electrostatic potential energy}}):
	
	Where $r$ is the distance between the center of the nucleus and the $\alpha$ particle. The potential energy therefore decreases with distance since the force is repulsive.

	Now, let's have a qualitative approach to the phenomenon. We put that the probability $T$ of tunelling as being proportional, according to our results in the section of Wave Quantum Quantum Physics, to:

	
	knowing that it is, following the approximations we made during the proof, of a lower limit.

	If we model the potential barrier of the core by a non-rectangular profile as presented below:
	\begin{figure}[H]
		\centering
		\includegraphics[scale=1]{img/atomistic/irregular_tunelling_barrier.jpg}
		\caption{Non-rectangular barrier profile of the nucleus}
	\end{figure}
	where we replace the real profile of the curve by a series of barriers of thickness $\Delta x$ and where the potential equals $E_p$ at the point $\xi_i$

	The probability of passing a barrier will therefore be proportional to:
	
	and we know (\SeeChapter{see section Probabilities page \pageref{joint probability}}) that the probability of passing one of the barriers is an independent event (mutually exclusive). We can therefore multiply the probabilities such that:
	
	And passing to the limit it comes:
	
	and if $x$ is assimilated to a radius of a spherical symmetry configuration:
	
	In the case of a $\alpha$ nucleus, the potential barrier ranges from $r_1$ where it begins to $r_2$ value where the potential barrier is considered as negligible.

	But, the potential energy of the $\alpha$ at any remote point $r$ from the outside of the border of the nucleus of the radioactive atom will be equal, as we have seen a earlier before, to:
	
	We thus have for $r_1\leq r \leq r_2$:
	
	To determine $E_\text{tot}$ of the emitted $\alpha$ nucleus, it is necessary to know that its total energy is supposed preserved in this simplified model. It is therefore the same as before it passes through the nuclear potential barrier when $r_1\leq r$, during, and after $r_2$.

	Moreover, in this model, kinetic energy is also assumed constant when $r\leq r_2$. In other words, since the $\alpha$ nucleus pre-exists in the nucleus of the radioactive atom, it already has the final velocity that it will have at the point of crossing the barrier of the nuclear potential ...

	So under all these very simplifying assumptions... if we know how to determine the total energy of the $\alpha$ nucleus in $r_2$ (for example), at the output of the barrier, we have its total energy during the whole crossing phenomenon of the barrier.

	Conversely, its total energy required to exit in $r_2$ of the potential barrier by tunnel effect starting from the nucleus of the radioactive atom (and then going away at infinity and gaining kinetic energy and losing all its potential Coulomb energy) is the same assuming that the total energy obtained by calculating the work of the force which from an infinite distance of the nucleus of the radioactive atom would make the $\alpha$ nucleus return inside the nucleus of the radioactive atom at the aforementioned velocity to the minimum exit point $r_2$ (minimum exit radius taken as constant because very distant in orders of magnitude with respect to the nucleus of the radioactive atom).

	This then corresponds to the potential energy difference between a point at infinity and an $r_2$. And since the potential energy is zero at infinity for a repulsive system, there remains only the term:
	
	and finally:
	
	still valid only for $r_1\le r \le r_2$ (it is as if during the crossing of the barrier, the $\alpha$nucleus restored kinetic energy to the vacuum as it approach of the point $r_2$, that said, in Quantum Physics one we cannot use the Classical Mechanics interpretation...).

	Now, very often in laboratories, $r_2$ is expressed as a constant sufficiently far from the nucleus of the radioactive atom. It is then relatively natural (even if it is do-it-yourself physicist way of doing maths...) to take $r$ as an integration variable such as:
	
	and it is traditional to put:
	
	which brings us to:
	
	Let us now do the following change of variables (the derivation of the $\cos^2(u)$ is detailed in the sectionofDifferential and Integral Calculus):
	
	hence:
	
	and by putting:
	
	The integral:
	
	becomes:
	
	Concerning the terminals we have for recall:
	
	So if $r$ is is equal to $r_1$ we write the bound as $u_0$ and if $r$ equals $r_2$ then:
	
	Therefore:
	
	We have seen in the section of Differential and Integral Calculus:
	
	Therefore:
	
	Then:
	
	What makes:
	
	But, we also have (\SeeChapter{see section Trigonometry page \pageref{remarkable trigonometric identities}}):
	
	Therefore:
	
	Let us recall again that:
	
	But, $r_2 \gg r_1$, therefore, $r_1/r_2\cong 0$.
	
	If we develop in Maclaurin series (\SeeChapter{see section Sequences and Series page \pageref{usual maclaurin developments}}) to the first order:
	
	Therefore:
	
	We then have:
	
	If we take the Maclaurin development to the first order:
	
	Therefore:
	
	So all this to write finally:
	
	Relation in which we can put again the coefficient of the exponential that we had determined in the section of Wave Quantum Quantum Physics. It is the exponential factor in the above relation which explains the great variation of the radioactive periods of the different nuclides, whereas the energy of the particles varies relatively little.
	
	Typically for the Uranium $^{238}\mathrm{U}$ nucleus, we take the values in the tables of the physical and universal constants that are in the previous relation to get a certain value of $T$ (I will omit this numerical application for now since the constant and physic tables are not all of agree between them still in this 21st century...).
	
	Then, in the semi-classical approximation, the $\alpha$ nucleus has, in the potential well, a velocity of the order of:
	
	And it goes back and forth in a nucleus whose radius is of the order of $r_1$. These round trips therefore correspond to a certain number of oscillations per second. Indeed, if we denote by $\tau$ the average duration between of two successive shocks, we then have:
	
	Therefore the frequency is equal to:
	
	Each time it has a probability $T$ to cross the potential barrier. This probability per unit time is thus determined by:
	
	and gives the disintegration constant $\lambda$ of the isotope by $\alpha$ emission with a relatively large error if we make the calculation with the numerical values. Otherwise, the order of magnitude is by contrast exact what is not bad at all! Therefore the simplified academic approach above gives satisfactory results.
	
	\pagebreak
	\paragraph{Beta- Disintegration (4)}\mbox{}\\\\\
	\textbf{Definition (\#\mydef):} When a nucleus is unstable due to an overflow of neutrons (like Carbon $14$ for example) it will not emit neutrons. On the other hand, it will have the ability to change one of its neutrons into a proton. During this transformation, to keep the total electrical charge of the system, an electron will be created. This transformation is named "\NewTerm{$\beta^{-}$ radioactivity}\index{beta- radioactivity}\index{$\beta^{-}$ radioactivity}" (because the electron has a negative charge in this disintegration).
	
	The so-named $\beta^{-}$ disintegration is therefore a characteristic of nuclei with an excess of neutrons. The isotopes concerned become more stable by transforming a neutron into a proton with emission of an electron $\beta^{-}$ and a particle named "\NewTerm{antineutrino}\index{antineutrino}" which we will justify the presence further below.

	We then have for the concerned neutron:
	
	We have put in right exponent the spin of the concerned  particle and in right index the sign of the electric charge of the particle. Thus, we observe that the spin is a conserved quantity, as well as the electric charge.

	We have for the concerned isotope:
	
	\begin{tcolorbox}[colframe=black,colback=white,sharp corners]
	\textbf{{\Large \ding{45}}Example:}\\\\
	The famous case of Radiocarbon dating:
	
	\end{tcolorbox}
	The energy released during the transmutation is calculated by means of the mass default:
	
	By neglecting the binding energy of the electrons, we have:
	
	Caution!!! The $Z$ in the equality of $M_X$ is the same as that found in the expression of $M_Y$ hence the $Z + 1$.

	We have then:
	
	Each pure $\beta^{-}$ disintegration is characterized by a fixed decay energy $Q$. Since the kinetic energy of the nucleus is negligible by its mass compared to that of the combined electron and antineutrino, the released energy $Q$ is shared between the kinetic energies of the $\beta^{-}$ and $\bar{}$. Since the mass of the antineutrino is very far below that of the electron, the kinetic energy of the antineutrino can also be neglected. However, the energy of the $\beta^{-}$ is not fixed and can take any value between $0$ and $Q$. We thus observe an energy spectrum unlike other types of radioactivity (because the electron may have variable kinetic energy).

	The shape of the observed distributions makes it possible to give an average energy value to the $\beta^{-}$ which is around:
	
	The existence of antineutrino was postulated in 1933 ($3$ years after the neutrino that we will see below) by Wolfgang Pauli in order to satisfy the spin conservation. The introduction of such a strange particle was very controversial and not widely accepted (zero charge, non-zero spin, negligible mass) and it continues to pose some problems in the contemporary physics of the 21st century.

	Independently of the "\NewTerm{electron neutrino}\index{electron neutrino}" (denoted usually $\nu_e$) accompanying the particles $\beta^{-}$ and $\beta^{+}$ (the latter having several names "\NewTerm{positron}\index{positron}" or "\NewTerm{positive electron}\index{positive electron}") there exists a neutrino of the meson $\mu$ (muon) and a neutrino of the tau (tauon) that are respectively denoted: $\nu_\mu$ and $\mu_\tau$ not to confuse them. Subsequently, as we will not be confronted with the neutrino of meson or tau neutrinos, we will simply denote the electron neutrino $\nu$ instead of $\nu_e$ or of $\upsilon$.
	\begin{tcolorbox}[title=Remark,colframe=black,arc=10pt]
	At the beginning of its discovery, ther disintegration $\beta^{-}$ was seen as a transmutation of the nucleus ... in small classes, even today, it is seen as the transformation of a neutron into a proton. In contemporary theories, it is seen as a the transformation of a quark $d$ into quark $u$ and it has led physicists to develop the theory of weak interaction to explain its origin.
	\end{tcolorbox}
	Betavoltaic devices, also known as betavoltaic cells, are generators of electric current, in effect a form of battery, which use energy from a radioactive source emitting beta particles (electrons). A common source used is the hydrogen isotope, tritium. Unlike most nuclear power sources, which use nuclear radiation to generate heat, which then is used to generate electricity (thermoelectric and thermionic sources), betavoltaics use a non-thermal conversion process; converting the electron-hole pairs produced by the ionization trail of beta particles traversing a semiconductor.
	
	Betavoltaic power sources (and the related technology of alphavoltaic power sources) are particularly well-suited to low-power electrical applications where long life of the energy source is needed, such as implantable medical devices or military and space applications.
	
	The actual technologie provide a source of continuous nanowatt-to-microwatt battery power that is resistant to a broad range of temperatures and other environmental conditions for extended periods of $20$ or more years.
	\begin{figure}[H]
		\centering
		\includegraphics[scale=0.7]{img/atomistic/citylabs_betavoltaic_battery.jpg}
		\caption{City Labs Nanotritium™ Battery}
	\end{figure}
	For example, the amount of electricity is tied with half-life. For example, if $^{63}\mathrm{Ni}$ has half-life of $100$ years this means, that mole of Nickel (63 grams) will produce Avogadro/$2$ electrons during that $100$ years. This means $10^{21}$ electrons per year and $10^{14}$ electrons per second.

	This means up to $0.1$ [mA] or electric current.
	
	The total decay energy of $^{63}\mathrm{Ni}$ is $67$ [keV] and is approximately the maximum energy of electron. But, since the decay also produces neutrino the mean energy of electron is much smaller: $17$ [keV] ($2.732\cdot 10^{-15}$ [J]).  This means that each electron has $17$ kilovolts of electric tension.
	
	So, the power of electricity from one mole of $^{63}\mathrm{Ni}$:
	
	This is not sufficient to provide iPhone 4 peak power consumption, which is about $1.5$ [W]. So atomic batteries can actually not supersede conventional batteries, and serve for years.
	
	Also, it must be noticed that the usage of radioactive batteries definitely require regulations for handling and probably would not be allowed inside a single unit for unrestricted civilian use. We know that when properly used betavoltaics are safe. But what about im-proper use / improper disposal / potential for abuse? At any rate, current perception of nuclear power by general public is not that good, so marketing nuclear batteries will present certain challenge. Also the price of Plutonium or Americanium batteries (that are superior to batteries in almost every way) have cost that limits them to micro power devices (one gram is worth almost $1,000$ dollars...).
	
	\paragraph{Beta+ Disintegration (5)}\mbox{}\\\\\
	\textbf{Definition (\#\mydef):} When a nucleus is unstable due to an overflow of protons it will not emit protons. On the other hand, it will have the ability to change one of its protons into a neutron. During this transformation, to keep the total electrical charge of the system there are two ways for this disintegration to occur: 
	\begin{itemize}
		\item By capture of a negative electron an then we speak of "electronic capture radioactivity" (see further below)

		\item Or by emission of positive and then this transformation is named "\NewTerm{$\beta^{+}$ radioactivity}\index{beta+ radioactivity}\index{$\beta^{+}$ radioactivity}" (because the electron detected has a positive charge in this disintegration).
	\end{itemize}
	Therefore during a $\beta^{+}$ disintegration a proton is dissociated into a neutron, a positive electron ("\NewTerm{positron}\index{positron}") denoted $\beta^{+}$  and of a neutrino (which we will justify the presence a little lower).
	
	This transformation has a ridiculously low probability since the inverse of the emission of an electron and an antineutrino would be the simultaneous capture of these two particles... and such a reunion of circonstances would be a miracle. To overcome this difficulty, the nucleus uses a quantum subterfuge: the emission of a particle is equivalent to the capture of its antiparticle. This joker then offers the aforementioned possibilities to the surplus nucleus in protons. Indeed, to carry out the inverse of the $\beta^{-}$ disintegration, the solution consists for the nucleus to use the conservation of the energy and of the spin by emitting a positron and capturing in the quantum energy of vacuum (\SeeChapter{see section Quantum Field Theory page \pageref{quantum field theory}}) an antineutrino and to emit in exchange a neutrino.

	We write this:
	
	or:
	
	The energy released during the transmutation is calculated by means of the mass default:
	
	By neglecting the binding energy of the electrons, we have:
	
	Caution!!! The $Z$ in the equality of $M_X$ is the same as that found in the expression of $M_X$ hence the $Z + 1$.

	We have then:
	
	The $\beta^+$ disintegration can therefore take place only if $Q_{\beta^{+}}\ge 0$, that is to say if:
	
	The mass energy of the electron $E=m_ec^2$ is important because it is the energy of one of the two photons resulting from an annihilation of a positron $\beta^{+}$ with an electron $\beta^{-}$.

	As for the $\beta^{-}$ disintegration, the energy of the $\beta^{+}$ is not fixed and can take any value between $0$ and $Q$. We thus observe an energy spectrum.

	\paragraph{Electronic capture (6)}\mbox{}\\\\\
	\textbf{Definition (\#\mydef):} When a nucleus is unstable because of an overflow of protons with respect to neutrons, we know now that a favorable solution from the point of view of its energy is to transform one of its protons into a neutron, ie to realize the inverse of the $\beta^{-}$ radioactivity. We saw just earlier that a possibility was for the nucleus via the disintegration $\beta^{+}$ to catch an antineutrino of the vacuum and to emit a positron (loss of its electric charge) and a neutrino. But it can also capture an electron from the electronic cloud (neutralizing its electrical charge) instead of emitting a positron, this is why we name this an "\NewTerm{electronic capture}".

	This will most often be an electron of layer $K$. What is written:
	
	The energy released during the transmutation is calculated by means of the mass default:
	
	assuming that the binding energy of the electron of the layer K and that of recoil of the nucleus are negligible.
	
	It is therefore the electron neutrino that carries all the energy, hence the necessity that Wolfgang Pauli had to introduce this new particle (what had horrified him ...!). Since the captured electron occupies a precise energy level in the atom, the neutrinos resulting from the disintegration of an isotope by electronic capture have a determined energy and therefore have a discrete spectrum.
	
	By neglecting the binding energy of electrons, we have:
	
	Therefore:
	
	The disintegration by electronic capture is in competition with the $\beta^{+}$ disintegration only if:
	
	in the case where:
	
	only the disintegration by electronic capture is possible.

	However, the hole left by the absorbed electron requires a rearrangement of the atomic cloud and the emission of radiation.
	
	
	\paragraph{Gamma emission (7)}\mbox{}\\\\\
	\textbf{Definition (\#\mydef):} For the nucleus, the emission of electromagnetic radiation $\gamma$ is a possibility of gaining stability. This emission generally follows a phenomenon of disintegration $\beta^{-}$, $\beta^{+}$, $\alpha$ or electronic capture. We can imagine that, in such types of disintegration, the topology of the nucleons in the nucleus is not ideal and that the rearrangement of the latter will be accompanied by a reduction of energy; The latter emitted in the form of one or more photons.

	So we have a schema ($\beta^{-}$) disintegration:
	
	then:
	
	where the $m$ means "metastable" or "\NewTerm{isomer}\index{isomer}" (this last term is used when the emission of radiation takes place long time after disintegration).
	
	\begin{tcolorbox}[title=Remark,colframe=black,arc=10pt]
	"Isomer" means that the nucleus is excited. It will de-energize with a period $T_{1/2}$. Generally the $T_{1/2}$ is extremely small and the photon(s) is (are) emitted immediately after the electron in the case of our example of a $\beta^{-}$ disintegration. We then talk about "metastable state" or "isomer". It should be noted that these radioisotopes are particularly interesting in medical imaging.
	\end{tcolorbox}
	The energy of the $\gamma$ photon is equal to:
	
	Now let us suppose that a proton is confined in a box of width $L = 1.00\cdot 10^{-14}$ [m] (a typical nuclear radius). If we assume that the proton confined in the nucleus can be modeled as a quantum particle in a box, all we need to do is to use te relation proven in the section of Wave Quantum Physics for an infinite potential wall box to find its energies $E_1$ and $E_2$:
	
	Therefore the emitted photo has for energy:
	
	Hence:
	
	This is the typical frequency of a gamma ray emitted by a nucleus. The energy of this photon is about $10$ million times greater than that of a visible light photon!
	
	\paragraph{Internal conversion (8)}\mbox{}\\\\\
	\textbf{Definition (\#\mydef):} The  "\NewTerm{internal conversion (I.C.)}\index{internal conversion}" is a process related to the emission of a $\gamma$ photon. Indeed, it is possible that the energy is transmitted directly to an electron of the electronic cloud, generally of the layer K, which is then ejected from the atom. This electron is named the "\NewTerm{conversion electron}". The place left in the electronic cloud is subsequently filled by an electron of the upper layers and so on. Thus, as in the case of an electronic capture decay process, there is a rearrangement of the electronic cloud characterized by the emission of X-rays characteristic by the element $Y$.

	The energy transmitted is:
	
	with $E_{e^{-}}$ being the kinetic energy of the emitted electron, $E_\gamma$ the photon energy percutting the electron, $E_b$, the binding energy of the electron considered (K, L, M, ...).
	
	The energy of the $\gamma$ photon is transmitted directly to an electron that is ejected. The process is followed by the rearrangement of the electrons (an emission of X-rays will follow). The ejected electron is named "\NewTerm{Auger electron}\index{Auger electron}" and sometimes the whole effect is named the "\NewTerm{Auger effect}\index{Auger effect}".
	
	The emission of a Auger electron is thus a process similar to the internal conversion process (IC), but the electromagnetic radiation does not come from a de-excitation of the nucleus (it is not a $\gamma$ photon) but of an X-ray produced during the rearrangement of the electronic cloud. In a radioactive process, this electronic rearrangement can come either from an electronic capture (EC) or from an internal conversion (IC).

	The ejected Auger electron comes mainly from an internal orbit and its energy is the characteristic energy of the X-ray minus its binding energy. The energy of the Auger electrons is therefore quite small (some [keV]) with respect to a  $\beta^{-}$ particle and these electrons are therefore often reabsorbed inside the source. The process of emission of an Auger electron is favored for elements with low atomic number because of their low energies of electronic connection.

	During a rearrangement of the electronic cloud such as the passage of an electron from the layer L to the layer K, the energy of the X-ray emitted will be equal to $E_\text{K}-E_\text{L}$. Since this energy difference is greater than the binding energy of another electron on layer L, the latter will then be emitted with the kinetic energy:
	
	In turn, the $2$ vacations left on layer L are filled with electrons from the upper layers. Fluorescence and Auger electron are then in competition. It is even possible that several Auger electrons are emitted during the de-excitation of the atom. This is named "\NewTerm{Auger cascade}\index{Auger cascade}" leaving the atom considered highly ionized, which can lead to the Coulomb explosion of the molecule of which it is a part.
	
	To conclude on all these radioactive phenomena, let us indicate the order of magnitude of the radioactive periods of some natural and artificial elements:
	\begin{table}[H]
		\centering
		\begin{tabular}{|l|c|l|}
		\hline
		\rowcolor[HTML]{C0C0C0} 
		\textbf{Nucleid} & \multicolumn{1}{l|}{\cellcolor[HTML]{C0C0C0}\textbf{Radioactivity}} & \multicolumn{1}{c|}{\cellcolor[HTML]{C0C0C0}\textbf{Period}} \\ \hline
		Thorium 232 & $\alpha$ & $\backsim 10^{10}$ years \\ \hline
		Uranium 238 & $\alpha$ & $\backsim 10^{9}$ years \\ \hline
		Uranium 235 & $\alpha$ & $\backsim 10^{8}$ years \\ \hline
		Uranium 233 & $\alpha$ & $\backsim 10^{5}$ years \\ \hline
		Plutonium 239 & $\alpha$ & $\backsim 10^{4}$ years \\ \hline
		Plutonium 238 & $\alpha$ & $\backsim 88$ years \\ \hline
		Radium 226 & $\alpha$ & $\backsim 10^{3}$ years \\ \hline
		Curium 242 & $\alpha$ & $160$ days \\ \hline
		Potassium 40 & $\beta^{-}$ & $\backsim 10^{9}$ years \\ \hline
		Carbone 14 & $\beta^{-}$ & $\backsim 10^{3}$ years \\ \hline
		Tritium & $\beta^{-}$ & $21$ years \\ \hline
		Cobalt 60 & $\beta^{-}$ & $\backsim 5.3$ years \\ \hline
		Iode 131 & $\beta^{-}$ & $\backsim 8$ days \\ \hline
		Azote 16 & $\beta^{-}$ & $\backsim 7.1$ seconds \\ \hline
		Technétium 97 & EC & $\backsim 10^{6}$ years \\ \hline
		Cobalt 58 & $\beta^{+}$ & $\backsim 20$ minutes \\ \hline
		Fluor 18 & $\beta^{+}$ & $\backsim 110$ minutes \\ \hline
		\end{tabular}
		\caption{Amplitudes of the ractioactive periods of a few nuclides}
	\end{table}
	
	\subsection{Radiation protection}
	In nuclear physics (and also in "high energy astrophysics"), it is very important to know how the various $\alpha$, $\gamma$, X-ray or neutron radiation interact with matter (roughly uncharged or charged radiation). This allows us to know how their kinetic energy is distributed or dissipated in the material they encounter along their path and to protect themselves in a suitable way.
	
	"\NewTerm{Radiation protection}\index{radiation protection}", sometimes known as "\NewTerm{radiological protection}\index{radiological protection}", is defined by the International Atomic Energy Agency (IAEA) as "\textit{The protection of people from harmful effects of exposure to ionizing radiation, and the means for achieving this}". The IAEA also states "\textit{The accepted understanding of the term radiation protection is restricted to protection of people. Suggestions to extend the definition to include the protection of non-human species or the protection of the environment are controversial}".

	Ionizing radiation is widely used in industry and medicine, and can present a significant health hazard. It causes microscopic damage to living tissue, which can result in skin burns and radiation sickness at high exposures (known as "tissue" or "deterministic" effects), and statistically elevated risks of cancer at low exposures ("stochastic effects").

	Fundamental to radiation protection is the reduction of expected dose and the measurement of human dose uptake. For radiation protection and dosimetry assessment the International Committee on Radiation Protection (ICRP) and International Commission on Radiation Units and Measurements (ICRU) have published recommendations and data which is used to calculate the biological effects on the human body, and set regulatory and guidance limits.

	Before investing this subject we must prove some relations and study some phenomena related to high energy radiations and behaviors of particles. So let us tackle this first!

	\subsubsection{Bether formula}
	The "\NewTerm{Bethe formula}\index{Bethe formula}" describes the mean energy loss per distance traveled of swift charged particles (protons, alpha particles, atomic ions) traversing matter (or alternatively the stopping power of the material). For electrons the energy loss is slightly different due to their small mass (requiring relativistic corrections) and their indistinguishability, and since they suffer much larger losses by Bremsstrahlung, terms must be added to account for this. Fast charged particles moving through matter interact with the electrons of atoms in the material. The interaction excites or ionizes the atoms, leading to an energy loss of the traveling particle.
	
	The non-relativistic version was found by Hans Bethe in 1930. The Bethe formula is sometimes named "\NewTerm{Bethe-Bloch formula}", but this is quite misleading.
	
	A heavy charged particle having an energy of one or more MeV loses its energy mainly by collisions with the electrons cloud of atoms, electrons which appear to it as quasi-free. The process by which electrons are thus ejected during the passage of an ionizing particle is named "\NewTerm{primary ionization electrons}". An electron can escape if it receives an energy higher than its binding energy.

	The maximum energy transfer $E_{\max}$ that can occur in a non-relativistic and elastic collision (where the system energy is conserved because there is by definition no heat - radiation - dissipation) is calculated simply by using the conservation principle of the linear momentum and energy:

	Given $M$,$v_M$ and $m_e$, $v_e$ be the masses and velocities of the incident particle and the electron respectively. We assume that the electron is immobile in its orbit and that its initial velocity is zero ($v_e=0$). After the shock, we will assume that the incident particle will have transferred all its kinetic energy to the electron and will in turn find itself at rest such as $v_M^{'}=0$.
	
	Let us first consider the conservation of linear momentum:
	
	The conservation of energy also allows us to write:
	
	Hence, after regrouping and simplification:
	
	Or otherwise written:
	
	Then, after dividing the second equation by the first we have:
	
	We then have the system:
	
	we then deduce the expression of the speed after the impact:
	
	relatively to our initial hypotheses ($v_e=0$, $v'_M=0$), we then have:
	
	Let us manipulate this relation a bit:
	
	For a heavy particle, with $M\gg m_e$ we can write:
	
	An ionization can only occur if the $_\text{max}E^e_c$ is at least equal to the ionization threshold of the electron that we will denote by $I_0$ and whose calculation was seen during the study of the Bohr model (\SeeChapter{see section Corpuscular Quantum Physics page \pageref{bohr model}}).

	The energy of the incident particle must therefore at least be equal to:
	
	Thus, when passing through matter, the electrically charged body of charge $Z\cdot e$ and speed $v_M$ yields its energy in numerous collisions with the electrons of the atoms encountered. The interaction is of the Coulomb type and each time a diffusion occurs. The recoil energy of the electron, assumed to be free, can be calculated accurately. To estimate the loss of energy, we will here approximate that the quantity of linear momentum transferred $\Delta p$ is equal to the product of the interaction force at the distance $r$ multiplied by the time necessary for the projectile to traverse the path $2r$. We have the Coulomb for $F$ given for recall by (\SeeChapter{see section Electrostatics page \pageref{coulomb force}}):
	
	and the linear momentum:
	
	The kinetic energy transferred to an electron of mass $m_e$ will be:
	
	The total energy loss will be obtained by integrating on all the electrons encountered. At the distance range that is between $r$ and $r + \mathrm{d}r$ of the trajectory and on the path of length $\mathrm{d}x$, there are:
	
	electrons, where $N$ is the number of atoms of atomic number $Z'$p er unit volume. The loss of energy per unit of distance is therefore:
	
	The value of $r_{\min}$ is evaluated by noting that this "\NewTerm{impact parameter}\index{impact parameters}" should corresponds in practice to the maximum energy transfer. Using the equations we have proved previously:
	
	With $E_c^M=\dfrac{1}{2}Mv_m^2$, we can get the parameter $r_{\min}$ by:
	
	and we get:
	
	When $r$ becomes very big, the energy transfer is smaller than the mean ionization energy denoted $\bar{I}$ of the electrons and the process is no longer efficient. We must therefore have the following relation:
	
	We derive a value for $r_\text{max}$:
	
	By replacing the values of $r_{\min}$ and $r_{\max}$ of the above relations into the equation:
	
	we get:
	
	A more rigorous quantum treatment would show that the root of the argument of the logarithm should be suppressed by taking into account the relativistic effects as well as the intrinsic properties of the electron (fine structure constant). We would then obtain the following Bethe formula:
	
	where $\beta=v/c$ and $c_k$ is a correction term that depends on energy and of $Z$ when we take into account the complete structure of the nucleus (layered model) of the material.

	Finally, we see that the loss of linear energy is proportional to the atomic number of the incident radiation and the penetrated matter. Thus, to come back on radioprotection..., protections composed of materials with high atomic numbers (high density) will have a high decelerating power and will be advantageous in radiation protection.
	
	\subsubsection{Compton scattering}\label{compton scattering}
	The Compton effect is observed when a photon is scattered inelastically by a charged particle. In fact, the photon is absorbed and then re-emitted by the particle, thus giving up part of its energy. It is this transfer of energy that justifies the inelastic character of the diffusion.

	Thus if the charged particle is an electron, this effect can take place indifferently on an electron of any electronic layer or even on a free electron. The energy of the photon and that of the electron depend on the direction of emission of these particles. Since this effect depends on the number of electrons available per target atom, the Compton diffusion probability increases linearly with the atomic number $Z$ of the absorbent. But since this effect is in competition with the production of an electron-positron pair that we shall see later, the Compton effect is especially important at average energies and at average atomic numbers.

	We have recall that in the section of Special Relativity we have proved that:
	
	And let us recall that we have thus for the momentum of a photon:
	
	and we have also proved in the Special Relativity section that, starting from the total energy, the linear momentum is given by:
	
	Hence the relation, which we shall use further below:
	
	Before the interaction, photon-electron, we have (we roughly consider the electron as being at rest) for the total energy of the system:
	
	and after collision:
	
	Conservation of energy leads us to write:
	
	By considering only the kinetic energies, we neglect that of the electron before the shock:
	
	Either the figure below:
	\begin{figure}[H]
		\centering
		\includegraphics[scale=1]{img/atomistic/compton_scattering.jpg}
		\caption[]{Simplified illustration of the Compton effect}
	\end{figure}
	The conservation of the linear momentum give us according to the $x$-axis:
	
	and according to the $y$-axis:
	
	The sum of these two relations squared gives us the total linear momentum:
	
	Then by substituting $p_e^2c^2$:
	
	and as $E_{c,e}=E_1-E_2$:
	
	When the energy of the photon is high enough, that is $E_1\gg m_0c^2$, that of the scattered photon tends to a limit given by (see Hospital rule in the section of Differential and Integral Calculus):
	
	The energy acquired by the Compton electron is finally equal to:
	
	It is interesting to notice that we can not have $E_e=E_1$. Indeed this would suppose that:
	
	and we see well that in fact, whatever the value of $\phi$, we always have $E_2\neq 0$.
	
	The frequency of the scattered photon is lower than that of the incident photon because its energy $E_2$ is always lower and hence its wavelength $\lambda_2$ larger. Therefore:
	
	and since:
	
	We have:
	
	That is:
	
	which is also written using the definition of the Planck constant and the usual trigonometric relations:
	
	We name the factor $\dfrac{h}{m_0c^2}$ the "\NewTerm{Compton wavelength}\index{Compton wavelength}\label{compton wavelength}" and its value is equal to:
	
	Notice that if the angle $\phi$ is equal to zero, then the wavelength variation is zero and if the angle $\phi$ is equal to $180^\circ$ ($\pi$ [rad]) then the variation of the incident photon wavelength is twice the length of the Compton's wavelength!
	
	\begin{figure}[H]
		\centering
		\includegraphics[scale=0.8]{img/atomistic/compton_scattering_final.jpg}
		\caption[Compton scattering]{Compton scattering (source: ? )}
	\end{figure}
	
	\paragraph{Thomson scattering}\mbox{}\\\\\\
	Thomson scattering is the elastic scattering of electromagnetic radiation by a free charged particle, as described by classical electromagnetism. It is just the low-energy limit of Compton scattering: the particle kinetic energy and photon frequency do not change as a result of the scattering.[1] This limit is valid as long as the photon energy is much less than the mass energy of the particle: $h\nu \ll m_ec^{2}$, or equivalently, if the wavelength of the light is much greater than the Compton wavelength of the particle.
	
	First consider scattering of (classical) electromagnetic waves by free electrons, named "Thomson scattering". The electron experiences a force due to the incident electric field:
	
	The equation of motion for the charged particle is:
	
	We recall that in the dipole approximation (valid since the wavelength of light is large compared to the scatterer) the power emitted per unit solid angle\label{solid angle atomic physics} is (\SeeChapter{see section Electrodynamics page \pageref{bremsstrahlung}}):
	
	where $\langle a^2 \rangle$ represents the time-averaged squared acceleration. The equation of motion gives the acceleration directly and we assume that it is equal to the time averaged squared acceleration such that:
	
	This yields, upon rearranging:
	
	The term in the brackets is known as the "\NewTerm{classical electron radius}\index{classical electron radius}":
	
	Now recall that the time-averaged Poynting vector which is the incident energy per unit area per unit time is (\SeeChapter{see section Electrodynamics page \pageref{poynting vector}}):
	
	We can astutely fall back on the "\NewTerm{differential scattering cross section}\index{differential scattering cross section}" define earlier above by using dimensional analysis:
	
	and is the area of the wavefront which delivers the same power as is scattered into a given solid angle $\mathrm{d}\Omega$. So we get after simplification:
	
	The total cross section is obtained by integrating over $\mathrm{d}\Omega$:
	
	But from the section Trigonometry we know that:
	
	Therefore (identically as the integral we did in the Trigonometry section):
	
	Therefore:
		
	This is known as the "\NewTerm{classical Thomson cross section}\index{classical Thomson cross section}".
	
	Ithe classical Thomson scattering formula is know to be in disagreement with experiment by the early 1920s. The Klein-Nishina formula based on Quantum Electrodynamics (QED) agreed perfectly with the available experiments in the late 1920s. However at ultra relativistic energies it looked wrong. It was not until many years later the difference was shown to be due to the production of electron/positron pairs, with the positron annihilating into some other electron, and to Bremsstrahlung.
	
	\subsubsection{Photoelectric effect}\label{photoelectric effect}
	The photoelectric effect is the ejection of electrons (the named "\NewTerm{photoelectrons}\index{photoelectrons}") from the surface of various metals exposed to radiation energy. This radiation can come from the rearrangement of the nucleus of the atom as well as from an external radiation.

	Moreover, it was by quantitative measurements of the photoelectric effect that Albert Einstein proposed to test the quantum theory of light (Wave Quantum theory) and therefore the theoretical explanation that bring him to receive Nobel prize of Physics.

	Let us first describe the experiment: the emission of electrons by a metal does not contradict the electromagnetic theory of light. If we consider a uniform beam, its energy is uniformly distributed over the entire wave front. The more intense the light is, the greater the amplitudes of the electric and magnetic fields at each point of the wavefront and the greater the energy per second transmitted by the wave is. These fields exert forces on the electrons in the metal and can even tear them from its surface.

	Here is the experience set up:
	\begin{figure}[H]
		\centering
		\includegraphics[scale=0.7]{img/atomistic/photoelectric_effect.jpg}
		\caption[Experiment for measuring the photoelectric effect]{Experiment for measuring the photoelectric effect (source: ?)}
	\end{figure}
	If the collecting anode is at a positive potential relative to the emitting cathode, the photoelectrons traverse the tube and constitute the current measured by the ammeter. We then observe experimentally a proportionality between the intensity of the incident beam and the current.
	However, at least three problems persist between the theoretical model and the experimental observation:
	\begin{enumerate}
		\item The wave aspect of light is not suitable for explaining the time required to absorb the extraction energy.

		Indeed, let us consider a $100$ [W] lamp with a $15\%$ luminous efficiency placed at $0.5$ [m] of a potassium-coated plate (chemical symbol of potassium is $\mathrm{K}$) have a minimum extraction energy of $2.25$ [eV] assuming a diameter of $1\cdot 10^{-10}$ [m] for the Potassium atom.
		
		We then have:
		
		The luminous power absorbed by the atom by its half-surface which faces the radiation is then:
		
		The time required for absorption is then:
		
		Which is in contradiction with the experience where we observe that the phenomenon is almost instantaneous (time for light to travel to the metal).
		
		\item If we reverse the poles, the electrons emitted by the metal are pushed back by the negative electrode, but if the reverse voltage is not to big the faster ones can still reach the negative electrode and a current will occur. At a negative potential, specific for each metal, named the equation "\NewTerm{stop potential $U_a$}", all the emitted electrons are pushed back and the current is zero. The maximum kinetic energy of these photoelectrons is then:
		
		However, we see experimentally that this stopping potential is independent of the intensity of the radiation. In the Wave theory, the increase in intensity should increase the number of electrons extracted (regardless of their energy level) and their maximum kinetic energy. A greater intensity implies a greater amplitude of the electric field: $I \propto E^2$. Thus, a bigger electric field should eject the electrons at higher velocities from all layers of the electronic cloud as the intensity increases.
		
		\item When we vary the frequency $v$ of incident light and we measure $U_a$, we observe that the photoelectric effect does not occur if $\nu<\nu_0$ ($\nu_0$ is named the "\NewTerm{frequency threshold}") and this whatever the intensity of the light . What is rather annoying... because in Wave theory, we must always be able to eject electrons whatever the frequency, we just have to increase the intensity.
	\end{enumerate}
	Each problem can be solved by adopting the following point of view:
	\begin{enumerate}
		\item In the Wave theory approach, the source is seen as propagating like a spherical wavefront whose surface density of energy decreases as a $1/r^2$ (\SeeChapter{see section Electrodynamics page \pageref{electromagnetic emissions}}). In order to explain the experimental observation, we must see the experiment from a corpuscular point of view where the wavefront is a front of corpuscles whose surface density of photons decreases in $1/r^2$ but where the energy of each photon remains $h\nu$ (according to the Planck-Einstein relation).
		
		\item If we think in terms of photons, when we increase the intensity, we increase the number of photons, but the energy per photon $h\nu$ remains unchanged. Thus, the energy $E_{c,\max}$ that each photon can have does not change. Hence the stopping potential is independent of the field intensity.
		
		\item If we think in terms of photons again, the electrons in the target are retained by the attraction forces, extracting an electron from the surface requires a minimal energy $E_l$ that depends on each material ($E_l$ is also named "\NewTerm{extraction work}" denoted $W_i$ which is of the order of some electronvolts). If the energy of the incident photon $E_i=h\nu$ is greater than $E_l$, an electron can be extracted, but if it is inferior, no electron can be extracted. The energy input $h\nu$ is equal to the kinetic output energy of the electron plus the energy required to extract it from the metal, therefore:
		
		Thus, if the frequency of light is increased, the maximum kinetic energy of the electrons increases linearly. R.A. Millikan made rigorous experiments between 1913 and 1914, the results were matching perfectly with Albert Einstein's theory.
	\end{enumerate}
	The light spreads from place to place as if it were a wave. But light interacts with matter in absorption and emission processes as if it were a particle stream. We already know that this is the  "wave-particle duality" (\SeeChapter{see section Wave Quantum Physics page \pageref{wave quantum physics}}). Thus, as that latter can found in the mass particles as suggested by the de Broglie hypothesis that we have seen in the section of Wave Quantum Physics, is finally verified also for the light.
	\begin{figure}[H]
		\centering
		\includegraphics[scale=1]{img/atomistic/photoelectric_effect_bohr_model.jpg}
		\caption{Principle of the photoelectric effect on the Bohr model of the atom}
	\end{figure}
	A photon of incident energy $E_i$ which interacts with an electron of a target atom can eject this electron from its orbit by communicating to it a kinetic energy equation:
	
	where $E_l$ is the binding energy of the electron ejected from its orbit (this relation is given in the form $T=E-W_i$ in the figure above).
	
	If the energy of the incident photon is lower than the binding energy of the electron of layer K (\SeeChapter{see section Corpuscular Quantum Physics page \pageref{corpuscular quantum physics}}), the photoelectric effect is done with an electron of the layer L, etc.

In the case where the radiation is absorbed, the atom is said to be "excited", because its energy state is not the minimum state. There follows a "relaxation" (or "de-excitation"): an electron of a higher layer fills the quantum space left vacant by the ejected electron.

	If the transition energy is moderate (ie, if the incident radiation has a moderate energy), relaxation causes the emission of a low-energy (visible or ultra-violet) photon, this is the phenomenon of fluorescence. 
	
	\pagebreak
	\subsubsection{Rutherford scattering}\label{rutherford scattering}
	Let us now consider the diffusion that a charged particle undergoes when it is subjected to an electrostatic repulsive force inversely proportional to the square of the distance between the moving particle and a fixed point or center of force. This problem is particularly interesting because of its application to atomic and nuclear physics. For example, when a proton, accelerated by a machine such as a cyclotron, passes near a core of the target's material, it is deflected by a force of this type from electrostatic repulsion of the nucleus. This study is referred to as the "\NewTerm{Rutherford scattering}\index{Rutherford scattering}" or "\NewTerm{Coulomb scattering}\index{Coulomb scattering}" and is manly based on the following figure:
	\begin{figure}[H]
		\centering
		\includegraphics[scale=1]{img/atomistic/rutherford_scattering.jpg}
		\caption{Rutherford scattering}
	\end{figure}
	Let O be a center of force and a particle thrown against O from a great distance with the velocity $v_0$ (see figure above) moving from right to left and entering the figure above at point $A$. We will choose The axis of $X$ passing through O and parallel to $v_0$. The distance $b$, named the "\NewTerm{impact parameter}", is the distance between the $x$-axis of the abscissa and the point $A$. Assuming that the force between $A$ and O is repulsive and central, the particle will follow the curved trajectory $AMB$. The shape of the curve depends on how the force varies with the distance. If the force is inversely proportional to the square of the distance, that is to say if:
	
	The trajectory is a hyperbola (indeed, the equation and solution are the same as Kepler problem, we just need to flip the sign of the constant $K_e$). Of course (\SeeChapter{see section Electrostatics page \pageref{coulomb force}}):
	
	When the particle is in $A$ its angular momentum is $mv_0b$. In any position such that $M$, its angular momentum (\SeeChapter{see section of Classical Mechanics page \pageref{angular momentum}}), is also given by $mr^2\dfrac{\mathrm{d}\theta}{\mathrm{d}t}$. Since the angular momentum must remain constant since the force is central:
	
	The equation of motion in the O$Y$ direction is obtained by injecting this relation into:
	
	so we can get rid of $r^2$ and write therefore:
	
	To find the deviation of the particle, we must integrate this equation from one end of the trajectory to the other. In $A$ the value of $v_y$ is zero because the initial motion is parallel to the $X$ axis and we also have $\theta=0$. In $B$ we have $v_y=v_0\sin(\phi)$ and $\theta=\phi$ or $\pi-\theta=\pi-\phi$. We notice that in $B$ the speed is again $v_0$ because, by symmetry, the speed lost when the particle approaches $O$ must be regained when it moves away from it. Therefore:
	
	Which give:
	
	thus by using the common trigonometric relations:
	
	and after rearrangement:
	
	Let us recall that (\SeeChapter{see section Trigonometry page \pageref{remarkable trigonometric identities}}) that:
	
	What then gives us:
	
	Either more explicitly:
	
	This relation gives the angle of deviation $\phi$ as a function of the impact parameter $b$:
	
	Of course, in undergraduate cases, $Q = q$ is often put, which simplifies the previous relation slightly, but we then loses in generalization.
	
	This equation is applied to the analysis of the deviation of an electrically charged particle by a nucleus. Let us notice that this result is only valid for a force inversely proportional to the square of the distance. If the force depends on the distance according to another law, the deflection angle satisfies another equation. The deviation experiments are therefore also very useful when we want to determine the law of force involved in the interactions between particles!
	\begin{figure}[H]
		\centering
		\includegraphics[scale=1]{img/atomistic/rutherford_scattering_closest_approach.jpg}
		\caption{Representation of the Rutherford diffusion of alpha nucleus}
	\end{figure}
	In nuclear physics laboratories, diffusion experiments are carried out by accelerating electrons, protons or other particles by means of a cyclotron, a Van de Graaf accelerator or some other similar device, and by observing the angular distribution of the deflected particles.

	It is clear that an incident particle in a surface defined by a radius between $b$ and $b + \mathrm{d}b$ will be respectively included in the solid diffusion angle:
	
	with (\SeeChapter{see section Trigonometry page \pageref{solid angle}}):
	
	\begin{figure}[H]
		\centering
		\includegraphics[scale=1]{img/atomistic/solid_diffusion_angle.jpg}
		\caption{Representation of the solid diffusion angle}
	\end{figure}
	The "\NewTerm{cross-section}\index{cross-section}" or "\NewTerm{differential cross-section}\index{differential cross-section}" being defined in Nuclear Physics and Particle physics by:
	
	Starting from (relation proved just earlier above):
	
	and using the following usual derivative proved in the section of Differential and Integral Calculus:
	
	we have:
	
	Hence:
	
	Therefore:
	
	Thus:
	
	We have then, remembering that the member on the left is nothing but the effective section:
	
	and it is customary to take the absolute value to define the "\NewTerm{effective (differential) section of Rutherford (or Coulomb)}\index{effective (differential) section of Rutherford (or Coulomb)}":
	
	Interestingly, a fully quantum mechanical calculation (albeit non-relativistic) would yield exactly the same answer!
	
	We notice several interesting things:
	\begin{enumerate}
		\item For an angle of incidence equal to zero, the effective cross-section diverges (because of the sinus in the denominator)

		\item The effective section decreases according to the square of the kinetic energy of the incident particle

		\item The expression is valid whatever the charges involved (positive or negative)
	\end{enumerate}
	Thanks to the Rutherford / Coulomb scattering, Rutherford was able to determine an approximation of the size of the nucleus of the atom (bombardment of a gold leaf\footnote{His best estimate came not from his famous gold foil but form lighter aluminum foil, a much less exotic material.} using alpha nuclei) as we did notice it at the beginning of the section of Corpuscular Quantum Physics. The reasoning applied is as follows to determine a lower bound of the radius of the nucleus:
	
	The total energy of a rotating system is the kinetic energy of translation summed with the kinetic energy of rotation, summed to the potential energy. Which gives us:
	
	By denoting by $L$ the angular momentum given by $L=mr^2\dot{\theta}$ (\SeeChapter{see section Classical Mechanics page \pageref{angular momentum}}) we have:
	
	hence:
	
	It therefore follows:
	
	From this it follows that the angle associated with two radial distances $r_1$, $r_2$ is given by:
	
	The figure below shows a process of collision by a central potential $U(r)$. The incident particle has an initial speed:
	
	in $t=-\infty$ with $z(t=-\infty)$ and $y(t=+\infty)$ by symmetry again.
	\begin{figure}[H]
		\centering
		\includegraphics[scale=1]{img/atomistic/rutherford_radius_target.jpg}
		\caption[]{Schematic approach for determining the radius of the target}
	\end{figure}
	The angle $\Theta$ is the angle of deflection when the incident particle approaches the diffuser at the minimum distance $r_{\min}.$

	Let us return to our equations where the angular momentum is linked to the impact parameter by the relation $L=mvb$ or written differently:
	
	We can therefore write after simplifications:
	
	Where we put $E\cong E_c$ (the rotation energy and the potential energy are considered negligible with respect to the kinetic energy) and:
	
	The minimum approach distance is thus determined by the largest zero of the denominator:
	
	That is to say (trivial):
	
	So we have:
	
	As we see in this last relation, the incident particle will undergo a frontal collision when $b=0$. Therefore, the value of the maximum approach is:
	
	Rutherford's experiment made it possible to estimate the size of the atomic nucleus. Indeed, the particles that have bounced back on the nucleus with a diffusion angle of $180^\circ$ (we are then talking of "\NewTerm{backscattering}\index{backscattering}") are those that have approached nearest to the latter. Since we have:
	
	with an initial kinetic energy of $7.7$ [MeV], Rutherford found for the radius of the gold atom ($Z = 79$) with alpha particles ($Z = 2$) a value of:
	
	Thus, the nucleus is not punctual but of the order of ten tens of femtometers
	
	\subsubsection{X-rays and Gamma rays}
	The fundamental difference of this type of radiation, with respect to the $\alpha$, $\beta^-$, $\beta^+$ is that it does not carry an electric charge and therefore does not have a coulombic interaction with the electronic cloud of the medium traversed. Consequently, the photon follows a straight path without loss of energy until it encounters a particle on its path (electron, nucleus) where it will undergo an interaction that profoundly alters its state.

	"\NewTerm{Gamma radiation}\index{Gamma radiation}" is a high-energy electromagnetic radiation produced by a nuclear phenomenon, while "\NewTerm{X-rays radiation}\index{X-rays radiation}" are high-energy electromagnetic radiation produced during atomic or molecular phenomena. The photon is the elementary particle that is associated with these electromagnetic waves. The X and gamma photon are therefore of the same nature but of different origins, they therefore have identical properties that depend on their energy.

	Let us recall that:
	
	By crossing mater a photon can interact with:
	\begin{itemize}
		\item One of the electrons of the atom cloud

		\item The nucleus of the atom

		\item  The electric field of charged atomic particles

		\item The meson field of the nucleons (strong interaction)
	\end{itemize}
	The result of the interaction can be schematized as follows:
	\begin{itemize}
		\item The photon is deflected by conserving its energy, then there is "\NewTerm{total diffusion}\index{total diffusion}" of the energy and the process is say to be "\NewTerm{coherent}" (elastic).

		\item The photon is deflected and its energy decreased, then there is "\NewTerm{partial diffusion}\index{total diffusion}" of energy, the other part is absorbed by matter, the process is then say to be "\NewTerm{incoherent}" (inelastic).

		\item The photon disappears, there is "\NewTerm{total absorption}\index{total absorption}" of its energy by matter.
	\end{itemize}
	We can prove that the macroscopic characteristics of these interactions in the framework of a thin and collimated beam lead to an exponential law of attenuation of photonic radiation in matter. This means that for photons there is no finite path (!) as for charged particles; It can never be assured that at a given distance all the photons of a beam have interacted.

	The number of particles interacting with the material obviously depends on the intensity $I$ and the type of material traversed (characterized by the "\NewTerm{linear attenuation coefficient $\mu$}\index{linear attenuation coefficient}") and its thickness $x$.

	We have:
	
	the "$-$" sign being there to show a decrease. We easily solve this differential equation (it is simply the Beer-Lambert law that we have already studied in the section of Geometric Optics):
	
	with $I_0$ being the initial intensity or "\NewTerm{flow rate of fluence}" and $\mu$ being the coefficient of linear attenuation in [cm$^{-1}$] which takes into account all the possible attenuation effects.
	\begin{tcolorbox}[title=Remark,colframe=black,arc=10pt]
	Often in the tables, we find the coefficient of attenuation mass $\mu/rho$ expressed in [cm$^2\cdot$g$^{-1}$]. We have then:
	
	\end{tcolorbox}
	In the case of an absorbent containing several homogeneously distributed chemical elements, the attenuation coefficient is equal to:
	
	where $\mu_T$ is the absorption coefficient of the absorbent, $\mu_i$ the absorption coefficient of the element $i$, $\rho_T$ the density of the absorbent, $\rho_i$ the density of the element $i$, $w_i$ being the mass fraction of the element $i$ in the absorbent.

	Let us now take a microscopic approach:
	
	Let a beam of $I$ [particles$\cdot$ cm$^{-2}\cdot$s$^{-1}$] striking perpendicularly the surface of a material of thickness $\mathrm{d}x$ and of atomic density $N$ [atoms$\cdot$cm$^{-3}$]. If we consider the particles striking tje surface $S$, the latter can theoretically encounter $N\cdot S\cdot \mathrm{d}x$ target atoms in this layer. The number of particles interacting will be proportional to the intensity times this number and we will have:
	
	where $\sigma$ is the constant of proportionality, named "\NewTerm{microscopic cross section}\index{microscopic cross section}". Its units are often expressed in [barn]\index{barn} ($1$ [barn]=$10^{-24}$ [cm$^2$]). 
	\begin{tcolorbox}[title=Remark,colframe=black,arc=10pt]
	\textbf{R1.} The atomic density $N$ is equal to $\rho N_\text{Av.}\cdot M_m^{-1}$ where $\rho $is the density in [g$\cdot$ cm$^{-3}$] of the target, $N_\text{Av.}$ the Avogadro constant ($6.022\cdot 10^{23}$ [atoms$\cdot$mole$^{-1}$]) and $M_m$ is the molar mass of the target expressed in [g$\cdot$ mole$^{-1}$].\\

	\textbf{R2.} If we admit that the diffusion centers are the electrons and not the target atoms, then we have to replace $N$ by $N_e=N\cdot Z$.
	\end{tcolorbox}
	From where we get:
	
	By identifying the macro and microscopic aspect, we see that $\mu$ plays the same role as $ZN_\text{Av.}\rho\sigma/M_m$ and that we find that the effective section can be written as:
	
	and in the hypothesis that the electron constitutes a "\NewTerm{sphere of action}" having a frontal surface $\pi r_e^2$, where $r_e$is the radius of the sphere of action then:
	
	and we have:
	
	By definition, we name the "\NewTerm{Half-Attenuation Elbow (HAE)}"  the thickness of the material that divides the fluence rate $I_0$ by a factor of two. Therefore:
	
	In radioprotection, we sometimes use the notion of attenuation layer to the tenth TVL (Tenth Value Layer) given by:
	
	We sometimes also use the "\NewTerm{relaxation length}", which represents the thickness from which the intensity of a monoenergetic beam is decreased by a factor $e$, and which is therefore given by:
	
	This value is much more useful than the others, because it is also the average distance at which the first photon collision takes place.
	\begin{tcolorbox}[title=Remark,colframe=black,arc=10pt]
	The Gamma irradiation is anecdotally used in the conservation of the heritage of organic objects. Indeed, during the discovery of archaeologists of ancient works or vestiges, the latter are attacked by microorganisms that will destroy these objects over time. The gamma radiation will allow, without destroying the objects, to kill by gamma irradiation all these microorganisms. The best known example is the irradiation of the Tutankhamun's mummy during $10$ hours in the CEA laboratories.
	\end{tcolorbox}
	The known microscopic causes of the attenuation of a photon beam (neutral from the Coulombian point of view) which deserve our attention in the energy domain of gamma or X-ray photons are:
	\begin{itemize}
		\item Coherent Thomson scattering

		\item Coherent Rayleigh scattering

		\item Coherent Delbrück scattering

		\item Raman scattering (inelastic equivalent of Rayleigh scattering)

		\item Coherent Compton scattering (already partially seen earlier above)

		\item Photoelectric absorption (already partially seen earlier above)

		\item Photonuclear reaction

		\item Creation of electron-positron pairs (already partially seen earlier above)
	\end{itemize}
	Although we can now discuss these effects, it is impossible for us in the present state of this book to introduce the mathematical formalism used to determine the effective cross-section of each of these scattering.
	
	\paragraph{Electron-Positron pair production}\mbox{}\\\\\
	The concept of the wave-particle duality which means all matter has both "particle-like" and "wave-like" nature has been the inception of the quantum physics. Especially, for the "particle-like" nature, there are some examples that can be an empirical proof: first two processes as proves are the photoelectric effect and the Compton effect. Those two effects are well-known processes of photon absorption or scattering that photon loses its energy by interacting with other matter as a "particle-like" things.
	
	The final one for proving "particle-like" nature of all matter is "\NewTerm{pair production}\index{pair production}" or especially the special case of "\NewTerm{Electron-Positron pair production}\index{electron-positron pair production}". It was first observed by the physicist, Patrick Blackett, who was the winner of the 1948 Novel Prize in physics. While both the two processes mentioned above - the photoelectric effect and the Compton effect - usually occur in low energy and middle energy condition, pair production interaction is known to occur when photon has high enough energy before its collision.
	
	The pair production is a crucial example that photon energy (gamma ray) can convert into kinetic energy as well as rest mass energy:
	
	Schematic diagram about the process of pair production is shown in the figure below. The high-energy photon that has energy $h\nu$ loses its entire energy when it collides with nucleus. Then, it makes pair of electron and positron and gives kinetic energy to each particle.
	
	Basically, these interactions are ruled by three kind of the law of conservation: total energy, momentum, and electric charge.
	
	During the creation of pairs, the photon absorbed in the electric field of the nucleus can generate an electron-positron pair. For the interaction to take place, the energy of the photon must be greater than the $2m_0c^2$ (about $1.02$ [MeV]) that is to say the energy at rest of the electron-positron pair.

	This effect is important for high energies and high atomic numbers. The positron created is braked in matter just like an electron and, at the end of the course, it annihilates with an electron to give rise to two photons of $0.511$ [MeV] ("\NewTerm{annihilation photons}") emitted almost at $180^\circ$ (the whole linear momentum being transformed into energy explains the value of this angle, so the final linear momentum is zero).
	
	The creation of a pair obviously costs at least the mass energy of the electron and the positron, or $2m_0c^2$. The energy balance is then divided into the kinetic energy of the two particles:
	
	The necessity of simultaneously satisfying the conditions of conservation of the mass energy and the linear momentum the other hand imposes on the materialization effect to take place in the vicinity of a material particle (nucleus most of time) which participates in the phenomenon. Indeed, in vacuum, the two conditions are contradictory! The linear momentum of each electron is:
	
	We have proved in the section of Special relativity that:
	
	And therefore:
	
	Therefore:
	
	The original photon has:
	
	which we introduce into the energy conservation equation and with the help of the relation $p_{\beta^+,\beta^-}$ we have:
	
	which shows that by the term $c^2/v>1$ that the nucleus must take away a part of the linear momentum since:
	
	\begin{figure}[H]
		\centering
		\includegraphics[scale=1]{img/atomistic/pair_creation_linear_moment_conservation.jpg}
	\end{figure}
	The presence of an electric field of a heavy atom such as lead or uranium is then essential in order to satisfy conservation of momentum and energy. In order to satisfy both conservation of momentum and energy, the atomic nucleus must receive some momentum.
	
	\begin{figure}[H]
		\centering
		\includegraphics[scale=1]{img/atomistic/positron_electron_pair_creation.jpg}
		\caption{Electron-Positron pair production in a bubble chamber}
	\end{figure}
	\begin{tcolorbox}[title=Remark,colframe=black,arc=10pt]
	Pair production is invoked to predict the existence of hypothetical Hawking radiation. According to quantum mechanics, particle pairs are constantly appearing and disappearing as a quantum foam. In a region of strong gravitational tidal forces, the two particles in a pair may sometimes be wrenched apart before they have a chance to mutually annihilate. When this happens in the region around a black hole, one particle may escape while its antiparticle partner is captured by the black hole.
	\end{tcolorbox}
	
	\pagebreak
	\subsection{Liquid Drop Model of Nucleus}\label{liquid drop model}
	At the beginning of the 21st century there is no general theory that underlies all the properties experimentally discovered relative to the nuclei. The nuclei has multiple facets which to this day can not be reconciled in a single theory (quantum chromodynamics not being able to model the nucleus).

	The most accurate information on the radii of nuclear nuclei and more generally on the charge density of nuclei comes from measurements of electron scattering. The latter, considered to date as elementary particles, do not undergo the strong nuclear force guaranteeing the cohesion of the nucleus and can be considered in theoretical treatments as punctual.

	Here is basically the state of our knowledge about the nuclei at the beginning of the 21st century:
	\begin{enumerate}
		\item The nucleon-nucleon nuclear interaction potential is a priori attractive at the nucleus level but sometimes also repulsive if the distance becomes too small between its constituents.

		\item The nuclear interaction is small in scope and presents a phenomenon of saturation such as if each nucleon were directly linked only to its direct neighbors (after having bound to a few nucleons its possibility of binding is exhausted), unlike  the electrostatic force.

		\item The nuclear force is a priori independent of the electric charge. It acts as well between neutrons, as between neutrons and protons and protons-protons.
	\end{enumerate}
	In the naive model that we will study here (which explains quite well the binding energy of the nucleons), the nucleus can be assimilated to a nuclear liquid drop in a first spherical approximation and incompressible, of constant density by volume:
	
	where $A$ still represents in the framework of this section the number of mass.

	Which leads to:
	
	with:
	
	However, the modeling is already failing here because the isotopes of mercury have larger radii than those predicted in the framework of this model and this with strong variations (these variations being irreconcilable with the regular evolution of a drop as a function of its number of constituents). There are even cases where by removing neutrons, the radius of the nucleus increases very significantly!

	Let us now examine the different energies involved. We will construct on the path the famous semi-empirical formula of von Weizsäcker.
	
	\subsubsection{Volumic binding energy}
	The strong nuclear interaction confers a "\NewTerm{volumic binding energy}" of the form:
	
	named the "\NewTerm{volume term}", where is $a_V$ is a predetermined constant (in a first time...) experimentally being approximately equal to:
	
	and we will prove how to theoretically determine its value a little further below during our determination of the "Pauli energy".

	The schematic representation of the idea of "volumic binding energy" (with directly related neighbors) gives:
	\begin{figure}[H]
		\centering
		\includegraphics[scale=1]{img/atomistic/liquid_drop_model_volumic_binding_energy.jpg}
	\end{figure}
	where the colors (red, white) are there only to illustrate the set of binding energies that we will also see later. However, these are obviously nucleons (protons and neutrons).
	
	\subsubsection{Superficial binding energy}
	The nucleons close to the outer surface of the nucleus are less linked via the strong interaction than the deep nucleons since they have fewer direct neighbors. It is therefore necessary to depart from the idea that each constituent has the same volumic binding energy and subtract from the totality of the latter a "\NewTerm{superficial binding energy}" proportional to the surface of the nucleus which is:
	
	We then have for the superficial energy:
	
	named the "\NewTerm{surface term}" where $a_s$ is a constant determined experimentally as being approximately equal to:
	
	The schematic representation of the idea of "superficial binding energy" (with directly related neighbors) gives:
	\begin{figure}[H]
		\centering
		\includegraphics[scale=1]{img/atomistic/liquid_drop_model_superficial_binding_energy.jpg}
	\end{figure}
	It is this term that makes we think to a liquid drop. Indeed, in a liquid drop, the forces are also assumed to be short-range (Van der Waals forces), and thus saturate what immediately causes a surface tension.
	
	We thus have up to now the total potential binding energy of the nucleus which is given by a part of the "von Weizsäcker semi-empirical formula" (or "Bethe-Weizsäcker formula"):
	
	
	\subsubsection{Coulomb electrostatic repulsive energy}
	We must also take into account the usual "\NewTerm{electrostatic binding energy}" or "\NewTerm{Coulomb binding energy}" which results from the electrostatic repulsion force between protons:
	\begin{figure}[H]
		\centering
		\includegraphics[scale=1]{img/atomistic/liquid_drop_model_coulomb_repulsive_energy.jpg}
	\end{figure}
	As it is repulsive, it decreases the binding energy (so this will be a negative term). To get the electrical potential energy, let us recall that we have already proved in the section of Astronomy that for the gravitational energy we had:
	
	It then comes immediately for the electrostatic (coulombian) case:
	
	named the "\NewTerm{Coulomb term}" with the computable constant:
	
	We thus have up to now the total potential binding energy of the nucleus which is given by a part of the von Weizsäcker semi-empirical formula:
	
	
	\subsubsection{Energy of asymmetry (Pauli energy)}
	This energy term is inspired by the model of the nucleus based on the Fermi gas where we consider the nucleus as a set of $A$ (quasi) free nucleons enclosed in a rectangular box having the dimensions of the nucleus and respecting quantization rules (what experiment seems to confirm). Quantum physics is therefore not unique to the atomic world but also to the structure of the nucleus (we could suspect it).

	We have then proved in the section of Electrokinetics (in our study of the theory of bands) that under certain (strong!) precise conditions, the maximum number of states in a spherical volume was given for fermions by:
	
	where for recall $k_F$ is the Fermi wave number\index{Fermi wave number} that it is more usage in the nuclear field to write in another way using the de Broglie relation (\SeeChapter{see section Wave Quantum Physics page \pageref{de Broglie relation}}):
	
	Thus:
	
	We can then have a number of neutrons ($N$) and a number of protons ($Z$) respectively equal at most to:
	
	Knowing the expression of the volume of a nucleus, under the hypothesis of modeling by a spherical and incompressible liquid drop, we have explicitly:
	
	We have, therefore, in extenso:
	
	Assuming that:
	
	We then have for the linear momentum (of the nucleus considered as (quasi) free in the nucleus ... by construction of the assumptions of the theory of bands):
	
	Knowing experimentally $R_0$, we extract a numerical value of it. From this we can derive, taking the classical formulation of energy (hence non-relativistic), that the state of maximum energy state (fermi level of the nucleus) is then:
	
	This being done, let us now calculate the average kinetic energy per nucleon. We then:
	
	The total kinetic energy of the nucleus is then (by approximating the mass of the neutron as being equal to that of the proton):
	
	However, as we have proved above that:
	
	Then it comes:
	
	And if we put:
	
	Then we have:
	
	We know that:
	
	We will seek to get a similar relation by making an clever approximation. For this, let us recall that $A = Z + N$, we will consider the ratio:
	
	Which will be supposed small ... We have then:
	
	We have also:
	
	It then comes:
	
	The Taylor's second-order expansion in $I$ gives:
	
	and therefore:
	
	We then have:
	
	Thus we have kill two birds with one stone: on the one hand we have identified the value of the coefficient of the term of volumetric bonding energy $a_V$ as we promised earlier and on the other hand we deduce from it the term energy of asymmetry:
	
	We then have a term of potential energy which appears in the form:
	
	with:
	
	\begin{figure}[H]
		\centering
		\includegraphics[scale=1]{img/atomistic/liquid_drop_model_asymmetry_energy.jpg}
	\end{figure}
	Since this term is zero when the number of neutrons is equal to the number of protons, we can understand the origin of the valley of stability a little bit better.
	
	We thus have up to now the total potential binding energy of the nucleus which is given by a part of the on Weizsäcker semi-empirical formula:
	
	
	\subsubsection{Pairing Energy}
	A systematic study of nuclei shows that they are more stable when they consist of an even number of neutrons or protons. Empirically, we write this fact by subtracting the following pairing energy:
	
	where $\delta$ is equal to $-11.2$ [MeV] if $Z$ and $N$ are even, $0$ if $A$ is odd and $11.2$ [MeV] if $Z$ and $N$ are odd.
	\begin{figure}[H]
		\centering
		\includegraphics[scale=1]{img/atomistic/liquid_drop_pairing_energy.jpg}
	\end{figure}
	Finally, we have the total potential binding energy of the nucleus given by the "\NewTerm{semi-empirical formula of von Weizsäcker}\index{semi-empirical formula of von Weizsäcker}\index{von Weizsäcker semi-empirical formula}":
	
	that we also found in some textbooks in the following form:
	
	The mass energy of the nucleus can then be written:
	
	\begin{tcolorbox}[title=Remark,colframe=black,arc=10pt]
	The numerical values of the constants are not yet internationally standardized and we can find in the literature several sets of different values.
	\end{tcolorbox}
	Let us notice that the model of the liquid drop also fails to explain that the elements of fissions are not of symmetrical size (the model of liquid drop favoring a fission in two cores of the same size).

	We have for the theoretical model above the corresponding graphical representation:
	\begin{figure}[H]
		\centering
		\includegraphics[scale=1]{img/atomistic/von_weizsacker_energies_plot.jpg}
		\caption{Schematic representation of the semi-empirical formula of von Weizsäcker}
	\end{figure}
	We see above that the average binding energy can be considered very approximately as constant. This can also be interpreted as a force exerted by a limited number of partners. We speak of "saturating force". By "saturating force" we mean that for a given force there is a limit to the number of nucleons placed side by side from which the addition of a nucleon only provides a constant additional binding energy. It is therefore the close neighbors who bring the force to its level, the arrival of new neighbors only subsequently supporting the average value reached. This is why we say that the energy of volume is calculated only with direct neighbors and that the model is also assimilable to a liquid drop (the molecules of a drop of water being sensitive only to directly adjacent molecules).

	This approach also explains the increase in binding energy per nucleon for low masses. Indeed, consider a so-named force of type "F2". We have then the construction of the nuclei thanks to this type of geometry:
	\begin{figure}[H]
		\centering
		\includegraphics[scale=1]{img/atomistic/saturing_f2_force_type.jpg}
		\caption[]{Schematic representation of the saturating force of type F2}
	\end{figure}
	Each nucleon has two bonds (except for the case $A = 2$ which explains that the binding energy increases for the small $A$) and exhaust the possibilities of the force of the F2 type. This limitation imposes no restrictions on the size of the objects to be built and we can imagine constructions as large as we want, stabilized by this type of force.
	
	If we calculate the total binding energy, we will have $1\cdot E_p$ for $A = 2$, where $E_p$ is the potential binding energy provided by a bond. Similarly, we will have $3\cdot E_p$ for $A = 3$, $4\cdot E_p$ for $A = 4$, $5\cdot E_p$ for $A = 5$, and so on... The binding energy per nucleon will then be $E_p/2 $for $A = 2$ and then $E_p$ for all the other masses, for a constant force of type F2. Therefore this force saturates beyond $A = 2$.

	If we take a F3 force type, we will have:
	\begin{figure}[H]
		\centering
		\includegraphics[scale=1]{img/atomistic/saturing_f3_force_type.jpg}
		\caption[]{Schematic representation of the saturating force of type F3}
	\end{figure}
	If we calculate the total binding energy, we will have $1\cdot E_p$ for $A = 2$, where $E_p$ still remains the potential binding energy provided by a bond. Similarly, we will have the $3\cdot E_p$ for $A = 3$, $6\cdot E_p$ for $A = 4$, $7\cdot E_p$ for $A = 5$, etc.
	
	\begin{tcolorbox}[title=Remark,colframe=black,arc=10pt]
	We notice that for $A = 5$ only $2$ bounds can start from the last vertex of the pentagon, otherwise another vertex of the pentagon would have $4$ bounds and not $3$ (force F3).
	\end{tcolorbox}
	The asymptotic value of the binding energy per nucleon will then be $1.5\cdot E_p$ for all masses greater than $A = 3$, for a constant force F3. Therefore this force saturates beyond $A = 3$.

	We can continue so on with nuclei that include more and more nucleons.

	In the particular case of a force that could interact with all other surrounding nucleons, there will be as we have proved in the section of Graph Theory for complete graphs ("graph size"):
	
	bounds and thus a behavior in $(A-1)/2$ in the average binding energy (since it is simply a matter of dividing by $A$ the number of bonds). Thus, in the classical case, the binding energy would only increase what is not compatible with the experiment.
	
	\pagebreak
	\begin{tcolorbox}[colframe=black,colback=white,sharp corners]
	\textbf{{\Large \ding{45}}Example:}\\\\
	Let us see some applications of this model starting by examining the predictions on the fission of the uranium $_{92}^{236}\mathrm{U}$ in two equal sub-products in mass and in charge:
	
	and this by using the liquid drop model by neglecting the terms of asymmetry and pairing energy.

	For this, let us recall that:
	
	and we will take as values of the constants:
	
	Let us evaluate the difference of energy between on the one hand the $2$ nuclei resulting from the fission and on the other hand the starting nucleus.

	We have then:
	
	as:
	
	The development is simplified in:
	
	\end{tcolorbox}
	
	
	\begin{tcolorbox}[colframe=black,colback=white,sharp corners]
	If there is fission, we will have $\Delta E\ge 0$, therefore:
	
	thus:
	
	Or rearranged:
	
	Either by putting the constants, it gives:
	
	But for $_{92}^{236}\mathrm{U}$, we have:
	
	So according to the above result $_{92}^{236}\mathrm{U}$ can fission symmetrically but this is not the case in the experiment, because in reality we have:
	
	\end{tcolorbox}
	There is, however, another possible theoretical approach which gives a result more in agreement with the experiment. Indeed, we can imagine that a nucleus can undergo fission if the force derived from the superficial energy is exactly compensated by the Coulomb force.

	In the end, we will compare the ratio $Z^2/A_m$ with the result we will get with the new approach..

	Let us first recall that we have seen in the section of Electrostatics (among others ...) that the electrostatic force derives from the electrostatic potential:
	
	where the radius $R$ has in our case the value of the nuclear radius, with for recall:
	
	We then have:
	
	Therefore:
	
	If the Coulomb repulsion prevails, the fission prevails and we then have:
	
	Therefore:
	
	After simplification, it remains:
	
	Therefore:
	
	Since this ratio for $_{92}^{236}\mathrm{U}$ is close to $36$ and therefore less than $52$, this approach naively explains why there can be no symmetrical fission (as experience confirms it!).
	
	\begin{figure}[H]
		\centering
		\includegraphics[scale=0.8]{img/atomistic/liquid_drop_model.jpg}
		\caption[Liquid drop model of Uranium 235 nuclear fission]{Liquid drop model of Uranium 235 nuclear fission (source: OpenStax)}
	\end{figure}
	
	\begin{figure}[H]
		\centering
		\includegraphics[scale=0.55]{img/atomistic/nuclear_plant.jpg}
		\caption[Pressurized water reactor]{Pressurized water reactor (author: Gloria Faccanoni)}
	\end{figure}
	
	\begin{figure}[H]
		\centering
		\includegraphics[scale=0.5]{img/atomistic/atom_size.jpg}
		\caption[Typical atom size]{Typical atom size (source: ?)}
	\end{figure}
	
	\begin{figure}[H]
		\centering
		\includegraphics[scale=0.8]{img/atomistic/chain_reaction.jpg}
		\caption[Uranium 235 fission chain reaction]{Uranium 235 fission chain reaction (source: OpenStax)}
	\end{figure}
	
	
	\begin{flushright}
	\begin{tabular}{l c}
	\circled{80} & \pbox{20cm}{\score{4}{5} \\ {\tiny 95 votes,  71.58\%}} 
	\end{tabular} 
	\end{flushright}

	%to make section start on odd page
	\newpage
	\thispagestyle{empty}
	\mbox{}
	\section{Quantum Field Theory}\label{quantum field theory}
	\lettrine[lines=4]{\color{BrickRed}B}efore the formulation of quantum physics, particles and fields were considered separate but linked. The particles have certain intrinsic characteristics (such as mass and electric charge) and produce fields (electromagnetic and gravitational). Each force field emanating from the particles and fills the space around them. The fields can store and transport energy; they are, in this sense, the real continuum that bind particles and communicate the interactions between them. It was considered that the particles were composed of matter and energy fields were compounds. The concept of force field was the alternative of the 19th century old and mysterious action at a distance. Particles that don't react to any force field are unobservable and physically are no sense. Similarly, force fields that act on no particles are also meaningless. The concepts of particles and fields thus have a sense only when they are connected.
	
	The notion of field began to be fundamentally changed with the introduction by Albert Einstein of the concept of "photon". Under this new concept, the electromagnetic field has no energy distributed in a continuous manner in space. The photon is the "\NewTerm{quantum of the electromagnetic field}\index{quantum of the electromagnetic field}". It carries the energy and momentum of the field. The electromagnetic interaction between two charged particles and the transfer of energy and the amount of movement of a particle to another must therefore take place by the exchange of electromagnetic quanta of energy: the photons. The theory of such interactions (between charged particles), named "\NewTerm{quantum electrodynamics}\index{quantum electrodynamics}" (QED), was the first successful application of these ideas (it allows to prove the fine structure of the Sommerfeld model, to explain the spin of the electron, etc.) and it is to it that we will look here.
	
	In this section, we do not wish to make a complete course of quantum filed theory because for reminder, the entire book is only intended to give the basics of what an engineer must know in the early 21st century and incidentally the author to have fun studying subjects that he had not seen during his school career. As such, the interested reader to deepen over this matter may be read to best book we've ever had hands so far (detailed, simple and pedagogical developments with many practical cases) on the subject which is \cite{desai2010quantum}.
	
	\begin{tcolorbox}[title=Remark,colframe=black,arc=10pt]
	Quantum field theory is the application of quantum mechanics to fields. It provides a framework widely used in particle physics and condensed matter physics. The basics of quantum field theory to which we will limit our study were developed between 1935 and 1955, mainly by Paul Dirac, Wolfgang Pauli, Sin-Itiro Tomonaga, Julian Schwinger, Richard Feynman and Freeman Dyson.
	\end{tcolorbox}
	
	Before we launch into the calculations (see below), let us show that the approach suggested above, can be considerate with a simple formalism (pedagogical).
	
	Readers should however remember that simple approaches require sometimes erroneous mental constructions (overly simplistic) compared to reality, but which meet the objective: to have a more or less understandable and intuitive model.
	
	Consider at this purpose the figure below (representation of the elastic collision of two electrons):
	\begin{figure}[H]
		\begin{center}
		\includegraphics{img/atomistic/feynman_pseudo_diagram.jpg}
		\end{center}	
		\caption{Example of pseudo Feynman-diagram}
	\end{figure}
	This figure is named, and wrongly (!), in many educational books \NewTerm{Feynman diagram}\index{Feynman diagram}" (in reality it is only a diagram that looks like a little bit because the real Feynman diagrams are intended to calculate Wick products in a perturbation series).
	
	Let us suppose that two electrons which are represented, initially move at the same speed. They approach first and then move away from each other along a straight line in space that is projected on the time axis, in the direction of increasing time. The left electron emits a photon (the wavy line), and for a time $\Delta t$, there are two electron and a photon. The electron on the right then absorbs the photon and the interaction is temporarily ended; other photons will eventually go back and forth between the electrons. The average force is proportional to the transfer rate of the amount of movement due to the exchange of photons. The probability of the emission or absorption of photons by a particle is connected to its electric charge. The force must be proportional to the product of the charges in interaction (according to Coulomb's law). Think of the repulsive force between two astronauts floating in space and sharing a ball in one direction and then the other (this is an educational approach to the problem but does not apply for example to the attraction between two particles of opposite charges!). However, the opposite phenomenon of attraction cannot be viewed in this way but only in formal mathematical form.
	
	The collision shown in the figure above is elastic; the energy of each electron is unchanged in the collision. Despite this, for a time $\Delta t$, the system contains an additional energy $h\nu$ corresponding to the photon. Meanwhile this time $\Delta t$, conservation of energy is apparently non respected! Can we tolerate this situation? The answer given by modern physics, is: Yes! But it can never be observed (...). In other words, because there is always some uncertainty $\Delta E$ on the measured value of the energy of a system. The Heisenberg uncertainty principle involving (see the simplified proof in the section of Wave Quantum Physics) that:
	
	A violation of the energy conservation law until a given amount $\Delta E$ will be hidden by uncertainty on energy provided that the time available to make the observation is sufficiently small as having for upper value:
	
	obviously a value less than $\Delta t$ also satisfies the condition. We can therefore write:
	
	The uncertainty on the energy exceeds the energy of a photon $h\nu$ if the photon exists for a shorter time than:
	
	The photon can then be observed on a maximum distance of:
	
	and as the frequency $v$ can be arbitrarily small, the scope of the transmitted force by the photon without mass is unlimited. It may seem in this relation that the scope is limited to a free photon. But this would be forgetting (\SeeChapter{see section Wave Quantum Physics page \pageref{free particles}}) that a free photon does not exist, because it would have a totally unknown frequency. Therefore the interaction distance would be also undetermined.
	
	These exchange of quanta, that are unobservable, are named "\NewTerm{virtual photons}\index{virtual photons}". Since photons are not charges, we also say that the interaction is made by "\NewTerm{neutral current}\index{neutral current}".
	
	A much more satisfactory approach is that of using mass as energy term:
	
	Thanks to this relation, it is possible to know the time during which a virtual particle can travel a distance that would be:
	
	We will see later how to approximately determine the mass of virtual particles that mediate in the nuclear forces which will allow us to estimate the duration of interactions as being of the order of $10^{-24}\; [\text{s}]$.
	
	In the late 1920s, it became clear that we could consider each known particle (proton, electron, etc.) as the quantum of a specific field. In this vision, there is an electron field, a proton field, and so on as will be proven below (the Universe would be a set of unified field). Any object is in reality a set of manifestations of observables of field quantas.
	
	Furthermore, we saw that the writing of wave equations for relativistic particles (Dirac equation and Klein-Gordon equation proven in the section of Relativistic Quantum Physics) brings intractable problems in a classical point of view, including negative energies. In fact, this approach is not justified, because according to Einstein equation mass and energy are equivalent, and if we add to that the energy-time Heisenberg's uncertainty principle we find that an infinite number of particles can be created or annihilated, hence the need for a model to take into account not only the properties of a single particle, but of a set of particles, both real and virtual!
	\begin{tcolorbox}[title=Remark,colframe=black,arc=10pt]
	When Fermi formulated his theory of weak interactions in 1932, he used the same principles those of quantum electrodynamics (this is one reason why QED is named the "jewel of physics" - the Standard Model is also based on this theory). Two years later, the Japanese physicist Hideki Yukawa proposed that the weak interaction was due to the exchange of a massive virtual boson.
	\end{tcolorbox}
	
	\subsection{Yukawa potential}
	The best way to argue the example of quantum remains the "proof" of the Coulomb's law (and also of Newton's Law) from the results we have achieved in wave quantum physics (we own these developments to the physicist Hideki Yukawa).

	A simplified version of this proof as we like them in this book is to first remember the free Klein-Gordon equation (\SeeChapter{see section Wave Quantum Physics page \pageref{free Klein-Gordon equation}}):
	
	this equation describes the amplitude dynamics of presence of a particle without spin in time in a given potential.
	
	Consider a static component equation (time-independent) with spherical symmetrical:
	Let us consider a static component of $\Psi$ (time-independent) with spherical symmetry:
	
	The Klein-Gordon equation is then reduced to:
	
	If we divide on the both side of the equalitby by $\hbar^2 c^2$ we get obviously:
	
	Let us recall (\SeeChapter{see section Vector Calculus page \pageref{scalar laplacian}}) the notation of the Laplacian of a scalar field:
	
	and also its expression in spherical coordinates where $r=0$ is identified to the source of the field (\SeeChapter{see section Vector Calculus page \pageref{scalar laplacian in spherical coordinates}}):
	
	As the field $U(r)$ has a spherical symmetry  (depending on $r$ only) the Laplacian is reduced to:
	
	So the equation of the field U (r) is written:
	
	This differential equation has for solution (we guess easily enough that the exponential is a possible solution):
	
	where $C$ is an integration constant.
	
	In the context of the use of natural units (which is most common at this level in the scientific literature) this potential is written:
	
	and is named "\NewTerm{Yukawa potential}\index{Yukawa potential}\label{yukawa potential}".

	The reader will notice that apart from the distance $r$, the other variable in the exponential is the mass (the other terms are universal constants). Consequence: the Yukawa potential is as good a "\NewTerm{scalar field}\index{scalar field}" in the case where mass is zero (see example below) that a "\NewTerm{mass field}\index{mass field}" if the mass is not zero!
	
	This leads us to the following hypothesis: if it is the electric field that holds charged particles in the atom between them (see the treatment of non-mass fields below), it is the mass field which keeps the uncharged particles together in the atom.

	In other words, if particles interact via a mass field of mass $m_0$ (instead of interacting with massless photons), their mutual strength will decrease exponentially (which is very fast).

	Yukawa's approach allows a new point of view to the interpretation of nuclear phenomena, but it is however too naive to adequately explain the strong interactions in general.
	
	\pagebreak
	\subsubsection{Mass fields}
	The physicist Hideki Yukawa therefore proposed in 1935 that nuclear power was own its very short range by the fact that it is transmitted by massive particles (more the mass of exchanged quanta is big, the greater the range of the interaction is reduced) described by the mass field above.
	
	\begin{tcolorbox}[title=Remark,colframe=black,arc=10pt]
	In the historical context of the time, these hypothetical particles were "\NewTerm{mesons}\index{mesons}". But we will see that this assumption did not hold up very long.
	\end{tcolorbox}
	Let's take a closer look to this. Let us write the Yukawa potential as follows:
	
	with:
	
	This notation is not innocent, because as we will see it in detail later, when $R\rightarrow +\infty$ (the case of electromagnetic and gravitational interaction) then $m_0\rightarrow 0$ and then we fall back on the fundamental law of electrodynamics or of gravitation where the particle interaction is respectively the photon (null mass) and the graviton.
	
	Thus, assuming that the radius of the strong nuclear force (cohesion of nucleons between them) is around $R\cong 1\cdot 10^{-15}\;[\text{m}]$ and that of the weak nuclear interaction (which should be the cause of the beta disintegration as we already mentioned in the section of Nuclear Physics) around $R\cong 1\cdot 10^{-18}\;[\text{m}]$, we get the binding interaction energies and their approximate weight immediately:
	\begin{itemize}
		\item For the "\NewTerm{strong interaction}\index{strong interaction}":
		
		that is to say approximately $386$ times the mass of the electron ($m_e$) and $1/5$ the mass of the proton ($m_p$).
	
		Two years after this prediction of Hideki Yukawa, physicists discovered a particle corresponding to this mass: the meson $\mu^{-}$. It was later that it was not good the good particle but a particle of the same type as the electron particle, that is to say a lepton and therefore a fermion\footnote{Go see the table of fundamental particles in the section Elementary Particle Physics to see how physicists classify elementary particles of the standard model} (so it can not be a messenger particle!!). Furthermore, the experiences of diffusion and collisions with protons, deuterons, etc. at energies higher and higher showed that there was a change in the intensity and shape of the strong interaction incompatible with the hypothesis of a single meson. Furthermore, the hadronic resonances showed that there was meson excited states which is hard to imagine for particles considered fundamental in analogy with the photon!!
	
		The particles detected in the laboratory and that seemed to be the best candidates at that time time (since there were several ...) for the strong nuclear force were "\NewTerm{pions}\index{pions (particle)}" (or "\NewTerm{pi mesons}\index{pi mesons}") which are know under three forms (at least at the time we write these lines):
		\begin{gather*}
			\pi^+,\pi^-,\pi^0
		\end{gather*}
		
		and which are almost $270$ times more massive than the electron. So this mass difference clearly indicates that the model of Yukawa is not entirely accurate.

		Before the discovery of quarks (which mesons are made of), the pi mesons (pions) were therefore considered to be the vectors of the strong interaction (today we know that in fact these vectors are gluons).
	
		\item For the "\NewTerm{weak interaction}\index{weak interaction}":
		
		It is therefore a colossal mass, a hundred times the mass of the proton! The vectors of interaction have been candidates that were identified in 1983 in the CERN accelerator. These carrier particles of the weak nuclear force are named "\NewTerm{intermediate bosons}\index{intermediate bosons}" and denoted by $W^+$, $W^-$, $Z^0$.
	\end{itemize}
	These observations led the hypothesis that Yukawa's theory was not a fundamental theory enough even if it explain quite well some properties...

	
	\subsubsection{Non-mass fields}
	Let us now imagine a static scalar field with a spherical symmetric which the photon (particle without spin) is the hypothetical quantum exchange. As the photon's mass is zero, the expression $U (r)$ is reduced 
to:
	
	If we interpret $U (r)$ as the electrostatic potential source of a quantity $Q=nq$ of elementary electric charges $q$, then the constant $C$ in our metric is:
	
	Such that:
	
	As we have:
	
	We get:
	
	Which give us:
	
	We can conclude that if a particle is in a potential field with spherical symmetry $U (r)$, whose photon is assumed to be initially the interacting quantum, then we are dealing with an electrostatic field whose expression is identical to the Coulomb law (this valid once again in a big way the theory Wave Quantum Physics).
	\begin{tcolorbox}[title=Remark,colframe=black,arc=10pt]
	Hubert Reeves and his colleagues astrophysicists proved that at the time of the genesis of the Universe, the smallest deviance in coupling constants would have caused the instability of nucleons and would have condemned cosmic evolution.
	\end{tcolorbox}
	The photon is therefore indeed the quantum interaction of the electric field with spherical symmetry (when the electric charges have a relativistic speed the electric field is not spherical anymore, and the equations become a bit more complicated - see the section of Special Relativity for some details about this) and we should not talk of electric charges but of "transparency" to photons. Indeed, the neutron being globally neutral it should not interact with the electric field, but as it is composed of charged particles (quarks) experiments show an influence in the presence of an electromagnetic field (which photon is the quantum interaction).
	That said, applying the same reasoning, we can still find the Newton's gravitational potential:
	
	This would imply that the quantum interaction of the gravitational field is massless (at least in the case of small masses as we know the Newton's potential  is only an approximation of General Relativity in the case of small masses). Since the gravitational field does not seem to interact with the presence of a magnetic or electrostatic field, this leads us to speculate that the quantum of interaction is not the photon but another particle, we name the "\NewTerm{graviton}\index{graviton}".
	
	\pagebreak
	\subsection{Euler-Lagrange equation for Fields}
	How field theory was introduced from the elementary particles by Dirac is known for historical reasons under the name of "\NewTerm{second quantization}\index{second quantization}".

	It may be useful to identify a possible source of confusion: the fields are not related to the wave-particle duality. What we mean by "field" is a concept that allows the creation or annihilation of particles at any point in space as we will discussed it in the mathematical developments below.

	Recall that we have defined in the section of Wave Quantum Physics during the study of the Schrödingerps evolution equation the Heisenberg's operator necessary for the de Broglie normalization condition:
	
	Differentiating this operator with respect to time, we trivially have trivially (once known the notation seen in the section of Wave Quantum Physics):
	
	where for recall, the commutator of two operators is given (as we have already seen it in our study of adjoint and  Hermitian operators in the section of Wave Quantum Physics) by definition:
	
	This is the Hamiltonian $H$ which appears first in the previous prior previous relation.
	
	Now, as we know, we can substitute to $\dot{X}(t)$ observable that are also well known to us such that we get in the cas of a constant hamiltonian:
	
	But we can just as easily substitute a time-dependent Hamiltonian $H(t)$ such that:
	
	For example, as we know (\SeeChapter{see section Wave Quantum Physics page \pageref{observables and operators}}), $X$ can be replaced by the operators $q(t)\rightarrow \hat{q}(t)$ or $p(t)\rightarrow -\mathrm{i}\hbar\partial_k$. Then in this special cases we see that we can simplify the previous relation by:
	
	This is named "\NewTerm{Heisenberg equation of motion}\index{Heisenberg equation of motion}". 
	
	What is interesting in the two relations $\dot{q},\dot{p}$ obtained previously, is the way in which is realized the connection between quantum physics and classical mechanics. Indeed, we proved in the section of Analytical Mechanics the following relations are will always be valid whatever the subject area:
	
	and also that:
	
	and assuming the generalization to several degrees of freedom as intuitive:
	
	The generalization to several degrees of freedom is immediate and gives us all the following relations (we simplify the notation by omitting the explicit time dependence):
	
	 and also:
	
	We still need two other important relations that we will immediately determined. First, according to the definitions of the commutators, we already prove in the section of Analytical Mechanics:
	
	By cons, it is a little more subtle to prove the value of $[q,p]$ (we joke...). Recall that we had prove in our study functional linear operators (we restrict ourselves to the case of the $x$ here):
	
	and that $q$ represents a generalized coordinate ($x$ for example ...). So we have (results already prove in the section of Wave Quantum Physics...):
	
	The last two relations can be generalized to all necessary components such as:
	
	with for recall (\SeeChapter{see section Tensor Calculus page \pageref{kronecker symbol}}):
	
	which is the Kronecker symbol.
	
	To arrive finally to quantum field theory, we still need to generalize to a continues infinite number of degrees of freedom. Even the simplest of fields is characterized, at a time $t$, by an infinity of continuous quantities:
	
	for any $\vec{x}$. We could then imagine to represent the function $\Phi$ by its values $\Phi(\vec{x},t)$ in a discrete set of points $\vec{x}_i$ that we will ultimately make infinitely dense (beware of the fact that we use the concept of density!). We can also work, to begin, not in all space, but in a finite volume that we will eventually make very large. By doing so, we can find how to generalize the canonical formalism and the quantification process. At the formal level, notwithstanding subtle issues of convergence (see the mathematical parts of this book), the generalization to continuous systems is mainly to replace the quantities with index $i$ by integrals of the arguments $\vec{x}$, and the deltas of Kronecker by the delta of Dirac (on space-time):
	
	Considering then the variational principle as we have studied it in the section of Analytical Mechanics:
	
	and the principle of least action requiring us to have:
	
	where the Lagrangian will be now a function of the field $\Phi(\vec{x},t)$ and of derivative with respect to $\phi(\vec{x},t)$ (since there is no concept of momentum for a field!).
	
	If we divide the above equation by $\delta q(t)=\delta q\neq 0$ we get:
	
	what gives us the right to write:
	
	and imposing an analogy with the concept of field:
	
	where $x:=(\vec{x},t)$ and $\mathrm{d}x^4=\mathrm{d}x_1\mathrm{d}x_2\mathrm{d}x_3\mathrm{d}x_t$.
	
	Finally, since all the following terms are equal to zero, they are equal (we introduce the Euler-Lagrange equation proved in the section of Analytical Mechanics):
	
	in analogy with the field $\Phi^2$ we get (thus it is an intuitive approach with the physicist way of life...):
	
	However, since this writing is not very convenient, it is customary to write the partial differentials (using the natural units of physics) of the components $\partial/\partial t$, $\partial/\partial y$, $\partial / \partial z$, $\partial / \partial x$ in the form $\partial_\mu$, which finally gives us:
	
	Obviously, if we do not use natural units, we should adopt the notation:
	
	The fact that the partial derivation $\partial_\mu$ now acts on all the components and not only on $t$ is due to the change of generalized coordinate $q$ function only from $t$ to a function of the field $\Phi^A$ dependent on $x$, $y$, $z$ and $t$. The reason for this lies in the fact that time and spatial coordinates play the same role, that of describing the space-time continuum on which the physical system evolves.

	This also leads us to write the principle of least action in the following form:
	
	With the action of fields denoted more traditionally:
	
	or to differentiate Lagrangian and Lagrangian density \label{lagrangian density} (we "stylized" sometimes the $L$ to makes the difference between the both):
	
	to be compared to the action of the particle:
	
	In analogy with $p_n=\dfrac{\partial L}{\partial \dot{q}_n}$, we will write:
	
	and in analogy with $H=\displaystyle\sum p_n\dot{q}_n-L$ we write:
	
	But a field is a continuous medium! The sigma sum  is therefore no longer adapted and we must move to an integration over all space-time such that:
	
	In analogy with the Heisenberg equations of motion (this way of doing things is often named the "\NewTerm{principle of correspondence}\index{principle of correspondence}"), we write:
	
	Let us now turn to quantum theory by postulating corresponding of "\NewTerm{Heisenberg field operators}\index{Heisenberg field operators}". Let us recall that we had obtained earlier above that:
	
	which gives us:
	
	If we summarize a little and we display the comparison with Wave Quantum Physics, we have finally:
	\begin{enumerate}
		\item In Wave Quantum Physics (it's nice to look at it isn't it?):
		
		
		\item And the equivalent by the principle of correspondence in Quantum Field Theory (here it don't just look nice... it becomes art!):
		
	\end{enumerate}
	And voila! We have just passed from the parameters of Wave Quantum Physics where the punctual bodies are described by wave functions, to a Quantum Physics where the punctual bodies become continuous fields.

	It remains only to apply this general scheme to concrete examples:

	We will start with a first example taking into account the relativistic aspect. Thus, the simplest non-trivial Lagrangian density that we can construct is of the form (you will see immediately what it will lead us to, which will confirm its theoretical validity - moreover, the development that follows could very well have been presented backwards):
	
	Or more explicitly:
	
	that the physicists name "\NewTerm{scalar field for a free and spinless particle}\index{scalar field for a free and spinless particle}" or "\NewTerm{Klein-Gordon Lagrangian}\index{Klein-Gordon Lagrangian}" for a spin-free particle where we use the usual condensed notations:
	
	and the natural units:
	
	Let us calculate the Euler-Lagrange equation relative to it (normally it is trivial) without forgetting that it is a functional derivative, which greatly simplifies the calculation:
	
	Following the request of a reader here are the details allowing to arrive at this result:
	
	Hence the equation of motion in natural units and daring a dangerous notation for the double partial derivative (the Einstein summation is then implicit...):
	
	or as we will quick see it, in order to fall back on the results obtained in the Relativistic Quantum Physics section, we are obliged to introduce the contravariant and covariant differential operator (\SeeChapter{see section Tensor Calculus page \pageref{contravariant and covariant components}}) with the signature $(-, + , +, +)$:
	
	Let us recall now for comparison purpose that in the section of Wave Quantum Quantum Physics we had obtained the following free Klein-Gordon equation:
	
	also with the signature $(-, +, +, +)$ thus with the Alembertien operator:
	
	We then have a perfect correspondence between the free Klein-Gordon equation and the field equation (written below in natural units as usual):
	
	and it is here that one can eventually feel a shiver in the back and stay admiring in front of the power of mathematical formalism opening new perspectives on how to see the gears of our Universe... Thus, in Quantum Field Theory, the Klein-Gordon equation can be reinterpreted as a field equation! 

	The form of the free Klein-Gordon equation involving fields is sometimes named the "\NewTerm{Klein-Gordon field equation}\index{Klein-Gordon field equation}".
	\begin{tcolorbox}[title=Remark,colframe=black,arc=10pt]
	We can generalize the Klein-Gordon field equation to curved space. Indeed, as:
	
	and therefore:
	
	Then the action $S$ must be corrected using the invariant volume introduced in the section of Tensor Calculus such that:
	
	\end{tcolorbox}
	And even... better ... you will see, we'll do it a little in a blind way ... and therefore!!!: Let us now consider the following Lagrangian (which we assume will be obtained by successive trials and errors... but again we could have done the development backwards), intending to express the "\NewTerm{interaction of an electromagnetic field with a current density}":
	
	where we recognize there the tensors of the electromagnetic field proved and determined in the section of Electrodynamics and Special Relativity and for which, for recall, we have:
	
	In this Lagrangian, let treat the vector potential as a field such that:
	
	Therefore, by decomposing the developments, we get very easily:
	
	In a first step, the reader will verify by doing some relatively elementary tensor calculus that:
	
	Then:
	
	Therefore, the equation of the field is written:
	
	hence:
	
	We must admit that the result is quite beautiful even if the approach is not the most rigorous one! We thus fall back on the Maxwell equation with sources with the same Lagrangian of the field (\SeeChapter{see section of Electrodynamics page \pageref{four vector current}}). Thus, this massless lagrangian is assimilated to the Lagrangian of the vector field of spin $1$ assimilated to the bosons (\SeeChapter{see section Statistical Mechanics page \pageref{fermi dirac distribution}}).

	Let us recall now that we had obtained in the section on Electrodynamics the following action for a charged particle in an electromagnetic field (before a long development which had brought us to the tensor of the electromagnetic field...):
	
	and remembering that (\SeeChapter{see section Electrodynamics page \pageref{action variation}}):
	
	it comes:
	
	The corresponding Lagrangian density is therefore:
	
	So we have finally:
	\begin{enumerate}
		\item The Lagrangian (Lagrangian density) of a charged particle in an electromagnetic field (which we have just obtained):
		
		\item The Lagrangian (Lagrangian density) we get just before (which allowed us to fall back on Maxwell's equations with sources):
		
	\end{enumerate}
	Hence, it is natural to write the "Lagrangian (total Lagrangian density) of the electromagnetic field":
	
	Hence, it is natural to write the "\NewTerm{Lagrangian (total Lagrangian density) of the electromagnetic field}\index{Lagrangian of the electromagnetic field}":
	
	Let us now continue our way with the free Dirac equation! Let us recall that in the Relativistic Quantum Physics section we obtained the free Dirac equation in the form (basically, let us also recall that it is a relativistic equation):
	
	Now let us recall (\SeeChapter{see section Linear Algebra page \pageref{transposed matrix}}) that $(AB)^\dagger=B^\dagger A^\dagger$. Therefore, it comes:
	
	where we have introduced the "\NewTerm{Feynman slash notation}\index{Feynman slash notation}" $\cancel{\partial}$ (less commonly known as the "\NewTerm{Dirac slash notation}\index{Dirac slash notation}").
	
	But, $\partial_\mu^\dagger=\partial_\mu$ and it is super easy to verify (do not forget that we use the Dirac representative form of Pauli matrices!!!):
	
	which leads us to write:
	
	It is then convenient to introduce the "\NewTerm{Dirac adjoint}\index{Dirac adjoint}" or "\NewTerm{adjoint spinor}\index{adjoint spinor}":
	
	\begin{tcolorbox}[title=Remark,colframe=black,arc=10pt]
	Let us recall that $\Psi$ is a matrix-column and $\Psi^\dagger$ a matrix-line. So it comes that $\overline{\Psi}$ is also a matrix-line!
	\end{tcolorbox}
	Using the fact that in the Dirac representation $\gamma^0 \gamma^0 =\mathds{1}$ we can write:
	
	by simplifying the $\gamma^0$ it comes the "\NewTerm{free adjoint Dirac equation}":
	
	What we traditionally write:
	
	The notation $\overleftarrow{\cancel{\partial}}$ meaning that the operator $\cancel{\partial}$ operates on $\overline{\Psi}$ on the left such that:
	
	\begin{tcolorbox}[title=Remark,colframe=black,arc=10pt]
	Some authors write $(\mathrm{i}\cancel{\partial}+m_0)\overline{\Psi}=0$ but this is false because $\overline{\Psi}$ is a line matrix as we pointed out above !!!
	\end{tcolorbox}
	Finally we have for the free Dirac equations:
	
	Let us now suppose that the "\NewTerm{Lagrangian of the free Dirac spinor field}\index{Lagrangian of the free Dirac spinor field}" is of the form (because in the end it is the Lagrangian that interests us!):
	
	where we have for recall $\bar{\Psi}=\Psi^\dagger \gamma^0$. It is therefore the Lagrangian of the spinor field for the particles of spin $1/2$ which are therefore free fermions.

	Considering the quantities $\bar{\Psi}$, $\Psi$ as independent (this is what they are anyway since they are orthogonal) and choosing the spinor field as being $\Psi^\dagger$, we have the Euler-Lagrange equation:
	
	The second term is equal to zero since the Dirac Lagrangian does not contain terms with $\partial_\mu\Psi^\dagger$. In fact it remains:
	
	So we fall back on the free Dirac equation (the same development that can be done for the free adjoint Dirac equation)! Thus, in this framework, the only way to explain the quantum properties of the material containing particles with spin $1/2$ is to involve the fields $\bar{\Psi}$, $\Psi$ representing electrically charged particles, electrons and positrons as we know. We then name then these entities "\NewTerm{Dirac field}\index{Dirac field}" also named \NewTerm{spinor field}\index{spinor field}" or \NewTerm{spin $1/2$ fermion field }\index{spin $1/2$ fermion field}".

	\begin{tcolorbox}[title=Remark,colframe=black,arc=10pt]
	Often, when using the Dirac equation one finds the slash notation introduced in the section of Relativistic Quantum Physics used on four-momentum. Using the Dirac basis for the gamma matrices:
	
	as well as the definition of four momentum (\SeeChapter{see section Special Relativity page \pageref{four momentum}}):
	
	we see explicitly that:
	
	Similar results hold in other bases, such as the Weyl basis.
	\end{tcolorbox}
	
	\subsection{Gauge Theories}
	We will now see a simple approach to a tool that revolutionized the approach of modern particle physics in the mid-$20$th century and that has earned several Nobel Prizes to those who contributed to it.

	We advise very strongly before reading what follows that the reader will also take a preliminary look at the sub-section of Gauge Theory of the Electrodynamics section, as it contains a first example of an invariance of Gauge making emerging a field (the vector potential) necessary to explain certain phenomena at the quantum scale as clearly explained by Pauli's equation (\SeeChapter{see section Relativistic Quantum Physics page \pageref{pauli equation}}).

	Since the early 1980s, popularization magazines have talked a lot when dealing about Quantum Physics on Gauge theories. Electromagnetic interactions and Weak interactions are described jointly by a Gauge theory developed by Glashow, Weinberg and Salam. Strong interactions also seem correctly described by a Gauge theory. It is within the framework of these gauge theories that theoretical physicists try to unify the various fundamental interactions of nature. It is therefore appropriate, even in this book which treats in an elementary way of Quantum Physics, to speak of Gauge theory within the framework of this study domain.
	
	To do this, we will already consider the context that led to the discovery of Gauge invariance in the context of Electrodynamics (see the section of the same name for details) and make a comparison with certain developments seen in the section of General Relativity and the role played by Weyl in the proof of the fundamental principles of a Gauge theory.

	Let us recall that the Special Relativity and General Relativity are based on the premise that there is no absolute reference frame in the universe. We have seen in the section of Special Relativity in a long and broad way that the relations which make it possible to pass the laws of physics from one reference frame to another one depend only on the relative speed between them. Thus, Special Relativity is a theory with global symmetry. We have also seen in the section of General Relativity that the affine connection is the link between the referentials of the local theory (weak field approximation) that is General Relativity.
	
	In 1919, the first experimental observation of the deviation of the light of a star by the gravitational field of the Sun took place. This spectacular confirmation of the theory of General Relativity inspired Hermann Weyl, who proposed the same year a revolutionary conception of Gauge invariance: If the effects of a gravitational field can be described by a connection expressing the relative orientation between local repositories of space-time, can other forces of nature such as electromagnetism also be associated with similar connections?

	We consider two types of Gauge symmetry: one known as "\NewTerm{global Gauge}\index{global gauge}" and the other named "\NewTerm{local Gauge}\index{local gauge}". They are distinguished by the parameter characterizing the phase change of the wave function (we will see this in detail further below).
	
	\subsubsection{Global Gauge invariance}
	We will therefore study the Gauge invariance using the Schrödinger equation and show that even if the results may seem confusing (in the context of complex set $\mathbb{C}$ applications), they nevertheless remain mathematically correct!
	\begin{tcolorbox}[title=Remark,colframe=black,arc=10pt]
	The global gauge invariance is rigorously referred to as "global symmetry".
	\end{tcolorbox}
	Let us consider the Schrödinger's equation:
	
	with as we have shown:
	
	with $\Psi=\Psi(\vec{r},t)$. Either in the case of a free particle:
	
	This operator is obviously invariant in the transformation which passes from $\Psi$ to $\Psi'$ with (change of the phase of a plane wave by an angle $\alpha$):
	
	where $g$ is a coupling constant (to ensure the homogeneity of the units and the amplitude) being considered as a real number and $\alpha$ a real parameter independent of the coordinates (in a first time ...) of space and time.
	
	Then:
	
	becomes:
	
	and as an $\alpha$ does not depend on $x$, $y$, $z$, $t$ then:
	
	Either after simplification:
	
	The form of the equation remained the same when we made the change from $\Psi$ to $\Psi'$.
	
	Thus, the description of a free system is not affected by the global phase change. In the language of group theory (\SeeChapter{see section Set Algebra page \pageref{unitary linear group}}), we speak of invariance under the $\text{U}(1)$ group of  phases.

	In other words to speak like the physicists...:
	
	defines a Gauge transformation by the rotation $\alpha$ (the parameter in the sense of the Lie groups) in the complex plane (but that corresponds to a phase change in the field of wave theory).
	
	The set of rotations forms a group denoted $\text{U}(1)$ which the usage make we name it the "\NewTerm{gauge group}\index{gauge group}" (isomorphic to $\text{SO}(2)$ as we have seen it in the section of Set Algebra).
	
	The set of all $e^{\mathrm{i}g\alpha}$ forms a monodimensional representation of the group $\text{U}(1)$ which we call the "representation $g$". There is, of course, an infinity of representations $g$ (as much as there are values of $g$!).
	
	Since the parameter $\alpha$ does not depend on position and time, we say that the system is "invariant by global gauge" transformation (everywhere at the same time) or simply an invariant of $\text{U}(1)$ in time and space.
	
	\begin{tcolorbox}[title=Remark,colframe=black,arc=10pt]
	In fact, global Gauge transformations are a subset of local gauge transformation: changing the same amount everywhere is a special case (ie, more restricting) of changing the phase of each point independently. A local Gauge transformation is therefore not a subset of a global Gauge transformation. In this respect the name is a bit misleading.
	\end{tcolorbox}
	
	\pagebreak
	\subsubsection{Local Gauge invariance}
	But but... either the global gauge invariance shows that we have an equation that remains valid as part of a fixed phase change. But now in a laboratory this Schrödinger equation must be valid even if the phase depends on position and time. This constraint is named a "\NewTerm{local gauge invariance}\index{local gauge invariance}".

	We consider this time that $\alpha$ is a function $\alpha(\vec{r},t)$ and the idea obviously is to check whether the Schrödinger equation remains invariant in the transformation:
	
	It is therefore obvious that Schrödinger's equation:
	
	is no longer invariant. Indeed, we quickly see that just the operator $\vec{\nabla}^2$ in the Hamiltonian will pose problem by making appear annoying terms that will not cancel out:
	
	To solve this problem, we introduce the force field associated with the vector potential and the electrical potential and we will see that it guarantees the local invariance (therefore it is impossible to envisage an invariant phase change without the presence of a force field of this type). The local invariance imposes that the particle is no longer free (there are therefore no free charged particles!).

	To do this, let us take again the Hamiltonian of the Pauli equation (\SeeChapter{see section Relativistic Quantum Physics page \pageref{pauli equation}}):
	
	and let us neglect the interaction between the spin and the magnetic field such that the Hamiltonian becomes:
	
	Therefore:
	
	We thus get the following Schrödinger equation:
	
	What in comparison to the free Schrödinger equation:
	
	involves the following correspondences (these are operators):
	
	Let us consider the following gauge transformation (\SeeChapter{see section Electrodynamics page \pageref{gauge theory}}) by denoting from now the electric potential by the letter $V$:
	
	where $f=f(\vec{r},t)$.
	
	First, we immediately see that the operators are invariant. Indeed, for the first one $\vec{D}$:
	
	But, if $g$ is put as being $q/\hbar$ and $f$ as $\alpha$ then we have:
	
	Thus simply:
	
	Similarly, knowing now that $f$ is $\alpha$ we have for the operator $D^0$:
	
	Therefore we have:
	
	Thus:
	
	The relation:
	
	Becomes then with the new correspondences:
	
	and with the previous developments, we have:
	
	Or written differently:
	
	Thus:
	
	Which gives after simplification:
	
	Thus, by requesting the local gauge invariance, we have make appear an interaction ... and we know well which interaction it is!

	The Schrödinger equation of a particle moving in an electromagnetic field is therefore invariant under the local phase transformation. The phase of a wave function is indeed a new local variable in the Weyl sense and the electromagnetic potential can be interpreted, according to Weyl, as a connection binding the phases to different points.

	We conclude that the electromagnetic field is a consequence of the local gauge invariance based on the group $\text{U}(1)$, a group of one-dimensional unitary matrices (\SeeChapter{see section Set Algebra page \pageref{unitary linear group}}). The interest is to build Gauge theories on more complicated (non-abelian) groups: these theories are named "\NewTerm{Yang-Mills theories}".
	
	So we can make an analogy: If a global gauge is like to turn all the hands of a watch of a constant angle around the circle the local gauge is similar as if we make this angle was vary continuously from one point to another of space-time. One might think that such an upheaval of the hands of a watch create tensions and compressions on the springs, sign of a modification of the electromagnetic field at each point!? But not at all: all the springs remain at rest, as if the electromagnetic field automatically adjusted to compensate for these local deformations:
	\begin{figure}[H]
		\centering
		\includegraphics[scale=1]{img/atomistic/local_global_gauge.jpg}
		\caption[Global and local gauge illustration concept]{Global and local gauge illustration concept (source: ?)}
	\end{figure}

	Now let us go a little further but without going too deep (as this book is only on \underline{Elementary} Applied Mathematics)... We showed above that the Lagrangian of the free Dirac equation was:
	
	Now, since this Lagrangian does not reveal the electromagnetic field, we strongly suspect that it does not carry in it a local Gauge invariance...
	
	Therefore the operator $\partial_\mu$ has somewhere a lack... But, the equivalent of the divergence operator $\vec{\nabla}$ in the free Schrödinger equation is here the covariant derivative $\partial_\mu$. So in the same way that we have associated the following operators to guarantee the local Gauge invariance of the free Schrödinger equation:
	
	It is tempting to combine the whole into a new operator:
	
	with:
	
	The Lagrangian of the free Dirac equation would then be written:
	
	or in Feynman notation notation:
	
	Therefore:
	
	with:
	
	It only remains to add the term of the field to obtain the total Lagrangian of the Dirac equation (it would have been relatively hard to find it in another way ...):
	
	which corresponds to the "\NewTerm{Dirac-Maxwell equations}\index{Dirac-Maxwell equations}" and which is the "\NewTerm{Quantum electrodynamic Lagrangian}\index{quantum electrodynamic Lagrangian}\label{quantum electrodynamics}" (in natural units) where on the left we have the term of the fermions and on the right the interaction part of the bosons of zero mass (photons).

	Thus the fact of having added to the free Lagrangian a condition of invariance by local transformations, has led us to a theory with interaction that we can write with more rigor and in developed form:
	
	Or in natural units and with the charge of the electron:
	
	However, quantum electrodynamics was lacking in the 1940s to describe a large number of particles that were detected by accelerators. Certainly, in a way, it has been extended to describe new particles. But many of them seemed to enjoy properties that quantum electrodynamics could not account for.

	In fact, the reason is simple ... it is a theory in which no exact solution is known so far, a situation that persists until the day these lines are written (2008). The only available calculation method is named "perturbative development". The idea is essentially the same as that of limited development practiced in the field of differential calculus. In this case, if we do not know how to calculate the value of a function, we decompose it into a sequence of polynomials and the approximation is refined as we take into account terms of higher order. Such a series development begins with a zero order term, which is just the value of the unknown function at a certain point where we know how to calculate this function.
	
	In the case of the perturbative development of quantum electrodynamics, the zero order term represents pure propagation without interaction (the intensity of the interaction between the electron and the magnetic field is set to zero). In this approximation, quantum electrodynamics is a theory of free particles and it is exactly calculable. We have electrons, positrons and photons but they cross without influencing each other. The next term in the series development, that of the first order, is also exactly calculable. In this approximation, the theory seems to reflect the real world fairly accurately. Very interesting physical phenomena appear in this first-order approximation of the real theory of photon-electron interaction and the theory agrees well with the experimental results.

	Unfortunately, it soon became apparent that the calculation of second-order terms and higher terms seemed meaningless as they yielded infinite values... today there still exist only approximate methods of resolution and that are not completely satisfactory. Since then physicists have been obliged to look for another technique of approximation based on a renormalization of the equations... and the results are extraordinarily good (to the $11$th decimal by!) but in the facts it looks like a bit like do-it-yourself physics...
	\begin{flushright}
	\begin{tabular}{l c}
	\circled{50} & \pbox{20cm}{\score{3}{5} \\ {\tiny 47 votes,  69.36\%}} 
	\end{tabular} 
	\end{flushright}

	%to make section start on odd page
	\newpage
	\thispagestyle{empty}
	\mbox{}
	\section{Elementary Particle Physics}\label{elementary particle physics}
	\lettrine[lines=4]{\color{BrickRed}W}e have already mentioned in the section of Nuclear Physics that we experiments  that the radioactive nuclei do not emit neutrons or protons. But we can ask ourselves: How do they synthesize an alpha particle, or transform a proton into a neutron or vice versa? To answer these questions, let us examine the forces in presence.
	
	Before the discovery of radioactivity, physicists had identified two fundamental forces: the gravitational force and the electromagnetic force. The discovery of radioactivity and studies on the atomic nucleus led physicists to introduce not one but two new fundamental forces!
	

	\begin{figure}[H]
		\centering
		\begin{tikzpicture} [scale=0.45,>=latex, inner sep=2pt, outer sep=2pt]
			\coordinate (Newton) at (168.7, 0);
			\coordinate (Einstein) at (191.5, 0);
			\coordinate (Coulomb) at (178, -5);
			\coordinate (Maxwell) at (186.4, -7.5);
			\coordinate (Biot) at (182, -10);
			\coordinate (Feynman) at (194.9, -7.5);
			\coordinate (Becquerel) at (189.6, -15);
			\coordinate (Fermi) at (193.4, -15);
			\coordinate (Unif1) at (196.1,-11.25);
			\coordinate (Yukawa) at (193.5,- 20);
			\coordinate (Unif2) at (197.3, -15.62);
			\coordinate (Unif3) at (201, -7.81);
			\draw[*-] (Newton)  node[above right]{\small 1687 - Newton} -- (Einstein);
			\draw[*-] (Einstein)  node[above right]{ \small 1915 - Einstein}--(201,0);
			\draw[dashed] (168.7,-5)--(Coulomb);
			\draw[*-] (Coulomb) node[below]{1780} node[above]{\small Coulomb} -|  (Maxwell);
			\draw[dashed] (168.7,-10) --(Biot);
			\draw[*-] (Biot) node[below]{1820} node[above]{\small  Biot and Savart} -|  (Maxwell);
			\draw[o-*] (Maxwell)  node[below right]{1864} node[above right]{\small Maxwell} --  (Feynman);
			\draw[] (Feynman) node[below]{1949} node[above]{\small Feynman \emph{et al}}-| (Unif1);
			\draw[dashed] (168.7,-15)--(Becquerel);
			\draw[*-] (Becquerel) node[below]{1896} node[above]{\small Becquerel} --(Fermi);
			\draw[*-] (Fermi) node[below]{1934} node[above]{\small Fermi} -|  (Unif1);
			\draw[o-] (Unif1) node[below right, fill=white]{1961}  -| (Unif2);
			\draw[dashed] (168.7,-20) --(Yukawa);
			\draw[*-] (Yukawa) node[below]{1935} node[above]{\small Yukawa}   -| (Unif2);
			\draw[o-] (Unif2) node[below right]{1973}  -| (201,0);
			\draw (Unif3) node[fill=white,text width=2cm,inner sep=5pt]{Physique Unifi\'ee ?} ;
			\draw (168.7,0) node[below right]{\small \textbf{Gravitation}};
			\draw (168.7,-5) node[below right]{\small \textbf{\'Electrostatique}};
			\draw (168.7,-10) node[below right]{\small \textbf{Magn\'etisme}};
			\draw (168.7,-15) node[below right]{ \small\textbf{Interaction Faible}};
			\draw (168.7,-20) node[below right]{\small \textbf{Interaction Forte}};
		\end{tikzpicture}
		\caption[Interactions discoveries]{Interactions discoveries (source:  http://femto-physique.fr author: Jimmy Roussel)}
	\end{figure}
	
	
	Even before knowing the exact composition of the nuclei, to explain the existence of these tiny systems and wearing sometimes strong positive charges, physicists had foreseen the need for a powerful cohesive force capable of dominating the electrostatic repulsion exerted between these charges (remember that we saw in the section of Classical Mechanics that the gravitational force between two bodies of equivalent masses to those of particles is completely negligible). Since the nucleus is small, this "nuclear force" had to have influence at very short distances. When J. Chadwick discovered the neutron, it was shown experimentally that attractive force is also well practiced between two neutrons, two protons and between a neutron and a proton. In 1935, Hideki Yukawa elaborated a theory about the nuclear force whose outlines seems to be still accepted, but must still be improved following some issues that have been identified (\SeeChapter{see section Quantum Field Theory}).
	
	However, as we already know, this nuclear force does not explain the transformation of a proton into a neutron, which takes place in the beta radioactivity. It was necessary to introduce a fourth fundamental force, of lower intensity, named for this reason "\NewTerm{weak interaction}\index{weak interaction}", the nuclear force becomes ipso facto the "\NewTerm{strong interaction}\index{strong interaction}".
	
	Thus, in principle, the radioactivity involves the four fundamental forces of nature: gravity and the electromagnetic force, as alpha and beta particles have mass and charge, and the two nuclear forces, strong and weak (in fact , the gravitation, of lesser intensity than the other three is often neglected  sub-atomic scales).
	
	Thus, in principle, the radioactivity involves the four fundamental forces of nature: gravity and the electromagnetic force, as alpha and beta particles have mass and charge, and the two nuclear forces, strong and weak (in fact , the gravitation, of lesser intensity than the other three is often neglected  sub-atomic scales).
	
	We partially addressed in the section of Quantum Field Theory the fundamental interactions and their interactions vectors. Let's do some reminders:
	\begin{enumerate}
		\item We have proved in the section of Special Relativity that there are two types of particles: those with a mass and will never go at the speed of light (because it will required an infinite energy to get them at this velocity) and those which have zero mass and which are therefore necessarily at the speed of light.
		\item We proved in the section of Quantum Field Theory that more a particle has energy, that more following the Heisenberg uncertainty, it may have long virtual life time and travel considerable distances. We distinguish therefore the interaction that have infinite range of interaction whose particles have no mass and a the interactions that have finite range whose interaction particles have a mass.
	\end{enumerate}
	Before starting with some arduous calculations, it is desirable first to acquire a vocabulary that is common use among theoretical physicists.
	
	The easiest concept to be addressed in the field of elementary particle physics is the comparison of the four elemental forces via their respective coupling constant (that is something that physicists like to do...).
	
	\begin{tcolorbox}[title=Remark,colframe=black,arc=10pt]
	Hubert Reeves and his colleagues astrophysicists proved that at the time of the genesis of the Universe, the smallest deviance in coupling constants would have caused the instability of nucleons and would have condemned cosmic evolution.
	\end{tcolorbox}
	
	\subsection{Coupling Constants}
	
	We will try here to rank the four forces according to their intensity through the use of "coupling constants".
	
	To do this, we need to calculate the four interactions for two same particles, for example two protons at identical distances, so of nuclear type and compare them to a common value of the units so that their ratio provides a dimensionless number.
	
	This common value will be chosen as the product:
	
	Thus we find:
	\begin{enumerate}
		\item For the gravitational force (\SeeChapter{see section Astronomy page \pageref{newton gravitational law}}) where:
		
		with the proton mass such that $m=M\cong 1.67\cdot 10^{-27}\; [\text{kg}]$, the coupling constant of the gravitational force is therefore by definition:
		
		\item For the electrical force (\SeeChapter{see section Electrostatic page \pageref{coulomb force}}) where:
		
		with the proton chart such that $q=Q\cong 1.602\cdot 10^{-19}\; [\text{C}]$, the coupling constant of the electric force is therefore by definition:
		
		\begin{tcolorbox}[title=Remark,colframe=black,arc=10pt]
		Here we find back the "\NewTerm{fine structure constant}\index{fine structure constant}" that we had already seen in the section of Corpuscular Quantum Physics. We therefore better understand the start choice for the relative comparison of interactions.
		\end{tcolorbox}
		\item For the strong nuclear force, where $F$ represents the "\NewTerm{strong nuclear charge}\index{strong nuclear charge}", the strong coupling constant has for value (caution! the value depends on the chosen theoretical model!):
		
		hence its name.
		\item For the weak nuclear force responsible for the decay of particles, $f$ represents the "\NewTerm{weak nuclear charge}\index{weak nuclear charge}", and it weak coupling constant value is then (caution! the value depends on the chosen theoretical model!):
		
	\end{enumerate}
	 So all this can be summarized in the following phenomenological table: 
	 %\setlength\extrarowheight{10pt}
	 \begin{table}[H]
		\begin{center}
		 \resizebox{\textwidth}{!}{\begin{tabular}{|m{5.5cm}|m{3.5cm}|m{3cm}|m{3cm}|}
		 \hline 
		 \centering\arraybackslash\ \cellcolor{black!30} \textbf{Four} & \centering\arraybackslash\ \cellcolor{black!30} \textbf{Phenomenological} & \centering\arraybackslash\ \cellcolor{black!30} \textbf{Macroscopic}  & \centering\arraybackslash\ \cellcolor{black!30}\textbf{Intensity} \\ 
		 \centering\arraybackslash\ \cellcolor{black!30} \textbf{Fundamentals Interactions} & \centering  \cellcolor{black!30} \textbf{Description} & \centering  \cellcolor{black!30} \textbf{Phenomenon}  & \cellcolor{black!30} \\ 
		 \hline 
		 \pbox{5cm}{Gravitational \\ \tiny (infinite range, massless particle interaction: graviton)} & \centering\arraybackslash\  \pbox{5cm}{$F=G\dfrac{m_1m_2}{R^2}$ \\ \tiny Newton/Einstein} &  \centering\arraybackslash\ \pbox{5cm}{\includegraphics{img/atomistic/interaction_gravitational.jpg} \\ \tiny Hydraulic Central} & \centering\arraybackslash\ $\alpha_G\cong 6\cdot 10^{-39}$ \\ 
		 \hline 
		 \pbox{5cm}{Electromagnetic \\ \tiny (infinite range, massless particle interaction: photon)} & \centering\arraybackslash\ \pbox{5cm}{$F=\dfrac{1}{4\pi\varepsilon}\dfrac{q_1q_2}{R^2}=qE$ \\ $\vec{F}=q\vec{E}+\vec{v}\times\vec{B}$ \\ \tiny Maxwell} & \centering\arraybackslash\ \pbox{5cm}{\includegraphics{img/atomistic/interaction_electromagnetic.jpg} \\ \tiny Generating Units} & \centering\arraybackslash\ $\alpha\cong \dfrac{1}{137}$ \\ 
		 \hline 
			\pbox{5cm}{Weak Force \\ \tiny (finite range, particle interaction with mass: bosons $W^{+}$, $W^{-}$, $Z^0$)} & \centering\arraybackslash\ \pbox{5cm}{$U(r)=\dfrac{C}{r}e^{-r/R}$ \\ $R=\dfrac{\hbar}{m_0c}$ \\ \tiny Yukawa} & \centering\arraybackslash\ \pbox{5cm}{\includegraphics{img/atomistic/interaction_weak.jpg} \\ \tiny Thermopile with radioactive isotope} & \centering\arraybackslash\ $\alpha_w=\dfrac{g_w^2}{\hbar c}\cong 10^{-6}$ \\ 
		 \hline 
		 \pbox{5cm}{Strong Force \\ \tiny (finite range, particle interaction with mass: gluons)} & \centering\arraybackslash\ \pbox{5cm}{$U(r)=\dfrac{C}{r}e^{-r/R}$ \\ $R=\dfrac{\hbar}{m_0c}$ \\ \tiny Yukawa} & \centering\arraybackslash\  \pbox{5cm}{\includegraphics{img/atomistic/interaction_strong.jpg} \\ \tiny Nuclear plant} & \centering\arraybackslash\ $\alpha_s=\dfrac{g_s^2}{\hbar c}\cong 1$ \\ 
		 \hline 
		 \end{tabular}}
		\caption[]{Radiometric and Photometric quantities}
		\end{center}
	\end{table}
	 %\setlength\extrarowheight{0pt}
	 or also with the following diagram (more interesting) where we can see, taking into account the results we have found during our study of the massic fields and non-massic fields in the context of the Yukawa model (\SeeChapter{see section Quantum Field Theory page \pageref{yukawa potential}}) : 
	\begin{enumerate}
		\item In ordinate the intensity of the forces as previously calculated according to the distance as given by the Yukawa model of massic fields (weak and strong interactions) and non-massic fields (electromagnetic and gravitational interactions).

		\item The representative interaction patterns (Feynman diagrams) in accordance with the results obtained and particles already mentioned in the section of Quantum Field Theory.
	\end{enumerate}
	\begin{figure}[H]
		\begin{center}
		\includegraphics[scale=0.8]{img/atomistic/fundamental_interactions.jpg}
		\end{center}	
		\caption{Feynman diagrams and interaction distance of the fundamental forces}
	\end{figure}
	 
	 \begin{figure}[H]
		\begin{center}
		\includegraphics{img/atomistic/summary_force_carriers.jpg}
		\end{center}	
		\caption{Force Carry subparticles}
	\end{figure}
	
	\begin{figure}[H]
		\centering
		\tikzset{%
	        brace/.style = { decorate, decoration={brace, amplitude=5pt} },
	       mbrace/.style = { decorate, decoration={brace, amplitude=5pt, mirror} },
	        label/.style = { black, midway, scale=0.5, align=center },
	     toplabel/.style = { label, above=.5em, anchor=south },
	    leftlabel/.style = { label,rotate=-90,left=.5em,anchor=north },   
	  bottomlabel/.style = { label, below=.5em, anchor=north },
	        force/.style = { rotate=-90,scale=0.4 },
	        round/.style = { rounded corners=2mm },
	       legend/.style = { right,scale=0.4 },
	        nosep/.style = { inner sep=0pt },
	   generation/.style = { anchor=base }
		}
		
		\newcommand\particle[7][white]{%
		  \begin{tikzpicture}[x=1cm, y=1cm]
		    \path[fill=#1] (0.1,0) -- (0.9,0)
		        arc (90:0:1mm) -- (1.0,-0.9) arc (0:-90:1mm) -- (0.1,-1.0)
		        arc (-90:-180:1mm) -- (0,-0.1) arc(180:90:1mm) -- cycle;
		    \ifstrempty{#7}{}{\path[fill=purple!50!white]
		        (0.6,0) --(0.7,0) -- (1.0,-0.3) -- (1.0,-0.4);}
		    \ifstrempty{#6}{}{\path[fill=green!50!black!50] (0.7,0) -- (0.9,0)
		        arc (90:0:1mm) -- (1.0,-0.3);}
		    \ifstrempty{#5}{}{\path[fill=orange!50!white] (1.0,-0.7) -- (1.0,-0.9)
		        arc (0:-90:1mm) -- (0.7,-1.0);}
		    \draw[\ifstrempty{#2}{dashed}{black}] (0.1,0) -- (0.9,0)
		        arc (90:0:1mm) -- (1.0,-0.9) arc (0:-90:1mm) -- (0.1,-1.0)
		        arc (-90:-180:1mm) -- (0,-0.1) arc(180:90:1mm) -- cycle;
		    \ifstrempty{#7}{}{\node at(0.825,-0.175) [rotate=-45,scale=0.2] {#7};}
		    \ifstrempty{#6}{}{\node at(0.9,-0.1)  [nosep,scale=0.17] {#6};}
		    \ifstrempty{#5}{}{\node at(0.9,-0.9)  [nosep,scale=0.2] {#5};}
		    \ifstrempty{#4}{}{\node at(0.1,-0.1)  [nosep,anchor=west,scale=0.25]{#4};}
		    \ifstrempty{#3}{}{\node at(0.1,-0.85) [nosep,anchor=west,scale=0.3] {#3};}
		    \ifstrempty{#2}{}{\node at(0.1,-0.5)  [nosep,anchor=west,scale=1.5] {#2};}
		  \end{tikzpicture}
		}
		\begin{tikzpicture}[x=1.2cm, y=1.2cm]
		  \draw[round] (-0.5,0.5) rectangle (4.4,-1.5);
		  \draw[round] (-0.6,0.6) rectangle (5.0,-2.5);
		  \draw[round] (-0.7,0.7) rectangle (5.6,-3.5);
		
		  \node at(0, 0)   {\particle[gray!20!white]
		                   {$u$}        {up}       {$2.3$ MeV}{1/2}{$2/3$}{R/G/B}};
		  \node at(0,-1)   {\particle[gray!20!white]
		                   {$d$}        {down}    {$4.8$ MeV}{1/2}{$-1/3$}{R/G/B}};
		  \node at(0,-2)   {\particle[gray!20!white]
		                   {$e$}        {electron}       {$511$ keV}{1/2}{$-1$}{}};
		  \node at(0,-3)   {\particle[gray!20!white]
		                   {$\nu_e$}    {$e$ neutrino}         {$<2$ eV}{1/2}{}{}};
		  \node at(1, 0)   {\particle
		                   {$c$}        {charm}   {$1.28$ GeV}{1/2}{$2/3$}{R/G/B}};
		  \node at(1,-1)   {\particle 
		                   {$s$}        {strange}  {$95$ MeV}{1/2}{$-1/3$}{R/G/B}};
		  \node at(1,-2)   {\particle
		                   {$\mu$}      {muon}         {$105.7$ MeV}{1/2}{$-1$}{}};
		  \node at(1,-3)   {\particle
		                   {$\nu_\mu$}  {$\mu$ neutrino}    {$<190$ keV}{1/2}{}{}};
		  \node at(2, 0)   {\particle
		                   {$t$}        {top}    {$173.2$ GeV}{1/2}{$2/3$}{R/G/B}};
		  \node at(2,-1)   {\particle
		                   {$b$}        {bottom}  {$4.7$ GeV}{1/2}{$-1/3$}{R/G/B}};
		  \node at(2,-2)   {\particle
		                   {$\tau$}     {tau}          {$1.777$ GeV}{1/2}{$-1$}{}};
		  \node at(2,-3)   {\particle
		                   {$\nu_\tau$} {$\tau$ neutrino}  {$<18.2$ MeV}{1/2}{}{}};
		  \node at(3,-3)   {\particle[orange!20!white]
		                   {$W^{\hspace{-.3ex}\scalebox{.5}{$\pm$}}$}
		                                {}              {$80.4$ GeV}{1}{$\pm1$}{}};
		  \node at(4,-3)   {\particle[orange!20!white]
		                   {$Z$}        {}                    {$91.2$ GeV}{1}{}{}};
		  \node at(3.5,-2) {\particle[green!50!black!20]
		                   {$\gamma$}   {photon}                        {}{1}{}{}};
		  \node at(3.5,-1) {\particle[purple!20!white]
		                   {$g$}        {gluon}                    {}{1}{}{color}};
		  \node at(5,0)    {\particle[gray!50!white]
		                   {$H$}        {Higgs}              {$125.1$ GeV}{0}{}{}};
		  \node at(6.1,-3) {\particle
		                   {}           {graviton}                       {}{}{}{}};
		
		  \node at(4.25,-0.5) [force]      {strong nuclear force (color)};
		  \node at(4.85,-1.5) [force]    {electromagnetic force (charge)};
		  \node at(5.45,-2.4) [force] {weak nuclear force (weak isospin)};
		  \node at(6.75,-2.5) [force]        {gravitational force (mass)};
		
		  \draw [<-] (2.5,0.3)   -- (2.7,0.3)          node [legend] {charge};
		  \draw [<-] (2.5,0.15)  -- (2.7,0.15)         node [legend] {colors};
		  \draw [<-] (2.05,0.25) -- (2.3,0) -- (2.7,0) node [legend]   {mass};
		  \draw [<-] (2.5,-0.3)  -- (2.7,-0.3)         node [legend]   {spin};
		
		  \draw [mbrace] (-0.8,0.5)  -- (-0.8,-1.5)
		                 node[leftlabel] {6 quarks\\(+6 anti-quarks)};
		  \draw [mbrace] (-0.8,-1.5) -- (-0.8,-3.5)
		                 node[leftlabel] {6 leptons\\(+6 anti-leptons)};
		  \draw [mbrace] (-0.5,-3.6) -- (2.5,-3.6)
		                 node[bottomlabel]
		                 {12 fermions\\(+12 anti-fermions)\\increasing mass $\to$};
		  \draw [mbrace] (2.5,-3.6) -- (5.5,-3.6)
		                 node[bottomlabel] {5 bosons\\(+1 opposite charge $W$)};
		
		  \draw [brace] (-0.5,.8) -- (0.5,.8) node[toplabel]         {standard matter};
		  \draw [brace] (0.5,.8)  -- (2.5,.8) node[toplabel]         {unstable matter};
		  \draw [brace] (2.5,.8)  -- (4.5,.8) node[toplabel]          {force carriers};
		  \draw [brace] (4.5,.8)  -- (5.5,.8) node[toplabel]       {Goldstone\\bosons};
		  \draw [brace] (5.5,.8)  -- (7,.8)   node[toplabel] {outside\\standard model};
		
		  \node at (0,1.2)   [generation] {1\tiny st};
		  \node at (1,1.2)   [generation] {2\tiny nd};
		  \node at (2,1.2)   [generation] {3\tiny rd};
		  \node at (2.8,1.2) [generation] {\tiny generation};
		\end{tikzpicture}
		\caption[Subparticles Standard Model]{Subparticles Standard Model (source: Wikipedia)}
	\end{figure}
	Above we can see the $12$ fundamental fermions and $4$ fundamental bosons. Brown loops indicate which bosons (red) couple to which fermions (purple and green). Please note that the masses of certain particles are subject to periodic reevaluation by the scientific community. The values currently reflected in this graphic are as of 2008 and may have been adjusted since. For the latest consensus, please visit the Particle Data Group website linked below.
	
	It should be precised for the general knowledge of the reader that these four forces are described by respectively four theories:
	\begin{enumerate}
		\item General Relativity (includes Classical Mechanics) for gravitation.

		\item The quantum electrodynamics (includes electrodynamics) for the electromagnetic force.

		\item The electroweak theory (which includes quantum electrodynamics) for the weak interaction

		\item The quantum chromodynamics for the strong interaction
	\end{enumerate}
	The last three are grouped in the "\NewTerm{Standard Model}\index{Standard Model}".
	
	\pagebreak
	\subsection{Spin magnetic resonance}
	We have hesitated to treat the subject of spin resonance in this section but after reflection, it's not nuclear physics because the calculations does not apply only well to the nucleus of atoms and this is not really pure relativistic quantum physics because it does not only apply well to elementary particles like electrons (as assumed in our developments in the section  of Relativistic Quantum Physics).

	During our proof of the Pauli equation in the section of Relativistic Quantum Physics using physicist art of life... we saw that for a particle of spin $\sfrac{1}{2}$ (which could be a nuclear nucleus of spin $\sfrac{1}{2}$), we had:
	
	where there is therefore a terme relatively to the spin in the Hamiltonian, that is:
	
	By putting:
	
	Being for reminder the "\NewTerm{Lande factor}\index{Lande factor}" or "\NewTerm{gyromagnetic factor}\index{gyromagnetic factor}" (\SeeChapter{see section Relativistic Quantum Physics page \pageref{gyromagnetic factor}}).
	
	Now let us dive the particle that will be named "probe" in a stable oscillating magnetic field following a horizontal plane and constant on the vertical plane of the type:
	
	where in practice the field $B_0$ goes form $0.1$ to $17$ Teslas.

	Therefore it comes:
	
	We then have by focusing only on this Hamiltonian:
	
	And as $\phi_{a0}$ is the double component of a spinor, let us write it explicitly:
	
	Thus after rearrangement:
	
	And let us recall that we proved at the very end of the section of Relativistic Quantum Physics that:
	
	It comes then naturally to write:
	
	This gives us:
	
	that is to say a system of two partial differential equation:
	
	If we assume that the spinor is orient to the top initially (initial conditions), then:
	
	Now to solve this system of differential equations we will use the laborious trial and error work already done by our predecessors by assuming that the answer is probably of the type:
	
	that is to say... wave functions!!!
	
	So by injecting these solutions in the system of differential equations we get:
	
	That is to say:
	
	and after simplification by the exponential and the pure imaginary number:
	
	Hence:
	
	We can get rid of the time dependence above by putting:
	
	Our system then reduces to:
	
	Thus by rearranging:
	
	which can be written in matrix form:
	
	To get a consistent solution (not all zero solutions) and unique one, the determinant of the matrix must be zero to be reversible (\SeeChapter{see section Linear Algebra page \pageref{determinant matrix inverse}}). But the determinant of this matrix is:
	
	But the roots of this polynomial of second degree in $\omega_{-}$ are (\SeeChapter{see section Calculus page \pageref{double root}}):
	
	So we deduce:
	
	So we have for summary:
	
	Results to inject in:
	
	So as each pulsation has two solution, the general solution will be the sum of the special solutions (\SeeChapter{see section Differential and Integral Calculus page \pageref{linear differential equations}}). Let us focus on the second relation. We then have:
	
	But let us recall the initial condition that we required for the second component of the spinor (due to the fact that the spinor is oriented to the top):
	
	Then we have:
	
	Therefore:
	
	with:
	
	and:
	
	Now, to determine $A$, we will use in a clever way the differential we started from earlier above:
	
	without forgetting the chosen initial conditions, we have:
	
	Therefore:
	
	We must have the equality between the both expressions:
	
	and this can be satisfied obviously if and only if:
	
	Then we have:
	
	Thus explicitly:
	
	As the second component of the spinor represents the angle and what we have above represents the angular position over time, the term:
	
	can be seen as the maximum value of the amplitude of the angle (Larmor rotation of the spin of the electron around the magnetic field). This maximum amplitude has itself a maximum if the denominator is as small as possible and therefore:
	
	Then we say that there is "\NewTerm{spin resonance}\index{spin resonance}".

	So the magnetic field may make switch the energy state of each spin if the oscillation amplitude is maximum. As we have just proved above for an isolated particle, the energy change from one state to the other is:
	
	and therefore when a spin flip it generates the emission of radiation named "\NewTerm{free induction decay}\index{free induction decay}". The collected signal depends on several parameters that characterize the particle or the nuclear core. In short, in practice it is $99\%$ engineering and $1\%$ theory!
	
	When we apply these theoretical results to an unpaired electron, that is to say, for the "\NewTerm{electronic magnetic resonance}\index{electronic magnetic resonance}", we have the following experimental data for a constant field $B_0\cong 0.3$ [T]:
	
	and therefore the electronic magnetic resonance technology is based on the detection of microwaves.
	When we apply these theoretical results to a nuclear core of a single proton, that is to say, for the "\NewTerm{nuclear magnetic resonance}\index{nuclear magnetic resonance}", we have the following experimental data for a constant field $B_0\cong 1$ [T]:
	
	and therefore the nuclear magnetic resonance technology is based on the detection of radio waves.
	
	Nowadays, nuclear magnetic resonance is also used to check the quality of food. For example, modern machines can detect whether orange juice is contaminated with juice from other fruit and can check whether the fruit were ripe when pressed. Other machines can check whether wine was made from the correct grapes and how it aged.
	\begin{figure}[H]
		\centering
		\includegraphics[scale=0.8]{img/atomistic/nuclear_magnetic_resonance.jpg}
		\caption[Machine to test fruit quality with
the help of nuclear magnetic resonance]{Machine to test fruit quality withthe help of nuclear magnetic resonance (source: Bruker)}
	\end{figure}
	
	\begin{flushright}
	\begin{tabular}{l c}
	\circled{10} & \pbox{20cm}{\score{3}{5} \\ {\tiny 26 votes,  56.92\%}} 
	\end{tabular} 
	\end{flushright}
	

\chapter{Cosmology}

	\textit{\textbf{Cosmology is the science that studies the structure, evolution and the general laws of the universe as a whole}}. (Larousse)
	\minitoc
	\pagebreak
		%to force start on odd page
	\newpage
	\thispagestyle{empty}
	\mbox{}
	\section{Astronomy (Celestial Mechanics)}
	\lettrine[lines=4]{\color{BrickRed}C}elestial mechanics is the consequence of the universal Newton's law of attraction  and of the fundamental principle of mechanics (\SeeChapter{see section Classical Mechanics}). Its main objective is the description of the motion of astronomical objects such as stars and planets using physical and mathematical theories.
	
	In this section we will approach the subject as always on this book, in the most elementary  possible way (to this day the topics in this section are not technically beyond the level of what was done in the beginning of 20th century in the field of astronomy).
	
	First we will make a warm up with a funny law on the living in the Universe ... (the Drake equation). Once completed this warm up, we will begin to "enumerate" Kepler's laws (often referring to the section of Classical Mechanics) and then study in detail the properties of Keplerian orbits thanks to our knowledge on classical mechanics and then to using Special Relativity, which will lead us to find a theoretical precessions of studied orbitals. Then we will have fun to model approximately the variation of the duration of the day (or night) on the Earth based on the month and latitude. Finally, to finish in style, we will launch the detailed calculation of the five Lagrangian points!
	
	\subsection{Drake Equation}
	
	This equation was invented (...) by F. Drake in the 1960s with the intention to discuss the number of extraterrestrial civilizations in our galaxy with which we might come in contact in the context of the SETI program (Search for ExtraTerrestrial Intelligence). The main purpose of this equation for scientists is to determine its factors, in order to know the likely number and (very) estimated extraterrestrial civilizations.
	
	This empirical equation which remains more something funny and provocative than something else... and  whose principle can be applied to a lot of different areas of physics and life is written:
	
	The terms of this formula (because it is a formula and not a relation!) are defined as follows:
	\begin{itemize}
		\item $N^{*}$ represents the number of stars in a single galaxy
		\item $f_p$ is the fraction of stars that would have an orbiting planet (between $0$ and $1$)
		\item $n_e$ is the number of planets per star that fulfill the conditions for the development of life
		\item $f_l$ is the fraction of planets whose life has emerged (between $0$ and $1$)
		\item $f_i$ is the fraction of those where an intelligent life has emerged (between $0$ and $1$)
		\item $f_c$ is the fraction of $f_i$ which has implemented radio communication technology (between $0$ and $1$)
		\item $f_l$ is the fraction of time during which the fraction $f_i$ civilizations will live (between $0$ and $1$)
	\end{itemize}
	In practice, it should be noted that this formula purpose is to try to determine an unknown amount from other amounts that are also unknown ... But it's a nice and funny formula to evaluate when you discuss with friends at the restaurant...
	
	There is therefore no guarantee that we are more knowledgeable after the estimate of this formula than before (method sometimes named in the literature "garbage in, garbage out"...).
	
	The resulting value can motivate the fact that following mathematical developments are not only applicable to only one solar (star) system in the universe... maybe... (it would make a lot of useless empty space otherwise...).
	
	Let us talk now about the "\NewTerm{Fermi paradox}\index{Fermi paradox}"  named after physicist Enrico Fermi, is the apparent contradiction between the lack of evidence and high probability estimates, e.g., those given by the Drake equation, for the existence of extraterrestrial civilizations. The basic points of the argument, made by physicists Enrico Fermi (1901–1954) and Michael H. Hart (born 1932), are:
	\begin{itemize}
		\item There are billions of stars in the galaxy that are similar to the Sun, many of which are billions of years older than Earth.
		\item With high probability, some of these stars will have Earth-like planets, and if the Earth is typical, some might develop intelligent life.
		\item Some of these civilizations might develop interstellar travel, a step the Earth is investigating now.
		\item Even at the slow pace of currently envisioned interstellar travel, the Milky Way galaxy could be completely traversed in a few million years.
	\end{itemize}
	According to this line of reasoning, the Earth should have already been visited by extraterrestrial aliens. In an informal conversation, Fermi noted no convincing evidence of this, leading him to ask, "Where is everybody?" There have been many attempts to explain the Fermi paradox, primarily either suggesting that:
	\begin{itemize}
		\item Extraterrestrial life is rare or non-existent
		\item No other intelligent species have arisen
		\item Civilizations lack advanced technology
		\item It is the nature of intelligent life to destroy itself
		\item It is the nature of intelligent life to destroy others
		\item There is periodic extinction by natural events
		\item Intelligent civilizations are too far apart in space or time
		\item It is too expensive to spread physically throughout the galaxy
		\item Human beings have not existed long enough
		\item Humans are not listening properly
		\item Civilizations broadcast detectable radio signals only for a brief period of time
		\item Civilizations tend to isolate themselves
		\item Everyone is listening, no one is transmitting	
		\item Earth is deliberately not contacted
		\item It is dangerous to communicate
		\item ...
	\end{itemize}
	
	\subsection{Kepler's Laws}
	
	In astronomy, Kepler's laws describe the main properties of the motion of planets around a main star, without explaining the reason (at least at the time these laws were developed!). They were discovered by Johannes Kepler based on the observations and measurements (in phenomenal amount) of the position of the planets made by Tycho Brahe, measures that were very accurate for its time.
	
	The first two Kepler's law seems to were published in 1609 and the third in 1618. The elliptical orbits, as set out in its first two laws can explain the complexity of the apparent motion of the planets.
	
	Soon after, in 1687 Isaac Newton discovered the law of gravitational attraction, deducting from it, by calculation, the three Kepler's laws.
	
	We will now try to present these laws in the most relevant possible way:
	
	\subsubsection{First Kepler's Law (conicity law)}
	
	The "\NewTerm{first law of Kepler}\index{first law of Kepler}", sometimes also named "\NewTerm{conicity law}\index{conicity law}" or "\NewTerm{law of orbits}\index{law of orbits}" is stated most of time as follow: The orbits of the planets are conics (ellipses) which the Sun (central star) occupies one of the focals.
	
	In fact, it should be noted that this is not really a "law" in the proper sense, since further below you will see that we can prove that:
	
	
	\begin{tcolorbox}[title=Remark,colframe=black,arc=10pt]
	The reader who has already read the section of Analytic Geometry will not be surprised by this relation...
	\end{tcolorbox}
	
	\subsubsection{Second Kepler's Law (area law)}
	The "\NewTerm{Kepler's second law}\index{Kepler's second law}", sometimes also named "\NewTerm{area law}\index{are law}" tells us that the line joining a planet to the Sun (central start) sweeps out equal areas in equal times (constant areal velocity) as:
	
	
	It is a relation that arises from the conservation of angular momentum as we have already shown it in the section of Classical Mechanics where we got:
	
	
	So again, the status of "law" is questionable in the language of modern physics!
	
	Now let us express this law in another form more conventional in the field of astronomy. Consider for this the movement in the plan in cylindrical coordinates by:
	
	Therefore:
	
	
	It comes therefore from the property of linearity of the vector product:
	
	Therefore taking the norm:
	
	And since it is equal to a constant, it is often customary to write this last equality in a condensed form (and putting the mass in the constant):
	
	Also, remember that we also got the result that the movement is and remains in a plane without any outside action!
	
	We note also that this law gives us the speed of the planet is variable. It is larger than at perihelion than a the aphelion:
	\begin{figure}[H]
		\begin{center}
		\includegraphics{img/cosmology/focus_aphelion_perihelion.jpg}
		\end{center}	
		\caption{Representation of surfaces conservation}
	\end{figure}
	This is true for the Earth for example. Indeed, this latter is closer to the sun in winter (for northern hemisphere) and then has a trajectory speed slightly higher than in summer; the travel time is therefore lower (winter has fewer days than the other seasons).
	
	\paragraph{Time of flight}\mbox{}\\\\
	We propose now to apply the second Kepler's law to determine the time $t$ from the passage to the perihelion as a function of the "\NewTerm{eccentric anomaly}\index{eccentric anomaly}" $\varphi$ in the case of an elliptic orbit  (thus special case!) using its two foci (one of them being assimilable for example to the position of the Sun) and the origin of the "\NewTerm{auxiliary circle}\index{auxiliary circle}" (also named "\NewTerm{apsidal circle}\index{apsidal circle}"):
	\begin{figure}[H]
		\begin{center}
		\includegraphics{img/cosmology/excentricity_anomaly.jpg}
		\end{center}	
		\caption{Schema for the study of eccentric anomaly angle}
	\end{figure}
	To determine the time $t$ between the passage at the perihelion $A$ and the point $P$ of a body following the trajectory of the ellipse of surface $\pi a b$ (see the Geometric Forms section for the proof of the calculation of the surface of the ellipse) in function of the eccentric anomaly $\varphi$, we will use the areas law just proved earlier above (second second Kepler's law) that give us the right to write:
	
	But, the surface of the ellipse is an affine transformation of the surface of the auxiliary circle such that:
	
 	We have then:
	
 	If $\varphi$ is, as it should, expressed in gradients, we have of course:
	
 	Therefore:
	
 	For $S_{F\text{O}Q}$, we have $\overline{F\text{O}}=a$ therefore equal to the radius of the auxiliary circle. The surface of the triangle $S_{F\text{O}Q}$, knowing its height given by $h=a\sin(\varphi)$ is then obtained by:
	
	Therefore we have:
	
	But, by definition of the eccentricity (\SeeChapter{see section Analytical Geometry}), we can write:
	
	Finally, we have:
	
	Therefore:
	
 	where the angle taken at the center of the ellipse is for recall named the "eccentric anomaly".

	\textbf{Definition (\#\mydef):} In the description of the Keplerian orbit of a celestial object, the "\NewTerm{eccentric anomaly}\index{eccentric anomaly}" is the angle between the direction of the periapse and the current position of an object in its orbit, projected on the circle extinct perpendicular to the major axis of the ellipse

	What would interest us now would be to find a relation of passage between this eccentric anomaly and the angle named "\NewTerm{true anomaly}\index{true anomaly}" $\theta$ as sometimes it is often more advantageous to use this last angle.

	For a relation between $\varphi$ and $\theta$, we will reuse our above schema but modified a bit:
	\begin{figure}[H]
		\begin{center}
		\includegraphics{img/cosmology/true_cosmology.jpg}
		\end{center}	
		\caption{Schema for the study of true-eccentric anomaly angle relation}
	\end{figure}
	We have obviously the $4$ below relations which a deduce from the above figure:
	
	We then have already in a first time (relation which will be useful to us a little later):
	
 	We have also proved in the section of Analytical Geometry that:
	
	this relation being valid at any border point of the ellipse. Thus, we also have:
	
 	Which leads us to write:
	
	Ideally, we could get rid of the radius in the denominator. For this, we will use the fact that (relations that we have just proved earlier above):
	
	We have:
	
 	The terms to the left of the equality are simplified immediately:
	
	hence:
	
 	therefore:
	
	thus finally:
	
	Finally notice that in the special case of an elliptical orbit, we deduce thanks to the second Kepler's law (see the proof of the calculation of an area of an ellipse in the section Geometric Shapes) the:
	
	\pagebreak
	\subsubsection{Third Kepler's Law (periods' law)} 
	The "\NewTerm{Kepler's third law}\index{Kepler's third law}", sometimes also named  "\NewTerm{Periods' law}\index{Periods' law}" or "\NewTerm{Kepler's harmonic law}\index{Kepler's harmonic law}", is stated as follow: The squares of the periods of revolution $T$ are proportional to the cube of the semi-major axes of the orbits $D$:
	
	The last ratio is then a constant in practice for all planets (in physics we also speak of "invariant") and the reason for this "harmony" was difficult to explain before Newton's theory.
	
	Again, we will see later that the status of "law" is no longer justified in our time as it is possible to prove that this relation, whose expression will be detailed, is in reality:
	
	and therefore we understand better when we see the term on the right why we had the previous constant ($m$ is the central body mass!).
	
	The last relation is more often written as following in the field of astronomy:
	
	where as we have seen in the section of Analytical Geometry, $a$ is the traditional notation for the apogee radius (semi-major axes).
	
	The most commonly used rearrangement of this la relation is obviously:
	
		
	Of course, Kepler did not immediately published his three laws in this provocative simplicity. Their current presentation order is also not the original one ... They are  in reality to find among  a profusion of physical speculations and reflections on world's harmony.
	
	\begin{tcolorbox}[title=Remark,colframe=black,arc=10pt]
	The Kepler's law are not limite to the gravitation force. They also apply for all acceleration (or force) of the type $1/r^2$. And this is also the case of the Coulomb's law (\SeeChapter{see section Electrostatic}). Kepler's law can therefore also be applied to an electron in orbit around a nucleus. The Borh-Sommerfeld model (\SeeChapter{see section Quantum Corpuscular Physics}) based also on Kepler's law gives also elliptic trajectories for electrons!
	\end{tcolorbox}	
	
	The three Kepler's law can be resume by the following small figure:
	\begin{figure}[H]
		\begin{center}
		\includegraphics{img/cosmology/keplerslaw.jpg}
		\end{center}	
		\caption{Summary of Kepler's law in image (source: ???)}
	\end{figure}
	
	The reader must take precautions with the image above because:
	\begin{enumerate}
		\item The planets are most of time in a movement that is not in the same plane. For the Solar system it is the tradition  at high school level to represent the planets in the "\NewTerm{ecliptic}\index{ecliptic}" plane that is the average plane described by the movement of Jupiter around the Sun:
		\begin{figure}[H]
			\begin{center}
			\includegraphics{img/cosmology/ecliptic_solar_system.jpg}
			\end{center}	
			\caption{Planets spin and angle relatively to the ecliptic plane}
		\end{figure}
		
		\item The orbits precess around the Sun as shown in the following figure (see further below for the mathematical proof in the cas of a 2-body system):
		\begin{figure}[H]
			\begin{center}
			\includegraphics{img/cosmology/orbit_precession.jpg}
			\end{center}	
			\caption{Orbit precession example}
		\end{figure}
		
		\item The planets have an helicity trajectory "behind" the movement of the Sun around the center of our Galaxy and the ecliptic has an angle of approximately $60^\circ$ ($\pi/3$ [rad]) relatively to the perpendicular of the Sun movement as visible in the figure below:
		\begin{figure}[H]
			\begin{center}
			\includegraphics{img/cosmology/solar_system_vortex.jpg}
			\end{center}	
			\caption{Planets with orbits following the Sun in its movement}
		\end{figure}
		and the planets are therefore sometimes in front of the Sun and sometimes behind.
		
		\item On the very very long term the orbits in a $n$-body system is a chaotic deterministic system that has period of quasi-stability but that sometimes diverges completely. This is great opportunity at the date we write these lines to be in such a period of quasi-stability.
	\end{enumerate}	
		
	\pagebreak
	\subsection{Newton Gravitational Law}
	To check the accuracy of his hypothesis, Newton (relatively long after Kepler) found Kepler's laws from the law of gravity, giving the explanation of the general movement of the planets.
	
	Newton considered to determine the law of gravitation a theoretical planet orbiting around the Sun in a circular orbit at a constant speed $v$. During a complete orbit the planet travels a distance equal to the circumference of the circle of radius $R$, or $2\pi R$, in a time (the period) equal to the distance divided by its velocity, either:
	
	Newton then relies on the third Kepler's law with always the assumption of a circular orbit.
	
	We therefore have:
	
	but as:
	
	Then we get by substitution:
	
	By comparing:
	
	and:
	
	and now assuming that $4\pi^2$ is divided by the constant is a new constant (which will be denoted in the same manner as the first although it it is not equal to...) we obtain:
	
	Therefore:
	
	Then, if we reverse the terms, this expression becomes (while noting that the inverse of the original is constant is, also, a constant):
	
	By another calculation, we have already established in the section of Classical Mechanics the expression of centrifugal force:
	
	by comparing this expression with the previous one:
	
	we get:
	
	There should therefore exist a force opposed to the centrifugal force that keeps the orbital cohesion and which can be written:
	
	remains to determine the value of the constant!
	
	It is trivial that the central mass $M$ of the orbital system has to intervene in one way or another in this constant. If the mass of the secondary body intervenes proportionally in the centrifugal force, the desire is great to do the same with the mass of the central body. So:
	
	Now there would be a priori more parameters to take into account. The remaining constant is here to meet the dimensional analysis so that we have "Newtons" (name given to the unit of force) on both sides of the equality. Scientists have determined with precision this "\NewTerm{gravitational constant}\index{gravitational constant}" denoted by $G$ that a priori seems universal and which in SI units, the 2014 CODATA-recommended value of the (with standard uncertainty in parentheses) is:
	
	Which brings us to write the "\NewTerm{Newton Gravitational law}\index{Newton Gravitational law}":
	
	Obviously it is not a true rigorous proof because based on experimental Kepler's observations. By cons, from General Relativity it is possible to prove it (under some given assumptions...)!
	\begin{tcolorbox}[colframe=black,colback=white,sharp corners]
	\textbf{{\Large \ding{45}}Example:}\\\\
	At the Earth's equator the radius is of $6378$ [km] and at the poles of $6357$ [km]. Therefore we have:
	
	Then the acceleration at the equator is equal to $9.800/9.865\cong 99.34 \%$ to that at the poles.
	\end{tcolorbox}
	It is very important to notice that the mutual forces of attraction acting on two mass spheres are always of equal size!
	
	Using Maple 17.00, we can simulate the plane trajectory of a satellite relative to $n$ number of fixed mass (thanks to Forhad Ahmed for his script). Here below is given the basic script that you can customize to your tastes and... feel free to give us your personal work if you have brought significant improvement to this script:
	
	\texttt{>restart; with(plots); with(DEtools)\\
	>G:=1; \#normalized gravitational constant to simplify\\
	>poles:=2; \#number of bodies/masses that we can play with...\\
	>M[1]:=10;M[2]:=1; \#mass of the first and second body (in relative values)\\
	>h[1]:=1;h[2]:=-1; \#X position of the first and second body X (in astronomical units)\\
	>k[1] := 1;k[2] := 1; \#Y position of the first and second bodies (in astronomical units)\\}
	
	\texttt{>\#differential equation of the satellite acceleration in X\\
	>Xeq := diff(x(t), t, t) = sum(-G*M[j]*(x(t)-h[j])/((x(t)-h[j])\string^2\\+(y(t)-k[j])\string^2)\string^(3/2), j = 1 .. poles);\\
	>\#differential equation of the satellite acceleration in X\\
	>Yeq := diff(y(t), t, t) = sum(-G*M[j]*(y(t)-k[j])/((x(t)-h[j])\string^2\\+(y(t)-k[j])\string^2)\string^(3/2), j = 1 .. poles);\\
	>\#position and initial velocity of the satellite\\
	>inits := x(0) = -2, y(0) = 0, (D(x))(0) = 0, (D(y))(0) = 2\\
	>\#numerical solution of the differential equation (you can play with the precision of the error as needed!)\\
	>g:=dsolve({Xeq,Yeq,inits},{x(t),y(t)},type=numeric,method=dverk78,abserr=0.1e-3, output= procedurelist);\\}
	
	\texttt{>n:=50; \#step of iterations\\
	>iter:=300; \#step of iterations\\}
	
	\texttt{>\#loop that resolves the differential equation at each new iteration\\
	>for i from 0 to iter do \\
	px[i]:=rhs(g(i/n)[2]);\\
	py[i]:=rhs(g(i/n)[4]);\\
	KE[i]:=1/2*(rhs(g(i/n)[3])\string^2+rhs(g(i/n)[5])\string^2);\\
	temp:=(rhs(g(i/n)[2])-h[j])\string^2+(rhs(g(i/n)[4])-k[j])\string^2;\\
	PE[i]:=sum(-G*M[j]/sqrt(temp), j = 1 .. poles);\\
	TE[i]:=KE[i]+PE[i]\\
	end do:\\}
	
	\texttt{>data:=seq(pointplot([px[i], py[i]], color = red), i = 0 .. iter):\\
	>\#mettre insequence à true pour avoir une animation\\
	>Anim:=display(data,insequence=false,scaling=constrained,axes=boxed):\\
	>stars:=display(seq(pointplot([h[i], k[i]], color = black), i = 1 .. poles))\\
	>display({Anim,stars},title=`Satellite orbiting a multipolar gravity field`);\\}

	\begin{figure}[H]
		\begin{center}
		\includegraphics{img/cosmology/trajectory_of_a_body_influenced_by_massive_body.jpg}
		\end{center}	
		\caption{Configuration for the study of relativistic effects}
	\end{figure}

	\texttt{>\#it is verified that the total energy of the satellite is always constant\\
	>print(`[Time] -- [Kinetic Energy] - [Potential Energy] - [Net Energy]`);\\
	>print(`======================================`);\\
	> for i by 3 to iter do\\
	print(evalf(i/n, 6), ` `, KE[i], ` `, PE[i], ` `, TE[i]);\\
	end do:\\
	>\#the last column of the table must always have normally an equal value...}
	
	\begin{tcolorbox}[title=Remark,colframe=black,arc=10pt]
	Equalizing the centrifugal force and gravitational force, it is quite easy to get an approximation of the speed of rotation of the planets in their orbits. The reader that will do the calculation will see that the value for the planets of our solar system is around a speed of about $100,000$ [km/h].
	\end{tcolorbox}	
	
	From this last relation, let us come back briefly on our third Kepler's law and detail it a little bit to show that it is valid for any type of conical orbit and to determine the expression of its constant.
	
	Expressed in the Frenet coordinate system (\SeeChapter{see section Differential Geometry}), and decomposed into its normal (centripetal) and tangential acceleration, the acceleration in respect to a geocentric reference frame (in the case of a referential located at the mass center of the system the expression change a little bit!) is written:
	
	From in previous developments (3rd Kepler's Law):
	
	and:
	
	the constant of Kepler's third law takes for value (it is a formulation sometimes used in practice but not a strictly necessary step in this development):
	
	but as we also have:
	
	then:
	
	Therefore:
	
	Finally, the third Kepler law can be found frequently in the literature as follows:
	
	But now let us consider again our figure of the center of mass study (\SeeChapter{see section Classical Mechanics}):
	\begin{figure}[H]
		\centering
		\includegraphics{img/atomistic/hydrogenoid_center_of_mass.jpg}
		\caption[]{Binary System Center of Mass (profile view)}
	\end{figure}
	And let us have a look at circular orbits in the center of mass frame! 
	
	First we look at the forces acting on body $M$:
	
	The forces balance, so:
	
	We know that in the simple case of a circular orbit (circular kinematics):
	
	We insert this into the previous equation and we get after some algebra:
	
	What is $r_M$? From the definition of center of mass we know that (\SeeChapter{see section Classical Mechanics}):
	
	We plug $r_M$ into our equation and now we get:
	
	or after rearranging terms:
	
	So we can use this 3rd Kepler's law to determine the total mass a binary pair if we know the period $T$!!! As the star masses are well estimated using the HR diagram we better understand how astronomers estimate orbiting planet mass knowing the period (in fact they also use the luminosity variation). In fact it is not as simple as there is no reason why we should be looking directly onto the orbital plane of the binary system. In other words, the apparent orbit is almost never the true orbit (which is what we need to do the calculation).
	
	This interlude performed, let us come back on our Newton's gravitation law:
	
	From the law of gravitation, we can find back Kepler's law. Besides, we have already done it for the second and third law of Kepler, since it is these that we used to get this latter relation (however it's a little bit the snake eating its tail...).
	
	In vector notation we have therefore:
	
	Identically to the electric field (\SeeChapter{see section Electrostatic}), we can develop:
	
	As the electric field is derived from an electric potential, identically, the gravitational field derived from a gravitational potential. By performing exactly the same development as in  our study of electromagnetism for the first Maxwell equation (\SeeChapter{see section Electrodynamics}), we prove that:
	
	where $\varphi$ is the "\NewTerm{gravitational potential}\index{gravitational potential}" that varies inversely with the relative distance of the body (this confirms what we had proved in our study of Noether's theorem in the section on Principles) and is therefore equal to:
	
	\begin{tcolorbox}[title=Remark,colframe=black,arc=10pt]
	We often encounter this potential in the section of General Relativity. It is therefore appropriate to remember it if possible!
	\end{tcolorbox}	
	Notation which obviously implies the following relation:
	
	\begin{tcolorbox}[title=Remark,colframe=black,arc=10pt]
	Obviously in the absence of field, we have $\varphi=c^{te}$ and therefore $\vec{a}$  will be zero.
	\end{tcolorbox}
	As in the section of Electromagnetism, again, we prove as we did for the first Maxwell equation:
	
	If we express this equation in terms of a gravitational potential $\varphi$ (also often denoted by the letter $U$ as in Electrostatic...), we get:
	
	that we write more aesthetically with the scalar Laplacian operator (\SeeChapter{see section Vector Calculus}):
	
	which is nothing else than the "\NewTerm{Newton-Poisson equation}\index{Newton-Poisson equation}" that we will  also meet again in our study of General Relativity (it has an important place for validation reasons of Einstein's Gravitation theory)!
	
	This equation means that the Newtonian gravitational theory can be resume to say that the gravitational field is described by a single potential $\varphi$ generated by the volume mass density and determining the acceleration of a test particle immersed in the outfield $\varphi$.
	
	Let's have a little bit fun now with the Newton's gravitation equation to get some interesting and curious results:
	
	\subsubsection{Gaussian Formulation of Newtonian Gravity}
	As we have just mentioned it and proved in the section of Electrodynamics, an alternative formulation of Newtonian gravity is: Gauss’s Law for gravity. It states that the acceleration $\vec{a}$ due to gravity of a mass $m$ (not necessarily a point mass) is given by:
	
	where the $-$ sign we have it's purpose of guarantee a positive scalar acceleration.
	
	For example, let's use Gauss's law to find the acceleration $a$ due to the gravity of a point mass $m$.

	We begin with a point mass m sitting in space. We now need to construct an imaginary closed surface $S$ surrounding $m$. While in theory any surface would do, we should pick a surface that will make the integral easy to evaluate. Such a surface should have these properties:
	\begin{enumerate}
		\item[P1.] The gravitational acceleration $vec{a}$ should be either perpendicular or parallel to $S$ everywhere.

		\item[P2.] The gravitational acceleration $\vec{a}$ should have the same value everywhere on $S$. (Or it may be zero on some parts of $S$).

		\item[P3.] The surface $S$ should pass through the point at which you wish to calculate the acceleration due to gravity.
	\end{enumerate}
	If we can find a surface $S$ that has these properties, the integral will be very simple to evaluate. For the point mass, we will choose $S$ to be a sphere of radius $r$ centered on mass $m$. Since we know $\vec{a}$ points radially
inward toward mass $m$, it is clear that $g$ will be perpendicular to $S$ everywhere. Also, by symmetry, it is not hard to see that $\vec{a}$ will have the same value everywhere on $S$. 

	Having chosen a surface $S$, let us now apply Gauss’s law for gravity. The law states that for recall that:
	
	Now everywhere on the sphere $S$, we have $\vec{a}\circ\vec{n}=-g$ (since $\vec{a}$ and $n$ are anti-parallel $g$ points inward, and $n$ points
outward). Since $g$ is a constant for a perfect sphere, the previous relation becomes:
	
	Now the integral is very simple: it is just $\mathrm{d}S$ integrated over the surface of a sphere, so it’s just the area of a sphere (\SeeChapter{see section Geometric Shapes}):
	
	or (canceling $-4\pi$ on both sides):
	
	and it's in agreement with the Gravitational Newton law!
	
	So the Flat-Earthers and some believers (following some holy books that we will not mention her) have to explain why everywhere in the world they can measure a falling object which acceleration corresponding to a spheric Earth if that latter is recall flat...

	As Flat-Earther sometimes challenge physicists to prove that the Newton law is not the same for a flat Earth (I was also challenged once... and this was a very bad idea from my opponent) here is the proof!
	\begin{figure}[H]
		\centering
		\includegraphics[scale=0.7]{img/cosmology/flat_earth.jpg}
		\caption[]{Schematic idea of a flat planet like... Earth......}
	\end{figure}
	In this case, the appropriate Gaussian surface $S$ is a "pillbox" shape - a short cylinder whose flat faces - (of area $A$) are parallel to the plane of mass. In this case, everywhere along the curved surface of $S$, the gravitational acceleration $\vec{a}$ is perpendicular to the outward normal unit vector $\vec{n}$, so the curved sides of $S$ contribute nothing to the integral. Only the flat ends of the pillbox-shaped surface S contribute to the integral. On each end, $\vec{a}$ is anti-parallel to $\vec{n}$, so $\vec{a}\circ\vec{n}=-g$ on the ends.
	
	Now apply Gauss's law to this situation:
	
	Here the integral needs only to be evaluated over the two flat ends of $S$. Since $\vec{a}\circ\vec{n}=-g$, we can bring $-g$ outside the integral to get:
	
	The integral in this case is just the area of the two ends of the cylinder, $2A$ (one circle of area $A$ from each end). This gives:
	
	Now let’s look at the right-hand side of this equation. The mass $m$ is the total amount of mass enclosed by surface $S$. Surface $S$ is sort of a "cookie cutter" that punches a circle of area $A$ out of the plane. The mass
enclosed by $S$ is a circle of area $A$ and surfacic density $\sigma$, so it has mass $\sigma A$. Then the previous relation becomes:
	
	Note that this is a constant: the acceleration due to gravity of an infinite plane of mass is independent of the distance from the plane...!

	So Flat-Earth have difficulties to only difficulties to explain this but also are not able to found the corresponding value of $g$ in their laboratory or home garage...

	\subsubsection{Shell Theorem}
	The shell theorem gives gravitational simplifications that can be applied to objects inside or outside a spherically symmetrical body. This theorem has particular application to astronomy.

	Isaac Newton proved the shell theorem and stated that:
	\begin{enumerate}
		\item A spherically symmetric body affects external objects gravitationally as though all of its mass were concentrated at a point at its center.
		\item If the body is a spherically symmetric shell (i.e., a hollow ball), no net gravitational force is exerted by the shell on any object inside, regardless of the object's location within the shell.
	\end{enumerate}
	A corollary is that, and we will prove it, that inside a solid sphere of constant density, the gravitational force varies linearly with distance from the center, becoming zero by symmetry at the center of mass.
	
	Given an object located outside of the Earth and $r$ is the distance of the object to the center of the Earth, we have:
	
	it comes:
	
	If the object is placed at the surface of the Earth or radius $R$, we have ($r=R$):
	
	From the two previous relations it comes therefore:
	
	At the surface we have then well (we expected this result...):
	
	Now, if the object is located inside the Earth by denoting the distance from the center by the letter $r$ and the central mass by $M'$, we have:
	
	Let us introduce the density $\rho$ that we will assume equal everywhere:
	
	By combining these last four relations, we get:
	
	\begin{figure}[H]
		\begin{center}
		\includegraphics{img/cosmology/gravity_profile.jpg}
		\end{center}	
		\caption{Internal/External gravitational acceleration profile of a mass body}
	\end{figure}

	For many people this result is quite counterintuitive (do a little survey around you, you'll see).
	
	\begin{tcolorbox}[title=Remark,colframe=black,arc=10pt]
	In addition to gravity, the shell theorem can also be used to describe the electric field generated by a static spherically symmetric charge density, or similarly for any other phenomenon that follows an inverse square law. The derivations below focus on gravity, but the results can easily be generalized to the electrostatic force. 
	\end{tcolorbox}
	
	\subsubsection{Orbital speed}
	We will prove now an obvious property of orbits that will be useful to us later to study of what seem to be an anomaly with structure of the size of galaxies.
	
	The orbital speed of a body, generally a planet, a natural satellite, an artificial satellite, or a multiple star, is the speed at which it orbits around the barycenter of a system, usually around a more massive body. It can be used to refer to either the mean orbital speed, i.e. the average speed as it completes an orbit, or the speed at a particular point in its orbit such as perihelia.
	
	We have proved above the origin of Newton's law. For planets thus considered as physical points in stable circular orbit, so there is balance between centrifugal and gravitational force. So we have:
	
	where we easily deduce:
	
	Which is approximately in good agreement with the experimental measurements as shown in the figure below:
	\begin{figure}[H]
		\begin{center}
		\includegraphics[scale=0.55]{img/cosmology/orbital_speed.jpg}
		\end{center}	
		\caption{Orbital speed characteristic curve}
	\end{figure}
	But as we will proved it in the section of Aerospace Engineering (Vis-Viva theorem) in a more general case the orbital speed is given by:
	
	
	\subsubsection{Asteroids/Meteors impact velocity}
	We have proved in the section of Classical Mechanics that the escape velocity was given by:
	
	and was therefore independent of the mass $m$ of the ejected object. Obviously the same relation can be applied for an object of mass $m$ coming from an infinite far distance.
	
	For the entry velocity of an asteroid in Earth's atmosphere we can assume that the minimum speed of a colliding asteroid is given by the above escape velocity relation for an asteroid returning to the zero potential of the Earth's gravitational field (a numerical application gives ${11 \;[\text{km}\cdot \text{s}^{-1}]}$.

	In fact, their real entering velocity depends on their direction that will determine their relative speed to Earth (which is ${30  \;[\text{km}\cdot \text{s}^{-1}]}$) plus eventually that of the Sun (which is around ${200  \;[\text{km}\cdot \text{s}^{-1}]}$. We can also take into account the escape veloctiy of the Solar System that is around  ${200  \;[\text{km}\cdot \text{s}^{-1}]}$.

	The sum gives therefore a speed between ${11 \;[\text{km}\cdot \text{s}^{-1}]}$ (for the optimistic case....) and ${300  \;[\text{km}\cdot \text{s}^{-1}]}$ (for the pessimistic case....) with a statistical peak that gives most observed entry at ${30 \;[\text{km}\cdot \text{s}^{-1}]}$.

	As we  will prove it in the section of Weather and Marine Engineering that at a height of $600$ [m] we can see at a distance of almost $80$ [km] we better understand why it is a joke in some movie to see huge asteroids entering the Earth atmosphere with people looking at it during $10$-$15$ seconds... and waiting almost $1$ minute before it hits the ground... (this is type of observation available for meteors but not for asteroids coming from very far!!!).

	A good example is to see all the YouTube videos about the small Chelyabinsk meteor (having a diameter of only $20$ meters) that entered Earth's atmosphere over Russia on 15 February 2013 and that had a speed of only almost  $30 \;[\text{km}\cdot \text{s}^{-1}]$ and which trajectory was visible during almost $20$ seconds only with the human eyes (so imagine with a speed $10$ times faster...).

	\subsubsection{Spherisation of Celestial Bodies}
	Thanks to Newton's law, we could answer to a lot of relevant questions in an approximated wayand giving us results quite convincing.

	A first example is to ask ourself at what scale there is a transition in the domain of irregular shapes (comets, asteroids, moons, etc.) to the field of spheres (moons, planets and stars)? Why the moons of Mars, Phobos and Deimos, have a potato shape like while our moon is roughly spherical. We will see below that this is due to the mass that is greater in the case of our moon. Indeed, from a certain mass, arbitrary geometric shapes are not possible anymore.

	To address this study, we will first estimate the maximum height of a mountain on a planet. Mount Everest has an altitude of $8.8$ [km] while Mount Olympus on Mars has a height of $27$ [km]. Why such mountains can not exist on Earth?

	To take a simplistic approach, we will assume that a mountain must be in hydrostatic equilibrium. We know experimentally the pressure limit in such a rocks lattice beyond which the rocks begin to "melt" (given in tables): $P_{\text{lim}}\cong 3 \cdot 10^8\;\text{[Pa]}$.

	We know from our study of continuum mechanics (\SeeChapter{see section Continuum Mechanics}) the pressure at the base of a mountain of height $h$ will be given in the hydrostatic approximation by:
	
	For the mountain to be stable, it is necessary that:
	For the mountain to be stable, it is necessary that:
	
	and therefore:
	
	Therefore:
	
	Assuming an average density of $rho=3,000\;[\text{kg}\cdot \text{m}^{-3}]$ (continental crust of the Earth) we get:
	\begin{enumerate}
		\item Earth: $h_0\cong 10\; [\text{km}]$
		\item Mars: $h_0\cong 27\; [\text{km}]$
	\end{enumerate}
	What is remarkable as approximate result!!!

	To estimate the minimum size $r_m$ of a body, starting the spherical shape becomes predominant compared to the surface deformation (that is to say where gravity has taken over the interatomic forces), we will require the size $r_m$ is greater than the maximum height of a mountain $h_0$. We also assume that the density $\rho$ remains constant through the body. Taking again the relation:
	
	we have:
	
	hence:
	
	The limit $r_m$ can after be estimated by fixing $r=r_m=h_0$ therefore:
	
	obviously for $r\gg r_m$ we will be even closer to the spherical shape.

	\paragraph{Flattening of Celestial Bodies (rotational flattening)}\mbox{}\\\\\
	Because of the symmetry of the gravitational potential a star or a planet should have a perfectly spherical form starting a given size, as we have just prove it. Now, the fact is... that it is not so for and especially for tellurice bodies.
	
	Because of the own rotation of the star or planet, a centrifugal term transforms potential. This term depends on the latitude which explains the ellipsoidal shape of most observed celestial bodies.
	
	Let us recall that:
	
	where $R$ is the equatorial radius of the star or planet, acceleration to which we have to add the centrifugal acceleration at a given latitude radius $r$ (\SeeChapter{see section Classical Mechanics}):
	
	Therefore the total acceleration given by:
	
	explains why the Earth is flattened at the poles (or depending of the point of view stretched to the equator ...) and that more one empty planet rotates, the more it will be flattened at the poles.
	
	On Earth, the equatorial radius is of $6,379$ [km] while the polar radius is of $6.357$ [km]. The difference is $22$ [km]. The "\NewTerm{flattening}\index{flattening}" of as star or planet is sometimes defines as:
	
	thus the difference between the equatorial radius and polar radius divided by the equatorial radius.
	
	Although an ellipsoid of revolution is the best description for the form of a planet:
	\begin{figure}[H]
		\begin{center}
		\includegraphics{img/cosmology/earth_with_atmosphere.jpg}
		\end{center}	
		\caption{Earth with its atmosphere and oceans}
	\end{figure}
	there are obviously imperfections between the model and the reality for some planets (in particular the terrestrial planets, satellites, and small rocky bodies). The geopotential of real body can be shaped much more complicated because of influences of the visible inhomogeneities on the surface as evidenced by this satellite image of the Earth omitting the liquid parts of it (the deformations are amplified by a factor $100,000$ in the image below!):
	\begin{figure}[H]
		\begin{center}
		\includegraphics{img/cosmology/earth_without_atmosphere.jpg}
		\end{center}	
		\caption{Earth without its atmosphere and oceans}
	\end{figure}
	The specialist of geodesics take into account these inhomogeneities. They measure and describe the shape of the planets that they name "\NewTerm{geoid}\index{geoid}".
	\begin{figure}[H]
		\begin{center}
		\includegraphics[scale=0.8]{img/cosmology/asteroid_spherisation.jpg}
		\end{center}	
		\caption{Evolution of asteroids shape in function of the radius}
	\end{figure}
	
	\subsubsection{Stability of Atmospheres}
	Comparing the liberation velocity and the velocities of various gases, we can explain the stability of certain atmospheres and the absence of others. We have proved in the section of Classical Mechanics that the liberation velocity of a spherical star was given by the following relation (on which we will come back in the section of General Relativity):
	
	For the Earth, a numerical application gives $v_L=11.2\cdot 10^3\;[\text{ms}^{-1}]$ and for the moon $v_L=2.37\cdot 10^3\;[\text{ms}^{-1}]$.

	Let us recall that we have proved in the section of Continuum Mechanics during our study of the kinetic temperature the following relation (Viriel's theorem):
	
	Using the molar mass (\SeeChapter{see section Thermochemistry}):
	
	A numerical application gives for nitrogen $v_\text{Az}=517\;[\text{ms}^{-1}]$ and for hydrogen $v_\text{Az}=1,934\;[\text{ms}^{-1}]$with an arbitrary temperature of $300 [\text{K}]$.
	
	So nitrogen is obviously trapped in the Earth's atmosphere. Hydrogen, light gas, more fast is less trapped. The two gases are even less trapped by the Moon.
	\begin{tcolorbox}[title=Remark,colframe=black,arc=10pt]
	In fact, the mean square speed is not the only speed of molecules. There is a distribution of velocities. We have indeed study the Maxwell-Boltzmann distribution of a gas at equilibrium in the section of Statistical Mechanics.
	\end{tcolorbox}
	

	\pagebreak	
	\subsection{Roche's Limit}
	The Roche limit is the theoretical distance below which a satellite would begin to break down under the action of tidal forces caused by the celestial body around which it orbit, these forces exceeding the satellite internal cohesion.
	
	We can simplify the problem by considering the satellite liquid, not rotating on itself (no spin), and decomposing it into two small masses $m$ of radius $r$ and volumic density $\rho_S$.
	\begin{figure}[H]
		\begin{center}
		\includegraphics{img/cosmology/roche_limit.jpg}
		\end{center}	
		\caption{Configuration for the study of the Roche limit}
	\end{figure}
	The planet is a sphere of radius $R$, mass $M$,  volume density $\rho_P$, located at a distance $D$ of the satellite axis.
	
	The planet exerts on the satellite the gravitational attraction:
	
	The difference of forces between the two masses is:
	
	We can consider that $r \ll D$, giving:
	
	So the difference in force is
	
	If the satellite cohesive force result in the gravitational attraction between the two masses:
	
	The satellite is destroyed if the difference in strength between the two masses is greater than the cohesive force:
	
	But we have the relations:
	
	Therefore we get:
	
	and we deduce of it the "\NewTerm{Roche limit}\index{Roche limit}":
	
	Depending on the approach and the approximations they can be a factor $3$ between some results.
	\begin{tcolorbox}[title=Remark,colframe=black,arc=10pt]
	For example the calculations given on Wikipedia consider only the difference in the primary's gravitational pull on the center of the satellite and on the edge of the satellite closest to the primary. This means that the main mass only apply one force momentum. But in fact this is not accurate as what interest us is the difference between the two extremities. This is why there is a factor $2$ between the Wikipedia calculations and ours (with our result the satellite will break twice the distance of that given by Wikipedia).
	\end{tcolorbox}
	
	Since in this calculation, we considered a satellite as a two point masses without rotation, and again we have assumed that the satellite's cohesion was provided exclusively by gravitational interactions, this value is an order of magnitude.
	
	\pagebreak
	\subsection{Keplerian Orbitals}
	Observation (main tool of the physicist and engineer for recall...) suggests at first glance, that the trajectories of celestial bodies in orbit around stars are indeed conical type (whew!) in the heliocentric reference frame. Knowing this, we can, in order to facilitate the calculation, anticipate the complexity of calculations and express the dynamics directly from a material point in polar coordinates.

	As we saw it in the section Vector Calculus, the speed in polar coordinate is expressed by the relation (we changed the angle Greek letter notation to adapt it to the tradition in astronomy):
	
	where to recall the first term is the radial velocity component and the second component the tangential (angular) velocity!

	For acceleration (the proof is still in the section of Vector Calculus):
	
	Now that we have the tools, let us get to the case of Keplerian orbits in the case of a static Newtonian field.

	There is to our knowledge the two main ways of doing the necessary mathematical developments but that do not gives (to our knowledge) the same level of detail results. The first approach provides finer results but is sometimes a bit do-it-yourself sometimes... is based on the use of the radial velocity and an important relation in astronomy, named the "first Binet formula". The second approach is simpler and most elegant, it uses the radial acceleration to approach the problem and a special relation named the "second Binet formula".
	
	\subsubsection{First Binet Formula}
	To start with this first approach to the problem, recall that we have already shown prove earlier that:
	
	However, it is unlikely that the main body is a perfect and homogeneous sphere ... so Astrophysicists have the habit of noting Newtonian potential $U$ under the form:
	
	where $\mu=GM$ is named "\NewTerm{gravitational constant of the star}\index{gravitational constant of a star}" (even if it is not always a star...) and where $f$ is a function representing the heterogeneity of the star.
	
	If there is one place in the universe where the laws of mechanics are perfectly verifiable, it is space, because the friction or causes of dissipation are extremely small. Within the field of a single force deriving from a potential, the movement satisfies the conservation of mechanical energy.

	Thus we end in the so-named "\NewTerm{energy equation}\index{energy equation}", wherein $E$ denotes the "\NewTerm{specific energy}\index{specific energy}" per unit weight (kilogram):
	
	Therefore:
	
	The Newtonian gravitational force is central, thus of having a null torque force at the center O of the main body. This results in the conservation of angular momentum in norm and direction, either:
	
	The vector $\vec{W}$ is the normalized vector of $\vec{b}$ or of $\vec{h}$ nalled the "\NewTerm{reduced momentum}\index{reduced momentum}". $K$ is the constant of areas (\SeeChapter{see section Classical Mechanics}) such that:
	
	We recall to the reader that the norm of the speed expressed in polar coordinates is given by the relation (remember that the both vectors of the polar base are orthogonal and that we can therefore apply the Pythagorean theorem to calculate the norm as it has been proved in the section of Vector Calculus):
	
	Which gives us the possibility to write the area constant $K$ as:
	
	Let us now put ourselves in the orbital plane, in polar coordinates.
	
	Given the relation already proved and known:
	
	and its squared norm:
	
	Or in the case of a central force (conservation of angular momentum):
	
	Let's put this in the prior-previous expression of $v^2$, then we have:
	Let us put this in the expression prior-previous expression of $v^2$, then we have:
	
	The relation:
	
	is named "\NewTerm{Binet's first formula}\index{Binet's first formula}".
	
	By equating with the expression of $v^2$ resulting from the conservation of energy that we get earlier above, we have:
	
	This gives us a rather complicated differential equation:
	
	And then we wonder how we can get out of such a situtation? After some hours of reflection ... we realize that takes to make a substitution. After another hour of neural chaos this ultimately leads to an end.... We decide to put (we have the right to do it!), knowing that $r$ is a function of $u$ and $\theta$:
	
	Let us derivate merrily relatively to $\theta$:
	
	Substituting in the differential equation:
	
	After simplification we get:
	
	We separate the variables to integrate:
	
	We have two solutions according to the sign we choose. However, at the end of the resolution, we notice that the only physically interesting choice is the negative sign. We have proved in the section of Differential and Integral Calculus And in our common derivatives that:
	
	We will chose the primitive in cosine and therefore we have:
	
	We leave, by approximation, the constant of integration that would involve very small oscillations in the orbit's path (if you do a study or a homework on this topic, you can transfer me your plots with or without the constant, it would interest me as I don't  have time to do it myself).

	This allows us to obtain:
	
	Now we see that our choice of the sign for the integration is fully justified because now, if we do a little recall on conics (\SeeChapter{see section Analytical Geometry}), we see that after rearrangement:
	
	So finally we have a relation of the form if we choose $\theta_0=0$:
	
	where by analogy with the section of Analytical Geometry $e$ is the eccentricity (let us recall that $e=c/a<1$ with $a$ the semi big axes and $c$ the distance to the center of the ellipse to the focal) and $p$ the focal parameter ($p=b^2/a$) of an ellipse. This corresponds well to the trajectories that follow celestial bodies in orbit.
	
	We thus fall back on our the first  Kepler "law"... so as we can see it, it can be proven!

	In our case, we have after simplification to resume:
	
	where (for recall) $K$ is the areas constant :
	
	and $\mu$ is the gravitation constant of the celestial body:
	
	and finally $E$ the specific energy:
	
	The reader could be able to check himslef as we have seen in the section of Analytical Geometry in our study conical that if:
	\begin{itemize}
		\item If $E=0$ such that $e=1$ we have an opened orbit in the form of a parabola

		\item If $E>0$ such that $e>1$ we have an opened orbit in the form of an hyperbola

		\item If $E\leq 0$ such that $0\geq e <1$ we have a closed orbit in the form of an ellipse or a circle
	\end{itemize}
	\begin{figure}[H]
		\begin{center}
		\includegraphics{img/cosmology/orbits.jpg}
		\end{center}	
		\caption[]{Reminders of conical but "orbit" oriented}
	\end{figure}
	Finally, if we inject:
	
	in the first Binet formula:
	
	then we get the velocity in any point of the ellipse based on the primary variable parameter which is therefore the angle.
	
	\subsubsection{Second Binet Formula}
	Let us now see the approach based on the radial acceleration which, while being more elegant, allows us to get a result less fine-tuned on the ellipse parameters.

	So we start from the expression of the acceleration in polar coordinates (\SeeChapter{see section Vector Calculus}):
	
	We can simplify the writing of the second term:
	
	Now we have seen just above that:
	
	and so:
	
	Then the acceleration is reduced to:
	
	We can eliminate the time by writing:
	
	and:
	
	Then we get:
	
	And so it comes to the standard "\NewTerm{Binet's second formula}\index{Binet's second formula}"
	
	But according to Newton's second law and his law of gravitation, we have:
		
	We then another form of the second Binet formula:
	
	Or after simplification and choosing the sign of the acceleration at our convenience to get rid of the "-" sign, we have:
	
	By isolating the constants, we get:
	
	After a change of variables we recognize the particular case of a differential equation of the second order we have already met several times so far in the various sections of this book and we will meet again:
	
	As it is customary, however, we will shoe the details of the resolution. The equation without second member is (\SeeChapter{see section of Differential and Integral Calculus}):
	
	We then have the discriminant that is negative since:
	
	We then saw in the section of Differential and Integral Calculus, that in this situation the solution of the homogeneous equation was of the form:
	
	Thus in the situation we are concerned, we have:
	
	We inject the solution into the homogeneous differential equation with second member:
	
	and we see immediately see that that for the equality to be satisfied, the general solution is:
	
	Or after rearrangement:
	
	And choosing the initial angle as zero, so we find well:
	
	at the difference with the first method of resolution that the value of the constant $A$ is unknown.
	
	Let us now come back on:
	
	By expliciting:
	
	And as (\SeeChapter{see section Classical Mechanics}):
	
	So if we choose a particular point of reference of the path (not necessarily circular path), we have:
	
	Then we have:
	
	If we put the phase shift as zero relatively to the reference radius choosed earlier above, the expression simplifies to:
	
	To determine the constant $A$, we place ourselves in the case where $\theta=0$ and imposes that the radius $r$ is the initial radius measured when this angle is zero. Then we have:
	
	This implies immediately:
	
	Thus after elementary rearrangements and simplifications:
	
	And therefore we have a direct correspondence:
	
	And as the eccentricity $e$ is known for a circular, parabolic, elliptical or another trajectory of the conical type... it gets us very easy to deduce the velocity at the particular point of the initial radius $r$ of the studied object.

	The closest distance of the object orbiting around its central star (focus), will be given by the value that can takes $r$ in the relation:
	
	if we impose $\theta=0$.

	In the case of an elliptical orbit it is the "\NewTerm{perigee}\index{perigee}" to be assimilated to the initial radius as the point where the measurement of the radial velocity was the most accurate.

	The farthest distance from the focus will be given by putting the angle as being $180^\circ$ ($\pi$) and then we name it "\NewTerm{apogee}\index{apogee}".
	
	\pagebreak
	\subsubsection{Keplerian orbital period}
	The Kepler's law of equal areas allows, as we already know, to calculate the Keplerian orbital period $T$. In fact, the area $S$ of the ellipse being equal to $S=\pi a b$ (\SeeChapter{see section Geometric Shapes}) and having already determined during our definition of the angular momentum that (\SeeChapter{see section Classical Mechanics}):
	 
	It comes naturally:
	
	Moreover, the study of conics (\SeeChapter{see section Analytic Geometry}) has showed us that:
	
	and we have defined above:
	
	So we have the relation:
	
	and then we fall back again on the third Kepler's law:
	
	which validates our previous calculations.
	
	Obviously the latter relation is only available for $T$ where the corresponding speed is non relativistic otherwise we have to use General Relativity tools.

	\pagebreak
	\subsubsection{Classical deflection of light}
	The calculations done previosuly can be applied to an interesting case: the deflection of light by a star in the Newtonian interpretation (of course!).

	Warning!!! Newton did not know at its time that the photon was massless. The following developments are therefore a wrong approach in our time and should be taken with precaution but are still taught today because it allows students that do not yet studied General Relativity or that will never study it  (in the section on General Relativity, the reader will find the contemporary detailed proof of the deflection of light that is a whole other level) to have a first approach... it's like everything in physics! Until we have reached the level of the university degree, we learn many things "wrong" because oversimplified. Then at the Master or PhD level, we learn a little more realistic and valid theories.

	Well this being recalled (following the remark of one of our reader), so we have proved above for recall that:
	
	In the case of a photon, we tend to put that $r\rightarrow +\infty$ (thus a hyperbolic trajectory) and therefore this requires that in the previous relation we have (which is equivalent to saying that $e$ is strictly greater than the unit as required by the hyperbolic trajectory):
	
	by putting $\varphi=2\theta-\pi$ the elementary trigonometric relations (\SeeChapter{see section Trigonometry}) give us:
	
	and therefore still using trigonometric identities:
	
	Therefore:
	
	And we know that:
	
	Hence:
	
	neglecting the potential energy of the photon since $r\rightarrow +\infty$ (caution !!! recall that according to what we saw in the section of Special Relativity, the photon has no mass strictly speaking but Newton knew nothing about this at his time!):
	
	Therefore:
	
	Hence:
	
	After simplification:
	
	and as $\theta$ is assumed to be small, we have using the Taylor expansion (\SeeChapter{see section Sequences and Series}) of the tangent function:
	
	So it finally comes:
	
	But, we have by definition:
	
	and we know that $v=\omega r=\dot{\theta}r$ (\SeeChapter{see section Classical Mechanics}). Thus we have:
	
	If the particle is a photon passing flush with to surface of the Sun then:
	
	a numerical application gives:
	
	Newtonian theory thus provides a $0.87$ arc seconds deviation for a ray of light passing flush to the Sun's surface. Which is twice less than what can be observed experimentally and that gives the theory of General Relativity (\SeeChapter{see section General Relativity})!
	
	\subsubsection{Classical precession of perihelia}
	Before studying the precession of the orbits, we would recall that the gravitational field is a conservative and center field. This implies that the angular momentum (\SeeChapter{see section of Classical Mechanics}) is constant and that the path is held in a plane whose normal vector to the surface always maintains the same direction (the angular momentum vector is constant in norm and direction for recall!).

	We will address here the analysis of the precession of the perihelion taking into account the results of the theory of special relativity (allowing it to be more accurate in the results and be able to apply these results to the orbiting electrons around the nucleus of the atom in the corpuscular model).
	
	First let us recall that:
	\begin{itemize}
		\item The "perihelion" is the point of the orbit of a celestial body (planet, comet, etc.) that is closest to the star around which it rotates.

		\item The "aphelion" is the point in the orbit of an object (planet, comet, etc.) where it is farthest from the star around which it rotates.

		\item The "equinox" is the moment (time) when the central star crosses the plane of the equator of the object that is in orbit around it.
	\end{itemize}
	\begin{tcolorbox}[title=Remark,colframe=black,arc=10pt]
	When the Sun passes from the southern hemisphere to the northern hemisphere of the Earth (in other words when the Sun is at the Zenith at midday at the equator), it is the spring equinox (20 or 21 March) in the opposite direction, this is the autumn equinox (22 or 23 September). At these dates, there is equality of day and night all over the Earth.
	\end{tcolorbox}
	Obviously, the result we get will here is not complete, since, as we know, we had to wait the development of General Relativity to give the exact value of the perihelion  precession of Mercury (see will come back on this subject further below).

	To calculate the effect of precession, we will seek the equivalent of the Binet formulas seen above in relativistic form (we will see the classical form in the section of General Relativity). For this we proceed as follows:

	The relativistic Lagrangian of the system is (\SeeChapter{see section Special Relativity}):
	
	\begin{tcolorbox}[title=Remark,colframe=black,arc=10pt]
	We subtract then energy at rest because only interest us here the study of the kinetic and potential energy. The potential energy is summed in the Lagrangian above (which is not consistent with the practice) but we will reverse the sign later below during the developments.
	\end{tcolorbox}
	With (see section Special Relativity and Vector Calculus):
	
	and the reduce mass for recall:
	
	The angular moment:
	
	in relativistic form and applied to our study is:
	
	Taking the norm, we have without forgetting that in our study $\vec{\omega}\bot\vec{r}$ and therefore $(\vec{\omega}\bot\vec{r})\bot\vec{r}$:
	and let us recall that we have adopted the notation $\omega=\dot{\theta}$ (in case you forget the definition...). Which finally gives us:
	
	To establish the relativistic equivalent of the Binet formulas:
	\begin{itemize}
		\item We deduce the expression of the angular momentum:
		

		\item We seek for a relation of the type $\dot{r}=\dot{r}(\theta)$ (as the trajectory is a conic):
		
		Indeed let us recall that in polar coordinates the speed is given by the following expression (\SeeChapter{see section Vector Calculus}):
		
		That is to say, $\dot{r}=f(r,\theta)$. The latter exprresion gives us the possibility to write that:
		
		
		\item We seek a relation of the type $\ddot{r}=\ddot{r}(\theta)$:
		
	\end{itemize}
	From the equations obtained previously, we have successively:
	
	Let us recall that we have defined in special relativity $\beta$ and that by using the speed in polar coordinates:
	
	With the previous relations, this gives us:
	With the above relations, this gives us:
	
	On the other hand:
	
	By introducing in the prior previous relationship in the latter:
	
	By putting $u=1/r$ and as:
	
	The prior-previous relationship becomes with this expression:
	
	Equating this relation with that of the Lagrangian:
	
	Differentiating the latter relation relatively to $\theta$:
	
	Indeed, the Lagrangian is constant over time (the system is assumed to be conservative), we then have:
	
	and also:
	
	But if we continue:
	
	By refering to:
	
	So we get:
	
	That gives after a few simplifications:
		
	By multiplying the latter by $\mu^2c^2/b^2$:
	
	In a gravitational potential:
	
	The Binet equation in special relativity is then:
	
	To find a solution to this differential equation, we will group the variable $u$ in the left side:
	
	We put:
	
	The differential equation then can be written:
	
	We put:
	
	By taking the second derivative:
	
	We then get a simple differential equation:
	
	whose solution is well known to us (\SeeChapter{see section Differential and Integral Calculus}):
	
	What can still be written as $\Omega^2$ is a constant:
	
	with $k_1,k_2=c^{te}$.

	To determine the constants $k_1,k_2$, we place ourselves irst in the situation for which $\theta=0$, where $r$ is minimal and therefore by $u$ is maximum by definition.
	
	We derivate relatively to $\theta$:
	
	Therefore $k_2=0$ which makes that the relation:
	
	becomes:
	
	Written differently (trying to return to a similar notation to that of the study of conic) then:
	
	And the interest to write this in this way is to notice that we fall ultimately on the equation of an ellipse with $p$ being the focal parameter of the conic, focal parameter given for recall by (\SeeChapter{see section Analytical Geometry}):
	
	where $a$ is the half major axes of the ellipse.
	
	Now let us put:
	
	In the first passage through the perihelion $\theta=0$ where:
	
	we therefore have:
	
	Now let us put:
	
	At the first passage through the perihelion $\theta=0$ where:
	
	we have:therefore
	
	At the second passage through the perihelion $\theta=2\pi$, we have:
	
	we also have:
	
	The trajectory is stille an ellipse but the angle $\Omega\theta_0$ that was zero initially has become $\Omega\theta_1=2\pi$.

	That is, if we have:
	
	Therefore:
	
	Which gives us:
	
	Since $G^2\ll c^2$, a development in Taylor series give us (\SeeChapter{see section Sequences And Series}):
	
	By limiting at the order $2$:
	
	So in conclusion, there is an advancement of the perihelion taking place in the satellite's direction of rotation. For a repository located in the satellite's rotation plane, the trajectory is always an ellipse.

	This advance is of:
	
	by period. Either by expliciting the momentum given for reminder by:
	
	It comes after simplification:
	
	We will now allow us a rough approximation (mixture of relativistic and non-relativistic). If we consider the last relation, we have obtained during our developments of the Keplerian orbital trajectories the relation:
	
	Therefore, injecting this into the relation of $\Delta \alpha$	we have:
	
	Unfortunately, the numerical values for Mercury precession give only a precessions of an angle of $7''$ per century and not the $43''$ as expected (...) there is therefore a lack of a factor $6$ that only the General 
Relativity (\SeeChapter{see section of General Relativity}) makes possible to found. It is nevertheless interesting that Special Relativity already gives an orbit that precesses where Newton sees stable ellipse and that this approximation works for all the planets except Mercury (the planet closest to the Sun and undergoing the brunt of curvature of space-time).
	\begin{tcolorbox}[title=Remark,colframe=black,arc=10pt]
	By applying exactly the same reasoning to corpuscular quantum physics (electrical potential) but with the ad hoc constants seen in the section of Electrostatics, we find:
	
	with $\vec{b}=\mu\vec{r}\times\vec{v}$ being the momentum and in the case of the atom, we will take (\SeeChapter{see section Corpuscular Quantum Physics}):
	
	with reduced mass equal to:
	
	\end{tcolorbox}
	If the positions of the perihelion (and therefore the aphelion) of the Earth-Moon center of gravity  were constant over time, the duration of the different seasons would be constant. But the orbit of the center of gravity Earth-Moon also rotates in its plane in the forward direction at about 12 '' per year (a revolution is about $108,000$ years).

	The precession of the equinox occurs in the opposite direction (retrograde direction) at about $50''$ per year (then a "\NewTerm{precession equinox}\index{precession equinox}" revolution is about $26,000$ years). The combination of these two movements permits to calculate the period of the passage of the perihelion of the Earth by the direction of the vernal equinox, this period of about $21,000$ years and is named the "\NewTerm{climatic precession}\index{climatic precession}.
	\begin{figure}[H]
		\begin{center}
		\includegraphics[scale=0.9]{img/cosmology/precession_orbit_earth.jpg}
		\end{center}	
		\caption{Effects of precession on the seasons using the Northern Hemisphere terms (source: Wikipedia)}
	\end{figure}
	Indeed, every $10,500$ years (half period of climatic precession) aphelion changes from summer to winter. But even if the Earth-Sun distance is by far not the predominant factor in the nature of the seasons, the combination of the passage of the Earth in the winter in aphelion gives winters a little bit mor harsh. Earth-Sun distance also depends on the variation in the eccentricity of Earth's orbit (due to external and inner planets). Thus, the ice ages are correlated with the minimum eccentricity of Earth's orbit.
	\begin{figure}[H]
		\begin{center}
		\includegraphics[scale=0.9]{img/cosmology/precession_orbit_earth_perspective.jpg}
		\end{center}	
		\caption[]{Simplified and perspective point of view of the previous figure (source: Latsis foundation (2001))}
	\end{figure}
	The work of the Celestial Mechanics Institute (France), since the 1970s, would have to definitively confirm the theoretical predictions as what the eccentricity of Earth's orbit undergoes wide variations formed numerous periodicals under which the most important one have periods near $100,000$ years, and for one of them, a period of $400,000$ years. These results confirm the climatic variations of the Earth during the Quaternary (\SeeChapter{see section Weather \& Marine Engineering}). The paleoclimatology models indeed show the correlation between changes in the Earth's orbit elements and large quaternary glaciation.
	\begin{tcolorbox}[title=Remark,colframe=black,arc=10pt]
	In the case of the hydrogen atom (\SeeChapter{see section of Corpuscular Quantum Physics}), for the case dealing with relativistic model of Sommerfeld, with $n=1,n_\theta=1,Z=1$ and and the fine structure constant approximately equal to $1/137$, we get to the precession of the perihelion of the orbit of the electron given by:
	
	according to a corpuscular point view of matter!
	\end{tcolorbox}
	
	\subsection{Duration of the diurnal arc}
	A diurnal arc is the time, as expressed in right ascension, it takes a planet, point, or degree to move from its rising point to its setting point. This takes place in many celestial bodies such as the sun and moon.
	
	So we will study here at the time length of the day, more exactly to the portion of day where we are illuminated by the Sun, as compared to the night when we are in the shade\footnote{Thanks to Xavier Hubaut for these very friendly developments}.
	
	In reality, the Earth revolves around the Sun and describes an almost circular orbit at the same time it turns on itself around its axis that is tilted (actually) by about $23^\circ 27'$ relatively to its orbital plane (the ecliptic):
	\begin{figure}[H]
		\begin{center}
		\includegraphics{img/cosmology/equinox_solstice.jpg}
		\end{center}	
		\caption{Representation of the rotation of the Earth on its orbit with its major phases}
	\end{figure}
	\begin{tcolorbox}[title=Remark,colframe=black,arc=10pt]
	It is obvious that, given the complexity of the problem, we will simplify it by considering a circular orbit without variations (precession, nutation) of the axis of rotation of the Earth. We will assume that the Sun is reduced to a point (no dawn or twilight, etc.).
	\end{tcolorbox}
	Let us first recall that the precession is the gradual change in direction of the axis of rotation of an object when a torque (force) is applied to it while the nutation is a periodic balancing of the axis of rotation of the Earth around its mean position in addition to the precession (\SeeChapter{see section Classical Mechanics}).
	\begin{figure}[H]
		\begin{center}
		\includegraphics{img/cosmology/nutation_precession_earth.jpg}
		\end{center}
	\end{figure}
	Let us represent the Earth with its vertical axis of rotation. Accordingly the equator will be located in a horizontal plane.

	Suppose that that day, the Earth is in such a position that the Sun's rays form an angle $\alpha$ with the equatorial plane (or conversely that the axis of the Earth form an angle with the equatorial plane). Notice that this angle $\alpha$ will always be according to actual measurements  between $-23^\circ 27'$ and $+23^\circ 27'$ at least... at a human life time scale...
	
	For our example we have chosen to focus our analysis on a day when $\alpha$ is positive. Thus, in the northern hemisphere, we are close to the summer solstice!

	We are looking for the day length at a place located at latitude $\lambda$! To fix ideas, we place ourselves around Brussels at $50^\circ$ north latitude.
	
	Let us now consider the following figures where the first is a view of the side of Earth at a time $t$ of its orbit when $\alpha>0$ and the second in to a cylindrical cutting of diameter $\overline{NJ}$ (corresponding to the diameter of the parallel of Brussels) of Earth's volume Earth at this same moment:
	\begin{figure}[H]
		\begin{center}
		\includegraphics{img/cosmology/diurnal_arc_duration.jpg}
		\end{center}
	\end{figure}
	On the figures above, $C$ denotes the center of the Earth, and O the center of the parallel of Brussels.

	Let us fix a time $t$ and denote by $M$ (morning) and $S$ (evening) the two points of the parallel of Brussels where the Sun rises and sets (these points will be considered fixed whatever the moment $t$, which is obviously wrong relatively to the reality), while $J$ (day) and $N$ (night) will be the points where it is noon and midnight respectively.

	$P$ will be the point on the disc corresponding to the Brussels parallel where the meridian noon plane  (the plan which of the sides is $\overline{NJ}$) cut the line $\overline{MS}$.

	Finally, $\gamma$ designate the angle $\widehat{M\text{O}S}$ (where O is the center of the disk generated by the parallel of Brussels) behind the illuminated part by the Sun and $r$ designate the radius $S\text{O}=M\text{O}$.

	To simplify the problem, let us also assume... that during $24$ hours the Earth rotates on itself without changing the position of its axis of rotation relative to the Sun....

	The angle $\gamma$ can be calculated by noting that $\overline{\text{O}P}$ is, in absolute value equal to:
	
	where $r$ represents for recall the radius of the parallel of Brussels.
	
	Using the properties of trigonometric functions (\SeeChapter{see section Trigonometry}), we have:
	
	But we still need to inject the parameter $\alpha$. Knowing the latitude $\lambda$ of Brussels, we have:
	
	where $R$ is the radius of the Earth.

	We have also:
	
	and in the triangle $C\text{O}P$:
	
	Finally, by comparing the values obtained for $\overline{P\text{O}}$, we get:
	
	and as:
	
	We finally get:
	
	and therefore:
	
	At the equinoxes (that is to say when the equator coincides with the ecliptic plane for recall...) we have $\alpha=0$ and therefore:
	
	However, as we have specified it at the beginning, we must take the absolute value thus:
	
	In other words, whatever the latitude we take, the angle formed by the night area is equal to the angle formed by the day area at equinoxes (both being equal to $\pi$).

	Let us now consider the summer solstice, when $\alpha=23^\circ 27'$ still considering the latitude of Brussels $\lambda=50^\circ$, we have:
	
	This translated into hours by:
	
	So the 24-hour day loses $7.9$ hours. Which is equivalent to a day light of approximately $16$ hours.
	
	In summary to calculate the duration of a "day", it is enough to know two things: the latitude and the angle at which the Sun falls on the plane of the equator to the chosen date. The value of this angle is well known at the equinoxes (it is $0^\circ$) and to the solstices (it is $+23^\circ 27'$ and $-23^\circ 27'$).

	But what about the other dates?
	
	The answer is quite simple. Let us imagine, sitting on the Sun watching throughout the year towards the center of the Earth.

	During its rotation around the Sun (the binome centroid in fact), the axis of rotation of the Earth maintains its inclination to the ecliptic. Seen from the Sun, this axis revolve aroundnormal to the plane of the ecliptic and therefore describe a cone whose half apex angle is $23^\circ27'$ (see figure below).

	The angle of attack $\alpha$ of the sunlight on the equator therefore vary according to the date $\delta$ (we associate to the date, the angle $\delta$ traveled by the Earth on its orbit, from its position the spring equinox)

	Therefore, the angle $\alpha$ vary according to the date $\delta$ sinusoidally.

	For those who are perhaps not convinced by this semi-intuitive reasoning, here's another approach:

	For readability of the diagram, we have greatly exaggerated the angle of the axis of rotation of the Earth with the ecliptic:
	\begin{figure}[H]
		\begin{center}
		\includegraphics{img/cosmology/cone_generated_by_earth_rotation.jpg}
		\end{center}
	\end{figure}
	Given $C$ the Earth's center, $A$ the end of a unitary vector $\overrightarrow{CA}$ oriented directed along the axis of rotation of the Earth (ie perpendicular to the plane of the equator) and another unit vector $\overrightarrow{CS}$ directed toward the Sun. Given now $\alpha$ the angle of the radius $\overline{CS}$  with the plane of the equator and $\beta$ the angle between the unit vectors  $\overrightarrow{CS}$ and  $\overrightarrow{CS}$. Then we have:
	
	Indeed, the vector $\overrightarrow{CA}$ being perpendicular to the plane of the equator, it forms a right angle with it. Therefore since the angle $\beta$ is the angle between this vector and the ecliptic, the angle $\alpha$ is then the complementary angle.
	
	Therefore we have:
	
	Let us decompose now $\overrightarrow{CA}$ in the sum of $\overrightarrow{CA'}$ directed perpendicular to the ecliptic plane and of $\overrightarrow{CA''}$ located in the ecliptic plane:
	
	Therefore:
	
	But:
	
	So finally:
	
	and as we have demonstrated that:
	
	We finally get:
	
	Now the problem is solved and the daylight time duration will depend on two variables: the date $\delta$ and the latitude $\lambda$.

	We just have so now to take again the relation:
	
	and to inject in it the new result to get a first simple version of "\NewTerm{equation of time}\index{equation of time}":
	
	With computer tools at our disposal, we can easily calculate the value $\gamma$. For example, we have below the variations in the length of the day over a year at latitudes of $0$ to $90^\circ$ spread by $10$ by $10^\circ$:
	\begin{figure}[H]
		\begin{center}
		\includegraphics{img/cosmology/equation_of_time.jpg}
		\caption{Equation of time plot}
		\end{center}
	\end{figure}
	From the latitude of the Arctic Circle, we see, in summer, periods with uninterrupted sun (midnight sun) and in winter whole days of night.

	For Brussels (latitude = $50^\circ$) we see from the figure that the length of the day varies approximately between the values of $16$ [h] (summer solstice) and $8$ (winter solstice).
	
	\subsubsection{Trigonometric parallax}
	Measuring distances to objects within our Galaxy is not always a straightforward task – we cannot simply stretch out a measuring tape between two objects and read off the distance. Instead, a number of techniques have been developed that enable us to measure distances to stars without needing to leave the Solar System. One such method is "\NewTerm{trigonometric parallax}\index{trigonometric parallax}", which depends on the apparent motion of nearby stars compared to more distant stars, using observations made $6$ months apart (corresponding to the measurement of diameter of the apparent approximated circular motion they have in the sky).

	A nearby object viewed from two different positions will appear to move with respect to a more distant background. This change is named a "\NewTerm{parallax}"\index{parallax}. A simple demonstration is to hold your finger up in front of your face and look at it with your left eye closed and then your right eye. The position of your finger will appear move compared to more distant objects.

	By measuring the amount of the shift of the object's position (relative to a fixed background, such as the very distant stars) with observations made from the ends of a known baseline, the distance to the object can be calculated.
	
	The trigonometric parallax method is very simple (but difficult to implement on the surface of our planet for very distant stars). Any amateur astronomer observed the flight of the star it observes with his eye. This movement is named as we have just seen the "\NewTerm{diurnal movement}\index{diurnal movement}" It is due to the rotation of the Earth itself. The star is also driven in an elliptical motion much less easily detectable: the "\NewTerm{parallactic motion}\index{parallactic motion}".

	It is due, as suggested in the figure below, to the rotation of the Earth around the Sun. So we measure the angle $p$ and we have obviously:
	
	If the angle is small (which is very often the case given the distance of stars...) we can take the first term of the Taylor expansion (\SeeChapter{see section Sequences And Series}) of the tangent function:
	
	Which allow us to write:
	
	\begin{figure}[H]
		\begin{center}
		\includegraphics{img/cosmology/parallax.jpg}
		\caption{Trigonometric parallax principle}
		\end{center}
	\end{figure}
	If the parallax angle, $p$ is measured in arcseconds (arcsec), then the distance to the star, $d$ in parsecs (pc) is given by:
	
	It is important to notice that in this example we assume that both the Sun and star are not moving with a transverse velocity with respect to each other. If they were this would complicate the picture as presented here. In practice stars with significant proper motions require at least three epochs of observation to accurately separate their proper motions from their parallax. Stars that are members of binaries further complicate the picture.

	The only star with a parallax greater than $1$ [arcsec] as seen from the Earth is the Sun - all other known stars are at distances greater than $1$ [pc] and parallax angles less than $1$ [arcsec] ($1/3600$ of degree... we understand better why this what impossible to measure before the 19th century). When measuring the parallax of a star, it is important to "account for the star’s proper motion, and the parallax of any of the fixed" stars used as references.

	Over a $4$ year period from 1989 to 1993, the Hipparcos Space Astrometry Mission measured the trigonometric parallax of nearly $120,000$ stars with an accuracy of $0.002$ [arcsec]. The GAIA mission, to be launched in 2010, will be able to measure parallaxes to an accuracy of $10^{-6}$ arcsec, allowing distances to be determined for more than $200$ million stars.
	
	In practice when we measure the parallax we must obviously take in account the obliquity of Earth on it's orbit that is in this beginning of the 21st century equal to $23^\circ 26'13.3$ otherwise me may think that stars have a huge parallax and therefore and are therefore at a small distance of us as illustrated be the figure below:
	\begin{figure}[H]
		\centering
		\includegraphics[scale=0.5]{img/cosmology/parallax_big_dipper_north_star.jpg}
		\caption[]{Shift angle of the Big Dipper that seems huge if we do not subtract the obliquity angle}
	\end{figure}
	In the figure above the "North star" may be any fixed star close to either celestial pole of any given planetary body. It might refer to any such star in the Earths remote history or future, situated along the path of the celestial poles in the course of the procession of the Earth's axis.
	
	The identity of the pole stars gradually changes over time because the celestial poles exhibit a slow continuous drift through the star field. The primary reason for this is the precession of the Earth's rotational axis, which causes its orientation to change over time. Precession causes the celestial poles to trace out circles on the celestial sphere approximately once every $26,000$ years, passing close to different stars at different times (with an additional slight shift due to the proper motion of the stars).
	\begin{figure}[H]
		\centering
		\includegraphics[scale=0.5]{img/cosmology/north_star_precession.jpg}
		\caption[]{The path of the north celestial pole amongst the stars due to the effect of precession, with dates shown (source: Wikipedia)}
	\end{figure}
		
	\subsection{Planets' Motion}
	We will briefly turn our attention to the movements of the planets in ideal and situations simplified in the point of view on an observer on Earth. We consider that all the movements will be in the same plane (coplanar) perfectly circular and constant...

	\textbf{Definition (\#\mydef):} The planets that are closer to the Sun than the Earth (whose radius is less than one astronomical unit AU\footnote{defined as an average of $149,597,870,700$ [m] ((about $150$ million kilometers). In ISO 80000-3, the symbol of the astronomical unit is "ua".}), that is to say the planets Mercury and Venus are "\NewTerm{inferior planets}\index{inferior planets}", the other planets (Mars and beyond) are named the "\NewTerm{outer planets}\index{outer planets}".
	
	\subsubsection{Synodic and Sidereal period}
	One of the many tools used in Astronomy are the formulas used to determine Orbital Motion. There are two basic forms of orbit periods:
	\begin{itemize}
		\item Sidereal Period
		\item Synodic Period
	\end{itemize}
	A "\NewTerm{sidereal period}\index{sidereal period}" is an actual measure of a complete orbit relative to the stars (since the stars are unmoving - or at least moving very slowly). A "\NewTerm{synodic period}\index{synodic period}" is a rotation of a planet so that it appears to be in the same place in the night sky.
	
	The synodic period of a planet (or satellite )is the time needed by this planet to return to the same configuration Earth-Planet-Sun (if we consider this particular case), that is to say in the same place in the sky relatively to the Sun, as seen from Earth. This period differs from the sidereal rotation period of the planet because the Earth itself moves around the Sun. Accordingly, it is the period of apparent revolution, the duration between two conjunctions Planet-Sun as viewed from Earth.

	The term generally refers to the time between two identical aspects of the object (opposition, conjunction, etc.) and thus depends on the three bodies involved.
	
	To mathematically study the problem in question, let us consider the following diagram with two planets describing a perfectly circular orbit at a constant angular velocity and in the same plane and in the same direction and where we have $\omega_1>\omega_2$ (thus the inner planet is faster than the outer planet):
	\begin{figure}[H]
		\begin{center}
		\includegraphics[scale=1]{img/cosmology/synodic_period_schema.jpg}
		\end{center}	
		\caption{Basic scheme for determining the synodical period}
	\end{figure}
	where $P_1$ and $P_2$ are two planets which we will denote the respective sidereal  periods by $T_1$, $T_2$ and for which we deduce the angular velocities:
	
	If we take as zero time, the time when the two planets are both aligned with the $X$ axis and at the same side of this axis (so in "inferior conjunction"), then the angle between this axis and each of the planets is:
	
	We have respectively:
	
	We seek therefore all the instants $t$ where the following relation is satisfied for a fixed $\alpha_{12}$:
	
	Therefore it comes:
	
	If we look from time zero the first (next) conjunction ("superior conjonction"), this is equivalent to put that $\alpha_{12}=\pi$ and therefore that:
	
	If we look from time zero the first (next) conjection ("inferior conjonction"),  this is equivalent to put that $\alpha_{12}=2\pi$ and therefore that:
	
	In the case where $\omega_2>\omega_1$ (typically Earth and one of its outer planets), the same reasoning leads us to:
	
	Here are some periods synodic and sidereal planets of the solar system relatively to Earth:
		
	As we can see from this table, we can make some of empirical observations:
	\begin{enumerate}
		\item For the inner planets: The closer we get to the Sun, the more the synodical period is short, indeed in the proved relation above, the more $T_1$ is small more $T$ decreases. So if there was a rotating planet very near the Sun, both sidereal and synodic periods are substantially equal.

		\item When we approach the Earth, the period increases. If there was a planet near to Earth, we would then have $T_1$ value close to $T_2$ value and the synodical period would be very large.

		\item For the outer planets: The synodic period decreases when the planet is farther from the Earth and approaches terrestrial sidereal period of $365$ days. We see well for Neptune, if we discovered a planet even further its synodic period would approach even more the $365$ days.
	\end{enumerate}
	
	\pagebreak
	\subsubsection{Planet's apparent retrograde motion}
	The "\NewTerm{retrograde motion}\index{retrograde motion}" of a planet is an apparent motion of this planet which gives the impression that it stop in his path in the "direct movement" to start reversing. This phenomenon is the result of the difference between the revolution speed of the planet and the Earth around the Sun.

	The example below shows roughly what a terrestrial observer (yellow dot) can be observed by monitoring month after month, the apparent motion of Mars (cyan point):
	\begin{figure}[H]
		\begin{center}
		\includegraphics[scale=0.6]{img/cosmology/retrograde_motion.jpg}
		\end{center}	
		\caption{Retrograde motion principle (source: Wikipedia)}
	\end{figure}
	Or more explicitly:
	\begin{figure}[H]
		\begin{center}
		\includegraphics[scale=0.8]{img/cosmology/retrograde_motion_mars.jpg}
		\end{center}	
		\caption{Apparent retrograde motion of Mars in 2003 as seen from Earth (source: Wikipedia, author: Eugene Alvin Villar)}
	\end{figure}
	To study this phenomenon mathematically, we will consider the following figure:
	\begin{figure}[H]
		\begin{center}
		\includegraphics[scale=0.8]{img/cosmology/retrograde_motion_study_figure.jpg}
		\end{center}	
		\caption{Basic scheme for the study of planet's retrograde motion}
	\end{figure}
	with two planets describing a perfectly circular orbit at a constant angular velocity and in the same plane and in the same direction and where we have. It is clear that the inner planet will therefore caught up the outer planet and it will seem to have a retrograde motion as shown in the figure below:
	\begin{figure}[H]
		\begin{center}
		\includegraphics[scale=0.8]{img/cosmology/retrograde_motion_explicative_sheme_for_time_zero_choice.jpg}
		\end{center}	
		\caption[]{Figure to illustrate the choice of zero time}
	\end{figure}
	As the reader can check it in the figure above we see that the retrograde motion with respect to the fixed stars begins when the angle between the two planets is zero and it ends when the angle between the two planets pass through a maximum.

	Therefore, in the prior previous figure, we have:
	
	So to know the time between when the moment where the angle is zero between the two planets, reaches a maximum and decreases again, we simply need determine when occurs the sign of change in the previous function. To do this we just search when the derivative is zero:
	
	By applying the derivation rules seen in the section of Differential and Integral Calculus:
	
	Hence after simplification:
	
	We develop all this:
	
	and we simplify a first time:
	
	and second:
	
	and finally a third one:
	
	and after rearrangement:
	
	We simplify using trigonometric identities proved in the section  Trigonometry:
	
	The values of $t$ that satisfy this relation gives us the sign change we were looking for.

	If $t_0$ is the first value of $t$ that satisfies the equation, we have:
	
	The next value of $t$ will be such that:
	
	and therefore:
	
	If we introduce the rotation periods, we have:
	
	To come back to:
	
	it may be more convenient to write it in the traditional following form:
	
	So far, we have only do geometry. No law of gravitation intervened in the calculations. As the radius are unknown or little known (at least historically), we will use the Kepler's third law (periods law) that is for recall:
	
	where for recall $D$ is the semi-major axis of the orbit, and if it is circular, it becomes a simple radius. So we have:	
	
	Therefore:
	
	hence:
	
	A numerical application with for Mercury $T_1\cong 87.95$ [d] and for Earth $T_2\cong 365.25$ [d] the value:
	
	Value we have represented in the diagram below:
	\begin{figure}[H]
		\begin{center}
		\includegraphics[scale=1]{img/cosmology/retrograde_motion_first_value.jpg}
		\end{center}	
	\end{figure}
	and therefore:
	\begin{figure}[H]
		\begin{center}
		\includegraphics[scale=1]{img/cosmology/retrograde_motion_second_value.jpg}
		\end{center}	
	\end{figure}
	and therefore we have:
	
	then a new cycle:
	
	etc. What gives in schematic form:
	\begin{figure}[H]
		\begin{center}
		\includegraphics[scale=1]{img/cosmology/retrogradiation_cycle_diagram_principle.jpg}
		\caption[]{Retrogradiation cycle diagram principle}
		\end{center}	
	\end{figure}
	\begin{tcolorbox}[title=Remark,colframe=black,arc=10pt]
	At specific points on Mercury's surface, an observer would be able to see the Sun rise part way, then reverse and set before rising again, all within the same Mercurian day. This apparent retrograde motion of the Sun occurs because, from approximately four Earth days before perihelion until approximately four Earth days after it, Mercury's angular orbital speed exceeds its angular rotational velocity. Mercury's elliptical orbit is farther from circular than that of any other planet in the Solar System, resulting in a substantially higher orbital speed near perihelion.
	\end{tcolorbox}
	\begin{figure}[H]
		\centering
		\includegraphics[scale=0.45]{img/cosmology/retrograde_motion_mars_saturn.jpg}
		\caption{Real sequence of exposures showing Mars and Saturn retrograde motion (author: Tunç Tezel)}
	\end{figure}
	
	\pagebreak
	\subsection{Lagrange Points}
	A "\NewTerm{Lagrange point}\index{Lagrange point}" (denoted by L), or "\NewTerm{libration point}\index{libration point}" is a position in space where the gravitational fields of two bodies in orbit around each other, and of substantial masses, combine to provide an equilibrium point to a third body of negligible mass, such that the relative positions of three bodies are fixed.

	We will in the developments that follow take time prove at best that such points are at the number of $5$ rated L1 to L5 respectively.

	It may be helpful to make a presentation of these points and their properties before going through the mathematical part. This may help in understanding the subject.

	We will immediately consider the following diagram:
	\begin{figure}[H]
		\begin{center}
		\includegraphics[scale=1]{img/cosmology/lagrange_points.jpg}
		\caption[]{Representation of the five Lagrangian point in the Sun-Earth system}
		\end{center}	
	\end{figure}
	There are five Lagrange points:
	\begin{enumerate}
		\item[L1:] On the line defined by the both masses between them (this is the most easy point to interpret intuitively: it is for example the point where the gravitational attraction of the Sun is compensated by that of the Earth).

		\begin{tcolorbox}[colframe=black,colback=white,sharp corners]
		\textbf{{\Large \ding{45}}Example:}\\\\
		We consider an object orbiting around the Sun, closer to the latter than the Earth but on the same line. This object undergoes a solar gravity greater than that of Earth, and therefore spins faster around the Sun than does the Earth. But Earth's gravity partially counteracts that of the Sun, which slows it down. The more we approaches this object of the Earth the more this counteract effect is important. At some point, the point L1, the angular speed of the object becomes exactly equal to that of the Earth.
		\end{tcolorbox}
		

		\item[L2:] On the line defined by the both masses, beyond the smaller (a bit less intuitive as is the point where the cumulative effect of the Sun and Earth will compensate the centrifugal force).

		\begin{tcolorbox}[colframe=black,colback=white,sharp corners]
		\textbf{{\Large \ding{45}}Example:}\\\\
		The principle is similar to the previous case, but on the other side of the Earth. The object should rotate more slowly than Earth because the solar gravity is lower, but the extra gravitational field due to the Earth tends to accelerate it. At some point, the point L2, the object rotates at exactly the same angular velocity as the Earth around the Sun.
		\end{tcolorbox}
		
		\item[L3:] On the line defined by the two masses, beyond the larger (intuitive based on physical considerations: it is clear that an object diametrically opposite to the Earth relatively to the Sun would have the same orbital period as the Earth and therefore would be fixed relative to the Earth-Sun system).

		\begin{tcolorbox}[colframe=black,colback=white,sharp corners]
		\textbf{{\Large \ding{45}}Example:}\\\\
		Identically to the L2 point, there exists a point a little further away than the Earth relatively the Sun, where a negligible mass object would be in equilibrium.
		\end{tcolorbox}
		

		\item[L4 \& L5:] On the apexes of two equilateral triangles whose base is formed by the two masses.
		
		\begin{tcolorbox}[colframe=black,colback=white,sharp corners]
		\textbf{{\Large \ding{45}}Example:}\\\\
		This is a subtle balance between the centripetal force exerted by the two main masses and the centrifugal force of the masses considered at the points of interest. L4 is ahead of the smaller mass in its orbit around the large one, and L5 is late. These two points are sometimes named "\NewTerm{triangular Lagrange points}\index{triangular Lagrange points}" or "\NewTerm{Trojans point}\index{Trojans point}".\\

	Remarkably, the last two points do not depend on the relative masses of the two bodies as we will prove it.
		\end{tcolorbox}

	\end{enumerate}
	For the first three Lagrangian points, stability appears only in the plane perpendicular to the line occupied by the two masses. For example, for the L1 point, if we move an object perpendicular to the line between the two masses, the two gravitational forces will play to bring it back to the starting position. The equilibrium is stable. However, if we move it near to one the two masses, then the field of that latter will prevail over the other and the object will tend to get closer. The equilibrium is unstable. For L4 and L5 points, stability is obtained due to Coriolis forces acting on the objects moving away from the point.
	
	Given the stability issues given above, we have no natural object around point L1, L2 and L3 at least in the solar system. However, they still represent an interest in scientific achievements because they allow savings of fuel for orbit control and attitude. This is not valid for point L3, due to its distance from Earth which only application what that utopic one made by Sci-Fi and comic books authors that place an Anti-Earth twin-planet but which mass was too high in relation to the theory stated above. However, space missions use L1 and L2: the case of the probe SOHO since 1995 (Solar and Heliospheric Observatory) a Sun observation station located at L1 (1.5 million kilometers from Earth) or WMAP (Wilkinson Microwave Anisotropy Probe) satellite or Planck satellite (to study the cosmic microwave background at $2.7$ [K]) close to the point L2 as will be the James Webb telescope in 2018 (the radiation of Earth there are relatively low and those of the Sun attenuated by the Earth which do a screening effect).
	
	The points L4 and L5 being stable, we find many natural celestial objects. In the Sun-Jupiter system, hundreds of asteroids, known as "Trojan asteroids", clump there together (around $1,800$ identified in April 2005). We count also  some in the Neptune-Sun systems and Mars-Sun system. Curiously, it seems that the Saturn-Sun system is not be able to accumulate such celestial objects because of the Jovian disruption. We also find objects to these points in the Saturn planetary system of Saturn: Saturn-Tethys with Telesto and Calypso with the L4 and L5 points and Saturn-Dione with Helen to the point L4 and Pollux at the L5 point. In the Sun-Earth system, there is no known large object to the Trojans points, but it was discovered a slight overabundance of dust in 1950. Slight dust clouds are also present for the system Earth-Moon; this make scientific abandon the idea to place there a space telescope as it was envisaged once.
	
	Strictly speaking, these 5 points exist only for two bodies in circular rotation one around the other. Once the orbit of the two bodies is elliptical, these points are no longer equilibrium points. In practice, if the orbit is slightly elliptical, as is the case for real planets, we can find stable orbits oscillating not departing too much of the regions corresponding to the Lagrangian points and this is well named a "\NewTerm{halo orbit}\index{halo orbit}". So halo orbit is a periodic, three-dimensional orbit near the L1, L2 or L3 Lagrange points in the three-body problem of orbital mechanics. Although the Lagrange point is just a point in empty space, its peculiar characteristic is that it can be orbited. The first mission to use a halo orbit was ISEE-3, launched in 1978. It traveled as we already mention to the Sun–Earth L1 point and remained there for several years. The next mission to use a halo orbit was in fact Solar and Heliospheric Observatory (SOHO), a joint ESA and NASA mission to study the Sun, which arrived at Sun–Earth L1 in 1996. It used an orbit similar to ISEE-3.
	
	So we will consider in space an isolated system of two bodies $A$ and $B$, of mass $M_A$ and $M_B$ in gravitational interaction. These two bodies are assumed to be in circular orbit (for simplicity!) one around the other, in the manner of a two-star system (binary system) or a planet-satellite like system (Saturn-Titan example) . We seek to determine if there are relative equilibrium point to the system of the two rotating body for  a third body also in circular motion in the same plane (of sufficiently low mass to avoid disturbing the motion of the system of the two main bodies).
	\begin{figure}[H]
		\begin{center}
		\includegraphics[scale=1]{img/cosmology/lagrange_points_configuraton_study.jpg}
		\end{center}	
	\end{figure}
	Let O be the centroid (\SeeChapter{see section Classical Mechanics}) of these two stars (or celestial objects in general). Let us consider a Galilean reference frame (in rectilinear and uniform motion therefore !) or origin O. Compared to this reference frame, we assume that the axis $AB$ rotates at a constant angular velocity $\omega$ of fixed axis $\vec{k}$ (perpendicular to the page in the figure and directed towards the reader) and that the distances $r_A=\overline{A\text{O}}$ and $r_B=\overline{B\text{O}}$ also remain constant.
 
 We know from our study of Classical Mechanics that a circular motion the centrifugal force is given by:
	
	So we have (equilibrium between centrifugal and centripetal forces) to guarantee the equilibrium:
	
	By simplifying and summing these two relations:
	By simplifying and summing these two relations:
	
	with in what will follow $AB=r_A+r_B=R$.
	
	Let us onsider a rotating reference frame $R'$ linked to our stars as shown in figure above: $\vec{i}$ will be a collinear unit vector to $AB$, $\vec{j}$ a unit vector perpendicular to $\vec{i}$ and in the rotating plane of the planets and finally $\vec{k}=\vec{i}\times\vec{j}$ co-linear to $\vec{omega}$.

	We consider in this rotating frame (with stars) a third star $S$ of mass $m$ negligible relatively ot $M_A$ and $M_B$, subject to the gravitational attraction of $A$ and $B$.

	Now let us denote by $\vec{a}_{R'}$ the acceleration of $S$ with respect to $R'$, $\vec{v}_{R'}$ its speed and $\vec{e}_r$ the collinear  unit vector to $\overrightarrow{\text{O}S'}$ where $S'$ is the projection of $S$ in the plane O$xy$, and $r=\overline{\text{O}S'}$ (in the figure above, we assumed $S$ in the plane O$xy$, so $S$ and $S'$ are indistinguishable).
	
	$S$ is thus subjected to two forces, one $\vec{F}_A$ directed along $A$ and the other $\vec{F}_B$ directed along $B$, forces of respective intensities :
	
	In a Galilean reference frame, these two forces apply to $S$ an acceleration given by the law of composition of accelerations in a circular reference frame (\SeeChapter{see section of Classical Mechanics}):
	
	But, in our configuration the pulsation (radial velocity) is assumed constant and the drive acceleration is zero since we assumed $R'$ as the main repository. Therefore it comes:
	
	We also have:
	
	where according to the figure all components are positive. The calculation of the cross product then gives (\SeeChapter{see section Vector Calculus}):
	
	So finally:
	
	Let us rather write this relationship into the form:
	
	We then obtain, by projecting on the three axes $x$, $y$ and $z$, the derivatives taken with respect to time $t$ the following system:
	
	with:
	
	so that the coordinates $(x,y,z)$ of the point $S$ are those of an equilibrium point, then it is trivial that in the rotating frame with the stars $A$ and $B$ that:
	
	We then get the following system:
	
	It is also immediately that the third equation has for only solution $z=0$ and thus ultimately the system reduces to:
	
	The third equation simply means that the equilibrium positions are in the plane O$xy$ (we could suspect it a bit ...). The other two, we will see it later, lead us to consider five solutions that are our five Lagrangian points L1, ..., L5.

	If we draw plot with an appropriate software the acceleration (respectively force) with the isoclines highlighted (curves on which the acceleration is equal) we get:
	\begin{figure}[H]
		\begin{center}
		\includegraphics[scale=1]{img/cosmology/lagrange_points_two_bodies_isoclines_plot_3d.jpg}
		\end{center}	
		\caption{Isoclines of the two-body system}
	\end{figure}
	where we see that a short distance of the bodies the gravitational potential energy dominates, but that a large distances the centrifugal potential predominates and the shape of the surface is similar to that of a paraboloid.

	By requesting the software to plot only the isoclines projected on a plane we get:
	\begin{figure}[H]
		\begin{center}
		\includegraphics[scale=1]{img/cosmology/lagrange_points_two_bodies_isoclines_plot_2d.jpg}
		\end{center}	
		\caption{Projected isoclines of the two body system on a plane}
	\end{figure}
	where we have highlighted the five Lagrange points and where the stars (or celestial objects) are represented by blue dots and the centroid of the system by a green dot. It seems that isoclines are named in the astronomy field "\NewTerm{Roche equipotentials lobes}\index{Roche equipotentials lobes}". Otherwise seen:
	\begin{figure}[H]
		\begin{center}
		\includegraphics[scale=1]{img/cosmology/lagrange_points_two_bodies_isoclines_plot_2d.jpg}
		\end{center}	
		\caption{Projected isoclines of the two body system on a plane}
	\end{figure}
	where we have highlighted the five Lagrange points and where the stars (or celestial objects) are represented by blue dots and the centroid of the system by a green dot. It seems that isoclines are named in the astronomy field "\NewTerm{Roche equipotentials lobes}". Otherwise seen:
	\begin{figure}[H]
		\begin{center}
		\includegraphics[scale=1]{img/cosmology/lagrange_points_two_bodies_isoclines_plot_2d_and_3d.jpg}
		\end{center}	
		\caption{Projected isoclines of the two body system on a plane}
	\end{figure}
	For those wishing to reproduce these figures with MATLAB here is how first proceed mathematically. From what we got previously, we have explicitly and in writing in a more academic form, the following relation:
	
	As the point $S$ is supposedly in equilibrium the last term vanishes (its speeds are zero in the rotating frame!). It then remains:
	
	We have proved for recall in the section of Classical Mechanics that:
	
	Then we have:
	
	Thus taken by unit mass of the satellite:
	
	The application of this relations in MATLAB 2013a then gives (sorry it's a bit long and its probably possible to do better ...):
	\begin{lstlisting}[language=MATLAB]
		%We build the grid plot that we will by anticipation densifiate where are the objects of interest
		x1=linspace(-7E8,-8E5,150);
		x2=linspace(8E5,1.2E8,150);
		x3=linspace(1.6E8,7E8,150);
		x=x1+x2+x3;
		y=linspace(-7E8,7E8,450);
		[X,Y]=meshgrid(x,y);
		%These masses and G are real but the rest is fictitious so that the plot is readable 
		f=@(x,y) -(1.3346E20)./(sqrt((x-450).^2+y.^2))-(1.0038E19)./(sqrt((x-449999550).^2+y.^2))-(6.9E-7.*(x.^2+y.^2));
		z=f(X,Y);
		%We eliminate the values that are too big on Z to have an esthetical result to see
		for i=1:450;
		   for j=1:150;
		      if (z(i,j)<-0.8E12) %to do with meshc a nice plot, limit to  -8E11
		         z(i,j)=-0.8E12;
		      end; 
		   end; 
		end; 
		contour(X,Y,z,100); mesh(X,Y,z); meshc(X,Y,z); 
		az = 100; el = 25; view(az, el);
		axis([-7E8 7E8 -0.8E9 0.8E9 -8E11 -4E11]);
		colorbar; light; camlight('right');
	\end{lstlisting}
	That gives:
	\begin{figure}[H]
		\begin{center}
		\includegraphics[scale=0.8]{img/cosmology/lagrange_points_two_bodies_3d_matlab.jpg}
		\end{center}	
		\caption{Lagrange plot and isoclines with MATLAB 2013a}
	\end{figure}
	The reader will notice that it is difficult to intuitively this configuration of the potential. In the rotating frame with the centroid of the two solid bodies, the potential resulting from the combination of rotational and gravitational potentials present 3 extrema L1, L2 and L3 on the right containing the both bodies. One of these maxima is between the two bodies as expected intuitively. The other two maxima are on the line connecting the two objects, but both pn either side... which is more surprising. They come from the contribution to the potential of the rotating frame which can be difficult to model intuitively.
	
	\subsubsection{Equilibrium points of the first type}
	What we mean by equilibrium positions of the first type are simply solutions located on the line $\overline{AB}$ such that $y=0$ which is equivalent to study only:
	
	with therefore:
	
	To this situation, we will consider three possible corresponding sub-cases respectively L1, L2 and L3 as we will immediately see it.
	
	\paragraph{L1 Lagrange point}\mbox{}\\\\\
	In this first sub-case, we consider:
	
	What is also equivalent to have:
	
	This allows us to write:
	
	in the following simplified form:
	
	Now to be able to say something about the possible solutions to this equation derive the left hand side. We then get:
	
	This term is strictly increasing from $-\infty$ to $+\infty$ when $x$ describes $]-r_A,r_B[$. So there is a unique solution and equilibrium point denoted L1 (first Lagrange point) between $A$ and $B$.

	If we typically consider the Sun-Earth case where $M_A>M_B$ and therefore $r_A<r_B$ then on $x=0$ we have:
	
	what is immediately negative. The equilibrium position will be obtained for a positive value of $x$ we will have to determine.

	This value can be obtained by considering a limit case: when $M_B$ tends to $0$ (corresponding to a massive celestial object $A$ turning around a mass $B$ of a much much smaller celestial object), then $A$ tends to O, $r_A$ tends to $0$ and therefore:
	
	with $R=\overline{AB}$. Therefore, in this limit case:
	
	becomes in approximation:
	
	and therefore:
	
	So the only value of $x$ satisfying this relation will be $x=R$.

	In other words, the equilibrium point L1 we are looking after here is between $A$ and $B$ moves near $B$, that is near the less massive celestial object (which corresponds well to the first figure that we used to show the location of the five Lagrange points).

	By this observation we can make the following calculations: 
	\begin{figure}[H]
		\begin{center}
		\includegraphics[scale=1]{img/cosmology/l1_point_configuration_study.jpg}
		\end{center}	
		\caption[]{Configuration to mathematically determine the position of point L1}
	\end{figure}
	We have from the definition of center of gravity (\SeeChapter{see section Classical Mechanics}):
	
	As our study is done relatively to the centroid we have $\vec{r}_G=\vec{0}$ and therefore:
	
	From the above relation by taking the norm, we have obviously:
	
	The distance between the two celestial objects $A$ and $B$ remaining constant and being equal to $r=r_A+r_B$ we write:
	
	We deduce trivially two relations (the second being obtained by exactly the same reasoning as the first):
	
	But since $M_A \gg M_B$ we can write roughly the first relation in the following approximate form (Taylor series):
	
	and since:
	
	we have also:
	
	So with $M_A\gg M_B$:
	
	According to the limiting case studied previously, we can assume $L$ at the neighborhood of $B$ such that it is possible to write:
	
	with $\varepsilon\ll 1$.
	
	Either using:
	
	Then we have:
	
	by neglecting the infinitely small therm of order $2$.

	Hence:
	
	Now in the mentioned  configuration the equilibrium is given by:
	
	Therefore:
	

	Now the third Kepler's law  gives us:
	
	Therefore:
	
	After simplification:
	
	Therefore:
	
	Hence:
	
	Since $1/\varepsilon^2$ is much greater than $1$ and assuming that $3M_A/M_B$ then we have also:
	
	Thus finally:
	
	and therefore:
	
	If we take the $A$ for the Sun and $B$ for the Earth, then:
	
	We find that the distance $\overline{LB}$ is then equal approximately to:
	
	which is the L1 point where was placed the satellite SOHO (since the latter will thus never  have its field of view obscured by the shadow of the Earth or the Moon).

	A special case of the L1 point to consider is when $M_A=M_B=M$, then $r_A=r_B=r$, then O is the midpoint of $\overline{AB}$. Then we have:
	
	Therefore:
	
	becomes:
	
	
	\paragraph{L2 Lagrange point}\mbox{}\\\\\
	In this second sub-case, we consider:
	
	Therefore we are looking for the equlibrium points beyond $B$.

	Thus we have:
	
	which becomes simply:
	
	The left side is a strictly increasing function of $x$ from $-\infty$ to $+\infty$ when $x$ describes $[r_B,+\infty]$. So there is a unique solution, and an equilibrium point beyond $B$. This point is denoted: L2.

	This value can be obtained by considering a limit case: when $M_B$ tends to $0$ (corresponding to a massive celestial object on $A$ around which a much smaller mass object $B$ turn around), then $A$ tends to O, $r_A$ to $0$ and therefore:
	
	with $R=\overline{AB}$. Therefore, in this limit case:
	
	becomes approximately:
	
	and so:
	
	So the only value of $x$ satisfying this relation will be $x=R$. The L2 point therefore ends up being being merged with $B$.

	Knowing this limit case, let us do a more detailed study. Consider the following diagram relatively to our previous limit case:
	\begin{figure}[H]
		\begin{center}
		\includegraphics[scale=1]{img/cosmology/l2_point_configuration_study.jpg}
		\end{center}	
		\caption[]{Configuration to mathematically determine the position of point L2}
	\end{figure}
	and let us consider $M_A\gg M_B$ without forgetting that in this scenario $x>r_B$.

	We then have almost the same developments as for L1 but with the difference that:
	
	becomes:
	
	and that instead of having:
	
	We have:
	
	and therefore:
	
	Still with:
	
	and therefore:
	
	which corresponds to the Lagrange point L2.

	A special case again about L2 is when $M_A=M_B=M$, then $r_A=r_B=r$, then O is at the midpoint of $\overline{AB}$. Then we have:
	
	Therefore:
	
	becomes:
	
	It is no longer possible to extract the roots here (at least to my knowledge). It must be done through a numerical approximation. In Maple 4.00b, we simply put:

	\texttt{>solve(-1/(r+x)\string^2-1/(x-r)\string^2=x/(8*r\string^3),x);allvalues(");}
	
	and the only feasible solution in $\mathbb{R}$ is then $x\cong 2.8r$ and the others being in $\mathbb{C}$.
	
	\pagebreak
	\paragraph{L3 Lagrange point}\mbox{}\\\\\
	In this third sub-case, we consider:
	
	So we look for the equilibrium points beyond $A$.

	Thus we have:
	
	which becomes simply:
	
	The left side is an increasing function of $x$ from $-\infty$ to $+\infty$ when $x$ describes $]-r_A,-\infty]$. So there is a unique solution, and one equilibrium point beyond $A$. This point is denoted: L3.

	This value can be obtained by considering a limit case: when $M_B$ tends to $0$ (corresponding to a massive celestial object $A$ turning around much smaller object $B$), then $A$ tends to O, $r_A$ to 0 and therefore:
	
	with $R=\overline{AB}$. Therefore, in this limit case:
	
	becomes approximately:
	
	and therefore:
	
	So the only value of $x$ satisfying this relation will be $x=R$. The point L3 will finish to merge with the position diametrically opposite to that of $B$.

	Knowing this limit case, let us do a more detailed study now. Consider the following diagram relative to our previous limit situation:
	\begin{figure}[H]
		\begin{center}
		\includegraphics[scale=1]{img/cosmology/l3_point_configuration_study.jpg}
		\end{center}	
		\caption[]{Configuration to mathematically determine the position of point L3}
	\end{figure}
	and let us still consider $M_A \gg M_B$ without forgetting that in this scenario $x<-r_A$.

	We will first consider the following approximation:
	
	and this one also (since $\overline{\text{O}A}$ tends to zero as the celestial object $A$ becomes very massive):
	
	Since then:
	
	We have also (...):
	
	when at the limit where the celestial object $A$ is really massive, we fall back on the first term ...

	With the last two relations, we have:
	
	if we neglect the terms of the second order.

	Furthermore, we have also:
	
	Let us recall the equilibrium condition:
	
	And let us put everything we got until now inside it:
	
	What becomes after simplifications:
	
	after a small approximation:
	
	after simplification:
	
	Hence:
	
	and finally:
	
	\begin{tcolorbox}[title=Remark,colframe=black,arc=10pt]
	Form some Si-Fi authors, for recall... this point L3 opposite to the Earth relatively to the Sun would hide us a hypothetical planet that we would be forever hidden to us by the Sun.
	\end{tcolorbox}
	
	\subsubsection{Equilibrium points of the second type}
	The equilibrium positions of the second type are those for which $y\neq 0$. In other words the points outside of the line $\overline{AB}$, but still in the plane $\text{O}xy$.

	Thus, our system of equations remains:
	
	
	\paragraph{L4, L5 Lagrange points}\mbox{}\\\\\
	To determine the remaining equilibrium points, we can divide the second equation of the system such that the system becomes:
	
	\begin{figure}[H]
		\begin{center}
		\includegraphics[scale=1]{img/cosmology/l4_l5_point_configuration_study.jpg}
		\end{center}	
		\caption[]{Configuration to mathematically determine the position of points L4,L5}
	\end{figure}
	where $\overline{AB}$ is obviously the distance between $A$ and $B$ and $D$ is the centroid of the system given by (\SeeChapter{see section Classical Mechanics}):
	
	which are the radii of gyration of the bodies $A$ and $B$.

	It is easy to verify that the sum of both previous distances is equal to $\overline{AB}$ and their proportion $M_B/M_A$. Another form of $\overline{DB}$ (which will be useful) is obtained by dividing the numerator and denominator by $M_A$:
	
	We know according to our previous calculations that $\overline{AS}=\overline{BS}$ but this is insufficient. We still want to know the angles of the vertices $A$, $B$, $S$, and this is what we will look for now.

	In this context, if a satellite $S$ is in equilibrium, there will always remain at the same distance of $A$ or $B$. The center of rotation of the $3$ points is the point $D$, the mass $A$ itself revolves around it. If the satellite, $S$, remains stable, the three bodies have the same orbital period $T$. If $S$ is immobile in this frame in rotation it will not be subject to the Coriolis force but only to centrifugal force of $A$ and of $B$.
	
	Let us denote by $v_B$ the roation speed of $B$ and $v_S$ the rotation speed of $S$. Then we have:
	
	and:
	
	From whose we get that:
	
	and:
	
	So we can equate these two expressions:
	
	This merely expresses the well known fact that if two objects rotate together, the furthest one from the centroid is the fastest. Speeds are proportional to the distances from the centroid.

	The centrifugal force on $B$ is in equilibrium with the gravitational force of $A$ and it is expressed by:
	
	Thus by simplifying:
	
	Similarly, the centrifugal force applied on $S$ is:
	
	It is balanced by the forces of attraction $\vec{F}_A$,$\vec{F}_B$ of the objects $A$ and $B$. However, only the components of these forces located on the line $R$ oppose efficiently to this centrifugal force. Hence:
	
	
	and as:
	
	We then have:
	
	In addition, the forces applied to $S$ and perpendicular to $R$ must vanish. If not, the object $S$ would follow the largest mass and would not remain in position and would therefore no longer be in equilibrium. We must then have:
	
	Or, after substitution and simplification:
	
	Of all the equations obtained up to now the only that bother us are those containing both speeds and angles $\alpha$,$\beta$. This requires that we must arrive to eliminate what is convenient to have only the last two parameters (that is to say: the angles).

	For this, we take the square:
	
	We multiply both sides by $\overline{AB}^2$ and we divide by $1+\dfrac{M_B}{M_A}$:
	
	which is similar to:
	
	Thus equaling:
	
	So we removed the speed of $B$. Now, let us multiply both sides by $\left(1+\dfrac{M_B}{M_A}\right)R$ and divide by $\overline{AB}^2$:
	
	which is similar to:
	
	Therefore:
	
	By dividing by the whole by $GM_A$ we find:
	
	And as we have proved at the beginning $\overline{AS}=\overline{BS}$ that we will denote by $R'$, then we have:
	
	and let us recall that we have:
	
	Therefore:
	
	This allows us to write:
	
	And multiplying by $\sin(\beta)$:
	We can now notice a thing (not easy to see...). If $R'=\overline{AB}$ (that is that the triangle $ABS $is equilateral) the previous relations simplifies to:
	
	But, if the triangle is really equilateral, then we have (\SeeChapter{see section Trigonometry}):
	
	Hence:
	
	What can finally write:
	
	Which is just the sine theorem for the triangle $SDB$ (\SeeChapter{see section Trigonometry}) and is therefore certain. Returning back, we can now prove that all previous equations are satisfied if and only if $ABS$ is equilateral. If we had not put $ABS$ as equilateral, we would have gotten a different relation of the sine theorem, without possible verification, and the set of equations required for equilibrium at the point $S$ could not be met.

	Conclusion of the thing ... the system gives as a solution:
	
	$ABS$ (or $ABL$ regardless the notation), then forms an equilateral triangle. The two equilibrium points are denoted L4 and L5. L4 is located in advance with respect to the less massive celestial object and and L5 is late relatively to it.
	\begin{figure}[H]
		\begin{center}
		\includegraphics[scale=1]{img/cosmology/l4_l5_final_point_configuration_study.jpg}
		\end{center}	
		\caption[]{L4 and L5 equilateral triangle}
	\end{figure}
	In 2000, $385$ asteroids in the L4 point and $188$ asteroids in the L5 point were counted on the orbit of Jupiter, but located precisely in an equilateral triangle with the Sun and Jupiter either side of Jupiter: these are the Trojan planets. It was also observed two objects at the point $L5$ of Mars discovered in 1990 and 1998.
	
	\pagebreak
	\subsection{Relativistic Doppler-Fizeau Effect}
	The Doppler effect is the difference between the frequency of the transmitted wave and the received wave when the transmitter and receiver are moving relative to each other (\SeeChapter{see section Music Mathematics}). This is an effect that must be take into account astronomy to calculate the distance of the body assuming its known (or estimated)  emission wavelength and measuring its received wavelength or for measuring the speed of rotation (radial velocity) of stars by observing very precisely and successively their opposite edges and measuring the shift of the spectrum obtained.

	In the early 21st century the precision and finesse of the spectra of measures has reached a level that allows to observe even minimal changes in the distance of stars and so speculate on possible planetary satellites (this may work if the plane of the orbit passes through the Earth):
	\begin{figure}[H]
		\begin{center}
		\includegraphics[scale=0.8]{img/cosmology/doppler_effect_radial_velocity.jpg}
		\end{center}	
		\caption{Doppler-Effect radial velocity method (source: ESO Press Photo 22e/05 2007-04-11)}
	\end{figure}
	The Doppler effect of electromagnetic waves must be discussed independently of the acoustic Doppler effect (also named "Galilean Doppler effect") study in the section of Music Mathematics. First, because the electromagnetic waves do not consist of a material movement and therefore the speed of the source relative to the medium does not enter into the discussion, then because their velocity is $c$ (the speed of light) and remains the same for all observers independently of their relative movements. The Doppler effect for electromagnetic waves is thus calculated necessarily using the principle of relativity and is symmetrical with respect to relative movement of the source and the observer (as opposed to acoustic cases).
	
	For an observer in an inertial reference frame, a plane and harmonic electromagnetic wave can be described by a function of the form:
	
	multiplied by an appropriate amplitude factor. For an observer attached to another inertial frame, the components $x$ and $t$ should be replaced with $x'$ and $t'$, obtained by the Lorentz transformation (\SeeChapter{see section Special Relativity}), and that latter will therefore write to describing plane wave:
	
	where $k'$ and $\omega'$ are not necessarily the same as that of the another observer (precisely this is what we want to determine). Moreover, the principle of relativity has allowed us to demonstrate in the section of Special Relativity that:
	
	This assumes that the expression:
	
	remains invariant when we move from one inertial observer to the other. We then have:
	
	Using the Lorentz transformation relations (\SeeChapter{see section Special Relativity}), we have immediately:
	
	By identification, it comes immediately:
	
	If we consider that:
	
	in the case of electromagnetic waves, we can write each of these relation in the form:
	
	The ratio visible in the both expression above is named the "\NewTerm{red shift}\index{red shift}" and is denoted by:
	
	for a movement of the observer relative to the source in the direction of propagation.
	\begin{figure}[H]
		\centering
		\includegraphics[scale=0.19]{img/cosmology/hubble_deep_space_redshift.jpg}
		\caption{High-redshift galaxy candidates in the Hubble Ultra Deep Field 2012 (source: NASA, ESA, R. Ellis (Caltech), and the HUDF Team)}
	\end{figure}
	Obviously if the source or observer don't move away but approach then we not have a red shift but a "\NewTerm{blue shift}\index{blue shift}".
	
	Furthermore, the last relation with the pulsations is most often written in the literature as follows:
	
	Which is written most often in the following form:
	
	Therefore if we measure the both frequencies (supposing that we know what should be the source), then we can also obviously determine the speed $v$ of the observed object.
	
	When a spectrum can be obtained, determining the red shift is rather straight-forward: if you can localize the spectral fingerprint of a common element, such as hydrogen, then the red shift can be computed using simple arithmetic. But similarly to the case of Star/Quasar classification, the task becomes much more difficult when only photometric observations are available:
	\begin{figure}[H]
		\centering
		\includegraphics[scale=0.8]{img/cosmology/red_shift_spectrum.jpg}
	\end{figure}
	It must be recalled that the pulsation offset (and therefore frequency offset) that takes place here is due to a relative motion of the observer with respect to the source and not to something else (or respectively of the source relatively to the observer). Indeed, in our study of General Relativity (\SeeChapter{see section General Relativity}), we will prove that there is a superposition of a shift because of the gravitational field surrounding the emitter that will be considered as caused by the spacetime curvature .

	Finally, for skeptics who want to check in another way that the Doppler phenomenon is well symmetric unlike the acoustic Doppler effect proved in the section of Music Mathematics, here's another approach:

	First, consider that it is the source moving away. If we calculated by the classical relation proved in the section of Music Mathematics the frequency of the signal at the reception would be:
	
	and we must take into account the time dilation for $f$ with (\SeeChapter{see section Special Relativity}):
	
	because the time interval of the fixed observer is longer than that of the source (time is faster for observer at rest).

	It comes then:
	
	and if it is the observer who moves away from the source we provec in the section of Music Mathematics that:
	
	both relationships are indeed symmetric in the special relativistic case (as expected for electrodynamics)!
	\begin{tcolorbox}[title=Remark,colframe=black,arc=10pt]
	Currently, astronomers have courses in astrophysics and their observations are generally studied in an astrophysical context, so there is less distinction between the two disciplines than before.
	\end{tcolorbox}
	A very good example of the application of the Doppler effect is to explore the limits given by measuring the apparent speed. Let's see what it is:
	
	\subsubsection{Apparent speed}
	By measuring the apparent speed of movement of very fast objects in the sky (plasma jets, etc.), astrophysicists have obtained apparent displacement speeds exceeding the speed of light in vacuum!

	In fact, it is an illusion that can occur if the speed of the object is very close to that of light it emits, so close enough to $c$.
	\begin{figure}[H]
		\centering
		\includegraphics[scale=1]{img/cosmology/apparent_speed.jpg}
		\caption{Schematic idea behing the apparent speed}
	\end{figure}
	The object emits light at time $t_0$, it does not instantly reach us but must travel a distance to get to us. We get it after the time:
	
	The object itself, moves with velocity $v$ at an angle $\theta$ with the viewing direction, so at time $t$, the object moved of a distance $vt$. The light emitted by the object at time $t$ must travel the distance (application of Pythagoras thereom)
	
	to reach us (the object has move of a distance $vt\cos(\theta)$ in the direction of observation but moved away from the axis of observation of the distance $vt\sin(\theta)$), so we receive light that was emitted by the object at time $t$ after a time $t_2$:
	
	Between the two positions of the object, it has elapsed the time $t$ but, viewed from the observer, the time interval between the reception of images of these two positions is:
	
	different from $t$!
	
	For a small time interval $t$, we have, by doing a limited Taylor development:
	
	During this time interval, always from the point of view of the observer, the object appears to have moved on the sky plane by a distance of $vt\sin(\theta)$.

	Thus, the apparent speed of the object is:
	
	If we set the angle $\theta$ as being very close to a right angle, then we have the second term of the denominator that is very small which allows us with a Taylor expansion to write a relation that can be found quit often in high-school textbooks :
	
	Let us seek the maximum of this function to understand how such observation is possible by deriving relatively to $\theta$ and by seeking for what value the derivative is zero:
	
	and this vanishes after simplification of the denominator for:
	
	Hence:
	
	The apparent velocity is then:
	
	and is equal to or greater than $c$ if:
	
	Therefore:
	
	Thus we see that it is possible to observe apparent movements faster than light, even though the subject is very fast indeed, but slower than $c$. As it is only an "illusion", there is no contradiction with the theory of relativity.

	Knowing the speed of movement of a celestial object obtained using the Doppler effect and the apparent speed with the observations, it is easy for astrophysicists to determine the angle  $\theta$ by doing a little bit elementary algebra from the following relation:
	
	
	\begin{flushright}
	\begin{tabular}{l c}
	\circled{80} & \pbox{20cm}{\score{3}{5} \\ {\tiny 47 votes,  64.68\%}} 
	\end{tabular} 
	\end{flushright}
	
	%to make section start on odd page
	\newpage
	\thispagestyle{empty}
	\mbox{}
	\section{Astrophysics}
	\lettrine[lines=4]{\color{BrickRed}A}strophysics is an interdisciplinary branch of astronomy which mainly concernes physics and the study of the properties of objects in the Universe (stars, planets, galaxies, interstellar medium for examples) as their luminosity, density, temperature and their chemical composition. The first scientific approaches in this area date from the early 19th century.
	
	\begin{tcolorbox}[title=Remark,colframe=black,arc=10pt]
	Currently, astronomers have courses in astrophysics and their observations are generally studied in an astrophysical context, so there is less distinction between the two disciplines than before.
	\end{tcolorbox}
	
	\subsection{Stars}
	Before addressing the mathematical formalism on the dynamics of stars, we wanted following readers requests, write a small popularized introduction to complete the general knowledge on this field.
	
	The stars are gaseous celestial body whose mass goes from $0.05$ solar mass to more than $100$ solar masses. The brightness of a star (its power radiation) ranges from $10^{-6}$ to $10^6$ times that of the Sun. Roughly, when the mass doubles, brightness is multiplied by. Most of the stars visible to the naked eye in our skies are blue giants of $10^4$-$10^5$ times more luminous than the Sun; they represent only $10\%$ of stars that inhabit our galaxy, the remaining $90\%$ being less luminous than the Sun.
	
	The Astronomers (of Harvard between 1918-1928) have developed a method of classification of stars based on their position in the spectrum, of the spectral absorption lines (spectroscopy). Formerly classified from A to Q, the evoluton of the spectrometry allowed their grouping and organization. Classes are now defined by the letters OBAFGKM, and each is divided into $10$ subclasses, rated from $0$ to $9$. The spectral classification (taken from a continuous spectrum which summarizes only certain lines of the spectrum after passing of light in a given medium) can be crossed with the lighting classes so that we can infer the temperature at the surface of the star (we will prove later how to get this information):
	
	\begin{figure}[H]
		\begin{center}
		\includegraphics[scale=0.9]{img/cosmology/hertzprung_russel_diagram.jpg}
		\end{center}	
		\caption{Hertzsprung-Russel Diagram example (source: Wikipedia)}
	\end{figure}
	And corresponding hypothesized path evolution of our Sun on this same diagram:
	\begin{figure}[H]
		\begin{center}
		\includegraphics[scale=0.6]{img/cosmology/hertzprung_russel_diagram_sunpath.jpg}
		\end{center}	
		\caption{Sun path on Hertzsprung-Russel}
	\end{figure}
	As it evolves, each star describes a particular curve on the HR diagram: it begins by following the "\NewTerm{Hayashi-path}\index{Hayashi-path}" (proto-star and after one of the existing tar) until it reaches the main sequence in which it operates as its core burns hydrogen. When beginning the burning of helium, it goes back up where red giants are concentrated and remains there until nuclear fusion stops; it then collapses on itself to join the white dwarfs or in the case of a certain value of solar masses, neutron stars, black holes, or if its mass is very high, exploding as supernovae.
	
	The O stars were discovered in the late 19th century. They are hot and their spectra look like nebulae. The B are helium stars, A hydrogen stars. The predominant component of F is calcium. G are of the same type as the Sun and K differ very little bit. M are characterized by titanium oxide and S of zirconium oxide, while R and N contain hydrocarbons and cyanogen.
	
	Therefore a star of the mass of the Sun after a stint on the main sequence, becomes a red giant, eventually a planetary nebula (ejection of fuel of the star at long distances), before ending his life as a white dwarf. The end as supernovae cannot be shown in this diagram because of their Luminosity that is to high. Neutron star and Black holes are in the same path than white dwarfs (lower on the right than Procyon B).
	
	A star is initially in hydrostatic equilibrium. Gravitational forces due to its mass are compensated by the internal pressure forces due to the elevated temperature maintained by thermonuclear reactions at low density and to the degeneracy pressure of electrons: 
	\begin{figure}[H]
		\begin{center}
		\includegraphics[scale=0.9]{img/cosmology/star_pressure.jpg}
		\end{center}	
	\end{figure}	
	A star spends almost $90\%$ of his life to fuse hydrogen into helium that builds up in the center. During this phase, it evolves into the "main sequence" of the Hertzsprung-Russian diagram.
	
	For a low mass Main Sequence star, hydrogen fusion is the first energy source that provides radiation pressure to maintain the hydrostatic equilibrium. When hydrogen fusion ends, the star begins to undergo structural changes, and it begins to become a red giant (helium fusion) through what we name a "\NewTerm{helium flash}\index{helium flash}". At the end of the life of a low mass star, the core collapses until the electrons provide a source of pressure to withstand the collapse, and at this stage the star is a white dwarf. For higher mass stars, the early stages of life are the same, but the core of these stars reach higher temperatures, so they can burn more massive species, like Carbon, Oxygen, Neon, Magnesium, and Silicon. Towards the end of its life, a high mass star's core will look like the layers of an onion.
	\begin{figure}[H]
		\begin{center}
		\includegraphics[scale=0.6]{img/cosmology/star_structure.jpg}
		\end{center}	
	\end{figure}
	The core of a high mass star will eventually create iron, but when the core tries to fuse iron, it will die in a catastrophic explosion. The problem is that unlike Hydrogen, Helium, Carbon, etc., the fusion of iron does not release energy (\SeeChapter{see section Nuclear Physics}). When the core contains enough iron, the star implodes in seconds, and all of the mass of the outer part of the star hits the core and rebounds, and the rebound sends a shockwave outward pushing all of the material outside of the core into space in a tremendous explosion, named a "\NewTerm{supernova}\index{supernova}":
	\begin{figure}[H]
		\begin{center}
		\includegraphics{img/cosmology/supernova.jpg}
		\caption{Region of the sky before and after the 1987 supernova in visible light}
		\end{center}	
	\end{figure}
	When the helium mass of a star becomes sufficient, the increase in pressure causes an increase of the temperature thereby initiating the fusion of helium ("\NewTerm{helium flash}\index{helium flash}") into carbon, oxygen and neon creating a second combustion front inside the first. For a star of one solar mass, reactions stop at this stage. The star radius increas and its surface temperature decrease until stabilization. It becomes a red giant $10^4$ times more luminous than before. It goes through various phases of instability and eventually gradually expel its outer layers, forming a "\NewTerm{planetary nebula}\index{planetary nebula}" (from a fraction of parsecs like in size like our Solar System to a little bit more of $1$ parsec - approximately $4$ light years - for the biggest known at this date). 
	\begin{figure}[H]
		\begin{center}
		\includegraphics[scale=0.09]{img/cosmology/planetary_nebula.jpg}
		\end{center}	
		\caption{Planetary Nebula gallery (source: Hubble Space Telescope)}
	\end{figure}
	Its core, with a density of several tons per cubic centimeter, cools down slowly: it become a "\NewTerm{white dwarf}\index{white dwarf}" (we will discuss this process mathematically below). The balance in its core is maintained by the pressure of degeneration of electrons.
	
	For a more massive star, the internal temperature becomes quite important so that the carbon and oxygen can fusion into silicon. In turn, if there is enough mass, silicon will fusion into iron. Combustion fronts develop in a pattern said of "onion skins" (see prior previous figure). As iron is the most stable nucleotide,  it is at the bottom of the valley of stability (\SeeChapter{see section Nuclear Physics}). It can not fusion or split. When the density reaches a critical value (this corresponds to a total mass of the star of more than $8$ solar masses!!!), electron degeneracy pressure can no longer maintain the balance against gravity. In a tenth of a second, the iron core collapses. The other layers of the heart of the star rush towards the collapsed core in the for a wave whose maximum speed corresponds to the sonic radius.
	
	The core density then becomes really huge. There occur inverse $\beta^-$ reactions where protons capture electrons forming neutrons (!!!) and releasing a flow of neutrinos. When the core of the star reaches the nuclear density of approximately $10^{18}\;[\text{kg}\cdot\text{m}^{-3}]$, the compaction stops roughly (the remaining radius at this stage is about $10$ [km] only!). The outer layers of the core bounce by a super elastic shock and come into expansion. When this reflected shock wave reached the sonic radius, the temperature rises so high that give him a value is almost meaningless. The material undergoes a complete photodisintegration (all nucleotides are disaggregated into nucleons gas). Finally by an unclear mechanism, all the outer layers of the star are ejected into space: it is a "\NewTerm{type II supernovae}\index{type II supernovae}".
	
	The collapsed core, made almost entirely of neutrons, will be rotating rapidly if the original star had a nonzero angular momentum (conservation of angular momentum oblige!). The magnetic field is also preserved and far exceeds anything that will probably never be feasible a laboratory. This causes a synchrotron beam which gives the illusion that the star flashes. This is why these young "\NewTerm{neutron stars}\index{neutron stars}" are named "\NewTerm{pulsars}\index{pulsars}".
	
	For very massive stars (above $50$ solar masses), the total mass of the core that collapses could exceed $3$ solar masses. In this case, gravity becomes such that its mass collapses beyond the last repulsive forces and compacted into a singularity. The curvature of space becomes such that almost (without going into the details of some theoreties that have until now not been verified) no material information or radiation can escape beyond the horizon or a volume named the "\NewTerm{Schwarzschild sphere}\index{Schwarzschild sphere}". This is a "\NewTerm{black hole}\index{black hole}". Anything that falls inside loses his identity. A black hole has only three properties: its mass, angular momentum and electric charge. We say that a black hole has "no hair". Moreover, such a singularity should always be hidden by a horizon, be: "dressed" (for more details see the section of General Relativity).
	
	To give an idea of the scales you can see the figure below:
	\begin{figure}[H]
		\begin{center}
		\includegraphics[scale=2]{img/cosmology/size_comparison.jpg}
		\end{center}	
		\caption{Comparison of various planets with various Stars}
	\end{figure}
	And here just for the solar system but without respecting distance but only the proportions of the planets and Sun (high definition image so you can zoom on):
	\begin{figure}[H]
		\begin{center}
		\includegraphics[width=\textwidth]{img/cosmology/solar_system.jpg}
		\end{center}	
		\caption{Solar System Proportions}
	\end{figure}
	
	\pagebreak
	\subsubsection{Stellar Physics}
	We will now see how new stars can be born from huge gas clouds that extend between the stars in galaxies. The interstellar medium is a potential source of new stars, which once completed their life (as a red giant or supernova) can inject some of their material in outer space.

	In fact, nobody really knows in this beginning of this 21st centuery the details of how an interstellar cloud leads to a star because it is a very difficult problem, mainly because of the emergence of a hierarchy of structures, sub-structures, etc. in the cloud as it collapses on itself. Turbulent motions appear, which can not be described simply by the hydrodynamic equations (\SeeChapter{see section Continuum Mechanics}). Further complications arise when we consider the magnetic field on the gas contraction, or supernova explosions in the cloud...

	At least, can we give the necessary conditions for a star to form in an interstellar cloud. For this, several barriers must actually be completed. A first thermal barrier. A second rotational barrier is: a protostar that contracts rotates faster and faster and can literally explode if its speed becomes too high (conservation of angular momentum). Let's examine these two effects.
	
	
	\paragraph{Collapse of an Interstellar Cloud}\mbox{}\\\\\
	Two opposing forces are present in a cloud of mass $M$ and radius $R$: an autogravitation a force which tends to contract the cloud, and thermal pressure force, which tends to explode it.
	
	We can quantify these two opposite forces in terms of energy: the cloud has a gravitational potential energy (negative) and a kinetic energy (positive) due to thermal agitation of the molecules.

	We know (\SeeChapter{see section Classical Mechanics}) that the gravitational potential energy of two masses $m$ and $m$ of particles separated from a distance $r$ is written:
	
	So the external potential energy of a spherical cloud (...) of mass $M$ and radius $R$ is of the order of:
	
	\begin{tcolorbox}[title=Remark,colframe=black,arc=10pt]
	Some practitioners (this is our case) prefer for the following developments use the internal potential energy that is given for recall by (see the proof in the section of Classical Mechanics):
	
	\end{tcolorbox}
	In a gas in thermodynamic equilibrium, a particle has a kinetic energy (\SeeChapter{see section Continuum Mechanics}) of $kT/2$ by degree of freedom (translation, rotation, etc.). So if $\mu$ is the average mass of a molecule of the cloud, the total kinetic energy of the latter will be expressed:
	
	The cloud then collapses if its total mechanical energy is negative, or (according to the previous approximation):
	
	The above equation defines the "\NewTerm{Jean's mass}\index{Jean's mass}" (assuming a spherical and homogeneous distribution). This is the minimum mass (limit) at a given temperature $T$ and density $\rho$ for a cloud begins to collapse until another physical process may intervene to stop the contraction of the gas.
	
	By eliminating the radius with:
	
	In the previous relation, we get:
	
	If we would not have make the choice of the external potential, but rather the internal one (more accurate in our point of view) and we did not approximate $2/3\cong 1$ the final result would have been:
	
	That is traditionally written as:
	
	Therefore if $M_\text{cloud}>M_J$ then the cloud collapse!
	
	As they are many approximations, astrophysicists prefer to write this last relation as:
	
	where $C$ is obviously a constant without units.
	
	\pagebreak
	\subparagraph{Limit Mass Cloud for Ionization (rogue planets)}\mbox{}\\\\\
	Now let us come back on the relation:
	
	A famous question is what is the mass required by a hydrogen cloud to start nuclear fusion and become a star. For this, we can say in a first approximation that for the fusion, hydrogen atoms must have a distance equal to their radius such that we have for the density:
	
	where for comparison, density for iron is $7,874\;[\text{kg}\cdot \text{m}^{-3}]$. We will also take $\overline{m}\cong m_p = 1.6726219\cdot 10^{-27}$ [kg] (we neglect the mass of electrons as it is almost $1,800$ smaller) and for temperature fusion of hydrogen\footnote{We have proved that the ionization energy of hydrogen in the section of Corpuscular Quantum Physics was $13.6$ [eV], hence $157,821$ [K] but to avoid recoupling we take a security factor of $10$} $T=T_i=10^6$ [K].
	
	Therefore:
	
	This value perfectly match the value given by then french version of Wikpedia (given without proof...).
	
	With the $10$ security factor for $T_i$ we would have:
	
	In comparison Jupiter is $0.1\%$ of the mass of the Sun...
	
	Therefore for an initial mass is less than $0.066 M_\odot $ the gas sphere liquefies or solidifies and stabilized in the form of a planet. Jupiter as we have just see has a mass that is not for very near this limit value and we observed with telescopes that the planet is still very slowly contracting. 
	
	Such bodies are named "\NewTerm{sub-brown dwarfs}\index{sub-brown dwarfs}", sometimes referred to as "\NewTerm{rogue planets}\index{rogue planets}".
	
	\pagebreak
	\subparagraph{Limit Mass Cloud for Fusion (black dwarf)}\mbox{}\\\\\
	The name "\NewTerm{black dwarf}\index{black dwarf}" has also been applied to substellar objects that do not have sufficient mass, less than approximately $0.08 M_\odot$, to maintain hydrogen-burning nuclear fusion. These objects are now generally called "\NewTerm{brown dwarfs}\index{brown dwarfs}", a term coined in the 1970s. Black dwarfs should not be confused with black holes or neutron stars.
	
	Beyond ionization mass limit, it is an ionized gas ball that will continue gravitational collapse. If during contraction, the temperature of $10^ 7$ [K] is not reached, the nuclear reactions can be triggered and it is the quantum nature of repulsive forces that will oppose gravity. Electrons are fermions (\SeeChapter{see section Statistical Mechanics}), the principle of Pauli exclusion (\SeeChapter{see section Corpuscular Quantum Physics}) prevents the stack of Electron in the same volume of phase space. This is equivalent to a high pressure which is well above the thermal pressure of the atoms.
	
	The mass that can be stabilized in this state is still:
	
	As $T_f\cong 10^7$ [K], the electrons are non-relativist. The linear moment is such that:
	
	From the incertitude principle (\SeeChapter{see section Wave Quantum Physics}):
	
	Roughly:
	
	to compare with the previous $r_0=0.5\cdot 10^{-10}$ [m]...
	
	Therefore:
	
	For the protons, that are $1,800$ times more massive, the minimum volume is much smaller and can be neglected at this level of our discussion.
	
	We use now the same relation as previously but where we take $T_f$ instead of $T_i$ and $\overline{m}=m_e=9.10938356(11)\cdot 10^{-31}$ [kg] as the proton mass as plasma experiences and intuition gives that electrons because of their small mass take the most kinetic energy (in comparison to the proton for the same charge... or as in heated liquids where small molecules are much more agitated than big one):
	
	Therefore theoretically we have so far:
	\begin{itemize}
		\item $M_i<0.066M_\odot$ we have a big gaz planet of the style of Jupiter or Saturn ("\NewTerm{sub-brown dwarfs}\index{sub-brown dwarfs}" or "\NewTerm{rogue planets}\index{rogue planets}" for recall).

		\item $0.066M_\odot<M_f<0.92M_\odot$ we have a ionized hot star at the limit of starting a nuclear fusion it's for recall a "\NewTerm{black dwarf}\index{black dwarf}" or more suited named a "\NewTerm{brown dwarf}\index{brown dwarf}".

		\item For $M_J>0.92M_\odot$ we have then a star able to initiate thermonuclear fusion.
	\end{itemize}
	Brown dwarf seems to be very difficult to observe. It seems as far as we know that the first one was directly observer in 2016 only (HD 4747 B)... because of their low surface energy emitting.
	\begin{figure}[H]
		\begin{center}
		\includegraphics[scale=0.55]{img/cosmology/brown_dwarf.jpg}
		\end{center}	
		\caption{Sun, brown dwarf and rogue planet (source: Wikipedia)}
	\end{figure}
	\begin{tcolorbox}[title=Remark,colframe=black,arc=10pt]
	The devlopements above are obviously approximations and depends on many other factors. For example, Proxima Centauri, located just $4.2$ light-years away has $12\%$ of the mass of the Sun, and it’s estimated to be just $14.5\%$ the size of the Sun with a diameter of Proxima Centauri about $200,000$ [km] (just for comparison, the diameter of Jupiter is $143,000$ [km], so Proxima Centauri is only a little larger than Jupiter).\\

	But that's not the smallest star ever discovered! The smallest known star right now is OGLE-TR-122b that is part of a binary stellar system. This red dwarf has its radius accurately measured!: $0.12$ solar radii. This works out to be $167,000$ [km]. That's only $20\%$ larger than Jupiter. You might be surprised to know that OGLE-TR-122b has $100$ times the mass of Jupiter.
	\end{tcolorbox}
	
	\paragraph{Nuclear Duration Life}\mbox{}\\\\\
	Once again remember that we have proved in the section of Classical Mechanics that insiste a massive body the gravitational potential energy was given by:
	
	What has this have to do with starshine?

	Well, notice that as $R$ gets smaller, $E_p$ gets more negative then energy is being converted to other forms, like heat. If a star can radiate this heat into space, then gravitational contraction might produce the luminosity of the star.
	
	How much of this gravitational energy can be radiated away? 
	
	Remember that we know that radiation is related to heat that is related to velocity (\SeeChapter{see section Statistical Mechanics}) and that in the section of Continuum Mechanics we proved during our study of Virial theorem that:
	 
	But back to the contracting Sun. The Virial theorem says that half the change in gravitational energy stays with the star (it heats the star throught the atomic agitation). The other half is radiated away.
	
	So, let's say that the Sun has been contracting and was originally much, much bigger.  Initially its gravitational potential energy was tiny (why?), so the change in gravitational energy is:
	
	Now, half of this energy could have been radiated as the Sun shrank:
	
	that gives with our Sun values: $\cong 10^{41}\;[J]$. That's a lot of energy! So how long could it sustain the luminosity of the Sun?:
	
	this time is named the "\NewTerm{Kelvin-Helmholtz timescale}\index{Kelvin-Helmholtz timescale}" and with the values of the sun it gives:
	
	and as we see the value is quite problematic... As our Earth would be older than our Sun. In fact this problem comes the fact that we don't take into account the fuel of start is the nuclear fusion. So let us see a little bit more accurate model:
	
	So the age of the stars is as we will see just now mainly a problem of calculation of nuclear fuel. The resolution of this problem was given by relativity, and in particular by the mass-energy equivalence (\SeeChapter{see section Special Relativity}).

	Even if the detailed description of nuclear reactions in the heart of the Sun was made in the mid-1930s by Hans Bethe, astrophysicists have suspected soon after Albert Einstein's work that the mass-energy equivalence could explain the brightness of the Sun on billions of years, for example through the fusion of ionized hydrogen (proton $p$) into ionized helium (two protons, two neutrons) via a series of steps (the specified energy is the kinetic energy of the different elements):
	
	The positron annihilates instantly with one of the electrons of a surrounding hydrogen and theire mass-energy is liberated in the form of two gamma photons:
	
	After this the deuterium produced in the first stage can fuse with another hydrogen nucleus to produce an isotope of helium:
	
	Finally, two isotopes of helium $^3\mathrm{He}$  may fuse and produce the normal isotope of helium $\tensor[^{3}_2]{\mathrm{He}}{}$ and also two hydrogen nuclei that can start again the reaction in three difference ways named PP1, PP2 and PP3:
	
	And these reactions do not occur all with the same probability and at the same temperatures ...
	
	The measurement of the mass of the proton gives $m_p\cong 1.673\cdot 10^{-27}$ [kg], while the helium mass is   $m_{\text{He}}\cong 6.645\cdot 10^{-27}$ [kg], that is to say a loss in atomic mass (we neglect the mass of positrons which is $10,000$ times smaller than that of the neutrino):
	
	So a relative loss of mass by fusion (this is the part of the reactions that escapes from the Sun in the form of kinetic energy):
	
	We will prove further below that the Sun emits a power output of:
	
	Therefore its mass consumption per second is:
	
	i.e. its mass decreases by $4.4$ million tonnes per second...
	
	Now we know that this value corresponds to only $0.72\%$ of the mass put in reaction in a fusion. The total mass reacted  is then (rule of three):
	
	Thus, at every second $627$ million tons of hydrogen 1(ionized) fuse into helium 4 with a weight loss of $4.4$ million tonnes which is converted into energy.
	
	Assuming that only the center of the Sun fills the thermal conditions for the fusion ($\cong 10\%$ of its total mass), this brings us to determine the time of nuclear life of the Sun (or any other Star of the same type whose mass is known):
	
	Transforming this in years we have:
	
	
	\paragraph{Internal Temperature}\mbox{}\\\\\
	The stars are assumed to be spherical clusters of hydrogen gas where the interactions between molecules are governed by the gravitational attraction.

	A star has no wall that delimits it, that is to say that there are no external forces coming from vacuum and therefore:
	
	and also not any bulk modulus.
	\begin{figure}[H]
		\begin{center}
		\includegraphics[scale=0.23]{img/cosmology/sun_corona.jpg}
		\end{center}	
		\caption{Sun Corona Zoom (source: NASA Goddard Space Flight Center)}
	\end{figure}
	\begin{figure}[H]
		\begin{center}
		\includegraphics[scale=0.75]{img/cosmology/global_sun.jpg}
		\end{center}	
		\caption{Global Sun overwivew (source: NASA)}
	\end{figure}
		
	Using the Virial theorem in the section of Continuum Mechanics that gives us:
	
	We have for a homogeneous spherical gas of radius $R$ and mass $N$ composed of $N$ bodies, the relations the following relations proved for the first one in the section of Continuum Mechanics of the second in the section of Classical Mechanics:
	
	Therefore:
	
	where for recall $k$ is the Boltzmann constant.

	Which gives:
	
	in order to not make the confusion between to constant of ideal gaz denoted $R$ and the radius it is more convenient to write the latter relation as (and making the Boltzmann constant more explicit):
	
	With for a given star $N$ being the ratio of the total mass of the star on the average mass of a molecule.
	
	For the Sun it comes that $T\cong 10^7$ [K].
	
	This is the central temperature of the Sun. Optical measurements measured from Earth only give the surface temperature (chromosphere), thus $6,000$ [K]. The calculated internal temperature is about $1,600$ times higher than at the surface. Independent methods based on nuclear reactions in the center of the Sun (measurement of solar neutrino flux) give the same order of magnitude, but the precise values differ by a factor of $2$-$3$.
	
	\paragraph{External temperature}\mbox{}\\\\\
	We have proved in the section of Thermodynamics that the Stefan-Boltzmann law permits to calculate the temperature of a heated body from its emittance or its internal energy in terms of density such as:
	
	with:
	
	being the Stefan-Boltzmann.

	Let us take an interesting example that concerns us directly:
	
	The average emittance also named "\NewTerm{average bolometric emittance}\index{average bolometric emittance}" received by the Earth outside the atmosphere, also named "\NewTerm{solar constant}\index{solar constant}" (which is in fact not constant ... on a scale of several billion years), is directly measurable in orbit and is equal to $\sim 1,373\;[\text{Wm}^2]$.
	
	Knowing the average distance from the Sun to be about $1.496\cdot 10^{11}\;[\text{m}]=1\;[\text{UA}]$  (Astronomical Unit), we can calculate the surface of the sphere $S$ at $R=1$ [UA] and thus the solar power $P$. Thus:
	
	and:
	
	Assuming known the radius of the Sun as being $r_{\odot}\cong 6.9599\cdot 10^8$ [m], we can calculate its surface $S$ and is solar radiative emittance $M_{\odot}(T)$. So:
	
	and:
	
	\begin{tcolorbox}[title=Remark,colframe=black,arc=10pt]
	The radiating surface of a star is named "\NewTerm{photosphere}\index{photosphere}". Indeed, as stars, excepting neutron stars, have no solid surface, the photosphere is typically used to describe the Sun's or another star's visual surface.
	\end{tcolorbox}
	Using the Stefan-Boltzmann law, we can now calculate approximately the thermodynamic temperature of the photosphere:
	
	which is very accurate to direct measurement!!! More generally the previous relation is written:
	
	or respecting the notation of optical geometry:
	
	So since we can estimate the luminosity and the temperature of a star (or something that looks like...) we can also estimate it's radius!
	\begin{figure}[H]
		\begin{center}
		\includegraphics{img/cosmology/photosphere.jpg}
		\end{center}	
		\caption{Layer's view of our Sun (source: NASA)}
	\end{figure}

	Planck's law (\SeeChapter{see section Thermodynamics}) applied at this temperature allow us to calculate the spectral distribution of solar radiation and then we see that the maximum of the intensity is in the visible range (our visible range!!!) spectrum which is from $400$ [nm] to $700$ [nm].
	
	\paragraph{Equation of Hydrostatic Equilibrium}\mbox{}\\\\\
	The absence of changes in most stars over timescales of hours or days indicates that the forces acting on the matter in the stars are essentially perfectly balanced (remember figure showed earlier above). Here we analyze this constraint in more detail.

	In the figure above the we have represented a piece mass shell in a spherically symmetric star where we consider in reality a very small cylindrical element between radius $r$ and radius $r + \mathrm{d}r$ hence the fact that the element surface $\mathrm{dS}$ is considered as being the same (but we can neglect the pressiur variation as it can be very big even in a small height difference):
	\begin{figure}[H]
		\centering
		\includegraphics{img/cosmology/histrostatic_equilibrium.jpg}	
	\end{figure}
	If we denote by $M(r)$ the mass of stat in the smaller radii and $\Delta m$ the mass in the cylinder, we get:
	
	Now for the pressure we have (net force due to difference in pressure between upper and lower faces):
	
	Using the definition of the derivative but applied to the pressure:
	
	Therefore:
	
	Now we have as mass element that is given as we know by:
	
	Applying Newton's second law to the cylinder:
	
	This sum must be equal to $0$ everywhere if the star is indeed static. Therefore:
	
	After simplification we get the "\NewTerm{equation of hydrostatic equilibrium}\index{equation of hydrostatic equilibrium}" or "\NewTerm{stellar structure equation}\index{stellar structure equation}":
	
	So far we already estimated roughly the core temperature of the Sun and of the photosphere. Let us now first estimate roughly its average pressure:
	
	A better estimation is given by using the equation hydrostatic equilibrium:
	
	and integrate (assuming density is constant):
	
	So we have an expression for the central pressure:
	
	That is to $6$ times more than the previous roughly approximation and compared to direct measurement method this seems more accurate!
	
	We also have for mean density of the sun:
	
	to compare with the density of pure water that is $1,000 \; [\text{kg}\cdot \text{m}^{-3}]$ or to that of iron that is $7,874 \; [\text{kg}\cdot m^{-3}]$. So we understand better why space image of the sun looks like a big liquid sphere of gas as because of the gravity conditions the pressure is such that the gas is reduce to a density greater than that of water in average!!!!!
	
	Here is a figure of what we think so far as comparison for Jupiter that is like a non-initiated star:
	\begin{figure}[H]
		\centering
		\includegraphics{img/cosmology/jupiter_layers.jpg}	
	\end{figure}
	
	Using the equation of hydrostatic equation we can estimate roughly the density of the photosphere:
	
	So it is obvious that the density of the sun decreases continuously outward from the center. The visible surface of the Sun (i.e. the photosphere) is a very thin layer, only about $500$ [km] thick as compared to the radius of the Sun. The density of the photosphere is very, very low, about $0.2\cdot 10^{-4} \;[\text{kg}\cdot \text{m}^{-3}]$ as estimated by observations. Therefore the average density of the Sun can only be explained by the density of its core that is very high, about $160,000\;[\text{kg}\cdot\text{m}^{-3}]$ , much higher than any material that we know.

	
	
	\pagebreak
	\paragraph{Brightness}\mbox{}\\\\\
	The "\NewTerm{intrinsic bolometric brightness}\index{intrinsic bolometric brightness}" of a star corresponds to the total power radiated in the entire electromagnetic spectrum in the direction of the observer expressed relatively to the total power radiated by the sun. Assuming all stars spherical and isotropic , we can express it in solar units:
	
	The radiated power $P$ is calculated, as we know, by multiplying the radiative emittance (Stefan-Boltzmann law) by the surface of the star:
	
	The intrinsic bolometric luminosity of a star is therefore proportional to the square of its radius and the fourth power of its surface temperature. Taking the Sun as a reference, the constants are simplified. We can the write:
	
	with $r_{\odot}\cong 6.9559\cdot 10^8$ [m] and $T_{\odot}\cong 5,780$ [K] hence $c^{te}\cong 1.85\cdot 10^{-33}\;[\text{K}^{-4}\text{m}^2]$.
	
	In astrophysics, we also use a logarithmic scale to express the bolometric luminosity of a star: the "absolute magnitude $M$". This unit has an empirical origin that will be explained below.
	
	\paragraph{Shining (apparent brightness)}\mbox{}\\\\\	
	Perhaps the easiest measurement to make of a star is its apparent brightness. I am purposely being careful about my choice of words. When I say apparent brightness, I mean how bright the star appears to a detector here on Earth.
	
	\textbf{Definition (\#\mydef):} The "\NewTerm{brilliance}\index{brilliance}" or "\NewTerm{shining}\index{shining}" or "\NewTerm{apparent brightness}\index{apparent brightness}" $b$ of a star is the density of radiation received by the observer, that is to say equal to the flow of energy divided (power of the star at its surface) divided by the sphere surface with the  radius equal to the distance which separates the observer from the star:
	
	The brilliance decreases therefore with the square of the distance (as in myna other filed of physics):
	\begin{figure}[H]
		\begin{center}
			\includegraphics{img/cosmology/apparent_luminosity_inverse_square.jpg}
		\end{center}
	\end{figure}
	It is important to notice that this quantity has no direct relation with the physical intrinsic  properties of the respective star (unlike the bolometric brightness!).
	
	Thus, two identical stars can have the same apparent brightness if (and only if) they lie at the same distance from Earth. However, as illustrated in Figure below, two different stars can appear equally bright if the more luminous one lies farther away. A bright star (that is, a star with large apparent brightness) is a powerful emitter of radiation, is near Earth, or both. A dim star is a weak emitter, is far from Earth, or both:
	\begin{figure}[H]
		\begin{center}
			\includegraphics{img/cosmology/apparent_luminosity.jpg}
		\end{center}	
		\caption{Apparent luminosity}
	\end{figure}	
	 The luminosity of a star, on the other hand, is the amount of light it emits from its surface. The difference between luminosity and apparent brightness depends on distance as we know now. Another way to look at these quantities is that the luminosity is an intrinsic property of the star, which means that everyone who has some means of measuring the luminosity of a star should find the same value. However, apparent brightness is not an intrinsic property of the star; it depends on your location. So everyone will measure a different apparent brightness for the same star if they are all different distances away from that star.
	
	Apparent brightness is the brightness perceived by an observer on Earth and absolute brightness is the brightness that would be perceived if all stars were magically placed at the same standard distance. There can be a great difference between the total amount of radiation a star emits and the amount of radiation measured at the Earth's surface.
	
	In astrophysics, we also use another scale of measurement where the apparent brightness is given by another magnitude of empirical origin: the apparent magnitude, which will be explained immediately below.
	
	\paragraph{Apparent magnitude}\mbox{}\\\\\
	Ptolemy in 137 AD had defined a scale of six magnitudes to express the brightness (shining) of stars, the first for the brightest and the sixth for the stars just visible to the naked eye ($6$ magntitudes and therefore $5$ gaps).

	During the 19th century, with the arrival of new photometric observations techniques (photographic and photoelectric), the scale of magnitude was replaced by that of "\NewTerm{apparent magnitude}\index{apparent magnitude}" $m$ that has been defined so that this new scale is close to the old one.

	The definition is the following:
	\begin{itemize}
		\item The scale is logarithmic in base $10$ (for convenience of the magnitude of manipulated quantities)

		\item There are $5$ magnitude gaps corresponding to an apparent brightness ratio of $1$ for $100$ ($1: 100$)

		\item The scale is inverse (high magnitude corresponds to a small apparent magnitude/ brightness).
	\end{itemize}
	Using these definitions, we can construct a relative way relating the shining (brilliance) of two stars to their apparent magnitude $m$.

	For a star $1$ two hundred times brighter than a star $2$, the stat $1$ is $5$ magnitude  units above the star $2$ (remember that the scale is reversed). So a ratio of:
	
	corresponds by definition to:
	
	We can then put the relations:
	
	By applying the rule of three, we build:
	
	By simplifying, we find the "\NewTerm{Pogson's Formula}\index{Pogson's Formula}" which expresses the relation between (visual) apparent apparent magnitudes and brilliance (shining) of two stars:
	
	Apart from small corrections, the brightness of Vega\footnote{Brightest star in the constellation Lyra at this day (21st century). It is actually a relatively close star at only $25$ light-years from Earth, and, together with Arcturus and Sirius, one of the most luminous stars in the Sun's neighborhood} ($\alpha$ Lyr) still serves as the definition of zero magnitude for visible and near infrared wavelengths. The brightness of Vega is exceeded by four stars in the night sky at the 21st century at visible wavelengths (and more at infrared wavelengths) as well as bright planets such as Venus, Mars, and Jupiter, and these must be described by negative magnitudes. For example, Sirius, the brightest star of the celestial sphere, has an apparent magnitude of $-1.4$.
	
	To get an idea of the (visual) apparent magnitudes relatively to Vega here are some examples: Sun $m_{\odot}=-26.74$, Full Moon $m=-15$, Venus maximum $m=-4.8$, Sirius $m=-1.4$ (spectral type A1 and distant of $8.6$ light years), limit perceived with the naked eye $6$, perception limit through an amateur telescope of $15$ [cm] at this date (2003) $m=13$ limits of perception Hubble space telescope $m=30$.
	\begin{tcolorbox}[colframe=black,colback=white,sharp corners]
	\textbf{{\Large \ding{45}}Example:}\\\\
	Now as we know that the apparent magnitude of the Sun is $-26.74$ (brighter), and the mean apparent magnitude of the full moon is $-12.74$ (dimmer) the difference in apparent magnitude is obviously that $\delta m=14.00$\\

	With this information reconsider that the Pogson formula also gives by construction the ratio of the luminosity. So relatively to our example we get:
	
	After rearranging we get therefore:
	
	Or better to have a nicer number:
	
	The Sun appears about $400,000$ times brighter than the full moon.
	\end{tcolorbox}
	It should be noticed that the (visual) apparent magnitude does not exactly match the real apparent magnitude, because the eye is not equally sensitive to all wavelengths. The blue or red stars seem less bright to the eye than they actually are because some of the radiation is in the ultraviolet, respectively in the infrared.

	It is therefore necessary to clarify whether it is a  visual or bolometric apparent magnitude. In general, astrophysicists use bolometric magnitudes in their publications.
	
	\paragraph{Absolute magnitude}\mbox{}\\\\\
	The absolute magnitude $M$ (not to be confused with the notation of emittance seen in the section of Geometrical Optics) of a star is also a logarithmic scale , expressing this time the bolometric luminosity $L$!!! It is the quantity presented in ordinate of the Hertzsprung-Russell diagram. The scale of this size is based however on the (visual) apparent magnitude.

	The apparent magnitude and absolute magnitude are bound by the distance from the star. At constant intrinsic apparent brightness, the apparent brightness therefore decreases obviously with the square of the distance as we have already seen. In order to establish a relation, we had to choose a reference distance by a new definition.

	\textbf{Definition (\#\mydef):} The "\NewTerm{absolute magnitude}\index{absolute magnitude}" $M$ of a star is equal to its apparent magnitude $m$ if it is distant of $10$ parsecs ($32.6$ light years).
	
	Therefore taking Pogon's formula that is for recall:
	
	And changing the notations to make it correspond the previous definition:
	
	we get:
	
	And as:
	
	But as it is the same star:
	
	In our case this becomes:
	
	Therefore:
	
	So finally:
	
	\begin{figure}[H]
		\begin{center}
		\includegraphics{img/cosmology/absolute_apparent_magnitudes.jpg}
		\end{center}	
		\caption{Sirius apparent VS absolute magnitudes}
	\end{figure}
	As the Sun-Earth distance in parsec is equal to $4.84814\cdot 10^{-6}$  we get:
	
	Therefore:
	
	\begin{tcolorbox}[title=Remark,colframe=black,arc=10pt]
	For the absolute magnitude $M$ to be accurate, we need stellar models, and know the temperature of the star as we will immediately see it. In practice, the only readily accessible quantity is obviously the observed magnitude, which is actually the combination of the apparent magnitude and the interstellar absorption.
	\end{tcolorbox}
	The absolute magnitude can be obviously rewritten with respect to the absolute bolometric luminosity of the Sun:	
	
	We put for the Sun that $L_{\text{bol},\odot}=1$. Therefore it remains:
	
	
	This latter relation of comparison of the absolute magnitude with the apparent magnitude (which is the actually magnitude observed on Earth) allows estimation $d$ of the distance of the object in astrophysics.
	
	Using the expression of the bolometric luminosity proved earlier above:
	
	the absolute magnitude of star being a direct function of its temperature and radius we can then write:
	
	This is the result we wanted to prove from the beginning: the absolute bolometric magnitude is directly related to the bolometric luminosity of the star, which is why it is one that most interests astrophysicists.
	\begin{tcolorbox}[title=Remark,colframe=black,arc=10pt]
	The distance to nearby stars could be determined by the satellite Hipparcos. By measuring the parallax (measurements of the star position at six-month intervals and applying basic trigonometric rules as seen in the section Astronomy). But beyond a few tens of parsecs, measuring the distance of stars by parallax becomes very imprecise. By studying the spectrum of the star, we can determine its spectral class, its surface temperature and place in the Hertzsprung-Russell diagram. It is therefore possible to estimate its absolute magnitude and roughly calculate its distance.
	\end{tcolorbox}
	This measurement trick is fundamental to cosmology. It is the way we determines the distance to nearby galaxies by measuring the period of some variable stars (we will focus a little bit on that further below).

	The distance of distant galaxies is calculated by measuring the apparent magnitude of supernovae that occur in it. Indeed, the absolute magnitudes of Type Ia supernovae (we recognize them by the lack of hydrogen spectrum lines, and by the decrease in brightness) are well calibrated because the energy released by these stellar explosions is relatively constant.
	
	The stars of the main sequence of the Hertzsprung-Russell diagram are very stable objects. The gravitational force, which tends to contract the star, is exactly compensated by the internal pressure forces, which tend to dilate it. It's when the star becomes a red giant that sometimes the balance is upset. Thus began a phase of instability which results in significant variations in the brightness of the star.

	The breaking of balance is caused by a complex phenomenon that involves variations of transparency of helium layers near the surface of the star. From there, the star begins to experience a series of expansions and contractions controlled by the forces that were formerly balance. When the pressure force prevails, the volume of the star increases. But the gravity slows the movement and eventually cause contraction. The volume of the star will pass below its average value, until the internal pressure opposes the contraction and managed to cause further expansion.

	It is not the size changes that cause the variations in brightness, but those of the temperature. Indeed, as we have prove it earlier above, the brightness of a star varies with the fourth power of the temperature, while it varies with the square of the radius following for recall:
	
	When the volume of the star, however, is lower than average, the temperature is slightly higher and the brightness maximum. At the opposite, the temperature is slightly lower than average and the brightness minimum . The brightness of the star thus changes periodically, hence the name of "variable star" or "pulsative variable star".

	It exists in the Hertzsprung-Russell diagram of a band of instability that crosses this diagram almost vertically just to produce the thermal phenomena in question.

	The two main types of pulsating variables are the Cepheids and RR Lyrae stars. These bodies play a central role in astrophysics. Cepheids are stars of a few solar masses. They are in the helium burning phase after reaching the red giant stage. The stars of solar mass arrived at this point become RR-Lyrae stars. Their brightness varies with a period of between one day and several weeks. The remarkable property of Cepheids is the existence of a relation between the average brightness and the period of their oscillations. For example, the average brightness is $1,000$ times that of the Sun for a period of days and $10,000$ times that amount for a period of several weeks. It is this relation that makes Cepheids one of the basic tools of astrophysics.
	
	\subsubsection{Pulsative Variable Stars}
	The stars of the main sequence of the Hertzsprung-Russell diagram are very stable objects. The gravitational force, which tends to contract the star, is exactly compensated by the internal pressure forces, which tend to dilate it. It's when the star becomes a red giant that sometimes the balance is upset. Thus began a phase of instability which results in significant variations in the brightness of the star.

	The breaking of balance is caused by a complex phenomenon that involves variations of transparency of helium layers near the surface of the star. From there, the star begins to experience a series of expansions and contractions controlled by the forces that were formerly balance. When the pressure force prevails, the volume of the star increases. But the gravity slows the movement and eventually cause contraction. The volume of the star will pass below its average value, until the internal pressure opposes the contraction and managed to cause further expansion.

	It is not the size changes that cause the variations in brightness, but those of the temperature. Indeed, as we have prove it earlier above, the brightness of a star varies with the fourth power of the temperature, while it varies with the square of the radius following for recall:
	
	When the volume of the star, however, is lower than average, the temperature is slightly higher and the brightness maximum. At the opposite, the temperature is slightly lower than average and the brightness minimum . The brightness of the star thus changes periodically, hence the name of "variable star" or "pulsative variable star".

	It exists in the Hertzsprung-Russell diagram of a band of instability that crosses this diagram almost vertically just to produce the thermal phenomena in question.

	The two main types of pulsating variables are the Cepheids and RR Lyrae stars. These bodies play a central role in astrophysics. Cepheids are stars of a few solar masses. They are in the helium burning phase after reaching the red giant stage. The stars of solar mass arrived at this point become RR-Lyrae stars. Their brightness varies with a period of between one day and several weeks. The remarkable property of Cepheids is the existence of a relation between the average brightness and the period of their oscillations. For example, the average brightness is $1,000$ times that of the Sun for a period of days and $10,000$ times that amount for a period of several weeks. It is this relation that makes Cepheids one of the basic tools of astrophysics.
	
	If we know this relationship for a variable star, it is relatively easy, by the determination its period to derive its absolute magnitude $M$. By then measuring its apparent magnitude $m$ we can then calculate the distance in parsec with of the relation proved earlier above:
	
	
	One of the main reasons for constructing the Hubble Space Telescope (HST) was to measure light curves of Cepheid variables in other galaxies. It is especially important to use Cepheids to measure distances to the galaxies in two nearby clusters: the Virgo Cluster (the nearest rich cluster), and the Fornax Cluster (a somewhat sparser collection of galaxies).

	HST can zoom in on a small portion of a galaxy to find and measure Cepheids:
	\begin{figure}[H]
		\begin{center}
		\includegraphics[scale=0.7]{img/cosmology/cepheid_hst_galaxy_ngc1365.jpg}
		\end{center}
	\end{figure}
	and the zoom inside the are of interest:
	\begin{figure}[H]
		\begin{center}
		\includegraphics[scale=0.7]{img/cosmology/cepheid_hst_galaxy_ngc1365_wfpc2_zoom.jpg}
		\caption{Hubble Space Telescope Cepheid Measurement}
		\end{center}
	\end{figure}
	The period-luminosity relation for classical Cepheids was discovered in 1908 by Henrietta Swan Leavitt in an investigation of thousands of variable stars in the Magellanic Clouds. She published it in 1912 with further evidence. Once the period-luminosity relationship is calibrated, the luminosity of a given Cepheid whose period is known can be established. Their distance is then found from their apparent brightness. The period-luminosity relationship has been calibrated by many astronomers throughout the twentieth century, beginning with Hertzsprung. Calibrating the period-luminosity relation has been problematic; however, a firm Galactic calibration was established by Benedict et al. 2007 using precise HST parallaxes for 10 nearby classical Cepheids. Also, in 2008, ESO astronomers estimated with a precision within $1\%$ the distance to the Cepheid RS Puppis, using light echos from a nebula in which it is embedded. However, that latter finding has been actively debated in the literature.

	The following relationship between a Population I Cepheid's period $P$ (in days) and its mean absolute magnitude $\bar{M}$ was established from Hubble Space Telescope trigonometric parallaxes for $10$ nearby Cepheids:
	
	\begin{figure}[H]
		\begin{center}
		\includegraphics[scale=0.7]{img/cosmology/period_cepheid_relation_plot.jpg}
		\caption{Cepheid absolute magnitude - period plot (source: Wikipedia)}
		\end{center}
	\end{figure}
	Cepheids aren't perfect distance indicators. For one thing, their brightness and periods of pulsation can vary with their chemical composition. There's also the problem of crowding and confusion: what if our view of a distant galaxy appears to show a single, varying Cepheid star... but is really a combination of light from the Cepheid and several nearby stars, all mixed together?
	
	\subsubsection{Neutron Stars (magnetars)}
	A neutron star is the collapsed core of a large star ($10$ to $29$ solar masses). Neutron stars are the smallest and densest stars known to exist. With a radius on the order of $10$ [km], they can, however, have a mass of about twice that of the Sun. They result from the supernova explosion of a massive star, combined with gravitational collapse, that compresses the core past the white dwarf star density to that of atomic nuclei. Most of the basic models for these objects imply that neutron stars are composed almost entirely of neutrons, which are subatomic particles with no net electrical charge and with slightly larger mass than protons. They are supported against further collapse by neutron degeneracy pressure, a phenomenon described by the Pauli exclusion principle (see further below). If the remnant has too great a density, something which occurs in excess of an upper limit of the size of neutron stars at $2$-$3$ solar masses, it will continue collapsing to form a black hole (see proof further below).
	
	\paragraph{Chandrasekhar limit}\mbox{}\\\\\
	We have already determined in the section of Classical Mechanics the Schwarzschild radius (in its classical form) that expresses the critical radius of a body for the release speed to it surface to be equal to that of speed of light. We obtained the following relation which typically expressed the radius that have a given celestial boject to have a release speed equal to that of light:
	
	
	In the figure below on the left we have schematic slice through a neutron star. Letters N, n, p, e, $\mu$ refer to the presence of nuclei, fluid neutrons and protons, electrons and muons, respectively. The inner core composition is still uncertain and various exotic possibilities exist, including hyperons and deconfined quark matter. 
	\begin{figure}[H]
		\centering
		\includegraphics[scale=1]{img/cosmology/neutron_star_slice.jpg}	
		\caption{Neutron star slice}
	\end{figure}
	On the figure below we have an overview of what we expect to be the composition of the inner crust:
	\begin{figure}[H]
		\centering
		\includegraphics[scale=0.9]{img/cosmology/neutron_star_inner_crust.jpg}	
		\caption{Neutron star inner crust}
	\end{figure}
	At lower densities, a lattice of superheavy, neutron-rich nuclei is immersed in a fluid of neutrons (which are likely to be superfluid) and a relativistic electron gas. At high enough densities the nuclei might deform and connect along certain directions to form extended tubes, sheets and bubbles of nuclear matter. These nuclear pasta phases might form a layer at the base of the neutron star crust, sometimes referred to as the mantle. Ranges of density and thickness given for each layer represent current uncertainties in the physics of neutron star crusts.
	
	
	\paragraph{Neutron star magnetic field}\mbox{}\\\\\
	As the star's core collapses, its rotation rate increases as a result of conservation of angular momentum, hence newly formed neutron stars rotate at up to several hundred times per second. Some neutron stars emit beams of electromagnetic radiation that make them detectable as pulsars. Indeed, the discovery of pulsars in 1967 was the first observational suggestion that neutron stars exist. The radiation from pulsars is thought to be primarily emitted from regions near their magnetic poles. If the magnetic poles do not coincide with the rotational axis of the neutron star, the emission beam will sweep the sky, and when seen from a distance, if the observer is somewhere in the path of the beam, it will appear as pulses of radiation coming from a fixed point in space (the so-called "lighthouse effect"). The fastest rotation rate for a neutron star was a rate of $716$ times a second or $43,000$ revolutions per minute, giving a linear speed at the surface on the order of $0.165 c$....
	
	So now let us focus on the simplified math approach of the impact of the angular momentum conservation on the magnetic field of the Star. 
	
	From the conservation of angular moment as the core collapses we have (\SeeChapter{see section Classical Mechanics}):
	
	Or,  for a sphere of constant density  (\SeeChapter{see section Geometric Shapes}):
	
	So the final spin frequency is:
	
	or the final spin period is:
	
	The magnetic flux $\Phi$ ($\vec{B}$ times area $S$) through the surface of the core is also conserved in collapse. So roughly   (\SeeChapter{see section Magnetostatics}):
	
	Which means that:
	
	
	Tensile strength against rotation \\
	
	\subsection{Galaxies}
	A galaxy is a gravitationally bound system of stars, stellar remnants, interstellar gas, dust, and (of the supposed...) dark matter. 

	Galaxies range in size from dwarfs with just a few billion ($10^9$) stars to giants with one hundred trillion ($10^{14}$) stars, each orbiting its galaxy's center of mass. Galaxies are categorized according to their visual morphology as elliptical, spiral or irregular:
	\begin{figure}[H]
		\begin{center}
		\includegraphics[scale=0.145]{img/cosmology/classification_galaxies_spitzer.jpg}
		\end{center}	
		\caption{Apparent stars speed anomaly in galaxies rotations}
	\end{figure}
	 Many galaxies are thought to have black holes at their active centers. 

	It seems that there is between $2\cdot 10^{11}$ galaxies in the observable Universe following the actual estimates. Most of the galaxies are $1,000$ to $100,000$ parsecs in diameter and usually separated by distances on the order of millions of parsecs (or megaparsecs). The space between galaxies is filled with a tenuous gas having an average density of less than one atom per cubic meter. The majority of galaxies are gravitationally organized into associations known as galaxy groups, clusters, and superclusters. At the largest scale, these associations are generally arranged into sheets and filaments surrounded by immense voids.
	
	\subsubsection{Radial Speed Anamoly}
	In 1978, Vera Rubin begins to observe that in galaxies, more the stars are distant from the galactic core, the more their angular velocity is high... The initial observation that uniformity of speed was unexpected because the theory of gravity Newton predicted that more distant objects have less speed. For example, the planets of the solar system orbit with a respective speed decreases while growing their respective distance from the sun. We are left with the same problem: how to explain a point measurement is greater than the theoretical value?
	\begin{figure}[H]
		\begin{center}
		\includegraphics{img/cosmology/apparent_anomaly_star_speed_galaxy_rotation.jpg}
		\end{center}	
		\caption{Apparent stars speed anomaly in galaxies rotations}
	\end{figure}
	According to Newton's laws, in a circular path, there is a as we know balance between the centripetal acceleration and gravitational attraction:
	
	The volume of a disk galaxy of radius $R$ and thickness $e$ is (\SeeChapter{see section Geometric Shapes}):
	
	If we consider the mass of the galaxy almost entirely within the radius $R _ {\max}$, corresponding to the maximum speed, of density $\rho$ is given then by:
	
	Making the approximation that the mass is substantially within the range corresponding to the maximum speed, we can write:
	
	Which, introduced into the first equation but rearranged:
	
	 gives:
	
	the law that the maximum speed varies as the $1/4$ power of the mass. After that the speed decrease of the stars should decrease following:
	
	But we must keep in mind that this is a two body relation! In the facts a galaxy should be considered as an isotropic fluid (like the rest of the universe) and therefore it is quite normal that the two-body assumption does not suite the observations. A galaxy can also not be consider as a solid cylinder otherwise by applying $v=\omega r$ the speed of the stars should increase in proportion to the distance to the center of the galaxy.
	
	\begin{flushright}
	\begin{tabular}{l c}
	\circled{90} & \pbox{20cm}{\score{4}{5} \\ {\tiny 28 votes,  80.71\%}} 
	\end{tabular} 
	\end{flushright}

	%to make section start on odd page
	\newpage
	\thispagestyle{empty}
	\mbox{}
	\section{Special Relativity}
	\lettrine[lines=4]{\color{BrickRed}W}e have always considered until now in all our developments that the  interactions (cause and effect) between the body were instantly and the observation of a phenomenon took place instantly after it had taken place. Now, two physicists (Michelson and Morley) during an experiment discovered something that would change radically all of classical physics: the velocity (speed) of light was invariant (constant) regardless of the movement that we had relatively to it!
	
	This observation is even more important that we know that is the light that allows us to perceive and feel things. It should also be taken into consideration that the electrostatic and magnetic fields are, as we have seen in the section of Quantum Field Theory, carried by the vector of interaction that is the photon that moves at the finite speed of light denoted by $c$. This fact also allows us to assume that the gravitational field also has an interaction vector (which would be the "graviton" whose existence seems indirectly proven) that propagates at the speed of light. It is therefore appropriate to take into account this non-immediacy and the consequences that this entails in the observed phenomena to finally be able to decide what is really of what seems to be.
	
	Before we start with the calculations, we need to define a little bit what will be studied in this section (which applies not only to cosmology but... it seemed to us better to put it in this chapter rather than in the chapter of Mechanics or Atomistic).
	
	\textbf{Definition (\#\mydef):} The "\NewTerm{Special Relativity}\index{special relativity}" is a theory confined to isolated inertial frames (Galileans), that is to say, the study of animated frames in a uniform (inertial) rectilinear motion. The reason of this will be given in the statement of Special Relativity principle (see below).
	
	\begin{tcolorbox}[title=Remarks,colframe=black,arc=10pt]
	\textbf{R1.} Restrict the study to inertial frames of course does not does not prohibit that within these, bodies can be animated of a uniform speed or not (an inertial rocket can have bodies inside itself that have a non-uniform movement)!\\
	
	\textbf{R2.} General relativity's purpose (see corresponding section) is to take into account non-inertial frames and in any coordinate system by making use of the power of the tensor calculus to be applicable in any type of space (other than flat one!).
	\end{tcolorbox}	
	Special Relativity is mainly based on three important concepts:
	\begin{enumerate}
		\item The invariance postulate of speed of light
		\item The cosmological principle (see below)
		\item The principle of Special Relativity (see below)
	\end{enumerate}
	It is also important to inform the reader that we will use here many concepts seen in the section of Linear Algebra, Tensor Calculus, Trigonometry, Analytical Mechanics, Classical Mechanics, Electrostatics, Magnetostatics and Electrodynamics. It is therefore strongly advised to have covered these topics before at risk of not understanding what follows.
	
	\subsection{Assumptions and Principles}
	Physics laws express relations between the fundamental physical quantities. If the laws of physics are invariant under Galilean referential change as we have seen in the section of Classical Mechanics, it is not necessarily the same for physical quantities! These can transform from Galilean frame to another according to  simple transformation law as we have seen in the section of Classical Mechanics for velocity for example. It is the same in Special Relativity, but we must now consider what we neglected in our study of Galileo's transformations: the time lag is not the same for two observers if the speed of the light is finite, but the concept of time interval is supposed to be kept invariant!
	
	\subsubsection{Postulate of Invariance}
	Laboratory measurements (Michelson-Morley experiment as we have already mentioned) have, for a long time, shown that the speed $c$ measured in an inertial frame (straight line and at constant speed) is constant regardless of its speed. Then we are taken to state the postulate of invariance of light: the speed of light (vector for the transport of ) can neither be added nor substracted, to the drive speed of the frame in which we measure it (more clearly it means that no matter how fast you move, you will always measure the speed of light as being numerically finite and equal to $c=299,792,458 [\text{km}\cdot  \text{s}^{-1}]$!).
	
	As corollary the principle of Galilean relativity (\SeeChapter{see section of Classical Mechanics}) according to this premise is completely at fault and then we have to develop a new theory that takes into account of this property of light.
	
	\begin{tcolorbox}[title=Remark,colframe=black,arc=10pt]
	It is important to note that we consider that light, within the framework of Special Relativity, the messenger of information from one body to another!!!
	\end{tcolorbox}
	
	\subsubsection{Cosmological Principle}
	We assume that our position in the Universe is typical not only in space as stated in the standard model of the Universe (\SeeChapter{see section Astrophysics}), but also in time. Thus, an astronomer located in a remote galaxy must observe the same general properties of the Universe that we, he lived a billion years ago, or that hed observed it in a billion years.
	
	In fact, it is quite natural to go further and state that: the Universe looks the same in every point, that is to say, it is homogeneous. This homogeneity is therefore sets as the "\NewTerm{Cosmological Principle}\index{Cosmological Principle}".
	
	This principle is not based actual 21th century observations because to much fragmentary compared to the huge size of the cosmos so that they can not establish its validity. It constitutes a presupposition for any physical study of our Universe. Its purpose is relative to its character, essential to any scientific cosmology study, and perhaps to a certain reaction to the geocentric or heliocentric old vision: it is assumed now that no place is special in the cosmos!
	
	\subsubsection{Special Relativity Principle}
	Let us recall (\SeeChapter{see section Classical Mechanics}) that the Galilean transformations tell us that no reference frame can not be considered as an absolute frame because the relations between the physical quantities are identical in all Galileans repositories ("Galilean relativity principle"). The Galilean motion is therefore relative!
	
	In the 20th century physicists noted that an important class of physical phenomena violated the Galilean relativity principle: the electromagnetic phenomena!
	
	By applying the Galilean transformations to Maxwell's equations, we get a different set of equations depending on whether the observer is in a fixed reference or a mobile reference frame.
	
	Indeed, we have proved in the section on Electrodynamics that the electric or magnetic field propagation equation could be written in one-dimensional space as the following d'Alembert equation:
	
	where $\psi$ represents any one of the two fields (electric or magnetic). We name this relation sometimes "\NewTerm{Hertz equation}\index{Hertz equation}".
	
	We also saw in the section of Classical Mechanics that an important factor in the validity of a theory is the invariance of the expression of its laws under a Galilean transformation by putting:
	
	We have also shown in the section of Differential and Integral Calculus that the total differential of a function was written (example with two variables):
	
	Therefore:
	
	Which brings us to simply write (using the physicist method way of life...):
	
	After elimination of $f$ and using the Schwarz theorem (see section Differential and Integral Calculus) and still the physicist way of life:
	
	If we write the same with the time variable:
	
	Ultimately the Galilean transformation of the wave equation supposedly have an invariant form becomes:
	
	To fix the situation, following this example, we can state at least three assumptions:
	\begin{enumerate}
		\item[H1.] Maxwell's equations are false. The correct equations remain to be discovered and must be invariant under a Galilean transformation.
		\item[H2.] Galilean invariance is valid for mechanics but not for electromagnetism (this is the historical solution before Albert Einstein, an "ether" determines the existence of a kind of absolute repository where Maxwell's equations do not change).
		\item[H3.] Galilean invariance is false. There is a more general invariance, it remains to be discovered, which preserves the form of the Maxwell equations. Classical mechanics is to be reformulated so that it is invariant under this new transformation.
	\end{enumerate}
	\begin{tcolorbox}[title=Remark,colframe=black,arc=10pt]
	It turns out that the first two assumptions are excluded by the experimental facts. Moreover, Maxwell's equations integrating the speed of light they are implicitly relativistic.
	\end{tcolorbox}	
	Albert Einstein did not accept the violation of the Galilean relativity principle by electromagnetism. From his perspective, it was necessary to generalize it to all natural laws.
	
	He postulated that the laws of physics should be the same in all repositories Galileans, which means, implicitly, that in the point of view of physical laws, it is not possible to distinguish one from another Galilean frame. This result is most commonly formulated as: no reference is privileged. This principle was named "\NewTerm{principle of relativity}\index{principle of relativity}". Indeed, this relativity is restricted to the case of Galileans frames (also named "inertial frames") exclusively.
	
	In other words, the physic laws should remain unchanged after a change of reference. We must therefore identify new adequate transformations that will substitute to the Galilean transformations.
	
	In the case of non Galileans frames repositories are not indistinguishable anymore. Indeed, imagine a person in a train moving at a constant speed and another person on land. Everyone can then say that it is the other who is in motion (relative) and indiscriminately. By cons, if the train begins to accelerate, although the two individuals can say that this is another speeding, only the one on the train will feel the effect of this acceleration ... and repositories are no more indistinguishable.
	
	Albert Einstein abolished as well as the idea that there is an absolute reference point that does not move and on which we can define an absolute time, an absolute length or absolute mass. However, one can define a privileged reference point for every object in the Universe. It is the frame moving at the same speed and in the same direction as the object. The time measured in this privileged reference frame is minimal and is named the "\NewTerm{proper time}\index{proper time}". Similarly, the size of the object is maximum, it is his "\NewTerm{proper dimension}\index{proper dimension}" or "\NewTerm{proper distance}\index{proper distance}", and its mass is minimal, it is its "\NewTerm{proper mass}\index{proper mass}" (we will do the corresponding detailed mathematical developments further below).
	
		\subsection{Lorentz Transformations/Boost}
		For make possible to $c$ to be invariant (light speed invariance postulate), we must admit that time does not flow the same way for the observer $\text{O}$ that is motionless than for the observer $\text{O}'$ in a reference frame in uniform translation (i.e.: an inertial frame) in the direction of $x$  with relative velocity (the term "relative" is important!) $v$ (caution! the relative speed between the repositories is often denoted in the literature by $u$).
		
	
		\begin{tcolorbox}[title=Remark,colframe=black,arc=10pt]
		A special case of disposal of referential frames in which the space axes are parallel leads to what we name the "\NewTerm{pure Lorentz transformations}\index{pure Lorentz transformations}" or "\NewTerm{special Lorentz transformations}\index{special Lorentz transformations}" and the relative displacement along a particular axis is often named a "\NewTerm{boost}\index{boost}".
		\end{tcolorbox}	
		
		To study the behavior of physic laws, we must bring two clocks that give the times $t$ and $t'$ (the referential frame that contains its clock/measuring instrument is named "\NewTerm{proper referential}\index{proper referential}" or "\NewTerm{proper frame}\index{proper frame}").
		
		Let's set up the following imaginary experiment:
		
		When the observers $\text{O}$ and $\text{O'}$ are superimposed, we set $t = 0$ and $t' = 0$ (clock time sync) and we emit a bright flash\footnote{In fact we should consider the emission of "an element of information". Using light as an example is just convenient for pedagogical purposes. As we will see the results we will get further belov involve a speed limit $c$ that is for sure the speed of light, but in reality we should consider light as a special case of the maximum possible speed of information transfer. The "$c$" can then be seen as the "causality speed".} in the direction of a point $A$ spotted by respectively $\vec{r}$ and $\vec{r}'$:
		\begin{figure}[H]
			\begin{center}
			\includegraphics{img/cosmology/lorentz_pure_transformations_experiment.jpg}
			\end{center}	
			\caption{Configuration for the study of relativistic effects}
		\end{figure}
		It is obvious that when the flash arrive in $A$, the observer $\text{O}$ will measure a time $t$ and $\text{O}'$ a time $t'$.
		
		The observer $\text{O}$ therefore concludes:
		
		The observer $\text{O}'$ therefore concludes:
		
		Since the displacement of $\text{O}'$ is made only along the $\text{O}x$ axis, we have for the two observers:
		
		Moreover, if the path of the light beam coincides within the axe $\text{Ox}$, we have:
		
		This gives us then:
		
		And therefore:
		
		these two relations are equal (zero) at any $x, x', t, t'$ between the two observers. These are the first "relativistic invariant" (equal values regardless of the frame) that we find in a more generalized form when applied to the whole space:
		
		Now iyou should be remembered that in the classical model (Galilean relativity), we would have written that the position of point $A$ for the observer $\text{O}$ from the information given by $\text{O}'$ would be given by $x=x'+vt$ and vice versa (\SeeChapter{see section of Classical Mechanics}) such as:
		
		In the relativistic model, we must admit that time $t$ which is related to $x$ is not the same as $t'$ which is related to $x'$, relativity principle oblige (otherwise it would be difficult to explain the invariance of the speed of light)!
		
		We are then led to try to write the above relation as follows:
		
		where $\lambda$ would be a numerical value to be determined from a given algebraic expression. Because to explain the constancy of the speed of light one possibility is that the space must constantly adjust according to our velocity $v$. What is revolutionary hypothesis as we have already mentioned!
	\begin{tcolorbox}[title=Remark,colframe=black,arc=10pt]
	A reader asked us why we could not write the last relation in the following simplified form (using the relation $x = ct$ obtained above) where point $A$ is on the $X$ axis:
	
	The only reason is that later we will introduce a vector (matrix) notation of this result showing the concept of quadrivector (four-vector) and that it is in the first form of writing (this making explicit reference to time) that we can clearly make the concept of space-time emerge.
	\end{tcolorbox}
	Furthermore, if $t\neq t'$, we must also be able to express $t'$ as a function of $t$ and $x$ in a similar way:
	
	Let us summarize the shape of the problem:
	
	with $\lambda$ to be determined and after:
	
	with $a,b$ to be determined.
	We then seek to determine the relation that give us to know the values of the coefficients $a$,$b$ and $\lambda$ that satisfy simultaneously:
	
	Remembering the previous developments and bearing in mind that in our special case $y '= y $and $z' = z$, the last equation becomes:
	
	Let us distribute:
	
	To satisfy the relation:
	
	We must have:
	
	It is easy to solve (2):
	
	We then introduce this result in (1) and (3) and we arrive at:
	
	If we divide (1') by (2'), we get:
	
	and introducing this latter result into the relation:
	
	we obtain the following remarkable result:
	
	That we frequently denote by:
	
	and which we name "\NewTerm{Michelson-Morley factor}\index{Michelson-Morley factor}" with:
	
	Also introducing:
	
	in:
	
	we get:
	
	Let us write now (to comply with the traditional notations in this field):
	
	with therefore the parameter:
	

	\subsubsection{Displacement four-vector}
	We derive the "\NewTerm{Lorentz transformation}\index{Lorentz transformation}" relations to pass from the values measured by $\text{O}'$ to those measured in $\text{O}$ and vice versa:
	
	who have for property to be covariant (that is to say their relations keep the same structure during a change of a Galilean reference system). We see through these relations that the concept of "time" is something individual relating to the movement we have over other (this is the "\NewTerm{proper time}"). This is why it is not possible to define a "common time" between two people moving relatively and that don't know their respective relative speed (and even here we do not take into account the gravity that distorts space-time ... that we will study in the section of General Relativity).
	\begin{tcolorbox}[title=Remark,colframe=black,arc=10pt]
	If $v$ is much smaller than $c$, we fall back on the Galilean transformation as $\gamma\cong 1$ and $v/c^2\cong 0$
	\end{tcolorbox}
	We can also write the last relations in a more useful way (the reader will notice that this time that for all relations the units of all the terms to the left of equality are the same: it is every time a distance!):
	
	Of course the difference is that the fourth dimension being the time coordinate of "space-time" seems at the contrary of the spatial coordinates to have a privileged direction: the "\NewTerm{arrow of time}\index{arrow of time}" (you can not go back to a given moment time given in the reality - as least as far as we know today - when it is possible when we traverse a purely spatial distance). The direction of time is imposed by the second law of thermodynamics as entropy can only increase (\SeeChapter{see section of Thermodynamics}). If this were not the case then all time could already exist and we could travel in time as we travel on distances and therefore the future should be already written (people that believe in destiny like this...) and we could also go back in time.... However, thermodynamics does not give a particular direction to the time ... so if our time has the direction it has today.... it is because our universe was organized at its creation (so it had a low entropy).
	
	By proceeding in a homogeneisation of units to be able to use more modern and generalized maths than just simple algebra we can see that in fact when we travel in time we travel in physical point of view a distance $ct$. But because $c$ and $t$ are measured in arbitrary human being units physicists prefer to put $c=1$ we mathematical development become more complicate rather than changing the definition of time.
	
	Now can put the Lorentz transformations of coordinate and time in the traditional following matrix form (\SeeChapter{see section Linear Algebra}) which defines the "\NewTerm{Lorentz matrix}\index{Lorentz matrix}" or "\NewTerm{Lorentz-Poincare matrix}\index{Lorentz-Poincare matrix}" or "\NewTerm{Lorentz boost}\index{Lorentz boost}":
	
	and reciprocally:
	
	and the reader can very easily control that with the two previous relations we fall back on:
	
	We have also obviously:
	
	In index form the matrix formulation is written:
	
	which can therefore be written in tensor (SeeChapter{see section Tensor Calculus}) form:
	
	\begin{tcolorbox}[title=Remark,colframe=black,arc=10pt]
	We can see the tensor (the matrix) of Lorentz transformation in some books in the condensed form $L(\beta)$ and sometimes $L_\nu^\mu$ or even $\Lambda_\nu^\mu$.
	\end{tcolorbox}
	The vector:
	
	is named "\NewTerm{space-time four-vector}\index{space-time four-vector}" or "\NewTerm{four-vector displacement}\index{four-vector displacement}".
	
	Notice that since:
	
	the transformation by the matrix $L_\nu^\mu$ conserves the norm (Lorentz invariance). In geometric terms it is thus a "\NewTerm{isometry}\index{isometry}" or an invariance of the dot product by Lorentz transformation as the Lorentz transformations are orthogonal (\SeeChapter{see section Vector Calculus}).
	
	Let us prove this explicitly following the request of a reader! We will take again for the proof only a movement along $x$ and we use:
	
	As $y'=y$ and $z'=z$ to simplify the development we will ignore these both components.
	
	We will also put to simply $c=1$ and therefore $v$ is expressed in percent of $c$ and becomes $v=\beta$:
	
	Now we calculate:
	
	and as $\gamma^2(1-\beta^2)=1$ we get indeed:
	
	
	\paragraph{Wave Equation Invariance}\mbox{}\\\\\
	Now that we have determined the Lorentz transformations, we can check whether the wave equation is invariant with respect to the latter (remember that we have proven earlier that it was not invariant under a Galilean transformation!!!).
	
	Starting from the Lorentz transformation written in explicitly:
	
	we calculate the partial derivatives with respect to $x$ and $t$ (the expression after the second equality has already been proven earlier in this section):
	
	These relation can also be written:
	
	Squared:
	
	In the Maxwell's equations, or rather in the propagation equation of the electric or magnetic field in vacuum, we have proven (\SeeChapter{see section Electrodynamics}) that the following operator appeared:
	
	Substituting in it the previous differential expressions:
	
	We therefore have well:
	
	which shows that a Lorentz transformation leaves invariant this operator (Jackpot!). So we got what we were looking for (the wave equation but in the other reference frame)!

	The reader will also have notice that this only works if and only if the wave propagation speed is the speed of light!

	\paragraph{Hypergeometric interpretation}\mbox{}\\\\\
	Now let us come back to our Lorentz transformations. Let us recall that we have restricted ourselves to the special case where the space axes were parallel (what brought us to define the "pure Lorentz transformations"). This special configuration has an interesting geometric property that sometimes many books use.

	Let us see what this is about:
	
	We have seen in the context of the study of the Lorentz transformations of lengths that we had a special transformation (boost) along one axis, ie the $x$-axis, requiring in this case for the other components:
	
	This allows us to  reduce the transformation matrix $L_\nu^\mu$ ($4\times 4$ matrix that we obtained earlier above) to a $2\times 2$ matrix of components $A$, $B$, $C$ and $D$ such that:
	
	We notice that the components $A$, $B$, $C$, $D$ respect by construction the following expressions:
	
	The first relation can be related with the remarkable identity in hyperbolic trigonometry (\SeeChapter{see section Trigonometry}):
	
	And therefore:
	
	and the second relation that there exists $\alpha_2$ such that:
	
	\begin{tcolorbox}[title=Remark,colframe=black,arc=10pt]
	The choice of the "$-$" sign for $B$ and $C$ is useful because as we always have $\beta \geq 0$ (same for $\gamma$ that is strictly positive) it will impose us at the end of the calculations to have $\alpha\geq 0$. Therefore, as $-\gamma\beta\leq 0$ and $\alpha\geq 0$ the only way for $C$ (and also for $B$) to be negative is to put a "$-$" sign.
	\end{tcolorbox}	
	The third then gives the remarkable addition relation:
	
	and therefore the difference $\alpha_1-\alpha_2$ that we will denoted more simply by $\alpha$ is equal zero. Which validate the relations:
	
	The matrix is therefore presented as follow:
	
	This is (by analogy to the classical one), a "\NewTerm{hyperbolic rotation matrix}\index{hyperbolic rotation matrix}". We will not go further on this analogy as it is not used for practical cases study in this book.
	
	Finally, the special Lorentz transformation of velocity $v$ along the $x$-axis can also be written:
	
	which brings us to write:
	
	The dimensionless quantity $\alpha$ is named "\NewTerm{rapidity}\index{rapidity}" by those who use physics in high energy. The advantage of working with angles is to make the combination of $2$ boosts easier.

	We will stop here regarding the geometric study of Special Relativity finding personally that it has less and less interest to proceed so today (even it is quite funny).
	
	\subsubsection{Velocity four-vector}
	We can also determine the Lorentz transformations for speed. Let us consider again a particle moving in an inertial reference frame O' such that at time $t'$, its coordinates are $(x ', y', z ').$:
	\begin{figure}[H]
		\begin{center}
		\includegraphics{img/cosmology/lorentz_pure_transformations_experiment.jpg}
		\end{center}	
	\end{figure}
	Therefore, the components of the velocity $v'$ are:
	
	So what are the components of the velocity in O (remember that $O'$ go away at speed $v$!)?
	
	Again, we write:
	
	We can differentiate by the time the components of the transformation equations we obtained before and thus we can write:
	
	Therefore we have:
	
	and also same:
	
	and:
	and also same:
	
	And as the constant speed of reference frame $O'$ is given by $\beta=v/c$, we then have:
	
	and vice versa:
	
	Within the limit of classical mechanics, where the speed of light was supposed instantaneous and therefore $c\rightarrow$, we fall back on:
	
	which are the Galilean transformations such as we have seen them in the section of Classical Mechanics.

	As we can see, the speeds transformations do not follow too much the shape of the Lorentz matrix that we determined above for the coordinates. Physicists, not liking what is inhomogeneous, sought to have the same transformations for both.

	Thus, let us take again the speed transformations and let us rewrite rewrite them as below:
	
	These relation can be written differently if we calculate:
	
	Thus simplifying a bit:
	
	Let us put:
	
	and:
	
	and:
	
	where the latter equality means that in order to simplify that the inertial speed and thus the study of only a single component is sufficient and that is the one collinear with the axis of movement.

	With this notation and simplifying it will be easy to determine the temporal component, indeed the relation:
	
	is the written:
	
	The reader will have perhaps notice that we therefore have three $\Gamma$: one related to the inertial speed, the second related the norm of the vector of the particle in the reference frame O and the third related to the norm of the vector in the reference frame O'. But actually following our simplification made earlier above we know that in the repository O' the particle is at the origin in $Y'$ and $Z'$.

	By doing the same for each of the spatial components, we will get in the end:
	
	and here we have reached our goal of homogenization that allows us to write if we put:
	
	the following system:
	
	that is written in tensor form  sometimes as:
	
	The vector:
	
	is itself named the "\NewTerm{four-vector velocity}\index{four-vector velocity}".
	
	\subsubsection{Current four-vector}
	We have defined naturally during our introduction of the electromagnetic tensor field (\SeeChapter{see section Electrodynamics}) the four-vector current:
	
	that we can write:
	
	This means that charge density is related to time, while current density is related to space.
	
	Therefore, considering $\rho_0$ as the charge density in the proper frame moving with velocity $v$ relative to reference frame O' and due to length contraction in the direction of the velocity, the volume occupied by a given load will be multiplied by the factor $\gamma(\vec{v})$ so that:
	
	which is none other than the "\NewTerm{four-current}\index{four-current}" where we see back the four-vector velocity previously determined.
	
	\subsubsection{Acceleration four-vector}
	Having previously obtained a four-vector velocity transformable thanks to the Lorentz matrix let us also look for the equivalent for acceleration.

	The four-vector acceleration is naturally expressed as the derivative with respect to the proper time of the four-velocity $u$ such that:
	
	Let us just recall that the proper time of a particle is the time measured in the coordinate system of the particle, that is to say, in the reference frame where it is motionless. The proper time in the literature is often denoted $\tau$.
	\begin{tcolorbox}[title=Remark,colframe=black,arc=10pt]
	We must be careful and check that the corollary of the assumption of the equivalence principle is true otherwise all General Relativity would collapse (in the early 21st century experiments are still going to try to show a default to this principle)!
	\end{tcolorbox}	
	The reader must first admit that (we will prove this further below) that:
	
	Therefore, we have:
	
	If we introduce the ordinary acceleration $\vec{a}=\mathrm{d}\vec{v}/\mathrm{d}t$ we see that:
	
	then:
	
	Using the vector identity (\SeeChapter{see section Vector Calculus}):
	
	we then find that the four-vector acceleration can be written:
	
	
	The vector:
	
	is named "\NewTerm{four-vector acceleration}\index{four-vector acceleration}" and therefore also transformed using the Lorentz matrix.
	
	We see that if this $v\ll c$ and $\vec{a}=\vec{a}_0$ the last relationship simplifies to:
	
	We thus fall back on the classic acceleration.

	Using the Minkowski metric (see definition further below), denoted $\nu_{uv}$, let us calculate the norm of the four-vector acceleration:
	
	\begin{tcolorbox}[title=Remark,colframe=black,arc=10pt]
	It must well understood that when we write $(\vec{a}+\vec{\beta}\times(\vec{\beta}\times\vec{a}))^2$ it is implicit in this case that we do sum of the squares of the components of the calculations in the brackets.
	\end{tcolorbox}	
	And as:
	
	and:
	
	we put this together to get:
	
	Now we develop the sum $a_ia^i$ of big parenthesis that becomes therefore:
	
	We simplify:
	
	Hence:
	
	But we have the relation:
	
	and the property of the cross product:
	
	Which finally gives us:
	
	Now imagine an object with a uniformly accelerated relative motion $\vec{}_0^2$ (constant acceleration) in our own repository. If we assume our repository as fixed, we have $\vec{v}=\vec{0}\Leftrightarrow \vec{\beta}=\vec{0}$. Therefore:
	
	Verbatim after rearranging the terms and taking the square root if the accelerated motion is made only along a single component:
	
	But, we also have:
	
	So finally, we can write:
	
	Which after integration gives:
	
	We see that the speed $u$ never reaches $c$ while the force (acceleration implicity) is always the same!

	So we have:
	
	which gives us:
	
	After rearranging, we write this:
	
	We are far from the relation of uniformly accelerated motion we have proved in the section of Classical Mechanics and that is for recall:
	
	However, for $t$ close to zero, we fall back on the same Classical Mechanics relation by taking the Taylor expansion to the second order of the square root (\SeeChapter{see section Sequences and Series}):
	
	However, this does not give us the relations of transformation of acceleration components in a simple form. Let's see how to get them.

	First recall that we have obtained for speed:
	
	Then it comes by differentiating:
	
	therefore:
	
	Let us recall now that we have proved that:
	
	differentiating it comes:
	
	We can write:
	
	After simplifying and rearranging we get obviously:
	
	Hence:
	
	hence finally:
	
	and for the components $y$, $z$:
	
	and therefore:
	
	So finally:
	So finally:
	
	Remember that these relations apply when the movements of the reference frames are in uniform translation!
	
	\subsubsection{Relativistic sum of velocities}
	As the speed of light is a speed supposed unsurpassable, we now come to ask ourselves what will be finally the speed of an object launched at a speed close to that of light (for example...) from a reference frame moving also close to that of the speed of light (why not...).

	We must then find a relationship that gives the real speed $V$ from the launch speed $v_2$ and speed of the repository $v_1$.

	We know that for the object launched:
	
	As the one who is concerned does not know the real speed $V$, it should use the Lorentz transformations. Thus, given the expression of $t'$ that we saw earlier it comes:
	
	and given the prior-previous expression of $x'$ we also have:
	
	therefore after rearranging and simplifying a bit:
	
	Hence:
	
	We know that $v=x/t$ so we finally the "\NewTerm{law of compositions relativistic speeds}\index{law of compositions relativistic speeds}" or simply "\NewTerm{velocity-addition formula}\index{velocity-addition formula}" or "\NewTerm{Einstein's velocity addition}\index{Einstein's velocity addition}" relation:
	
	which is then the speed of a moving body in the moving reference frame relatively to that seen as the rest frame (but that in fact should also move at any speed less then $c$).

	And conversely seen from the other moving frame of reference, we have by the same developments (with reverse signs and speed of course):
	
	which is the speed of a moving body in the rest frame relatively to that considered as being in motion (or in other words seen by the moving frame of reference).
	
	\subsubsection{Relativistic lengths variation (length contraction)}
	Let us consider now that the length of an object is given by the distance between its two ends $A$ and $B$. Let us consider this object $\overline{AB}$ motionless in the repository $\text{O}'$ in uniform translation and oriented along the axis $\text{O}'X'$:
	\begin{figure}[H]
		\begin{center}
		\includegraphics{img/cosmology/lorentz_pure_transformations_experiment.jpg}
		\end{center}	
	\end{figure}
	Its length is then the distance between its both ends:
	
	For the observer O, the object is moving. The positions $A$ and $B$ should therefore be measured simultaneously:
	
	So it comes using the relation proved earlier in this section:
	
	the following difference:
	
	hence the remarkable result:
	
	we also find the relation frequently in the literature as follows:
	
	Thus, the length of an observed rule in a moving frame relatively to the proper frame of the rule is less than its own length (which can assimilate in generality to a "\NewTerm{proper length}\index{proper length}"). In other words, the length of a moving object measured by the fixed reference frame will be measured shorter than its real proper size. This phenomenon is named "\NewTerm{length contraction}\index{length contraction}".
	\begin{figure}[H]
		\begin{center}
		\includegraphics[scale=0.95]{img/cosmology/start_trek.jpg}
		\end{center}	
		\caption{Length contraction principle for straight motion (source:?)}
	\end{figure}
	Due to superficial application of the contraction formula some paradoxes can occur. Examples are the ladder paradox and Bell's spaceship paradox. However, those paradoxes can simply be solved by a correct application of relativity of simultaneity. Another famous paradox is the Ehrenfest paradox (high relativistic speed "rigid" rotating disc\footnote{Circumference of a rotating disk should contract but not the radius, as radius is perpendicular to the direction of motion.}), which proves that the concept of rigid bodies is not compatible with relativity, reducing the applicability of Born rigidity, and showing that for a co-rotating observer the geometry is in fact non-euclidean and the we need then to use General Relativity.
	
	\subsubsection{Relativistic time variation (time dilatation)}
	An event is a phenomenon that occurs in a given place at a given time. The origin of time is difficult to determine, we often prefer to define the concept of time interval as the time elapsed between two events as it is often customary (\SeeChapter{see section Principia}).
	
	Let us now consider two consecutive events $A$ and $B$ that occur at the same location $x'$ (!) in the repository in uniform translation:
	\begin{figure}[H]
		\begin{center}
		\includegraphics{img/cosmology/lorentz_pure_transformations_experiment.jpg}
		\end{center}	
	\end{figure}
	For the observer in $\text{O}'$, the time interval is simply:
	
	To measure this time interval, the observer O in the fixed reference repository should also require that $x'$ is common to both events. Then using the relation proved earlier above:
	
	we get:
	
	hence the remarkable result:
	
	what we write under traditional condensed form:
	
	We also deduce taking an infinitesimal time element:
	
	So the observer O (stationary) measures a time interval much larger than that one measured in the moving repository where the phenomenon takes place as it moves quickly. The time in the fixed repository (thus the "\NewTerm{proper time}\index{proper time}" of the fixed reference frame!) seems like dilated compared to that in occurring in the mobile reference frame (that is to say relatively to the "proper time" of mobile reference frame!).
	
	Let us see two application examples that are so famous that they have their even a name so that we will consider them as an table of contents entry of our book:
	
	\paragraph{Hafele–Keating experiment}\mbox{}\\\\\
	In 1971, direct experimental verification of time dilation was performed. Two airplanes in whose had been placed a cesium atomic clock during their regular commercial flights (one flying to the east, the other to west) compared their clocks to a third  atomic clock remained on the ground. This experiment made famous by time is named today "\NewTerm{Hafele-Keating experiment}\index{Hafele-Keating experiment}" ( Joseph C. Hafele, a physicist, and Richard E. Keating, an astronomer).
	\begin{figure}[H]
		\begin{center}
		\includegraphics{img/cosmology/hafele_keating_experiment.jpg}
		\end{center}
	\end{figure}
	Because the Hafele–Keating experiment has been reproduced by increasingly accurate methods, there has been a consensus among physicists since at least the 1970s that the relativistic predictions of gravitational and kinematic effects on time have been conclusively verified.[7] Criticisms of the experiment did not address the subsequent verification of the result by more accurate methods, and have been shown to be in error.

	The idea is obviously that in a frame of reference at rest with respect to the center of the Earth, a clock aboard the plane moving eastward, in the direction of the Earth's rotation, has a greater velocity (resulting in a relative time loss) than one that remained on the ground, while a clock aboard the plane moving westward, against the Earth's rotation, had a lower velocity than one on the ground.
	
	The plane flying eastward lost $59$ [ns] while the plane flying westward gained $273$ [ns] (the Earth rotates on itself in a day, from West to East. It was therefore measured a total difference:
	
	between the two clocks and this difference is even statistically significantly greater than that the one implied by Special Relativity (see detailed calculations just below).

	Let us analyze the experience considering that all repositories are inertial (thus eliminating General Relativity).
	\begin{tcolorbox}[title=Remark,colframe=black,arc=10pt]
	Strictly speaking, the effect of General Relativity (slowing of clocks depending on the altitude in accordance with Einstein's effect proved in the section of General Relativity) is absolutely not negligible since it is equivalent in amplitude that of Special Relativity.
	\end{tcolorbox}
	Let us consider for our study  three inertial reference points, one at the North Pole, one on Earth (elsewhere apart from the North Pole in the idea!) and one in a plane. The time intervals $t_{\text{North}},t_{\text{Earth}}$ and $t_{\text{plane}}$ respectively (which we will abbreviated $t_N,t_E,t_P$ for the following developments), are connected by the previously proven relations (so the North pole is taken as the reference at rest in this experience and therefore the reference proper time!):
	
	where we have:
	
	The repository on Earth and in the plane so have equation relative speeds $v_E$ and $v_P$ relative to the North Pole. The time by plane and on Earth are then linked by:
	
	We gonna rewrite this relation:
	
	We will accept the following approximation:
	
	where we have assumed that at the denominator:
	
	For square roots whose value is anyway close to $1$ (since $c$ is much larger than the considered relative speeds), we can do a Maclaurin expansion to the second order as $x$ approaches zero (\SeeChapter{see section Sequences and Series}):
	
	Then we can write:
	
	Thanks to these tricky successive approximations, we can easily write the difference between the two clocks that is then:
	
	According to the initial assumptions, the cruising speed of the two aircraft from the ground is constant and is denoted $v$. The speed of each plane (!nonrelativistic according to the preceding approximations) is then:
	
	for the plane going eastwards and:
	
	for the aircraft going respectively westward. So:
	
	We will consider that (it's pretty rough ...):
	
	So it remains:
	
	We see well that obviously with the previous approximations we lost the asymmetry of time dilatation between East and West. The reader that this should disturb can then apply directly the numerical values in the prior previous relation.

	The previous result that we get we all successive approximations already lead us to see formally and quickly that the sign of the result will be in agreement with experimental results.

	For a practical numerical application, we will take the constant speed of the commercial planes at that time that was:
	
	and the total time travel of planes was of $41$ hours according to the measurement at the ground, thus:
	
	and a point at equator of the Earth's surface go at the speed:
	
	where the Earth's radius being of $6,371$ [km] (this suppose that the plans are above the equator radius). We then have applying all that numerical values:
	
	which leads to a result very close to the measurement that was performed.

	And using directly the non-approximate version:
	
	where we took this time the speed of the Earth at latitude consistent with the experience in 1971:
	
	So we see that the result is therefore not very consistent with the experience! Indeed, we must now consider in that approximated case the time dilatation due to gravity. We'll have to use the Einstein's effect relation proved in the section of General Relativity for approximated locally flat space:
	
	which expresses for recall the that at the ground time flows slower than the time at altitude $h$.
	
	According to the records of the experiment, the aircraft flew at $10,000$ [m] above sea level. What gives (the acceleration $g$ is not the same on the ground level than in altitude for recall!) and acceleration of time of:
	
	But, we see that the two aircraft were both at the same height, we always have:
	
	So either there are other effects, of the order of General Relativity, which should be taken into account to explain the $67$ [ns] of difference to the experience, or it is a accuracy problem of the time accuracy of the time clock at the time of the experiment.

	In fact, we will see a detailed study of this experience in the section of General Relativity and see that the theoretical values are in very good agreement with experimental results.
	
	\paragraph{Twins paradox}\mbox{}\\\\\
	The twin paradox is a thought experiment in special relativity involving identical twins, one of whom makes a journey into space in a high-speed rocket and returns home to find that the twin who remained on Earth has aged more. This result appears puzzling because each twin sees the other twin as moving, and so, according to an incorrect and naive application of time dilation and the principle of relativity, each should paradoxically find the other to have aged more slowly. However, this scenario can be resolved within the standard framework of special relativity: the traveling twin's trajectory involves two different inertial frames, one for the outbound journey and one for the inbound journey, and so there is no symmetry between the space-time paths of the two twins. Therefore, the twin paradox is not a paradox in the sense of a logical contradiction.
	\begin{figure}[H]
		\begin{center}
		\includegraphics[scale=0.5]{img/cosmology/twin_paradox.jpg}
		\end{center}
	\end{figure}
	We can already consider the famous twin paradox in the framework of Special Relativity to show that the twin paradox does not only apply to non inertial systems. This is a rough approach (knowing that will rigorously discussed the subject in the section of General Relativity).

	Let us consider a rocket taking off at time $t$ zero of the Earth and accelerating to $20$ times the acceleration of Earth's gravity $g$ up to a cruising speed of $90\%$ the speed of light $c$. Let us suppose that the rocket continues at this speed during a terrestrial year and decelerate with the same intensity to resume its journey to Earth and accelerates again for its approach to the Earth and decelerate once again to its final zero velocity.
	
	
	Thus, the total proper time spent for a human remained on Earth is:
	
	For the traveler in the rocket, the proper time during the acceleration phase will be given roughly by:
	
	Thus by integrating (using the usual primitive proved in the section of Differential and Integral Calculus) for one of the phase of acceleration of the rocket it gives:
	
	And the proper time for the part with the constant cruising speed:
	
	And therefore the total proper time in the rocket is:
	
	So compared to the person remained on Earth, the one that was in the rocket has aged about half !!! This is a paradox (rather a "sophism" in reality) because we can not accurately apply Special Relativity to non-inertial frames. Nevertheless, even with General Relativity, there is a time difference!
	
	\subsubsection{Apparent relativistic mass}
	First the reader must take care (!!) the title is misleading by tradition! We will see why a little further below.

	Meanwhile, imagine a frontal collision between two identical objects $(1)$ and $(2)$ having in the repository $R_0$ equal but opposite speeds. We will assume that the collision is elastic, that is to say that the kinetic energy and momentum are conserved.

	Before the shock (collision), the components of objects speeds $(1)$ and $(2)$ are:
	
	as shown below:
	\begin{figure}[H]
		\begin{center}
		\includegraphics[scale=1]{img/cosmology/relativistic_mass_collision_01.jpg}
		\caption{Configuration for the study of the apparent relativistic mass variation seen from $R_0$}
		\end{center}
	\end{figure}
	After the collision, we have:
	
	We will now apply the following Lorentz transformation:
	\begin{itemize}
		\item We give ourselves another repository $R$ and assume that the repositories $R_0$ and $R$ are in uniform translation speed $u_1$ along the $x$-axis in the positive direction (that is to say in the same direction and at the same horizontal speed than the particle $(1)$).
		
		\item For our particle $(1)$ its trajectory became such is present not visible speed anymore along the $x$-axis.
	\end{itemize}
	Let's go! Let us place ourselves in repository $R$ that moves relative to $R_0$ with the speed $u_1$ following the $x$-axis, the components of the speeds in this repository are ten before the collision:
	
	and after the collision:
	
	\begin{figure}[H]
		\begin{center}
		\includegraphics[scale=1]{img/cosmology/relativistic_mass_collision_02.jpg}
		\caption{Configuration for the study of the apparent relativistic mass variation seen from $R$}
		\end{center}
	\end{figure}
	So we have trivially in the reference frame $R$:
	
	but by applying the law of composition of speeds proved earlier above:
	
	for the components of the horizontal axis we always have in the reference frame $R$
	
	and for the vertical movement, we have earlier above that:
	
	Therefore we get:
	
	Passing from $R_0$ to $R$, the component following $y$ of the total momentum must remain zero (as it was the case in $R_0$ initially). But:
	
	To break this deadlock, we must admit that the respective apparent masses $m_1$ and $m_2$ may not be identical in $R$. So that brings us to request that:
	
	which leads us to:
	
	In $R$, the square of the norm of the two objects speeds gives:
	
	The last relation can be written:
	
	so that after rearrangement and factorization we get:
	
	Therefore:
	
	We thus found:
	
	In the case as assumed above where both object are identical we will put $m_1=m_2=m_0$ and therefore:
	
	And we will put them as an apparent relativistic denoted simply by $m$ such that:
	
	And as $V_1^2$ and $U_2^2+V_2^2$ are simply the square norm of the velocity, we can write:
	
	So that finally:
	
	So we see that when $v=0$, we have $m=m_0$ this is why we name $m_0$ the "\NewTerm{rest mass}" or "\NewTerm{invariant mass}\index{invariant mass}".
	
	Since the mass is a function of $v$ (at least in apparence), some physicists note the rest mass as a function, that is to say: $m(0)$. But it is rather more common to use the $m_0$ to not have too much in parentheses in developments...
	
	As the Michelson-Morley factor $\gamma$ tends to infinity when the speed $v$ approaches the speed $c$ of light in a vacuum we have an additional reason to say that $c$ is the upper limit assigned to the speed of any material object otherwise the apparent mass $m$ would be infinite also, which is consistent with both the experience and the consequences already formulated by the Lorentz transformations!!!
	
	It already follows an important conclusion: there are therefore two types of particles, those with a mass and will never go to the speed of light (as it then takes an infinite energy to get them there following our previous result), and those having a zero mass and which will therefore necessarily be at the speed of light.
	
	As we will see it in the section of Quantum Field Theory interaction forces are short-range precisely because of the uncertainty principle and of the above statement. The greater the distance is between large particles that interacts together, the more time will be longer and therefore the small will be the energy involved. But in the case where the particle of interaction have no mass, the "force" is a long range one.
	
	\pagebreak
	\paragraph{Mass–energy equivalence}\mbox{}\\\\\
	Under the action of a force $F$, the speed of a mass $m$ increases or decreases on each portion of the trajectory. The work of the component $F\mathrm{d}x$  can then be interpreted into kinetic energy $\mathrm{d}E_c$ ..

	In the relativistic theory, the mass varies with speed as we have just prove it, therefore:
	
	The integration by parts (\SeeChapter{see section Differential and Integral Calculus}):
	
	give us:
	
	The gain of kinetic energy of a particle can be considered as gain in its apparent mass. Since $m_0$ is the rest mass, the quantity $m_0c^2$ is named "\NewTerm{rest energy}\index{rest energy}" of the particle.

	We then have:
	
	where $E_c$ represents the energy of motion (kinetic energy).

	The sum of:
	
	therefore represents the total energy $E$ of the particle in the absence of the potential field. Which brings us to write:
	
	And therefore:
	
	\begin{figure}[H]
		\begin{center}
		\includegraphics[scale=0.6]{img/cosmology/eintein_trial.jpg}
		\end{center}
	\end{figure}
	Finally we could also have the same result in another way using lagrangien mechanics (\SeeChapter{see section Analytical Mechanics}):
	
	\paragraph{Relativistic Lagrangian}\mbox{}\\\\\
	The following developments will help us in the study of Electrodynamics (if this section has not been read yet), to determine the expression of the tensor of the electromagnetic field and in Relativistic Quantum Physics to determine the Klein-Gordon equation with magnetic field. So be sure to carefully read what follows.

	In special relativity, so we want the equations of motion have the same form in all inertial frames. For this, we need the action $S$ (\SeeChapter{see section Analytical Mechanics}) to be invariant with respect to Lorentz transformations. Guided by this principle, trying to get the action of a free particle. Suppose that the action is in the reference frame O':
	
	\begin{tcolorbox}[title=Remarks,colframe=black,arc=10pt]
	\textbf{R1.} The choice of the minus sign will be evident in our study of electrodynamics.\\
	
	\textbf{R2.} The notation $L_0$ instead of the $L$ of Lagrangian lets just emphasize that this is a case study where the system is free. This distinction of notation will be useful in our study of General Relativity and determination of the Tensor of the Electromagnetic field in the section of Electrodynamics.\\
	
	\textbf{R3.} We are not supposed to know what kind of mass we are dealing (inertial rest mass), hence the fact that in the ignorance, we will work with the inertial mass $m$ to perhaps correct this hypothesis later if necessary.
	\end{tcolorbox}
	And let us recall that:
	
	In the repository O, then we have the "\NewTerm{Lorentz invariant action}\index{Lorentz invariant action}":
	
	So according to our initial hypothesis, we have for the relativistic Lagrangian (in the absence of potential field... since the system is assumed to be "free"):
	
	In the non-relativistic approximation $v\ll c$, we have following the Maclaurin development of the square root (\SeeChapter{see section Sequences and Series}):
	
	We thus fall back on the usual Lagrangian of a free system in movement but more a constant $(-mc^2)$ that does not affect the equations of motion we get in Classical Mechanics but that will be absolutely necessary in to us in Electrodynamics.

	Let us recall now that the generalized momentum (\SeeChapter{see section Analytical Mechanics}) is defined by:
	
	We will now see that this definition is not accidental. Indeed:
	
	The Hamiltonian (\SeeChapter{see section Analytical Mechanics}) is equal to:
	
	Which gives:
	
	The Hamiltonian is in this case equal to the total energy of the particle. Its expression led us to change somewhat our initial hypothesis and finally to write $m_0$ instead of $m$ in the expression of the action $S$.

	So we finally we have:
	
	and:
	
	In the non-relativistic approximation $v\ll c$, $H_0$ becomes with a Maclaurin development (\SeeChapter{see section Sequences And Series}):
	
	We recognize the usual kinetic energy, plus a constant: the energy at rest. Which corresponds to the calculations we had made before where we got:
	
	
	\paragraph{Relativistic (linear) momentum}\mbox{}\\\\\
	The total energy $E$ and the (linear) momentum $p=mv$ of a particle can therefore take any positive value (when the speed approaches the limit value $c$, the apparent mass suits for the product $p=mv$ to not be bounded) .

	In the expression of $E$, we can replace the speed $v^2$ by a function $p^2$:
	
	introduced into:
	
	we have:
	
	Therefore:
	
	hence (we will come back on that relation of the utmost importance during our our proof of Einstein relation):
	
	We have not kept the negative part of the previous relation as it has no meaning in classical physics. However, when we will study Relativistic Quantum Physics, it will be essential to preserve it otherwise we will get absurdities.

	However, we can obviously write this last relation also in the following form named "\NewTerm{relativistic mass momentum relation}\index{relativistic mass momentum relation}":
	
	or also (ugly!):
	
	In other words, the total energy of a moving particle is equal to its mass energy added to its kinetic energy (basically nothing new).
	
	The relation above has two limit cases where we can simplify it:
	\begin{enumerate}
		\item For a particle at rest ($p = 0$), we can reduce the expression to:
		
		by omitting the negative energy ... at least for now.

		\item We can apply the equation to a particle without mass to eliminate the first term, which then gives us:
		
		A photon, for example, has a zero rest mass but it is never at rest ...by definition, it is a quantum of energy, kinetic energy is never zero and so it has a mass corresponding to its kinetic energy. Thus, a massless particle at rest moves at the speed of light, regardless of the chosen repository frame! Conversely, a particle with a non-zero rest mass can never reach the speed of light in any repository.
	\end{enumerate}
	\begin{tcolorbox}[title=Remarks,colframe=black,arc=10pt]
	\textbf{R1.} As we prove it further below (see the "Einstein relation"), from the construction of Planck's law (\SeeChapter{see section Thermodynamics}), we can write $E=pc=hv$.\\
	
	\textbf{R2.} The mass of the photon can hardly be non-zero! Indeed, quantum theory would be false otherwise. But it has never fail until now (\SeeChapter{see section Wave Quantum Physics}). We would also have a small change on th Electrostatic force following that it is given by the Yukawa potential (\SeeChapter{see section Quantum Field Theory}) and this would have been observed in laboratory since...
	\end{tcolorbox}
	Let us now look after the relations between $p$ and $p'$ and between $E$ and $E'$, to make it possible for O' to write:
	
	We then begin to get rid of the square root:
	
	If O write:
	
	O' must be able to write:
	
	Therefore we have:
	
	If we identify:
	
	we obtain similar expressions to those used for the Lorentz transformations of spatial and temporal components. We can then write, by similarity, that the changes to the (linear) momentum and energy are therefore given by:
	
	Again, if we take:
	

	We therefore have by expressing all previous relations of transformation in the same units by remembering that $E\equiv pc$ (for a photon!):
	
	We can the define a matrix such that:
	
	where we fall back on the "Lorentz matrix" or "symmetric Lorentz tensor" ${L'}_\mu^\nu$

	The vector:
	
	is meanwhile, named the "\NewTerm{four-vector energy-momentum}\index{four-vector energy-momentum}" or just "\NewTerm{four-momentu}\index{four-momentum}" Its utility is that its value is also conserved and this is especially useful for the study of nuclear reactions. If we add these vectors on all particles (without forgetting the photons as well!!!) before and after the reaction, we should found the same quantities for the $4$ components!
	\begin{tcolorbox}[title=Remarks,colframe=black,arc=10pt]
	\textbf{R1.} The inverse transformation being done obviously with the inverse matrix that we have already outlined earlier above.\\
	
	\textbf{R2.} We use in Relativistic Optistics the four-vector $(\omega/c,\vec{k})$, where $\omega$ is for recall the pulsation of the wave and $\vec{k}$ the wave vector (\SeeChapter{see sections Wave Mechanics and Wave Optics}). This four-vector is the equivalent for an electromagnetic wave of the four-vector $(E/c,\vec{p})$ for a particle multiplied by the reduced Planck's constant $\hbar=h/2\pi$. Indeed, the wave-particle duality (\SeeChapter{see section Wave Quantum Physics}) attributes to a wave an energy:
	
	and a linear momentum which norm is:
	
	\end{tcolorbox}
	Now let us come back on the following relation is central in some areas of quantum physics:
	
	Therefore:
	
	Which can be written in vector form (very common form):
	
	This latter relation will be very useful in the section of Relativistic Quantum Physics to calculate the energy of virtual photons exchange.

	For photons, since the mass is zero, we have:
	
	Finally let us also notice that the four-momentum is also related to a another quantity name "\NewTerm{four-wave vector}\index{four-wave vector}" as following:
	
	as we know that (for the first component):
	
	and that for all other components that as:
	
	
	\subparagraph{Einstein relation}\mbox{}\\\\\
	Following the principle of relativity, we whish that the relation between the linear momentum and energy of an electromagnetic wave can be written in the same way for two inertial observers in translation relative to the other:

	If O writes:
	
	then O' must be able to write:
	
	Let the take the first relation above and put it to the square without forgetting that the photon has a zero rest mass $m_0$. Therefore:
	
	and as $m_0=0$ for the photon:
	
	Given the known Planck relation (\SeeChapter{see section Thermodynamics}):
	
	we are led to write the famous "\NewTerm{Einstein's relation}\index{Einstein's relation}" that we will find very often in Quantum Physics and in Thermodynamics:
		
	So even if the photo has no mass at rest, it has a "\NewTerm{relativistic mass}\index{relativistic mass}". This relativist mass is in Special Relativity for the photon the equivalent to the electric charge in Electrodynamics.
	
	\subparagraph{Time of flight}\mbox{}\\\\\
	Suppose we get a beam made of massive particles. The rest mass  is $m_0$. The particle travels a distance $L$ in its inertial frame. The particle has an energy $E$ in that frame. Therefore, the so-called "\NewTerm{time of flight}\index{time of flight}” from two points at $x=0$ and $x=L$ will be:
	
 	where we use the relativistic definition of momentum:
	
	Now, knowing that the relativistic energy is:
	
	using the time that a massless light beam uses to travel the proper distance $L$, easily calculated to be:
	
	 we get:
 	
	But we have just proved that:
	
 	Therefore:
 	
	and then:
 	
	Thus, the time of flight is finally written as follows:
 	
	This equation is very important in practical applications. Specially in Astrophysics and baseline beam experiments, like those involving the neutrinos! Indeed, we usually calculate the difference between the photon (or any other massless) time of arrival and that of massive particles, e.g. the neutrinos. Several neutrino experiments can measure this difference using a well designed experimental set-up. The difference between those times of flight (the neutrino time of flight minus the photon time of flight) is:
	
	or equivalently:
 	
	or as well:
 	
	This last expression can also be expressed in terms of the speed of light and neutrinos, since:
	
	so:
	
 	Therefore:
 	
	In the case of know light left-handed neutrinos, the rest mass is tiny (likely sub-[eV] and next to the [meV] scale), and then we can make a Taylor expansion for $Q$ if:
 	
	Then:
	
	Then, we would expect, accordingly to Special Relativity, of course, that:
	
 	We can guess how large it is plugging "typical" values for the neutrino mass and energy. For instance, taking $m_\nu\cong 1$ [meV] and $E\cong 1$ [GeV]  the $Q$ value is about $10^{-24}$.
 
	Nowadays, we have no clock with this precision, so the neutrino mass measurement using this approach is impossible with current technology. However, it is clear that if we could make clocks with that precision, we would measure the neutrino mass with this "time of flight" procedure. It is a challenge. We can not do that in these times (circa 2012), and thus we don't measure any time delay in baseline experiments. Then, neutrinos move with $v_\nu=c$ and since there is no observed delay (beyond the OPERA result, already corrected), neutrinos are, thus, ultra-relativistic particles, and for them $E=pc$ with great accuracy.
	
	\subparagraph{Relativistic force}\mbox{}\\\\\
	Following the principle of relativity, we want that the relation between force and linear momentum to be written in the same way by two inertial observers in translation relative to each other!

	Therefore if O writes:
	
	O' must be able to write:
	
	The relation between $\vec{F}$ and $\vec{F}'$ is quite complicated in the general case. We will limit ourselves here to the particular case where a body is momentarily at rest in O' and therefore where the observer O' will only take into account the force $\vec{F}'$ that he applies. He will name this the "\NewTerm{proper force}\index{proper force}" because it has not to worry about other forces (such as centrifugal force, for example).

	It is necessary to substitute $p'$ and $t'$ by $p$ and $t$ in:
	
	Since:
	
	we will have:
	
	We have seen also previously that:
	
	Therefore it remains:
	
	The component of the force is therefore invariant in the direction of the movement.

	For the directions $y$ and $z$ perpendicular to the movement:
	
	So for summary:
	
	However, to change from one reference frame to another, it is better to use again the "\NewTerm{four-vector force}\index{four-vector force}" defined as the derivative of the four liner moment vector with respect to the proper time:
	
	Indeed, let us recall that:
	
	
	\pagebreak
	\subsubsection{Relativistic electrodynamics}
	With a mass spectrometer we establish that the ratio $m / q$ of the mass $m$ of a particle by its electrical charge $q$ varies in the same way as the mass $m$ when the velocity $v$ of the particle varies:
	
	Thus, it comes that:
	
	The charge of a particle is therefore independent of its velocity, as we have proved in the section of Electromagnetism (\SeeChapter{see section Electrodynamics}) when determining the charge conservation equation.

	Let us consider now two charges $q$ and $Q$ immobile a reference frame O' in translation at speed $v$ with respect to another one centered on O:
	\begin{figure}[H]
		\centering
		\includegraphics[scale=1]{img/cosmology/electric_field_lorenz_transformation.jpg}
		\caption{Configuration for the study of transformations of electric and magnetic fields}
	\end{figure}
	We will restric ourselves to the case where the velocity $\vec{v}$ is paralle to the O$x$-axis:
	
	and we write the vector at the horizontal to spare time and paper...

	The electric charge $Q$ is place on O' and is therefore fixed for O'. The observer O' makes the conclusions that an electrostatic force:
	
	act on the reference particle $q$ placed on $\vec{r}'$:
	
	The observer O also sees an electrostatic field $\vec{E}$ in $\vec{r}$, but he also sees that $Q$ is in movement along the O$x$-axis. It thus deduces the existence of a magnetic field $\vec{B}$ on $\vec{r}$ oriented in the plane $YZ$ plane (\SeeChapter{see section Electrodynamics}):
	
	It therefore measures the supposedly known Lorentz's force (\SeeChapter{see section Magnetostatics}):
	
	But:
	
	Therefore:
	
	We have now seen:
	
	The comparison of the expressions above gives the relativistic transformations of the electric field:
	
	As for the Lorentz transformation of the spatial and temporal components, we have obtained the inverse transformations by exchanging the fields and considering that O' sees O going back away (we therefore replace $v$ by $-v$).

	The above relations, sometimes named "\NewTerm{Joules-Bernoulli equations of the electric field}\index{Joules-Bernoulli equations of the electric field}", make it clear that if, for example, the electric field in one of the reference systems is zero but the magnetic field is not, then an electric field exists from the point of view of the other reference frame !!! It is therefore an absolute victory of relativity in comparison to classical mechanics!

	To obtain the relativistic transformations of the magnetic field, we proceed as follows:
	To get the relativistic transformations of the magnetic field, we proceed as follows:
	
	After some small manipulations of very elementary algebra, we get:
	
	We do the identically:
	
	After a few simple manipulations of very elementary algebra, we get:
	And so on. Finally, we get:
	
	The above relations, sometimes named "\NewTerm{Joules-Bernoulli equations of the magnetic field}\index{Joules-Bernoulli equations of the magnetic field}", make it clear that if, for example, the magnetic field in one of the reference frame is zero but the electric field is not, then a magnetic field exists from the point of view of the other reference fame !!! It is therefore once again an absolute victory of relativity in comparison to classical mechanics!

	Let us now study the behavior of the electromagnetic field of a moving charge:

	Let us consider two parallel referentials O and O', in translation at constant velocity $v$ along the axis $XX'$:
	\begin{figure}[H]
		\centering
		\includegraphics[scale=1]{img/cosmology/lorentz_configuration_study_for_electromagnetic_transformations.jpg}
		\caption{Configuration for the study of electrodynamic transformations}
	\end{figure}
	where a fixed electric charge $Q$ is placed at O '.

	It is clear that the observer O measures $\vec{B}=\vec{0}$ everywhere and that at the point $P$ of the plane $X 'Y'$, on $\vec{r}'=(x',y',0)$ he measures the electrostatic field (\SeeChapter{see section Electrostatic}):
	
	If the observer O is informed of the values of $\vec{E}'$ and of $\vec{B}'=\vec{0}$, he can introduce them into the relativistic transformation giving the electric field $\vec{E}$ that he observes:
	
	To write an expression of the field $\vec{E}$ at the point $P$, the observer O must determine, at a time $t$ of its local time, the components of the vector $\vec{r}=(x,y,0)$ which separates the point $P$ from the electric charge $Q$ (by summing the position vectors of the latter two material points).

	The coordinates of the point $P$ and of the charge $Q$ that he sees in the plane $XYZ$ are given by the usual Lorentz transformations:
	
	He thus easily deduces, by summation, the distances $x$, $y$.

	Another simpler possible method is that since the $x$ component is a length, it therefore undergoes Lorentz transformations and:
	
	Since for recall:
	
	The relativistic transformation of the electric field then gives:
	
	and:
	
	Written in vector form:
	
	We must also determine how to express $r'$ as a function of $r$:
	
	as (Pythagorean theorem):
	
	The writing is simplified if we use the angle formed by the electric field vector and the $x$-axis. We then denote then $\theta'$ in O' and the $\theta$ in O the angles given by:
	
	with $\theta\geq \theta'$ due to the expansion of the lengths along the $x$-axis.

	We eliminate $y$ with:
	
	Thus, the electric field $\vec{E}$ that sees O is given by:
	
	The factor containing $\sin(\theta)$ shows that the electric field $\vec{E}$ of a moving charge no longer has a spherical symmetry!!! It depends on the direction of the vector $\vec{r}$.

	At equal distances, the electric field is more intense in the vertical direction to that of the displacement ($\theta=\pi/2$) than in the direction of the displacement of the electric charge ($\theta=0$).

	If $v = 0$, we fall back on the classic known expression:
	
	\begin{tcolorbox}[title=Remark,colframe=black,arc=10pt]
	Let us recall that we have carried out (and continue in this sense) here a study of an electric charge in uniform rectilinear motion, that is to say at constant speed!
	\end{tcolorbox}
	To find now the expression of the magnetic field $\vec{B}$, we introduce:
	
	and:
	
	in:
	
	We therefore get:
	
	Which are the components of:
	
	To know $\vec{B}$ as a function of $\vec{r}$, we substitute the expression obtained for $\vec{E}$:
	
	In the case where the velocity is small, the relativistic term tends to $1$ and the field $\vec{B}$ of an electric charge $Q$ moving at the velocity $v$ becomes:
	
	because as we have in the section of Electrodynamics: 
	
	\begin{tcolorbox}[title=Remarks,colframe=black,arc=10pt]
	\textbf{R1.} At each location, the lines of the field $\vec{B}$ are contained in a plane perpendicular to the direction of motion of the electric charge $Q$ (vector product oblige...!).\\
	
	\textbf{R2.} If the moving electric charge is seen as a $\mathrm{d}Q$ attached to the point O', we can interpret its displacement at velocity $v$ as a current $I$ at a point of the referential O where O' is located. Therefore:
	
	Therefore:
	
	We then fall back here the "Biot and Savart law" as proved in the section of Electromagnetism. So this a success of the Special Relativity theory again!!!
	\end{tcolorbox}
	It is interesting to remember that an electric charged particle in motion will be seen in the frame of reference of the particle as emitting no electromagnetic field (there will be just an electrostatic field). This is not the case for a repository at rest. There is thus here a sort of flagrant counter-intuitive contradiction.

	But this poses another problem, in a fast-moving frame of reference, a charged particle normally emits an acceleration radiation (\SeeChapter{see section Electrodynamics}), this radiation in quantum mechanics must necessarily be accompanied by the emission of a quanta, which exists either or does not exist (a medium term does not exist). The very existence of photons would therefore be purely relative. And yet it is! Some particles have only a relative existence!!!! The complicated answer is therefore to know what the photons have become.

	But here we reach the limit of what we master perfectly in the physics of the end of the 20th century, because we speak of accelerated references frames (which implies to be in General Gelativity and not the special one) and quantum field theory . The rigorous framework for dealing with this (which would encompass Quantum Gravitation) does not yet exist as far as we know. But a first step has been taken with the development of the Quantum Field Theory in curved space.
	
	\pagebreak
	\paragraph{Tensor field transformation}\mbox{}\\\\\
	We have seen and proved in the section of Electrodynamics that the whole electromagnetic field was summarized by the tensor of the same name. It would then be good to look at how this tensor transforms itself and if it does so correctly in relation to the results obtained above.

	Let us consider the transformation (where the tensor of the electromagnetic field is in natural units !!!):
	
	with the tensor of the electromagnetic field in contravariant components in the Minkowski metric $+---$:
	
	And also by construction:
	
	Let us take, for example, the velocity parallel to the $x$-axis, then we have proved above that:
	
	Therefore:
	
	where as we can see, it is often customary in the field of Special Relativity and Electrodynamics to number the components of matrices / tensors starting from $0$ (instead of $1$ for most of the other chapters of this book) .

	We calculate the transformation (remember that the tensor of the electromagnetic field is antisymmetric!):
	
	We thus deduce, for the electric field (which corresponds perfectly to what we obtained above):
	
	We make a second calculation for the perpendicular component:
	
	hence:
	
	which again corresponds perfectly to what we had obtained earlier above (in natural units, do not forget that we then have $\beta=v$)!

	The same applies to the magnetic field:
	
	and:
	
	Which gives (in natural units, again do not forget that we then have $\beta=v$)! :
	
	etc.
	
	\subsection{Minkowski space-time}
		We have proved earlier above that:
	
	Let us write this in the form:
	
	Let us multiply the two members by $(ct)^2$:
	
	which gives us:
	
	If $v=c$ the equation vanishes:
	
	This result translates the fact that the dimensions of space and time are as stopped in the relativistic referential, because the relative speed of the object is equal to that of the light!

	Let us now imagine that a light beam is emitted at the instant $t=0$ and propagates from the origin of a referential. We know that in space-time (application of the Pythagoras theorem in three-dimensional Euclidean space for recall...) the distance traveled by the photon is:
	
	By changing $t$ of member and bringing the whole to the square to remove the root, we get:
	
	Therefore:
	
	\begin{tcolorbox}[title=Remark,colframe=black,arc=10pt]
	We can assimilate this relation to the representation of a spherical wavefront of a light wave propagating at the speed of light (see the equation of a sphere originally centered in the section of Analytical Geometry) .
	\end{tcolorbox}
	Let us now consider two coordinate events $(x_1,y_1,z_1,t_1)$ and $(x_2,y_2,z_2,t_2)$ and we denote by $\mathrm{d}s$ the "\NewTerm{space-time abscissa}\index{space-time abscissa}" (or "\NewTerm{proper-distance}\index{proper-distance}"). We can then write the spatio-temporal interval as such:
	
	By passing to the limit, we get the quadratic form:
	
	which has the same shape and value regardless of the reference system considered as we have already proved it earlier above. The infinitesimal interval of space-time $\mathrm{d}s^2$ between two infinitely neighboring events is therefore a relativistic invariant that we often name the "\NewTerm{space-time curvilinear abscissa}\index{space-time curvilinear abscissa}". It is the interval of space-time or, as Albert Einstein simply said, the "square of distance" .... The fact that this magnitude may be positive, negative (!) or zero is linked to the absolute character of the speed of light (we will come back to this later).

	We can also now turn our attention to the relativistic character of this metric. If it is invariant, it must also be invariant by Lorentz transformations. We then say that "the metric is invariant by Lorentz transformation". Such a transformation can be found on the basis of that used for the tensor of the electromagnetic field (see above). The reader will readily verify from the detailed example of the electromagnetic field that for the metric tensor we have the relation (as always we can detail on request if necessary):
	
	The curvilinear abscissa can also be expressed by the norm of the quadrivector displacement which we defined above as $(ct,x,y,z)$. Indeed, the norm (\SeeChapter{see section Tensor Calculus}) is written by taking down the indices using the "\NewTerm{Minkowski metric}\index{Minkowski metric}" $\eta_{\mu\nu}$ or "\NewTerm{pseudo-Riemannian metric}\index{Minkowski metric}":
	
	with the definition of the "\NewTerm{Minkowski's matrix}\index{Minkowski's matrix}" (we will return to this in detail at the beginning of our study of General Relativity):
	
	where as usual in this book we make the abuse of notation (already mentioned in the section of Tensorial Calculation) not to put $\eta_{\mu\nu}$ in brackets (since a tensor and its matrix form are normally two distinct things in rigorously speaking).

	If we put the following two relations in correspondence:
	
	we then have $\mathrm{d}s^2=0$ when the two events are connected to the speed of light.

	Moreover, if we put:
	
	we can then write:
	
	This is nothing more than the equation of a cone (\SeeChapter{see section of Analytical Geometry}) of axis of ordinate $c^2\mathrm{d}t^2$... the famous "\NewTerm{light cone of Universe}\index{light cone of Universe}" (to which we devote a study further below). Every event is therefore by extension in this cone and the evolution of any system can thus be described (by its spatial and temporal position), by what we name its "\NewTerm{line of Universe}\index{line of Universe}" or "\NewTerm{World line}\index{World line}". The Universe line of a particle is therefore the sequence of events it unfolds during its lifetime.
	
	\subsubsection{Four-vectors}
	We have just defined what Minkowski's metric was, we can now correctly define the concept of quadrivector that we have already addressed without always knowing what we were doing.
	
	\textbf{Definition (\#\mydef):} In a four-dimensional space of Minkowski type, the four quantities:
	
	(regardless of the order of terms for this definition or whether the indices are numbers or letters corresponding to the four spatio-temporal components) form a covariant "\NewTerm{four-vector}\index{four-vector}\index{quadrivector}" if they transform following the Lorentz transformation:	
	
	The "\NewTerm{pseudo-norm}\index{quadrivector pseudo-norm}\index{four-vector pseudo-norm}" of a quadrivector in a Minkowski space of metric $\eta_{\mu\nu}$ is then:
	
	where we see that the contravariant four-vector multiplied by the metric returns the contravariant four-vector (\SeeChapter{see section Tensor Calculus}).

	The following quantity being invariant by change of Galilean referential as we proved it almost at the beginning of this section:
	
	This property of invariance by change of Galilean referential of the four-vector is their main property. Thus, two observers in relative motion, which are uniform in relation to each other, must compare the results of the same measure using the norm of the four-vectors. Similarly, the laws they seek to determine to be as general as possible must use these invariant quantities! 
	
	We can also write the norm of a four-vector in the form:
	
	and the four-vector themselves:
	
	So for summary let us give the list of the four-vectors we have determined so far in this section but by standardizing the notations (and only those four-vectors that will be useful for other sections of this book!):
	\begin{itemize}
		\item The space-times four-vector ("four-position"):
		

		\item The velocity four-vector ("four-velocity"):
		
		
		\item The current four-vector ("four-current"):
		
		
		\item The acceleration four-vector ("four-acceleration"):
		
		
		\item The energy-momentum four-vector ("four-momentum"):
		

		\item The gradient four-vector ("four-gradient") introduced in the section of Tensor Calculus:
		
	\end{itemize}
	Obviously we have considered here four-vectors in the context of Special Relativity. Although the concept of four-vectors also extends to General Relativity, some of the results stated above require modification in General Relativity.
	
	\subsubsection{Universe light cone}
	The topology of the light cone has its origin in the relations of anteriority and posteriority of relativistic events, which makes it possible to distinguish between an event in the past of another or in the future of it.

	The principal objective of the light cones in the popularization works of theoretical physics is to map out the history of light pulses emitted at a point in the space where certain conditions may prevail. The points are represented in space by a series of snapshots at various times $t_1$, $t_2$, $t_3$, etc. (see figure below), the spherical wave front of the light magnifying in space. In space-time, the same event (at the bottom on the figure) is represented by a "\NewTerm{light cone}\index{light cone}", whose apex is the point of emission.

	On a sheet of paper, we have to remove one of the spatial dimensions. The spatial axes are drawn in the horizontal plane and the time axis directed upwards. The cone sections at the instants $t_1$, $t_2$, $t_3$ correspond to the snapshots of the spatial representation: the two-dimensional wavefronts are circles whose radius is that of the spherical wavefront at the instant considered. The light cone shows in a single diagram the continuous history of the wavefront of a light signal.
	\begin{figure}[H]
		\centering
		\includegraphics{img/cosmology/light_cone.jpg}
		\caption{Idea of light-cone}	
	\end{figure}
	More precisely, the "snapshots" mentioned above are named "\NewTerm{punctual events}" and these appear instantaneous (approximation based on geometric optics) to any observer capable seeing them. A collision between two point particles provides an example of a punctual event. It is quite possible that a non-punctual instantaneous event appears instantaneous to a certain observer but, because of the finite propagation velocity of the light, not instantaneous to another observer.

	\textbf{Definitions (\#\mydef):}
	\begin{itemize}
		\item[D1.] Two punctual events occupy the same "\NewTerm{time-space point}\index{time-space point}" if they appear simultaneously to any observer able to see them.

		\item[D2.] The set $M$ of all points of space-time is named the "\NewTerm{space-time}\index{space-time}".

		\item[D3.] The boundary defined by the Universe cone is named the "\NewTerm{cosmological horizon}\index{cosmological horizon}"
	\end{itemize}
	Let us recall that if no force acts on a point particle, we say from it that is is an "inertial" or "free" particle. We also say that it is in "inertial motion".

	Given the point $p$, $N(p)$ is an absolute geometric structure independent of the observer. Its future component will be denoted $N^{+}(p)$; Its past component $N^{-}(p)$ and it will be represented by the following cone:
	\begin{figure}[H]
		\centering
		\includegraphics{img/cosmology/past_future_light_cone.jpg}
		\caption{Past and future light cones}	
	\end{figure}
	Indeed, let us recall that the Minkowski equation is invariant since:
	
	We have, when we reduced to three parameters (we remove a spatial dimension to simplify the conceptualisation), if the punctual events are related to the speed of light (see earlier above):
	
	What we can also write in the form:
	
	to be compared with the equation of a cone (\SeeChapter{see section Analytical Geometry}):
	
	when we put $c = 1$ (which is frequent in theoretical physics as we have already mentioned many times).

	Therefore the Minkowski equation can be indeed presented by a cone.
	\begin{tcolorbox}[title=Remark,colframe=black,arc=10pt]
	If we would have keep the three spatial parameters and the time interval constant, the reader will then perhaps have notice that we would fall bacl neither on the equation of a cone but on that of a sphere. It is the "\NewTerm{celestial sphere}\index{celestial sphere}" where at a given instant, on its surface, multiple cones of light are created.
	\end{tcolorbox}
	The universe line of any observer which occupies instantly $p$ and whose line of the Universe passes through $p$ itself, is contained within $N(p)$ defined by a single point on its celestial sphere (the one that is described by the information vector - the photon - in all directions of space). This means that there can be, in extenso, as many null rays (foci of cones) passing through $p$ as points on a sphere.

	The following example will (we hope) appear more obvious:
	\begin{figure}[H]
		\centering
		\includegraphics{img/cosmology/light_cone_associated_universe_line.jpg}
		\caption{Universe Line Principle with its associated Cone}	
	\end{figure}
	As illustrated in the figure above, a light event at the point O of the space-time produces a beam of photons, all in the zero cone of the future O, $N^{+}(\text{O})$ (these photons have been emitted by atoms in various states of movements whose universe lines $l$ and $l'$ pass through O, but are entirely contained within the $N^{+}(\text{O})$). The universe line $n$ can only be described by a particle moving at the speed of light because it defines the boundary of the cone (we then say that the line of the Universe is of "light type").
	
	\begin{tcolorbox}[title=Remark,colframe=black,arc=10pt]
	The representation of the Universe lines in the lower part (inverted cone) comes from the fact that an event can also have a past... so the scheme generalizes the particular example.
	\end{tcolorbox}	
	Let $l_p$ be the universe line of a stationary person $P$ (hence the verticality of its Universe line in the figure above) and $n$ that of a light ray having the origin O. Both lie in the four dimension space and they intersect at a single point $P$. The points O and $P$ lie on a zero radius (of a cone of the future), $n$, of $N^{+}(\text{O})$. In $P$, the person $P$ sees a sudden flash in the direction defined by $n$, for him the direction of the luminous event (described only by its velocity therefore, so a universe line of an inertial particle can be described only by time and speed).
	
	An atom whose Universe line cuts $n$ at the point $Q$ absorbs a photon of the luminous event O and re-emits a beam of photons shortly after. These, in turn, form zero rays in $N^{+}(Q)$, but only those of direction $n$ will reach the person $P$ and will be seen by him at the point $P$.

	If $P$ is inside $N(\text{O})$, the zero cone of O, we will say that its universe line is of the "time type". In this case, O and $P$ are located on the universe line of an observer or a massive particle. There are, of course, two types of time displacement:
	\begin{enumerate}
		\item If $P$ is in the future of O (according to an observer whose universe line passes through O and $P$), we will say that $P$ "points to the future".

		\item If not, we will of course say that it "points to the past".
	\end{enumerate}
	If $P$ is on $N(\text{O})$ - that is to say on the surface of the cone - then we say that it is "null" or of "light type" and if $P$ is neither zero nor of time, then $P$ is outside of $N(\text{O})$ and the we say that it is of "space type":
	\begin{figure}[H]
		\centering
		\includegraphics{img/cosmology/type_of_universe_lines.jpg}
		\caption{Types of universe lines}	
	\end{figure}
	This is mathematically translated by remembering (see above) that:
	
	\begin{itemize}
		\item $\mathrm{d}s^2=0$ (then $r^2=c^2\Delta^2$): The universe line is therefore "light-like" and it is that latter which describes the surface of the cone by definition (according to what we have demonstrated previously and whatever the choice of the metric) is such that:
		
		which is the case of a photon (hence the name ...). In other words, the spatial separation is equal to the distance light travels.

		\item $\mathrm{d}s^2<0$ (then $r^2<c^2\Delta^2$: We then say that the Universe line is "space-like", therefore such that:
		
		Two events that take place simultaneously but at different places are therefore space-like. In other words, the spatial separation is less than the distance light travels.

		\item $\mathrm{d}s^2>0$ (then $r^2>c^2\Delta^2$: We then say that the Universe line is "time-like", therefore such that:
		
		In other words the spatial separation is greater than the distance light travels.
		\begin{figure}[H]
			\centering
			\includegraphics{img/cosmology/universe_line_type_with_equation.jpg}
		\end{figure}
		
		\item A "\NewTerm{causal line}" is a time-like or light-like line that is always oriented towards the future.
	\end{itemize}
	Let us return to our equations after this small interlude ... the equations therefore lead us to several observations. Thus, in the four-dimensional Euclidean Universe of Minkowski, the trajectories of objects in space-time are always straight lines. Indeed, the trivial example consists in considering that the object remains at rest, then only the time then continues to flow. We have therefore:
	
	by putting $v=0$, this gives us:
	
	therefore:
	
	hence:
	
	and also:
	
	The primitive being (integration constant taken as zero):
	
	which is indeed a straight line and therefore represents the universe line of the object considered in the universe cone. We can also observe that in this case, the evolution of the phenomenon is purely temporal when the interval is positive (which supports what we said earlier).
	\begin{tcolorbox}[title=Remarks,colframe=black,arc=10pt]
	\textbf{R1.} If the speed of light is infinite, we fall back on the particular case of the Newtonian universe, where a phenomenon can instantly occur. Time is absolute and there is no cosmological horizon because the cone has a maximum aperture (right angle).\\
	
	\textbf{R2.} If we put that the velocity of light as equal to $1$ (natural units), as we have done it already sometimes, the axis of the ordinate of the cone is named a "purely temporal axis".\\
	
	\textbf{R3.} It is necessary to understand that our Universe has its own cone of Universe (cone... if the space is of Minkowsky-like of course...).
	\end{tcolorbox}
	Finally, let us say that the theory of Special Relativity, like that of General Relativity, does not impose a given number of spatial dimensions in order to remain consistent: this is a pity for theoretical physicists who would like a theory which, imposes on itself a finite number of dimensions to remain consistent (that on the other hand the theory of the strings or superstring).
	
	\begin{flushright}
	\begin{tabular}{l c}
	\circled{95} & \pbox{20cm}{\score{4}{5} \\ {\tiny 48 votes,  71.25\%}} 
	\end{tabular} 
	\end{flushright}
		
	%to force start on odd page
	\newpage
	\thispagestyle{empty}
	\mbox{}
	\section{General Relativity}
	\lettrine[lines=4]{\color{BrickRed}A}s we saw it, in the previous section, Special Relativity is a remarkable achievement from a theoretical point of view as well as a practical point of view, forming a continuum of space-time where the space variables and time are given the same physical dimension (that of a distance metric for reminder!). However, this applies only to the Euclidean frames and to inertial/Galileans reference frames(constant speed reminder ...). It is therefore appropriate to first generalize the entire mechanic theory by expressing its principles and fundamental results in a generalized form independent of the type of coordinate system chosen (that is to say: independent of the space properties) using for this purpose tensor calculus and then to take into account the non-inertial systems. The equivalence of inertial systems by Special Relativity and the non-equivalence of inertial systems can then shortly be resume (a little bit basically...) saying that speed is relative but the acceleration is absolute. Thus, we can never rest distinguish a uniform motion, but we can distinguish them from an accelerated motion.
	
	It should also be consider the fact that Special Relativity applies only to Galileans frames is restrictive because any mass creates a gravitational field whose scope is endless. To find a true Galilean frame, it is therefore necessary to lie infinitely far from any mass. Relativistic mechanics built from Special Relativity therefore constitutes an approximation of the laws of nature, where the gravitational fields or accelerations are low enough. This application limitation is not  more suited to relativistic astrophysics whose activity has intensified in the late 20th century.
	
	\subsection{Assumptions and Principles}
	Albert Einstein and some others of his time believed in a physics that not to favor any frame system since that was in their eyes the reality of the Universe (we have already mentioned this point of view). But how can we subtract ourselves to the phenomenon of  acceleration? The brilliant idea was to state the "\NewTerm{equivalence postulate}\index{equivalence postulate}" below (which still in this early 21st century has not show any default by recent known experiences) plus the "invariance postulate" and "cosmological principle" we have already stated in the section of Special Relativity and the assumption that the motion of a particle that does not undergo any other interaction that gravitation follows a geodesic line (see below for the detailed proof).
	
	\subsubsection{Equivalence Postulates}
	At first, Albert Einstein will improve the equivalence postulate (also named "equivalence principle") whose older versions are due to Galileo and Newton:
	
	\textbf{Postulate:} The (uniform!) acceleration of a mass (outside gravitational field) due to application of mechanical force and the acceleration of that mass subjected to a gravitational field are supposed completely equivalent. Thus, the results of mathematical analysis in one case may apply to the other (here this is already smart but consistent ... the idea is very good still had to have it...!). 
	\begin{figure}[H]
		\begin{center}
		\includegraphics{img/cosmology/equivalence_principle.jpg}
		\end{center}	
	\end{figure}
	In other words, the gravity field has a fundamental property which distinguishes it from all other fields known in nature: the free fall movement of bodies is universal, independent of the mass and composition of the bodies.
	
	Corollary: The rest mass of a body must be the same whether it is measured in a frame within a gravitational field or outside a gravitational field (we speak then about "inertial mass" and of "gravitational mass" as we have already study at the beginning of our study in the section of Classical Mechanics).
	
	\begin{tcolorbox}[title=Remark,colframe=black,arc=10pt]
	We must be careful and check that the corollary of the assumption of the equivalence principle is true otherwise all General Relativity would collapse (in the early 21st century experiments are still going to try to show a default to this principle)!
	\end{tcolorbox}	
	
	So all static and uniform gravitational field is equivalent to an accelerated frame in vacuum. We can consider any physical gravitational field as static and uniform in a relatively small region of space, and for a relatively short period of time to avoid the tides effects. We are thus led to state the "\NewTerm{Weak Equivalence Principle WEF}\index{weak equivalence principle}": For any event in space-time in an arbitrary gravitational field, we can choose a frame named "\NewTerm{locally inertial frame}\index{locally inertial frame}" such as in the neighborhood of the event of interest the free movement of all body (which are also in the gravity field!) is straight and uniform as we are able to apply the Lorentz transformations (\SeeChapter{see section Special Relativity}). In other words, it is not possible to distinguish a system in vacuum space far away from any star (gravitational source) from as system falling in a constant homogeneous gravitational field.
	
	If we experimentally show that WEF fails, then we are put into default the equivalence principle itself ... which has never been achieved in the laboratory to this date!
	
	\begin{tcolorbox}[title=Remark,colframe=black,arc=10pt]
	The concept of "locality" is very important because it reality we don't know any natural uniform gravitational field. For example, on Earth, two body distant of a certain length dropped from a certain height will fall to the ground with a shorter distance than the distance between them when they were released. This is what we call in physics the "tide effect": the gravitational field is never uniform (as far as we know...).
	\end{tcolorbox}
	
	So the postulate of equivalence (which includes the principle of weak equivalence) finally asserts that the Newton's force on inertial mass $m_i$:
	
	and that of gravitation in the form of the Newton-Poisson law (\SeeChapter{see section Astronomy}) with gravitational mass $m_g$:
	
	are equivalent such as the inertial mass equals the gravitational mass and acceleration equal to gravity and that it is not possible to distinguish the both such that:
	
	In what this postulate allows to resolve all the problems therefore? It's simple! The idea is the following:
	
	When we will consider a body in acceleration, we first always equate it to the acceleration due to the fall in a gravitational field (by applying the postulate of equivalence). Then, we will assume, and will have to check (see proof further below) by rediscovering Newton's law, that acceleration due to the gravitational field is not due to the field itself but to the geometry of the deformed space by the presence of the mass (i.e. energy) that creates the gravitational field. Thus, the object is no longer in "free fall" but will be seen as sliding on the distorted spatial frame to acquire therefore its acceleration.
	\begin{enumerate}
		\item If the tensor calculus gives the possibility to express the laws of classical and relativistic mechanics in any coordinate system, it is then possible to see how the coordinate system (metric) acts on the expression of the laws of the Universe (Albert Einstein did not know that fact as he had not completed its calculations but had a presentiment about this)!
		\item If the natural tensor expression of the laws of mechanics shows slippage (i.e. acceleration) on the spatial frame following the (local) considered metric, then the bet is won and then the acceleration can be seen as an effect whose cause is purely geometrical.
	\end{enumerate}
	
	\begin{tcolorbox}[colframe=black,colback=white,sharp corners]
\textbf{{\Large \ding{45}}Example:}\\\\
		Suppose that two rockets, which we denoted by $A$ and $B$ are in a region of space away from any body. Their engines are stopped which physically results in a uniform motion. In each rocket, physicists are making mechanics experiments with objects which they know the inert mass. Suddenly the engine of the rocket $A$ starts and communicates to it an acceleration whose effects felt inside the spaceship is an inertia force that constraint objects going to the floor. For physicists $A$ rocket laws of mechanics are then the same as that observed in a gravitational field. They are logically led to interpret the force of inertia as the manifestation of a gravitational field. Using a balance, they can weigh their objects and assign to them a gravitational mass.\\
		
		Suppose now that the physicists in rocket $B$ could observe what happens in the rocket $A$. They know what their colleagues interpret as the weight of objects is in fact a force of inertia. The inertial force is proportional to the acceleration and the inertial mass. If the gravitational mass was different from the inertial mass the physicists of the rocket $A$ could distinguish the effects of inertial forces from those of a gravitational field because the measured masses are distinct. We know that the inertial and gravitational mass are equivalent (Galilean principle of equivalence). It follows that the physicists of the rocket $A$ have no way to differentiate between inertial forces resulting from an accelerated motion of their spaceship and the gravitational attraction forces.\\
		
		However, we must temper the conclusions from this experience: the real gravitational fields differ from an accelerated frame since the gravitational acceleration varies with the distance to the main body while in an accelerated reference frame, the acceleration is the same at any point in space. However, \underline{locally}, a gravitational field and an accelerated frame can not be differentiated!!!
	\end{tcolorbox}
	
	We are led now to state the "\NewTerm{Einstein's equivalence principle}\index{Einstein's equivalence principle}" (EPE) as did Albert Einstein: locally all the laws of physics are the same in a gravitational field and a uniformly accelerated frame.
	
	This has a consequence: If the mass (which is equivalent to the energy as we have proved in the section of Special Relativity) of an object is not differentiable that we are in a gravitational field or a uniformly accelerated frame that means that all types of energy (nuclear cohesion energy, electrostatic energy, proper gravitational energy of the object, etc.) of this object are indistinguishable. So the laws of Special Relativity are also valid whatever the considered frame!
	
	If the laws are not the same, then EPE (Einstein equivalence principle) is faulted, so verbatim WEF (weak equivalence principle) also and more globally the principle of equivalence in general but this has never happened experimentally as far as we know at this day.

	\begin{tcolorbox}[colframe=black,colback=white,sharp corners]
	\textbf{{\Large \ding{45}}Example:}\\\\
	By the WEF, it is interesting to note that the gravitational field also acts on the gravitational potential energy of the other bodies. We say then that the gravitational field is a "coupled field".
	\end{tcolorbox}
	
	Given that in General Relativity, the gravitational field is supposed to be described by the metric $g_{\mu\nu}$ (from which the 4-dimensional differentiable manifold that is space-time is supposed to be made), we can see a locally inertial frame as a coordinate system of spacetime in which the metric becomes flat (pseudo-Riemannian):
	
	using the notation introduce in the section of Tensor Calculus.
	
	Such a coordinate system will by hypothesis always exists, indicating the existence, for any gravitational field, of locally inertial frames!
	
	\subsubsection{Mach Principle}
	
	If the equivalence principle highlights the equality of inert and gravitational mass, it does not enlighten us about the nature of these two masses. Finally, what are the inert and gravitational mass?
	
	The deep nature of the inert mass should inform us about the inertia itself. The inertia is manifested in a passive form - the principle of inertia - and an active form - the second Newton's law. In general, it expresses a universal behavior of bodies to resist to the change of movement. But we know that inertial motion is relative, that is to say that there is no absolute referential frame. Is it the same with the accelerated movement? Consider, to illustrate this question, a rocket in which has taken place a physicist and let us carry two experiments:
	
	
	However when the metric is not flat the coordinates are named "\NewTerm{Riemann normal coordinates}\index{Riemann normal coordinates}" and then describes a Riemann metric space (curved space) and itself depends in a nontrivial way of the coordinates system (see sections Tensor Calculus and Non-euclidean Geometry).
	
	\begin{enumerate}
		\item First experience: The rocket accelerates and the physicist is subjected to inertia force oriented in the direction opposite to that of acceleration.
		
		\item Second experience. Now assume that we gives to the whole Universe - at the exception of the rocket that moves in an inertial motion - an acceleration exactly opposite to the one of the rocket of the preceding experiment.
	\end{enumerate}
	If the accelerated motion is relative then, for an observer, it is not possible to distinguish the two experiments. In particular, the physicist located inside the rocket must observe the emergence of an inertial force absolutely identical to the one he noted in the first experiment. The inert mass could then has its origin in the interactions of the gravitational mass of bodies with all the gravitational mass of the Universe! It is as if by moving all masses of the Universe, they dragged with them the objects in the rocket, the physicist therefore experienced a force that pulls in the same direction as the acceleration applied to stars.
	
	Following Ernst Mach, physicist and philosopher of the 19th century, the movement whatsoever inertial or accelerated, is relative.
	
	This theory was named by Albert Einstein "\NewTerm{Mach principle}\index{Mach principle}". At this date, Mach's principle has not been confirmed, but no more rejected. It is true that its experimental verification far exceeds actual human capacities!
	
	\subsection{Metrics}
	Albert Einstein assumed that gravity was only the manifestation of space-time distortions. To try to illustrate in the more possible simple and illustrated way the idea of Albert Einstein, consider a rolling gear at constant speed (say, one tooth at a second) on a rack. Imagine that we have the power to simultaneously change the pitch of the rack and the wheel when and where we wish. Let us do things such that the pitch of the rack slightly increases from one tooth to another. For fixed observers the gear is then driven with a uniformly accelerated motion as, in effect, at each turn thereof always travels a greater distance. On the other hand, if one chooses the rack as a reference and thereof the pitch as a standard to measure the movement of the wheel is then uniform (one tooth per second). The acceleration of the wheel is the consequence of the increase in the pitch of the rack.
	
	Let us continue the analogy: the pitch of the rack acts as a local measurement standard in our one-dimensional space that represents the rack. In geometry, it is named the "metric". The metric is what determines the distance between two points, it is somehow the standard infinitesimal space unit. In Euclidean geometry, the metric is constant, allowing us to create universal measurement standards. Bernhard Riemann, for example, invented a metric geometry which can vary from one point to another in space, which allowed him to describe curved spaces like the surface of a sphere, for example (\SeeChapter{see section Non-Euclidean Geometries}).
	
	During our study of tensor calculus, non-Euclidean geometries and differential geometry (section that the reading is more than recommended!!!) we have seen that the measurement of the curvilinear distance $ds$ between two points positioned in a two or three dimensions space can be made using a large number of coordinate system by the "\NewTerm{metric equation}\index{metric equation}" (\SeeChapter{see section Tensor Calculus}):
	
	In General Relativity, the idea is to make the theoretical model independent of the background and thus build it in a covariant form (which some physicists liken to assimilate to a postulate named the "\NewTerm{covariance principle}\index{covariance principle}"). An excellent candidate for this type of approach is to use the tensor formalism. This is the reason why the metric equation is therefore one of the pillars.
	
	\begin{tcolorbox}[colframe=black,colback=white,sharp corners]
	\textbf{{\Large \ding{45}}Examples:}\\\\	
	E1. Rectangular coordinates (in $\mathbb{R}^3$):
	
	If the squared distance satisfies this relation then we are in a flat space or at least locally flat (\SeeChapter{see section Non-Euclidean Geometries}).\\
	
	E2. Polar coordinates (in $\mathbb{R}^2$):
	
	Therefore:
	
	Therefore:
	
	If the squared distance satisfies this relation then we are in a flat space or at least locally flat (\SeeChapter{see section Non-Euclidean Geometries}).\\
	
	E3. Cylindrical coordinates (in $\mathbb{R}^3$):
	
	when we put this into $\mathrm{d}s^2=\mathrm{d}x^2+\mathrm{d}y^2+\mathrm{d}z^2$ we get in a similar way as before:
	
	If the squared distance satisfies this relation then we are in a curved space (cylindrical type) but that may may be locally flat (\SeeChapter{see section Non-Euclidean geometries}). In fact, to have the metric of the cylinder surface and not simply of the plane expressed in cylindrical coordinates, we must take the following metric:
	
	whose origin was proved in the section of Differential Geometry and also... just previously...
	\end{tcolorbox}
	
	\pagebreak
	\begin{tcolorbox}[colframe=black,colback=white,sharp corners]	
	E4. Spherical coordinates (in $\mathbb{R}^3$) for which we have:
	
	when we put this into $\mathrm{d}s^2=\mathrm{d}x^2+\mathrm{d}y^2+\mathrm{d}z^2$ we get:
	
	Now remember that (\SeeChapter{see section Algebra Calculus}):
	
	Therefore:
	
	After a first set of factorization and basic simplifications of identical terms, we obtain:
	
	If the squared  distance satisfies this relation then we are in a curved space (spherical type) but that locally may be flat (\SeeChapter{see section Non-Euclidean Geometries}). In fact, for the metric of the surface of the sphere and not simply of the plane expressed in spherical coordinates, we have to take the following metric:
	
	whose origin has been proved in the section of Differential Geometry. We also checked in the section of Tensor Calculus, that the Ricci curvature of the spherical prior-previous metric was zero. By cons, we had right after checked that if we took the previous metric of the surface of the sphere, the Ricci curvature was not zero (and it is still happy!).
	\end{tcolorbox}
	Until then, you may be wondering where we are going? In fact, we try to define from these relations, a mathematical being that consistent with the Einstein's hypothesis, expresses the geometric properties of given space.
	
	How we will do this?: We first change simply change the notations. Instead of using the symbols $(x,y,z,\theta,\phi,r)$ we will write $x^1,x^2,x^3,...$. Caution! The numbers suffixes are not powers!!! These are dummy values that are only there to symbolize the $x$-th coordinate of a given basis.
	
	Now let us write again our metric equations with this new notation by considering it is only specific examples that do not necessarily have relevant physical sense (we also mentioned it earlier!):
	\begin{itemize}
		\item Rectangular coordinates:
		
		\item Polar coordinates:
		
		\item Cylindrical coordinates:
		
		\item Spherical coordinates:
		
	\end{itemize}
	Now let us recall again that the "\NewTerm{metric tensor}\index{metric tensor}" (so named because it calibrates space-time) noted (\SeeChapter{see section Tensor Calculus}):
	
	is involved in the metric equation as follows in the Lorentz invariant:
	
	and notice that the components of the matrix are also dimensionless!
	
	This mathematical entity which is a tensor thus contains the parameters of the curvature (we also sometimes say of the "stress" or "tension") wherein a space is located. But then what contains the metric tensor of space-time for a flat Euclidean space?
	
	According to the summing writing Einstein's convention (\SeeChapter{see section of Tensor Calculus}), for example, for $\mu=\nu=2$ we have:
	
	So if we return to our tensor for the flat Euclidean space, we already know (\SeeChapter{see section Tensor Calculus}) that $m$ and $n$ goes from $1$ to $3$ and we have in our tensor $g_{\mu\nu}=0$ for $\mu\neq \nu$ and $g_{\mu\nu}=1$ for  $\mu= \nu$  (symmetrical tensor ). So:
	
	Therefore:
	
	which as usual in this book we make usage of the abusive notation (already indicated in the section of Tensor Calculus) to not put $g_{\mu\nu}$ (since a tensor and its matrix form are normally two separate things strictly speaking).

	This result is remarkable, because the metric tensor will therefore enable us to define the properties of a space from a simple mathematical being that can easily be handled formally as we already seen in the section of Tensor Calculus, Non-Euclidean Geometry and Differential Geometry.

	In polar coordinates the tensor $g_{\mu\nu}$ is:
	
	Check:
	
	And in cylindrical coordinates the tensor $g_{\mu\nu}$ is written:
	
	We will not do the check as the result is obvious (excepted on reader request).
	
	In spherical coordinates the tensor $g_{\mu\nu}$ is a little more complex and is written:
	
	We will also not do the check as the result is obvious (excepted as always on reader request). 
	\begin{tcolorbox}[title=Remark,colframe=black,arc=10pt]
	As we have mention it, in the section of Tensor Calculus, $g^{ij}=(g_{ij})^{-1}$ and the reader can quickly verify this with Maple 4.00b as inverting a matrix is always a boring work (here the code is given only for the spherical one but the idea is the same for the others):\\
	
	\texttt{>with(linalg):\\
	>A:=array([[1,0,0],[0,r\string^2,0],[0,0,r\string^2*sin(theta)\string^2]]);\\
	>inverse(A);}
	\end{tcolorbox}
	
	In Special Relativity, we have seen that the notions of space and time were implicitly bounded. Thus, to study modern physics (this does not really interest  the pure mathematician), we need to add to our metric tensor a time component  to get what we name the "\NewTerm{space-time metric tensor}\index{space-time metric tensor}".
	
	To determine the writing of this tensor, we will place us at first in a Minkowski space where we have for recall (\SeeChapter{see section Special Relativity}):
	
	which is the infinitesimal interval of space-time between two infinitely close events (or considered as it at a given scale ...).

	Thus by putting:
	
	We have:
	
	with the "\NewTerm{signature}\index{signature}":
	
	\begin{tcolorbox}[title=Remark,colframe=black,arc=10pt]
	For all metric tensor that we have determined before, if we express them in space-time (thus adding time component), the spatial components will all have a negative sign!
	\end{tcolorbox}
	We will see later other metrics that are much less intuitive once we will have proved far later below the Einstein's fields equation.
	
	\subsubsection{Schild Criteria (Einstein red-shift effect Newtonian approach)}
	We will prove later that gravitation as formulated in Newtonian mechanics is completely describable by a curvature formulation of space-time. But first we want to introduce to the reader what in our point of view the easiest development that can be done without General Relativity to compare it also later with the easiest development that can be done with General Relativity: the gravitational red-shift effect!
	
	Imagine first a very height tower of height $h$ built on the surface of the Earth. A man sits at the ground of the tower, and sends a signal of pulsation $\omega_A$ to a colleague $B$ at the top of the tower. There will be, and we will immediately prove it, that the pulsation $\omega_B$ of the wave received by $B$ differs of $\omega_A$ according to the relation:
	
	Hence:
	
	This shift of pulsation (frequencies respectively) in a gravitational field is what we name the "\NewTerm{Einstein's effect}\index{Einstein's effect}", or "\NewTerm{gravitational redshift}\index{gravitational redshift}".
	
	We will first prove this relation using conventional arguments and now well known to us. Later we will prove that in fact this is only an approximation of a result that we will get later using General Relativity curvature properties.
	
	A material body sent from the ground to the sky must fight against the gravitational force that pulls it down. So it will lose a certain amount of energy, equivalent to gravitational potential energy gained during the trip. The total energy $E_A$ of the body at the ground level is therefore its mass energy (\SeeChapter{see section Special Relativity}) to which we add the potential energy at the height of the tower:
	
	The energy of this body when you reach the top of the tower is simply its mass energy:
	
	because he had to spend the energy $mgh$ during the trip to go up. The ratio of energy is then:
	
	This ratio being independent of the mass $m$, we can take the limit $m\rightarrow 0$ in order to have the relation for the photon. We then get:
	
	which implies:
	
	That is to say the clocks run slower in a gravitational field as seen by a distant observer!
	
	We will now study this phenomenon in the context of the Minkowski space-time. We will see then a contradiction, what will motivate the transition to a curved space-time: this is the argument of a curved geometry that was used by Schild.

	Let us consider again the human experience of a human in $A$ which sends a wave to his friend positioned in $B$. Given $\Delta t_A$ the time taken by $A$ to emit exactly $1$ cycle of the wave (\SeeChapter{see section Wave Mechanics}):
	
	and $\Delta t_B$ the time taken for $B$ to receive this cycle:
	
	Because of the Einstein's effect just seen previously, we know that $\omega_A>\omega_B$ and therefore that $\Delta t_A<\Delta t_B$ in proper time! That is to say that time passes more slowly for someone on the ground ($A$) than another person in a mountain top ($B$)!

	But as we are in flat geometry and the gravitational field is assumed static, we deduce that space-time trajectories described by the signals must be parallel! This leads to the conclusion that the proper time interval would be $\Delta t_A=\Delta t_B$ (according to Special Relativity).
	
	If we opt for a curved space, we can preserve the relation $\Delta t_A<\Delta_B$, that is to say that time passes more slowly for $A$ than for $B$. This simply results in the fact that curved geometry, the proper time (!) of an observer depends on the metric.

	Let us now notice that same developments can be made by assimilating the previous experience with a train that moves with constant acceleration $g$ (horizontal situation of the previous one!). The observer $A$ is in the rear compartment (equivalent to the floor of the Earth in the preceding experiment) sends a wave to his colleague $B$ on the front of the train (at a distance $h$).
	
	The observer $B$ receives the wave after a time $\Delta t=h/c$. During this time, the train has accelerated, and its speed has increased of a value $\Delta v=g\Delta t=gh/c$. Therefore, the wave seen by $B$ will be altered by the conventional Doppler effect (\SeeChapter{see section Wave Mechanics}):
	
	We fall back on the initial results of the Einstein's effect by simply writing:
	
	giving gloriously:
	
	We find more often this relation in the form below in the literature using the relation between pulsation and frequency and Newton's gravitational force to explicit $g$:
	
	 and putting $h$ as being equal to $r$:
	 
	and after rearranging we also found sometimes in textbooks:
	
	or even more frequently:
	
	We also find this last relation in the following condensed form:
	
	The same result can be obtained using the Schwarzschild metric (see further below), hence the name of this effect that can also be obtained from the mathematical tools of Einstein's General Relativity. We will prove later, in a simple way, using this metric that time actually flows more slowly in a gravitational field (assumption we made a few paragraphs above).
	
	We see that in all cases:
	
	since the right term is positive and not zero. This simply means that the electromagnetic wave in analogy to the color spectrum shifts toward red. Thus, Einstein's effect is indeed a gravitational redshift!

	The frequency difference is very small and therefore difficult to measure even with the best spectroscopes. The slightest disturbance can completely mask the Einstein's effect. It will be necessary to wait until 1960 that the experience of Pound and Rebka to be capable of measuring a frequency offset with an accuracy of $1\%$ therefore leaving no doubt as to the reality of this phenomenon.
	
	\subsection{Equations of movement}
	We will prove here that the equation of motion of a free particle is constant along its world line by first limiting ourselves to the case of a flat space (Minkowski space type ). Then we will generalize this result in any kind of space using a simple development, to show quite clearly that the equation of motion is independent of the mass and follows the curvature of space !!! Finally, we will present a second proof in any kind of space using the variational principle.

	So let us start by proving the equation of motion of a free particle in a flat space.

	During our study of Special Relativity, we have proved the Relativistic Lagrangian of a free particle given by:
	
	and for this we started from the action (hypothetical):
	
	and we came to write:
	
	Now let us show something interesting! Let us recall that for the Minkowski space-time, we got:
	
	and restricting ourselves to one spatial dimension, we obtain as relation:
	
	Therefore:
	
	and then ... well that's the way, if we put:
	
	we finally have:
	
	so we fall back on the same action from a more general form (pure) action that is:
	
	result that we had also proved in the section of Electrodynamics !! We can even do better in terms of elegance ...! If we observe well the developments of the previous lines, we observe that in facts the relation:
	
	and is the special case to one dimension of the relation:
	
	with as defined earlier above:
	
	and therefore:
	
	Thus we have the "\NewTerm{Fitzgerald-Lorentz factor}\index{Fitzgerald-Lorentz factor}" or simply "\NewTerm{Lorentz factor}\index{Lorentz factor}" that is given in general form by:
	
	as a generalization of Special Relativity!

	This being done, let us come back on our topic... In an space without potential field, we have proved the section of Analytical Mechanics that the Lagrangian is reduced to the simplest expression of the kinetic energy such that:
	
	If we wish to generalize this relation for it to be valid in any type of space (curved or flat), we must introduce the curvilinear coordinates as we have studied them in the section of Tensor Calculus.

	In a first time, this gives:
	
	where for recall $\mathrm{d}s$ is the curvilinear abscissa of the path.
	
	And we have proved in the section of Tensor Calculus that:
	
	The latter relation is written in the context of relativistic mechanics in a most standard way:
	
	where $\tau$ is a parameter that in relativistic mechanics if for recall the proper time of the particle.
	
	Before we focus on curved spaces described by the metric $g_{\alpha\beta}$ (which we will do during our proof of the free generalized Lagrangian ), let us restrict us to Euclidean space with the metric (this will be a good exercise to understand) given by the Minkowski matrix (\SeeChapter{see section Special Relativity}):
	
	which we denote $\eta_{\alpha\beta}$ to differentiate it from others (because most often used). Finally we in Euclidean space:
	
	Now let us apply the variational principle:
	
	The variation $\mathrm{d}s$ can be found simply from the variation of $\mathrm{d}s^2$:
	
	we find:
	
	The factor "$2$" is because by symmetry of the Euclidean space, the variations of $\mathrm{d}x^\alpha$ and $\mathrm{d}x^\beta$ are equal. 
	\begin{tcolorbox}[title=Remark,colframe=black,arc=10pt]
	As we will see later, this relation $\delta(\mathrm{d}s)^2$ will not be the same anymore when dealing with curved spaces.
	\end{tcolorbox}
	Simplifying a bit, we get:
	
	Which is equivalent to write:
	
	We can now go back to the action:
	
	We rewrite the preceding integral as following (it will be easier to treat):
	
	Indeed, let check that this form is similar:
	
	So let us come back to our integral:
	
	We have then two integrals that it will be a bit easier to analyze. The first integral:
	
	simply gives an expression evaluated to the temporal extremities $(\tau_1,\tau_2)$. Therefore, as the values  $x^\alpha$ are perfectly known at the time ends, the variational $\delta x^\alpha$ is zero at the both extremities and this integral is therefore zero.
	
	Then we are left only with this integral:
	
	So for the variational principle (\SeeChapter{see section Analytical Mechanics}):
	 
	is respected, we must have:
	
	Now, we can write this expression explicitly. Indeed, we have:
	 
	Remember also that we have proved earlier above that:
	
	and that we have:
	
	Therefore:
	
	Now, let us recall that during our study of Special Relativity, we have proved the path that led us to define the linear momentum four-vector:
	
	So finally, what cancel the variational of the action integral can be written:
	
	We thus fall back on the conservation equation of linear momentum (momentum conservation) that we name in the framework of General Relativity "\NewTerm{equation of motion}\index{equation of motion}". This form of the equation of motion seems dependent on the mass but by digging a bit, we will see that it is fact not.
	
	Multiplying this relation by $\eta_{\mu\nu}$ we can also write:
	
	and the same for another observer:
	
	In other words, the linear momentum of the particle remains constant along its world line.

	But we can also write:
	
	Therefore:
	
	An even more important form of movement equation can be obtained. Indeed using the relations just proved above we can write:
	
	Therefore:
	
	Hence:
	
	this relation is therefore "massless" equation of motion in Euclidean space or in other words, in a Minkowski space-time type. In other words, there exists a falling coordinate system wherein the motion of the particle is a uniform movement in space-time.
	
	It will be very interesting to compare it later with the equation of motion in a curved space as we will see later (named "geodesic equation").
	\begin{tcolorbox}[title=Remark,colframe=black,arc=10pt]
	It is equivalent to write the relations of equations of motion with respect to the curvilinear abscissa $\mathrm{d}s$ or the proper time $\mathrm{d}t$ (traditionally denoted by $\mathrm{d}\tau$ in the field of General Relativity).
	\end{tcolorbox}
	We can now prove that the previous equation of motion, just like the geodesic equation that we will see afterwards, is invariant under Lorentz transformation. Indeed:
	
	Now let us see a more general form of the equation of motion for any kind of space. The aim is to highlight, and this in a few lines of calculations, that the movement followed by a free particle is independent of its mass (you can already anticipate the interpretation of the path of a photon in a curved space...!).

	Let us first recall that we have proved in the section of Tensor Calculus (and previously) that:
	
	giving us for the generalized Lagrangian of a free particle with $\mathrm{d}\alpha=\mathrm{d}\tau,u^i=x^\alpha,u^j=x^\beta$ (although we fall back on the general expression of the kinetic energy as there is no potential for a free particle):
	
	where for recall $\tau$ is the proper time\footnote{The proper time is for recall a kind of imaginary clock that travels on the particle and whatever observers watch the clock, they will mathematically agree on the value of the time interval between two "Tic" of the clock.} of the particle, it is an invariant!
	\begin{tcolorbox}[title=Remark,colframe=black,arc=10pt]
	This relation is named the "\NewTerm{geodesic lagrangian}\index{geodesic lagrangian}" by some text book authors.
	\end{tcolorbox}
	This allows us to write (caution! the reader must remember the different relations that we had determined during our study of the Lagrangian formalism in the section dealing with Analytical Mechanics):
	
	\begin{tcolorbox}[title=Remark,colframe=black,arc=10pt]
	The elimination of the $1/2$ Lagrangian factor results from the symmetry of the metric tensor. If that latter is not symmetric, we can always characterize it by a tensor that is.\\

	Indeed, for recall (\SeeChapter{see section Tensor Calculus}, given $\vec{x}$ a vector of coordinates $x_1,\ldots,x_n$ and given:
	
	The $T_{ij}$ are not symmetric a priori, but we can write:
	
	We put afterwards:
	
	Therefore:
	
	And the $B_{ij}$ are symmetric.\\
	
	The quadratic form $q$ can thus always be written with a symmetric matrix, there is even a bijection. The conclusion is that a metric tensor must be symmetric if we want to characterize it by the quadratic form it defines.
	\end{tcolorbox}
	The mathematical interlude having ended, let us continue our physical development. As a consequence of the last relation, the expression of the Hamiltonian obviously becomes:
	
	since we consider to be in a space without potential field anymore. Since the square of the velocity is therefore constant over the entire trajectory, we have:
	
	Let us now establish the equations of motion of any body. We have:
	
	and as:
	
	then:
	
	hence:
	
	By putting everything together we get:
	
	that we can write identically for the $\ddot{x}^\alpha$ by proceeding in the same way as above.

	The preceding relation therefore gives the trajectory of a body in motion, in a space without a potential field, as a function of its curvilinear coordinates and of the metric of the space under consideration.

	What is particularly interesting in this result is that mass $m$ (again) is eliminated identically in this equation of motion:
	
	Notice that we could have used another invariant parameter as well as the proper time $\tau$ such as the curvilinear abscissa $\mathrm{d}s$. Hence the preceding equation should be written:
	
	We can still simplify this relation, but we will keep this simplification for the second proof of the equation of motion in any space (by making use of the variational principle this time) just further below.

	It is very (very) interesting to observe that if we restrict the metric to that of a Euclidean space:
	
	with:
	
	We then have the following simplification:
	
	That it remains only:
	
	By lowering the indices with the signature it remains:
	
	We thus fall back on the first equation of the motion obtained for a flat space! The result is remarkable!

	The conclusions is that at the same initial conditions of curvilinear position and velocity in a space (flat or curved) without a potential field (this is what we could think at least according to our initial hypotheses ...), corresponds the same trajectory whatever the mass $m$ of the particle (even for photons - light - whose rest mass is zero!).

	We can now study the principle of least action in order to seek the shortest path (both spatially and temporally!) between two points in a given geometric space before addressing the much more complex case of the Lagrangian which takes In account the tensor field...
	
	\subsubsection{Geodesic equations}
	Let us now turn to the same result, but this time using the variational principle. We will fall on the same equation as before for any kind of space with the difference that this time we will take the time to simplify it to arrive at the "geodesic equation".

	Starting from (see previous developments):
	
	with a parametrization such that $x^i$ and $x^j$ depend of a temporal or spatial parameter.

	For a given surface in parametric form, we therefore seek to minimize the length of an arc $\mathrm{d}s$ by applying the variational principle (not dependent on time) because the photons can not have a faster path in the temporal sense of the term between two points but only a shorter path - in the metric sense of the term!):
	
	in natural units. Or:
	
	By developing, and as the indices have the same range of variation:
	
	hence (we have already multiplied the expression after the second equality by $\mathrm{d}s/\mathrm{d}s$ by anticipating the integral that follows):
	
	Then, we must introduce this development under the integral:
	
	Working on the second integral (after the equality), we put:
	
	So by integration by part (\SeeChapter{see sectoin Differential and Integral Calculus}):
	
	becomes:
	
	Thus finally:
	
	The non-integrated term below:
	
	is negligible because of the presence of the factor $\delta \mathrm{d}x^j$:
	Therefore:
	
	We make a change of index:
	
	Which allows us to factorize $\delta x^k$:
	
	As $\delta \mathrm{d}x^k$ and $\mathrm{d}s$ are different from zero, it is the integrande that must be zero:
	
	By developing the second term:
	
	Which can also be written (in the physicist way of life...)
	
	Which simplifies into:
	
	We fall back (again!) on the system of equations which defines the "\NewTerm{geodesics}\index{geodesic}", that is to say the straight lines of $\mathcal{E}^n$. These latter then constitute the extremities of the integral which measures the length of a curve arc joining two given points in $\mathcal{E}^n$.

	This last equation is the one which interests us in the case of the free Lagrangian. Indeed, if we take the extreme case of light (or photons if you prefer), the latter will not seek the fastest path at the temporal level. This would totally contradict the postulate of invariance to see the light accelerate according to the path !!! In this context, it means that on the spatio-temporal framework, the only thing that has meaning is the shortest spatial path and not the shortest temporal path! This is why the latter equation is named the "\NewTerm{geodesic equation}\index{geodesic equation}" or also "\NewTerm{generalized Euler-Lagrange equation}\index{generalized Euler-Lagrange equation}".

	However, we can write this last equation in a more condensed form by introducing the Christoffel symbols if the metric is a symmetric tensor, that is to say if $g_{\alpha\beta}=-g_{\beta\alpha}$.
	
	Indeed:
	
	And as the Christoffel symbol of the first kind (\SeeChapter{see section Tensor Calculus})
 is defined by:
	
	\begin{tcolorbox}[title=Remark,colframe=black,arc=10pt]
	It is important to remember that this symbol contains almost all information about the space-time metric. We will see an example below as what in a locally inertial frame this Christoffel symbol is equal to zero.
	\end{tcolorbox}
	Then the Euler-Lagrange equation is then written:
	
	The contracted multiplication (\SeeChapter{see section Tensor Calculus}) of the preceding relation in the canonical basis by $g^{kl}$ gives us:
	
	Hence
	
	In the literature a change of index is often carried out in order to at the end (it is still the same expression given that the indices have the same range of variation!):
	
	with $\Gamma_{\alpha\beta}^\mu$ being the Christoffel symbol of the second kind (\SeeChapter{see section Tensor Calculus}) given by:
	
	and is named in the context of General Relativity the "\NewTerm{affine connection}\index{affine connection}" or "\NewTerm{connection coefficients}\index{connection coefficients}" and which makes it possible to find the system of coordinates (through the resolution of a system of differential equations) in free fall in which the particle equation is that of a uniform movement in space-time as a function of a reference system (the two systems are therefore connected by the affine connection).

	This relation, of the highest importance, allows us to determine how a moving body will naturally move in a curved space and this perhaps ... regardless of its mass !!! It therefore gives us the metric in which we must set a frame of reference so that it is inertial with respect to the body in question.

	The previous equation of geodesics is also the differential equation of the second order which must therefore satisfy the parametric representation of a line on a surface where $s$ is the length along the line so that its total length is extremal!!!

	According to the principle of equivalence, we are therefore entitled to interpret this relation as the equation of motion in any gravitational field of and thus to interpret the second additional term of the equation as the opposite of a gravitational term force per unit mass, that is to say as the opposite of a gravitational field!
	\begin{tcolorbox}[title=Remark,colframe=black,arc=10pt]
	We can also write the equation of the geodesics and using the proper time. Indeed:
	
	or by using the four-vector velocity:
	
	\end{tcolorbox}
	Again, if we restrict ourselves to a flat space-time, we see trivially that we fall back on the first equation of motion that we had obtained since for the Minkowski metric $\eta_{\mu\nu}$ we have immediately $\Gamma_{\alpha\beta}^\mu=0$:
	
	because the components of the Minkowski metric being constant the Christoffel coefficients are all zero.
	
	The solutions of the latter equation are ordinary straight lines given by:
		
Obviously, in a general curved space-time, the geodesics can not be globally represented by straight lines. However, with a second-order approximation in Taylor's development (\SeeChapter{see section Sequences and Series}), we fall back on straight lines (which is equivalent to bringing the curved space back to a flat space).

	The important thing in all this is that the equation of geodesics makes it possible to observe that the curvature of space determines the trajectories of the bodies which move there whatever their mass, whether they are in uniform motion or not (observe the second derivative in the geodesic equation!). All that remains is then to complete the work and to relate the curvature of space-time with the energy that is there!
	
	\subsubsection{Newtonian Limit}
	We have shown above (Shild's argument) that to study gravitation (in particular the Einstein's effect), curved geometry is necessary. We promised also to show that it was enough. Now is the time to do it!

	\textbf{Definition (\#\mydef):}  The "\NewTerm{Newtonian limit}" is a physical situation where the three conditions below are satisfied:
	\begin{enumerate}
		\item[C1.] The particles move slowly with respect to the speed of light. This is expressed as the fact that the variations of the spatial components of their quadrivector are much less than those of the temporal component ($t$ being the proper time):
		

		\item[C2.] The gravitational field is static. In other words, any time derivative of the metric is zero!

		\item[C3.] The gravitational field is weak, that is, it can be seen as a weak perturbation of a flat space:
		
		with $|h_{\mu\nu}|\ll 1$ and where $\eta_{\mu\nu}$ is constant (only $h_{\mu\nu}$ depends on the coordinates).
	\end{enumerate}
	Let us consider the geodesic equation obtained previously:
	
	The first condition (C1) leads us to simplify it in the form:
	
	The two other conditions (C2 and C3 whose application has been shown in the development below) offer us several simplifications in the expression of the symbol of Christoffel of the second kind:
	
	The geodesic equation then becomes:
	
	and is then equal for the temporal component to ($\mu=0$):
	
	But (recall of the Minkowski metric):
	
	for $\lambda>0$ and for $\lambda=0$ we have (static metric):
	
	Therefore, we must conclude that $\mathrm{d}x^0/\mathrm{d}t$ is a constant (whatever the choice of the signature of the Minkowski metric).

	And for the spatial components, we know that $\eta^{\mu\nu}$ when reduced to its spatial part is a simple unitary $3\times 3$ matrix, which gives for each spatial component in the case where we choose (by tradition only!) the signature $- + + +$ of the Minkowski metric:
	
	Obviously, the reader can have fun making the development that follows with the inverse signature ($+ - - -$) and he will see that it only changes the sign of potential in the final result of the development):
	Let us now rearrange the above relation:
	
	By dividing by $(\mathrm{d}x^0/\mathrm{d}\tau)^2$ and restoring $x^0=c\tau$, we get by making as sequence of simplifications:
	
	Starting from here we put (because our illustrious predecessors have tried before us):
	
	such as (a relation which will be very useful to us when studying the Schwarzschild metric further below):
	
	where $\varphi$ is the gravitational potential. We fall back here on the expression of the gravitational acceleration (Newton-Poisson equation) of the Newtonian mechanics (\SeeChapter{see section Astronomy}):
	
	with $i=1,2,3$.

	This development, simple but nevertheless remarkable by its interpretation, proves that the curved geometry is sufficient to describe the gravitation (and therefore the theory of Newton)!!!!!!!!!!!! This verification is named by some people the "\NewTerm{principle of correspondence}\index{principle of correspondence}".
	
	\subsection{Stress-Energy Tensor}
	The "\NewTerm{Stress-Energy Tensor SET}\index{Stress-Energy Tensor}" (sometimes named "\NewTerm{stress–energy–momentum tensor}\index{stress–energy–momentum tensor}" or "\NewTerm{energy–momentum tensor}\index{energy–momentum tensor}") is a mathematical tool used (in particular) in General Relativity to represent the density and flux of energy and moment in space-time, generalizing the stress tensor of Newtonian physics of mass and energy. It is therefore an attribute of matter, radiation, and non-gravitational force fields. The stress–energy tensor is the source of the gravitational field in the Einstein's field equations of general relativity, just as mass density is the source of such a field in Newtonian gravity.
	
	Let us take for example the SET which considers matter in General Relativity as being able to be approximated by a perfect fluid. In the section Continuum Mechanics we have proved:
	
	where $N_i$ has for recall the units of a force and $n_j$ those of a surface. Thus with a more conventional writing:
	
	In variational form this gives:
	
	Let us now calculate:
	
	\begin{tcolorbox}[title=Remark,colframe=black,arc=10pt]
	We do not work with differential elements to avoid being trapped later. It is completely a physicist Do It Yourself approach, but it works well (confirmed by experience...).
	\end{tcolorbox}
	Assuming that only the volume and the time makes that the force varies (which assume a constant density and the to be inertial) we then have:
	
	This gives simply the tensor product of the velocities (\SeeChapter{see section of Tensor Calculus}):
	
	If we generalize this relation to the velocity quadrivectors of Special Relativity with the corresponding notations, then we have by definition the "\NewTerm{energy-momentum tensor}\index{energy-momentum tensor}" or "\NewTerm{Stress–energy tensor}\index{Stress–energy tensor}":
	
	or in index form:
	
	Either in contravariant form (most common form in textbooks):
	
	This relation is the justification for which General Relativity is also indicated as a theory of continuous mechanics by some specialists.

	Now let us prove that the derivative:
	
	\begin{tcolorbox}[title=Remark,colframe=black,arc=10pt]
	What we have already pointed out in the section of Tensorial Calculus is written $T^{0j}_{,j}$ in old books or in modern textbooks where the author want to show its technical level...
	\end{tcolorbox}	
	First, let us recall that (\SeeChapter{see section Special Relativity}):
	
	and let us admit that we are in low speeds such as $\gamma=01$. Then, in a Minkowski metric of type $(+, -, -, -)$ we have:
	
	But, we recognize in the parentheses the equation of continuity (conservation of the mass) which we have proved in the section of Thermodynamics and which we know is equal to zero in a system without sources! Therefore:
	
	Let us also look that what contains the componant $T^{00}$ of the stress-energy tensor:
	
	In terms of units, this is an energy density (we see directly that this quantity can only be positive).

	Let us now look at the other components with $i=0$ and $j=1,2,3$:
	
	where $p^i$ has the units of linear momentum density.

	Let us now consider the components of the tensor when $i,j=1\ldots 3$ (we omit then the first row and the first column):
	
	We thus fall back on the components of the stress tensor of a perfect fluid.

	So finally, the stress-energy tensor can be written in the form of a symmetric real $4\times 4$ matrix:
	
	
	This tensor is also sometimes represented as following:
	
	We thus fall back in this tensor on the following interpretations of the physical quantities (although rigorously all the components have units which can be seen as density of energy or as a pressure):
	\begin{figure}[H]
		\centering
		\includegraphics[scale=0.4]{img/cosmology/stress_energy_tensor.jpg}	
	\end{figure}
	We then understand better why the this matrix is named "Energie-Momentum Tensor" or "Stress-Energy-Momentum Tensor" since implicitly it is a question of modeling the space by a perfect fluid under shear stresses (tangential forces) and tensions (normal forces).
	\begin{tcolorbox}[title=Remark,colframe=black,arc=10pt]
	The sub-matrix of spatial components:
	
	is the matrix named the "\NewTerm{matrix of moments flows}\index{matrix of moments flows}" (a name that is quite debatable ...). In Continuum Mechanics (see section of the same name), we have proved that its diagonal corresponds to the pressure, and the other components to the tangential forces due to the dynamic viscosity.
	\end{tcolorbox}
	Let us prove that the covariant derivative (\SeeChapter{see section Tensor Calculatus}) of the stress-energy tensor is zero such that:
	
	Therefore:
	
	Let us begin by developing the first term:
	
	But we have:
	
	hence:
	
	We find in the squared brackets the equation of continuity which is zero in the absence of sources. On the other hand, the first term in parentheses is non-zero as we saw in our study of the four-accelerator acceleration in the section of Special Relativity:
	
	But according to the weak principle of equivalence (WPE), we can always place ourselves in a repository such that locally the acceleration is null, that is to say such that (for recall, we do not put vector arrows for the quadrivectors):
	
	And it comes then:
	
	So we now have:
	
	Let us look at what this last term gives but first recalling that in the section of Special Relativity we had proved that the quadri-acceleration was expressed according to:
	
	Therefore (we take only the first two components as examples):
	
	We will now in fact prove that:
	
	for this we start first to prove that:
	
	For this we calculate first:
	
	But:
	
	Therefore:
	
	Now let us prove that:
	
	the other components $a^2$, $a^3$ are then verified automatically.

	For this we do little bit algebra:
	
	and therefore we have indeed:
	
	but according to the WEP, $a^\nu=0$ therefore:
	
	and finally we have indeed under the assumptions stated above:
	
	Which is the expression of the conservation of energy in general relativity! By lowering the indices it comes:
	
	
	\pagebreak
	\subsection{Einstein's Field Equations}
	It is now time to tackle one of the most beautiful, one of the most famous equations of our time and that shines the eyes of many young students and science passionate: Einstein's field equations. The one that explains why matter (energy) curves space!!! There are several ways to obtain these equations. The two most common ones are either:
	\begin{enumerate}
		\item To have an engineer approach: That is to say we proceed by comparison with a known limiting result which is the law of gravitation of Newton (it is the one that we have chosen)

		\item To have a pure mathematic approach (very elegant but a little fallen from the sky with some circular reasoning): That is to say that we use the Lagrangian formalism and seek by trial and errors a Lagrangian density which allows us to fall back on something known.
	\end{enumerate}
	Well this having been said, let us recall before starting some results that we have obtained so far. First, we have succeeded in proving brilliantly that every particle (assumed to be free but left to interpretation ... in a curved space ...) follows the equation of motion of geodesics:
	
	In the section of Tensor Calculus, we have proved (not without difficulty...) what we name the "\NewTerm{Einstein's tensor}\index{Einstein's tensor}" (which is a constant in a given Riemannian space) is given by:
	
	where $R^{\mu\nu}$ is for recall the Ricci tensor (\SeeChapter{see section Tensor Calculus}).

	Since the covariant derivative of the Einstein's tensor is zero (\SeeChapter{see section Tensor Calculus}) and we have proved that the covariant derivative of stress-energy tensor is also, then it is tempting to put:
	
	where $\kappa$ is a normalization constant and must satisfy the relation so that it is homogeneous at the level of the units. So it comes (we should better say: "we think we can write...") after simplification:
	
	To find the expression of the constant, we will place ourselves in the Newtonian limit and request that the preceding relation reproduce the Poisson's equation for the gravitational potential $\Phi$ (\SeeChapter{see section Astronomy}):
	
	\begin{tcolorbox}[title=Remark,colframe=black,arc=10pt]
	This relation shows that the gravitational potential is connected to the matter density linearly through its second derivatives. Albert Einstein thought, therefore, that the first member of the equations of the field in General Relativity, member supposed to describe the geometry of space-time, must therefore somehow include the second derivatives, not of the gravitational potential, but of the potentials of the metric. In fact, Albert Einstein tried to generalize the right-hand side of the Poisson equation: the desired quantity must include not only the density of matter but also the momentum (as soon as the body is moving, its energy increases and therefore its mass). To evaluate the gravitational effect of a body, it was therefore necessary to combine its mass at rest with its momentum. It was finally the stress-energy tensor of rank $2$ which is the generalization of the quadrivector momentum of Special Relativity.
	\end{tcolorbox}
	We have proved earlier above that in the Newtonian limit (weak field approximation):
	
	and in our definition of stress-energy tensor, for a distribution of matter at rest (or in a coordinate frame according to...) only the following component is non-zero:
	
	It follows that the Poisson equation can be written:
	
	Now let us return to the relation:
	
	By contracting the two members of the preceding relation, it comes:
	
	that is to say more explicitly (\SeeChapter{see section Tensor Calculus}):
	
	But, the Ricci scalar (\SeeChapter{see section Tensorial Calculus}) is given by:
	
	It comes therefore:
	
	Now in the special case of the Minkowski metric (with the signature $(-, +, +, +)$) it is immediate that:
	
	Therefore (implicity we continue the Minkowski metric!):
	
	Using this last relation, the equation:
	
	can finally be written:
	
	Let us focus on the component $\rho=\sigma=0$ (not to be confused with the notation of shear stress and density!!!) such that the preceding relation is written:
	
	Let us write explicitly this last relation by using the definition of the Ricci tensor (\SeeChapter{see section of Tensor Calculus}) that is for recall:
	
	Then it comes:
	
	But, the Riemann-Christoffel tensor developed in this particular case is given for recall by (\SeeChapter{see section of Tensor Calculus}):
	
	\begin{tcolorbox}[title=Remark,colframe=black,arc=10pt]
	In the absence of a gravitational field and in Cartesian coordinates, it is logical that all the Christoffel symbols are null. Indeed, the Christoffel symbols translate nothing more than the forces of inertia. But when we have a field of gravitation, the trajectories followed are no longer straight lines, even in the Newtonian case, then the Christoffels symbols are non-zero.
	\end{tcolorbox}
	In the approximation of the weak field slowly variable over time, the Christoffel symbols are of order $\mathcal{O}^1$ and their products are of order $\mathcal{O}^2$ and the temporal derivatives are negligible in front of the spatial derivatives. It therefore remains only the terms of order $\mathcal{O}^1$ such that:
	
	But, we have proved in the section of Tensor Calculus that:
	
	Since then:
	
	But in the weak field approximation, the variation of the metric with respect to time is negligible compared to the spatial variation (the approximation is somewhat pulled by the hair it must be said ...):
	
	Therefore, the relation:
	
	becomes:
	
	and we immediately notice that we fall back on the Poisson's equation if and only if:
	
	Constant which is sometimes named "\NewTerm{Einstein's constant}\index{Einstein's constant}". It follows immediately that the Ricci scalar is positive and therefore that we are locally in a spherical curvature space.

	The "\NewTerm{Einstein's field equations EFE}\index{Einstein's field equations}" is therefore in definitive form:
	
	or more conventionally:
	
	The left-hand part represents the curvature of space-time as determined by the metric and the right-hand expression represents a modelization of the space-time content of mass / energy. This equation can then be interpreted as a set of equations describing how the curvature of space-time is related to the mass-energy content of the Universe. These equations, as well as the geodesic equation, form the core of the mathematical formulation of General Relativity.
	
	The EFE  is therefore a dynamic equation describing how matter and energy modify the geometry of space-time. This curvature of the geometry around a source of matter is then interpreted as the gravitational field of this source. The movement of objects in this field is described very precisely by the equation of its geodesic.
	
	Similar to the way that electromagnetic fields are determined using charges and currents via Maxwell's equations, the EFE are used to determine the spacetime geometry resulting from the presence of mass–energy and linear momentum, that is, they determine the metric tensor of space-time for a given arrangement of stress–energy in the space-time. 

	On the other hand, we have just seen that Einstein's equation reduces to the laws of Newton's gravity by using the approximation of weak fields and slow movements. 
	
	These differential equations are in general a nightmare to solve, the Ricci scalars and tensors are contractions of the Riemann tensor, which include the derivatives and products of the Christoffel symbols, which are themselves constructed on the inverse metric tensor and on the derivatives of it. To compute the whole, it is possible to construct energy-momentum tensors that can invoke the metric as well. It is therefore very difficult to solve the Albert Einstein equations of fields in the general case. Exact solutions for the EFE can only be found under simplifying assumptions such as symmetry. Special classes of exact solutions are most often studied as they model many gravitational phenomena, such as rotating Black Holes and the expanding universe. Further simplification is achieved in approximating the actual space-time as flat space-time with a small deviation, leading to the linearised EFE. These equations are used to study phenomena such as gravitational waves.

	Since the stress-energy tensor has $16$ components, $10$ of which are actually unique (independent) since the tensor is symmetric, we can see the Einstein equation of the fields as $10$ second-order differential equations coupled on field tensor metric $g_{ij}$.
	
	Some people are confused about how the curvature of space-time and gravity are related. I am going to explain mainly that starting with simpler examples, and moving to more complicated ones.

	Okay, let's say we have a sheet of rubber. This is the classic example of spacetime. Let's say wet ake a bowling ball, and set it on the taut sheet of rubber. It has a large mass (compared to what else we'll be putting on the sheet), therefore the sheet curves a lot for the bowling ball. We now have an image in our head like the one below:
	\begin{figure}[H]
		\centering
		\includegraphics[scale=0.8]{img/cosmology/general_relativity_2d_space_curvature.jpg}	
		\caption{2D naive representation of space curve near Earth}
	\end{figure}
	So mass leads to curvature. Then, let us take a baseball, say, and set it near the bowling ball. It rolls toward the bowling ball, right? This occurs because of the curvature of the sheet. So, then, curvature leads to gravity. So, if an object has large mass, it will curve space-time dramatically, leading to strong gravity.

	This is, of course, an overly simplistic example. It is 2D, and it doesn't take into account other factors. Let us move to 3D (keeping in mind the universe is accepted to be at 4D, ignoring the holographic principle). The mass of a bowling ball now sucks in space around it, sort of like in the picture below:
	\begin{figure}[H]
		\centering
		\includegraphics[scale=0.9]{img/cosmology/general_relativity_3d_space_curvature.jpg}	
		\caption{3D naive representation of space curve near Earth}
	\end{figure}
	And now, in this case, we can see (or understand) that more mass still leads to more curvature. The greater the mass, the more space-time will "contract" around the object. So we still think that mass leads to curvature. Now, if we set an object near this massive object (like the moon next to Earth) it is "sucked in" sort of, by the curvature of space-time, though of course the moon contracts space-time around it as well. At this point, we can reasonably still conclude that in 3D, mass leads to curvature which leads to gravity.
	
	A quick glance at the oconstant of prooertionality in the Einstein field equtions gives one a roguht feeeling of much stres-energy is needed to curve space. In SI units, the gravitational $G$ is about $6.67\cdot 10^{-11}\;[\text{m}^3\cdot\text{kg}^{-1}\cdot\text{s}^{-2}]$ while the spped fo light $c$ is approximately $3.00\cdot 10^8\;[\text{m}\cdot \text{s}^{-1}]$. The field equation reat then in numertical value:
	
	The sun has an average mass-energy density (the dominant component of the stress-energy tensor) of $T^{00}\cong 1.27\cdot 10^{20}\;[\text{kg}\cdot\text{m}^{-1}\cdot\text{s}^{-2}]$. The corresponding component of the Einstein Tensor is therefore $G_{00}\cong 2.64\cdot 10^{-23}\;[\text{m}^{-2}]$. By comparison the Einstein tensor for the flat Minkowski metric is identically zero. So to see a curvature we need to look at hyperenergetic phenomena, like a collapsing star, to fin an Einstein tensor component appreciable greater than this. Even though the space-time metric $g_{\mu\nu}$ is not generally flat, throughout most of the universe it is flat enough to be considered as small perturbation of a flat background metric:
	
	But, as I said earlier, the Universe is generally thought of as 4D. What does our picture look like when we add time? Well, the time dimension is contracted around a massive object. So let us picture our previous example but that the fabric of space-time has a few clocks embedded in it occasionally. As the space stretches and contracts, so will the clocks (the "time") and so the time on those clocks will be "wrong" - it'll differ from the other clocks. And in this case, as the Earth contracts space and time around it, it changes the time and space (it curves space-time) and so when another object enters our region of space-time, it is "sucked in" still, but so is it's time. This is, of course, a very extreme example, but I hope this shows that we can conclude that mass leads to curvature which leads to gravity. 
	
	\pagebreak
	\subsubsection{Cosmological Constant}
	Albert Einstein modified his original field equations to include a cosmological constant term $\Lambda$ proportional to the metric that led afterwards the Universe model to be static (\SeeChapter{see section Cosmology})
	To see how this constant was introduced let us recall that we have proved so far that:
	
	or more explicitly:
	
	That is to say:
	
	But we have proved in the section of Tensor Calculus that the covariant derivative kills the metric, that is to say for recall:
	
	Therefore if we choose a constant $\Lambda$ the latter relation can also we written:
	
	Obviously:
	
	So nothing avoid us to put this covariant derivative in:
	
	as we can write:
	
	and replacing the $0$ by the covariant derivative of the metric:
	
	After factorization we get:
	
	And simplifying we get the "\NewTerm{general Einstein's field equation}\index{general Einstein's field equation}":
	
	where $\Lambda$ is the so named "\NewTerm{cosmological constant}\index{cosmological constant}" (a.k.a. dark energy in conteporary physics).
	
	The latter relation can be found also in many textbooks in natural units (\SeeChapter{see section Principia}) and rearranged a little bit as following:
	
	The effort from Albert Einstein to introduce this constant was unsuccessful because:
	\begin{itemize}
		\item The universe described by this theory was unstable
		\item Observations by Edwin Hubble confirmed that our Universe is expanding
	\end{itemize}
	So, Albert Einstein abandoned $\Lambda$, calling it the "biggest blunder [he] ever made".

		Despite Albert Einstein's motivation for introducing the cosmological constant term, there is nothing inconsistent with the presence of such a term in the equations. For many years the cosmological constant was almost universally considered to be $0$. However, recent improved astronomical techniques have found that a positive value of $\Lambda$  is needed to explain observations that seems to give an accelerating universe.	
	\begin{figure}[H]
		\centering
		\includegraphics[scale=1]{img/cosmology/einstein_efe_leiden.jpg}	
		\caption{Diagram of gravitational lensing  with formula of Albert Einstein on a wall of Museum Boerhaave, Leiden in Netherlands (source: Wikipedia, author: Stichting Tegenbeeld,  photograph: Vysotsky)}
	\end{figure}
	
	\pagebreak
	\subsubsection{Schwarzschild Solution}
	The "\NewTerm{Schwarzschild metric}\index{Schwarzschild metric}" is an approximate solution of the EFE  in the case of an isotropic non-rotating gravitational field, without electric charge, zero universal cosmological constant  and at a great distance from the source. It provides the three main proofs of General Relativity: the shift of clocks, the deviation of light by a dense celestial body and the advance of the perihelion of Mercury. These three proofs are very important because Einstein's equation was not experimentally demonstrated at the time.  The solution is a useful approximation for describing slowly rotating astronomical objects such as many stars and planets, including Earth and the Sun. The solution is named after Karl Schwarzschild, who first published the solution in 1916.

	To introduce this metric, let us imagine a source (for example the Sun) which produces a gravitational field by means of its mass $M$. We seek, in order to compare with the experiment, the solutions of Einstein's equation (in other words: the metric) outside the source (of the Sun therefore ...) of mass $M$.
	\begin{tcolorbox}[title=Remark,colframe=black,arc=10pt]
	There are several mathematical techniques to introduce the Schwarzschild metric. The reader will be able to search, for example, in the literature or on the Internet the one using a gauge transformation ("Einstein gauge" with the "harmonic gauge") for the local perturbation constraint. This method is very elegant but very math oriented and we prefer as the reader alreaydy know it the "engineer" method...
	\end{tcolorbox}
	In other words, this is like assuming to have in the region of space that interests us (considering that there is only the star in question and nothing else around, not even the energy / mass specific to the gravitational field) the following property:
	
	So the EFE proved just above without cosmological constant:
	
	then becomes:
	
	But we proved above that this last relation can also be written using the definition of the Ricci scalar that is given for recall by:
	
	as following:
	
	and since the parenthesis is not null since we have proved above that in the Minkowski metric:
	
	it remains:
	
	and therefore in extenso the scalar of Ricci is also null. This last relation is named the "\NewTerm{vacuum field equations}\index{vacuum field equations}".
	\begin{figure}[H]
		\centering
		\includegraphics[scale=1]{img/cosmology/einstein_coin_vacuum_equation.jpg}	
		\caption{Swiss commemorative coin showing the vacuum field equations with zero cosmological constant (top) and action minimization}
	\end{figure}
	We must therefore find the metric that satisfies this relation (in other words, a metric that far from the source corresponds to a flat space since the Ricci tensor is zero). As there are several possibilities let us focus on a particularly elegant case with as the physicists like it ... full of symmetries.

	The idea is therefore to find a metric, if possible independent of time (therefore the gravitational field as well will be independent of time!) and ... with spherical symmetry (a star or planet being itself of this form), taking into account the mass of the star (this is the major objective!) and such that far enough from the source (...) or when the mass is zero we fall back on the classical metric known and see earlier above:
	
	But this is not totally accurate! Indeed, we work in space-time. But, we have seen that the equation of the curvilinear metric is given in a flat space-time by:
	
	by passing in spherical coordinates we then have:
	
	And it is on this equation of the metric that we must fall back when we are far from the source or that the mass is extremely small ($M=0$). That is to say the Schwarzschild metric must therefore be asymptotically flat, that is to say corresponding to the flat space of Minkowski.

	So let's get to the task. First, we start from what we know (it's better if we can...!). Which means:
	
	And in spherical coordinates including time we have the components $r,\theta,\phi,t$. Rigorously, we denote by:
	
	the "\NewTerm{Schwarzschild coordinates}\index{Schwarzschild coordinates}".
	
	On a total of $16$ terms implied by the prior-previous relation, we finally retain $10$ namely: the $4$ terms of the diagonal and the $6$ other terms of interaction so as to obtain:
	
	where $A$, $B$, $C$, ... are coefficients to be determined.

	Before tackling this work, we know that according to one of our starting constraints, when the mass is weak or we are far from a  non high-speed rotating source, we must therefore fall back on:
	
	therefore intuitively we can already write:
	
	what we must admit it ... is a clear progress ...!

	If as we have imposed it to ourselves at the beginning, the equation of the metric is independent of time, we can by symmetry of time (hypothesis ...) make the following change of variable:
	
	without this changing anything in our $\mathrm{d}s^2$. But, we realize very quick that this will not be the case. Immediately, for this to be satisfied we see that we must have:
	
	Which brings us (it's already better!) to:
	
	Now if the system is indeed spherical, the equation of the metric must be invariant by the transformation $\mathrm{d}\phi=-\mathrm{d}\phi$ (the opposite would be known for a long time if this were not the case experimentally) and/or also for the transformation $\mathrm{d}\theta=-\mathrm{d}\theta$.

	So for this to be correct, we see immediately that in the preceding relation we must impose:
	
	So finally it only remains:
	
	where $A$, $B$, $C$, $D$ will obviously be independent of time (the opposite would contradict our initial constraint) but may by symmetry of the sphere may be dependent of $r$ such that:
	
	Now, let us imagine on the sphere (rigorously it is a hypersphere but it helps anyway...) at a fixed distance $r$ from the center of the source of the field at a given instant $t$ fixed. We then only have:
	
	since $\mathrm{d}t$ is zero (fixed time) and $\mathrm{d}r$ also (fixed distance $r$).

	We have also on the way removed the sign $-$ because we anticipated that it will be eliminated in the third equality that will follow and we will put it then back.

	Now, let imagine we close to the north pole of the sphere ($\theta=0$) we then only have in first approximation:
	
	and at equator ($\theta=\pi/2$):
	
	By symmetry of the field, an infinitesimal angular displacement in each of these two particular zones must, however, be equal. From then on, we can only put:
	
	Hence the equation of the metric is reduced to:
	
	Let us now show that we can choose a system of coordinates for which $C(r)=1$.

	Let us introduce for this a distance defined by:
	
	hence:
	
	Therefore it comes:
	
	hence:
	
	This is further simplified by:
	
	Let's put it all to the square and divide it by left and right by $C(r)r^2=\bar{r}^2$:
	
	Therefore after rearranging a bit:
	
	hence:
	
	hence:
	
	Hence the equation of the metric is written:
	
	It is therefore as if $C(r)=1$:
	
	Therefore:
	
	Therefore:
	
	and the corresponding contravariant metric tensor (that we will further below):
	
	such that for recall (\SeeChapter{see section Tensor Calculus}):
	
	Now, to determine the remaining coefficients (that is, $A$ and $B$) we are going to use the relation that must satisfy metric if it is locally of the Minkowski type:
	
	and therefore the first Bianchi's identity (\SeeChapter{see section Tensor Calculus}) will be automatically satisfied.

	Either in a developed form (\SeeChapter{see section Tensor Calculus}):
	
	with obviously (\SeeChapter{see section Tensor Calculus}):
	
	That is to say that we have quite a lot of work to do... OK! First since the metric is simple the only non-zero derivatives are:	
	
	We then simply deduce the $9$ non-zero elements of the connection (the details are given following the request of a reader):
	
	
	
	
	
	
	
	
	
	
	
	
	
	
	
	
	
	To summarize (we have taken the results with the signature $-, +, +, +$ of the metric instead of $+, -, -, -$ to conform ourselves to the tradition but this does not change the final result):
	
	Now that we have these terms of the connection, we have to calculate their derivative in order to be able to express the first two terms of:
	
	There are then $10$ non-zero terms which are:
	
	We finally have for each component of the Ricci tensor:
	
	The only elements directly non-zero are then:
	
	In a more conventional form (according to the literature) we can simplify a little and moreover keep only the first three equations:
	
	If we add the first two equations, we have:
	
	which equals:
	
	And this also gives us:
	
	We have therefore:
	
	which becomes:
	
	Where we have divide by $2A$ when passing from the second to the third line.

	The reader can verify that a solution of the differential equation is (we can provide the details on request):
	
	Where $S$ is a non-zero real constant. Consequently, the metric for a static solution, symmetrically spherical and in the vacuum (...), is written:
	
	It remains for us to determine a coefficient. But as:
	
	It comes:
	
	Hence:
	
	Finally:
	
	Let us notice that the space-time represented by this metric is asymptotically flat, or, in other words, when $r\rightarrow +\infty$ the metrich approaches that of Minkowski and the space-time variety resembles to that of the Minkowski's space.

	To calculate the constants $K$ and $S$, we use the weak field approximation. In other words, we place ourselves far from the center, where the gravitational field is weak. In this case, the component $g_{tt}$ of the metric can be calculated.

	Indeed, we had studied the Newtonian limit above and obtained the following relation:
	
	with (\SeeChapter{see section Astronomy}) $\varphi=GM/rc^2$. So in extenso we can put without too much fear:
	
	Therefore:
	
	Finally we have for the "\NewTerm{Schwarzschild metric}\index{Schwarzschild metric}":
	
	That is to say in natural units:
	
	What ultimately gives the Schwarzschild metric tensor:
	
	Caution!!! Some reference books have the Schwarzschild metric with different signs because they take the metric $-, +, +, +$ instead of the metric $+, -, -, -$.
	
	Caution!!! The Schwarzschild metric is a solution of Einstein's field equations in \underline{empty space}, meaning that it is valid only outside the gravitating body. That is, for a spherical body of radius $R$ the solution is valid for $r > R$. To describe the gravitational field both inside and outside the gravitating body the Schwarzschild solution must be matched with some suitable interior solution at $r = R$, such as the "\NewTerm{interior Schwarzschild solution}\index{interior Schwarzschild solution}".

	An all (physically) apparent singularity appears when:
	
	Or in other words, when the coordinate of the radius $r$ is equal to:
	
	This radius, which we had already determined during our study of Classical Mechanics, is named the "\NewTerm{Schwarzschild radius}\index{Schwarzschild radius}".

	Therefore the Schwarzschild solution appears to have singularities at $r = 0$ and $r = 2GM/c^2$; some of the metric components "blow up" at these radii. Since the Schwarzschild metric is only expected to be valid for radii larger than the radius $R$ of the gravitating body, there is no problem as long as $R > 2GM/c^2$. For ordinary stars and planets this is always the case. For example, the radius of the Sun is approximately $700'000$ [km], while its Schwarzschild radius is only $3$ [km].
	
	The Schwarzschild radius is defined as the critical radius provided by the Schwarzschild geometry, below which nothing can escape: if a Star or other object reaches a radius equal to or less than its Schwarzschild radius Then it becomes a "\NewTerm{Black Hole}\index{Black Hole}", and any object approaching at a distance from it less than the Schwarzschild's ray will not escape from it. The term is used in physics and astronomy to give an order of magnitude of the characteristic size to which general relativity effects become necessary for the description of objects of a given mass. The only objects that are not Black Holes and whose size is of the same order as their Schwarzschild radius are neutron stars (or pulsars), thus, curiously, also the observable Universe as a whole...
	
	\begin{tcolorbox}[title=Remarks,colframe=black,arc=10pt]
	\textbf{R1.} The singularity in the metric when the Schwarzschild radius is reached is apparent because it is only an effect of the coordinate system used. It is an instance of what is named a "\NewTerm{coordinate singularity}\index{coordinate singularity}". As the name implies, the singularity arises from a bad choice of coordinates or coordinate conditions. When changing to a different coordinate system (for example Lemaitre coordinates, Eddington–Finkelstein coordinates, Kruskal–Szekeres coordinates, Novikov coordinates, or Gullstrand–Painlevé coordinates) the metric becomes regular at $r=2GM/c^2$\\
	
	\textbf{R2.} A remarkable theorem states that the Schwarzschild metric is the only solution to Einstein's equations in vacuum possessing spherical symmetry. As the Schwarzschild metric is also static, this shows that in fact in vacuum any spherical solution is automatically static. One interesting consequence of this theorem is that any pulsating star that remains spherically symmetrical can not generate gravitational waves (since the space-time region outside the star must remain static).\\
	
	\textbf{R3.} As the previous developments are based on the assumption of mathematical tools (Bianchi's identity) that requires a zero torsion tensor (\SeeChapter{see section Tensor Calculus}), there are more complete models that can not by extension use the Einstein equation of fields.
	\end{tcolorbox}	
	Now that we have the Schwarzschild metric we come back to the Schild criterion that we saw in our classical study of the Einstein effect.

	If we rewrite the Schwarzschild metric for an static body, we have the metric which is simplified into:
	
	By using the gravitational potential (\SeeChapter{see section Astronomy}):
	
	The metric is written:
	
	hence by introducing the proper time:
	
	hence:
	
	Therefore:
	
	Maclaurin's second-order expansion in series (\SeeChapter{see section Sequences and Series}) of the negative root gives:
	
	Therefore we have:
	
	Thus, this proof that the curvature (gravitation) generates a larger time dilation (in the sense that it flows faster) that the field of gravity is intense (mass $M$ is large) or that we are close to the body under the influence of the field (small radius $r$).

	For the Earth, the term:
	
	is relatively small. But for a Black Hole or a Neutron star, this is no longer the case and the dilation becomes important and the effects accessible to the measure.
	
	\subsection{Experimental Tests}	
	We will now review the $4$ classical experimental checks of the $20$th century of the General Relativity theory which are:
	\begin{enumerate}
		\item The precession of the perihelion which, in terms of numerical results, posed a problem for us with the tools of Classical Mechanics (\SeeChapter{see section Astronomy}).

		\item The deflection of electromagnetic waves (light) passing close to a massive stellar body which in the numerical results also posed a problem to us with the tools of Classical Mechanics (\SeeChapter{see section Astronomy}).

		\item The proof of the Schild criterion (already made in the preceding paragraphs) as the only way to explain rigorously the gravitational redshift and the hypothesis of slowing down time in a gravitational field.

		\item The delay of electromagnetic signals propagating near dense bodies. Delay referred to as "Shapiro effect" whose numerical applications are used for the operation of the G.P.S and which will be discussed later.
	\end{enumerate}
	\subsubsection{The Precession of Mercury's Perihelion}
	Let us now treat one of the most famous examples of General Relativity: the precession of the Mercury's perihelion. We had already dealt with this case in the Astronomy section, but we had mentioned that the theoretical numerical result did not correspond to the experimental observations. We shall see in the equivalent of almost ten A4 pages of detailed developments how General Relativity makes it possible to reconcile theory and experience.

	To study this case, we will use the Lagrangian formalism seen in the section of Analytical Mechanics.

	First, let us recall that we obtained for the metric of Schwarzschild:
	
	What we will write by dividing by $\mathrm{d}s^2$:
	
	And to abbreviate the notations, we put $l=GM/c^2$ such that:
	
	Now let us recall that (\SeeChapter{see section Analytical Mechanics}) in natural units:
	
	So (it's very rude but it works ... This is physics!...):
	
	Finally it means that the Lagrangian is:
	
	The equations of Lagrange give us for the $\theta$ coordinate \SeeChapter{see section Analytical Mechanics}):
	
	with therefore:
	
	Hence:
	
	and:
	
	From where finally for the coordinate $\theta$:
	
	Let us do the same for $\phi$. First, we have:
	
	and:
	
	And it comes immediately from the application of the Euler-Lagrange equation:
	
	Let us do the same for $t$:
	
	And it comes here also immediately:
	
	Therefore:
	
	Now let us assume that the motion of Mercury is in the equatorial plane such as $\theta=\pi/2$. Hence, the relation obtained above:
	
	simplifies into:
	
	hence:
	
	We have, therefore, the expression of the Universe line, which, for recall, is:
	
	Which since $\theta=\pi/2$ (which is therefore a constant) is simplified into:
	
	Let us now do the following replacement:
	
	Which is therefore a constant as we have proved just above and also the following replacement (which is also a constant as we proved just above):
	
	In the universe line element and we get:
	
	Let us consider also $r$ as a function of $\phi$ then:
	
	hence:
	
	Thus, we can rewrite the universe line in the form:
	
	Let us make a change of variable by putting:
	
	hence:
	
	Which gives for our universe line:
	
	or:
	
	By differentiating:
	
	Or written differently:
	
	Which simplifies and factorize itself into:
	
	The first possible solution is obviously:
	
	Hence as $r=1/u$:
	
	The circular motion is thus also a solution of Kepler's problem in general relativity in a Schwarzschild field (ouf!).

	The other solution will be:
	
	Or written differently:
	
	it corresponds to the orbit of Kepler's problem.

	Let us do the comparison by considering in Newton's mechanics the motion of a particle of mass $m$ in a potential $V$. The Lagrangian (\SeeChapter{see section Analytical Mechanics}) is then:
	
	In polar coordinates we have already seen in different section (Vector Calculus and Astronomy) that the speed is then written:
	
	Using the Euler-Lagrange equation we have the equation of motion:
	
	Which give:
	
	hence:
	
	And as we have seen in the section Astronomy:
	
	Is the constant of areas. Let us introduce:
	
	Hence:
	
	and therefore:
	
	So:
	
	The equation:
	
	therefore becomes:
	
	But:
	
	hence:
	
	therefore:
	
	where:
	
	It is therefore the "\NewTerm{non-relativistic Binet formula}\index{non-relativistic Binet formula}" which gives the relation between $u = 1 / r$ and $\phi$ for a central force (\SeeChapter{see section Astronomy}). In the case of a Newtonian potential:
	
	Hence:
	
	with for recall:
	
	Now let us recall the form of that which we had obtained just before with the General Relativity:
	
	Thus, we see that the analogous term in relativity is:
	
	and that general relativity adds the term $3lu^2$. Now, as in General Relativity:
	
	Then:
	
	However, in the case of the approximation of weak fields:
	
	hence:
	
	So finally:
	
	That said, it is really interesting to note that the equation for General Relativity:
	
	can be interpreted as Binet's equation for Classical Mechanics:
	
	with the potential:
	
	with $\gamma=lK^2$.
	
	Let us now return to our equation:
	
	We would like to know if the second term on the right of the equality is negligible or not with respect to the first term on the right of the equality in order to be able to apply the theory of perturbations.

	We will first put with the help of the weak field approximation given above:
	
	Now let us calculate the ratio:
	
	Recall that in polar coordinates (\SeeChapter{see section Vector Calculus}):
	
	In approximation, we can roughly put that:
	
	Therefore for Mercury ...:
	
	So we see immediately that we can apply the variational theories to the term $3lu^2$. Thus, let us put:
	
	The equation:
	
	takes the shape:
	
	To solve this differential equation, we will use the perturbation theory approach (\SeeChapter{see section Differential and Integral Calculus}). We will therefore focus on a solution of the form of a Taylor expansion in second order only in $\varepsilon$:
	
	where $u_0$, $u_1$ are obviously dependent on $\phi$ and will have to be determined! To do this, we know that we must replace the previous expression in the differential equation such that:
	
	Which simplifies into:
	
	where let us recall that:
	
	is the classical equation obtained earlier above:
	
	Let us consider the solution of the type:
	
	where $D$ is an arbitrary constant. Now, as we have seen in the section of Astronomy in the case of the precession of perihelion:
	
	is actually an ellipse. Which means that any solution of the form:
	
	is also an ellipse!
	
	For the equation in $\varepsilon$:
	
	which is simplifies into:
	
	Since (\SeeChapter{see section of Trigonometry}):
	
	It comes:
	
	To determine $u_1$, let us decompose it into three terms:
	
	This gives us immediately (by injecting the three terms respectively into the second derivative and the term alone):
	
	So finally:
	
	The solution sought is finally:
	
	It is therefore with:
	
	that it is necessary to calculate the displacement of the perihelion (we arrive soon... pfiuuuuu...).
	
	We see relatively quickly by observing the preceding relation that the only term whose amplitude is not constant is $\varepsilon D\phi\sin(\phi)$.

	Let us then recall that (\SeeChapter{see section Trigonometry}):
	
	This can also be roughly written as a first approximation using Maclaurin's first-order expansion (\SeeChapter{see section Sequencers and Series}):
	
	We know that the zero order orbit is:
	
	The effect of the last term:
	
	is therefore to introduce a small periodic variation in the radial distance. This term does not affect the displacement of the perihelion. This is the term $\varepsilon\phi$ in:
	
	which introduces a non-periodicity which can be non-negligible in the case where $\phi$ is large.
	
	The perihelion (the point closest to the Sun for recall) therefore appears when $r$ is the minimum therefore $u=1/r$ maximum. But, $u$ is maximum when the term which interests us is maximum, that is to say:
	
	We have approximately:
	
	For two successive perihelions, we have an interval:
	
	instead of $2\pi$. Thus, the displacement for a revolution is:
	
	where $K$ is therefore the constant of the areas and $M$ the mass of the central star and since:
	
	Finally we have in the end:
	
	Relation to be compared with that obtained in the section of Astronomy with a Classical Newtonian treatment:
	
	We thus fall back at the perfection on the factor $6$ which was lacking in the conventional treatments!

	For Mercury a numerical application gives:
	
	and the experiment gives $\delta\phi\cong 42.5''\pm 1.0''$. By Albert Einstein's own admission, in obtaining this result he had palpitations and the impression of grazing a heart attack and satisfied with his Herculean effort which had exhausted him he took a long period of rest.
	\begin{tcolorbox}[title=Remark,colframe=black,arc=10pt]
	It is perhaps useful for the reader to know that Albert Einstein and Michele Besso took almost $2$ years (!!!) by trials and errors to found the good result above. The first time they had an error of $4000\%$ in comparison to the experimental observed value, the second time an error of $400\%$ and finally the value above (after that Albert Einstein had identified that he choose the wrong Tensor for his theory).
	\end{tcolorbox}
	To conclude on this subject, let us mention a second frequent writing in the literature concerning the result obtained. Indeed, we have proved in the section of Astronomy that the focal parameter was given by:
	
	It therefore remains:
	
	and we have also proved in the section of Analytical Geometry that:
	
	It thus comes in the end the most classic form:
	
	
	\subsubsection{Deflexion of Light}
	We have just proved that:
	
	By replacing the factors by their respective values, we have:
	
	But we have seen above that:
	
	and as $K$ is the areas constant given by the conservation of the momentum itself constant (\SeeChapter{see section Classical Mechanics}):
	
	We then have for a photon $m\rightarrow 0\Rightarrow K\rightarrow +\infty$.
	
	Let us put now to simplify the notations:
	
	Then:
	
	The term to the right of the equality is small (considering the constants that intervene therein) so that an approximate form of the differential equation is:
	
	of which a particular solution, which we know in advance, is interesting:
	
	We carry this approximated solution in the initial differential equation and we get:
	
	Therefore:
	
	Hence:
	
	What follows is going to be very subtle (how to guess something like that ...?). First we will create a new differential equation:
	
	The trick is to multiply this equation by $\mathrm{i}$ and sum it to the original differential equation:
	
	What we will denote by:
	
	Another trick is to look for a particular solution of the previous relation in the form:
	
	Then we have:
	
	This injected into our new differential equation gives:
	
	We deduce immediately:
	
	A particular solution of the original differential equation is thus:
	
	Either by using the remarkable trigonometric relations (\SeeChapter{see section Trigonometry}):
	
	It comes:
	
	The general solution is:
	
	If we admit that the light is very weakly deviated by the Sun, the radius of curvature ($1/r$) of its trajectory will be very small.

	Therefore:
	
	such that:
	
	The first term is predominant relatively to the second because of the factor $r_g$ that is very small on the second. For what will follows, we will proceed as in the in the section Astronomy (only the notations change) for the study of the deflexion angle (if you don't come back to it, it can be difficult to understand the justification of what will follow!). We put without loosing in generality:
	
	Therefore:
	
	and as:
	
	it comes:
	
	Using trigonometric identities again:
	
	It comes:
	
	$\theta$ being supposed as very small we do a Maclaurin development (\SeeChapter{see section Sequences and Series}) to the first order of the trigonometric functions:
	
	Which gives:
	
	Therefore after a series of approximation... and of hypothesis at the limit of what is acceptable..., we get ($R$ is sometimes named the "\NewTerm{impact parameter}, that is to say he distance of nearest approach of the light-beam to the center of mass):
	
	instead of the result that we get following the Newtonian approach in the section Astronomy:
	
	We thus founded the factor $2$ that was missing in the classical treatment, relatively to experimental observations, that we have proved in the section of Astronomy:
	
	What is often pictured in the media by the following drawing:
	\begin{figure}[H]
		\centering
		\includegraphics[scale=1]{img/cosmology/light_deflexion.jpg}	
	\end{figure}
	This deviation have been observed experimentelly by measuring the position of stars in the vicinity of the solar disk during the 1919 eclipse by Arthur Eddington and his team. After the advance of the perihelion of Mercury, this was the second test successfully passed by the General Relativity. It was this event that made Albert Einstein famous among the general public. Today, the deviation of light rays can be measured with much greater precision by considering radio signals emitted by extragalactic sources (quasars, AGN, etc.): the prediction of the General Relativity has been confirmed to the nearest thousandth.

	The deviation of light rays is today very important in observational cosmology,
Since it is at the origin of the phenomenon of gravitational mirage, also named "gravitational lens".

	It is interesting to notice that the whole theory of gravitational mirages is based on the relation:
	
	at least for a point detector. It is the only ingredient of General Relativity used in the calculation of images.
	
	In observational astronomy an "\NewTerm{Einstein ring}\index{Einstein ring}", also known as an "\NewTerm{Einstein-Chwolson}\index{Einstein-Chwolson}" ring or "\NewTerm{Chwolson ring}\index{Chwolson ring}", is the deformation of the light from a source (such as a galaxy or star) into a ring through gravitational lensing of the source's light by an object with an extremely large mass (such as another galaxy or a Black Hole). This occurs when the source, lens, and observer are all aligned.
	\begin{figure}[H]
		\centering
		\includegraphics[scale=0.55]{img/cosmology/einstein_ring_lrg_3_757.jpg}	
		\caption{Einstein Ring LRG 3 757}
	\end{figure}
	
	\subsubsection{Shapiro Effect (delay)}
	In 1964,  Irwin Shapiro demonstrated that a ray of light was not only deflected by passing near a mass, but also that the duration of its path was lengthened in relation to a Euclidean geometry. He calculated that the delay should be about $200$ microseconds, therefore perfectly measurable, for a line of sight shaving the Sun. He then suggested systematically measuring the time taken by a radar signal to make the round trip between the Earth and a planet passing behind the Sun (so that the effect is maximal). This was first accomplished with radar echoes on Mars, Venus or Mercury, with an accuracy of the order of $20\%$. The result was very clear: the time required for a radar signal to make the go and come back between the Earth and the other Planet increases suddenly just before the planet passes behind the Sun and decreases just as suddenly when it reappears.
	\begin{tcolorbox}[title=Remark,colframe=black,arc=10pt]
	We sometimes also talk of "slowing down of the light" near the Sun to describe the Shapiro effect, but it is an awkward and erroneous expression. As we have already been mention it, the speed of light is constant in General Relativity as well as in Relativity for all observers (but for recall this doesn't mean that the speed of light in constant during the life of our Universe!). In the case of the Shapiro effect (and in other similar cases), what changes is the flow of time where the light passes, in relation to what it is where the observer is located.
	\end{tcolorbox}
	Although this is a weak effect, it has been verified precisely since the arrival of the Viking probes on Mars in 1976, using signals sent from Earth to Mars and reflecting on the latter by the probes (see the principle of the experiment in the figure further below). In addition, there is now even an increasingly common object for which the Shapiro effect must be taken into account: the "G.P.S." (Global Positioning System). Indeed, despite the weakness of the field of gravitation, a geographical precision of a few meters requires such details in the calculation! However, a satellite has recently been launched to verify in the Earth's gravitational field an even lower effect predicted by General Relativity and which does not even intervene in GPS: the drag over of space also known as the "\NewTerm{Lense-Thirring effect}" due to the rotation of Earth.
	
	Let us point out for the GPS that two phenomena of error are known within the framework of the Relativity:
	\begin{enumerate}
		\item The satellites rotating around the Earth at a speed of approximately $20,000$ kilometers per hour then delay $7$ millionths of a second per day (Relativity).

		\item At an altitude of $20,200$ kilometers, that of the satellite orbit, the lower gravitational field advances the satellite clocks by $45$ millionths of a second per day.
	\end{enumerate}
	The sum of the two corrections gives a drift of $38$ millionths of a second per day, a staggering figure for a GPS system whose precision must be $50$ billionths of a second per day!!!

	Let us make the calculation for a ray touching the surface of the Sun. For this, we take up our Schwarzschild's metric given for recall by:
	
	with:
	
	For a photon, we know that $\mathrm{d}s=0$ and therefore the equation of the Schwarzschild's metric is then written:
	
	The trajectory of the photon taking place in the equatorial plane of the Sun, we put:
	
	which simplifies even more the equation of the metric by:
	
	To simplify even more, we make the hypothesis that the trajectory (in polar coordinates) of the photon shaving the Sun is rectilinear such that (for one of the polar components of the plane):
	
	where $r_\odot$ is the ray of the Sun. We will use this assumption to simplify the equation of the metric. For this we rearrange:
	
	We derive (\SeeChapter{see section Differential and Integral Calculus}):
	
	If we square everything:
	
	hence:
	
	We can now rewrite the equation of the metric:
	
	Taking the square root:
	
	Since $r>r_\odot$ and $r_g\ll 0$ then:
	
	Therefore, we have using the Maclaurin developments (\SeeChapter{see section Sequences and Series}) to the first order:
	
	We have then:
	
	Finally, we get once condensed:
	
	What it is traditional to write (we take out the $1 / c$ of the different terms):
	
	If there is no mass then space-time is flat and $r_g=0$. Therefore:
	
	We can thus distinguish the classical time from the extra time generated by the curved space. The "delay" will therefore be given by:
	
	Then, to integrate the four functions of $r$, we must place ourselves in a repository placed if possible at the center of the main body (the Sun typically) since the Schwarzschild metric is based on this hypothesis for recall. Thus, to know the delay of a luminous ray starting from the Sun and travelling to the Earth, we logically choose as the radius of departure that of the Sun itself and as the radius of arrival, the distance Sun-Earth (this will correspond once the primitives computed at the integration terminals).
	\begin{figure}[H]
		\centering
		\includegraphics[scale=1]{img/cosmology/shapiro_effect.jpg}	
		\caption{Round-trip time of a signal as a function of the position of Mars}
	\end{figure}
	Well that says it's nice to know the notations of use, but it's even better to do a numerical application! We will therefore first determine the primitive of each of the terms below:
	
	The first two primitives are simple because they are usual primitives proved in detail in the section of Differential and Integral Calculus:
	
	where for the last primitive we have preserved the constant of integration (contrary to what was done in the section of Differential and Integral Calculus because $r_\odot\neq 1$).

	Now it remains to us the last two integrals. Let's start in the order by:
	
	By putting:
	
	and using the results proved in the section of Differential and Integral Calculus, we then have:
	
	Since we have (\SeeChapter{see section Trigonometry}):
	
	Then:
	
	Finally, it remains the last primitive:
	
	We put for what will follow:
	
	Therefore it comes:
	
	In the section of Differential and Integral Calculus we have proved that:
	
	and that:
	
	Therefore:
	
	To return to the integral of the beginning we remember that $r=\dfrac{r_\odot}{x}$. Therefore:
	
	We thus finally have by taking all the primitives calculated above and by choosing a starting and finishing terminal for the calculation:
	
	We see in the Newtonian limit case where $r_g=0$ that this relation is reduced to:
	
	So for a round trip (between planet and satellite for example), then it comes in this simplified case:
	
	In November 1976, when the two Viking spacecraft were operating on the surface of Mars, the planet went
behind the Sun as seen from Earth (see figure below). Scientists had preprogrammed Viking to send a radio wave toward Earth that would go extremely close to the outer regions of the Sun. According to General Relativity there would be a delay because the radio wave would be passing through a region where time ran more slowly. The experiment was able to confirm Einstein’s theory to within $0.1\%$.
	\begin{figure}[H]
		\centering
		\includegraphics[scale=0.7]{img/cosmology/shapiro_effect_viking.jpg}	
		\caption{Radio signals from the Viking lander on Mars were delayed when they passed near the Sun (source: OpenStax}
	\end{figure}
	
	\pagebreak
	\subsubsection{Black Holes}
	Always staying with at our Schwarzschild metric .... a radial trajectory of light-type implies:
	
	therefore:
	
	and in a direct radial trajectory (by definition!) we also have:
	
	therefore:
	
	Therefore:
	
	hence:
	
	so that:
	
	Let us change to natural units $c=1$. It then comes:
	
	When $r\rightarrow 2GM$ the right-hand side of this equality tends to $\pm \infty$, then the evolution of time $t$ (external observer) as a function of $r$ tends to infinity with respect to the proper time of light.

	The sphere given by the radius:
	
	defines the "\NewTerm{horizon}\index{horizon}" of a "\NewTerm{Schwarzschild Black Hole}\index{Black Hole}".
	
	Towards this limit boundary, the light seems to put an infinite time compared to an external observer to move when approaching a Black Hole. It therefore never really reaches it in relation to the observer, hence the fact that the Black Holes can be surrounded according to their environment by a luminous halo near the Schwarzschild radius. Moreover, since time seems to be stopped, the frequency of the light surrounding the Black Hole tends towards zero and therefore towards the infrared.

	A Black Hole is therefore a region of spacetime exhibiting such strong gravitational effects that nothing—not even particles and electromagnetic radiation such as light—can escape from inside it. The theory of general relativity predicts that a sufficiently compact mass can deform spacetime to form a black hole. The boundary of the region from which no escape is possible is named the "event horizon". 

	Objects whose gravitational fields are too strong for light to escape were first considered in the 18th century by John Michell and Pierre-Simon Laplace. The first modern solution of general relativity that would characterize a black hole was found by Karl Schwarzschild in 1916, although its interpretation as a region of space from which nothing can escape was first published by David Finkelstein in 1958. Black Holes were long considered a mathematical curiosity; it was during the 1960s that theoretical work showed they were a generic prediction of general relativity. The discovery of neutron stars sparked interest in gravitationally collapsed compact objects as a possible astrophysical reality.

	Black Holes of stellar mass are expected to form when very massive stars collapse at the end of their life cycle. After a Black Hole has formed, it can continue to grow by absorbing mass from its surroundings. By absorbing other stars and merging with other Black Holes, supermassive Black Holes of millions of solar masses may form. There is general consensus that supermassive Black Holes exist in the centers of most galaxies.

	Despite its invisible interior, the presence of a Black Hole can be inferred through its interaction with other matter and with electromagnetic radiation such as visible light. Matter that falls onto a black hole can form an external accretion disk heated by friction, forming some of the brightest objects in the universe. If there are other stars orbiting a black hole, their orbits can be used to determine the black hole's mass and location. Such observations can be used to exclude possible alternatives such as neutron stars. In this way, astronomers have identified numerous stellar black hole candidates in binary systems, and established that the radio source known as Sagittarius A*, at the core of our own Milky Way galaxy, contains a supermassive black hole of about $4.3$ million solar masses. 
	\begin{tcolorbox}[title=Remark,colframe=black,arc=10pt]
	If the Sun or any other celestial object collapse into a Black Hole, this doesn't affect its orbiting element as the Black Holes still have the same mass (only the density change) and Newton's law still remains valid at a quite significant distant of it. So it is wrong to imagine that if the Sun collapse into a Black Hole of $6$ [km] diameter it would exert a more big gravitational force and that all planets turning around it will be sucked in.
	\end{tcolorbox}
	
	On 11 February 2016, the LIGO collaboration announced the first observation of gravitational waves; because these waves were generated from a Black Hole merger it was the first ever direct detection of a binary Black Hole merger. On 15 June 2016, a second detection of a gravitational wave event from colliding black holes was announced.
	\begin{figure}[H]
		\centering
		\includegraphics[scale=0.6]{img/cosmology/black_hole.jpg}	
		\caption{2D naive representation of space curve near some celestial objects (source: OpenStax)}
	\end{figure}
	The Galactic Center Group members have been measuring the positions of thousands of stars in the vicinity of the Galactic Center for more than $20$ years. This unique data set allowed us to measure directly short-period orbits of stars. In particular, a full phase coverage has been measured for two stars: S0-2 with an orbital period of $15.56$ years, and S0-102 with $11.5$ years. At the closest approach, S0-2 is only $17$ light hours away from the center of the Galaxy, about four times the distance of Neptune from the Sun. From these orbital data, we can determine the mass of the central black hole in our own Galaxy.

	The Milky Way Galaxy center is the best candidate to what seems to a black hole, and especially and example of the closest supermassive black holes (SMBH), located at $\sim 25,000$ light years away from us and corresponds with the location of Sagittarius A* a bright and very compact astronomical radio source at the center of the Milky Way. Its mass is estimated to be $4$ million times the mass of the Sun, which implies that the Schwarzschild radius is about $17$ times that of Sun's radius. As a comparison, Mercury's orbit is located at a distance of $\sim 83$ solar radii. Because the Galactic Center is the site of the closest supermassive black hole by a factor of $100$, it is a unique laboratory for solving some of the greatest mysteries associated with the fundamental physics of supermassive black holes and the role that they play in the formation and evolution of galaxies. Furthermore, it is the only galactic nucleus in which direct measurements of stellar orbits is possible, with either the current or the next-generation instruments.
	
	Observations of stellar orbits around the Galactic black hole also yields precision measurement of the distance to the Galactic Center, which is important as it affects almost all questions not only of Galactic structure, dynamics and mass, but those of extragalactic distance scales and the value of Hubble's constant as well.
	\begin{figure}[H]
		\centering
		\includegraphics[scale=1]{img/cosmology/galactic_black_hole_center.jpg}	
		\caption{The orbits of stars within the central 1.0 X 1.0 arcseconds of our Galaxy (source: Galactic Group Center)}
	\end{figure}
	The origin of supermassive black holes  and also the assumption that each galaxy center is a black hole remains an open field of research. Astrophysicists agree that once a black hole is in place in the center of a galaxy, it can grow by accretion of matter and by merging with other black holes this is why some observations gives for example for the hyperluminous quasar S5 0014+81 a weight of approximately $40\cdot 10^10$ times the mass of the Sun.
	
	Are there any other magnitudes we should note or calculate in Black Hole physics and thermodynamics! Yes, there are. Assuming spherical symmetry, we can calculate the Schwarzschild area or event horizon/surface area of the Schwarzschild's Black Hole simple by:
   
	We can also calculate the surface gravity $\kappa$, if the gravitational field of the Black Hole reads
	
	then, at the Schwarzschild radius it becomes the mentioned surface gravity $\kappa=g(R=R_S)$:
	
	Interestingly, this surface gravity is $1/M$ the maximal force $c^4/4G$ allowed by natural units... What else? Surface tides, or more precisely, the tidal acceleration at the Black Hole surface. The tidal acceleration is calculated with (the reader notice that we take the version of the tidal force with the factor $2$ instead of factor $4$ as discusses during the proof of this relation in the section of Astronomy):
	
	If it is evaluated at $R_S$ we get:
	
	
	\pagebreak
	\subsubsection{Gravitational waves}
	Before we focus on the maths, let us do a simple introduction.
	
	"\NewTerm{Gravitational waves}\index{gravitational waves}" are ripples in the curvature of space-time that propagate as waves at the speed of light, generated in certain gravitational interactions that propagate outward from their source. The possibility of gravitational waves was discussed in 1893 by Oliver Heaviside using the analogy between the inverse-square law in gravitation and electricity. In 1905 Henri Poincaré first proposed gravitational waves emanating from a body and propagating at the speed of light as being required by the Lorentz transformations. Predicted in 1916 by Albert Einstein on the basis of his theory of General Relativity, gravitational waves transport energy as gravitational radiation, a form of radiant energy similar to electromagnetic radiation. Gravitational waves cannot exist in the Newton's law of universal gravitation, since that law is predicated on the assumption that physical interactions propagate at infinite speed.
	
	Gravitational-wave astronomy is an emerging branch of observational astronomy which aims to use gravitational waves to collect observational data about sources of detectable gravitational waves such as binary star systems composed of white dwarfs, neutron stars, and Black Holes; and events such as supernovae, and the formation of the early universe shortly after the Big Bang.
	
	On February 11, 2016, the LIGO Scientific Collaboration and Virgo Collaboration teams announced that they had made the first observation of gravitational waves, originating from a pair of merging black holes using the Advanced LIGO detectors. On June 15, 2016, a second detection of gravitational waves from coalescing Black Holes was announced. Besides LIGO, many other gravitational-wave observatories (detectors) are under construction.
	\begin{figure}[H]
		\centering
		\includegraphics[scale=0.75]{img/cosmology/ligo_measurements.jpg}	
		\caption{LIGO measurement of the gravitational waves at the Hanford (left) and Livingston (right) detectors}
	\end{figure}
	\begin{figure}[H]
		\centering
		\includegraphics[scale=0.75]{img/cosmology/ligo.jpg}
	\end{figure}
	In real life it looks like this:
	\begin{figure}[H]
		\centering
		\includegraphics[scale=0.95]{img/cosmology/ligo_washington.jpg}
		\caption[Hanford LIGO detector Washington state (source: LIGO)]{Hanford LIGO detector - October 30, 2000 - Washington state (source: LIGO)}
	\end{figure}
	The most typical illustration that we can see of gravitational wave of newspapers are that generated by the special case of a high speed dynamic gravitation field due to the rotation of two massive object around each other. Otherwise, and it is obvious, a collapsing Star at the end of its life also generated a variational gravitational field but that is quite difficult to detect with actual existing instruments as a star alone is not massive enough to generate detectable gravitation waves:
	\begin{figure}[H]
		\centering
		\includegraphics[scale=0.8]{img/cosmology/gravitational_wave.jpg}	
		\caption{Pseudo-3D (wrong) common visualization of gravitational wave of a binary system}
	\end{figure}
	But obviously gravitational waves don't make things go up and down like ocean waves as illustrated above, and they're definitely not like that planet on a trampoline — after all, there's nothing "below" to pull things downward so there can't be a dent.  And gravitational waves don't do spirals, much....
	
	In a gravitational wave, space itself is compressed and stretched.  A particle caught in a gravitational wave doesn't get pushed back and forth.  Instead, it shrinks and expands in place.  If you encounter a gravitational wave, you and all your calibrated measurement gear (yardsticks, digital rangers, that slide rule you’re so proud of) shrink and expand together.  You would only notice the experience if you happened to be comparing two extremely precise laser rangers set perpendicular to each other (LIGO!).  One would briefly register a slight change compared to the other one.
	
	
	Now let us deal with the maths. As almost always in science there are multiple ways to introduce a new tool. In this book, as most of times, we will use what we consider the most easy one and that is "physicist" or "engineer" oriented...
	
	So first, remember that earlier we have proved under some assumptions that:
	
	That what we have seen afterwards can be written more generally as:
	
	If we explicit it by keeping in mind Classical Mechanics it would be written in 3D:
	
	or a bit better:
	
	But... but...! We are in General Relativity and we have to introduce the 4-th dimensions:
	
	This is better but now let u explicit this using in cartesian coordinates using the metric $(- + + +)$. We then get:
	
	Assuming that we a observing a piece (volume) of space where there is no matter and no radiation in any form, then $T_{\mu\nu}=0$ and we get:
	
	Using the d'Alembertian already introduced in the section of Electrodynamics but now in its General Relativity form named "\NewTerm{flat-space d'Alembertian}\index{flat-space d'Alembertian}":
	
	that latter relation is commonly written in textbooks:
	
	Taking back and rearranging the explicit relation we get (we see that this equation also give us that space transmits gravitational waves at the speed of light!):
	
	and named the "\NewTerm{gravitational wave equation}\index{gravitational wave equation}", "\NewTerm{gravitational propagation equation}\index{gravitational propagation equation}" or "\NewTerm{gravitational d'Alembert's equation}\index{gravitatgional d'Alembert's equation}".
	
	As gravitational waves pass through boundaries that light cannot. They can transport information about what happens inside the event horizons of Black Holes, and they can pass through the cosmic background (CMB) radiation (\SeeChapter{see section Cosmology}), the barrier of light that prevent us from seeing our Universe before it turned $380,000$ years old!
	\begin{figure}[H]
		\centering
		\includegraphics[scale=0.8]{img/cosmology/gravitational_wave_cmb.jpg}	
		\caption{Gravitational cosmic background "radiation"}
	\end{figure}
	
	So for summary General Relativity has successfully with high accuracy pass the following experimental tests in order of verification (top oldest):
	\begin{enumerate}
		\item Mercury perhelie precession
		\item Light deviation
		\item Black Holes
		\item Universe expansion
		\item Time Dilatation
		\item Gravitational waves
	\end{enumerate}	

	Let us point out once again a very important point. Before Albert Einstein, geometry was considered an integral part of the laws. Albert Einstein has shown that the geometry of space evolves in time according to other, even deeper, laws. It is important to understand this point. The geometry of space is not part of the laws of nature (this was criticized at his time by many French physicists). Therefore, nothing that we can find in these laws tells us what geometry of space we are working in. Thus, before we begin to solve the equations of Einstein's General Theory of relativity, we have absolutely no idea what geometry we ear dealing with. We only discover it once the equations are solved!
	
	In extenso, the choice of $4$ dimensions is part of the background. Could it be possible that another deeper theory does not require presupposing the number of dimensions? 

	To sum up, the idea of independence relatively to the background, in its most general formulation, is a wise way of making physics: made up of better theories, in which the things which before were postulated, will be explained in allowing such things to evolve over time according to new laws.

	This is also a difficulty of Quantum Theory, that latter is essentially background dependent at the opposite of General Relativity that is "\NewTerm{background independent}\index{background independent}".
	
	\begin{flushright}
	\begin{tabular}{l c}
	\circled{90} & \pbox{20cm}{\score{3}{5} \\ {\tiny 33 votes,  61.82\%}} 
	\end{tabular} 
	\end{flushright}
	
	%to force start on odd page
	\newpage
	\thispagestyle{empty}
	\mbox{}
	\section{Cosmology}
	\lettrine[lines=4]{\color{BrickRed}C}osmology is concerned with understanding the birth and evolution of the Universe by the scientific method. It is only through this game between physical theories, models and observations that we will discuss this issue here. We will try to avoid carefully any metaphysical digression. The specific problems of cosmology fit in its definition: Statistics that is one of the great scientific methods is apparently poor: we only have one Universe visible to us at this day. Furthermore, we can only observe the past of our Universe. Can we speak about "predictions" in these conditions? The theories, however, are reliable since they predict behaviors that can be tested by observations.
	
	Cosmology mainly uses the arsenal of mathematics, theoretical physics, particle physics, nuclear physics, physics of detectors and astrophysics. It is interdisciplinary! Cosmology deals with scales larger than the size of a galaxy to the scales defined as itself as "Horizon". Even if the limit is deliberately vague, cosmology does not address the internal details of the birth and evolution of astrophysical objects (such as galaxies, globular clusters, and clusters of galaxies) that fall more into the study field of "\NewTerm{cosmogony}\index{cosmogony}".
	
	\begin{fquote}[]If you can explain a God without a creator, you can explain a Universe without a creator...
 	\end{fquote}
	
	\subsection{Newtonian Cosmological Model}
	A cosmological model is a mathematical representation of the Universe that seeks to explain the reasons for its present appearance, and describe its evolution over time (named "\NewTerm{cosmological time}\index{cosmological time}") but not its creation!
	
	The Newtonian model applies under the assumptions of Newtonian mechanics (instantaneous action for example). The results we are going to study here were discovered before the development of General Relativity but published after! But this model has the advantage of simplicity while being able to identify and discuss the dynamics of the Universe and to prepare for the study of the Universe models making use of the results of General Relativity afterwards. Its disadvantage, besides the fact that it is not quite fit the experimental results, it is to beno longer valid under extreme conditions and therefore cannot be extrapolated to the instant of the Big Bang.
	
	Before we begin, we must define the "\NewTerm{cosmological principle}\index{cosmological principle}" consisting of the following two assumptions (basically, it ensures that we are not privileged observers, and that what we are seeing is a representative of the whole of the Universe) :
	
	\pagebreak
	\begin{itemize}
		\item[H1.] The space (the Universe) is homogeneous, that is to say, it has the same properties in all its regions. This must be at a very large scale, beyond the thousand Mpc (megaparsecs). It is clear that small-scale inhomogeneities exist... we for example... \Winkey
		
		\item[H2.] The space (Universe) is isotropic, ie there is no specific direction in space, such as flattening a direction or an overall movement on a Universal scale for example and all its physical properties are almost identical in any point.
	\end{itemize}
	
	\begin{tcolorbox}[title=Remark,colframe=black,arc=10pt]
	This hypothesis of the isotropy of the Universe and that works relatively well in theoretical models (see below) requires an interesting fact if we admit a beginning to the Universe. This fact implies that the Universe had a phase in its history where it did not leave the time to the matter to clump together to form from its beginnings inhomogeneous and anisotropic material groups that are visible today in our telescopes. From this it follows that at a moment of its history, the Universe had an non-quasistatique expansion rate that we could make match with the speed of light (this is badly formulated but I hope it is still acceptable in the idea).
	\end{tcolorbox}
	The figure below shows the Automated Plate Measurement (APM) Galaxy Survey. Over $2$ million galaxies are depicted in a region of $100$
degrees across centered toward the Milky Way's south pole:
	\begin{figure}[H]
		\begin{center}
		\includegraphics[scale=0.55]{img/cosmology/apm_galaxy_survey.jpg}
		\end{center}	
		\caption{Automated Plate Measurement (APM) Galaxy Survey}
	\end{figure}
	\pagebreak
	We will also request other working assumptions/hypothesis:
	\begin{itemize}
		\item[H1.] The Universe is a non viscous gaseous fluid whose particles are galaxies. Assuming the cosmological principle, the movement of galaxies, constituents in-fine of this "fluid" by construction, are statistically at rest.
		
		\item[H2.] The Universe is a thermodynamically closed system, without work and adiabatic (no heat exchange with the outside).
		
		\item[H3.] The Universe in a homothetic expansion (in proportional expansion in all its dimensions) is taken as having a spherical shape with a center (yes this is the Newtonian model...).
		
		\item[H4.] Its density is only a function of time and there is mass conservation (and therefore energy). Therefore the amount of material is constant (yes this is always the Newtonian model...)!
		
		\item[H5.] We accept the Newtonanian dynamics (approximation of General Relativity) to build the Newtonian models that will follow.
		
		\item[H6.] The origin of time is treated as the origin of creation (horizon) of the universe and repository of study is comoving with the particles (and therefore moves with the galaxies placed on the space-time pattern) and named "\NewTerm{reference material}\index{reference material}" (galaxies are stationary in this repository!).
	\end{itemize}

	\subsubsection{Hubble's Law}
	Assuming the cosmological principle and the above assumptions, the distance from an origin point O to any point $M$ of the universe can vary in function of time (in a undetectable way to the human scale) in the form:
	
	where $F(t)$ is the "\NewTerm{scaling factor}\index{scaling factor}" (denoted by $R(t)$ depending on the context...).
	
	In writing this relation, we consider that the points O and $M$ are on a plane of zero curvature. Indeed, if we imagine two points on a circular curved surface (e.g. the surface of a sphere) let us see what happens:
	\begin{figure}[H]
		\begin{center}
		\includegraphics{img/cosmology/newtonian_universe_model.jpg}
		\end{center}	
		\caption{Illustration of the validity limit of the model}
	\end{figure}
	The distance between two points on the circle (i.e. spherical space) is given by (\SeeChapter{see section Trigonometry}):
	
	We see very well in this relation that if the radius (of the spherical Universe) changes by a factor $F$, then the change in the distance between the two points is not linearly proportional to this factor!! Which is not the case in a zero curvature plane!
	
	Consequence: Our Newtonian model is valid only in a flat universe where General Relativity or purely classical energy approach (see further below) can take into account different types of curvature!
	
	We see immediately that the relation:
	
	is independent of the chosen origin, in fact, if we apply it on any two points $A, B$ we have:
	
	Then by difference:
	
	\begin{tcolorbox}[title=Remarks,colframe=black,arc=10pt]
	At the time $t_0$ it is obvious that the above equation is:
	
	and imposes $F(t_0)=F(0)=1$. This is important and we will come back on it many times during the developments that will follow.
	\end{tcolorbox}
	The law above therefore applies to any segment $\overline{AB}$ in the Universe. This is why the Universe has no geometrical center (at least as far as we know) and that we can represent ourselves the expansion of the frame of the Universe: consider a half-inflated balloon on whose surface we draw two marks (eg: two crosses drawn with ink). Inflating the balloon more we find that these two cross diverge from each other and therefore the distance between them will increase. This is what we are seeing with the galaxies:
	\begin{figure}[H]
		\begin{center}
		\includegraphics{img/cosmology/inflated_universe.jpg}
		\end{center}	
		\caption{Illustration of the inflated balloon Universe model}
	\end{figure}
	
	Deriving with respect to time the relation:
	
	The first member then gives the particle velocity (or other any object) to the point $\vec{r}(t)=\overrightarrow{\text{O}M}(t)$:
	
	Therefore eliminating $\overrightarrow{\text{O}M}(t_0)$:
	
	We put to simplify the notations:
	
	Therefore we have:
	
	This relation is known as the "\NewTerm{Hubble law}\index{Hubble law}" (and which, according to historical research should have for paternity rather Georges Lemaître...).
	
	Before going further, it is necessary to pause on this equation for the present moment:
	
	This equation says that the object of the Universe recede with a speed proportional to their distance in all points of the Universe without special repository (no galaxy seems to be fixed while they are in the material repository!).
	\begin{tcolorbox}[title=Remark,colframe=black,arc=10pt]
	This relation allows for speeds higher than those of light... But this is not a violation of Special Relativity regarding the constancy of the speed of light! Indeed, we must not forget that the Hubble law takes into account the expansion of the "frame" of space-time on which light moves. Also Special Relativity deal with speed of light and not with speed of space itself. Therefore if the frame extends along an expansion factor $F$ greater than one, it gives the impression that light travels faster than $c$, and this is what gives the sometimes a redshift of $4$ or $5$!
	\end{tcolorbox}
	The constant $H(t_0)=H_0$ being of course being identifiable to the "\NewTerm{Hubble constant}\index{Hubble constant}" as currently measured in the early 2000s as being about: $\sim 75\; [\text{km s}^{-1} \text{Mpc}^{-1}]$.
	
	In IS units, since one megaparsec is almost equal to $\sim 3.085\cdot 10^{22}\; [\text{m}]$ then we have:
	
	Thus, a current estimate of the age (horizon) of the Universe could be interpreted as the inverse of the Hubble constant that gives the "\NewTerm{Hubble time}\index{Hubble time}":
	
	that is to say about 13 billion years ago (we will further below a better approach).
	
	Conversely, we can have fun to calculate the distance from which we can reach the speed of light $299,792/75\cong 3,997\;[\text{Mpc}]$ thanks to the equation:
	
	and a numerical application gives roughly $13$ billion light years. This is the distance of the "\NewTerm{cosmological horizon}\index{cosmological horizon}".
	
	In cosmology, a "\NewTerm{Hubble volume}\index{Hubble volume}, or "\NewTerm{Hubble sphere}\index{Hubble sphere}", is a spherical region of the Universe surrounding an observer beyond which objects recede from that observer at a rate greater than the speed of light due to the expansion of the Universe. Regarding the relation we get above, the corresponding radius today is given by $r_{\text{HS}}(t_0)=c/H_0$ and almost equal to $13$ billion light years as mentioned just earlier above. As observations seems to indicate that our Universe is accelerating, so that some objects that we can currently exchange signals with will one day cross our Hubble limit.
	
	\subsection{Friedmann Equations}
	Consider now a spherical ring of material of radius $r$ and of constant mass $m$ expanding at velocity $v$, and containing a ball of material of mass $M$ (also in expansion at speed $v$).
	
	where $k_1$ is a constant. By dividing by $m$ and replacing each member $M$ by its expression as a function of the density, we obtain:
	
	\begin{tcolorbox}[title=Remark,colframe=black,arc=10pt]
	If it can help the reader to understand what we did with the term of the potential energy, it can refer to section Classical Mechanics when we developed the calculations of the potential energy of a material sphere.
	\end{tcolorbox}
	
	and:
	
	We get:
	
	That we simplify in:
	
	However, $k_1,m,r_0$ are constants. We introduce a new constant $k$ defined by (to simplify the notations):
	
	So we get the equation:
	
	which is none other than the "\NewTerm{Friedmann's first equation}\index{Friedmann's first equation}" that we frequently find in the literature as follows (among others notations...):
	
	It is possible to obtain the same equation from Einstein's Field equations (\SeeChapter{see section General Relativity}) and the metric of Friedmann-Robertson-Walker (see further below).
	
	Let us still notice a very common form of that relation. When using Einstein's mass and energy equivalence ($E=mc^2$) the above density $\rho$ is no longer a mass density but an energy density, we will have to divid it again by the squared speed of light to have a mass density again. It is the same if at the denominator of the constant $k$, the mass $m$ is replaced by energy, then we will have to multiply $k$ by the squared speed of light to fall back on a masse We then noting the radius $a$ (as is often customary in the literature) and redistributing the terms, the following form of the first Friedmann equation:
	
	where $a$ is also known as the "\NewTerm{cosmic scale factor}\index{cosmic scale factor}" or sometimes the "\NewTerm{Robertson-Walker scale factor}\index{Robertson-Walker scale factor}".
	
	\pagebreak
	\begin{tcolorbox}[title=Remark,colframe=black,arc=10pt]
	Albert Einstein added to this equation for personal and quasi-religious beliefs a cosmological constant equation that allowed him to make static the scale factor of the Universe. I (as the main redactor of the book) reject this arbitrary constant (at least until this date), even if in the contemporary physics, it has become a trendy constant (its value was defined more mathematically rather than religiously) because it would explain the origin of a supposed "dark matter" that seem to accelerate the expansion of our Universe (this acceleration could also be a "local" departure of the Hubble constant from its globally averaged value perhaps caused by a local void in the mass density of our space neighborhood named a "\NewTerm{Hubble bubble}\index{Hubble bubble}"), the current laws of our Universe, the inflationary period of our universe and its geometry. Thus, the first Friedmann equation with the cosmological constant, which is a total artifice of work, is then: 
	
	with:
	
	This is Andrei Sakharov who defined the value of the above cosmological constant, which supposedly corresponds to the quantum energy of vacuum (depending on the Higgs fields).\\
	
	In quantum physics the equations of the field associated with elementary particles that are used to define the Big Bang theory are one of the main flow of the beginning of this 21st century. The famous Einstein's field equation tells us that energy creates a gravitational field like the electron in motion causes an electromagnetic field. It follows from these two observations that by measuring the gravitational field we have a way to determine the energy of vacuum. The gravitational field is no longer about the matter but about the energy density of vacuum. But the cosmological constant is directly proportional to the constant of gravity $G$. Its measurement is a quite very dangerous game because its value depends on several fundamental laws of physics and of significant properties on the dynamics of our Universe. The debate remains completely open and if I (the main redactor of this book) find a valid and rigorous proof of this constant, we will provide to the reader the  consequences of this constant on the models that we will see below.
	\end{tcolorbox}
	Let us now use the first law of thermodynamics (\SeeChapter{see section of Thermodynamics}) for a system by definition that will be closed and isolated, which the sum of kinetic and potential energy is constant (and therefore the amount of total energy variation is zero for these two energies). We then have that the change in total energy is only given by the variation of internal energy (the most common case in thermodynamics for macroscopic objects):
	
	and we have also proven in the section of Thermodynamics the characteristic equation of a fluid in equilibrium:
	
	If the system is adiabatic (no heat transfer between the system and outside), then we have also proven in the section of Thermodynamics that the variation of entropy was:
	
	Then:
	
	Since the universe is spherical assumed in our model, we have:
	
	and in the material repository where galaxies (cosmic fluid particles) are immobile:
	
	Therefore:
	
	which simplifies in:
	
	taking the derivative with respect to the cosmic time $t$:
	
	therefore:
	
	Now let us take again the first Friedmann equation, obtained above, in the form:
	
	and let us write it as follows:
	
	If we differentiate:
	
	Then we get:
	
	Let us inject:
	
	in the relation:
	
	Then we get:
	
	Thus:
	
	The following relation:
	
	is the "\NewTerm{Friedmann's second equation}\index{Friedmann's second equation}" that is also sometimes named "\NewTerm{Raychaudhuri equation}\index{Raychaudhuri equation}".
		
	\paragraph{Critical Density}\mbox{}\\\\\
	Let us come back to our first Friedmann equation without cosmological constant. So we have shown above that:
	
	We obtain then by injecting the latter relation in the first Friedmann equation the following relation:
	
	which rearranges with:
	
	into:
	
	The exponent of the left term requires that the right term is positive or zero as:
	
	Recall that the initial conditions impose us that at time $t_0=0$ we have:
	
	Indeed:
	
	Then it comes:
	
	This term should be accessible to observation, sadly $H_0^2$ is very poorly known and $\rho_0$ even more. In other words, given the "$-$" sign in the expression of $k$, we do not even know today the sign of this constant.
	
	However, it may be important to notice that there is a value $\rho_0$ named "\NewTerm{critical density}\index{critical densit}" that cancels the $k$ above and therefore also (see above):
	
	as it is also equal to $k$. This imply that the thotal Energy of the Universe would be zero (following considerations of quantum cosmology).

	This value $\rho_0$ is given immediately by:
	
	For $H_0=67.80\pm 0.7\;[\text{km} \cdot\text{s}^{-1}\cdot\text{Mpc}^{-1}]$ (actual value) we get (for the value in SI units see our tables of constants in the section Principia):
	
	In comparison, a hydrogen atom weighs $1.7\cdot 10^{-27}\;[\text{kg}]$, the critical density would therefore correspond to six hydrogen atoms per cubic meter.

	Physicists have defined a constant (time-varying... so therefore not so constant...) denoted by the Greek letter $\Omega$ and named "\NewTerm{cosmological density parameter}\index{cosmological density parameter}" and given by the ratio of the mass density (or energy densities since the ratio will be the same!):
	
	astrophysicists often break the cosmological density parameter in three terms:
	
	estimated  experimentally in this early 21st century at $\Omega=1.00\pm0.02$.
	
	It is interesting to work with this constant because in the case:
	\begin{itemize}
		\item $\Omega=1$:
		We have:
		
		which gives by replacing in the Friedmann $k=0$ (flat Universe as we shall see in our study of the relativistic model).

		\item $\Omega>1$:
		By performing the same reasoning, and still using inequality, we have then: $k>0$ (a Universe with positive curvature (closed) as we shall see in our study of the relativistic model).

		\item $\Omega<1$:
		By performing the same reasoning, and still using inequality, we have then: $k<0$ (a Universe with negative curvature (open) as we shall see in our study of the relativistic model).
	\end{itemize}
	These three situations can be summarized geometrically by the following well known figure:
	\begin{figure}[H]
		\begin{center}
		\includegraphics[scale=0.5]{img/cosmology/type_universe.jpg}
		\end{center}	
		\caption{Illustration of the different types of curvature (source: Wikipedia)}
	\end{figure}
	All measurement which have been made so far have failed to show a curvature of the Universe (anisotropy). The measurements of the microwave background (see mathematical details further below) by the BOOMERANG balloon and COBE satellite however, tend to support the hypothesis of a relatively flat Universe validating therefore numerical simulations:
	\begin{figure}[H]
		\begin{center}
		\includegraphics{img/cosmology/anisotropy_boomerang.jpg}
		\end{center}	
		\caption{Illustration of what would give observations depending on the curvature type}
	\end{figure}
	Also satellites sensibility don't stop to increase as how the figure below but it's still hard to conclude anything about the temperature background anisotropy:
	\begin{figure}[H]
		\begin{center}
		\includegraphics[scale=0.12]{img/cosmology/anisotropy_performance.jpg}
		\end{center}	
		\caption{Comparison of CMB results from COBE, WMAP and Planck – 2013-03-21 (source: Wikipedia)}
	\end{figure}
	\begin{tcolorbox}[title=Remark,colframe=black,arc=10pt]
	The concept of Universe topology and openness are actually normally two separate concepts. When we speak of "open" or "closed" Universe we do not normally talk about its topology but his destiny! Thus, an "open" Universe is expanding indefinitely and a "closed" Universe recontracte on itself after a given time. That said, in the models that we study in this section (cosmological constant equal to zero), the curvature is directly related to the density, and thus to its openness.
	\end{tcolorbox}
	Let us come back to the relation:
	
	We can write:
	
	By adopting the notation:
	
	\begin{tcolorbox}[title=Remark,colframe=black,arc=10pt]
	The actual measurement gives (year 2007): $A\cong 0.2793923067\cdot 10^{-35}$.
	\end{tcolorbox}
	Therefore:
	
	It is now appropriate for us to consider three situations:
	
	thus correspond respectively to the cosmological density parameters:
	
	\begin{tcolorbox}[title=Remark,colframe=black,arc=10pt]
	We can't put $\rho_0=0$ because our initial assumptions was the principle of conservation of energy.
	\end{tcolorbox}
	
	\pagebreak	
	\subsection{Cosmological models of Friedmann-Lemaitre}
	The Euclidean cosmological models of Friedmann-Lemaitre consist in the Newtonian  limit to study the  "\NewTerm{fundamental equation of Friedmann models}\index{fundamental equation of Friedmann models}":
	
	considering the three situations:
	
	\begin{tcolorbox}[title=Remark,colframe=black,arc=10pt]
	It is possible within the framework of General Relativity to rigorously find a solution to Einstein's field equations named the "\NewTerm{Robertson-Walker metric}\index{Robertson-Walker metric}" which in the case of a Newtonian approximation gives us the Friedmann equations obtained in the present text (often these are the approximations that are used in the literature for because the exact solution is out of the  framework of the traditional universities courses of the 21st century).
	\end{tcolorbox}
	\subsubsection{Flat spaces ($k=0$)}
	The flat (Euclidean) space model consist to assume that $k=0$. In other words, we are in a Universe whose density is named "\NewTerm{critical density}\index{critical density}" or also simply "\NewTerm{flat}\index{flat}" (as we will see with the relativistic model).

	The we have the following equation:
	
	By arranging appropriately the terms:
	
	and by integrating, it comes:
	
	Which simplifies in (we raise to the square, hence the removal of the double sign $\pm$):
	
	So we have in this model the relation:
	
	to which we must add a constant for the condition corresponding to today:
	
	that remains satisfied. Therefore:
	
	This gives us a plot function that looks like this (do not trust the values shown on the horizontal axis as they are arbitrary):
	\begin{figure}[H]
		\begin{center}
		\includegraphics{img/cosmology/friedmann_flat_space.jpg}
		\end{center}	
		\caption{Evolution of the scale factor for a zero curvature space}
	\end{figure}
	We put the area where $F(t)<1$ in evidence to remember that this part of the solution is to reject.

	So we have a model of Universe in which the scale factor is growing exponentially and this and indefinitely.
	\begin{tcolorbox}[title=Remark,colframe=black,arc=10pt]
	More $\rho_0$ is big, more the scale factor incrase fast (meaning that the slope is obviously larger).
	\end{tcolorbox}
	
	\paragraph{Flat space dominated by matter}\mbox{}\\\\\
	There is also another approach much more elegant and subtle from my point of view than the previous proof (I have discovered that many years after writing the previous version). It has the benefit of highlighting a hypothesis that does not appear with previous developments.

	We start from the first Friedmann equation:
	
	put always putting $k$ as being equal to zero and then we use the trick that consist to start from the relation prove also earlier above (used to proved the second Friedmann equation):
	
	to require that the fluid pressure $P$ (whatever it is: gas or radiation) is zero. We then say that the Universe is a universe dominated by matter and we deduce:
	
	which gives us:
	
	Under these conditions, the first Friedmann equation becomes:
	
	rearranging and simplifying, we then have:
	
	which gives:
	
	At the time $t=0$, we have the scale factor that is equal to $R=0$ and thus the constant is zero. Then we have:
	
	By putting that at time $t=0$, the scaling factor was unitary, the latter relationis simplified:
	
	Using the common notation in theoretical cosmology we get the famous relation often given without proof:
	
	If we assume that the scale factor is today taken as unitary, then it comes:
	
	and by replacing in it the numerical values currently known of the Hubble constant, it follows that the universe is now aged about $8.6$ billion years (compared to $13$ billion of the Hubble time obtained earlier above!).
	
	\paragraph{Flat space dominated by radiation}\mbox{}\\\\\
	We have proved in the section of Thermodynamics dur our study of the Stefan-Boltzmann law that the pressure of radiation was related to the energy density by the following equation:
	
	In a universe dominated by radiation, the relation:
	
	 proved earlier, expressed with a density of energy and not a density of mass becomes:
	
	Therefore:
	
	and using the relationship between radiation pressure and energy density, we get:
	
	After a little rearrangement, we get:
	
	from which we get that:
	
	Under these conditions, the first Friedmann equation:
	
	becomes first by changing into energy density and with $k$ being put as equal to zero:
	
	and so we can replace the energy density by the result we get just before:
	
	Then we have:
	
	hence:
	
	The primitive is obvious and is immediate:
	
	At the time of the Big Bang in $t=0$, we have the scale factor that is supposely $R=0$ and therefore the constant is zero. Then we have:
	
	But assuming that at $t=0$ we had $R_0=1$ the we have:
	
	Therefore:
	
	Thus after simplification it remains only:
	
	Using the common notation in theoretical cosmology we get the famous relation often given without proof:
	
	Then a flat universe dominated by radiation has a scaling factor that is growing slightly more slowly than a flat universe dominated by matter.

	In comparison with Maple 4.00b (blue: flat universe dominated by matter, in red: flat universe dominated by radiation):

	\texttt{>plot([t\string^(2/3),t\string^(1/2)],t=0..2*Pi,0..3,color=[blue,red]);}\\
	\begin{figure}[H]
		\begin{center}
		\includegraphics{img/cosmology/universe_scale_factor_evolution_flat_matter_radiation_maple.jpg}
		\end{center}	
		\caption[]{Evolution of $R$ for a zero curvature space dominated by matter or radiation.}
	\end{figure}
	
	\subsubsection{Spherical spaces ($k>0$)}
	In this model (also sometimes named "elliptical model"), we consider $k>0$. So the equation to deal with remains:
	
	Which can also be written:
	
	Let us recall that we have assumed that for $t=t_0$ we had $F(t_0)=1$, if we make the change of variable $U=1/F(t)$, we get the following integral:
	
	So we are looking of a primitive of:
	
	and we will discuss the sign $\pm$ after having found the primitive.

	We still carry a change of variable:
	
	thus:
	
 	which gives us the following primitive to calculate:
	
	by doing again a change of variable:
	
	Therefore to a given multiplicative constant $2Ak^{-3/2}$ we have finally the following integral:
	
	In the section of Differential andIntegral Calculus we proved that this form of primitive is resovled by (we add the constant of integration at the end because we do physics and we must satisfy to some initial conditions at which are not necessarily interested to in mathematics):
	
	with:
	
	hence:
	
	We still have to calculate $I_1$ (\SeeChapter{see section of Differential and Integral Calculus}):
	
	Finally:
	
	by inverting all the changes of variables and introducing the multiplicative constant again, we have finally in the case where $k>0$:
	
	Between the two terminals of integration $(1/F,1)$ we therefore have (the integration constant cancels and we take back the $\pm$ which was originally in the primitive):
	
	where for recall the theory request that $\Delta t>0$ (otherwise it's pure speculation and philosophy...).
	
	We see that as expected, if we $F=1$ in the above relation we have indeed $\Delta t=0$ (that means that at the moment in time we take the Universe actual size as reference, the time variation is indeed equal to $0$ as expected).
	
	If we plot this function for a fixed value $k>0$. We have the following  animated plot Maple in 17.00 for the negative sign:
	
	\texttt{>restart:\\
	>f:=(A,k,F)->-[(F*sqrt(A/F-k)/k+A/k\string^(3/2)*arctan(sqrt(A/f-k)/sqrt(k)))\\
	-(sqrt(A-k)/k+A/k\string^(3/2)*arctan(sqrt(A-k)/sqrt(k)))]:\\
	>plots:-animate(plot3d,[f(A,k,F),k=0.1..1,F=0..10,],A=0..1)
	}
	\begin{figure}[H]
		\begin{center}
		\includegraphics{img/cosmology/spherical_universe_maple_animation_minus_sign_solution.jpg}
		\end{center}
	\end{figure}
	We can see with this solution that as time increase the Universe reach an asymptote size whatever the value of $k$, but the bigger is $k$ the faster the asymptote is reached. In other words... we must be lucky not to be in a Universe with a to big positive curvature...
	
	And for the positive sign:\\
	
	\texttt{>restart:\\
	>f:=(A,k,F)->+[(F*sqrt(A/F-k)/k+A/k\string^(3/2)*arctan(sqrt(A/f-k)/sqrt(k)))\\
	-(sqrt(A-k)/k+A/k\string^(3/2)*arctan(sqrt(A-k)/sqrt(k)))]:\\
	>plots:-animate(plot3d,[f(A,k,F),k=0.1..1,F=0..10,],A=0..1)
	}
	\begin{figure}[H]
		\begin{center}
		\includegraphics{img/cosmology/spherical_universe_maple_animation_plus_sign_solution.jpg}
		\end{center}
	\end{figure}
	We can see with that this solution see to be the symmetric one of the above plot. That means as time reached back to the zero value, the Universe contracts on single point back and once again this effect is slower as $k$ is big! 
	
	\begin{tcolorbox}[title=Remark,colframe=black,arc=10pt]
	The time $\Delta t$ in the above plots is always represented on the vertical axis and also for all the following charts further below (you have to turn your head a little if as usually you want to put the time on the horizontal axis...).
	\end{tcolorbox}
	By fixing a small value of $A$ and for $k$, we get the following two-dimensional plot first for the negative sign:
	
	\texttt{>k:=0.0001;A:=1;\\
	 >plot([-(F*sqrt(A/F-k)/k+A/k\string^(3/2)*arctan(sqrt(A/F-k)/sqrt(k)))-(sqrt(A-k)/k\\
	+A/k\string^(3/2)*arctan(sqrt(A-k)/sqrt(k)))],F=1..10000,labels=[F,t])}
	\begin{figure}[H]
		\begin{center}
		\includegraphics[scale=0.6]{img/cosmology/universe_factor_evolution_for_constant_A_negative_sign.jpg}
		\end{center}
	\end{figure}
	where we as have already mention, the read must keep in mind that the values below $1$ must be rejected!
	
	And for the positive sign:
	\begin{figure}[H]
		\begin{center}
		\includegraphics[scale=0.6]{img/cosmology/universe_factor_evolution_for_constant_A_positive_sign.jpg}
		\end{center}
	\end{figure}
	By looking at the both plots above it is obvious to see that we have after a rotation and putting each one next to the other (we also could change the time reference to have a logical time axis):
	\begin{figure}[H]
		\centering
		\includegraphics[scale=0.5]{img/cosmology/universe_big_bang_big_crunch.jpg}
		\caption{Big Bang and Big Crunch plot side by side}
	\end{figure}
	and and what we see here is the Big Ban and the Big Crunch!
	
	Now let us recall that to build the previous model we started from:
	
	A limit condition (condition of integration) is that the right term to be positive. That is:
	
	or:
	
	So for our previous $2$D plots this limit is locate at $F=k/A=10,000$ and this is according to the maximum value before the Universe turn into a Big Crunch.
	
	So if $F^{-1}$ is smaller than $F_{\lim}^{-1}$, we are not in a valid (real) domain model anymore.
	
	In fact, beyond the time limit $t_{\lim}$ corresponding to this $F_{\lim}$, what does not know the computer that has drawn our function plot is that it should switch to the scaling function with the "$+$". So when we execute the plot of both functions we should get the previous figure.
	
	We see then that for $t<t_{\lim}$ the Universe is entering a phase of contraction that we commonly name the "\NewTerm{Big Crunch}\index{Big Crunch}". After this phase of contraction, it is possible that either the Universe disappears completely in a singularity or that it re-enters a cyclical dynamic phase (mathematically the two outcomes seems to be possible).
	
	\paragraph{Spherical space dominated by matter}\mbox{}\\\\\
	Just as the model for flat space, there is also another approach much more elegant and subtle for my taste than the previous proof (I have also discovered that many years after writing the previous text). It also has the advantage of highlighting a hypothesis that has not appear with previous developments and allows to plot more simply in Maple 4.00b the behavior of the scale factor of the Universe. We thus find exactly the famous plot representing the evolution of the scale factor of the Universe available in almost all popular books on the subject (without proofs obviously...)

	We start again from the first Friedmann equation:
	
	It is customary for this model equation to put $k=1$ (even take any positive number at least pick one that is friendly...) and we have shown that when matter dominates, we had:
	
	Since then:
	
	and it comes immediately:
	
	Therefore:
	
	If we move now to the comoving time also named "\NewTerm{conformal time}\index{conformal time}" (already introduced at the beginning of this section) defined mathematically by (don't forget that $R$ is note a radius but a ratio of two radius!):
	
	Then we have:
	
	Let us write that in the form:
	
	where $A$ is strictly positive. Let us do the substitution:
	
	Then we have:
	
	and we have proved in the section of Differential and Integral Calculus that the primitive is:
	
	Therefore:
	
	As at time $\eta=0$ we have $R=0$, it is necessary that the constant is such that:
	
	Therefore:
	
	Hence:
	
	Now, let us recall that:
	
	Therefore:
	
	Hence:
	
	and as we must have at time $t=0$ the comoving time that is also zero, the constant is then zero. Therefore we have the following parametric system in the end (something very strange with this result... it is the parametric function of the brachistochrone curve!!!!):
	
	With Maple 4.00b we then have by comparing the flat Universe dominated by matter (blue), the flat Universe dominated by radiation (red) and finally the positive curvature Universe dominated by matter (green) and putting artificial coefficients to better distinguish the plots:
	
	\texttt{>plot([t\string^(2/3),t\string^(1/2),[0.5*(t-sin(t)),0.5*(1-cos(t)),t=0..2*Pi]],\\
	t=0...Pi,0..2.5,color=[blue,red,green]);}
	\begin{figure}[H]
		\centering
		\includegraphics[scale=0.8]{img/cosmology/universe_models_01_maple_plot.jpg}
		\caption[]{Evolution of the $R$ factor for the resulting space configurations studied so far with Maple 4.00b}
	\end{figure}
	
	\paragraph{Spherical space dominated by radiation}\mbox{}\\\\\
	Let us now consider a universe dominated by radiation. We have proved earlier above that in this situation we had:
	
	and:
	
	In this case the Friedmann equation in terms of energy density can be written by putting $k=1$:
	
	What becomes:
	
	By injecting $R^4\rho_E=R_0^4\rho_{E,0}$, it comes:
	
	Therefore:
	Therefore:
	
	Let us write this in the form:
		
	If we change to the comoving time again:
	
	Then we have:
		
	In the section of Differential and Integral Calculus we have proved how to determine exactly the same primitive (because it is a usual primitive). We have:
	
	For at time $\eta=0$ we have $R=0$, it is necessary that the constant is such that:
	
	Hence:
	
	Now, remember that:
	
	Therefore:
	
	Hence:
	
	and as at time $t=0$ time comoving time is also zero, the constant is therefore equal to $\sqrt{A}$. Therefore we have finally the following parametric system:
	
	With Maple 4.00b we then comparing the flat Universe dominated by matter (blue), the flat Universe dominated by radiation (red), the Universe with positive curvature dominated by matter (green), the Universe with positive curvature dominated by radiation (black):\\
	
	\texttt{>plot([t\string^(2/3),t\string^(1/2),[0.5*(t-sin(t)),0.5*(1-cos(t)),t=0..2*Pi],[0.5*(1-cos(t)),\\0.5*sin(t),t=0..2*Pi]],t=0...Pi,0..3,color=[blue,red,green,black]);}
	\begin{figure}[H]
		\centering
		\includegraphics[scale=0.8]{img/cosmology/universe_models_01_maple_plot.jpg}
		\caption[]{Evolution of the $R$ factor for the resulting space configurations studied so far with Maple 4.00b}
	\end{figure}
	
	\subsubsection{Hyperbolic spaces ($ky0$)}
	In this model, we consider $k<0$. So the equation to be treated can be written:
	
	Which is also written:
	
	Let us recall that we assumed that $t=t_0$ that $F(t_0)=1$. If we make the change of variable $U=1/F(t)$, we get the following integral:
	
	So we are looking for a primitive of:
	
	and we will discuss the sign $\pm$ after finding the primitive.

	We still carry a change of variable by putting:
	
	 therefore:
	
	which gives us the following primitive to calculate:
	
	Doing again a change of variable:
	
	hence to a given multiplicative constant:
	
	We have:
	
	In the section of Differential and Integral Calculus we saw that this form of primitive is resolved by the relation (we added the constant of integration in the end because we do physics and must satisfy the initial conditions to which we were not interested to in pure mathematics):
		
	with:
	
	hence:
	
	hence:
	We still need to calculate $I_1$:
	
	Finally:
	
	by putting back all the changes of variables and introducing the multiplicative constant again, we have therefore in the case $k>0$:
	
	Between the two terminals of integration $(1/F,1)$ so we have (the integration constant is zero):
	
	We must obviously have (we take back the $\pm$ which was originally in the integral):
	
	If we plot this function for a fixed value $k>0$. We have the following  animated plot Maple in 17.00 for the negative sign (as the positive one has no physical meaning):
	
	\texttt{>restart:\\
	>f:=(A,k,F)->-(((-F(sqrt(A/F+abs(k)))/k+A/2(2*abs(k)\string^(3/2))*ln(abs(((sqrt(\\
	abs(k))+sqrt(A/F+abs(k)))/(sqrt(abs(k))-sqrt(A/F+abs(k))))/((sqrt(abs(k))+sqrt(A+abs(k)))/(sqrt(abs(k))-sqrt(A+abs(k))))))))):\\
	>plots:-animate(plot3d,[f(A,k,F),k=0.1..1,F=0..10,],A=0.1..1)
	}
	\begin{figure}[H]
		\begin{center}
		\includegraphics{img/cosmology/hyperbolic_universe_maple_animation_minus_sign_solution.jpg}
		\end{center}
	\end{figure}
	We see that small the constant $A$ is, the fastest the Universe increases indefinitely quickly. Furthermore for a fixed value of $k$, some values of $A$ are prohibited (it is in fact still the integration condition).

	Again we see that the in the equation above the criterion $F(t_0)=F(0)=1$ is naturally fully respected. All values $F (t) $ below $1$ are to be rejected!

	So we have in this hyperbolic model a Universe that grows indefinitely in an exponential away (as the flat Friedmann-Lemaitre mdoel) because since $k<0$, there is no integration condition limit anymore (unlike the previous spherical model).
	
	\subsubsection{Matter dominated hyperbolic space}
	Just as for the models for flat and spherical space, there is also another approach much more elegant and subtle for my taste than the previous proof (I have also discovered that one many years after writing the previous version) . It also has the advantage of highlighting a hypothesis that has not occurred with previous developments and allows to draw more simply in Maple 4.00b the behavior of the scale factor of the Universe. We thus find exactly the famous plot representing the evolution of the scale factor of the Universe available in almost all popular books on the subject but without proof.

	We always start from the first Friedmann equation:
	
	It is customary for this model to put $k=-1$ (as we have to choose any negative number at least we pick one that is friendly ...) and we have proved that when the radiation dominates, we had:
	
	In this case the Friedmann equation in terms of energy density can be written by putting $k=-1$:
	
	The first Friedmann equation then becomes:
	
	Then we have:
	
	Hence:
	
	This is exactly the same integral than that of the spherical Universe dominated by matter at the difference that in the root, we $+1$ instead of $-1$. We will proceed in the same manner using the time comoving time:
	
	It comes then:
	
	Let us write that in the form:
	
	where $A$ is strictly positive. Let us make the substitution:
	
	Then we have:
	
	Using the usual primitive proved in the section of Differential and Integral Calculus it comes:
	then we have:
	
	Using the usual primitive proved in the section of Differential and Integral Calculus it comes:
	
	Or redoing the change of variables:
	
	Therefore:
	
	So that at time $\eta=0$ we have $R=0$ , it is necessary that the constant is such that:
	
	Which brings us to that the constant is zero and thus:
	
	Therefore:
	
	Hence:
	
	and as:
	
	We have:
	
	Which gives:
	
	As at time $t=0$, we must have $\eta=0$, it follows that the constant must be zero. So finally, we have:
	
	with Maple 4.00b we then have by comparing the flat Universe dominated by matter (blue), the flat Universe dominated by radiation (red), the Universe with positive curvature dominated by matter (green), the Universe with positive curvature dominated by radiation (black), the negatively curved Universe dominated by matter (gray):
	
	\texttt{>plot([t\string^(2/3),t\string^(1/2),[0.5*(t-sin(t)),0.5*(1-cos(t)),t=0..2*Pi],}\\
	\texttt{[0.5*(1-cos(t)),0.5*sin(t),t=0..2*Pi],[0.5*(sinh(t)-t),0.5*(cosh(t)-1)}
	\texttt{,t=0..2*Pi]],t=0...Pi,0..3,color=[blue,red,green,black,gray]);}
	\begin{figure}[H]
		\centering
		\includegraphics[scale=0.8]{img/cosmology/universe_models_03_maple_plot.jpg}
		\caption[]{Evolution of the $R$ factor for the resulting space configurations studied so far with Maple 4.00b}
	\end{figure}
	We can therefore observe that for a negative curvature (hyperbolic type), expansion is growing significantly faster than for a flat Universe and this without end.
	
	\subsubsection{Hyperbolic space dominated by radiation}
	Let us now consider a Universe dominated by radiation. We have proved that in this situation we had:
	
	and:
	
	The first Friedmann equation the becomes:
	
	Then we have:
	
	hence:
	
	This is exactly the same integral than that of the spherical universe dominated by matter at the difference that in the root, we have $+1$ instead of $-1$. We will proceed in the same manner using the comoving time:
	
	It comes then:
	
	Let us write this in the form:
	
	where $A$ is strictly positive.In the section of Differential and Integral Calculus we have proved how to determine exactly the same primitive (because it is a usual primitive). We have:
	
	So that at time $\eta=0$ we have $R=0$ , it is necessary that the constant is zero. Therefore:
	
	Hence:
	
	and as:
	
	We have:
	
	Which gives:
	
	As at the time $t=0$, we must have $\eta=0$, it follows that the constant must be equal to $-\sqrt{A}$. So finally, we have:
	
	with Maple 4.00b we then have by comparing the flat Universe dominated by matter (blue), the flat Universe dominated by radiation (red), the Universe with positive curvature dominated by matter (green), the Universe with positive curvature dominated by radiation (black), the Universe with negative curvature dominated by matter (gray), the negative curvature Universe dominated by radiation (brown):
	
	\texttt{>plot([t\string^(2/3),t\string^(1/2),[0.5*(t-sin(t)),0.5*(1-cos(t)),t=0..2*Pi],}\\
	\texttt{[0.5*(1-cos(t)),0.5*sin(t),t=0..2*Pi],[0.5*(sinh(t)-t),0.5*(cosh(t)-1)}\\
	\texttt{t=0..2*Pi],[0.5*(cosh(t)-1),0.5*(sinh(t)),t=0..2*Pi]],t=0...Pi,0..3
,color=[blue,red,green,black,grey,brown]);}
	\begin{figure}[H]
		\centering
		\includegraphics[scale=0.8]{img/cosmology/universe_models_04_maple_plot.jpg}
		\caption[]{Evolution of the $R$ factor for the resulting space configurations studied so far with Maple 4.00b}
	\end{figure}
	We can therefore observe that for a negative curvature (hyperbolic type), the expansion of a Universe dominated by radiation grows slower than a universe dominated by matter (it's a bit intuitive...).

	Finally to summarize a little better all this with captions it we get the following important plot (its important to be implicated by this plot as the Universe affects us all ...):
	\begin{figure}[H]
		\centering
		\includegraphics[scale=1]{img/cosmology/summary_newtonian_universe.jpg}
		\caption{Summary of Newtonian Universe models}
	\end{figure}
	
	\subsection{Observable Universe}
	We have determined previously a possible interpretations of the current estimate of the age (horizon) of our Universe as being as the inverse of the Hubble constant that has given us for recall:
	
	that is to say about $13$ billion years.
	\begin{tcolorbox}[title=Remarks,colframe=black,arc=10pt]
	\textbf{R1.} It is important to know that popular research articles in cosmology  often use the term "Universe" in the sense of "observable Universe".\\
	
	\textbf{R2.} There should be more rigorous in fact when we speak of the Universe "age". In fact, we should rather say that the horizon of the Universe, for a comoving observer since the earliest times, is $13$ billion years. In other words, it's time that someone would have measure if he has remained an inertial observer (in free fall: not subjected to any force other than gravity) throughout the evolution of the Universe and in a repository such that he would always perceived this Universe as homogeneous and isotropic.
	\end{tcolorbox}
	The word "observable" used in this sense does not depend on whether modern technology actually permits detection of radiation from an object in this region (or indeed on whether there is any radiation to detect). It simply indicates that it is possible in principle for light or other signals from the object to reach an observer on Earth. In practice, we can see light only from as far back as the time of photon decoupling in the recombination epoch. That is when particles were first able to emit photons that were not quickly re-absorbed by other particles. Before then, the Universe was filled with a plasma that was opaque to photons. The detection of gravitational waves indicates there is now a possibility of detecting non-light signals from before the recombination epoch.
	
	At the beginning of the early 21st century, we do still do not know if the Universe is finite or infinite, although the majority of theorists currently favor an infinite universe.

	The observable Universe is thus composed of all locations that could have affected us since the Big Bang (beware!... despite its name, the Big Bang theory has nothing to say on its start! It merely describes the evolution and the expansion of the Universe).

	The current size (the "\NewTerm{comoving distance}\index{comoving distance}") of the observable Universe is larger, since the Universe continued to expand during the time that light takes to reach us, we believe it is about $40$ billion light years.

	This value can be obtained by taking the actual most distant visible object which is $13.39$ billion years of Earth. This will therefore have needed $13.39$ billion years to get away from us, its light will have needed $13.39$ billion years to reach us and during the time of light travel, it will have move away of $13.39$ billion years (since objects at cosmological horizon are going at the speed of light). Thus a total of about $40$ billion years.
	
	This observable Universe contains according to today's heuristics  estimates (year 2011) about $7\cdot 10^{22}$ stars, distributed in approximately $10^{10}$ galaxies, themselves organized into clusters and superclusters of galaxies. The number of galaxies may be even larger, as the "Hubble Ultra-Deep Field" observed with the Hubble space telescope seems to indicate us. 
	
	The Hubble Ultra-Deep Field is an image of a region of the observable universe (equivalent sky area size shown in bottom left corner), near the constellation Fornax. Each spot is a galaxy, consisting of billions of stars. The light from the smallest, most red-shifted galaxies originated nearly $14$ billion years ago.
	\begin{figure}[H]
		\centering
		\includegraphics[scale=0.19]{img/cosmology/hubble_deep_space.jpg}
		\caption{Hubble Ultra-Deep (source: Wikipedia, author: NASA and ESA)}
	\end{figure}
	However it is difficult to imagine what that represents. As we found on the Internet some wonderful series of illustrations we would like to share them with the reader before the disappear from the Internet.

	First here is a high-resolution summary of various structures you can found at different scales of our Universe (you can considerably zoom in!):
	\begin{figure}[H]
		\centering
		\includegraphics[scale=0.09]{img/cosmology/universe_scales.jpg}
		\caption{Universe scales (source: Wikipedia, author: Andrew Z. Colvin)}
	\end{figure}
	And a more detailed way to discover its structure:
	\begin{enumerate}
		\item The universe to $14$ billion light years (the visible Universe as we approximately know it today):
		\begin{figure}[H]
			\centering
			\includegraphics[scale=0.75]{img/cosmology/universe_zoom_0.jpg}
			\caption{Simplified illustration of the observable Universe (source: http://atunivers.free.fr, author: Richard Powell)}
		\end{figure}
		This illustration attempts to show the entire visible Universe. The galaxies in the universe tend to collect into vast sheets and "supercluster" of galaxies surrounding large vacuums zones, giving the universe a cellular appearance. Because light in the Universe only travel a finite speed, we see objects at the edge of the Universe as when it was very young, there is $14$ billion years ago

		Some numbers (estimates):
		\begin{itemize}
			\item Number of superclusters in the visible universe: $10$ million
			\item Number of galaxy groups in the visible universe: $25$ billion
			\item Number of large galaxies in the visible universe: $350$ billion
			\item Number of dwarf galaxies in the visible universe: $7$ trillion
			\item Number of stars in the visible universe: $30$ billion trillion  ($3\cdot 10^{22}$)
		\end{itemize}
		
		\item After a $\times 14$ zoom we get the Universe withing a $1$ billion light years, that is the neighboring superclusters:
		\begin{figure}[H]
			\centering
			\includegraphics[scale=0.75]{img/cosmology/universe_zoom_1.jpg}
			\caption{Simplified illustration of the neighboring superclusters (source: http://atunivers.free.fr, author: Richard Powell)}
		\end{figure}
		Galaxies and clusters of galaxies are not distributed uniformly in the Universe. Instead, they gathered in large clusters, sheets and walls of galaxies separated by large gaps in which few galaxies appear to be. The illustration above shows a number of these superclusters including the Virgo - a rather small supercluster of which our galaxy is part of. The entire map is approximately $7\%$ of the diameter of the visible universe. The galaxies are too small to appear individually on this map, each point there is a group of galaxies.
		
		Some numbers (estimates):
		\begin{itemize}
			\item Number of superclusters within $1$ billion light years: $100$
			\item Number of galaxy groups within $1$ billion light years: $240,000$
			\item Number of large galaxies within $1$ billion light years: $3$ million
			\item Number of dwarf galaxies within $1$ billion light years: $60$ million
			\item Number of stars within $1$ billion light years: $250,000$ trillion
		\end{itemize}
		
		\item After a $\times 10$ zoom we get the Universe within $100$ million light years or the Virgo Supercluster:
		\begin{figure}[H]
			\centering
			\includegraphics[scale=0.75]{img/cosmology/universe_zoom_2.jpg}
			\caption{Simplified illustration of the Virgo Supercluster (source: http://atunivers.free.fr, author: Richard Powell)}
		\end{figure}
		Our galaxy is just one of thousands that lie within $100$ million light years. The above illustration shows how galaxies tend to gather in groups, the largest nearby cluster is the Virgo cluster (Virgo), a concentration of several hundred galaxies which dominates the surrounding groups of galaxies. Collectively, all of these groups is known to Supercluster Virgo. The second richest cluster in this volume is the Fornax cluster (Fornax), but it is as rich as that of the Virgin. Only bright galaxies are drawn here, our galaxy is the point at center.	
		
		Some numbers (estimates):
		\begin{itemize}
			\item Number of galaxy groups within $100$ million light years: $200$
			\item Number of large galaxies within $100$ million light years: $2,500$
			\item Number of dwarf galaxies within $100$ million light years: $50,000$
			\item Number of stars within $100$ million light years: $200$ trillion
		\end{itemize}
		
		\item After a $\times 20$ zoom we get the Universe within $5$ million Light Years, that is the Local Group of Galaxies:
		\begin{figure}[H]
			\centering
			\includegraphics[scale=0.75]{img/cosmology/universe_zoom_3.jpg}
			\caption{Simplified illustration of the Local Group (source: http://atunivers.free.fr, author: Richard Powell)}
		\end{figure}
		The Milky Way is one of three large galaxies in the group named "\NewTerm{Local Group}\index{Local Group}" which also contains several dozen of dwarf galaxies. Most of these galaxies are plotted on the illustration above, but note that many of these dwarf galaxies a very small magnitude, so that there are certainly more to discover.	
		
		Some numbers (estimates):
		\begin{itemize}
			\item Number of large galaxies within $5$ million light years: $3$
			\item Number of dwarf galaxies within $5$ million light years: $46$
			\item Number of stars within $5$ million light years: $700$ billion
		\end{itemize}
		
		\item After a $\times 10$ zoom we get the Universe within $500,000$ light years, that is the Satellite Galaxies:
		\begin{figure}[H]
			\centering
			\includegraphics[scale=0.75]{img/cosmology/universe_zoom_4.jpg}
			\caption{Simplified illustration of Satellite Galaxies (source: http://atunivers.free.fr, author: Richard Powell)}
		\end{figure}
		The Milky Way is surrounded by several dwarf galaxies, each containing tens of millions of stars, which is insignificant compared to the population of the Milky Way itself. The map above shows all of the nearest dwarf galaxies that are gravitationally bound to the Milky Way, and revolve around it in a few billion years.
		
		Some numbers (estimates):
		\begin{itemize}
			\item Number of large galaxies within $500,000$ light years: $1$
			\item Number of dwarf galaxies within $500,000$ light years: $12$
			\item Number of stars within $500,000$ light years: $225$ billion billion
		\end{itemize}
		
		\item After a $\times 10$ zoom we get the Universe within $50,000$ light years, that is the Milky Way Galaxy:
		\begin{figure}[H]
			\centering
			\includegraphics[scale=0.75]{img/cosmology/universe_zoom_5.jpg}
			\caption{Simplified illustration of the Milky Way Galaxy (source: http://atunivers.free.fr, author: Richard Powell)}
		\end{figure}
		This map shows the Milky Way as a whole - a spiral galaxy of at least two hundred billion stars. Our Sun is buried deep within the Orion Arm about $26,000$ light years from the center. Toward the center of the galaxy, stars are much closer to each other than at the periphery where we live. Also notice the presence of small globular clusters far outside the galactic plane, and the presence of a neighboring dwarf galaxy - named "Sagittarius" - which is slowly being swallowed by our own Galaxy.
		
		Some numbers (estimates):
		\begin{itemize}
			\item Number of stars within $50,000$ light years: $200$ billion billion
		\end{itemize}
		
		\item After a $\times 10$ zoom we get the Universe within $5,000$ light years, that is to say the Orion Arm:
		\begin{figure}[H]
			\centering
			\includegraphics[scale=0.75]{img/cosmology/universe_zoom_6.jpg}
			\caption{Simplified illustration of the Orion Arm (source: http://atunivers.free.fr, author: Richard Powell)}
		\end{figure}
		This is a map of our corner of the Milky Way. The Sun is located in the Orion Arm - a fairly small arms compared to the Sagittarius Arm, which is closer to the galactic center. The map shows several stars visible to the naked eye, located far away in the Orion arm. The most notable group of stars is composed of the main stars of the constellation of Orion - from which the spiral arm gets its name. All these stars are bright giant and supergiant stars, thousands of times more luminous than the sun. The brightest star of the map is Rho Cassiopeia - to $4,000$ light-years from us is just barely visible to the naked eye star, but in reality it is a supergiant $100,000$ times brighter than our Sun.
		
		Some numbers (estimates):
		\begin{itemize}
			\item Number of stars within $5,000$ light years: $600$ million
		\end{itemize}
		
		\item After a $\times 20$ zoom we get the Universe within $250$ light years, that is to say the solar neighborhood:
		\begin{figure}[H]
			\centering
			\includegraphics[scale=0.75]{img/cosmology/universe_zoom_7.jpg}
			\caption{Simplified illustration of the solar neighborhood (source: http://atunivers.free.fr, author: Richard Powell)}
		\end{figure}
		This map shows the $1,500$ most luminous stars within $250$ light years. All these stars are much more luminous than the Sun, and most are visible to the naked eye. About a third of the stars visible to the naked eye are within $250$ light years, even though that area represents only a small part of our galaxy.
		
		Some numbers (estimates):
		\begin{itemize}
			\item Number of stars within $250$ light years: $260,000$
		\end{itemize}
		
		\item After a $\times 20$ zoom we get the Universe within $12.5$ light years (the nearest stars):
		\begin{figure}[H]
			\centering
			\includegraphics[scale=0.75]{img/cosmology/universe_zoom_8.jpg}
			\caption{Simplified illustration of the nearest stars (source: http://atunivers.free.fr, author: Richard Powell)}
		\end{figure}
		This map shows some stars up to a distance of $12.5$ light years from our Sun (there would be $33$ identified to this date). Most of these stars are red dwarfs - stars with a tenth of the mass of the Sun and a hundred times less bright. About $80\%$ of stars in the Universe are red dwarfs, and the nearest star - Proxima Centaure - is a typical example.
		
		The map below show all known stars within $20$ light years. There are a total of $77$ systems containing $110$ stars:
		\begin{figure}[H]
			\centering
			\includegraphics[scale=0.75]{img/cosmology/universe_zoom_9.jpg}
		\end{figure}
		The distances between stars are huge. The distance from the Sun to Proxima Centauri is $4.22$ light years, or $40$ trillion kilometers. Walk this distance would take a billion years. Even the fastest space probes in this early 21st would need $6,000$ years to make the trip. There are currently four probes leaving the solar system - Pioneer 10 and 11 and Voyager 1 and 2 but we will likely lose contact with them within the next two years (if it's not already done when the reader see these lines). The following diagram attempts to show these distances by broadening the scope from the inner solar system to Alpha Centauri:
	\begin{figure}[H]
		\centering
		\includegraphics[scale=0.75]{img/cosmology/universe_zoom_10.jpg}
		\end{figure}
	\end{enumerate}
	And finally an artist's logarithmic scale conception of the observable universe with the Solar System at the center, inner and outer planets, Kuiper belt, Oort cloud, Alpha Centauri, Perseus Arm, Milky Way galaxy, Andromeda galaxy, nearby galaxies, Cosmic Web, Cosmic microwave radiation and the Big Bang's invisible plasma on the edge.
	\begin{figure}[H]
		\centering
		\includegraphics[scale=0.4]{img/cosmology/observable_universe_logarithmic_illustration.jpg}
		\caption{Artist's logarithmic scale conception of the observable universe (source: Wikipedia, author: ?)}
	\end{figure}
	
	\pagebreak
	\subsection{Cosmic Microwave Background (CMB)}
	We have already mention the cosmic microwave background earlier above and we have given numerous illustrations of it and the corresponding satellites names and observations related to its experimental study. The purpose here is to have a mathematical approach of that latter.
	
	The existence and properties of the cosmic radiation discovered by Arno Penzias and Robert Woodrow Wilson were mainly due to the two physical phenomena that we will now describe in broad outline.

	The expansion of the Universe has for consequence in its gradual cooling. From the fantastically high values that have reigned immediately after the Big Bang that created the Universe, the temperature gradually decreased. When it reaches about $3,000$ [K] occurs the first of two crucial phenomena that interest us here: the radiation, which until then was in thermal equilibrium with the material particles practically ceases to interact with them and became independent. In the "standard model" of evolution of the Universe, we calculate that this crucial moment is situated $0.68\cdot 10^6$ years after the Big Bang (see the proofs further below).
	\begin{figure}[H]
		\centering
		\includegraphics[scale=0.35]{img/cosmology/planck_history_of_universe.jpg}
		\caption{2D illustration of Universe past history following 21st century hypothesis (source: ESA)}
	\end{figure}
	We can first qualitatively understand the physical reasons for this (further below we will approach this with maths stuff!). Shortly before, when for example the temperature was $100,000$ [K], the Universe contained mostly photons, electrons and bare atomic nuclei (mostly protons, and, to a lesser extent, $\alpha$ particles, helium $4$ nuclei). The temperature was too high so that the electrons and nuclei may form stable atoms. The interaction between the photons and charged particles (mainly electrons, the lighter of them) is sufficiently intense, and the density of the latter was then sufficiently strong, so that the photons were continuously diffused, transmitted and absorbed . Despite its expansion, the Universe was at every moment at equilibrium; its temperature $T$ was consistently well defined, although decreasing over time, the photon energy, that is to say the pulse of radiation, was therefore distributed according to Planck's law for this temperature $T$!
	
	The temperature decrease then permit the formation of atoms from the electrons and nuclei. This process led to a rapid drop in average cross section of interaction between photons and material particles (mainly due to the disappearance of free electrons), so that the Universe became transparent to photons. A quantitative assessment of the characteristics of the phenomenon is this decoupling occurred when the temperature dropped to $3,000$ [K] (see the simple mathematical approach further below).

	At the moment of decoupling, the volume density of the radiation energy is distributed in the pulsations spectrum according to Planck's law (\SeeChapter{see section Thermodynamics}):
	
	where we assume that $T$ is the temperature ($3,000$ [K] approximately - ionization temperature of the simplest atoms) at the moment of decoupling. This distribution will then evolve under the influence of the expansion of the Universe.

	Let us consider the photons located at time $t$ in the volume $r^3$, and whose pulsation $\omega$ with a variation $\mathrm{d}\omega$. Their number is then using previous relation equal to:
	where we assume that $T$ is the temperature ($3,000$ [K] approximately - ionization temperature of the simplest atoms) at the moment of decoupling. This distribution will then evolve under the influence of the expansion of the Universe.

	Let us consider the photons located at time $t$ in the volume $4/3r^3\cong r^3$, and whose pulsation $\omega$ with a variation $\mathrm{d}\omega$. Their number is then using previous relation equal to:
	
	As there is no absorption or emission of photons at this temperature (it is a hypothesis but as experimental measurements seem to confirm this model...), this number will remain constant. But because of the expansion of the Universe, these photons constant number will occupy a larger volume, and gain greater wavelength $\lambda$ (following the expansion of the structure of space due to the positive value of the Hubble constant) that is to say, a rather smaller pulsation $\omega$ (the equivalent of the Doppler effect). To clarify, let consider the situation at time a further time $t'$. All lengths of the Universe have increased between, between $t$ and $t'$, by the same scaling factor $F$ following the Hubble's law: the chosen radius $r$ of our previous selected sphere volume has become obviously:
	
	and the wavelength of the photons considered:
	
	so that their pulsation is equal at the instant $t'$ to:
	
	So the energy contained at this time in the volume $V\cong {r'}^3$ and in the pulsation range $(\omega',\omega'+\mathrm{d}\omega')$ that is given obviously by:
	
	is given by:
	
	The volumetric energy density $R'(\omega',T')\mathrm{d}\omega')$ at time $t'$, for the pulsation range $(\omega',\omega'+\mathrm{d}\omega')$, is then written:
	
	It follows that the spectral energy distribution is still at the instant $t'$ that of the black body:
	
	where the corresponding temperature $T'$ is immediately such that:
	
	that is more often written:
	
	Thus, after decoupling with matter, the cosmic radiation evolves maintaining the distribution of a black body whose temperature decreases regularly in the same proportion as the distances are increased during the expansion of the Universe.

	From the moment of decoupling, the $F$ factor of scale is very close to $1,000$ since from to the estimated $3,000$ [K] to go to the $2.7$ [K] measured today there is a factor of:
	
	also named "\NewTerm{recombination redshift}\index{recombination redshift}".
	
	 This value of approximately $1,000$ allows us from the Friedmann-Lemaitre model we have introduced earlier above to easily calculate at what time (horizon) of the Universe this decoupling occurred.
	 
	 If we consider a radiation dominate flat Universs we have proved roughly that the scale factor $R(t)$ fwas proportion to the $t^{2/3}$:
	
	So clearly we have the age age ratio on:
	
	That is:
	
	So to get the age of the universe at $z=1100$ we would therefore have to divide the age now, by the factor $36482$. This gives:
	
	Or more generally:
	
	 Thus we find a value of approximately $380,000$ years. This is according to what most textbooks gives without proof.
	 
	 Now let us see how we can get an an approximation of the $3,000$ [K]. Actually observations give that the density of radiations seems to be:
	
	and for that of matter:
	
	Then we have the photon to baryon ratio::
	
	\begin{tcolorbox}[title=Remark,colframe=black,arc=10pt]
	It is more common in textbooks to define the baryon to photo ratio as:
	
	\end{tcolorbox}
	Now let us consider that the first atoms absorbing photons and avoiding the transparency of the Universe were hydrogen atoms. As electrons "orbiting" the hydrogen proton never had by assumption the time to "relax" (reached their fundamental state) because of the density of the early Univers we can assume that all absorbed photons were for electrons between the first and second main quantum number. But as we have proved in the section of Corpuscular Quantum Physics:
	
	and for Hydrogen:
	
	So keep the value $13.605\;[\text{eV}]$ in mind.
	
	\subsection{Einstein Cosmological Model}
	Ok... We have seen a lot of stuff and all this without involving General Relativity! Let us see now a more robust way to derivate a more complete version of Friedman equations using a special metric AND General Theory of relativity. The result will be what we name the "\NewTerm{standard cosmological model}\index{standard cosmological model}".
	
	\subsubsection{Robertson-Walker metric}
	We will now introduce an important metric for General Relativity and Comsology that is the "\NewTerm{Friedmann–Lemaître–Robertson–Walker (FLRW) metric}\index{Friedmann–Lemaître–Robertson–Walker metric}\index{Friedmann–Lemaître–Robertson metric}\index{Robertson-Walker metric}".
	
	To see how we derive this metric with first consider the elemetary distance on a $2$D hypersurface:
	
	on if the $2$D hypersurface is a sphere, we have:
	
	For that latter relation if $R=c^{te}$ we have:
	
	After simplification:
	
	Therefore:
	
	We see that if $R\rightarrow +\infty$, then:
	
	Now let us consider a $3D$ hypersurface in a $4D$ space such that:
	
	The trick that explain with we put ourselves on the $\mathcal{S}^3$ sphere is that we anticipate in advance that by doing a change of variable we will reduce this metric to three coordinates only!
	
	We have also obviously:
	
	If we assume $R=c^{te}$ we get by differentiation:
	
	After simplification:
	
	That is:
	
	Now we use spherical coordinates (\SeeChapter{see section Vector Calculus}):
	
	But as we know that in the general:
	
	Therefore having all this, we can rewrite:
	
	as following:
	
	We will denote obviously $||\vec{r}||$ as $r$, and $||\mathrm{d}\vec{r}||$ as $\mathrm{d}r$, and as they are colinear, we have $\cos(\alpha)=1$, therefore:
	
	It is traditional to put:
	
	Therefore:
	
	As for the comoving time introduce earlier before, let us introduce the "\NewTerm{comoving coordinates}\index{comoving coordinates}" as being:
	
	Therefore:
	
	If we had made the same development but with the hyperbolic metric (\SeeChapter{see section Differential Geometry}), we would have obtain:
	
	This is why we write more generally:
	
	with:
	\begin{itemize}
		\item If $k=0$ we fall back on the flat Euclidean space (but don't forget we are on comoving coordinates!)
		
		\item If $k=+1$ we are on a spherical space
		
		\item If $k=-1$ we are in a hyperbolic space
	\end{itemize}
	Then the metric tensor including time coordinates with $-,+,+,+$ signature is given by:
	
	So that we can write obviously if $x^\mu=(ct,x^i)$:
	
	But $R$ can be an arbitrary function of time so we write traditionally:
	
	and as we have seen earlier it usage to denote the radius $a$ so therefore:
	
	and this is the final form of the Friedmann–Lemaître–Robertson–Walker (FLRW) metric.
	
	It is important to remember that $x$ has no units and is often denoted in texbooks with the following letter $\chi$ and that rigorously the scale factor $a(t)$ is defined by $R(t)/R_0$.

	The main results of the FLRW model were first derived by the Soviet mathematician Alexander Friedmann in 1922 and 1924. Although his work was published in the prestigious physics journal Zeitschrift für Physik, it remained relatively unnoticed by his contemporaries. Friedmann was in direct communication with Albert Einstein, who, on behalf of Zeitschrift für Physik, acted as the scientific referee of Friedmann's work. Eventually Einstein acknowledged the correctness of Friedmann's calculations, but failed to appreciate the physical significance of Friedmann's predictions.
	
	Friedmann died in 1925. In 1927, Georges Lemaître, a Belgian priest,	astronomer and periodic professor of physics at the Catholic University of Leuven, arrived independently at similar results as Friedmann had and published them in Annals of the Scientific Society of Brussels. In the face of the observational evidence for the expansion of the universe obtained by Edwin Hubble in the late 1920s, Lemaître's results were noticed in particular by Arthur Eddington, and in 1930–31 his paper was translated into English and published in the Monthly Notices of the Royal Astronomical Society.

	Howard P. Robertson from the US and Arthur Geoffrey Walker from the UK explored the problem further during the 1930s. In 1935 Robertson and Walker rigorously proved that the FLRW metric is the only one on a spacetime that is spatially homogeneous and isotropic (as noted above, this is a geometric result and is not tied specifically to the equations of general relativity, which were always assumed by Friedmann and Lemaître).

	Because the dynamics of the FLRW model were derived by Friedmann and Lemaître, the latter two names are often omitted by scientists outside the US. Conversely, US physicists often refer to it as simply "Robertson–Walker". The full four-name title is the most democratic and it is frequently used.[citation needed] Often the "Robertson–Walker" metric, so-called since they proved its generic properties, is distinguished from the dynamical "Friedmann-Lemaître" models, specific solutions for a(t) which assume that the only contributions to stress-energy are cold matter ("dust"), radiation, and a cosmological constant.
	
	
	
	
	\pagebreak
	\subsection{The Black Hole Universe}
	A recent hypothesis in the history of cosmology (since the 1970 as far as we know...) which is at the heart of many theoretical research (Stephen Hawking, Roger Penrose and others) is the possibility of assimilating our Universe to a Black Hole (\SeeChapter{see section of General Relativity}).

	The origin of the idea can be made from a very simple calculation:

	We know that approximately the radius of the (current) Universe is given according to our previous calculations by:
	
	But we have proved in the section of General Relativity (and Classical Mechanics) that the Schwarzschild radius is given by:
	
	What we can write for the Universe in the following form (under many assumptions: isotropy, homogeneity, spherical, etc.):
	
	which with the values of the critical density and the radius of the cosmological horizon calculated earlier above gives:
	
	So, roughly speaking, knowing all the uncertainties that we have accumulated in particular that on the Hubble constant we see that the Schwarzschild radius is not very far from the radius of the present Universe.

	As curious as it may seem, this question is not so far-fetched and is very seriously studied. It is therefore theoretically possible that our whole universe is encapsulated in a gigantic Black Hole (therefore of very large mass and very low density as we see it with our numerical values) of another inaccessible Universe ...

	What is certain is that if this were the case, the expansion of the Universe (now observed) could not continue beyond the horizon of this super Black Hole because nothing coming from within can cross this horizon. However, recent observations seem to show that the expansion of the Universe is far from slowing and tends to accelerate with time, which is in contradiction with such a Black Hole Universe...

	\begin{flushright}
	\begin{tabular}{l c}
	\circled{90} & \pbox{20cm}{\score{3}{5} \\ {\tiny 17 votes,  70.59\%}} 
	\end{tabular} 
	\end{flushright}

	%to make section start on odd page
	\newpage
	\thispagestyle{empty}
	\mbox{}
	\section{String Theory}
	\lettrine[lines=4]{\color{BrickRed}I}t must be considered in this section that string theory (and verbatim superstring theory) is currently speculative and could not be verified (confirmed) or falsified by experience following the scientific method. We should therefore take with caution the developments that follow and to be the most critical possible!
	
	It is also a theory (we can not talk currently about "model") of unification of the forces that is not new since it soon have over thirty years and is trying to bridge the issues of the standard model of particles and also to unite General Relativity and quantum physics (which is not without difficulty since the latter is dependent on the background unlike General Relativity). She is one of the many theories that exist in modern physics and is trying in the early 21st century this unification (there are dozens of other more or less known).
	
	\begin{tcolorbox}[title=Remark,colframe=black,arc=10pt]
	If this subject is covered in the Cosmology chapter and not Atomistic one it is only for an pedagogical reason. Indeed, the basic formalism of string theory is much closer to relativistic mechanics (special and General Relativity) than that of the wave quantum physics or quantum physics fields. It seemed therefore more suited to this day (!), to provide continuity in the mathematical formalism and its interpretation rather than a thematic continuity with a relatively different approach than the usual formalism of quantum physics.
	\end{tcolorbox}
	Well, string theory is a theoretical framework in which the point-like particles of particle physics are replaced by one-dimensional objects named "\NewTerm{strings}\index{strings}" as its name suggest it... It describes how these strings propagate through space and interact with each other. On distance scales larger than the string scale, a string looks just like an ordinary particle, with its mass, charge, and other properties determined by the vibrational state of the string. In string theory, one of the many vibrational states of the string corresponds to the graviton, a quantum mechanical particle that carries gravitational force. Thus string theory is a theory of quantum gravity.

	String theory is a broad and varied subject that attempts to address a number of deep questions of fundamental physics. String theory has been applied to a variety of problems in black hole physics, early universe cosmology, nuclear physics, and condensed matter physics, and it has stimulated a number of major developments in pure mathematics. Because string theory potentially provides a unified description of gravity and particle physics, it is a candidate for a theory of everything, a self-contained mathematical model that describes all fundamental forces and forms of matter. Despite much work on these problems, it is not known to what extent string theory describes the real world or how much freedom the theory allows to choose the details.
	
	The undeniable advantage of string theory, besides the fact that mathematically it is quite indigestible but not really worse than General Relativity, is that it avoids in a certain order... many singularities in the calculations unlike other contemporary theories that consider the objects as points (so zero volume and length...).
	
	The undeniable advantage of string theory, besides the fact that mathematically it is quite indigestible but not really worse than General Relativity, is that it avoids in a certain order ... many singularities in the calculations unlike other contemporary theories that consider the objects as points (so zero volume and length...).
	
	This theory, although aesthetic and remarkable in that it uses for its calculations bases foundations that have over 200 years, is lacking in our opinion to work with successive analogies, as we shall see, with current relativistic and quantum theories. While this is not dramatic in itself, the theory may seem to lose a little of its own autonomy even though the fact it is not. The reader should therefore not be badly surprised in the development that will follow...
	
	The main particularity of string theory is that his ambition does not stop to this reconciliation, but it claims to successfully unify the four known elementary interactions, we speak therefore about "theory of everything", while staying on two hypothesis/assumptions:
	
	\begin{enumerate}
		\item[H1.] The fundamental building blocks of the universe would not be point particles, but a variety of vibrating string having a given stress in the manner of an elastic. What we perceive as particles with special characteristics (mass, charge, etc.) are merely distinct strings vibrating differently. With this assumption, string theories admit a minimum scale and make it easy to avoid the emergence of some infinite amounts that are inevitable in usual quantum field theories.
		\item[H2.] The Universe could contain more than three spatial dimensions. Some of them, folded on themselves, being invisible to our scales (by a procedure named "\NewTerm{dimensional reduction}\index{dimensional reduction}").
	\end{enumerate}
	
	Despite promising first partial promising results and also remarkable rich mathematical background the string theory remains however incomplete. On the one hand, a multitude of solutions to string theory equations exist, which poses a selection problem for our Universe and, secondly, even if many similar models were obtained, none of them allows to reproduce accurately the standard model of particle physics...
	
	That said ... let us begin our initiation:
	
	\subsection{Wave equation of a transervsal string}
	The aim here will be in a first step to determine the non-relativistic wave equation of a string excited transversely using calculations that we made in the section of Wave Mechanics. Once this work done, we will pass to the study of relativistic strings and see their wave equation, as well as the non-relativistic version, can be assimilated to the current conservation equation we had proved in the section of Electrodynamics.
	
	We begin by recalling the form of the action that we proved in the section of Wave Mechanics for a non-relativistic string:
	
	with:
	
	Now, in the same way as we did in section of Analytic Mechanics (and in that of Quantum Field Theory), we will define a notation by an analogy to the canonical moments of the string:
	
	with $y'=\partial y/\partial x$. It is simply the derivatives from the Lagrangian density with respectively the first and second argument. More explicitly, we get then directly by doing the calculation (\SeeChapter{see section Wave Mechanics}):
	
	So if we rewrite the variational of the action we obtained in the section of Wave Mechanics with this canonical notation, we get:
	
	Making use of the same methods as in the section of Wave Mechanics, our variational can be express after simplification again in the form of three terms:
	
	The conditions to find the extreme value (according to the principle of least action) are the same as those seen in the section of Wave Mechanics. Thus, for the third term, we therefore have the wave equation of a transversely excited string with the canonical form given by:
	
	\begin{tcolorbox}[title=Remark,colframe=black,arc=10pt]
	It obviously should be noted that this form of writing will greatly facilitate our work!
	\end{tcolorbox}
	It must be observed (as it is remarkable!) also that as in the sectionof Analytical Mechanics, the canonical moment $\mathcal{P}^t$ as defined above, coincides perfectly (the hazard makes things well) with the density momentum that we obtained in the section of Wave Mechanics. Effectively:
	
	Thus, by analogy with the Analytical Mechanics (where we recall, the derivative of the Lagrangian with respect to the speed gives the momentum), $\dot{y}$ plays well the role of speed and thus the derivative of the Lagrangian density by this latter gives the momentum density $\mathcal{P}^t$!!!
	
	Remember also another point that has been seen in the section of Wave Mechanics, the extremum of the action ($\delta S=0$) imposes use the Neumann boundary conditions, which leads us to write $\mathcal{P}^x=0$.
	\begin{tcolorbox}[title=Remark,colframe=black,arc=10pt]
	In the context of string theory to relativistic with more than 3 dimensions, it is possible to generalize the concept of boundary conditions considering the constraints in space like hypersurfaces named Dp-branes with $p$ dimensions. The usual Dirichlet boundary conditions then correspond to the situation where the ends of a strings are constraint by a 0-brane. The Neumann condition for a free string in $p$ dimensions corresponds to a constraint on a Dp-brane.
	\begin{figure}[H]
		\begin{center}
		\includegraphics{img/cosmology/dp_brane.jpg}
		\end{center}	
		\caption[]{Illustration of Dp-branes}
	\end{figure}
	\end{tcolorbox}
	
	\subsection{Non-relativistic Wave equation of a transversal string}
	We will now determine the action of a relativistic string. We can, to lay the foundations of our study, remember that a point particle draws a line in space-time (each point on the line is marked by a time coordinate and three space coordinates). Therefore, by extension, a string that is a two-dimensional element (if we consider it with no thickness) plots a surface in the spacetime.
	
	Thus, just like the line that that draw a point particle in space-time is named a "world line" (\SeeChapter{see section Special Relativity}), the surface traced by a string will by analogy be named a "\NewTerm{surface Universe}\index{surface Universe}".
	
	A string in a closed space-time Minkowski trace, for example, a tube, while an open string trace a band:
	\begin{figure}[H]
		\begin{center}
		\includegraphics{img/cosmology/open_closed_string.jpg}
		\end{center}	
		\caption{Universe surface generated by respectively an open/closed string}
	\end{figure}
	in the figure above, with two spatial dimensions and one implicit temporal dimension, the string is motionless in our current space. It moves in space-time (as time goes on the vertical axis) but not in space in the example above (it would take an additional spatial component to see such a movement).
	\begin{tcolorbox}[title=Remark,colframe=black,arc=10pt]
	\textbf{R1.} Caution! Remember that the diagram above is in three dimensions while the space-time has four dimensions.\\
	
	\textbf{R2.} Remember also that the time vector of the orthogonal basis is always perpendicular to all other spatial components (this remark will be useful during our proof of the Nambu-Goto action).
	\end{tcolorbox}
	
	During our proof of the equation of motion in the section of General Relativity, we reparameterized the particle's Universe line with a parameter that was the proper time of the particle $t$. Indeed, you only have to remember parametric equations that represent curves. For example with Maple 4.00b:
	
	\texttt{>with(plots):}\\
	\texttt{>spacecurve([cos(t),sin(t),t],t=0..4*Pi,axes=boxed);}
	
	\begin{figure}[H]
		\begin{center}
		\includegraphics{img/cosmology/curve_parametrization.jpg}
		\end{center}	
		\caption{Elementary illustrated refresh of a parametric curve}
	\end{figure}
	and the same procedure is valid for a line in four dimensions (time + space).
	
	We were thus arrived to construct the expression of the action $S$ of it before applying the variational principle.
	
	We will do the same for a relativistic string with the difference that we will reparametrized the surfaces generated by the strings this time. The constraints we impose are that the chosen parameters will also have to be (in reference to the case of the particle) relativistic invariants.
	
	As we have therefore seen in the section of General Relativity, a world line can be reparametrized naturally using only one parameter (curvilinear abscissa). A surface in space, however, is a two-dimensional object, we assume by extension that it requires two parameters $\zeta^1,\zeta^2$ (one more) to be described completely.
	
	Indeed, we guess, that one of the two parameters will be the proper time (to make the surface evolve in the time), the second parameter will give a "thickness" of what would be only a Universe line if it did not exist. It would be sufficient in a three-dimensional space that the second parameter had, to generate a surface, the dimensions of length but in the four dimension space-time the second parameter must have the units of a surface.
	
	Given a parameterized surface, we can draw on them isolines of the parameters (lines where the two parameters $\zeta_,\zeta_2$ are constant over the entire surface). These contours cover the surface as a grid (see figure a little bit further below).
	
	The parametric equation of a volume in space requires three parameters as we saw in the section of Analytic Geometry. Thus, if a parametrized area in Euclidean space can be represented by a vector of the type:
	
	during a reparameterization and making use of the tensor notation of Minkowski space-time as seen in the section of General Relativity, we have (by restricting for the moment to the particular case of two spatial dimensions and one of time):
	
	Thus, the surface is the image of the parameters $(\xi^1,\xi^2)$. Alternatively, we can see the components $(\xi^1,\xi^2)$ as the time and space coordinates of the surface, at least locally!
	
	We now want to calculate the area of an element of any type of space in the same way as we did for the curvilinear abscissa of any world line in the section of General Relativity. This raises the question of the form of the differential surface element??? Should we take the multiplication of the differential of the two previously selected parameters as being a square, rectangle, circle, or other?
	
	In fact, we will put our choice on a parallelogram! This choice may seem completely arbitrary for now but as we'll see a few lines later, this choice coincides for mathematical reasons with that we name the "\NewTerm{induced metric}\index{induced metric}" of the surface itself (rather remarkable result!).
	
	Thus, let us denote by $\mathrm{d}\vec{v}_1$ and $\mathrm{d}\vec{v}_2$ the sides of the parallelogram. They are the image by $\vec{x}$ of the couples $(\xi^1,0)$ and $(0,\xi^2)$ respectively:
	\begin{figure}[H]
		\begin{center}
		\includegraphics{img/cosmology/elementary_surface_study_configuration.jpg}
		\end{center}	
		\caption{Configuration for the study of an elementary surface element}
	\end{figure}
	Therefore we can write:
	
	and then:
	
	Now let us calculate the surface $\mathrm{d}A$ (we will not take the letter $S$ to avoid confusion with the variable representing the action in this section) of the parallelogram (\SeeChapter{see section Vector Calculus}):
	
	using the dot product, this can be rewritten as:
	
	using the previously established relations this can be written:
	
	the latter relation is the general shape of a surface element of a parametrized pattern. The total surface is obviously given by:
	
	Just as in the framework of the study of the principle of least action (\SeeChapter{see section Analytical Mechanics}) we searched the optimum path for a particle browsing a universe line, for a string, we have to search for the optimum of the surface $A$ by minimizing the function $\vec{x}=\left(\xi^1,\xi^2\right)$.
	
	This latter form is, however, a little heavy and does not show anything known particularly or is not similar to any form already known in another field of physics. We will see, however that by digging a little bit however it is possible to get something pretty interesting.
	
	Consider now a vector $\mathrm{d}\vec{x}$ and its squared length (norm) given by the scalar product:
	
	\begin{tcolorbox}[title=Remark,colframe=black,arc=10pt]
	This approach of separating the wave function into the composition of a wave function of the center of mass and the relative movement is also used in the context of the study of poly-electronic atoms, but with one difference: as the nucleus is much more massive than the processing electrons (in approximation ...), the center of mass is assimilated to the nucleus of the atom and the relative motion to the entire electron cloud. This approximate approach is well known under the designation "\NewTerm{Born-Oppenheimer approximation}\index{Born-Oppenheimer approximation}".
	\end{tcolorbox}
	Careful in the future not to "see" the $s$ as squared in the $\mathrm{d}s$ (as it is the case in Special and General Relativity) but remember well that it is the $\mathrm{s}$ that is squared (the notation may lead to confusion...).
	
	Thus, the squared length of $\mathrm{d}\vec{x}$ can be expressed in tensor form:
	
	what we will note by convention in the future:
	
	The quantity $g_{ij}(\xi)$  is named the "\NewTerm{induced metric of the parametrized area}\index{induced metric of the parametrized area}" (because contains a scalar product which quite generally uses a metric... hence the term "induced") and is therefore a matrix of dimension $2 \times 2$. It is obvious that the choice of this name comes from the resemblance with the usual metric as we have defined in our study of tensor calculus and from its use in special and General Relativity.
	
	The matrix $g_{ij}(\xi)$ has therefore by design and definition the form:
	
	Now let us come back to our expression of the surface generated by the string:
	
	and let us quickly calculate the determinant (\SeeChapter{see section Linear Algebra}") of the matrix $g_{ij}(\xi)$:
	
	and so what? Well here it is! $A$ can now been expressed as:
	
	Thus, the choice of the parallelogram as elementary surface is best explained here!
	
	Now we will adopt the traditional notations of string theory in relation to the expression of the surface. Thus, just as the time-space coordinates are described in Special Relativity the space-time four-vector:
	
	we will describe the surfaces Universe by (we now turn to the notation making use of the four dimensions of space-time):
	
	This notation will save us in the future to have to confuse, if the theory leads us there, the traditional space-time coordinates $x^\mu$ with the image function of the surface Universe $x^\mu(\tau,\sigma)$  and this especial because physicists being sometimes a little lazy shorten this latter $x^\mu$... hence the choice of the capital letter.
	
	It is then much more appropriate and wise to change the notation.
	
	From now on we will name "\NewTerm{string coordinates}\index{string coordinates}" the surface Universe described by $X^\mu$.
	
	This small change in notation obviously not change the interpretation of the image of the function. Given a couple $(\tau,\sigma)$ involving proper-time element and surface element of the pre-image, this point is projected onto a surface element of the space-time coordinates of the string:
	
	
	\subsubsection{Nambu-Goto Action}
	The Nambu–Goto action is the simplest invariant action in bosonic string theory, and is also used in other theories that investigate string-like objects (for example, cosmic strings). It is the starting point of the analysis of zero-thickness (infinitely thin) string behavior, using the principles of Lagrangian mechanics. Just as the action for a free point particle is proportional to its proper time—i.e., the "length" of its world-line—a relativistic string's action is proportional to the area of the sheet which the string traces as it travels through spacetime.
	
	In the case of a Universe surface the parameters are then by convention $\tau$ and $\sigma$, where as in special and General Relativity, the proper time may be in the range:
	
	the second can only be positive since this is a surface:
	
	and the coordinates of this surface corresponding to the parameter space is therefore:
	
	where once again, for recall, the parameter $\tau$ is considered as the variable describing the time (there must be one!), and $\sigma$ the variable describing the spatial extension of a string (that is to say that the condition $\sigma\in]0,\sigma_1[$ involving the finite length of the string).

	The parameters $(\tau,\sigma)$ describe therefore a surface in the space of pre-images:
	\begin{figure}[H]
		\begin{center}
		\includegraphics{img/cosmology/space_time_surface_parametrization.jpg}
		\end{center}	
		\caption{Parameterization of a space-time surface}
	\end{figure}
	The ends of the string have a constant value $\sigma$. However, as time passes and that the ends of the string on the Universe surface move we must notice an essential condition of the Universe surface concerning the both ends of an open string:
	
	\begin{tcolorbox}[title=Remark,colframe=black,arc=10pt]
	This condition is made on the component $X^0$ because it corresponds to the component $x^0$ of the space-time four-vector which is nothing else, in natural units, $t$ (the proper time). Therefore, time passes and is never constant, this is why we  impose this derivative as being nonzero.
	\end{tcolorbox}
	And using the standard conventions in physics for writing the derivatives with respect to time or space components, we agree to adopt also now the following notations:
	
	since as:
	
	then:
	
	The surface is therefore written:
	
	However, there is a problem here! Indeed, let us look if the radicand (term under the root) has a tangible physical reality ...

	For this, we must first consider the left side of the figure below representing the surface patch described by an open string:
	\begin{figure}[H]
		\begin{center}
		\includegraphics{img/cosmology/surface_patch_for_study_radicand_string_action.jpg}
		\end{center}	
	\end{figure}
	At each point $P$ of this surface patch (assumed differentiable at every point) there are endless tangents, all in the same plane, which we will denote for example by $\vec{v}$ and thus form a surface tangent at $P$.

	Now, as the space in which the surface patch of the string held in a spatial and temporal orthonormal basis, the tangent vectors $\vec{v}$ can then in turn be decomposed into a two-dimensional spatial and temporal local orthogonal base at the point $P$ such that the vectors of the base are two vectors (see our study of surface patches in the section of Differential Geometry):
	
	all other tangent vectors expressing as a linear combination thereof.
	
	However, a problem remains in our decomposition (...): the unit vectors of the local orthogonal basis at $P$ have units that differ... For this let us add a dimensional factor  to the spatial component (this is arbitrary because the conclusion will be the same regardless of the component on which you put the sizing factor) as we did in Special Relativity with the time axis:
	
	This dimensional factor can also be used to get all the tangent vectors such as:
	
	Indeed, if $\lambda\in [-\infty,+\infty]$, then for $\lambda=0$ we get the vector $\partial X^\mu/\partial \tau$ and for $\lambda=+\infty$ the vector $\partial X^\mu/\partial \sigma$ . And all intermediate values, we get all the tangent vectors as shown on the left side of the previous figure.

	Now, let us recall that we saw in the section of Special Relativity that there exist according to the curvilinear abscissa:
	
	of the Universe line of the light type ($\mathrm{d}s^2$), space ($\mathrm{d}s^2<0$) or time ($\mathrm{d}s^2>0$) if we consider the four-vectors $x^\mu$.

	It must be true by analogy the same for the tangent vectors to the surface and given by:
	
	Therefore:
	
	that corresponds to an equation of the second degree on $\lambda$, and that must, to have negative values (Universe surface patch of the type of space) or positive (Universe surface type of the type time) have at least two roots (see the right part of the previous figure ). This brings us back to the condition that the discriminant is strictly positive (\SeeChapter{see section Calculus}):
	
	Therefore:
	
	Into condensed form this is equivalent as writing:
	
	The surface must therefore be written as:
	
	if we want the radicand has a physical sense.
	
	By analogy with the Lagrangien of General Relativity we write the latter:
	
	that we will justify a little more robustly further below.
	
	Recall now that the action $S$ of a point particle is proportional to its world line (proper time). Thus by analogy, the action $S$ of string will be proportional to the Universe area:
	
	which gives:
	
	Which brings us very frequently in the literature to find the action of a string in the following form:
	
	or more stylized:
	
	Relation to compare with the Lagrangian of a free particle (\SeeChapter{see section Analytical Mechanics}) and with the Lagrangian density of a field (\SeeChapter{see section of Quantum Field Theory}):
	
	The functional $S$ has for units the one of a surface. This because the $X^\mu$ have a unit of length and inside the root each is at the fourth power and the units of $\tau,s\sigma$ cancel between the inner root and the differential that are outside it.
	
	Now, by definition of the action, the units that we have to get should match that of energy multiplied by time. That is to say joules [J] or using the international system of units $[\text{kg}\cdot\text{m}^2\cdot s^{-1}]$. For now, we have:
	
		To get to the action units we want, then we have to multiply the expression of the surface $A$ by a quantity whose units are $[\text{kg}\cdot\text{s}^{-1}]$. To chose these quantities, we will inspire us from our study of Wave Mechanics. When we worked with (non-relativistic) strings we saw that the properties to be considered were the stress and the velocity of string wave propagation. We'll try to take the following stress/speed ratio:
	
	where appears therefore that the stress of the string at rest $T_0$ and the speed of light $c$. 
	
	\begin{tcolorbox}[title=Remark,colframe=black,arc=10pt]
	This is similar to the point material physics where in the action we find the rest mass (equivalent to the tension at rest of the string) and the speed of light (\SeeChapter{see section Special Relativity}).
	\end{tcolorbox}
	Thus, the "\NewTerm{Nambu-Goto action}\index{Nambu-Goto action}" can now be written:
	
	\begin{tcolorbox}[title=Remark,colframe=black,arc=10pt]
	We will prove later why we put a factor "$-$". However, a small analogy with the action of a point particle, for which we also have a "$-$" sign (\SeeChapter{see section of Special Relativity}), can easily be done...
	\end{tcolorbox}
	Let us define for what will follow:
	
	What we can also write in matrix form:
	
	using the determinant of the matrix, it comes:
	
	So we can then write the action of a relativistic string in the final following condensed form:
	
	which is nothing else than the "\NewTerm{Nambu-Goto condensed form}\index{Nambu-Goto condensed form}" form of a relativistic string.
	
	We will now obtain the equation of motion by varying the action. For this, we will exactly inspire us of the methods seen when determining the non-relativistic wave equation of string in the section of Wave Mechanics.

	Thus, we rewrite the Nambu-Goto action by defining a Lagrangian density $\mathcal{L}$ such that:
	
	where $\mathcal{L}$ is therefore defined by:
	
	We will now apply the variational principle on the action in order to get the equation of movement of a string. The development and approximation are perfectly similar to those seen at the beginning of this section for the non-relativistic string. Let us recall that we obtained as Lagrangian density and as an expression of the action:
	
	and that the application of variational gave us:
	
	But what we did not see in the section Wave Mechanics is that the latter relation could easily be written also from the Lagrangian density:
	
	Therefore, for the relativistic string, we have an identical form by applying developments in all points similar (even if the Lagrangian density has a different form):
	
	and as we did at the beginning of this section for non-relativistic strings, we will introduce the canonical momentum (pulse density/momentum if you prefer) of the string by choosing for the notation:
	
	where in the details, we get very easily (it's a simple derivative but if you wish by contacting us, we can detail the developments as always in this book) the longitudinal and transverse momentum:
	
	\begin{figure}[H]
		\begin{center}
		\includegraphics{img/atomistic/string_momentum_density.jpg}
		\end{center}	
	\end{figure}
	by making use of this notation, we can then write:
	
	Making use once again of exactly the same methods as those seen in the section Wave Mechanics, our variational can be written, after simplification, again in the form of three terms:
	
	The conditions to find the extrem8k (according to the principle of least action) remain the same as in Wave Mechanics. Thus, for the third term, we have well the wave equation of a transversely excited stringwith the following canonical form:
	
	As far as we know this equation is horribly difficult to solve but choosing an appropriate parameterization can nevertheless simplify the task.
	
	\subsection{Lagrangian of a String}
	Let us recall that we have:
	
	and that with this choice, we have:
	
	\begin{gather*}
		\begin{aligned}
		&=-\dfrac{T_0}{c}\int \mathrm{d}t\mathrm{d}\sigma\sqrt{\left(\dfrac{\partial \vec{X}}{\partial t}\circ\dfrac{\partial \vec{X}}{\partial \sigma}\right)^2+c^2\left(\dfrac{\partial \vec{X}}{\partial \sigma}\right)^2-\left(\dfrac{\partial \vec{X}}{\partial t}\right)^2\left(\dfrac{\partial \vec{X}}{\partial \sigma}\right)^2}\\
		\end{aligned}
	\end{gather*}
	Now let use what we have seen in section of Differential Geometry with the Frenet's triad:
	
	where $\vec{T}$ is the tangent to the Universe surface at a time $t$ at the neighborhood of a given point. We had also notice in this same section that by definition:
	where $\vec{T}$ is the tangent to the Universe surface at a time $t$ at the neighborhood of a given point. We had also notice in this same section that by definition:
	
	But we can write:
	
	where it should be remembered that $\partial \vec{X}/\partial \sigma$ is taken at a fixed time $t$. As the lines of the surface Universe of constant $ $t describe the string, then $\partial \vec{X}/\partial \sigma$ is tangent to the string.

	And as:
	
	Then $\mathrm{d}\vec{X}/\mathrm{d}s$ is collinear to $\partial \vec{X}/\partial \sigma$ and thus also tangential to the string (information that we did not have a few lines before!). These small discovers being made, let us go back to:
	
	it already gets a little more interesting!

	Now let consider the following figure:
	\begin{figure}[H]
		\begin{center}
		\includegraphics{img/cosmology/recall_dot_product.jpg}
		\end{center}	
		\caption[]{Illustrated recall of the dot product}
	\end{figure}
	where $\vec{u}$ is any vector and $\vec{n}$ a unit (dimensionless) vector and $\vec{v}$, the orthogonal projection of $\vec{u}$ on $\vec{n}$. We then have (\SeeChapter{see section Vector Calculus}):
	
	Now if we seek for the vector $\vec{v}$ we will have to multiply  the whole $\vec{n}$:
	
	Finally, if we seek the expression of the vector $\vec{}$ it comes immediately:
	
	Then by similarity, we can write:
	
	where $\vec{w}$ is perpendicular to $\partial\vec{X}/\partial s$ and as the unit of speed. By construction, $\vec{w}$ is therefore the transverse velocity of the speed at a time $t$ t since $\partial\vec{X}/\partial s$ is tangent thereto. We will denote then:
	
	Let us now take, for future needs, the square norm of this last relation (be careful we process the components of the vectors directly by generalizing the vector notation!):
	
	and if we now go back to:
	
	The associated Lagrangian is then directly (not to be confused with the Lagrangian density!):
	
	as:
	
	The Lagrangian of the prior-previous relation is considered by the specialists in string theory as a natural generalization of the Lagrangian of the free particle as we get it in the section of Special Relativity and that was for recall given by:
		
	
	\begin{flushright}
	\begin{tabular}{l c}
	\circled{20} & \pbox{20cm}{\score{3}{5} \\ {\tiny 37 votes,  80\%}} 
	\end{tabular} 
	\end{flushright}

	
\chapter{Chemistry}

	\textit{\textbf{Chemistry is the science that studies the nature and properties of simple substances, the molecular action of these bodies on each other and combinations due to this action.}}(Larousse)
	\minitoc
	\pagebreak 
		%to force start on odd page
	\newpage
	\thispagestyle{empty}
	\mbox{}
	\section{Quantum Chemistry}
	\lettrine[lines=4]{\color{BrickRed}T}hroughout human history, people have tried to convert matter into more useful forms. Our Stone Age ancestors chipped pieces of flint into useful tools and carved wood into statues and toys. These endeavors involved changing the shape of a substance without changing the substance itself. But as our knowledge increased, humans began to change the composition of the substances as well—clay was converted into pottery, hides were cured to make garments, copper ores were transformed into copper tools and weapons, and grain was made into bread.

	Humans began to practice chemistry when they learned to control fire and use it to cook, make pottery, and smelt metals. Subsequently, they began to separate and use specific components of matter. A variety of drugs such as aloe, myrrh, and opium were isolated from plants. Dyes, such as indigo and Tyrian purple, were extracted from plant and animal matter. Metals were combined to form alloys - for example, copper and tin were mixed together to make bronze - and more elaborate smelting techniques produced iron. Alkalis were extracted from ashes, and soaps were prepared by combining these alkalis with fats. Alcohol was produced by fermentation and purified by distillation.

	Attempts to understand the behavior of matter extend back for more than $2500$ years. As early as the sixth century BC, Greek philosophers discussed a system in which water was the basis of all things. You may have heard of the Greek postulate that matter consists of four elements: earth, air, fire, and water. Subsequently, an amalgamation of chemical technologies and philosophical speculations were spread from Egypt, China, and the eastern Mediterranean by alchemists, who endeavored to transform “base metals” such as lead into "noble metals" like gold, and to create elixirs to cure disease and extend life.

	From alchemy came the historical progressions that led to modern chemistry: the isolation of drugs from natural sources, metallurgy, and the dye industry. Today, chemistry continues to deepen our understanding and improve our ability to harness and control the behavior of matter. This effort has been so successful that many people do not realize either the central position of chemistry among the sciences or the importance and universality of chemistry in daily life as every other scientific field presented in this book do!
	
	Before the reader to go further in reading this chapter of the book, we want to remind that the site deals mainly with applied mathematics and theoretical physics. Thus, we will address in this section only of theoretical chemistry (theoretical quantum  chemistry, theoretical thermochemistry, theoretical kinetic chemistry, etc).
	
	This choice follows the changes of chemistry since the years 1980: form a largely descriptive science descriptive, it tends to become deductive. That is to say that in addition to experience, calculation methods are constantly growing and particularly since the development of modern computing that greatly helps chemists to numerical modeling.	
	
	Theoretical chemistry, also named "\NewTerm{physical chemistry}\index{physical chemistry}" - application of methods from physics to chemistry - is too often seen as a discipline in itself. In fact, under this term any modern chemistry field is included. Thus, the investigation of any problem in advanced chemistry requires the assistance of theoretical chemistry (and this is lucky...) and the chemist must have a thorough knowledge of it. At the level of chemistry teaching as secondary branch, the role of physical chemistry is already evident: the result is an increase in the level of students, increase in the abstraction and therefore a risk in alienating the average student. Finally, the purpose is not to burden the knowledge by incorporating more new elements, but to convert the mode of approach of this discipline by substituting the most often encyclopedic knowledge statements by rational developments based on only a few assumptions and hypothesis that permits to deduce thanks to mathematics many properties thanks to colorraries.
	
	A good understanding of physical chemistry requires in our point of view necessarily to be familiar with quantum physics (\SeeChapter{see chapter Atomistic}) to have at least one approach to what an atom is and its different electron orbits before talking about connections, different filling methods of electron orbits, redox, filling layers, and others...
	
	In this sense, we will begin by studying the particular case of the hydrogen atom, which is crucial for the whole that will follow (study of polyelectronic atoms). It is therefore necessary for the reader to browse the next lines with all possible attention and to understand as best as possible the subtleties!
	
	\subsection{Infinite three-dimensional rectangular potential}
	We studied in details in the section of Corpuscular Quantum Physics the Bohr-Sommerfeld hydrogen atom using the results proved in the section of Special Relativity. This model emerged in a simplistic quantification (but not too much wrong as will discussed later below) of certain properties of matter.
	
	In the section of Wave Quantum Physics, we studied alos in details the rectilinear infinite potential wall and the harmonic oscillator without giving many more examples. Now we will move towards to resolve problems closer to those useful in chemistry with the objective of studying the hydrogen-like atom.
	
	We will now consider a particle moving freely in the three dimensional box below:
	\begin{figure}[H]
		\begin{center}
		\includegraphics{img/chemistry/box_quantum_chemistry.jpg}
		\end{center}	
		\caption[]{Three dimensional imaginary box in which the particle moves}
	\end{figure}

	The potential electric energy of the system is given by:	

	
	As in the one-dimensional case (\SeeChapter{see section Wave Quantum Physics page \pageref{quantum potential well}}), the walls of infinite potential prevent the particle from leaving the box, and the wave function is nonzero only for position vector $\vec{r}$ being inside the box. It  necessarily vanish when one of the walls is touched. The Schrödinger equation we have to solve is (\SeeChapter{see section Wave Quantum Physics page \pageref{schrodinger wave equation}}):
	
	and boundary conditions are:
	
	Notice that the Hamiltonian can be written as the sum of the Hamiltonian in each axis (we speak of the hamiltonian operators of course!). So we have:
	
	where
	
	relations that we have proved the origin in details in the section of Wave Quantum Physics of this book.
	
	Such a form is named a "\NewTerm{separable form}\index{chemical separable form}": the Hamiltonian is the sum of individual operators $H_i$ each depending only on one variable or degree of freedom $q_i$. This form reflects the independent nature of the movements described by the variables $q_i$.
	
	Remember that the joint probability of two independent events is the product of the individual probabilities of the two events separately (\SeeChapter{see section Probabilities page \pageref{joint probability}}). We therefore expect that the presence probability density in space (\SeeChapter{see section Wave Quantum Physics page \pageref{first postulate wave quantum physics}}) with multidimensional configuration is, if the Hamiltonian of separable form, a simple individual probability density product. In fact, the separable form of the Hamiltonian permits the separation of variables on the wave function itself.

	Let us write the solutions of the Schrödinger equation under the form:
	
	of a product of three factors each depending only of one coordinated.

	Substituting this notation in the Schrödinger equation, we get without technical developments (elementary algebra):
	
	or, by dividing both sides of this equation by $\xi(x)\vartheta(y)\zeta(z)$:
	
	which is a much more aesthetic and easier to remember.
	
	This equation requires that the sum of the three terms in the left-hand side is equal to a constant in the context of a conservative system (that is what often interested chemists)! Each of these three terms depending only on one and only one variable, so that their sum is equal to a constant, it is necessary that each term is itself constant! In fact, by taking the derivative of both sides of the above equation with respect to $x$, for example, we have:
	
	meaning that equation although  must be a constant which we will denote equation (as this term expresses an energy). We then have (surprise...):
	
	Similarly, we get:
	
	Notice that each of the separate equations that we have just obtained, for the movement of the particle in the three spatial directions, is a Schrödinger equation in a one-dimensional box. Thus, the three relations previously obtained independently describe each movement in the respective $x, y, z$ directions, limited to the respective ranges:
	
	and must be respectively  solved with boundary conditions:
	
	The results obtained in the section of Wave Quantum Physics when solving the Schrödinger equation in the case of straight wells give us directly:
	
	In summary the stationary states of the particle in the three-dimensional box are specified by three positive integers quantum numbers $\lambda, \mu, \nu$. The wave function is finally:
	
	and its respective energies (eigenvalues):
	
	The variable separation technique detailed above, is applicable only because the Hamiltonian is in separable form. It comes automatically therefore the three-dimensional probability density $\vert \Psi(x,y,z) \vert^2$ is the product of probability density $\vert \xi_\lambda(x)\vert^2,\vert \vartheta_\mu(y)\vert^2,\vert \zeta_\nu(z)\vert^2$, as we had anticipated it. We also notice that the energy of movement in three dimensional space is the sum of energy movements in all three spatial directions: the independence of these three directions or degrees of freedom, implies the additivity of their energy.
	
	\subsection{Molecular Vibrations}
	We studied in the Wave Quantum Physics section the harmonic oscillator. Now it is molecular chemistry that we will use all the power of the results obtained during the study of this system.
	
	The harmonic oscillator is a model of molecular vibrations, and is represented by a type of parabolic potential as:
	
	for a diatomic molecule. But we have proved in the section of Nuclear Physics that $c^{te}=m\omega_0^2$ so that we finally have for a diatomic molecule:
	
	For a polyatomic molecule, we have verbatim (by the additivity of energy):
	
	Quantities $\omega_0$ and $\omega_i$ are the vibration frequencies (or rather more correctly: the pulsation) of a molecule, diatomic in the first case and polyatomic in the second case. In the first equation, the variable $x$ represents the elongation of the bond between the two atoms $A$ and $B$ (as with a spring) in a diatomic molecule, that is to say $x=R-R_{eq}$, where $R$ is the instantaneous length of this bond, and $R_{eq}$ is its equilibrium value.
	
	In the case of a polyatomic molecule, the potential describing molecular vibrations takes a separable form in terms of summation above only if one considers special variables $q_i$ denoting collective motions of nuclei, and which are named "\NewTerm{normal vibration modes}\index{normal vibration modes}".
	
	We also saw in the section of Wave Quantum Physics that the Hamiltonian of a diatomic molecule (problem of the harmonic oscillator) can be written as
	
	For a polyatomic molecule that relationship becomes logically:
	
	The Hamiltonian above is clearly a type of separable form: it is a sum of one-dimensional Hamiltonians, each depending only on a single mode $q_i$ as variable, describing this mode as a unique spring or harmonic unit mass ($m=1$) oscillator and of pulse oscillation $\omega_i$. Therefore, a separation of variables $q_i$ is possible, reducing the Schrödinger time independant into a number of equations of the same type as that of a one-dimensional harmonic oscillator. So we need just o know the expression of the wave function for a one-dimensional harmonic oscillator, what we already have done in the section of Wave Quantum Physics where we got:
	
	and:
	
	The figure below shows the graph of the first wave functions of the above relation as well as that of their respective presence probability densities. We can see the same modal structures as those specific to functions of a particle in a one-dimensional box:
	\begin{figure}[H]
		\begin{center}
		\includegraphics{img/chemistry/one_dimensionnal_oscillator.jpg}
		\end{center}	
		\caption{Wave functions and probability density of a one-dimensional harmonic oscillator}
	\end{figure}
	Above the first energy levels of a one-dimensional oscillator with \texttt{\textbf{(a)}} their associated eigenfunction, \texttt{\textbf{(b)}} the  associated probability distribution of presence.
	
	In the limit of very large values of $n$, the probability distribution approximates more and more of that predicted by classical mechanics, the oscillator lies for the most of the time in the vicinity of the turning points defined by the intersection of potential $E_{p}$ with the level of $n$. This trend is illustrated below:
	\begin{figure}[H]
		\begin{center}
		\includegraphics{img/chemistry/one_dimensionnal_oscillator_limit.jpg}
		\end{center}	
		\caption{Probability density function of a one-dimensional harmonic oscillator for large $n$}
	\end{figure}
	For a polyatomic molecule the expression of quantified energy therefore becomes:
	
	and eigenfunctions/eigenstates become:
	
	with:
	
	The last two relations are very important because they allow among others to:
	\begin{itemize}
		\item Predict the spectrum of the molecule (spectroscopy)
		\item To study the energy bands (where does the bands of valence and conduction comes from)
		\item To locate the bonds between atoms and thus the chemical properties
	\end{itemize}
	
	\subsection{Hydrogenoid Atom}
	We consider here the quantification of a generic system made of two bodies (particles) interacting with each other and moving in a three-dimensional space. We will be prove at first that, even if the separation of dynamic variables describing individually each of the two bodies is impossible, for cons, the overall movement system (the center of mass) and internal movement, also said "relative motion", are separable. In addition, if the potential is centrosymmetric, the internal movement may also be decomposed into a rotational movement and radial movement. The quantification of the rotational movement is intimately connected to that of angular momentum.
	
	Here we focus on the mechanics of an atomic system having only one electron. This is a two-particle system: a nucleus of mass $M$ and charge $+Zq_e$, and an electron of mass $m_e$ and of charge $-q_e$.
	
	The atomic system is described by the following Hamiltonian
	
	Remember that in the section of Wave Quantum Physics we had proved during our study of functional operators:
	
	and remember also that $\vec{r}_e$ and $\vec{R}_n$ are respectively the position vectors of the electron and nucleus in the prior-previous relation.
	
	The potential electric energy being given by (\SeeChapter{see section Electrostatic page \pageref{electrostatic potential energy}}):
	
	The movements of the two particles are correlated because the two charges interact through their mutual electrical field. We can not make a separation between variables $\vec{r}_e$ and $\vec{R}_N$. By cons, a separation of variables is possible with the coordinate of the center of mass (see the definition of the center of mass in the section of Classical Mechanics):
	
	and the relative coordinate of the electron relative to the nucleus:
	
	We get therefore:
	
	and:
	
	The Hamiltonian in the center of mass repository will therefore be written:
	
	where $M_{\text{tot}}=m_e+M$ is the total mass of the system and:
	
	is its reduced mass.
	
	We clearly see that the Hamiltonian $H$ is this time set in a separable form and we can write it as follows:
	
	with:
	
	In terms of the coordinates $\vec{R}_{\text{CM}}$ and $\vec{r}_{\text{rel}}$, the function describing a stationary state of the two-body system is a product of individual wave functions (recall that the joint probability of two events is the product of probabilities), one for the movement of the center of mass and the other for the relative movement:
	
	and the energy of this state is the sum of the respective energies of movements:
	
	with:
	
	\begin{tcolorbox}[title=Remark,colframe=black,arc=10pt]
	This approach of separating the wave function into the composition of a wave function of the center of mass and the relative movement is also used in the context of the study of poly-electronic atoms, but with one difference: as the nucleus is much more massive than the processing electrons (in approximation ...), the center of mass is assimilated to the nucleus of the atom and the relative motion to the entire electron cloud\index{electron cloud}\footnote{The electron cloud is the region of negative charge surrounding an atomic nucleus that is associated with an atomic orbital. The region is defined mathematically, describing a region with a high probability of containing electrons. As we know it, the electron cloud model differs from the more simplistic Bohr model, in which electrons orbit the nucleus in much the same way as planets orbit the Sun. In the cloud model, there are regions where an electron may likely be found, but it's theoretically possible for it to be located anywhere, including inside the nucleus.}. This approximate approach is well known under the designation "\NewTerm{Born-Oppenheimer approximation}\index{Born-Oppenheimer approximation}".
	\end{tcolorbox}
	Where the Hamiltonian appearing in the first of these relations has been defined above as being:
	
	This movement is that of a particle of mass $M_{tot}$ in a three-dimensional box of infinite volume. Eigenvalues and eigenfunctions for this movement has already been obtained in our previous study, we will restrict ourselves to the study of separate equation for the relative movement, or internal movement. As no confusion will be possible between different Hamiltonians, we let down, to simplify the notations, the "rel" word in subscript.
	With $H_{\text{rel}}$ given by the relation that we have proved previously:
	
	and the relation (as proved above):
	
	then we obtain the Schrödinger equation for the relative motion:
	
	or written differently:
	
	Notice that in the case where the potential energy $E_{p}$ is a centrosymmetric source, that is to say it depends only on the length of the position vector $\vec{r}$, and not its orientation, the previous equation, written in Cartesian coordinates, is inseparable. Indeed, in Cartesian coordinates, the length $\vec{r}$ is given by:
	
	and the potential energy can not be separated into three components, each depending only one of the three variables $x, y, z$. The Hamiltonian is therefore still not a separable form and so we did not meet our target. However, the above equation is separable at the moment we make a change of coordinates to spherical coordinates. Indeed, in this coordinate system, the potential depends on only on one of the three spherical variables: the radius $r$. It is independent of the two angles $\theta$ and $\phi$.
	
	Referring to the result obtained in the study of the Laplace expressions in different coordinate systems, in the section of Vector Calculus, we got for the Laplacian of a scalar field in spherical coordinates the following expression:
	
	The hamiltonien:
	
	then becomes (simple distribution and new way to note):
	
	where:
	
	is the kinetic energy operator for the radial movement of the electron relative to the nucleus, and $L^2$ is the squared "associated" operator of the angular momentum vector:
	
	The term:
	
	where $J=\mu r^2$ is therefore an energy associated with the angular momentum $L$ (\SeeChapter{see section Classical Mechanics page \pageref{moment of inertia} and page \pageref{angular momentum}}).
	
	To understand the nature of this operator $L^2$ a detour by the notion of rigid rotor will help.
	
	\subsection{Rigid Rotator}
	If we now consider the case of a system named "\NewTerm{rigid rotor}\index{rigid rotor}" where we neglect ("restrict" would be a more appropriate term ...) the degrees of freedom of oscillation (this is the system that are the study case for linear diatomic or polyatomic molecules), the only coordinates being into play are the angles $\theta$ and $\phi$ which fix the orientation of the rotator.
	
	Thus, in this case $r$ is fixed and we have:
	
	and in view of the constraints on the potential, it is normally quite easy to understand why the rotator is said to be "rigid". In the above case, the Hamiltonian is reduced to:
	
	
	For the rest, we associate the operator $L^2$ to the square of an angular momentum, for the simple reason that he has the units of it... Indeed, let us recall that we have prove in the section of Wave Quantum Physics that when the spin is zero (so as part of our study of the hydrogenoid atom here, the spin will not be taken into account in the first instance) and that we are dealing with a single particle then the angular momentum (which we will denote by $L$ instead of $b$) is given by:
	
	where the components of the vector $\vec{l}$ are also natural numbers. By doing this similarity, we can then write the Schrödinger equation in the form:
	
	Let us recall we got in the section of Wave Quantum Physics that:
	
	by the vector product.
	
	We go now to rectangular coordinates $x, y, z$ coordinates to spherical coordinates $r,\theta,\phi$. Remember for this (\SeeChapter{see section Vector Calculus page \pageref{spherical coordinates}}) that:
	
	and that:
	
	Now let us express the total differentials:
	
	These relations can be written as an orthogonal transformation of the total differential $\mathrm{d}r,r\mathrm{d}\theta,r\sin(\theta)\mathrm{d}\phi$ by:
	
	or by the inverse transformation (if required ... it is enough to check that the two transformation matrices multiplied together give the identity matrix):
	
	It results of this for example:
	
	and finally (the method for the second and third lines is the sam as for the first!):
	
	Thus, taking into account these relationships, we obtain for example, in the case of the operator:
	
	the following developments:
	
	which gives the following result:
	
	By doing the same with:
	
	by doing the same developments:
	
	we have the following result:
	
	And for finish with:
	
	by doing the same developments:
	
	we get the following result:
	
	Finally, we have only little freedom for the movement of our rigid rotor (as it is very rigid ...) and we can write for the Schrödinger equation:
	
	where $H_\text{rot}$ is for recall, seen as the functional linear operator, and the total energy $E$ as its corresponding eigenvector.

	Therefore, we can write that angular momentum operator is given by (we change the notation so to not confuse subsequently operator and eigenvalue according to the comments we made during the satement of the postulates of Wave Quantum Physics in the corresponding section):
	
	Thus, the eigenfunctions $\Phi(\phi)$ of $\hat{L}_z$ are solutions of the equation to the eigenvalues and eigenfunctions:
	
	that is to say the differential equation:
	
	where $L_z$ is obviously the eigenvalue of $\hat{L}_z$. A simple solution to this differential would be:
	
	with for uniformity condition, depending on the properties of complex number (\SeeChapter{see section Numbers page \pageref{complex numbers}}):
	
	This mathematical condition imposes the obvious and remarkable following quantification:
	
	where (recall) $m_l$ is the magnetic quantum number.

	Knowing that (\SeeChapter{see section Corpuscular Quantum Physics page \pageref{quantum number of orbital angular momentum interval}}):
	
	We can write:
	
	Therefore, we falls back on the result(s) that we get in the section of Corpuscular Quantum Physics and Wave Quantum Physics:
	
	Which is quite satisfactory, even remarkable and enjoyable (to not say it ...).
	
	Thus, the measurement of a component of the angular of $\hbar$ which appears as a natural unit of angular momentum.
	The common eigenfunctions (!!!) to the operators $\hat{L}^2$ and $\hat{L}_z$ are in a more general framework necessarily of the form (method of separation of variables ):
	
	As the rotator is rigid, we have $R(r)=c^{e}$. This factor will eliminate itself in the equation of eigenvalues and eigenfunctions that we will determine further below. So we can not take it into account if we ant. Finally, we can write thanks to previous developments:
	
	Which brings us to the equation to the eigenvalues and eigenfunctions:
	
	That is to say
	
	Therefore:
	
	By putting:
	
	and therefore:
	
	we get a "Fuchs" like differential equation given by:
	
	Therefore finally:
	
	Whose coefficients have poles (singularities) in $\xi=\pm 1$. But, let us recall that we have:
	
	So that we often find the previous differential equation in the following form in the books after elementary algebra factorization of some terms:
	
	A nontrivial solution being, knowing the Fuchs of differential equations, that it is customary to name the "\NewTerm{associated Legendre polynomials}\index{associated Legendre polynomials}\label{legendre polynomial}" (although this is not strictly speaking a polynomial ....) because containing partly Legendre polynomials (\SeeChapter{see section Calculus page \pageref{legendre polynomials}}):
	
	that you can check by injecting this solution in prior-previous differential equation.
	
	...Following the request of a reader is an example of verification before continuing:
	
	The $m_l=l=0$ is immediate. Then let us consider the case where $m_l=l=1$:
	
	Thus:
	
	And we inject in it the associated Lagrange polynomial:
	
	Therefore:
	
	Which give after a small simplification:
	
	By derivating:
	
	Let's focus on the left part to see what it is equal to by putting everything to a common denominator:
	
	By simplifying the numerator, it should be zero. Let us see this by simplifying a first time:
	
	by distributing:
	
	Which is indeed equal to zero!!!
 	\begin{figure}[H]
		\centering
		\includegraphics[width=\textwidth]{img/chemistry/image_spherical_harmonics.jpg}	
		\caption{Legendre spherical harmonics plot}
	\end{figure}
	With the corresponding MATLAB™ 2013a script\footnote{It is a bit long but we really thinks it helps for a better understanding} (we give the script here as it is not given in the MATLAB™ companion book) provided on Internet by Sanjay Sekaran (big thanks!):
	\begin{lstlisting}[language=MATLAB]
		theta = 0:pi/40:pi;                   % polar angle
		phi = 0:pi/20:2*pi;                   % azimuth angle
		
		[phi,theta] = meshgrid(phi,theta);    % define the grid
		
		degree = 0;
		order = 0;
		amplitude = 0.5;
		radius = 5;
		
		Ymn = legendre(degree,cos(theta(:,1)));
		Ymn = Ymn(order+1,:)';
		yy = Ymn;
		
		for kk = 2: size(theta,1)
		    yy = [yy Ymn];
		end
		
		yy = yy.*cos(order*phi);
		
		order = max(max(abs(yy)));
		rho = radius + amplitude*yy/order;
		
		r = radius.*sin(theta);    % convert to Cartesian coordinates
		x = r.*cos(phi);
		y = r.*sin(phi);
		z = radius.*cos(theta);
		
		subplot(5,5,1)
		surf(x,y,z, rho);
		title('$\ell=0, m=0$')
		
		shading interp
		
		axis equal off      % set axis equal and remove axis
		view(0,30)         % set viewpoint
		
		%%%%%%%%%%%%%%%%%%%%%%%%%%%%%%%%%%%%%%%%%%%%%%%%%%%%%%%%
		degree = 1;
		order = 0;
		amplitude = 0.5;
		radius = 5;
		
		Ymn = legendre(degree,cos(theta(:,1)));
		Ymn = Ymn(order+1,:)';
		yy = Ymn;
		
		for kk = 2: size(theta,1)
		    yy = [yy Ymn];
		end
		
		yy = yy.*cos(order*phi);
		
		order = max(max(abs(yy)));
		rho = radius + amplitude*yy/order;
		
		r = radius.*sin(theta);    % convert to Cartesian coordinates
		x = r.*cos(phi);
		y = r.*sin(phi);
		z = radius.*cos(theta);
		
		subplot(5,5,6)
		surf(x,y,z, rho);
		title('$\ell=1, m=0$')
		shading interp
		
		axis equal off      % set axis equal and remove axis
		view(0,30)         % set viewpoint
		
		%%%%%%%%%%%%%%%%%%%%%%%%%%%%%%%%%%%%%%%%%%%%%%%%%%%%%%%%
		
		degree = 1;
		order = 1;
		amplitude = 0.5;
		radius = 5;
		
		Ymn = legendre(degree,cos(theta(:,1)));
		Ymn = Ymn(order+1,:)';
		yy = Ymn;
		
		for kk = 2: size(theta,1)
		    yy = [yy Ymn];
		end
		
		yy = yy.*cos(order*phi);
		
		order = max(max(abs(yy)));
		rho = radius + amplitude*yy/order;
		
		r = radius.*sin(theta);    % convert to Cartesian coordinates
		x = r.*cos(phi);
		y = r.*sin(phi);
		z = radius.*cos(theta);
		
		subplot(5,5,7)
		surf(x,y,z, rho);
		title('$\ell=1, m=\pm 1$')
		shading interp
		
		axis equal off      % set axis equal and remove axis
		view(0,30)         % set viewpoint
		
		%%%%%%%%%%%%%%%%%%%%%%%%%%%%%%%%%%%%%%%%%%%%%%%%%%%%%%%%
		
		degree = 2;
		order = 0;
		amplitude = 0.5;
		radius = 5;
		
		Ymn = legendre(degree,cos(theta(:,1)));
		Ymn = Ymn(order+1,:)';
		yy = Ymn;
		
		for kk = 2: size(theta,1)
		    yy = [yy Ymn];
		end
		
		yy = yy.*cos(order*phi);
		
		order = max(max(abs(yy)));
		rho = radius + amplitude*yy/order;
		
		r = radius.*sin(theta);    % convert to Cartesian coordinates
		x = r.*cos(phi);
		y = r.*sin(phi);
		z = radius.*cos(theta);
		
		subplot(5,5,11)
		surf(x,y,z, rho);
		title('$\ell=2, m=0$')
		shading interp
		
		axis equal off      % set axis equal and remove axis
		view(0,30)         % set viewpoint
		
		%%%%%%%%%%%%%%%%%%%%%%%%%%%%%%%%%%%%%%%%%%%%%%%%%%%%%%%%
		
		degree = 2;
		order = 1;
		amplitude = 0.5;
		radius = 5;
		
		Ymn = legendre(degree,cos(theta(:,1)));
		Ymn = Ymn(order+1,:)';
		yy = Ymn;
		
		for kk = 2: size(theta,1)
		    yy = [yy Ymn];
		end
		
		yy = yy.*cos(order*phi);
		
		order = max(max(abs(yy)));
		rho = radius + amplitude*yy/order;
		
		r = radius.*sin(theta);    % convert to Cartesian coordinates
		x = r.*cos(phi);
		y = r.*sin(phi);
		z = radius.*cos(theta);
		
		subplot(5,5,12)
		surf(x,y,z, rho);
		title('$\ell=2, m=\pm 1$')
		shading interp
		
		axis equal off      % set axis equal and remove axis
		view(0,30)         % set viewpoint
		
		%%%%%%%%%%%%%%%%%%%%%%%%%%%%%%%%%%%%%%%%%%%%%%%%%%%%%%%%
		
		degree = 2;
		order = 2;
		amplitude = 0.5;
		radius = 5;
		
		Ymn = legendre(degree,cos(theta(:,1)));
		Ymn = Ymn(order+1,:)';
		yy = Ymn;
		
		for kk = 2: size(theta,1)
		    yy = [yy Ymn];
		end
		
		yy = yy.*cos(order*phi);
		
		order = max(max(abs(yy)));
		rho = radius + amplitude*yy/order;
		
		r = radius.*sin(theta);    % convert to Cartesian coordinates
		x = r.*cos(phi);
		y = r.*sin(phi);
		z = radius.*cos(theta);
		
		subplot(5,5,13)
		surf(x,y,z, rho);
		title('$\ell=2, m=\pm 2$')
		shading interp
		
		axis equal off      % set axis equal and remove axis
		view(0,30)         % set viewpoint
		
		%%%%%%%%%%%%%%%%%%%%%%%%%%%%%%%%%%%%%%%%%%%%%%%%%%%%%%%%
		
		degree = 3;
		order = 0;
		amplitude = 0.5;
		radius = 5;
		
		Ymn = legendre(degree,cos(theta(:,1)));
		Ymn = Ymn(order+1,:)';
		yy = Ymn;
		
		for kk = 2: size(theta,1)
		    yy = [yy Ymn];
		end
		
		yy = yy.*cos(order*phi);
		
		order = max(max(abs(yy)));
		rho = radius + amplitude*yy/order;
		
		r = radius.*sin(theta);    % convert to Cartesian coordinates
		x = r.*cos(phi);
		y = r.*sin(phi);
		z = radius.*cos(theta);
		
		subplot(5,5,16)
		surf(x,y,z, rho);
		title('$\ell=3, m=0$')
		shading interp
		
		axis equal off      % set axis equal and remove axis
		view(0,30)         % set viewpoint
		
		%%%%%%%%%%%%%%%%%%%%%%%%%%%%%%%%%%%%%%%%%%%%%%%%%%%%%%%%
		
		degree = 3;
		order = 1;
		amplitude = 0.5;
		radius = 5;
		
		Ymn = legendre(degree,cos(theta(:,1)));
		Ymn = Ymn(order+1,:)';
		yy = Ymn;
		
		for kk = 2: size(theta,1)
		    yy = [yy Ymn];
		end
		
		yy = yy.*cos(order*phi);
		
		order = max(max(abs(yy)));
		rho = radius + amplitude*yy/order;
		
		r = radius.*sin(theta);    % convert to Cartesian coordinates
		x = r.*cos(phi);
		y = r.*sin(phi);
		z = radius.*cos(theta);
		
		subplot(5,5,17)
		surf(x,y,z, rho);
		title('$\ell=3, m=\pm 1$')
		shading interp
		
		axis equal off      % set axis equal and remove axis
		view(0,30)         % set viewpoint
		
		%%%%%%%%%%%%%%%%%%%%%%%%%%%%%%%%%%%%%%%%%%%%%%%%%%%%%%%%
		
		degree = 3;
		order = 2;
		amplitude = 0.5;
		radius = 5;
		
		Ymn = legendre(degree,cos(theta(:,1)));
		Ymn = Ymn(order+1,:)';
		yy = Ymn;
		
		for kk = 2: size(theta,1)
		    yy = [yy Ymn];
		end
		
		yy = yy.*cos(order*phi);
		
		order = max(max(abs(yy)));
		rho = radius + amplitude*yy/order;
		
		r = radius.*sin(theta);    % convert to Cartesian coordinates
		x = r.*cos(phi);
		y = r.*sin(phi);
		z = radius.*cos(theta);
		
		subplot(5,5,18)
		surf(x,y,z, rho);
		title('$\ell=3, m=\pm 2$')
		shading interp
		
		axis equal off      % set axis equal and remove axis
		view(0,30)         % set viewpoint
		
		%%%%%%%%%%%%%%%%%%%%%%%%%%%%%%%%%%%%%%%%%%%%%%%%%%%%%%%%
		
		degree = 3;
		order = 3;
		amplitude = 0.5;
		radius = 5;
		
		Ymn = legendre(degree,cos(theta(:,1)));
		Ymn = Ymn(order+1,:)';
		yy = Ymn;
		
		for kk = 2: size(theta,1)
		    yy = [yy Ymn];
		end
		
		yy = yy.*cos(order*phi);
		
		order = max(max(abs(yy)));
		rho = radius + amplitude*yy/order;
		
		r = radius.*sin(theta);    % convert to Cartesian coordinates
		x = r.*cos(phi);
		y = r.*sin(phi);
		z = radius.*cos(theta);
		
		subplot(5,5,19)
		surf(x,y,z, rho);
		title('$\ell=3, m=\pm 3$')
		shading interp
		
		axis equal off      % set axis equal and remove axis
		view(0,30)         % set viewpoint
		
		%%%%%%%%%%%%%%%%%%%%%%%%%%%%%%%%%%%%%%%%%%%%%%%%%%%%%%%%
		
		degree = 4;
		order = 0;
		amplitude = 0.5;
		radius = 5;
		
		Ymn = legendre(degree,cos(theta(:,1)));
		Ymn = Ymn(order+1,:)';
		yy = Ymn;
		
		for kk = 2: size(theta,1)
		    yy = [yy Ymn];
		end
		
		yy = yy.*cos(order*phi);
		
		order = max(max(abs(yy)));
		rho = radius + amplitude*yy/order;
		
		r = radius.*sin(theta);    % convert to Cartesian coordinates
		x = r.*cos(phi);
		y = r.*sin(phi);
		z = radius.*cos(theta);
		
		subplot(5,5,21)
		surf(x,y,z, rho);
		title('$\ell=4, m=0$')
		shading interp
		
		axis equal off      % set axis equal and remove axis
		view(0,30)         % set viewpoint
		
		%%%%%%%%%%%%%%%%%%%%%%%%%%%%%%%%%%%%%%%%%%%%%%%%%%%%%%%%
		
		degree = 4;
		order = 1;
		amplitude = 0.5;
		radius = 5;
		
		Ymn = legendre(degree,cos(theta(:,1)));
		Ymn = Ymn(order+1,:)';
		yy = Ymn;
		
		for kk = 2: size(theta,1)
		    yy = [yy Ymn];
		end
		
		yy = yy.*cos(order*phi);
		
		order = max(max(abs(yy)));
		rho = radius + amplitude*yy/order;
		
		r = radius.*sin(theta);    % convert to Cartesian coordinates
		x = r.*cos(phi);
		y = r.*sin(phi);
		z = radius.*cos(theta);
		
		subplot(5,5,22)
		surf(x,y,z, rho);
		title('$\ell=4, m=\pm 1$')
		shading interp
		
		axis equal off      % set axis equal and remove axis
		view(0,30)         % set viewpoint
		
		%%%%%%%%%%%%%%%%%%%%%%%%%%%%%%%%%%%%%%%%%%%%%%%%%%%%%%%%
		
		degree = 4;
		order = 2;
		amplitude = 0.5;
		radius = 5;
		
		Ymn = legendre(degree,cos(theta(:,1)));
		Ymn = Ymn(order+1,:)';
		yy = Ymn;
		
		for kk = 2: size(theta,1)
		    yy = [yy Ymn];
		end
		
		yy = yy.*cos(order*phi);
		
		order = max(max(abs(yy)));
		rho = radius + amplitude*yy/order;
		
		r = radius.*sin(theta);    % convert to Cartesian coordinates
		x = r.*cos(phi);
		y = r.*sin(phi);
		z = radius.*cos(theta);
		
		subplot(5,5,23)
		surf(x,y,z, rho);
		title('$\ell=4, m=\pm 2$')
		shading interp
		
		axis equal off      % set axis equal and remove axis
		view(0,30)         % set viewpoint
		
		%%%%%%%%%%%%%%%%%%%%%%%%%%%%%%%%%%%%%%%%%%%%%%%%%%%%%%%%
		
		degree = 4;
		order = 3;
		amplitude = 0.5;
		radius = 5;
		
		Ymn = legendre(degree,cos(theta(:,1)));
		Ymn = Ymn(order+1,:)';
		yy = Ymn;
		
		for kk = 2: size(theta,1)
		    yy = [yy Ymn];
		end
		
		yy = yy.*cos(order*phi);
		
		order = max(max(abs(yy)));
		rho = radius + amplitude*yy/order;
		
		r = radius.*sin(theta);    % convert to Cartesian coordinates
		x = r.*cos(phi);
		y = r.*sin(phi);
		z = radius.*cos(theta);
		
		subplot(5,5,24)
		surf(x,y,z, rho);
		title('$\ell=4, m=\pm 3$')
		shading interp
		
		axis equal off      % set axis equal and remove axis
		view(0,30)         % set viewpoint
		
		%%%%%%%%%%%%%%%%%%%%%%%%%%%%%%%%%%%%%%%%%%%%%%%%%%%%%%%%
		
		degree = 4;
		order = 4;
		amplitude = 0.5;
		radius = 5;
		
		Ymn = legendre(degree,cos(theta(:,1)));
		Ymn = Ymn(order+1,:)';
		yy = Ymn;
		
		for kk = 2: size(theta,1)
		    yy = [yy Ymn];
		end
		
		yy = yy.*cos(order*phi);
		
		order = max(max(abs(yy)));
		rho = radius + amplitude*yy/order;
		
		r = radius.*sin(theta);    % convert to Cartesian coordinates
		x = r.*cos(phi);
		y = r.*sin(phi);
		z = radius.*cos(theta);
		
		subplot(5,5,25)
		surf(x,y,z, rho);
		title('$\ell=4, m=\pm 4$')
		shading interp
		
		axis equal off      % set axis equal and remove axis
		view(0,30)         % set viewpoint
		
		%%%%%%%%%%%%%%%%%%%%%%%%%%%%%%%%%%%%%%%%%%%%%%%%%%%%%%%%
		
		map = makeColorMap([0.2 0.2 0.6],[1.0 0.99 0.72],[0.8 0.25 0.33],80);
		colormap(map);
		cd(Figures)
	\end{lstlisting}
	
	So finally, we have common eigen functions (because remember that the Legendre polynomials are orthogonal to each other) that will be:
	 
	\begin{tcolorbox}[title=Remark,colframe=black,arc=10pt]
	It is not necessary to make complicated calculations to calculate the normalizaton factor of the exponential, as in the context of an integration over all space, the three factors of $Y_{m_l,l}(\theta,\phi)$ are independent of each other. Thus the integral is the product of the integrals (\SeeChapter{see section of Differential and Integral Calculus page \pageref{integral calculus}}).
	\end{tcolorbox}
	Finally, we must find $N_{m_l,l}$ such that:
	
	and we will see (what we will prove just below) that:
	
	In summary, we write (we should rather write "we will write"...)
	
	where we have omitted the factor $(-1)^l$ since in any case this term in the module of this function this term multiplies himself and then gives $(-1)^{2l}=1$.

Let us check now the framed previous boxed relation (warning this is a bit long and it is advisable to read it several times):

	We consider the functions defined by:
	
	where:
	
	with:
	
	The aim will therefore be to prove that these functions are orthogonal first and then find the constants $N_{m_l,l}$ such that $||Y_{m_l,l}||$. In short we will have to roll up the sleeves... of our brain...

	First, let us prove for future needs that:
	
	\begin{dem}
	If and only if $l=m_l=0$ the equality is obvious. Let us suppose that $l\geq 1$ (thus the general case outside the obvious previous case) and given $P$ a real polynomial of degree $\leq l-1$.

	Let us put:
	
	Let us prove (functional dot product):
	
	in  $\mathcal{C}(x\in[-1,1],\mathbb{R})$.
	
	Indeed, let us recall that we made the change of variable:
	
	Integrating by parts, we get:
	
	let us notice that for any $0\leq j\leq m_l-1$, $\dfrac{\mathrm{d}^j}{\mathrm{d}x^j}Z(x)$ is equal to zero in $x=\pm 1$ that is to say:
	
	Therefore (by extension), the above relation simplifies to:
	
	After $m_l$ integration by parts equation, we get:
	
	If $\deg(P)<m_l$ then the above expression shows that trivially:
	
	If $\deg(P)\geq m_l$ then by putting:
	
	We get:
	
	let us notice once again that $\dfrac{\mathrm{d}^j}{\mathrm{d}x^j}h(x)$ vanishes in $x=\pm 1$ for any $j\leq m_l-1$, that is say:
	
	By integrating by parts $m_l$the previous expression, we find:
	
	but $h$ is an polynomial of degree $m_l+\text{deg}(P)$.
	
	Indeed, the first factor is of degree $2m$ and the $m_l$th derivative of $P(x)$" is of degree $P-m_l$, therefore:
	
	So $\dfrac{\mathrm{d}^{m_l}}{\mathrm{d}x^{m_l}}h(x)$ is a polynomial of degree $\deg(P)\leq l-1$ and knowing that $\dfrac{\mathrm{d}^l}{\mathrm{d}x^l}(1-x^2)^l$ is to a given constant equal to the $l$-th Legendre polynomial (\SeeChapter{see section Calculus page \pageref{legendre polynomials}}) we then have:
	
	So we have just proved that $\dfrac{\mathrm{d}m_l}{\mathrm{d}x^{m}_l}$ is orthogonal to any polynomial of degree $\leq l-1$.
	\begin{flushright}
		$\square$  Q.E.D.
	\end{flushright}
	\end{dem}
	$\dfrac{\mathrm{d}^{m_l}}{\mathrm{d}x^{m_l}}Z$ is a polynomial of degree $l$ (its just enough to check for some values) so therefore let us search if there is a constant $c^{te}\in\mathbb{R}$ such that:
	
	with for recall:
	
	We can determine the constant $c^{te}$ by comparing the dominant coefficients of the polynomials:
	
	The dominant coefficient of $\dfrac{\mathrm{d}^{m_l}}{\mathrm{d}x^{m_l}}Z$ is:
	
	and the dominant coefficient of $\dfrac{\mathrm{d}^l}{\mathrm{d}x^l}(1-x^2)^l$ is:
	
	Therefore:
	
	That is to say:
	
	So we would have for $l\geq m_l\geq 0$ (we integrate parts we integrate as many times as necessary to the left and right - necessarily - to achieve this result):
	
	Now let us establish a remarkable relation that should perhaps exist between $P_{-m_l,l}$ and $P_{m_l,l}$ (and which will be useful to us later). Let us assume for this $0\leq m_l\leq l$ and remember that at the base:
	
	So that brings us to write (nothing special):
	
	By the previous results ($(-1)^{m_l}=(-1)^{-m_l}$):
	
	this leads us to write:
	
	Therefore we get:
	
	
	First, let us prove that:
	
	where $P_l$ is the $n$-th Legendre polynomial (hence the origin of the name of "associated  Legendre polynomial" ...).
	\begin{dem}
	First, we have proved that the Legendre polynomials satisfy the following recurrence relation (\SeeChapter{see section Calculus page \pageref{legendre polynomials}}):
	
	for $n\geq 1$.
	Multiplying the above equation by $x^{n-1}$ and integrating, we get:
	
	But:
	
	Let us recall that the $P_{n+1}$ polynomials form an orthogonal basis of which the polynomials that generate it are of increasing degree from $0$ to $n$, so a lower order polynomial - expressed in a sub vector space - will always be perpendicular to the vectors (polynomials) generating the higher dimensions. So if we take the example of $\mathbb{R}^3$ generated by the basis $(\vec{e}_1,\vec{e}_2,\vec{e}_3)$, then a vector $\vec{v}$ expressed by the linear combination of $(\vec{e}_1,\vec{e}_2)$ will always be perpendicular to $\vec{e}_3$ and therefore a zero scalar product with it.

	And  therefore it follows:
	
	Let us put:
	
	The previous expression becomes (remember $P_0(x)=1$):
	
	Thus by induction:
	
	Furthermore as:
	
	We then for for the prior-previous relation the denominator which can obviously be rewritten:
	
	Then we have:
	
	So in the end we can simplify the denominator as follows:
	
	and:
	
	So we have well proved that (just in case ... you would not follow anymore the initial target ...) that:
	
	\begin{flushright}
		$\square$  Q.E.D.
	\end{flushright}
	\end{dem}
	Let us attack us finally to what interests us. That is to say, prove that:
	
	\begin{dem}
	If $m_l\neq j$ then:
	
	where:
	
	\begin{tcolorbox}[title=Remark,colframe=black,arc=10pt]
	Let us recall that the Jacobian in spherical coordinates is $r^2\sin(\theta)$ (\SeeChapter{see section of Differential and Integral Calculus page \pageref{jacobian spherical coordinates}}) and as the integrated function above is not dependent on $r$, we have take out the term $r^2\mathrm{d}r$ of this integral (by cons we will meet again the same term in the function $R(r)$ present in the Schrödinger equation).
	\end{tcolorbox}
	And with:
	
	If $l>k$ and $m_l=j$ then the dot product of:
	
	is simplified to:
	
	By doing the change of variable $x=\cos(\theta)$ we get:
	
	Let us suppose that $m_l\geq 0$:
	
	where $P_l(x)$ is the $n$-th Legendre polynomial. Thus the expression of the dot product becomes:
	
	If we put:
	
	then the relation becomes:
	
	Integrating by parts $m$ times the expression above we get:
	
	But $\dfrac{\mathrm{d}^{m_l}}{\mathrm{d}x^{m_l}}h(x)$ is a polynomial of degree $k$. Knowing that $l>k$, the latter integral is zero for the same reasons as those mentioned above. Therefore:
	
	If $m_l<$ then we have proved that:
	
	and therefore:
	
	as $-m_l\geq 0$.
	
	It only remains to us to treat the case $m_l=j,l=k$. Let us suppose again that $m_l\geq 0$. So as before we have:
	
	and:
	
	Let us put:
	
	The relation then becomes:
	
	By integrating $m$ times by parts, we find:
	
	$\dfrac{\mathrm{d} ^{m_l}}{\mathrm{d} x^{m_l}}h(x)$ is a polynomial of degree $l$ which dominant coefficient is equal to:
	
	$P_l$ being orthogonal to any polynomial of degree strictly less that $l$, the expression can be written:
	
	But, we have proved that:
	
	therefore:
	
	If $m_l\leq 0$ we know that we get the result.
	\begin{flushright}
		$\square$  Q.E.D.
	\end{flushright}
	\end{dem}
	Finally this result gives us also the normalization condition:
	
	And so finally:
	
	is indeed an orthonormal family. Either explicitly (we reintroduce the factor $(-1)^l$):
	
	Finally, after this highly mathematical interlude (but instructive for the methodology of approach), we see (which is logical) that ot each value of $l$ correspond therefore $2l + 1$ eigenfunctions $Y_{m_l,l}(\theta,\phi)$. We also say that the value $\hbar l(l+1)$ is $2l + 1$ times degenerated since:
	
	Here are some values of the function  $Y_{m_l,l}(\theta,\phi)$ that generates what we commonly name "\NewTerm{spherical harmonics}\index{spherical harmonics}":
	
	Let's see some plots of these beautiful spherical harmonics that can be obtained with Maple 4.00b by using the following command (this is the $6$th spherical harmonic function above):\\

	\texttt{>plot3d(Re(sqrt(15/(8*Pi))*(sin(theta)*cos(theta)*exp(I*phi)))\string^2,phi=0..2*Pi,\\theta=0..Pi, coords=spherical,scaling=constrained,numpoints=5000,axes=frame);}
	\begin{figure}[H]
		\centering
		\includegraphics{img/chemistry/orbit_rigid_rotator_hydrogen_y12_maple.jpg}	
		\caption{Plot of the spherical harmpnics $Y_{1,2}$}
	\end{figure}
	
	\begin{itemize}
		\item $Y_{0,0}$ (corresponding to $n=1$!) gives a sphere (constant value regardless $\theta,\phi$) which the probability density can be represented by the "\NewTerm{photographic card}\index{photographic card (chemistry)}" or "\NewTerm{density map}\index{density map (chemistry)}" (the density in a given state is represented by the density of light spots on a dark background):
		\begin{figure}[H]
			\centering
			\includegraphics{img/chemistry/density_map1s.jpg}	
			\caption{$1s$ density map}
		\end{figure}
		Representing the possible $1s$ orbits.

		\item $Y_{0,1},Y_{1,1},Y_{-1,1}$ give (for $n=2$  at least!):
		\begin{figure}[H]
			\centering
			\includegraphics{img/chemistry/harmonic_functions_2p.jpg}	
			\caption{$2p$ orbitals (spherical harmonics)}
		\end{figure}
		Which represents the possible $2p$ orbits, the probability density function can be represented by its density and isodensity maps:
		\begin{figure}[H]
			\centering
			\includegraphics{img/chemistry/density_map2p.jpg}	
			\caption{$2p$ density map}
		\end{figure}

		\item $Y_{-2,2},Y_{-1,2},Y_{1,2},Y_{2,2},Y_{0,2}$ give (for $n=3$  at least!):
		\begin{figure}[H]
			\centering
			\includegraphics{img/chemistry/harmonic_functions_3d.jpg}	
			\caption{$3d$ orbitals (spherical harmonics)}
		\end{figure}
		Representing $5$ possible $3d$ centrosymmetric  orbits, which the probability density can be represented by (the last two maps represent $Y_{0,2}$) the following density maps:
		\begin{figure}[H]
			\centering
			\includegraphics{img/chemistry/density_map3d.jpg}	
			\caption{$3d$ density map}
		\end{figure}

		\item $Y_{-3,3},Y_{-2,3},Y_{-1,3},Y_{1,3},Y_{2,3},Y_{3,3},Y_{0,3}$ give (for $n=4$  at least!):
		\begin{figure}[H]
			\centering
			\includegraphics{img/chemistry/harmonic_functions_4f.jpg}	
			\caption{$3d$ orbitals (spherical harmonics)}
		\end{figure}
		Representing $7$ possible $3f$ anti-centrosymmetric  orbits, which the probability density can be represented by (in the order: $Y_{0,3},Y_{\pm 1,3},Y_{\pm 2,3},Y_{\pm 3,3}$) the following density maps:
		\begin{figure}[H]
			\centering
			\includegraphics{img/chemistry/density_map4f.jpg}	
			\caption{$4f$ density map}
		\end{figure}
	\end{itemize}
	The above results thus lead us to write:
	
	Substituting this in the Schrödinger equation:
	
	We get ($T_r=0$ in the rigid rotor but $\neq 0$ in the case of the hydrogen atom):
	
	As there is in this relation no operator which acts on $Y_{m_l,l}(\theta,\phi)$, we can simplify to obtain:
	
	which we see in the general case of the isolated atom that energy levels are no longer dependent of $m_l$ (due to the spherical symmetry of the potential). Then we say that the levels corresponding to the same values of $n$ and of $l$ are all merged whatever the values of $m_l$.
	
	In the case where $E_p$ derived from the $1 / r$ Coulomb potential , this radial equation leads us to a normalizing solution of $R (r)$ (different from zero then...) only for values of the energy corresponding to the following quantization law (well ... what a coincidence, we fall back on the expression proved in the old models of Corpuscular Quantum Physics!):
	
	where $R_H$ is the Rydberg constant as we determined in the section of Corpuscular Quantum Physics. Thus, in this case the energy levels corresponding to the same values of $n$ are all merged regardless of the value of $l$.

	For a given value of the principal quantum number $n$ (recall that we saw in the section of Corpuscular Quantum Physics that $l\leq n-1$), it is possible to verify that there are several solutions to the function $R(r)$ according to the value of the azimuthal quantum number $l$. Hence the identification of the solutions by the pair $(n, l)$. We note them $R_{n,l}(r)$. These are real functions of the variable $r$ (it just enough to check ... because if they work then they satisfy the Schrödinger equation, we will make an example a little further below):
	
	where (beware some books give this value in natural units!):
	
	is the equivalent of the Bohr radius (for the reduced mass) that we have determined in the section of Corpuscular Quantum Physics  with the difference that here we have a reduced mass instead of a single mass.
	
	However let us see if our Schrödinger equation is satisfied (taking $n=1,l=0$ for example):
	
	Which corresponds well to the expected result.

	Which graphically gives us the radial part $R_{n,l}(r)$:
	\begin{figure}[H]
		\centering
		\includegraphics{img/chemistry/radial_functions.jpg}	
		\caption{Plot of few radial functions $R_{n,l}(r)$}
	\end{figure}
	Let us study a little more in detail the radial function in the case of the hydrogen atom!:

	In the case of the atomic orbital $1s$ (!special case but we could do the same calculations as the following with all other orbital!) so we have for the hydrogen atom:
	
	So it is well a decreasing exponential function as shown in the graphic above. Before continuing let us recall that (\SeeChapter{see section Wave Quantum Physics page \pageref{first postulate wave quantum physics}}) :
	
	But, in spherical coordinates (see the beginning of this section):
	
	It then comes as we have seen earlier above:
	
	Then if follows that:
	
	With this result we can calculate the radial probability of finding the electron on each atomic orbital! So, it comes immediately with the previous result:
	
	So in the case of our $1s$ atomic orbital:
	
	It is now super interesting to calculate the point $r$ point where the probability of finding the electron is maximum on the $1s$ orbital!

	For this, we notice that $\dfrac{P_r}{\mathrm{d}r}$ reaches a maximum when we have trivially:
	
	Therefore:
	
	Therefore:
	
	Which is remarkable, because we find the result of the Bohr model (\SeeChapter{see section Corpuscular Physics page \pageref{bohr model}})!!!

	To summarize a little all this, the stationary states of the hydrogen atom are specified by three quantum numbers $n\in\mathbb{N}^{*},l\leq n-1,|m_l|\leq l$ and the Schrödinger wave function given finally by:
	
	We then the following traditional nomenclature in the case of the hydrogen atom:
	\begin{table}[H]
	\begin{center}
		\definecolor{gris}{gray}{0.85}
			\begin{tabular}{|c|c|c|c|c|}
				\hline
				\multicolumn{1}{c}{\cellcolor{black!30}\textbf{$n$}} & 
  \multicolumn{1}{c}{\cellcolor{black!30}\textbf{$l$}} & 
  \multicolumn{1}{c}{\cellcolor{black!30}\textbf{$m_l$}}  & 
  \multicolumn{1}{c}{\cellcolor{black!30}Function} & 
  \multicolumn{1}{c}{\cellcolor{black!30}Nomenclature}\\ \hline
				1 & 0 & 0 & $\Psi_{1,0,0}$ & $1s$\\ \hline
				   &  &  &  & \\ \hline
				2 & 0 & 0 & $\Psi_{2,0,0}$ & $2s$\\ \hline
				   & 1 & 1 & $\Psi_{2,1,1}$ & $2p_1$\\ \hline
				   &   & 0 & $\Psi_{2,1,0}$ & $2p_0$\\ \hline
				   &   & -1 & $\Psi_{2,1,-1}$ & $2p_{-1}$\\ \hline
				   &  &  &  & \\ \hline
				3 & 0 & 0 & $\Psi_{3,0,0}$ & $3s$\\ \hline
				   & 1 & 1 & $\Psi_{3,1,1}$ & $3p_1$\\ \hline
				   &   & -1 & $\Psi_{3,1,-1}$ & $3p_{-1}$\\ \hline
				   &  2 & 2 & $\Psi_{3,2,2}$ & $3d_2$\\ \hline
				   &     & 1 & $\Psi_{3,2,1}$ & $3d_1$\\ \hline
				   &     & 0 & $\Psi_{3,2,0}$ & $3d_0$\\ \hline
				   &     & -1 & $\Psi_{3,2,-1}$ & $3d_{-1}$\\ \hline
				   &     & -2 & $\Psi_{3,2,-2}$ & $3d_{-2}$\\ \hline

		\end{tabular}
	\end{center}
	\caption{Nomenclature of layers and sub-layers of the hydrogen atom}
	\end{table}
	We can include the spin of the electron in the description of the electronic structure of the atom. If we treat the spin as an additional degree of freedom then the lack of interaction term between conventional degrees of freedom (positions in real space) and the spin interaction named "\NewTerm{spin-orbit coupling}\index{spin-orbit coupling}" in the previous Hamiltonian implies that we can write the total wave function, spin included, in the form of a product:
	
	
	So taking into account everything seen so far we have the following density plots:
	\begin{figure}[H]
		\centering
		\includegraphics[scale=0.8]{img/chemistry/hydrogen_full_wave_function.jpg}	
	\end{figure}
	
	where we added the spin quantum number $m_s=\pm 1/2$ (\SeeChapter{see section Corpuscular Quantum Physics page \pageref{spin}}).

	The same remark we made in the section of Corpuscular Quantum Physics then applies: the levels remain $2n^2$ times degenerated.

	Let us do example. So we have:
	
	Thus for $1$ proton:
	
	Now let us apply the 5th postulate of wave quantum physics (see section of the same name page \pageref{fifth postulate of wave quantum physics}) for unlike earlier, not calculate the modal radius (most likely one), but the average radius! Then, as the operator position is the position itself (\SeeChapter{see section of Wave Quantum Physics page \pageref{observables and operators}}), the average value of the radius will be given by (do not forget that we are in spherical coordinates!):
	
	Using the Fubini theorem proved in the section of Differential and Integral Calculus we can write (well in this case it's even trivial that we have the right to write this... we should not even have to mention Fubini theorem normally...):
	
	For the last integral, we will use integration by parts:
	
	Thus finally:
	
	Or more explicitly:
	
	Therefore the average distance of the electron to the core is equal to $3/2$ times that of the Bohr radius so further that the most likely radius we have calculated earlier above (and which corresponds to Bohr radius)!
	
	\subsubsection{Potential Profile}
	Let us come back on an important point that is often used in physics book but never proved (as far as we know): the quantum potential profile of the hydrogen-like atom. Many books sometimes speak of "\NewTerm{harmonic model of the atomic bonding}\index{harmonic model of the atomic bonding}" but it seems that this is a priori rather a misnomer.

	So we saw much earlier in this section that:
	
	In view of the interpretation of the three terms of the Hamiltonian, it is customary to say that the two terms:
	
	constitute the "\NewTerm{effective potential energy}\index{effective potential energy}", thus explicitly:
	
	So the first term is (logically) repulsive while the second is attractive. A plot in Maple 4.00b of the effective potential energy gives with real experimental values for the radius with the real values of the constants:\\
	
	\texttt{>plot([-2.31E-28/r+6.11E-39*0*(0+1)/r\string^2,-2.31E-28/r+6.11E-39*1*(1+1)/r\string^2,\\-2.31E-28/r+6.11E-39*2*(2+1)/r\string^2,-2.31E-28/r+6.11E-39*3*(3+1)/r\string^2,\\-2.31E-28/r+6.11E-39*10*(10+1)/r\string^2],r=5E-11..10E-10,\\y=-0.5E-17..0.5E-17,thickness=2);}
	\begin{figure}[H]
		\begin{center}
		\includegraphics{img/chemistry/effective_potential_energy.jpg}
		\end{center}	
		\caption{Plot of the effective potential energy with Maple 4.00b for various $l$ and $Z$}
	\end{figure}
	where the legends were added afterwards with a text processor software. The reader will notice especially the case where $l=1$ that matches to the case of the figure indicated by the majority of graduate books of physics. Either with a zoom:\\

	\texttt{>plot(-2.31E-28/r+6.11E-39*1*(1+1)/r\string^2,r=5E-11..10E-10,thickness=2, color=green);}
	\begin{figure}[H]
		\begin{center}
		\includegraphics{img/chemistry/effective_potential_energy_l_equal_1.jpg}
		\end{center}	
		\caption{Plot of the famous effective potential energy with Maple 4.00b for  $l=1$ and $Z=1$}
	\end{figure}

	The first graph also tells us quite clearly that for $l= 0$ the electron has a negative potential energy that firmly holds it in the orbit of the proton. By cons already at $l= 1$ we guess that the point of stability of the electron is where the derivative is zero. Beyond the $l= 1$, in the case of a nucleus with a single proton, the electron is not naturally linked anymore since its potential energy tends to be positive. The reader can also have fun with Maple by making vary $Z$ and $l$. He will see that the effective potential energy is very sensitive to these parameters. For example, the plot below shows the effective potential energy with $l= 4$ and $Z = 1$ (thus unstable atom) and then with $l = 4$ and $Z = 6$ (which corresponds rather to an excited state):\\

	\texttt{>plot([-2.31E-28*1/r+6.11E-39*4*(4+1)/r\string^2,-2.31E-28*6/r+6.11E-39*4*(4+1)/r\string^2],\\r=5E-11..10E-10,thickness=2);}
	\begin{figure}[H]
		\begin{center}
		\includegraphics{img/chemistry/effective_potential_energy_l_equal_1_varous_z.jpg}
		\end{center}	
		\caption{Plot of the effective potential energy for $l=1$ and various $Z$}
	\end{figure}
	
	It is customary in practice to consider that:
	
	is to a given factor (electric charge factor) an "\NewTerm{effective electrical potential}\index{effective electrical potential}" or "\NewTerm{electric screened potential}\index{electric screened potential}". Indeed by defining the electric potential (\SeeChapter{see section Electrostatics page \pageref{electric potential}}), there are only an electri charge factor ratio between the electric potential energy and the electric potential. So we have:
	
	
	\begin{flushright}
	\begin{tabular}{l c}
	\circled{90} & \pbox{20cm}{\score{3}{5} \\ {\tiny 49 votes,  66.12\%}} 
	\end{tabular} 
	\end{flushright}

	%to make section start on odd page
	\newpage
	\thispagestyle{empty}
	\mbox{}
	\section{Molecular Chemistry}\label{molecular chemistry}
	\lettrine[lines=4]{\color{BrickRed}B}olecular chemistry is the central area that interconnects thanks to the study of molecules many promising advanced technologies of the early 21st century which are to name only the best known: molecular biology, molecular materials, molecular electronics, polymers, etc.
Orbital approximation

	Knowing it was found experimentally that a single molecule can have several very different functions, its theoretical study allows to use them better (sometimes better performance in terms of R\&D) in its areas of application. The reader will therefore understand that, as usual in this book, that we will focus here only on the theoretical aspect (mathematical) of molecular chemistry even if we limit ourselves only to theoretical developments made between the years 1910 and about 1935 (beyond the complexity of theories require too many pages to a general book as ours).
	
	We are in the beginning of the 21st century at the infancy of the discovery of what nature has done with plenty of time and chance (probabilities): that is to say complex molecules working as nanomachines capable locally (active site) to filter, oxidize, to make catalysis ... and many other manipulations (there is just to observe your own body!).
	
	A molecule is often treated in school classes with the Schrödinger equation (so no relativistic case and no consideration of the spins) in the usual form (\SeeChapter{see section Wave Quantum Physics page \pageref{schrödinger hamiltonian}}):
	
	or also in a stationary from (time-independent) where as a reminder $\Psi$ is a eigenfunctions and $E$ an eigenvalue of the application $H$.
	
	In reality, the wave functions are impossible to calculate normally with contemporary mathematical tools and the only thing we can do are numerical calculations (perturbation method). This is why some chemistry centers are transformed over time into data centers where the predictive character (and inexpensive) of quantum chemistry is becoming more and more important.
	
	It remains of course essential, as always, to understand how the theoretical models are built and their underlying assumptions.
	
	But we can still thanks to calculations predict the form of reasonable size of molecules, the energy of their internal connections, their energy capacity under stress deformation, the shape of the molecular orbitals (M.O.), energy state transitions (when parts of the molecule move therein), their reactivity vis-a-vis of a reaction medium...
	
	We commonly distinguish two cases of study of the molecular chemistry:
	\begin{enumerate}
		\item Quantum mechanics: all interactions between particles are taken into account under the assumption of some acceptable simplifications.
		\item Molecular mechanics: For large molecules, we are note concerned anymore over the electronic problem, but the interaction of certain parameters on which we want to focus.
	\end{enumerate}
	For example, hemoglobin (protein carrying oxygen carrying in the muscles) is a huge molecular structure which we will study only active site with the tools of quantum mechanics. The overall behavior of the molecule itself is treated with the molecular mechanics tools.
	
	It follows that excepts for hydrogen-like atoms, we can not analytically describe a molecule from a purely quantum point of view! All current quantum methods rely on one or more approximations. The wave functions are therefore approximated and the level of calculation is adjusted according to what we want to show and the precision that we seek (seeking to minimize the computation time for cost problems...). The good understanding of approximations permits to express simple models requiring only a minimum of calculations (often trivial).
	
	We propose here to show two common models (and the most simplest):
	
	\subsection{Orbital Approximations}
	A molecule is obviously an extremely complex problem: $N$ nuclei, $n$ electrons and everything is moving!
	\begin{figure}[H]
		\begin{center}
		\includegraphics{img/chemistry/vibrating_molecule.jpg}
		\end{center}	
		\caption{Example of molecule where a almost everything is moving}
	\end{figure}
	The Hamiltonian (\SeeChapter{see section Wave Quantum Physics page \pageref{hamiltonian operator wave quantum physics}}):
	
	is then a nightmare but in the intuitive form (the subscript $G$ of the Hamiltonian means "General") below:
	
	where:
	
	\begin{enumerate}
		\item $\displaystyle-\sum_{k=1}^{N}\frac{\hbar^2}{2M_k}\vec{\nabla}_k^2$ is the kinetic energy of the $k$ nuclei of mass $M_k$ in the molecule.

		\item $\displaystyle-\sum_{i=1}^{n}\frac{\hbar^2}{2m_e}\vec{\nabla}_i^2$  is the kinetic energy of the $n$ electrons n mass $m_e$.

		\item $\displaystyle-\sum_{k=1}^{N}\sum_{i=1}^{n}\frac{Ze^2}{4\pi\varepsilon_0 r_{ik}}$ is the potential energy due to the attraction electron(-)/nucleus(+).

		\item $\displaystyle\mathop{\sum_{i=1}}_{j>1}^{n-1}\frac{e^2}{4\pi\varepsilon_0 r_{ik}}$ is the potential energy of the repulsion electron(-)/electron(-).

		\item $\displaystyle\mathop{\sum_{k=1}}_{i>k}^{N-1}\frac{Z_kZ_ie^2}{4\pi\varepsilon_0 r_{ik}}$ is the potential energy of repulsion nucleus(+)/nucleus(+).
	\end{enumerate}

	Often we find these terms in the following form of the Schrödinger equation in the literature:
	
	A first approximation we might try is to decouple the movement of the nuclei of the electrons. Indeed, as the nucleus is much more massive (about $2,000$ times) than the cloud of electrons, the center of mass is assimilated to the nucleus of the atom and all the motion to the entire electron cloud. This approximate approach is well known under the name "\NewTerm{Born-Oppenheimer approximation}\index{Born-Oppenheimer approximation}":
	
	which then allows us to study the molecular orbitals. But unfortunately this approximation is not sufficient because of the repulsion interelectronic term (the double sum) that prevents using the separation of variables technique as we did in the section of Quantum Chemistry with the hydrogenoid-atom.
	
	Moreover, this latter equation is also written as the first line of the couple of equation below (Schrödinger equation of electrons and nuclei):
	\begin{subequations}
		\begin{align}
		&\underbrace{(T_e+V_{ee}+V_{en})}_{H_{\text{el}}}\Psi_{\text{el}}=E\Psi_{\text{el}}\\
		&\underbrace{(T_n+V_{nn})}_{H_{\text{nuclei}}}\Psi_n\Psi_{\text{el}}=E\Psi_n\Psi_{\text{el}}
		\end{align}
	\end{subequations}
	This system of equations is what some name the "\NewTerm{adiabatic approximation}\index{adiabatic approximation}" (???).
	
	The idea that then comes to mind will be using the following property:
	
	Given two operators $A$ and $B$, $f (u)$ and $g(v)$ their respective eigenfunctions associated with eigenvalues $a$ and $b$. Then $f (u) g (v)$ is an eigenfunction of the operator $A + B$ with associated eigenvalue $a + b$.

	Which is written:
	
			
	\begin{dem}
		We have:
		
		\begin{flushright}
			$\square$  Q.E.D.
		\end{flushright}
	\end{dem}
	And that's what we will use to break the $n$-electronic Hamiltonian $H_{\text{el}}$ into a sum of independent-electron Hamiltonian knowing of the above that if we find the eigenfunction for each (which is relatively easier) if will bu sufficient to simply multiply them to get the overall eigenfunction.

	Thus, we write:
	
		and therefore we have to find for each $i$:
	
	To then have:
	
	with therefore:
	
	This approach by one-electron Hamiltonian approach will lead us to replace:
	
	by the sum of Hamiltonian for an electron named "\NewTerm{effective Hamiltonian}\index{effective Hamiltonian}":
	
	This approximation method is sometimes named in theoretical chemistry "\NewTerm{independent electron approximation}\index{independent electron approximation}" or "\NewTerm{orbital approximation}\index{orbital approximation}". It consists therefore to include the electron-electron interactions and to write that each electron move in an average potential resulting from the presence of all other electrons.
	
	The "\NewTerm{Slater method}\index{Slater method}" consists by definition to write the latter relation in the form:
	
	where $\sigma$ is named the "\NewTerm{screen constant}\index{screen constant}".
	
	The Slater method basically means replacing the purely electronic terms by a constant. It can be regarded as a parametric method since the constants were determined purely experimentally.
	
	The principle of empirical calculation of the screening constant is relatively simple: In a poly-electronic atom, the core electrons are on much contracted orbits  while the valence electrons that will be responsible for the chemical properties of the atom in question are on orbits much more "relaxed".
	
	The attraction of the nucleus on the latter electrons is much lower than that exerted on the core electrons and these electrons only receive a portion of the atomic charge.
	
	Slater then proposed that the effective charge, which is usually denoted by $Z^*$ could be calculated by taking into account the screening constant. This constant represents then the average effect of the other electrons on the considered electron  of the effective Hamiltonian $i$:
	
	For a peripheral electron, we will need to consider its screen constant is due to all electrons placed on orbits equal or below its own. The tradition (or rather the "trick") is that the calculation is done by combining atomic orbitals in several groups $1s/2s, 2p/3s, 3p/3d/4s, 4p/4d/4f/5s, 5p/$ etc.
	
	Then the calculation is simple because it is based on an array of predefined values and we simply have to add the screening contributions of all the electrons following the table below:
	\begin{table}[h!]\centering
		\begin{tabular}{ccccc}\hline
		& $n'<n-1$ & $n'=n-1$ & $n'=n$ & $n'>n$ \\\hline
		1$s$ & & & $0.30$ & $0$ \\
		n$s$, n$p$ & $1$ & $0.85$ & $0.35$ & $0$ \\
		n$d$, n$f$ & $1$ & $1$ & $0.35$ & $0$\\ \hline
		\end{tabular}
		\caption{Screening contributions of electrons}
	\end{table}
	This table deserves some explanation of course !:
	
	The index indicates the number of the group that contributes to the screening constant while $n$ is the number of the group of electron that we consider.
	
	\pagebreak
	\begin{tcolorbox}[colframe=black,colback=white,sharp corners]
	\textbf{{\Large \ding{45}}Example:}\\\\
	In the case of the Carbon of configuration $1s^2 2s^2 2p^2$, the nuclear charge is $Z=6$. One electron $1s$ is shielded by onlye the another $1s$ electron, the effective charge it sees is therefore:
	
	A $2s$ or $2p$ electron is shielded by the two $1s$ electrons and by the other $3$ electrons $2s$ and $2p$. The effective charge by which it is attracted is then:
	
	So we see that the effective charge experienced decreases rather quickly!
	\end{tcolorbox}
	
	\subsection{LCAO Method}
	A linear combination of atomic orbitals or LCAO is a quantum superposition of atomic orbitals and a technique for calculating molecular orbitals in quantum chemistry. In quantum mechanics, electron configurations of atoms are described as wave functions. In mathematical sense, these wave functions are the basis set of functions, the basis functions, which describe the electrons of a given atom. In chemical reactions, orbital wave functions are modified, i.e. the electron cloud shape is changed, according to the type of atoms participating in the chemical bond.
	
	So as already mention, this method, rather qualitative, considers that the molecular wave function is a "\NewTerm{Linear Combination of Atomic Orbitals LCAO}\index{linear combination of atomic orbitals}" unlike the previous method where we multiply the effective Hamiltonian.
	
	This method is important because it is the basis of much of the current vocabulary of chemists when the chemistry done is cutting edge one!
	
	Let us take the example of the dihydrogen molecule $H_2$. The idea is then following:
	
	If we have the function of the atomic orbital $1s_A$ of $H_A$ and respectively the function $1s_B$ of $H_B$, then we assume that the dicentric molecular orbital (linked to two atoms) thereof is given by:
	
	which defines a quantum system with two eigenstates.

	But as we well know, in reality, only the square of the wave function has a physical sense (probability of presence). Thus, if we assume that the wave function has no value in $\mathbb{C}$, we have for the single electron of interest ($1s$):
	
	where we assume that:
	\begin{itemize}
		\item $a^2\Psi_A^2$ represents the probability of presence to be near $A$.
		\item $b^2\Psi_B^2$ represents the probability of presence to be near $B$.
		\item $2ab\Psi_A\Psi_B$ represents the probability of presence of the electron that do the link $A-B$.
	\end{itemize}
	In the particular case of the symmetric diatomic molecule we have chosen as an example, the atoms $A$ and $B$ perform the same function and there is no reason that the electron is closer to $A$ than to $B$ or vice versa.

	Thus, the probability of finding the electron near $A$ is equal to the probability of finding it near $B$.
	
	Moreover, in this case the orbitals $\Psi_A$ and $\Psi_B$ are completely identical ($1s$ orbitals, both of the same atom) and there is therefore no need to distinguish them. So we have:
	
	We have two solutions for $\Psi_{AB}$ that are (these two solutions can be found in very different notations in the literature):
	
	and:
	 
	\begin{tcolorbox}[title=Remark,colframe=black,arc=10pt]
	Caution! We can not put for the last two relations that $\Psi_A=\Psi_B$. The latter equality occurs at any point only if the distance between the two nucleus is zero (which is unlikely) or, if they are spaced a distant of a certain value $D$ in the middle thereof.
	\end{tcolorbox}
	These two expressions are simultaneously solutions of the Schrödinger equation. So we get two molecular orbitals from the two atomic orbitals in the case of symmetrical diatomic molecule.
	
	The function:
	
	is named "\NewTerm{bonding function}\index{bonding function}" because it corresponds to a reinforcement of the probability of presence of the electron between atoms $A$ and $B$ which corresponds to the creation of the bond!
	\begin{figure}[H]
		\begin{center}
		\includegraphics{img/chemistry/bonding_link.jpg}
		\end{center}	
	\end{figure}
		
	Conversely, the function:
	
	is named "\NewTerm{anti-bonding function}\index{anti-bonding function}" because it corresponds to a reduction of the probability of presence of the electron between atoms $A$ and $B$ which corresponds to the destruction of the bond!
	\begin{figure}[H]
		\begin{center}
		\includegraphics{img/chemistry/bonding_unlink.jpg}
		\end{center}	
	\end{figure}
	
	Ultimately, by overlapping, the two atomic orbitals with the same energy give birth to two molecular orbitals of different energy, a stabilized binding and the other antibonding destabilized.

	We have obviously from what we see just above that, in more complex cases, the energy level of the bonding molecular orbital is smaller than the antibonding (we will prove this rigorously in details below).

	Thus, it takes more energy to ionize respectively the electron of the binding orbital $\sigma$ than to ionize the electron of the antibonding orbital $\sigma^{*}$. It is commonly accepted that the energy of the bond function is stronger than the antibonding one (but we will make the proof further below).

	Let us also indicate that in chemistry, a chemical bond wherein each of the bonded atoms is sharing an electron from one of its outer layers to form a pair of electrons linking two atoms is commonly known as "\NewTerm{covalent bond}\index{covalent bond}".
	
	The chemists then say the covalent bond involves the equitable sharing of only one pair of electrons, named "\NewTerm{bonding pair}\index{bonding pair}" (but in fact where only one electron is really shared). Each atom provides an electron, the electron pair is then delocalized between two atoms as we have shown.

	These are the reasons why we commonly say that the bond $\sigma$ is a covalent chemical bond between two atoms created by orbital axial overlap.

	Now let us in-deep this approach! The molecular orbitals are to be normalized as we know. Which means that:
	
		What gives, since the atomic orbitals are normalized for $\Psi_1$ and are real functions:
	
	Since $a$ (real number in our case) is imposed as a constant, it comes immediately:
	
	Therefore for $\Psi_{AB}^1$:
	
	Identically, we have for $\Psi_{AB}^2$:
	
	If we have $S_{12} 1$, it comes the following format that we find in many books:
	
	Let us make a small example using as orbital, the lowest atomic orbital (1$s$) of the hydrogen atom in the case of a dihydrogeneous bond $H_2$ for which we have proved at the end of that section of Quantum Chemistry of quantum chemistry that:
	
	Therefore it comes:
	
	with for recall:
	
	It comes then for the molecular binding orbital of level $s$:
	
	and for the antibonding of also the $s$ level:
	
	We then see immediately that $\sigma^*_s$ vanishes in the middle of the two protons because in this place $r_1=r_2$. The molecular antibonding orbital therefore has a nodal plane and the electrons are mainly located on the protons.
	
	By cons, for the molecular orbital $\sigma_s$ the density does not vanish. Then we understand easily that an electron of $\sigma_s$ ensures the stability of the molecule and is therefore responsible for the chemical bond.
	
	We therefore conclude that the electronic stabilization due to the two identical orbital interaction is proportional to their recovery. More the recovery is big, the more the stabilization is important.
	
	There is a more technical approach using Dirac notation (\SeeChapter{see section Wave Quantum Physics page \pageref{dirac formalism}}) and that has the advantage of allowing the determination of the eigenvalues of energy.

	First we write the general expression of the time independant Schrödinger equation with the Bra-Ket notation for one molecular orbital, superposition of two atomic orbitals:
	
	Either in explicit form:
	
	If we multiply by the bra $\langle \Psi_A|$  on the left and taking into account that $a$, $b$ and the specific eigenvalues of the energy are constants, we get the following equation:
	
	Similarly, we get the bra $\langle \Psi_B|$:
	
	Let us simplify the notations even more:
	
	By symmetry of the problem in the case of dihydrogen, we put:
	
	which are named "\NewTerm{resonance integrals}\index{resonance integrals}" because it is a term relating to the combination (resonance) of the both atomic orbital relative to the two atoms that made the molecular structure.

	We also have:
	
	which are named "\NewTerm{Coulomb integrals}\index{Coulomb integrals}" because they correspond according to the fifth postulate of Wave Quantum Physics (see section of the same name page \pageref{fifth postulate of wave quantum physics}) to the average value of the total energy of the electron.

	We have obviously:
	
	which are named "\NewTerm{recovery integrals}\index{recovery integrals}" because the two atomic orbitals of the same type of each atom overlap.
	
	And finally, we have always have by symmetry of our particular case:
	
	We can then write, since the recovery integrals are unitary:
	
	These two equations are named "\NewTerm{secular equations}\index{secular equations}". The trivial solution is a priori not physical because it would mean that the electron has a zero probability density at any point in the space corresponding at $a=b=0$.

	There is a nontrivial solution and unique solution if and only if the following determinant (\SeeChapter{see section Linear Algebra page \pageref{determinant}}), known in molecular chemistry under the name "\NewTerm{secular determinant}\index{secular determinant}", is equal to zero:
	
	As we have by symmetry in our particular case:
	
	Therefore it comes:
	
	Hence:
	
	This gives us two solutions ($+$):
	
	and minus ($-$):
	
	Therefore we have:	
	
	But to be able to calculate the energy levels in detail, we must still have the shape of the Hamiltonian... and that using the both electrons of the dihydrogen molecule is quite difficult... To simplify the study, we reduce ourselves to the case of the cation (positive ion) $H_2^{+}$ consisting of two protons and one electron:
	\begin{figure}[H]
		\begin{center}
		\includegraphics{img/chemistry/dihydrogen_cation.jpg}
		\end{center}	
		\caption{Simplified study of the dihydrogen cation $H_2^+$}
	\end{figure}
	We then have base on the relation we ahve obtained at the beginning of this section:
	
	The following relation:
	
	where the first two terms in the brackets are for recall associated with the potential energy of the electron and the last to the potential repulsion energy of proton (the first term on the right of the equality is the kinetic energy of the electron).

	Now let us try to sort the energy of these two molecular orbitals. For this, we write:
	
	Let us recall that for a system to be stable, the energies  $E_n$ must be negatives, this corresponding to the stable states (we need a supply of energy to take them out) and request from us because of the shape of $E_2$:
	
	Knowing this it comes:
	
	Therefore, we see that the notations are not consistent with the use in quantum physics because normally the index $1$ is reserved to the lowest energy. So we will write in the future:
	
	with the associated eigenfunctions  $\Psi_1$ and $\Psi_2$ and therefore:
	
	We can also noticed an important thing! This is that if we consider the atoms in isolated, the interaction terms cancel and we have:
	
	Therefore we have the qualitative difference between a single atom and a simple diatomic (ionized) system:
	
	This means that the energy of the lowest level of a diatomic ionized molecule is less than the energy of a single atom which is near $\alpha$. This observation confirms that the system is stabilized in energy compared to two isolated atoms, which seems consistent with the experimental determination of the existence of such molecules.
	
	The traditional is that chemists represent the energy differences in the following form for our particular case:
	\begin{figure}[H]
		\begin{center}
		\includegraphics{img/chemistry/dihydrogen_cation_energy_levels.jpg}
		\end{center}	
		\caption{Energy levels of the dihydrogen cation $H_2^+$}
	\end{figure}
	We therefore conclude - by generalizing a little bit... - that when two atoms (each contributing with an electron) combine, their atomic orbitals will combine to generate two molecular orbitals, one of energy level $\Psi_1$ and the second of higher energy level $\Psi_2$ than that of the isolated atoms. Thus, the split up that will make leave one of the electron with one of atoms will be exothermic in comparison to the single atoms.
	
	Up to now we have discussed the electronic states of rigid molecules, where the nuclei are clamped to a fixed position. In this section we will improve our model of molecules and include the rotation and vibration of diatomic molecules.

	\pagebreak
	\subsection{Molecular Rotational Energy Levels}
	As we have seen in the section of Quantum chemistry, for analytical reasons we consider molecules als rigid rotators.

	The rigid rotators are commonly classified into four types:
	\begin{itemize}
		\item Spherical rotors: have equal moments of inertia (e.g., $\mathrm{CH}_4$).
		\begin{figure}[H]
			\centering
			\includegraphics{img/chemistry/molecule_ch4.jpg}
		\end{figure}
		
		\item Symmetric rotors: have two equal moments of inertial (e.g., $\mathrm{NH}_3$).
		\begin{figure}[H]
			\centering
			\includegraphics{img/chemistry/molecule_nh3.jpg}
		\end{figure}
		
		\item Linear rotors: have one moment of inertia equal to zero (e.g., $\mathrm{CO_2}$, $\mathrm{HCl}$).
		\begin{figure}[H]
			\centering
			\includegraphics{img/chemistry/molecule_co2.jpg}
		\end{figure}
		\begin{figure}[H]
			\centering
			\includegraphics{img/chemistry/molecule_hcl.jpg}
		\end{figure}

		\item Asymmetric rotors: have three different moments of inertia (e.g., $\mathrm{H}_2\mathrm{O}$).
		\begin{figure}[H]
			\centering
			\includegraphics{img/chemistry/molecule_h2o.jpg}
		\end{figure}
	\end{itemize}
	Let us now recall that have proved in the section of Quantum Chemistry that for the rigid rotator the part of the Hamiltonian dedicated to the rotation of energy is:
	
	Where $L^2$ was is an operator but from which we know from our study ow Wave Quantum Physics that the eigenvalues are:
	
	and where $r$ is the distance between the two corpuscules (nucleus and electron in the context of our study of the hydrogenous atom in the section of Quantum Chemistry) and
	
	In the context of diatomic molecules $A$ and $B$ it is more common to write $r_{AB}$ and:
	

	In the section of Wave Quantum Physics we have seen that we must consider the spin we have have to write the more general form:
	
	Therefore:
	
	In the old style spectroscopic literature, the rotational term values $F(J) = E(J)/hc$ are used instead of the energies....The previous relation is then written:
	
	with the "\NewTerm{rotational constant}\index{rotational constant}":
	
	
	We also know from the section of Classical Mechanics that:
	
	Therefore:
	
	That simplifies to:
	
	Therefore:
	
	
	The energy separation between the rotational levels $J$ and $J+1$ is given obviously by:
	
	and increase linearly with $J$.
	
	Let us now calculate the moment of inertia, that we will denoted $I$ to avoid the confusion with the orbital kinetic momentum $J$ used above, of a diatomic molecule. Let us imagine the diatomic molecule as a system of two tiny spheres at either end of a thin weightless rod.
	\begin{figure}[H]
		\centering
		\includegraphics{img/chemistry/diatomic_molecule_moment_inertia.jpg}	
		\caption{Construction for the study of inertia momentum of a diatomic molecule}
	\end{figure}
	Let $C$ be the center of mass of the molecule. Let $r_1$ and $r_2$ be the distances of the two atoms of respective masses $m_1$, $m_2$ from the center of mass $C$ of the molecule:
	We see that:
	
	and we have (\SeeChapter{see section Classical Mechanics page \pageref{center of mass}}):
	
	Therefore:
	
	Hence:
	
	After rearranging we get:
	
	or:
	
	Similarly:
	
	Let $I$ be the moment of inertia of the diatomic molecule about an axis passing through the center of mass of the molecule and perpendicular to bond length.

	Then we have seen in the section of Classical Mechanics that:
	
	or:
	
	thus:
	
	after simplification:
	
	Hence:
	
	So finally for a diatomic molecule (or any pair of object turning around a common center) we get the following moment of inertia\index{moment of inertia of a diatomic molecule}:
	
	Hence the fact that we often found in the literature the previous main relations under the form:
	
	and:
	
	Therefore, as: 
	
	the frequencies at which transitions can occur are given by :
	
	Notice that for $J_z=0$ we have a non-null zero point energy and frequency:
	
	
	\begin{tcolorbox}[colframe=black,colback=white,sharp corners]
	\textbf{{\Large \ding{45}}Example:}\\\\
	The molecule $\mathrm{NaH}$ is found to undergo a rotational transition from  $J=0$ to $J=1$ when it absorbs a photon of frequency $2.94 \times 10^{11}$ [Hz]. We want to know the equilibrium bond length of the molecule.\\

	For this purpose we use $J_z=0$ in the formula for the transition frequency 
	
	Solving for $r_0$ gives:
	
	The reduced mass is given by:
	
	which is in atomic mass units or relative units. In order to convert to kilograms, we need the conversion factor $1\;[\text{au}]= 1.66\cdot 10^{-27}$ [kg]. Multiplying this by $0.9655$ gives a reduced mass of $1.603\cdot 10^{-27}$ [kg]. Substituting in for $r_0$ gives:
	
	\end{tcolorbox}

	\pagebreak
	\subsection{Molecular Vibrational Energy Levels}\label{molecular vibrations}
	Let us consider the simple case of a vibrating diatomic molecule, where restoring force is proportional to displacement such that (\SeeChapter{see section Mechanics page \pageref{spring tension}}):
	
	The potential energy is a we proved it in the previously mentioned section, but with the notation of Quantum Physics:
	
	Now remember that we have proved in the section of Wave Quantique Physique page \pageref{schrodinger wave equation}, that the Schrödinger equation was given by:
	
	After rearrangement:
	
	And using the conventional notations in chemistry and quantum physics:
	
	As we consider a linear vibration mode, the know that we can use the reduced mass to analyze the system (\SeeChapter{see section Classical Mechanics page \pageref{center of mass}}). Therefore we have:
	
	Hence:
	
	And as we have prove it in the section of Wave Quantum Physique we have:
	
	with for recall $n\in \mathbb{N}$.
	
	So we can combine the previous results to get the "\NewTerm{vibrational-rotational energies level}\index{vibrational-rotational energies level}":
	
	
	\begin{flushright}
	\begin{tabular}{l c}
	\circled{90} & \pbox{20cm}{\score{3}{5} \\ {\tiny 23 votes,  64.35\%}} 
	\end{tabular} 
	\end{flushright}

	%to make section start on odd page
	\newpage
	\thispagestyle{empty}
	\mbox{}
	\section{Analytical Chemistry}\label{analytical chemistry}
	\lettrine[lines=4]{\color{BrickRed}C}hemistry is a very complex $n$-body science that mathematics can not explained without the input of numerical computer simulations or approximations regarding the use of quantum theory (\SeeChapter{see Atomistic section}). Until these tools are powerful enough and accessible to everyone, chemistry remains a primarily experimental science based on the observation of different properties of matter and we would like here give some very important definitions (which we find also elsewhere in other fields as chemistry).
	
	Analytical chemistry is concerned with the chemical characterization of matter and the answer to two important questions: what is it (qualitative analysis) and how much is it (quantitative analysis). Chemicals make up everything we use or consume, and knowledge of the chemical composition of many substances is important in our daily lives. Analytical chemistry plays an important role in nearly all aspects of chemistry, for example, agricultural, clinical, environmental, forensic, manufacturing, metallurgical, and pharmaceutical chemistry. The nitrogen content of a fertilizer determines its value. Foods must be analyzed for contaminants (e.g., pesticide residues) and for essential nutrients (e.g., vitamin content). The air we breathe must be analyzed for toxic gases (e.g., carbon monoxide). Blood glucose must be monitored in diabetics (and, in fact, most diseases are diagnosed by chemical analysis). The presence of trace elements from gun powder on a perpetrator's hand will prove a gun was fired by that hand. The quality of manufactured products often depends on proper chemical proportions, and measurement of the constituents is a necessary part of quality assurance. The carbon content of steel will influence its quality. The purity of drugs will influence their efficacy.

	In this section, we will focus only on the mathematical tools and techniques for performing these different types of analyses.

	\textbf{Definitions (\#\mydef):}	
	\begin{enumerate}
		\item[D1.] A "\NewTerm{subjective property}\index{subjective property}" is a property based on personal / individual printing, for example: beauty, sympathy, color, utility, etc.
		
		\item[D2.] An "\NewTerm{objective property}\index{objective property}" is an experienced property (which can not be contradicted), for example: mass, volume, shape, etc.
		
		\item[D3.] A "\NewTerm{qualitative property}\index{qualitative property}" is a descriptive property given using words. For example: oval, magnetic, conductive, etc.
		
		\item[D4.] A "\NewTerm{quantitative property}\index{quantitative property}" is a property that can be measured. For example: mass, volume, density, etc.
		
		\item[D5.] A "\NewTerm{characteristic property}\index{characteristic property}" is an exclusive property that identifies a pure substance. It does not change even if it is physically transformed material, for example: its density, its boiling point, its melting point, etc.
		
		\item[D6.] A "\NewTerm{characteristic property}\index{characteristic property}" is an exclusive property that identifies a pure substance. It does not change even if it is physically transformed material, for example: its density, its boiling point, its melting point, etc.
		
		\begin{tcolorbox}[title=Remark,colframe=black,arc=10pt]
	We know about $2,000,000$ different pure substances in the early 21st century (that is to say ... there is work behind it).
		\end{tcolorbox}
		
		\item[D7.] We name "\NewTerm{compound bodies}\index{compound bodies}", the bodies, that subjected to chemical processes, restore their components in the form of pure substances.
		
		\item[D8.] If we make the separation of mixtures and the decomposition of compositions, we finally get the bodies that are non-decomposable by conventional chemical methods; we name them "\NewTerm{elements}\index{elements}" or "\NewTerm{simple bodies}\index{simple bodies}".
		
		\item[D9.] The smallest part of a chemical combination yet having all of the properties thereof is the "\NewTerm{molecule}\index{molecule}" of this combination. The smallest part of an element or simple body is the "\NewTerm{atom}\index{atom}" of that element.
	\end{enumerate}
	Although there are just over $100$ elements, tens of millions of chemical compounds result from different combinations of these elements. Each compound has a specific composition and possesses definite chemical and physical properties by which we can distinguish it from all other compounds. And, of course, there are innumerable ways to combine elements and compounds to form different mixtures. A summary of how to distinguish between the various major classifications of matter is shown in the figure below:
	\begin{figure}[H]
		\centering
		\includegraphics[scale=0.42]{img/chemistry/chemical_mixtures_and_substances.jpg}
		\caption[Homogeneous mixture, heterogeneous mixture, compound or element]{Depending on its properties, a given substance can be classified as a homogeneous mixture, a heterogeneous mixture, a compound, or an element (source: OpenStax)}
	\end{figure}

	Let us also give some reminders of what we saw at the very start of the Mechanics chapter:
	\begin{enumerate}
		\item A mixture is named "\NewTerm{heterogeneous}\index{heterogeneous}" in chemistry if the components are immediately discernible to the naked eye or through the microscope.
		
		\item A mixture is said to be "\NewTerm{homogeneous}\index{homogeneous}" in chemistry if the components are not discernible to the naked eye or through the microscope.
		
		\item A system or body is said to be "\NewTerm{isotropic}\index{isotropic}" if it has identical values of a property in all directions otherwise it is said "\NewTerm{anisotropic}\index{anisotropic}".
	\end{enumerate}
	
	\subsection{Analytical chemistry process}
	The general analytical process is shown in the figure below. The analytical chemist should be involved in every step. The chemical analyst is like the physicist, mathematician and engineer a problem solver, a critical part of the team deciding what, why, and how. The unit operations of analytical chemistry that are common to most types of analyses are considered in more detail below that should in part belongs to process to other science jobs.
	\begin{figure}[H]
		\centering
		\includegraphics[scale=0.91]{img/chemistry/analytical_chemistry_process.jpg}
		\caption[Steps in Analytical Chemistry]{Steps in Analytical Chemistry (source: \cite{christian2013analytical})}
	\end{figure}	

	\subsection{Simple Mixtures}
	Before going into more or less complicated equations, the simplest case of application of mathematics to chemistry by which we can start is the management of mixtures for analysis and control operations of simple chemical reactions with two mixtures.
	
	Let us consider two typical and particular examples as theoretical introduction:
	\begin{enumerate}
		\item Given a solution (yellow) of $10$ milliliters of a solution containing an acid concentration at $30\%$. How many milliliters of pure acid (blue) should we add to increase the concentration (green) to $50\%$?
		\begin{figure}[H]
			\begin{center}
			\includegraphics{img/chemistry/chemistry_simple_mixture.jpg}
			\end{center}	
			\caption{The joy of mixtures...}
		\end{figure}
		Since the unknown is the amount of pure acid to be added, we will denote it by $x$. Then we have:
		
		That gives:
		
		It comes the obviously:
		
		Therefore $4$ milliliters of acid should be added to the original solution.
		
		\item  A canister contains $8$ liters of gasoline and oil to run an aggregate. If $40\%$ of the initial mixture is of the essence, how much should we remove of the mixture (pink) to replace it with pure gasoline (green) so that the final mixture (light green) contains $60\%$ gasoline?
		\begin{figure}[H]
			\begin{center}
			\includegraphics{img/chemistry/chemistry_simple_mixture_gazoline.jpg}
			\end{center}	
			\caption{The joy of mixtures by for diyers and military ...}
		\end{figure}
		We denote the unknown $x$ that is the number of liters of the initial mixture to remove and replaced by the pure essence being of equal amount also $x$. Then we have:
		
		That gives:
		
		We have then obviously:
		
		So approximately $2.6$ liters should be removed from the original mixture and be replaced by approximately $2.6$ liters of pure essence.
	\end{enumerate}
	In short this is for all mixtures in this book until now. We can go much further and do much more complicated with more unknowns but we'll stop there for now.
	
	\subsection{Reactions}\label{chemical reactions}
	Since the main study in chemistry is to observe the results of pure substances mixtures and/or of compounds mixtures, it is first necessary to deal with  the basic rules governing these mixtures under normal conditions of pressure and temperature (N.C.P.T).
	
	We should first clarify that we are not going to study in this section what creates the connections between the elements, as this is the role of quantum and molecular chemistry (see previous sections). Furthermore, we insist on the fact that every theoretical element will be illustrated with a practical example which can be useful sometimes to better understand.
	
	But before, for prevention reason, let us introduce a symbol shown in the figure below and that we can see on containers of chemicals in a laboratory or workplace. Sometimes named a "\NewTerm{fire diamond}\index{fire diamond}" or "\NewTerm{hazard diamond}\index{hazard diamond}", this chemical hazard diamond provides valuable information that briefly summarizes the various dangers of which to be aware when working with a particular
substance:
	\begin{figure}[H]
		\centering
		\includegraphics[scale=0.55]{img/chemistry/fire_diamond.jpg}
		\caption[National Fire Protection Agency (NFPA) hazard diamond]{National Fire Protection Agency (NFPA) hazard diamond summarizes the major hazards of a chemical substance (source: OpenStax)}
	\end{figure}
	
	Let us now consider a closed chemical system (without mass transfer therefore!). We translate the change in the composition (if applicable and if have there is one) of the chemical system with a reaction equation of the form (the system does not always go both ways!):
	
	but most of time written as:
	
	named "\NewTerm{reaction equation}\index{reaction equation}" where the coefficients $v_i \in \mathbb{N}^*$ are named  "\NewTerm{stoichiometric coefficients}\index{stoichiometric coefficients}" in the sense that they indicate the "golden proportions", strictly named "\NewTerm{stoichiometric ratio}\index{stoichiometric ratio}" necessary such that under normal conditions the reaction can take place and where the $A_i$ are the reactants (pure or compounds) and the ${A'}_i$ the formed products.
	
	The sum of the coefficients of the reactants minus the sum of the coefficients of the products is the "\NewTerm{stoichiometric sum}\index{stoichiometric sum}\label{stoichiometric sum}". If this is zero, the equation is say to be a "\NewTerm{balanced chemical reaction}\index{balanced chemical reaction}". In this case the reaction can be written:
	
	where the convention is that stoichiometric coefficients are positive for reactants and negative for products. The stoichiometric sum is $\sum v_i$.
	\begin{figure}[H]
		\centering
		\includegraphics[scale=0.55]{img/chemistry/chemical_reactions.jpg}
		\caption{Some typical chemical reactions (source: ?)}
	\end{figure}
	Caution! In the writing of the above equation, we require that all the $A_i$ without exception react to the chemical reaction and that therefore all the $v_i$ are dependents.
	
	If the "golden proportions" are respected (such that the coefficients are well stoichiometric) and exist when writing of the reaction equation, then for any $\alpha \in \mathbb{N}$ we have:
	
	this proposal can be proven only if the stoichiometric coefficients on one side or the other of the reaction vary proportionally. Experience shows that in normal conditions of temperature and pressure (N.C.T.P.) this is the case!
	
	Therefore, the stoichiometry of the reaction requires that if it disappears $x_1$ moles of $A_1$, $x_2$ moles of $A_2$  respectively with a variation of material of the products $\mathrm{d}n_1,\mathrm{d}n_2,\ldots $, it will appear accordingly ${x'}_1$ moles of ${A'}_1$, ${x'}_2$ moles of ${A'}_2$, ... with respectively a variation of material of the products $\mathrm{d}{n'}_1,\mathrm{d}{n'}_2,\ldots $... by respecting the proportionalities of the stoichiometric coefficients such that we can write the "\NewTerm{material balance equation}\index{material balance equation}":
	
	where $\mathrm{d}\xi$ is named the "\NewTerm{elementary reaction progress}\index{elementary reaction progress}" (frequently we will take the absolute values of the ratios to not have to think about the sign of the variations).
	
	The division of the variations $\mathrm{d}n_1,\mathrm{d}n_2$ and $\mathrm{d}{n'}_1,\mathrm{d}{n'}_2$  by their stoichiometric coefficients is justified only for normlization reasons having for purpose to bring $\mathrm{d}\xi$ to a value between $0$ and $1$ (between $0\%$ and $100\%$...).
	
	These last equalities simply indicate that if one of the reactive products disappear in a given quantity, the other reactants have their quantity that decreased in relation to their stoichiometric coefficient so as to maintain the golden proportions of the reaction.
	
		The writing of the energy balance can be simplified by the introduction of algebraic stoichiometric coefficients $v_i$ such that: $v_i>0$ for a formed product, $v_i<0$ for a reactive product.

	Finally we can write:
	
	we also often find in the literature with the absolute value at the numerator!

	Therefore, with this algebraic convention, the reaction equation as it exists, can be written:
	
	which means that the algebraic sum of the total number of pure compounds of the reactants and products formed is always zero.

	It is clear that at the initial time of the reaction we choose for the progress the value $\xi=0$ (its maximum value being equal to unity), time at which the quantities of material are equal to $n_{i,0}$.

	The integration of the differential expression of material balance obviously gives:
	
	Therefore:
	
	relation that we found in chemical progress tables (see further below), without forgetting that $v_i>0$ for a formed product and, $v_i<0$ for a reactive product.

	This bring us to the question: What is the maximum value $\xi_{\max}$ of the progress of a reaction? 

	Well the answer to that is in fact quite simple: The maximum progress value of a reaction having the stoichiometric proportions and such that it occurs when the reactants will have all disappear and therefore it is necessarily given by:
	
	for what we name the "\NewTerm{limiting reactant}\index{limiting reactant}", that is to say, the reactant that disappears (has always the smallest value of molarity) first and stops the expected reaction (the other one bieng named the "\NewTerm{excess reactant}\index{excess reactant}")! If there is no limiting reactant, that is that at the end of the reaction all reactants have been transformed: then we say that all reactants were in stoichiometric proportion.
	\begin{figure}[H]
		\centering
		\includegraphics[scale=0.6]{img/chemistry/limiting_reactant.jpg}
		\caption[Limiting reactant illustration]{When $\mathrm{H}_2$ and $\mathrm{C}_{l2}$ are combined in nonstoichiometric amounts, one of these reactants will limit the amount of $\mathrm{HCl}$ that can be produced. This illustration shows a reaction in which hydrogen is present in excess and chlorine is the limiting reactant.}
	\end{figure}
	It may be helpful to define the "\NewTerm{percentage of completion}\index{percentage of completion (chemistry)}" $A_i$ given by the intensive quantity:
	
	
	which gives with a more formal notation:
	
	\begin{tcolorbox}[colframe=black,colback=white,sharp corners]
	\textbf{{\Large \ding{45}}Example:}\\\\
	Let us consider to illustrate these concepts the reaction (dinitrogen and hydrogen giving ammonia):
	
	where the Latin letters represent the pure substances (atoms) whose name does not matter to us in this book (notation proposed by Jöns Jacob Berzelius in 1813). The indices simply represent the number of combination of atoms to obtain a molecule. \\

	We then have in this reaction:
	
	The reader will have notice that we have well following our convention for the mass balance:
	
	If we consider that there is one mole of each compound body, it gives us for the stoichiometric proportions (to a given factor $x\in\mathbb{R}^{*}$ for all values):
	
	If at any a given time $t\neq t_0$, we get by measurement:
	
	What is the progress of that reaction?

	The answer is:
	
	or in other words, we are at $10\%$ of progress (logical!).\\
	\end{tcolorbox}
	
	\begin{tcolorbox}[colframe=black,colback=white,sharp corners]
	The conversion rate of $\mathrm{NH}_3$ is thereto:
	
	And what is the maximum progress value $\xi_{\max}$ of the limiting reactant?\\

	So in the context of the above example where we have $n_{1,0}=1\;[\text{mol}]$ for the $\mathrm{N}_2$ then:
	
	\end{tcolorbox}
	
	Chemists also often use what they name a "\NewTerm{reaction progress table}\index{reaction progress table}".
	
	Let us take our previous example to introduce this table. We have:
	\begin{table}[H]
		\begin{center}
		\definecolor{gris}{gray}{0.85}
		\begin{tabular}{|l|l|c|c|r|}
		\hline 
		{\cellcolor{black!30}} & {\cellcolor{black!30}Equation} & {\cellcolor{black!30}$ \boldsymbol{\mathrm{N}_2}$} & {\cellcolor{black!30}$\boldsymbol{+3\mathrm{H}_2}$} & {\cellcolor{black!30}$\boldsymbol{+2\mathrm{NH}_3}$}\\ 
		\hline 
		{\cellcolor{black!30}Initial State} & $n_{i,0}$ & $1$ & $3$ & $0$ \\  \hline
		{\cellcolor{black!30}Intermediate State} & $n_i=n_{i,0}+v_i\xi$ & $1-1\cdot\xi$ & $3-3\cdot \xi$ & $0+2\cdot\xi$ \\  \hline
		{\cellcolor{black!30}Final State} & $\xi_{\max}$ & $1-1\cdot \xi_{\max}$ & $3-3\cdot\xi_{\max}$ & $0+2\cdot\xi_{\max}$\\  \hline
		\end{tabular} 
		\end{center}
		\caption{Table progress of a chemical reaction}
	\end{table}
	Let us seek $\xi_{\max}$ from this table. The limiting reactant is either $\mathrm{N}_2$ or $3\mathrm{H}_2$.

	So for $\mathrm{N}_2$:
	
	and for $3H_2$:
	
	Each reactant having the same $\xi_{\max}$ progress, it is thus also the minimum $\xi_{\max}$. Consequently, according to the definition of limiting reactant, as the proportions are stoichiometric in the given example no reactant is limiting.

	\begin{flushright}
	\begin{tabular}{l c}
	\circled{10} & \pbox{20cm}{\score{3}{5} \\ {\tiny 25 votes,  55.20\%}} 
	\end{tabular} 
	\end{flushright}

	%to force start on odd page
	\newpage
	\thispagestyle{empty}
	\mbox{}
	\section{Thermochemistry}
	\lettrine[lines=4]{\color{BrickRed}T}hermochemistry is the branch that historically focuses on thermic phenomena and to equilibrium accompanying chemical reactions. It mainly has its foundations in the thermodynamics. More technically, thermochemistry is the study of the energy and heat associated with chemical reactions and/or physical transformations. A reaction may release or absorb energy, and a phase change may do the same, such as in melting and boiling. Thermochemistry focuses on these energy changes, particularly on the system's energy exchange with its surroundings. Thermochemistry is useful in predicting reactant and product quantities throughout the course of a given reaction. In combination with entropy determinations, it is also used to predict whether a reaction is spontaneous or non-spontaneous, favorable or unfavorable.
	
	We can only strongly recommend the readers to have read or to read the section on Thermodynamics in the Mechanics chapter because many concepts that have been seen there will be assumed to be known in this section.
	
	Moreover, it is strongly recommended to read this chapter in parallel to that of Analytical Chemistry (this can be a boring but you must do with...).
	
	\subsection{Chemical transformations}
	Chemical reactions, such as those that occur when you light a match, involve changes in energy as well as matter. Societies at all levels of development could not function without the energy released by chemical reactions. In 2012, about $85\%$ of US energy consumption came from the combustion of petroleum products, coal, wood, and garbage. We use this energy to produce electricity ($38\%$); to transport food, raw materials, manufactured goods, and people ($27\%$); for industrial production ($21\%$); and to heat and power our homes and businesses ($10\%$). While these combustion reactions help us meet our essential energy needs, they are also recognized by the majority of the scientific community as a major contributor to global climate change.

	Useful forms of energy are also available from a variety of chemical reactions other than combustion. For example, the energy produced by the batteries in a cell phone, car, or flashlight results from chemical reactions. This chapter introduces many of the basic ideas necessary to explore the relationships between chemical changes and energy, with a focus on thermal energy.

	Given the closed system of the following chemical reaction (\SeeChapter{see section Analytical Chemistry page \pageref{chemical reactions}}):
	
	We will consider for simplicity that the chemical reaction is complete and that the reactants are used in stoichiometric amounts (state 1: $\Sigma_1$) to give the products formed, also in stoichiometric quantities (state 2: $\Sigma_2$).
	
	If the transformation is done in (quasi-)steady volume steady, work on the surrounding atmosphere is zero because (\SeeChapter{see section Thermodynamics page \pageref{work of mechanical forces}}):
	
	The application of the first law of thermodynamics is reduced and allows then us to write:
	
	where $Q_v$ is within the thermal chemistry framework named "\NewTerm{heat of reaction at constant volume}\index{heat of reaction at constant volume}", of course exchanged between the system and the external environment (we do not write the delta $\Delta$ in front of $Q_V$ to indicate that it is a variation... by tradition...).
	
	Let us recall that:
	
	\begin{enumerate}
		\item If $Q_V>0$ the reaction is said to be "\NewTerm{endothermic}\index{endothermic}" (the system receives heat from the external environment).
		
		\item If $Q_V<0$ the reaction is said to be "\NewTerm{exothermic}\index{exothermic}" (the system gives heat to the external environment).
		
		\item If $Q_V=0$ the reaction is said to be "\NewTerm{athermic}\index{athermic}" (the system do not exchange any heat with the environment).
	\end{enumerate}
	\begin{tcolorbox}[title=Remark,colframe=black,arc=10pt]
	Let us also recall that a closed system is not an isolated system! For a review of different definitions, the reader is referred once again to the section of Thermodynamics.
	\end{tcolorbox}
	
	If the reaction is carried out at constant pressure (the most usual case in practice), that is to say isobaric, then we have:
	
	\begin{tcolorbox}[title=Remark,colframe=black,arc=10pt]
	The choice of integration indices are different to previously to differentiate the fact that a reaction a pressure or constant volume are not necessarily identical.
	\end{tcolorbox}
	
	The application of the first law of thermodynamics, between the two states, gives:
	
	where $Q_p$ is the amount of heat, named "\NewTerm{constant-pressure reaction heat}\index{constant-pressure reaction heat}", exchanged between the system and the external environment ($Q_P$ is a variation... even if the traditional unfortunate notation of thermodynamician does not put that in evidence...).
	
	Using the definition of enthalpy, we can write the last relation in the form:
	
	If we work with the molar volumes, those of condensed phases (therefore solid and liquid) is negligible compared to the gas molar volume, only the gas components have a very different enthalpy of their internal energy (see the example in the section of Thermodynamics) . We would therefore have under the ideal gas approximation (\SeeChapter{see section Thermodynamics page \pageref{enthalpy}}):
	
	In the context of the ideal gas, the prior-previous relation can be written:
	
	But, as (\SeeChapter{see section Continuum Mechanics page \pageref{virial theorem}}) $U_2$ and $U_3$ are both the same final states of a single complete reaction and that we know for a monatomic gas we have:
	
	therefore the internal energy $U_2$ and $U_3$ only depends on the number of components but ... they are equal since they are the same final state of the same reaction!
	
	Therefore we have:
	
	By putting $\Delta n=n_2-n_1$ (the difference between the number of moles of gas of formed products and those of reacting products), we can write for a chemical reaction:
	
	that gives the possibility to differentiates the energy involved between isobaric and isochoric reaction and look for the best choice in terms of industrial objectives. It is interesting to notice that if the $\Delta$ of moles is zero. Isobaric or isochoric heat variations are equal and there is no a priori reason to prefer one or the other transformations.
	
	Obviously in practice the problem is to know the values of the different variables of the latter relation. These values can be found on huge databases that chemists have access to... This relationship is only very rarely used in practice and in any case it is based on too simplifying and restrictive assumptions to be of real practical interest.
	
	\subsection{Molar Quantities}
	\textbf{Definitions (\#\mydef):}
	\begin{enumerate}
		\item[D1.] By convention, the "\NewTerm{mole}\index{mole}" is the quantity of substance of a system which contains as many chemical species as there are Carbon atoms in $12$ [g] of Carbon $12$ (\SeeChapter{see section Nuclear Physics page \pageref{atomic mass unit}}).
		
		The number of carbon atoms contained in $12$ [g] is equal to the Avogadro's \underline{number} given approximately by (notice that this number if by far bigger than the number of humans that are on earth!):
		
		This means verbatim and by construction that a mole of water, of iron, of electron, respectively always contains a number of atoms equal to the Avogadro's number.
		
		Most of time the mole is simply denoted $n$ and has its value in $\mathbb{R}^{+}$.
		
		Notice that with a mixed system it is a mathematical nonsense to do the sum of the molar masses of the constituents for the total molar mass. The molar mass is an intensive quantity!
		\begin{tcolorbox}[title=Remark,colframe=black,arc=10pt]
		\textbf{R1.} Hydrogen-1 was once used as a standard but given the inaccuracy that can occur because of its low mass, it was later disregarded. Once mass spectrometry was made available, physicists were using Carbon-12 for it's stability and abundance, and basically to stop everybody from fighting. Carbon-12 also more accurately defines a mass for hydrogen, and it is unbound in it's ground state and also is the most common and readily available isotope to have exactly the same number of protons and neutrons, 6 of each, and thus provides a perfect average when divided by the total number of protons and neutrons (electron is so small as to be considered negligible). \\
		
		\textbf{R2.} The "Avogadro project" aims to redefine Avogadro's constant (currently defined by the kilogram: the number of atoms in 12 g of Carbon-12) and reverse the relationship so that the kilogram is precisely specified by Avogadro's constant. This method required creating the most perfect sphere on Earth. It is made out of a single crystal of silicon 28 atoms. By carefully measuring the diameter, the volume can be precisely specified. Since the atom spacing of silicon is well known, the number of atoms in a sphere can be accurately calculated. This allows for a very precise determination of Avogadro's constant.
		\end{tcolorbox}	
		
		\item[D2.] The "\NewTerm{molar mass (MM)}\index{molar mass}\label{molar mass}" is the mass of one mole of atoms of the chemical elements involved. Therefore by definition the molar mass of $\mathrm{C}_{12}$ is equal to $12$ grams (yes historically we use the gram to express molar mass because for application purposes is is obviously more convenient...
		\begin{figure}[H]
			\centering
			\includegraphics{img/chemistry/moles.jpg}
			\caption[]{Each sample contains one mole of atoms. From left to right (top row): $65.4$ [g] zinc, $12.0$ [g] carbon, $24.3$ [g] magnesium, and $63.5$ [g] copper. From left to right (bottom row): $32.1$ [g] sulfur, $28.1$ [g] silicon, $207$ [g] lead, and $118.7$ [g] tin (source: Mark Ott)}
		\end{figure}
		\begin{tcolorbox}[title=Remark,colframe=black,arc=10pt]
		We find these atomic molar masses in the periodic classification. But above all it must be known that those that are indicated take into account the natural isotopes (which is normal since they are chemically indistinguishable excepted for the nuclear chemist or nuclear physicist). So the value indicated in the tables is calculated as the sum of the respective proportions of the molar masses of the different corresponding isotopes (the validity of this method of calculation is obviously relative...).
		\end{tcolorbox}	
		
		\item[D3.] The "\NewTerm{atomic molar mass}\index{atomic molar mass}" is the molar mass of a given element divided by the Avogadro number. Thus:
		
		Therefore the atomic (molar) mass is the mass of $1$ atom of a particular element and the molar mass is the mass of $1$ mole of an atom or molecule.
		
		We therefore have the following graph:
		\begin{figure}[H]
			\begin{center}
				\includegraphics{img/chemistry/mole_mass_avogadro.jpg}
			\end{center}	
		\end{figure}
		
		\item[D4.] The "\NewTerm{Molecular molar mass (MMM)}\index{molecular molar mass}" is equal to the sum of the atomic molars  masses of the chemical elements that constitutes it.
		
		It comes therefore immediately the following observation: the mass $m$ of a sample consisting of an amount of $n$ moles of identical chemical species of molar mass $M_m$ is given by the relation:
		
		Somewhat in a little bit more formal way and in a thermodynamic aspect, here is also is how we can define the molar mass:
		
		Let $X$ be an extensive quantity on a single-phase system (see the section of Thermodynamics for precisions about the vocabulary used) and given a volume element $\mathrm{d}V$ of this system around a common point $M$ and containing the amount of material $\mathrm{d}n$. We associate it the extensive quantity $\mathrm{d}X$ proportional to $\mathrm{d}n$ such that:
		
		so that $X_m$ is an intensive quantity (ratio of two extensive quantities according to what was seen in the section of Thermodynamics) which we will name by definition the "\NewTerm{associated molar size}\index{associated molar size}" to $X$.
		
		We conclude that:
		
		the integral applying on the whole monophasic system.
		
		In the case of a uniform phase, $X_m$ being constant at any point, we can simply write the latter as:
		
		\begin{tcolorbox}[title=Remark,colframe=black,arc=10pt]
		Basically the idea is to say that the mass of a single-phase chemical system is proportional to the molar mass of it to closely to a given integer factor representing the number of its constituents (or the number of moles to be more exact).
		\end{tcolorbox}	
		
		\item[D5.] When the system is heterogeneous, we use the concept of "\NewTerm{mole fraction}\index{mole fraction}", defined by:
		
		$x_i$ being the mole fraction of a species $A_i$ whose the quantit of material (the number of moles for example) is $n_i$ with $n=\sum_i n_i$ being the total quantity of matter of the studied phase.
		
		As a result, for all chemical species of the studied phase, $\sum_i x_i=1$ which means that if there are $n$ chemical species, it is enough to  know $n-1$ molar titles to know them all.
	
		If the studied phase is a gas and assuming a perfect gas according to Boyle's law (approximation of the Van der Waals equation proved in the section of Statistical Mechanics) we have:
		
		we therefore have the possibility in the case of gaseous phases to express the mole fraction as:
		
		\begin{tcolorbox}[title=Remark,colframe=black,arc=10pt]
		We can do obviously the same for the volume $V$.
		\end{tcolorbox}	
		
		\item[D6.] We define the "\NewTerm{mass content associated with the species $A_i$}\index{mass content associated a species}" by the ration:
		
		with $m=\sum_i m_i$ being the total mass of the studied phase. We also have of course $\sum_i w_i=1$.
		
		\item[D7.] We define the "\NewTerm{volumic molar concentration}\index{volumic molar concentration}" or "\NewTerm{molarit}\index{molarit}" the ratio (do not confuse the notation with the specific heat):
		
		\begin{tcolorbox}[title=Remark,colframe=black,arc=10pt]
		There are other composition variables used much less used than $x_i$ or $c_i$. We can cite the "\NewTerm{mass concentration density}\index{mass concentration density}" $m_i/V$, the "\NewTerm{molality}\index{molality}" (ratio of the amount of material of the species $A_i$ by the total mass of solvent), etc.
		\end{tcolorbox}
		
		\item[D8.] We say that a (perfect) gas is in the "\NewTerm{standard state}\index{standard state}" if its pressure is equal to the standard pressure:
		
		
		\item[D9.] We name "\NewTerm{standard molar quantity}\index{standard molar quantity}" of a constituent $X_m^\circ$ the value of the molar quantity of this same component taken in the standard state, that is to say under the pressure $P^\circ$.
		\begin{tcolorbox}[title=Remarks,colframe=black,arc=10pt]
		\textbf{R1.} Any standard molar quantity is obviously intensive: the pressure being set by the standard state , it depends only on the temperature.\\
		
		\textbf{R2.} Any standard quantity is denoted with the superscript "${}^\circ$". $V_m^\circ$ is then standard molar volume. For cons, the standard molar quantity is not always specified with the small index $m$, we must sometimes be careful with what is handled in the equations (as always anyway!).
		\end{tcolorbox}	
		In the case of the ideal gas, the molar volume is calculated using the ideal gas equation of state. Then we get:
		
		We see of course that the standard molar volume of an ideal gas depends on the temperature.
		
		If we do that calculation at the "\NewTerm{standard conditions of temperature and pressure}\index{standard conditions of temperature and pressure}" (abbreviated STP), that is to say at a temperature of $273.15$ [K] (i.e. $0$ [$^\circ$C]) and a pressure of $1$ [atm] (i.e. $101,325$ [kPa]), then we find a volume of $22.4$ [L$\cdot$mol$^{-1}$] which is a well known value by chemists.
	\end{enumerate}
	
	\begin{tcolorbox}[title=Remarks,colframe=black,arc=10pt]
	\textbf{R1.} In a wide range of temperatures and pressures, the molar volume of real gases is generally not very different from that of an ideal gas.\\
	
	\textbf{R2.} In the case of a condensed state, we do not have in general a state equation but we can measure the molar volume..
	\end{tcolorbox}
	We can define then by extension other standard quantities resulting from those we had defined in the section Thermodynamics:
	\begin{enumerate}
		\item The "\NewTerm{standard molar internal energy}\index{standard molar internal energy}" (intensive quantity as expressed by molar unit) and denoted by $U_m^\circ$.

		\item The "\NewTerm{standard molar enthalpy}\index{standard molar enthalpy}" (intensive quantity as expressed by molar unit) with:
		
		It is important that the reader notice that the enthalpy depends only on the temperature (and the internal energy).
		
		Enthalpies of combustion for many substances have been measured. A few of these are listed in the table below. Many readily available substances with large enthalpies of combustion are used as fuels, including hydrogen, carbon (as coal or charcoal), and hydrocarbons (compounds containing only hydrogen and carbon), such as methane, propane, and the major components of gasoline.
		\begin{table}[H]
			\centering
			\begin{tabular}{|l|c|c|}
			\hline
			\rowcolor[HTML]{C0C0C0} 
			\multicolumn{1}{|c|}{\cellcolor[HTML]{C0C0C0}\textbf{Substance}} & \multicolumn{1}{|c|}{\cellcolor[HTML]{C0C0C0}\textbf{Combustion Reaction}} & \textbf{\parbox{5.4cm}{Combustion Molar Enthalpy \\ $\Delta H_{c,m}^\circ$ in $[\text{kJ}\cdot\text{mole}^{-1}]$ at $25^\circ$}} \\ \hline
			carbon & $\mathrm{C}\text{(s)}+\mathrm{O}_2\text{(g)}\rightarrow \mathrm{CO}_2\text{(g)}$ & $-393.5$ \\ \hline
			hydrogen & $\mathrm{H}_2\text{(g)}+\dfrac{1}{2}\mathrm{O}_2\text{(g)}\rightarrow \mathrm{H}_2\mathrm{O}\text{(l)}$ & $-285.8$ \\ \hline
			magnesium & $\mathrm{Mg}\text{(s)}+\dfrac{1}{2}\mathrm{O}_2\text{(g)}\rightarrow \mathrm{MgO}\text{(s)}$ & $-601.6$ \\ \hline
			sulfur & $\mathrm{S}\text{(s)}+\mathrm{O}_2\text{(g)}\rightarrow \mathrm{SO}_2\text{(g)}$ & $-296.8$ \\ \hline
			carbon monoxide & $\mathrm{CO}\text{(g)}+\dfrac{1}{2}\mathrm{O}_2\text{(g)}\rightarrow \mathrm{CO}_2\text{(g)}$ & $-283.0$ \\ \hline
			methane & $\mathrm{CH}_4\text{(g)}+2\mathrm{O}_2\text{(g)}\rightarrow \mathrm{CO}_2\text{(g)}+2\mathrm{H}_2\mathrm{O}\text{l}$ & $-890.8$ \\ \hline
			acetylene & $\mathrm{C}_2\mathrm{H}_2\text{(g)}+\dfrac{5}{2}\mathrm{O}_2\text{(g)}\rightarrow 2\mathrm{CO}_2\text{(g)}+\mathrm{H}_2\mathrm{O}\text{(l)}$ & $-1301.1$ \\ \hline
			ethanol & $\mathrm{C}_2\mathrm{H}_5\mathrm{OH}\text{(l)}+3\mathrm{O}_2\text{(g)}\rightarrow 2\mathrm{CO}_2\text{(g)}+3\mathrm{H}_2\mathrm{O}\text{(l)}$ & $-1366.8$ \\ \hline
			methanol & $\mathrm{CH}_3\mathrm{OH}\text{(l)}+\dfrac{3}{2}\mathrm{O}_2\text{(g)}\rightarrow \mathrm{CO}_2\text{(g)}+2\mathrm{H}_2\mathrm{O}\text{(l)}$ & $-726.1$ \\ \hline
			isooctane & $\mathrm{C}_8\mathrm{H}_{18}\text{(l)}+\dfrac{25}{2}\mathrm{O}_2\text{(g)}\rightarrow 8\mathrm{CO}_2\text{(g)}+9\mathrm{H}_2\mathrm{O}\text{(l)}$ & $-5461$ \\ \hline
			\end{tabular}
			\caption[Different molar enthalpy of combustion]{Different molar enthalpy of combustion (source: OpenStax)}
		\end{table}
	\end{enumerate}
	\begin{tcolorbox}[title=Remark,colframe=black,arc=10pt]
	For condensed states the standard volume is very low in S.I. units so that $H_m^\circ\cong U_m^\circ$. However, it is very difficult to speak of pressure for so condensed states so this approximation has to be used with caution.
	\end{tcolorbox}	
	If we now consider an extensive function $X$ (as for example the volume!) defined on a chemical gaseous evolving system. We can a priori express $X$ based on two intensive variables $T$, $P$ (because an extensive function is always a product or ratio of two intensive quantities, or a sum of extensive quantities) and of the different quantity of materials $n_i,{n'}_i$ of $A_i,{A'}_i$ such that:
	
	If all products (reagents and resulting one) are in their standard state, the extensive function, therefore denoted $X^\circ$, gets the form:
	
	where the pressure is no longer involved as attached to its standard value. The gas is then described by its temperature and the quantity of its constituents!
	
	However, if we consider an infinitesimal evolution of the system at constant temperature and pressure (because assume a very slow transformation) the different quantities of materials vary therefore following the exact total differential (\SeeChapter{see section of Differential Calculus and Integral page \pageref{total exact differential}}):
	
	where obviously are taken into account, as $T$ and $P$ are are supposed constant, only the quantity of materials that could vary (yes don't forget we are doing chemistry!!!).

	We can then define artificially (nothing avoid us to do so, it's not false!) the intensive standard molar quantity that depends only on the temperature:
	
	Therefore:
	
	but we have also defined in the section Analytical Chemistry the relation:
	
	expressing, for recall, the variation in the quantity of matter of one of the compounds of a chemical reaction relatively toits stoichiometric ratio (constant) and the progress of the reaction. We therefore have:
	
	and also that:
	
	By definition, we name this algebraic sum "standard quantity reaction associated with the extensive function $X$" and denote it by (notation badly chosen by chemists in our point of view...):
	
	which is an intensive quantity that depends only on the temperature and represents a relative change (hence the subscript $r$!). This relation can also be written:
	
	In general, chemists name "\NewTerm{Lewis operator}\index{Lewis operator}", denoted $\Delta_r$, the derivative of a quantity $X$ (standardized or not), with respect to the progress of the reaction $\xi$ with constant temperature and pressure.
	\begin{tcolorbox}[title=Remark,colframe=black,arc=10pt]
	The symbol $\Delta_r$ appears with the letter $r$ in subscript to show that this is a relative reaction quantity. In other words, it is the standard variation of the molecular quantity during the concerned reaction for a given reaction progress of one mole at a pressure of $1$ bar for a perfect gas.
	\end{tcolorbox}
	We must also not forget that the stoichiometric coefficients of the reactants are positive and those of the resulting products are negative (\SeeChapter{see section of Analytical Chemistry page \pageref{stoichiometric sum}}).

	There are two reaction quantities that play important roles in chemistry:
	\begin{enumerate}
		\item The internal molar energy of reaction, named often "\NewTerm{internal energy of standard reaction}\index{internal energy of standard reaction}" of a chemical system:
		

		\item The molar enthalpy of reaction, often named more "\NewTerm{standard enthalpy of reaction}\index{standard enthalpy of reaction}" of a chemical system:
		
	\end{enumerate}
	
	\subsubsection{Standard enthalpy of reaction}
	Therefore we can consider the following two cases after knowing the relation (\SeeChapter{see section Thermodynamics page \pageref{enthalpy}}):
	
	\begin{enumerate}
		\item If the $A_i$ are in a condensed state, since the internal pressure does not apply we have:
		
		which still remains to be taken with precaution following the scenarios!

		\item If the $A_i$ are in the gaseous state (assumed perfect gas):
		
	\end{enumerate}
	We conclude that only gaz will intervene in this relation:
	
	That we write conventionally:
	
	It follows that in the special case where:
	
	(Which is in fact an unfortunate notation... for the algebraic sum of the stoichiometric ratio that would equal to zero) for a given temperature then we have:
	
	where it must be remembered that the stochiometric coefficients of the products are counted as positive, while those of the reactants are counted as negative (\SeeChapter{see section Analytical Chemistry page \pageref{stoichiometric sum}}).
	
	Thus, the variation of the enthalpy function corresponds to the variation of the quantity of heat absorbed or emitted in an isobaric transformation at a given temperature $T$. This is why it is sometimes denoted $\Delta_r H_{T,P^\circ}$.

	A chemical reaction that has an enthalpy reaction (which is for recall the instantaneous change in enthalpy during a reaction) that is negative is said to be "\NewTerm{exothermic}\index{exothermic}", since it releases heat into the environment (constant pressure obligedby the definition of enthalpy reaction!), then a chemical reaction whose reaction enthalpy is positive is say to be "\NewTerm{endothermic}\index{endothermic}" since it then requires a supply of heat to occur (so the vocabulary is the same as in the section of Thermodynamics).

	Thus, according to the preceding developments, if we denote with an index $p$ the products and with an index $i$ the reactants, we often find the standard enthalpy of reaction as follows if the stochiometric coefficients are counted as positive:
	
	Into this form, then we see well that the standard reaction enthalpy corresponds to the difference partial molar enthalpies between the products and reactants of the transformation. This is nothing more than the "\NewTerm{Hess's law}\index{Hess's law}" set in the 19th century by the Swiss chemist Henri Hess. The law the states that the total enthalpy change during the complete course of a chemical reaction is the same whether the reaction is made in one step or in several steps and can be understood as an expression of the principle of conservation of energy, also expressed in the first law of thermodynamics, and the fact that the enthalpy of a chemical process is independent of the path taken from the initial to the final state (i.e. enthalpy is a state function).

	Because in a system at equilibrium, the initial energy is always greater than or equal to the final energy (all systems tend to move towards to a more stable state with minimum energy as we have study it in the section of Thermdynamics), then the standard enthalpy of reaction $\Delta_r H^\circ$ may be only negative or zero.

	If a chemical reaction at constant pressure and at a specific temperature gives only a single chemical compound (product) then the standard enthalpy of reaction is named  "\NewTerm{standard enthalpy of formation}\index{standard enthalpy of formation}" and is denoted by $\Delta_f H^\circ$.

	In fact, the interest of prior-previous relation is that the chemist can simply, without having to know the quantities of material involved, determine just by knowing the stochiometric coefficients of an isobaric  gas or condensed chemical reaction (if he agrees that it will then be an approximation for the latter case) that the instantaneous variation of the molar internal energy during the progress of the reaction at a given temperature is equal to the instantaneous variation of the molar enthalpy .

	Two different situations arise then:
	\begin{enumerate}
		\item The difference between the instantaneous variation of the molar internal energy and the molar enthalpy is zero: therefore, the chemical reaction (at a given temperature) does not instantaneously occupies a larger volume and thus don't loss energy to push ("unnecessarily") the pressure of the gas surrounding the studied reaction (this can be seen as a money saving in terms of energy in the chemical industry). In this case, the standard enthalpy of reaction is simply equal to the heat of reaction at constant pressure $Q_p$.

		\item The difference between the instantaneous variation of the molar internal energy and the molar enthalpy is positive: Therefore, the chemical reaction (at the given temperature) instantly occupies a greater volume and thus loses some energy to push ("unnecessarily") the pressure of the gas surrounding the studied reaction  (this can be seen as a waste of money in terms of energy efficiency in the chemical industry).
	\end{enumerate}
	\begin{tcolorbox}[title=Remark,colframe=black,arc=10pt]
	Obviously, it is possible to imagine a company that takes advantage of the change in volume of a reaction (case 2 above) that pushes the surrounding gas with a piston system to then produce mechanical energy... so it would be possible in certain situations to lose much less money (verbatim energy ...).
	\end{tcolorbox}
	However a small difficulty arise, ... the standard enthalpy of a simple pure body (body formed of a single type of atom) can not be calculated in absolute terms because it depends on the internal energy which is very difficult to calculate (you must use the tools of quantum physics which raise insurmountable problems even in the beginning of the 21st century). This means we must define an arbitrary scale of molar enthalpies by setting an arbitrary zero enthalpy and adopted internationally (which is unfortunately not the case as far as we know!).

	Thus, in order to set up tables of standard molars enthalpy, he was chosen to define the scale of enthalpy as follows: the standard molar enthalpy of a simple steady pure body in the standard state is equal to $0$ at $298$ [K]. It follows that the enthalpy of formation of a simple standard pure body is always equal to zero.
	\begin{tcolorbox}[colframe=black,colback=white,sharp corners]
	\textbf{{\Large \ding{45}}Example:}\\\\
	Given the reaction:
	
	That is to say, the dissociation of chlorine and phosphorus pentachloride in phosphorus trichloride. The tables give us at the temperature of $T=1000$ [K] the following value of the standard molar enthalpy of this reaction:
	
	The variation of the value of the molar enthalpy of reaction being positive, it follows that the reaction is endothermic (requires a heat input hence the dissociation temperature of $1000$ [K]) and therefore the product is more volatile than the initial reactant.

	We have the following algebraic sum of the stochiometric coefficients of the reaction:
	
	That is (which is dimensionless since enthalpy is in molar value!):
	
	therefore the reaction increase the pressure by creating an additional mole per mole of reactant.

	Since:
	
	\end{tcolorbox}
	
	\begin{tcolorbox}[colframe=black,colback=white,sharp corners]
	then it comes:
	
	This is then the part of internal energy absorbed by the system on the $156 \;[\text{kJ}\cdot \text{mol}^{-1}]$. The remainder (difference) is has just been used to push the surrounding atmosphere of the chemical reactor.
	\end{tcolorbox}
	
	\paragraph{Kirchhoff's Enthalpy Law}\mbox{}\\\\\
	Kirchhoff's Enthalpy Law describes the enthalpy of a reaction's variation with temperature changes. In general, enthalpy of any substance increases with temperature, which means both the products and the reactants' enthalpies increase. The overall enthalpy of the reaction will change if the increase in the enthalpy of products and reactants is different.
	
	In other words, in a more practical way, the latent heat - energy required to evaporate a liquid - is not the same at every temperature! . The difference between the Gas and Liquid energy levels increases at higher temperatures. Thus, the Kirchhoff's Enthalpy law enables the calculation of a new latent heat from an existing one with a known temperature change.
	\begin{figure}[H]
		\centering
		\includegraphics[scale=0.9]{img/chemistry/eirchooff_enthalpy_law.jpg}	
		\caption{Kirchoff's enthalpy law illustration}
	\end{figure}
	So the Kirchhoff enthalpy law idea is to express the variations of the enthalpy of reaction (molar or not) in function of the temperature from the knowledge of the heat capacity at constant pressure of the gaseous reactants.

	He have built in previous developments the following relation:
	
	which is the standard enthalpy of reaction at a given temperature in a system with a standard pressure.

	We had also mentioned earlier above that $\Delta_r$, for recall, is somewhat an unfortunate notation for the differential (Lewis) operator of progress of the reaction $\mathrm{d}/\mathrm{d}\xi$.
	
	If we focus on the influence of the temperature $T$ on $\Delta_r H^\circ$ we have just to write the exact differential:
	
	since the algebraic variation of the standard enthalpy by definition depends only on the temperature.

	The stochiometric coefficients $v_i$ are not dependent of the temperature at least until this latter does not changes the essence itself of the studied transformation.
	
	We then have under this approximation (assumption):
	
	Now we have defined in the section of Thermodynamics the heat capacity at constant pressure which is written:
	
	So if the conditions are standard (the enthalpy therefore depends only on the temperature), we get is the exact differential:
	
	Then we have:
	
	We can of course integrate the latter relations to get the common form use in practice and available in many books:
	
	Then we have:
	
	where $T_0$ is a particular temperature for which $\Delta H^\circ (T_0)$ is known.

	In a temperature range very close to $T_0$ chemists sometimes approximate the variation as being linear. That is equivalent to put:
	
	It then immediately comes from the prior-previous relation:
	
	\begin{tcolorbox}[title=Remark,colframe=black,arc=10pt]
	Quite often, the variation of the enthalpy of reaction with temperature is negligible!
	\end{tcolorbox}
	\begin{tcolorbox}[colframe=black,colback=white,sharp corners]
	\textbf{{\Large \ding{45}}Example:}\\\\
	For the reaction (graphite + oxygen yielding to carbon dioxide) we would like to know $\Delta_r H^\circ$ at $1000$ [K]:
	
	For this, it is given in the tables for this reaction at $298$ [K]:
	
	and:
	
	We write in lowercase the heat capacities above as the are enough subscripts to not add a third one ($m$) that mean these are molar heat capacities.
	\begin{tcolorbox}[title=Remark,colframe=black,arc=10pt]
	When the enthalpy of reaction is given at the reference temperature (nowadays...) at $298$ [K] chemists then speak as we have already mention earlier above the "standard enthalpy of formation".
	\end{tcolorbox}
	The value of the molar enthalpy of reaction being negative, it follows that the reaction is exothermic (it is tendency of nature to favor exothermic reactions to stabilize systems in their minimal energy states).
	\end{tcolorbox}
	
	\begin{tcolorbox}[colframe=black,colback=white,sharp corners]
	We then have immediately:
	
	Thus the variation is of $-560\;[\text{kJ}\cdot \text{mol}^{-1}]$, that is a variation of about $+0.1\%$. It follows that the higher the temperature increase, more is the reaction exothermic. In fact, the choice of this particular temperature of $1000$ [K] is not innocent because it is from this temperature that experiments shows that the reaction also produces carbon monoxide.
	\end{tcolorbox}
	We can also conclude that some exothermic reactions and having a enthalpy of reaction that decreases rapidly with temperature can blow up!
	
	Finally, let us indicate that in practice we often use the term "\NewTerm{calorific power}\index{calorific power}" or "\NewTerm{heat of combustion}\index{heat of combustion}" ,which is simply the fact ... the enthalpy  of reaction per unit mass of fuel or the energy obtained by combusting a kilogram of fuel.

	Thus,  for Gasoline, we have following what give tables (under the assumption that this number is correct):
	
	And we can have fun by calculating the amount of Gasoline needed to accelerate a car of $1000$ [kg] from $0$ to $100\;[\text{km}\cdot \text{h}^{-1}]$ with a yield of $\eta=35\%$ at a temperature of $293$ [K]. Thus we have:
	
	and to get the amount of fuel in liters the tables give us the for Gasoline density about $700\;[\text{kg}\cdot \text{m}^{-3}]$ which gives finally a volume in liters of:
	
	
	\begin{flushright}
	\begin{tabular}{l c}
	\circled{20} & \pbox{20cm}{\score{3}{5} \\ {\tiny 23 votes,  58.26\%}} 
	\end{tabular} 
	\end{flushright}
	
	
\chapter{Theoretical Computing}

	\textit{\textbf{The theoretical computer science is the branch of science that deals with the development of algorithms and theoretical tools that can be apply to IT to solve formal problems related to the simulation of physical phenomena or data treatments and the exchange of information.}}(Larousse)
	\minitoc
	\pagebreak 
	 	%to make section start on odd page
	\newpage
	\thispagestyle{empty}
	\mbox{}
	\section{Numerical Methods/Analysis}\label{numerical methods}
	\lettrine[lines=4]{\color{BrickRed}T}he numerical analysis is a mathematical discipline. It is interested in both theoretical foundations as the implementation methods to resolve by purely numerical calculations and empirical approach, mathematical analysis problems.\\

	\textbf{Definition (\#\mydef):} "\NewTerm{Numerical Methods/Analysis}\index{numerical methods/analysis}" is the study of algorithms or empirical scientific methods for solving or analysing mathematical continuous or discrete problems. This means that it mainly deals to respond numerically to real or complex variables questions like numerical linear algebra over the real or complex fields, looking for numerical solutions of differential equations and other problems occurring in the physical sciences or financial/statistical engineering.
	
	Some continuous mathematical problems can be solved exactly by an algorithm. These algorithms are then named "\NewTerm{direct methods}\index{direct methods}". Examples are the elimination of Gauss-Jordan for solving a system of linear equations or the simplex algorithm of linear programming (see further below). However, for some problems no direct method are known (and is even proved that for a class of problems known as "NP complete" - see further below - there is no direct calculation with finished algorithm in a polynomial time). In such cases, it is sometimes possible to use an iterative method to attempt to determine an approximation of the solution. Such a method starts from a guessed value or roughly estimated one and finds successive approximations that should converge to the solution under certain conditions. Even when a direct method exists, however, an iterative method may be preferable because it is often more effective and often more stable (in particular it allows most often to correct minor errors in intermediate calculations). 

	The use of numerical analysis has been greatly facilitated by modern computers. The Increasing availability and power of computers since the second half of the 20th century allowed the application of numerical methods in many scientific, technical and economic, often with revolutionary, accurate and significant effects.

	\begin{figure}[H]
		\centering
		\fbox{\includegraphics[scale=0.75]{img/computing/meaning_life.eps}}
	\end{figure}
	
	In numerical simulations of physical systems (multi-physics), the initial conditions are very important in solving differential equations (see the various sections of this book where chaotic effects appears). The fact that we can not know them exactly implies that the results of calculations can never be perfectly accurate (we know this fact very well for the weather forecasting which is the most glaring example known). This effect is a consequence of the results of fundamental physics (based on pure mathematics) which demonstrates that we can not perfectly know a system by performing measurements since it directly disrupts it (Heisenberg uncertainty principle as see in the section of Quantum Wave Theory) and these disturbances are the subject of Chaos Theory (classical or quantum).

With new computer tools available in the early 21st century, it became practical and exciting to know the numerical methods to have fun with some softwares or programming languages (OpenGL, 3D Studio Max, Blender, Maple, MATLAB™, Mathematica, Comsol, R, C++, Python, etc.) to simulate 2D or 3D physical systems graphically.

	\begin{tcolorbox}[title=Remarks,colframe=black,arc=10pt]
	\textbf{R1.} Many numerical methods used in computer science are based on arguments which we have already been studied in other sections of this books. We won't come back on this methods.\\
	
	\textbf{R2.} This section being on the boundary between engineering and Applied Mathematics, we decided to give some application examples of developed tools with various programming languages or software.\\
	
	\textbf{R3.} Many of the techniques presented below are available as complete code in the C++ book \cite{oliveira2015practical} that we strongly recommend to the reader.
	\end{tcolorbox}	
	
	\textbf{Definitions (\#\mydef):}
	\begin{enumerate}
		\item[D1.]  An "\NewTerm{algorithm}\index{algorithm}\label{algorithm}" is a finite sequence of rules to be applied in a specific order in a finite number of data to arrive in a finite number of steps (including the amount, or conversely the execution time is defined by the term "\NewTerm{cost}\index{algorithm cost}") to a certain result, independently of the data type.
		
		\item[D2.] The algorithms are integrated into computers through "\NewTerm{programs}\index{programs}" (including stuff like "functions", "objects", "classes", "methods", "properties", "pointers", etc. but this is more related to a programming language course) that are the realization (implementation) of an algorithm using a given language (on a given architecture). This is the implementation of the principle.
	\end{enumerate}
	
	Axioms of programming (anecdotal):

	\begin{enumerate}
		\item[A1.] More we write code, more errors we will produce.
		
		\item[A2.] There are no programs without possible errors (due to the program itself, to the underlying electronics or most often to the user himself).
	\end{enumerate}
	
	\begin{tcolorbox}[title=Remark,colframe=black,arc=10pt]
	Basically there a is minimum steps to follow when developing an algorithm and its corresponding code. A very good process to follow is the one proposed by the ISO/CEI 9126 norm.
	\end{tcolorbox}	

	When developing a scientific algorithm, it may be interesting, rigorous and even an obligation in very high level cases to analyse the "complexity" of the algorithm. Without going too far, let's see what it is exactly:
	
	\subsection{Computer Representation of Numbers}\label{computer representation of numbers}
	
	In a Computer numbers are represented by binary digits $0$ and $1$. Computers employ binary arithmetic for performing operations on numbers. Since it gets cumbersome to display large numbers in binary form computers usually display them in hexadecimal or octal or decimal system. All of these number systems are positional 	systems. In a positional system a number is represented by a set of symbols. Each of these symbols denote a particular value depending on its position. The number of symbols  used in a positional system depends on its 'base'. Let us now discuss about	various positional number systems:
	
	\subsubsection{Decimal System}
	
	The decimal system uses $10$ as its base value and employs ten symbols $0$ to $9$ in representing numbers. Let us consider a decimal number $7402$ consisting of four symbols $7$,$4$,$0$,$2$. In terms of base $10$ it can be expressed as follows:
	
	$$7402=7\times10^{3}+4\times10^{2}+0\times10^{1}+2\times10^{0}$$
	
	So each of the symbols from a set of symbols denoting a number is multiplied with power of the base ($10$) depending on its position counted from the right. The count always begins with $0$.
	
	In general a decimal number $d_{m}d_{m-1}...d_{1}d_{0}$ consisting of $(m+1)$ symbols can be expressed as:
	$$
	d_{m}\times10^{m}+d_{m-1}\times10^{m-1}+....+d_{1}\times10^{1}+d_{0}\times10^{0}=\sum\limits_{i=0}^{m}
	d_{i}10^{i}
	$$
	where $0\leq d_{i}\leq 9$ with $i=0,1,\ldots, m$
	
	Similarly, a fractional part of a decimal number can be expressed as $\sum\limits_{i=1}^{m}d_{i}10^{-i}$
	
	\subsubsection{Binary system} 
	Binary system is the positional system 	consisting of two symbols i.e. $\{0,1\}$, named "\NewTerm{bits}\index{bits}\index{binary digit}\label{bit}", and '$2$' as its base. Any binary number $d_{m}d_{m-1}...d_{1}d_{0}$ actually represents a decimal value given by
	$$
	d_{m}2^{m}+d_{m-1}2^{m-1}+...+d_{0}2^{0}=\sum\limits_{i=0}^{m}
	d_{i}2^{i}
	$$
	where $d_{i}=0\quad or\quad 1, \quad i=0,1,..m. $
	
	Consider the binary number 10101. The decimal equivalent of 10101 is given by
	\begin{eqnarray}
	(10101)_{2} & = & 1\times 2^{4}+0\times 2^{3}+1\times
	2^{2}+0\times 2^{1}+1\times 2^{0} \nonumber \\
	& = & 16+0+4+0+1=(21)_{10} \nonumber
	\end{eqnarray}
	
	Here are the representation of some integer positive values in binary notation:
	\begin{alignat*}{6}
	1&\qquad&1& \qquad\quad\qquad&  13&\qquad&1101&\qquad\quad\qquad &25&\quad&11001&\\
	2&&10&&  14&&1110&&   26&&11010&\\
	3&&11&&  15&&1111&&   27&&11011&\\
	4&&100&& 16&&10000&&   28&&11100&\\
	5&&101&& 17&&10001&&   29&&11101&\\
	6&&110&& 18&&10010&&   30&&11110&\\
	7&&111&& 19&&10011&&   31&&11111&\\
	8&&1000&&20&&10100&&  32&&100000&\\
	9&&1001&&21&&10101&&  33&&100001&\\
	10&&1010&&22&&10110&& 34&&100010&\\
	11&&1011&&23&&10111&& 35&&100011&\\
	12&&1100&&24&&11000&& 36&&100100&
	\end{alignat*}

	Notice that:
	\begin{itemize}
		\item The  number of different values representable in $n$ bits is  $2^n$,
		
		\item An $n$-bit binary number $X=x_{n-1}x_{n-2}\cdots x_1 x_0$ can represent  any integer value in the range $0 \le X \le 2^n-1$ (e.g., if $n=3$,  then $0 \le X \le 7$). Note that $n+1$ bits are needed to represent the value $X=2^n$.
		
		\item The highest (left-most) bit $x_{n-1}$ of an $n$-bit number is named the "\NewTerm{most significant bit}\index{most significant bit}" (MSB) and the lowest bit (right-most) $x_0$ the "\NewTerm{least significant bit}\index{least significant bit}" (LSB).
	\end{itemize}
	\begin{figure}[H]
		\centering
		\includegraphics[scale=1]{img/arithmetics/binary_joke.jpg}
	\end{figure}
	It should be obvious at this level for the reader that:
	\begin{itemize}
		\item $1000$ has $10$ bits because $512\leq 1000\leq 1024$, or $2^9\leq 1000\leq 2^{10}-1$
		\item $1344$ has $11$ bits because $1024\leq 1344 \leq 2047$, or $2^{10}\leq 1344\leq 2^{11}-1$
		\item $2527$ has $12$ bits because $2048\leq 2527\leq 4095$, or $2^{11}\leq 2527\leq 2^{12}-1$
		\item $5019$ has $13$ bits because $4096\leq 5019\leq 8191$, or $2^{12}\leq 5019\leq 2^{13}-1$
		\item $9999$ has $14$ bits because $8192\leq 9999\leq 16383$, or $2^{13}\leq 9999\leq 2^{14}-1$
	\end{itemize}
	Hence the minimum number of bit required $b_\mathrm{min}$ for a $d$ digit integer is computed simply by using the specific number formula on the minimum $d$ digit value:
	
	So for example, in 2018 the biggest prime number (important in cryptography) discovered had $23,249,425$ digits, thus corresponding using the above relation to $b_\mathrm{min}=25$.
	
	\paragraph{Binary arithmetic}\mbox{}\\\\\
	Here are the addition facts that you need when adding numbers in binary notation:
	
	
	Doing binary addition is like doing decimal addition excepted that we work on a binary set. 
	\newcommand*{\carry}[1][1]{\overset{#1}}
	\newcolumntype{B}[1]{r*{#1}{@{\,}r}}
	
	Here are the subtractions facts that you need when subtracting numbers in binary notation:
	
	The fact that $10-1=1$ will lead to borrowing since you can not take $1$ from $0$ for a certain place value, and so you must borrow from the $1$ in the next place value, much like how you need to borrow in decimal notation:
	
	We will come back more in details on binary arithmetic during our study of Boolean algebra in the section of Logical Systems.
	
	\subsubsection{Hexadecimal System} 
	
	The Hexadecimal system is the 	positional system consisting of sixteen symbols, $0$,$1$,$2$...$9$,$A$,$B$,$C$,$D$,$E$,$F$, and '$16$' as its base. Here the symbols $A$ denotes $10$, $B$ denotes $11$ and so on. The decimal equivalent of the 	given hexadecimal number $d_{m} d_{m-1}...d_{0}$ is given by:
	
	 For example consider $(15ACB)_{16}$:
	
	We can convert a binary number directly to a hexadecimal number by grouping the binary digits, starting from the right, into sets of four and converting each group to its equivalent hexadecimal digit. If in such a grouping the last set falls short of four binary digits then do the obvious thing of prefixing it with adequate number of binary digit '$0$'. For example let us find the 	hexadecimal equivalent of $(111\;011\;0101\;0010\;1110)_{2}$:
	
	The vice versa is also true.
	
	\subsubsection{Octal System}  
	The octal system is the positional
	system that uses 8 as its base and $\{0,1,...7\}$ as its symbol set of size 8. The decimal equivalent of an octal number $(d_{m}d_{m-1}...d_{0})_{8}$ is given by:
	
	 For example consider $(6741)_{8}$:
	 
	
	We can get the octal equivalent of a binary number by grouping the binary digits, starting from the right, into sets of three binary digits and converting each of these sets to its octal equivalent.
	
	If such a grouping results in a last set having less number of digits it may be prefixed with adequate number of binary digit 0.
	
	As an example the octal equivalent of $(1010\;110\;111\;001)_{2}$.
	
	So we have:
	
	
	\pagebreak
	\subsubsection{Conversion of decimal system to non-decimal system:}
	To convert a decimal number to a number of any other system we should consider the integer and fractional parts separately and follow the following procedure:
	
	Conversion of integer part:
	\begin{enumerate}
		\item Consider the integer part of a given decimal number and divide it by the base $b$ of the new number system. The remainder will constitute the rightmost digit of the integer part of the new number.

		\item Next divide the quotient again by the base $b$. The remainder will constitute second digit from the right in the new system
	\end{enumerate}
	
	Continue this process until we end up with a zero-quotient. The last remainder is the leftmost digit of the new number.
	
	Conversion of fractional part:
	\begin{enumerate}
		\item Consider the fractional part of the given decimal number and multiply it with the base $b$ of the new system. The integral part of the product constitutes the leftmost digit of the fractional part in the new system.

		\item Now again multiply the fractional part resulting in step (a) by the base $b$ of the new system. The integral part of the resultant product is the second digit from the left in the new system.
	\end{enumerate} 
	
	Repeat the above step until we encounter a zero-fractional part or a duplicate fractional part. The integer part of this last product will be the rightmost digit of the fractional part of the new number.
	
	\begin{tcolorbox}[colframe=black,colback=white,sharp corners]
	\textbf{{\Large \ding{45}}Example:}\\\\
	We want to convert $54.45$ into its binary equivalent.\\
	
	\begin{enumerate}
		\item Consider the integer part i.e. $54$ and apply the steps listed under conversion of integer part i.e.

		\item conversion of fractional part:
		\vskip 10pt
	
		\begin{center}
			\begin{tabular}{ccccc}
			                  &   &  Product & integral part &
			Binary number \\ \hline $0.45$ $\times$ $2$   & = &  $0.90$    &   $0$ &
			$(.01\overline{1100})_2$ \\
			0.9 $\times$ 2   & = &  1.80    &    1 &  \\
			0.8 $\times$ 2   & = &  1.6     &    1 &  \\
			0.6 $\times$ 2   & = &  1.2     &    1 &  \\
			0.2 $\times$ 2   & = &  0.4     &    0 &  \\
			0.4 $\times$ 2   & = &  0.8     &    0 &  \\
			0.8 $\times$ 2   & = &  1.6     &    1 &  \\
			0.6 $\times$ 2   & = &  1.2     &    1 &  \\
			0.2 $\times$ 2   & = &  0.4     &    0 &  \\
			0.4 $\times$ 2   & = &  0.8     &    0 &  \\
			0.8 $\times$ 2   & = &  1.6     &    1 &  \\
			$\dots$          &   &  $\dots$ &    $\dots$ &  \\
			$\dots$          &   &  $\dots$ &    $\dots$ &  \\
			\hline
			\end{tabular}
		\end{center}
	\end{enumerate} 
	Therefore:
	
	and finally:
	
	\end{tcolorbox}
	
	Here the overbar denotes the repetition of the binary digits.
	
	Using binary system as an intermediate stage we can easily convert octal numbers to hexadecimal numbers and vice-versa:
	\begin{tcolorbox}[colframe=black,colback=white,sharp corners]
	\textbf{{\Large \ding{45}}Example:}\\\\
	(a)$(423)_{8}=(100\quad 010\quad 011)_{2}$
	
	$\qquad\qquad\qquad=(0001\quad 0001\quad 0011)_{2}=(113)_{16}$\\
	
	(b) $(93AF)_{16}=(1001\quad0011\quad1010\quad1111)_{2}$
	
	$\qquad\qquad\qquad=(010\quad010\quad011.\quad101\quad011\quad110)_{2}$
	
	$\qquad\qquad\qquad=(223536)_{8}$
	\end{tcolorbox}
	
	In the above two examples we have grouped the binary digits suitably either to quadruplets or triplets to convert octal to hexadecimal and hexadecimal to octal numbers respectively.



	\pagebreak	
	\subsection{Algorithm Complexity}\label{algorithm complexity}

\textbf{Definition (\#\mydef):} The "\NewTerm{complexity}\index{complexity}" of an algorithm is the measure of the number of fundamental operations it performs in the worst case on a dataset.

Measure the exact complexity is most of time irrelevant because often too complex given the size of the programs (too big algorithms). To avoid calculating in detail the complexity of an algorithm, we identify the fundamental operations. These basic operations can be: an assignment, a comparison between two variables, an arithmetic operation between two variables, etc.

Thus, the classical used hypothesis for the calculation of the complexity are:

	\begin{enumerate}
		\item[H1.] The four fundamental operations have the same time cost: $+ \equiv - \equiv \times \equiv \div$
		\item[H2.] A memory access has a cheaper time cost than an arithmetic operation.
		\item[H3.] A comparison check has a cheaper time cost than an arithmetic operation.
		\item[H4.] We work with only one single processor.
	\end{enumerate}
	
\textbf{Definitions (\#\mydef):}
	\begin{enumerate}
		\item[D1.] We note $D_n$ the sets of data of size $n$ and $T(n)$ the cost (in time) of the algorithm on the data or the data set of size $n$.
		
		\item[D2.] The "\NewTerm{complexity at the best}\index{complexity at the best}" is given by the function:
			
		This is the smallest time that will need an algorithm to run on a data set (lines of code) of fixed size, here equal to $n$, that the cost (duration) execution is $C(d)$. This is a lower bound on the complexity of the algorithm on a data set of size $n$.
		
		\item[D3.] The "\NewTerm{complexity at worst}\index{complexity at worst}" (the most interesting for the practitioner because it is the one to minimize!):
			
		This is the biggest time that will need an algorithm to run on a data set (lines of code) of fixed size, here equal to $n$, that the cost (duration) execution is $C(d)$. This is a upper bound on the complexity of the algorithm on a data set of size $n$. The algorithm will always finish at this time or before but never after.
		\item[D4.] The "\NewTerm{average complexity}\index{average complexity}":
			
This is the average of the complexities of the algorithm on the data sets of size $n$ (strictly speaking, we must obviously take into account the probability of occurrence of each of the data sets). This average reflects the general behaviour of the algorithm if extreme cases are rare or complexity varies slightly depending on the data. However, the complexity in practice on a particular data set may be significantly greater than the average complexity; in this case the average complexity does not give a good indication of the behaviour of the algorithm.
	
		\item[D5.] The "\NewTerm{order of complexity}\index{order of complexity}" is defined as the complexity at worst of an algorithms that contains several terms (additions or subtractions) so that we only keep the term that is growing the fastest. Thus, an algorithm having a complexity of type:
			
will be said to have a complexity of order $\mathcal{O}(n!)$. This is the "\NewTerm{big $\mathcal{O}$ notation}". It is a member of a family of notations invented by Paul Bachmann, Edmund Landau, and others, collectively named "\NewTerm{Bachmann–Landau notation}\index{Bachmann–Landau notation}" or "\NewTerm{asymptotic notation}\index{asymptotic notation}".
	\end{enumerate}
	
	\begin{tcolorbox}[title=Remark,colframe=black,arc=10pt]
	Just as worst cases are given in terms of upper bounds, best cases are given in terms of lower bounds, which are specified with "\NewTerm{big $\Omega$ notation}". A best case of $\Omega(n)$ would mean that the function is always at least linear, but could grow faster.
	\end{tcolorbox}

	\begin{tcolorbox}[colframe=black,colback=white,sharp corners]
	\textbf{{\Large \ding{45}}Examples:}\\\\
	Consider $N$ as the size of the data. For a decimal number this is the numbers $N$ of digits. Consider two decimal numbers like $0.a_1a_2a_3...$ and $0.b_1b_2b_3...$\\
	
	E1. Addition has linear complexity because in developed way addition is expressed as (\SeeChapter{see section Numbers page \pageref{number power decomposition}}):\\
		
	
	E2. Multiplication has quadratic complexity because in developed way multiplication is expressed as:\\
	
	\end{tcolorbox}

\textbf{Definition (\#\mydef):} An algorithm is said to be "\NewTerm{optimal algorithm}\index{optimal algorithm}" if its complexity has the minimal complexity among all other algorithms in its class.

As we made it understand implicitly earlier, we focus almost exclusively on the time complexity of the algorithms. Sometimes it is interesting to focus on other of their characteristics, such as space complexity (size of the memory used), the bandwidth required, etc.

For the result of the analysis of an algorithm to be relevant, we must have a model of a machine on which the algorithm will be implemented (as a program). We usually take as a reference, the "random access machine (RAM)" with a single processor where instructions are executed one after the other, without concurrent operations and without stochastic processes (in contrast to possible future quantum computers).

Most common algorithms can be classified into a number of broad classes of complexity whose order $\mathcal{O}$ varies somehow:
	\begin{enumerate}
		\item The algorithms with "\NewTerm{constant complexity $\mathcal{O}(1)$}\index{complexity!constant complexity}" that just do a boolean control (comparison).
		
		\item The algorithms with "\NewTerm{linear complexity $\mathcal{O}(n)$}\index{complexity!linear complexity}" and those in complexity in $\mathcal{O} (n log (n))$ that are considered as fast.
		
		\item The algorithms with "\NewTerm{"polynomial complexity in $\mathcal{O}(n^k)$}\index{complexity!polynomial complexity}" (for $k>3$ are considered as slow).
		
		\item The algorithms with "\NewTerm{sub-linear complexity}\index{complexity!sub-linear complexity}" whose complexity is usually of the order $\mathcal{O}(\log (n))$
		
		\item The algorithms with "\NewTerm{factorial complexity}\index{complexity!factorial complexity}" whose complexity is usually of the order $\mathcal{O}(n!)$
	\end{enumerate}
and so on... (we give here definition only for complexity for which we already have given a detailed example or will give examples further below).

It is important to see that (focusing only on complexities we have seen with examples or will see later below) for $n>4$:
	

	\begin{tcolorbox}[title=Remark,colframe=black,arc=10pt]
	In practice a good complexity is considered as being of order $\mathcal{O}(n^k)$ for $k>3$. Poor complexity is considered as being of type $\mathcal{O}(e^n),\mathcal{O}(n!)$ or equation.
	\end{tcolorbox}
	For people that like better visual stuff:
	\begin{figure}[H]
		\centering
		\begin{tikzpicture}[scale=1.8]
		\begin{axis}[domain=0:8, samples=100,grid=major,
		    restrict y to domain=0:30,xlabel=$x$,ylabel=$\mathcal{O}(g(x))$, legend pos=north east]
		\addplot [color=red]    {1};
		\addplot [color=cyan]  {x*log10(x)};
		\addplot [color=green]  {log10(x)};
		\addplot [color=purple] {(x)^0.5}; 
		\addplot [color=blue]   {x};
		\addplot [color=yellow]   {(x)^(x)};
		\addplot [color=orange]   {factorial(x)};
		\legend{$\scriptstyle\mathcal{O}(1)$,$\scriptstyle\mathcal{O}(x\log(x))$, $\scriptstyle\mathcal{O}(\log(x))$, $\scriptstyle\mathcal{O}(\sqrt{x})$, $\scriptstyle\mathcal{O}(x)$,$\scriptstyle\mathcal{O}(x^x)$,$\scriptstyle\mathcal{O}(x!)$}
		\end{axis}
		\end{tikzpicture}
		\caption{Big $\mathcal{O}$ notation complexities plot}
	\end{figure}
	We already saw above an example with linear complexity (addition of scalars) and of polynomial (quadratic) complexity (with multiplication of scalars).


Let us see some other very common examples. The first example use the results of the previous examples:

	\begin{tcolorbox}[colframe=black,colback=white,sharp corners]
	\textbf{{\Large \ding{45}}Examples:}\\\\
	E1. Evaluation of a polynomial:
	
	The direct estimate of the value of $P(x)$ leads to a complexity:
	
	\end{tcolorbox}
	
	\begin{tcolorbox}[colframe=black,colback=white,sharp corners]
	E2. Another important example is the inversion matrix problem that is used a lot in supercomputer megaflops benchmark (sometimes named "\NewTerm{LINPACK test}\index{LINPACK test}". As you can see in the MATLAB™ companions book there is a division by $n^3$ as the matrix inversion complexity is of order $n^3$. Indeed:\\
	
	At the beginning, when the first row has length $n$, it takes $n$ operations to zero out any entry in the first column (one division, and $n-1$ multiply-subtracts to find the new entries along the row containing that entry. To get the first column of zeroes therefore takes $n(n-1)$ operations.\\
   
    In the next column, we need $(n-1)(n-2)$ operations to get the second column zeroed out.\\

   In the third column, we need $(n-2)(n-3)$ operations.\\
   
   The sum of all of these operations is:
   
	which goes as $\mathcal{O}(n^3)$.
	\end{tcolorbox}
	
	Thanks to William George Horner we have a more efficient algorithm that uses a factorization of the polynomial in the form:
	
We can see almost easily that this factorization holds the same number of additions to $(n)$ but reduces the number of multiplications to $(n)$.

The resulting complexity is $\mathcal{O} (n)$. The gain is unquestionably important. In addition, this  factorization avoids computing power.

Let us see now another well known example that almost all students that have learn scripting a little bit have experiences:

	\begin{tcolorbox}[colframe=black,colback=white,sharp corners]
	\textbf{{\Large \ding{45}}Example:}\\\\
	The famous example of algorithmic complexity is the search for an  information in a sorted column (list). A simple algorithm named "dichotomous search", also named "binary search", is to take the cell to mid-column and see if we find the desired value. Otherwise, research must continue on the same method in the top or bottom of the column (depending on the lexicographical order).\\
	
	The algorithm is recursive and allows at each step, to divide by two the size of the search space.
	\end{tcolorbox}
	
	\begin{tcolorbox}[colframe=black,colback=white,sharp corners]
	If this size is initially of $n$ cells in the column, it is of size $\dfrac{n}{2}$ in step $1$, of size $\dfrac{n}{2^2}$ in step 2, and more generally of size $\dfrac{n}{2^k}$ in step $k$.\\
	
	At worst, the search ends when there is only one cell in the column to explore, i.e. when $k$ is such that $n<2^k$.\\
	
	We deduce the maximum number of steps: it is the smallest $k$ such that $n<2^k$, written in another way $\log_2(n)<k$, i.e. the sub-linear complexity:
	
	That result is strongly related to a famous guess game. The game is that if you have a set of integers $\{1,2,...n\}$. Host picks at random one of the numbers and gives you $k$ chances to guess it. After each guess the host says "higher," "lower," or "correct". The question is what is the optimal strategy (number of guesses)?\\
	
	If we use binary search algorithm with guesses lower or higher on every turn, the maximum number of guesses that is required to guess a number between $1-n$ is $\lceil\log_{2}(n)\rceil$ and this must be less than $k$ to guarantee that we guess the number within $k$ guesses. So:
	
	Therefore:
	
	Hence:
	
	So we have just proved that if $n\leq 2^{k-1}$ then there is a strategy that guarantees we will guess the number by the $k$th guess.
	\end{tcolorbox}
	The result of this last example is to compared with a sequential search (useful when sorting is too costly in resources). For example, in a column of $25,000$ data the complexity is $\mathcal{O}(n)$ thus $25,000$ while with the dichotomous method, the sub-linear complexity gives $\log_2(25,000)=15$. The gain is considerable (at the condition that data is sorted)!

	Other complexity that the reader must absolutely know are some elementary Linear Algebra calculations (see the section of the same name)! So without proof (because normally trivial given the previous examples) if we consider two square matrices $A$ and $B$ of dimensions $n$ the main operations have the following complexities:
	\begin{itemize}
		\item Read the components (nested loops): $\mathcal{O}(n^2)$
		
		\item Calculation of the trace: $\text{tr}(A)=\displaystyle\sum_{i=1}^{n}a_{ii}\rightarrow \mathcal{O}(n)$ 
		
		\item Addition $A+B=C$ so that $c_{ij}=a_{ij}+b_{ij}\rightarrow \mathcal{O}(n^2)$
		
		\item Product $A\cdot B=c$ so that $c_{ij}=\displaystyle\sum_{k=1}^na_{ik}b_{kj}\rightarrow \mathcal{O}(n^3)$
		
		\item Solving the travelling salesman problem via brute-force search is $\mathcal{O}(n!)$.
		
		\item Determinant (by the direct method of Cramer as detailed in the section of Linear Algebra page \pageref{Cramer's rule}). We can thus show that the complexity order of the determinant of a square matrix of dimensions $n$ is $n$ products, $n-1$ additions plus $n$ times the complexity of the determinant of a matrix of dimensions $n-1$ so that finally we have: $\text{det(A)}=\mathcal{O}(n\cdot n!)$
	\end{itemize}
Assuming that the computer performs an elementary operation in equation seconds (which is already a good computer), we obtain the following calculations time for several values of $n$ for the determinant:

	\begin{table}[H]
	\begin{center}
		\definecolor{gris}{gray}{0.85}
			\begin{tabular}{|p{1cm}|p{2cm}|p{2cm}|p{2cm}|p{2cm}|p{2cm}|}
				\hline
				{\cellcolor{black!30}$n$} & $5$ &  $10$ & $15$ & $20$ & $50$\\ \hline
				 {\cellcolor{black!30}}$t$ & $6\cdot 10^{-7}[s]$ & $0.04[s]$ & $5.5 \text{ h.}$ & $1,543 \text{ y.}$ & $4.8\cdot 10^{46} \text{ y.}$ \\ \hline
		\end{tabular}
	\end{center}
	\caption[]{Various costs}
	\end{table}	
	hence the need to make sometimes a time complexity calculation before starting an algorithm (unless you are working exclusively for future generations, provided that there will still be future generations...).
	
	To close this subject let us indicate that there are other notations than just the big $\mathcal{O}$ and the big $\Omega$. So let us give a summary and at the same time the definition for the other one:
	\begin{itemize}
		\item $f(x) = \mathcal{O}(g(x))$ means that the growth rate of $f(x)$ is asymptotically less than or equal to the growth rate of $g(x)$
		
		\item $f(x) = \Omega(g(x))$ means that the growth rate of $f(x)$ is asymptotically greater than or equal to the growth rate of $g(x)$
		
		\item $f(x) = \mathcal{o}(g(x))$ means that the growth rate of $f(x)$ is asymptotically less than the growth rate of $g(x)$.
		
		\item $f(x) = \omega(g(x))$ means that the growth rate of $f(x)$ is asymptotically greater than the growth rate of $g(x)$
		
		\item $f(x) = \Theta(g(x))$ means that the growth rate of $f(x)$ is asymptotically equal to the growth rate of $g(x)$
	\end{itemize}
	\begin{tcolorbox}[title=Remark,colframe=black,arc=10pt]
	Here's a simple way to remember which notation means what. All of the big $\mathcal{O}$ notations can be considered to have a bar. When looking at a $\Omega$, the bar is at the bottom, so it is an (asymptotic) lower bound. When looking at a $\Theta$, the bar is obviously in the middle. So it is an (asymptotic) tight bound. When handwriting $\mathcal{O}$, you usually finish at the top, and draw a squiggle. Therefore $\mathcal{O}(g(x))$ is the upper bound of the function. To be fair, this one doesn't work with most fonts, but it is the original justification of the names.
	\end{tcolorbox}



	\subsubsection{NP-Completude}\label{np completude}
	We will introduce now for the general culture the concept of "\NewTerm{NP-completeness}\index{NP-completeness}", that is to say that we will try to define without too much formalism (as usual in this book).
	
	First let us introduce two non-formal definitions of a Turing machine.
	
	\textbf{Definitions (\#\mydef):}
	\begin{enumerate}
		\item[D1.] An algorithm is said to be a "\NewTerm{deterministic Turing machine}\index{deterministic Turing machine}" when the set of its rules prescribes at most one action to be performed for any given situation.
		
		\item[D2.] A "\NewTerm{non-deterministic Turing machine}\index{non-deterministic Turing machine}" may have a set of rules that prescribes more than one action for a given situation. For example, a non-deterministic Turing machine may have both "If you are in state 2 and you see an 'A', change it to a 'B' and move left" and "If you are in state 2 and you see an 'A', change it to a 'C' and move right" in its rule set.
	\end{enumerate}
	
	Now let us define what type of problems are frequently considered:
	
	\textbf{Definitions (\#\mydef):}
	
	\begin{enumerate}
		\item[D1.] "\NewTerm{Logarithmic problems L}\index{problem!logarithmic problems L}" contains all algorithms (problems) that can be solved by a deterministic Turing machine using a logarithmic amount of computation time.
		
		\item[D2.] "\NewTerm{Polynomial problems P}\index{problem!polynomial problems P}" contains all algorithms (problems) that can be \underline{solved} by a deterministic Turing machine using a polynomial amount of computation time.
		
		\item[D3.] "\NewTerm{Non-deterministic polynomial time problems NP}\index{problem!non-deterministic polynomial time problems NP}" are the set algorithms (problems) where the result instances can be \underline{controlled} in a polynomial time complexity by a non-deterministic Turing machine. 
	\end{enumerate}
	
	It should be noted at this stage of the discussion that the class P is included in the NP class so that $\text{P} \subset \text{NP} $. Indeed if we know a polynomial algorithm to solve a problem then we can at worst check the solution with a polynomial complexity algorithm also.

	But a difficult question is the following reciprocal: if a problem is NP (known solution can be controlled in polynomial time) but it seems we can not found yet an algorithm in a polynomial time (or less) to find the solution, does it always exist a polynomial algorithm to find the solution so that $P=NP$ that thus bring us to write  $\text{P} \subseteq \text{NP}$?
	
	In other words: If the solution of a problem can be quickly verified as being the correct one, can the solution also be found quickly?
	\begin{tcolorbox}[colframe=black,colback=white,sharp corners]
	\textbf{{\Large \ding{45}}Examples:}\\\\
	E1. The problem of finding a Hamiltonian cycle (cycle that passes once and only once by all the vertices of the graph - see section Graphs Theory page \pageref{hamiltonian cycle}) in a graph belongs to NP since, given a known cycle it is trivial to check in linear time $\mathcal{O}(n)$ it contains a well and once each vertex but found the cycle is at worst a factorial complexity $\mathcal{O}(n!)$.\\
	
	E2. Factoring an integer $n$ in product of prime factors (important in cryptography) is an NP problem. Indeed, given the prime factors $p_1,p_2,...,p_n$ it is trivial to control the solution $n=p_1,p_2,...,p_n$ as it is of order $\mathcal{O}(n^2)$. But we do not know if it exist a polynomial algorithm to find the prime numbers (to find the solution). So we do not know if the problem of finding prime numbers of an integer is a P-problem.
	\end{tcolorbox}
	
	\begin{tcolorbox}[title=Remark,colframe=black,arc=10pt]
	Apparently (we were not able to find the proof of this result and neither had opportunity to do so) the complexity of the best factorization algorithm for prime numbers is in the year 2007 of the type:
	
therefore there is still work to do (if a reader could provide us with the details that led to this result, we are interested).
	\end{tcolorbox}
	A problem $x$ that is in NP is also in NP-Complete if and only if every other problem in NP can be quickly (i.e. in polynomial time) transformed into $x$.

	In other words a problem $x$ is "\NewTerm{NP-Complete NPC}\index{NP-Complete NPC}" if:
	\begin{enumerate}
		\item $x$ is in NP
		\item Every problem in NP is reducible to $x$
	\end{enumerate}
	So, what makes NP-Complete so interesting is that if any one of the NP-Complete problems was to be solved quickly, then all NP problems can be solved quickly.

	A NPC problem is complete in that it contains most of the complexity of problems belonging to NP, and a polynomial solution to this problem involves a polynomial solution to all NP type problems.

	In other words: NPC problems have an exponential complexity and they all have the same complexity class (modulo polynomials).

	Finally, what is important to understand and retain about this idea is that if we find one day a polynomial algorithm for one of these really difficult problems that are the NPC problems, then in one stroke NP becomes equal to P and all difficult problems become easy such that 
	
	In other words: can we find in a polynomial time what can be proved (controlled) in polynomial time?

	\begin{tcolorbox}[title=Remark,colframe=black,arc=10pt]
	This issue is so important in computing science to reduce energy consumptions and time computations that it belongs (arbitrarily) to the 7 millennium problems, whose resolution is prized \$1 million  by the Clay Mathematics Institute. 
	\end{tcolorbox}
	
	Ok this definitions given let us now study of some typical applications of numerical methods which are very often used in the industry. We will go from simple to more complicated and not forgetting that many methods that are not in this section can sometimes be found in other sections of the book!

\subsection{Integer Part}\label{integer part}

	The biggest integer less than or equal to a real number x is expressed in this book by $[x]$, which will be read "\NewTerm{integer part of $x$}\index{integer part}" and according to the norm ISO 80000-2:2009 \textit{Mathematical signs and symbols to be used in the natural sciences and technology} it should denoted as: $\text{int} \; x$.

	The number $M$ is an integer if and only if $[M] = M$. Similarly, the natural number $A \in \mathbb{N}$ is divisible in the natural set by the natural number $B \in \mathbb{N}$ if and only if:
	
	We also denote by $\left\lbrace x\right\rbrace $ the fractional part of $x$ such that:
	
	That is to say:
	
	with $\mid \left\lbrace x\right\rbrace \mid < 1$.
	
	Consider $x,y \in \mathbb{R}$. Then we have the following properties (normally most of them don't need any proof):
	\begin{enumerate}
		\item[P1.] $[x]\leq x <[x]+1\Leftrightarrow 0\leq x-[x] <1$
		
		\item[P2.] $[x]=\displaystyle\sum_{n\leq x} 1$ for $x \geq 0$
		
		\item[P3.] $[x+m]=[x]+m$ if $m \in \mathbb{Z}$
		
		\item[P4.] $[x]+[y] \leq [x+y] \leq [x]+[y]+1$
		\begin{dem}
			A reader asked for the proof of this property. So let us see how to proceed. First we write:
			
			where $n,m \in \mathbb{Z}$ and where $0 \leq \theta < 1$ and $0 \leq \phi <1$. Thus:
			
			By writing $x=n+\theta$, where $0\leq 0 <1$, we have:
			
			where $0 < 1 -\theta \leq 1$.
			It follows that:
			
			if $\theta = 0$ and:
			
			if $0<\theta<1$.
			\begin{flushright}
				$\blacksquare$  Q.E.D.
			\end{flushright}
		\end{dem}
		
		\item[P5.] $[-x]=-[x]$ if $x \in \mathbb{Z}$ otherwise $[-x]=-[x]-1$ if $x \not\in \mathbb{Z}$
			\begin{dem}
			 The proof is already given at the end of the proof of property P4.
			\begin{flushright}
				$\blacksquare$  Q.E.D.
			\end{flushright}		
			\end{dem}
			
		\item[P6.] $\left[\dfrac{[x]}{m}\right]=\left[\dfrac{x}{m}\right]$ if $m \in \mathbb{N}$ 
			\begin{dem}
				For this proof we will write:
				
				where $0 < \theta <1$
				and:
				
			where $0 \leq r < m$ (\SeeChapter{see section Number Theory page \pageref{euclidean division}}). So we get:
				
				because $0 \leq r + \theta \leq m$. Besides :
				
				and thus we have the expected result.
				\begin{flushright}
					$\blacksquare$  Q.E.D.
				\end{flushright}
			\end{dem}

		\item[P7.] If $a \in \mathbb{N}$ then $\left[\dfrac{x}{a} \right]$ represents the number of integers less than or equal $x$ that are divisible by $a$.
		\begin{dem}
		For the last part, we observe that if $a, 2a,...,ma$ are all positive integers $\leq x$ that are divisible by $a$, it suffices to prove that $\left[\dfrac{x}{a}\right]=m$. 
		
		Since $(m+1)a>x$, then:
			
		That is to say:
			
		and thus we have the expected result.					
		\begin{flushright}
			$\blacksquare$  Q.E.D.
		\end{flushright}			
		\end{dem}
	\end{enumerate}
	\begin{tcolorbox}[title=Remark,colframe=black,arc=10pt]
	The rounding method of real values will be given in the section on Economy.
	\end{tcolorbox}
	
	\subsection{Heron's Square Root Algorithm}\label{Heron square root algorithm}
	
	One of the first things that many people learn in an introduction to Computer Sciences course is the algorithm for calculating the square root of a number. 
	
	There is such a simple algorithm for this purpose named "\NewTerm{Heron's algorithm}\index{Heron's algorithm}" or "\NewTerm{algorithm of Heron of Alexandria}\index{algorithm of Heron of Alexandria}" or even "\NewTerm{Babylonian method}\index{Babylonian method}" which converge to the value of this square root.
	
	Thus want to calculate the square root:
	
	\begin{dem}
		Here is a pseudo-proof because historically the algorithm was built on purely intuitive considerations (since 100 BC algebra did not exist...). In high-school classes the result is given as a definition and just convergence is observed.
		 
		\begin{tcolorbox}[title=Remark,colframe=black,arc=10pt]
		The "\NewTerm{convergence ratio}\index{convergence ratio}" of a sequence $x^{(k)}$ to a constant $x_{0}$ is:
		
		if this limit exists. If the convergence ratio is greater than $0$ and less than $1$, the sequence is said to "\NewTerm{converge linearly}". If the convergence ratio is $0$, the sequence is said to "\NewTerm{converge superlinearly}".
		\end{tcolorbox}
	
		So for the demonstration, we will proceed as follows:
		
		And the trick is to write:
		
		This is named a "\NewTerm{fixed-point iteration}\index{fixed-point iteration}" (related to the theorem of the same name proved in the section of Sequences and Series page \pageref{fixed point theorem}). It is used in mainly situations in computer science. We will encounter it quite a lot of time further below...!
		\begin{flushright}
			$\blacksquare$  Q.E.D.
		\end{flushright}
	\end{dem}
	
	\begin{tcolorbox}[colframe=black,colback=white,sharp corners]
	\textbf{{\Large \ding{45}}Example:}\\\\
	We want to calculate:
	
	We take $A=10$ and this gives us the following table of iterations:
	
		\begin{table}[H]
	\begin{center}
		\definecolor{gris}{gray}{0.85}
			\begin{tabular}{|p{1cm}|p{2cm}|p{1.8cm}|p{2.5cm}|p{2cm}|}
				\hline
				\multicolumn{1}{c}{\cellcolor{black!30}\textbf{Iteration}} & 
\multicolumn{1}{c}{\cellcolor{black!30}\textbf{$\dfrac{x_i}{2}$}} & \multicolumn{1}{c}{\cellcolor{black!30}\textbf{$\dfrac{A}{2x_i}$}} & \multicolumn{1}{c}{\cellcolor{black!30}\textbf{$x_{i+1}$}} & \multicolumn{1}{c}{\cellcolor{black!30}\textbf{Gap}}\\ \hline
		1 & 5 & 0.5 & 5.50 & $\cong$ 2 \\ \hline
		2 & 2.750 & 0.90909 & 3.659090909 & $\cong$ 0.49 \\ \hline
		3 & 1.82954 & 1.3664 & 3.196005083 & $\cong$ 0.033 \\ \hline
		4 & 1.59800 & 1.5644 & 3.162455624  & $\cong$ 0.0002 \\ \hline
		5 & 1.58122 & 1.5810 & 3.162277665 & $\cong$ 0.5$\cdot 10^{-8}$ \\ \hline
	\end{tabular}
	\end{center}
	\caption[]{Iterations for Heron's square root algorithm}
	\end{table}	
	\end{tcolorbox}

	In the case of the cubic root, the proof is similar and we obtain:
	
	
	and so on...
		
	We can prove algebraically that the babylonian method converge. Note that all $x_n>0$ and:
	 
	Thus, all $x_{n}$ past the first satisfy $x_{n}^{2} \geq a .$ Therefore:
	 
	says that $x_{n+1} \leq x_{n} .$ Therefore, $\left\{x_{n}\right\}$ is a decreasing sequence, bounded below. Thus, the sequence
	converges. Furthermore, since the sequence converges:
	 
	Therefore:
	 
	To conclude it is perhaps interesting to know that the reader can find in the section of Number Theory the method used in the antiquity (at least an analogy) using continued fractions.
	
	\begin{tcolorbox}[title=Remark,colframe=black,arc=10pt]
	We will see later that this above method is a special case of Newton's method introduced further below page \pageref{newton method}.
	\end{tcolorbox}
	
	\pagebreak
	\subsection{Archimedes Algorithm}
	
	The calculation of the universal constant "pi" denoted by $\pi$ is certainly the algorithm with the greatest interest since we found this constant almost everywhere in physics and mathematics (there are numerous books on the subject available on the market).
	
	We recall that we did not give the value of $\pi$ in the section on Geometry or in other sections of this book until now. So we will now tackle this task.
	
	We define in geometry the number named "pi", independently of the metric used, as the ratio of half the circumference of a circle with its radius such that:
	
	It seems that we own the first algorithm of the calculation of this constant by Archimedes (287-212 BC.) and whose proof is given below:
	\begin{dem}
		Consider an $n$-polygon inscribed in a circle (we start with a square of $n$ sides of length $2u_n$ and if we go from a square to an octogone, two times more side, the sides length becomes $2u_{2n}$ and so on...):
		\begin{figure}[H]
			\begin{center}
			\includegraphics[scale=0.75]{img/computing/archimedes_algorithm.eps}
			\end{center}	
			\caption{The illustrated principle of Archimedes algorithm}
		\end{figure}
	The principle of Archimedes algorithm is as follows:
	
	Given the perimeter of a regular polygon of $n$ sides inscribed in a circle of radius $1/2$ we can see in the figure above that by induction (\SeeChapter{see section Trigonometry page \pageref{spherical trigonometry}}):
	
	With have for the perimeter of an $n$-polygon:
	
	and:
	
	with:
	
	hence:
	
	Therefore:
	
	We then just need a computer and several iterations to evaluate with a good accuracy the value of $\pi$. Obviously, we use the Heron algorithm to calculate the square root...
	\begin{flushright}
		$\blacksquare$  Q.E.D.
	\end{flushright}
	\end{dem}
	
	\begin{tcolorbox}[title=Remark,colframe=black,arc=10pt]
	There are a large number of algorithms to compute $\pi$. The one shown above, without being the most aesthetic, seems historically to be the first and most simple one.
	\end{tcolorbox}
	
	\pagebreak
	\subsection{Euler's Number $e$}\label{euler number computation}
	Including the constant $\pi$, there are many other important mathematical constant that we need to generate with a computer (nowadays most constant values are stored as is and are not recalculated automatically). Among them, is the "\NewTerm{Euler number}\index{Euler number}" denoted by $e$ (\SeeChapter{see section Functional Analysis page \pageref{Euler number}}). Let's see how to calculate this number.
	
	Consider the Taylor series (\SeeChapter{see section Sequences and Series page \pageref{taylor series}}) for an infinitely differentiable function $f$ given by:
	
	As (\SeeChapter{see section Differential and Integral Calculus page \pageref{usual derivatives}}):
	
	Therefore we have:
	
	So finally:
	
	This relation provides an easy algorithm to calculate the Euler number to a given order $n$ of precision.

	\subsection{Stirling's factorial approximation}\label{stirling}
	Obviously, the factorial may be calculated with a simple iteration. However, this kind of method generates an exponential complexity algorithm which is not the best. Then there exists another method:

	In mathematics, Stirling's approximation (or Stirling's formula) is an approximation for factorials. It is a very powerful approximation, leading to a very useful result in Statistical Mechanics (see section of the same name page \pageref{statistical physics distributions}). 

	There are various approach that lead to different results. We will focus here on the only one that we will use later in theoretical physics and that is the worst easiest approximation.
	
	Either the definition of the factorial:
	
	
	And according to the properties of logarithms:
	
	
	If $n$ is very large (but very large) then the previous sum can be approximately written as an integral:
	
	
	Solving this integral we get (\SeeChapter{see section Differential and Integral Calculus page \pageref{usual primitives}})
	
	
	When $n \gg 1$, the lower limit is negligible and then (approximation that will be very useful in the section of Statistical Mechanics):
	
	\begin{figure}[H]
		\centering
		\includegraphics[scale=0.7]{img/computing/stirling_factorial_approximation.jpg}
		\caption{Comparison of logarithmic formulations using Stirling formula}
	\end{figure}
	
	After a small elementary simplification, we obtain:
	
	
	The latter relation is useful of course only if we assumes that Euler's constant is a value stored in the machine...
	
	Let us consider the Gamma function (\SeeChapter{see section Differential and Integral Calculus page \pageref{gamma euler function}}):
	
	and recall that for integers $\lambda$ we have:
	
	We get:
	
	Substituting $y=t/\lambda$ and letting $g(x)=y-\log (y)$ We get: 
	
	Now we will use Laplace's method of integration (\SeeChapter{see section Differential and Integral Calculus page \pageref{laplace method of integration}}) and for this we differentiate twice and get:
	
	so that $y^*=1,g(y^*)=1$ and $g''(y^*)=1$. Laplace's method now
	yields to:
	
	The relation:
	
	which is known as "\NewTerm{Stirling's improved formula}\index{Stirling's improved formula}".
	
	\pagebreak
	\subsection{Linear Systems of Equations}\label{linear systems of equations}
	
	There are many methods for solving systems of linear equations. Most of them have been developed to address particular systems. We will here study for the moment only one, named the "\NewTerm{Gauss reduction method}\index{Gauss reduction method}" or "\NewTerm{Gauss pivot}\index{Gauss pivot}" or "\NewTerm{Gauss reduction algorithm}\index{Gauss reduction algorithm}" or also  "\NewTerm{Gauss elimination}\index{Gauss elimination}" which is well suited for solving small systems of linear equations (up to $50$ unknowns).
	
	\begin{tcolorbox}[title=Remarks,colframe=black,arc=10pt]
	\textbf{R1.} The validity of some of the operations that we will perform here to solve linear systems is proved in the section on Linear Algebra. In fact, to be brief, the whole method use vector spaces whose columns are linearly independent vectors.\\

	\textbf{R2}. Recall that linear systems admit a solution if and only if the rank of the augmented matrix is less than or equal to the number of equations (\SeeChapter{see section Linear Algebra page \pageref{rank of a matrix}}).
	\end{tcolorbox}
	
	\subsubsection{One equation with one unknown}
	
	We begin for sure with the easiest and smallest possible example... of one equation with one unknown:
	
	Where $a$ and $b$ are the coefficients of the equation and $x$ is unknown. Solving this equation consists to determine $x$ is according to $a$ and $b$. If $a$ is not equal to $0$ then:
		
	is the solution of the equation. If $a$ is equal to zero and if $b$ is different from $0$ then the equation above admits no solution. If $a$ and $b$ are equal to zero, then the equation has infinitely many solutions.
	
	\subsubsection{Two equations with two unknowns}
	
	A linear system of two equations with two unknowns can be written as we know:
		
		Where $a_{11},a_{12},a_{21},a_{22}$ are the coefficients of the equations, $x_1$ and $x_2$ are the unknowns.
		
		\begin{tcolorbox}[title=Remark,colframe=black,arc=10pt]
The above-used notations have nothing to do with the tensor calculus!
		\end{tcolorbox}	
		
		To solve the system following the algorithm that interest us we proceed as follows:
		
		Using elementary algebraic manipulations (addition or subtraction of the various equalities between themselves - operations authorized by the linear independence of vectors-lines) we transform the system into another with one unknown less for one of the both equations given by (for example):
		
		The transformation between the two systems:
		
		is simply done by multiplying each coefficient of the first equality by $\dfrac{a_{21}}{a_{11}}$ and subtracting the second line with the resulting equation. This procedure is named "\NewTerm{row reduction}\index{row reduction}" or "\NewTerm{Gaussian elimination}\index{Gaussian elimination}\label{Gaussian elimination}". 
		
		In our case, the element $a_{11}$ is named the "\NewTerm{pivot}\index{pivot}" or "\NewTerm{leading coefficient}\index{leading coefficient}". In others words, if the $L_i$ is the notation for the each line number of the system, what we made is:
		
		Then, we solve the equation with only one unknown:
		
		We can therefore conclude with:
		
		But this is not the real algorithm. The real one use the augmented form:
		
		 Now we apply $L_2-\dfrac{a_{21}}{a_{11}}L_1\rightarrow L_2$ and $L_1-\dfrac{a_{12}}{a_{22}}L_2 \rightarrow L_1$ such that:
		
		As you can see the matrix has been put in diagonal form!
		Now to continue, we write:
		
		
		Now we just apply $\dfrac{1}{a_{11}^{\prime}}L_1 \rightarrow L_1$ and  $\dfrac{1}{a_{12}^{\prime}}L_2 \rightarrow L_2$ such that:
		
		Finally:
		
		
		\subsubsection{Three equations with three unknowns}
	Now consider the case of the linear systems of three equations with three unknowns:
		
	We can subsequently by elementary operations (see section Linear Algebra page \pageref{linear systems} and the previous case) reduce this linear system in the following echelon form system:
		
		And therefore we can trivially solve the last line:
		
		And afterwards the second line:
		
		And finally:
		
		Let us return to the systems transformations. It is carried out in two stages:
		\begin{enumerate}
			\item In the first line, we choose $a_{11}$ as the pivot and we eliminate the coefficients $a_{21}$ and $a_{31}$ as follows:
			
			We have to multiply each coefficient of the first line by $\dfrac{a_{21}}{a_{11}}$ and subtract this result of the second line and therefore $a_{21}$ disappears.
			
			Similarly, by multiplying the coefficients of the first line by $\dfrac{a_{31}}{a_{11}}$, and subtracting the result obtained from the third line, $a_{31}$ disappears.
			
			The linear system of equations can therefore be written as:
			
			\item The second step is to treat the linear system of two equations with two unknowns formed by the second and third lines of the previous system and that, in choosing $a_{22}^{\prime}$ as pivot. This method of resolution can seem complicated but it has the advantage of being generalized and be applied to solve linear systems of $n$ equations in $n$ unknowns.
		\end{enumerate}
		
		\begin{tcolorbox}[colframe=black,colback=white,sharp corners]
		\textbf{{\Large \ding{45}}Example:}\\\\
		Let us see an example taken on Wikipedia (helps to copy/paste boring \LaTeX ...):
		
		Now we put:
		
		Therefore we get:
		
		Now we put:
		
		Therefore we get:
		
		\end{tcolorbox}
		\begin{tcolorbox}[colframe=black,colback=white,sharp corners]
		To continue we put:
		
		Therefore we get:
		
		We put:
		
		Therefore we get:
		
		And finally we put:
		
		To get:
		
		So this is again an application of the row reduction method (or Gaussian elimination).
	\end{tcolorbox}
	For sure when you know that such systems can be solved using just a matrix inversion with a vector multiplication (\SeeChapter{see section Linear Algebra page \pageref{linear systems}}) all that stuff will be almost useless for most employees working in non-maths jobs. 
	
	\begin{tcolorbox}[title=Remark,colframe=black,arc=10pt]
	By extension if the procedure of solution showed earlier above and gives a line where $0=\text{'something non-zero'}$ or $0=0$, then it means that the system has no solution and respectively infinitely many solutions. We will see in the section of Linear Algebra that this is explained by the fact that the matrix of the system is a singular matrix (i.e. there is no inverse).
	\end{tcolorbox}

		\pagebreak
		\subsubsection{$n$ equations with $n$ unknowns}
	
		To simplify the writing, the coefficients will always be noted $a_{ij}$ and not $a_{ij}^{\prime},a_{ij}^{\prime\prime}$, etc. at each stage of the calculation.
		
		Given the linear system (we could also represent it as an augmented matrix to simplify the notations):
		
We will choose $a_{11}$ as the pivot to eliminate $a_{21},a_{31},...,a_{n1}$. Then, removing $a_{32},a_{42},...,a_{n2}$ is performed by taking $a_{22}$ as a pivot. Last pivot to consider is obviously $a_{n-1,n-1}$, it helps eliminate $a_{n,n-1}$. The system then takes the form:
		
		And we can therefore solve the last equation, then the fore last equation, and so on up to the first one.
		
		This method must however be fine-tuned to avoid pivots with $0$ values. The trick is therefore to switch the order in which the equations are written to choose the pivot coefficient whose absolute value is the largest. Thus, in the first column, the better is pivot is the coefficient such that $a_{j1}$:
		
It is taken to $a_{11}$ by permutation of first lines and $j$-th lines. Removal of the rest of the first column can then be performed. Then, we begin again with the $n-1$ remaining equations.

	\begin{tcolorbox}[colframe=black,colback=white,sharp corners]
	\textbf{{\Large \ding{45}}Example:}\\\\
	A last example with $4$ unknowns with a different representation that may help (we hope so!). Consider the following linear system:
	
	Let us apply the Gauss elimination algorithm:
	
	And the last steps should be obvious to the reader (if not we can detail them on request as always!).
	\end{tcolorbox}

	\pagebreak
	\subsection{Polynomials}
	The basics polynomials with real coefficients has been studied in the section of Functional Analysis in detail. Here we will address only the digital aspect of some problems related to polynomials (that is to say elementary algorithms or formulas useful for some operations not included by default in most computer programming languages).
	
	Apart from the addition and subtraction of polynomials which we assume as trivial (aside the optimization of the complexity aside as the Horner scheme), we will see how to multiply and divide two polynomials.
	
	Let's first see how to multiply two polynomials.
	
	Let:
	
	Therefore:
	
	With for $k=0,1,2,...,n+m$:
	
	it was easy...
	
	The second case of interests to us now is the Euclidean division of polynomials (\SeeChapter{see section Calculus page \pageref{polynomials division}}).
	
	Let us take again:
	
	but with the condition that $n \geq m$ that is to say $\text{deg}(f(x))<\text{deg}(g(x))$.
	
	The division can be written as we know (see the section of Number Theory or Calculus):
	
	with:
	
	otherwise $r(x)=0$
	It is normally known beforehand (since proved in the section Calculus of the chapter Algebra) that we have:
	
	and:
	
	We have therefore by definition $q(x)$ that is the quotient of the division and $r(x)$ the remainder of the Euclidean division of $f(x)$ by $g (x)$.
	
	Therefore, nothing prevents us from writing in the most general possible way:
	
	To prove the expression of different $q_i$, we have preferred for educational reasons to use a specific example (see below) whose result will be generalized.
	
	\begin{tcolorbox}[colframe=black,colback=white,sharp corners]
\textbf{{\Large \ding{45}}Example:}\\\\
	Let:
	
	So of what we have said before, we get (starting point):
	
	Using the fact that (as a reminder):
	
	So we have almost immediately:
	
	Then (still proceeding in the same way):
	
	And finally:
	
	\end{tcolorbox}
	So in general:
	
	As:
	
	The first remainder is then:
	
	After:
	
	The second remainder is then:
	
	and so on... we continue until $\deg(r_k(x))<m$.	
	
	\pagebreak
	\subsection{Regression Techniques}\label{regression techniques}
	Regressions are very useful and very important tools for statisticians, engineers, computer scientists, marketing analysts, physicists, physicians, economists wishing to establish a law of correlation between two (or more) variables, do a qualitative analysis, an extrapolation or even to separate signal form noise (as every point outside the regression will be considered as being noise).
	
	There are many regression  methods: the simple solution of first degree equations (when only two points of measurement are known) to equations that permits to obtain from a large number of points information that are essential to the establishment of a linear regression, polynomial, logistic or other law (or function).
	
	Let us give a list of the most used regressions techniques used in business and administrations (whose mathematical models are not all shown in this section yet but will be when we will have more time):
	\begin{enumerate}
		\item \textbf{Simple Linear Regression (SLR) model}\footnote{Also named \textbf{Classical Linear Regression (CLR)}} with one (\textbf{Univariate Linear Regression (CLR)}) or more variables (\textbf{Multiple Linear Regression (MLR)}) based on the method of least\footnote{Don't confuse "estimation methods" like OLS (ordinary least squares), TLS (total least squares), ML (maximum likelihood), REML (restricted maximum likelihood), PQL ( penalized quasi-likelihood) with "regression methods"!} squares with binary or continuous variables with response variable and coefficients\footnote{If the coefficients are required to be non-negative, we speak then of \textbf{Non-Negative Least Squares} and if we require the coefficient to be in some bounds then we speak of \textbf{Bounded-Variable Least-Squares} or \textbf{Constrained Least Squares}.} belonging to $\mathbb{R}$. Presented in detail in this section of the book (implicitly this model contains the interactions between variables and also some non-linear models).
		
		\item \textbf{Gaussian linear regression model} (statistical approach of linear regression based on the method of least squares) with binary or continuous variables with response variable also belonging to $\mathbb{R}$. When a regression model assumes that the errors are normally distributed, we speak commonly of \textbf{general linear model}. Presented in detail in this section of the book. A gaussian linear regression model is a special case of the generalized linear model (GLM).
		
		\item \textbf{Weighted least squares regression model} is used in the case where the assumption $\text{V}(\varepsilon_i)=\sigma_\varepsilon^2$ of Gaussian linear regression is not satisfied. The purpose is to apply a transformation trick using simple weights (see further below for more details), to keep this assumption valid.
		
		\item \textbf{Nonlinear regression models} with binary or continuous variables with response variable in $\mathbb{R}$. Presented in detail in this section of the if they can be reduced to linear case or not but then no interactions of explanatory variables. Otherwise based on quasi-Newton techniques type or Gauss-Newton presented also this section.
		
		\item \textbf{Polynomial regression model by the method of B-splines} or of the collocation polynomial with response variable in $\mathbb{R}$. The most common case is to restrict ourselves to the restriction of third order splines and then we speak of \textbf{restricted cubic splines regression (RCS)}. Presented in detail in this section of the book.
		
		\item \textbf{Logistic regression models} (binomial/multinomial/ordinal regressions) with binary, nominal variables (categorical) or ordinal or continuous with response variable bounded between $0$ and $1$. Presented summarily and naively in this section of the book. A logistic, probit or robit (robust version of the logit model) regression model is a special case of the generalized linear model (GLM)
		
		\item \textbf{Logic regression model} is a (generalized) regression methodology that is primarily applied when most of the covariates in the data to be analysed are binary. The goal of logic regression is to find predictors that are Boolean (logical) combinations of the original predictors.
		
		\item \textbf{Principal Component Regression (PCR)} instead of regressing the dependent variable on the explanatory variables directly, the principal components of the explanatory variables are used as regressors. Therefore PLS can reduces the number of predictors (dimensional reduction) to a smaller set of uncorrelated components and performs least squares regression on these components.
		
		\item \textbf{Partial Least squares Regression (PLR)} is a statistical method that bears some relation to principal components regression.  They differ in the methods used in extracting factor scores. In short, principal components regression produces like weight matrix reflecting the covariance structure between the predictor variables, while partial least squares regression produces like a weight matrix reflecting the covariance structure between the predictor and response variables.
		
		\item \textbf{Counting Poisson regression} (Poisson MLE, PMLE, GLM, Poisson-quasi-Lindley) or \textbf{negative binomial regression} (binomial MLE and QGPMLE) model with binary, nominal (categorical) or ordinal or continuous variables with positive integer answer in $\mathbb{N}$. All the regression models that do not assume the errors to me normally distributed, but following other well-known statistical distributions, are known under the name of \textbf{Generalized Linear regression Models (GLM)}.
		
		\item \textbf{Beta regression} is commonly used by practitioners to model variables that assume values in the standard unit interval $[0,1]$. It is based on the assumption that the dependent variable is beta-distributed and that its mean is related to a set of regressors through a linear predictor with unknown coefficients and a link function.  A beta regression model is also a special case of the generalized linear model (GLM).
		
		\item \textbf{Simplex regression} is actually (2019) considered as a more robust and flexible alternative to the beta regression dedicated of rates or proportions. This regression is based on the simplex distribution, hence it's name...
		
		\item \textbf{Orthogonal linear regression model} (or Deming regression) that is used as complement to the paired $T$-test to check the stability of the measuring instruments in laboratories. This is a case where the explanatory and dependent variables are tainted with uncertainty. Notice that at the opposite of all regression techniques listed above and below that have no assumptions about the distribution of the predictor (independent) variables, the orthogonal linear regression is as far as we know, the only regression technique that has an assumption of the distribution of the predictor!
		
		\item \textbf{Quantile regression model} (very useful in the medical and economic fields) based on the same idea as the regression by the method of least squares, but where we do not minimizes the sum of squared errors from the average, but the sum of absolute errors from a given quantile (median or other). It's also used sometimes to get rid of extreme values.
		
		\item \textbf{Theil–Sen estimator method}\index{Theil–Sen estimator method} also named Sen's slope estimator\index{Sen's slope estimator} or Slope selection method\index{slope selection method} or Single median method\index{single median method} or Kendall robust line-fit method\index{Kendall robust line-fit method} or Kendall–Theil robust line\index{Kendall–Theil robust line}... that is a very simple method for robustly fitting a line to a set of points (simple linear regression) that chooses the median $a_M$ of the slopes of all lines through pairs of two-dimensional sample points. For the estimate of the intercepts the USGS recommends the following calculation $b=y_M-a_Mx_M$. That's all... it's quite simple in fact.
		
		\item \textbf{Least Absolute Deviation (LAD) regression model} that use absolute values rather than square errors.  But absolute values are difficult to work with in mathematics (especially Calculus) as absolute values results in discontinuous derivatives that cannot be treated analytically
		
		\item \textbf{LOESS (LOcally Estimated Scatterplot Smoothing)} and \textbf{LOWESS (LOcally WEighted Scatterplot Smoothing)} are two strongly related nonparametric regression methods. These methods are purely numerical (does therefore not provide any unique formula) and are performed by fitting simple models to localized subsets (we speak then of "segmented regression"). In fact, one of the chief attractions of this method is that is not required to specify a global function of any form to fit a model to the data, only to fit segments of the data.
		
		\item \textbf{Multivariate adaptive regression splines (MARS)} (the term "MARS" is trademarked and licensed to Salford Systems. In order to avoid trademark infringements, many open source implementations of MARS are named "Earth"...) and as it names describes it well... is uses splines to interpolate and also extrapolate know data.
		
		\item \textbf{Bayesian linear regression model} is an approach to linear regression in which the statistical analysis is undertaken within the context of Bayesian inference, that means we have some prior knowledge about the regression coefficients and the error term.
		
		\item \textbf{Ridge regression model}\footnote{Technically speaking it's not a regression but a method of regularization! It's only a procedure affecting (sometimes improving) the method of fitting. It can be applied to almost any kind of regression.} that use a trick on the information matrix $X^TX$ (see further below) by adding a constant in it. But the constraints of usage are quite boring (no intercept and coefficients normalized) that make it quite difficult to interpret. In fact it is a special case of a regression family named \textbf{regularized linear regression models} and the underlying mathematical trick can be applied ton any general linear model (simple linear regression, logistic regressions, etc.).
		
		\item \textbf{Least Absolute Shrinkage and Selection Operator model (LASSO)}\footnote{Technically speaking it's also not a regression but a method of regularization!} that is based on the same ideas as that of the Ridge regression model and therefore also a special case of the Regularized Linear Regression models family (the only point that differs with the Ridge is that the theoretical model has somewhere a exponent equal to $1$ for LASSO when for it is equal to $2$ for Ridge).
		
		\item \textbf{Elastic net regression model}\footnote{Technically speaking it's also not a regression but a method of regularization!} is only a mathematical mixture of the LASSO and Ridge regression. This model therefore also belongs to the family of Regularized Linear Regression models.
		
		\item \textbf{Two-Stage Least Squares (2SLS) Regression} is used when the dependent variable's error terms are correlated with the independent variables. Additionally, it is useful when there are feedback loops in the model. In a first stage this regression builds instrumental variables that are uncorrelated with the error terms to compute estimated values of the problematic predictor(s), and then in a second stage uses those computed values to estimate a linear regression model of the dependent variable. This technique is used mainly in \textbf{Structural equation modelling (SEM)}.
		
		\item \textbf{Moderated Multiple Regression (MMR)} is used when one of the explicative variable has an interaction with another variable (product) that is continuous or dichotomous. The variable that interacts with the normal explicative variable is named the "moderator variable". A common error in practice is to make a confusion between Moderated Multiple Regression and a nested hierarchical model (ie Nested ANOVA).
		
		\item \textbf{Independent Component Regression (ICR)} is just a simple linear regression but based on a previous dimensional reduction of the exogenous variables based on an independent component analysis (ICA). That means the final model used the most explaining variance independent component as exogenous variables. Indeed, the reader must remember that PCA creates uncorrelated components but "uncorrelated" doesn't mean "independent" (!!!) as we have proved it in the section Statistics.
		
		\item \textbf{Support Vector Machine regression (SVMr)} use the concept of Support Vector Machine classification technique to evaluate existing values and interpolate new one.
		
		\item \textbf{Bootstrap or Jackknife regression model}\footnote{Technically speaking they are not "regression models" but methods related to the process of estimation!} using all previous models and then by resampling techniques (see further below) gives the possibility to have robust estimators that could in some situations be hard to get analytically. 
		
		\item \textbf{Isotonic regression} is a technique of fitting a free-form line to a sequence of observations under the following constraints: the fitted free-form line has to be non-decreasing (or non-increasing) everywhere, and it has to lie as close to the observations as possible.
	\end{enumerate}
	... and ... for a given number of these approaches we differentiate mathematical models taking into account the censored data and uncensored data, constraints coefficients (like the fact that they should be non-negative, like non-negative least square NNLS, or sum up to $1$ or any other specific constraint that we categorize under the name of Bounded-Variable Least Squares (BVLS) or Constrained Least Square). This makes a bunch of theories/models to study in final and this is why this subsection is one of the biggest of the whole book and that some models are given in some other sections of the book (especially in the Statistics section).
	
	\begin{tcolorbox}[title=Remark,colframe=black,arc=10pt]
	There are also submodels families. Like the Panel Ordinary Least Squares (Panel OLS), linear mixed models (LMM) that includes fixed effects models (FE models) and random effects models (RE models) and mixed effect models (ME models) and longitudinal regression models. There is also Nonlinear Mixed Model (NLMM) that includes all linear mixed models, and Generalized Linear Models (to not be confuse with "general linear models) that includes NLMM. And also convex regression, etc.
	\end{tcolorbox}
	
	Here is a figure the resumes quite well the different regressions and their relationships (thanks to Adrian Olszweski for having authorized us to reproduced it!):
	\begin{figure}[H]
		\centering
		\includegraphics[width=1.0\textwidth]{img/computing/regression_models.pdf}
		\caption[Synoptic summary and relationships between various regression models]{Synoptic summary and relationships between various regression models (author: Adrian Olszweski)}
	\end{figure}
	
	\begin{tcolorbox}[colback=red!5,borderline={1mm}{2mm}{red!5},arc=0mm,boxrule=0pt]
	\bcbombe Caution!!!!!!!!! There is a huge difference in science by only adjusting a mathematical function (having no real physical variables) to observations, and modelling observations with a physical model, ie including real physical variables and physics laws (the first one is naive, and most of time wrong, and the second one is obviously robust and most of time less wrong)!
	\end{tcolorbox}
	
	Finally, note that in the linear regression, explanatory variables form a linear expression but that does not mean they are themselves linear. Thus, if we consider the two expressions below:
	
	the first is linear in the parameters but the second is not!
	
	Finally, a word on a technique sometimes used for qualitative interpretation of the influential of explanatory variables trough their coefficients for simple or multivariate linear regressions:
	
	When the amplitudes of some explanatory variables (continuous!) have very different orders of magnitudes this raises a big problem of interpretation of the influence of each variable by reading their coefficient and also generates problems of calculations precision in algorithms because of differences in size order and thus also generates rounding errors!
	
	The traditional idea is then to center-reduce all the values of explanatory variables which helps greatly to interpretation of the influence of these variables (but we must then leave out the interpretation of the numerical value of the response variable). But take care to a common trap!!! Once the theoretical model obtained from been normalized (center-reduce) variables, the new values to explain should be obtained by having previously centered-reduced the new explanatory values but by subtracting the old average and reducing by the old standard deviations of respectively each of the injected explanatory variable in the model!
	
	\pagebreak
	\subsubsection{Univariate linear regression model}\label{simple linear regression}
	
	We will present here several algorithms (methods) useful and known in experimental science (we have already discussed about some of them during our study of statistics). The goal is to express the linear relation between two variables $x$ (explanatory variable) and $y$ (response variable) independently by a "\NewTerm{linear model LM}\index{linear model}" as simple as possible (otherwise it would take hundreds of pages to introduce the topic!).
	
	\begin{tcolorbox}[title=Remark,colframe=black,arc=10pt]
	Writing this book we have hesitated a long time to put regression techniques in the Statistics section. But because in practice the choice of the type of regression is empirical it has seem to us most convenient to put this subject here.
	\end{tcolorbox}	
	
	\textbf{Definition (\#\mydef):} In univariate regression we have:
	\begin{enumerate}
		\item[D1.] $x$ is the independent variable or "\NewTerm{explanatory variable}\index{explanatory variable}" also named "\NewTerm{covariate}\index{covariate}" or "\NewTerm{predictor}\index{predictor}" (in economics "\NewTerm{exogenous variable}"...). The $x$ values are set by the experimenter and are assumed to be known without error.
		
		\item[D2.] $y$ is the dependent variable or "\NewTerm{explained variable}\index{explained variable}" (e.g. the answer of the analyser) also named in economy "\NewTerm{endogenous variable}\index{endogenous variable}". $y$ values are most of time measured with an error (bias) of measurement. One goal of regression is to estimate precisely this error.
	\end{enumerate}
	We seek a relation of the form:
	
	This is the equation of a straight line (affine function), hence the term "\NewTerm{linear regression}\index{linear regression}" where $a$ is named in the study framework of regression techniques: "\NewTerm{regression coefficient}\index{regression coefficient}" instead of "slope" as seen in previous sections.
	
	In real life, linear relations are an exception because most phenomenon are nonlinear in reality and even non-continuous in certain situations... Furthermore, it is not because they are linear in a given interval of measurements that are still linear at a smaller-scale or larger scale (zoom bias)!
	
	However, in practice we make transformation to linearise functions either by elementary algebraic transformations like those used by spreadsheets softwares (e.g. Microsoft Excel) like for example the linearisation of a logarithmic function by making a simple change of variables:
	
	or for power and exponential functions by also making a small algebraic manipulation with the properties of logarithms as proved in the section of Functional Analysis (under the assumption that $a$ is strictly positive)\label{logarithmic and exponential linearization}:
	
	or by making Taylor series approximations (\SeeChapter{see section Sequences and Series page \pageref{taylor series}}).
	
	\begin{tcolorbox}[title=Remark,colframe=black,arc=10pt]
	If we seek to determine the value of $y$ for an unmeasured $x$ and lying beyond the original interval of measurement, then we speak of "\NewTerm{extrapolation}\index{extrapolation}" or in more complicated cases of "\NewTerm{forecast with prediction interval for $x$}\index{forecast with prediction interval}". We will see that further below.
	\end{tcolorbox}	
	
	\paragraph{Regression line}\mbox{}\\\\\
	In the common way to make an univariate linear regression of the type:
	
	there exist multiple methods.
	
	This first and most simple method in our point of view relies on the properties of covariance and mean (\SeeChapter{see section Statistics page \pageref{covariance}}) and is widely used among others in elementary finance (but in fact in any filed where there are some statistics).
	
	Consider $X, Y$ two variables, one of which depends on the other (often it is $y$ that depends on $x$). According to the covariance bilinearity property (\SeeChapter{see section Statistics page \pageref{bilinearity of the variance}}) we recall that we have from:
	
	the following relation:
	
	So it comes for the regression coefficient (we will reuse this relation during our study of yield of a portfolio according to Sharpe model in the section Economy):
	
	
	Either in a most explicit form which that we will use later (using what explicit relation determined in the section Statistics):
	
	To determine the intercept we use the properties of the expected mean as proved in the section Statistics:
	
	Therefore we have $b$ as:
	
	
	\paragraph{Ordinary Least Squares Method (OLSM)}\label{least squares method}\mbox{}\\\\\
	Due to the error on $y$, the experimental points, of coordinates $(x_k,y_k)$ do not lie exactly on the theoretical line. We can therefore find the equation of the experimental  line passing closest to these points.
	
	The "\NewTerm{least squares method LSM}\index{least squares method}" will be under the particular study we are interested in to look for the values of the parameters $a, b$ that minimize the sum of squares of residual $e_i$ (SSR: Sum of Squared Residuals) between the observed values $y_k$ and the theoretical calculated values $y_k^{\prime}$. We then speak sometimes about the "\NewTerm{least squares method of ordinate deviations}\index{least squares method of ordinate deviations}"...:
	
	where $n$ is the number of measured points and the theoretical values given by:
	
	Therefore written explicitly:
	
	This relation shows the sum of squared deviations as a function of the parameters $a, b$. When this function is minimal (extremal), the derivatives with respect to these parameters cancels:
	
	
	\begin{tcolorbox}[title=Remark,colframe=black,arc=10pt]
	This method of minimum research (optimization) is named "\NewTerm{method of Lagrange multipliers}\index{method of Lagrange multipliers}" in the world of Economy (we will detail this method further below). In our example SSR is the scalar value that will be used as Lagrange multiplier.
	\end{tcolorbox}	
	
	Therefore after simplification and rearrangement:
	
	The above system is named "\NewTerm{normal equations}\index{normal equations}". This is a linear system of two equations with two unknowns. Let us write to simplify the notation:
	
	The system becomes:
	
	From the second line we get without surprise:
	
	By replacing in the first line, we get:
	
	From there we get from the second line:
	
	Thus, the terms of the slope and intercept of the straight line equation are:
	
	The last two relations are used by a majority of spreadsheet softwares such as in the English version of Microsoft Excel 11.8346 when using the \texttt{REGRESSION( )} function. The term $b$ (the $y$-intercept) may be obtained directly with the \texttt{INTERCEPT( )} function and $a$ with the \texttt{SLOPE( )} function and the whole with the \texttt{LINEST( )} function.
	
	Here for information an interesting little list of some very practical case with this spreadsheet software (because requested a lot):
	\begin{itemize}
		\item For a straight line:\\
		
		$a$: \texttt{=SLOPE(y, x)}\\
		$b$: \texttt{= INTERCEPT(y, x)}
		
		\item For a logarithmic function (we see here again the change of variable given at the before):\\
		
		$a$: = \texttt{INDEX(LINEST(y, LN (x)), 1)}\\
		$b$: = \texttt{INDEX(LINEST(y, LN (x)), 1, 2)}
		
		\item For a power function (once again wee see the change of variable given at the beginning):\\
		
		$a$: = \texttt{EXP(INDEX(LINEST(LN(y),LN(x),,),1,2))}\\
		$b$: = \texttt{INDEX(LINEST(LN(y),LN(x),,),1)}
		
		\item For an exponential function (we also find the change of variable given at the beginning):\\
		
		$a$: = \texttt{EXP(INDEX(LINEST(LN(y),x),1,2))}\\
		$b$: = \texttt{INDEX(LINEST(LN(y),x),1)}\\
	\end{itemize}
	
	\begin{tcolorbox}[title=Remark,colframe=black,arc=10pt]
We must keep in mind that the line of least squares, which can best summarize the cloud of observations points by minimizing SSR, necessarily passes through the center of gravity of the cloud, that is to say, by an average point that corresponds rarely to an observation (mean average of abscissas and ordinates).
	\end{tcolorbox}
	
	\pagebreak
	\paragraph{Univariate Regression Variance Analysis}\label{univariate regression variance analysis}\mbox{}\\\\\
	Before starting it is important that the reader abandons immediately the possible reflex that would be to try to bring by successive analogies the regression ANOVA we will see now to the categorical ANOVA we have study in the section Statistics!
	
	either in discrete form:
	
	as well as by the construction of the least square method we have the following relation:
	
	Now we assume that each measured value is attached by a residual error such that:
	
	Either by subtracting the last two relations:
	
	Now let us go through an intermediate result. Remember that we obtained earlier:
	
	And therefore:
	
	By replacing $b$ by its value:
	
	We therefore get:
	
	Multiplying the second line above by $\bar{x}$ and by subtracting from the first we get:
	
	Therefore after rearrangement:
	
	Now we go back to:
	
	If we put it all square and summing for all observations, we get:
	
	Therefore:
	
	But we have shown just before the double product was equal to zero. Therefore:
	
	This last relation is named "\NewTerm{ANOVA equation}\index{ANOVA!ANOVA equation}" or "\NewTerm{variance analysis equation}\index{ANOVA!variance analysis equation}". In fact, it is the sums of squares. We would need to divided it by $n$ to obtain biased variances.
	
	This last relation is often written:
	
	where SST is the "\NewTerm{total sum of squares}\index{total sum of squares}", SSE the "\NewTerm{sum of square errors}\index{sum of square errors}" and SSR "\NewTerm{sum of squares residuals}\index{sum of squares residuals}".
	
	Let us note now the estimated $y_k$ that minimize the errors such that the error is null in a different way and let us named that the "\NewTerm{a priori linear model}\index{a priori linear model}":
	
	Hence the equality we will reuse several times (it is just previous relations without error term):
	
	It is indeed important in practice to differentiate the a priori model that does not take into account the errors of the real model that does!
	
	Because of the previous equality the relation:
	
	Can be written:
	
	More explicitly:
	
	This last relation can be represented graphically as follows:
	\begin{figure}[H]
		\centering
		\includegraphics[width=1.0\textwidth]{img/computing/sst_sse_ssr_detailed.jpg}
		\caption{Graphical representation of respectively SST, SSE, SSR}
	\end{figure}

The last relation is sometimes denoted also in the literature in a most educational way as follows:

	which is just another way to write the variance decomposition (implicit variance):
	
	
	and it then comes immediately the relation sometimes used in practice to calculate residues (knowing the calculated values and measured values):
	
	It is important to remember that the above relation between SST, SSE and SSR are valid only in the case of a linear model!
	
	It is also important to note that in this particularly variance decomposition we have:
	
	Remember now that we have proved in the section Statistics, we found that the correlation coefficient was given (defined) by\label{correlation coefficient numerical methods}:
	
	Or else since we have shown above that (remember that the indicated variance is an estimated variance in practice!):
	
	We can therefore write the correlation coefficient in the form\label{slope and correlation coefficient relation}:
	
	So we deduce from this using the relations established in the section of statistics:
	
	Remember that we proved above that:
	
	Therefore:
	
	that is to say\label{correlation linear regression model}
	
	\begin{tcolorbox}[title=Remark,colframe=black,arc=10pt]
This formulation of the correlation coefficient is extremely useful because, unlike the statistical formulation, the latter generalizes immediately to the multiple linear regression we will see a further below.
	\end{tcolorbox}
	and in the framework of regression models here are some typical cases of the value of the linear correlation coefficient with the first two lines and non-linear for the third line:
\begin{figure}[H]
	\centering
	\includegraphics[scale=0.6]{img/arithmetics/correlation_coefficients.jpg}
	\caption[Some values of the linear correlation coefficient]{Some values of the linear correlation coefficient (source: Wikipedia)}
\end{figure}
	Finally let us indicate that we also find very often the linear correlation coefficient as follows in softwares and literature:
	
	The last form highlights better that if the sum of the squares of residues SSR residues is zero, the measures are perfectly modelised by a linear relation in the range of study considered.
	
	\begin{tcolorbox}[title=Remark,colframe=black,arc=10pt]
	Some softwares communicate the "\NewTerm{predicted R-squared}\index{predicted R-squared}" defined as:
	
	 where PRESS, the "\NewTerm{predicted residual sum of squares}\index{predicted residual sum of squares}", is a form of cross-validation (see further below page \pageref{cross-validation}) used in regression analysis to provide a summary measure of the fit of a model to a sample of observations that were not themselves used to estimate the model. It is calculated as the sums of squares of the prediction residuals for those observations:
	 
	Therefore PRESS differs from the sum of squares of the residual error in that each fitted value, $\hat{y}_k$, for PRESS is obtained from the remaining $n-1$ observations.\\
	
	Given this procedure, the PRESS statistic can be calculated for a number of candidate model structures for the same dataset, with the lowest values of PRESS indicating the best structures. Models that are over-parametrised (over-fitted) would tend to give small residuals for observations included in the model-fitting but large residuals for observations that are excluded.\\
	
	Modern textbooks use the average of the PRESS such that we define the "\NewTerm{leave-one-out cross-validation score}\index{leave-one-out cross-validation score}\label{loocv}" (LOOCV) or "\NewTerm{ordinary cross-validation}\index{ordinary cross-validation}" (OCV):
	
	\end{tcolorbox}
	
	Finally, note that the ordinate value is not involved in the value of the correlation coefficient since (bilinearity property of covariance as proved in the section Statistics):
	
	
	\pagebreak
	\paragraph{F-test for Regression (significance test for linear regression)}\label{dummy variable regression}\label{anova for linear regression}\mbox{}\\\\\
	Now comes a part that interests us mainly for practical scientific laboratories tests!!! 
	
	If your graduate statistical training was anything like mine, you learned ANOVA in one class and Linear Regression in another. My professors would often say things like "\textit{ANOVA is just a special case of Regression}" but give vague answers when pressed. Let us see why!
	
	Let us recall that in the section of Statistics we proved that for the one controlled factor ANOVA, the test of equality of means (through the use of the variances) was written (we just change the letter $x$ to $y$ to avoid confusion for the developments that will follow) is given by:
	
	which is used to compare, for example, the means of the cash-flows of two supposed independent chosen months (ie with $k=2$) over $10$ years (obviously in the case $k=2$ we could use a Student test if the conditions are of course satisfied!) supposing that for every year for the given month the values are normally distributed:
	\begin{table}[H]
	\centering
		\definecolor{gris}{gray}{0.85}
			\begin{tabular}{|c|c|}
				\hline
				\cellcolor{black!30}\textbf{January} & \cellcolor{black!30}\textbf{February} \\ \hline
				$304$ & $280$ \\ \hline
				$284$ & $303$ \\ \hline
				$290$ & $294$ \\ \hline
				$310$ & $270$ \\ \hline
				$320$ & $276$ \\ \hline
				$270$ & $310$ \\ \hline
				$309$ & $290$ \\ \hline
				$293$ & $312$ \\ \hline
				$315$ & $260$ \\ \hline
				$301$ & $325$ \\ \hline
		\end{tabular}
	\end{table}
	Therefore a simple ANOVA with one controlled factor will give according to the calculations proven in the section of Statistics (see page \pageref{anova one way fixed factor}) and using a spreadsheet software like Microsoft Excel 14.0.7106 the following results:
	\begin{figure}[H]
		\centering
		\includegraphics{img/arithmetics/anova_one_factor_for_comparison_with_regression_anova_excel.jpg}
	\end{figure}
	With for recall:
	\begin{table}[H]\small
		\renewcommand{\arraystretch}{1.2}
		\centering
		\begin{tabular}{llcccc}\hline
		\textbf{Source} & \textbf{Sum of squares (SSE)} & $\chi^2$ \textbf{df} & \textbf{Mean squares} & $F$ & \textbf{Critical} $F$\\ \hline
		Inter-Class & $Q_A=\displaystyle\sum_{i}n_i\left(\bar{x}_{i}-\bar{\bar{x}}\right)^2$ & $k-1$ & $\text{MSk}=\displaystyle\dfrac{Q_A}{k-1}$ &
		$\displaystyle\dfrac{\text{MSk}}{\text{MSE}}$ & $P(F> F_{k-1,N-k})$ \\
		Intra-Class & $Q_R=\displaystyle\sum_{ij}\left(x_{ij}-\bar{x}_i\right)^2$ & $N-k$ & $ \text{MSE}=\displaystyle\dfrac{Q_R}{N-k}$  & & \\
		Total & $Q_T=\displaystyle\sum_{ij}\left(x_{ij}-\bar{\bar{x}}\right)^2$ & $N-1$ & & &\\ \hline
		\end{tabular}
		\caption[]{One way fixed factor ANOVA table}
	\end{table}
	
	But you might think: Why we go back on such an example ????

	Well simply because we can very well build a linear regression model of the cash-flows of these two months, which we had not mentioned in the Statistics section! Thus, when we have an ANOVA with a controlled factor, nothing avoid us from making a linear regression of the data with binary explanatory variables (and vice versa)! To do this, it is enough that we rewrite the table above in the following form (technique named "dummy coding" or "one hot encoding"):
	\begin{table}[H]
	\centering
		\definecolor{gris}{gray}{0.85}
			\begin{tabular}{|c|c|c|c|}
				\hline
				\cellcolor{black!30}\textbf{Observation} & \cellcolor{black!30}\textbf{Cash-Flow $\pmb{y_i}$} & \cellcolor{black!30}\textbf{January $\pmb{x_{i1}}$} & \cellcolor{black!30}\textbf{February $\pmb{x_{i2}}$} \\ \hline
				$1$ & $304$ & $1$ & $0$ \\ \hline
				$2$ & $284$ & $1$ & $0$ \\ \hline
				$3$ & $290$ & $1$ & $0$ \\ \hline
				$4$ & $310$ & $1$ & $0$ \\ \hline
				$5$ & $320$ & $1$ & $0$ \\ \hline
				$6$ & $270$ & $1$ & $0$ \\ \hline
				$7$ & $309$ & $1$ & $0$ \\ \hline
				$8$ & $293$ & $1$ & $0$ \\ \hline
				$9$ & $315$ & $1$ & $0$ \\ \hline
				$10$ & $301$ & $1$ & $0$ \\ \hline
				$11$ & $280$ & $0$ & $1$ \\ \hline
				$12$ & $303$ & $0$ & $1$ \\ \hline
				$13$ & $294$ & $0$ & $1$ \\ \hline
				$14$ & $270$ & $0$ & $1$ \\ \hline
				$15$ & $276$ & $0$ & $1$ \\ \hline
				$16$ & $310$ & $0$ & $1$ \\ \hline
				$17$ & $290$ & $0$ & $1$ \\ \hline
				$18$ & $312$ & $0$ & $1$ \\ \hline
				$19$ & $260$ & $0$ & $1$ \\ \hline
				$20$ & $325$ & $0$ & $1$ \\ \hline
		\end{tabular}
	\end{table}
	Obviously we not always have in reality binary $(0,1)$ explanatory values but we can always normalize the control variables to fall back on such a situation!
	
	This necessarily can be summarized to:
	\begin{table}[H]
	\centering
		\definecolor{gris}{gray}{0.85}
			\begin{tabular}{|c|c|c|c|}
				\hline
				\cellcolor{black!30}\textbf{$\pmb{n_i}$} & \cellcolor{black!30}\textbf{Mean cash flow $\pmb{\bar{y}_i}$} & \cellcolor{black!30}\textbf{January $\pmb{x_{i1}}$} & \cellcolor{black!30}\textbf{February $\pmb{x_{i2}}$} \\ \hline
				$10$ & $299.11$ & $1$ & $0$ \\ \hline
				$10$ & $292.00$ & $0$ & $1$ \\ \hline
		\end{tabular}
	\end{table}
	Therefore the regression model associated with this ANOVA with $1$ control factor at $2$ levels can therefore be written:
	
	\begin{tcolorbox}[title=Remark,colframe=black,arc=10pt]
	Notice that this last relation can also be written:
	
	And here maybe you recognize something well known to us when we have studies all the ANOVA!!! So as we then see, checking that the means are $\hat{\mu}_i$ all equal or not in the ANOVA such that $\mu_1=\mu_2=\ldots=\mu$ is in the case of the regression equivalent to check that $\beta_1=\beta_2=0$. So under this form, the hypotheses are then:
	
	This is what we named the "\NewTerm{$F$-test of Overall Significance in Regression Analysis}\index{$F$-test of Overall Significance in Regression Analysis}" (unlike $T$-tests that can assess only one regression coefficient at a time as we will see further below page \pageref{variable importance GML}).
	\end{tcolorbox}
	But, with the preceding relation and the table summarized above, we have a system of two equations with three unknowns ... which is obviously insoluble for a least squares approach. Therefore, the idea consists in sacrificing one of the explanatory variables as (generalizable to more than two variables obviously):
	
	and therefore (special choice):
	
	and then we have indeed two equations with two unknowns:
	
	This explains the reason why statistical softwares will always give the coefficient of one of the two binary explanatory variables as zero (which obviously can be problematic in some cases and therefore it is enough to force the ordinate at the origin to be zero to have the two non-zero coefficients since then we fall back on a system of two equations with two unknowns). In the case of $k$ binary (dichotomous) explanatory variables, there will be $k - 1$ whose coefficients are non-zero (since one can always be explained by all others).
	\begin{tcolorbox}[title=Remark,colframe=black,arc=10pt]
	The use of ANOVA, you probably guess, is only feasible if the residues are almost normally distributed and homoscedastic... This is why statistical softwares have outputs giving such analysis when running a F-test for regression. 
	\end{tcolorbox}
	Finally, all this to say that the ANOVA is only a special case (with binary explanatory variables rather than continuous one and let us recall that $\mathbb{N}\in\mathbb{R}$ not the inverse!) of the linear regression.

	Let us notice then that we can write:
	
	and as they are either $0$ or $1$ (or normalized to be such as). Then there remain only the terms $x_{ij}\hat{\beta}_j$ where $x_{ij}\neq 0$ in a quantity that for each $j$ we will denote by $n_j$. It comes then since we only keep the $x_{ij}=1$ (for the last equality if it is not obvious the reader can make an example on a sheet of paper with a simple numerical application!):
	
 	with obviously\footnote{In our above example $N=20$ and $n_1=n_2=10$} $\sum_{j=1}^k n_j=N$.

	By the method of the ordinary least squares and the values that can take the $x_{ij}$ we quickly see that the coefficients (parameters) of the regression will be given by (we can detail again on reader request):
	
	and therefore:
	
	and as in our case in comparison with the one fixed factor ANOVA we have the obvious correspondences:
	
	it follows, therefore, that there is a correspondence between the numerator of the Fisher test of ANOVA and the equivalent expression of the linear regression (and therefore with the same number of degrees of freedom) that brings us to write:
	
	But it remains for us to find the equivalent also of the denominator for the regression. For this we will proceed by similarity. Let us call that in the one fixed factor ANOVA we have proved that:
	\begin{table}[H]
		\centering
		\begin{tabular}{ccccc}
		$Q_T$ & $=$ & $Q_A$ & $+$ & $Q_R$  \\
		$\displaystyle\sum_i (y_{ij}-\bar{\bar{y}})^2$ & $=$ & $\displaystyle\sum_i n_i(\bar{y}_i-\bar{\bar{y}})^2$ & $+$  & $\displaystyle\sum_{ij}(y_{ij}-\bar{y}_i)^2$  \\
		 $N-1$ & $=$ & $k-1$ & $+$ & $\displaystyle\sum_i (n_i-1)$  \\
		\end{tabular}
	\end{table}
	and let us recall that for the linear regression, we have proved earlier above (with the corresponding notations in usage):
	\begin{table}[H]
		\centering
		\begin{tabular}{ccccc}
		SST & $=$ & SSE & $+$ & SSR  \\
		$\displaystyle\sum_i (y_{i}-\bar{y})^2$ & $=$ & $\displaystyle\sum_i (\hat{y}_i-\bar{y})^2$ & $+$  & $\displaystyle\sum_i(y_i-\hat{y}_i)^2$  \\
		\end{tabular}
	\end{table}
	From what we have seen above we know that degrees of freedom of $\sum_i (\hat{y}_i-\bar{y})^2$ are $k-1$. If follows immediately that for $\sum_i (y_{i}-\bar{y})^2$ the degrees of freedom are $N-1$. Then we can write:
	\begin{table}[H]
		\centering
		\begin{tabular}{ccccc}
		SST & $=$ & SSE & $+$ & SSR  \\
		$\displaystyle\sum_i (y_{i}-\bar{y})^2$ & $=$ & $\displaystyle\sum_i (\hat{y}_i-\bar{y})^2$ & $+$  & $\displaystyle\sum_i(y_i-\hat{y}_i)^2$  \\
		$N-1$ & $=$ & $k-1$ & $+$ & $????$  \\
		\end{tabular}
	\end{table}
	It then follows that the degrees of freedom of the sum of the squares residuals is then of $N-k$ such that:
	\begin{table}[H]
		\centering
		\begin{tabular}{ccccc}
		SST & $=$ & SSE & $+$ & SSR  \\
		$\displaystyle\sum_i (y_{i}-\bar{y})^2$ & $=$ & $\displaystyle\sum_i (\hat{y}_i-\bar{y})^2$ & $+$  & $\displaystyle\sum_i(y_i-\hat{y}_i)^2$  \\
		$N-1$ & $=$ & $k-1$ & $+$ & $N-k$  \\
		\end{tabular}
	\end{table}
	Sum of the squares of residues that it is customary to write (useful for later !!!):
	
	and that is given in the output of some statistical softwares (sadly!) under the misnamed "\NewTerm{residual standard error}"\index{residual standard error} (instead than "\NewTerm{residual standard deviation}"):
		 
	As $Q_T$ doesn't appear in the Fisher test there is no a priori reason that the equivalent which is SST for regression appears there. By elimination, the correspondence is then immediate:
	
	The Fisher test then becomes for the linear regression:
	
	and is often the most famous "\NewTerm{omnibus test for OLS regression}"\index{omnibus test for OLS regression} (ie that a significant percentage of statistical softwares give below the diagnostic table of the regression analysis!).
	
	It follows than that as for the one-fixed factor ANOVA, with linear regression, we can also make a table of the ANOVA (\SeeChapter{see section Statistics page \pageref{anova}}) as we will see later with an example!
	
	Now, let us prove a common and important form of this last relation. We have proved earlier above that the correlation coefficient could also be written in the form:
	
	However, let us notice that:
	
	However, let us recall once again that:
	
 	Thus explicitly as we have just seen:
	
	Therefore we can write:
	
	What we usually can found in some textbooks in the following form:
	
	We thus see that if the coefficient of determination $R^2$ is large (close to $1$) then the value of $F$ is large, the linear model will be considered as explaining significantly well the variable explained with respect to the explanatory variable (but this still doesn't mean that there is causality!).
	
	Let us come back to our companion example! A simple ordinary least squares regression with a spreadsheet software like Microsoft Excel of the below table:
	\begin{table}[H]
	\centering
		\definecolor{gris}{gray}{0.85}
			\begin{tabular}{|c|c|c|}
				\hline
				\cellcolor{black!30}\textbf{Observation} & \cellcolor{black!30}\textbf{Cash-Flow $\pmb{y_i}$} & \cellcolor{black!30}\textbf{January $\pmb{x_{i1}}$} \\ \hline
				$1$ & $304$ & $1$\\ \hline
				$2$ & $284$ & $1$ \\ \hline
				$3$ & $290$ & $1$ \\ \hline
				$4$ & $310$ & $1$ \\ \hline
				$5$ & $320$ & $1$ \\ \hline
				$6$ & $270$ & $1$ \\ \hline
				$7$ & $309$ & $1$ \\ \hline
				$8$ & $293$ & $1$ \\ \hline
				$9$ & $315$ & $1$ \\ \hline
				$10$ & $301$ & $1$ \\ \hline
				$11$ & $280$ & $0$ \\ \hline
				$12$ & $303$ & $0$ \\ \hline
				$13$ & $294$ & $0$ \\ \hline
				$14$ & $270$ & $0$ \\ \hline
				$15$ & $276$ & $0$ \\ \hline
				$16$ & $310$ & $0$ \\ \hline
				$17$ & $290$ & $0$ \\ \hline
				$18$ & $312$ & $0$ \\ \hline
				$19$ & $260$ & $0$ \\ \hline
				$20$ & $325$ & $0$ \\ \hline
		\end{tabular}
	\end{table}
	gives (we make the choice not to force the ordinate at the origin and to take January as an explanatory variable with coefficient not zero otherwise we will not find the value of the calculation of the classic ANOVA seen earlier above):
	\begin{figure}[H]
		\centering
		\includegraphics[scale=0.8]{img/computing/anova_approach_univariate_regression_plot_ms_excel.jpg}
		\caption{Graphical representation of an univariate regression with the ANOVA approach in Microsoft Excel 14.0.7177}
	\end{figure}
	that is to say:
	
	We then have:
	\begin{table}[H]
		\centering
		\definecolor{gris}{gray}{0.85}
		\resizebox{\textwidth}{!}{\begin{tabular}{|c|c|c|c|c|c|}
		\hline
		\cellcolor{black!30}\textbf{Observation} & \cellcolor{black!30}\textbf{Cash flow $\pmb{y_i}$} & \cellcolor{black!30}\textbf{Cash flow $\pmb{\hat{y}_i}$} & \cellcolor{black!30}\textbf{SSR $\pmb{(y_i-\hat{y}_i)^2}$} & \cellcolor{black!30}\textbf{SST $\pmb{(y_i-\bar{y})^2}$} & \cellcolor{black!30}\textbf{SSE $\pmb{(\hat{y}_i-\bar{y})^2}$}  \\ \hline
		$1$ & $304$ & $299.6$ & $19.36$ & $67.24$ & $14.44$ \\ \hline
		$2$ & $284$ & $299.6$ & $243.36$ & $139.24$ & $14.44$ \\ \hline
		$3$ & $290$ & $299.6$ & $92.16$ & $33.64$ & $14.44$ \\ \hline
		$4$ & $310$ & $299.6$ & $108.16$ & $201.64$ & $14.44$ \\ \hline
		$5$ & $320$ & $299.6$ & $416.16$ & $585.64$ & $14.44$ \\ \hline
		$6$ & $270$ & $299.6$ & $876.16$ & $665.64$ & $14.44$ \\ \hline
		$7$ & $309$ & $299.6$ & $88.36$ & $174.24$ & $14.44$ \\ \hline
		$8$ & $293$ & $299.6$ & $43.56$ & $7.84$ & $14.44$ \\ \hline
		$9$ & $315$ & $299.6$ & $237.16$ & $368.64$ & $14.44$ \\ \hline
		$10$ & $301$ & $299.6$ & $1.96$ & $27.04$ & $14.44$ \\ \hline
		$11$ & $280$ & $292$ & $144$ & $249.64$ & $14.44$ \\ \hline
		$12$ & $303$ & $292$ & $121$ & $51.84$ & $14.44$ \\ \hline
		$13$ & $294$ & $292$ & $4$ & $3.24$ & $14.44$ \\ \hline
		$14$ & $270$ & $292$ & $484$ & $665.64$ & $14.44$ \\ \hline
		$15$ & $276$ & $292$ & $256$ & $392.04$ & $14.44$ \\ \hline
		$16$ & $310$ & $292$ & $324$ & $201.64$ & $14.44$ \\ \hline
		$17$ & $290$ & $292$ & $4$ & $33.64$ & $14.44$ \\ \hline
		$18$ & $312$ & $292$ & $400$ & $262.44$ & $14.44$ \\ \hline
		$19$ & $260$ & $292$ & $1024$ & $1281.64$ & $14.44$ \\ \hline
		$20$ & $325$ & $292$ & $1048$ & $852.64$ & $14.44$ \\ \hhline{|=|=|=|=|=|=|}
		&  & \textbf{Total} & $\mathbf{5976.4}$ & $\mathbf{6265.2}$ & $ \mathbf{288.8}$\\ \hline
		\end{tabular}}
	\end{table}
	with obviously:
	
	The calculations of all the classical ANOVA terms of the regression then give still with the same version of Microsoft Excel (the reader can verify by hand using the relations proved earlier above that we find the values given by this spreadsheet software):
	\begin{figure}[H]
		\centering
		\includegraphics{img/computing/anova_for_regression_in_ms_excel.jpg}
	\end{figure}
	Or more explicitly:
	\begin{table}[H]\small
		\renewcommand{\arraystretch}{1.2}
		\centering
		\begin{tabular}{llcccc}\hline
		\textbf{Source} & \textbf{Sum of squares (SSE)} & $\chi^2$ \textbf{df} & \textbf{Mean squares} & $F$ & \textbf{Critical} $F$\\ \hline
		Inter-Class & $Q_A=\displaystyle\sum_{i}\left(\hat{y}_{i}-\bar{y}\right)^2$ & $k-1$ & $\text{MSk}=\displaystyle\dfrac{Q_A}{k-1}$ &
		$\displaystyle\dfrac{\text{MSk}}{\text{MSE}}$ & $P(F> F_{k-1,N-k})$ \\
		Intra-Class & $Q_R=\displaystyle\sum_{i}\left(y_i-\hat{y}_i\right)^2$ & $N-k$ & $ \text{MSE}=\displaystyle\dfrac{Q_R}{N-k}$  & & \\
		Total & $Q_T=\displaystyle\sum_{i}\left(y_i-\bar{y}\right)^2$ & $N-1$ & & &\\ \hline
		\end{tabular}
		\caption{ANOVA table for regression}
	\end{table}
	\begin{tcolorbox}[title=Remarks,colframe=black,arc=10pt]
	Notice that in the above ANOVA table we recognize something known to us:
	
	That is... the $\text{SEE}^2$, where for recall SEE is named the "standard error of estimate" or "standard regression error" (see page \pageref{standard error of estimate})! This is why in all statistical softwares you can simply rely the standard error of estimate of the regression with the intra-class mean sum of square (ie mean residuals errors) just by taking the square root of that latter such that: $\text{SEE}=\sqrt{\text{MSE}}$.
	\end{tcolorbox}
	And running the one-fixed factor ANOVA we had already introduced a little earlier above we have for comparison:
	\begin{figure}[H]
		\centering
		\includegraphics{img/arithmetics/anova_one_factor_for_comparison_with_regression_anova_excel.jpg}
	\end{figure}
	With for recall (again!) its general table:
	\begin{table}[H]\small
		\renewcommand{\arraystretch}{1.2}
		\centering
		\begin{tabular}{llcccc}\hline
		\textbf{Source} & \textbf{Sum of squares (SSE)} & $\chi^2$ \textbf{df} & \textbf{Mean squares} & $F$ & \textbf{Critical} $F$\\ \hline
		Inter-Class & $Q_A=\displaystyle\sum_{i}n_i\left(\bar{x}_{i}-\bar{\bar{x}}\right)^2$ & $k-1$ & $\text{MSk}=\displaystyle\dfrac{Q_A}{k-1}$ &
		$\displaystyle\dfrac{\text{MSk}}{\text{MSE}}$ & $P(F> F_{k-1,N-k})$ \\
		Intra-Class & $Q_R=\displaystyle\sum_{ij}\left(x_{ij}-\bar{x}_i\right)^2$ & $N-k$ & $ \text{MSE}=\displaystyle\dfrac{Q_R}{N-k}$  & & \\
		Total & $Q_T=\displaystyle\sum_{ij}\left(x_{ij}-\bar{\bar{x}}\right)^2$ & $N-1$ & & &\\ \hline
		\end{tabular}
		\caption[]{One way fixed factor ANOVA table}
	\end{table}
	We see then the obvious similarity that there are with the two approaches!!! This is where some people say sometimes: "\textit{running a regression or an ANOVA are two equivalent thinks (assumed under some given conditions)}" as in the ANOVA, the categorical variable is effect coded, which means that each category's mean is compared to the grand mean. In the regression, the categorical variable is dummy coded, which means that each category intercept is compared to the reference group intercept! Since the intercept is defined as the mean value when all other predictors $= 0$, and there are no other predictors, the three intercepts are just means.
	
	Indeed, we have:
	
	and:
	
	
	\subparagraph{Test for Lack of Fit}\label{test for lack of fit}\mbox{}\\\\\
	The "\NewTerm{test for lack-of-fit}\index{test for lack of fit}\index{lack of fit}" compares the variation around the model with "pure" variation within replicated observations (some statistical softwares use this analysis only for the adequacy of quadratic response surface and not for simple linear models...). In particular, if there are $n_i$ replicated observations $y_{i1}, ... ,y_{in_i}$ of the response all at the same values $\vec{x}_i$ (vector of dimension $n$) of the factors, then we can predict the true response at $\vec{x}_i$ either by using the predicted value $\hat{y}_i$ based on the model or by using the mean $\bar{y}_i$ of the replicated values. The test for lack-of-fit decomposes the residual error into a component due to the variation of the replications around their mean value (the "\NewTerm{pure error}\index{pure error}"), and a component due to the variation of the mean values around the model prediction (the "\NewTerm{lack of fit}" also named the "\NewTerm{bias error}\index{bias error}"):
	
	or written differently:
	
	\begin{tcolorbox}[title=Remarks,colframe=black,arc=10pt]
	\textbf{R1.} The reader should easily notice that without repeated measurements, $\text{SSPE}=0$ and hence we cannot conduct a lack-of-fit analysis!\\
	
	\textbf{R2.} Obviously for one $x_i$ we have multiple $y_i$, hence the notation $y_{ij}$. But the model for a given $x_i$ gives only one and only one $\hat{y}_i$. Hence $\forall j$ and a given $i$, we have $\hat{y}_{ij}=\hat{y}_i$.
	\end{tcolorbox}	
	If the model is adequate, then both components estimate the nominal level of error; however, if the bias component of error is much larger than the pure error, then this constitutes evidence that there is significant lack of fit. The concept can be illustrated as following:
	\begin{figure}[H]
		\centering
		\includegraphics[scale=1]{img/computing/lack_of_fit_pure_error.jpg}
		\caption{Lack of fit and pure error}
	\end{figure}
	
	
	\begin{itemize}
		\item As before, the degrees of freedom associated with SSE (ie $Q_R$) is as we know very well given by $n(k-1)=N-k$ (keep in mind that the $k$ comes from the fact that you estimate $k$ parameters - the slopes plus the intercept - whenever you fit a line to a set of data.)
	
		\item The degrees of freedom associated with SSLF is $n-k$, where $c$ denotes the number of distinct $x$ values we have.
	
		\item The degrees of freedom associated with SSPE is $nk-n=N-n$, where again $n$ denotes the number of distinct $x$ values we have.
	\end{itemize}
	Therefore the degrees of freedom breakdown as:
	
	It then follows that the statistic:
	
	has an $F$-distribution with the corresponding number of degrees of freedom in the numerator and the denominator, provided that the model is correct. 
	
	To understand this test keep in mind that under the null hypothesis:
	\begin{itemize}
		\item The sum of squares due to pure error, divided by the error variance $\sigma_\varepsilon^2$, has a chi-squared distribution with $N-n$ degrees of freedom;

		\item The sum of squares due to lack of fit, divided by the error variance $\sigma_\varepsilon^2$, has a chi-squared distribution with $n-k$ degrees of freedom 

		\item The two sums of squares are probabilistically independent
	\end{itemize}
	If the model is wrong, then the probability distribution of the denominator is still as stated above, and the numerator and denominator are still independent. But the numerator then has a noncentral chi-squared distribution (\SeeChapter{see section Statistics page \pageref{noncentral chi-square distribution}}) and consequently the quotient as a whole has a noncentral $F$-distribution.
	
	One uses this $F$-statistic to test the null hypothesis that there is no lack of linear fit. Since the non-central F-distribution is stochastically larger than the (central) $F$-distribution, one rejects the null hypothesis if the F-statistic is larger than the critical $F$ value.
	
	So we complete the previous analysis of variance table using the following relations:
	\begin{table}[H]
		\centering
		\begin{tabular}{|l|l|l|l|l|}
		\hline
		\rowcolor[HTML]{9B9B9B} 
		\textbf{Source} & $\chi^2$ \textbf{df} & \textbf{SS} & \textbf{Means squares} & \textbf{$\pmb{F}$} \\ \hline
		Regression & $k-1$ & $\text{SSR}=\displaystyle\sum_{i=1}^{n}n_i(\hat{y}_{i}-\bar{y})^2$ & $\text{MSR}=\displaystyle\dfrac{\text{SSR}}{k-1}$ & $F=\displaystyle\dfrac{\text{MSR}}{\text{MSE}}$ \\ \hline
		Residual error & $N-k$ & $\text{SSE}=\displaystyle\sum_{i=1}^{n}\sum_{j=1}^{n_i}(y_{ij}-\hat{y}_{i})^2$ & $\text{MSE}=\displaystyle\dfrac{\text{SSE}}{N-k}$ &  \\ \hline
		Lack of Fit & $n-k$ & $\text{SSLF}=\displaystyle\sum_{i=1}^{n}n_i(\bar{y}_{i}-\hat{y}_{i})^2$ & $\text{MSLF}=\dfrac{\text{SSLF}}{n-k}$ & $F=\dfrac{\text{MSLF}}{\text{MSPE}}$ \\ \hline
		Pure error & $N-n$ & $\text{SSPE}=\displaystyle\sum_{i=1}^{n}\sum_{j=1}^{n_i}(y_{ij}-\bar{y}_{i})^2$ & $\text{MSPE}=\dfrac{\text{SSPE}}{N-k}$ &  \\ \hline
		Total & $N-1$ & $\text{SSTO}=\displaystyle\sum_{i=1}^{n}\sum_{j=1}^{n_i}(y_{ij}-\bar{y})^2$ &  &  \\ \hline
		\end{tabular}
		\caption{Lack-of-Fit ANOVA table}
	\end{table}
	The reader must be careful with the "Total" in the above table (last row). Indeed, the total must not take into account the row with the degrees of freedom $n-k$ and $N-n$ as their are the decomposition of the row with the $N-k$.
	
	\pagebreak
	\subsubsection{Univariate linear regression Gaussian Model}\label{univariate linear regression gaussian model}
	We will assume that for an individual $k$ picked randomly from the population, $x_k$ is known without error, and that $y_k$ is a realization of a random variable that we will now denote $Y_k$ and the theoretical least squares regression line will be written now under the general form of a "\NewTerm{stochastic linear model}\index{stochastic linear model}":
	
	where $\varepsilon_k$ is assumed identically distributed and independent residue (no correlation) for each point $k$ according to a centered Normal distribution (zero mean and standard deviation $\sigma$ for all $k$) such as $\forall i\neq j$:

	So in other words\footnote{The conditions $\text{cov}(\varepsilon_i,\varepsilon_j)=0$, and $\text{E}(\varepsilon_k)=0$ and $\text{V}(\varepsilon_k)=\sigma_\varepsilon^2$ are named the "\NewTerm{Gauss-Markov assumptions}" and are at the basis of the "\NewTerm{Gauss-Markov Linear Model}\index{Gauss-Markov linear model}" (GMLM).}:
	
	where the residue is defined by the difference between the theoretical ordinates $Y_k$ (considered as random variable) and the measured (experimental) ordinates $y_k$:
	
	and since by hypothesis $\varepsilon_k=\mathcal{N}(0,\sigma)$, it immediately comes by the stability of the Normal law (\SeeChapter{see section Statistics page \pageref{stability of the sum in statistics}}) that:
	
	This is why this specific model is named "\NewTerm{Gaussian linear model}\index{Gaussian linear model}" .... Explicitly, we have:
	
	This is why theoretical model is formally denoted: 
	
	We will choose the symbol $\sim$ to say "follows the law ..." in what follows immediately in order to avoid any confusion:
	
	Which graphically is equivalent to have (so normally the statistical analysis of the Gaussian regression occurs only if and only if we have taken several measures of the dependent variable for fixed and identical values of the independent variables!!!!!!!!):
	\begin{figure}[H]
		\centering
		\includegraphics{img/arithmetics/gaussian_regression.jpg}
		\caption{Graphical representation of the idea behind the Gaussian linear regression}
	\end{figure}
	
	 \begin{tcolorbox}[colback=red!5,borderline={1mm}{2mm}{red!5},arc=0mm,boxrule=0pt]
	\bcbombe Caution! Since the model is Gaussian, the variable to explain has its domain of definition which is unbounded (support of the Normal distribution). Some companies (especially auditing firms...) wish sometimes to create a simple linear model to model a  probability (that for reminder is bounded in [0,1]) typically for the probability of bankruptcy/default based on various factors (explanatory variables ). Therefore the measured probability must first be turn into $Z$ values (quantiles) of the Normal distribution to have again an infinite range. This type of approach is then named "\NewTerm{linear $Z$-score model}\index{linear $Z$-score model}".
	\end{tcolorbox}
	

	Almost all statistical analysis software provide a pattern (figure) of residues according to the $x$ values. Thus, these type of figures help to accept or reject the use of a linear Gaussian model:

	\begin{figure}[H]
		\centering
		\includegraphics{img/computing/residuals.jpg}
		\caption{Examples of "plot" of residuals}
	\end{figure}

	In the above figure, the graph on the top left is what we should expect to be able to apply statistical tests the Gaussian linear model. The graph on the top right shows that the residuals variance is not constant and therefore the violation of the assumption homoscedasticity. The graph at the bottom left shows that the variance is constant but that our model has missing endogenous variables that added could perhaps explain the shift that grows up linearly. The graph at the bottom right indicates a constant variance, but the model look like to be more nonlinear than linear.

	Previous assumptions about the moments of residues (mean, variance) are named "\NewTerm{Gauss-Markov assumptions}\index{Gauss-Markov assumptions}" and the particular hypothesis of equal variances is named as we saw it in the section Statistics "\NewTerm{homoscedasticity}\index{homoscedasticity}" (while the fact that the variances are not equal is named for reminder "heteroskedasticity").

	\begin{tcolorbox}[title=Remarks,colframe=black,arc=10pt]
\textbf{R1. }The majority of softwares (including Microsoft Excel 11.8346) propose a graph which shows the residuals in function of the ordinate values x. Obviously, it is better that the points representing the residues are not too divergent ... otherwise the homoscedasticity assumption will be not satisfied.\\

\textbf{R2.} Practitioners sometimes transform the endogenous and exogenous variables by a logarithmic or exponential function or other to try to stabilize the residuals as most as possible.
	\end{tcolorbox}	
	
	\begin{theorem}
	We have through the property of the mean (\SeeChapter{see section Statistics page \pageref{properties of the mean}}):
	
	\end{theorem}
	Then under the above assumptions, we will show that $a$ and $b$ are unbiased estimators (\SeeChapter{see section Statistics page \pageref{unbiased estimator}}) of $\alpha$ and $\beta$ and it is possible to estimate the standard deviation from the SSR which is an important result named "\NewTerm{Gauss-Markov theorem}\index{Gauss-Markov theorem}".
	
	Before seeing the proof let us do a recall and give some definitions of the variables that we have already handled and the new one that we will handle (if the vocabulary seems technical to the reader then it should read or reread the sections of Probabilities and Statistics):
	
	\begin{table}[H]
	\begin{center}
		\definecolor{gris}{gray}{0.85}
			\begin{tabular}{|p{1cm}|p{10cm}|}
				\hline
				\multicolumn{1}{c}{\cellcolor{black!30}\textbf{Variable}} & 
  \multicolumn{1}{c}{\cellcolor{black!30}\textbf{Description}} \\ \hline
				$a$ & Slope (director coefficient) of the linear model of the least squares method (LSM). This is a punctual estimate value (deterministic). \\ \hline
				$b$ & Intercept of the linear model of the least squares method (LSM). This is a punctual estimate value (deterministic).\\ \hline
				$A$ & Random variable of the slope (director coefficient) according to the statistical Gaussian model approach and for which $a$ is a realization. The expected mean of $A$ being an unbiased estimator of $\alpha$.\\ \hline
				$B$ & Random variable of the ordinate at the origin according to the statistical approach of the linear Gaussian model and for which $b$ is a realization. The expected mean of $B$ being an unbiased estimator of $\beta$.  \\ \hline
				$\alpha$ &	Unbiased expected mean of the variable $A$ representing the slope (director coefficient) within the framework of the statistical approach of the Gaussian linear model.  \\ \hline
				$\beta$ & Unbiased expected mean of the variable $B$ representing the $y$-intercept (ordinate at origin) in the framework of the statistical approach of linear Gaussian model.  \\ \hline
		\end{tabular}
	\end{center}
	\caption{Reminder of notations for the study of the linear regression}
	\end{table}
	
	\begin{dem}
	According to the adopted model, $a$ must now be regarded as a realization of the random variable given by (shown above as the ratio of the covariance and variance):
	
	and $b$ as a realization of the random variable given by:
	
	So we differentiate random and non-random coefficient values by passing the lowercase notation in uppercase (as it is the tradition in the field of Statistics).
	
	Taking into account that theoretical dependent (endogenous) variable is considered as the realization of a random variable is therefore given by:
	
	we can put $A$ in the form\label{estimator slope OLS}:
	
	Therefore:
	
	
	And for $B$:
	
	Therefore and using also the definitions:
	
	with the same conclusion.
	\begin{flushright}
		$\blacksquare$  Q.E.D.
	\end{flushright}
	\end{dem}
	So the expected mean of $A$ and $B$ are unbiased estimators (i.e.. with minimum variance as seen in the section of Statistics) of. As they are estimators, in the literature, they are often noted $\hat{a},\hat{}$ and therefore it comes the common alternative notation:
		
	Finally, we must also calculate the variances of $A$ and $B$ using its properties (\SeeChapter{see section Statistics page \pageref{properties of the variance}}) and the assumptions on the residuals, we have:
	
	As by assumption we have all $\text{V}(\varepsilon_k)$ are equal and there is no autocorrelation, we can write:
	
	And if $n$ is large enough we will write:
	
	Before determining the variance of $B$, remember that by hypothesis:
	
	therefore by the property of linearity of the Normal distribution, the random variables $A$ and $B$ also follow a Normal distribution.
	
	After the recall of this assumption, it follows immediately (\SeeChapter{see section Statistics page \pageref{standard error}}):
	
	Therefore:
	
	Remember the Huygens theorem (\SeeChapter{see section Statistics page \pageref{huygens relation}}):
	
	in the case of equitable probabilities (Normal Law estimator of the mean as prove in the section Statistics).
	
	Finally we have:
	
	where obviously the notation of the variance in the denominator is very unfair (because $x$ is not a random variable in this model) but very convenient to condense the notation.
	
	The problem now lies in determining $\sigma_\varepsilon^2$. Obviously to do this we will be forced to go through a statistical estimator.
	
	We know we can write according to what was seen in the section Statistics regarding to estimators:
	
	because the Normal distribution is centered for residuals and therefore $\bar{\varepsilon}=0$... and the residue is implicitly dependent of two the sum of two random variable that are $A$ and $B$ this is why we have a $-2$ and the denominator of the first fraction (we have to take away two degrees of freedom).
	
	Let us also indicate that in practice we frequently note the last result by mixing the notations of the random and deterministic appearance (hence noting everything with lower case):
	
	where SEE\footnote{Sadly sometimes (like in Minitab outputs) just denoted by the letter "S"...} means "\NewTerm{Standard Error of Estimate}\index{standard error of estimate}\label{standard error of estimate}" or sometimes also named  the "\NewTerm{standard regression error}\index{standard regression error}" which is obtained with the English version of Microsoft Excel trough square of the function \texttt{STEYX( )}.
	
	\begin{tcolorbox}[title=Remark,colframe=black,arc=10pt]
	This bring us obviously to consider that standard R squared estimator uses biased estimators of $\sigma^{2}_\varepsilon$ and $\text{V}(Y)$, by using the divisor $n$ (or $n-1$) for both. These estimators are biased because they ignore the degrees of freedom used to estimate the regression coefficients and mean of $Y$. If instead, we estimate $\text{V}(Y)$ by the usual unbiased sample variance (which uses $n-1$ as the divisor) and $\sigma^{2}_\varepsilon$ by its unbiased estimator which uses $n-p-1$ (where $p$ is the number of predictors) as the divisor, we obtain the "\NewTerm{adjusted R-squared $R^2_\text{adj.}$}\index{adjusted R-squared}\label{adjusted R-squared}" estimator:
	
	From this relation, we can see that the adjusted estimator is always less than the standard one. We can also see the larger $p$ is relative to $n$, the larger the adjustment. A final point: although the adjusted $R$ squared estimator uses unbiased estimators of the residual variance and the variance of $Y$, it is not unbiased! This is because, as we know it, the expectation of a ratio is not generally equal to the ratio of the expectations.\\
	
	Furthermore, notice, as $\hat{R}^{2}=1-\text{SSE}/\text{SST}$, we have:
	
	But for any value of $p\geq 1$:
	
	If we multiply both side by $(1-\hat{R}^{2})$, we get:
	
	Subtracting $-1$:
	
	multiplying by $-1$:
	
	Therefore:
	
	From this relation, we can that the adjusted estimator is always less than the standard one!
	\end{tcolorbox}
	
	So we have to summarize the unbiased estimators variances of $A$ and $B$:
	
	relations which are therefore only valid for a linear regression with one unique explanatory variable (and under the assumptions of the Gaussian linear model). Knowing that by construction of the initial hypothesis that $A$ and $B$ follow a Normal distribution of respective expected mean $\alpha,\beta$ and whose variance is given just above, so we know completely the distribution that characterizes them.
	
	What is nice knowing these variances is that we can also therefore easily estimate the variance of the dependent variable in our regression (using the properties of the variance proved in the section Statistics).
	
	It would be interesting to make statistical inference on the mean of the parameters $A$ and $B$ (i.e. the slope and intercept) given their known empirical mean (i.e. average). For this, remember that we have proved in the section Statistics the following confidence interval:
	
	It follows by making a parallel like engineers and physicists like to do ... that as $A$ is an unbiased estimator of the average of the slope $a$ and that:
	
	is in fact the standard error of the mean $A$, then by analogy:
	
	and then (this is an reasoning to be take with caution an it is better to use the developments that will follow later):
	
	which therefore gives the confidence interval of the slope of a linear Gaussian with one unique explanatory variable (that's what gives Microsoft Excel 11.8346 for each coefficient). The approach is the same for the intercept.
	
	Warning! If the explanatory variable is a random variable, then we use naively (the are more accurate and correct model that we will see later when the explanatory variable is a also a random variable):
	
	
	In the case of a linear regression with several explanatory variables, therefore assimilate to the concept of "\NewTerm{degrees of freedom DoF}\index{degrees of freedom}", the idea is the same but the calculations are longer (we don't have yet the will to do the mathematical developments for this case).
	
	Finally, remember that we got for the empirical correlation coefficient:
	
	We then verbatim get the famous confidence interval of the correlation coefficient:
	
	You should know that as calculating the confidence interval for the slope or for the correlation coefficient two equivalent thing, many softwares (Tanagra, Minitab, Microsoft Excel, etc.) give only the value of the Student's distribution at the critical value of this latter only for the slope and... they then assume that the reader knows that it is the same for the correlation coefficient.
	
	\paragraph{Pearson Correlation Coefficient Test}\index{Pearson's test}\mbox{}\\\\\
	The calculation obtained above for the confidence interval of the correlation coefficient is a little difficult in practice. It is for this reason that many practitioners and statistical softwares implements a very simple alternative communicated only in the minimal form that is the $p$-value.
	
	To see this approach, remember that we proved in the section of Statistics that (this time we will adopt the correct notation...):
	
	And we saw just above that:
	
	Similarly, we have the Pearson correlation coefficient estimator which is (using here the various possible notations that we can find in the literature...):
	
	and therefore:
	
	The hypothesis test we want to do is therefore:
	
	and therefore equivalent to:
	
	the null hypothesis obviously being that the Pearson correlation is statistically significantly different from zero. So this is a bilateral test!
	
	To find a simple form of the test, remember that we obtained:
	
	and also:
	
	which brings us mixing the two to have:
	
	Therefore:
	
	But we remember that we have also proved that if $n$ is large enough:
	
	But if $n$ is small we fall back on:
	
	Therefore:
	
	and with the null hypothesis $a=0$, we get:
	
	
	Be careful with the use of this test often and logically named "\NewTerm{Student's $T$-test for univariate regression slope}\index{Student's $T$-test for univariate regression slope}" or "\NewTerm{$T$-test for coefficient slope}\index{$T$-test for coefficient slope}", depending if the Pearson correlation coefficient is negative or positive and don't forget that it is bilateral!
	
	\begin{tcolorbox}[colframe=black,colback=white,sharp corners]
	\textbf{{\Large \ding{45}}Examples:}\\\\
	E1. We have calculated for a series of data a positive Pearson correlation coefficient $R$ of value $0.298$ and the explanatory variable has $7$ values. So we have with the English version Microsoft Excel 11.8346 the $p$-value that is given by (we find exactly the same value as with Minitab 15.1.1):\\
	
	\texttt{=2*(1-T.DIST(0.298/SQRT((1-0.298\string^2)/(7-2));7-2;1))\\=2*(1-0.741869)=0.51626}\\
	
	In this case we sadly can't reject the null hypothesis as what the Pearson correlation coefficient is equal to zero at a threshold of $5\%$.\\

	E2. We have calculated for a data set a positive Pearson correlation coefficient $R$ of  $-0.084$ and the explanatory variable contains $19$ values. So we have with the English version of Microsoft Excel 11.8346 the $p$-value that is equal to (we find exactly the same value as with Minitab 15.1.1):\\
	
	\texttt{=2* T.DIST((-0.084)/SQRT((1-(-0.084)\string^2)/(19-2));19-2;1)\\=2*(1-0.366)=0.74186}\\
	
	In this case we also sadly can't reject the null hypothesis as what the Pearson correlation coefficient is equal to zero at a threshold of 5%.
	\end{tcolorbox}
	This small trap makes that finally we take the absolute value of the Pearson correlation coefficient and we therefore use the always the same calculation method.
	
	\pagebreak
	\paragraph{Confidence interval of predicted values}\mbox{}\\\\\
	We wish for each measured value of the dependent variable, know the confidence interval. In other words, we would like to know the statistical variance estimator of $Y$ (we do not write the subscripts anymore to save time):
	
	Unfortunately, we will go into the wall because the covariance is difficult to calculate ($A$ and $B$ are not independent as shown by the expressions we got previously).
	
	By cons, being a good observer, we see that if we use the result seen above:
	
	Then:
	
	The problem being circumvented, we now have using the properties of variance:
	
	Therefore:
	
	Then we have finally:
	
	Now let us recall (\SeeChapter{see section Statistics page \pageref{reduced centered variable}}) that:
	
	and as $Y$ is distributed according to a Normal distribution for whose the unbiased estimator of the mean and standard deviation are given by the prior-previous  relation it comes immediately:
	
	In practice you must also check that this ratio follows a Normal distribution in order to make the confidence intervals and statistical tests that follow.
	
	Let us recall now that we have proved in the section Statistics that:
	
	follows a Student law of degrees of freedom $k$ and the variable $U$ follows a chi-square law of degree of freedom $k$.
	
	Now let us come back to the expression of the $Z$ obtained above and remember that:
	
	Therefore:
	
	But as we have the assumption:
	
	Therefore:
	
	and obviously:
	
	corresponds to a sum of squares of reduced centered Normal laws. And therefore according to what we proved in section Statistics it follows that:
	
	Therefore:
	
	That is to say:
	
	This is one reason why many statisticians note directly and without detours:
	
	
	Which is not necessarily obvious at first glance. This is why, following the request of a reader, we have detailed a little bit exaggeration, the mechanism behind this involvement\label{likelihood binomial logistic regression}.
	
	This done, we have finally:
	
	and we deduce from this immediately a bilateral confidence interval of a given threshold level $\alpha$ for a fixed $x$ whose expression is:
	

	\begin{tcolorbox}[title=Remark,colframe=black,arc=10pt]
	The same kind of development can be done for the slope and the intercept. This is why software like Microsoft Excel, SPSS, Minitab, Statistica, etc. give the value of the Student t-distribution as well as the confidence interval for a given threshold level $\alpha$. But for this to be meaningful, you must never forget that all the assumptions of the model must be met.
	\end{tcolorbox}
	\begin{figure}[H]
		\centering
		\includegraphics{img/computing/regression_confidence_interval.jpg}
		\caption{Print screen of a confidence interval obtained with Minitab 15}
	\end{figure}
	The reader will may have noticed that:
	\begin{itemize}
		\item It is very boring to obtain without software or without coding the plot of the confidence interval for the ordinary least squares since we have to calculate it for each point...
		
		\item The confidence interval is curved which is sometimes considered as common sense, at least in the temporal version of the regression: furthest is the forecast less accurate it will be
	\end{itemize}
	
	The true value of $Y$ is given by:
	
	with the variance:
	
	which is independent of the estimator $Y$. Therefore, the difference between $Y$ and $y$ (therefore between estimator and real value) has for variance:
	
	Therefore it is customary to consider that the  "\NewTerm{prediction interval}\index{prediction interval}" (not to be confused with the confidence interval of the estimator) is taken as:
	
	\begin{tcolorbox}[title=Remark,colframe=black,arc=10pt]
	The same kind of development can be done for the slope and the intercept. This is why software like Microsoft Excel, SPSS, Minitab, Statistica, etc. give the value of the Student t-distribution as well as the confidence interval for a given threshold level $\alpha$. But for this to be meaningful, you must never forget that all the assumptions of the model must be met.
	\end{tcolorbox}
	This gives us the following type of chart with a software such as Minitab 15:
	\begin{figure}[H]
		\centering
		\includegraphics[scale=1]{img/computing/regression_prediction_interval.jpg}
		\caption{Print screen of a prediction interval (PI) in green obtained with Minitab 15}
	\end{figure}
	Where we can see the prediction interval PI in green and the confidence interval CI in red and its is obvious to see by the property of the variance that CI<PI.
	
	These two intervals are often confused. To understand these, it's important to realize that a regression model fits a relationship between the input variables and the mean value of the outcome variable - this is the red line in the below chart. The confidence interval indicates a range of certainty around the mean value of the outcome variable - usually $95\%$ - and is illustrated by the green lines above. The prediction interval indicates a range of certainty around any value of the outcome variable and is illustrated by the purple lines above. Another way of looking at this is to imagine taking a  vertical slice of the below chart. This would translate - under the most common models - to a bell curve for the value of the outcome variable at that specific  value of the input variable. The red point indicates the most likely center of the bell curve, the green lines represents the $95\%$ likelihood range for that center  and the purple lines represents the likelihood for the $95\%$ range (ie width) of all these bell curves.
	
	So what must be well understand by the reader so far, is that if we are dealing with univariate linear regression or any other type of regression and want to plot the following data that corresponds to a sample for each $x$-value:
	\begin{figure}[H]
		\centering
		\includegraphics[scale=0.7]{img/computing/regression_excel_list_data_sample.jpg}
	\end{figure}
	Therefore using a spreadsheet software to make a plot of that the average of each year as following (most common case see in Fortune 500 companies by top managers and also by auditing companies like KPM and EY):
	\begin{figure}[H]
		\centering
		\includegraphics{img/computing/regression_excel_sample_plot.jpg}
	\end{figure}
	using a smooth chart is everything but scientific. Indeed:
	\begin{enumerate}
		\item No spreadsheet software is able to summary the data list above automatically in quick an efficient way (with Microsoft Excel even Pivot Charts cannot do smooth scatter charts of such a list of data...).
		
		\item Smooth scatter plot in spreadsheet softwares use splines interpolation and these are not statistical models that can be used for samples! So reading such charts in a spreadsheet software introduce a huge deterministic bias in the mind of the board committee or any customer/supplier.
		
		\item As the averages come from samples there is an implicit confidence interval THAT HAS TO BE SHOWN to the board committee or customer/supplier. Do not show a confidence interval on sample data is a high-school level error that is not acceptable when done by a senior manager or business analyst or even a consultant of a big $5$ auditing companies.
	\end{enumerate}
	So following the detailed steps given in our \texttt{R} companion book we get something truly scientific and robust that is:
	\begin{figure}[H]
		\centering
		\includegraphics[scale=0.7]{img/computing/regression_r_sample_plot.jpg}
	\end{figure}
	
	\pagebreak
	\subsubsection{Linear univariate regression forced through the origin (RTO)}
	One very common case and requested case in laboratories (and generally in other departments) of companies, is to force the linear regression through the origin (named also "\NewTerm{regression through origin}\index{regression through origin}" and abbreviated RTO).
	
	We will see now (in the univariate case) that the approach is only a simplified variant of the method of the ordinary least squares.
	
	We use as before:
	
	where $n$ is the number of points. But this time, let us write:
	
	then:	
	
	This relation make appears the sum of squared deviations as a function of the parameter $a$. When this function is minimal (extremal), the derivatives with respect to its parameters are cancelled:
	
	After simplification:
	
	Finally:
	
	You can also easily check with any spreadsheet software (Microsoft Excel for example) that the calculations corresponds well.
	
	However the reader must know now that the RTO subject is surprisingly
controversial among statisticians. Indeed,  forcing the regression line through the origin is generally inconsistent with the best fit.

	\begin{tcolorbox}[colback=red!5,borderline={1mm}{2mm}{red!5},arc=0mm,boxrule=0pt]
	\bcbombe Caution! The text below is mainly a reproduction with some adaptations of the following article \cite{eisenhauer2003regression}. Thanks to the donations we were able to buy the rights (225 US\$...) to John Wiley \& Sons, Inc. to reproduce it in this book. But in no manner the reader is authorized to reproduce the same text without the agreement of John Wiley \& Sons, Inc. !!!
	\end{tcolorbox}
	
	The proper method for evaluating RTO has long been disputed. To appreciate the controversy, note the familiar identity:
	
	Squaring both sides and summing across all observation gives:
	
	but, as we have proved earlier above, the cross-product term must be equal to zero in the case of OLS. The remaining terms therefore constitute the usual analysis of variance decomposition:
	
	And we know that then the coefficient of determination for OLS is then defined by:
	
	Some authors maintain that because this diagnostic measure is based on an identity, it should not depend on the inclusion or exclusion of a constant term in the regression. From that perspective, the previous relation is equally valid for RTO and OLS.
	
	However, when there is no constant in the regression, $\sum(y_k-\hat{y}_k)(\hat{y}_k -\bar{y})$ will generally take a non-zero value, so the variance decomposition:
	
	is not a valid basis for analysis of variance in RTO. And if the RTO model provides a sufficiently poor fit, the data may exhibit more variation around the regression line than around $y$, in which case:
	
	Heedlessly applying:
	
	would then result in an implausibly negative (and thus non-interpretable) coefficient of determination as well as a negative $F$ ratio. Moreover, it is often argued that defining SST as the sum of squared deviations from the mean is inappropriate when the regression line is forced 	through the origin but does not necessarily pass through $(\bar{x},\bar{y})$, when so viewed, the starting relation:
	
	is replaced by identity:
	
	Squaring and summing yields:
	
	but the final (cross-product) term in this equation equals zero under RTO, because:
	
	Thus, the variance OLS decomposition:
	
	is replaced by:
	
	Applying this latter relation one rather than the classical OLS one to RTO, one finds that SSE is unchanged, but $\text{SST}=\sum y_k^2$ and $\text{SSR}=\sum \hat{y}_k^2$. Redefining SST and SSR in this manner results in:
	
	ie, also a strictly non-negative coefficient of determination that equals or exceeds the classical $R^2$ measurement. Of course, these definitions also affect the adjusted $R^2$ and $F$ statistics.
	
	Note that, without a constant, the degrees of freedom for SST, SSR and SSE are $n$, $k$ and $n – k$, respectively, where $n$ is the sample size and $k$ is the number of independent variables. Thus:
	
	regardless of how SST is defined.
	
	The controversy over SST is not merely academic: practitioners (and students) running RTO got various outputs depending on which computer packages they use a few dozens of years ago (since they almost all give the same result based on the adapted definition of the correlation coefficient).
	
	\pagebreak
	\subsubsection{Deming regression (orthogonal regression)}
	As we have already mention it, the "\NewTerm{orthogonal linear regression model}" \index{orthogonal linear regression} or "\NewTerm{Deming regression}"\index{Deming regression}  is used as complement to the paired $T$-test to check the stability of the measuring instruments in laboratories. This is a case where the explanatory and dependent variables are tainted with uncertainty (we speak then of "\NewTerm{stochastic regressors}" at the opposite classical case where we have "\NewTerm{nonstochastic regressors}"). In other words, this regression is used when working in a situation where it is desired to compare two linearly increasing or decreasing variables and where each is tainted with measurement errors including error values (of the two variables! ) are independent.
	\begin{tcolorbox}[title=Remark,colframe=black,arc=10pt]
	Chemists do this analysis sometimes in qualitative form using the "Bland-Altman plots".
	\end{tcolorbox}	

	We will see now the derivation of the maximum likelihood estimates related to the Deming regression model. It is based on the book \textit{Models in regression and related topics} (chapter three), from 1969 by Peter Sprent, but with more detailed calculations included (big thanks to Anders Jensen for the \LaTeX code of the proof!)
	.
	\paragraph{Deming orthogonal regression log-likelihood derivation}\mbox{}\\\\
	The mathematical model $\eta=\alpha+\beta\xi$ describes a linear
	relationship between two variables $\xi$ and $\eta$. Observations $x$
	and $y$ of two variables are usually described by a regression of $y$
	on $x$ where $x$ is assumed to be observed without error (or,
	equivalently using the conditional distribution of $y$ given $x$). In
	linear regression with observations subject to additive random
	variation on both $x$ and $y$ and observed values for individuals
	$(x_i,y_i), i=1,\ldots,n$, a model may be written:
	
	where $e_{xi}$ and $e_{y_i}$ denotes the random part of the model.
	This is known as a functional relationship because the $\xi_i$'s are
	assumed to be fixed parameters, as opposed to a structural
	relationship where some distribution for the $\xi_i$'s is assumed. In
	the following it is assumed that the $e_{xi}$s are iid with:
	
	and that the $e_{yi}$s are iid with:
	
	for some $\lambda>0$. Furthermore $e_{xi}$ is assumed to be independent of $e_{yi}$.
	
	The aim of this document is to derive the maximum
	likelihood estimates for $\alpha, \beta, \xi_i$ and $\sigma^2$ in the functional model stated above.

	The likelihood function:
	
	denoted $f$ is:
	
	and the log-likelihood, denoted $\mathcal{L}$, is:
	
	It follows that the likelihood function is not bounded from above when $\sigma^2$ goes to $0$, so in the following it is assumed that $\sigma^2>0$.

	Let us solve for $\xi_i$ now! The differentiation of $\mathcal{L}$ with respect to $\xi_i$ gives:
	
	Setting $\frac{\partial \mathcal{L}}{\partial \xi_i}$ equal to zero yields:
	
	So to estimate $\xi_i$, estimates for $\beta$ and $\alpha$ are needed. Therefore focus is turned to the derivation of $\hat{\alpha}$.

	Let us now solve for $\alpha$. The differentiation of $\mathcal{L}$ with respect to $\alpha$ gives:
	
	and putting $\frac{\partial \mathcal{L}}{\partial \alpha}$ equal to zero yields to:
	
	Now one can use:
	
	to dispense with $\xi_i$:
		
	
	Hence the estimate for $\alpha$ becomes:
	

	Solving for $\beta$ by differentiating  $\mathcal{L}$ with respect to $\beta$ gives:
	
	Setting $\frac{\partial \mathcal{L}}{\partial \beta}$ equal to zero yields:
	
	and using again:
	
	we get:
	
	This implies that:
	
	Dividing with $\lambda$ and using the fact that:
	
	it is seen that:
	
	Splitting up the sums even more gives:
	
Finally the terms are sorted and collected according to powers of $\beta$:
	
	Since:
	\begin{itemize}
		\item
		$\displaystyle\sum_{i=1}^n\overline{x}^2
		-\overline{x}\displaystyle\sum_{i=1}^nx_i=0$
		\item
		$\overline{y}\displaystyle\sum_{i=1}^nx_i
		-\displaystyle\sum_{i=1}^nx_iy_i
		-2\displaystyle\sum_{i=1}^n\overline{y}\overline{x}
		+2\displaystyle\sum_{i=1}^ny_i\overline{x}=-\text{SPD}_{xy}$
		\item
		$\displaystyle\sum_{i=1}^ny_i^2
		-\lambda\displaystyle\sum_{i=1}^nx_i^2
		+\displaystyle\sum_{i=1}^n\overline{y}^2
		-2\displaystyle\sum_{i=1}^ny_i\overline{y}
		+\lambda\overline{x}\displaystyle\sum_{i=1}^nx_i
		=\text{SSD}_y-\lambda\text{SSD}_x$
		\item
		$\displaystyle\sum_{i=1}^nx_iy_i
		-\overline{y}\displaystyle\sum_{i=1}^nx_i=\text{SPD}_{xy}$
	\end{itemize}
	it is clear that the derivation of $\beta$ comes down to solve:
	
	For $\text{SPD}_{xy}\neq 0$ this implies that:
	
	Since:
	
	there is always a positive and a negative solution to (\ref{equation}). The desired solution should always have the same sign as $\text{SPD}_{xy}$, hence the solution with the positive numerator is selected. Therefore:
	

	Let us solve for $\xi_i$ - again... With estimates for $\beta$ and $\alpha$ it is now possible to estimate $\xi_i$ using:
	
	we get:
	

	Let us now solve for $\sigma^2$ by differentiating $\mathcal{L}$ with respect to $\sigma^2$ gives:
	
	and setting $\frac{\partial \mathcal{L}}{\partial \sigma^2}$ equal to zero yields:
	
	To get a central estimate of $\sigma^2$ one must divide by $n-2$
instead of $2n$ since there are $n+2$ parameters to be estimated,
namely $\xi_1,\xi_2,\ldots,\xi_n,\alpha$ and $\beta$. Hence the
degrees of freedom are $2n-(n+2)=n-2$. Therefore:
	
	Finally summing up:
	
	These results are implemented in Minitab 16.1.2 and also in the \texttt{Deming()} function in the MethComp package of \texttt{R} as the ready can see it in the corresponding companion books.

	\subsubsection{Multiple linear regression Gaussian Model}\label{multiple linear regression gaussian model}
	Of course, in some situations, linear regression is too simple or just not suitable. The most typical case that will concern us now in what follows are the situations where we have several explanatory variables (multivariate case!).
	
	The idea of multiple linear regression is relatively simple. We want to determine the dependent variable $y$ from $p-1$ independent variables (i.e. in the absence of "colinearity"!) - and therefore of $p$ parameters to determine - connected by a linear relation of the general form:
	
	In a sample of $n$ individuals we measure $y_i,x_{i,1},...,x_{i,p-1}$ for $i=1..n$:
	\begin{table}[H]
	\begin{center}
		\definecolor{gris}{gray}{0.85}
			\begin{tabular}{|p{2cm}|p{2cm}|p{2cm}|p{2cm}|p{2cm}|}
				\hline
				\multicolumn{1}{c}{\cellcolor{black!30}\textbf{Observation}} & 
  \multicolumn{1}{c}{\cellcolor{black!30}\textbf{$y_i$}}  & \multicolumn{1}{c}{\cellcolor{black!30}\textbf{$x_{i,1}$}} & \multicolumn{1}{c}{\cellcolor{black!30}\textbf{$\ldots$}} & \multicolumn{1}{c}{\cellcolor{black!30}\textbf{$x_{i,p-1}$}} \\ \hline
				\centering\arraybackslash\ $1$ & \centering\arraybackslash\ $y_1$ & \centering\arraybackslash\ $x_{1,1}$ & \centering\arraybackslash\ $\ldots$ & \centering\arraybackslash\ $x_{1,p-1}$ \\ \hline
				\centering\arraybackslash\ $2$ & \centering\arraybackslash\ $y_2$ & \centering\arraybackslash\ $x_{2,1}$ & \centering\arraybackslash\ $\ldots$ & \centering\arraybackslash\ $x_{2,p-1}$ \\ \hline
				\centering\arraybackslash\ $\vdots$ & \centering\arraybackslash\ $\vdots$ & \centering\arraybackslash\ $\vdots$ & \centering\arraybackslash\ $\ldots$  & \centering\arraybackslash\ $\ldots$\\ \hline
				\centering\arraybackslash\ $n$ & \centering\arraybackslash\ $y_n$ & \centering\arraybackslash\ $x_{n,1}$ & \centering\arraybackslash\ $\ldots$ & \centering\arraybackslash\ $x_{n,p-1}$ \\ \hline
		\end{tabular}
	\end{center}
	\end{table}
	In fact, the to estimate the parameters $\beta_0,...,\beta_{p-1}$ (estimated values which we denote by $\hat{\beta}_0,...,\hat{\beta}_{p-1}$ to respect traditions) the approach is very simple because it is just a generalization of the method of least squares we saw earlier for the simple univariate linear regression.
	
	So in the end we rewrite the relation of Sum of Squared Residuals seen above by modifying a little bit because we now have the multi-linear stuff:
	
	with the estimated theoretical model:
	
	So we have to minimize:
	
	The above parentheses can be rewritten in the form (using what we learned in Vector Calculus and Linear Algebra sections):
	
	Whose condensed form is:
	
	where $X$ is named the "\NewTerm{designed matrix}", also known as "\NewTerm{regressor matrix}\index{regressor matrix}" or "\NewTerm{model matrix}\index{model matrix}" or "\NewTerm{data matrix}\index{data matrix}".
	
	We then have the vector of residues that can be written:
	
	As we know, the least squares method is to find the vector $\vec{\hat{\beta}}$ that minimizes:
		
	Therefore explicitly:
	
	Notice that we have:
	
	and:
	
	as each of the elements of the multiplication is a simple vector!
	
	Therefore we have (caution! do not forget that some multiplications in the relation that will follow are dot products!!!) the quadratic multivariate function of coefficients of vectors:
	
	Let us derivate this latter "object function $F$" at the order of vector $\hat{\beta}$ (it is as internal derivative component by component). What we will write:
	
	Now let us rewrite this form a vector notation to a pure matrix notation:
	
	We seek the $\vec{\hat{\beta}}$ that cancel this derivative. Therefore we must solve the following equation:
	
	Therefore:
	
	Remember before we continue that:
	
	Therefore the prior-previous relation can be written:
	
	As Linear Algebra is associative, let us write without the parenthesis:
	
	We can obviously not simplify right and left by $\vec{u}^TX^T$ as it is not a squared matrix, this term is then by obligation not reversible. The only one thing we cam do is identify the elements such that:
	
	this implies obviously:
	
	We find then that if the squared matrix $X^TX$ is reversible then:
	
	The matrix $X^TX$ that we will see again in the field of multiple linear regression and in the field of design of experiments (\SeeChapter{see section Industrial Engineering page \pageref{doe}}) is named "\NewTerm{information matrix}\index{information matrix}\label{information matrix}" or "\NewTerm{dispersion matrix}\index{dispersion matrix}" for a reason that will be very obvious later.
	
	The expression $(X^TX)^{-1}X^T$ is named the "\NewTerm{left pseudo-inverse}\index{left pseudo-inverse}", or simply "\NewTerm{pseudoinverse}\index{pseudoinverse}", and even sometimes "\NewTerm{Moore-Penrose pseudoinverse}\index{Moore-Penrose pseudoinverse}" of $X$ and is denoted $X^-$ in some textbooks.
	
	\begin{tcolorbox}[title=Remark,colframe=black,arc=10pt]
	We say that the regression is "\NewTerm{balanced and orthogonal}\index{balanced and orthogonal}" when the information matrix is diagonal. We say that the regression is "\NewTerm{orthogonal}\index{orthogonal regression}" when the sub-matrix of the information matrix excluding the first row and first columns is orthogonal. We say that the regression is just "\NewTerm{balanced}\index{balanced regression}" when all the values of the first row and of the first column excepted the one at the intersection are equal to zero.
	\end{tcolorbox}	
	To show that this seem correct a priori, let us fall back on the results we get of the simple univariate regression:
	
	Then supposing 2 observations, we have therefore:
	
	Using the relation proved in the section of Linear Algebra to calculate in generality the inverse of a matrix $A(a_{ij})$ in $A^{-1}(b_{ij})$:
	
	We have in our special case:
		
	Therefore we have:
	
	and since we have a square matrix of dimension $2$ only, the calculation of the four determinants is reduced to selecting the components of $X^TX$ (\SeeChapter{see section Linear Algebra page \pageref{determinant of two by two matrix}}):
	
	Therefore:
	
	So to a change in notations for the indices and experimental measurements, we fall back on the results that we obtained during our study of the simple linear univariate regression that was (for refresh...):
	
	Now, we need a quality indicator regarding our multi-linear regression. Remember that in the context of our study of the univariate linear regression, we proved that the linear correlation coefficient can be written as (see page \pageref{correlation linear regression model}):
	
	and in fact it also applies directly to the multiple linear regression, since it does not presuppose the number of explanatory variables!!
	
	But it is the tradition to rewrite that latter in matrix form. So at it is beautiful, let us do the work.
	
	First we start with:
	
	and finally:
	
	Hence:
	
	
	\begin{tcolorbox}[title=Remark,colframe=black,arc=10pt]
	The reader interested in the practical application of these results may, just as for simple univariate regression, refer to the server exercises on the companion website - Numerical Methods section - where there are practical examples with Microsoft Excel.
	\end{tcolorbox}	
	Obviously with multiple linear regression, we can now, with a small tip, do linear regression ... of polynomials (we will see later how to apply directly the ordinary least squares method on a polynomial). Indeed, consider a polynomial of the form:
	
	That we can consider, when rewritten, as:
	
	So in spreadsheet softwares like Microsoft Excel, just use the Regression Analysis Tool with the input variable $x$ column, a second column that we will have taken care to create with inside the square of $x$ (i.e. $x^2$) and which will be considered as the explanatory variable $w$ and a third column we have also taken care to create as the cube of $x$ (i.e. $x^3$) and which will be considered as the explanatory variable $z$.
	
	We can also directly obtain the polynomials coefficients with functions already presented earlier above with Microsoft Excel (but you will not have all the relevant results of the Analysis Tool). For example for a polynomial of second degree:
	
	\texttt{a: =INDEX(LINEST(y,x\string^{1,2}),1)\\
	b: =INDEX(LINEST(y,x\string^{1,2}),1,2)\\
	c: =INDEX(LINEST(y,x\string^{1,2}),1,3)
	}
	
	and for a third-degree polynomial:
	
	\texttt{a: =INDEX(LINEST(y,x\string^{1,2,3}),1)\\
	b: =INDEX(LINEST(y,x\string^{1,2,3}),1,2)\\
	c: =INDEX(LINEST(y,x\string^{1,2,3}),1,3)\\
	d: =INDEX(LINEST(y,x\string^{1,2,3}),1,4)}
	
	This is this trick that allows us to understand why and how the linear correlation coefficient also applies to polynomials in most spreadsheets softwares and statistical softwares. However we can have a more direct approach that does not require this transformation but which therefore is a little longer.
	
	Now let us go back to the Gaussian linear model with a concept a little more rigorous and adapted to the multi-linear case and in particular to highlight the distinction between estimators of the slope of the regression and exact values:
	
	But under this notation convention we have:
	
	and written it in vector form:
	
	Now, we use the technique of maximum likelihood (\SeeChapter{see section Statistics page \pageref{likelihood estimators}}) and we seek the coefficients that maximizes therefore:
	
	What we can write in matrix form:
	
	Taking as in the section of Statistics the log-likelihood to facilitate future calculations and using the property of the transposed matrices proved in section of Linear Algebra, we get:
	
	Now let us look to the expression of $\vec{\beta}$ that maximizes the log-likelihood. It comes then:
	
	Let us use the property of the transposed matrix proved in the section of Linear Algebra:
	
	Therefore we have:
	
	For reasons which will appear evident a little further below we chose the second equivalence. Therefore, we have (and remembering that seeking for an optimum is equivalent to have partial derivative equal to zero):
	
	Therefore:
	
	That is to say, after rearrangement:
	
	That is to say exactly the same expressions that we got just a little earlier above with the least squares method in the multi-linear case and that was for reminder:
	
	This shows that the statistical of the linear model by the maximum likelihood can fall back into the multi-linear case (and thus also the univariate case) on the results of the least squares method. This has something almost divine ...
	
	Finally, let us indicate that we find here the almost famous "\NewTerm{hat matrix}\index{hat matrix}\label{hat matrix}" or "\NewTerm{influence matrix}\index{influence matrix}" $H$ that connects alone all information between the real and explained theoretical values (and depending only on $X$!) and therefore verbatim the error of the theoretical model:
	
	We have then something interesting to observe:
	
	By definition, the influence of the observation $i$ in the regression and named the  "\NewTerm{leverage}\index{leverage}" (or "\NewTerm{leverage score}\index{leverage score}") is defined by:
	
	$H_i$ is therefore the influence of $y_i$ on its own fitted value! It is a qualitative method (available in many statistical softwares) to judge the influence of points that could be considered as outliers. The idea is to compare the values of leverage between them.
	
	Let us now give the famous explicit expression of the leverage of the simple linear regression.

	We start be remembering that by definition:
	
	and we have proved earlier above that in the simple linear regression that we have:
	
	and also:
	
	Let us put by definition (to make notations more light...):
	
	Hence we may rewrite:
	
	Hence:
	
	Finally:
	
	This relation has a nice interpretation in the SLR model: if the distance from $x_i$ to $\bar{x}$ is large relative to the other $x$'s then $H_{ii}$ will be close to $1$.

	Leverages have nice mathematical properties, for example, they satisfy:
	
	and their sum is:
	
	A rule of thumb is to consider leverage values to be large if they are more than double their average size (which is $2/n$ according to the previous relations). So leverages larger than $4/n$ are suspect. Another rule of thumb is to say that values bigger than $0.5$ indicate high leverage, while values between $0.3$ and $0.5$ indicate moderate leverage.

	\paragraph{Fisher Matrix}\mbox{}\\\\
	We have proved just earlier that:
	
	And we have afterwards also prove that (changing the notation just a bit):
	
	And we also get immediately:
	
	From the both relation above we immediately deduce (note that $\hat{\sigma}^2$ is biased!):
	
	To compute $\mathcal{I}(\vec{\theta})$, notice that we have immediately:
	
	Since:
	
	Then the expected mean of the Hessian is (\SeeChapter{see section Statistics page \pageref{Fisher information matrix}}):
	
	Without forgetting that (\SeeChapter{see section Statistics page \pageref{Fisher information matrix}}):
	
	And finally:
	
	It may seems to the reader in a first view that it is a $2\times 2$ matrix. But this is forget that the term in the upper left corner is itself a $p\times p$ (where $p$ is the number of predictors and we put $+1$ because of the intercept). Hence the matrix dimension is $(p+1)\times (p+1)$.

	\paragraph{$R^2$ decomposition in multiple regression}\mbox{}\\\\
	Let us see now how to decompose $R^2$ into components that capture the percentage of variation explained by predictor in a multiple linear regression!

	Consider $n$ predictors and $K$ observations in the sample:
	
	where all variables have been standardized to have mean zero and variance one. That is, we have centered and rescaled the observations such that for $i = 1, . . . ,n$:
	
	Standardizing all variables in this manner is without loss of generality since $R^2$ is manifestly invariant to centering and rescaling of variables. Our multiple linear model is then given by:
	
	since standardization ensures that the intercept in the regression above is identically zero.
	
	We estimate the slope coefficient and obtain the fitted values:
	
	where $\hat{\beta}_i$ are the estimated slope coefficients for predictors $i=1,\ldots,n$. Let $\text{cov}(\cdot,\cdot)$ denote the sample covariance operator, defined for centered vectors $x$ and $y$ by:
	
	Then by definition and construction:
	
	That is, we have the decomposition:
	
	We can therefore define the percentage of variation explained by predictor $i$, denoted by $R_i^2$ by:
	

	\paragraph{Influential points}\label{influential points}\mbox{}\\\\
	An "outlier\index{outlier}" as we already know is a data point which is very far, somehow, from the rest of a group of the data. They are often worrisome, but not always a problem. When we are doing regression modelling, in fact, we don't really care about whether some data point is far from a group of data, but whether it breaks a pattern the rest of
the data seems to follow and this is what we name an "\NewTerm{influential point}\index{influential point}". We will look here some empirical ways of quantifying how much influence particular data points have on the model. 

	The points marked in red and blue in the figure below are clearly not like the main cloud of the data points, even though their $x$ and $y$ coordinates are quite typical of the data as a whole: the $x$ coordinates of those points aren't related to the $y$ coordinates in the right way, they break a pattern. On the other hand, the point marked in green, while its coordinates are very weird on both axes, does not break that pattern - it was positioned to fall right on the regression line.
	
	\begin{figure}[H]
		\centering
		\includegraphics[scale=0.7]{img/computing/outliers_vs_influentials.jpg}
		\caption[Outliers vs Influential points]{Outliers (red) vs Influential (blue) points}
	\end{figure}
	
	\begin{tcolorbox}[colback=red!5,borderline={1mm}{2mm}{red!5},arc=0mm,boxrule=0pt]
	\bcbombe Be very very very careful with influential points!!! Even if they follow the global pattern, they may quite significantly (and surprisingly) modify the regression equation and the importance of the different coefficients of that same equation (see further below for an example)!
	\end{tcolorbox}
	
	We have seen just earlier the "leverage". But let us introduce other influential indicators that are very common in the 20th century and beginning of 21st century in many statistical softwares and textbooks!
	
	The "\NewTerm{semistudentized residuals}\index{semistudentized residuals}" are defined as:
	
	But... Let us recall that we have proved just earlier that:
	
	Hence:
	
	Let us rewrite this in term of noise:
	
	Since $\text{E}(\vec{\varepsilon})=\vec{0}$ and $\text{V}(\vec{\varepsilon})=\sigma^2_\varepsilon\vec{\mathds{1}}$, we have:
	
	and:
	
	What does this imply for the residual at the $i$th data point? First, and obviously, it has expectation of $0$:
	
	and it has variance which depends on $i$ through the hat matrix:
	
	In other words, the bigger the leverage of $i$, the smaller the variance of the residual is. This is yet another sense in which points with high leverage are points which the model tries very hard to fit.
	
	Previously, when we looked at the residuals, we expected them to all be of roughly the same magnitude. This rests on the leverages $H_{ii}$ being all about the same! If there are substantial variations in leverage across the data points, it's better to scale the residuals by their expected size.

	The usual way to do this is through the "\NewTerm{standardized residuals}" or "\NewTerm{internally studentized residuals}\index{internally studentized residuals}" or just simply "\NewTerm{studentized residuals}\index{studentized residuals}":
	
	Why "studentized"? Because we're dividing by an estimate of the standard error, just like in Student's $T$-test for differences in means.
	
	However the single $e_i$ and $\hat{\sigma}_\varepsilon$ are non independent, so $r_i$ can't have a $T$ distribution.
	
	\begin{tcolorbox}[title=Remark,colframe=black,arc=10pt]
	Indeed, let us recall that in the section Statistics we have seen that given $Z$ a random variable of Normal law centered and reduced $\mathcal{N}(0,1)$ and $U$ a variable independent of $Z$ and distributed according to the law of $\chi^2$ at $k$ degrees of freedom. By definition, the variable:
	
	follows a Student law of $k$ degrees of freedom.
	\end{tcolorbox}
	 The procedure is then to delete the $i$th observation, fit the regression function to the remaining $n-1$ observations, and get new $\hat{y}$'s which can be denoted by $\hat{y}_{i(i)}$. The difference:
	
	and named "\NewTerm{deleted residual}\index{deleted residual}". 
	
	Then we define the "\NewTerm{studentized deleted residuals}\index{studentized deleted residuals}" or "\NewTerm{externally studentized residuals}\index{externally studentized residuals}":
	
	Then if the errors are independent and normally distributed with expected value $0$ and variance $\sigma^2_\varepsilon$, then the probability distribution of the $i$ externally studentized residual is a Student's $t$-distribution with $N-k-1$ degrees of freedom:
	
	
	\begin{tcolorbox}[title=Remark,colframe=black,arc=10pt]
	In some statistical softwares we can found a influential indicator named DFITS (or also sometimes DFFITS) that is the acronym of "\NewTerm{DiFference In fiTS}\index{difference in fits (DFIT)}" and defined by:
	
	DFITS is very similar to the externally studentized residual, and is in obviously equal to the latter times $\sqrt{H_{ii}/(1-H_{ii})}$.
	\end{tcolorbox}
	Probably one of the most famous influential indicator is the "\NewTerm{Cook's distance}\index{Cook's distance}". Omitting point $i$ will generally change all of the fitted values, not just the fitted value at that point. Hence the idea to measure the total change and to define this distance as the sum of all the changes in the regression model when observation $i$ is removed from it:
	
	To make that relation more comparable across data sets, it's conventional to divide it by $p$, since there are really only $p$ independent coordinates here, each of which might contribute something on the order of $\hat{\sigma}_{\varepsilon}$:
	
	There are different opinions regarding what cutoff values to use for spotting highly influential points. A simple operational guideline of $D_{i}>1$ has been suggested. Others have indicated that $D_i>4/n$ where $N$ is the number of observations, might be used.
	
	\begin{figure}[H]
		\centering
		\includegraphics[scale=0.7]{img/computing/influencial_point.jpg}
		\caption{Importance of an Influence point on regression equation}
	\end{figure}
	
	\paragraph{Significance of regression coefficients (variable importance)}\label{variable importance GML}\mbox{}\\\\
	When you study variable importance, you should always first remember to standardize all your predictors (thus, centered and scaled by the sample standard deviations).
	
	Now let us deal first with the Gaussian multivariate regression case! Let us recall that (see page \pageref{law of GLM coefficients}):
	
	Thus:
	
	under the null $H_0:\beta_i=0$ we would actually have a simple $Z$ test:
	
	This relation is again a kind of "\NewTerm{Wald statistics}\index{Wald statistics}" and the related hypothesis also belong to the family of a Wald tests\index{Wald test} (not the first we meet in this book). Also often denoted:
	
	or:
	
	Now let us introduce a necessary intermediate result. Remember that for the bivariate regression, we have proved that:
	
	Let us divide by the supposed true value of $\sigma_\varepsilon$:
	
	We recognize here a sum of square of Normal distribution that led us to write:
	
	That we will guess can be generalized to the multivariate case by:
	
	Let us rearrange and take the square root:
	
	Let us divide:
	
	by this previous result. Then:
	
	We recognize here well the term on the right. It's the definition itself the Student distribution (\SeeChapter{see Statistics section page \pageref{student distribution}}). Then we can write:
	
	That latter relation is also sometimes denoted:
	
	It explains why, in regression analysis and in variable importance, greater the Student $T(n-k)$ value is (in absolute value) the more the coefficient is significant ($H_1:\;\beta_i\neq 0$) and the more the corresponding predictor is important (hence the expression "\NewTerm{variable importance}\index{variable importance}")!
	
	In logistic (and Poisson) regression that we will see later, the variance of the residuals is related to the mean. If $Y \sim \mathcal{B}(n, p)$, the mean is $\text{E}(Y)=n p$ and the variance is $\text{V}(Y)=n p(1-p)=\text{E}(Y)(1-p)$ so the variance and the mean are related. In logistic and Poisson regression, but not in regression with gaussian errors, we then know the expected variance and don't have to estimate it separately. 
	
	In statistics, the "dispersion parameter" $\phi$ indicates if we have more or less than the expected variance. If $\phi=1$ this means we observe the expected amount of variance, whereas $\phi<1$ means that we have less than the expected variance (named "underdispersion") and $\phi>1$ means that we have extra variance beyond the expected variance (named "overdispersion"). The dispersion parameter in logistic and Poisson regression is fixed at $1$ which means that we can use the $Z$-score (that's why statistical softwares give the $Z$-score for the coefficients of such regression models!). In other regression types such as Normal linear regression, we have to estimate the residual variance and thus this is why, a $T$-value is used for calculating the $p$-values (that's why statistical softwares give the $T$-score for the coefficients of such regression models!). 

	\begin{itemize}
		\item Linear Regression:
		
		The use of $T$-tests in linear regression comes from the distribution of Normally distributed error terms: $y_{i}=X_{i}^T \beta+\varepsilon_{i}$ where $\varepsilon_{i} \sim \mathcal{N}(0,1)$ iid. It follows then as seen above that $\frac{\hat{\beta}_{j}-\beta_{j 0}}{\widehat{\text{se}}\left(\hat{\beta}_{j}\right)} \sim T(n-k)$. 
		
		Note that the default in most regression software packages test the hypothesis that $\hat{\beta}_{j}=0,$ i.e. setting $\beta_{j 0}$ equal to zero.


		\item Logistic Regression:
		
		Logistic regression assumes errors follow the logistic distribution. Consequently, the term $\frac{\hat{\beta}_{j}-\beta_{j 0}}{\widehat{\text{se}}\left(\hat{\beta}_{j}\right)}$ does not follow a $T$-distribution. Instead, we can use the Wald test, which relies on asymptotic normality as is implied by the Central limit Theorem.
	\end{itemize}
	
	Note, though, that even if the $T$-test and Wald test are asymptotically equivalent (i.e. as the sample size, $n \rightarrow +\infty,$ they will reject the same cases); certainly some people - if a bit loosely - name a test based on a $T$-statistic a Wald test, whether the statistic is compared with the asymptotic Normal distribution or the small sample result ($T$-distribution).
	
	
	\paragraph{Variance Inflation Factor (multicollinearity detection)}\mbox{}\\\\
	In statistics, the "\NewTerm{variance inflation factor}\index{variance inflation factor}\label{variance inflation factor}" (VIF) quantifies the severity of multicollinearity in an ordinary least squares regression analysis. It provides an index that measures how much the variance (the square of the estimate's standard deviation) of an estimated regression coefficient is increased because of collinearity.

	To start, the reader must know that during our study of Principal Component Analyses (\SeeChapter{see section Statistics page \pageref{principal component analysis}}) we have proved in a trivial way that the variance-covariance matrix is given by:
	
	with for recall (for a regression model with $p$ terms, and hence $p-1$ variables):
	
	Therefore for example (and for recall again) for the special case of a $2\times 2$ design matrix:
	
	Hence:
	
	And for a $3\times 3$ matrix:
	
	And as the matrix of correlations is given by (\SeeChapter{see section Statistics page \pageref{correlation matrix}}):
	
	Then we get:
	
	The ideal would also be to be able to put the inverse of the standard deviations in a matrix form in order to have a totally matrix expression. After some trial and error we quickly find that we can write the correlation matrix in the form:
	
	where $S^{-1}$ is the usual notation to say that it is the diagonal matrix containing the inverse of the $\sigma_j$, or explicitly (so that it is clear to the reader!):
	
	So for now, we have two relations (out of three ... waiting for the proof of the third one) that will be very useful to us (attention the components of the vectors of the explanatory variables are always centered henceforth what implies a forced linear regression by the origin!):
	
	\begin{tcolorbox}[title=Remark,colframe=black,arc=10pt]
	Because $X_c^TX_c$ has the first row and column only with zeros (due to recall by the constant of the model) that generates issues when inverting the matrix (ie $(X_c^TX_c)^{-1}$), we then eliminate this column and row. To highlight this choice, a few textbooks write instead:
	
	where the $(-1)$ is here to indicate that the first row and first column were removed. For what will follow we did not make this choice of notation, so keep in mind now that when you see $X_c$, it is the "design matrix" but without the first column, or said differently: $X_c^TX_c$ without the first column \underline{and} without the first row.
	\end{tcolorbox}
	Now let's start from (don't forget that starting from now $\vec{\beta}$ does not contain $\beta_0$ anymore!!!):
	
 	and let us rewrite this in a somewhat more explicit form:
	
	Therefore:
	
	We understand now better why $(X_c^TX_c)^{-1}$ is often named "\NewTerm{dispersion matrix}\index{dispersion matrix}" as we already mentioned above.

	Let us now take the matrix of the covariance-variances matrix of the coefficients (measurement of multicollinearity):
	
 	Using the following property of the transposed matrices proved in the section Linear Algebra:
	
	we then have using this property this property a first time:
	
	and using it a second time:
	
	Using now the following property proven in the section Linear Algebra:
	
	We then have:
	
	and reusing:
	
	it comes:
	
	Using the associativity property of the matrix product, we will write:
	
	And by taking out the elements which are not random variables of the expected mean, it comes:
	
	It is sadly often traditional (in spite of the confusion) to write this last result in the following form which we will see (and use) in the section of Industrial Engineering during our study of the Box-Behnken experimental designs:
	
	\begin{tcolorbox}[title=Remark,colframe=black,arc=10pt]
	At this point, notice something important! As by assumption for the Gaussian multivariate linear regression we have $Y=\mathcal{N}(X\vec{\beta},\sigma_\varepsilon^2\mathds{1}_k)$ as a consequence, we then have using the result above\label{law of GLM coefficients}:
	
	This relation will be important for us later in our study of variable importance in the case of regression models and statistical significant test for the coefficient of the regression!
	\end{tcolorbox}
	This last matrix $\text{V}\left(\vec{\hat{\beta}}\right)$ contains in the diagonal the general expression in the general case of multi-linear regression the "\NewTerm{standard error of the regression coefficients}\index{standard error of the regression coefficients}\label{standard error of the regression coefficients}" (relation that uses among other the spreadsheet software Microsoft Excel in the case of the multi-linear regression for the column "Standard error" of the coefficients ... except that it takes the square root to have the standard deviation and not the variance!). If the linear regression is not forced at the origin, it will be necessary to use obviously:
	
	with for recall:
	
	where $k$ is for recall the number of coefficients of the multiple linear model.

	We have now three important relations\footnote{There is a fourth one in reality that is also important...! The bias-variance tradeoff of the ordinary least squares regression as derived in details at page \pageref{bias-variance tradeoff ols}.}:
	
	Let us take up the first relation and manipulate it a little bit:
	
	It comes then:
	
	Then we have:
	
	By substituting the right-hand term for equality in the third of the equalities, we then have:
	
	If we denote by $\text{V}\left(\vec{\hat{\beta}}\right)_j$ the $j$-th diagonal element of the covariance-variance matrix of the $\vec{\hat{\beta}}$ and $\text{VIF}_j$, the "\NewTerm{Variance Inflation Factor (VIF)}\index{variance inflation factor}", the $j$-th diagonal element of the matrix $R^{-1}$ then it trivially follows that:
	
	The "\NewTerm{scaled Variance Inflation Factor}" is sometimes used instead and defined by:
	 
	\begin{tcolorbox}[title=Remark,colframe=black,arc=10pt]
	Remember that if the variances are all equal to the unit ($\text{V}_i=1$) then the covariance matrix is equal to the correlation matrix, ie $R=\Sigma$. Hence if we work with the $\text{VIF}_{js}$, all the information is then also contained in the inverse of the covariance matrix, ie $\Sigma^{-1}=\Theta$ named the "\NewTerm{precision matrix}\index{precision matrix}".
	\end{tcolorbox}
	However, we must find the explicit formulation of the $\text{VIF}_j$. For this, let us recall that $R$ is a positive definite symmetric matrix (\SeeChapter{see section Linear Algebra page \pageref{positive definite matrix}}). Let us, for example, calculate the inverse of such a matrix of dimension $2\times 2$ explicitly by using the relation proved in the section Linear Algebra that is for recall:
	
	In the special case that interests us, we have then:
	
	and since the correlation matrix is symmetric and unit diagonal, we have:
	
	It comes then that if we focus on the elements of the diagonal:
	
	Repeating the job with a $3\times 3$ matrix, we have:
	
	Let us focus on the first diagonal component and rearrange it a bit to get something similar to the $2\times 2$ case:
	
	Now we put:
	
	So that:
	
	Where we define:
	
	as the "\NewTerm{multiple correlation coefficient}\index{multiple correlation coefficient}" (for three variables...). This is by definition the way we calculate the correlation between $x_1$ and the other explanatory variables:
	
	
	And the reader can check that regardless of the size of the matrix, the $\text{VIF}_j$ can always be put in the form:
	
	In the case of $2\times 2$ correlation matrix there is therefore only one coefficient $\text{VIF}_j$ for the simple and good reason that the associated linear regression has only two coefficients ... 
	\begin{tcolorbox}[title=Remark,colframe=black,arc=10pt]
	Some textbooks give the following relation for the general case:
	
	But this is false as far as I know!
	\end{tcolorbox}
	
	Most statistical softwares simply use the following relation:
	
	So we understand (how the name suggests...) why the variance inflation factor (VIF) quantifies how much the variance of the coefficients is inflated when multicollinearity exists.
	
	Therefore:
	
	Either in the case where the data are such that the regression is not forced at the origin:
	
	Therefore the version I prefer to use and to teach (especially for customers and students having only a spreadsheet software at their disposition is):
	
 	and as all terms are known to us in practice, the calculation of the $\text{VIF}_j$ doesn't make us problems anymore!
 	
	A rule of thumb is that if $\text{VIF}_j>10$ then multicollinearity is high (solving backwards this translates into an $R_j^2$ value of $0.90$). Obviously a VIF of $1$ means that there is no correlation among the $k$th predictor and the remaining predictor variables, and hence the variance of $\beta_k$ is not inflated at all. VIFs exceeding $4$ should warrant further investigation.
	\begin{tcolorbox}[title=Remark,colframe=black,arc=10pt]
	Some softwares instead calculates the tolerance which is just the reciprocal of the VIF. The choice of which to use is a matter of personal preference.
	\end{tcolorbox}
	The square root of the variance inflation factor finally indicates how much larger the standard error is, compared with what it would be if that variable were uncorrelated with the other predictor variables in the model.
	\begin{tcolorbox}[colframe=black,colback=white,sharp corners]
	\textbf{{\Large \ding{45}}Example:}\\\\
	If the variance inflation factor of a predictor variable were $5.27$ ($\sqrt{5.27} = 2.3$) this means that the standard error for the coefficient of that predictor variable is $2.3$ times as large as it would be if that predictor variable were uncorrelated with the other predictor variables.
	\end{tcolorbox}
	Regardless of your criterion for what constitutes a high VIF, there are situations in which a high VIF is not a problem and can be safely ignored. For example, if you specify a regression model with both $x$ and $x^2$, there is a good chance that those two variables will be highly correlated. 
	
	\paragraph{Weighted Least Squares (WLS)}\mbox{}\\\\
	The method of ordinary least squares\footnote{A rule of thumb for OLS regression is that it isn't too impacted by heteroscedasticity as long as the maximum variance is not greater than four times the minimum variance.} assumes that there is constant variance in the errors (which is called homoscedasticity). The method of "\NewTerm{weighted least squares}\index{weighted least squares}" (WLS) can be used when the ordinary least squares assumption of constant variance in the errors is violated (which is called heteroscedasticity). The model under consideration is:
	
	where we consider now that the $\varepsilon_i =\mathcal{N}(0,\sigma_i)$. Hence the errors may not be homoscedastic!
	
	Let us recall the likelihood for the multivariate linear gaussian regression:
	
	In the heteroscedatic case that interest us here the likelihood can only be written as:
	
	named the "\NewTerm{generalized least squares regression likelihood}\index{generalized least squares regression likelihood}\label{generalized least squares regression likelihood}".
	
	So in the case of standard linear regression we have $\sigma_i=\sigma\; \forall i$ and all of diagonal terms that are zero, therefore:
	
	If we take the logarithm of the likelihood and ignoring the terms that doesn't content the $\beta_i$ we then get:
	
	The idea is to rewrite the previous relation as:
	
	with:
	
	Or in matrix form:
	
	\begin{tcolorbox}[title=Remark,colframe=black,arc=10pt]
	The reader must understand that we statistical softwares require the matrix $W$ this is just because they will replace the calculated $\hat{\sigma}_i$ by the corresponding $w_i$ and not multiply them together! And by the way... as all off-diagonal terms of $W$ are zero, most statistical softwares require only a vector corresponding to the values of the diagonal of $W$.
	\end{tcolorbox}
	Or in matrix notation this will be denoted:
	
	
	So for weighted regression, it is the weighted residuals that we try to make homoscedastic! Such that:
	
	
	Let us consider the univariate case:
	
	To determine the intercept we then calculate:
	
	Hence:
	
	And similarly:
	
	Hence:
	
	And similarly:
	
	Notice that for the two relations:
	
	If $w_i=1$ for all $i$ then we fall back on OLS.
	
	Now let us generalize that result. First notice that:
	
	will be rewritten in matrix form (we get rid of the factor $-1/2$ that doesn't bring anything):
	
	Let us expand the last equality (using the property of transposed matrices as seen on page \pageref{transposed matrix} in the section of Linear Algebra):
	
	Now we seek the minimum of the latter relation (Weighted Sum of Squares Errors) with respect to the $\beta_i$:
	
	For this we differentiate with respect to $\vec{\beta}$. Obviously the first term will vanish so it remains only:
	
	Notice now that we have also using the properties of transposed matrices:
	
	without forgetting that as $W$ is diagonal we have $W=W^T$. Therefore:
	
	Rearranging we get:
	
	Again we notice that if $W$ is a diagonal unitary matrix, we fall back on the ordinary multivariate regression coefficients. 
	
	\begin{tcolorbox}[title=Remark,colframe=black,arc=10pt]
	Notice that exactly the same developments can lead us to the coefficient of the general ordinary least squares then given by:
	
	Since we can write $\Sigma=SS^T$, where $S$ is a triangular matrix using the Cholesky decomposition (\SeeChapter{see section Linear Algebra page \pageref{Cholesky decomposition}}), we have for the general sum of squares errors:
	
	So GLS is like regressing $S^{-1}X$ on $S^{-1}\vec{y}$. Furthermore:
	
	So we have a new regression equation $y'$ where as we can see from the GSSE can be assimilated a simple regression equation which has uncorrelated errors with equal variance.
	\end{tcolorbox}
	
	In practice, for other types of dataset, the structure of $W$ is usually unknown, so we have to perform an ordinary least squares (OLS) regression first. Provided the regression function is appropriate, the $i$-th squared residual from the OLS fit is an estimate of $\sigma^2_i$ and the $i$-th absolute residual is an estimate of $\sigma_i$ (which tends to be a more useful estimator in the presence of outliers). The residuals are much too variable to be used directly in estimating the weights, $w_i$, so instead we use either the squared residuals to estimate a variance function or the absolute residuals to estimate a standard deviation function. We then use this variance or standard deviation function to estimate the weights.
	
	Some key points regarding weighted least squares are:
	\begin{itemize}
		\item The difficulty, in practice to determine the right weights (in some cases, the values of the weights may be based on theory or prior research).
		
		\item Weighted least squares estimates of the coefficients will usually be nearly the same as the "ordinary" unweighted estimates\footnote{Same applies for the ANOVA of the regression obviously.}. In cases where they differ substantially, the procedure can be iterated until estimated coefficients stabilize (often in no more than one or two iterations); this is named "\NewTerm{iteratively reweighted least squares}\index{iteratively reweighted least squares}".
		
		\item In designed experiments with large numbers of replicates, weights can be estimated directly from sample variances of the response variable at each combination of predictor variables.
		
		\item Use of weights will (legitimately) impact the widths of statistical intervals.
		
		\item In the transformed model, the interpretation of the coefficient estimates can be difficult. In weighted least squares the interpretation remains the same as before.

		\item Weighted least squares gives us an easy way to remove one observation from a model by setting its weight equal to 0.

		\item We can also downweight outlier or influential points to reduce their impact on the overall model.
	\end{itemize}
	
	\begin{tcolorbox}[title=Remark,colframe=black,arc=10pt]
	The reader must keep in mind that this technique of weights can be applied to numerous techniques of regression. Not only linear, but also polynomial, logistic and so on!
	\end{tcolorbox}
	
	\paragraph{Model Selection Criterias}\mbox{}\\\\
	Much of modern scientific enterprise is concerned with the question of model choice. An experimenter or researcher collects data, often in the form of measurements on many different aspects of the observed units, and wants to study how these variables affect some outcome of interest. Which measures are important to the outcome? Which are not? Are there interactions between the variables that need to be taken into account? Statisticians are also naturally involved in the question of model selection, and so it should come as no surprise that many approaches have been proposed over the years for dealing with this key issue. Both frequentist and Bayesian schools have weighed in on the matter, spawning methods such as $F$ tests, Akaike information criterion (AIC), Mallows's $C_p$, exhaustive search, stepwise, backward, and forward selection procedures, cross-validation, Bayes factors of various flavours (partial, intrinsic, pseudo, fractional, posterior), Bayesian information criterion (BIC), and Bayesian model averaging, to name some of the more popular and well-known methods. Some of these, such as stepwise selection, are algorithms for picking a 'good' (or maybe useful) model; others, for example, AIC, are criteria for judging the quality of a model. Given this wealth of choices, how is a statistician to decide what to do? An approach that cannot be implemented or understood by the scientific community will not gain acceptance. This implies that, at the very least, we need a method that can be carried out easily and yields results that can be interpreted by scientifically and numerically literate end users. From a statistical point of view, we want a method that is coherent and general enough to handle a wide variety of problems. Among the demands we could make on our method would be that it obeys the likelihood principle, that it has some frequentist (asymptotic) justification, and that it corresponds to a Bayesian decision
problem. So let's begin from easier do derive to the most complicated one keeping in mind that there is actually in this beginning of the 21st century no magic procedures to get you the "best model" (this still remains an unsolved problem)!
	
	\subparagraph{Mallows's $C_p$ Linear Regression Prediction}\mbox{}\\\\
	We estimated our model by minimizing the mean squared error on our data\footnote{The whole text and developments below are mostly a copy paste of the PDF of the professor Cosma Shalizi}:
		
	Different linear models will amount to different choices of the design matrix $X$ --- we add or drop variables, we add or drop interactions or polynomial terms, etc., and this adds or removes columns from the design matrix (typical how works the "subset method" in many statistical softwares).  We might consider doing selecting among models themselves by minimizing the MSE.  This may sometimes (but not always!) be a very bad idea, for a fundamental reason: Every model can be too optimistic about how well it will actually predict (sometimes the ultimate validation of a model is to test its predictive capacity).
	
	Let's be very clear about what it would mean to predict well.  The most challenging case would be that we see a new random point, 	with predictor values $X_1, \ldots X_p$ and response $Y$, and our old $\widehat{\vec{\beta}}$ has a small expected squared error:
	
	Here both $Y$ and the $X$'s are random (hence the capital letters), so we might be asking the model for a prediction at a point it never saw before (of course if we have multiple identically distributed $(X,Y)$ pairs, say $q$ of them, the expected MSE over those $q$ points is just the same as the expected squared error at one point.)
	
	An easier task would be to ask the model for predictions at the same values of the predictor variables as before, but with different random noises. That is, we fit the model to:
	
	The test data consist of new outcome data drawn from the same true model and at the same $x$-locations as the training data (Note, this is a practical construction to derive the criterion, but not a necessity to
use this for model selection. However, you should not use model selection criteria for a specific range of $x$ and assume you can predict well on a different range of $x$!).
	
	Let us note for the training data the model as following:
	
	with $\varepsilon = \mathcal{N}(0,\sigma_\varepsilon^2)$ and for the test data:
	
	where $\varepsilon$ and $\varepsilon'$ are independent but identically distributed!  The design matrix is the same, the true parameters $\vec{\beta}$ are the same, but the noise is different\footnote{If we really  are in an experimental setting, we really could get a realization of $\vec{Y}^{\prime}$ just by running the experiment a second time.  With  surveys or with observational data, it would be harder to actually realize $\vec{Y}^\prime$, but mathematically at least it's unproblematic.}.  We now want to see if the coefficients we estimated from $(X, \vec{Y})$ can predict $(X, \vec{Y}')$.  Since the only thing that's changed is the noise, if the coefficients can't predict well any more, that means that they were really just memorizing the noise, and not actually doing anything useful.
	\begin{tcolorbox}[title=Remark,colframe=black,arc=10pt]
	If we have $p-1$ variables, there are $2^{p-1}$ possible subset models!
	\end{tcolorbox}
	Our "\NewTerm{out-of-sample expected MSE}", then, is:
	
	It will be convenient to break this down into an average over data points, and to abbreviate $X\hat{\vec{\beta}} = \hat{\vec{m}}$, the vector of fitted values.  Notice that since the predictor variables and the coefficients aren't changing, our predictions are the same both in and out of sample. At point $i$, we will predict $\vec{m}_i$.
	
	In this notation, then, the expected out-of-sample MSE is:
	
	We will compare this to the "\NewTerm{expected in-sample MSE}":
	
	Notice that $\hat{m}_i$ is a function of the $y_i$ (among other things), so those are dependent random variables, while $\hat{m}_i$ and $y_i^\prime$ are completely statistically independent\footnote{That might sound weird, but remember we're holding $X$ fixed in this exercise, so what we mean is that knowing $\hat{m}_i$ doesn't give us an extra information about $y_i^\prime$ beyond what we'd get from knowing the values of the $X$ variables.}.
	
	Break this down term by term.  What's the expected value of the
	$i^{\mathrm{th}}$ in-sample squared error? Using Huygens relation given for recall by $\text{V}(X)=\text{E}(X^2)-\text{E}(X)^2$ we then have:
	
	The covariance term is not (usually) zero, because, as already mentioned, $\hat{m}_i$ is a function of, in part, $y_i$.
	
	On the other hand, what's the expected value of the $i$-th squared error on new data? Using again Huygens relation we get:
	
	The $y_i'$ is independent of $y_i$, but has the same distribution.  This tells us that:
		
	So:
	
	Averaging over data points:
	
	
	Clearly, we need to get a handle on that sum of covariances.
	
	For a linear model, though (using the results proved also during our study of influential points on page \pageref{hat matrix}):
	
	as for recall:
	
	So, for linear models:
	
	and we know that with $p$ predictors and one intercept\footnote{Be careful, on some Internet forum, people use $X_c$ instead of $X$ then the result lead to $p$ and not $p+1$ as the first component of the diagonal for $X_c$ is equal to $0$.}:
		
	Thus, for linear models:
	
	Of course, we don't actually know the expectation on the right-hand side, but we do have a sample estimate of it, which is the in-sample MSE.  If the law of large numbers is still our friend:
	
	
	The second term on the right, $(2/n)\sigma_\varepsilon^2 (p+1)$, is the "\NewTerm{optimism}" of the model, ie the amount by which its in-sample MSE systematically under-estimates its true expected squared error.  Notice that this:
	\begin{itemize}
		\item Grows with $\sigma_\varepsilon^2$: more noise gives the model more opportunities to seem to fit well by capitalizing on chance.
	
		\item Shrinks with $n$: at any fixed level of noise, more data makes it harder to pretend the fit is better than it really is.
	
		\item Grows with $p$: every extra parameter is another control which can be adjusted to fit to the noise.
	\end{itemize}
	Minimizing the in-sample MSE completely ignores the bias from optimism, so it is guaranteed to pick models which are too large and predict poorly out of sample. If we could calculate the optimism term, we could at least use an unbiased estimate of the true MSE on new data. From one point of view, the optimism is just an estimate of the bias. 	From another point of view, it's a cost we're imposing on models for having extra parameters. 
	
	If $p$ regressors are selected from a set of $k>p$, the "\NewTerm{Mallows's $C_p$}" statistic for that particular set of regressors is defined as (it's the previous relation multiplied by $n$ and divided by $\sigma_\varepsilon^2$):
	
	
	The $C_p$ criterion, that compares the precision and bias of the full model to models with a subset of the predictors, suffers from two main limitations:
	\begin{enumerate}
		\item The $C_p$ approximation is only valid for large sample size
			
		\item The $C_p$ cannot handle complex collections of models as in the variable selection (or feature selection\footnote{Feature subset selection is the process of identifying and removing as much irrelevant and redundant information as possible. This reduces the dimensionality of the data and may allow learning algorithms to operate faster and more effectively. In some cases, accuracy on future classification can be improved; in others, the result is a more compact, easily interpreted representation of the target concept.}) problem.
	\end{enumerate}
	
	The $C_p$ statistic is often used as a stopping rule for various forms of stepwise regression (don't forget that it is more a feature selection technique than a regression one!). 
	\begin{tcolorbox}[title=Remark,colframe=black,arc=10pt]
		One way to automatically select a model is to begin with the largest model you can, and the prune (simplify) it, which can be done in several way:
	\begin{itemize}
		\item Eliminate the least-significant coefficient (thanks to the $p$-value)
	
		\item Pick your favourite model selection criterion, consider deleting each coefficient in turn, and pick the sub-model with the best value of the criterion.
	\end{itemize}
	Have eliminated a variable, one then re-estimates the model, and repeats the procedure. Stop when either all the remaining coefficients are significant (under the first option), or nothing can be eliminated without worsening the criterion. That latter procedure is named "\NewTerm{backward stepwise regression}\index{backward stepwise regression}". The method that starts with intercept only and adds variables in the same fashion is named the "\NewTerm{forward stepwise regression}\index{forward stepwise regression}" (don't forget that these two methods are more feature selection techniques than regression one!). There are naturally forward-backward hybrids.\\
	
	Because the inter-correlation between the regressors affect the order of term entry and removal.  Since we are approaching the final model from two different directions this aspect can cause the methods to converge on different models. 
	\end{tcolorbox}	
	Colin Lingwood Mallows proposed the statistic as a criterion for selecting among many alternative subset regressions. Under a model not suffering from appreciable lack of fit (bias), $C_p$ has expectation nearly equal to $p$. Indeed, if your model with $p$ parameters is correct it holds that $\text{SSE}_p\cong (n-p)\sigma_\varepsilon^2$. If your other model is already correct as well, it holds $\text{SSE}_q\cong (n-q)\sigma_\varepsilon^2$, therefore:
	
	
	Otherwise the expectation is roughly $p$ plus a positive bias term. 

	Nevertheless, even though it has expectation greater than or equal to $p$, there is nothing to prevent $C_p<p$ or even $C_p < 0$ in extreme cases. It is suggested that one should choose a subset that has $C_p$ approaching $p$, from above, for a list of subsets ordered by increasing $p$. In practice, the positive bias can be adjusted for by selecting a model from the ordered list of subsets, such that $C_p < 2p$.
	
	To summarize, we should look for models where Mallows's $C_p$ is small and close to the number of predictors in the model plus the constant $p$. A small Mallows's $C_p$ value indicates that the model is relatively precise (has small variance) in estimating the true regression coefficients and predicting future responses. A Mallows's $C_p$ value that is close to the number of predictors plus the constant indicates that the model is relatively unbiased in estimating the true regression coefficients and predicting future responses. Models with lack-of-fit and bias have values of Mallows's $C_p$ larger than $p$.
	
	\pagebreak
	\subparagraph{Akaike Information Criterion (AIC)}\mbox{}\\\\
	The Akaike information criterion, AIC, was developed by Akaike to estimate the expected Kullback-Leibler discrepancy (relative quality of statistical models for a given set of dataset) between the model generating the data and a fitted candidate model. In instances where the sample size is large and the dimension of the candidate model is relatively small, AIC serves as an approximately unbiased estimator. In other settings, AIC may be characterized by a large negative bias which limits its effectiveness as a model selection criterion\footnote{For such instances, Hurvic and Tasi (1989) proposed the corrected Akaike information criterion, AICc.}.
	
	Sometimes we have a set of possible models and we want to choose the best model. Model selection methods help us choose a good model. Here are some examples:
	\begin{tcolorbox}[colframe=black,colback=white,sharp corners]
	\textbf{{\Large \ding{45}}Examples:}\\\\
	E1. Suppose we use a polynomial to model the regression function:
	
	We will need to choose the order of polynomial $p$. We can think of this as a sequence of models $\mathcal{M}_1,\ldots,\mathcal{M}_k$ index by $k$.\\
	
	E2. Suppose you have data $y_1,\ldots,y_n$ on age at death of $n$ people. You want to model the distribution of $y$. Some popular models are:
	\begin{itemize}
		\item $\mathcal{M}_1$: The exponential distribution: $p(y,\theta)=\theta e^{-\theta y}$
		
		\item $\mathcal{M}_2$: The gamma distribution: $p(y,a,b)=(b^a/\Gamma(a))y^{a-1}e^{-by}$
		
		\item $\mathcal{M}_3$: The log-normal distribution: we take $\log(y)\cong \mathcal{N}(\mu,\sigma^2)$
	\end{itemize}
	
	E3. Suppose you have time series data $y_1, y_2,\ldots$ A common model is the AR (autoregressive model):
	
	where $\varepsilon_t\cong \mathcal{N}(0,\sigma^2)$. The number $k$ is the order of the model as we know. We need to choose $k$.\\
	
	E4. In a linear regression model, you need to choose which variables to include in the regression. This is called variable selection.
	\end{tcolorbox}
	\begin{tcolorbox}[title=Remark,colframe=black,arc=10pt]
	The AICc is useful for selecting between models in the same class. For example, we can use it to select an ARIMA model (\SeeChapter{see section Economy page \pageref{arima}}) between candidate ARIMA models or an ETS model between candidate ETS models (\SeeChapter{see section Economy page \pageref{ETS models}}). However, it cannot be used to compare between ETS and ARIMA models because they are in different model classes, and the likelihood is computed in 
	\end{tcolorbox}
	Suppose we have models $\mathcal{M}_1,\ldots,\mathcal{M}_k$ where each model is a set of densities (to simplify the notations we will consider only univariate models):
	
	We have data $y_1,\ldots,y_n$ drawn from some density $f$. We do not assume that $f$ is any of the models $\mathcal{M}_j$!

	Let $\hat{\theta}_j$ be the maximum likelihood estimator from model $j$. An estimate of $P$, based on model $j$ is $\hat{p}_j(y)=p(y,\hat{\theta})$. The quality of $\hat{p}_j(y)$ as an estimate of $f$ can be measured by the Kullback-Leibler divergence given for recall in its continuous form by (\SeeChapter{see section Statistical Mechanics page \pageref{kullback-leibler divergence}}):	
	
	That we will write here as:
	
	The first term does not depend on $j$. So minimizing $D_\text{KL}(p,\hat{p}_j)$ over $j$ is the same as maximizing (don't forget that we have proved that $D_\text{KL}$ was positive definite such that $D_\text{KL}\geq 0$):
	
	We need to estimate $D_\text{KL,j}$. Intuitively, a parametric estimate of $D_\text{KL,j}$ is given by the arithmetic average (not such as good as taking the median but more easy to deal for mathematical developments):
	
	 However, apart from the fact that this estimate is non-robust, it is also probably quite biased. Akaike proved that the bias is approximately $d_j/n$ where $d_j=\text{dim}(\vec{\Theta}_j)$. Therefore we use for the model $\mathcal{M}_j$:
	
	That is to say a measure of a in-sample performance plus a penalty (or bias). Defining on the way the "\NewTerm{Akaike information criterion\footnote{Actually, in his original paper, Hirotugu Akaike, proposed using the factor $2$ to simplify some calculation involving chi-squared distribution. Many subsequent specialists have since kept the factor of $2$ which of course will not change which model is selected. Also some authors define AIC as negative of this, and then minimize it; again, clearly the same thing!}}\index{Akaike information criterion}\label{Akaike information criterion}" (AIC):
	
	Notice that maximizing $\hat{D}_{\text{KL},j}$ is the same as maximizing $\text{AIC}(j)$ over $j$. Why do we multiply by $2n$? Just for historical reasons! We can multiply by any constant; it won't change which model we pick. In fact, different texts use different versions of AIC.
	
	\begin{dem}
	Let us recall that during the study of the Fisher information matrix (\SeeChapter{see section Statistics page \pageref{Fisher information matrix}}) we have defined the score function for a parameter $\theta_j$ of a distribution function depending of a set of parameters $\vec{\theta}$ as (keep in mind that we can take the $\ln$ or the $\log$, the result remains the same!):
	
	We know that the score function is used to build the estimate of parameters of the distribution (when we put the score equal to zero). And that therefore if the true parameters are known, then it always equal to zero (at least of non-strange distribution functions)!
	
	Let us recall that also during the study of this of bivariate Taylor series  (\SeeChapter{see section Sequences and Series page \pageref{hessian matrix}}), we have introduced the Hessian matrix, the matrix of second derivatives, that we will denoted $H(\vec{\theta})$ (meaning implicitly that it is evaluated at $\vec{\hat{\theta}}$.
	
	Let us recall that during our study of the central limit theorem (\SeeChapter{see section Statistics page \pageref{central limit theorem}}) we have proved that:
	
	Therefore:
	
	For a bivariate or multivariate normal distribution (\SeeChapter{see section Statistics page \pageref{bivariate normal distribution}}), this will be written:
	
	For the case that interest us here, let us denote this:
	
	With as we have seen during our study of the Fisher information Matrix:
	
	Now that we have finish the recalls. Let us take the Taylor series of:
	
	considering $\vec{\hat{\theta}}$ as the variable and doing the development around the true value of the estimator that we will denote $\vec{\theta}_0$:
	
	At the true value $\vec{\theta}_0$ we have (don't forget that this partial derivative is the score, ie the partial derivative of the likelihood and that latter is always equal to zero for the true model with the real parameters!):
	
	Then it remains:
	
	That we can also write:
	
	The $\vec{Z}_n$ doesn't depend explicitly on $y$ then we can write:
	
	Now let us recall that we have seen during our study of the Fisher Information matrix that:
	
	So in the previous integral between the parenthesis we have the same expression apart of the notation: $p$ instead of $L$ and $\log$ instead on $\ln$ but however the manipulated objects are the same! Hence:
	
	That we will denote to simplify the next developments as following:
	
	Now let us do the same Taylor series but for the approximate model, with the real data:
	
	Now we assume (...) when $n\rightarrow +\infty$ that:
	
	Therefore:
	
	\begin{tcolorbox}[title=Remark,colframe=black,arc=10pt]
	Notice that for obscure reasons we don't assume that for $n\rightarrow +\infty$:
	
	\end{tcolorbox}
	Then:
	
	Now let us recall that during our study of the Fisher information matrix we have proved that:
	
	And we have shown earlier above that:
	
	As $Z_n$ is a vector, and $-\text{E}(H)$ is a non-random matrix, according to the following relation we have proved during our study of the variance-covariance matrix:
	
	We then have for the transformed random variable $-\text{E}(H)Z_n$:
	
	As $\text{E}(H)$ is symmetric, we have $\text{E}(H)^T=\text{E}(H)$ then:
	 
	But we have proved during our study of the Fisher information matrix the information matrix equality $-\text{E}(H)(\vec{\theta})=\mathcal{I}(\vec{\theta})$. Therefore:
	 
	ie:
	
	and we recognize here $\sqrt{n}\vec{\hat{\mathcal{S}}}(\theta_0)$. So that finally:
	
	So:
	
	Now we will use the following relation also proved during our study of the variance-covariance matrix (\SeeChapter{see section Statistics page \pageref{quadratic relation for akaike information criterion}}):
	
	Therefore:
	
	But by construction $\vec{Z}_n$ we have $\vec{\mu}=\vec{0}$ and also using the matrix information equality and the definition of the Fisher information matrix:
	 
	ie:
	
	So that finally:
	
	Remembering as we have proved it that the bigger is the better!
	\begin{tcolorbox}[title=Remark,colframe=black,arc=10pt]
	As we can factorize the $1/n$ and that when we compare two or more AIC by subtracting them the constants vanishes (the number of parameters $d$ of different models may not be the same!), the latter relation is often used in the following form:
	
	that will be used as the definition of the AIC itself (without any $2n$ factor)! These different possible choices explains with software packages may well give completely different AICs on the same data for the same model.
	\end{tcolorbox}
	\begin{flushright}
		$\blacksquare$  Q.E.D.
	\end{flushright}
	\end{dem}
	There, as you have very likely (...) noticed it, several steps where we are making a bunch of approximations and assumptions. Some of these approximations (especially those involving the Taylor expansions) can be shown to be OK asymptotically (ie as $n\rightarrow +\infty$) by more careful maths. The last steps however, where we invoke the Fisher information matrix and the score are rather more dubious. So AIC is a very crude indicator. Cross-validation is much more reliable!
	\begin{tcolorbox}[colframe=black,colback=white,sharp corners]
	\textbf{{\Large \ding{45}}Examples:}\\\\
	E1. Let:
	
	We want to compare two models (notice that the first one has zero parameters to estimate and the second one has only one parameter to estimate: $\mu$):
	
	We want to test:
	
	The test statistics is as we already know:
	
	We reject $H_0$ if $|Z|>z_\alpha/2$. For $\alpha=0.05$, we reject $H_0$ if $|Z|>2$, ie if:
	
	The likelihood is obviously proportional to (we ignore the value of the variance as for both models it is equal to $1$):
	
	Hence:
	
	Recall that one of the definition of the Akaike information criterion is $\text{AIC} = \mathcal{L}(Y_i,\hat{\theta}_j)-d $. The AIC scores are then:
	
	And:
	
	\end{tcolorbox}
	
	\begin{tcolorbox}[colframe=black,colback=white,sharp corners]
	After, depending on the experimental values of the $Y_i$, we have to compare:
	
	Keeping the one that has the biggest value (as in this example their both negative, that means the one that is the nearest to zero, or the smallest AIC in absolute value)!\\
	
	E2. Remember that during our study of multiple linear regression, we have proved that: 
	
	And also earlier we have proved that:
	
	denoted also sometimes:
	
	Recall that another definition of the Akaike information criterion (the most common in statistical softwares) is $\text{AIC} = -2\mathcal{L}(Y_i,\hat{\theta}_j)+2d$. The AIC score is then when $\sigma$ is unknown for a multiple regression:
	
	What interest us when we use the AIC is to compared different models (subtract the different AIC). As $n$ remains constant between all models, we can eliminate some useless terms that will automatically vanish, this is why in many textbooks we will found:
	
	And we have seen earlier that the Fisher information matrix of the multiple linear regression had $p+1$ dimension, therefore:
	
	What is of denoted in some textbooks as:
	
	\end{tcolorbox}
	
	\begin{tcolorbox}[title=Remark,colframe=black,arc=10pt]
	A software like \texttt{R} deals with $\text{AIC} = -2\mathcal{L}(Y_i,\hat{\theta}_j)+2d$ that we have used above. Therefore the smaller is better!
	\end{tcolorbox}
	
	
	The most common model selections methods are:
	\begin{enumerate}
		\item $p$-value
		\item AIC (and related methods like $C_p$)
		\item Cross-validation (see further below)
		\item BIC (see further below).
	\end{enumerate}
	AIC is motivated by the estimation of the generalization error (like Mallows's $C_p$, BIC,...). If you want the model for predictions, better use one of these criteria. If you want your model for explaining a phenomenon, use $p$-values.
	We need to distinguish between two goals:
	\begin{enumerate}
		\item Find the model that gives the best prediction (without assuming that any of the models
	are correct).
		\item Assume one of the models is the true model and not the "true" model.
	\end{enumerate}
	Generally speaking, AIC and cross-validation are used for goal (1) while BIC is used for goal (2).
	
	AIC is founded on information theory. When a statistical model is used to represent the process that generated the data, the model will almost never be exact; so some information will be lost by using the model to represent the process. AIC estimates the relative information lost by a given model: the less information a model loses, the higher the quality of that model.

	To apply AIC in practice, we start with a set of candidate models, and then find the models' corresponding AIC values. There will almost always be information lost due to using a candidate model to represent the "true model" (i.e. the process that generated the data). We wish to select, from among the candidate models, the model that minimizes the information loss. We cannot choose with certainty, but we can minimize the estimated information loss.
	
	\subparagraph{Cross-Validation metrics}\mbox{}\\\\
	In Machine Learning, "\NewTerm{cross-validation}\index{cross-validation}\label{cross-validation}" (CV) is a resampling method used for model evaluation to avoid testing a model on the same dataset on which it was trained. It is mainly used in settings where the goal is prediction, and one wants to estimate how accurately a predictive model will perform in practice. This is a common mistake, especially that a separate testing dataset is not always available. However, this usually leads to inaccurate performance measures (as the model will have an almost perfect score since it is being tested on the same data it was trained on). To avoid this kind of mistakes, cross validation is usually preferred.
	
	Two types of cross-validation can be distinguished, exhaustive and non-exhaustive cross-validation.
	\begin{itemize}
		\item Exhaustive cross-validation methods are cross-validation methods which learn and test on all possible ways to divide the original sample into a training and a validation set.
		
		\item Non-exhaustive cross validation methods do not compute all ways of splitting the original sample.
	\end{itemize}
	\begin{figure}[H]
		\centering
		\includegraphics[width=1.0\textwidth]{img/computing/cross_validation_techniques.jpg}
		\caption[Some famous techniques of cross-over validation]{Some famous techniques of cross-over validation (source: ?)}
	\end{figure}

	The concept of cross-validation is actually simple: Instead of using the whole dataset to train and then test on same data, we could randomly divide our data into training and testing datasets.

	There are several types of cross-validation methods (LOOCV – Leave-one-out cross validation\footnote{We already know that latter as it is just the Jackknife method}, the holdout method, $K$-fold cross validation). Here, we will discuss the "\NewTerm{$K$-Fold cross validation method}\index{$K$-fold cross validation method}".

	The idea is illustrated as following:
	\begin{figure}[H]
		\centering
		\includegraphics[width=1.0\textwidth]{img/computing/cross_over_kfold.jpg}
		\caption[$K$-fold cross-over validation]{$K$-fold cross-over validation (author: ?)}
	\end{figure}	
	
	So following the above illustration, the $K$-fold basically consists of the below steps:
	\begin{enumerate}
		\item Split the data in a train and a test set
		
		\item Randomly split the train set into $K$ subsets, also named "folds".
		
		\item Train the model all $K$ subsets excepted the first one
		
		\item Fit the model on the first set that was ignored previously and keep its quality metric (ie "cross validated metric)
		
		\item Repeat the procedure (3) to (4) (but by ignoring the second set, after the third, and so on...)
		
		\item Average all the quality metrics (ordinary cross-validation: OCV)
		
		\item At the end, use all the $K$-folds to train the model and fit it on the test set
		
		\item Compare if the last fit perform better that the average of all previous fits
		
		\item If not, use one of the $K$ as reference model based on the quality metric. If yes, keep the model made in step (8)
	\end{enumerate}
	
	The typical basis of cross-over quality metrics is the mean square error MSE. Then the average quality metric, or the "\NewTerm{ordinary cross-validation}\index{ordinary cross-validation}", for a $K$ fold will be (assuming that each fitted model has a sub-sample size of $n_k$):
	
	If we denoted the model (in a univariate case) by $\hat{y}_i^{[k]}$ for the subset (submodel) $k$ and the true values by $y_i$ we then have:
	
	Therefore the OCV for an univariate model based one the MSE is :
	
	For example with LASSO, ridge or elastic net regularized regression, we have:
	
	And the "best lambda" $\lambda_0$ is the one that minimize the $\text{OCV}_\text{MSE}$ such that:
	
	
	Let us now speak about a very common relation in statistical textbooks, the cross-validation for one of the subsets of the Leave One Out method (ie Jackknife):
	
	Here, $y_i$ is the actual label value of training point $i$, ${y}_i^{[k]}$ is the value predicted by the cross-validation model trained on all points except $i$, $\hat{y}^i$ is the value predicted by the regression model trained on all points (including point $i$), and $H_{ii}$ is as we know the leverage of point $i$, given for recall by:
	
	and:
	
	Notice that the left side of the above relation is the LOOCV sum of squares error (the quantity we seek), while the right can be evaluated given only the model trained on the full data set. Fantastically, this allows us to evaluate the LOOCV error using only a single regression!!!
	\begin{dem}
	Consider the LOOCV step where we construct a model trained on all points except training example $k$. Using a linear model of form $y^{[k]}=\vec{x}^T\vec{\beta}^{[k]}$ - with $\vec{\beta}^{[k]}$ the coefficient vector - the sum of squares that must be minimized is:
	
	Here, we are using a exponent $k$ on $\vec{\beta}^{[k]}$ to highlight the fact that the above corresponds to the case where example $k$ is held out. We minimize the above relation by taking the gradient with respect to $\vec{\beta}^{[k]}$. Setting this to zero gives the following equation:
	
	Rearranging and simplifying a bit gives:
	
	Similarly, the full model (trained on all points) coefficient vector $\vec{\beta}$ (here the notation $\vec{\beta}$ means in fact $\hat{\vec{\beta}}$ obviously!) satisfies:
	
	But we have  that:
	
	can we rewritten as for the left term:
	
	and for the right term:
	
	Therefore:
	
	But as we have proved just earlier that:
	
	Then substituting in the previous relation, this lead us to:
	
	Rearranging:
	
	That can be written using the definition of $y_i^{[k]}$:
	
	Hence:
	
	Left multiplication by $\vec{x}_k^T\left(\sum_i\vec{x}_i\vec{x}_i^T\right)^{-1}$ (without forgetting that the term into parenthesis is a scalar) gives:
	
	So it only remains (rearranging a bit):
	
	We recognize here the vector definition of $H_{kk}$. Therefore:
	
	that is:
	
	That latter can be rewritten as:
	
	Hence:
	
	Squaring and summing finally leads to:
	
	\begin{flushright}
		$\blacksquare$  Q.E.D.
	\end{flushright}
	\end{dem}
	Again... Fantastically, this allows us to evaluate the LOOCV error using only a single regression!!!
	
	A minor variant on cross validation is, so-called "\NewTerm{generalized cross validation}\index{generalized cross validation}\label{generalized cross validation}", which, of course, like most things statisticians call "generalized" isn't...
	
	It replaces the $H_{kk}$ in the denominator with their average trace $\text{tr}(H_{kk})/n$ giving:
	
	
	\pagebreak
	\subsubsection{Ridge, LASSO and Elastic Net regularization}\label{regularization}
	"\NewTerm{Regularization}\index{Regularization}" also named "\NewTerm{regularized linear regression}\index{regularized linear regression}" or "\NewTerm{penalized linear regression}\index{penalized linear regression}" has been intensely studied on the interface between statistics and computer science.  

	There are two types of regularization as follows:
	\begin{itemize}
		\item $L_1$ Regularization or LASSO Regularization (see details further below)
		
		\item $L_2$ Regularization or Ridge Regularization (see details further below)
	\end{itemize}
	Where for recall $L_p:\mathbb{R}^n\mapsto\mathbb{R}$ (\SeeChapter{see section Topology page \pageref{distance}}) is defined:
	
	with $p>0$. For $p=2$, this is the familiar Euclidean distance (don't confuse with the notation of the lost function $L$ if possible...). 
	
	$L_1$ and $L_2$ are the most common types of regularization. These update the general cost function by adding another term known as the "regularization term":
	
	Or more formally if we denote by $R(f)$ the regularization term :
	
	where $V$ is an underlying loss function (most of times for regression models its the residual sum of squares RSS) that describes the cost of predicting $f(x)$ when the label is $y$ and $\lambda$ is a parameter which controls the importance of the regularization term. $R(f)$ is typically chosen to impose a penalty on the complexity of $f$. Concrete notions of complexity used include restrictions for smoothness and bounds on the vector space norm.
	
	Notice that softwares compute estimates for a large number of values for $\lambda$ at once.  The optimal $\lambda$ is selected by cross validation of some sort... The most common one in the end of the 21st century being the generalized cross-validation metric (see further below page \pageref{generalized cross validation}).
	
	From a Bayesian point of view, many regularization techniques correspond to imposing certain prior distributions on model parameters. 
	
	\begin{dem}
	Consider that the regularized loss function $L$ has a similar role as the probability of a parameter configuration $\beta$ given the observations $X$,$\vec{y}$. Applying the Bayes theorem (\SeeChapter{see section Probabilities page \pageref{bayes formula}}), we get:
	
	Taking the log of the expression gives us:
	
	Now, let's say $L(\theta)$ is the negative\footnote{Negative since we want to maximize the probability but minimize the cost.} log-posterior, $-\log(P(\theta|X,\vec{y}))$. Since the last term does not depend on $\theta$, we can ignore it without changing the minimum. We are left with two terms: 1) the likelihood term $\log (P(X,\vec{y}|\theta))$ depending on $X$ and $\vec{y}$, and 2) the prior term $\log (P(\theta))$ depending on $\theta$ only. These two terms correspond exactly to the loss and the regularization term.
	\begin{flushright}
		$\blacksquare$  Q.E.D.
	\end{flushright}
	\end{dem}
	
	\begin{tcolorbox}[title=Remark,colframe=black,arc=10pt]
	It can be proved (see further below) that $L_2$ regularization is the case with a Gaussian prior analytically and the $L_1$ is equivalent to a Laplacian prior.
	\end{tcolorbox}
	
	We can also see regularization as an application of the Lagrange Multiplier method (see previously page \pageref{Lagrange multipliers method}).

	\paragraph{$L_2$ Regularization or Ridge Regularization}\mbox{}\\\\
	When learning a linear function $f$, characterized by an unknown vector $\vec{\beta}$ such that $f(x)=\vec{\beta}\circ \vec{x}$, the $L_2$-norm loss corresponds to "\NewTerm{Tikhonov regularization}\index{Tikhonov regularization}". This is one of the most common forms of regularization, is also known as "\NewTerm{ridge regression}\index{ridge regression}", and is expressed as:
	
	where $\lambda$ is the "\NewTerm{tuning parameter}\index{tuning parameter}" or "\NewTerm{shrinkage parameter}\index{shrinkage parameter}".
	
	\begin{tcolorbox}[title=Remark,colframe=black,arc=10pt]
	If the different predictor variables don't have physically comparable units it's a good idea to standardize them first, so they all have mean $0$ and variance $1$ and also no physical units. Otherwise penalizing the $p$ predictor with the $L_2$-norm using $\sum _{i=1}^{p}\beta_i^2$ seems to be adding up apples, oranges, and the occasional bout of regret (some people always pre-standardize the predictors).
	\end{tcolorbox}
	
	The learning problem with the least squares loss function and Tikhonov regularization can be solved analytically. Written in matrix form, the optimal $\vec{\beta}$ will be the one for which the gradient of the loss function with respect to $\vec{\beta}$ is $\vec{0}$.
	
	Let us write this explicitly:
	
	which is just the usual least squares criterion with a penalty determined by $\lambda$ for large coefficient estimates. This is why it's named "\NewTerm{penalized residual sum of squares}\index{penalized residual sum of squares}" (PRSS). If $\lambda=0$ the lasso is the same as OLS; as $\lambda$ increases, shorter coefficients are preferred. Therefore we can say that most of times (but now always!) if we increase the value of $\lambda$ then the magnitude of the coefficients decreases.
	
	 Notice also that this is a well defined, convex, differentiable optimization problem even after including $L_2$ function. The $L_2$ function acts as a penalizer: $\|\vec{\beta}\|^{2}$ grows large as the coefficients $\vec{\beta}$ grow large, thus the minimization program tries to make w smaller.
	 
	 There are many variations on this procedure, including application of it to other-than the linear model.
	
	We take the derivative:
	
	To find the optimum we put:
	
	Therefore:
	
	By construction of the optimization problem, other values of $\vec{\beta}$  would give larger values for the loss function. Notice that in particular the regularization term solves the problem of non-invertibility of the first term between parenthesis.
	
	Notice that the solution is indexed by the parameter $\lambda$. So for each $\lambda$, we have a solution. The fact that we add $\lambda n\mathds{1}$ to the diagonal of ${X}^{T}{X}$ (which correspondence to the correlation matrix if $X$ is previously centered and reduced) is why this method is named "ridged" (and the term $\lambda n\mathds{1}$ is often named the "ridged").
	
	The reader may also notice that with:
	
	excepted is some very special cases, the ridge regression will never take the coefficients $\hat{\vec{\beta}}_\lambda^\text{ridge}$ to zero as the denominator will never be practically infinite! But we can see obviously that as $\lambda \rightarrow +\infty$ then $\hat{\vec{\beta}}_\lambda^\text{ridge}\rightarrow \vec{0}$.

	\begin{tcolorbox}[title=Remark,colframe=black,arc=10pt]
	Inclusion of $\lambda$ makes problem non-singular even if $X^{T}X$ is not invertible. It seems that this was the original motivation for ridge regression (Hoerl and Kennard, 1970).
	\end{tcolorbox}
	
	Let us prove now that the ridge regression corresponds indeed to a Gaussian prior on the coefficients.
	\begin{dem}
	Let us imagine that you want to infer some parameter $\beta$ from some observed input-output pairs $(x_1,y_1),\ldots,(x_n,y_n)$. Let us assume that the outputs are linearly related to the inputs via $\beta$ and that the data are corrupted by some noise $\varepsilon$:
	
	where $\varepsilon$ is Gaussian noise with as we know mean $0$ and variance $\sigma^2_\varepsilon$. This gives rise to a Gaussian likelihood:
	
	Let us regularise parameter $\beta$ by imposing the Gaussian prior $\mathcal{N}(\beta|0,\lambda^{-1})$, where $\lambda$ is a strictly positive scalar. Hence, combining the likelihood and the prior we simply have:
	
	Let us take the logarithm of the above expression. Dropping some constants we get:
	
	If we maximise the above expression with respect to $\beta$, we get the so called "\NewTerm{maximum a-posteriori estimate}\index{maximum a-posteriori estimate}" for $\beta$, or "MAPE for ridge regression" for short. In this expression it becomes apparent why the Gaussian prior can be interpreted as a $L_2$ regularisation term.
	\begin{flushright}
		$\blacksquare$  Q.E.D.
	\end{flushright}
	\end{dem}
	Let us now prove that the $\hat{\vec{\beta}}_\lambda^\text{ridge}$ are biased. 
	\begin{dem}
	First let us recall that we have proved that:
	
	Now let us rewrite this:
	
	Remember that we have proved in the section of Linear Algebra that if $A$ and $B$ are invertible, then:
	
	Then:
	
	So:
	
	Therefore if $\lambda \neq 0$ then:
	
	\begin{flushright}
		$\blacksquare$  Q.E.D.
	\end{flushright}
	\end{dem}
	We know that if we actually have the "right" model, then, say, OLS parameter estimates will be unbiased and have minimal variance among all unbiased (linear) estimators (they are BLUE). Predictions from an OLS model will be best linear unbiased predictions (BLUPs). That sounds good.
	
	So we have $\hat{\vec{\beta}}_\lambda^\text{ridge}=(X^TX+\lambda\mathds{1})^{-1}X^TY$ which is biased; but if $X$ is ill conditioned then $\text{V}(\hat{\beta})\propto (X^TX)^{-1}$ may be monstrous whereas $\text{V}(\hat{\vec{\beta}}_\lambda^\text{ridge})$ can be much more modest. Indeed, let us recall that we have proved during our study of multiple linear regression that (see above page \pageref{standard error of the regression coefficients}):
	
	Exactly the same developments would lead us to:
	
	So increasing $\lambda$ decrease indeed the variances of the $\hat{\vec{\beta}}_\lambda^\text{ridge}$.
	
	\pagebreak
	\paragraph{$L_1$ Regularization or LASSO Regularization}\mbox{}\\\\
	We describe the basic idea through the "\NewTerm{LASSO regularization}\index{LASSO regularization}\label{LASSO regularization}" (LASSO stands for: Least Absolute Shrinkage and Selection Operator) as applied in the context of linear regression.  The method starts by assuming a model like:
	
	which is just the usual least squares criterion with a penalty determined by $\lambda$ for large coefficient estimates.  Again, if $\lambda=0$ the lasso is the same as OLS; as $\lambda$ increases, shorter vectors are preferred.
	
	\begin{tcolorbox}[title=Remark,colframe=black,arc=10pt]
	If the different predictor variables don't have physically comparable units it's a good idea to standardize them first, so they all have mean $0$ and variance $1$ and also no physical units. Otherwise penalizing the $p$ predictor with the $L_1$-norm using $\sum _{i=1}^{p}|\beta_i|$ seems to be adding up apples, oranges, and the occasional bout of regret (some people always pre-standardize the predictors).
	\end{tcolorbox}
	
	Immediately we lose the differentiability of the optimization function: the absolute values are non-differentiable. 
	
	Thus we cannot solve the optimization problem using standard calculus techniques in the most general case. What do we gain then? First we can consider only the case $\vec{\beta}\neq\vec{0}$.
	
	We take the derivative:
	
	To find the optimum we put:
	
	Therefore:
	
	If you observe the numerator, it may be equal to zero, since we are subtracting some value of $\lambda$. And therefore the values of $\hat{\vec{\beta}}_\lambda^\text{LASSO}$ may be all equal to zero at the opposite of the ridge regression!
	
	One can show that for some large enough $\lambda$, the solution is a sparse solution. A sparse solution means a solution for $\vec{\beta}$ that has many zeros in it, effectively removing the corresponding variable from the system. There are then two immediate consequences:
	\begin{enumerate}
		\item The system will give a coefficient of zero to non-relevant variables, instead of some insanely small coefficient. This means independent or near-independent variables will be discarded instead of included.
		
		\item It gives us the most important variables in the regression model! We can speak confidently that variable $i$ has a strong connection with the response variable $\vec{y}$.
	\end{enumerate}

	Again notice that softwares compute estimates for a large number of values for $\lambda$ at once.  The optimal $\lambda$ is selected by cross validation of some sort.
	
	Let us prove now that the LASSO regression corresponds to a Laplace prior on the coefficients.
	\begin{dem}
	Let us imagine that you want to infer some parameter $\beta$ from some observed input-output pairs $(x_1,y_1),\ldots,(x_n,y_n)$. Let us assume that the outputs are linearly related to the inputs via $\beta$ and that the data are corrupted by some noise $\varepsilon$:
	
	where $\varepsilon$ is Gaussian noise with as we know mean $0$ and variance $\sigma^2_\varepsilon$. This gives rise to a Gaussian likelihood:
	
	First, let us recall that a Laplace distribution is given by (difference of two exponential distribution as we have proved it in the section of Statistics):
	
	\begin{tcolorbox}[title=Remark,colframe=black,arc=10pt]
	Notice that the Normal and Laplace prior have a strong similitude. This is why they are commonly denoted:
	
	Ridge regression ($p=2$) corresponds to a Gaussian prior and the LASSO ($p=1$) to a Laplacian (or
double-exponential) prior. 
	\end{tcolorbox}
	Let us regularise parameter $\beta$ by imposing the Laplace prior $\mathcal{L}(\beta|0,b)$, where $b$ is a strictly positive scalar (and mean $\mu=0$). Hence, combining the likelihood and the prior we simply have:
	
	Let us take the logarithm of the above expression. Dropping some constants we get:
	
	Let $1/b=\lambda$, we get then:
	
	If we maximise the above expression with respect to $\beta$, we get the so called "\NewTerm{maximum a-posteriori estimate}\index{maximum a-posteriori estimate}" for $\beta$, or "MAPE for LASSO regression" for short. In this expression it becomes apparent why the Laplace prior can be interpreted as a $L_1$ regularisation term.
	\begin{flushright}
		$\blacksquare$  Q.E.D.
	\end{flushright}
	\end{dem}
	
	\paragraph{$L_1+L_2$ Regularization or Elastic Net Regularization}\mbox{}\\\\
	In statistics and, in particular, in the fitting of linear or binomial logistic regression models, the elastic net is a regularized regression method that linearly combines the $L_1$ and $L_2$ penalties of the lasso and ridge methods.
	
	The elastic net method overcomes the limitations of the LASSO (least absolute shrinkage and selection operator) method which uses a penalty function based for recall on:
	
	Use of this penalty function seems to have several limitations. For example, in the "large $n$, small $N$" case (high-dimensional data with few examples), the LASSO seems to select at most $n$ variables before it saturates. Also if there is a group of highly correlated variables, then the LASSO tends to select one variable from a group and ignore the others (ie LASSO tends to select only one predictor among the predictors that are highly correlated). To overcome these limitations, the elastic net adds a quadratic part to the penalty $\|\beta \|^{2}$, which when used alone is ridge regression. The estimates from the elastic net method are defined by:
	
	The quadratic penalty term makes the loss function strictly convex, and it therefore has a unique minimum. The elastic net method includes the LASSO and ridge regression: in other words, each of them is obviously a special case where $\lambda_{1}=\lambda ,\lambda_{2}=0$ or $\lambda_{1}=0,\lambda_{2}$.
	
	Another very common notation is: 
	
	where $0\leq \alpha \leq 1$ is a compromise between ridge ($\alpha=0$) and lasso ($\alpha=1$).
	
	\begin{tcolorbox}[title=Remark,colframe=black,arc=10pt]
	Depending on the softwares, the algorithms implement some variations of the relation above. The most common one are:
	
	or:
	
	and so on.... Hence the sometimes the huge differences in the optimal $\lambda$ value.
	\end{tcolorbox}
	
	A common question about regularization is that traditional statistics, while building a model, we check for multicollinearity using methods such as estimates of the variance inflation factor (VIF), but in Machine Learning, we instead use regularization for feature selection and don't seem to check whether features are correlated at all. Why do we do that?
	
	The reason is because the goals of "traditional statistics" are different from many Machine Learning techniques.

	In traditional statistics regressions, we are trying to understand the impact the independent variables have on the dependent variable. If there is strong multicollinearity, this is simply not possible. No algorithm is going to fix this. If studiousness is correlated with class attendance and grades, we cannot know what is truly causing the grades to go up - attendance or studiousness.

	However, in Machine Learning techniques that focus on predictive accuracy, all we care about is how we can use a set of variables to predict another set. We don't care about the impact these variables have on each other.

	Basically, the fact that we don't check for multicollinearity in Machine Learning techniques isn't a consequence of the algorithm, it's a consequence of the goal!
	
	\begin{tcolorbox}[title=Remark,colframe=black,arc=10pt]
	For more information we strongly recommend the reader to refer to the excellent lecture notes of Wessel N. van Wieringen \cite{1509.09169}.
	\end{tcolorbox}

	
	\pagebreak
	\subsubsection{Polynomial regression}
	We will now see how to determine the for example of the best second degree polynomial that passes by any number of points, but without transforming the function contrary to what we have just done just before! As we like physics in this book, we'll take a classic case of the cinematic to join business with pleasure...
	
	So consider that we are looking for a polynomial of the second degree of the form:
	
	knowing that the method is easily applicable to higher order polynomials (at least as far as we know).
	
	Relation that it is customary to write in the field of polynomial regression as follows:
	
	where $i$ represents the number of points available to us.
	
	For the rest, let us once again base our developments on the least squares method. In other words, we seek the coefficients $c_1,c_2,c_3$ that minimize the error:
	
	and we attack again with partial derivatives for each coefficient:
	
	Therefore after a small rearrangement and simplification:
	
	Similarly:
	
	Therefore after a small rearrangement and simplification:
	
	And finally:
	
	Therefore after a small rearrangement and simplification:
	
	So using the notation of linear algebra, we finally have to solve the following system:
	
	and so we just have to solve this simple linear system either by hand using the relations proved in the section of Linear Algebra, or with a simple spreadsheet software (like Microsoft Excel for example).
	
		
	\pagebreak
	\subsubsection{Kernel regression}\label{kernel regression}
	We have already discussed and introduce quite in details the concepts of kernel smoothing in the section Statistics at page \pageref{kernel smoothing}.
	
	We estimate a regression model $\hat{y}(x)$ by the "\NewTerm{kernel regression estimate}\index{kernel regression estimate}" defined by:
	
	Note that each $\hat{y}(x)$ is a weighted average of the $y_i$, which is the operation:
	
	where the $p_i$ are non-negative constants that sum to one and obviously given by:
	
	Again, a very common smoother is for example the gaussian Kernel:
	
	For example of such kernel regression, the reader can refer to our \texttt{R} companion book.
	
	An improvement to kernel smoothing is "\NewTerm{local polynomial smoothing}\index{local polynomial smoothing}" or also named "\NewTerm{locally estimated scatterplot smoothing}\index{locally estimated scatterplot smoothing}" (LOESS), which does the following: suppose we choose to use a first degree polynomial (that is, use linear regression), then for each $x$ we gave $\hat{\alpha}(x)$ and $\hat{\beta}(x)$ that minimize the weighted residual sum of squares:
	
	and set:
	
	Note that we have to do this for each $x$ for which we wish to evaluate $\hat{y}(x)$.
	
	Note that in the general case, the above relation we will be written:
	
	
	Again, for an example of such kernel regression, the reader can refer to our \texttt{R} companion book. 
	
	\begin{tcolorbox}[title=Remark,colframe=black,arc=10pt]
	As far as we know there is no closed form inferential techniques for the coefficients or any other regression diagnostics.
	\end{tcolorbox}
	
	\pagebreak
	\subsubsection{Logistic Regressions (LOGIT)}\label{logistic regression logit}
	Often, statistics data are relative to the qualitative informations (prediction if a customer is at risk for a credit, prediction if an individual is at risk for sickness, prediction of customer crunch\footnote{Prediction if a customer will leave your business}, prediction of buying a next given item, etc.). However, as we shall see, the traditional inference methods do not allow to model and study this type of variables (in fact they can do but performs poorly most of time!).
	
	A well known simple model that performs most of time quite poorly for binary classification is the  "\NewTerm{signum least squares classifier}\index{signum least squares classifier}" defined as:
	
	where we used the signum function already defined in the section of Arithmetic Operators page \pageref{signum function}.
	
	The intuition behind this least squares classifier is simple. The value $f(\vec{x})$ is a number, which ideally is near $+1$ when $y_i=+1$, and near $-1$ when $y_i=-1$. When $f(\vec{x})$ is near $+1$ we have confidence in our guess $y_i=+1$; when it is small and negative (say, $f(x)=-0.03$), we guess that $y_i=-1$ but our confidence in the guess will be low. So we won't pursue this idea further as we can't associate any probability with it and furthermore the signum function is not continuous (of class $\mathbb{C}^1$) and this can make problems in some computer implementations).
	
	
	Specific methods should be used taking into account for example the lack of continuity of the processed variables or absence of the natural order between the terms that can take the qualitative variable. We will therefore see now the most simple of these methods.
	
	\paragraph{Binomial Logistic Regression}\mbox{}\\\\
	As we have seen above, the simple linear regression therefore aims to model the relation between an unbounded quantitative dependent variable and an unbounded quantitative explanatory variable.
	
	When the "\NewTerm{class variable}\index{class variable}" to explain $Y$ is binary (yes-no, presence-absence, 0-1, etc.), the idea is to approach it at first by a probability function $P(Y=1)$ which gives at the opposite the probability of belonging to the class $Y=0$ or $Y=1$, which we will name the "\NewTerm{logistic binomial regression}\index{logistic binomial regression}" or "\NewTerm{logit regression}\index{logit regression}" or simply "\NewTerm{binomial regression}\index{binomial regression}" and also "\NewTerm{binary regression}\index{binary regression}" (often used in the context of artificial neural networks that we will see later). Then, in a second step, we define a for a binary case a "\NewTerm{cutoff}\index{cutoff}" value. For example, if we take a cutoff of $0.5$ then the cases for which $P(Y=1)>0.5$ will belong to the class $1$ (and vice versa in the opposite case). This is why the binomial logistic regression is often qualified of "\NewTerm{binary classifier}\index{binary classifier}".
	
	\begin{tcolorbox}[title=Remarks,colframe=black,arc=10pt]
	\textbf{R1.} In fact, logistic regression is a simple probability distribution law in our case (we will see another logistic regression in the section Economy during our study of time series and yet another one in the section of Populations Dynamics).\\
	
	\textbf{R2.} It is obviously not possible to apply systematically logistic regression to any type of data sample! Sometimes we have to look elsewhere...\\
	
	\textbf{R3.} When the number of modalities is equal to $2$, we talk about "\NewTerm{dummy variable}\index{dummy variable}" (yes-no) or a "\NewTerm{dichotomic model}\index{dichotomic model}" or even of "\NewTerm{indicator variable}\index{indicator variable}"; if it is greater than $2$, we talk about "\NewTerm{polytomous variables}\index{polytomous variables}" (polytomous logistic regression) or "\NewTerm{multinomial logistic regression}\index{multinomial logistic regression}" (see further below). Therefore the logit binomial model is a "dichotomous model".
	\end{tcolorbox}
	For example, consider the dichotomous variable: "graduation". This takes two forms: "ongoing", "finished". Age is a possible predictor of this variable and we seek to model the probability of completing studies as a function of age.
	\begin{tcolorbox}[colframe=black,colback=white,sharp corners]
	\textbf{{\Large \ding{45}}Example:}\\\\
	To build the graph below, we calculated and shown on the ordinate, for youth of various ages $x$, the percentage of those who have left school.
	\begin{figure}[H]
		\centering
		\includegraphics{img/computing/binomial_logistic_idea.jpg}
		\caption{Part at school according to Age}
	\end{figure}
	But how do we obtain such a graph with a dichotomic (dummy) variable ??? In fact it's relatively simple ... Imagine a sample of $100$ individuals. For these $100$ individuals assume for a given age that $70\%$ "has finished" and $30\%$ are "ongoing". Well the curve is simply the ratio of the two classes for a given age $x$. It is sometimes given the size of classes with circles over the length of the horizontal asymptotes to mean that this is a dichotomous variable.
	\end{tcolorbox}
	Points are distributed according to an $S$-curve (\NewTerm{"sigmoid"}\index{sigmoid}): there are two horizontal asymptotes as the proportion is between $0$ and $1$. We see immediately that a linear model would be manifestly inadequate (especially as the dependent variable a linear model sweeps the whole real numbers $\mathbb{R}$ and is not confined to the range $[0, 1]$).
	
	This curve evokes for some, rightly, a cumulative curve representing a distribution function (of a Normal distribution for instance, but other continuous distributions have almost the same shape). Thus, to fit a curve to this representation, we could move towards the distribution function of a Normal distribution, and instead of estimating the parameters $a$ and $b$ of the linear regression, we could estimate the parameters $\mu,\sigma$ of the Normal law (which is very similar to the logistic law as will be shown below). We then speak of a "\NewTerm{probit model}\index{probit model}" (probability-unit). We could also move towards a Student distribution. We then speak of a "\NewTerm{robit model}\index{robit model}" (robust-unit model) because the $T$-distribution has fatter tails than the Normal distribution, the model allows for occasional large errors, and as a result the estimate for $\beta$ is less affected by outliers. Robit regression is equivalent to probit regression when its degrees of freedom $k \rightarrow +\infty$, and is close to logistic regression when $k = 7$.
	
	\begin{tcolorbox}[title=Remark,colframe=black,arc=10pt]
	Is logit better than probit, or vice versa? Both methods will yield similar inferences (though not identical: in practice the end result of these different distributional assumptions is that coefficients differ, usually logit estimate should be divided by approx $1.6$ to match probit estimate of the same data). But if we look at marginal effects (meaning the effects on the predicted mean of the outcome holding other covariates at the mean or averaging over observed values) the logit and probit models will make essentially the same predictions.\\
	
	On the other hand, if we are not going to go about calculating the margins then logit has the obvious advantage of generating coefficients that can be transformed into the familiar odds ratio by exponentiating the coefficient. Probit and Robit coefficients are essentially uninterpretable.\\
	
	Logit – also known as logistic regression – is more popular in health sciences like epidemiology partly because coefficients can be interpreted in terms of odds ratios. Probit models can be generalized to account for non-constant error variances in more advanced econometric settings (known as heteroskedastic probit models) and hence are used in some contexts by economists and political scientists. If these more advanced applications are not of relevance, than it does not matter which method you choose to go with.
	\end{tcolorbox}	
	
	The law that will interest us, however, is the logistic law. Unlike the Normal distribution, we know how to evaluate the expression of its distribution function (cumulative probability) that is of the type (his first advantage!):
	
	for a single predictor variable (predictor) $x$ where $P$ is obviously a probability between $0$ and $1$. We will see a little further the historical reason for this choice.
	
	We immediately see that this last relation being the integral of a density function (see the proof a little bit further below) it is therefore indeed a cumulative function as:
	
	If there are several predictor variables then we write:
	
	When we choose the logistic distribution function, we get the logistic regression model, or "\NewTerm{logit model}\index{logit model}" for the choice of the "\NewTerm{link function}\index{link function}" and this is its second advantage (the most important one in fact!): we can make statistics on binary variable as if we were doing a simple linear regression!
	
	Thus, to come back on our previous example, we estimate the cumulative probability for an individual of age $x$ to have finish his studies (there are several ways to write the law following the traditions and the context) with the following logistic distribution function\label{logistic distribution}\index{logistic distribution}:
	
	
	\begin{tcolorbox}[title=Remark,colframe=black,arc=10pt]
	Again the logit and probit mainly differ on the link function:
	\begin{itemize}
		\item In Logit: $P(Y=1|X)=[1+e^{-X^T\vec{\beta}} ]-1$
	
		\item In Probit: $P(Y=1|X)=\Phi(X^T\vec{\beta})$ (cumulative Normal probability density function)
	\end{itemize}
	In other way, logistic has slightly flatter tails. i.e the probit curve approaches the axes more quickly than the logit curve.\\
	
	Notice that in probit models latent variable is assumed to be of the form: $Y=a+bx+\varepsilon$ with $\varepsilon = \mathcal{N}(0,1)$.\\

	A question that may arise is what if $\varepsilon = \mathcal{N}(0,\sigma)$? In fact as we know by the property of the Normal distribution it is equivalent to: $Y=a+bx+\sigma\varepsilon$ with $\varepsilon = \mathcal{N}(0,1)$.
	
	Then:
	
	The only thing that change is the parameters.
	\end{tcolorbox}
	
	It therefore follows the distribution function\footnote{If the reader is looking for the second derivative and inflection point of the sigmoid function, see page \pageref{sigmoid second derivative and inflection point}}:
	
	Obviously, depending on the value of the probability we associate with the age $x$ the fact of not having finished his studies (state associated with the binary value: $0$) or have them finished (state associated with the binary value: $1$).	
	\begin{tcolorbox}[title=Remark,colframe=black,arc=10pt]
	After a change of variables, we find fall back on the logistic law as defined on Wikipedia:
	
	\end{tcolorbox}	
	Let us indicate that if $a$ is set as unit, and $b$ as non null, then we have the "\NewTerm{standard logistic law}\index{standard logistic law}" given by:
	
	We can also calculate the mean of the distribution of the function by applying what has already been study in the Statistics section but part of this integral can only be solved numerically (at least as far as we know)... if we put:
	
	as being the random variable then we can formally calculate the mean of the logistic law (the reader may have noticed that it is as if we posed as $a=1$ and $b=0$). Indeed, starting from:
	
	Therefore it comes:
	
	that after a numerical integration gives $0$. We then also get the following result:
	
	Let us calculate the integral:
	
	Thus we see that if we put:
	
	We fall back on a distribution function with the same position and dispersion parameters of a Normal distribution centered reduced variable (zero mean and unit variance).
	
	The distribution function:
	
	can also be transformed in a very known and important form!:
	
	Therefore:
	
	In fact this is where lies the historically the trick of the origin of logistic regression. We transform a variable $P$ taking values in the in the range $[0,1]$ thanks to the logarithm of the ratio $P / (1-P)$ in a variable taking its values on the set $\mathbb{R}$ and therefore it is possible to associate to it a standard linear regression. Certainly it is empirical, but the idea was pretty good!

	What some also write...:
	
	The result of the last transformation is named the "\NewTerm{logit}\index{logit}". It is equal to the logarithm of the "\NewTerm{odds}\index{odds}" (which will be discussed in more detail soon):
	
	So this is just the ratio of a likelihood of an event on the probability of the complementary event (or opposite event if you prefer).
	
	\begin{tcolorbox}[title=Remark,colframe=black,arc=10pt]
	So what logit and probit do, in essence, is take the linear model and feed it through a function to yield a non-linear relationship. Whereas the linear regression predictor looks like:
	
	The logit and probit predictor can be written as:
	
	Both functions will take any number and rescale it to fall between $0$ and $1$. Hence, whatever $ax+b$ equals, it can be transformed by the function to yield a predicted probability.
	\end{tcolorbox}	
	
	So when the coefficients $a$ and $b$ have been determined, the above expression is used to determine $P$ knowing $x$ easily (it comes to solve a linear equation) and vice versa! Moreover, since $x$ is a dichotomic (dummy) variable coefficients are easily interpretable.
	
	\pagebreak
	\begin{tcolorbox}[title=Remark,colframe=black,arc=10pt]
	The odds is also sometimes named the "\NewTerm{rating}\index{rating}" by analogy to the rating of horses to the triple forecast. For example, if a horse has $3$ chances on $4$ to be a winner (thus verbatim $1$ on $4$ chance of being non-winner) its rating is a $3$ against $1$ ratio, that is to say an odds equal to $3$. We can also introduce the concept of "\NewTerm{odds ratio}\index{odds ratio}\label{odds ratio logistic regression}" (O.R.) for the rating ratio which is a very widely used indicator in medicine. Thus, if the occurrence of an event in a group $A$ is $p$, and $q$ in the group $B$, the odds ratio is then simply given by:
	
	The odds ratio is always by design greater than or equal to zero. If the odds ratio is close to $1$, the event is independent of the group, if it is greater than $1$, the event is more common in group $A$ than in group $B$, if it is less than $1$ the event is less frequent in group $A$ than in group $B$.\\
	
	Odds ratio or "\NewTerm{crude ratio}" are obtained when we consider the effect of only one predictor variable. However when we include more variables in the analysis we get what is named the "\NewTerm{adjusted odds ratio}".
	\end{tcolorbox}
	Let us come back over the odds because it is possible to introduce the concept of logistics function by doing the opposite approach from the one presented above (i.e. to start with the definition of odds to get to the logit) and this can sometimes be even more educational.
	
	Let us suppose we start from the size (height) of a person to predict whether this person is a man or a woman. So we can talk about probability of being a man or a woman. Suppose the probability of a man for a given height is $90\%$. So the odds of being a man is:
	
	In our example, the odds will be $0.90 / 0.10$ therefore equal to $9$. Now, the probability of being a woman will be $0.10 / 0.90$ therefore equal to $0.11$. This asymmetry of values is not talking because the odds of being a man should be the opposite of the odds of being a woman ideally. We solve precisely this asymmetry using the natural logarithm. Thus we have:
	
	and:
	
	In this way the logit (logarithm of the odds) is exactly the opposite of this of being a woman by the property of the logarithm:
	
	
	\pagebreak
		To introduce this tool let us suppose a bank wants to make a scoring of its debtors. As it has several subsidiaries it (the bank) built the following data tables for some of them (all subsidiaries are then not presented):
	\begin{itemize}
    	\item 1st Subsidiary:
		\begin{table}[H]
		\begin{center}
			\definecolor{gris}{gray}{0.85}
				\begin{tabular}{|p{2cm}|p{2cm}|p{2cm}|}
					\hline
					\multicolumn{1}{c}{\cellcolor{black!30}\textbf{Credit Amount}} & 
	  \multicolumn{1}{c}{\cellcolor{black!30}\textbf{Paid}}  & \multicolumn{1}{c}{\cellcolor{black!30}\textbf{Not Paid}}\\ \hline
					\centering\arraybackslash\ $27,200$ & \centering\arraybackslash\ $1$ & \centering\arraybackslash\ $9$ \\ \hline
					\centering\arraybackslash\ $27,700$ & \centering\arraybackslash\ $7$ & \centering\arraybackslash\ $3$ \\ \hline
					\centering\arraybackslash\ $28,300$ & \centering\arraybackslash\ $13$ & \centering\arraybackslash\ $0$ \\ \hline
					\centering\arraybackslash\ $28,400$ & \centering\arraybackslash\ $7$ & \centering\arraybackslash\ $3$ \\ \hline
					\centering\arraybackslash\ $29,900$ & \centering\arraybackslash\ $10$ & \centering\arraybackslash\ $1$ \\ \hline
			\end{tabular}
		\end{center}
		\caption[]{Debtors Scoring by credit amount of subsidiary 1}
		\end{table}
		\item 2nd Subsidiary:
			\begin{table}[H]
			\begin{center}
				\definecolor{gris}{gray}{0.85}
				\begin{tabular}{|p{2cm}|p{2cm}|p{2cm}|}
						\hline
						\multicolumn{1}{c}{\cellcolor{black!30}\textbf{Credit Amount}} & 
		  \multicolumn{1}{c}{\cellcolor{black!30}\textbf{Paid}}  & \multicolumn{1}{c}{\cellcolor{black!30}\textbf{Not Paid}}\\ \hline
						\centering\arraybackslash\ $27,200$ & \centering\arraybackslash\ $0$ & \centering\arraybackslash\ $8$ \\ \hline
						\centering\arraybackslash\ $27,700$ & \centering\arraybackslash\ $4$ & \centering\arraybackslash\ $2$ \\ \hline
						\centering\arraybackslash\ $28,300$ & \centering\arraybackslash\ $6$ & \centering\arraybackslash\ $3$ \\ \hline
						\centering\arraybackslash\ $28,400$ & \centering\arraybackslash\ $5$ & \centering\arraybackslash\ $3$ \\ \hline
						\centering\arraybackslash\ $29,900$ & \centering\arraybackslash\ $8$ & \centering\arraybackslash\ $0$ \\ \hline
				\end{tabular}
			\end{center}
			\caption[]{Debtors Scoring by credit amount of subsidiary 2}
			\end{table}
		\item 3rd Subsidiary:
		\begin{table}[H]
			\begin{center}
				\definecolor{gris}{gray}{0.85}
				\begin{tabular}{|p{2cm}|p{2cm}|p{2cm}|}
						\hline
						\multicolumn{1}{c}{\cellcolor{black!30}\textbf{Credit Amount}} & 
		  \multicolumn{1}{c}{\cellcolor{black!30}\textbf{Paid}}  & \multicolumn{1}{c}{\cellcolor{black!30}\textbf{Not Paid}}\\ \hline
						\centering\arraybackslash\ $27,200$ & \centering\arraybackslash\ $1$ & \centering\arraybackslash\ $8$ \\ \hline
						\centering\arraybackslash\ $27,700$ & \centering\arraybackslash\ $6$ & \centering\arraybackslash\ $2$ \\ \hline
						\centering\arraybackslash\ $28,300$ & \centering\arraybackslash\ $7$ & \centering\arraybackslash\ $1$ \\ \hline
						\centering\arraybackslash\ $28,400$ & \centering\arraybackslash\ $7$ & \centering\arraybackslash\ $2$ \\ \hline
						\centering\arraybackslash\ $29,900$ & \centering\arraybackslash\ $9$ & \centering\arraybackslash\ $0$ \\ \hline
				\end{tabular}
			\end{center}
			\caption[]{Debtors Scoring by credit amount of subsidiary 3}
		\end{table}
	\end{itemize}
	We can see that the total proportion of good debtors in the three subsidiaries is of $91/136 \cong 0.67$.
	
	When the credit is less than $27,500$, the percentage of good debtors is of $2/27\cong 0.07$. When the amount of credit is less than $28,000$ the proportion of good debtors is of $19/51\cong 0.37$.
	
	When the amount of credit is less than $28,500$, the percentage of good debtors is of $64/108 \cong 0.59$ and for amounts below $30,000$ the proportion is of $91/936\cong 0.67$.
	
	We will put for this logistic regression that $Y=1$ is a good credit risk and that $Y=0$ is a bad risk. Next, we create the following table that is a summary of data from all subsidiaries:
	\begin{table}[H]
		\begin{center}
			\definecolor{gris}{gray}{0.85}
			\begin{tabular}{|p{2cm}|p{4cm}|}
					\hline
					\multicolumn{1}{c}{\cellcolor{black!30}\textbf{Credit Amount}} & 
	  \multicolumn{1}{c}{\cellcolor{black!30}\textbf{Proportion $P$}} \\ \hline
					\centering\arraybackslash\ $27,200$ & \centering\arraybackslash\ $=2/27=0.0741$  \\ \hline
					\centering\arraybackslash\ $27,700$ & \centering\arraybackslash\ $=17/24=0.7083$ \\ \hline
					\centering\arraybackslash\ $28,300$ & \centering\arraybackslash\ $=26/30=0.8667$ \\ \hline
					\centering\arraybackslash\ $28,400$ & \centering\arraybackslash\ $=19/27=0.7037$ \\ \hline
					\centering\arraybackslash\ $29,900$ & \centering\arraybackslash\ $=27/28=0.9643$ \\ \hline
			\end{tabular}
		\end{center}
		\caption[]{Proportion of good debtors}
	\end{table}
	Which gives graphically in Kilo-dollars:
	\begin{figure}[H]
		\centering
		\includegraphics{img/arithmetics/logistic_regression_debtors_percentage.jpg}
		\caption{Cumulative percentage of good debtors based on credit}
	\end{figure}
	Once done, we use the logit transformation:
	
	Which gives:
	\begin{table}[H]
		\begin{center}
			\definecolor{gris}{gray}{0.85}
			\begin{tabular}{|p{2cm}|p{2cm}|p{2cm}|}
					\hline
					\multicolumn{1}{c}{\cellcolor{black!30}\textbf{Credit Amount}} & 
	  \multicolumn{1}{c}{\cellcolor{black!30}\textbf{Proportion $P$}}  & \multicolumn{1}{c}{\cellcolor{black!30}\textbf{Logit}}\\ \hline
					\centering\arraybackslash\ $27,200$ & \centering\arraybackslash\ $0.0741$ & \centering\arraybackslash\ $	
-2.5257$ \\ \hline
					\centering\arraybackslash\ $27,700$ & \centering\arraybackslash\ $0.7083$ & \centering\arraybackslash\ $0.8873$ \\ \hline
					\centering\arraybackslash\ $28,300$ & \centering\arraybackslash\ $0.8667$ & \centering\arraybackslash\ $1.8718$ \\ \hline
					\centering\arraybackslash\ $28,400$ & \centering\arraybackslash\ $0.7037$ & \centering\arraybackslash\ $0.8650$ \\ \hline
					\centering\arraybackslash\ $29,900$ & \centering\arraybackslash\ $0.9643$ & \centering\arraybackslash\ $3.2958$ \\ \hline
			\end{tabular}
		\end{center}
		\caption[]{Proportion of good debtors and logit}
	\end{table} 
	\begin{tcolorbox}[title=Remark,colframe=black,arc=10pt]
	The reader must keep in mind the social responsibilities when running such models! First their accuracy (wrong rate of classification), but also especially the confidence interval of the predicted proportion (probability). A statistician or data scientist should never communicate punctual values alone, but always with their confidence interval!\\
	
	Be careful also to have balanced (unbiased and correctly sampled) datasets, or to weight them correctly if necessary, to not have biased results especially when your classification methods or algorithms have important social impacts on people!!!
	\end{tcolorbox}
	A linear regression by least squares method gives:
	\begin{figure}[H]
		\centering
		\includegraphics{img/arithmetics/logistic_regression_olg.jpg}
		\caption{Logit of good debtors based on the credit amount}
	\end{figure}
	with for equation:
	
	The logistics function with its representation comes then immediately (the $x$ units are in thousands of francs)
	
	\begin{figure}[H]
		\centering
		\includegraphics{img/arithmetics/logistic_cumulative_distribution.jpg}
	\end{figure}
	Thus, it is possible to say in this example, what is the proportion $P$ of good or bad debtors according to the credit value $X$ than or equal to a certain given value. Since $0$ is a bad credit risk, we see that the more is the credit, the lower the risk is big (in this hypothetical case...). Moreover, with software like Minitab (see the corresponding companion book), the difference between the quick calculations carried out by hand and those made with the binary logistic regression tool  of the software is in the order of $10\%$ (because of course ... Minitab uses the concept of maximum likelihood estimators seen in the Statistics section to determine the coefficients and the constant).
	
	A software like Minitab gives automatically a sympathetic output which is the "\NewTerm{confusion matrix $\mathcal{C}$}\index{confusion matrix}". It compares the model to reality with a traditional cutoff set at $50\%$ (obviously if the model perfectly match to the sample, the following matrix is a diagonal one):
	\begin{figure}[H]
		\centering
		\includegraphics{img/arithmetics/confusion_matrix.jpg}
		\caption{Confusion matrix example}
	\end{figure}
	where the good debtors have the value $1$ (no credit risk) and the bad one the value $0$ (credit risk). We will detail further below how to get this matrix with a spreadsheet software. Finally let us indicate that in many Data Mining softwares, it is customary to define the "\NewTerm{score}\index{score}" of the model as (in the specific case of one unique explanatory variable but can easily be generalized to many):
	
	\begin{tcolorbox}[title=Remark,colframe=black,arc=10pt]
	In some statistical softwares, when you run a logistic regression, you can get sometimes a message of the type: \textit{complete separation of data points}. Indeed, take a look at the following data set consisting of the response variable, $Y$, and one predictor variable, $X$:
	\begin{table}[H]
		\centering
		\begin{tabular}{|c|c|c|c|c|c|c|c|c|}
		\hline
		$X$ & $1$ & $2$ & $3$ & $4$ & $4$ & $5$ & $5$ & $6$ \\ \hline
		$Y$ & $0$ & $0$ & $0$ & $0$ & $0$ & $1$ & $1$ & $1$ \\ \hline
		\end{tabular}
	\end{table}
	Notice the key pattern... This data set can be simply described as follows: if $X\leq 4$, then $Y=0$ without fail. Similarly, if $X>4$, then $Y=1$, again without fail. This is hat is knows as "separation". This perfect prediction of the response is what causes the estimates, and thus the model, to fail.
	\end{tcolorbox}
	Following the request of a reader, here is the maximum likelihood approach of the logistic regression.

	Let us consider the notation:
	
	Assuming that the observations are independent of one another, we can define $L(D,\vec{\beta})$, the likelihood of the data $D$ with respect to the logistic model parameter by $\vec{\beta}$. It is the product of the probabilities according to this model that each individual in $D$ belongs to the observed class with $y_i=1$ or $y_i=0$:
	
	It is therefore necessary to determine the vector of parameters $\hat{\vec{\beta}}$ maximizing the likelihood of the data $D$:
	
	To identify $\hat{\vec{\beta}}$, the first thing to do is to derive the likelihood with respect to each component of $\vec{\beta}$, $\dfrac{\partial }{\partial \beta_j}L(D,\vec{\beta})$, then solve:
	
	since $L(D,\vec{\beta})$ is convex in $\vec{\beta}$ and $\dfrac{\partial }{\partial \beta_j}L(D,\vec{\beta})$ vanishes when the maximum is reached.
	
	However, the logarithm is a strictly increasing function, maximizing the likelihood is equivalent to maximizing the logarithm of the likelihood as we already know it, whose expression is easier to take the derivative (the logarithm transforming the products into sums and the exponents into factors):
	 
	Notice that some statistical softwares return the "\NewTerm{deviance of the proposed model}\index{deviance}" defined as:
	 
	The deviance is always larger or equal than zero, being zero only if the fit is perfect.
	
	A benchmark for evaluating the magnitude of the deviance is the "\NewTerm{deviance of the null model}\index{deviance of the null model}":
	 
	which is the deviance of the worst model, the one fitted without any predictor (ie only with the intercept)! The null deviance serves for comparing how much the model has improved by adding the predictors.
	
	Here is a summary of what you can see in some statistical software (be careful! the concept of "null deviance" must not be confused with the concept of "deviance of the null"!):
	
	where:
	\begin{itemize}
		\item The "saturated model" is a model that assumes each data point has its own parameters (which means you have $n$ parameters to estimate).
	
		\item The "null model" assumes the exact opposite, in that is assumes one parameter for all of the data points, which means you only estimate $1$ parameter.
	
		\item The "proposed model" assumes you can explain your data points with $p$ parameters + an intercept term, so you have $p+1$ parameters.	
	\end{itemize}
	Having defined these two deviances we can if we wish build a likelihood-ratio test from them (\SeeChapter{see section Statistics page \pageref{likelihood ratio tests}}). That test is sometimes named the "\NewTerm{deviance test}\index{deviance test}".
	
	Let us now just prove that the log-likelihood of the saturated model is always zero. AS the saturated model is the model that \underline{perfectly} fits the observed response.
	
	Let us remind that the likelihood as already seen above is given by:
	
	Therefore as the model is perfect:
	
	and clearly the logarithm of this is $0$.
	
	\begin{tcolorbox}[title=Remark,colframe=black,arc=10pt]
	Notice that we define the "\NewTerm{cross-entropy}\index{cross-entropy}\label{cross-entropy}":
	
	where $p$ and $q$ denote a "true" and an "empirical/estimated" distribution, respectively. Both are discrete distributions, hence we can sum over their individual components, denoted with $i$.  Above, when we refer to a "distribution", it means with respect to a single training data point, and not the "distribution of training data points". That's a different concept.
	\end{tcolorbox}
	So we have now therefore to determine the vector $\hat{\vec{\beta}}$ maximizing the log-likelihood of the data $D$:
	
	We shall therefore have to express $\dfrac{\partial }{\partial \beta_j}\ln(L(D,\vec{\beta}))$. Before that, we can agree a slightly different notation, in order to improve the readability of the formulas. We consider that each vector $\vec{x}\in D$ has an additional component, $x_0$, always equal to $1$ (same technique as the one used to generalize the univariate regression). Thus we can rewrite $P$, so that the parameter of the sigmoid function is the dot product of the vectors $\vec{\beta}$ and $\vec{x}$ (both of dimension $d + 1$):
	
	Where $\vec{\beta}\circ\vec{x}$ is commonly known as the "\NewTerm{prognostic index}\index{prognostic index}" (PI). The expression to be derived is thus:
	
	First we compute the first term:
	
	Now we compute the second term:
	
	So finally:
	
	As far as we know... we can not solve this analytically:
	
	so we will have to approach the solution by a numerical method anyway... (gradient descent again)!
	
	\subparagraph{Binomial Logistic regression relative variable importance}\label{variable importance BLM}\mbox{}\\\\
	We have seen for the Gaussian multivariate linear model how to calculate variable importance earlier above (see page \pageref{variable importance GML}). Now for logistic regression we use rather "\NewTerm{relative variable importance}". 
	
	To introduce that latter let us first calculate the partial derivative of the logistic regression relatively to one predictor (we used the quotient rule proved in the section of Differential and Integral Calculus page \pageref{quotient rule}):
	
	So this result measures the effect of $x_i$ on $P(Y)$. This effect is a function of $x_i$. However, the relative importance of two predictors is:
	
	which is independent of $\vec{x}$. Thus, provided we have standardized all predictors, we can look at the estimates of the model coefficients as indicators of the relative importance of the predictors\index{variable importance} for what it concerns the variation of the output.
	
	\subparagraph{Binomial logistic regression residuals}\label{binomial logistic regression residuals}\mbox{}\\\\
	Residual analysis for logistic regression is more difficult than for linear regression models because the responses $y_i$ take only the values $0$ and $1$. Consequently the $i$th ordinary residual between the real value $y_i$ and the model fitted value, that we will denote now by $\hat{\pi}$, is given obviously by:
	
	And obviously we have that (as $y_i$ take only the values $0$ and $1$):
	
	The ordinary residuals will not be normally distributed and, indeed, their distribution under the assumption that the fitted model is correct is unknown. Plots of ordinary residuals against fitted values or predictor variables will generally be uninformative.
	
	The ordinary residuals can be made more comparable by dividing them by the estimated standard deviation of $y_{i},$ namely for the binomial distribution (as we have proved it during our study of the binomial distribution):  
		
	The resulting "\NewTerm{Pearson residuals}\index{Pearson residuals}" are given by:
	
	If we square:
	
	Now notice that this can be written:
	
	So we recognize here that all the terms are of the form:
	
	Therefore the same applies to the sum of square of the Pearson residuals and we can write the "\NewTerm{Pearson chi-square statistic}\index{Pearson chi-square statistic}":
	
	Hence, we wee that the sum of the squares of the Pearson residuals is numerically equal to the Pearson chi-square test statistic. Therefore the square of each Pearson residual measures the contribution of each binary response to the Pearson chi-square test statistic and can be used as an adequation test for our model.
	
	\begin{tcolorbox}[title=Remark,colframe=black,arc=10pt]
	Notice that some practitioners run not only a chi-square test on residuals but also on the resulting contingency classification table of all discrete $x$ values of the binomial regression. For our example above the corresponding contingency table being:
	\begin{table}[H]
		\centering
		\definecolor{gris}{gray}{0.85}
		\begin{tabular}{|p{2cm}|p{2cm}|p{2cm}|}
			\hline
			\multicolumn{1}{c}{\cellcolor{black!30}\textbf{Credit Amount}} & 
  \multicolumn{1}{c}{\cellcolor{black!30}\textbf{Expected}}  & \multicolumn{1}{c}{\cellcolor{black!30}\textbf{Observed}}\\ \hline
			\centering\arraybackslash\ $27,200$ & \centering\arraybackslash\ $0$ & \centering\arraybackslash\ $2$ \\ \hline
			\centering\arraybackslash\ $27,700$ & \centering\arraybackslash\ $0$ & \centering\arraybackslash\ $17$ \\ \hline
			\centering\arraybackslash\ $28,300$ & \centering\arraybackslash\ $30$ & \centering\arraybackslash\ $26$ \\ \hline
			\centering\arraybackslash\ $28,400$ & \centering\arraybackslash\ $27$ & \centering\arraybackslash\ $10$ \\ \hline
			\centering\arraybackslash\ $29,900$ & \centering\arraybackslash\ $28$ & \centering\arraybackslash\ $27$ \\ \hline
		\end{tabular}
	\end{table}
	Unfortunately, it is common that there are not enough observations (or even worse...: zero observations) for each possible combinations of values of the $x$ variables, so the Pearson chi-squared statistic cannot be readily calculated (see a practical example in our \texttt{R} companion book). A solution to this problem is the Hosmer-Lemeshow statistic (see below). The key concept of the Hosmer-Lemeshow statistic is that, instead of observations being grouped by the values of the $x$ variable(s), the observations are grouped by expected probability. That is, observations with similar expected probability are put into the same group, usually to create approximately $10$ groups.
	\end{tcolorbox}
	
	\subparagraph{Hosmer–Lemeshow test}\label{Hosmer–Lemeshow test}\mbox{}\\\\
	The Hosmer–Lemeshow test is a statistical test for goodness of fit for binary logistic regression models (however a generalized Hosmer–Lemeshow goodness-of-fit test for multinomial logistic regression models exists!). It is used frequently in risk prediction models. The test assesses whether or not the observed event rates match expected event rates in subgroups of the model population. The Hosmer–Lemeshow test specifically identifies subgroups as the deciles of fitted risk values. Models for which expected and observed event rates in subgroups are similar are called well calibrated.
	
	The Hosmer-Lemeshow test statistic is given by:
	
	Here $O_{1 g}, E_{1 g}, O_{0 g}, E_{o g}, N_{g}$, and $\pi_{g}$ denote the observed $Y=1$ events, expected $Y=1$ events, observed $Y=0$ events, expected $Y=0$ events, total observations, predicted risk for the $g^{\text {th }}$ risk decile group, and $G$ is the number of groups. The test statistic asymptotically follows a $\chi^{2}$ distribution with $G-2$ degrees of freedom (see remark further below!!!). Hence:
	
	The number of risk groups may be adjusted depending on how many fitted risks are determined by the model. This helps to avoid singular decile groups.
	
	The Hosmer–Lemeshow test seems to have some limitations. Frank Harrell describes several in an answer on the forum \textit{Cross Validated}:
	\begin{itemize}
		\item The Hosmer-Lemeshow test is for overall calibration error, not for any particular lack of fit such as quadratic effects. It does not properly take overfitting into account, is arbitrary to choice of bins and method of computing quantiles, and often has power that is too low.
		
		\item For these reasons the Hosmer-Lemeshow test is no longer recommended. Hosmer et al have a better one df omnibus test of fit, implemented in the \texttt{R} software rms package residuals.lrm function.
		
		\item Other alternatives have been developed to address the limitations of the Hosmer-Lemeshow test. These include the Osius-Rojek test and the Stukel test.
	\end{itemize}
	
	\begin{tcolorbox}[title=Remark,colframe=black,arc=10pt]
	In the 1982 paper by Lemeshow and Hosmer \textit{A review of goodness of fit statistics for use in the development of logistic regression models}, American Journal of Epidemiology 115:92-106. In this they write (page 96) in reference to their statistic that:
	
	\begin{center}
	\og \textit{The theoretical development given by Hosmer and Lemeshow (1980) requires only that $G>(p+1)$.} \fg{}
	\end{center}
	
	where $p$ is the number of parameters in the model and following this that:
	
	\begin{center}
	\og \textit{Hosmer and Lemeshow (1980) have shown via computer simulations that if the number of covariates plus one is less than the number of groups (i.e. $p+1 < G$), then the statistic $C^*_G$ has a distribution which is closely approximated by a chi-square distribution with $G-2$ degrees of freedom when $H_0$ is true.} \fg{}
	\end{center}
	
	To be more precise, in Hosmer and Lemeshow's 1980 paper, Theorem 2 states that the asymptotic distribution of $\hat{C}_{g}^{*}$ (the usual Hosmer-Lemeshow test statistic) is:
	$$
	\chi_{2 g-g-(p+1)}^{2}+\sum_{i=1}^{p+1} \lambda_{i} \chi_{i}^{2}(1)
	$$
	where the $\lambda_{i}$ are eigenvalues of a matrix (specified in the paper, not relevant to this question). Then, they show through simulations that $\sum_{i=1}^{p+1} \lambda_{i} \chi_{i}^{2}(1)$ is approximately $\chi_{p-1}^{2}$, which leads to the usual $g-2$ degrees of freedom in the Hosmer-Lemeshow test.
	
	This means that given our fitted model, the p-value can be calculated as the right hand tail probability of the corresponding chi-squared distribution using the calculated test statistic. If the $p$-value is small, this is indicative of poor fit.
	\end{tcolorbox}
	It should be emphasized, same as any other null hypothesis statistical test, that a large $p$-value does not mean the model fits well, since lack of evidence against a null hypothesis is not equivalent to evidence in favour of the alternative hypothesis. In particular, if our sample size is small, a high $p$-value from the test may simply be a consequence of the test having lower power to detect misspecification, rather than being indicative of good fit.
	
	The reader can refer to our \texttt{R} companion book where we provide simulation evidence that the above claim is quite accurate!
	
	
	
	\paragraph{Multinomial Logistic Regression}\label{multinomial logistic regression}\mbox{}\\\\
	"\NewTerm{Multinomial logistic regression}\index{multinomial logistic regression}\label{multinomial logistic regression}" is a classification method that generalizes logistic regression to multiclass problems, i.e. with more than two possible discrete outcomes. That is, it is a model that is used to predict the probabilities of the different possible outcomes of a categorically distributed dependent variable, given a set of independent variables (which may be real-valued, binary-valued, categorical-valued, etc.).

	Multinomial logistic regression is known by a variety of other names, "\NewTerm{including polytomous LR}\index{including polytomous LR}", "\NewTerm{multiclass LR}\index{multiclass LR}", "\NewTerm{softmax regression}\index{softmax regression}", "\NewTerm{multinomial logit}\index{multinomial logit}", the "\NewTerm{maximum entropy (MaxEnt) classifier}\index{maximum entropy (MaxEnt) classifier}", and the "\NewTerm{conditional maximum entropy model}\index{conditional maximum entropy model}".
	
	The basic setup is the same as in logistic regression, the only difference being that the dependent variables are categorical rather than binary, i.e. there are $K$ possible outcomes rather than just two. The following description is somewhat shortened.
	
	As in other forms of linear regression, multinomial logistic regression uses a linear predictor function $f(k,i)$ to predict the probability that observation $i$ has outcome $k$, of the following form:
	
	where $\beta _{m,k}$ is a regression coefficient associated with the $m$th explanatory variable and the $k$th outcome. As it is common, the regression coefficients and explanatory variables are normally grouped into vectors of size $M+1$, so that the predictor function can be written more compactly:
	
	where $\vec{\beta}_k$ is the set of regression coefficients associated with outcome $k$, and $\vec{x}_i$ (a row vector) is the set of explanatory variables associated with observation $i$.
	
	One fairly simple way to arrive at the multinomial logit model is to imagine, for $K$ possible outcomes, running $K-1$ independent binary logistic regression models, in which one outcome is chosen as a "pivot" and then the other $K-1$ outcomes are separately regressed against the pivot outcome. This would proceed as follows, if outcome $K$ (the last outcome) is chosen as the pivot:
	
	Note that we have introduced separate sets of regression coefficients, one for each possible outcome.
	
	That latter relation is also often denoted:
	
	for each $j=1,\ldots,K-1$.

	If we exponentiate both sides, and solve for the probabilities, we get:
	
	Using the fact that all $K$ of the probabilities must sum to one, we find:
	
	That latter relation is also often denoted (notation that we will use further below):
	
	Re-injecting in the previous relations sets, we get:
	
	Hence:
	
	Often denoted (as the sum in the denominator is a constant\footnote{However, it is definitely not constant with respect to the explanatory variables, or crucially, with respect to the unknown regression coefficients $\vec{\beta}_k$, which we will need to determine through some sort of optimization procedure.}):
	
	The fact that we run multiple regressions reveals why the model relies on the assumption of independence of irrelevant alternatives!
	
	Note that the prior-previous relation is also denoted:
	
	So we have:
	
	\begin{tcolorbox}[title=Remark,colframe=black,arc=10pt]
	Note that when $K=2$ the multinomial and logistic regression models become one and the same. Also for information, the softmax function is used in multi-class classification neural networks as activation function as we will see further below during our study of neural networks (page \pageref{neural network}).
	\end{tcolorbox}
	
	The following function:
	
	is referred to as the "\NewTerm{softmax function}\index{softmax function}\label{softmax functiion}". The reason is that the effect of exponentiating the values $x_{1},\ldots ,x_{n}$ is to exaggerate the differences between them. As a result $\text{softmax}(k,x_{1},\ldots ,x_{n})$ will return a value close to $0$ whenever $x_{k}$ is significantly less than the maximum of all the values, and will return a value close to $1$ when applied to the maximum value, unless it is extremely close to the next-largest value.
	
	When using multinomial logistic regression, one category of the dependent variable is chosen as the reference category. Separate odds ratios are determined for all independent variables for each category of the dependent variable with the exception of the reference category, which is omitted from the analysis. The exponential beta coefficient represents the change in the odds of the dependent variable being in a particular category vis-a-vis the reference category, associated with a one unit change of the corresponding independent variable.
	
	Following the request of a reader, here is the maximum likelihood approach of the logistic regression.

	Let us recall that the multinomial distribution is given (\SeeChapter{see section Statistics page \pageref{multinomial distribution}}):
	
	In other words if there are $m$ random variables, i.e. $X_i$, $i\in[1,n]$, where $X_i$ represents the number of occurrences of item $i$ in a choice of $n$ items, with entry $i$ in the vector of probabilities $\vec{P}$, $P_i$ giving the probability of drawing item $i$. The probability of selecting $k_1$ of item $1$ .. $k_m$ of item $m$ is then given by $\mathcal{M}$ above! 
	
	\begin{tcolorbox}[title=Remark,colframe=black,arc=10pt]
	Remember that if $m=2$  then the above relation becomes:
	
	But as $k_1+k_2=n$ then:
	
	and as $p_1+p_2=1$ we finally have:
	
	We recognize here the binomial distribution!
	\end{tcolorbox}
	Let us denote the joint probability density (likelihood) function for the multinomial logistic function as:
	
	Since we want to maximize the above relation with respect to $\vec{\beta}$, the factorial terms that do not contain any of the $\pi_{ij}$ terms can be treated as constants. Thus, the kernel of the log likelihood function for multinomial logistic regression models is:
	
	Replacing the $J$th terms, the previous relations becomes:
	
	Since $a^{x+y}=a^xa^y$, the sum in the exponent in the denominator of the last term becomes a product over the first $J-1$ terms of $j$. Continue by grouping together the terms that are raised to the $y_{ij}$ power for each $j$ up to $J-1$:
	
	Now we substitute for $\pi_{ij}$ and $\pi_{iJ}$ the both relations that we have derived earlier:
	
	This give us then:
	
	Taking the logarithm of the above relation gives us the log-likelihood function for the multinomial logistic regression model:
	
	
	\pagebreak
	\subparagraph{$F_1$ score}\label{F1 score}\mbox{}\\\\
	Now that we have introduce a well known multiclass classifier, it is time to introduce the "\NewTerm{$F_1$ score}\index{$F_1$ score}" (also named "\NewTerm{$F$-score}" or "\NewTerm{$F$-measure}") which is a measure of a classifier quality (among many others empirical quality indicators...). It considers both the precision $P$ and the recall $R$ of the test to compute the score. To introduce these two indicators, let us consider for companion example, the following generic three-class ($n=3$) confusion matrix $\mathcal{C}$ for a total of $N$ records to classify:
	\begin{table}[H]
		\centering
		\begin{tabular}{|
		>{\columncolor[HTML]{C0C0C0}}l |
		>{\columncolor[HTML]{EFEFEF}}c |c|c|c|}
		\hline
		 & \multicolumn{4}{c|}{\cellcolor[HTML]{C0C0C0}\textbf{Predicted}} \\ \hline
		\cellcolor[HTML]{C0C0C0} &  & \cellcolor[HTML]{EFEFEF}\textbf{Class $1$} & \cellcolor[HTML]{EFEFEF}\textbf{Class $2$} & \cellcolor[HTML]{EFEFEF}\textbf{Class $3$} \\ \cline{2-5} 
		\cellcolor[HTML]{C0C0C0} & \textbf{Class $1$} & $\mathcal{C}_{11}$ & $\mathcal{C}_{12}$ & $\mathcal{C}_{13}$ \\ \cline{2-5} 
		\cellcolor[HTML]{C0C0C0} & \textbf{Class $2$} & $\mathcal{C}_{21}$ & $\mathcal{C}_{22}$ & $\mathcal{C}_{23}$ \\ \cline{2-5} 
		\multirow{-4}{*}{\cellcolor[HTML]{C0C0C0}\textbf{Actual}} & \textbf{Class $3$} & $\mathcal{C}_{31}$ & $\mathcal{C}_{32}$ & $\mathcal{C}_{33}$ \\ \hline
		\end{tabular}
		\caption[]{A typical three class confusion matrix}
	\end{table}	
	Let us recall first that the "\NewTerm{accuracy}\index{accuracy}" is given by:
	
	The "\NewTerm{precision}\index{precision}" $P_i$ for a multiclass classifier (of $n$ classes) is defined as the fraction of correct predictions for a certain class\footnote{For a practical example, see our \texttt{R} companion book on the $F$-score.}!:
		
	and the "\NewTerm{recall}\index{recall}" $R_i$ for a multiclass classifier (of $n$ classes) is the number of correct positive results divided by the number of all relevant samples (all samples that should have been identified as positive):
	
	This can nicely be illustrated as following:
	\begin{figure}[H]
		\centering
		\includegraphics[width=1.0\textwidth]{img/computing/precision_and_recall.jpg}
		\caption[Precision and Recall]{Precision and Recall (source: Wikipedia)}
	\end{figure}
	The $F_1$ score is the harmonic average of the precision and recall (for a given class $i$) where an $F_1$ score reaches its best value at $1$ (perfect precision and recall) and worst at $0$ (it will seems obvious to the reader with the details given further below).
	
	The $F$-score is also used in Machine Learning as a suitable measure of models tested with unbalanced datasets. Note, however, that the $F$-measures do not take the true negatives into account, and that measures such as the Matthews correlation coefficient, Informedness or Cohen's kappa may be preferable to assess the performance of a binary classifier.
	
	David Hand and others criticize the widespread use of the $F$-score since it gives equal importance to precision and recall. In practice, different types of miss-classifications incur different costs. In other words, the relative importance of precision and recall is an aspect of the problem.
	
	\begin{tcolorbox}[title=Remark,colframe=black,arc=10pt]
	The text that follows is completely inspired and copied from the excellent publication of Professor Yutaka Sasaki (see \cite{sasaki2007truth}) with it's kind authorization.
	\end{tcolorbox}
	
	
	The $F$-measure is defined as a harmonic mean (\SeeChapter{see section Statistics page \pageref{harmonic mean}}) of precision $P$ and recall $R$:
	
	If you are satisfied with this definition and need no further information, that's it. However, if you are deeply interested in the definition of the $F$-measure, you should recap the definitions of the arithmetic and harmonic means.
	
	The arithmetic mean $\mu_a$ (an average in a usual sense) and the harmonic mean $\mu_h$ are defined for recall as follows:
	
	When $x_1 = P$ and $x_2 = R$, $\mu_a$ and $\mu_h$ will be:
	
	Let us recall that harmonic mean is more intuitive than the arithmetic mean when computing a mean of ratios. Indeed, suppose that you have a finger print recognition system and its precision and recall be $1.0$ and $0.2$, respectively. Intuitively, the total performance of the system should be very low because the system covers only $20\%$ of the registered finger prints, which means it is almost useless.

	The arithmetic mean of $1$ and $0.2$ is $0.6$ whereas the harmonic mean of them is:
	
	As you see in this example, the harmonic mean ($0.\bar{3}$) is a more reasonable score than the arithmetic mean ($0.6$).
	
	Some researchers named the definition of the $F$-measure given earlier above the "$F_1$-measure". What is stands the $1$ of $F_1$ for?
	
	The full definition of the $F$-measure is given as follows:
	
	$\beta$ is a parameter that controls a balance between $P$ and $R$. When $\beta=1$, $F_1$ comes to be equivalent to the harmonic mean of $P$ and $R$. If $\beta> 1$, $F$ becomes more recall-oriented and if $\beta < 1$, it becomes more precision-oriented, e.g., $F_0 = P$.
	
	However it seems that the real original definition comes from the $E$ "\NewTerm{effectiveness function}" given by:
	
	where:
	
	Let's remove $\alpha$ using $\beta$:
	
	Now you see that:
	
	Note that $F$ rises if $R$ or $P$ gets better whereas $E$ becomes small if $R$ or $P$ improves. This seems the reason why $F$ is more commonly used than $E$.
	
	Some people use $\alpha$ as a parameter of $F$:
	
	There is nothing wrong with this definition of $F$ but use of this definition might cause an unnecessary confusion because $F_{\alpha=0.5}=F_{\beta=1}$. An attention is needed that the commonly used notation $F_1$ means $F_{\beta=1}$, not $F_{\alpha=1}$.
	
	Still, some of you are not sure why $\beta^2$ is used instead of $\beta$ in $\alpha=\frac{1}{\beta^2+1}$ let my try an explanation of the reason.
	
	 $\beta$ is the parameter that controls the weighting between $P$ and $R$. Formally, is defined as follows:
	
	where:
	
	The motivation behind this condition is that at the point where the gradients of $E$ with respect to $P$ and $R$ are equal, the ratio of $R$ against $P$ should be a desired ratio $\beta$.
	
	Please recall that $E$ is defined as follows:
	
	Now we calculate $\frac{\partial E}{\partial P}$ and $\frac{\partial E}{\partial R}$. By the quotient rule on the derivative of a composite function:
	
	For conciseness, let;
	
	Then:
	
	 Then:
	 
	 is equivalent to:
	 
	which can be simplified to:
	
	As $\beta=R / P$, we can replace $R$ with $\beta P^{2}$:
	
	\begin{tcolorbox}[title=Remark,colframe=black,arc=10pt]
	The per-class metrics above (precision $P_i$, recall $R_i$ and $F_{1,i}$-score) can be averaged over all the classes $n$ resulting in "\NewTerm{macro-averaged precision}" (MP), \NewTerm{macro-averaged recall}" (MR) and "\NewTerm{macro-averaged $F_1$}" ($\text{MF}_1$):
	
	When the instances are not uniformly distributed over the classes, it is useful to look at the performance of a multi-class classifier with respect to one class at a time before averaging the metrics. The idea is to compute the "\NewTerm{one-vs-all confusion matrix for each class}" ($3$ matrices in the case of our companion example here!). You can think of the problem as three binary classification tasks where one class is considered the positive class while the combination of all the other classes make up the negative class.\\
	
	Summing up the values of all these matrices results in a unique confusion matrix and allows us to compute weighted metrics such as "\NewTerm{average accuracy}" and "\NewTerm{micro-averaged $R$, $P$ and $F_1$ metrics}". Similar to the overall accuracy, the average accuracy is defined as the fraction of correctly classified instances in the sum of one-vs-all matrices matrix!
	\end{tcolorbox}
	
	\paragraph{Ordinal Logistic Regression}\label{ordinal logistic regression}\mbox{}\\\\
	In statistics, ordinal regression (also called "ordinal classification") is a type of regression analysis used for predicting an ordinal variable, i.e. a variable whose value exists on an arbitrary scale where only the relative ordering between different values is significant.
	
	To introduce the "\NewTerm{ordinal logistic regression}\index{ordinal logistic regression}", let us recall that given a response variable $Y$ having $K$ ordered categories $j=1,2,\ldots,K$, with probabilities:
	
	the multinomial logistic model, considered the $K-1$ ratios:
	
	for $j=1,2,\ldots,K-1$.
	
	Now we will consider the $K-1$ cumulative probabilities instead!
	
	for $j=1,\ldots,K-1$ and writ down a model for each of them.
	
	\begin{tcolorbox}[title=Remark,colframe=black,arc=10pt]
	Note that $\gamma_i^{(K)}=P(Y_i\leq K)=1$ always, so it need not be modelled!
	\end{tcolorbox}
	The following holds:
	 
	 In other words, the ordinal logistic model considers as for the multinomial version, a set of dichotomies, one for each possible cutoff of the response categories into two sets, of "high" and "low" responses. But this concept of "high" and "low" at the opposite of multinomial regression is meaningful only if the categories of $Y$ do have an ordering!
	 
	\begin{tcolorbox}[title=Remark,colframe=black,arc=10pt]
	In Strata and in the \texttt{polr} package of \texttt{R} the ordinal logistic regression model is parametrized as:
	
	instead of:
	
	where $\eta_i=-\beta_{1}$ hence the notation above with the minus signs.
	\end{tcolorbox}
	
	If we consider the ordered labels: Strongly disagree (SD), Disagree (D), Agree (A) and Strongly Agree (SA), the cutoffs are:
	\begin{itemize}
		\item SD vs (D, A or SA)
		\item (SD or D) vs (A or SA)
		\item (SD, D or A) vs SA
	\end{itemize}
	A binary logistic model is then defined for the log-odds of each of these cuts!
	
	The model for the cumulative probabilities is (we omit the index $i$ without loss of generality but just to simplify the notations!):
	
	The intercept terms must be $\alpha^{(1)}<\alpha^{(2)}<\ldots<\alpha^{(K-1)},$ to
guarantee that $\gamma^{(1)}<\gamma^{(2)}<\ldots<\gamma^{(K-1)}$. The $\beta_1,\beta_2,\ldots,\beta_n$ are assumed to the same for each value of $j$.

	From this the practitioner must remember that:
	\begin{itemize}
		\item There is thus only one set of regression coefficients, not $K-1$ as in a multinomial logistic model
		
		\item The curves for $\gamma^{(1)},\gamma^{(2)},\ldots,\gamma^{(K-1)}$  are translated (ie "proportional") as seen in the plots below
		
		\item This is the assumption of "proportional odds". This is why ordinal
logistic model is also known as the "\NewTerm{proportional odds model}\index{proportional odds model}" or "\NewTerm{parallel regression model}\index{parallel regression model}"!
	\end{itemize}
	This is good for the parsimony of the model, because it means that the effect of an explanatory variable on the ordinal response is described by one parameter! However, it is also a restriction on the flexibility of the model, which may or may not be adequate for the data.
	
	In other words, one of the assumptions underlying ordinal logistic (and ordinal probit) regression is that the relationship between each pair of outcome groups is the same. In other words, ordinal logistic regression assumes that the coefficients that describe the relationship between, say, the lowest versus all higher categories of the response variable are the same as those that describe the relationship between the next lowest category and all higher categories, etc. Because the relationship between all pairs of groups is the same, there is only one set of coefficients! If this was not the case, we would need different sets of coefficients in the model to describe the relationship between each pair of outcome groups. 
	
	Thus, in order to assess the appropriateness of our model, we need to evaluate whether the proportional odds assumption is tenable. Statistical tests to do this are available in some software packages. However, it seems that these tests have been criticized for having a tendency to reject the null hypothesis (that the sets of coefficients are the same), and hence, indicate that there the parallel slopes assumption does not hold, in cases where the assumption does hold (see \cite{harrell2001regression} page 335 ). However, Harrell does recommend a graphical method for assessing the parallel slopes assumption. The values displayed in his proposed graph are essentially (linear) predictions from a logit model, used to model the probability that $y$ is greater than or equal to a given value (for each level of $y$), using one predictor $(x)$ variable at a time.

	
	\begin{tcolorbox}[colframe=black,colback=white,sharp corners]
	\textbf{{\Large \ding{45}}Example:}\\\\
	Suppose the proportions of members of the statistical population who would answer "poor", "fair", "good", "very good", and "excellent" are respectively $p_1, p_2, p_3, p_4, p_5$. Then the logarithms of the odds (not the logarithms of the probabilities) of answering in certain ways are:
	
	The proportional odds assumption is that the number added to each of these logarithms to get the next is the same in every case. In other words, these logarithms form an arithmetic sequence. The model states that the number in the last column of the table - the number of times that the logarithm must be added - is some linear combination of the other observed variables.
	\end{tcolorbox}

	The probabilities of individual categories are:
	
	Illustrated below with plots for a case with $K = 4$ categories, first the cumulative probabilities:
	\begin{figure}[H]
		\centering
		\includegraphics[width=0.75\textwidth]{img/computing/ordinal_logistic_regression.jpg}
	\end{figure} 
	and the probabilities of individual categories:
	\begin{figure}[H]
		\centering
		\includegraphics[width=0.75\textwidth]{img/computing/ordinal_logistic_regression_individual_categories_probabilities.jpg}
	\end{figure} 
	Everything here is unchanged from binary logistic models:
	\begin{itemize}
		\item Parameters are estimated using maximum likelihood estimation
		
		\item Hypotheses of interest are typically of the form $\beta_j=0$, for one or more coefficients $\beta_j$
		
		\item Exponentiated coefficients are interpreted as partial odds ratios for being in the higher rather than the lower half of the dichotomy
		
		\item Wald tests, likelihood ratio tests and confidence intervals are the same defined and used as before
	\end{itemize}
	
	\begin{tcolorbox}[colframe=black,colback=white,sharp corners]
	\textbf{{\Large \ding{45}}Example:}\\\\
	In Stata and \texttt{R} (package \texttt{polr}) the ordinal logistic regression model is parametrized a:
	
	where for recall $\eta_{i}=-\beta_{i}$.\\
	
	Suppose we want to see whether a binary predictor parental education (\textit{pared}) predicts an ordinal outcome of students who are unlikely, somewhat likely and very likely to apply to a college (\textit{apply}).\\

	Due to the parallel odds assumption, even though we have categories, the coefficient of parental education (\textit{pared}) stays the same across the two categories. Then the two equations for pared $=1$ and pared $=0$ are:
	
	Then:
	
	To run an ordinal logistic regression in \texttt{R} we run (see our \texttt{R} companion book for more details):
	\begin{figure}[H]
		\centering
		\includegraphics[scale=0.6]{img/computing/ordinal_logistic_regression_example.jpg}
	\end{figure}
	The output shows that for students whose parents attended college, the log odds of being unlikely to college (versus somewhat or very likely) is actually $-\hat{\eta}_{1}=-1.13$ or 1.13 points lower than students did not attend college. Recall that $-\eta_{i}=\beta_{i}$ for $j=1,2$ only since logit $(P(Y \leq 3))$ is undefined. So the formulations for the first and second category becomes:
	
	To see the connection between the proportional odds assumption, exponentiate both sides of the equations above and use the property that $\log (b)-\log (a)=\log (b / a)$ to calculate the odds of \textit{pared} for each level of \textit{apply}:
	\end{tcolorbox}
	
	\begin{tcolorbox}[colframe=black,colback=white,sharp corners]
	 
	From the odds of each level of pared, we can calculate the odds ratio of \textit{pared} for each level of \textit{apply}:
	
	The proportional odds assumption ensures that the odds ratios across all $J-1$ categories are the same. In our example, the proportional odds assumption means that the odds of being unlikely versus somewhat or very likely to apply $(j=1)$ is the s the odds of being unlikely and somewhat likely versus very likely to apply $(j=2)$.\\
	
	The proportional odds assumption is not simply that the odds are the same but that the odds ratios are the same across categories. These odds ratios can be derived by exponentiating the coefficients (in the log-odds metric), but the interpretation is a bit unexpected. Recall that the coefficient  $-\eta_1$ represents a one unit change in the log odds of applying for students whose parents went to college versus parents who did not:
	
	Since the exponent is the inverse function of the log, we can simply exponentiate both sides of this equation, and by using the property that $\log(b)-\log(a)=\log(b/a)$:
	
	For simplicity of notation and by the proportional odds assumption, let:
	
	Then the odds ratio is defined as:
	\end{tcolorbox}
	
	\begin{tcolorbox}[colframe=black,colback=white,sharp corners]
	
	In our example, $e^{-1.13}=0.323$, which means that students whose parents attended college have a $67.6\%$ lower odds of being less  likely to apply to college.\\
	
	But you won't get this value with \texttt{R} and Stata (they retrieve the value $3.087$ instead). Let's see why!\\
	
	Since:
	
	From the output $\eta=1.13$, which means that $3.086$ come from actually:
	
	This suggests that students whose parents did not go to college have higher odds of being less likely to apply.\\
	
	Double negation can be logically confusing. Suppose we wanted to interpret the odds of being more likely to apply to college. We can perform a slight manipulation of our original odds ratio:
	
	Since $ e^{-\eta_{1}}=1/ e^{\eta_{1}}$, then we have:
	
	Instead of interpreting the odds of being in the $j$ th category or less, we can interpret the odds of being greater than the $j$ th category by exponentiating $\eta$ itself. In our example, $e^{\eta_{1}}=e^{1.127}=3.086$ means that students whose parents went to college have $3.086$ times the odds of being very likely to apply (vs. somewhat or unlikely) compared to students whose parents did not go to college. The results here are consistent with our intuition because it removes double negatives. As a general rule, it is easier to interpret the odds ratios of $x_{1}=1$ vs. $x_{1}=0$ by simply exponentiating $\eta$ itself rather than interpreting the odds ratios of $x_{1}=0$ vs. $x_{1}=1$ by exponentiating $-\eta$. However by doing so, we flip the interpretation of the outcome by placing $P(Y>j)$ in the numerator.
	\end{tcolorbox}
	
	\pagebreak
	\paragraph{Log-Loss}\mbox{}\\\\
	Let us recall the relation that we have introduced just earlier for the binomial logistic regression:
	
	It is often rewritten as following and defined as the "\NewTerm{Log-loss}\index{Log-loss}" (often used in Data Mining/Data Science that like to use the log in base $10$ instead of the natural logarithm...):
	
	with $y_i\in\{0,1\}$ type of data with estimated $\hat{y}_i\in \{0,1\}$, thus $\text{LL}>0$ (the Log-loss gradually declines as the predicted probability improves as the reader can notice it). In other words, Log-loss is used when we have $\{0,1\}$ response. This is usually because when we have $\{0,1\}$ response, the best models give us values in terms of probabilities.
	
	It is also sometimes written as following:
	
	
	In simple words (even if the Log-loss function is simply implicitly the objective function to minimize, in order to  fit a log linear probability model to a set of binary labelled examples!), Log-loss measures the uncertainty of the probabilities of your model by comparing them to the true labels. Let us look closely at its formula and see how it measures the uncertainty 

	Now the question is, your training labels are $0$ and $1$ but your training predictions are $0.4$, $0.6$, $0.89$, $0.1122$, etc. So how do we calculate a measure of the error of our model ? If we directly classify all the observations having values $> 0.5$ into $1$ then we are at a high risk of increasing the misclassification. This is because it may so happen that many values having probabilities $0.4$, $0.45$, $0.49$ can have a true value of $1$.

	This is where Log-loss comes into picture!
	
	Now let us closely follow the formula of Log-loss. There can be four major cases for the values of $y_i$ and $p_i$:
	\begin{enumerate}
		\item[C1.] $y_i=1$ , $p_i =$ High , $1-y_i=0$ , $1-p_i =$ Low
	
		\item[C2.] $y_i=1$ , $p_i =$ Low , $1-y_i=0$ , $1-p_i =$ High
	
		\item[C3.] $y_i=0$ , $p_i =$ Low , $1-y_i=1$ , $1-p_i =$ High
	
		\item[C4.] $y_i=0$ , $p_i =$ High , $1-y_i=1$ , $1-p_i =$ Low
	\end{enumerate}
	\begin{itemize}
		\item Case 1:

		In this case $y = 1$ and $p =$ high implies that we have got things right! Because the true value of the response agrees with our high probability. Now look closely! Occurrence of Case 1 will significantly decrease the sum because, $y_i\log(p_i)$ would be small and simultaneously the other term in the summation would be zero since $1 - y_i = 1 - 1 = 0$. So more occurrences of Case 1 would decrease the sum and consequently decrease the mean.
		
		Also note that this is possible because if:
		
		
		\item Case 2:
		
		In this case $y = 1$ and $p =$ Low. This is a totally undesirable case because our probability of $y$ being $1$ is low but still the true value of $y$ is $1$. Now again looking at the Log-loss closely, the second term in the summation would be very small since $1-y_i$ would almost equal to zero. And since $p =$ Low, $y_i\log(p_i)$ would inflate the sum at the opposite of Case 1. So Case 2 would ultimately affect the sum a lot.
	\end{itemize}
	
	Similarly the occurrences of Case 3 would not change the sum and occurrences of Case 4 would significantly.
	
	Now coming back to the main question, how does log loss measure uncertainty of your model? The answer is simple! Suppose we have more of Case 1 and Case 3, then the sum inside the Log-loss relation would be small (would tend to decrease). This would imply that the mean ($/n$) would also tend to decrease and will be substantially small in comparison to what it would have been if Case 2 and Case 4 got added. So now this value is as small as possible at Case 1 and Case 3 which indicates a good prediction. This is why smaller the value, better is the model i.e. smaller the Log-loss, better is the model i.e. smaller the uncertainty.
	
	\paragraph{Odds Ratio Confidence Interval}\label{odds ratio confidence interval}\mbox{}\\\\
	We have seen earlier above during our study of the binomial logistic regression (see page \pageref{odds ratio logistic regression}) that the odds ratio was defined by:
	
	What we can also write as:
	
	But that many practitioners write it as following:
	
	Since the ratio of proportions is equivalent to the ratio of the size of the concerned populations, it is customary to write the estimator of the odds ratio as being:
	
	Which then relates to a table typically of the following kind in the medical field (and other fields obviously!):
	\begin{table}[H]
		\centering
		\begin{tabular}{|l|c|c|c|}
		\hline
		 & \multicolumn{1}{l|}{\cellcolor[HTML]{9B9B9B}\textbf{Sick}} & \multicolumn{1}{l|}{\cellcolor[HTML]{9B9B9B}\textbf{Non-Sick}} & \multicolumn{1}{l|}{\cellcolor[HTML]{9B9B9B}\textbf{Odds}} \\ \hline
		\cellcolor[HTML]{9B9B9B}\textbf{Treated Group} & $n_1$ & $n_2$ & $n_1/n_2$ \\ \hline
		\cellcolor[HTML]{9B9B9B}\textbf{Non-Treated Group} & $n_3$ & $n_4$ & $n_3/n_4$\\ \hline
		\cellcolor[HTML]{9B9B9B}\textbf{Totals} & $n_1+n_3$ & $n_2+n_4$ & $\widehat{\text{O.R.}}=\dfrac{\frac{n_1}{n_2}}{\frac{n_3}{n_4}}$\\ \hline
		\end{tabular}
	\end{table}
	In the field of survival analysis (\SeeChapter{see section Statistics page \pageref{survival analysis}}) such a table will be represented as following:
	\begin{table}[H]
		\centering
		\begin{tabular}{|l|c|c|c|}
		\hline
		 & \multicolumn{1}{l|}{\cellcolor[HTML]{9B9B9B}\textbf{Observed}} & \multicolumn{1}{l|}{\cellcolor[HTML]{9B9B9B}\textbf{Expected}} & \multicolumn{1}{l|}{\cellcolor[HTML]{9B9B9B}\textbf{Relative failure rate}} \\ \hline
		\cellcolor[HTML]{9B9B9B}\textbf{Survival Curve 1} & $O_1$ & $E_1$ & $O_1/E_1$ \\ \hline
		\cellcolor[HTML]{9B9B9B}\textbf{Survival Curve 2} & $O_2$ & $E_2$ & $O_2/E_2$\\ \hline
		\cellcolor[HTML]{9B9B9B}\textbf{Totals} & $O_1+O_2$ & $E_2+E_4$ & $\widehat{\text{O.R.}}=\dfrac{\frac{O_1}{E_1}}{\frac{O_2}{E_2}}$\\ \hline
		\end{tabular}
	\end{table}
	\begin{tcolorbox}[title=Remark,colframe=black,arc=10pt]
	 An O.R. of $1.00$ means that the two groups were equally likely to remain sick. An O.R. higher than $1.00$ means that a group is more likely to experience the event than the second group. An O.R. of less than $1.00$ means that a group was less likely to experience the event. However, an O.R. value below $1.00$ is not directly interpretable. It is important to put the group expected to have higher odds of the event in the first row.
	\end{tcolorbox}
	Now, let us consider that the $n_i$ follow an asymptotically Normal law such as according to what we had seen in our study of the construction of the Normal Law:
	
	Now the idea will be to make an additional approximation ... otherwise we will not find anything simple. We will put that:
	
	It is very very approximate as approach .... since rigorously:
	
	Simulations show that we have a Normal law beyond $n$ equating $30$, but although it is a Normal centered law ... it is hardly reduced (therefore not of unit variance...).
	
	Regarding the last choice established, we have:
	
	This last approximation then allows us to write (we detail the steps a maximum on the request of one of our reader):
	
	Now let us take the logarithm (it is customary to take the natural logarithm):
	
	The second term can be developed in Taylor series of the first order (\SeeChapter{see section Sequences and Series  page \pageref{usual maclaurin developments}}):
	
	It comes then after many approximations ...:
	
	We then have:
	
	However, we observe that (do not forget that the variance of a random variable multiplied by a constant puts the constant squared):
	
	We then first consider that the variance part of the random variables as independent (we will take into account the covariance just after):
	
	We now have to take into account the following six covariances (refer to the section Statistics for the proof that when we sum $n$ random variables there are as many covariances to compute as possible pairs of possible variables):
	
	Recall that during our study of the multinomial law and as part of an application for the McNemar test (page \pageref{mcnemar test}), we proved that:
	
	which in our context with a more general notation is written:
	
	If we center these two random variables (as is the case in our developments so far), the covariance does not change:
	
	If we reduce the random variables by $\sqrt{n}$, then we have:
	
	and therefore:
	
	Among the six covariances that we need to calculate we have therefore:
	
	So the sum is worth $-2$. It is necessary to multiply by $2$ the whole (reminder of the proof of the general case which makes a factor $2$ appear before each term of the covariance) to obtain the global value of the covariance terms. So in the end we have a value of $-4$ for the sum of the pure covariances:
	
	The total variance is ultimately given by:
	
	And finally we have the approximation sometimes named "\NewTerm{Woolf method}\index{Woolf method}":
	
	More explicitly (notice that the lower and upper bound are symmetrical):
	
	Relationship that is often written as:
	
	Some softwares calculate the upper and lower bound use the exponential (especially softwares plotting the famous "Four-fold plot"), and therefore the upper and lower bound are no longer symmetrical:
	
	
	\begin{tcolorbox}[colframe=black,colback=white,sharp corners]
	\textbf{{\Large \ding{45}}Example:}\\\\
	Let us consider the following data:
	\begin{table}[H]
		\centering
		\begin{tabular}{|l|c|c|}
		\hline
		 Group / Heart Attack & \multicolumn{1}{l|}{\cellcolor[HTML]{9B9B9B}\textbf{Yes}} & \multicolumn{1}{l|}{\cellcolor[HTML]{9B9B9B}\textbf{No}} \\ \hline
		\cellcolor[HTML]{9B9B9B}\textbf{Placebo} & $53$ & $58$ \\ \hline
		\cellcolor[HTML]{9B9B9B}\textbf{Treated Group} & $11$ & $40$ \\ \hline
		\end{tabular}
	\end{table}
	The odds ratio is then:
	
	So the Placebo group is almost three times more likely to experience the event of interest (Heart Attack).\\
	
	You must keep in mind the three following scenarios (refresh!):
	\begin{itemize}
		\item The odds ratio is almost or equal to $1.0$. means that the odds of exposure among cases is the same as the odds of exposure among controls. In other words, the exposure is not associated with the disease.
	
		\item The odds ratio is greater than $1.0$ means that the odds of exposure among cases is greater than the odds of exposure among controls. In other words, the exposure may be a risk factor for the disease.
	
		\item The odds ratio is less than $1.0$ means that the odds of exposure among cases is lower than the odds of exposure among controls. In other words, the exposure may be protective against the disease.
	\end{itemize}
	
	The confidence interval of $\ln(\text{O.R.})$ at a confidence level of $95\%$ is then
approximately given by:
	
	Thus:
	
	Hence:
	
	and so taking the exponential we finally have the interval that interests us (read further below the text on the corresponding four-fold plot!):
	
	\end{tcolorbox}
	
	\begin{tcolorbox}[colback=red!5,borderline={1mm}{2mm}{red!5},arc=0mm,boxrule=0pt]
	\bcbombe Caution!!! While the estimator of the natural logarithm of the odds ratio is in the middle of the confidence interval, this is obviously not the case with the standard odds ratio!
	\end{tcolorbox}
	
	
	\subparagraph{Four-fold plot}\mbox{}\\\\
	The "\NewTerm{four-fold plot}\index{four-fold plot}" is a graphic designed to display the frequencies in a $2\times 2$ contingency table. In this display the frequency in each cell is shown by a quarter circle, whose area is proportional to the cell count in a given $2\times 2$ layer, in a way that depicts the odds ratios for $1$ strata. Confidence rings for the odds ratio can be superimposed to provide a visual test of the hypothesis of no association in each stratum.
	
	Before we deal with the maths let us look with the \texttt{R} 3.4.2 software how to generate a four-fold plot for the previous example:
	\begin{figure}[H]
		\centering
		\includegraphics[width=1.0\textwidth]{img/computing/fourfold_plot.jpg}
		\caption{Four-fold plot with \texttt{R} 3.4.2}
	\end{figure}
	In this display the frequency $n_i$ in each cell of a four-fold table is shown by a quarter circle (on an vertical and horizontal axis of range $[-1,+1]$ with major ticks each $0.2$ and minor ticks each $0.1$), whose radius is proportional to $\sqrt{n_i}$ so the are is proportional to the cell count (otherwise it would be proportional to the square of the cell count as $S=\pi R^2$).
	
	The four-fold display is constructed so that the four quadrants will align vertically and horizontally when the odds ratio is $1$. Confidence rings for the observed odds ratio provide a visual test of the hypothesis of no association. They have the property that rings for adjacent quadrants overlap if and only if the observed counts are consistent with this null hypothesis. In the above figure the confidence intervals in do not overlap, indicating a significant association between \textit{Heart attack} and \textit{Group}. 
	
	The radius of the circles are given by:
	\begin{table}[H]
		\centering
		\begin{tabular}{|l|c|c|}
		\hline
		 Group / Heart Attack & \multicolumn{1}{l|}{\cellcolor[HTML]{9B9B9B}\textbf{Yes}} & \multicolumn{1}{l|}{\cellcolor[HTML]{9B9B9B}\textbf{No}} \\ \hline
		\cellcolor[HTML]{9B9B9B}\textbf{Placebo} & $\sqrt{\dfrac{n_1}{\max n_i}}$ & $\sqrt{\dfrac{n_2}{\max n_i}}$ \\ \hline
		\cellcolor[HTML]{9B9B9B}\textbf{Treated Group} & $\sqrt{\dfrac{n_3}{\max n_i}}$ & $\sqrt{\dfrac{n_4}{\max n_i}}$ \\ \hline
		\end{tabular}
	\end{table}
	Explicitly:
	\begin{table}[H]
		\centering
		\begin{tabular}{|l|c|c|}
		\hline
		 Group / Heart Attack & \multicolumn{1}{l|}{\cellcolor[HTML]{9B9B9B}\textbf{Yes}} & \multicolumn{1}{l|}{\cellcolor[HTML]{9B9B9B}\textbf{No}} \\ \hline
		\cellcolor[HTML]{9B9B9B}\textbf{Placebo} & $\sqrt{\dfrac{53}{58}}=0.955$ & $\sqrt{\dfrac{58}{58}}=1.000$ \\ \hline
		\cellcolor[HTML]{9B9B9B}\textbf{Treated Group} & $\sqrt{\dfrac{11}{58}}=0.435$ & $\sqrt{\dfrac{40}{58}}=0.830$ \\ \hline
		\end{tabular}
	\end{table}
	
	\paragraph{Risk Ratio and its Confidence Interval}\label{risk ratio}\mbox{}\\\\
	The "\NewTerm{risk ratio}\index{risk ratio}" or "\NewTerm{relative risk}\index{relative risk}" is defined quite naturally by:
	
	Consider the typical case in the medical field:
	\begin{table}[H]
		\centering
		\begin{tabular}{|l|c|c|c|}
		\hline
		 & \multicolumn{1}{l|}{\cellcolor[HTML]{9B9B9B}\textbf{Sick}} & \multicolumn{1}{l|}{\cellcolor[HTML]{9B9B9B}\textbf{Non-Sick}} & \multicolumn{1}{l|}{\cellcolor[HTML]{9B9B9B}\textbf{Total}} \\ \hline
		\cellcolor[HTML]{9B9B9B}\textbf{Treated Group} & $n_1$ & $n_2$ & $n_A$ \\ \hline
		\cellcolor[HTML]{9B9B9B}\textbf{Non-Treated Group} & $n_3$ & $n_4$ & $n_B$\\ \hline
		\end{tabular}
	\end{table}
	Thus the risk of illness for the treated group (TG) is:
	
	and for the non-treated group (NTG) (or "control group"):
	
	and hence the risk ratio is defined by:
	
	Risk ratios (RR) and odds ratios (OR) are both used to analyse factors associated with increased risk of disease. They are different in that a risk ratio is the ratio of two risks and an odds ratio is the ratio of two odds. Risks and odds are calculated differently. For example, with one toss of a fair die, the risk of rolling a $1$ is $1/6 = 0.167$. The odds of rolling a $1$ are $1/5 = 0.200$. In textbook terms, risk is successes divided by total trials and odds are successes divided by failures. In this example rolling a $1$ is a success and rolling $2$ through $5$ is a failure.
	
	\begin{tcolorbox}[colframe=black,colback=white,sharp corners]
	\textbf{{\Large \ding{45}}Example:}\\\\
	Let us consider a R.R of $1.50$. That means obviously an increase of the risk of $50\%$ in the treatment group in comparison of the control group. So if the control group has a risk of $20\%$, then the treatment group has a risk of $20\%+1.5\cdot 20\% = 30\%$.
	\end{tcolorbox}
	
	\begin{tcolorbox}[title=Remark,colframe=black,arc=10pt]
	If $n_1$ is small relatively to $n_2$ and $n_3$ is small relatively to $n_4$ we fall back on the odds ratio:
	
	we then better understand why the odds ratio is sometimes named "\NewTerm{approximate relative risk}".\\
	
	So in the medical field, we say that if the disease is rare in treated and non-treated groups the interpretation of relative risk and odds ratio is the same.
	\end{tcolorbox}
	Let us take the log of the relative risk:
	
	It then comes for the variance since $ p_1 $ and $ p_2 $ are assumed to be independent:
	 
	To determine the variance of the natural logarithm of the proportions we will have to use a technique that we name the "\NewTerm{univariate delta method}\index{delta method}\index{univariate delta method}\label{delta method}" (indeed, because of the presence of a sum at the denominator, we will not be able to take the same approach as the one used for the odds ratio) which consists of approximating the exact variance by the corresponding asymptotically normal variance.

	To see what the univariate delta method is, let's remember that we have proved  in the section Statistics (see page \pageref{normal centered reduced variable}) that:
	
	Now, let us consider the following linear function:
	
	where $a$ and $b$ are any two constants. From then on, using the linearity property of the mean, we immediately have:
	
	Now, let us ask ourselves the question, what is the limit of:
	
	We then have:
	
	We immediately deduce that:
	
	Now, let us develop in series of Taylor in the first order around $\mu$ the function $g(t)$, which would not necessarily be linear, then we have:	
	
	By identification with the simple linear case, we then have:
	
	It then comes in the case of a function $g (t)$ that is not necessarily linear, but approximated as such by a series development of Taylor and under the assumption that the variable is continuous that:
	
	So don't forget what this method means (!!): If $X_n$ is (asymptotically) Normal with mean $0$ and variance $\sigma^2$, $g$ is differentiable and $g'(\mu)\neq 0$, then $g(X_n)$ is approximately given by:
	
	For the purpose of demonstrating the confidence interval of the risk ratio, let us apply that to the binomial distribution with:
	
	It then comes from the previous results that the approximation of the variance by a Normal law and from the knowledge that the variance and the mean of the binomial law are given for recall by (\SeeChapter{see section Statistics page \pageref{binomial distribution}}):
	
	that the univariate delta method gives:
	
	So the variance corresponding to $\ln(p_i)$ is given asymptotically by:
	
	Therefore:
	
	Hence:
	
	Thus, we have asymptotically the following interval for the risk ratio if the assumptions are satisfied (independence and underlying Normal distribution):
	
	or in condensed form:
	
	Note that in the medical literature, the reader will also find the standard deviation of the risk ratio in many other forms that are here (which are of course all equivalent):
	
	If the risk ratio is the same for both target groups, the Risk Ratio will obviously be equal to $100\%$ (ie equal to $1$). If it is less than $100\%$ (ie less than $1$), it simply means that the risk of illness in the treated group is lower than that of the non-treated group, etc.
	
	The typical range of values of the odds ratio and risk ratio are designated (wrongly!) traditionally as following in scientific publications:
	\begin{figure}[H]
		\centering
		\includegraphics[scale=0.7]{img/computing/odds_ratio_risk_ratio_range_values.jpg}
		\caption{Odds Ratio and Risk Ratio range values subjective qualifications}
	\end{figure}
	
	\begin{tcolorbox}[colback=red!5,borderline={1mm}{2mm}{red!5},arc=0mm,boxrule=0pt]
	\bcbombe Caution!!! While the natural logarithm Risk Ratio estimator is in the middle of the confidence interval, this is obviously not the case for the standard Risk Ratio!
	\end{tcolorbox}
	
	
	\begin{tcolorbox}[colframe=black,colback=white,sharp corners]
	\textbf{{\Large \ding{45}}Example:}\\\\
	Let us consider the following data:
	\begin{table}[H]
		\centering
		\begin{tabular}{|l|c|c|c|}
		\hline
		Group/Heart attack & \multicolumn{1}{l|}{\cellcolor[HTML]{9B9B9B}\textbf{Yes}} & \multicolumn{1}{l|}{\cellcolor[HTML]{9B9B9B}\textbf{No}} \\ \hline
		\cellcolor[HTML]{9B9B9B}\textbf{Placebo} & $53$ & $58$ \\ \hline
		\cellcolor[HTML]{9B9B9B}\textbf{Aspirin} & $11$ & $40$ \\ \hline
		\end{tabular}
	\end{table}
	The relative risk is then (which corresponds to what the MedCalc software returns):
	
	The confidence interval of $\ln(\widehat{\text{R.R.}})$ at a confidence level of $95\%$ is then approximately given by:
	
	Hence:
	
	So finally:
	
	and so taking the exponential we finally have the interval that interests us:
	
	This perfectly matches the data returned by the MedCalc software!
	\end{tcolorbox}
	The reader or practitioner must keep in mind that the odds ratio, as the risk ratio, are tools as we have just seen, used to make null hypothesis statistical testing (NHST) as they involve confidence intervals. So don't forget that we are seeking is to reject the null hypothesis $H_0$ ! Because of this, a lot of people read medical studies wrongly as they forget how NHST works!
	
	So for recall:
	\begin{figure}[H]
		\centering
		\includegraphics[scale=1]{img/computing/risk_ratio_odds_ratio_normal_tests.jpg}
		\caption{Risk Ratio (relative risk), Odds Ratio or Hazard Ratio rules}
	\end{figure}
	So far we have seen that in calculation of patient risks there are two indicators. But that's quite wrong in practice... Indeed, it is quite common to find the following indicators in many scientific papers:
	\begin{itemize}
		\item Odds ratio:
		
	
		\item Relative risk:
		
	
		\item Experimental event rate:
		
	
		\item Control event rate:
		
	
		\item Absolute risk increase (also named "attributable risk"):
		
	
		\item Absolute risk reduction:
		
	
		\item Number needed to harm:
		
	\end{itemize}
	And keep always in mind what we already mentioned during our study of Chemistry:
	\begin{fquote}[Paracelsus]All things are poisons, for there is nothing without poisonous qualities. It is only the dose which makes a thing poison!
 	\end{fquote}
 	\begin{tcolorbox}[title=Remark,colframe=black,arc=10pt]
	A famous example of personal patient risk appreciation is about industrial drugs or medicines usage risk! Indeed, it seems that on the $18,000$ drugs commercially released this last century around the World only approximately $177$ of them (source: \url{https://en.wikipedia.org/wiki/List_of_withdrawn_drugs}) were withdrawn because of risks to patients, but also because of commercial reasons (e.g. lack of demand and relatively high production costs).
	\end{tcolorbox}
	
	Let us take an example taken during the COVID-19 pandemic published the 2021-04-03 by FranceInfo: According to the numbers that the EMA unveiled on Wednesday, there were at this stage $62$ cases of cerebral venous thrombosis in the world, including $44$ in the $30$ countries of the European Economic Area (EU, Iceland, Norway, Liechtenstein) for $9.2$ million doses of vaccine administered. $9$ deaths have been recorded in Germany out of $2.8$ millions AstraZeneca vaccine injected. In France $4$ death have been reported out of $1.9$ million injections, $3$ death cases in Norway out of $120,000 injections$ in the Netherlands. But as with any drug, knowing the risk is not enough, it must be compared with the benefits provided by the product. The Risk Ratio of the AstraZeneca vaccine in preventing COVID-19, which leads to hospitalizations and much more deaths, obviously outweigh the risks of side effects.
	
	\paragraph{ROC and Lift curves}\label{ROC and Lift curves}
	\subparagraph{ROC curve}\mbox{}\\\\
	The "\NewTerm{ROC curve}\index{ROC curve}", that stands for "\NewTerm{Receiver Operating Characteristic curve}\index{ROC curve}", is a measure of the performance of a binary classifier, that is to say a system that aims to categorize entities in two distinct groups on the basis of one or more of their characteristics. 

	Graphically, we often represents the ROC measurement  in the form of a curve which gives the true positive rate (sensitivity: fraction of positives that are correctly detected) according to the rate of false positive rate (proportion of negatives which are incorrectly detected) for the same group. ROC curves are often used in statistics to show the progress realized with a binary classifier when "\NewTerm{discrimination threshold}\index{discrimination threshold}" (cutoff) varies.
	\begin{tcolorbox}[title=Remark,colframe=black,arc=10pt]
	We present this tool because many statistical software return automatically a ROC curve as output but the interest of this tool is in my personal opinion very questionable. However, we will show a more useful tool right after.
	\end{tcolorbox}
	Let us see practical case by taking our previous example (the Bank and its subsidiaries) with the spreadsheet software Microsoft Excel 14.0.7106. First to build the ROC curve we must obtain the confusion matrix $\mathcal{C}$ that we presented just a little higher with Minitab. For this with a spreadsheet software and without doing code, here is a simple solution (but this is not the more condensed solution for pedagogical reasons).
	
	The goal will be first to build the following table (this is only a part of the table of the previous example since in reality there are 136 rows of data) whose columns \texttt{B} and \texttt{C} comes from the three small tables of Credits used just before, column \texttt{A} is just the cumulative frequency of individuals $1/136$ ... $2/136$ ... $3/136$ ... and so on):
	\begin{figure}[H]
		\centering
		\includegraphics[width=1.0\textwidth]{img/computing/roc_curve_excel_data.jpg}
		\caption{List to obtain in Microsoft Excel to obtain for the ROC curve}
	\end{figure}
	Now let us see what is in columns \texttt{D}, \texttt{E} and \texttt{F} and that are directly related to the result obtained above, which was for refresh:
	
	but that we have refined with a specialized statistical software to obtain:
	
	which then allows us to build the three previous mentioned columns (here we have only a few of the first lines because it is sufficient to just increment the formulas to the bottom of the list):
	\begin{figure}[H]
		\centering
		\includegraphics{img/computing/roc_curve_excel_model_and_score.jpg}
		\caption{Spreadsheet formulas for the probability of the model and the score}
	\end{figure}
	The formula in column \texttt{F} refers to the cell \texttt{M3} which as we shall see a little further contains the empirical choice of the value of the cutoff we had mentioned during the presentation of the theoretical model of the logistics regression (by default we defined it at $50\%$).
	
	Then we must build the columns of true positive, false positive, true negative, false negatives using basic spreadsheet formulas:
	\begin{figure}[H]
		\centering
		\includegraphics[width=1.0\textwidth]{img/computing/roc_curve_excel_positive_and_negative.jpg}
		\caption{Spreadsheet formulas for true / false positives and negatives}
	\end{figure}
	Once we have this data, we can rebuild the confusion matrix $\mathcal{C}$ that was given to us by Minitab and we will need to develop the ROC curve:
	\begin{figure}[H]
		\centering
		\includegraphics{img/computing/confusion_matrix_formulas.jpg}
		\caption{Spreadsheet formulas to build the confusion matrix}
	\end{figure}
	which explicitly gives:
	\begin{figure}[H]
		\centering
		\includegraphics{img/computing/confusion_matrix_values.jpg}
		\caption{Corresponding values of the construction of the confusion matrix}
	\end{figure}
	compared with the confusion matrix that we gave and coming from Minitab:
	\begin{figure}[H]
		\centering
		\includegraphics{img/arithmetics/confusion_matrix.jpg}
		\caption{Confusion matrix Minitab 16.1.1}
	\end{figure}
	Now observe the ROC curve given by Minitab:
	\begin{figure}[H]
		\centering
		\includegraphics{img/computing/minitab_roc_curve.jpg}
		\caption{ROC curve given by Minitab 16.1.1}
	\end{figure}
	Why the abscissa-represents:
	
	you will probably ask yourself when the interpretation would have been easier if we had just the:
	
	Well for two reasons: the first is that practitioners like strictly increasing functions... and the second reason, the most important, is that if the binary classifier is fully effective, the surface is then equal to 1$ $(that is to say, $100\%$). Which admittedly is nicer than to say that a null surface corresponds to a $100\%$ efficiency. Well that being said ... let us continue.
	
	Before learning how to interpret this chart, how to get the same curve in a spreadsheet software? Well just simply observe our column \texttt{D} of our Microsoft Excel list. Logically we have there $5$ different cumulative probabilities (as there were only $5$ credits values) which are respectively:
	
	Therefore the idea of the construction of the ROC curve is to take each of the accumulated probabilities as cutt-off values. This give respectively:
	\begin{figure}[H]
		\centering
		\includegraphics{img/computing/confusion_matrix_cutoff1.jpg}
		\includegraphics{img/computing/confusion_matrix_cutoff2.jpg}
		\includegraphics{img/computing/confusion_matrix_cutoff3.jpg}
		\includegraphics{img/computing/confusion_matrix_cutoff4.jpg}
		\includegraphics{img/computing/confusion_matrix_cutoff5.jpg}
		\caption{Values of the confusion matrix for various cutoff values}
	\end{figure}
	So we get each time the coordinates of points of the ROC curve in function of cutoff values. Well this done, now let us turn to the interpretation by taking the below manually completed graph:
	\begin{figure}[H]
		\centering
		\includegraphics{img/computing/minitab_roc_curve_completed.jpg}
		\caption{ROC curves (blue: average classifier, red: good classifier, green: perfect classifier, black: poor classifier)}
	\end{figure}
	Then as shown in the graph above and the intuition, a perfect binary classifier is one whose true positive rate is constant and always $100\%$. So the binary classifier in our example is moderately good.
	
	But from my personal point of view a good binary classifier is one that for a given value of the cutoff maximizes the sum of the rate of true positives and true negatives. Thus, with a spreadsheet software such as Microsoft Excel in "evolutionary mode" it is very easy to find this cutoff value that maximizes this objective. But however be careful to the trap: the value may not be unique, it can also be an interval (which is the case in our example!).
	
	Finally, if we denote by TP the true positive, TN the true negative, FP the false positives or Type I error (in terms of decision theory or tests theory), and FN false negative or Type II error we can then define a set of indicators to judge the quality of our predictor/classifier (or rather of our score):
	
	\begin{itemize}
		\item $\text{TPR} = \dfrac{\text{TP}}{\text{P}} = \dfrac{\text{TP}}{\text{TP+FN}}$ named "\NewTerm{sensitivity}\index{classification!sensitivity}", corresponding to the rate of true positives.
		
		\item $\text{RPF} = \dfrac{\text{FP}}{\text{N}} = \dfrac{\text{FP}}{\text{FP+TN}}$ corresponding to the rate of false positives.
		
		\item $\text{ACC} = \dfrac{\text{TP+TN}}{\text{P+N}}$ named "\NewTerm{accuracy}\index{classification!accuracy}" (fraction of instances that are correctly classified).
		
		 \item $\text{SPC} = \dfrac{\text{TN}}{\text{N}} = \dfrac{\text{TN}}{\text{FP+TN}}=1-\text{FPR}$ named "\NewTerm{specificity}\index{classification!specificity}" or true negative rate.
		 
		 \item $\text{PPV} = \dfrac{\text{TP}}{\text{TP+TP}}$ the "\NewTerm{positive predictive value}\index{classification!positive predictive value}".
		 
		 \item $\text{NPV} = \dfrac{\text{TN}}{\text{TN+FN}}$ the "\NewTerm{negative predictive value}\index{classification!negative predictive value}".
		 
		 \item $\text{FDR} = \dfrac{\text{FP}}{\text{FP+TP}}$ corresponding to the  "\NewTerm{false discovery rate}\index{classification!false discovery rate}".
	\end{itemize}
	And we can build many quality metrics like (for the most famous one!) the Matthews correlation coefficient (\SeeChapter{see section Statistics page \pageref{Matthews correlation coefficient}}), the $F_1$ score (see page \pageref{F1 score}) or the "\NewTerm{Rand Index}\index{classification!Rand index}" (also sometimes named "\NewTerm{model accuracy}\index{classification!model accuracy}") defined by:
	
	
	\begin{tcolorbox}[title=Remark,colframe=black,arc=10pt]
	In practice, the sensitivity and specificity\index{sensitivity and specificity} of a test give an assessment of its intrinsic validity. It is considered in practice that taken separately, they do not mean anything. For example, it should be obvious for the reader a test with $95\%$ sensitivity has no value if its specificity is only $5\%$. It is considered in practice that if the sum of sensitivity and specificity is equal to $100\%$ the test is as good a random.
	\end{tcolorbox}
	\begin{tcolorbox}[colframe=black,colback=white,sharp corners]
	\textbf{{\Large \ding{45}}Example:}\\\\
	Consider a disease test sensibility and sensitivity analysis with the following values:
	\begin{table}[H]
		\centering
		\begin{tabular}{ll|c|c|l}
		\cline{3-4}
		 &  & \multicolumn{2}{c|}{\begin{tabular}[c]{@{}c@{}}Patients with bowel cancer\\ (as confirmed by endoscopy)\end{tabular}} &  \\ \cline{3-4}
		 &  & \cellcolor[HTML]{EFEFEF}\textbf{Positive} & \cellcolor[HTML]{EFEFEF}\textbf{Negative} &  \\ \hline
		\multicolumn{1}{|l|}{\cellcolor[HTML]{FFFFFF}} & \multicolumn{1}{c|}{\cellcolor[HTML]{EFEFEF}\textbf{Positive}} & \cellcolor[HTML]{9AFF99}TP$=2$ & \cellcolor[HTML]{FFCCC9}FP$=18$ & \multicolumn{1}{l|}{\parbox{1.9cm}{$\dfrac{\text{TP}}{\text{TP}+\text{FP}}\\=\dfrac{2}{2+18}\\=10\%$}} \\ \cline{2-5} 
		\multicolumn{1}{|l|}{\multirow{-2}{*}{\cellcolor[HTML]{FFFFFF}Disease Test}} & \multicolumn{1}{c|}{\cellcolor[HTML]{EFEFEF}\textbf{Negative}} & \cellcolor[HTML]{FFCCC9}FN$=1$ & \cellcolor[HTML]{9AFF99}TN$=182$ & \multicolumn{1}{l|}{\parbox{1.9cm}{$\dfrac{\text{TN}}{\text{TN}+\text{FN}}\\=\dfrac{182}{1+182}\\=99.5\%$}} \\ \hline
		 &  & \multicolumn{1}{l|}{\parbox{3.6cm}{Sensitivity\\$=\dfrac{\text{TP}}{\text{TP}+\text{FN}}\\=\dfrac{2}{2+1}=66.\bar{6}\%$}} & \multicolumn{1}{l|}{\parbox{3.6cm}{Specificity\\$=\dfrac{\text{TN}}{\text{FP}+\text{TN}}\\=\dfrac{182}{18+182}=91\%$}} &  \\ \cline{3-4}
		\end{tabular}
		\caption{Example of sensibility and specificity analysis}
	\end{table}
	There are some cases where sensitivity is important and need to be near to $1$. There are business cases where specificity is important and need to be near to $1$ (ideally we want to maximize both sensitivity \& specificity, but this is not always possible always). We need to understand the business problem and decide the importance of sensitivity and specificity.
	\end{tcolorbox}
	\begin{tcolorbox}[title=Remark,colframe=black,arc=10pt]
	Current generation HIV tests are considered extremely accurate. The blood-based HIV ELISA has a demonstrated sensitivity of between $99.3\%$  to 99.7 percent, with a specificity of between $99.91\%$ and $99.97\%$. When combined with a Western blot, this translates to approximately one false positive out of every $250,000$ tests in the general U.S. population. Newer, fourth generation combination tests, which test for both HIV antibodies and antigens, are reported to have a clinical sensitivity of $99.9\%$.
	\end{tcolorbox}
	
	As you may have notice in the previous chart above. Below the title of the chart there was a mention of something named "\NewTerm{Area Under the Curve}\index{area under the curve (AUC)}" and that is well know under the abbreviation AUC.

	Obviously, the names mean what it is: It measures the entire two-dimensional area underneath the entire ROC curve (think integral calculus) from $[0,0]$ to $[1,1]$.

	AUC provides an aggregate measure of performance across all possible classification thresholds. One way of interpreting AUC is as the probability that the model ranks a random positive example more highly than a random negative example. For example, given the following examples, which are arranged from left to right in ascending order of logistic regression predictions:
	\begin{figure}[H]
		\centering
		\includegraphics[width=1.0\textwidth]{img/computing/auc_illustration.jpg}
	\end{figure}
	AUC represents the probability that a random positive (green) example is positioned to the right of a random negative (red) example. The corresponding mathematical expression is quite obviously given by:
	
	Here $i$ runs over all $m$ data points with true label $1,$ and $j$ runs over all $n$ data points with true label $0 ; p_{i}$ and $p_{j}$ denote the probability score assigned by the classifier to data point $i$ and $j$, respectively. $1$ is always the indicator function: it outputs $1$ if and only if the condition ($p_{i}>p_{j}$) is satisfied!
	
	AUC ranges in value from $0$ to $1$. A model whose predictions are $100\%$ wrong has an AUC of $0.0$; one whose predictions are $100\%$ correct has an AUC of $1.0$. A model that doesn't better than change will have an AUC of $50\%$.
	
	Testing the fact evidence that the AUC is significantly different of $0.5$ is like the null hypothesis $H_0$: \textit{The distribution of the ranks in the two groups are equal}. It's then the equivalent of running a Wilcoxon rank sum test (\SeeChapter{see section Statistics page \pageref{Wilcoxon rank sum test}}).
	
	\begin{tcolorbox}[title=Remark,colframe=black,arc=10pt]
	In logistic regression if we consider then probabilities $\left\lbrace \left(\hat{\pi}(\hat{\vec{\beta}}|\vec{x}_i\right),\hat{\pi}\left(\hat{\vec{\beta}},\vec{x}_j\right)\right\rbrace$, the $C$-index can be defined as:
	
	Which is equivalent to:
	
	\end{tcolorbox}
	
	AUC is desirable for the following two reasons:
	\begin{itemize}
		\item AUC is scale-invariant. It measures how well predictions are ranked, independently of their absolute values.

		\item AUC is classification-threshold-invariant. It measures the quality of the model's predictions irrespective of what classification threshold is chosen.
	\end{itemize}
	Notice that there is a simple relation between the AUC and the Gini index. Indeed, as we have proved it in the section of Quantitative Management (see page \pageref{gini index}), we have:
	
	But as:
	
	We then have:
	
	
	\begin{tcolorbox}[title=Remark,colframe=black,arc=10pt]
	In multi-class models, we can plot $N$ number of AUC ROC curves for $N$ number classes using the "One vs All methodology". So for example, if you have three classes named $A$, $B$ and $C$, you will have one ROC for $A$ classified against $B$ and $C$, another ROC for $B$ classified against $A$ and $B$, and a third one $C$ classified against $A$ and $B$.
	\end{tcolorbox}
	
	Having done this, let us move to the second curve.
	
	\subparagraph{Lift curve}\mbox{}\\\\
	The principle of the "\NewTerm{lift curve}\index{lift curve}" is very simple but is based on a questionable assumption (which fortunately has not however a great importance) that is that without the theoretical model (logistic model in this case) we will absolutely know nothing of the real behaviour that will have the customers (debtors). The hypothesis is to consider that if we took $50\%$ of the individuals in our sample, we would have $50\%$ of true positives (real bad debtors), if we took $25\%$ of our sample, we would have $25\%$ of true positives (true bad debtors) and so on. This initial hypothesis (considered by practitioners as the worst case) will be represented graphically by the following straight line:
	\begin{figure}[H]
		\centering
		\includegraphics{img/computing/lift_curve_at_worst.jpg}
		\caption[]{Lift curve at worst (as assumption)}
	\end{figure}
	But that's just to have an empirical reference when working with a single model of statistical clustering. In fact, given the multiplicity of statistical modelling methods, where almost each has its own empirical indicators of quality, statisticians have sought general criteria for the performance of a model (this obviously in the idea to be able to compare different the accuracy of various models between them).
	
	Thus, an empirical and intuitive enough choice is to say that a statistical method is better than another if for a given subset of the sample, its predictive power (the cumulative number of true positives for example) is better or no. This is the purpose of the lift curve.
	
	The lift curve is therefore a variation of the ROC curve. The lift curve classify the individuals by descending score (again for reasons of simplification of interpretation of the surface under the curve and especially to have in first the target group of interest to monitor), by grouping them by percentiles for example, determining the percentage of events of interest in each percentile (normally the true positives) and then by plotting the cumulative curve of these percentages, so that a point of the coordinates $(n, m)$ on the curve means that the $n\%$ of individuals with the highest score concentrate $m\%$ of events. This is the way to build this curve which we makes we speak of predictive performance of "targeted marketing".
	
	Let's see how to build such a curve, always with the same spreadsheet software and always the same data. The start of construction is to take some columns that we had used for the ROC curve (first 31 records out of 136):
	\begin{figure}[H]
		\centering
		\includegraphics[scale=0.8]{img/computing/lift_curve_basis_formula.jpg}
		\caption[]{Formulas of the first columns in Microsoft Excel to get the lift curve}
	\end{figure}
	where the reader may have noticed that we have arbitrarily chosen to set the cutoff at $50\%$ (thus in reality, and the reader must never forget it, the whole lift curve changes for each value of the cut- off !!!).
	
	Which explicitly provides (remember that column \texttt{A}, that as for the ROC curve is trivial, as its simply represent the cumulative effective):
	\begin{figure}[H]
		\centering
		\includegraphics[scale=0.8]{img/computing/lift_curve_values.jpg}
		\caption[]{Explicit value in Microsoft Excel to obtain for the lift curve}
	\end{figure}
	One difference now compared to the ROC curve is that we want the events of interest first (the bad debtors). Then we must sort the column Model (that is to say the: Score) in descending order. This will give:
	\begin{figure}[H]
		\centering
		\includegraphics[scale=0.8]{img/computing/lift_curve_values_sorted.jpg}
		\caption[]{Explicit sorted value in Microsoft Excel to obtain for the lift curve}
	\end{figure}
	Then we add the column of \%Cumulative of true positives:
	\begin{figure}[H]
		\centering
		\includegraphics[scale=0.8]{img/computing/lift_curve_values_sorted_cumulated_formula.jpg}
		\caption[]{Column of \%Cumulatived of true positives for the lift curve}
	\end{figure}
	What gives globally:
	\begin{figure}[H]
		\centering
		\includegraphics[scale=0.7]{img/computing/lift_curve_global_values.jpg}
		\caption{Overview of all constructed columns for the lift curve}
	\end{figure}
	The last step is now simply to build a graph of \%cumulative number of true positive (column G) relative to the \%cumulative of the sample size (column \texttt{A}) to get:
	\begin{figure}[H]
		\centering
		\includegraphics{img/computing/lift_curve.jpg}
		\caption[]{Lift curve (in red at worst, in blue our model)}
	\end{figure}
	So obviously we can not do in this case make an interpretation of the predictive power in terms of "lift" in comparison to another statistical model. By cons we can compare the predictive power of the logistic model used in this case in terms of lift compared to the case considered at worst (the diagonal for reminder ...). So in this case, if we take the $20\%$ of customers (debtors) having the highest score, that's what we have:
	\begin{figure}[H]
		\centering
		\includegraphics{img/computing/lift_curve_completed.jpg}
		\caption[]{Lift Analysis}
	\end{figure}
	So in this particular example, this means that if we focus our analysis / research / target marketing or other... than on the $20\%$ of the sample being the best score (for example ... for cost reasons), we have an overperformance of $37\% / 20\% = 1.85$. Then we say that the model has a lift of $1.85$. The idea then when we compare several models is to keep the one with the greatest lift (leverage).
	
	\subsubsection{General Linear Models (GLM)}\label{general linear models}
	So far we have seen multiple linear regression techniques (simple, multiple linear regression, Gaussian linear model, linearisation of non-linear regressions, logistic regressions, etc.).
	
	There are, however, still other types of regressions named "\NewTerm{general linear models}\index{general linear models}" (not to be confused with "generalized linear models") which are intended for cases where the regression of the variable to be explained (and therefore the residuals) takes values whose range is not necessarily in $]-\infty, +\infty[$ as it is the case for the Gaussian linear model, but only takes integer positive values in $\mathbb{N} $ , or that takes real values in the interval $ [0,1] $, etc.
	
	\begin{tcolorbox}[title=Remark,colframe=black,arc=10pt]
	We will focus here on the essential concepts of the generalized linear regression models because for each of the submodels that we will see below it is possible to find textbooks of 300 to 500 pages which are dedicated to it! In the same way as for classical linear regression (Gaussian model) we one can found textbooks of almost 1,000 pages.
	\end{tcolorbox}
	
	To introduce this type of model, recall what characterizes a general regression model (think by reading the following lines to the particular case of Gaussian linear model as it may help for the understanding):
	\begin{enumerate}
		\item[C1.] The response variable $Y$ is a random component with a probability distribution: Normal (the most frequent case for continuous data), Binomial (positive integer successes in $n$ trials), Poisson (positive integer successes for rare events), Bernoulli (whole positive success with one unique try), Binomial-Negative (positive integer success in a sequence of independent and identically distributed Bernoulli trials before a specified (non-random) number of failures occurs). We note that these are laws belonging to the exponential family (the underlying idea being to linearise it with logarithm) as we will see later.
		
		For example, for the linear Gaussian model seen earlier above, we have shown that for the case with a unique explanatory variable:
		
		was a model with which we relied on the least squares method via the log-likelihood maximum.
		
		\item[C2.] The explanatory variables $ x_1, x_2, \ ldots, x_k $ with the intercept constant used as predictors in the model define as a linear (or non-linear  by multiplying the variables by pairs) the deterministic components.
		
		\item[C3.] The link describing the functional relationship between the linear combination of the explanatory variables and the expectation of the response variable $Y$ is named the "\NewTerm{link function}\index{link function}" or "\NewTerm{canonical parameter}\index{canonical parameter}" and is traditionally denoted $g(\text{E}(Y_i))$.
	\end{enumerate}
	
	In general, when the relation between $\text{E}(Y_i)$ and the explanatory variables is linear we speak of "linear model", when the relation between a function $g(\text{E}(Y_i))$ and the parameters is linear, we speak then of "general linear model".
	
	For example, in the case of the linear Gaussian model with an explanatory variable, we have (it is immediate by the property of the underlying distribution):
	
	Thus, in the case of the Gaussian linear model, the expectation is modelled directly. Thus, the link function for the Normal distribution is the mean itself! We talk then about an "\NewTerm{identity link relation}".
	
	We will obviously not be able to use the linear Gaussian model given the definition domain taken by the variable to be explained when that latter is defined only on integers. On the other hand
we could try an approach by saying that each count:
	
	Follows a Poisson distribution:
	
	of mean (still in the special case with a unique explanatory variable!):
	
	The studied model is then written:
	
	and we deduce from it:
	
	We say that $\log (N_i)$ is an "\NewTerm{offset variable}" (in the industry it is often a value representing the time spent until the event of interest) and that the "link function" is then the logarithm for the Poisson regression. We speak more generally of "\NewTerm{log-linear Poisson model}\index{log-linear Poisson model}" which is associated with a "\NewTerm{logarithmic link}".
	
	Obviously the above approach is very far-fetched but a little further we will see the rigorous approach to achieve get this link function.
	
	Let us recall some important things in this model which are that the hypothesis are similar to the Gaussian linear model: the $Y_i$ are all supposed to follow a Poisson distribution, to have all the same variance and to be independent, and so on...
	
	Note that since we have:
	
	The mean is therefore exponential, influenced by the sign of $\alpha$ (it is logical but it can always be useful to specify it).
	In the general case, we will see that the law of probability density associated with can be written in the general form:
	
	and we say then that the law belongs to the "exponential family" (the Normal, Poisson, Bernoulli and Gamma laws are part of it for example).
	
	\begin{tcolorbox}[title=Remark,colframe=black,arc=10pt]
	Very few specialists designate by the abbreviation "GLZM" the GLM other than those based on the Normal law.
	\end{tcolorbox}
	
	\paragraph{Normal GLM}\mbox{}\\\\
	Let's see this with the Normal law by analogy (the homoscedastic case to simplify...). We therefore start from:
	
	So it comes by identification with:
	
	that:
	
	Note also that (this is very important for what will follow!):
	
	So we fall back on the expected mean and the variance of the Normal law. It would be interesting to check if we have the same kind of property for another law of the exponential family to see if it is generalizable or not ???
	
	Let us also indicate that compared to the famous "canonical parameter" (since mean and variance ultimately depend on it), we have in this case:
	
	so the canonical function is then (without too much surprise in this case...):
	
	that it is customary to write in a relatively general case as we already know in vector notation (see the study of multiple linear regression for the notation):
	
	In this case the canonical function is the unitary application. We speak then of "\NewTerm{canonical identity link}" and it is the only case where the method of ordinary least squares is confused (to understand: is an identical approach) to the use of the GLM.
	
	We note that the assumptions of use of the Normal GLM trivially remain the independence of the explanatory variables, the homoscedasticity, the linearity of the model and the linearity of the expected mean.
	
	\paragraph{Poisson GLM}\mbox{}\\\\
	Let's do the same as before but with the Poisson's law. The idea is therefore that the variable to be explained is discrete and whereas in the same order of idea where with the Normal regression we have:
	\begin{figure}[H]
		\centering
		\includegraphics{img/computing/glm_normal.jpg}
	\end{figure}
	then with the Poisson regression, the idea is to have (notice that as the mean increase, the variance seems also to increase and we will prove this further below even if it's quite obvious regarding to the moment of the Poisson distribution):
	\begin{figure}[H]
		\centering
		\includegraphics{img/computing/glm_poisson.jpg}
	\end{figure}
	However, the reader must be keep in mind that even a continuous variable can have an excellent fit to a Poisson's law, which is why Poisson's regression is also sometimes used for variables to be explained that are continuous (typically for heteroscedastic regressions where weighted regression is to difficult to apply or fails).
	
	We will therefore look for a regression model which, like Gaussian GLM regression, gives the slip of the expected mean of the Gaussian law as a function of explanatory variables but this time with a model that gives the slip of the expected mean of the Poisson's law (yes indeed as a reminder ... the two moments of the Poisson's law are directly connected to each other). We start from:
	
	Last arrangement that we had to guess ... to make an identification with:
	
	Therefore it comes:
	
	Let us indicate for the small parenthesis that some authors use the Euler Gamma function (\SeeChapter{see section Differential and Integral Calculus page \pageref{gamma euler function}}) and write:
	
	Now, let's have a look at what gives the derivatives used above:
	
	So we can once again find the expected mean and variance but this time of the Poisson's law. It would be interesting to check again if we have the same kind of property for another law of the exponential family in order to see if it is generalizable or not ???
	
	So we notice a second time that the parameter $\theta$ controls the mean and the variance through its first and second derivative. The variance is therefore a function of the mean for all the laws of the family of exponential laws.
	
	Let us also indicate that relatively to the famous "canonical parameter" (since mean and variance ultimately depend on it), we have in this case:
	
	The canonical function for the Poisson's law is then:
	
	So the canonical function is the logarithmic application. We then talk about "\NewTerm{canonical logarithmic link}".
	
	Before going further let us compare, the generalized form of the Normal law of the regression with that of Poisson:
	
	\begin{tcolorbox}[title=Remark,colframe=black,arc=10pt]
	We will not go too much further with the Poisson's law for regression because according to the experience feedback of the senior practitioners it seems that it suffers from the major problem that as the expected mean is equal to the variance whereas in reality the variance tends to be higher than the expected mean. The negative binomial model (see below) should be preferred to that of Poisson.
	\end{tcolorbox}
	For the parameters to be estimated, it is customary to also use the maximum log-likelihood as we have done for many previous regression models. Then the likelihood is given in this case by:
	
	The log-likelihood becomes:
	
	Now to proceed as we did with the Gaussian linear regression, let us recall that the $y_i$ are given by the experiment. We will write then:
	
	and that, as for Gaussian linear regression, the explicit expression of the $\mu_i$ must allow us to determine the expression of parameters $\beta_j$ of the regression.
	
	However we have seen earlier above that the link function for the Poisson regression was the natural logarithm which in the general case is written as follows:
	
	 
	So we have so far:
	
	Therefore for the Poisson logarithm link:
	
	Hence the log-likelihood will be written:
	
	To solve this kind of log-likelihood, we then use Newton-Raphson type algorithms (see further below in this section page \pageref{newton raphson method}).
	
	We note that the hypotheses of use of the GLM Poisson remain trivially the independence of the explanatory variables, the homoscedasticity is on the other hand no more assumed since the variance depends on the expected mean, the linearity of the model is always a hypothesis and the expected mean is linear only under a logarithmic transformation.
	
	Finally, let us say that, just as for the Gaussian regression, we can use the result of the Poisson regression to make the probabilistic inference. Indeed, since the density probability function of the Poisson distribution is given by:
	
	and since we have:
	
	 Then:
	
	and in this case you have to change the notation so that it makes sense:
	
	Thus, for any abscissa value $x_i$ of our model (which in the present case is univariate), we can calculate the prior probability of the realization of an event of interest for an ordinate value $y_i \in \mathbb{N}$.
	
	\paragraph{Binomial GLM}\mbox{}\\\\
	Let's do the same with the binomial law. We start from:
	
	Now, let's do the identification with:
	
	Therefore we get (take a particular care at how looks like the first parameter...):
	
	Now, let's have a look at what gives the derivatives used earlier above:
	
	and:
	
	So we can once again fall back on the expected mean and variance but this time the of the binomial law!
	
	Let us also indicate that relatively to the famous "canonical parameter" (since expected mean and variance ultimately depend on it), we have in this case:
	
	So the canonical function is:
	
	in this case the canonical function is named, as we already know, the "\NewTerm{logit link}" or simply the "\NewTerm{logit}" (which we had already discovered otherwise in our study of logistic regression).
	
	Let us compare now the generalized form of the Normal law of the regression with that of Poisson and the Binomial law:
	
	and also the link functions:
	
	We note that the assumptions of use of the Binomial GLM remain trivially the independence of the explanatory variables, the homoscedasticity is on the other hand no more assumed, the linearity of the model is always a hypothesis and the expected mean is linear under a logit transformation .
	
	We will not develop here the likelihood model of the binomial regression as we already did it during our study of the binomial logistic regression on page \pageref{likelihood binomial logistic regression}.
	
	\paragraph{Binomial Negative GLM}\mbox{}\\\\
	Negative binomial regression (sometimes abbreviated in general as N.B.R.M. for Negative Binomial Regression Model) is probably the most widely used in practice outside binomial logistic regression or Gaussian regression. However it suffers from a major problem ... Indeed there are about twenty derived models such as: NB-C (the simplest that we will study here), NB-1, NB-2 (generalization of GLM Poisson), truncated NB / NB-1 / NB-C, NB with zero inflation (ZINB), censored NB, NB with obstacle, NB-P, heterogeneous-NB (NBH), finite mixed model, conditional fixed effects model, random effects model, NB endogenous stratification, NB sampling, NB latent class, NB bivariate, etc. In short enough to write several thousand pages and study for several years ...
	
	Well this being said, let's take a look at the case of canonical negative binomial regression.
	
	We thus start from one of the possible expressions of the density function of the negative binomial law (see the section Statistics page \pageref{negative binomial distribution} for the demonstration of its origin):
	
	We then adopt this writing with to the GLM sauce, admitting that we are interested in failures $E$:
	
	Now, let's do the identification with:
	
	Therefore:
	
	Now, let's have a look at what gives the derivatives used earlier above:
	
	Which corresponds well to the expected mean of the number of failures before the $R$-th success (result to be compared with the expected mean of the number of successes of the negative binomial law as demonstrated in the section Statistics) since it is this that we seek to model by linear regression. Do not forget that we can reverse the roles failures / successes.
	
	And:
	
	Which corresponds well to the variance of the number of failures (result to be compared with the variance of the number of successes of the negative binomial law as demonstrated in the section Statistics) since this is what we seek to model by the regression.
	
	We then have:
	
	Let's now make a usual change of notation:
	
	Which is also sometimes denoted:
	
	In this form, we can rewrite:
	
	If we were more interested in successes rather than failures we would have (this is the case most often chosen in the specialized literature):
	
	Then, it is customary to write this with the Gamma Euler function (\SeeChapter{see section Differential and Integral Calculus page \pageref{gamma euler function}}). What then gives (with different common notations):
	
	
	In this form, we immediately see that expected mean is written then:
	
	Now, to return to our previous development, let us recall that we have proved in the section of Differential and Integral Calculus during our study of the Euler's Gamma function that:
	
	We then have by extension:
	
	He then comes in this case:
	
	Let us integrate with the following variable change:
	
	We then have:
	
	We find under the integral two density functions which are respectively the Poisson density function and the Gamma density function (\SeeChapter{see section Statistics page \pageref{poisson distribution} and page \pageref{gamma distribution}}):
	
	We have a result that it is customary to associate with a generalization of the Poisson's law and which we name "\NewTerm{Poisson-Gamma density law}\index{Poisson-Gamma density law}" or "\NewTerm{mixed Poisson-Gamma law}\index{mixed Poisson-Gamma law}".
	
	
	\begin{tcolorbox}[title=Remark,colframe=black,arc=10pt]
	Thus, the negative binomial distribution can be considered as a generalization of the Poisson distribution. Indeed, the negative binomial law can be seen as a Poisson distribution where the Poisson parameter is itself a random variable, distributed according to a Gamma distribution !!!\\
	
	An important application is in the area of risk insurance where $k$ the number of undesirable events is modelled by a Poisson law of parameter $\lambda$ which itself is therefore tainted by an uncertainty that is characteristic of the risk category of the insured. Thus, the parameter $\lambda$ can itself be seen as a random variable traditionally denoted $\Theta$ in the field of actuarial science and following a gamma law such that the density function of $\Theta$ is then given by (with the traditional notation in actuarial science):
	
	\end{tcolorbox}
	For the parameters to be estimated, it is customary to also use the maximum log-likelihood as we did for all previous regressions. Then, in order to determine the maximum of the log-likelihood, it is traditional in the case of the negative binomial law to start from the following form demonstrated above:
	
	We then have:
	
	The log-likelihood then becomes:
	
	Now to proceed as we did with Gaussian linear regression, let us remember that $y_i$ are given by the experience. We will write then:
	
	and that, as for the Gaussian linear regression, the explicit expression of the $\mu_i$  must allow us to determine the expression of the parameter $\alpha$ of the regression.
	
	Now, let us recall that:
	
	and that:
	
	and let us denoted this $\lambda$. Then we get:
	
	Therefore:
	
	Now let us put that $R \rightarrow +\infty$. It comes then:
	
	since each term of the product tends to $1$ and since there is a constant number of terms in the product (that is, $k$ terms) the whole tends to $1$.
	
	And as we saw in the chapter of Functional Analysis (see page \pageref{natural exponential function}):
	
	Therefore:
	
	We thus fall back on this limit case the Poisson law!
	
	Therefore, as the Poisson law is a special case of the Negative Binomial law, a possible model for the link function of the Negative Binomial Law is to take up the logarithm (hence the Poisson regression link function).
	
	The canonical function is then that of the Poisson function:
	
	From then on, the log-likelihood will be written:
	
	it is customary to name the "\NewTerm{NB2 log-likelihood}\index{NB2 log-likelihood}" (which is the one used in the majority of statistical softwares).
	
	But there is another version of log-likelihood that uses a combination of approximation and exact value. Indeed, remember that we had obtained:
	
	Therefore, using:
	
	and by mixing (rather dubious mix...) with the approximation approach using Poisson link function:
	
	We have (happy mix ...):
	
	The log-likelihood is written then:
	
	that is customary named the "\NewTerm{NB-C log-likelihood}\index{NB-C log-likelihood}".
	
	To solve this kind of log-likelihood, we then use again Newton-Raphson type algorithms (see further below in this section page \pageref{newton raphson method}).
	
	Let us compare, the generalized form of the Normal law of regression with that of Poisson, of the Binomial and Negative Binomial law:
	
	and also the link functions:
	
	
	We note that the hypotheses of use of the Binomial Negative GLM remain trivially the independence of the explanatory variables, the homoscedasticity is on the other hand no more assumed, the linearity of the model is always a hypothesis and the expected mean is linear only under a logarithmic  transformation chosen a little bit empirically in the canonical case...
	
	\paragraph{Gamma GLM}\mbox{}\\\\
	Let's do the same as before but now with the Gamma law (\SeeChapter{see section Statistics page \pageref{gamma distribution}}). We therefore start from:
	
	by just changing the notation a bit to not confuse with one of the GLM parameters:
	
	assuming that the scale parameter $\lambda$ is known (...). Why? Because the expected mean of the Gamma law is for recall given by (see the proof in the section Statistics page \pageref{gamma distribution}):
	
	and therefore controlling only one of the two parameters is enough to scan the entire desired definition domain (yes, simply!).
	
	The use of the Gamma law is justified, besides by its domain of definition which is not in the negatives, by the fact that the relation between its expected mean and its variance gives:
	
	So the relationship between expected mean is the variance is constant regardless of their respective values (so when mean increases, the variance increase too as for the Poisson distribution)! Another way of looking at this is that the variance is proportional to the expected mean since:
	
	The reason why the GLM Gamma is interesting is that we can "adjust" the relationship between variance and expected mean by choosing the value of $\lambda$, which other models seen so far can not do!
	
	Therefore we get:
	
	Now, let us as always the identification with:
	
	We therefore get:
	
	Now, let's have a look at what gives the derivatives used earlier above:
	
	and:
	
	So we can once again fall back on the expected mean and variance but this time of the Gamma law!
	
	Let us also indicate that compared to the famous "canonical parameter" (since expected mean and variance ultimately depend on it), we have in this case:
	
	so the canonical function is:
	
	in this case, the canonical function is named the "\NewTerm{negative inverse link}" and is often denoted in the following form (because finally put that $-\alpha_i=1$ is simply equivalent to change the values of the parameters $\beta_j$):
	
	So in view of this expression we can make regression of very various type of curves.
	
	Let us now compare the generalized form of the Normal law of the regression with that of Poisson and the Binomial law and the Gamma law:
	
	and also the link functions:
	
	We note that the hypotheses of use of the GLM Gamma remain trivially the independence of the explanatory variables, the homoscedasticity is on the other hand no more assumed, the linearity of the model is always a hypothesis and the expected mean is linear only under an inverse transformation.
	
	\begin{tcolorbox}[colframe=black,colback=white,sharp corners]
	\textbf{{\Large \ding{45}}Examples:}\\\\
	E1. Let us now see after all these developments a first practical example using the most important case: the negative binomial regression. So that everyone can practice, we are going to do this regression with the spreadsheet software of Microsoft (a version made with \texttt{R} is also available in the corresponding companion book).\\

	We collected a dataset of days of absence from $316$ young college students (below the reader can see only the 32 first). Our explanatory variables of interest are their average score in foreign languages and their gender (boy or girl: B/G):
	\begin{figure}[H]
		\centering
		\includegraphics{img/computing/raw_data_nb2_glm_microsoft_excel.jpg}
	\end{figure}
	\end{tcolorbox}
	
	\begin{tcolorbox}[colframe=black,colback=white,sharp corners]
	Before going any further, it is first trivial to notice that the days of absence are non-negative with integer values and that the majority are never absent, a Gaussian regression is not the most suitable because its support sweeps away all the positive and negative real numbers (this does not necessarily mean that the result would be bad, however!). In addition, the standard deviation of days of absence is significantly different from the average of these same days of absence, therefore in extenso the Poisson regression is not the most suitable ... (however, this also does not mean necessarily say that the result would be bad!).\\
	
	So here is how to proceed with Microsoft Excel 14.0.7106:\\

	First we prepare an area which contains the $2$ coefficients, the intercept and the parameter $\alpha$ of the negative binomial model (which for reminder is the inverse of the number of successes or respectively of failures and whose value must be at least equal the unit!!!) in an area of the sheet as visible below to which we associate a weak positive value written in hard on the keyboard (therefore no formulas here!):
	\begin{figure}[H]
		\centering
		\includegraphics{img/computing/glm_negative_binomial_initial_values.jpg}
	\end{figure}
	Then, in column \texttt{D}, we will write the relation of the NB-2 log-likelihood  which for reminder is given by:
	
	and since this spreadsheet software does not have Euler's Gamma function built in at the day we writ these lines, we will rewrite it as follows:
	
	Which gives with the Microsoft Excel formula syntax:\\
	\end{tcolorbox}
	
	\begin{tcolorbox}[colframe=black,colback=white,sharp corners]
	\texttt{=LN(LN(FACT(1/\$G\$2+C2-1)))-LN(LN(FACT(1/\$G\$2-1)))
-IFERROR(LN(LN(FACT(C2+1-1)));1)
-\\(C2+1/\$G\$2)*LN(\$G\$2*EXP(\$G\$3+\$G\$4*A2+\$G\$5*B2)+1)\\
+C2*LN(\$G\$2*EXP(\$G\$3+\$G\$4*A2+\$G\$5*B2))}\\

	By writing this formula into column \texttt{D} and dragging it down we get first the following values:
	\begin{figure}[H]
		\centering
		\includegraphics{img/computing/raw_data_poisson_glm_with_log_likelihood_microsoft_excel.jpg}
	\end{figure}
	Now in a cell, we will report the sum of the NB-2 log-likelihood values. In this case we have chosen to put it in \texttt{G7}:
	\end{tcolorbox}
	
	\pagebreak
	\subsubsection{Robust, $M$-estimators and $W$-estimators}\label{m-estimators}
	The $M$-Estimators naturally appear in statistics when we see the problem of the aspect of the variance which is a function of squared deviations from the mean (hence a parabola) that is therefore a little too sensitive to extreme values (this is especially the case of  many linear regression techniques, hence the fact that $M$-estimators are introduced mainly in textbooks related to regression techniques!). The idea then is to ask if it is not possible to construct something other than a square of deviations (hence a more robust statistic!) and then to define the corresponding sample mean, everything being distribution free, that we will name a "\NewTerm{$M$-estimator of location}\footnote{Then by extension $M$-estimators are a broad class of extremum estimators for which the objective function is a sample average!}\index{$M$-Estimator of location}" (the reader should know that there are also $M$-estimators for the sample variance named "\NewTerm{$M$-estimator of scale}" like the $\tau$-estimator!).
	
	We have placed this subject in the section of Theoretical Computing instead of the section of Statistics because for most of these estimators, as we will see it further below, no closed form solution exists and an iterative approach to computation is required\footnote{It is possible to use standard function optimization algorithms, such as Newton–Raphson (solver!)}, the mathematics are quite ugly and most of the time there are different worldwide definitions for the same thing and also computer implementations that differ from their theoretical definition! .

	\begin{tcolorbox}[title=Remark,colframe=black,arc=10pt]
	Peter J. Huber (1964, 1967) introduced $M$-estimators and their asymptotic properties (the proposed estimating the center of symmetry of symmetric distributions). They played an important part of the development of modern robust statistics. Liang and Zeger (1986) helped popularize $M$-estimators in the biostatistics literature under the name "\NewTerm{generalized estimating equations}" (GEE). Since then many other robust estimators has been developed like MM-estimators, $S$-estimators, GM-estimators, $\tau$-estimators, etc. (basically, MM-estimators are $M$-estimators initialized by an $S$-estimator...).
	\end{tcolorbox}

	The best-known example of $M$-estimator is the arithmetic mean $\mu_a$, as we have just mentioned, indeed it is it that trivially maximizes the function of the sum of squares of the deviations as we have seen earlier above:
	
	But some statistician have had the idea (in their purpose to always generalize everything thank to a lot of coffee) to define an $M$-estimator as the expression of the average that maximizes any function $f$ whose argument is by definition $x_i-\theta$ such that:
	
	And for example, in the case of the most familiar variance, this function $f$ is the square of the argument (ie $f(x_i-\theta)=(x_i-\theta)^2$. It is customary, however, in statistics to denote the function $f$ with the letter $\rho$ (so do not confuse with the correlation coefficient or anything of the same kind!). We will continue our explanations by aligning ourselves with this traditional notation!
	
	What ultimately makes us look for a $\theta$ that for a particular function $\rho'$ gives:
	
	where $\psi$ is named the "\NewTerm{influence function}\index{influence function}" (both $\rho$ and $\psi$ should be obviously symmetric!). When the partial derivative exist, we speak of "$\psi$-type $M$-estimator" otherwise of "$\rho$-type $M$-estimator".
	
	\begin{tcolorbox}[title=Remark,colframe=black,arc=10pt]
	If $\rho=f$, the probability density function related to the random variable $X$, then the reader may have noticed that we fall back on the MLE (Maximum Likelihood Estimator) as introduced in the section of Statistics page \pageref{maximum likelihood estimators}!
	\end{tcolorbox}
	We would like any good $M$-estimator of location to be location and scale equivariant, that is, we would like it to respond in a reasonable manner to linear changes across the sample. An $M$-estimator is "location equivariant" if, when every observation is shifted by some amount $a$, the $\theta$ of that shifted sample also shifts by $a$. An $M$-estimator of location is scale equivariant if, when every observation is multiplied by some non-zero constant $b$, the $\theta$ of that altered sample is $b$ times the $\theta$ of the unaltered sample. An $M$-estimator of location that possesses both these characteristics is location-and-scale equivariant. In other words, we want
the following to hold:
	
	In order for an $M$-estimator of location to be location-and-scale-equivariant, it is often necessary to scale the observations when computing $\rho$ and  $\psi$. This is often done whether rescaling is necessary or not because it makes notation simpler.
	
	The matter of location-equivariance is fairly simple! It can be satisfied by making the input of the form $x_i-\theta$. Adjusting for scale is not as straightforward. We must pick some estimator of the scale of the sample, noted (without surprise...) $s_i$, which is a function of the observations $x_1,x_2,\ldots,x_n$. $s$ is scale-equivariant and location invariant, meaning that it is unaffected by a shift in the sample as described above.

	It is then common to define the standardized residual:
	
	where $s_i$ is a strictly positive robust estimator of scale (typically the median absolute deviation but it's not the only choice!) and $c$ is a tuning constant, then we can write obviously:
	
	
	\textbf{Definition (\#\mydef):} Thus, more generally, a "\NewTerm{$M$-estimator of location}" may be defined to the value of a sample mean $\hat{\theta}$ that leads to a zero value of an estimating function given by:
	
	
	Ideally, it is desirable that the function $\rho'$ had as main property of being bounded (which immediately eliminates the arithmetic mean as an ideal $M$-estimator) which avoids typical deviations from the standard deviation (where $\rho(r_i)=r_i^2/2$ and hence $\psi(r_i)=r_i$):
	\begin{figure}[H]
		\centering
		\includegraphics{img/computing/m_estimator_influence_function_L_2.jpg}
		\caption{$L^2$-norm (influence function)}
	\end{figure}
	But we also require at least the four following properties for the $\rho(r_i)$ function:
	\begin{itemize}
		\item[P1.] $\rho(r_i)\geq 0$ for all $r_i$ and has a minimum at $0$
		
		\item[P2.] $\rho(r_i)=\rho(-r_i)$ for all $r_i$
		
		\item[P3.] $\rho(r_i)$ increases as $r_i$ increases from $0$, but doesn't get too large as $r$ increases
		
		\item[P4.] $\rho(r_i)$ should be differentiable
	\end{itemize}
	and the following four properties for the $\psi(r_i)$ function:
	\begin{itemize}
		\item[P1.] Be piece-wise continuous function
		
		\item[P2.] Be an odd function, ie $\psi(-r_i)=-\psi(r_i)$
		
		\item[P3.] $\psi(r_i)\geq 0$ for $r_i\geq 0$ and $\psi(r_i)>0$ for $0<r_i<k$
		
		\item[P4.] Its slope is $1$ at $0$, ie $\psi'(0)=1$
	\end{itemize}
	Note that the last property is not strictly required mathematically, but we use it for standardization in those case where $\psi$ is continuous at $0$. Then if follow from the first property that $\psi(0)=0$, and we require $\psi(0)=0$ also for the case where $\psi$ is discontinuous in $0$, as it is, e.g., for the $M$-estimator defining the median.
	
	Albert E. Beaton and John W. Tukey in their article \cite{beaton1974} also define a new weighted function $\omega$ (without justifying it, but it's probably a mathematical trick to simplify some calculations that we will see further below) such that:
	
	Thus 
	
	What E. Beaton and John. W. Tukey write explicitly in their original article:
	
	We can eliminate the $c\cdot s_i$ of the second fraction if we assume, same like Beaton and Tukey did, that it is equal for all $i$ such that (relation that we use later):
	

	\textbf{Definition (\#\mydef):} The "\NewTerm{breakpoint of an estimator}" is the maximum proportion of observations (hence a number between $0$ and $1$) that can be changed without
changing the estimator. For example the breakpoint of the arithmetic average is $0$, that of the median is (obviously) $0.5$.
	
	\begin{tcolorbox}[title=Remark,colframe=black,arc=10pt]
	Andrews et al. (1972) (the Princeton Robustness Study), at which time it was expected that all statistical analyses would, by default, be robust wrote: \guillemotleft \textit{any author of an applied article who did not use the robust alternative would be asked by the referee for an explanation}\guillemotright  and also that \guillemotleft\textit{From the 1970s to 2000 we would see... extensions to linear models, time series, and multivariate models, and widespread adoption where every statistical package would take the robust method as the default...}\guillemotright\\
	
	Sadly, we see that even in 2018 (the day we write these lines), it's always not the case even if all softwares have largely the possibility to do so using nonparametric or brute force statistical methods (human inertia to change issue...).
	\end{tcolorbox}
	 
	Before introducing a list of some well-known  $\rho$, $\psi$ and $\omega$ functions, let us introduce two one well know robust estimator and one well known $M$-estimators of location (these both being introduced in the form of the objective function introduced just earlier above!).
	
	\begin{enumerate}
		\item The "\NewTerm{trimmed mean}\index{trimmed mean}\label{trimmed mean}", also named "\NewTerm{truncated mean}" or "\NewTerm{pruned mean}", is the robust estimator consisting in the arithmetic mean obtained by removing from the observed values those below and above a given quantile. It is customary to denote $\alpha/2$ the corresponding percentile and that if we have $n$ values then we return the $[\alpha n]$ largest values and the $n-[\alpha n]$ smaller (remember that $[\ldots]$ is the notation for "integer value" as seen on page \pageref{integer part}).
		
		Mathematically, we denote elegantly the pruned average of ordered realizations $x_i$ in descending order to increasing of a random variable $X$ in the following form:
		
		\begin{tcolorbox}[title=Remark,colframe=black,arc=10pt]
		We can found trimmed mean in many spreadsheets softwares for example under the function name \texttt{TRIMMEAN( )}, or in statistical software like \texttt{R} (just by passing an argument to the default \texttt{MEAN( )} function!
		\end{tcolorbox}
		
		It is quite obvious that this robust estimator is an $M$-Estimator as it respects the definition seen above since we are only replacing some values of the original series with other values and that's finally equivalent to an arithmetic mean!
		
		\item The "\NewTerm{Winzorised (Huber) mean}\index{Winzorised (Huber) mean}" is the robust estimator consisting in the arithmetic mean calculated after that the $[\alpha n]$ smallest values have been replaces by the value $x_{[\alpha n]+1}$ and the $[\alpha n]$ biggest values by the value $x_{n-[\alpha n]}$.
		
		Mathematically we denote elegantly the Winsorized mean of the ordered realizations $x_i$ in descending order to increasing of a random variable $X$ in the following way (discrete version):
		
		Again it is obvious that this robust estimator is an $M$-Estimator as it respects the definition seen above since we are only replacing some values of the original series with other values and that's finally equivalent to an arithmetic mean!
		
		\begin{tcolorbox}[title=Remark,colframe=black,arc=10pt]
		We can found the Huber's mean in software like \texttt{R} using the function \texttt{HUBER( )} of the MASS package
		\end{tcolorbox}
		
		\item The "\NewTerm{median}\index{median}", that we already well know, that satisfies (discrete version under the form of an $M$-estimator):
		
		where:
		
		is for recall named the "\NewTerm{signum function}\index{signum function}\label{signum function}" or simply "\NewTerm{sign function}\index{sign function}".
		
		\begin{figure}[H]
			\centering
			\includegraphics{img/computing/m_estimator_median.jpg}
			\caption{Median (influence function)}
		\end{figure}
	\end{enumerate}
	
	\begin{tcolorbox}[colframe=black,colback=white,sharp corners]
	\textbf{{\Large \ding{45}}Example:}\\\\
	Consider the values $2,4,5,10,200$. We then have first:
	
	We then have for the $20\%$ trimmed mean:
	
	and for the $20\%$ Winsorized mean:
	
	\end{tcolorbox}
	
	\begin{tcolorbox}[title=Remark,colframe=black,arc=10pt]
	The mean $\mu_a$, the trimmed mean $\mu_{T_\alpha}$, the Winsorized mean $\mu_W$ and the median $M_e$ are particular cases of "\NewTerm{$L$-statistic}\index{$L$-statistic}" (linear combination of order statistics):
	
	For example:
	
	lead to the mean $\mu_a$:
	
	leads to the median $M_e$.
	\end{tcolorbox}
	
	Let's look at some of $\rho$, $\psi$ and $\omega$ functions that build some common $M$-estimators:

	\begin{table}[H]
		\centering
		\begin{tabular}{|c|c|c|c|}
		\hline
		\rowcolor[HTML]{C0C0C0} 
		\multicolumn{1}{|c|}{\cellcolor[HTML]{C0C0C0}\textbf{Type}} & \multicolumn{1}{c|}{\cellcolor[HTML]{C0C0C0}\pmb{${\rho(r_i)}$}} & \multicolumn{1}{c|}{\cellcolor[HTML]{C0C0C0}\pmb{${\psi(r_i)}$}} & \multicolumn{1}{c|}{\cellcolor[HTML]{C0C0C0}\pmb{${\omega(r_i)}$}} \\ \hline
		$L_2$ (least-squares) & $r_i^2/2$ & $r_i$ & $1$ \\ \hline
		$L_1$ (least-absolute) & $|r_i|$ & $\mathrm{sgn}(r_i)$ & $\dfrac{1}{|r_i|}$ \\ \hline
		$L_1-L_2$ & $2\left(\sqrt{1+r_i^2/2}-1\right)$ & $\dfrac{r_i}{\sqrt{1+r_i^2/2}}$ & $\dfrac{1}{\sqrt{1+r_i^2/2}}$ \\ \hline
		$L_p$ & $\dfrac{|r_i|^{\nu}}{\nu}$ & $\mathrm{sgn}(r_i)|r_i|^{\nu-1}$ & $|r_i|^{\nu-2}$ \\ \hline
		"Fair" & $c^{2}\left[\dfrac{|r_i|}{c}-\log \left(1+\dfrac{|r_i|}{c}\right)\right]$ & $\dfrac{r_i}{1+|r_i| / c}$ & $\dfrac{1}{1+|r_i| / c}$ \\ \hline
		Huber $\left\{\begin{array}{l}{\text { if }|r_i| \leq k} \\ {\text { if }|r_i| \geq k}\end{array}\right.$ & $\left\{\begin{array}{l}{r_i^{2} / 2} \\ {k(|r_i|-k / 2)}\end{array}\right.$ & $\left\{\begin{array}{l}{r_i} \\ {k \mathrm{sgn}(r_i)}\end{array}\right.$ & $\left\{\begin{array}{l}{1} \\ {k /|r_i|}\end{array}\right.$ \\ \hline
		Cauchy & $\dfrac{c^{2}}{2} \log \left(1+(r_i / c)^{2}\right)$ & $\dfrac{r_i}{\left(1+r_i^{2}\right)^{2}}$ & $\dfrac{1}{\left(1+r_i^{2}\right)^{2}}$ \\ \hline
		Geman-MacClure & $\dfrac{r_i^{2} / 2}{1+r_i^{2}}$ & $\dfrac{r_i}{\left(1+r_i^{2}\right)^{2}}$ & $\dfrac{1}{\left(1+r_i^{2}\right)^{2}}$ \\ \hline
		Welsch & $\dfrac{c^{2}}{2}\left[1-\exp \left(-(r_i / c)^{2}\right)\right]$ & $r_i \exp \left(-(r_i / c)^{2}\right)$ & $\left.\exp \left(-(r_i / c)^{2}\right)\right)$ \\ \hline
		\multicolumn{1}{|l|}{Tukey $\left\{\begin{array}{l}{\text { if }|r_i| \leq c} \\ {\text { if }|r_i|>c}\end{array}\right.$} & \multicolumn{1}{l|}{$\left\{\begin{array}{l}{\dfrac{c^{2}}{6}\left(1-\left[1-(r_i / c)^{2}\right]^{3}\right)} \\ {\left(c^{2} / 6\right)}\end{array}\right.$} & \multicolumn{1}{l|}{$\left\{\begin{array}{l}{r_i\left[1-(r_i / c)^{2}\right]^{2}} \\ {0}\end{array}\right.$} & \multicolumn{1}{l|}{$\left\{\begin{array}{l}{\left[1-(r_i / c)^{2}\right]^{2}} \\ {0}\end{array}\right.$} \\ \hline
		\end{tabular}
		\caption{Table of a few common dispersion $M$-estimators}
	\end{table}
	As far as I know (as redactor of this topic on $M$-estimators), they are all empirical and there is no "beautiful" way to derive them from a logical process. 
	
	Here is a graphical representation of the different functions of the above table:
	\begin{figure}[H]
		\centering
		\includegraphics[width=1.0\textwidth]{img/computing/m_estimators.jpg}
		\caption{Graphic representation of a few common dispersion $M$-estimators}
	\end{figure}
	Briefly we give a few description of these functions:
	\begin{itemize}
		\item The "\NewTerm{$L_2$ (least-squares) estimators}", associated to the classic well know Pearson variance, are not robust because their influence function is not bounded.
		
		\item The "\NewTerm{$L_1$ (least-absolute) estimators}", associated to the mean absolute deviation, are not stable because the $\rho$-function $|r_i|$ is not strictly convex in $r_i$. Indeed, the second derivative at $r_i=0$ is unbounded, and an indeterminate solution may result. This estimator in comparison of the $L_2$ reduce the influence of large errors, however they still have an influence because the influence function has no cutoff point.
		
		\item The "\NewTerm{$L_1-L_2$ estimators}" take both the advantage of the $L_1$ estimators to reduce the influence of large errors and that of $L_2$ estimators to be convex.
		
		\item The "\NewTerm{$L_p$ (least-power) estimators}" represents a family of functions. It is $L_2$ with $p=2$ and $L_1$ with $p=1$. It appears that $p$ must be fairly moderate to provide a relatively robust estimator or, in other words, to provide an estimator scarcely perturbed by outlying data. The selection of an optimal $p$ seems to have been investigated, and for $p$ around $1.2$, a good estimate may be expected. However, many difficulties are encountered in the computation when parameter $p$ is in the rage of $1<p<2$, because zero residuals are troublesome.
		
		\item The "\NewTerm{fair estimators}" have everywhere defined continuous derivatives of first three orders, and yields a unique solution. The $95\%$ asymptotic efficiency on the standard Normal distribution is obtained with the tuning constant $c=1.3998$.
		
		\item The "\NewTerm{Huber's estimators}" are parabola in the vicinity of zero, and increases linearly at a given level $|r_i|>k$. The idea is to penalize small residuals quadratically, and large residuals linearly. The $95\%$ asymptotic efficiency on the standard Normal distribution is obtained with the tuning constant $k=1.345$. These estimators are so satisfactory that it has been recommended for almost all situations; very rarely they have been found to be inferior to some other $\rho$-function. However, from time to time, difficulties are encountered, which may be due to the lack of stability in the gradient values of the $\rho$-function of discontinuous second derivative:
		
		
		\item The "\NewTerm{Cauchy's estimators}", also known as the "\NewTerm{Lorentzian estimators}", do not guarantee a unique solution. With a descending first derivative, such a function has a tendency to yield erroneous solutions in a way which cannot be observed. The $95\%$ asymptotic efficiency on the standard Normal distribution is obtained with the tuning constant $c=2.3849$.
		
		\item The other remaining functions have the same problem as the Cauchy function. As can be seen from the influence function, the influence of large errors only decreases linearly with their size. The "\NewTerm{Geman-McClure estimators} and "\NewTerm{Welsch estimators}" try to further reduce the effect of large errors, and the "\NewTerm{Tukey's biweight estimators}" even suppress the outliers. The $95\%$ asymptotic efficiency on the standard Normal distribution of the Tukey's biweight function is obtained with the tuning constant $c=4.6851$; that of the Welsch function, with $c=2.9846$.
	\end{itemize}
	There still exist many other $\rho$-functions, such as Andrew's cosine wave function, the GGW (Generalized Gauss-Weight), the LQQ (Linear Quadratic Quadratic), the "Optimal", etc.
	
	\textbf{Definition (\#\mydef):} "\NewTerm{Redescending $M$-estimators}" are $\psi$-type $M$-estimators which have $\psi$ functions that are non-decreasing near the origin, but decreasing toward $0$ far from the origin. Their $\psi$ functions can be chosen to redescend smoothly to zero, so that they usually satisfy $\psi(r_i) = 0$ for all $r_i$ with $|r_i| > k$, where $k$ is referred to as the "\NewTerm{minimum rejection point}". Due to these properties of the $\psi$ function, these kinds of estimators are very efficient, have a high breakdown point and, unlike other outlier rejection techniques, they do not suffer from a masking effect. They are efficient because they completely reject gross outliers, and do not completely ignore moderately large outliers (like median).
	
	Regarding to the previous figure, the Cauchy, Geman-McClure, Welsch, Tukey but also Andrew $M$-estimators are Redescending $M$-estimators!

	Let us see now practically how explicitly are computationally implemented two of these estimator typically with a language like \texttt{R} as it not really obvious (notice the iteration in the Huber estimator, it is the iterative reweighted least squares (IRLS)!):
	\begin{figure}[H]
		\centering
		\includegraphics[width=0.8\textwidth]{img/computing/m_estimators_r_script.jpg}
		\caption[Implementation of Tukey's biweight and Huber M-estimator in R]{Implementation of Tukey's biweight and Huber M-estimator in \texttt{R} (source: \texttt{R} affy package)}
	\end{figure}	
	As the reader may have noticed by reading these script, we are very far from what we have theoretically introduced earlier before! So to understand why, let us focus for example on Tukey's biweight, and the reader will see that it will lead us to the concept of "$W$-estimators"!
	
	For this, the reader let us recall that Beaton and Tukey had introduced in their article:
	
	We can extract $\theta$ quite easily :
	
	In the case of Tukey's biweight we have then (this corresponds to the \texttt{t.bi.mu} in the script above):
	
	As there is a circular reference between the $\theta$ on the left and the $\theta$ on the right, the idea of some Tukey's biweight computer implementation is to take an estimate of the $\theta$ on the right (same idea to estimate $S$). Then we can see for example that Tukey's biweight $M$-estimator of location is computed in the affy package of the \texttt{R} software as:
	
	where we put $w_i=0$ if:
	
	Such an approach (one-shot iteration using a weight function trick) to calculate an $M$-estimator should, according to Beaton and Tukey article, be named a "\NewTerm{$W$-estimator}". But we can iterate the process by writing:
	
	where the weights for a particular $\theta^{[k]}$ depend upon the previous values of $\theta^{[k-1]}$. We simply start with a reasonable guess of the values of $\theta^{[0]}$ (as we did just above) and also especially of the first initial estimate $\theta^{[0]}$, and iterate until the estimates have converged to within whatever we consider to be a reasonable margin of accuracy, that is, until $(\theta^{[k+1]}-\theta^{[k]})^2$ is less than some predicted values. This procedure is an example of the process of iteratively reweighted least squares (IRLS) that Beaton and Tukey in their article recommend to name $W^k$-estimators depending on the number $[k]$ of iterations...
	
	\begin{tcolorbox}[title=Remark,colframe=black,arc=10pt]
	IRLS can have some negative effects. Indeed, when IRLS is employed, more and more points can be eliminated from the data set with each iteration. At each step, the most extreme outliers are dropped from the sample because they are outside of a certain main body of the data. Left unchecked, IRLS can sometimes give zero weight to every point and whittle the data set away to nothing, so IRLS estimators need constraints applied to them that will counteract the effect of the redescending weight function. Such constrained estimators are named "\NewTerm{$S$-estimators}".
	\end{tcolorbox}
	
	So far, the ready should have a better understanding why $M$-estimators are more a subject related to "statistical engineering" rather than "pure Statistics": different implementations and also even sometimes different definitions, quite arbitrary parameters, etc...!!!!
	\begin{tcolorbox}[title=Remark,colframe=black,arc=10pt]
	It can be proved that like the trimmed mean, $M$-estimators, $\tau$-estimator and MM estimator are unbiased and asymptotically Normal at symmetric distributions (useful property to calculate their confidence intervals!).
	\end{tcolorbox}
	
	\pagebreak
	\subsection{Interpolation Techniques}
	The main difference between interpolation and regression, is the definition of the problem they solve.

	Given $n$ data points, when you interpolate, you look for a function that is of some predefined form that has the values in that points exactly as specified. That means given pairs $(x_i,y_i)$ you look for $f$ of some predefined form that satisfies $f(x_i)=y_i$. Most commonly $f$ is chosen to be polynomial, spline (low degree polynomials on intervals between given points).
	
	When you do regression, you look for a function that minimizes some cost, usually sum of square of errors, You don't require the function to have the exact values at given point, you just want a good approximation. In general, your found function $f$ might not satisfy $f(x_i)=y_i$ for any data point, but the cost function, i.e. $\displaystyle\sum_{i=1}^n(f(x_i)-y_i)^2$ will be the smallest possible of all the functions of given form.
	
	There are numerous polynomials interpolation techniques more or less complex and sophisticated . We propose here to present some in ascending order of difficulty.
	
	\subsubsection{Bezier Curves (B-Splines)}
	The Russian engineer Pierre Bezier, at the beginnings of the Computer Aided Design (CAD), in the 60s, gave a way to define curves and surfaces from points. This allows direct manipulation, geometric, of curves without having to give a series of equation to the machine to describe a complex curve!!
	
	The application field of Bezier curves is a multifaceted to, very rich, at the crossroads of many diverse mathematical areas: Analysis, Cinematic, Differential Geometry,  Affine Geometry, Projective Geometry, Fractal Geometry, Probabilities, Statistics, Finance (rates curve), etc.
	
	The Bezier are also become essential in their concrete applications in industry, and computer graphics. Most non-engineer and non-scientific people know them through the usage of some drawing tools included in Adobe Illustrator, Adobe Photoshop, 3D Studio Max, Blender, Rhino, Adobe InDesign, Microsoft Office Visio and even Microsoft Office Word and Microsoft Office PowerPoint.
	
	Let us present and study in detail the mathematical approach of this technique:
	
	First, we know that the equation of a segment line which we will denote in this field of study $M$ (with respect for the tradition) joining two points $A(x_1,y_1),B(x_2,y_2)$ is:
	
	This is right since when $t=0$ we are on $A$ and when $t=1$ we are on $B$. Therefore $t\in[0,1]$ and the point $M$ describes the whole straight segment $[AB]$.
	
	By definition, the straight segment denoted $[AB]$ is a "\NewTerm{Bezier curve of degree $1$}\index{Bezier curve}" with control points $A$ and $B$ and the polynomials on $t$ or on $t-1$ are the "\NewTerm{Bernstein polynomials of degree $1$}\index{Bernstein polynomials}\label{bernstein polynomial}".
	
	Let us now build a parametric curve by adding a second step to what we just see:
	\begin{figure}[H]
		\centering
		\includegraphics{img/computing/b_splines_three_points.jpg}
		\caption{Idea behind 2nd order Bézier curves}
	\end{figure}
	\begin{enumerate}
		\item 1st step:
		\begin{enumerate}
			\item Given $M_1(t)$ the barycenter of $(A,1-t)$ and $(B,t)$ and where $M_1(t)$ describes the segment $[AB]$.
			\item  Given $M_2(t)$ the barycenter of $(B,1-t)$ and $(C,t)$ and where $M_2(t)$ describes the segment $[BC]$.
		\end{enumerate}
		\item 2nd step:
		
		Given $M(t)$ the barycenter of $(M_1(t),1-t),(M_2(t),t)$.\\
			
		By construction $M (t)$ is therefore at the same proportion of the segment $[M_1(t),M_2(t)]$ relative to the segment $[AB]$ or to $M_2(t)$ compared to the segment $[BC]$.\\
			
		The curve obtained is then the envelope of the segments $[M_1(t),M_2(t)]$: at any point $M(t)$, the tangent to the curve is therefore the segment $[M_1(t),M_2(t)]$.
	\end{enumerate}
	$M (t)$ then describes a Bézier curve of order two, which, by construction starts at $A$ and ends at $C$, and has for tangents $[AB]$ on $A$ and $[CB]$ on $C$.
	
	This is in fact a parabolic arc (that we denote logically by $[ABC]$ in this field of study):
	\begin{figure}[H]
		\centering
		\includegraphics{img/computing/b_splines_three_points_result.jpg}
		\caption[]{Result of 2nd order Bézier curve}
	\end{figure}
	By the same pattern, we can define a Bézier curve of $n$ points $P_i$ with $i=1...n$. This is what we name the "\NewTerm{Casteljau algorithm}\index{Casteljau algorithm}". Thus, given:
	
	We get:
	
	The recurrence ending for:
	
	Thus, for $n=2$ we have:
	
	Therefore:
	
	Thus, we have necessarily with two points the equation of a straight line.
	
	Now consider $M_3(t)$ the Bézier curve of order $3$, therefore we have the points always defined by:
	
	We then have by the recurrence relation:
	
	where we have eliminated the terms containing unspecified points.
	
	We have therefore:
	
	Therefore it comes:
	
	or in vector form (more consistent with the usual mathematical notation and therefore we also better understand the meaning of "\NewTerm{vectorial drawing}\index{vectorial drawing}"):
	
	and in matrix form:
	
	By the same reasoning, we get for a Bézier curve of order $4$:
	
	or in vector form:
	
	Which corresponds generically to:
	\begin{figure}[H]
		\centering
		\includegraphics{img/computing/b_splines_order_4.jpg}
		\caption{Example of Bézier curve of order $4$}
	\end{figure}
	\pagebreak
	\begin{tcolorbox}[colframe=black,colback=white,sharp corners]
	\textbf{{\Large \ding{45}}Example:}\\\\
	Let us plot now a Bézier curve of order $4$ with Maple 4.00b. For this purpose we will use the following commands:\\
	
	\texttt{>restart:with(plots):\\
	>x[0]:=1: y[0]:=4:\\
	>x[1]:=6: y[1]:=6:\\
	>x[2]:=1: y[2]:=2:\\
	>x[3]:=8: y[3]:=2.5:\\
	>f:=t->x[0]*(1-t)\string^3+3*x[1]*t*(1-t)\string^2+3*x[2]*t\string^2*(1-t)+x[3]*t\string^3;\\
	>g:=t->y[0]*(1-t)\string^3+3*y[1 ]*t*(1-t)\string^2+3*y[2]*t\string^2*(1-t)+y[3]*t\string^3\\
	>G:=plot([f(t),g(t),t=0..1],thickness=2):\\
	>t0:=textplot([x[0],y[0],`P[0]`],align=ABOVE):\\	
	>t1:=textplot([x[1],y[1],`P[1]`],align=RIGHT):\\	
	>t2:=textplot([x[2],y[2],`P[2]`],align=RIGHT):\\
	>t3:=textplot([x[3],y[3],`P[3]`],align=ABOVE):\\
	>Or:=textplot([0,0,`Origin`],align=ABOVE):\\
	>pp:=pointplot({[x[0],y[0]],[x[1],y[1]],[x[2],y[2]],[x[3],y[3]]},\\
	symbol=circle,color=navy):\\
	>display(G,t0,t1,t2,t3,pp,Or);\\}
	
	This will give:
	\begin{figure}[H]
		\centering
		\includegraphics{img/computing/b_splines_order_4_maple.jpg}
	\end{figure}
	\end{tcolorbox}
	Now let us take again the previous Bezier curve with a different notation:
	
	We first notice easily the following proportionality:
	
	and if we look more closely at the coefficients, we note that we have also:
	
	It is neither more nor less than the Pascal's triangle !! So the coefficients are simply the binomial coefficients (\SeeChapter{see section Calculus page \pageref{binomial theorem}}) given for the order $n$ in our example by:
	
	Thus, "\NewTerm{Bernstein polynomials}\index{Bernstein polynomials}" are defined by:
	
	and finally the Bernstein curves of order $n$ are given by:
	
	In fact, if we have noted previously the sum as follows:
	
	We would then have the Bernstein polynomials that are given (which is more respectful of the traditions ...) by:
	
	This is a very practical relation because it allows easily and quickly to calculate the polynomial corresponding to a sequence of Bezier curve of order $n$.
	
	Then we have:
	
	\begin{tcolorbox}[title=Remark,colframe=black,arc=10pt]
	A Bezier curve is completely changed as soon as at least on point is moved. Then we say that the method of Bezier is a "global method".
	\end{tcolorbox}	
	A well-known example of order Bézier curves of order $3$ is the Pen tool of Adobe Photoshop or Adobe Illustrator softwares. Indeed, these tools create a series of Bézier curves of order $3$ whose point $P_2$ is set afterwards with the mouse using handles named "Handles" in Adobe practitioners language... Here's an example taken from one of these programs done with a plot with the Pen tool with $5$ points (thus $4$ splines):
	\begin{figure}[H]
		\centering
		\includegraphics{img/computing/adobe_spline.jpg}
		\caption{Examples of splines with a drawing program}
	\end{figure}
	As the user does not move the points handles all points are aligned on the straight line. We then feel like to have a spline of order $2$.
	
	\begin{tcolorbox}[colframe=black,colback=white,sharp corners]
	\textbf{{\Large \ding{45}}Example:}\\\\
	A circle, drawn by a professional drawing software is in practice composed of 4 Bezier arcs. To observe this particularity, simply draw a circle with Adobe Illustrator for example, and then select it to reveal the Bezier control points arcs that defines it.\\
	
	We'll look at the best way to choose the control points of these arcs so that they look like circle quadrants, and then we will see the difference between the drawing produced (vector circle) and true (bitmap) circle:
	\begin{figure}[H]
		\centering
		\includegraphics{img/computing/spline_vector_circle.jpg}
		\caption{Example of constructing a circle with Bézier curves}
	\end{figure}
	Let us take the first quadrant or radius $1$ centered at the origin:
	
	\end{tcolorbox}
	
	\pagebreak
	\begin{tcolorbox}[colframe=black,colback=white,sharp corners]
	\begin{figure}[H]
		\centering
		\includegraphics{img/computing/spline_vector_circle_first_quadrant.jpg}
	\end{figure}
	It is approached by a Bézier arc whose control points are $P_1,P_2,P_3,P_4$. The ends of the Bézier arc being $P_1$ and $P_4$, it is natural to choose $P_1(1,0)$ and $P_4(0,1)$. Intuition leads us to choose $P_2(1,k)$ and $P_3(k,1)$ and it remains to find a positive value of $k$ so that the Bezier curve looks like a circular arc.\\
	
	We thus obtain the parametric equation of the Bézier curve:
	
	Therefore:
	
	We can look for example the value of $k$ for which the arc passes through the point:
	
	in $t=0.5$. It then becomes very simple from the parametric equation to determine $k$. It is simply for $x$ (or $y$) of a simple equation of one unknown.
	\end{tcolorbox} 
	\begin{tcolorbox}[title=Remark,colframe=black,arc=10pt]
	Keep in mind, as the vocabulary may be confusing, that:
	\begin{itemize}
		\item If there is only one (polynomial) segment, the spline is often named a "Bézier curve".
	
		\item If each segment is expressed in Bézier form (using Bernstein basis functions), then you might say that the spline is a "Bézier spline".
	
		\item If each polynomial segment has degree $3$, the spline is called a cubic spline.
	
		\item If each segment is described by its ending positions and derivatives, it is said to be in "Hermite form".
	
		\item The B-spline approach gives a way of ensuring continuity between segments. In fact, we show that every spline can be represented in B-spline form. So, in that sense, every spline is a B-spline.
	\end{itemize}
	\end{tcolorbox}	
	
	\subsubsection{Linear ordering isotonic regression}\index{linear ordering isotonic regression}
	"\NewTerm{Linear ordering isotonic regression}\index{isotonic regression}\label{isotonic regression}" can be understood as approximating given series of $1$-dimensional observations with non-decreasing function. It is similar to inexact smoothing splines, with the difference that we use monotonicity, rather than smoothness, to remove noise from the data.
	
	Given values $\vec{a}\in\mathbb{R}^n$ and weights $\vec{w}\in \mathbb{R}^n$ such that $w_i>0$ for all $i$, we want to approximate them by $y_1, \ldots y_n$ as close as possible, subject to a set of constraints of kind $y_i\geq y_j$, ie we want to minimize (obviously sometimes other metrics are used instead of the Euclidean metric!):
	
	with respect to $\vec{y}\in\mathbb{R}^n$ subject to $y_1 \le y_2\le\;\ldots\;\le y_n$.
	
	If all weights equal to $1$, the problem is named "\NewTerm{unweighted isotonic regression}", otherwise it is called "\NewTerm{weighted isotonic regression}".
	
	In pseudo-code (non-unique and very likely not optimized):\\\\	
	\begin{algorithm}[H]
	\KwData{$\vec{a}$,$\vec{w}$}
	\KwResult{$\vec{y}$}
	initialization\;
	$a'_1 := a_1$\;
	$w'_1 := w_1$\;
	$j:=1$\;
	$S_0:=0$\;
	$S_1:=1$\;
	\For{$i=2,3,\ldots ,n$}{
        $j:=j+1$ \;
        $a'_j:= a_i$ \;
		$w'_j := w_i$\;
		\While{$j>1$  AND  $a'_j < a'_{j-1}$}{
			$a'_{j-1} =\displaystyle {w'_j a'_j + w'_{j-1} a'_{j-1} \over w'_j + w'_{j-1}}$\;
			$w'_{j-1} = w'_j + w'_{j-1}$\;
			$j := j-1$\;
		}
		$S_j := i$\;
	}
	\For{$k=1,2,\ldots,j$}{
		\For{$l=S_{k-1}+1,\ldots,S_k$}{
		$y_l=a'_k$\;
		}
	}
	\caption{Pool Adjacent Violators Algorithm (PAVA) pseudo-code algorithm}
	\end{algorithm}
	Here $S$ defines to which old points each new point corresponds.

	Let us see a very detailed companion example (thanks to Wojciech Kotłowski for having authorized us to reproduce it!).
	
	Let us consider:	
	\begin{table}[H]
	\centering	
	\begin{tabular}{c|ccccccccccccc}
    \toprule
  $x$ & $7$ & $-1$ & $-2$ & $9$ & $2$ & $0$ & $6$ & $3$ & $-3$ & $5$ & $-3$ & $7$ & $-5$  \\
  $y$ & $1$ & $0.4$ & $0.2$ & $0.7$ & $0.7$ & $0.6$ & $0.8$ & $0.2$ & $0.3$ & $0.6$ & $0.4$ & $1$ & $0$ \\
   \bottomrule
 	\end{tabular}
 	\end{table}
 	
 	\begin{figure}[H]
		\centering
		\includegraphics[width=1.0\textwidth]{img/computing/isotonic_regression_example_initial.jpg}
	\end{figure} 
	Step 1: We sort the data in the increasing order of $x$.
	\begin{table}[H]
	\centering
	\begin{tabular}{c|ccccccccccccc}
	    \toprule
	  $x$ & $7$ & $-1$ & $-2$ & $9$ & $2$ & $0$ & $6$ & $3$ & $-3$ & $5$ & $-3$ & $7$ & $-5$  \\
	  $y$ & $1$ & $0.4$ & $0.2$ & $0.7$ & $0.7$ & $0.6$ & $0.8$ & $0.2$ & $0.3$ & $0.6$ & $0.4$ & $1$ & $0$ \\
	   \bottomrule
	 \end{tabular}
	\end{table}
	\vspace*{-1cm}	
	 $$\Downarrow \qquad \Downarrow \qquad \Downarrow$$	 
	\begin{table}[H]
	\centering
	 \begin{tabular}{c|ccccccccccccc}
	    \toprule
	  $x$ & $-5$ & $-3$ & $-3$ & $-2$ & $-1$ & $0$ & $2$ & $3$ & $5$ & $6$ & $7$ & $7$ & $9$ \\
	  $y$ & $0$ & $0.4$ & $0.3$ & $0.2$ & $0.4$ & $0.6$ & $0.7$ & $0.2$ & $0.6$ & $0.8$ & $1$ & $1$ & $0.7$ \\
	   \bottomrule
	 \end{tabular}
	\end{table}
	
	Step 2: We split the data into blocks $B_1,\ldots,B_r$, such that points  with the same $x_t$ fall into the same block.
  
  We assign value $f_i$ to each block ($i=1,\ldots,r$) which is the average of labels in this block.
  
	\begin{table}[H]
	\centering
	\begin{tabular}{c|ccccccccccccc}
	    \toprule
	  $x$ & $-5$ & \color{blue}{$-3$} & \color{blue}{$-3$} & $-2$ & $-1$ & $0$ & $2$ & $3$ & $5$ & $6$ & \color{red}{$7$} & \color{red}{$7$} & $9$ \\
	  $y$ & $0$ & \color{blue}{$0.4$} & \color{blue}{$0.3$} & $0.2$ & $0.4$ & $0.6$ & $0.7$ & $0.2$ & $0.6$ & $0.8$ & \color{red}{$1$} & \color{red}{$1$} & $0.7$ \\
	   \bottomrule
	 \end{tabular}
	\end{table}

	\vspace*{-0.5cm}

  $$\Downarrow \qquad \Downarrow \qquad \Downarrow$$

	\begin{table}[H]
	\centering
	{\setlength{\tabcolsep}{0.4em}
	\begin{tabular}{c|ccccccccccc}
	\toprule
	block & $B_1$ & \color{blue}{$B_2$} & $B_3$ & $B_4$ & $B_5$ & $B_6$ & $B_7$ & $B_8$ & $B_9$ & \color{red}{$B_{10}$} & $B_{11}$   \\[1mm]
	  data & $\{1\}$ & \color{blue}{$\{2,3\}$} & $\{4\}$ & $\{5\}$ & $\{6\}$ & $\{7\}$ & $\{8\}$ & $\{9\}$ & $\{10\}$ & \color{red}{$\{11,12\}$} & $\{13\}$ \\[1mm]
	  $f_i$ & $0$ & \color{blue}{$0.35$} & $0.2$ & $0.4$ & $0.6$ & $0.7$ & $0.2$ & $0.6$ & $0.8$ & \color{red}{$1$} & $0.7$ \\
	\bottomrule
	\end{tabular}}
	\end{table}
	
	Step 3: While there exists a "violator", i.e. a pair of blocks $B_i, B_{i+1}$ such that $f_i > f_{i+1}$, we merge $B_i$ and $B_{i+1}$ and assign a weighted average:
 	
 	
 	\begin{table}[H]
 	 \centering
  {\setlength{\tabcolsep}{0.4em}
 \begin{tabular}{c|ccccccccccc}
    \toprule
  block & $B_1$ & \color{red}{$B_2$} & \color{red}{$B_3$} & $B_4$ & $B_5$ & $B_6$ & $B_7$ & $B_8$ & $B_9$ & $B_{10}$ & $B_{11}$  \\[1mm]
  data & $\{1\}$ & \color{red}{$\{2,3\}$} & \color{red}{$\{4\}$} & $\{5\}$ & $\{6\}$ & $\{7\}$ & $\{8\}$ & $\{9\}$ & $\{10\}$ & $\{11,12\}$ & $\{13\}$ \\[1mm]
  $f_i$ & $0$ & \color{red}{$0.35$} & \color{red}{$0.2$} & $0.4$ & $0.6$ & $0.7$ & $0.2$ & $0.6$ & $0.8$ & $1$ & $0.7$ \\
   \bottomrule
 \end{tabular}}
\end{table}

\vspace*{-1cm}

  $$\Downarrow \qquad \Downarrow \qquad \Downarrow$$

  \begin{table}[H]
  \centering
  {\setlength{\tabcolsep}{0.4em}
 \begin{tabular}{c|cccccccccc}
    \toprule
  block & $B_1$ & \color{red}{$B_2$} & $B_3$ & $B_4$ & $B_5$ & $B_6$ & $B_7$ & $B_8$ & $B_9$ & $B_{10}$ \\[1mm]
  data & $\{1\}$ & \color{red}{$\{2,3,4\}$} & $\{5\}$ & $\{6\}$ & $\{7\}$ & $\{8\}$ & $\{9\}$ & $\{10\}$ & $\{11,12\}$ & $\{13\}$ \\[1mm]
  $f_i$ & $0$ & \color{red}{$0.3$} & $0.4$ & $0.6$ & $0.7$ & $0.2$ & $0.6$ & $0.8$ & $1$ & $0.7$ \\
   \bottomrule
 \end{tabular}}
\end{table}

 	We iterate again:
 	
 	  \begin{table}[H]
 	  \centering
  {\setlength{\tabcolsep}{0.4em}
 \begin{tabular}{c|cccccccccc}
    \toprule
  block & $B_1$ & $B_2$ & $B_3$ & $B_4$ & \color{red}{$B_5$} & \color{red}{$B_6$} & $B_7$ & $B_8$ & $B_9$ & $B_{10}$ \\[1mm]
  data & $\{1\}$ & $\{2,3,4\}$ & $\{5\}$ & $\{6\}$ & \color{red}{$\{7\}$} & \color{red}{$\{8\}$} & $\{9\}$ & $\{10\}$ & $\{11,12\}$ & $\{13\}$ \\[1mm]
  $f_i$ & $0$ & $0.3$ & $0.4$ & $0.6$ & \color{red}{$0.7$} & \color{red}{$0.2$} & $0.6$ & $0.8$ & $1$ & $0.7$ \\
   \bottomrule
 \end{tabular}}
\end{table}

\vspace*{-1cm}

  $$\Downarrow \qquad \Downarrow \qquad \Downarrow$$

  \begin{table}[H]
  \centering
  {\setlength{\tabcolsep}{0.4em}
 \begin{tabular}{c|ccccccccc}
    \toprule
  block & $B_1$ & $B_2$ & $B_3$ & $B_4$ & \color{red}{$B_5$} & $B_6$ & $B_7$ & $B_8$ & $B_9$ \\[1mm]
  data & $\{1\}$ & $\{2,3,4\}$ & $\{5\}$ & $\{6\}$ & \color{red}{$\{7,8\}$} & $\{9\}$ & $\{10\}$ & $\{11,12\}$ & $\{13\}$ \\[1mm]
  $f_i$ & $0$ & $0.3$ & $0.4$ & $0.6$ & \color{red}{$0.45$}  & $0.6$ & $0.8$ & $1$ & $0.7$ \\
   \bottomrule
 \end{tabular}}
\end{table}

	We iterate once again:
	
	 \begin{table}[H]
	 \centering
  {\setlength{\tabcolsep}{0.4em}
 \begin{tabular}{c|ccccccccc}
    \toprule
  block & $B_1$ & $B_2$ & $B_3$ & \color{red}{$B_4$} & \color{red}{$B_5$} & $B_6$ & $B_7$ & $B_8$ & $B_9$ \\[1mm]
  data & $\{1\}$ & $\{2,3,4\}$ & $\{5\}$ & \color{red}{$\{6\}$} & \color{red}{$\{7,8\}$} & $\{9\}$ & $\{10\}$ & $\{11,12\}$ & $\{13\}$ \\[1mm]
  $f_i$ & $0$ & $0.3$ & $0.4$ & \color{red}{$0.6$} & \color{red}{$0.45$}  & $0.6$ & $0.8$ & $1$ & $0.7$ \\
   \bottomrule
 \end{tabular}}
\end{table}

\vspace*{-1cm}

  $$\Downarrow \qquad \Downarrow \qquad \Downarrow$$

  \begin{table}[H]
  \centering
  {\setlength{\tabcolsep}{0.4em}
 \begin{tabular}{c|cccccccc}
    \toprule
  block & $B_1$ & $B_2$ & $B_3$ & \color{red}{$B_4$} & $B_5$ & $B_6$ & $B_7$ & $B_8$ \\[1mm]
  data & $\{1\}$ & $\{2,3,4\}$ & $\{5\}$ & \color{red}{$\{6,7,8\}$} & $\{9\}$ & $\{10\}$ & $\{11,12\}$ & $\{13\}$ \\[1mm]
  $f_i$ & $0$ & $0.3$ & $0.4$ & \color{red}{$0.5$}  & $0.6$ & $0.8$ & $1$ & $0.7$ \\
   \bottomrule
 \end{tabular}}
\end{table}

	We iterate once again:
	
	\begin{table}[H]
	\centering
  {\setlength{\tabcolsep}{0.4em}
 \begin{tabular}{c|cccccccc}
    \toprule
  block & $B_1$ & $B_2$ & $B_3$ & $B_4$ & $B_5$ & $B_6$ & \color{red}{$B_7$} & \color{red}{$B_8$} \\[1mm]
  data & $\{1\}$ & $\{2,3,4\}$ & $\{5\}$ & $\{6,7,8\}$ & $\{9\}$ & $\{10\}$ & \color{red}{$\{11,12\}$} & \color{red}{$\{13\}$} \\[1mm]
  $f_i$ & $0$ & $0.3$ & $0.4$ & $0.5$  & $0.6$ & $0.8$ & \color{red}{$1$} & \color{red}{$0.7$} \\
   \bottomrule
 \end{tabular}}
\end{table}

\vspace*{-1cm}

  $$\Downarrow \qquad \Downarrow \qquad \Downarrow$$

  \begin{table}[H]
  \centering
  {\setlength{\tabcolsep}{0.4em}
 \begin{tabular}{c|cccccccc}
    \toprule
  block & $B_1$ & $B_2$ & $B_3$ & $B_4$ & $B_5$ & $B_6$ & \color{red}{$B_7$}\\[1mm]
  data & $\{1\}$ & $\{2,3,4\}$ & $\{5\}$ & $\{6,7,8\}$ & $\{9\}$ & $\{10\}$ & \color{red}{$\{11,12,13\}$} \\[1mm]
  $f_i$ & $0$ & $0.3$ & $0.4$ & $0.5$  & $0.6$ & $0.8$ & \color{red}{$0.9$} \\
   \bottomrule
 \end{tabular}}
\end{table}

	There is no more violators yet!
	
	Reading out the solution:
	
	\begin{table}[H]
	\centering
  {\setlength{\tabcolsep}{0.4em}
 \begin{tabular}{c|cccccccc}
    \toprule
  block & $B_1$ & $B_2$ & $B_3$ & $B_4$ & $B_5$ & $B_6$ & \color{red}{$B_7$}\\[1mm]
  data & $\{1\}$ & $\{2,3,4\}$ & $\{5\}$ & $\{6,7,8\}$ & $\{9\}$ & $\{10\}$ & \color{red}{$\{11,12,13\}$} \\[1mm]
  $f_i$ & $0$ & $0.3$ & $0.4$ & $0.5$  & $0.6$ & $0.8$ & \color{red}{$0.9$} \\
   \bottomrule
 \end{tabular}}
\end{table}

	\vspace*{-1cm}
  $$\Downarrow \qquad \Downarrow \qquad \Downarrow$$

\begin{table}[H]
\centering
  {\setlength{\tabcolsep}{0.5em}
  \begin{tabular}{c|ccccccccccccc}
    \toprule
  $x$ & $-5$ & $-3$ & $-3$ & $-2$ & $-1$ & $0$ & $2$ & $3$ & $5$ & $6$ & $7$ & $7$ & $9$ \\[1mm]
  $y$ & $0$ & $0.4$ & $0.3$ & $0.2$ & $0.4$ & $0.6$ & $0.7$ & $0.2$ & $0.6$ & $0.8$ & $1$ & $1$ & $0.7$ \\[1mm]
  \color{red}{$f^*$} & \color{red}{$0$} & \color{red}{$0.3$} & \color{red}{$0.3$} & \color{red}{$0.3$} & \color{red}{$0.4$} & \color{red}{$0.5$} & \color{red}{$0.5$} & \color{red}{$0.5$} & \color{red}{$0.6$} & \color{red}{$0.8$} & \color{red}{$0.9$} & 
  \color{red}{$0.9$} & \color{red}{$0.9$}\\
   \bottomrule
 \end{tabular}}
\end{table}

	Which gives:
	
	\begin{figure}[H]
		\centering
		\includegraphics[width=1.0\textwidth]{img/computing/isotonic_regression_example_final.jpg}
	\end{figure} 
	This corresponds perfectly to what we get with the \texttt{R} software (see our companion book on the subject!).
	
	\subsubsection{Euler Method}\label{Euler method}
	Strictly speaking the Euler Method is much more than an interpolation method. But it can be introduced first as a liner piece-wise interpolation technique. In fact it provides an approximation (in the broadest sense) of a function $f (x)$ whose first derivative is known.
	
	The interpolated points are not necessarily equidistant as the interpolation is piece-wise but as soon we extrapolate the step is by tradition equidistant and denoted by $h$ and the distance between two points: $x_i=x_0+ih$. We denote by $f(x)$ the exact value and $y_i$ the approximate value.
	
	The idea is quite simple. If in $\mathbb{R}^2$ we know two points $(y_i,x_i)$ and $(y_{i+1},x_{i+1})$ we can calculate piece-wise the slope as:
	
	In more complicate example and real application of the Euler methods we know the true derivative on $x_i$.
	
	The previous relation can be written
	
	And therefore for small $h$ we can do a linear interpolation between two point.
	
	If the derivative is know (and the derivative can be of non-linear function!) we have the more general relation:
	
	That is traditionally written:
	
	named also the "\NewTerm{Euler difference equation}\index{Euler difference equation}". This relation can also be used to solve some simple numerical differential equation as illustrated by the following example:
	
	\begin{tcolorbox}[colframe=black,colback=white,sharp corners]
	\textbf{{\Large \ding{45}}Example:}\\\\
	Let's solve the following initial value problem ("Cauchy problem" style with $y'(x)=f'(x)$ and $y(x_0)=y_0$):
	
	using the Euler's method. The recurrent relation can be written in is this special case:
	
	i.e.:
	
	For a given $x$ we have:
	
	and thus:
	
	For  $h\rightarrow 0_+$ the approximate solution $y_n$ converges to the exact solution ${e}^{x}$.
	\end{tcolorbox}

	The "\NewTerm{local truncation error}\index{local truncation error}" (LTE) of the Euler method is error made in a single step. It is the difference between the numerical solution after one step.
	
	To calculate this error we can write first:
	
	For the exact solution (considered as exact...), we use the Taylor expansion (\SeeChapter{see section Sequences and Series page \pageref{taylor series}}):
	
	Also sometimes written:
	
	This relation can also be used to solve some simple numerical differential equation as illustrated by the following example:
	\begin{tcolorbox}[colframe=black,colback=white,sharp corners]
	\textbf{{\Large \ding{45}}Example:}\\\\	
	We want to use the method of Taylor's expansion of the third order to solve the following initial value problem (again a "Cauchy problem" style with $y'(x)=f'(x)$ and $y(x_0)=y_0$):
	
	For $n=0,1,2,\dots$ we have:
	
	Here:
	 \begin{eqnarray}
	 \hspace{-10mm}
	 x_0=1 \,, & & \qquad y_0=0 \,, \nonumber \\
	 y'(x_n) & = & \displaystyle\frac{4}{x_n^2}- y_n^2
	 -\displaystyle\frac{y_n}{x_n} \,, \nonumber \\
	  y''(x_n) & = &  -\displaystyle\frac{8}{x_n^3}-2 y_n y'(x_n)-
	  \displaystyle\frac{y'(x_n) x_n-y_n}{x_n^2}
	   \,=\, -\displaystyle\frac{12}{x_n^3}-
	  \displaystyle\frac{6y_n}{x_n^2}
	 +\displaystyle\frac{3y_n^2}{x_n}+2 y_n^3 \,, \nonumber \\
	 y'''(x_n) & = &\displaystyle\frac{24}{x_n^4}-2(y'(x_n))^2-2y_n
	 y''(x_n)-\frac{y''(x_n) x_n^2-2(y'(x_n)x_n-y_n)}{x_n^3}\nonumber\\
	 & = &\displaystyle\frac{12}{x_n^4}-
	  \displaystyle\frac{42y_n}{x_n^3}
	 +\displaystyle\frac{21y_n^2}{x_n^2}-\displaystyle\frac{12y_n^3}{x_n}
	 -6y_n^4 \,.\nonumber
	 \end{eqnarray}
	The table below	shows the computed values of the solution in the point$x_N=2$ for various $N$ (and thus for various $h=1/N$):
	 \begin{table}[H]
	   \centering
	 $$
	 \begin{array}{|l|c|c|c|c|c|c|c|} \\[-10mm] \hline
	 & x& 1 & 1.2 & 1.4 & 1.6 & 1.8 & 2 \\ \hline
	 h=0.2& & 0 & 0.576000 & 0.835950 & 0.920226 & 0.920287 & 0.884745
	 \\
	  h=0.1 & y(x)  & 0 & 0.581645 & 0.838338 & 0.919251 & 0.918141 &
	  0.882631 \\
	  h=0.05 & & 0 & 0.582110 & 0.838443 & 0.919062 & 0.917872 &
	 0.882386 \\ \hline
	 \multicolumn{8}{|c|}
	 {\mbox{Exact solution:}\qquad y(x)=
	 \displaystyle\frac{2(x^4-1)}{x(x^4+1)}\,,
	 \qquad y(2)= 0.882353} \\ [+0.2cm] \hline
	 \end{array}
	 $$
	 \end{table}
	\end{tcolorbox}
	
	The local truncation error (one step!!!) introduced by the Euler method is given by the difference between these equations:
	
	This shows that for small $h$, the local truncation error is approximately proportional to $h^2$. This makes the Euler method less accurate (for small $h$) than other higher-order techniques such as Runge-Kutta methods that we will see later.
	
	\pagebreak
	\subsubsection{Polynomial of collocation}
	In the section of Sequences and Series we discussed the general unsuitability of Taylor polynomials for approximation. These polynomials are useful only over small intervals for functions whose derivatives exist and are easily evaluated. In this section we find approximating polynomials that can be determined simply by specifying certain points on the plane through which they must pass.

	Given $y=f(x)$ a known function in explicit form or in tabulated form, and suppose that a given number of values:
	
	are given. The known points $(x_i,f(x_i))$ are named "\NewTerm{support points}\index{support points}".
	
	"\NewTerm{Interpolate $f$}\index{interpolate}" in this context means to estimate the values of $f$ for the horizontal $x$ knows values located between $x_0$ and $x_n$, that is to say in the range of interpolation by an approximate function $y=P(x)$, which satisfies the "collocations conditions" (nothing to do with your room-mate...!):
	
	\begin{figure}[H]
		\centering
		\includegraphics{img/computing/collocation_polynomial.jpg}
		\caption{Illustration of the interpolation concept with the collocation technique}
	\end{figure}
	The function $P$ is named "\NewTerm{collocation function}\index{collocation function}" on the $x_i$. When $P$ is a polynomial, we speak of "\NewTerm{collocation polynomial}\index{collocation polynomial}" or of "\NewTerm{interpolation polynomial}\index{interpolation polynomial}".
	
	"\NewTerm{Extrapolate}\index{extrapolate}" a function means study the $f(x)$ by $P(x)$ for $x$-located "outside" of the interpolation interval.
	\begin{tcolorbox}[title=Remark,colframe=black,arc=10pt]
	It goes without saying that the interpolation is a very important tool for all researchers, statisticians and others.
	\end{tcolorbox}	
	When we know a polynomial of degree $n$ on $n + 1$ points, we can know the by simple way (but not very fast - but there are several methods) this polynomial completely.
	
	To determine the polynomial, we will use the results presented above in our study of systems of linear equations. The disadvantage of the method presented here is that you have to guess to what type of polynomial you are dealing with (the order at least) and know what are the good points to chose (when there is a choice...).
	
	A particular example should suffice for understanding this method, the generalization being quite simple (see further below).
	
	Given the univariate second degree polynomial:
	
	and we know the following point (you will notice the ingenuity of the points selected by the authors of these lines ...):
	
	We deduce therefore the following system of equations:
	
	System that once solved using the state of the art techniques (\SeeChapter{see section Linear Algebra page \pageref{linear systems}}) gives us:
	
	Let us see the general case:
	\begin{theorem}
	Given  $(x_i,f(x_i))$  support points, with $x_i\neq x_j$ if $i\neq j$. Then there exists a polynomial $P_n(x)$ of degree less than or equal to $n$, and only one, such as $P_n(x_i)=f_i$ for $i=0,1,...,n$.
	\end{theorem}
	\begin{dem}
	Let us write:
	
	The collocation conditions:
	
	are thus written:
	
	This is a system of $n + 1$ equations in $n + 1$ unknowns.
	
	Its determinant is written (\SeeChapter{see section Linear Algebra page \pageref{determinant}}):
	
	relation that we name "\NewTerm{Vandermonde determinant}\index{Vandermonde determinant}". We know that if the system has a solution, the determinant of the system must be non-zero (\SeeChapter{see section Linear Algebra page \pageref{determinant matrix inverse}}).
	
	Let us show by an example (by taking a polynomial of the same degree as the one we used above) that the determinant is calculated using the previous relation (the reader will generalize by induction):
	
	Therefore in the case $n=2$, we consider the determinant:
	
	thus corresponding to the system (for reminder):
	
	Let us calculate this determinant following the column $1$ (by making use of co-factors as proved in the section of Linear Algebra):
	
	This polynomial can be written:
	
	Which is written:
	
	As the $x_9$ is in the statement of our problem are all different such that $x_i\neq x_j$ then the system has a unique solution! This proves that there is always an interpolation polynomial.
	\begin{flushright}
		$\blacksquare$  Q.E.D.
	\end{flushright}	
	\end{dem}
	It should be noted however that it is not a method of polynomial regression. Indeed, with a method of polynomial regression, we might choose a higher degree for the polynomial as the number of points we have.
	
	\pagebreak
	\subsubsection{Lagrange polynomials}\label{lagrange polynomial interpolation method}
	The interpolation  polynomial Lagrange method (used a lot in practice because the algorithm is very simple and therefore effective) consider that we initially have (know) $n + 1$ points such as:
	
	and that we are looking for a collocation polynomial that passes through all the points. The idea of polynomial interpolation of Lagrange is therefore simple and very clever (as always someone must have think to it first...). Let us observe the chart below where we have $5$ points by which we seek to pass a collocation polynomial:
	\begin{figure}[H]
		\centering
		\includegraphics{img/computing/lagrange_polynomial.jpg}
		\caption[Illustration of concept of Lagrange Interpolation]{Illustration of concept of Lagrange Interpolation (source: Wikipedia)}
	\end{figure}
	To find the collocation polynomial in red (we obviously suppose we don't know it initially), the idea is that for every $x_i$ we associate a different polynomial of degree $n$ and not null on $f(x_i)$ but null on all other points $(x_j)_{j\neq i}$. For this, as we see in the graph above, it is necessary that for each $i$ we associate a polynomial that has $n$ roots as:
	
	So we see that we have $5$ polynomials, each with $4$ roots (hence of order $4$) and which are respectively zero on all the points $x_j\neq x_i$ (as you can see on the graph). The coefficients $A_i$ are constants to be determined.
	
	Now nothing avoid us to sum all polynomial together since we will always have the right value on $f(x_i)$ (since the other polynomials are zero on this same point). Either by generalization the notation, the sum becomes:
	
	By injecting respectively $x_0,x_1,...,x_i,...,x_n$, we get in the general case:
	
	We deduce then immediately in the general case:
	
	Substituting the values of the constants in the initial expression of the sum, we get:
	
	Which can be written in condensed form (this relation will be useful to us later for some numerical integration methods!):
	
	where the term:
	
	is named "\NewTerm{Lagrange polynomial interpolation coefficients}\index{Lagrange polynomial interpolation coefficients}". 
	
	Notice that if we use the Lagrange polynomial to approximate a known analytic and continuous function (hence with an infinite number of points), then obviously there may be an approximation error. This will be denoted:
	
	
	\begin{tcolorbox}[colframe=black,colback=white,sharp corners]
	\textbf{{\Large \ding{45}}Example:}\\\\
	Approximate function $y=f(x)=x\,\sin(2x+\pi/4)+1$ by a polynomial $P_n$ of degree $n=3$, based on the following $n+1=4$ points:
	\begin{table}[H]
		\centering
		\begin{tabular}{|l|l|l|l|l|}
		\hline
		$i$ & $0$ & $1$ & $2$ & $3$ \\ \hline
		$x_i$ & $-1$ & $0$ & $1$ & $2$ \\ \hline
		$y_i=f(x_i)$ & $1.937$ & $1.000$ & $1.349$ & $-0.995$ \\ \hline
		\end{tabular}
	\end{table}
	Based on these points, we construct the Lagrange polynomials as the basis functions of the polynomial space:
	
	Note that indeed $L_0(x)+L_1(x)+L_2(x)+L_3(x)=1$. The interpolating polynomial can be obtained as a weighted sum of these basis functions:
	
	\centering
		\includegraphics[scale=0.90]{img/computing/lagrange_interpolation_example.jpg} 
	\end{tcolorbox}
	
	\subsubsection{Newton polynomials (divided differences)}
	As we have just seen, the Lagrange interpolation relies on the $n+1$ interpolation points:
	
	all of which need to be available to calculate each of the basis polynomials $l_i(x)$.
	
	The goodness of an approximation depends on the number of approximating points and also
on their locations. One problem with the Lagrange interpolating polynomial is that we need $n$
additions, $2n^2+2n$ subtractions, $2n^2+n-1$ multiplications, and $n+1$ divisions to evaluate $P(x)$ at a given point $x$. Even after all the denominators have been calculated once and for all we still need $n$ additions, $n^2+n$ subtractions, and $n^2+n$ multiplications.
	
	Furthermore, if additional points are to be used when they become available, all basis polynomials need to be recalculated. Another problem is that in practice, one may be uncertain as to how many interpolation points to use. So one may want to increase them over time and see whether the approximation gets better. In doing so, one would like to use the old approximation. It is not clear how to do that easily with the Lagrange form. In this sense, the Lagrange form is not incremental (plus it is also awkward to program).

	In comparison, in the Newton interpolation, when more data points are to be used, additional basis polynomials and the corresponding coefficients can be calculated, while all existing basis polynomials and their coefficients remain unchanged. Due to the additional terms, the degree of interpolation polynomial is higher and the approximation error may be reduced (e.g., when interpolating higher order polynomials).

	Specifically, the basis polynomials of the Newton interpolation are calculated as below:
	
	and the Newton interpolating polynomial is constructed:
	
	Note that the last data point $(x_n, y_n)$ is not used in any of the basis polynomials, but it is used for calculating the last coefficient $c_n$, as shown below. For the $n$th degree polynomial $N_n(x)$ to pass all $n+1$ points $(x_i,\;y_i),\;(i=0,\ldots,n)$, it needs to satisfy the following $n+1$ equations:
	
	The last relation is named the "\NewTerm{Newton's divided-difference formula}\index{Newton's divided-difference formula}".

	The relation can also be expressed in matrix form:
	
	
	The $n+1$ coefficients $c_0,\ldots,c_n$ can be obtained by solving these $n+1$ equations in the triangular equation system progressively from top down:
	
	
	In general, we have the:
	
	which is the expanded form of the $k$th divided differences $f[x_0,\ldots,x_k]$ of the first $k+1$ points. Now the Newton polynomial interpolation can be written as:
	

	Due to the uniqueness of the polynomial interpolation (here "uniqueness" means only in the way that all previous polynomial interpolation methods are such that for some given points $\{(x_i,y_i)\}_{i=0}^n$ we just have $p_n(x_i)=y_i$... and nothing more!), this Newton interpolation polynomial is the same as that of the Lagrange and the power (ie Taylor) function interpolations:
	
	They are the same $n$th degree polynomial but expressed in terms of different basis polynomials weighted by different coefficients.

	We can now consider some important facts all related to the Newton polynomial interpolation.	
	
	\begin{itemize}
		\item When an additional node point $(x_{n+1},\,y_{n+1})$ is available and to be used, all previous basis polynomials and their corresponding coefficients remain unchanged, we only need to obtain a new basis polynomial of degree $n+1$:
		
		together with its coefficient, the $(n+1)$th divided difference $c_{n+1}=f[x_0,\ldots,x_n,x_{n+1}]$. The new interpolation polynomial of degree $n+1$ can then be obtained by including an extra term in the summation above:
		
		As $N_{n+1}(x)$ passes through the new point $(x_{n+1},\,y_{n+1})$, we have:
		
		But $x_{n+1}$ is just an arbitrary point, we can replace it by $x$, and get:
		
		
		\item If all $n+1$ points $x_0=a\le x_1\le\ldots\le x_{n-1}\le x_n=b$ are equally spaced, i.e.:
		
				then Newton's divided difference interpolation can take a simpler form. For any point $x\in(a,\,b)$, we let $c=(x-x_0)/h$ so that it can be represented as $x=x_0+ch$, and $x-x_i=(x_0+ch)-(x_0+ih)=(c-i)h$, now the Newton polynomial can be written as:
		
		where we used the binomial-coefficient notation:
		
	\end{itemize}
	
	\begin{tcolorbox}[colframe=black,colback=white,sharp corners]
	\textbf{{\Large \ding{45}}Example:}\\\\
	Approximate function $y=f(x)=x\,\sin(2x+\pi/4)+1$ by a polynomial of degree $n=3$, based on the following $n+1=4$ points:
	\begin{table}[H]
		\centering
		\begin{tabular}{|l|l|l|l|l|}
		\hline
		$i$ & $0$ & $1$ & $2$ & $3$ \\ \hline
		$x_i$ & $-1$ & $0$ & $1$ & $2$ \\ \hline
		$y_i=f(x_i)$ & $1.937$ & $1.000$ & $1.349$ & $-0.995$ \\ \hline
		\end{tabular}
	\end{table}
	Based on $f[x_i]=f(x_i),\;(i=0,\cdots,n)$, we can find all other divided differences recursively in tabular form as shown below. In general, $f[x_i,\cdots,x_j]$ can be found based on its left neighbour $f[x_{i+1},\cdots,x_j]$ and top-left neighbour $f[x_i,\cdots,x_{j-1}]$:
	
	\begin{table}[H]
		\resizebox{\textwidth}{!}{\begin{tabular}{|l|l|l|l|l|}
		\hline
		$x_i$ & $0$th & $1$st & $2$nd & $3$rd \\ \hline
		$x_0=-1$ & $f[x_0]=1.937$ &  &  &  \\ \hline
		$x_1=0$ & $f[x_1]=1.000$ & $f[x_0,x_1]=-0.937$ &  &  \\ \hline
		$x_2=1$ & $f[x_2]=1.349$ & $f[x_1,x_2]=0.349$ & $f[x_0,x_2]=0.643$ &  \\ \hline
		$x_3=2$ & $f[x_3]=-0.995$ & $f[x_2,x_3]=-2.343$ & $f[x_1,x_3]=-1.346$ & $f[x_0,x_3]=-0.663]$ \\ \hline
		\end{tabular}}
	\end{table}
	The coefficients are the four divided differences along the diagonal:
			
	and $c_3=f[x_0,x_1,x_2,x_3]=-0.663$. Alternatively, they can also be represented in the expanded form:
	\end{tcolorbox}
	
	\begin{tcolorbox}[colframe=black,colback=white,sharp corners]
	
	Now the Newton interpolating polynomial can be obtained as:
	
 	\begin{figure}[H]
		\centering
		\includegraphics{img/computing/newton_polynomial_interpolation_example.jpg}
	\end{figure}
	\end{tcolorbox}
	Using the Newton interpolating polynomials is then usually the best choice in comparison of Taylor, Vandermonde (ie collocation polynomial\footnote{Using  the Vandermonde  matrix is  indeed not  a  very  good  method  for  any  situation. The system is ill-conditioned and therefore the coefficients may be calculated very inaccurately. Also the amount of work is excessive!}) or Lagrange polynomials.  It has the advantage that data pairs can be added and interpolated by merely adding one additional term to the previous interpolating polynomial.  Under other restrictions the coefficients give information about the derivatives of a function being approximated as well as the error.
	
	\subsubsection{Errors in Polynomial Interpolation}
	There are multiple ways to derive the approximation error of polynomial interpolation. In our personal and subjective point of view, all proofs known to us so far are not really convincing...
	
	So let us introduce the error term in an engineer way rather than in a tricky mathematician one. For this, let us recall that during our Study of Taylor series, we have proved that the Lagrange remainder (\SeeChapter{see section Sequences and Series page \pageref{Lagrange Remainder}}) was given by:
	
	For the case of polynomial case where we have multiple points $x_0,x_1,\ldots,x_n$ rather than a unique one $x_0$ we can guess that the error may be written:
	
	In the field of polynomial interpolation (that we use the Taylor, Lagrange or Newton methods) that latter is often written:
	
	where $\xi$ is some (unknown) point in the interval $[a,b]$. The precise location of this point depends on $\{x_i\}_{i=0}^n$. Here $f^{(n+1)}(\xi)$ is the $(n+1)$st derivative of $f(x)$ evaluated at the point $x=\xi_x$.
	
	Therefore, for example (an example that will be useful to us when we will study numerical integration), the Lagrange polynomial given for recall by:
		
	Will be written in it's full form as:
	
	
	\pagebreak
	\subsection{Roots search}
	Many equation encountered in practice or in theory cannot be solved by closed form or analytical methods. Consequently, only a numerical approach can be obtained in a finite number of operations.

	The mathematician Évariste Galois has proved, in particular, that the polynomial equation $P_n(x)$ (\SeeChapter{see section Calculus page \pageref{polynomial}}) has no algebraic solution if $P_n(x)$ is of degree $n>4$.

	There are numerous algorithms that gives the possibility to calculate the roots of equations of the type $f(x)=0$ with an almost arbitrary precision. We will see in this section only the main one.

	\begin{tcolorbox}[colback=red!5,borderline={1mm}{2mm}{red!5},arc=0mm,boxrule=0pt]
	\bcbombe Caution! The implementation of such algorithms need always at least an approximate knowledge (randomly or deterministic) of the searched value and also the behaviour (shape) of the function near the root. A table of the function values (\SeeChapter{see section Functional Analysis page \pageref{table of variations}}) or its graphical representation (see same section) can in simple cases help to acquire this prerequisites knowledge.
	\end{tcolorbox}
	
	
	If the univariate equation to solve is under the form $g(x)=h(x)$, we plot the curves representing $g$ and $h$. The roots of the equation $g(x)=h(x)$ being given by the abscissa of the intersection points of the both curves.
	
	\begin{tcolorbox}[title=Remark,colframe=black,arc=10pt]
	Before solving by numerical methods the equation $f(x)=0$, we have to check that the function $f$ satisfy some constraints. For example, the function $f$ has to be at least strictly monotone near the root that we will denote here by $\bar{x}$, when the Newton's method is applied. It is many times useful, even absolutely necessary, to determine an interval $[a,b]$ such that:
	\begin{itemize}
		\item $f$ is continuous on $[a,b]$ of class $\mathcal{C}^1$.

		\item $f(a)f(b)<0$
		
		\item $\exists \bar{x}$ unique, $\bar{x}\in [a,b],f(r)=0$
	\end{itemize}
	\end{tcolorbox}

	\subsubsection{Proportional parts methods}
	The implementation, on a computer, of this method is particularly simple. The conditions to be satisfied being only that in the interval $[a,b]$:
	\begin{itemize}
		\item $f$ must be continuous

		\item $f$ must be monotone near the root $r$
		
		\item $f(a)f(b)<0$ to sure that there is a root
	\end{itemize}
	As we already know, in a small interval, we can replace a curve by a straight line segment. There are several possible situations but here is a special one but that can be generalized easily to anything:
	
	\begin{figure}[H]
		\centering
		\includegraphics{img/computing/root_proportional_method.jpg}
		\caption{Local approximation of a curve by a straight line segment}
	\end{figure}
	In this figure, we get by using the theorems of Thales (\SeeChapter{see section Euclidean Geometry page \pageref{thales theorem}}):
	
	therefore:
	
	If $|f(a)| \ll |f(b)|$, we can neglect $f(a)$ at the denominator and it comes:
	
	The algorithm consists therefore in performing the following steps:
	\begin{enumerate}
		\item We fix $\varepsilon>0$ as upper bound of the admissible error tolerance.		
		
		\item Choose $a$ and $b$ with opposite signs, calculate $f(a)$ and $f(b)$
		
		\item We calculate $x_1=a-(b-a)k$. If $|f(x_1)|$ is small enough, we stop the calculation and we display $x_1$ and $f(x_1)$.

		\item Otherwise we proceed as following:
		\begin{itemize}
			\item We replace $b$ by $a$ and $f(b)$ by $f(a)$
			\item We replace $a$ by $x_1$ and $f(a)$ by $f(x_1)$
		\end{itemize} 
		and we go back to point (2).
	\end{enumerate}
	In pseudo-code (non-unique and not optimized):\\\\
	\begin{algorithm}[H]
	 \KwData{$a$,$ b$, $\varepsilon$ expression of $f$ }
	 \KwResult{$x_1$}
	 initialization\;
	$f(a)$,$f(b)$\;
	 \While{$|f(x_1)|>\varepsilon$}{
	  $x_1=a-(b-a)\displaystyle\frac{f(a)}{f(b)}$\;
	  \If{$|f(x_1)|>\varepsilon$}{
	   $b:=a$\;
	   $f(b):=f(a)$\;
	   $a:=x_1$\;
	   $f(a):=f(x_1)$\;
	   }
	  Display $x_1$\;
	 }
	 \caption{Proportional Parts pseudo-code algorithm}
	\end{algorithm}
	Obviously this is not the best algorithm especially if there is no root or if the computer is slow. Then a good advice is to take as input a number of limited iterations that have to be done (we will see a more elaborated way to handle such situations in the next algorithm).
	
	\subsubsection{Bisection method (interval-halving)}\label{bisection method}
	The bisection method in mathematics is also root-finding method that repeatedly bisects an interval and then selects a subinterval in which a root must lie for further processing. It is a very simple and robust method, but it is also relatively slow. Because of this, it is often used to obtain a rough approximation to a solution which is then used as a starting point for more rapidly converging methods. The method is also named the "\NewTerm{interval halving method}\index{interval halving method}", the "\NewTerm{binary search method}\index{binary search method}", or the "\NewTerm{dichotomy method}\index{dichotomy method}".
	
	The method is applicable for numerically solving the equation $f(x) = 0$ for the real variable $x$, where $f$ is a continuous function defined on an interval $[a, b]$ and where $f(a)$ and $f(b)$ have opposite signs such that $f(a)f(b)<0$. In this case $a$ and $b$ are also said to bracket a root since, by the intermediate value theorem, the continuous function f must have at least one root in the interval $[a, b]$.

	At each step the method divides the interval in two by computing the midpoint $c = (a+b) / 2$ of the interval and the value of the function $f(c)$ at that point. Unless $c$ is itself a root (which is very unlikely, but possible) there are now only two possibilities: either $f(a)$ and $f(c)$ have opposite signs and bracket a root, or $f(c)$ and $f(b)$ have opposite signs and bracket a root. The method selects the subinterval that is guaranteed to be a bracket as the new interval to be used in the next step. In this way an interval that contains a zero of $f$ is reduced in width by $50\%$ at each step. The process is continued until the interval is sufficiently small.

	\begin{figure}[H]
		\centering
		\includegraphics{img/computing/root_bissection_method.jpg}
		\caption{Bisection method scheme}
	\end{figure}

	Explicitly, if $f(a)$ and $f(c)$ have opposite signs, then the method sets $c$ as the new value for $b$, and if $f(b)$ and $f(c)$ have opposite signs then the method sets $c$ as the new $a$. (If $f(c)=0$ then $c$ may be taken as the solution and the process stops.) In both cases, the new $f(a)$ and $f(b)$ have opposite signs, so the method is applicable to this smaller interval.
	
	The implementation, on a computer, of this method is particularly simple. The conditions to be satisfied being only that in the interval $[a,b]$:
	\begin{itemize}
		\item $f$ must be continuous

		\item $f$ must be monotone near the root $\bar{x}$
		
		\item $f(a)f(b)<0$ to sure that there is a root
	\end{itemize}
	
	The algorithm consists therefore in performing the following steps:
	\begin{enumerate}
		\item We fix $\varepsilon>0$ as upper bound of the admissible error tolerance.
		
		\item We calculate $x=(a+b)/2$

		\item We evaluation $f(x)$
		
		\item If $|f(x)|<\varepsilon$ then the job is done, we have to display $x$ and $f(x)$
	
		\item Otherwise we proceed as following
			\begin{enumerate}
				\item we replace $a$ by $x$ if $f(x)f(a)>0$.
	
				\item we replace $b$ by $x$ if $f(x)f(b)>0$ or $f(x)f(a)<0$.
				
				\item we go back in (2)
			\end{enumerate}
	\end{enumerate}
	The previous step (4) imposes the condition for stopping the calculations. Sometimes it is better to choose another criterion calculation ending. It requires the calculated solution to be contained in an interval of length equation containing the root $x^{*}$. This test is enunciate as follows:
	\begin{enumerate}
		\item[4'.] If $|b-a|<\varepsilon$, the job is finished and $x=(a+b)/2$ is displayed. It is for sure obvious that $|x-x^{*}|<\varepsilon/2$
	\end{enumerate}
	In pseudo-code (non-unique and not optimized):\\\\
	\begin{algorithm}[H]
	 \KwData{$a$,$ b$, $\varepsilon$ expression of $f$ }
	 \KwResult{$x^{*}$}
	 initialization\;
	$x=(a+b)/2$\;
	\While{$|f(x)|>\varepsilon$}{
	    \uIf{$f(x)f(a)>0$}{
     		$a:=x$\;
	 	}
		\uElseIf{$f(x)f(b)>0\; \vee \; f(x)f(a)<0$}{
			$b:=x$\;
		}
		$x=(a+b)/2$\;
	 }
	 Display $x,f(x)$\;
	 \caption{Proportional Parts bisection pseudo-code algorithm}
	\end{algorithm}
	The equivalent Maple 4.00b code is given by:
	
	\texttt{>zero:=proc(f,a,b,pre)
	local M;\\
	M:=f((a+b)/2);\\
	if abs(M)<pre then \\
	     RETURN((a+b)/2)\\
	elif M>0 then\\
	     zero(f,a,(a+b)/2,pre)\\
	else zero(f,(a+b)/2,b,pre)\\
	     fi\\
	end:}
	
	\pagebreak
	\subsubsection{Secant method (Regula Falsi or False Position)}
	The "\NewTerm{secant method}\index{secant method}" also named or "\NewTerm{regula falsi}\index{regula falsi}" (for: regularly false) or also named "\NewTerm{false position method}\index{false position method}" is still a root search algorithm to zero. To introduce it, consider the following figure:
	\begin{figure}[H]
		\centering
		\includegraphics{img/computing/root_regula_falsi.jpg}
		\caption{Regula falsi method scheme}
	\end{figure}
	The implementation, on a computer, of this method is particularly simple. The conditions to be satisfied being only that in the interval $[a,b]$:
	\begin{itemize}
		\item $f$ must be continuous

		\item $f$ must be monotone near the root $x^{*}$
		
		\item $f(a)f(b)<0$ to sure that there is a root
	\end{itemize}
	If $P_n$ is the point of coordinates $(x_n,0)$, then the points $(A,B,P_n)$ are aligned on the secant. The following proportion (application of Thales theorem) is then true:
	
	hence we deduce that:
	
	The algorithm consists therefore in performing the following steps:
	\begin{enumerate}
		\item We fix $\varepsilon>0$ as upper bound of the admissible error tolerance.
	
		\item Calculation of $x_n=\dfrac{af(b)-bf(a)}{f(b)-f(a)}$

		\item Evaluation of $f(x_n)$

		\item If $|f(x_n)|<\varepsilon$, the job is done. We have to display $x_n$

		\item Otherwise we proceed as following:
			\begin{enumerate}
				\item we replace $a$ by $x_n$ if $f(x_n)f(a)>0$.
	
				\item we replace $b$ by $x_n$ if $f(x_n)f(b)>0$ or $f(x_n)f(a)<0$.
				
				\item we go back in (2)
			\end{enumerate}
 	\end{enumerate}
 	Once again, the previous step (4) imposes the condition for stopping the calculations. Sometimes it is better to choose another criterion calculation ending. It requires the calculated solution to be contained in an interval of length equation containing the root $x^{*}$. This test is enunciate as follows:
	\begin{enumerate}
		\item[4'.] If $|b-a|<\varepsilon$, the job is finished and $x_n=\dfrac{af(b)-bf(a)}{f(b)-f(a)}$ is displayed. It is for sure obvious that $|x-\bar{x}|<\varepsilon/2$
	\end{enumerate}
 	In pseudo-code (non-unique and not optimized):\\\\
	\begin{algorithm}[H]
	 \KwData{$a$,$ b$, $\varepsilon$ expression of $f$ }
	 \KwResult{$\bar{x}$}
	 initialization\;
	$x_n=\dfrac{af(b)-bf(a)}{f(b)-f(a)}$\;
	\While{$|f(x)|>\varepsilon$}{
	    \uIf{$f(x)f(a)>0$}{
     		$a:=x_n$\;
	 	}
		\uElseIf{$f(x_n)f(b)>0\; \vee \; f(x_n)f(a)<0$}{
			$b:=x$\;
		}
		$x_n=\dfrac{af(b)-bf(a)}{f(b)-f(a)}$\;
	 }
	 Display $x_n,f(x_n)$\;
	 \caption{Regula-Falsi pseudo-code algorithm}
	\end{algorithm}
	Notice that the initial above relation can also be rewritten:
	
	If we change a little bit the notation we get:
	
	often also denoted:
	
	We then use this new value of $x$ in the prior previous relation as $x_2$ and repeat the process using $x_1$ and $x_2$ instead of $x_0$ and $x_1$. We continue this process, solving for $x_3$, $x_4$, etc., until we reach a sufficiently high level of precision (a sufficiently small difference between $x_n$ and $x_n - 1$):
	
	This again an application of the "\NewTerm{fixed-point iteration}\index{fixed-point iteration}" (related to the theorem of the same name proved in the section of Sequences and Series page \pageref{fixed point theorem}).
	
	\pagebreak
	\subsubsection{Newton's method}\label{newton method}
	A few years after its discovery of the theory of gravitation, Newton developed a special technique to fast compute the solutions of any equation. This "supernaturally" fast convergence has been used to prove some of the most significant theoretical results of the 20th century: the Kolmogorov stability theorem, the isometric embedding theorem of Nash ... Alone, this technique transcends the distinction between pure mathematics and applied mathematics!!!
	
	To study the "\NewTerm{Newton's method}\index{Newton's method}" (also named "\NewTerm{Newton-Raphson method}\index{Newton-Raphson method}\footnote{Newton's method was first published in 1685 and in 1690, Joseph Raphson published a simplified description. Raphson also applied the method only to polynomials, but he avoided Newton's tedious rewriting process by extracting each successive correction from the original polynomial. This allowed him to derive a reusable iterative expression for each problem. Finally, in 1740, Thomas Simpson described Newton's method as an iterative method for solving general non-linear equations using calculus. In the same publication, Simpson also gives the generalization to systems of two equations and notes that Newton's method can be used for solving optimization problems by setting the gradient to zero.}" or "\NewTerm{Newton approximation scheme"}\index{Newton approximation scheme}) in the plane (hence in the univariate case), consider the following figure:
	\begin{figure}[H]
		\centering
		\includegraphics{img/computing/root_newton_method.jpg}
		\caption{Newton's method in the plane}
	\end{figure}
	If $x_0$ is an approximation of the root $\bar{x}$, we notice that $x_1$ is as better one. The point $x_1$ is the intersection of the tangent to the curve $(x_0,f(x_0))$ with the $x$-axis. The point $x_2$ is even a better approximate of $\bar{x}$, the point $x_2$ is obtained in the same manner as $x_1$ but from $(x_1,f(x_1))$.
	
	We may ask ourselves at this point what is the difference between Newton-Raphson and secant method on the basis of geometric interpretation? The two methods are almost the same, from a geometric perspective. The difference is that Newton's method uses a line that is tangent to one point, while the secant method uses a line that is secant to two points.
	
	To use this technique, remember that if we take a function $f$ that is differentiable on $x_0$, then we can rewrite it in the form (\SeeChapter{see section Sequences and Series page \pageref{taylor series}}):
	
	where $f'(x_0)$ is the derivative of $f$ in $x_0$ and $\mathcal{O}(x-x_0)$ is a function that tends to $0$ as $(x-x_0)^n$ when $n \geq 2$ when $x$ tends to $x_0$ (this is as we know a corrective term for the superior orders of the Taylor series).
	
	Applying this result to solve $f(x)=0$, we get:
	
	The function does not permit the resolution of this equation relatively to $\bar{x}$. By neglecting this term we get obviously:
	
	This is just the expression of the discrete derivative. Indeed:
	 
	The latter relation is easily resolved with respect to $\bar{x}$. To see this, let us begin by putting $\bar{x}=x_1$:
	
	But $x_1$ does not satisfy, in general, the equality $f(x_1)=0$. But as we have already pointed it out, $|f(x_1)|$  is smaller than $|f(x_0)|$ if the function $f$ satisfies some conditions.
	
	Newton's method consists in replacing the relation:
	
	by:
	
	and to iteratively solve this relation.
	
	The following conditions are sufficient to ensure the convergence of the method in an interval $[a, b]$ including $x_0$ and $\bar{x}$:
	\begin{enumerate}
		\item The function is twice differentiable

		\item The derivative $f'$ does not vanish (monotony)
		\item The second derivative $f''$ is continuous and does not vanish (no inflection point)
	\end{enumerate}
	\begin{tcolorbox}[title=Remark,colframe=black,arc=10pt]
	It is often sufficient to check the conditions (1) and (2) for the process to converge.
	\end{tcolorbox}
	The condition (2) is obvious, indeed if $f'(x)=0$ the iteration can lead to a calculation overflow (singularity).
	
	The third condition (3) is less obvious, but the following figure shows a case of non-convergence. In this case, the process loops calculating alternately $x_i$ and $x_j$:
	\begin{figure}[H]
		\centering
		\includegraphics{img/computing/root_newton_method_non_convergence.jpg}
		\caption{Non-convergence example of Newton's method}
	\end{figure}
	
	\begin{tcolorbox}[colframe=black,colback=white,sharp corners]
	\textbf{{\Large \ding{45}}Example:}\\\\
	Consider the problem of finding the positive number $x$ with $\cos(x) = x^3$. We can rephrase that as finding the zero of $f(x) = \cos(x) - x^3$. We have $f'(x) = -s\in(x) - 3x^2$. Since $\cos(x) \leq 1$ for all $x$ and $x^3 > 1$ for $x > 1$, we know that our solution lies between $0$ and $1$. We try a starting value of $x_0 = 0.5$. (Note that a starting value of $0$ will lead to an undefined result, showing the importance of using a starting point that is close to the solution!).
	
	The correct digits are underlined in the above example. In particular, $x_6$ is correct to the number of decimal places given. We see that the number of correct digits after the decimal point increases from $2$ (for $x_3$) to $5$ and $10$, illustrating the quadratic convergence.
	\end{tcolorbox}
	If the function $f$ is given analytically, its derivative can be determined analytically. But in many cases it is advisable or even necessary to replace $f'(x_n)$ by the differential quotient:
	
	where $h$ should be chosen as small enough so that the difference:
	
	is also small enough.

	The iteration is then written:
	
	If the resolution method is converging, the gap between $x_{n+1}$ and $\bar{x}$ decreases at each iteration. This is ensured, for example, if the interval $[a, b]$ containing $x_{n+1}$, sees its length decreasing at each step. 

	\begin{theorem}
	Newton's method is interesting because the convergence is quadratic:
	
	while the convergence of other methods is linear such that:
	
	Let us consider, for example, the method of bisection seen previously. At each iteration the length of the interval $[a, b]$ is halved. This ensures us that the gap $|x_{n+1}-\bar{x}|$ is halved at each step of the calculation:
	
	\end{theorem}
	\begin{dem}
	To prove the quadratic convergence of the Newton's method, we have to make use of the limited Taylor series of $f$ and $f'$ in the neighbourhood of $\bar{x}$:
	
	But:
	
	therefore:
	
	Subtracting $\bar{x}$ left and right of the equality and putting the two terms of the second member to the same denominator, we get:
	
	and when $x_n-\bar{x}$ is small enough, the denominator can be simplified.
	
	which shows that convergence is quadratic.
	\begin{flushright}
		$\blacksquare$  Q.E.D.
	\end{flushright}
	\end{dem}
	
	\begin{tcolorbox}[colframe=black,colback=white,sharp corners]
	\textbf{{\Large \ding{45}}Example:}\\\\
	Here is an application with Maple 4.00b of this method:\\
	
	\texttt{
	>with(plots): with(plottools):\\
	>f:=x->exp(x)*x\string^2-36;\\
	>D(f)(x);\\
	>x[0]:=3;\\
	>n:=7;\\
	>g:=x->f(x[i-1])+D(f)(x[i-1])*(x-x[i-1]);\\
	>for i from 1 by 1 to n do;\\
	>x[i]:=evalf(solve(g(x)=0,x));\\
	>od;\\
	>lines:={}:\\
	>for i from 1 by 1 to n do;\\
	>lines:=lines union \{line([x[i-1],0],[x[i-1],f(x[i-1])],color=green), line([x[i-1],f(x[i-1])],[x[i],0],color=green)\};\\
	>od:\\
	>display({plot(f(x),x=2..3.01)} union lines);}
	\begin{figure}[H]
		\centering
		\includegraphics{img/engineering/newton_method_with_maple4.jpg}
		\caption{Maple 4.00b application of Newton's method}
	\end{figure}
	\end{tcolorbox}
	Now let us prove that Newton's method is reduced to the Babylonian method seen earlier above at page \pageref{Heron square root algorithm}! Indeed. we want to solve $\sqrt{A}$, i.e. looking for a root of $f(x)=x^2-A$. Then:
	
	That's it!
	
	\begin{tcolorbox}[title=Remark,colframe=black,arc=10pt]
	Let us finally communicate that we will study the Newton's method with several variables during our study of non-linear optimization. It is an educational choices that seemed to us the best choice.
	\end{tcolorbox}	
	
	\pagebreak
	\subsection{Numerical Differentiation}
	In numerical analysis, numerical differentiation describes algorithms for estimating the derivative of a mathematical function or function subroutine using values of the function and perhaps other knowledge about the function.
	
	Many modelling techniques or numerical resolution technique that we will see further use derivatives as for example the search of optimums (see further below), the finite element methods (see also further below). For example, to name the most famous case, the solver of Microsoft Office Excel 2007 and earlier offers some of the most elementary numerical derivatives that we will study here and reuse further:
	\begin{figure}[H]
		\centering
		\includegraphics{img/computing/excel_solver_derivatives.jpg}
		\caption{Screenshot of Microsoft Excel 2003 Solver}
	\end{figure}
	To allow a computer processing, the various derivatives  present in many algorithms must be approximated numerically. To do this, we use in the most basic case the principle of centered finite difference which is based on the following Taylor series expansions (\SeeChapter{see section Sequences and Series page \pageref{taylor series}}):
	
	We then based on the base of this principle the following development of the second order:
	
	It comes then when neglecting the higher order terms and subtracting and simplifying the two series above:
	
	Relation that we name "\NewTerm{first centered derivative with tangent estimate}\index{first centered derivative with tangent estimate}" (because we neglect all non-linear terms) or also "\NewTerm{Symmetric difference quotient}\index{symmetric difference quotient}". We also find often this latter relation in the following equivalent form:
	
	Now let us see what we name the "\NewTerm{right first derivative}\index{right first derivative}" also named "\NewTerm{forward derivatives}\index{forward derivatives}" (or "forward difference"), which consist simply in the application of the following intuitive  algorithm:
	
	and incidentally we can also define the "\NewTerm{left first derivative}\index{left first derivative}" also named "\NewTerm{backward derivative}\index{backward derivative}" (or "backward difference"):
	
	We see therefore the central derivatives require more calculations but are also more accurate. This below figure gives a quite summary of the previous relations with a special case:
	\begin{figure}[H]
		\centering
		\includegraphics{img/computing/forward_backward_derivative.jpg}
		\caption{Forward/Backward derivatives illustration}
	\end{figure}
	We can also develop more elaborate relations through Taylor expansions with superior orders, do averages between different methods and so on... it's quite endless...
	
	\pagebreak
	\subsection{Numerical Integration}\label{numerical integration}
	In numerical analysis, "\NewTerm{numerical integration}\index{numerical integration}" constitutes a broad family of algorithms for calculating the numerical value of a definite integral, and by extension, the term is also sometimes used to describe the numerical solution of differential equations. This subsection focuses on calculation of definite integrals. 
	
	The basic problem in numerical integration is to compute an approximate solution to a definite integral overt the interval $[a,b]$ by evaluating $f(x)$ at a finite number of sample points:
	
	to a given degree of accuracy (with an error $\varepsilon(f)$. If $f(x)$ is a smooth function integrated over a small number of dimensions, and the domain of integration is bounded, there are many methods for approximating the integral to the desired precision.
	
	\textbf{Definition (\#\mydef):} Suppose that $a=x_0<x_1<\ldots<x_n=b$. A relation of the form:
	
	with the property that:
	
	is named a "\NewTerm{numerical integration}\index{numerical integration}" or "\NewTerm{quadrature formula}\index{quadrature formula}\footnote{In mathematics, quadrature is a historical term which means the process of determining area. This term is still used nowadays in the context of differential equations, where "solving an equation by quadrature" means expressing its solution in terms of integrals.}".  The term $\varepsilon(f)$  is named the "\NewTerm{truncation error for integration}".  The values $\{x_i\}_{i=0}^n$ are named the quadrature nodes and $\{w_i\}_{i=0}^n$ are named the "\NewTerm{weights}".

 	Depending on the application, the nodes  $\{x_i\}_{i=0}^n$  are chosen in various ways.  For the Trapezoidal Rule, Simpson's Rule, and Boole's Rule, the nodes are chosen to be equally spaced.  For Gauss-Legendre quadrature, the nodes are chosen to be zeros of certain Legendre polynomials.  When the integration formula is used to develop a predictor formula for differential equations, all the nodes are chosen less than b.  For all applications, it is necessary to know something about the accuracy of the numerical solution. 
	
	Let us consider the following figure:
	\begin{figure}[H]
		\centering
		\includegraphics{img/computing/numerical_integration_interval.jpg}
		\caption{Illustration of an interval under a curve}
	\end{figure}
	We would like to calculate the area between the $x$ axis, the curve $f$ and the straight vertical lines of equations $x=a$ and $x=b$. We assume in this case that the function $f$ is with positive values:
	
	We would like to calculate the area between the $x$ axis, the curve $f$ and the straight vertical lines of equations $x=a$ and $x=b$. We assume in this case that the function $f$ is with positive values:
	
	This problem, in its generality, is difficult or impossible to solve analytically in the most general cases. Below we will see some mainstream numerical methods for the approximate calculation of this area in increasing complexity order (sometimes these methods are used in corporations by employees who have only spreadsheets softwares like Microsoft Excel or OpenOffice Calc to calculate integrals...):
	\begin{itemize} 
		\item Newton-Cotes formulas: In this case, we obtain methods for numerical integration which can be derived from the Lagrange interpolating method. Alternatively the formulas can also be derived from Taylor expansion. The idea is similar to the way we obtain numerical differentiation schemes. We can easily derive not just integration formulas but also their errors using this technique. The schemes which we develop here will be based on the assumption of equidistant points.
	
		\item Composite, Newton - Cotes formulas (open and closed): These methods are composite since they repeatedly apply the simple formulas derived previously to cover longer intervals. This idea allows for piecewise estimates of the integral thus improving the error of our integration (we will also assume equidistant nodes in our introduction of these methods).
	
		\item Romberg Integration: This method allows us to improve the error of our integration methods by doing minimal extra work. The idea is based on the Richardson extrapolation (actually we don't present this method here).
	
		\item: Adaptive Integration: Here we are free to choose the points over which we calculate the numerical integral of $f(x)$ so as to minimize our error. Adaptive integration does not therefore require equidistant nodes. Thus if the function is not very smooth at some interval the step size $h$ of the numerical integration method decreases to make sure we do not accumulate too much error in our calculation (actually we don't present this method here). 
	
		\item Gaussian Integration: We explore methods which can achieve optimal error reduction provided we place the nodes at specific locations. Computing the best weights for our numerical quadratures guarantees optimal approximation of our integral (actually we don't present this method here).
	
		\item Monte Carlo Integration: Use random number generation and a ratio of the target random points relatively to all random generated points.  At this method employs a non-deterministic approach: each realization provides a different outcome! There are different methods to perform a Monte Carlo integration, such as uniform sampling, stratified sampling, importance sampling, sequential Monte Carlo (also known as a particle filter), and mean field particle methods (see page \pageref{monte carlo simulations}).
	\end{itemize}
	
	\subsubsection{Rectangles method}\label{rectangle integration method}
	The "\NewTerm{rectangle methods}\index{rectangle methods}" computes an approximation to a definite integral, made by finding the area of a collection of rectangles whose heights are determined by the values of the function by different approach.
	
	We divide the interval $[a,b]$ into $n$ subintervals which bounds are the $x_i$. The lengths of these subintervals are $h_i=x_{i+1}-x_i$. We build rectangles which sides are $h_i$ and in the case of the "\index{left rectangle methods}" the height is $f(x_i)$.
	
	\begin{figure}[H]
		\centering
		\includegraphics{img/computing/numerical_integration_left_inferior_rectangle_method.jpg}
		\caption{Approach of the area under a curve by lower left rectangles method}
	\end{figure}
	The area of these rectangles is:
	
	If the $h_i$ are small enough, $A_G$ is a good approximation of the sought approached area by the left method.

	We can start this exercise again by choosing $h_i$ and $f(x_{i+1})$ as sides of the rectangles (so the approach is named the "right rectangle method\index{right rectangle methods}"). We then get:
	
	The correspondent figure is therefore the following:
	\begin{figure}[H]
		\centering
		\includegraphics{img/computing/numerical_integration_right_superior_rectangle_method.jpg}
		\caption{Approach of the area under a curve by upper right rectangles method}
	\end{figure}
	Again, the area of these rectangles approaches the area searched. To simplify computer code, it is useful to choose identical length intervals:
	
	If we have $n$ rectangles, $h$ is then equal to $(b-a)/n$. The areas $A_D$ and $A_G$ become:
	
	We can also mixed the both method above as we will illustrate it further below in the figure that will summarize the for more common methods.
		
	
	\subsubsection{Trapezoidal method}\label{trapezoidal numerical integration}
	In the purpose to increase the accuracy, it is possible to calculate:
	
	In the case where all the intervals are of equal length, $A_T$ is equal to:
	
	That we often find in the academic literature, in the form:
	
	There are many other methods for solving this type of problem (including the Monte Carlo method that we will see further below).
	
	In the case where the function $f$ is not made of only positive values, we no longer speak anymore about "area" but of "\NewTerm{Riemann sum}\index{Riemann sum}". The sum to calculate are then:
	
	and:
	
	Hence to summarize we have the following four following more typical univariate integral numerical methods:
	\begin{figure}[H]
		\centering
		\includegraphics[width=1.0\textwidth]{img/computing/numerical_integration_methods.jpg}
		\caption[]{\textbf{A}: left rectangle method, \textbf{B}: right rectangle point, \textbf{C}: mid-point method, \textbf{D}: Trapezoidal method}
	\end{figure}
	
	\subsubsection{Newton–Cotes formulas}
	The "\NewTerm{Newton–Cotes formulas}\index{Newton–Cotes formulas}", also named the "\NewTerm{Newton–Cotes quadrature rules}" or simply "\NewTerm{Newton–Cotes rules}", are a group of formulas for numerical integration (also called "\NewTerm{quadrature formulas} as we have already mention it) based on evaluating the integrand at equally spaced points (hey are named after Isaac Newton and Roger Cotes) and where the weights $w_i$ of:
	
	re derived from the Lagrange basis polynomials (see page \pageref{lagrange polynomial interpolation method}).
	
	Let us see now three mainstream of the many "\NewTerm{Newton-Cotes quadrature formulas}". They are all based on Lagrange Polynomials (with or without the error term):
	
	Since the Taylor polynomials (\SeeChapter{see section Sequences and Series page \pageref{Taylor polynomial}}) have the property that all the information used in the approximation is concentrated at the single point $x_0$, it is not uncommon for these polynomials to give inaccurate approximations as we move away from $x_0$. This limits Taylor polynomial approximation to the situation in which approximations are needed only at points close to $x_0$.
	
	The more the simple Newton-Cotes methods are based on high degree polynomials, the slower they are and the more difficult they are to code, but the more precise they are. Most often in practice, the total integration domain $[a, b]$ is much too large and the function varies too much on this domain for these methods to give satisfactory results. They are therefore almost never used as such.

	The total domain $[a, b]$ is therefore subdivided into a large number of small intervals over each of which the simple Newton-Cotes methods can be successfully applied. We then speak of "\NewTerm{composite Newton-Cotes method}\index{composite Newton-Cotes method}".
	
	\paragraph{Trapezoidal Rule}\mbox{}\\\\
	We derive the Trapezoidal rule for approximating $\int_a^b f(x)\mathrm{d}x$  using the Lagrange polynomial method, with the linear Lagrange polynomial.
	
	Let $x_0 = a$, $x_1 = b$, and $h = b-a$:
	
	Thus, the Trapezoidal rule is:
	
	We fall back here on the relation that we have proved earlier but with another approach (a more analytic one...!).	
	
	Since the error term for the Trapezoidal rule involves $f''$, the rule gives the exact result when applied to any function whose second derivative is identically zero. That is, the Trapezoidal rule gives the exact result for polynomials of degree up to or equal to one.
	
	\paragraph{Composite Trapezoidal Rule}\mbox{}\\\\
	When the trapezoidal rule is applied on the subintervals it is called a composite trapezoidal rule.
	
	To see this, let us define:
	
	Then:
	
	Thus, the Composite Trapezoidal rule is:
	
	
	\paragraph{Simpson's Rule}\label{Simpson's rule}\mbox{}\\\\
	Simpson's rule can be derived by integrating the second Lagrange polynomial. However, this derivation gives only an $\mathcal{O}\left(h^{4}\right)$ error term involving $f^{(3)}$. To get a better error term, we use Taylor polynomial to derive the Simpson's rule.

	Let $x_{0}=a, x_{2}=b,$ and $x_{1}=\dfrac{a+b}{2}=a+\dfrac{b-a}{2}=a+h .$ That is, $x_{1}-x_{0}=h$ and $x_{2}-x_{1}=h$. Then:
	
	Thus, the Simpson's rule is:
	
	
	\begin{tcolorbox}[title=Remark,colframe=black,arc=10pt]
	When Simpson's rule is applied on two subintervals it is named a "\NewTerm{composite Simpson's  rule}".
	\end{tcolorbox}
	
	\subsubsection{Multidimensional integrals}
	Integrals of functions of several variables, over regions with dimension greater than one, are not easy. There are two reasons for this. First, the number of function evaluations needed to sample an $N$-dimensional space increases as the $N$th power of the number needed to do a one-dimensional integral. If you need $30$ function evaluations to do a one-dimensional integral crudely, then you will likely need on the order of $30\times 30\times 30=27000$ evaluations to reach the same crude level for a three-dimensional integral. Second, the region of integration in $N$-dimensional space is defined by an $N-1$ dimensional boundary which can itself be terribly complicated: It need not be convex or simply connected, for example. By contrast, the boundary of a one-dimensional integral consists of two numbers, its upper and lower limits.
	
	The first question to be asked, when faced with a multidimensional integral is if can it be reduced analytically to a lower dimensionality one?
	
	Alternatively, the function may have some special symmetry in the way it depends on its independent variables. If the boundary also has this symmetry, then the dimension can be reduced. In three dimensions, for example, the integration of a spherically symmetric function over a spherical region reduces, in polar coordinates, to a one-dimensional integral.
	
	The next questions to be asked will guide your choice between two entirely different approaches to doing the problem. The questions are: Is the shape of the boundary of the region of integration simple or complicated? Inside the region, is the integrand smooth and simple, or complicated, or locally strongly peaked? Does the problem require high accuracy, or does it require an answer accurate only to
a percent, or a few percent?

	If your answers are that the boundary is complicated, the integrand is not strongly peaked in very small regions, and relatively low accuracy is tolerable, then your problem is a good candidate for Monte Carlo integration. This method is very straightforward to program, in its cruder forms. One needs only to know a region with simple boundaries that includes the complicated region of integration, plus a method of determining whether a random point is inside or outside the region of integration. Monte Carlo integration evaluates the function at a random sample of points, and estimates its integral based on that random sample.

	If the boundary is simple, and the function is very smooth, then the remaining approaches, breaking up the problem into repeated one-dimensional integrals, or multidimensional Gaussian quadratures, will be effective and relatively fast. If you require high accuracy, these approaches are in any case the only ones available to you, since Monte Carlo methods are by nature asymptotically slow to converge.
	
	For low accuracy, use repeated one-dimensional integration or  
Gaussian quadratures when the integrand is slowly varying and smooth in the region of integration, Monte Carlo when the integrand is oscillatory or discontinuous, but not strongly peaked in small regions.

	If the integrand is strongly peaked in small regions, and you know where those regions are, break the integral up into several regions so that the integrand is smooth in each, and do each separately. If you don't know where the strongly peaked regions are, you might as well (at the level of sophistication of this book) quit: It is hopeless to expect an integration routine to search out unknown pockets of large contribution in a huge $N$-dimensional space.
	
	If, on the basis of the above guidelines, we decide to pursue the repeated one-dimensional integration approach, here is how it works (the text below about the repeated one-dimensions approach is a bit a small summary of multivariate integral that we have already study in-deep in the section of Differential and Integral Calculus page \pageref{double integral}). For definiteness, we will consider the case of a three-dimensional integral in $x$, $y$, $z$-space. Two dimensions, or more than three dimensions, are entirely analogous.

	The first step is to specify the region of integration by:
	\begin{enumerate}
		\item Its lower and upper limits in $x$, which we will denote $x_1$ and $x_2$
		
		\item Its lower and upper limits in $y$ at a specified value of $x$, denoted $y_1(x)$ and $y_2(x)$
		
		\item Its lower and upper limits in $z$ at specified $x$ and $y$, denoted $z_1(x, y)$ and $z_2(x, y)$. 
	\end{enumerate}
	In other words, find the numbers $x_1$ and $x_2$, and the functions $y_1(x)$, $y_2(x)$, $z_1(x, y)$, and $z_2(x, y)$ such that:
	
	Now we can define a function $G(x, y)$ that does the innermost integral:
	
	and a function $H(x)$ that does the integral of $G(x,y)$:
	
	and finally our answer as an integral over $H(x)$:
	
	The methods that we have introduced earlier to calculate simple integrals can be generalized to multiple integrals. Les us consider first the following integral:
	
	where $\mathcal{R}$ is a rectangular area of the plane, that is of the type:
	
	the limits of integration $a,b,c$ and $d$ being (for the moment...) constants.
	
	To determine an approximated value of this integral, let us write it under the form:
	
	and let us use the Simpson's method that we have proved earlier (see page \pageref{Simpson's rule}):
	
	for each of the integrals. Notice that every Newton-Cotes  can be used and that it is not necessary to use the same method for the two variables.
	
	Let us choose two integers $m$ and $n$ to subdivide the interval $[a,b]$ into $2m$ sub-intervals with the step:
	
	and $[c,d]$ into $2n$ sub-intervals with the step:
	
	That is $y_0=c$, $y_1=c+k$, $y_{2n}=d$.
	
	The first step has for purpose to calculate the integral:
	
	with the Simpson's method by keeping $x$ constant! We put $y_j=c+jk$, with $j=0,1,\ldots,2n$, therefore $y_0=c$ and $y_{2n}=d$. We then get (without the error term):
	
	and for the double integral:
	
	The Simpson's method can now be used to calculate each integral that leads on the variables $x$. Let us put for this purpose $x_i=a+ih$ with $i=0,1,\ldots,2m$, therefore $x_0=a$ and $x_{2m}=b$. For each value of the index $j$ ($j=0,1,2,\ldots,2n$), that is by $y=y_j$, we then get without error term:
	
	By substituting the latter in the first integral, we get:
	
	This can be written also in matrix form:
	
	The use of numerical approximation methods for double integrals is not limited to rectangular areas. The integrals of the shape:
	
	can be computed by modifying a bit the previous described method.
	
	As example, let us use the Simpson's method to determine an approximation of the following double integral:
	
	in the simplest case with $m=n=1$. Therefore the integration step of the variables $x$ is then:
	
	when that of the step of $y$ varies with $x$:
	
	The estimation of the integral then takes the following shape (without the error term):
	
	
	\pagebreak
	\subsubsection{Numerical solution of ordinary differential equations}
	Numerical integration of ordinary differential equations is a frequent task of numerical analysis. Numerical integration of differential equations is used if the equations are non-linear or if we have a large system of linear equations with constant coefficients, where the analytical solution can be found, but it is in the form of long and complicated expressions containing exponential functions. Numerical integration of such systems is more efficient both in human time and in computer time. 

	Numerical integration of linear equations with non-constant coefficients is also more efficient than the analytical solution; in the case of inner diffusion in porous catalyst with a
chemical reaction of the first order the analytical solution contains Bessel functions, which can be evaluated more conveniently when we use numerical integration of the original equations
than to evaluate Bessel functions.

	We have already seen of to use Euler's method and the method of Taylor's expansion at the page \pageref{Euler method}, let us see more elaborated techniques.
	
	 \paragraph{Runge-Kutta methods}\index{Runge-Kutta method}\label{Runge-Kutta methods}\mbox{}\\\\
	 In numerical analysis, the Runge–Kutta methods are a family of implicit and explicit iterative methods, which include the well-known "Euler's method" (we will prove that further below), used in temporal discretization for the approximate solutions of ordinary differential equations. These methods were developed around 1900 by the German mathematicians Carl Runge and Wilhelm Kutta.
	 
	  The techniques discussed in these pages approximate the solution of first order ordinary differential equations (with initial conditions) of the form ("Cauchy's problem type):
	 
	 In the following developments you will understand why we say sometimes that Runge-Kutta methods "simulate" the Taylor's method!
	
	\subparagraph{1st order Runge-Kutta method}\mbox{}\\\\
	The first order Runge-Kutta method is already known to us. It's simply the "Euler's method" already derived earlier before.
	
	Let us review that latter introducing a notation that is really mainstream and will be useful to understand the higher Runge-Kutta order methods.
	
	For this we start again from:
	
	 and we write the approximation to the derivative as:
	 
	 We expand $y(t)$ around $t_0$ assuming a time step $h$ using Taylor expansion given for recall by:
	 
	This give us:
	 
	and drop all terms after the linear term. Because all of the dropped terms are multiplied by $h^2$ or greater, we say that the algorithm is accurate to order $h^2$ locally, or $\mathcal{O}(h^2)$ (if $h$ is small the other terms that are multiplied by $h^3$, $h^4$... which will be even smaller, and can be dropped as well):
	
	This gives us our approximate Euler's method solution at the next time step:
	
	Since the number of steps over the whole interval is proportional to $1/h$ (or $\mathcal{O}(h^{-1})$) we might expect the overall accuracy to be the $\mathcal{O}(h^2)\cdot \mathcal{O}(h^{-1})=\mathcal{O}(h)$. A rigorous analysis proves that this is true.
	
	With a little more work we can develop a method that is accurate to higher order than $\mathcal{O}(h)$. And this is the second order Runge-Kutta method.
	
	 \subparagraph{2nd order Runge-Kutta method}\mbox{}\\\\
	 In the following derivation we will use two math facts that are reviewed here. You should be familiar with this from a course in multivariate calculus.

	First, let us recall the relation we have derived in the section of Sequences and Series on bivariate Taylor series:
	
	Hence:
	
	If $f$ function of two variables $f(x,y)$, where $x=r(t)$ and $y=s(t)$, then by the chain rule for partial derivatives:
	
	In particular if:
	
	then:
	
	To start, recall that we are expressing our differential equation as
	 
	 Now define two approximations to the derivative:
	 
	 In all cases $\alpha$ and $\beta$ will represent fractional values between $0$ and $1$. These equation state that $k_1$ is the approximation to the derivative based on the estimated value of $y(t)$ at $t=t_0$ (i.e., $y^*(t_0)$) and the time at $t_0$. The value of $k_2$ is based upon the estimated value, $y^*(t_0)$, plus some fraction of the step size, $\beta h$, times the slope $k_1$, and the time at $t_0+\alpha h$ (i.e., some time between $t_0$ and $t_0+h$).
	 
	 To update our solution with the next estimate of $y(t)$ at $t_0+h$ we use then the relation:
	  
	 That latter relation can be found in some textbooks under the form:
	 
	 This equation states that we get the value of $y^*(t_0+h)$ from the value of $y^*(t_0)$ plus the time step, $h$, multiplied by a slope that is a weighted sum of $k_1$ and $k_2$. In the method described previously $a=0$ and $b=1$, so we used only the second estimate for the slope. 
	 
	 Note that Euler's Method (first Order Runge-Kutta) is a special case of this method with $a=1$, $b=0$, and $\alpha$ and $\beta$ don't matter because $k_2$ is not used in the update equation!
	 \begin{tcolorbox}[title=Remark,colframe=black,arc=10pt]
	In general a Runge-Kutta method of order $s$ will be written as:
	
	 where:
	 
	\end{tcolorbox}
	
	Our goal now is to determine, from first principles, how to find the values $a$, $b$, $\alpha$ and $\beta$ that result in low error. Starting with the update equation above:
	 
	We can use now the bivariate-dimensional Taylor Series (where the increment in the first dimension is $\beta hk_1$, and the increment in the second dimension is $\alpha h$) to expand the rightmost term:
	 
	 In the last line we used the fact the $k_1=f$. Now we substitute this in the update equation:
	 
	 To finish we compare this approximation with the expression for a Taylor Expansion of the exact solution (going from the first line to the second we used the chain rule for partial derivatives):
	 
	 Comparing this expression with our final expression for the approximation:
	 
	 we see that they agree up to the error terms (third order and higher) if we define the constants, $a$, $b$, $\alpha$ and $\beta$ such that:
	 
	 Also written:
	 
	 This system is underspecified, there are four unknowns, and only three equations, so more than one solution is possible. 
	 
	 The following choices are the most common one:
	 \begin{itemize}
	 	\item $a=0$, $b=1$ and $\alpha=\beta=1/2$, we get:
	 	
	 	This is named the "\NewTerm{middle-point method}".
	 
		\item $a=b=1/2$ and $\alpha=\beta=1$, we get:
	 	
		This is named the "\NewTerm{improved Euler's method}" or "\NewTerm{modified Euler's method}". Notice that the term:
		
		is equivalent to the trapezoidal approximation of an integral of $y^*(y(t),t)$.
		
		\item $a=1/4$, $b=3/4$ and $\alpha=\beta=2/3$, we get:
		
		This is named the "\NewTerm{Heun's method}".
	 \end{itemize}
	More complicated and more accurate methods can be derived by a similar approach and this can be the subject of a whole book.
	
	\subparagraph{4th order Runge-Kutta method}\mbox{}\\\\
	It is the most widely known member of the Runge–Kutta family and generally referred to as "\NewTerm{RK4}\index{RK4}", the "\NewTerm{classic Runge–Kutta method}\index{classic Runge–Kutta method}" or simply as the "\NewTerm{Runge–Kutta method}\index{Runge–Kutta method}".
	
	To derive it, let us recall that general a Runge–Kutta (empirical) method of order $s$ can be written as:
	
	where:
	
	are increments obtained evaluating the derivatives of $y_{t}$ at the $i$-th order.
	
	We develop the derivation for the Runge–Kutta fourth-order method using the general formula with $s=4$ evaluated, at the starting point, the midpoint and the end point of any interval $(t,\ t+h)$. Thus, we choose:
	
	and $\beta _{ij}=0$ otherwise. We begin by defining the following quantities:
	
	where:
	
	If we define:
	
	and for the previous relations we can show that the following equalities hold up to $\mathcal {O}(h^{2})$:
	
	where:
	
	is the total exact differential of $f$ with respect to time.
	
	If we now express the general formula using what we just derived we get:
	
	and comparing this with the Taylor series of $y_{t+h}$ around $y_{t}$:
	
	we get a system of constraints on the coefficients:
	
	which when solved gives:
	
	as stated above.
	
	\pagebreak
	\subsection{Optimization}\label{operational research}
	In mathematics, computer science and operations research, mathematical optimization (alternatively "\NewTerm{mathematical programming}\index{mathematical programming}") is the selection of a best element (with regard to some criteria) from some set of available alternatives.
	
	In the simplest case, an optimization problem consists of maximizing or minimizing a real function by systematically choosing input values from within an allowed set and computing the value of the function. The generalization of optimization theory and techniques to other formulations comprises a large area of applied mathematics. More generally, optimization includes finding best available values of some objective function given a defined domain (or a set of constraints), including a variety of different types of objective functions and different types of domains.
	
	In mathematics, conventional optimization problems are usually stated in terms of minimization (or changed to be as!). A large number of algorithms proposed for solving optimization problems are not capable of making a distinction between local optimal solutions and rigorous optimal solutions. The branch of applied mathematics and numerical analysis that is concerned with the development of deterministic algorithms that are capable of guaranteeing convergence in finite time to the actual optimal solution is named "\NewTerm{global optimization}\index{global optimization}".
	\begin{tcolorbox}[title=Remark,colframe=black,arc=10pt]
	We speak of "\NewTerm{convex optimization}\index{convex optimization}" when any local minimum must be a global minimum. In other word there is only one unique solution.
	\end{tcolorbox}
	In the context of problem solving which involved two variables and their products, we then speak logically "\NewTerm{quadratic programming (QP)}\index{quadratic programming}" or simply "\NewTerm{non-linear programming}\index{non-linear programming}". This is typically the case in financial engineering in portfolios modelling (\SeeChapter{see section Economy page \pageref{markowitz overall minimum variance portfolio}}) or in forecasting. We will study in details also further below simplified and particular version of the corresponding models that are the: Newton's method, quasi-Newton's method, conjugate gradient method and non-linear GRG.
	
	\pagebreak
	\subsubsection{Linear programming (Linear Optimization)}\label{linear programming}
	The objective of the linear programming (LP-programming) is to find the optimum value of a linear function subject to a system of equations consisting in inequalities constraints that are also linear. The objective function is named "\NewTerm{economic function}\index{economic function}" (because used a lot in Economy) and we solve this type of system using typically, among others, a method named "\NewTerm{simplex method}\index{simplex method}\label{simplex method}" (see below), the corresponding graph is a "\NewTerm{polygon constraints}\index{polygon constraints}" (when the number of variable is obviously equal to $2$).
	
	The reader can remember the following diagram that we saw in the section Calculus:	
	\begin{center}
	\begin{tikzpicture}[scale=2]
    \draw[gray!50, thin, step=0.5] (-1,-3) grid (5,4);
    \draw[very thick,->] (-1,0) -- (5.2,0) node[right] {$x_1$};
    \draw[very thick,->] (0,-3) -- (0,4.2) node[above] {$x_2$};

    \foreach \x in {-1,...,5} \draw (\x,0.05) -- (\x,-0.05) node[below] {\tiny\x};
    \foreach \y in {-3,...,4} \draw (-0.05,\y) -- (0.05,\y) node[right] {\tiny\y};

    \fill[blue!50!cyan,opacity=0.3] (8/3,1/3) -- (1,2) -- (13/3,11/3) -- cycle;

    \draw (-1,4) -- node[below,sloped] {\tiny$x_1+x_2\geq3$} (5,-2);
    \draw (1,-3) -- (3,1) -- node[below left,sloped] {\tiny$2x_1-x_2\leq5$} (4.5,4);
    \draw (-1,1) -- node[above,sloped] {\tiny$-x_1+2x_2\leq3$} (5,4);

	\end{tikzpicture}
	\end{center}
	Corresponding to the following system of inequalities:
	
	\begin{tcolorbox}[title=Remark,colframe=black,arc=10pt]
	Linear programming is widely used (to name only the most famous case) in Logistics (maximal flow problem also named "\NewTerm{transport problem}\index{transport problem}"), in corporate finance or also in decision theory when we solve a mixed strategy game (see the section of Game and Decision Theory  for a practical example). That's why Microsoft Excel 12.0 and earlier includes a tool named the "solver" in which there is an option named "Assume Linear Model" which then requires the use of the simplex model that we will study below:
	\begin{figure}[H]
		\centering
		\includegraphics{img/computing/excel_solver_lp.jpg}
	\end{figure}
	or since the 2010 version of the software (the user interface has completely changed):
	\begin{figure}[H]
		\centering
		\includegraphics[scale=0.8]{img/computing/excel_solver_lp_2010.jpg}
	\end{figure}
	\end{tcolorbox}
	We will focus in particular in the section on the most widely used algorithm for linear optimization named the "\NewTerm{simplex algorithm}\index{simplex algorithm}\label{simplex algorithm}".
	
	When a problem can be modelled as an economic function to be maximized with respect to certain constraints that are purely additive, so are typically in the context of linear programming.
	
	So an economic function $Z$ as:
	
	where the $x_i$ are variables that affect the value of $Z$, and the $c_i$ the weights of these variables modelling the relative importance of each of them on the value of the economic function.
	
	The constraints  related to the variables are expressed by the following linear system:
	
	Under general and matrix form this problem is written as:
	
	
	To see the different method of resolution let us use an example as theoretical introduction:
	
	A factory produces two types of pieces $P1$ and $P2$ machined in two workshops $A1$ and $A2$. Machining times are for $P1$ or $3$ hours in the workshop $A1$ and $6$ hours in the workshop $A2$ and of for $4$ hours for $P2$ in the workshop $A1$ and $3$ hours in the workshop $A2$.
	
	Weekly up-time of human resources (workers) of the $A1$ workshop is $160$ hours and that of the workshop $A2$ is $180$ hours.
	
	The profit margin is of $1,200.-$ for the pieces $P1$ and $1,000.-$ for pieces $P2$.

	The question is how much of each kind of piece should we make to maximize weekly margin?
	
	This will be formalized as follows (canonical formulation):
	
	
	\paragraph{Graphical LP resolution}\mbox{}\\\\
	The graphical method is will adapted for problem with $2$ or $3$ variables but not more as our perception of hyper-volume is quite limited for humans.
		
	When translate our optimization problem into graphical form, we speak also of "\NewTerm{polygon of constraints}\index{polygon of constraints}". Indeed, the economical constraints are represented by half planes. The solutions, if they exists, belongs to the intersection set name "\NewTerm{set of admissible solutions}\index{set of admissible solutions}" and is quite trivially represented in our case by:	
	\begin{figure}[H]
		\centering
		\includegraphics{img/computing/lp_graph.jpg}
		\caption{Illustration of a simple operational research problem with area of feasible solutions}
	\end{figure}
	\begin{tcolorbox}[title=Remark,colframe=black,arc=10pt]
	In the general case, for those who love the language of mathematicians ..., the information of a linear constraint geometrically corresponds to a half a space of $n$-dimensional space ($n$ being the number of variables). In the elementary case, all the points in space that satisfy all constraints is limited by convex portions of hyperplane (see the case with $2$ variables, easy to illustrated), that is why this is named also "\NewTerm{convex optimization}\index{convex optimization}". If the cost function is linear, the extreme point is a vertices (easy to see). The basic algorithm simplex algorithm (see further below) start from one vertices and goes to the next vertices which locally maximizes the cost, and restarts the procedure as long as necessary.
	\end{tcolorbox}	
	To find the coordinates of the vertices, we can use the graph if the points are easy to determine.

	It is therefore to seek inside this area (connex), the pair  $(x_1,x_2)$ maximizing the economic function.

	However, the equation $Z$ is represented by a constant line of constant slope ($-1.2$) which all points $(x_1,x_2)$ provide the same $Z$ value for the objective function.
	
	In particular, the straight line $1200x_1+1000x_2$ pass trough the origin and it provides a zero value to the economic function. To increase the value of $Z$ and therefore of the economic function, we have to take away from the origin (in the quarter $x_1>geq 0,x_2\geq 0$ that is to say in the \texttt{I}st quadrant) the line of slope $-1.2$. Obviously then we see very quickly that the simplex method will not work if the constraints of the polygon does not contain the origin point!
	
	To meet the constraints, this straight line will be moved until the limit where it will not have a point of intersection anymore in common with the aree of admissible solutions (eventually a segment).
	\begin{figure}[H]
		\centering
		\includegraphics{img/computing/lp_graph_detailed.jpg}
		\caption{Finding solutions graphically with the economic function}
	\end{figure}
	The optimal solution is therefore necessarily located on the periphery of the region of admissible solutions and the parallel formed by translating the economic function are named "\NewTerm{isoquants lines}\index{isoquants lines}" or "\NewTerm{isocost lines}\index{isocost lines}"...
		
	\paragraph{Algebraic LP resolution}\mbox{}\\\\	
	Let us now see how to solve this problem analytically before moving to the theoretical part.
	
	So we have the "\NewTerm{canonical system}\index{canonical system}":
	
	with:
	
	We first introduce the "\NewTerm{slack variables}\index{slack variables}" to transform the $2$ inequalities in equalities. The system of equations takes then "\NewTerm{standard form}\index{standard form}":
	
	Therefore, for $x_1,x_2\geq 0$ fixed, the slack variable whose coefficients are always unit, measure the distance to travel to reach the vertices.

	It goes without saying that the technique of slack variables may be used for linear (or non-linear) systems. Therefore, an constraint optimization system with inequalities, can always be reduced to an optimization system with equalities.
	\begin{tcolorbox}[title=Remark,colframe=black,arc=10pt]
	Obviously there is as much slack variable as inequalities.
	\end{tcolorbox}
	For the remaining part, we have noticed, after a review of this section, that the technique using tables (that we will see later) often presented in books and websites finally brought nothing to a deep understanding of the resolution mechanism (even if to program the computational method that is most convenient). Since the purpose of this book is to always prove with a maximum detail the operating principle of things so it goes without saying that we will opt for a first purely algebraic approach. Let us see it by returning to the system with the slack variables and the economic function but slightly rearranged:
	
	The $A1$ constraint then becomes:
	
	and the constraint $A2$ respectively becomes:
	
	Therefore, the problem consist to maximize $Z$ with the constraints:
	
	Let us start with an obvious feasible solution given the constraints that is trivially:
	
	Therefore with the system:
	
	we find immediately:
	
	The parameters in the actual state can be summarized as:
	
	To go forward, the goal will be to make $Z$ grow and for this purpose we will increase only one single variable, choosing the one with the largest coefficient (weight) in:
	
	that is to say $x_1$ (because implicitly we think this is how the $Z$ will increase the faster). We speak then of $x_1$ as the "\NewTerm{pivot direction}\index{pivot direction}". We keep then $x_2=0$ and we increase $x_1$ with the system which then reduces to:
	
	Therefore with $x_2=0$ and to begin $x_1=1$, we have:
	
	and we see that the constraints $x_1,x_2,x_3,x_4\geq 0$ are still respected, it is the same if $x_1$ is equal to $2$, $3$, $4$, $5$, ... and this until $31$, because after:
	
	and one of the slack variable has become negative, the constraints $x_1,x_2,x_3,x_4\geq 0$ are not all met and therefore this solution is not feasible.
	
	The question in the general case is to ask ourselves until what value (the most constraint value, verbatim the smallest) we can increase $x_1$ while maintaining the condition $x_1,x_2,x_3,x_4\geq 0$ when $x_2=0$? And the answer is quite simple:
	
	and therefore it is:
	
	then we speak sometimes of the "\NewTerm{pivot step}\index{pivot step}". We then have the actual solution:
	
	Which gives:
	
	Graphically, this is what we have just do:
	\begin{figure}[H]
		\centering
		\includegraphics{img/computing/lp_graph_detailed_with_pivot.jpg}
		\caption[]{Direction of the pivot head and arrival at point $(30, 0)$}
	\end{figure}
	To continue to increase $Z$ such simply (by increasing only one variable), we need a new system of equations similar to the original system:
	
	where we had expressed the variables that takes are non-zero value depending on others who take a zero value, that is to say $x_3,x_4$, in function of $x_1,x_2$ since we had for recall:
	
	For the remaining part, we must express $x_1,x_3$ and also $Z$ in function of $x_2,x_4$ since we just get for reminder:
	
	Before getting the new system reaction function, making some algebraic manipulations:
	
	which gives after simplification:
	
	and therefore it comes:
	
	which gives after simplification:
	
	and we have identically:
	
	So finally the system is:
	
	from which we reiterate the process (we increase only one  variable in $Z$ keeping the other to $0$). When we can not increase $Z$ as all coefficients are negative, well it is that we are at a maximum (thank you convexity...). Let us see this...
	
	In $Z$ the biggest coefficient is now $x_2$ and so it leads us to put $x_4=0$. The most constraint value of $x_2$ that allows to always respect the constraints $x_1,x_2,x_3,x_4\geq 0$ is therefore:
	
	And for this value, we have:
	
	The original economic function then takes the value:
	
	which therefore corresponds graphically to:
	\begin{figure}[H]
		\centering
		\includegraphics{img/computing/lp_graph_detailed_with_pivot_second_iteration.jpg}
		\caption{Pivot direction with arrival point at $(16, 28)$ for the second iteration}
	\end{figure}
	So we see above that we arrived at the optimum value visible on the graph given at the beginning of this example. But how do we know that we arrived at the final point if we do not have plots or if we work in higher dimensions?
	
	In fact, the process is terminated either when all the coefficients of the economic function are negative or that the most constraint value that respects the constraints is equal to zero !!! Let's see if this is the case! We therefore have in our example:
	
	So we'll rewrite the system:
	
	with this time $x_1,x_2$ and $Z$ dependant to $x_3,x_4$. We then have first:
	
	and we have:
	
	Therefore for $Z$ we have (the coefficients are all negative so we guess what comes ...):
	
	We then have the new system:
	
	As all coefficients of $Z$ are now negative, we're blocked because we would go in the wrong direction if we continue. We must therefore stop here and we adopt finally the solution:
	
	The method of resolution using tables that is often presented in the literature is only useful to write the coefficients of the variables of the system a table, but the changes that we made are exactly those we just made algebraically before (but a the opposite of the tables the method we used don't hide the logic of the method).
	
	\paragraph{Simplex algorithm LP resolution}\mbox{}\\\\
	To implement the  simplex algorithm, we must write the problem in a "standard" form and introduce the concept of "base program" that is the algebraic expression corresponding to the notion of "extreme point of the polyhedron of eligible programs" presented earlier above. Indeed, we will see that the solution of a problem of the linear programming type it exists, can still be obtained with a base program. The simplex method will therefore be to find a first base program and to build a following base programs constantly improving the economic function and thus leading to the optimum (this is what we name "dynamic programming").
	
	An LP problem is said to be placed in its "\NewTerm{standard form}\index{standard form}" if it involves the search for the minimum of the objective function, the latter being subject to constraints in the form of linear equations and conditions of non-negativity of the variables, that is, say we can write it in the form earlier before:
	
	That is to say, using matrix notation:
	
	where the matrices $C:n\times 1, A:m\times n,b:m \times 1$ respectively correspond to the activity coefficients of objective function, to the technical coefficients of activities and to second members of the constraints.

	We will now see how a LP general problem can always be reduced to a standard form. As we guess it the concept of "slack variable" will be essential to perform this "reduction".

	Find the maximum of a function $f (x)$ is equivalent to find the minimum of the opposite sign function $f (x)$. Moreover, a constraint which is presented as an inequality:
	
	can be replaced by the system:
	
	where as we already know $s_i$ is the slack variable constraint such that $s_i\geq 0$.
	
	Of course, if the system is such that:
	
	can be replaced by the system:
	
	implying again to add a slack variable  and always with the constraint that $e_i\geq 0$.

	This work of putting in standard form work us to find a system of linear equations to solve system (we saw previously at the beginning of this section how to solve this kind of system with the pivot algorithm).

	The matrix $A$ representing the components of the system of equations can be, as we know, expressed in different ways depending on the chosen vector basis (\SeeChapter{see section Vector Calculus page \pageref{vector basis}}). We will introduce now the concept of "\NewTerm{canonical associated usable form}\index{canonical associated usable form}" by choosing a special base and show that this reformulation of constraint system will enable us to move towards the optimum.
	
	The matrix $A$ can, after introduction of the slack variables be decomposed into two sub-matrices $[D|B]$, one containing the initial variables $D$ and the other with the slack variables $B$ such that:
	
	\begin{tcolorbox}[title=Remark,colframe=black,arc=10pt]
	The slack variables are variable and not constant!! In a system where the variables are in quantity $n$ and the equations in quantity $m$ to have as a system where one of the equation would be written:
	
	To add a slack variable such that:
	
	where $x_{n+1}=e_i$ on each row $m$, the added slack variable being different of all variables already existing in the system. This is why we can decompose the matrix in two sub matrices.
	\end{tcolorbox}
	The columns of the matrix $B$ are obviously, by definition of the method, units columns, linearly independent. These columns form a basis of the vector space of columns of $m$ elements (or dimensions) - the number of  lines of the system. We name $B$ the "\NewTerm{base matrix}\index{base matrix}".
	
	Right now it can seem a little bit confusing. So let us continue the theory with a companion example as we have for habit to do it in this book.
	
	The text below describing the simplex algorithm has been taken to Marcel Oliver (April 12, 2012) and formalized by our own work.
	\begin{enumerate}
	\item Step 1: Write the linear programming problem in standard form
	
	Turning a problem into standard form involves the following steps.
	\begin{enumerate}
		\item Turn Maximization into minimization and write inequalities in
		standard order.
		
		This step is obvious.  Multiply expressions, where appropriate, by	$-1$.
		
		\item Introduce slack variables to turn inequality constraints into equality constraints with non-negative unknowns.
		
		Any inequality of the form
		
		can be replaced by:
		  
		with $s \geq 0$.
		
		\item Replace variables which are not sign-constrained by differences.
		
		Any real number $x$ can be written as the difference of non-negative
		numbers $x=u-v$ with $u,v\geq 0$.
	\end{enumerate}

	Consider the following example.
	
	subject to 
	
	Written in standard form, the problem becomes:
	
	subject to 
	

	\item Step 2: Write the coefficients of the problem into a "\NewTerm{simplex tableau}\index{simplex tableau}".
	
	The coefficients of the linear system are collected in an augmented	matrix as known from Gaussian elimination for systems of linear equations; the coefficients of the objective function are written in a separate bottom row with a zero in the right hand column.
	
	For our example, the initial tableau reads:
	\newcolumntype{B}{%
	  >{\columncolor[gray]{.8}[.5\tabcolsep]}c}
	\begin{center}
	\begin{tabular}{BcccBBB|c}
	  $x_1$ & $x_2$ & $u$ & $v$ & $s_1$ & $s_2$ & $s_3$ \\
	  \hline
	  $1$ & $1$ & $-1$ & $1$ & $0$ & $0$ & $0$ & $1$ \\
	  $2$ & $-1$ & $-2$ & $2$ & $1$ & $0$ & $0$ & $5$ \\
	  $1$ & $-1$ & $0$ & $0$ & $0$ & $1$ & $0$ & $4$ \\
	  $0$ & $1$ & $1$ & $-1$ & $0$ & $0$ & $1$ & $5$ \\
	  \hline
	  $-1$ & $-2$ & $-3$ & $3$ & $0$ & $0$ & $0$ & $0$ 
	\end{tabular}
	\end{center}
	So following what we have above, we get for:
	
	that:
	
	The variables associated to the column components of the matrix $S$ will now be named "\NewTerm{bases variables}\index{bases variables}". In our case the bases variables are then essentially the slack variables that we will write now $x_{n+1},x_{n+2},\ldots,x_{n+m}$. The variables associates to the column of the matrix $X$ are named "\NewTerm{off-base variables}\index{off-base variables}", these are the variables $x_1,x_2,\ldots,x_n$.
	\begin{tcolorbox}[title=Remark,colframe=black,arc=10pt]
	Let us recall that in the expression of the economic function, only the off-base variables appear.
	\end{tcolorbox}
	In the following steps, we will act on the tableau by the rules of Gaussian elimination, where the pivots are always chosen from the columns corresponding to the bases variables.
	
	Before proceeding, we need to choose an initial set of basic variables	which corresponds to a point in the feasible region of the linear programming problem.  Such a choice may be non-obvious, but we shall defer this discussion for now.  In our example, $x_1$ and $s_1, \dots, s_3$ shall be chosen as the initial bases variables, indicated by gray columns in the tableau above.
	
	\item Step 3: Gaussian elimination

	For a given set of basic variables, we use Gaussian elimination (see page \pageref{Gaussian elimination}) to reduce the corresponding columns to a permutation of the identity matrix.  This amounts to solving $A\vec{x}=\vec{b}$ in such a way that the values of the non-basic variables are zero and the values for the basic variables are explicitly given by the entries in the right hand column of the fully reduced matrix. In addition, we eliminate the 	coefficients of the objective function below each pivot.
	
	Our initial tableau is thus reduced to:
	\begin{center}
	\begin{tabular}{BcccBBB|c}
	  $x_1$ & $x_2$ & $\pmb{u}$ & $v$ & $s_1$ & $s_2$ & $s_3$ \\
	  \hline
	  $1$ & $1$ & $\pmb{-1}$ & $1$ & $0$ & $0$ & $0$ & $1$ \\
	  $0$ & $-3$ & $\pmb{0}$ & $0$ & $1$ & $0$ & $0$ & $3$ \\
	  \textit{0} & \textit{-2} & \textbf{\textit{1}} & 
	   $-$ \textit{1} & \textit{0} & \textit{1} & \textit{0} & \textit{3} \\
	  $0$ & $1$ & $\pmb{1}$ & $-1$ & $0$ & $0$ & $1$ & $5$ \\
	  \hline
	  $0$ & $-1$ & $\pmb{-4}$ & $4$ & $0$ & $0$ & $0$ & $1$ 
	\end{tabular}
	\end{center}
	The solution expressed by the tableau is only admissible if all basic variables are non-negative, i.e., if the right hand column of the reduced tableau is free of negative entries.  This is the case in this example.  At the initial stage, however, negative entries may come up;	this indicates that different initial basic variables should have been chosen.  At later stages in the process, the selection rules for the	basic variables will guarantee that an initially feasible tableau will remain feasible throughout the process.

	\item Step 4: Choose new basic variables

	If, at this stage, the objective function row has at least one negative entry, the cost can be lowered by making the corresponding variable basic.  This new basic variable is named the "\NewTerm{entering variable}\index{entering variable}". Correspondingly, one formerly basic variable has then to become non-basic, this variable is named the "\NewTerm{leaving variable}\index{leaving variable}". We use the following standard selection rules.

	\begin{enumerate}
		\item The entering variable shall correspond to the column which has the most negative entry in the cost function row.  If all 	cost function coefficients are non-negative, the cost cannot be lowered and we have reached an optimum. The algorithm then terminates.
		
		\item Once the entering variable is determined, the leaving variable shall be chosen as follows.  Compute for each row the ratio of its right hand coefficient to the corresponding coefficient in the entering variable column. Select the row with the smallest finite
		positive ratio.  The leaving variable is then determined by the column which currently owns the pivot in this row.  If all coefficients in the entering variable column are non-positive, the cost can be lowered	indefinitely, i.e., the linear programming problem does not have a
		finite solution.  The algorithm then also terminates.
	\end{enumerate}
	If entering and leaving variable can be found, go to Step~3 and	iterate.
	
	Note that choosing the most negative coefficient in rule (i) is only a heuristic for choosing a direction of fast decrease of the objective function.  Rule (ii) ensures that the new set of basic variables remains feasible.  
	
	Let us see how this applies to our problem.  The previous tableau holds the most negative cost function coefficient in column $3$, thus $u$ shall be the entering variable (marked in boldface).  The smallest positive ratio of right hand column to entering variable column is in row $3$, as $\tfrac31<\tfrac51$.  The pivot in this row	points to $s_2$ as the leaving variable.  Thus, after going through the Gaussian elimination once more, we arrive at
	\begin{center}
	\begin{tabular}{BcBcBcB|c}
	  $x_1$ & $\pmb{x_2}$ & $u$ & $v$ & $s_1$ & $s_2$ & $s_3$ \\
	  \hline
	  $1$ & $-$\textbf{1} & $0$ & $0$ & $0$ & $1$ & $0$ & $4$ \\
	  $0$ & $-$\textbf{3} & $0$ & $0$ & $1$ & $0$ & $0$ & $3$ \\
	  $0$ & $-$\textbf{2} & $1$ & $-1$ & $0$ & $1$ & $0$ & $3$ \\
	  \textit{0} & \textit{\textbf{3}} & \textit{\textbf{0}} & \textit{0} 
	  & \textit{0} & $-$\textit{1} & \textit{1} & \textit{2} \\
	  \hline
	  $0$ & $-$\textbf{9} & $0$ & $0$ & $0$ & $4$ & $0$ & $13$ 
	\end{tabular}
	\end{center}
	At this point, the new entering variable is $x_2$ corresponding to the
	only negative entry in the last row, the leaving variable is $s_3$.
	After Gaussian elimination, we find
	\begin{center}
	\begin{tabular}{BBBcBcc|c}
	  $x_1$ & $x_2$ & $u$ & $v$ & $s_1$ & $s_2$ & $s_3$ \\
	  \hline
	  $1$ & $0$ & $0$ & $0$ & $0$ & $\tfrac23$ & $\tfrac13$ & $\tfrac{14}3$ \\
	  $0$ & $0$ & $0$ & $0$ & $1$ & $-1$ & $1$ & $5$ \\
	  $0$ & $0$ & $1$ & $-1$ & $0$ & $\tfrac13$ & $\tfrac23$ & $\tfrac{13}3$ \\
	  $0$ & $1$ & $0$ & $0$ & $0$ & $-\tfrac13$ & $\tfrac13$ & $\tfrac23$ \\
	  \hline
	  $0$ & $0$ & $0$ & $0$ & $0$ & $1$ & $3$ & $19$ 
	\end{tabular}
	\end{center}
	Since there is no more negative entry in the last row, the cost cannot
	be lowered by choosing a different set of basic variables; the
	termination condition applies.

	\item Step 5: Read off the solution

	The solution represented by the final tableau has all non-basic
	variables set to zero, while the values for the basic variables can be
	can be read off the right hand column.  The bottom right corner gives
	the negative of the objective function.
	
	In our example, the solution reads $x_1=\tfrac{14}3$, $x_2=\tfrac23$,
	$x_3=u=\tfrac{13}3$, $s_1=5$, $v=s_2=s_3=0$, which corresponds to
	$\zeta=-19$, which can be independently checked by plugging the
	solution back into the objective function.
	
	As a further check, we note that the solution must satisfy the initial equation and inequations.  This can obviously be checked by direct computation.
	\end{enumerate}
	In summary, any LP once put in standard form is such that:
	\begin{itemize}
		\item There is a square sub-matrix of matrix $A$, which is named the "base matrix" and is equal to the square unit matrix $\mathds{1}$ of size $m$ (indeed there are as many slack variables that lines in the original equations - at the number of $m$ - and as many columns as each slack variable has a different index).

		\item The basic variables involved does not appear in the expression of the economic function.

		\item The second member of the constraints consists of non-negative values.
	\end{itemize}
	We say then that the problem is put under a "a canonical form associated with the basis $B$, corresponding to the basis variables $x_{n+1},x_{n+2},\ldots,x_{n+m}$".
	
	\subsubsection{Nonlinear programming (Nonlinear optimization)}\label{nonlinear optimization}
	Nonlinear programming is the process of solving an optimization problem defined by a system of equalities and inequalities, collectively termed constraints, over a set of unknown real variables, along with an objective function to be maximized or minimized, where some of the constraints or the objective function are non-linear. 
	
	A non-linear optimization program (NLOP) is a generalization of linear programming (simplex algorithm) but about non-linear functions and can also include non-linear constraints and non-linear economic functions.

	The purpose of what follows is to understand in outline but with an acceptable level of rigour the optimization tools that offer many spreadsheets softwares like the previous versions of Microsoft Excel to the version 2007 (since the version 2007 we cannot make a fine-tuning of these options anymore):
	\begin{figure}[H]
		\centering
		\includegraphics{img/computing/excel_solver_nonlinear_optimization.jpg}
		\caption{Microsoft Excel 2003 Solver Options}
	\end{figure}
	 We will especially see now in what consist the \textit{Newton} Search (meaning implicitly: "Gauss-Newton's method") with the \textit{Tangent} and \textit{Quadratic} estimates. After which we will study also the Conjugate Gradients Search also with the tangent and quadratic methods respectively.
	 \begin{tcolorbox}[title=Remark,colframe=black,arc=10pt]
	We will stop at the study of the above cited models because there is an excessive quantity of empirical models such as for example the best known models (algorithms): substitution method, method of Lagrange multipliers, Nelder-Mead algorithm, Broyden-Fletcher-Goldfarb-Shanno (BFGS) algorithm, algorithm  of simultaneous annihilation (SA), methods of interior points... and see Wikipedia for a more complete list (there are over a dozen of methods without taking into account the variations including empirical adjustments).
	\end{tcolorbox}
	We will see it further below, but we already guess that the choice \textit{Tangent} use a linear approximation (tangent) of the function to be optimized at the point considered when at the opposite the \textit{Quadratic} option will make an estimation of a function of the second degree at the considered point (typically a parabola). If at the considered point, the function is well modelled by a quadric, then the \textit{Quadratic} option can save time by choosing a better starting point that will require fewer steps on each additional research. If you have no idea of the behaviour of a priori function, then the  \textit{Tangent} option is slower but safer.
	
	A well known example in the literature to introduce the search of optimums of non-linear functions, before moving on to the part taking into account constraints on the system, is the "humpback whale function" of that consist to find the minimum of:
	
	With the range constraints:
	
	what we can indeed check visually:
	\begin{figure}[H]
		\centering
		\includegraphics{img/computing/humpback_whale_function.jpg}
		\caption{Plot of the humpback whale function with minima already visible}
	\end{figure}
	Either with Maple 4.00b:
	
	\texttt{>plot3d(x\string^2*(4-2.1*x\string^2+1/3*x\string^4)+x*y+y\string^2*(-4+4*y\string^2),\\
 x=-2..2,y=-1..1,contours=20,style=patchcontour,axes=boxed);}

	That gives:
	\begin{figure}[H]
		\centering
		\includegraphics[scale=0.65]{img/computing/humpback_whale_function_maple.jpg}
	\end{figure}

	As we can see, this function is a great example of multiple local minimum but there is also a must more vicious one that we will refer to when we will study evolutionary algorithms, the "Rastgrini's function":
	
	\texttt{>plot3d(20+x\string^2+y\string^2-10*(cos(2*Pi*x)+cos(2*Pi*y)),\\ 	x=-5..5,y=-5..5,contours=20,style=patch,axes=boxed,numpoints=10000);}
	
	That gives:
	\begin{figure}[H]
		\centering
		\includegraphics[scale=0.65]{img/computing/rastgrini_function_maple.jpg}
	\end{figure}
	
	\paragraph{Substitution Method}\mbox{}\\\\
	The least complex method for solving a non-linear programming problem is named the "\NewTerm{substitution method}\index{substitution method}".

	This method is restricted to models containing a single constraint and must be in addition to the equality type.

	Let us consider a companion example by considering the following economic function to maximize:
	
 	with the constraint (remember that it must be an equality !!!):
	
 	The first step is then to arbitrarily substitute:
	
 	In the economic function to get:
	
 	This therefore brings us back to a function to be maximized which is unconstrained, so we can differentiate it and put it as zero. Then it comes:
	
	that gives:
	
	From this we can deduce immediately that
	
 	By injecting those two values in the economic function, we then have:
	
	We see by this example very quickly the limitations of this technique. The first being that the economic function was not too complex, the final equation to solve was therefore not a problem, the second being that there were only two variables, the third being that the constraint must be an equality as we have already mentioned.
	
	
	\paragraph{Lagrange Multipliers Method}\mbox{}\\\\	
	The "\NewTerm{Lagrange multipliers method}\index{Lagrange multipliers method}\label{Lagrange multipliers method}" is a rather general technique for solving non-linear programming problems with one or more constraints with linear or non linear economic function with inequalities (rather than strict equations only) and with more than two variables (for Other examples than those presented here the reader can go to the corresponding page of Wikipedia).

	We will begin to present this technique by a simple case which consists simply of taking again the example used during our study of the method of substitution:
	
	The first step is to write the constraint function in Lagrangian form as we do in Analytical Mechanics (\SeeChapter{see section  Analytical Mechanics page \pageref{euler lagrange}}) at the difference that there are no general variables depending on the time:
	
	For this, we write first:
	
	Then the idea is that since this expression is null, nothing prevents us from summing it or subtracting it from the economic function with why not an empirical multiplier that we will denote $\lambda$ and which we will name the "\NewTerm{Lagrange multiplier}".

	It comes then if we choose to subtract for example (in fact the choice of the subtraction is made by anticipation of an interpretation of the Lagrange multiplier that we will see immediately after):
	
	Which gives us our Lagrangian function. In generic form the latter is often written as following:
	
 	Now the idea is to determine the values of the variables where the partial derivatives of the Lagrangian with respect to the variables vanish at the same time (corresponding in Analytical Mechanics as the sum of all the Lagrangians relatively to one variable to be equal to zero):
	
 	What is generally written as:
	
	We therefore have a system of three equations with three unknowns which we know trivially how to solve (\SeeChapter{see section Linear Algebra page \pageref{linear systems}}). We then get as solutions:
	
 	Notice that if we had not taken into consideration $\lambda$ and therefore that implicitly that latter had been equal to $1$ since the beginning, we would have had the following system of equations:
	
	and therefore we would not have obtained the results of the substitution method seen previously ... hence the multiplicative factor!

	The final value of $Z$ is then the same as for the substitution method taking into account $\lambda$!!!

	Although the Lagrange multiplier method is powerful, its increasing complexity with a large number of variables makes it a difficult tool to manipulate in practice.

	We will now focus on the interpretation of $\lambda$. As we shall see, the latter represents the local variation rate per unit of positive variation of the constant of the constraint function. Thus, if the constraint function becomes (we have changed the $40$ into a $41$):
	
	By doing again the same calculations as before, we then get
	
 	Therefore, by having increased of one unit the constant of the constrained function, we have:
	
 	In general, if the Lagrange multiplier is positive, then the economic function will increase if the constant of the constraint is also incremented positively and vice versa.

	Let us now consider a case much more elaborate and useful for our study of Economics (especial Portfolio Optimization). We want to solve by the method of the Lagrange Multipliers the Markowitz portfolio problem (\SeeChapter{see section Economy page \pageref{markowitz overall minimum variance portfolio}}):
	
	with for recall:
	
	To facilitate the developments that will follow we will write this system with the following notation (for the details on the notations see the section Economy):
	
	The Lagrangian function will therefore be written:
	 
	Therefore:
	 
 	Now we calculate:
	 
	This gives us the system of three equations:
	  
	By rearranging the first equation:
	 
 	We have:
	 
 	Let us now take the two equations:
	
	Thus in an equivalent way:
	
	By injecting in it the explicit relation of the weights of the portfolio, we have:
	
	We can put this system in matrix form:
	
	Which gives us:
	
	Therefore:
	
	So once we have the values of these two Lagrange multipliers, we just have to inject them into:
	
 	To have weights that minimize the portfolio variance:
	
	So the reader will see during our study of Modern Portfolio management that getting the optimal weights is much more easy using the Lagrange multiplied method (and less time consuming) than using a software optimizer!
	\begin{tcolorbox}[title=Remark,colframe=black,arc=10pt]
	By the way the attentive reader will perhaps have noticed that finally making all the development with the factor $1/2$ is useless since during the final substitution, the latter is neutralized with itself since $2$ is multiplied by $1/2$.
	\end{tcolorbox}
	
	\pagebreak
	\paragraph{Newton-Raphson Method (Quadratic Newton)}\label{quadratic programming}\mbox{}\\\\
	The "\NewTerm{Newton-Raphson method}\index{Newton-Raphson method}\label{newton raphson method}", also named "\NewTerm{Newton's method for unconstrained optimization}", is a technique for searching the extremum of a function or also, as we will see it when we will compare with a special example the difference between the Gauss-Newton's method with that of Newton, for non-linear regression.

	The Newton-Raphson, who in earlier versions to Microsoft Excel 2007 was activated in the solver by selecting Newton and Quadratic option uses the second order Taylor approximation (ie with second order derivatives) to have a quadratic function (parabola) which converges if the origin point of the research is close to the optimum. This approximation is repeated to each iteration.

	To start the formal approach let us recall that we have proved in the section Sequences and Series that a Taylor expansion for a function of two variables could be written in quadratic approximation by:
	
	where for recall $h$ and $k$ are variables and $x_0,y_0$ are fixed and where we have the Hessian matrix (\SeeChapter{see section Sequences and Series page \pageref{hessian matrix}}):
	
	that American experts in the field have a habit (unfortunate in my opinion ...) Note:
	
	the latter expression being the most common can be very misleading with the notation of the Laplacian.

	In the field of numerical methods it is customary to write the Taylor series above with few notations changes by putting first:
	
	This gives us a more condensed and technical form of the Taylor series around $\vec{x}$:
	
	By changing again a little bit the notations:
	
	We thus fall back on the usual expression of a function of $\mathbb{R}^2\rightarrow \mathbb{R}$ evaluated in Taylor series centered on $\vec{x}$.

	But if we seek for a local extrema (also sometimes named "\NewTerm{critical point}\index{critical point}"), we will need in first time that the derivative of the whole Taylor series be equal to zero. That is to say:
	
	and that the determinant of the Hessian matrix is positive (\SeeChapter{see section Sequences and Series page \pageref{hessian matrix}}). And to know if we are on a local maximum or local minimum, we must look at the sign of $\partial_x^2 f(\vec{x})$.

	Let us rewrite the above relation explicitly as we proved it in the section of Sequences and Series for pedagogical reasons:
	
	And let us recall that all terms $x_0,y_0$ are constants because it is either the function $f$ evaluated the particular point $(x_0,y_0)$, or the partial derivative evaluated at the same point, either the partial second derivative always evaluated at the same point, etc.

	So finally the gradient will give:
	
	and returning traditional notations in the field of numerical methods, we have then:
	
	And so as the gradient has to be equal zero, we have:
	
	and after a first rearrangement:
	
	and a second rearrangement:
	
	which is often written:
	
	and by American specialists:
	
	Finally, before moving on to a concrete example it is important that the reader remembers the relation just seen above:
	
	
	\pagebreak
	\begin{tcolorbox}[colframe=black,colback=white,sharp corners]
	\textbf{{\Large \ding{45}}Example:}\\\\
	We are seeking a local extremum of the "humpback whale" function shown earlier above:
	
	with the starting point (arbitrary):
	
	To do the search, we calculate the gradient:
	
	and the Hessian matrix:
	
	We then have:
	
	and:
	
	and:
	
	and therefore:
	
	and we start again (with less detail):
	
	\end{tcolorbox}
	
	\pagebreak
	\begin{tcolorbox}[colframe=black,colback=white,sharp corners]
	
	and therefore:
	
	and once again (with again less details):
	
	and therefore:
	
	and again (with even less detail):
	
	and values will not move anymore. But if we look at the original graphic where we highlighted the convergence point by a red point:
	\begin{figure}[H]
		\centering
		\includegraphics[scale=0.85]{img/computing/convergence_point_humpback_whale_example.jpg}
		\caption[]{Highlight of the convergence point in the humpback whale function}
	\end{figure}
	we see that this system does not search a global extremum but a local extremum as we already specified it. In fact, as the reader may test itself, convergence is very sensitive to initial starting point.
	\end{tcolorbox}
	Keep really in mind that this method (and also the next one) works well only for smooth convex optimization problem (obviously twice differentiable and the opposite of the next method that need to be only once differentiable):
	\begin{figure}[H]
		\centering
		\includegraphics[scale=0.7]{img/computing/optimization_paths_convex_nonconvex.jpg}
		\caption[]{Optimization paths with different starting points are illustrated in different colours. In the case of strictly convex function (Figure a.), the paths starting from any points all lead to the global optimum. Conversely, in the case of non-convex function, different paths may end up at different local optima!}
	\end{figure}
	
	\paragraph{Gauss-Newton method's (Tangent Newton)}\label{gradient descent}\mbox{}\\\\
	The Gauss-Newton's method is a powerful approximation without the derivatives of the second order of the Newton-Raphson method that in the prior versions of Microsoft Excel 2007 was activated in the solver by selecting the option \textit{Newton} and \textit{Tangent}.

	To study this method, let us do use a companion concrete example! Suppose we obtained the following data:
	\begin{table}[H]
		\centering
		\definecolor{gris}{gray}{0.85}
		\begin{tabular}{|p{2cm}|p{2cm}|}
			\hline
			\multicolumn{1}{c}{\cellcolor{black!30}$t_i$} & 
  \multicolumn{1}{c}{\cellcolor{black!30}$y_i$}  \\ \hline
			\centering\arraybackslash\ $1$ & \centering\arraybackslash\ $3.2939$ \\ \hline	
			\centering\arraybackslash\ $2$ & \centering\arraybackslash\ $4.2699$ \\ \hline	
			\centering\arraybackslash\ $4$ & \centering\arraybackslash\ $7.1749$ \\ \hline	
			\centering\arraybackslash\ $5$ & \centering\arraybackslash\ $9.3008$ \\ \hline	
			\centering\arraybackslash\ $8$ & \centering\arraybackslash\ $20.259$ \\ \hline	
		\end{tabular}
	\end{table}
	and we suppose "a priori" that the data follow the following theoretical model (we could also try any other function):
	 
	We look then for $x_1,x_2$ that minimize the sum of squares (the square of the euclidean distance - $L^2$ - that is obviously not the only choice of distance in practice!) between the experimental and theoretical values such that:
	
	with therefore:
	
	The previous relations is also sometimes (especially in the field of Machine Learning) denoted as following (but with a notation quite generalized):
	
	and named the "\NewTerm{cost function}\index{cost function}". The $1/2$ is here by anticipation to eliminate a useless factor $2$ that will appear later below (as it change nothing to the result of minimization anyway!). The way to put a problem in this form and minimize the cost function is generally named a "least square optimization problem" or in finance and ecology a "model calibration\index{model calibration}" (that will result to a "calibrated model"...).
	
	\begin{tcolorbox}[title=Remark,colframe=black,arc=10pt]
	We can also use this method to just minimize a function:
	
	without using any sum of square. This depends on the problem! The methods remains exactly the same. 
	\end{tcolorbox}
	Let us write (following the tradition in the field of optimization):
	
	We then have the following common notation:
	
	Now, imagine that we found a bipoint $(x_1,x_2)=(\vec{x})$ that gives this minimum and let us write it $(\vec{x}_{*})$ and without forgetting that it will remain a local minimum and with luck a global one...! 

	Let us consider a special case that we will name "\NewTerm{compatible solution}\index{compatible solution}" and define by the fact the bi-point that minimize the sum of squares of errors is also such that for all $i$ we have:
	
	Therefore it is immediate that:
	
	Before going further, let us notice for example that for a component $j$ (which corresponds in our case to each variable of the a priori supposed theoretical function of our model):
	
	where the last condensed equality is many times less obvious as it makes usage of the gradient of a vector field (\SeeChapter{see section Vector Calculus page \pageref{gradient of vector field}}) that we see rarely in practice (note the factor $2$ that we were speaking about earlier!). The reader that should be destabilized can refer directly to the numerical example further below to makes things more crystal clear.
 	\begin{tcolorbox}[colframe=black,colback=white,sharp corners]
	\textbf{{\Large \ding{45}}Example:}\\\\
	In the field of Machine Learning, considering (this a liner weighted input of a neural network as we will see later during our study of these objects):
	
	where $x_0$ refers something named a "bias unit" such that $x_0=1$.\\
	
	Here, our cost function is also the following sum of squared errors (SSE) with the traditional notation of Machine Learning field that fits more when we do detailed developments:
	
	We then have for one component $j$:
	
	So we have the correspondence for this special (famous) case:
	
	\end{tcolorbox}
	So to continue... we deduce that:
	
	and the "compatible solution" brings us obviously to:
	
	Following the same step, we have:
	
	So finally we have the following two relations:
	
	Given that for the "compatible solution" we have:
	
	it follows that in this case the second relation becomes:
	
	where $H$ is the Hessian matrix (\SeeChapter{see section Sequences and Series page \pageref{hessian matrix}}) what American practitioners write simply:
	
	So we can approximate in the case of the compatible solution, the Hessian that contains derivatives of the second order by derivatives of the first order.

	So we have finally in this special case the two relations that are the pillar of the Gauss-Newton's method:
	
	Now let us recall the basic relation of the Newton-Raphson method obtained earlier above:
	
	and for information, any mathematical technique (because they are many of them!) that simplifies the Hessian matrix (\SeeChapter{see section Sequences and Series page \pageref{hessian matrix}}) to the right of equality becomes part of the family named "\NewTerm{quasi-Newton's methods}\index{quasi-Newton's methods}".
	
	Well the Gauss-Newton's method that interest us here and is therefore one of the techniques of the family of the "quasi-Newton's methods" consists simply in a first step in getting rid of the second derivatives of the Hessian (\SeeChapter{see section Sequences and Series page \pageref{hessian matrix}}) of the Newton-Raphson method at the right of the equality thanks to the previously established relations such that (attention to remember the abuse of writing!):
	
	and in a second time rewrite the gradient on the left of the equality thanks also to the previously established relation. Which gives us:
	
	The factor $2$ being not very aesthetic, almost all textbooks optimize such optimization problem with the type of notation as already mentioned:
	
	So by simply multiplying by a factor $1/2$ (which does not change the result as we already know!) we have then:
		
	Let us recall again that a spreadsheet software like Microsoft Excel can not determine the derivatives it will calculate them using the numerical methods of right or centered derivatives as we have presented a earlier above.

	Now let us come back to our example of the beginning! So we have:
	
	We start with a bipoint that seems the closest to the desired solution:
	
	Then we have:
	
	and therefore we have:
	
	Then it comes:
	
	What we can therefore rewrite as:
	
	We also have by extension:
	
	Then we apply the relation proved earlier above:
	
	Therefore:
	
	After a minor simplification:
	
	Therefore:
	
	and therefore the next bipoint for the iteration will be:
	
	Hence formally:
	
	\begin{tcolorbox}[title=Remark,colframe=black,arc=10pt]
	In the field of Machine Learning, that latter relation is written as following:
	
	but where $\eta$ is fixed (in the field of Machine Learning!) and named the "\NewTerm{learning rate}\index{learning rate}" and the whole relation (corresponding to a minimization problem for recall!) is named the "\NewTerm{gradient descent method}\index{gradient descent method}".\\
	
	Obviously we see that the weight change is then defined by:
	
	
	In component notation, for the component $j$ this gives:
	
	In numerical computing this is often denoted:
	
	where $k=1,2,\ldots$ is iteration number, $t_k$ is step size (or step length) at iteration $k$.
	\end{tcolorbox}
	Which of corresponds well to the values of the first iteration:
	
	We will not do again explicitly also the other iterations. So this is what we get in the end:
	\begin{table}[H]
		\begin{center}
			\definecolor{gris}{gray}{0.85}
				\begin{tabular}{|p{2cm}|p{2cm}|}
					\hline
					\multicolumn{1}{c}{\cellcolor{black!30}$i$} & 
	  \multicolumn{1}{c}{\cellcolor{black!30}$\mathop{\min}_{x_1,x_2} f(x_1,x_2)$}  \\ \hline
					\centering\arraybackslash\ $0$ & \centering\arraybackslash\ $2\cdot 10^0$ \\ \hline	
					\centering\arraybackslash\ $1$ & \centering\arraybackslash\ $4\cdot 10^{-3}$ \\ \hline	
					\centering\arraybackslash\ $2$ & \centering\arraybackslash\ $2\cdot 10^{-8}$ \\ \hline	
					\centering\arraybackslash\ $3$ & \centering\arraybackslash\ $3\cdot 10^{-9}$ \\ \hline	
					\centering\arraybackslash\ $4$ & \centering\arraybackslash\ $3\cdot 10^{-9}$ \\ \hline	
				\end{tabular}
		\end{center}
		\caption[]{Gauss Newton iterations of our example}
	\end{table}
	with therefore for local solution at the 4th iteration:
	
	With Maple 4.00b we get:
	
	\texttt{>with(plots):}\\
	\texttt{>points:=plot([[1,3.2939],[2,4.2699],[4,7.1749],[5,9.3008],[8,20.259] ],style=point,color=blue,symbol=circle):}\\
	\texttt{>plot\_GN:=plot(2.5411*exp(0.2595*x),x=0..8):}\\	
	\texttt{>display([pict1,pict2]);}
	
	\begin{figure}[H]
		\centering
		\includegraphics[scale=0.85]{img/computing/gauss_newton_example_plot.jpg}
		\caption[]{Points and our interpolated Gauss-Newton function}
	\end{figure}
		
	To close this subject let us do a comparison with the Newton-Raphson method for the first iteration using the same starting bipoint. Let us recall again that for this latter method, the iterations are based on the relation:
	
	and we will write the function as following for the Newton-Raphson method:
	
	and:
	
	Then we have:
	
	that becomes:
	
	Then we have:
	
	That becomes:
	
	Thus:
	
	Therefore:
	and therefore the next bipoint for the iteration will be:
	
	which corresponds obviously to the values of the first iteration:
	
	We will not do again explicitly also other iterations. So this is what it gives finally about our function to be minimized:
	\begin{table}[H]
		\begin{center}
			\definecolor{gris}{gray}{0.85}
				\begin{tabular}{|p{2cm}|p{2cm}|}
					\hline
					\multicolumn{1}{c}{\cellcolor{black!30}$i$} & 
	  \multicolumn{1}{c}{\cellcolor{black!30}$\mathop{\min}_{x_1,x_2} f(x_1,x_2)$}  \\ \hline
					\centering\arraybackslash\ $0$ & \centering\arraybackslash\ $2\cdot 10^0$ \\ \hline	
					\centering\arraybackslash\ $1$ & \centering\arraybackslash\ $1\cdot 10^{-1}$ \\ \hline	
					\centering\arraybackslash\ $2$ & \centering\arraybackslash\ $2\cdot 10^{-4}$ \\ \hline	
					\centering\arraybackslash\ $3$ & \centering\arraybackslash\ $5\cdot 10^{-9}$ \\ \hline	
					\centering\arraybackslash\ $4$ & \centering\arraybackslash\ $6\cdot 10^{-9}$ \\ \hline	
					\centering\arraybackslash\ $5$ & \centering\arraybackslash\ $3\cdot 10^{-9}$ \\ \hline	
				\end{tabular}
		\end{center}
		\caption[]{Gauss Newton iterations of our example}
	\end{table}
	So the Newton-Raphson method converges in this case slower than that of Gauss-Newton.
	
	So this was a very detailed step-by-step numerical example. For people interested in Machine Learning and for a more formal example (without numerical values) with a famous case (univariate linear regression) let us consider:
	
	The cost function for any guess of $\theta_0, \theta_1$ can be computed as:
	
	where $x^{(i)}$ and $y^{(i)}$ are the $x$ and $y$ values for the $i$th component in the learning set. If we substitute for $h_\theta(x)$:
	
	Then as we know, in Machine Learning, the goal of gradient descent\index{gradient descent} can be expressed as:
	
	\begin{tcolorbox}[title=Remark,colframe=black,arc=10pt]
	At the opposite of the "\NewTerm{gradient ascent}\index{gradient ascent}" where we seek:
	
	\end{tcolorbox}
	Finally, each step in the gradient descent can be described as we have seen earlier above for a component $j$:
	
	for $j=0$ and $j=1$ in this special case!
	
	It is straightforward that:
	
	and:
	
	Or in vector form:
	
	
	In order to minimize the above cost function $J(\theta_0,\theta_1)$, notice that the gradient  is calculated from the \underline{whole training set} (hence the sum!). This is why this approach of "gradient descent" is more specifically referred to as "\NewTerm{batch gradient descent}\index{batch gradient descent}".
	
	\begin{tcolorbox}[title=Remark,colframe=black,arc=10pt]
	The use of the traditional complete method using the Hessian is in the early 21st century impractical for most deep learning applications because computing (and inverting) the Hessian in its explicit form is a very costly process in both space and time. For instance, a Neural Network with one million parameters would have a Hessian matrix of size $[1,000,000 \times 1,000,000]$, occupying approximately $3,725$ gigabytes of RAM. Hence, a large variety of quasi-Newton methods have been developed that seek to approximate the inverse Hessian!
	\end{tcolorbox}
	
	As we need to calculate the gradients for the whole dataset to perform just one update, batch gradient-descent can be very slow and is intractable for datasets that do not fit in memory.  Batch gradient  descent also does not allow us to update our model online, i.e. with new examples on-the-fly

	So, in "\NewTerm{stochastic gradient descent method}\index{stochastic gradient descent method}" (SGD), also known as "\NewTerm{incremental gradient descent}\index{incremental gradient descent}", instead of updating the weights $\vec{\theta}$ based on the sum of the $m$ accumulated errors over all samples $x^{(i)}$ :
	
	we can use the following update:
	
	Note that we now update the weights incrementally with a single training sample but not with the whole training set. 
	
	It is called "stochastic" because samples are selected randomly (or shuffled) instead of as a single group (as in standard gradient descent) or in the order they appear in the training set.
	
	\begin{tcolorbox}[colframe=black,colback=white,sharp corners]
	\textbf{{\Large \ding{45}}Example:}\\\\
	Let's suppose we want to fit a straight line $y_i=\theta_0+\theta_1x_i$ to a training set with observations $(x_{1},x_{2},\ldots ,x_{n})$ and corresponding estimated responses $({\hat {y_{1}}},{\hat {y_{2}}},\ldots ,{\hat {y_{n}}})$ using least squares. The objective function to be minimized is:
	
	Then we will have using the stochastic gradient descent method:
	
	Note again that in each iteration, the gradient is evaluated at a single point $x^{(i)}$ instead of being evaluated on the set of all samples.
	\end{tcolorbox}
	The key difference compared to standard (Batch) Gradient Descent is that only one piece of data from the dataset is used to calculate the step, and the piece of data is picked randomly at each step.  SGD  does  away  with  this  redundancy  by performing one update at a time.  It is therefore usually much faster and can also be used to learn online.  SGD performs frequent updates with a high variance that cause the objective function to fluctuate heavily.
	
	While batch gradient descent converges to the minimum of the basin the parameters are placed in, SGD's fluctuation, on the one hand, enables it to jump to new and potentially better local minima. On the other hand, this ultimately complicates convergence to the exact minimum, as SGD will keep overshooting. However, it has been shown that when we slowly decrease the learning rate, SGD shows the same convergence behaviour as batch gradient descent, almost certainly converging to a local or the global minimum for non-convex and convex optimization respectively.
	
	\begin{tcolorbox}[title=Remark,colframe=black,arc=10pt]
	"\NewTerm{mini-batch gradient descent}\index{mini-batch gradient descent}" initially takes the best of both worlds and performs an update for every mini-batch of $n$ training examples. Common mini-batch sizes range between $50$ and $256$, but can vary for different applications. Mini-batch gradient descent is typically the algorithm of choice when training a neural network and the term SGD usually is employed also when mini-batches are used
	\end{tcolorbox}
	
	\subparagraph{Gradient Descent/Ascent techniques summary}\mbox{}\\\\
	At this point it may be useful to also introduce the gradient descent\index{gradient descent} and gradient ascent\index{gradient ascent} in a simpler way (if it can help!) as it is quite and important topic.
	
	The reader may have already understand that the gradient descent/ascent is a general framework for solving optimization problems where we want to maximize or minimize functions of continuous (differentiable) parameters.

	We will illustrate the technique here for maximization (ascent) in terms of a single free variable, but the generalization is quite simple and there is already plenty of example above!
	
	Suppose we want to find a maximum of some function $y=f(x)$. The standard procedure is to find $f'(x)$, set it to zero and solve for $x$. But what if $f'(x) = 0$ is a difficult equation to solve?

	We can still find the maximum algorithmically!
	
	Let $y=f(x)$ have a maximum at $x_{\max}$. Pick an arbitrary value for $x$, say $x_1$. Compute $f '(x_1)$. If the slope of $y$ is positive at $x_1$, i.e. $f'(x_1) > 0$, then $x_{\max} > x_1$ lies to the right of $x_1$. Likewise if $f'(x_1) < 0$, then $x_{\max} < x_1$ lies to its left!

	Thus we know the direction in which $x_1$ should be updated in order to approach $x_{\max}$. In fact this direction is given by $f'(x_1)$. So we can use the update rule:
	
	where $\eta$ is a positive constant. If $\eta$ is sufficiently small, and there is indeed a maximum for $f$, the above update rule will converge to it after a finite number of iterations (when $f'(x_1) = 0$).
	
	This is the so-called gradient ascent technique to determine maxima when actual solution of $f' = 0$ is difficult. To find minima, the corresponding gradient descent rule can set up to update $x_1$ away from $x_{\max}$.
	
	For $y = f(x_1, \cdots, x_n)$ the partial derivatives with respect to each $x_i$ indicate how rapidly $y$ is changing along that axis and so the gradient, $\nabla y$, in fact denotes the precise direction which promises maximum increase of $y$.
	
	Collectively these, among others, are named "\NewTerm{gradient techniques}".
	
	We will summarise now the common gradient descent optimisation algorithms used in popular deep learning frameworks in the early 21st century.
	
	Recall that the vanilla stochastic gradient descent (SGD) updates weights by subtracting the current weight by a factor (i.e. $\eta$, the learning rate) of its gradient:
	
	Variations in this equation are commonly known as "\NewTerm{stochastic gradient descent optimisers}\index{stochastic gradient descent optimisers}" . There are 3 main ways how they differ:
	\begin{enumerate}
		\item Adapt the "gradient component" ($\frac{\partial L}{\partial w}$)

		Instead of using only one single gradient like in stochastic vanilla gradient descent to update the weight, take an aggregate of multiple gradients. Specifically, these optimisers use the exponential moving average of gradients.
		
		\item Adapt the "learning rate component" ($\eta$)

		Instead of keeping a constant learning rate, adapt the learning rate according to the magnitude of the gradient(s).
		
		\item Both (1) and (2)

		Adapt both the gradient component and the learning rate component.
	\end{enumerate}
	As you will see later, these optimisers try to improve the amount of information used to update the weights, mainly through using previous (and future) gradients, instead of only the present available gradient.

	Below is a table that summarises which components are being adapted:
	\begin{table}[H]
		\centering
		\begin{tabular}{|l|l|c|c|}
		\hline
		\textbf{Optimiser} & \textbf{Year} & \textbf{Learning Rate} & \textbf{Gradient} \\ \hline
		Momentum & 1964 &  & \checkmark \\ \hline
		AdaGrad & 2011 & \checkmark  &  \\ \hline
		RMSprop & 2012 & \checkmark  &  \\ \hline
		AdaDelta & 2012 & \checkmark  &  \\ \hline
		Nesterov & 2013 &  & \checkmark  \\ \hline
		Adam & 2014 & \checkmark  & \checkmark  \\ \hline
		AdaMax & 2015 & \checkmark  & \checkmark  \\ \hline
		Nadam & 2015 & \checkmark  & \checkmark  \\ \hline
		AMSGrad & 2018 & \checkmark  & \checkmark  \\ \hline
		\end{tabular}
		\caption{Gradient descent optimisers}
	\end{table}
	Here are the detailed expression of each of the empirical method visible in the above table (the knowledgeable reader may recognize some techniques inspired by time series forecasting techniques):
	\begin{itemize}
		\item Vanilla SGD:
		
		
		\item Momentum:
		
		
		\item Adagrad:
		
		
		\item RMSprop:
		
		
		\item Adadelta:
		
		
		\item Nesterov:
		
		
		\item Adam:
		
		
		\item AdaMax:
		
		
		\item Nadam:
		
		
		\item AMSGrad:
		
	\end{itemize}
	
	Below the reader can see animations (credit: Alec Radford) that may help the reader intuitions about the learning process dynamics (we also strongly recommend to read the paper \textit{An overview of gradient descent optimization algorithms} to get more detailed on the underlying ideas \cite{ruder2016overview}).
	
	The first Flash animation below depicts the contours of a loss surface and time evolution of different optimization algorithms. Notice the "overshooting" behaviour of momentum-based methods, which make the optimization look like a ball rolling down the hill:
	\begin{center}
		\includemedia[activate=pageopen,width=\textwidth,height=380pt,
	]{}{swf/contours_evaluation_optimizers.swf}
	\end{center}
	The animation above will run for people having a PDF reader with Adobe Flash player installed and activated (otherwise see here: \url{https://vimeo.com/575749745}).
	
	The second and last Flash animation is a visualization of a saddle point in the optimization landscape, where the curvature along different dimension has different signs (one dimension curves up and another down). Notice that SGD has a very hard time breaking symmetry and gets stuck on the top. Conversely, algorithms such as RMSprop will see very low gradients in the saddle direction. Due to the denominator term in the RMSprop update, this will increase the effective learning rate along this direction, helping RMSProp proceed:
	\begin{center}
		\includemedia[activate=pageopen,width=\textwidth,height=380pt,
	]{}{swf/saddle_point_evaluation_optimizers.swf}
	\end{center}
	The animation above will run for people having a PDF reader with Adobe Flash player installed and activated (otherwise see here: \url{https://vimeo.com/575752619}).
	
	\pagebreak
	\subsubsection{Expectation-Maximization (EM) algorithm}\label{EM algorithm}
	The "\NewTerm{Expectation-Maximization algorithm}\index{Expectation-Maximization algorithm}" is a stochastic optimizations algorithm often used in machine learning and very quite different of the methods that we have seen before. We will use it explicitly in this book later below for various applications like Factor Analysis (see page \pageref{factor analysis}) but also for the $T$-distributed Stochastic Neighbour Embedding (see page \pageref{tsne}) and others during our study of Data Mining.
	
	The EM algorithm seems to have been introduced in the early 1950 by R. Ceppellini, M. Siniscalco and C.A. Smith in the context of gene frequency estimation. The expectation maximization algorithm was analysed more generally by H. Hartley and by L.E. Baum in the context of hidden Markov models, where it is commonly known as the "Baum-Welch algorithm".
	
	\begin{tcolorbox}[title=Remark,colframe=black,arc=10pt]
	The text that follows is inspired of the following paper \cite{brayexpectation} but also of the following web page \url{https://indowhiz.com/articles/en/the-simple-concept-of-expectation-maximization-em-algorithm} itself inspired of \cite{do2008expectation}.
	\end{tcolorbox}
	
	 To summarize, EM algorithm is actually an iterative method, involving expectation (E-step) and maximization (M-step) to find the local maximum likelihood from the data. Commonly, EM algorithm is used on several distributions or statistical models, where there are one or more unknown variables named the "latent variables".
	 
	 To easily understand EM algorithm, we will use first a companion example of the coin tosses distribution that is widely used in quite a few textbooks, articles and blogs and afterwards we will deal with the underlying mathematical formalism.
	 
	 So let us consider that we have $2$ coins: Coin $A$ and Coin $B$, where both have a different head-up probability. We will randomly choose a coin $5$ times, whether coin $A$ or $B$. Then, each coin selection is followed by tossing it $10$ times. Therefore, we have the following outcomes:
	 \begin{itemize}
	 	\item Set 1 (\textcolor{blue}{Coin $B$}): \texttt{H T T T H H T H T H} (5H5T)
	 	\item Set 2 (\textcolor{red}{Coin $A$}): \texttt{H H H H T H H H H H} (9H1T)
	 	\item Set 3 (\textcolor{red}{Coin $A$}): \texttt{H T H H H H H T H H} (8H2T)
	 	\item Set 4 (\textcolor{blue}{Coin $B$}): \texttt{H T H T T T H H T T} (4H6T)
	 	\item Set 5 (\textcolor{red}{Coin $A$}): \texttt{T H H H T H H H T H} (7H3T)
	 \end{itemize}
	 where the image side of the coin will be denoted by \texttt{H} or Head-up, and the number side will be denoted by \texttt{T} or Tail-up. Then, the probability of a coin to land with head-up for each of these coins, will be denoted as $\theta$.
	 
	 The question is, how do we estimate $\theta$ for each coin?
	 
	 Because we already have all the data we need, we can easily calculate the probability of getting head-up for each coin. Therefore, the calculation of $\theta$ will be done through the maximum likelihood estimation (MLE) of the proportion that can naively be obtained by:
	 \begin{itemize}
	 	\item By knowing the result of coin $A = 24\texttt{H}6\texttt{T}$, in $3$ sets with a total of $30$ tosses, then:
	 	
	 	
	 	\item Similarly to the result of coin $B = 9\texttt{H}11\texttt{T}$, in $2$ sets with a total of $20$ tosses, then:
	 	
	 \end{itemize}
	 Before returning to the original case, let us explain briefly the different the stages of the EM algorithm as it will give us an overall picture of the EM algorithm flow and may help to understand what will follow:
	 \begin{figure}[H]
		\centering
		\includegraphics[scale=2]{img/computing/em_flowchart.jpg}
		\caption{Flowchart of EM algorithm}
	\end{figure}
	Actually, the main point of EM is the iteration between E-step and M-step, which could be seen in the figure above. The E-step will estimate your hidden variables, and the M-step will re-update the parameters, based on the estimation of the hidden variables. In other words, this iteration aims to re-improve the estimation of current parameters.
	
	First, what we must define is what variables are required; but are not observed directly in the data.

	The goal is to estimate the probability of getting heads-up for each coin. However, it cannot be calculated directly if we don't know the identity of the coin used in each set. Therefore, it is necessary to know which coin is used in each set. In other words, this coin identity will be our latent variable.

	Right now, besides the $5$ sets of outcomes above, we only know that we used two coins; which are coin $A$ and coin $B$.
	
	As stated before, the probability of getting head-up for each of these coins is denoted as  $\theta$. Currently, there are only coin $A$ and coin $B$; with unknown parameter values of $\theta_{A}$ and $\theta_{B}$.
	
	\begin{tcolorbox}[title=Remark,colframe=black,arc=10pt]
	By the way, please note that there is no relationship between parameters guessing; both $\theta_{A}$ and $\theta_{B}$. For example, if you think that the sum of $\theta_{A}$ and $\theta_{B}$ must be $= 1$, NO! This probability represents the individual value of getting heads-up on each coin.
	\end{tcolorbox}
	 Okay enough! Now, let's say we have randomly set both initial values, which are:
	 
	 Now we have the required variables. So, we can start estimating the identity of coin used in each set.

	Each set, which contains the outcomes of heads and tails, can be denoted by $E$ notation. Then, the probability of \textit{using coin $A$} denoted as $Z_{A}$, and the probability of \textit{using coin $B$} denoted as $Z_{B}$.
	 
	At the beginning of E-Step, we need to know the probability of a set using coins $A$ or $B$. Take an example from the set 3 earlier above, where $E = \text{HTHHHHHTHH}$ (8H2T). If $\hat{\theta}_{A} = 0.6$, it means the probability of getting head is $0.6$ (and tail $0.4$) uses coin $A$. Then, how much probability of coin $A$ will give $8\texttt{H}2\texttt{T}$ in $10$ tosses (a set)? This is what we need to estimate first. Then, this probability can be denoted by $P(E|Z_{A})$. Similarly, the probability of coin $B$ giving $E=8\texttt{H}2\texttt{T}$, could be denoted by $P(E|Z_{B})$. 
	
	This actually relates to the probability distribution of a binomial random variable, which definition is for recall:
	
	 Or denoted:
	 
	 Then, we can calculate the probability of getting $E$ using coins $A$ and $B$ as follows:
	 
	 Next, we can compare the probabilities of both coin $A$ and coin $B$ giving $E$. So, according to Bayes’ theorem and the law of total probability, we can determine the ratio of probability (posterior value of latent variables( using the following relation:
	 
	 where $P(Z_{x}|E)$ is the probability of coin $x$ giving $E$ (compared to coin $y$), $P(Z_{x})$ is the probability of choosing coin $x$ and $P(Z_{x})$ is the probability of choosing coin $y$.
	 
	 However, because our case is only using $2$ coins, coins $A$ and $B$, then the probability that we will choose one of them is $50/50$. Then $P(Z_{A}) = P(Z_{B}) = 0.5$.
	 
	 Accordingly, for coin $A$, we can simplify the relation above into:
	 
	 Hence:
	 
	 Similarly, for coin $B$, we have:
	 
	 Previously, the ratio of coins $A$ and $B$ in giving each $E$ needs to be calculated. In this case, it is calculated from set $1$ to $5$. Then the results could be seen in the following table:
	 \begin{table}[H]
	 	\centering
		\begin{tabular}{|c|c|c|c|}
		\hline
		\rowcolor[HTML]{C0C0C0} 
		\multicolumn{1}{|l|}{\cellcolor[HTML]{C0C0C0}\textbf{Coin Tosses}} & \multicolumn{1}{l|}{\cellcolor[HTML]{C0C0C0}\textbf{$E$}} & \multicolumn{1}{l|}{\cellcolor[HTML]{C0C0C0}\textbf{Coin $A$ probability}} & \multicolumn{1}{l|}{\cellcolor[HTML]{C0C0C0}\textbf{Coin $B$ probability}} \\ \hline
		\texttt{HTTTHHTHTH} & 5\texttt{H}5\texttt{T} & $0.45$ & $0.55$ \\ \hline
		\texttt{HHHHTHHHHH} & 9\texttt{H}1\texttt{T} & $0.80$ & $0.20$ \\ \hline
		\texttt{HTHHHHHTHH} & 8\texttt{H}3\texttt{T} & $0.73$ & $0.27$ \\ \hline
		\texttt{HTHTTTHHTT} & 4\texttt{H}6\texttt{T} & $0.35$ & $0.65$ \\ \hline
		\texttt{THHHTHHHTH} & 7\texttt{H}3\texttt{T} & $0.65$ & $0.35$ \\ \hline
		\end{tabular}
	\end{table}
	After that, we need to estimate the total number of \texttt{H} for each coin. It is calculated based on the coin ratio above. To calculate \textit{total heads and tails} for coin $x$, it is similar to the complete data. The method is quite easy, we just need to multiply the ratio of each coin to the number of heads in each $E$, the results are shown in the following table:
	\begin{table}[H]
		\centering
		\begin{tabular}{|c|c|c|c|}
		\hline
		\rowcolor[HTML]{C0C0C0} 
		\multicolumn{1}{|l|}{\cellcolor[HTML]{C0C0C0}\textbf{Coin Tosses}} & \multicolumn{1}{l|}{\cellcolor[HTML]{C0C0C0}\textbf{$E$}} & \multicolumn{1}{l|}{\cellcolor[HTML]{C0C0C0}\textbf{Estimated \texttt{H} for Coin $A$}} & \multicolumn{1}{l|}{\cellcolor[HTML]{C0C0C0}\textbf{Estimated \texttt{H} for Coin $B$}} \\ \hline
		\texttt{HTTTHHTHTH} & 5\texttt{H}5\texttt{T} & $5\cdot 0.45=2.25$ & $5\cdot 0.55 = 2.75$ \\ \hline
		\texttt{HHHHTHHHHH} & 9\texttt{H}1\texttt{T} & $9\cdot 0.8 =7.2$ & $9\cdot 0.20=1.8$ \\ \hline
		\texttt{HTHHHHHTHH} & 8\texttt{H}3\texttt{T} & $8\cdot 0.73 = 5.84$ & $8\cdot 0.27 = 2.16$ \\ \hline
		\texttt{HTHTTTHHTT} & 4\texttt{H}6\texttt{T} & $4\cdot 0.35 = 1.4$ & $4\cdot 0.65=2.6$ \\ \hline
		\texttt{THHHTHHHTH} & 7\texttt{H}3\texttt{T} & $7\cdot 0.65 = 4.55$ & $7\cdot 0.35=2.45$ \\ \hline
		\end{tabular}
	\end{table}
	If we want, we can also calculate the tails for each coin as in the table above. But that is not necessary, because we can use another straightforward approach.

	Until this step, we already have the E-Step calculations. Just a little bit more effort to finish our calculation in the M-Step.
	
	The results from the E-step can be used to improve the $\hat{\theta}_x$ parameters. Here, we can use the maximum likelihood estimation (MLE) relation similar to the completed data.

	As we said before, it is not necessary to calculate the tails for each coin; then we sum the heads and tails for each coin. Because each set contains $10$ tosses, we just need to multiply the coin ratio with $10$. That way, we could get the total estimated tosses from. Therefore (we will prove in details just further below where these both relations comes from):
	
	Finally, the parameter of $\hat{\theta}_{A}$ and $\hat{\theta}_{B}$ for the first iteration have been improved. For the next iteration, the E-Step will use this new parameter value; and re-improved at next M-step. This iteration will always repeating the E-step and M-step, until it reaches any stop condition as illustrated below:
	 \begin{figure}[H]
		\centering
		\includegraphics[scale=0.85]{img/computing/em_algorithm.jpg}
		\caption[EM parameter estimation for complete and incomplete data]{EM parameter estimation for complete and incomplete data (source: Nature Biotechnology Volume 26 Number 8 August 2008)}
	\end{figure}
	The iteration of the E-Step and M-Step, will be repeated until they meet the stopping condition. Commonly, the EM algorithm has two options of stopping condition, which are:
	\begin{itemize}
		\item Maximum iteration: means that, the EM Algorithm will stop if a certain number of iterations has been reached. For example the maximum iteration is set to $10$ iterations, then the EM Algorithm will not be more than $10$ iterations
		
		\item Convergence threshold: means that, the M-step gives no significant parameter improvement; compared to the improvement in the previous iteration. The changes are very small below our threshold
	\end{itemize}
	In the EM case example above, the parameter improvement in each iteration can be seen in the following table:
	\begin{table}[H]
		\centering
		\begin{tabular}{|c|c|c|c|}
		\hline
		\rowcolor[HTML]{C0C0C0} 
		\multicolumn{1}{|l|}{\cellcolor[HTML]{C0C0C0}\textbf{Iteration $i$}} & \multicolumn{1}{l|}{\cellcolor[HTML]{C0C0C0}\textbf{$\theta_{A}^{(i)}$}} & \multicolumn{1}{l|}{\cellcolor[HTML]{C0C0C0}\textbf{$\theta_{B}^{(i)}$}} & \multicolumn{1}{l|}{\cellcolor[HTML]{C0C0C0}\textbf{Differences}} \\ \hline
		$0$ & $0.6$ & $0.5$ & $0.7180$ \\ \hline		
		$1$ & $0.713$ & $0.581$ & $0.1390$ \\ \hline
		$2$ & $0.745$ & $0.569$ & $0.0342$ \\ \hline
		$3$ & $0.768$ & $0.550$ & $0.0298$ \\ \hline
		$4$ & $0.783$ & $0.535$ & $0.0212$ \\ \hline
		$5$ & $0.791$ & $0.526$ & $0.0120$ \\ \hline
		$6$ & $0.795$ & $0.522$ & $0.0057$ \\ \hline
		$7$ & $0.796$ & $0.521$ & $0.0014$ \\ \hline
		$8$ & $0.796$ & $0.520$ & $0.0010$ \\ \hline
		$9$ & $0.796$ & $0.520$ & $0.0000$ \\ \hline
		\end{tabular}
	\end{table}
	To calculate the differences or improvements in each iteration, we can use the Euclidean Distance such that:
	
	 So for example:
	 
	As many algorithm that one also has weaknesses, without too much doubts. Indeed, every iteration in the EM algorithm, in general, will always improve the parameter closer to the local maximum likelihood. In other words, the EM algorithm will guarantee convergence, but will not guarantee to give a global maximum likelihood. And also there is no guarantee that you will get the maximum likelihood estimation (MLE).

	Now let us formalize what we have seen. For this purpose we will use the personal notes of Benjamin Bray \cite{brayexpectation} who authorized us to reproduce them below (we did some minor modifications).
	
	Suppose we still have two coins, each with a different probability of heads, $\theta_A$ and $\theta_B$, unknown to us. We collect data from a series of $N$ trials in order to estimate the bias of each coin. Each trial $k$ consists of flipping the same random coin $Z_k$ a total of $M$ times and recording only the total number $X_k$ of heads.
	
	Assuming all samples in the observed dataset $X=[\vec{x}_1,\ldots,\vec{x}_N]$ are independent and identically distributed (i.i.d.), we can find the "\NewTerm{complete data likelihood}\label{complete data likelihood}\index{complete data likelihood}" function of ${\theta}$ (using the Bayes' relation established in the section of Probabilities page \pageref{bayes relation for complete data likelihood}):
	
	and the log-likelihood:
	
	Or in scalar form (for non-multivariate distributions):
	
	The second term can be dropped in our companion example as $P(z_n\vert \theta)$ is independent of the model parameters in ${\theta}$ and therefore irrelevant to the maximization of the log likelihood with respect to ${\theta}$. Indeed, the reader must remember that in our companion example, $z_n$ is the realization of the random (hidden) variable that describes the probability to deal with coin $A$ either coin $B$. We assume that this probability is constant and equal to $1/2$, hence:
	 
	The remaining term in our companion example is:
	
    Now that we have specified the probabilistic model and worked out all relevant probabilities, we are ready to derive an Expectation-Maximization algorithm.
    
	Our general approach will be to reason about the hidden variables through a proxy distribution $q$, which we use to compute a lower-bound on the log-likelihood.  This section is devoted to deriving one such bound, named the "\NewTerm{Evidence Lower Bound}\label{evidence lower bound}" (ELBO).  We can expand the data log-likelihood by marginalizing over the hidden variables:
    
	Through Jensen's inequality (\SeeChapter{see section Statistics page \pageref{jensen inequality}}), we obtain the following bound:
	
	The lower bound can be rewritten as follows:
	
    Where the last term is often a constant at each iteration that we can ignore for the maximization problem.
    
    Now that we have this result, the \textbf{E-Step} is straightforward (calculation of the lower bound on the observed log-likelihood).  The \textbf{M-Step} computes a new parameter estimate $\theta_{t+1}$ by optimizing over the lower bound found in the E-Step, as:
    
    Now, because each trial is conditionally independent of the others, given the parameters:
	
	Let $a_k = q(z_k = A)$ and $b_k = q(z_k = B)$.  Note that:
	
	 To maximize the above expression with respect to the parameters, we take derivatives with respect to $\theta_A$ and $\theta_B$ and set to zero:
	
Solving for $\theta_A$ and $\theta_B$, we get
    
    These are the both relations we used earlier above in our companion example!

    We will see more complicated and practical business applications of the EM algorithm further below like:
    \begin{itemize}
    	\item For our study of Gaussian Mixture Classification (related to $K$-mean)
    	
    	\item For our study of Factor Analysis (Exploratory Factor Analysis (EFA))
    	
    	\item For our study of $T$-distributed Stochastic Neighbour Embedding ($T$-SNE)
    \end{itemize}
    
	
	\pagebreak
	\subsection{Resampling statistics}	
	Resampling statistics refers to the use of the observed data or of a data generating mechanism (such as a die) to produce new hypothetical samples (resamples) that mimic the underlying population, the results of which can then be analysed. With numerous cross-disciplinary applications especially in the sub-disciplines of the life science, resampling methods are widely used since they are options when parametric approaches are difficult to employ or otherwise do not apply. 
	
       Resampled data is derived using a manual mechanism to simulate many pseudo-trials. These approaches were difficult to utilize prior to 1980s since these methods require many repetitions. With the incorporation of computers, the trials can be simulated in a few minutes and is why these methods have become widely used.  The methods that will be discussed are used to make many statistical inferences about the underlying population. The most practical use of resampling methods is to derive confidence intervals and test hypotheses. This is accomplished by drawing simulated samples from the data themselves (resamples) or from a reference distribution based on the data; afterwards, you are able to observe how the statistic of interest in these resamples behaves. Resampling approaches can be used to substitute for traditional statistical (formulaic) approaches or when a traditional approach is difficult to apply. These methods are widely used because their ease of use. They generally require minimal mathematical formulas, needing a small amount of mathematical (algebraic) knowledge. These methods are easy to understand and stray away from choosing an incorrect formula in your diagnostics.
      \begin{figure}[H]
		\centering
		\includegraphics[scale=0.6]{img/computing/random_sampling.jpg}
		\caption[Summary of resampling in different methods]{Summary of resampling in different methods (source: ?)}
	\end{figure}
	
	\subsubsection{Monte Carlo Simulations}\label{monte carlo simulations}
	Monte Carlo methods, also named Monte Carlo experiments or Monte Carlo simulations\index{Monte Carlo simulations} (MCS), are a broad class of computational algorithms that rely on repeated random sampling to obtain numerical results. They are often used in physical and mathematical problems (finance, supply chain, decisioneering, quality, etc.) and are most useful as workaround when it is difficult or impossible to use other mathematical methods.
	
	It finds applications in various fields including the following examples:
	\begin{itemize}
		\item Problems related to the neutron bomb (or any other problem of  the same family: neutron diffusion)

		\item Calculations of integrals or various parameters of random variables (finance, insurance, risk, forecasting)
	
		\item Resolution of elliptic or parabolic equations

		\item Solving linear systems

		\item Optimization Problem Solving (operations research, project management, supply chain)

		\item Creation of statistical tests (Anderson-Darling, Kolmogorov, Levene, Brown-Forsythe, etc.)
	\end{itemize}
	Thus there are two types of problems that can be treated by the Monte Carlo probabilistic method:
	\begin{enumerate}
		\item Problems, which have a random behaviour 

		\item Deterministic problems, which do not have a random behaviour
	\end{enumerate}
	
	About the probabilistic case, the idea is to observe the behaviour of a series of random numbers that simulates how the real problem behaves and derive statistical solutions/conclusions. We then speak of "\NewTerm{Monte Carlo estimation}\index{Monte Carlo estimation}".

	In the deterministic case, the studied problem is completely defined and we can in principle predict its evolution, but some parameters of the problem can be treated as if it were random variables (this is typically the case in "vector regression" technique in Economy). The deterministic problem them becomes probabilistic and still solvable numerically. We then speak of "\NewTerm{Monte Carlo elaborated estimation}\index{Monte Carlo elaborated estimation}".
	
	Let us begin with the most used one in business: generating draws from a probability distribution. For this purpose we need to introduce the Inverse transform sampling (we will treat mostly only univariate case in this book!).
	
	\begin{tcolorbox}[title=Remark,colframe=black,arc=10pt]
	The name of "Monte Carlo method" date around 1944. Isolated researchers have used however long before similar statistical methods: for example, Edwin Herbert Hall for the experimental determination of the speed of light (1873), or Kelvin in a discussion of the Boltzmann equation (1901), but the real use of Monte Carlo methods began with research on the atomic bomb.\\

	During the immediate postwar period, John Von Neumann, Encrio Fermi and Stanislaw Ulam warned the scientific community of the applicability of the Monte Carlo methods (eg for the approximation of the eigenvalues of the Schrödinger equation). The systematic study was made by Harris and Herman Khan in 1948. After an eclipse caused by too intensive use during the 1950s, the Monte Carlo method is back since almost everybody can run complex management or business strategic simulations on office computers (with softwares like @Risk or CrystalBall).  in short, wherever it is profitable to use simulation processes.
	\end{tcolorbox}
	
	\pagebreak
	\paragraph{Inverse Transform Sampling}\label{inverse transform sampling}\mbox{}\\\\
	Inverse transform sampling (also known as inversion sampling,  inverse probability integral transform, inverse transformation method, Smirnov transform, golden rule,) is a basic method for pseudo-random number sampling, i.e. for generating sample numbers at random from any probability distribution given its cumulative distribution function.
	
	Inverse transformation sampling takes uniform samples of a number $\mathcal{U}$ between $0$ and $1$, interpreted as a probability, and then return the largest number $x$ from the domain of the distribution $P(X)$ such that $P(-\infty < X < x) \le \mathcal{U}$. 
	
	Computationally, this method involves computing the quantile function of the distribution — in other words, computing the cumulative distribution function (CDF) of the distribution (which maps a number in the domain to a probability between $0$ and $1$) and then inverting that function.
	
	To use this we method we go through he "\NewTerm{probability integral transform}\index{probability integral transform}" that states that if $X$ is a continuous random variable with cumulative distribution function $F_X$, then the random variable $Y=F_X(X)$ has a uniform distribution on $[0, 1]$. The inverse probability integral transform is just the inverse of this: specifically, if $Y$ has a uniform distribution on $[0, 1]$ and if $X$ has a cumulative distribution $F_X$, then the random variable $F_X^{-1}(Y)$ has the same distribution as $X$.
	\begin{theorem}
	Suppose that a random variable $X$ has a distribution for which the cumulative distribution function (CDF) is $F_X$. Then the random variable $Y$ defined as:
	
	has a uniform distribution.
	\end{theorem}
	\begin{dem}
	Given any random variable $X$, define $Y = F_X (X)$. Then (it is not always easy to read this proof through the first time even if afterwards it is obvious):
	
	Therefore $F_Y$ is just the CDF of uniform random variable $\mathcal{U}[0,1]$. Thus, $Y$ has a uniform distribution on the interval $[0, 1]$ (or in other words, if $X$ is a random variable with CDF $F_X(X)$ then $F_X(X)=\mathcal{U}[0,1]$).
	\begin{flushright}
		$\blacksquare$  Q.E.D.
	\end{flushright}
	\end{dem}
	In other words, the percentiles are uniformly distributed (ie the frequency plot of the percentiles is uniformly distributed) as illustrated below:
	\begin{figure}[H]
		\centering
		\includegraphics[width=1.0\textwidth]{img/computing/percentiles_uniformly_distributed.jpg}
	\end{figure}
	
	\begin{tcolorbox}[colback=red!5,borderline={1mm}{2mm}{red!5},arc=0mm,boxrule=0pt]
	\bcbombe Caution! In NHST (Null Hypothesis Statistical Tests) the $p$-value itself is considered as a realization of random variable. Therefore if we replace the random variable $X$ above by the random variable $p$, we get the famous result that $p$-value is uniformly distributed when the null hypothesis is true (indeed under the null $H_0$ there is no reason that the value of $p$ is more located in given place than in another one \underline{by definition}!).
	\end{tcolorbox}
	
	The problem that the inverse transform sampling method solves is as follows:
	\begin{itemize}
		\item Let $X$ be a random variable whose distribution can be described by the cumulative distribution function $F_X$.
		\item We want to generate values of $X$ which are distributed according to this distribution.
	\end{itemize}
	The inverse transform sampling method works as follows:
	\begin{enumerate}
		\item Generate a random number $\mathcal{U}$ from the standard uniform distribution in the interval $[0,1]$.
		\item Compute the value $x$ such that $F_X(x) =\mathcal{U}$ (using $F_X^{-1}(\mathcal{U})$).
		\item Take $x$ to be the random number drawn from the distribution described by $F_X$.
	\end{enumerate}
	Expressed differently, given a continuous uniform variable $\mathcal{U})$ in $[0, 1]$ and an invertible cumulative distribution function $F_X$, the random variable $X = F_X^{-1}(\mathcal{U})$ has distribution $F_X$ (or, $X$ is distributed $F_X$).
	\begin{figure}[H]
		\centering
		\includegraphics[scale=0.9]{img/computing/inverse_transform_method.jpg}
		\caption[Principle of the inverse transform method]{Principle of the inverse transform method (source: ?)}
	\end{figure}
	
	\pagebreak
	\paragraph{Random number generation}\mbox{}\\\\
	The best way to understand the method of Monte Carlo is to make examples (even small one should be enough). But for this, we must first have a good random number generator (which is quite difficult depending on the job). This is a very delicate and sensitive field for which an international standards is published (ISO 28640:2010 \textit{Random variate generation methods}).
	
	A random-number generator (RNG) is a computational or physical device designed to generate a sequence of numbers or symbols that cannot be reasonably predicted better than by a random chance.
	
	Several computational methods for random-number generation exist. Many fall short of the goal of true randomness, although they may meet, with varying success, some of the statistical tests for randomness intended to measure how unpredictable their results are.
	
	There are two principal methods used to generate random numbers. The first method measures some physical phenomenon that is expected to be random and then compensates for possible biases in the measurement process. Example sources include measuring atmospheric noise, thermal noise, and other external electromagnetic and quantum phenomena. For example, cosmic background radiation or radioactive decay as measured over short time-scales represent sources of natural entropy.
	
	The second method uses computational algorithms that can produce long sequences of apparently random results, which are in fact completely determined by a shorter initial value, known as a seed value or key. As a result, the entire seemingly random sequence can be reproduced if the seed value is known. This type of random number generator is often named a "pseudo-random number generator". This type of generator typically does not rely on sources of naturally occurring entropy, though it may be periodically seeded by natural sources. This generator type is non-blocking, so they are not rate-limited by an external event, making large bulk reads a possibility.
	
	Let us take, to begin, an example with the Maple 4.00b random generator:
	\begin{figure}[H]
		\centering
		\includegraphics{img/computing/maple_random_generator.jpg}
		\caption{Pseudo-random generator with Maple 4.00b}
	\end{figure}
	and:
	\begin{figure}[H]
		\centering
		\includegraphics{img/computing/maple_random_generator_restart.jpg}
		\caption[]{Pseudo-random generator restart with Maple 4.00b}
	\end{figure}
	So we see that the default random number generator used by default in Maple 4.00b should be used with extreme caution since a system reset is enough to find... equal random values! This is therefore as we already said "\NewTerm{pseudo-random generator}\index{pseudo-random generator}" that gives the possibility to make what we name "\NewTerm{pseudo Monte Carlo method}\index{pseudo Monte Carlo method}".
	\begin{figure}[H]
		\centering
		\includegraphics{img/computing/maple_random_generator_special_library.jpg}
		\caption[]{Pseudo-random library with Maple 4.00b}
	\end{figure}
	The \texttt{RAND( )} and \texttt{RANDBETWEEN( )} functions of the of Microsoft Excel 14.0.6123 are also pseudo-random generators which here is a sample of 100 simulations (of course in Microsoft Excel the chart below will change each time you press on the keyboard button: F9):
	\begin{figure}[H]
		\centering
		\includegraphics{img/computing/random_generator_plot_excel.jpg}
		\caption[]{Illustration of a sequence of pseudo-random numbers with Microsoft Excel 14.0.6123}
	\end{figure}
	Unfortunately, it may happen with pseudo-random numbers that the numbers generated are presented in bunches, that is to say by sets of numbers close to each other, which reduces the effectiveness of the Monte Carlo simulation .

	An empirical technique is to use then sequences of numbers generated by algorithms that scan almost surely the range $[0,1]$. We speak then of "\NewTerm{quasi-random numbers}\index{quasi-random numbers}" to make simulations sometimes named "\NewTerm{quasi-Monte Carlo}\index{quasi-Monte Carlo}". In almost all Microsoft Excel, you can create a Visual Basic Application function that will replace the pseudo-random generators that are the \texttt{RAND( )} or \texttt{RANDBETWEEN( )}.

	Here is an example of such a V.B.A. function which generates quasi-random number named "\NewTerm{Fauré random number sequence}\index{Fauré random number sequence}":
	
	\begin{lstlisting}[language={[Visual]Basic}, caption={VBA Fauré sequence code}]
		Function SequenceFaure(n) As Double
    		Dim f As Double, sb As Double
    		Dim i As Integer, n1 As Integer, n2 As Integer
    
    		n1 = n
    		sb = 1 / 2
    		Do While n1 > 0
		        n2 = Int(n1 / 2)
		        i = n1 - n2 * 2
		        f = f + sb * i
		        sb = sb / 2
		        n1 = n2
		    Loop
		    SequenceFaure = f
		End Function
	\end{lstlisting}
	This will gives the following sequence for a sample of $100$ simulations:
	\begin{figure}[H]
		\centering
		\includegraphics{img/computing/faure_pseudo_random_sequence_excel.jpg}
		\caption[]{Illustration of a Fauré sequence of pseudo-random numbers with Microsoft Excel 14.0.6123}
	\end{figure}
	where we see well that the sequence covers well the whole area between $0$ and $1$ (we say then that: it covers faster the integration surface). This technique is sometimes preferred because it has the advantage of keeping the values off the simulation every time we restart the simulation (therefore in Microsoft Excel the chart above will not change when you press the keyboard button F9).

	By cons the sequence generators have a great weakness: they are only applicable (to my knowledge at least) for problems of simulations with a single random variable (typically pricing single option strategy following Black \& Scholes model). Indeed if we have several random variables (and this is the most common case!), then the variables are artificially correlated (correlation coefficient = $1$) because they travel all the area between $0$ and $1$ in the same way! So a good simulation with several variables is a simulation including the treated variables have a correlation coefficient which approaches zero!!!!!!

	In addition, sequence generators require algorithms that are very time consuming when there are many variables relatively to a pseudo-random generator, this is why in most situations we prefer the old methods.
	\begin{tcolorbox}[title=Remark,colframe=black,arc=10pt]
	As we already mention it, engineers should refer to the international standard ISO 28640: 2010 when they need to implement random number generators in their softwares.
	\end{tcolorbox}
	Once the pseudo-random or random generator created and tested, we can see some application of the Monte Carlo method before continuing on performance tricks relatively to this method. Thus, in the calculation of integrals, this method is very useful and very fast in terms of convergence speed.
	
	\pagebreak
	\paragraph{Monte Carlo integration}\label{monte carlo integration}\mbox{}\\\\
	In numerical integration, methods such as the Trapezoidal rule use a deterministic approach as we already know. Monte Carlo integration, on the other hand, employs a non-deterministic approaches: each realization provides a different outcome. In Monte Carlo, the final outcome is an approximation of the correct value with respective error bars, and the correct value is within those error bars.
	
	Consider for example the calculation of the following univariate defined integral of a function $f$ and positive over the interval $[a, b]$:
	
	Given:
	
	the maximal value of the function $f$ between the bounds $[a,b]$.

	We consider the rectangle bounding function on the interval $[a, b]$ defined by vertices $\{(a,0),(b,0),(b,m),(a,m)\}$:
	\begin{figure}[H]
		\centering
		\includegraphics{img/computing/monte_carlo_onedimensional_integration.jpg}
		\caption{Basic principle of the univariate integral calculation with Monte Carlo}
	\end{figure}
	We draw a large number $N$ of random points in this rectangle. For each point, $\xi_i=(x,y)$ we test if it is below the blue curve. Given $P$ the proportion of points below this curve, we have:
	
	This approach seems to be sometimes named the "\NewTerm{Monte Carlo Stochastic Point Method}" or \NewTerm{Monte Carlo hit-or-miss approach}" or "\NewTerm{Monte Carlo poor man's sampling}".
	
	The corresponding Maple 4.00b algorithm  is given by:
	
	\texttt{>intmonte:=proc(f,a,b,N)}\\
	\texttt{local i,al,bl,m,P,aleaabs,aleaord,isabove;}\\
	\texttt{m:=round(max(f(a),f(b))*10\string^4);}\\
	\texttt{al:=round(a*10\string^4);}\\
	\texttt{bl:=round(b*10\string^4);}\\
	\texttt{aleaabs:=rand(al..bl);}\\
	\texttt{aleaord:=rand(0..m);}\\
	\texttt{k:=0;}\\
	\texttt{for i from 1 to N do}\\
	\texttt{     isabove:=(f(aleaabs()/10\string^4)-aleaord()/10\string^4)>=0;}\\
	\texttt{     if isabove then}\\
	\texttt{          k:=k+1;}\\
	\texttt{     fi}\\
	\texttt{od:}\\
	\texttt{RETURN((b-a)*max(f(a),f(b))*k/N)}\\
	\texttt{end:}\\
	
	To call this procedure in Maple, just write \texttt{>intmonte(f,a,b,N)} but replacing the first argument passed as a parameter with the expression of a function and the other arguments by numerical values (obviously!).
	
	Notice that the relation above can be rewritten
	
	where $\hat{p}$ is the estimated parameter of a Bernoulli distribution and the corresponding random variable is $\xi_i$ if we reject or accept the generated point:
	
	Then  $\{\xi_i\}$ are outcomes of independent duplicate trials and $\xi_1,\cdots,\xi_N\overset{i.i.d.}{\sim}\text{B}(1,\hat{p})$. So if $\{\xi_i\}$ follows a Bernoulli distribution, then:
	
	follows a binomial distribution $\mathcal{B}(N,k)$.
	
	We know that the mean standard deviation of a variable (without factors) following a binomial distribution is:
	
	But here we have a factor $(b-a)m/N$, so that the mean becomes:
	
	and the standard deviation will be:
	
	The both latter relation are often written in textbooks by considering (implicitly) that the range of integration has been transformed to have an interval $a=0,b=1$ and the function transformed such that maximum $m=1$. Therefore:
	
	To estimate the variance (or precision), we use the Central Limit Theorem and get\footnote{Remember that the denominator is the standard error of the mean $\sigma_\mu=\sigma/\sqrt{N}$} (don't forget that this relation is valid only if the integration has been transformed to have an interval $a=0,b=1$ and the function transformed such that maximum $m=1$):
	
	That's a useful result to build confidence interval for that Monte Carlo method.
	
	Let us consider another very common approach named "\NewTerm{Monte Carlo mean value method}" (m.v.) or "\NewTerm{Monte Carlo uniform sampling method}"!

	Let $g(x)$ be a function and suppose that we want to compute:
	
	Recall that if $X$ is a random variable with density $f(x)$, then the mathematical expectation of the random variable $Y=g(X)$ between $0$ and $1$ is given by:
	
	If a random sample is available from the distribution of $X$, an unbiased estimator of $\text{E}(g(X))$ is the sample mean:
	
	If we take a distribution $f(X^{(i)}_{[0,1]})$ that is equally likely between $0$ and $1$ like the Uniform distribution $\mathcal{U}_{[0,1]}$, then the above relation reduces to:
	
	To compute $\int_{a}^{b} g(t) \mathrm{d}t$ we make a change of variables so that the limits of integration are from $0$ to $1$. The linear transformation is $y=(t-a) /(b-a)$ and $\mathrm{d}t=(b-a)\mathrm{d}y$. Therefore:
	
	Hence:
	
	By the fundamental theorem of Monte Carlo (\SeeChapter{see section Statistics page \pageref{fundamental theorem of Monte Carlo}}) it converges with probability $1$.
	
	Notice also that by the Central Limit Theorem we have that the standard error of $I_{\text{m.v.}}$ is given by:
	
	Or using Huygens relation, given for recall by $\text{V}(X)=\text{E}(X^2)-\text{E}(X)^2$, we get:
	
	This is useful to put confidence limits or error bounds on the Monte Carlo estimate of the integral, and check for convergence.

	However, the m.v. method has two major limitations:
	\begin{itemize}
		\item I it does not apply to unbounded intervals.
		
		\item I it can be inefficient to draw samples uniformly across the interval if the function $g(x)$ is not very uniform.
	\end{itemize}
	
	\begin{tcolorbox}[title=Remark,colframe=black,arc=10pt]
	The reader can refer to our \texttt{R} or MATLAB™ free companion books to see examples with the respective script language of these two softwares for the both method presented above.
	\end{tcolorbox}
	
	Now let us just compare:
	
	Moreover, many textbooks claim that:
	
	But this is neither true nor false!
	
	\paragraph{Monte Carlo estimation of $\pi$}\mbox{}\\\\
	For the calculation of $\pi$ the principle is the same and therefore consist to use the proportion of the number of points in a quarter of circle area (this simplifies the algorithm by restricting the calculations to strictly positive coordinates) inscribed in a square of side $1$ (so the radius of the circle is also equal to $1$ obviously) relatively to the total number of points (to test if a point is outside the circle, we obviously use the Pythagorean theorem) such that:
	
	\begin{figure}[H]
		\centering
		\includegraphics{img/computing/monte_carlo_pi.jpg}
		\caption{Monte Carlo $pi$ estimate}
	\end{figure}
	The corresponding Maple 4.00b algorithm is given by:\\

	\texttt{>isinside:=proc(x,y) x\string^2+y\string^2<1 end:}\\
	\texttt{>calculatepi:=proc(N)}\\
	\texttt{local i,P,abs,ord,alea;}\\
	\texttt{alea:=rand(-10\string^4..10\string^4);}\\
	\texttt{P:=0;}\\
	\texttt{for i from 1 to N do}\\
	\texttt{     abs:=alea()/10\string^4;ord:=alea()/10\string^4;}\\
	\texttt{       if isinside(abs,ord) then}\\
	\texttt{            P:=P+1;}\\
	\texttt{       fi}\\
	\texttt{od;}\\
	\texttt{RETURN(4*P/N)}\\
	\texttt{end:}\\
	\texttt{>evalf(calculatepi(100));evalf(calculatepi(1000));\\evalf(calculatepi(10000));evalf(calculatepi(100000));}\\
	

	In terms of convergence it looks typically as:
	\begin{figure}[H]
		\centering
		\includegraphics{img/computing/monte_carlo_pi_convergence.jpg}
	\end{figure}
	
	\begin{tcolorbox}[title=Remark,colframe=black,arc=10pt]
	The reader can refer again to our \texttt{R} or MATLAB™ free companion books to see examples with the respective script language of these two softwares for the example above.
	\end{tcolorbox}
	
	\paragraph{Monte Carlo Modelling}\mbox{}\\\\
	The most common application of the Monte Carlo method in business and industry is certainly the study of random variables. Furthermore, this method is part of the ISO 31010 Risk Management standard under the name of "\NewTerm{Monte Carlo analysis}\index{Monte Carlo analysis}". Many cutting edge tech companies make Monte Carlo modelling with a spreadsheet softwares like Microsoft Excel (even multinationals!) and with a lesser extent with professional oriented softwares or add-ins such as @RISK, CrystalBall, TreeAge, Isograph or MATLAB™.
	
	The advantages of this method in modelling random variables are:
	\begin{itemize}
		\item We can integrate any distribution  including empirical one and not continuous one!

		\item Models are very easy to implement and can be expanded as needed without too much effort.

		\item All influences or relation occurring in reality (at least the identified one....) may be represented and implemented in the model.

		\item Sensitivity analysis (\SeeChapter{see section Quantitative Management Techniques page \pageref{sensitivity analysis}}) can be applied.

		\item The models are easily understandable and provide a measure of the accuracy of the result.

		\item Many inexpensive software are available (at least inexpensive in comparison to criticality of the business analysed that is most of time in the order of the billion of dollars).
	\end{itemize}
	Let us consider a simple but concrete case (widely used in business) that I like to use in my introduction training of a small project of two tasks denoted by $A$ and $B$ which follow each other without free margin (or free slack). Let us imagine that the duration of each task has been estimated in accordance with the recommendation of the Project Management Institute with a beta distribution (\SeeChapter{see section Statistics page \pageref{beta distribution}}) as learn it almost all good project managers in their training curriculum (\SeeChapter{see section Quantitative Management Techniques page \pageref{probabilitic pert}}).

	For this example, the task $A$ has an optimistic duration of $5$ days and a pessimistic duration of $8$ days. Task $B$ an optimistic duration of $1$ day and a pessimistic duration of $4$ days. We would like in the spreadsheet software Microsoft Excel using a pseudo Monte Carlo simulation (therefore necessarily based on a pseudo-random variable) introduce three traditional minimum information:
	\begin{figure}[H]
		\centering
		\includegraphics{img/computing/monte_carlo_tasks.jpg}
	\end{figure}
	\begin{itemize}
		\item A table with $3$ columns (duration of $A$, $B$ and sum of both) and $10,000$ simulations (rows)
	
		\item The graphical distribution function of the sum of the two random variables 

		\item The convergence of the 95th percentile on the $100$ first simulations (useful for the subject further below).
	\end{itemize}
	We then construct the following table of $10,000$ row (the screenshot shows only the first rows...) where all cells from row $2$ to row $10,000$ of column \texttt{A} contains the following function (Microsoft Excel 14.0.7166):
	\begin{center}
	\texttt{=BETA.INV(RAND(),3+SQRT(2),3-SQRT(2),5,8)}
	\end{center}
	and for column \texttt{B} contains the following function:
	\begin{center}
	\texttt{=BETA.INV(RAND(),3+SQRT(2),3-SQRT(2),1,4)}
	\end{center}
	and finally the cell \texttt{C1} contains the following function that was pull down until cell \texttt{C10000}:
	
	\begin{center}
	\texttt{=A2+B}
	\end{center}
	
	Obviously the values in Microsoft Excel 14.0.6123 will change each time you press the \texttt{F9} key on the keyboard.

	Then this gives us the histogram still made with the same software version (we will not detail how to building such a chart it is a basic subject of Microsoft Office knowledge and has nothing to do in a scientific book) where the $x$-axes are the number of days:
	\begin{figure}[H]
		\centering
		\includegraphics{img/computing/monte_carlo_histogram_tasks.jpg}
	\end{figure}
	and the convergence of the $95$th percentile of the first $100$ simulations (because as this example is simple, the system converges quickly enough so that we do not to need to take more than $100$ simulations as example) where the $y$-axes is the number of days:
	\begin{figure}[H]
		\centering
		\includegraphics{img/computing/monte_carlo_convergence_tasks.jpg}
	\end{figure}
	Obviously by default, in Microsoft Excel 14.0.6123 the chart above will change each time you press the \texttt{F9} key on the keyboard.
	\begin{tcolorbox}[title=Remark,colframe=black,arc=10pt]
	In the case of the simulations of random variables, we can in simple cases involving only sums or subtractions of random variables, as is the case for the example above, determine the mean and the standard deviation of the results analytically using the property of linearity of the mean and variance (because normally for the variance of two independent random variables, the covariance is zero). By analysing the difference between the analytical value and that obtained by numerical simulation, the offset can be corrected certain other statistical indicators by simply adding or subtracting the differential. This is known as the technique of "\NewTerm{control variables}\index{control variables}" that we will detail further below.
	\end{tcolorbox}
	There are other variance reduction techniques (ie: the standard deviation) that control variables technique Carlo to reduce the variance of the Monte Carlo estimators in specific conditions:

	\begin{itemize}
		\item One of these techniques is the use of "\NewTerm{antithetic variables}\index{antithetic variables}" which consists very simply (programming this technique at the high school level as you can see in the MATLAB™ companion book) to decorrelate simulations to make the covariance between the variables negative  and so reduce the global variance (such as we have seen in the section Statistics, the variance of the sum of two random variables make appear a covariance term). Unfortunately, this technique works satisfactorily with symmetric distributions this is why to my knowledge it is not implemented in simulation software available on the market.

		\item There are also the technique named "\NewTerm{stratified sampling}\index{stratified sampling}" that consist to cut the  pre-image space of the random variable in regular intervals (the programming of this technique is also at the high school level as you can see in the MATLAB™ companion book). This technique works very well when the number of simulations must be small but only in the case of a single variable. This also why, as far as we I know, it is not implemented in simulation softwares available on the market.

		\item There exist is a generalization of stratified sampling (the programming of this technique is also at the high school level) for simulations with multiple variables and that is named "\NewTerm{Latin Hypercube}\index{Latin Hypercube}" (abbreviated as "LHS" for Latin Hypercube Stratification). This technique ensures that each $n$-tuple of random variables (corresponding to a space of $n$ dimensions) uses a unique pre-image at each iteration, hence the name of that technique (Latin: refers to magic squares where each value appears uniquely, Hypercube because is an $n$-dimensional generalization of a magic square). Some simulation software available on the market implement this technique (@RISK, CrystalBall).
	\end{itemize}
	To summarize, whether that its the technique of Faure sequence generators, of antithetic variables, of control variables, of stratified sampling or Latin Hypercube even if these techniques are all easy to program, the method using the pseudo-random variables is privileged because is the most suitable for the majority of common situations in the business and in non-cutting edge scientific applications.
	
	\begin{tcolorbox}[title=Remark,colframe=black,arc=10pt]
	For the reader interested to have a deeper overview, however with out too much mathematical details, we recommend the following actual reference: \textit{Introducing Monte Carlo Methods with R} (see \cite{robert2010introducing}).
	\end{tcolorbox}
	
	\subsubsection{Bootstrapping}\label{bootstrap}
	In statistics, "\NewTerm{bootstrap techniques}\index{bootstrap techniques}" can refer to any test or metric that relies on random sampling with replacement of a population of data to run statistical inference on small samples. This methods is relatively intensive in computations still for expensive office computers at this beginning of the 21st century.  
	\begin{figure}[H]
		\centering
		\includegraphics{img/computing/bootstrap_sampling.jpg}
	\end{figure}
	The goal of bootstrapping is to find some indication of a statistic: its estimate of course, but also its dispersion (variance, standard deviation), confidence intervals or hypothesis testing. This method is based on simulations, such as Monte Carlo methods, with the difference that the bootstrap does not require additional information than the one that is available already in the initial sample. In general, it is based on new samples obtained by sampling with replacement from the original sample (then we speak also of "\NewTerm{resampling}\index{resampling}").
	
	We distinguish generally two types of bootstrap:
	\begin{enumerate}
		\item The bootstraps which make no assumptions about the probability distribution of the data analysed. We then speak then as in statistics of "\NewTerm{nonparametric bootstrap}\index{nonparametric bootstrap}".

		\item The bootstraps replacing each data measured by those corresponding to the analytical expression of the law of probability distribution assumed. We speak then of "\NewTerm{parametric bootstrap}\index{parametric bootstrap}". Once all the original values replaced, the process is exactly that of the nonparametric bootstrap.
	\end{enumerate}
	We will illustrate the principle of the bootstrap on the example of the confidence interval of the mean $\mu$ of a random variable (this special case is named a "\NewTerm{studentized bootstrap}\index{studentized bootstrap}". For this example, the confidence interval for the mean of a random variable is completely determined from the mean and the variance calculated on the sample (\SeeChapter{see section Statistics page \pageref{likelihood estimators}}).

	We consider a sample of the random variable composed of $10$ estimates:
	
	The arithmetic average of this sample is:
	
	and it standard deviation (maximum unbiased likelihood estimator of the standard deviation):
	 
	As we are in the situation of a known sample mean and an unknown sample variance, to do the calculation of a confidence interval, then we have proved in the section Statistics that we had to use:
	
	where $S$ is for recall another traditional notation in some areas of statistics for the notation of the empirical standard deviation (\SeeChapter{see section Statistics page \pageref{empirical standard deviation}}). We then have for the confidence interval at $95\%$ of the mean:
		
	Therefore:
		
	Which gives:
	
	The confidence interval can also be calculated by bootstrap (this is especially useful for complicate distributions that are not symmetric and when we focus on the median rather than on the mean). Then it is therefore obtained by the following algorithm (and abbreviated BCI for "bootstrapped confidence interval"):
	\begin{enumerate}
		\item From the initial sample, we simulate new samples of the same size, named "\NewTerm{bootstrap replicates}\index{bootstrap replicates}" of size $n$, by random draws with replacement (see figure above). For example with the previous series, we could get the following replicate:
		
		in which, by definition, some of the original sample values do not appear, and where others appear several times (yes it's a sampling with replacement therefore...). Several samples are simulated in this way. So we can form a number of replicas (arrangements with replacement) equal to (\SeeChapter{see section Probabilities page \pageref{simple arrangements with repetitions}}):
		
		Therefore with $10$ values we have $10,000,000,000$ possibilities...
		
		\item For each simulated sample, an average\footnote{in fact the process is the same for any estimate of any statistical indicator $\hat{\theta}$} is calculated (so we will have several thousand of averages!). 
	
		\item The $95\%$ confidence interval is then calculated on this set of averages (or any other estimator) by typically using the percentile calculation (through the functions of a spreadsheet software or a programming/scripting language). This grouping is named the "\NewTerm{bagging}\index{bagging}" that stands for the abbreviation of "\NewTerm{Bootstrap Aggregating}\index{bootstrap aggregating}".
	\end{enumerate}
	Obviously for each set of bootstrap, the percentiles themselves will not be the same so it is even possible to create a confidence interval for the percentiles themselves and this is named "\NewTerm{percentile bootstrap}\index{percentile bootstrap}"!  But one important assumption is that such a distribution is pivotal. This means that if the underlying parameter changes, the shape of the distribution is only shifted by a constant, and the scale does not necessarily change.
	
	It is quite easy (just like the Monte Carlo methods) to create replicas with spreadsheet software like Microsoft Excel (at least for people that know a little bit how to use a spreadsheet software) without computer programming or scripting (see below an example with Microsoft Excel)! Furthermore, the bootstrap technique is very powerful because it does not use any assumptions about the underlying statistical distribution. 

	The most common field of application of bootstrapping in "direct" business (I don't mind about Data Mining for Marketing that is not what I mean about "direct business") that I know is in project management during meetings where a dozen people estimates the duration of a project task or project phase.
	
	Bootstrapping can therefore be applied to any estimator other than the average, as the median, the correlation coefficient between two random variables or the principal eigenvalue of a variance-covariance matrix (for principal component analysis for example!), or the slope and intercept of a regression and this is its great strength!!! Indeed, for these estimators, there is no general mathematical relations that defines the standard error or confidence interval. The only methods applicable are resampling methods to which bootstrapping belongs to and this is intensively used since almost any home or office computer at the beginning of the 21st century is powerful enough to bootstrap small databases.
	\begin{tcolorbox}[colframe=black,colback=white,sharp corners]
	\textbf{{\Large \ding{45}}Example:}\\\\
	As example let us first use a spreadsheet software like Microsoft Excel table 14.0.6123 and taking the theoretical companion example above as practical software example (we prohibit ourselves of doing VBA programming). We then build a small table with the previous sample:
	\begin{figure}[H]
		\centering
		\includegraphics{img/computing/boostrapping_excel_initial_dataset.jpg}
	\end{figure}
	At the opposite of the companion theoretical example  would be able to determine a confidence interval for the median instead as for the arithmetic mean (we purposely take a statistical indicator for which there is simple analytical confidence interval). For this, we calculate the median of several thousand of replications in the column \texttt{F} (random choice!), where each replication corresponds to a row :
	\begin{figure}[H]
		\centering
		\includegraphics[scale=0.8]{img/computing/boostrapping_excel_resampling_median.jpg}
	\end{figure}
	\end{tcolorbox}
	
	\begin{tcolorbox}[colframe=black,colback=white,sharp corners]
	with the following quite long formula for Microsoft Excel Next 14.0.6123 to put in cell \texttt{F5} and then pull down to the end of the sheet:\\
	
	\texttt{=MEDIAN(INDEX($A$5:$A$14,RANDBETWEEN(1,10),1),\\
INDEX($A$5:$A$14,RANDBETWEEN(1,10),1),\\
INDEX($A$5:$A$14,RANDBETWEEN(1,10),1),\\
INDEX($A$5:$A$14,RANDBETWEEN(1,10),1),\\
INDEX($A$5:$A$14,RANDBETWEEN(1,10),1),\\
INDEX($A$5:$A$14,RANDBETWEEN(1,10),1),\\
INDEX($A$5:$A$14,RANDBETWEEN(1,10),1),\\
INDEX($A$5:$A$14,RANDBETWEEN(1,10),1),\\
INDEX($A$5:$A$14,RANDBETWEEN(1,10),1),\\
INDEX($A$5:$A$14,RANDBETWEEN(1,10),1))
	}\\
	
	So we can not have more that $10$ billion more... corresponding to the $\bar{A}_n^n$ calculate above (Microsoft Excel 14.0.6123 and after is limited  to $17,179,869,184$ cells...).\\
	
	Then simply in a cell of your choice we write:
	\begin{center}
	\texttt{=PERCENTILE(F5:F2003,0.025)}
	\end{center}
	and in another cell:
	\begin{center}
	\texttt{=PERCENTILE(F5:F2003,0.975)}
	\end{center}
	
	which will give will $2,000$ replication respectively $7$ and $29.5$.\\

	With basic knowledge of a spreadsheet software, it is possible to graphically show the convergence of the median in function the number of replications (below we used only the first $100$ replications):
	\begin{figure}[H]
		\centering
		\includegraphics[scale=0.7]{img/computing/boostrapping_excel_median_convergence.jpg}
		\caption[]{Convergence of the median in function of the number of replications}
	\end{figure}
	Obviously, this chart will look different every time you restart the simulation in Microsoft Excel 14.0.6123 by pressing the F9 key.
	\end{tcolorbox}
	
	Finally let us indicate that after having study earlier above many linear regression models this does avoid the fact that in many cases no theoretical model is adapted either to interpolate or verbatim to extrapolate some data. Therefore, if we have for each abscissa point (exogenous variable) a given quantity of values  of the output (endogenous) variable, we can therefore use the bootstrapping method which will give us the bootstrapped regression coefficients and also bootstrapped interpolated or extrapolated values! This is an extremely interesting technique  in practice of nonparametric regression! We can do such bootstrap regressions in softwares like SPSS (with additional module) or SAS but also in the free \texttt{R} software with the right packages (see the companion book on \texttt{R}).
	
	\begin{tcolorbox}[title=Remark,colframe=black,arc=10pt]
	Bootstrapping has enormous potential in statistics education and practice, but there are subtle issues and ways to go wrong. For example, the common combination of nonparametric bootstrapping and bootstrap percentile confidence intervals seems very likely to be less accurate than using $T$-intervals for small samples, though more accurate for larger samples. We therefore strongly recommend anyone to read the following paper to avoid some traps about the usage of bootstrapping: Hesterberg TC.\textit{ What teachers should know about the bootstrap: Resampling in the undergraduate statistics curriculum.} The American Statistician. 2015 Oct 2;69(4):371-86.
	\end{tcolorbox}	
	
	\begin{tcolorbox}[colback=red!5,borderline={1mm}{2mm}{red!5},arc=0mm,boxrule=0pt]
	\bcbombe Caution!!!!!!!!! Use bootstrap mainly for confidence intervals or parameters estimation. Not for testing hypotheses (under the null), unless you really know what you are doing! For NHST you should prefer permutations techniques that we will introduce further below.\\
	
	Indeed, we use bootstrap to generate data under the empirical distribution of the observed data. To get a $p$-value by bootstrap, we need to generate bootstraps under the null hypothesis!!!! So when conducting bootstrapped hypothesis tests like on the mean for example we must first subtract the empirical mean $\hat{\mu}$ of the sample and secondly add to it the mean under $H_0=\mu$. Then we compare the original observe sample mean $\hat{\mu}$ to our bootstrapped null distribution $H_0=\mu$ and see how many bootstrapped means are at least as extreme as $\hat{\mu}$ is. This gives us a $p$-value. 
	\end{tcolorbox}
	
	\pagebreak
	\subsubsection{Jackknife Resampling}\label{jackknife resampling}
	The jackknife was proposed by M.H. Quenouille in 1949 and later refined and given its current name by John Tukey in 1956 (it predates other common resampling methods such as the bootstrap). M.H. Quenouille originally developed the method as a procedure for correcting bias. Later, Tukey described its use in constructing confidence limits for a large class of estimators. It is similar to the bootstrap in that it involves resampling, but instead of sampling with replacement, the method samples without replacement.

	So in statistics, the "\NewTerm{jackknife}\index{jackknife}" is a resampling technique especially useful for variance and bias estimation these estimator converge quick enough (the number of resampling is much more limited with jackknifing than with bootstrapping). The jackknife estimator of a parameter is found by systematically leaving out each observation from a dataset and calculating the estimate and then finding the average of these calculations. Given a sample of size $N$, the jackknife estimate is found by aggregating the estimates of each $N-1$ estimate in the sample.
	\begin{figure}[H]
		\centering
		\includegraphics{img/computing/jacknife_permutation.jpg}
	\end{figure}
	This method is especially useful when:
	\begin{enumerate}
		\item Computer is now powerful enough to run a bootstrap
		\item When we don't trust the resampling method as not suited to do the situation
	\end{enumerate}
	\textbf{Definition (\#\mydef):} The "\NewTerm{delete-1 Jackknife samples}\index{delete-1 Jackknife samples}" are selected by taking the original data vector and deleting one observation from the set. Thus, there are n unique Jackknife samples, and the $i$th Jackknife sample vector is defined as:

	This procedure is obviously generalizable to $k$ deletions.

	The $i$th Jackknife replicate is defined as the value of the estimator $s(\cdot)$ evaluated at the $i$th Jackknife sample.
	
	As we will prove it further below, the jackknife standard error of the estimator is (given typically by the bootstrap package of \texttt{R} as you can see it in the corresponding companion book):
	
	where $\hat{\theta}_{(\cdot)}$ is the empirical average of the Jackknife replicates:
	
	with:
	
	For the proof let us consider the special case where the Jackknife estimator above is an unbiased estimator of the variance of the sample mean.
	\begin{dem}
	So to prove the previous relation with the sample mean we just need to prove that (\SeeChapter{see section Statistics page \pageref{standard error}}):
	
	is equal to:
	
	And to prove that latter, we write it craftily:
   
   Once the term is squared, the equation is complete, and is identically equal to the right hand term above. Thus, in the case of the sample mean, the Jackknife estimate of the standard error reduces to the regular, unbiased estimator commonly used.\\
   
   	We say sometimes then that $n-1$ is the "\NewTerm{standard error jackknife bias inflation factor}\index{standard error jackknife bias inflation factor}".
	\begin{flushright}
		$\blacksquare$  Q.E.D.
	\end{flushright}
	\end{dem}
	As practical business oriented example of the use of Jackknife let me give the example of a customer (Fortune 500 company) that has to analyse worldwide counterfeiting of its products by sampling and controlling completely each year a given area of a given city chosen randomly in various countries and calculating the sum by country of counterfeit products as it has a major influence on the strategy of the company, national politics and borders controls (the counterfeiting being manly done by the mafia). As the geographical sampling error cannot be calculated as it is not guarantee that the chosen city has the same heterogeneity overall cities of the country it is as far as we know impossible to calculate with a closed form equation a tolerance interval for the real total counterfeit. So an easy way to get a tolerance interval is to make a Jackknife resampling as it is easily acceptable for the board committee to make an analysis of what will have be the sum if for example we remove $50\%$ of the sampling (at the condition that the a posteriori power of the test is still big enough!).
	
	\subsubsection{Permutation Tests}
	A "\NewTerm{permutation test}\index{permutation test}" (also named a "\NewTerm{randomization test}\index{randomization test}" or "\NewTerm{re-randomization test}") is a type of statistical significance test in which the distribution of the test statistic under the null hypothesis is obtained by calculating all possible values of the test statistic under rearrangements of the labels on the observed data points. 
	
	To illustrate the basic idea of a permutation test, suppose we have two groups (we take the special case where the size of the groups are equal!):
	 
	whose sample means are $\bar{x}_{A}$ and $\bar{x}_{B}$, and that we want to test, at $5\%$ significance level, whether they come from the same distribution. Let $n$ be the sample size corresponding to each group. The permutation test is designed to determine whether the observed difference between the sample means is large enough to reject the null hypothesis:
	
	that the two groups have identical probability distributions (ie identical mean).
	
	The test proceeds as follows:
	\begin{enumerate}
		\item The difference in means between the two samples is calculated as we do normally. This lead to the observed value of the test statistic $T_n^{\text{(obs)}}$.
		
		\item Then the observations of groups $\vec{A}$ and $\vec{B}$ are pooled and mixed formally using a "\NewTerm{random sampling operator}\index{random sampling operator}" $\vec{\delta}_n^R$ defined by:
		
		where $r_i$ is a binary random variable defined by $r_i\in\mathbb{Z}_2$, where $\mathbb{Z}_2=\{0,1\}$ and we also define its binary negation by:
		
		Then we create a permutation of $\vec{A}$ denoted $\vec{A}'$ (same respectively for $\vec{B}$) by applying the random sampling operator:
		
		
		\item Next, the difference in sample means is calculated and recorded for this new groups $\vec{A}'$ and $\vec{B}'$ and we get a new $T_{n}^{'\text{(obs)}}$.
		
		\item We repeat the above procedure for every possible way of combining the samples (in practice we do that completely randomly then some cases - pairs of samples - may repeat).
		
		The number of ways in which $m\cdot n$ different items can be divided equally into $m$ groups ($m=2$ in our example) each containing $n$ objects and the order of the groups is important is:
		
		and the number of ways in which $m\cdot n$ different items can be divided equally into $m$ groups, each containing $n$ objects and the order of the groups is NOT important is:
		
		
		\item Finally the one-sided $p$-value of the test is calculated as the proportion of sampled permutations where the difference in means was greater than or equal to $T_{n}^{\text{(obs)}}$. The two-sided $p$-value of the test is calculated as the proportion of sampled permutations where the absolute difference was greater than or equal to $|T_{n}^{'\text{(obs)}}|$.
	\end{enumerate}
	
	The major down-side to permutation tests are that they:
	\begin{itemize}
		\item Can be computationally intensive and may require "custom" code for difficult-to-calculate statistics. This must be rewritten for every case.
		\item Are primarily used to provide a $p$-value. The inversion of the test to get confidence regions/intervals requires even more computation.
	\end{itemize}
	Hopefully it is quite easy to do that with modern software. You can see the \texttt{R} companion book for an example of a Student-$T$ permutation test.
	
	Keep in mind that as the assumption behind a permutation test is that the observations are exchangeable under the null hypothesis (assumption of "data exchangeability"), an important consequence of this assumption is that tests of difference in location (like a permutation $T$-test) require equal variance. 
	\begin{tcolorbox}[title=Remark,colframe=black,arc=10pt]
	Permutation tests are a subset of nonparametric statistics as the basic premise is to use only the assumption that it is possible that all of the treatment groups are equivalent, and that every member of them is the same before sampling began (i.e. the slot that they fill is not differentiable from other slots before the slots are filled).
	\end{tcolorbox}
	
	\pagebreak
	\subsection{Finite difference method (F.D.M.)}
	The "\NewTerm{finite element method}\index{finite element method}" (FEM) is a numerical technique for finding approximate solutions to boundary value problems for partial differential equations. It is also referred to as finite element analysis (FEA). FEM subdivides a large problem into smaller, simpler, parts, named "finite elements". The simple equations that model these finite elements are then assembled into a larger system of equations that models the entire problem. FEM then uses variational methods from the calculus of variations to approximate a solution by minimizing an associated error function.
	
	\subsubsection{One space dimension F.D.M.}\label{one space dimension fdm}
	Let us recall that we have proved in the section of Thermodynamics the following heat diffusion equation (we present here the equation reduced to only one spatial dimension):
	
	and let us notice that this equation is not very general ... (it is not relativistic and does not take into account the heat generated in the form of radiation by the concerned material concerned or many other factors ...).

	We can consider (\SeeChapter{see section Differential and Integral Calculus page \pageref{differential calculus}}) that:
	
	and:
	
	Also:
	
	Then the heat equation becomes:
	
	After rearranging, we have
	
	If we look at this relation more closely, we see that this is a simple recursion. We just need to know the initial distribution (initial conditions) $T(x,0)$ to determine the distribution then all other values as:
	
	and:
	
	etc.
	It is possible to implement such a simulation with nothing but a small spreadsheet software and a little time as we will we see just after... (using a spreadsheet software to understand the mechanism is better than using a blackbox like Maple or MATLAB™ by my experience).

	For information $h$ and $k$ are named then the "\NewTerm{mesh step}\index{mesh step}" or "\NewTerm{space step}\index{space step}" of the model.
	
	Let us see an application example with a spreadsheet software like Microsoft Excel as it is quite a good practical exercise to understand how to implement the method.
	
	So let us consider the following worksheet\footnote{source: \url{http://excelcalculations.blogspot.co.at/2011/04/solving-1d-heat-equation-using-finite.html}} where we consider a bar that is initially at a temperature of $0$ [C]and that is heated on the left-hand side at a constant temperature of $100$ [C] and where we use the relation proved earlier above:
	
	So this gives (first rows only of $1,000$ rows): 
	\begin{figure}[H]
		\centering
		\includegraphics[scale=0.62]{img/computing/heat_equation_1d_excel_calculations.jpg}
		\caption{1D Heat Equation FDM calculations in Microsoft Excel 14.0.7172}
	\end{figure}
	Explicitly for only a few rows a and few columns visible, we have:
	\begin{figure}[H]
		\centering
		\includegraphics[scale=0.42]{img/computing/heat_equation_1d_excel_formulas.jpg}
		\caption{1D Heat Equation FDM explicit formulas in Microsoft Excel 14.0.7172}
	\end{figure}
	We have above taken for boundary conditions:
	 
	
	All this with a chart view (famous figure that is painful to obtain in a spreadsheet software...):
	\begin{figure}[H]
		\centering
		\includegraphics[scale=0.7]{img/computing/heat_equation_1d_excel_plot.jpg}
		\caption{1D Heat Equation FDM plot in Microsoft Excel 14.0.7172}
	\end{figure}
	

	For readers wishing to practice with real values ... a longitudinal Iron bar of $1$ [kg] has a specific heat capacity of $450\;[\text{J}\cdot\text{kg}^{-1}\cdot\text{K}^{-1}]$, an density of almost $7.88\;[\text{kg}\cdot \text{m}^{-3}]$ and a thermal conductivity of $82\;[\text{J}\cdot\text{s}^{-1}\cdot\text{m}^{-1}\cdot\text{K}^{-1}]$.
	
	However with such a file a above the reader will see that the FDM is stable   after some trials and errors if and only if:
	
	This is what we will study now, first with a simple example, and afterwards with a more elaborated one:
	
	\paragraph{Numerical instability}\mbox{}\\\\
	Numerical instability is a huge issues in numerical computing that occurs in many situation (the most know one being the finite elements method). To explain this let us consider the following companion example:
	
	with $n\in\mathbb{N}$.
	
	An immediate calculation gives:
	
	This allow us to calculate $I_n$ by recurrence with:
	
	This problem apparently well posed mathematically leads digitally to catastrophic results. Indeed we have:
	
	even if we neglect the rounding error on $1/n$. The error on $I_n$ explode exponentially, the initial error on $I_0$ being multiplied by $10^n$ at the step $n$.
	
	\paragraph{von Neumann stability}\mbox{}\\\\
	In numerical analysis, "\NewTerm{von Neumann stability analysis}\index{von Neumann stability analysis}" (also known as "\NewTerm{Fourier stability analysis}\index{Fourier stability analysis}") is a procedure used to check the stability of finite difference schemes as applied to linear partial differential equations. The analysis is based on the Fourier decomposition of numerical error and was developed at Los Alamos National Laboratory after having been briefly described in a 1947 article by British researchers Crank and Nicolson. This method is an example of explicit time integration where the function that defines governing equation is evaluated at the current time. Later, the method was given a more rigorous treatment in an article co-authored by John von Neumann.
	
	The stability of numerical schemes is closely associated with numerical error. A finite difference scheme is stable if the errors made at one time step of the calculation do not cause the errors to be magnified as the computations are continued. A neutrally stable scheme is one in which errors remain constant as the computations are carried forward. If the errors decay and eventually damp out, the numerical scheme is said to be stable. If, on the contrary, the errors grow with time the numerical scheme is said to be unstable. The stability of numerical schemes can be investigated by performing von Neumann stability analysis. For time-dependent problems, stability guarantees that the numerical method produces a bounded solution whenever the solution of the exact differential equation is bounded. Stability, in general, can be difficult to investigate, especially when the equation under consideration is non-linear.
	 
	The von Neumann method is based on the decomposition of the errors into Fourier series. To illustrate the procedure, consider the one-dimensional heat equation:
	
	defined on the spatial interval $L$, which can be discretized as we have just proved as (in a very condensed form):
	
	where as we have just proved earlier:
	
	and the solution $T_j^n$ of the discrete equation approximates the analytical solution $T(x,t)$ of the PDE on the grid.
	
	Let us define the round-off error $\varepsilon_j^n$ as:
	
	where $T_j^n$ is the solution of the discretized PDE as we knot it that would be computed in the absence of round-off error, and $N_j^n$ is the numerical solution obtained in finite precision arithmetic. Since the exact solution $T_j^n$ must satisfy the discretized PDE exactly, the error $\varepsilon_j^n$ must also satisfy the discretized equation (superposition principle). Here we assumed that $N_j^n$ satisfies the PDE, too (this is only true in machine precision). Thus:
	
	is a recurrence relation for the error. 

	Equations:
	
	show that both the error and the numerical solution have the same growth or decay behaviour with respect to time. For linear differential equations with periodic boundary condition, the spatial variation of error may be expanded in a finite Fourier series, in the interval $L$, as:
	As we have proved it in the section Thermodynamics the solution of the PDE for the errors can be written in a condensed way for a given term:
	
	Since the difference equation for error is linear (the behaviour of each term of the series is the same as series itself), it is enough to consider the growth of error of a typical term:
	
	and as the reader will see with the next development we can simplify already the constant such that it remains:
	
	The stability characteristics can be studied using just this form for the error with no loss in generality. To find out how error varies in steps of time, substitute the relation above into:
	
	after noting that:
	
	to yield (after simplification):
	
	Using the identities (\SeeChapter{see section Trigonometry page \pageref{hyperbolic trigonometry}}):
	
	Then the prior previous relation can the be written as:
	
	Let us now define the amplification factor:
	
	The necessary and sufficient condition for the error to remain bounded is that $|G|<1$. However in our case:
	
	this is the explicit condition for stability of the numerical scheme.
	
	Note that the term:
	
	is always positive. Thus, to satisfy the prior previous relation we have:
	
	For the above condition to hold at all:
	
	we have:
	
	gives the stability requirement for the FTCS scheme as applied to one-dimensional heat equation. This is exactly the value we found when playing with the Microsoft Excel worksheet.
	
	\subsubsection{Space-time F.D.M (finite-volume method)}
	The finite-volume method (FVM) is a method for representing and evaluating partial differential equations in the form of algebraic equations]. Similar to the finite difference method or finite element method, values are calculated at discrete places on a meshed geometry. 
	
	"Finite volume" refers to the small volume surrounding each node point on a mesh. In the finite volume method, volume integrals in a partial differential equation that contain a divergence term are converted to surface integrals, using the divergence theorem. These terms are then evaluated as fluxes at the surfaces of each finite volume. Because the flux entering a given volume is identical to that leaving the adjacent volume, these methods are conservative. Another advantage of the finite volume method is that it is easily formulated to allow for unstructured meshes. The method is used in many computational fluid dynamics computing packages as illustrated below:
	\begin{figure}[H]
		\centering
		\includegraphics[scale=0.95]{img/computing/fdm_car.jpg}
		\caption{Space-time FDM for car $C_x$ study}
	\end{figure}
	\begin{figure}[H]
		\centering
		\includegraphics{img/computing/fdm_airplane_wing.jpg}
		\caption{Space-time FDM for airplane wing profile study}
	\end{figure}
	\begin{figure}[H]
		\centering
		\includegraphics{img/computing/fdm_nasa_spaceshuttle_launch.jpg}
		\caption[Space-time FDM for NASA space shuttle launch]{Space-time FDM for NASA space shuttle launch (source: NASA)}
	\end{figure}
	The F.D.M. is a therefore a veeeeery important numerical method in practice because it also gives the possibility to solves directly Maxwell's equations in the time domain and space (and also General Relativity situations). It is then the classified in the $3$D (three-dimensional quantification of space) and temporal computational methods and finds its main industrial applications in the fields of design (antennas and circuits), of electromagnetic compatibility, of the diffraction and of the propagation and electromagnetic dosimetry (living beings and waves interactions).
	
	We will discuss now the basics of the concept in a special case as in practice, programming the F.D.M. is a whole team job in itself (like the rest of this book obviously but sometimes it is useful to recall that). Indeed a loot of problems must be resolved when dealing with computer programs using F.D.M. (convergence criteria, meshing methods, boundary conditions, user input, programming language methods, etc.).

	In a electrodynamics problem treated by F.D.M., the first necessary step is to define the volume $V$ of the space and the time interval $I = [0, T]$ for which the resolution is desired (it is unrealistic at this day to hope to solve Maxwell's equations for an infinite space and for an unlimited period of time!). The volume of calculation contains the object (antenna circuit, ...) that it is desired to characterize, in response to a given excitation. Secondly, the space (meshing of $V$) and time should be discretized to allow a numerical implementation of the resolution (and in reality the meshing is not uniform...). The problem then becomes the one of determining the field at any point of the mesh for any discrete moment of the observation time interval. The spatial and temporal discretization will be specified in what will follow below and will naturally come from physics equations to solve. They obviously condition both the accuracy of the calculation results and the computing resources required to carry it out.

	The structuring of the F.D.M. mesh and the resolution method directly result of the equations to solve.
	\begin{figure}[H]
		\centering
		\includegraphics[scale=0.9]{img/computing/fdm_mesh_01.jpg}
		\caption{Airplane typical FDM meshing}
	\end{figure}
	\begin{figure}[H]
		\centering
		\includegraphics{img/computing/fdm_mesh_02.jpg}
		\caption{Mechanical element FDM meshing}
	\end{figure}
	In a linear, homogeneous, isotropic, non-dispersive and non-magnetic (...) material, the Maxwell equations will be written explicitly based on the third Maxwell's equation (\SeeChapter{see section Electrodynamics page \pageref{third maxwell equation}}):
	
	Either explicitly with the negative sign put at the other side of the equality:
	
	And we will also use the fourth Maxwell equation without sources:
	
	Thus explicitly and rearranged:
	
	That is to say for summary:
	That is to say for summary:
	
	In what follows, we will only concerned with the first equation, the other leading to similar developments.

	To allow a computer processing, the various derivatives present in the equation must be approximated numerically as we already know. To do this, we use the principle of centered finite difference which is based on the following Taylor series expansions for recall (\SeeChapter{see section Sequences and Series page \pageref{taylor series}}):
	
	We then have on the basis of this principle:
	
	If we neglect the terms of the second order, it comes by subtracting the two series:
	
	where $\varepsilon(\mathrm{d}_x^2$ is an error or order $2$, neglected thereafter (we notice that this is the centering that, allowing compensation of second derivatives,minimizes the error in the approximation).
	
	Applying this principle to temporal and spatial derivatives of:
	
	it comes:
	
	or after rearrangement:
	
	This relation shows that if we know the components $E_y,E_z$ of electric field at time $t$ and the component $B_x$ of the magnetic field at the earlier time $t-\mathrm{d}_t/2$, it is possible to determine $B_x$ at the time $t+\mathrm{d}_t/2$. Obviously the process is exactly the same for all other components and shows the same time lag. This result suggests using an iterative numerical solution, in which the electric and magnetic fields are evaluated alternately, respectively at the discrete time $n\mathrm{d}_t$ and $(n+1/2)\mathrm{d}_t$, $\mathrm{d}_t$ being the time step (denoted $\Delta t$ by the computer scientists). It is customary in the literature to denote by $B_x^{n+\frac{1}{2}}$ the component of the magnetic field at the time $(n+1/2)\mathrm{d}_t)$.
	
	The same analysis applies for the spatial distribution of the field on the observed points. Thus, evaluating $B_x$ at the point $(x,y,z)$ is based on the knowledge of $E_y$ at the points $(x,y,z+\mathrm{d}_z/2)$ and $(x,y,z-\mathrm{d}_z/2)$  and of $E_z$ at the points $(x,y+\mathrm{d}_y/2,z)$ and $(x,y-\mathrm{d}_y/2,z)$.

	So we can summarize this geometrically in the following figure named "\NewTerm{Yee cell}\index{Yee cell}":
	\begin{figure}[H]
		\centering
		\includegraphics{img/computing/yee_cell_generic.jpg}
		\caption{Generic Yee cell in parallelepiped mesh element}
	\end{figure}
	The electric field components are evaluated at the centers of the edges of the mesh and the components of the magnetic field at the centers of the faces so as to ensure the alternation imposed by the equations (as mentioned previously we name "Yee cell" the unit cell with this distribution of points).
	
	In the special case of a propagating electromagnetic wave, the $\vec{E}$ and $\vec{B}$ fields are always perpendicular (\SeeChapter{see section Electrodynamics page \pageref{perpendicularity electric magnetic field wave}}) in a homogeneous, linear, anisotropic medium (this type of media includes many things like air, water, glass without stress or tempering) and when the engineers prefers to represent the "real" magnetic field $\vec{H}$ instead of the magnetic excitation $\vec{H}$ we can found in the literature the following parallelepiped Yee cell figure:
	\begin{figure}[H]
		\centering
		\includegraphics{img/computing/yee_cell_special.jpg}
		\caption{Yee cell in parallelepiped mesh element for EM wave propagating in vacuum}
	\end{figure}
	So globally in vacuum without the presence of any object we have a meshing of the space that can be represented as follows:
	\begin{figure}[H]
		\centering
		\includegraphics{img/computing/yee_cell_meshing.jpg}
		\caption{Typical simple meshing with multiple Yee cells}
	\end{figure}
	Obviously, we can perform the calculation routines with a value of the permittivity and permeability that are not necessarily equal in all the cells. Allowing in addition to model the propagation of electromagnetic waves in heterogeneous and non isotropic media.
	
	Finally, it is important to notice that the spatial and temporal mesh step must be configured by the user running such simulations. This for computing resources reasons as well as accuracy goals. Indeed, we do not do the same simulations for  a multi-physics system in the low frequency in  vacuum than for non-isotropic material at high frequency and, either on a desktop computer or on a supercomputer.
	
	\pagebreak
	\subsection{Data Mining}\label{data mining}
	Data Mining also popularly referred to as "knowledge discovery from data (KDD)" or "statistical learning" is just a term to group a family of mathematical techniques that use data to extract value whose first levels of granularity is given by\footnote{As said by Arthur Samuel, a computer program is said to learn (ie machine learning) from experience $E$ with respect to some class of tasks $T$ and performance measure $P$ if its performance at tasks in $T$, as measured by $P$, improves with experience $E$.}:
	\begin{figure}[H]
		\centering
		\includegraphics[scale=0.25]{img/computing/datamining.jpg}
	\end{figure}
	The main purpose of Data Mining is to ascend the following analytics levels:
	\begin{figure}[H]
		\centering
		\includegraphics[width=1.0\textwidth]{img/computing/descriptive_diagnostic_projective_prescriptive.jpg}
		\caption{Descriptive, Diagnostic, Predictive and Prospective analytics}
	\end{figure}
	Or an extended and detailed version of the above figure but as a table:
	\begin{table}[H]
		\begin{tabular}{l|l}
		\rowcolor[HTML]{C0C0C0}
		\textbf{Analytics Steps} & \textbf{Typical Questions Addressed} \\ \hline
		\begin{tabular}[c]{@{}l@{}}Descriptive\\ analytics\end{tabular} & \begin{tabular}[c]{@{}l@{}}
		\textbullet What is the current situation? What's happening?\\
		\textbullet What has change? What's new?\\
		\textbullet What should we focus on? What should we worry about?\\
		\textbullet Is this the right data to use for making the prediction?\\
		\textbullet How reliable is the data we are using here?
		\end{tabular} \\ \hline
		\begin{tabular}[c]{@{}l@{}}Predictive\\ analytics\end{tabular} & \begin{tabular}[c]{@{}l@{}}
		\textbullet If we do not change what we are doing, what will (probably)\\
		happen next? When? How likely are the different possibilities?\\
		\textbullet Given observed (or assumed) values for some variables, what are\\
		the probabilities for values of other variables?\\
		\textbullet  How well can some variables be predicted from others?\\
		\textbullet How well can future outcomes be predicted now?\\
		\textbullet  Are we certain that's what we want to predict?\\
		\textbullet  Are we really aware of the limits of the predictions?
		\end{tabular} \\ \hline
		\begin{tabular}[c]{@{}l@{}}Causal\\ analytics\end{tabular} & \begin{tabular}[c]{@{}l@{}}
		\textbullet Diagnosis, explanation, and attribution: What explains the current\\
		situation?\\
		\textbullet What can we do about it?\\
		\textbullet How would different actions change the probabilities of\\
		 different future outcomes?\\
		\textbullet What decisions will this be used to make?\\
		\textbullet What's the cost of getting those decisions wrong?\\
		\textbullet How much will decisions affect the targeted people\\ 
		(especially the vulnerable one!)?
		\end{tabular} \\ \hline
		\begin{tabular}[c]{@{}l@{}}Prescriptive\\ analytics\end{tabular} & \begin{tabular}[c]{@{}l@{}}
		\textbullet What should we do next? What decisions and policies implemented\\
		now will most improve probabilities of future outcomes?
		\end{tabular} \\ \hline
		\begin{tabular}[c]{@{}l@{}}Evaluation \\ analytics\end{tabular} & \begin{tabular}[c]{@{}l@{}}
		\textbullet How much will it cost to deploy and maintain the models?\\
		\textbullet How well are our current decisions and policies working?\\
		\textbullet What effects have our decisions and policies actually caused?\\
		\textbullet How do different policies affect behaviour and outcomes for different\\
		people?
		\end{tabular} \\ \hline
		\begin{tabular}[c]{@{}l@{}}Learning \\ analytics\end{tabular}& \begin{tabular}[c]{@{}l@{}}
		\textbullet What decisions or policies might work better than our current ones?\\
		\textbullet How can we use data and experimentation to find out?\\
		\textbullet By how much do different items of information improve decisions?\\
		\textbullet What is the value of information for different measurements?
		\end{tabular} \\ \hline
		\begin{tabular}[c]{@{}l@{}}Collaborative \\ analytics\end{tabular} & \begin{tabular}[c]{@{}l@{}}
		\textbullet How can we best work together to improve probabilities of future\\
		outcomes?\\
		\textbullet Who should share what information with whom, how and when?\\
		\textbullet What actions should each division of an organization or each member\\
		of a team take?		
		\end{tabular}
		\end{tabular}
		\caption{Component of risk analytics}
	\end{table}
	A second level of granularity would be (we will come back on a more precise definition further below):
	\begin{figure}[H]
		\centering
		\includegraphics[width=0.6\textwidth]{img/computing/data_science_map.pdf}
		\caption{Data Mining/Machine Learning non-exhaustive orgchart techniques and tools}
	\end{figure}
	and a third level of granularity would give\footnote{this list is strongly inspired by Wikipedia, SPSS, SAS, \texttt{R}, Rapid Miner and Tanagra softwares options. For a neural networks list see the figure further below of Fjodor van Veen where $27$ different neural networks type are given.} (sorry it's quite long but "data scientists", managers and IT staff in my teachings ask me many times to have an exhaustive one-place list):
	\begin{itemize}
		\item \textbf{Variables transformations:}
			\begin{small}
			\begin{multicols}{2}
			\begin{enumerate}
				\item Percentage ("normalization")
				\item Logarithmic transformation
				\item Center transformation (ie mean normalization)
				\item $Z$-transformation (centered-reduced / standardized)
				\item Rank transformation
				\item Percentile transformation (scaling)
				\item Range transformation (ie Min-Max transformation)
				\item Robust scaling
				\item Box-Cox transformation
				\item Modified Box-Cox transformation
				\item Johnson transformation (Yeo-Johnson)
				\item Variance stabilizing transformation
				\item Fisher transformation
				\item Lambert W x F transformation
				\item ORQ normalization
				\item Manly's exponential transformation
				\item John/Draper’s modulus
				\item Bickel/Doksum's modified Box-Cox
				\item Whitening
				\item ...
			\end{enumerate}
			\end{multicols}
			\end{small}
		\item \textbf{Distance metrics:}
			\begin{small}
			\begin{multicols}{2}
			\begin{enumerate}
				\item Euclidean distance (special case of Hölder's distance)
				\item Mahalanobis distance
				\item Manhattan distance
				\item Canberra distance
				\item Chebyshev distance
				\item Earth mover's distance
				\item Hölder distance
				\item Cosine similarity
				\item Minkowski distance
				\item Haversine distance
				\item Hamming distance
				\item Jaccard distance
				\item Chi-2 distance
				\item Kendall rank correlation distance
				\item Heterogeneous euclidean overlap Metric (HEOM)
				\item...
			\end{enumerate}
			\end{multicols}
			\end{small}
		\item \textbf{Sampling and validation techniques:}
			\begin{small}
			\begin{multicols}{2}
			\begin{enumerate}
				\item Simple random sampling with/without replacement (SRWR)
				\item Cluster sampling
				\item Systematic sampling
				\item Stratified random sampling
				\item Bootstrap sampling
				\item Conveniance sampling
				\item Judgemental or purposive sampling
				\item Snowball sampling
				\item Quota sampling
				\item Jackknife sampling
				\item Latin Hypercube sampling
				\item V-Fold cross-validation
				\item Leave-one-out cross-validation
				\item Monte Carlo cross-validation
				\item Group V-fold cross-validation
				\item Rolling origin forecast resampling
				\item Nested or double resampling
				\item ...
			\end{enumerate}
			\end{multicols}
			\end{small}
		\item \textbf{Dimensionality reduction (feature extraction+feature selection):}
			\begin{small}
			\begin{multicols}{2}
			\begin{enumerate}
				\item Correlation coefficients (Pearson's, Kendall, MIC, etc.)
				\item Stepwise regression
				\item Forward/Backward-logit
				\item CFS (Correlation Feature Selection) filtering (Hall \& Smith CFS)
				\item FCBF filtering (Yiu \& Liu Fast Correlation Based Filter)
				\item mRMR (Minimum-redundancy-maximum-relevance) feature selection
				\item Chi-2 feature ranking
				\item Fisher Criterion Scoring
				\item Battiti's MIFS (Mutual Information-based Feature Selection Method) feature filtering
				\item NIMFS (Normalized mutual information feature) selection
				\item MODTree (Multivalued Oblivious Decision) filtering (Lallich \& Rakotomalala MODTree)
				\item Non-negative matrix factorization dimension reduction
				\item ReliefF (Kira \& Rendell)
				\item Runs filtering
				\item Stepdisc (Wilk's partial lambda)
				\item Oracle Minimum Description Length (MDL)
				\item PCA (Principal Component Analysis)
				\item SVD (Singular Value Decomposition)
				\item CCA (Curvilinear Component Analysis)
				\item ICA (Independent Component Analysis)/Kernel ICA			
				\item Parallel Factor Analysis (PARAFAC)
				\item Tucker Decomposition (PARAFAC generalization)
				\item Sammon mapping
				\item Kernel PCA / SVD
				\item Projection Pursuit (PP) 
				\item Laplacian eigenmaps
				\item Mixture Discriminant Analysis (MDA)
				\item Regularized Discriminant Analysis (RDA)
				\item Flexible Discriminant Analysis (FDA)
				\item Latent Dirichlet allocation
				\item Isomap
				\item Locally linear embedding
				\item Maximum variance unfolding
				\item Synthetic Minority Over-sampling Technique (SMOTE)
				\item Adaptive Synthetic Sampling Approach (ADASYN)
				\item Autoencoder
				\item $T$-SNE ($T$-distributed stochastic neighbour embedding) 
				\item Generalized Low Rank Models (GLRM)
				\item Multivariate (Soft) Self Modelling Curve Resolution (MCR)
				\item Recursive Feature Selection
				\item ...
			\end{enumerate}
			\end{multicols}
			\end{small}
		
		\item \textbf{Statistical indicators (parametric and nonparametric\footnote{indicators with the exponent $^\text{np}$ are nonparametric}):}
			\begin{small}
			\begin{multicols}{2}
			\begin{enumerate}
				\item Mean (arithmetic, geometric, harmonic, weighted, LS...)
				\item Bias
				\item M-Estimators$^\text{np}$ / L-Estimators$^\text{np}$ / W-estimators / S-estimators
				\item Maximum, Minimum, Range$^\text{np}$
				\item Median$^\text{np}$, Pseudomedian$^\text{np}$ (Hodges-Lehmann estimator)
				\item Median absolute deviation (MAD)
				\item Quantiles, Upper and Lower Hinge, Interquartile range$^\text{np}$			
				\item Mode$^\text{np}$
				\item Variance, semivariance, standard deviation/MSE (biased or unbiased)
				\item Fluctuation interval
				\item Skewness (Pearson's, Bowleys's, Kelly's, Groeneveld \& Meeden's, Medcouple)
				\item Kurtosis
				\item Bimodal coefficient
				\item Pearson correlation, adjusted correlation, partial correlation, semi-partial correlation, biweight correlation
				\item Blomqvist’s correlation$^\text{np}$
				\item Spearman$^\text{np}$, Kendall$^\text{np}$ correlations
				\item McFadden, Cox \& Snell's, Nagelkerke, Efron, McKelvey \& Zavoina, Count, Adjusted count pseudo-correlations\footnote{Pseudo-correlations are used in logistic regression}
				\item Lin' correlation
				\item Shepherd’s Pi correlation
				\item d-variable Hilbert-Schmidt independence criterion (dHSIC)
				\item $p$-value
				\item $\beta$ of a NHST (power of test)
				\item Coefficient of variation
				\item Effect size ($\omega^2$, partial $\omega^2$,$\eta^2$, Cohen's $f^2$, Cohen's $q$, Cohen's $w$, Cohen's $h$, Glass' $\Delta$, $\Psi$, Cliff's $\Delta$, Risk ratio, odds ratio)
				\item Cohen's $\kappa$
				\item Yule Q coefficient
				\item Gini impurity index (ie  Somers' D)
				\item Area Under the Curve (AUC)
				\item Harrell’s $C$-index
				\item $F_1$ score, $F_\beta$ score
				\item Piotroski score
				\item Intraclass correlation coefficient
				\item Bangdiwala's B
				\item Hartley entropy
				\item Rényi entropy
				\item Moran's I correlation
				\item Fleiss Kappa
				\item Pearson's measure of mean square contingency
				\item Cramér'V ($\phi_c$)
				\item Tschuprow's T 
				\item Scott's $\pi$
				\item $\phi$-coefficient (Matthews correlation coefficient)
				\item Point-Biserial correlation coefficient
				\item Confusion matrix $\mathcal{C}$
				\item Mallow's $C_p$
				\item Akaike information criterion (AIC)
				\item Corrected Akaike information criterion (AICc)
				\item Bayesian information criterion (BIC)
				\item Bayes factor (BF)
				\item Deviance information criterion (DIC)
				\item Hannan–Quinn information criterion (HQC)
				\item Focused information criterion (FIC)
				\item Mutual information criterion (MIC)
				\item Bayesian predictive information criterion (BPIC)
				\item Covariance inflation criterion (CIC)
				\item Risk inflation criterion (RIC)
				\item Relative error, Absolute error, Final prediction error (FPE)
				\item Widely applicable information criterion (WAIC)
				\item Shannon entropy (+perplexity)
				\item Specificity
				\item Log-loss (LL)
				\item Sensitivity
				\item Jensen–Shannon divergence
				\item Kullback–Leibler divergence
				\item Aggregation indices (econometric)
				\item Helsel's coefficient
				\item Minimum message length (MML)
				\item Freeman's $\theta$
				\item $\varepsilon$-squared
				\item Pearson contingency coefficient
				\item Goodman's Gamma
				\item Guttman monotonicity coefficient$^\text{np}$
				\item Rand index
				\item Concentration coefficient
				\item Uncertainty coefficient
				\item ...
			\end{enumerate}
			\end{multicols}
			\end{small}
		
		\item \textbf{Statistical tests\footnote{Some of these tests are proved in the section Statistics page \pageref{statistics}. Some items in the lists below are not really tests but more "procedures" or "methods". And obviously some tests are just special cases of others tests!!!} (parametric, semi-parametric and nonparametric\footnote{Tests with the exponent $^\text{sp}$ are semi-parametric, those with $^\text{np}$ are nonparametric}):}
			\begin{small}
			\begin{multicols}{3}
			\begin{enumerate}
				\item (Brown-)Mood's test$^\text{np}$
				\item (Siegel-)Tukeys HSD test$^\text{np}$
				\item Abelson-Tukey score test
				\item Ahsanullah test$^\text{np}$
				\item Adjacency test$^\text{np}$
				\item ADF-GLS test
				\item 2-AFC/3-AFC (alternative force choice) tests
				\item Agostino test of skewness for normality
				\item Agresti–Pendergrast test$^\text{np}$
				\item (Hodges)-Ajne's test$^\text{np}$
				\item Anscombe-Glynn test of kurtosis
				\item Anderson-Darling's adequation test$^\text{np}$
				\item ANCOVA test
				\item ANOVA/MANOVA tests (+Welch's ANOVA, Friedman's ANOVA$^\text{np}$, Kruskal-Wallis ANOVA$^\text{np}$, Sheirer-Ray-Hare ANOVA, Van der Waerden ANOVA$^\text{np}$)
				\item Ansari-Bradley's test$^\text{np}$
				\item Anscombe-Glynn test
				\item Aroian test
				\item (Aspin-)Welch test
				\item Baumgartner-Weiss-Schindler test$^\text{np}$
				\item Baringhaus-Henze-Epps-Pulley (BHEP) test
				\item Bartholomew's likelihood ratio test
				\item Barnard's test$^\text{np}$
				\item Bartels rank test$^\text{np}$ of randomness
				\item Bartlett-(Kendall) test for variances
				\item Bartlett's test for sphericity
				\item Bartlett-Diananda test
				\item Bayesian A/B tests
				\item Beran's tests
				\item Behrens-Fisher test
				\item Begg test
				\item Besag-Newell's R test
				\item BDS (Brock-Dechert-Scheinkman) test for non-linear serial dependence in time series
				\item Bimodality test
				\item (Binomial) sign test$^\text{np}$
				\item Binomial test$^\text{np}$ 
				\item Bithell’s linear rank score test$^\text{np}$ 
				\item Birch test$^\text{np}$ 
				\item Bivariate sign test$^\text{np}$ 
				\item Bhapkar test$^\text{np}$
				\item Blumen's test$^\text{np}$
				\item Blum-Kiefer-Rosenblatt independence test
				\item BMP (Boehmer, Musumeci and Poulsen) cross-sectional test
				\item Bonett-Seier test of kurtosis
				\item Bonferroni outlier test
				\item Boschloo's test$^\text{np}$
				\item Bowker's test for symmetry$^\text{np}$
				\item Bowman and Shenton combination tests
				\item Box's (M) test
				\item Box–Pierce test
				\item Brant test (proportional odds test)
				\item Breslow-Day test
				\item Breusch-Pagan-Godfrey's test for homogeneity of variances
				\item Brown–Forsythe's test
				\item Brunk's$^\text{np}$ test
				\item Brunner-Munzel generalized Wilcoxon test$^\text{np}$
				\item Buishand range and U test
				\item Buy-and-hold abnormal return test (BHAR)
				\item Butler-Smirnov test
				\item Capon test$^\text{np}$
				\item Chacko's test$^\text{np}$
				\item Chen-Wolfe test$^\text{np}$
				\item Chi-squared test for association / independence or homogeneity$^\text{np}$
				\item Chi-2 test for independence with/without Yates correction
				\item Chi-squared test for outlier
				\item Chi-2 test for adequation$^\text{np}$
				\item Chow's test
				\item Chakraborti-Desu test$^\text{np}$
				\item Circular-cone test
				\item Clark-Evans' test
				\item Cliff-Ord tests
				\item Cochran's C-test
				\item Cochran's Q-test$^\text{np}$
				\item Cochran-Armitage's test$^\text{np}$ (aka Chi square test for trend)
				\item Cochran-Mantel-Haenszel's test$^\text{np}$
				\item Conover's test$^\text{np}$
				\item Corner test for association$^\text{np}$
				\item Comparing correlation coefficients of overlapping (CCO) samples
				\item Cox's F-test$^\text{np}$	
				\item Cox and Oakes test$^\text{np}$
				\item Cox–Small test		
				\item Cox-Stuart trend test$^\text{np}$	
				\item Corrado rank test$^\text{np}$	
				\item Correlation for categorical variables test
				\item Cramer–von Mises' test$^\text{np}$
				\item Cross sectional $T$-test
				\item CAR (cumulative abnormal return) test
				\item Cusum test for structural change
				\item Cuzick–Edwards test
				\item Cuzick's trend test$^\text{np}$
				\item d-variable Hilbert Schmidt independence criterion test$^\text{np}$
				\item Daniel's test for trend$^\text{np}$
				\item Davidson-MacKinnon test for comparing non-nested models
				\item David-Barton test
				\item David-Hellwig test
				\item Davis-Quade's test$^\text{np}$
				\item Demsar test$^\text{np}$
				\item Deshpande test$^\text{np}$
				\item Dickey–Fuller's test$^\text{np}$
				\item Difference between two non-overlapping dependent correlation coefficients test
				\item Diggle and Chetwynd’s test
				\item Dixon's test
				\item Doornik-Hansen test
				\item Duckworth's test$^\text{np}$
				\item Dudewicz-van der Meulen test 
				\item Duncan's multiple range comparison (MRT) test
				\item Dunn's test$^\text{np}$
				\item Dunnett's $T$, C or T3 test
				\item Duo-Trio test
				\item Durbin's rank test$^\text{np}$
				\item Durbin-Watson autocorrelation test
				\item Durbin-Wu-Hausman test
				\item Dwass-Steele-Critchlow paired test
				\item Dzhaparidze-Nikulin test
				\item Egger's test
				\item Elliott-Rothenberg stock test
				\item Engelman-Hartigan test
				\item Engle test
				\item Engle-Granger cointegration test
				\item Epstein's test
				\item Epps and Pulley test
				\item Epps-Singleton (ES) test
				\item Excess mass test$^\text{np}$
				\item Extreme rank sum test$^\text{np}$
				\item Fagerland-Hosmer-Bofin test$^\text{np}$
				\item Fieller's test (A/B test)
				\item Filliben's probability plot correlation test
				\item Fisher-Behrens test
				\item Fisher's Exact (Fisher-Irwin) test$^\text{np}$
				\item Fisher-Freeman-Halton Exact test$^\text{np}$
				\item Fisher's LSD (Least Significant Difference) test
				\item Fisher-Hayter's LSD test
				\item Fisher-Pitman's permutation test$^\text{np}$
				\item Fisher's $G$-test for periodicity
				\item Fisher's variance test
				\item Fisher's cumulant test
				\item Fisher-Yates test
				\item Fisher-Yates-Terry-Hoeffding test$^\text{np}$
				\item Fligner(-Killeen)'s test for homogeneity of variances$^\text{np}$
				\item Fligner–Wolfe's test$^\text{np}$
				\item Fligner–Policello's test$^\text{np}$
				\item Foutz' $F_n$ test
				\item Fung-Paul test
				\item Freeman–Tukey's test
				\item Freund-Ansari-Bradley test for scale$^\text{np}$
				\item Friedman correlation test$^\text{np}$
				\item Friedman–Rafsky test 
				\item Frosini test
				\item $G$-test
				\item Gabriel's pairwise test
				\item Galton's rank order test$^\text{np}$
				\item Gart's test
				\item Gel-Gastwirth test
				\item Generalized extreme studentized deviate test$^\text{np}$ (GESD)
				\item Generalized rank $T$-test$^\text{np}$
				\item Generalized rank $Z$-test$^\text{np}$
				\item Generalized sequential probability ratio test$^\text{np}$ (GSPR)
				\item Generalised sign test$^\text{np}$
				\item Games-Howell test$^\text{np}$
				\item Geary's test
				\item Gehan's generalized Wilcoxon test$^\text{np}$
				\item Gini test (Gail and Gastwirth)
				\item Glejser's test
				\item Gnedenko $F$-test$^\text{np}$
				\item Goldfeld–Quandt's test
				\item Goodman-Kruskal's Gamma test$^\text{np}$
				\item Goodman-Kruskal's Lambda test$^\text{np}$
				\item Goodman-Kruskal's tau test$^\text{np}$
				\item Gore's test$^\text{np}$
				\item Grambsch-Therneau test of proportionality
				\item Granger causality test
				\item Greenhouse-Geisser test
				\item Greenwood's test
				\item Grubbs' test
				\item Gupta's test$^\text{np}$
				\item Harrington and Fleming's Gp tests
				\item Harris-Gnedenko test$^\text{np}$
				\item Harrison-McCabe test for heteroskedasticity
				\item Harrison-Kanji-Gadsden test
				\item Hartigan's Dip (HDS) test
				\item Hartley's (Fmax) test
				\item Harbord's test
				\item Harvey-Collier test for linearity
				\item Haugh's test
				\item Hayter-Stone test$^\text{np}$
				\item Hayter-Nasimoto-Wright test
				\item Hegazy-Green test
				\item Henze-Zirkler test
				\item Hettmansperger–McKean test$^\text{np}$
				\item Hirsch-Slack correlation test$^\text{np}$
				\item Hochberg's GT2 pairwise test$^\text{np}$
				\item Hodges' bivariate sign test$^\text{np}$
				\item Hodges-Lehmann test
				\item Hoeffding's independence test$^\text{np}$
				\item Hollander's bivariate symmetry test$^\text{np}$
				\item Hollander's parallelism test$^\text{np}$
				\item Hollander-Proshan test$^\text{np}$
				\item Hopkins-Skellam test
				\item Hosmer–Lemeshow test$^\text{np}$
				\item Hosmer-Le Cessie test$^\text{np}$
				\item Hosking test
				\item Hotelling's T2-test 
				\item Hsu's MCB test
				\item Hudson-Martin-Arguadé test (HKA)
				\item Huynh-Feldt test
				\item Imam test$^\text{np}$
				\item J-tests
				\item Jacquez’s $k$-nearest neighbours test
				\item Jalal and Jamshidian test of homscedasticity
				\item Jarque-Bera Normality test
				\item Jennrich test of the equality of two matrices\footnote{Correlation matrices typically!}
				\item Johansen cointegration test
				\item Jonckheere-Terpstra trend test$^\text{np}$
				\item Johnson-Mehrotra test$^\text{np}$
				\item Inner-Wedge test
				\item K test (Kleibergen)
				\item Kaplan-Meier test$^\text{np}$
				\item Kaiser-Meyer-Olkin test
				\item Kendall rank correlation test$^\text{np}$
				\item Kendall's tau-b test$^\text{np}$
				\item Kendall's tau-c test$^\text{np}$
				\item Kendall's concordance W test$^\text{np}$
				\item Kendall's tau test$^\text{np}$
				\item Kenward-Roger tests
				\item Kimber-Michael test
				\item KPSS (Kwiatkowski, Phillips, Schmidt, and Shin) test
				\item Klotz scale test$^\text{np}$
				\item Knox's tests
				\item Kolmogorov-Smirnov test$^\text{np}$
				\item Kowalski's bivariate line test
				\item Koziol-Nemec test$^\text{np}$
				\item Kruskal-Wallace test 
				\item Kuiper's test of uniformity$^\text{np}$
				\item $k$-ratio $T$-test
				\item LaBreque's tests for non-linearity
				\item Lagrange multiplier (LM) score test
				\item Lanzante's test$^\text{np}$
				\item Larocque and Labarre sign test$^\text{np}$
				\item Le's test$^\text{np}$
				\item Lehmacher test$^\text{np}$
				\item Leybourne-McCabe test
				\item Lepage test
				\item Levene's test
				\item Li-Mak test 
				\item Likelihood ratio test (LRT)
				\item Lilliefors test
				\item Lim–Wolfe test$^\text{np}$
				\item Linear Time MMD$^\text{np}$
				\item Link–Wallace test
				\item Lin-Mudholkar test
				\item Lin-Wang test
				\item Lin-Xu test
				\item Lipsitz goodness of fit test$^\text{np}$
				\item Little’s test for data MCAR
				\item Ljung-Box test
				\item Locally most powerful rank order test$^\text{np}$
				\item Log-rank (Mantel-Cox) test$^\text{np}$
				\item Locke and Spurrier tests
				\item Lu-Smith Normal score test
				\item Macaskill test
				\item Mack–Wolfe's test$^\text{np}$
				\item Madhava Rao-Raghunath test$^\text{np}$
				\item Mann-Kendall trend test$^\text{np}$
				\item Mann-Whitney's test$^\text{np}$
				\item Mantel-Haenszel log-rank test$^\text{np}$
				\item Mantel's test
				\item Mardia's test of multivariate normality
				\item Mardia–Watson–Wheeler's test
				\item Martinez-Iglewicz test
				\item Mathisen's test$^\text{np}$
				\item Mauchly's test
				\item max-combo test
				\item McLeod-Li test
				\item McCabe–Tremayne's test
				\item McDonald-Kreitman test$^\text{np}$
				\item McNemar's test$^\text{np}$
				\item McCulloch test
				\item Mean slippage test (Schwager-Margolin)
				\item Median test$^\text{np}$
				\item Miller test$^\text{np}$
				\item Michael's test
				\item Mood's runs test$^\text{np}$
				\item Mood's scale test$^\text{np}$
				\item Mojena's test
				\item Moran's I test
				\item Moses test$^\text{np}$
				\item Mosteller's $k$-sample slippage test
				\item Mudlarks-McDermott test of ordered variance
				\item Mukerjee-Robertson-Wright multiple-contrast test
				\item Multi-binomial test
				\item Murakami $k$-Sample BWS normal test$^\text{np}$
				\item Nagarwalla’s scan statistic
				\item Nearest distance test (Andrews-Bickel-Hampel-Huber-Rogers-Tukey)
				\item Nemenyi-Damico-Wolfe test$^\text{np}$
				\item (Student)-Newman–Keuls (SNK) test (contested)
				\item Neuhauser's test (Murakami's B2 test)
				\item Neumann trend test\footnote{Also named "adjacency test" or "mean successive difference test".}
				\item Neyman's smooth goodness of fit test
				\item Nikulin-Rao-Robson test
				\item Noether's test for cyclical trend
				\item Norton's test$^\text{np}$
				\item O'Brien's test$^\text{np}$
				\item O'Brien's two-sample tests
				\item Oja's tests
				\item Omega square test
				\item Orthogonal $T$-test
				\item Osius-Rojek test
				\item Page's test$^\text{np}$ (paired and not paired versions)
				\item Park test
				\item Partial Pearson and Spearman correlation trend test
				\item Patell test
				\item Partial correlation test
				\item Partial Theil U$^\text{np}$
				\item Pearson correlation $T$-test
				\item Pearson's chi-squared test
				\item Pearson-D'Asgostino-Bowman test
				\item Permutation tests
				\item Peters test
				\item Peto-Prentice test
				\item Pettitt’s test  
				\item Peto and Peto test
				\item Phillips-Perron (PP) test$^\text{np}$
				\item Phillips-Ouliaris test
				\item Pillai's trace test
				\item Pitman-Morgan test$^\text{np}$
				\item Pocock's test
				\item Poisson test
				\item Pulkstenis-Robinson test$^\text{np}$
				\item Potthoff and Whitlinghill's test
				\item Potthoff's test
				\item Precedence test
				\item Priestly's lambda-test
				\item Priestly's $P$-test
				\item Proportional Mass test$^\text{np}$
				\item Puri's expected Normal scores test$^\text{np}$
				\item Pustejovsky test
				\item $P$-test
				\item Quade test$^\text{np}$
				\item Quadratic Time MMD$^\text{np}$
				\item Quadrant test
				\item Quenouille's test
				\item Rainbow test for linearity
				\item Ramsey Regression Equation Specification Error test (RESET)
				\item Randles-Fligner-Policello-Wolfe test$^\text{np}$ 
				\item Rank product test$^\text{np}$
				\item Rao's score test 
				\item Rao's spacing test of uniformity
				\item Rao's spacing test of homogoneity
				\item Rayleigh's test of uniformity
				\item Renyi test
				\item RESET test (REgression Specification Error)
				\item Robust rank-order test$^\text{np}$
				\item Rohlf generalized gap test
				\item Rosenbaum's test
				\item Rosner outlier test
				\item Royston's test
				\item Roy’s largest root test statistic
				\item Roy and Kastenbaum test$^\text{np}$
				\item Ryan-Einot-Gabriel-Welsch range (REGWQ) test
				\item Ryan-Einot-Gabriel-Welsch $F$ (REGWF) test
				\item Runs test$^\text{np}$
				\item Sandvik-Olsson test
				\item Sargan–Hansen test
				\item Sarkadi-Kosik test
				\item Satterthwaite's test
				\item Savage's test
				\item Schach's two-sample tests
				\item Scheffé's test
				\item Schuster's test
				\item Scott-Knott Test 
				\item Scree test
				\item Schmidt-Phillips test
				\item Schwarzer's test
				\item Seasonal Hybrid ESD test
				\item Sen's slope test$^\text{np}$
				\item Semi-partial correlation test
				\item Sequential test$^\text{np}$
				\item Serial correlation test$^\text{np}$
				\item Shapiro-Wilk adequation test
				\item Shapiro–Francia test
				\item Sherman's test
				\item Shirley-Williams test$^\text{np}$
				\item Siegel test$^\text{np}$
				\item Silverman’s bandwidth test$^\text{np}$
				\item Skewness tests
				\item Skillings-Mack test$^\text{np}$
				\item Slivka's test
				\item Smith quartile means test
				\item Smith and Jain's test
				\item Spatial scan statistic$^\text{np}$
				\item Spearman test$^\text{np}$
				\item Sobel test
				\item Sommer's d$^\text{np}$
				\item Squared rank test$^\text{np}$
				\item Standard normal homogeneity (SNH) test
				\item Steel test$^\text{np}$
				\item Stone’s test
				\item Stuart–Maxwell test$^\text{np}$
				\item Stukel test
				\item Sukhatme's test
				\item Sup-Wald test
				\item Swartz' entropy test
				\item Tango’s score test
				\item Tarone's $Z$ test
				\item Taha's test
				\item Tajima’s D test
				\item Tamhane's T2 test
				\item Tamhane-Dunnett test
				\item Tango's maximized excess events test
				\item Tetrad test
				\item Tharone-Ware 
				\item Theil's $U$ test$^\text{np}$
				\item Tietjen-Moore test
				\item Tiku's test
				\item Time-series standard deviation test
				\item Triangle test
				\item Triples test
				\item $T$-test for coefficient slope
				\item $T$-test (heteroscedastic or homoscedastic version - paired or unpaired version)
				\item Trim \& Fill test
				\item Trimmed Means $T$-test
				\item Two out of five test
				\item Tsai-Koziol correlation test
				\item Tukey's B test
				\item Tukey's quick test
				\item Tukey's test for non-additivity
				\item Tukey-Kramer test
				\item Turning point test$^\text{np}$
				\item Ury-Wiggins-Hochberg test
				\item Vasicek's sample entropy test
				\item V-test (modified Rayleigh)
				\item Vuong's test$^\text{np}$
				\item W/S-test
				\item Waerden Normal-Scores test
				\item Wald Chi-square test
				\item Wald's F-test
				\item Wald–Wolfowitz test$^\text{np}$ for continuous/dichotomous data
				\item Walker's test$^\text{np}$
				\item Wall test$^\text{np}$
				\item Waller-Duncan $T$-test
				\item Waller and Lawson’s score test
				\item Wallis-Moore phase frequency test
				\item Wang and Tsiatis' test$^\text{np}$
				\item Watson's A-test
				\item Watson's $U^2$-test
				\item Watson–Williams test$^\text{np}$
				\item Wei–Lachin test
				\item Weisberg–Bingham test
				\item Westenberg's interquartile range test
				\item White neural network test for non-linearity
				\item White test for homogeneity of variances 
				\item Whittemore's test
				\item Wilcoxon's rank sum test$^\text{np}$
				\item Wilcoxon's signed rank test$^\text{np}$
				\item Wilk's multivariate outlier test
				\item Wilk's lambda test
				\item William's trend test
				\item Woolf logit test$^\text{np}$
				\item Yuen's test
				\item Yuen-Welch test
				\item $Z$-test for mean
				\item $Z$-test for the difference between independent correlations
				\item Zelen's test$^\text{np}$
				\item Zivot-Andrews test
				\item ...
			\end{enumerate}
			\end{multicols}
			\end{small}
		
		\item \textbf{Regression techniques\footnote{see section Numerical Methods page \pageref{regression techniques}} (regularization):}
			\begin{small}
			\begin{multicols}{2}
			\begin{enumerate}
				\item Linear regression model (LSM)
				\item Gaussian linear regression model		
				\item Non-linear regression models with binary or continuous variables 
				\item Polynomial regression model 
				\item Weighted least squares regression model
				\item B-spline or of collocation polynomial regression		
				\item Logistic regression models (binomial, multinomial, ordinal)
				\item Counting Poisson regression (Poisson MLE, PMLE, GLM, Poison-quasi-Lindley) 
				\item Negative binomial (binomial MLE and QGPMLE) regression model		
				\item Beta regression
				\item Orthogonal linear regression model (or Deming regression)
				\item Quantile regression model	
				\item Partial least squares regression (PLS)
				\item Structural Equation Modelling (Two-Stage Least Squares/2SLS)
				\item Moderated Multiple Regression
				\item Mixed linear/non-linear or generalized models (nested or not, splitted or not)
				\item Panel regression
				\item LAD (Least Absolute Deviation) regression 		
				\item LOESS (LOcally Estimated Scatterplot Smoothing) and LOWESS (LOcally Weighted Scatterplot Smoothing)		
				\item Least absolute shrinkage and selection operator (LASSO)
				\item Multivariate adaptive regression splines (MARS)		
				\item Bayesian linear regression model
				\item Logic regression
				\item Ridge regression model 		
				\item Bootstrap or Jackknife regression model
				\item Backward elimination regressions
				\item Forward entry regression
				\item C-RT regression tree
				\item DfBetas
				\item Epsilon SVR (support vector regression)
				\item Nu SVR
				\item Least-angle regression (LARS)
				\item Elastic net regularization 
				\item Simplex regression
				\item Independent component regression
				\item Theil-Sen estimator regression method
				\item Isotonic regression
				\item ...
			\end{enumerate}
			\end{multicols}
			\end{small}
		
		\item \textbf{Factor Analysis (FA):}
			\begin{small}
			\begin{multicols}{2}
			\begin{enumerate}
				\item FADM (Mixed data factor analysis)
				\item Bootstrap eigenvalues
				\item Linear discriminant analysis (canonical LDA)/Kernel LDA
				\item Correspondence analysis
				\item Discriminant correspondence analysis
				\item Principal Component Analysis with/without Factor rotation (VariMax)
				\item Harris component analysis
				\item Multiple correspondence analysis
				\item NIPALS (Non-linear Iterative Partial Least Squares)
				\item Parallel analysis
				\item Principal factor analysis
				\item ...
			\end{enumerate}
			\end{multicols}
			\end{small}
				
		\item \textbf{Clustering (unsupervised):}
			\begin{small}
			\begin{multicols}{2}
			\begin{enumerate}
				\item CatVARCHA (categorical variable hierarchical agglomerative clustering)
				\item Clustering Tree (CT) with/without post-pruning
				\item Fuzzy clustering tree
				\item Kernel Density estimation (density based clustering)
				\item Kernel PCA (Principal Component Analysis)
				\item Singular Value Decomposition (SVD)
				\item EM-Clustering (Expectation-Maximization)
				\item HAC (Hierarchical Agglomerative Clustering)
				\item $K$-Means, $K$-Medians or $K$-Medoids clustering, Fuzzy clustering, $K$-Modes, $K$-Prototypes
				\item Kernel $K$ means
				\item Mean shift clustering
				\item Spectral clustering
				\item Multidimensional scaling (MDS)
				\item Kohonen-SOM (Self Organization Map)
				\item Kohonen-LVQ (Learning Vector Quantizer)
				\item GMM (Gaussian Mixture Model)
				\item VARCLUS (top down approach)
				\item VARHCA (clustering variables using Hierarchical Cluster Analysis)
				\item VARKMeans (clustering variable using $K$-Means)
				\item Density-based spatial clustering of applications with noise (DBSCAN)
				\item Soft independent modelling of class analogies (SIMCA)
				\item ...
			\end{enumerate}
			\end{multicols}
			\end{small}
		
		\item \textbf{Supervised (Spv) Learning:}
			\begin{small}
			\begin{multicols}{2}
			\begin{enumerate}
				\item Binary logistic regression
				\item BVM (Ball Vector Machine)
				\item Iterative Dichotomiser 3 (ID3 Quinla algorithm)
				\item C4.5 (Quinlan algorithm extension of ID3)
				\item M5 (Quinlan algorithm)
				\item Cubist (extension of M5)
				\item C-PLS (PLS for classification)
				\item C-RT (Regression Tree for classification)
				\item CS-CRT (Cost Sensitive Classification RT)
				\item C-MC4 (M-Estimates based and Laplace smoothed CS-CRT)
				\item C-SVC (Continuous Supervised Classification)
				\item CVM (Core Vector Machine)
				\item Decision List (One-Rule, Zero-Rule, CN2)
				\item Repeated Incremental Pruning to Produce Error Reduction (RIPPER) 
				\item CHAID (Chi-squared Automatic Interaction Detection)
				\item Gradient Boosting
				\item Gaussian processes
				\item $k$-nn (Nearest Neigbhors)
				\item LAD (Linear Discriminant Analysis)/Kernel LDA
				\item QDA (Quadratic Discriminant Analysis)
				\item Log-Reg TRIRLS 
				\item Multilayer perceptron (MLP neural network)
				\item Multinomial Logistic Regression
				\item Naive bayes categorical variables classification
				\item Naive bayes continuous variables classification
				\item PLS-DA (PLS Discriminant Analysis)
				\item PLS-LDS (PLS Linear DA)
				\item Prototype-NN (Nearest Neighbours)
				\item Radial basis function (RBF neural network)
				\item Random Tree
				\item Rule Induction, Fuzzy rule induction
				\item Support Vector Machine (SVM)
				\item Genetic classification
				\item Locally Weighted Learning (LWL)
				\item ...
			\end{enumerate}
			\end{multicols}
			\end{small}
		
		\item \textbf{Semi-Supervised learning:}
			\begin{small}
			\begin{enumerate}
				\item Contrastive Pessimistic Likelihood Estimation (CPLE)
				\item semi-supervised vector machine (S3VM)
				\item Transductive Support Vector Machines (TSVM)
				\item Label propagation
				\item Label spreading
				\item ...
			\end{enumerate}
			\end{small}
			
		\item \textbf{Meta Supervised learning (ensemble learning):}
			\begin{small}
			\begin{multicols}{2}
			\begin{enumerate}
				\item Random forest
				\item Extremely Randomized trees
				\item Rotation forest
				\item Gradient Boosting Machines (GMB)
				\item Boosting
				\item Bootstrapped Aggregation (bagging)
				\item CatBoost, AdaBoost, XGBoost
				\item Light GBM
				\item Stacked Generalization (blending)
				\item Gradient Boosted Regression Trees (GBRT)
				\item Arcing (Arc-x4) bagging with weights
				\item Bagging with/without cost sensitivity
				\item Boosting  with/without cost sensitivity
				\item MultiCost sensitive supervised learning
				\item Stacked generalization
				\item ...
			\end{enumerate}
			\end{multicols}
			\end{small}
			
		\item \textbf{\text{Supervised learning assessment:}}
			\begin{small}
			\begin{multicols}{2}
			\begin{enumerate}
				\item Bias-variance decomposition (Wolpert \& Kohavi)
				\item Bootstrap
				\item Cross-validation
				\item Leave-One-Out
				\item Test set assessment
				\item Learning/Train assessment test
				\item ...
			\end{enumerate}
			\end{multicols}
			\end{small}
		
		\item \textbf{Scoring:}
			\begin{small}
			\begin{multicols}{2}
			\begin{enumerate}
				\item Lift curve
				\item ROC curve
				\item Precision-Recall curve
				\item Log-loss (and ie log likelihood)
				\item Log-pointwise predictive density (LPPD)
				\item Reliability diagram
				\item ...
			\end{enumerate}
			\end{multicols}
			\end{small}
			
		\item \textbf{Association:}
			\begin{small}
			\begin{multicols}{2}
			\begin{enumerate}
				\item A priori (MR version or not)
				\item A priori PT (Borgelt's algorithm)
				\item Assoc outlier (association rule mining principle)
				\item Frequent itemsets (Borgelt's algorithm)
				\item ECLAT (equivalence class transformation algorithm)
				\item FP-growth (frequent pattern growth)
				\item RElim (recursive elimination)
				\item SaM (Split and Merge)
				\item JIM (Jaccard Itemset Mining)
				\item Spv association rule 
				\item Spv association tree
				\item ...
			\end{enumerate}
			\end{multicols}
			\end{small}
			
		\item \textbf{Anomaly detection:}
			\begin{small}
			\begin{multicols}{2}
			\begin{enumerate}
				\item Chauvenet's criterion
				\item Cochran C test
				\item Dixon's test
				\item Grubb test
				\item Cook's distance
				\item Peirce's criterion
				\item Studentized residual
				\item Isolation forest
				\item Control Charts\footnote{see section Industrial Engineering for the details on control charts page \pageref{quality control charts}} (P, NP, C, U, R-R, X-R, etc.)
				\item Markov modulated Poisson process (MMPP)
				\item Local outlier factor (FOL)
				\item Online Over-Sampling Principal Component Analysis
				\item ...
			\end{enumerate}
			\end{multicols}
			\end{small}
		
		\item \textbf{Reinforcement learning:}
			\begin{small}
			\begin{multicols}{2}
			\begin{enumerate}
				\item Markov Decisions Processes
				\item Case-Based Reasoning 
				\item Genetic algorithm
				\item Asynchronous Advantage Actor Critic (A3C) 
				\item State–Action–Reward–State–Action (SARSA)
				\item Q-learning 
				\item TD($\lambda$) algorithm
				\item TD($0$) Actor-Critic
				\item ...
			\end{enumerate}
			\end{multicols}
			\end{small}
			
		\item \textbf{Imputations methods:}
			\begin{small}
			\begin{multicols}{2}
			\begin{enumerate}
				\item Predictive mean matching
				\item Weighted predictive mean matching 
				\item Random sample from observed values
				\item Classification and regression trees 
				\item Hot or Cold deck imputation
				\item Random forest imputations
				\item Unconditional mean imputation
				\item Bayesian linear regression
				\item Linear regression ignoring model error
				\item Linear regression using bootstrap
				\item Linear regression, predicted values
				\item Imputation of quadratic terms
				\item Laplace smoothing
				\item Random indicator for non-ignorable data
				\item Logistic regression
				\item Logistic regression with bootstrap
				\item Proportional odds model
				\item Polytomous logistic regression
				\item Linear discriminant analysis
				\item Level-1 normal heteroscedastic
				\item Level-1 logistic
				\item Level-2 class mean/normal
				\item Level-2 class predictive mean matching
				\item Multivariate imputation by Chained Equations 
				\item ...
			\end{enumerate}
			\end{multicols}
			\end{small}
			
		\item \textbf{Forecastings:}
			\begin{small}
			\begin{multicols}{2}
			\begin{enumerate}
				\item Linear and polynomial regressions (see above)
				\item Moving Average
				\item Simple exponential smoothing
				\item Double exponential smoothing (Brown)
				\item Triple exponential smoothing (Holt \& Winters)
				\item ETS (Error Trend Seasonal) models
				\item Logistic forecasts
				\item Croston's method / Adjusted Croston's method
				\item ARIMA\footnote{AutoRegressive Integrated Moving Average} processes (AR, ARMA DARMA, INARMA, ARIMA, SARIMA, ARFIMA, SARFIMA, ARIMAX)
				\item GARCH (General-ARCH\footnote{AutoRegressive Conditional Heteroskedasticity}) processes (GARCH-X, EGARCH, ARCH, sGARCH, RUGARCH, iGARCH, MGARCH)
				\item NARCH (Non-linear-ARCH) processes
				\item TARCH (Threshold-ARCH) processes
				\item Independent Component Analysis (ICA)
				\item ...
			\end{enumerate}
			\end{multicols}
			\end{small}
			
		\item \textbf{Sequence mining:}
			\begin{small}
			\begin{enumerate}
				\item GSP (Generalized Sequential Patterns)
				\item SPADE (Sequential Pattern Discovery)
				\item FreeSpan
				\item HMM (hidden Markov models)
				\item ...
			\end{enumerate}
			\end{small}
			
		\item \textbf{Neural Networks/Deep learning:}
			\begin{small}
			\begin{multicols}{2}
			\begin{enumerate}
				\item Perceptron
				\item Feed-Forward Neural Network (FFNN)
				\item Radial Basis Network (RBF)
				\item Recurrent Neural Network (RNN)
				\item Gated Recurrent Unit (GRU)
				\item Long Short-Term Memory (LSTM)
				\item Deep Boltzmann Machine (DBM)
				\item Deep Feed Forward Networks (DBN)
				\item Deep Belief Networks (DBN)
				\item Deep Belief Convolutional Network (DCN)
				\item Deep Residual Network (DRN)
				\item Differentiable Neural Computer (DNC)
				\item Neural Turing Machine (NTM)
				\item Capsule Network (CN)
				\item Kohonen Network (KN)
				\item Attention Network (AN)
				\item Deconvolutional Network (DN)
				\item Deep Convolutional Inverse Graphics Network (DCIGN)
				\item Liquid State Machine (LSM)
				\item Extreme Learning Machine (ELM)
				\item Echo State Network (ESN)
				\item Convolutional Neural Networks (CNN)
				\item Fuzzy Neural Network (FNN)
				\item Spectral-Residual CNN (SR-CNN)
				\item Super Resolution CNN (SRCNN)
				\item Stacked Auto-Encoders (SAE)
				\item Variational Auto-Encoder (VAE)
				\item Denoising Auto-Encoder (DAE)
				\item Sparse Auto-Encoder (DAE)
				\item Hopfield Network (HN)
				\item Restricted Boltzmann Machine Network (RBM)
				\item Adaptative Neuro-Fuzzy Inference System (ANFIS)
				\item Generative Adversarial Networks (GAN)
				\item Wasserstein Generative Adversarial Networks (WGAN)
				\item Spectral Normalization Generative Adversarial Networks (SN-GAN)
				\item Self-attention Generative Adversarial Networks (S-GAN)
				\item Progressive Generative Adversarial Networks (PROGAN)
				\item Contextual RNN-GANs (Context-RNN-GAN)
				\item Continuous recurrent neural networks With Generative Adversarial Networks (C-RNN-GAN)
				\item Conditional Sequence Generative Adversarial Networks (CS-GAN)
				\item Fine-Grained Image Generation through Asymmetric Training (CVAE-GAN)
				\item Cycle-consistent Adversarial Networks (CycleGAN)
				\item Unsupervised Cross-Domain Image (DTN)
				\item Unsupervised Learning With Deep Convolutional Generative Adversarial Networks (DCGAN)
				\item Discover Cross-Domain Relations Generative Adversarial Networks (DiscoGAN)
				\item Disentangled Representation Learning GAN for pose-lnvariant Recognition (DR-GAN)
				\item Unsupervised Dual Learning for Translation (DualGAN)
				\item Energy-based Generative Adversarial Network (EBCAN)
				\item Training Generative Neural Samplers Variational Divergence Minimization (f-GAN)
				\item Face Frontalization Generative Adversarial Network (FF-GAN)
				\item Generative Adversarial What-Where Network (GAWWN)
				\item Learning Object Transfiguration and Attribute Subspacefrom unpaired Data (GeneGAN)
				\item Geometric Generative Adversarial Networks (GGAN)
				\item Generative Adversarial Networks With Maximum Margin Ranking (Hogan)
				\item Towards Realistic Blending (GP-GAN)
				\item Introspective Adversarial Networks (IAN)
				\item Natural Image Manifold Generative Adversarial Networks (iGAN)
				\item Invertible Conditional Generative Adversarial Networks (ICGAN)
				\item Image De-raining using a Conditional Generative Adversarial Network (ID-CGAN)
				\item Interpretable Representation Learning by Information Maximizing Generative Adversarial Nets (InfoGAN)
				\item Location-Aware Generative Adversarial Networkss (LAGAN) 
				\item Laplacian Pyramid Of Adversarial Networks (LAPGAN)
				\item 3D Generative-Adversarial Modelling (3D-GAN)
				\item Face Aging With Conditional Generative Adversarial Networks (acGAN)
				\item Auxiliary Classifier Generative Adversarial Networks (AC-GAN )
				\item Boosting Generative Models (AdaGAN)
				\item Autoencoder based Generative Adversarial Networks (AEGAN)
				\item Amortised MAP Inference for Image Super-resolution (AffGAN)
				\item Attributes and Semantic Layouts Conditional Generative Adversarial Networks (AL-CGAN)
				\item Adversarial Learned Inference (ALI)
				\item Activation Maximization Generative Adversarial Networks (AM-GAN)
				\item Anomaly Detection Generative Adversarial Networks (AnoGAN)
				\item Artwork Synthesis Generative Adversarial Networks (ArtGAN)
				\item Brainstorming Generative Adversarial Networks (b-GAN)
				\item Bayesian GAN (BGAN)
				\item Boundary Equilibrium Generative Adversarial Networks (BEGAN)
				\item Bidirectional (Generative Adversarial Networks)
				\item Boundary-Seeking Generative Adversarial Networks (BS-GAN)
				\item Conditional Generative Adversarial Nets (CGAN)
				\item Calorimeters With Generative Adversarial Networks (CaloGAN)
				\item Context-Conditional Generative Adversarial Networks (CCCAN)
				\item Categorical Generative Adversarial Networks (CatGAN)
				\item Coupled Generative Adversarial Networks (COGAN)
				\item ...
			\end{enumerate}
			\end{multicols}
			\end{small}
			
		\item \textbf{Cost functions:}
			\begin{small}
			\begin{multicols}{2}
			\begin{enumerate}
				\item Quadratic cost
				\item Cross-entropy cost
				\item Exponential cost
				\item Hellinger cost
				\item Kullback-Leibler cost
				\item Generalized Kullback-Leibler cost
				\item Bregman cost
				\item Jensen-Shannon cost
				\item Itakura-Saito cost
				\item ...
			\end{enumerate}
			\end{multicols}
			\end{small}
		
		\item \textbf{Text Mining:}
			\begin{small}
			\begin{multicols}{2}
			\begin{enumerate}
				\item Latent Semantic Analysis (LSA)
				\item Latent Semantic Indexing (LSI)
				\item Latent Dirichlet allocation (LDA)
				\item Probabilistic Latent Semantic Indexing (PLSI)
				\item Vector space model (term vector model)
				\item Sentiment Analysis
				\item Language detection
				\item Pattern detection/correlation
				\item ...
			\end{enumerate}
			\end{multicols}
			\end{small}
	\end{itemize}
	
	\begin{tcolorbox}[title=Remark,colframe=black,arc=10pt]
	Keep in mind that Machine Learning cannot be used for everything related to statistics. Indeed, in drug assessment and approval we are limited by a set of statistical methods forming an industry standard and approved by regulatory agencies (FDA, EMA, etc.), by the accepted approach named "confirmatory data analysis" via inferential statistics, both frequentist and bayesian, by small and extremely small data (starting from $5$ observations in early trial phases) by demand for interpretability (not rarely we prefer here simpler but more "transparent" methods rather than efficient ones). In clinical research, every single possible aspect is regulated by guidelines, controlled by validation, every step of statistical analysis must be well enough justified under the threat of being rejected by regulatory or statistical reviewers and lead the whole trial to fail.
	\end{tcolorbox}
	
	The reader must keeping in mind some possible limitations of predictive model based on data fitting:
	\begin{itemize}
		\item History cannot always predict future
		\item The issue of unknown unknowns
		\item Bad model choice
		\item Over-fitting model
		\item Overconfidence of humans on algorithms results
		\item ...
	\end{itemize}
	
	So we see through these lists that Data Mining and Machine Learning are child of statistics, computer science, and mathematical optimization. Along the way, it took inspiration from information theory, neural science, theoretical physics, and many other fields.
	
	This was the scientist point of view of Data Mining... From the point of view of business, Data Mining is more considered as the following data life cycle:
	\begin{figure}[H]
		\centering
		\includegraphics[scale=0.7]{img/computing/data_mining_life_cycle.jpg}
		\caption[Data Mining life cycle as seen by SAS™]{Data Mining life cycle as seen by SAS™ (source: SlideShare SAS)}
	\end{figure}
	ad also let us introduce the 4V's of Data Mining/Machine Learning that resume quite well the most common cases of data to which the Data Scientist is faces to:
	\begin{figure}[H]
		\centering
		\includegraphics[scale=0.9]{img/computing/four_v_data_mining.jpg}
		\caption[Data Mining/Machine Learning 4V's]{Data Mining/Machine Learning 4V's (source: ?, author: ?)}
	\end{figure}
	
	\begin{tcolorbox}[title=Remarks,colframe=black,arc=10pt]
	\textbf{R1.} The data scientist must take care of not becoming a "Data Pusher" and not to do just "infotainement" (almost no-sense statistics) in order to remain employed. This is ethically non-scientific and as the reader of this book already knows it, as going at the opposite of the Archimedes Oath!\\
	
	\textbf{R2.} Corporations and managers have to understand that a data steward cannot be a data analyst, that a data analyst that cannot be a data scientist and that latter is also most of time not at the level of statistician. Think the opposite show an evident lack of technical and scientific knowledge from the management, especially when the data scientist is supposed alone to install Big Data servers, clean data, develop new mathematical models, do programming for implementing the models and put them in production.\\
	
	\textbf{R3.} Data is not a currency, most data is garbage. Actionable information is currency. Extracting information form modern big data sets requires the equivalent processing infrastructure and time of extracting a nugget of gold from a mountain of dirt.
	\end{tcolorbox}
	The reader must also know that it is commonly accepted as a definition that Data Mining is not Data Science as the commonly definition of "\NewTerm{Data Science}\index{data science}" is represented by the following Venn Diagram:
	\begin{figure}[H]
		\centering
		\includegraphics[scale=0.7]{img/computing/data_science_venn_diagramm.jpg}
		\caption[Data science]{source: Palmer, Shelly. Data Science for the C-Suite. New York: Digital Living Press, 2015 Print}
	\end{figure}
	And small, medium and international companies but not be surprised when they hire a Data Scientist that should "master" programming languages, data retrieval,data wrangling, data archiving, data governance and statistics as it is impossible for one person to master all these subjects at the same time (or even to have an undergraduate level at the same time of each of these topics) furthermore when consider the four types of data where some need very specific high level mathematical skills to deal with:
	\begin{figure}[H]
		\centering
		\includegraphics[scale=0.7]{img/computing/types_of_data_in_data_science.jpg}
		\caption[Types of data in Data Science]{Types of data in Data Science (source: ?)}
	\end{figure}
	So don't be surprised if by hiring only one person in your company to do all the following jobs:
	\begin{figure}[H]
		\centering
		\includegraphics[scale=0.7]{img/computing/data_science_roles.jpg}
		\caption[Data science roles]{Data science roles (source: ?)}
	\end{figure}
	that as we put bullshit inside... we get bullshit outside...
	
	It should be notice that business intelligence is quite different from Data Science as illustrated below:
	\begin{figure}[H]
		\centering
		\includegraphics[scale=0.7]{img/computing/business_intelligence_vs_data_science.jpg}
		\caption[Comparing Business Intelligence with Data Science]{Comparing Business Intelligence with Data Science (source: \cite{emc2015data})}
	\end{figure}
	Finally, as requested by a customer, here is also a list of visuals (charts) used in Data Mining:
	\begin{figure}[H]
		\centering
		\includegraphics[width=1.0\textwidth]{img/computing/graphic_continuum.jpg}
		\caption[Graphic Continuum]{Graphic Continuum (authors: Jon Schwabish, Severino Ribecca)}
	\end{figure}
	And keep in mind as Data Scientist or as a manager of a Data Scientist to be careful with most common data fallacies (\SeeChapter{see section Statistics page \pageref{data fallacies}}).
	
	\begin{figure}[H]
		\centering
		\includegraphics[scale=0.4]{img/computing/path_to_data_science_adventure.jpg}
		\caption[]{source: \url{www.xkcd.com} (author: Randall Munroe)}
	\end{figure}
	
	\subsubsection{Training vs Test Data}
	We typically split the input data into learning and testing datasets. The then run the Machine Learning algorithm on the learning dataset to generate the prediction model. Later, we use the test dataset to evaluate our model.

	It is important that the test data is separate from the one used in training otherwise we will be kind of cheating because may for example the generated model memorizes the data and hence if the test data is also part of the training data then our evaluation scores of the model will be higher than they actually are.
	
	The data is usually split $75\%$ training and $25\%$ data or $2/3$ training and $1/3$ testing. It is important to note that: the smaller the training set the more challenging it is for the algorithm to discover the rules.
	
	In addition, when splitting the dataset, we need to maintaining class proportions and population statistics otherwise we will have some classes that are under represented in the training dataset and over represented in the test dataset.
	
	A crucial step when building our Machine Learning model is to estimate its performance on that the model hadn't seen before. We want to make sure that the model generalizes well to new unseen data.
	\begin{figure}[H]
		\centering
		\includegraphics[scale=0.6]{img/computing/overfitting.jpg}
		\caption{Underfitting and overfitting errors}
	\end{figure}
	In statistics and Machine Learning, the "\NewTerm{bias–variance tradeoff}\index{bias–variance tradeoff}" (or dilemma) is the problem of simultaneously minimizing two sources of error that prevent supervised learning algorithms from generalizing beyond their training set:
	 \begin{itemize}
	 	\item The bias is an error from erroneous assumptions in the learning algorithm. High bias can cause an algorithm to miss the relevant relations between features and target outputs (underfitting).

		\item The variance is an error from sensitivity to small fluctuations in the training set. High variance can cause an algorithm to model the random noise in the training data, rather than the intended outputs (overfitting).
	\end{itemize}
	This tradeoff applies to all forms of supervised learning: classification, regression (function fitting), and structured output learning. It may has also been invoked to explain the effectiveness of heuristics in human learning.
	
	The bias-variance tradeoff is a central problem in supervised learning. Ideally, one wants to choose a model that both accurately capture the regularities in its training data, but also generalizes well to unseen data. Unfortunately, it is typically impossible to do both simultaneously. High-variance learning methods may be able to represent their training set well but are at risk of overfitting to noisy or unrepresentative training data. In contrast, algorithms with high bias typically produce simpler models that don't tend to overfit but may underfit their training data, failing to capture important regularities.
	
	Models with low bias are usually more complex (e.g. higher-order regression polynomials), enabling them to represent the training set more accurately. In the process, however, they may also represent a large noise component in the training set, making their predictions less accurate - despite their added complexity. In contrast, models with higher bias tend to be relatively simple (low-order or even linear regression polynomials) but may produce lower variance predictions when applied beyond the training set.
	
	Suppose that we have a training set consisting of a set of points $x_{1},\dots ,x_{n}$ and real values $y_{i}$ associated with each point $x_{i}$. We assume that there is a function with noise $y=f(x)+\varepsilon$, where the noise, $\varepsilon$, has zero mean and variance $\sigma^{2}_\varepsilon$.
	
	We want to find a function $\hat{f}$, that approximates the true function $f(x)$ as well as possible, by means of some learning algorithm. We make "as well as possible" precise by measuring the mean squared error between $y$ and $\hat{f}(x)$: we want $(y-{\hat {f}}(x))^{2}$ to be minimal, both for $x_{1},\dots ,x_{n}$ and for points outside of our sample. Of course, we cannot hope to do so perfectly, since the $y_{i}$ contain noise $\varepsilon$; this means we must be prepared to accept an irreducible error in any function we come up with.
	
	\begin{theorem}
	Finding an $\hat{f}$ that generalizes to points outside of the training set can be done with any of the countless algorithms used for supervised learning. It turns out that whichever function $\hat{f}$ we select, we can decompose its expected error on an unseen sample $x$ as follows:
	
	named "\NewTerm{bias-variance tradeoff}\index{bias-variance tradeoff}", where:
	
	and:
	
	\end{theorem}
	and where PE denotes the "\NewTerm{prediction error}\index{prediction error}" also sometimes named the "\NewTerm{expected loss}\index{expected loss}".
	\begin{dem}
	The derivation of the bias-variance decomposition for squared error proceeds as follows. For notational convenience, abbreviate $f = f(x)$ and $\hat{f}=\hat{f}(x)$. First, recall that, by definition, for any random variable $X$, we have:
	
	Rearranging, we get:
	
	If we assume a model where $f$ is deterministic:
	
	This, given $y=f+\varepsilon$ and $\text{E}(\varepsilon)=0$, implies:
	
	Also, since $\text{V}(\varepsilon)=\sigma_\varepsilon^2$:
	
	Thus, since $\varepsilon$ and $\hat{f}$ are independent, we can write:
	
	So we get finally the bias-variance tradeoff relation\label{bias-variance tradeoff}:
	
	That latter is sometimes denoted as following in some textbooks (the letter $D$ denotes the known data):
	
	\begin{flushright}
		$\blacksquare$  Q.E.D.
	\end{flushright}
	\end{dem}
	The trade-off between bias and variance is that:
	\begin{itemize}
		\item Simple Models: High Bias, Low Variance (over-fitting)
		\item Complex Models: Low Bias, High Variance (under-fitting)
	\end{itemize}
	
	We will derive later the bias and variance for the (vanilla) $k$ nearest neighbours clustering but also for the bagging and bootstrap of random forests (when we will introduce these topics!). Let us however see right now as an example how to derive the prediction error for the linear regression!
	
	\begin{tcolorbox}[colframe=black,colback=white,sharp corners]
	\textbf{{\Large \ding{45}}Example:}\\\\
	Our goal\label{bias-variance tradeoff ols} is first to derive the bias of the ordinary least square regression:
	
	For that we assume obviously the true model to be:
	
	Therefore:
	
	Hence $\hat{\theta}_{\text{MLE}$ is unbiased! Let's compute the variance now:
	
	\end{tcolorbox}
	With the above example we see therefore that:
	
	
	\newcommand{\target}[1]{%
	  \foreach \r in {2.5, 2, 1.5, 1, 0.5, 0.05} {
	    \draw [
	      fill = black,
	      fill opacity = 0.05
	    ] (0, 0) circle (\r cm);
	  }
	}%
	
	\begin{figure}[H]
		\centering
		\begin{tikzpicture}[
		  every path/.style = {draw, > = stealth'},
		  nodetext/.style = {
		    draw, rounded corners = 2pt, fill = white,
		  },
		  labeltext/.style = {
		    draw, minimum width = 5.25cm, minimum height = 1.5em, fill = gray!20,
		  },scale=0.8]
		
		\def\dist{5.5};
		
		\begin{scope}[shift = {(0, 0)}, scale = 1.25]
		
		  \target
		
		  \foreach \x/\y in {0.825/1.414, 1.827/0.407, 1.068/0.855, 2.016/0.645, 2.131/-0.294, 0.341/1.3, 1.306/-0.008, 2.035/-0.416, 1.353/0.156, 1.163/1.409} {
		    \draw[fill = orange] (\x, \y) circle (2pt);
		  }
		
		  \node[text = cyan] (mean) at (1.4065, 0.5468) {$\times$};
		  \node[text = magenta] (theta) at (0, 0) {$\times$};
		
		  \path[->] (mean) to[out = 90, in = 180] ++(1, 1) node[nodetext, right] {$\text{E}(\hat{\theta)}$};
		  \path[->] (theta) to[out = 90, in = 0] ++(-1, 1) node[nodetext, left] {$\theta$};
		  \path[->] (1.163, 1.409) to[out = 90, in = 180] ++(1, 1) node[nodetext, right] {$\hat{\theta}_i$};
		
		  \node[nodetext] at (0, -3) {$\textcolor{magenta}{\text{PE}(\hat{\theta})} = \textcolor{cyan}{\text{V}(\hat{\theta})} - \textcolor{red}{\text{Bias}(\hat{\theta})}^2$};
		
		\end{scope}
		
		\begin{scope}[shift = {(7, 0)}]
		
		  \target
		
		  \foreach \x/\y in {0.825/1.414, 1.827/0.407, 1.068/0.855, 2.016/0.645, 2.131/-0.294, 0.341/1.3, 1.306/-0.008, 2.035/-0.416, 1.353/0.156, 1.163/1.409} {
		    \draw[fill = orange] (\x, \y) circle (2pt);
		    \draw[dashed, magenta] (\x, \y) -- (0, 0);
		  }
		
		  \node[nodetext] at (0, -1) {$\textcolor{magenta}{\text{PE}(\hat{\theta})} = 1/n\sum (\hat{\theta}_i - \theta)^2$};
		
		  \node[text = cyan] at (1.4065, 0.5468) {$\times$};
		  \node[text = magenta] at (0, 0) {$\times$};
		
		\end{scope}
		
		\begin{scope}[shift = {(12, 2.75)}]
		
		  \target
		
		  \foreach \x/\y in {0.825/1.414, 1.827/0.407, 1.068/0.855, 2.016/0.645, 2.131/-0.294, 0.341/1.3, 1.306/-0.008, 2.035/-0.416, 1.353/0.156, 1.163/1.409} {
		    \draw[fill = orange] (\x, \y) circle (2pt);
		    \draw[dashed, cyan] (\x, \y) -- (1.4065, 0.5468);
		  }
		
		  \node[nodetext] at (0, -1) {$\textcolor{cyan}{\text{V}(\hat{\theta})} = 1/n\sum (\hat{\theta}_i - \text{E}(\hat{\theta}))^2$};
		
		  \node[text = cyan] at (1.4065, 0.5468) {$\times$};
		  \node[text = magenta] at (0, 0) {$\times$};
		
		\end{scope}
		
		\begin{scope}[shift = {(12, -2.75)}]
		
		  \target
		
		  \foreach \x/\y in {0.825/1.414, 1.827/0.407, 1.068/0.855, 2.016/0.645, 2.131/-0.294, 0.341/1.3, 1.306/-0.008, 2.035/-0.416, 1.353/0.156, 1.163/1.409} {
		    \draw[fill = orange] (\x, \y) circle (2pt);
		  }
		
		  \node[nodetext] at (0, -1) {$\textcolor{red}{\text{Bias}(\hat{\theta})} = \text{E}(\hat{\theta}) - \theta$};
		
		  \node[text = cyan] at (1.4065, 0.5468) {$\times$};
		  \node[text = magenta] at (0, 0) {$\times$};
		
		  \draw[dashed, red] (0, 0) -- (1.4065, 0.5468);
		
		\end{scope}
		
		\end{tikzpicture}
	\end{figure}
	
	One case, the Machine Learning algorithm has different parameters and we want to tune these parameters to achieve the best performance (the parameters of the Machine Learning algorithm are named "\NewTerm{hyperparameters}" as we will see further below\footnote{The double problem in Data Mining to find the best algorithm + its better hyperparameters is named the "CASH problem", where CASH stands for "Combined Algorithm Selection and Hyperparameters"}). Another case, sometimes we want to try out different algorithms and choose the best performing one. Here is a typical technique:
	
	We simply split the data into training and testing datasets. Then, the training data is further split into training and validation sets.
	
	The training data is used to train different models. Then the validation data is used to compute performance of each of them and we select the best one. Finally, the model is then used for the test set to evaluate performance. The next figure illustrates this idea:
	\begin{figure}[H]
		\centering
		\includegraphics[scale=0.9]{img/computing/machine_learning_validation.jpg}
		\caption{Holdout Validation in Machine learning}
	\end{figure}
	However, because we use the validation set multiple times, Holdout validation is sensitive to how we partition the data and that is what $K$-fold cross validation tries to solve.
	
	Initially, we split the data into training and testing dataset. Furthermore, the training dataset is split into $K$ chunks.

	Suppose we will use 5-fold cross validation, the training data set is split into 5 chunks and the training phase will take place over 5 iterations. In each iteration we use one chunk as the validation dataset while the rest of the chunk are grouped together to form the training dataset.

	This is very similar to Holdout validation except in each iteration the validation data is different and this removed the bias. Each iteration generates a score and the final score is the average score of all iteration. As before we select the best model and use the test data for the final performance evaluation.
	
	It is interesting to know that this practice of splitting data seems to come first form an article of David Freedman \cite{freedman1983note} in 1983 (hence sometimes the name "\NewTerm{Freedman paradox}"). Freedman shows in an impressive way the dangers of data reuse in statistical analysis. The potentially dangerous scenarios include those where the results of one statistical procedure performed on the data are fed into another procedure performed on the same data. As a concrete example Freedman considers the practice of performing variable selection first, and then fitting another model using only the identified variables on the same data that was used to identify them in the first place. 
	
	The figure below shows the distributions of the considered model statistics (variable selection and afterwards a regression) by David Freedman across the $1000$ repetitions for model fits with and without data reuse (the code producing this figure is given at the bottom of this post):
	\begin{figure}[H]
		\centering
		\includegraphics[scale=0.8]{img/computing/freedman_paradox.jpg}
		\caption{Freedman paradox illustration}
	\end{figure}
	Well, the $R$ squared statistic shows a moderate change between models with or without data reuse (average of $0.3093018$ vs. $0.5001641$). The F test statistic however grows immensely to an average of $3.2806118$ (from $1.0480097$), and the $p$-values fall after data reuse to an average of $0.0112216$ (from $0.5017696$), below the widely used (but arbitrary) $5\%$ significance level.

	Obviously the model with data reuse is highly misleading here, because in fact there are absolutely no relationships between the predictor variables and the response (as per the data generation procedure).
	
	OK! This done let us now begin first with the basics of clustering methods! 
	
	\pagebreak
	\subsubsection{Hyperparameters vs Parameters}
	You may have heard of terms like "\NewTerm{hyperparameter}\index{hyperparameter}" search, autotuning (which is just a shorter way of saying hyperparameter search), or grid search (a possible method for hyperparameter search). Where do those terms fit in? To understand hyperparameter search, we have to talk about the difference between a model parameter and a hyperparameter. In brief, model parameters are the knobs that the training algorithm knows how to tweak; they are learned from data. Hyperparameters, on the other hand, are not learned by the training method, but they also need to be tuned.
	
	A famous example are neural networks (see further below) where weights and bias are autotuned parameters, but the number of layers, the type of activation function of the bias values are hyperparameters:
	\begin{figure}[H]
		\centering
		\includegraphics[scale=0.63]{img/computing/hyperparameters.jpg}
		\caption{Parameters vs Hyperparameters}
	\end{figure}
	"\NewTerm{Bagging}\index{bagging}" and "\NewTerm{Boosting}\index{boosting}" are both ensemble methods in Machine Learning that also use hyperparameters (number of iterations/trees, sampling with or without replacement, minimum leaf size, minim parent size, maximum number of decision splits, learning rate for shrinkage, etc).
	
	Bagging and Boosting are similar in that they are both ensemble techniques, where a set of weak learners are combined to create a strong learner that obtains better performance than a single one. So, let's start from the beginning\footnote{The whole text introducing the difference between bagging and boosting in this book is taken form the QuantDare blog article \url{https://quantdare.com/what-is-the-difference-between-bagging-and-boosting/} written by Ana Porras Garrido who works for the company ETS Asset Management Factory (\url{https://www.etsfactory.com/})}.
	
	"\NewTerm{Ensemble learning}\index{Ensemble learning}" is a Machine Learning concept in which the idea is to train multiple models using the same learning algorithm. The ensembles take part in a bigger group of methods, named "\NewTerm{multiclassifiers}\index{multiclassifiers}", where a set of hundreds or thousands of learners with a common objective are fused together to solve the problem.

	The second group of multiclassifiers contain the "\NewTerm{hybrid methods}\index{hybrid methods}
". They use a set of learners too, but they can be trained using different learning techniques. 

	The main causes of error in learning are due to noise, bias and variance. Ensemble helps to minimize these factors. These methods are designed to improve the stability and the accuracy of Machine Learning algorithms. Combinations of multiple classifiers decrease variance, especially in the case of unstable classifiers, and may produce a more reliable classification than a single classifier.

	To use Bagging or Boosting you must select a base learner algorithm. For example, if we choose a classification tree, Bagging and Boosting would consist of a pool of trees as big as we want. 
	
	Bagging and Boosting get $N$ learners by generating additional data in the training stage. $N$ new training data sets are produced by random sampling with replacement from the original set. By sampling with replacement some observations may be repeated in each new training data set.
	
	In the case of Bagging, any element has the same probability to appear in a new data set. However, for Boosting the observations are weighted and therefore some of them will take part in the new sets more often: 
	\begin{figure}[H]
		\centering
		\includegraphics[width=1.0\textwidth]{img/computing/bagging_and_boosting_01.jpg}
	\end{figure}
	These multiple sets are used to train the same learner algorithm and therefore different classifiers are produced.
	
	At this point, we begin to deal with the main difference between the two methods. While the training stage is parallel for Bagging (i.e., each model is built independently), Boosting builds the new learner in a sequential way:
	\begin{figure}[H]
		\centering
		\includegraphics[width=1.0\textwidth]{img/computing/bagging_and_boosting_02.jpg}
	\end{figure}
	To predict the class of new data we only need to apply the $N$ learners to the new observations. In Bagging the result is obtained by averaging the responses of the $N$ learners (or majority vote). However, Boosting assigns a second set of weights, this time for the $N$ classifiers, in order to take a weighted average of their estimates.
	\begin{figure}[H]
		\centering
		\includegraphics[width=1.0\textwidth]{img/computing/bagging_and_boosting_03.jpg}
	\end{figure}
	In the Boosting training stage, the algorithm allocates weights to each resulting model. A learner with good a classification result on the training data will be assigned a higher weight than a poor one. So when evaluating a new learner, Boosting needs to keep track of learners’ errors, too. Let's see the differences in the procedures:
	\begin{figure}[H]
		\centering
		\includegraphics[width=1.0\textwidth]{img/computing/bagging_and_boosting_04.jpg}
	\end{figure}
	Some of the Boosting techniques include an extra-condition (hyperparamater) to keep or discard a single learner. For example, in AdaBoost, the most renowned, an error less than $50\%$ is required to maintain the model; otherwise, the iteration is repeated until achieving a learner better than a random guess.

	The previous image shows the general process of a Boosting method, but several alternatives exist with different ways to determine the weights to use in the next training step and in the classification stage (AdaBoost, LPBoost, XGBoost, GradientBoost, BrownBoost).
	
	We may ask ourselves what is the best, Bagging or Boosting?
	
	There's not an outright winner; it depends on the data, the simulation and the circumstances. Bagging and Boosting decrease the variance of your single estimate as they combine several estimates from different models. So the result may be a model with higher stability.
	
	Indeed, let us recall that the bias-variance tradeoff relation that we have derived at page \pageref{bias-variance tradeoff}:
	
	As bagging consist to average the different models, we know that averaging reduces the variance according:
	
	So bagging $B$ times leads to:
	
	As we know decreasing the variance therefore increase the bias (as the sum of the both must always be equal). In practice the bagging models are correlated, and as we know it, correlation makes variance even smaller!

	If the problem is that the single model gets a very low performance, Bagging will rarely get a better bias. However, Boosting could generate a combined model with lower errors as it optimises the advantages and reduces pitfalls of the single model.

	By contrast, if the difficulty of the single model is over-fitting, then Bagging is the best option. Boosting for its part doesn't help to avoid over-fitting; in fact, this technique is faced with this problem itself. For this reason, Bagging is effective more often than Boosting.
	
	Here is a very nice summary of the most three common ensemble learning method in one picture:
	\begin{figure}[H]
		\centering
		\includegraphics[width=1.0\textwidth]{img/computing/ensemble_learning.pdf}
		\caption[Three most common ensemble learning methods]{Three most common ensemble learning methods\\ (source: \url{https://www.cheatsheets.aqeel-anwar.com}, author: Aqeel Anwar)}
	\end{figure}
	
	\pagebreak
	\subsubsection{Association Rules}
	A classic story tells about the increase in beer sales after a store decided to place the beer next to the diapers. What is the relationship between baby diapers and beers? We would be wondering...
	
	Several explanations follow to understand this observation, but that's about the purchases made by the male parents. The latter, when they go to take diapers profit to buy a few beers. This little anecdote actually reveals the existence of a rule of association between these two products which at first sight have nothing in common.
	\begin{figure}[H]
		\centering
		\includegraphics[scale=0.6]{img/computing/diapers_beer.jpg}
	\end{figure}
	Association rule learning is a unsupervised rule-based Machine Learning method for discovering interesting relations between variables in large databases. It is intended to identify strong rules discovered in databases using some measures of interestingness.
	
	 For example, the rule  ${\displaystyle \{\mathrm {diapeer} \}\Rightarrow \{\mathrm {beer} \}}$ or ${\displaystyle \{\mathrm {onions,potatoes} \}\Rightarrow \{\mathrm {burger} \}}$  found in the sales data of a supermarket would indicate that if a customer buys onions and potatoes together, they are likely to also buy hamburger meat. Such information can be used as the basis for decisions about marketing activities such as, e.g., promotional pricing or product placements. In addition to the above example from "market basket analysis\index{market basket analysis}" association rules are employed today in many application areas including Web usage mining, intrusion detection, continuous production, and bioinformatics. In contrast with sequence mining, association rule learning typically does not consider the order of items either within a transaction or across transactions.
	
	Here are some possible questions that association rules can answer:
	\begin{itemize}
		\item Which products tend to be purchased together?
		\item Of those customers who are similar to this person, what products do they tend to buy?
		\item Of those customers who have purchased this product, what other similar products do they tend to
		view or purchase?
	\end{itemize}
	Some well-known algorithms are OneR (OneRule), Apriori, ECLAT (Equivalence Class Transformation) and FP-Growth (first pass), but they only do half the job, since they are algorithms for mining frequent itemsets. Another step needs to be done after to generate rules from frequent itemsets found in a database.
	
	\paragraph{ZeroR}\mbox{}\\\\
	"\NewTerm{ZeroR}\index{ZeroR}", short for "Zero Rule", is the simplest supervised classification method which relies on the target and ignores all predictors. ZeroR classifier simply predicts the majority category (class). Although there is no predictability power in ZeroR, it is useful for determining a baseline performance as a benchmark for other classification methods.
	
	A nice story about ZeroR algorithm is that we use it unconsciously everyday when our brain chooses the most frequent seen item as the most probable event (ZeroR classification) or when we talk about averages (ZeroR Regression).
	
	The ZeroR algorithm is (it's not a joke!): Construct a frequency table for the target and select its most frequent value.
	
	The "\textit{Play Golf} $=$ \textit{Yes}" in the example below\footnote{Thanks to Dr. Saed Sayad for having authorized us to reproduce its web page: \url{http://www.saedsayad.com/zeror.htm}} is the ZeroR rule for the learning table ($14$ rows) with an accuracy of $9/14\cong 0.64$.
	\begin{figure}[H]
		\centering
		\includegraphics[scale=1]{img/computing/zeror.jpg}
		\caption{ZeroR Data Mining (association rule) example}
	\end{figure}
	There is nothing to be said about the predictors contribution to the model because ZeroR does not use any of them.
	
	The following confusion matrix $\mathcal{C}$ (more complete than the one we have introduce during our study of the binomial logistic regression)
	\begin{table}[H]
		\centering
		\begin{tabular}{|l|l|c|c|l|l|}
		\hline
		\multicolumn{2}{|l|}{\cellcolor[HTML]{FFFFFF}} & \multicolumn{2}{c|}{\cellcolor[HTML]{C0C0C0}\textbf{Play Golf}} & \multicolumn{2}{l|}{} \\ \cline{3-4}
		\multicolumn{2}{|l|}{\multirow{-2}{*}{\cellcolor[HTML]{FFFFFF}Confusion matrix}} & \cellcolor[HTML]{C0C0C0}Yes & \cellcolor[HTML]{C0C0C0}No & \multicolumn{2}{l|}{\multirow{-2}{*}{}} \\ \hline
		\cellcolor[HTML]{C0C0C0} & \cellcolor[HTML]{C0C0C0}Yes & $a$ & $b$ & \cellcolor[HTML]{FFFFC7}\textit{Positive Predictive Value} & \cellcolor[HTML]{FFFFC7}$\dfrac{a}{a+b}$ \\ \cline{2-6} 
		\multirow{-2}{*}{\cellcolor[HTML]{C0C0C0}\textbf{ZeroR}} & \cellcolor[HTML]{C0C0C0}No & $c$ & $d$ & \cellcolor[HTML]{FFFFC7}\textit{Negative Predictive Value} & \cellcolor[HTML]{FFFFC7}$\dfrac{d}{c+d}$ \\ \hline
		\multicolumn{2}{|l|}{} & \cellcolor[HTML]{FFFFC7}\textit{Sensitivity} & \cellcolor[HTML]{FFFFC7}\textit{Specificity} & \multicolumn{2}{l|}{} \\ \cline{3-4}
		\multicolumn{2}{|l|}{\multirow{-2}{*}{}} & \cellcolor[HTML]{FFFFC7}$\dfrac{a}{a+c}$ & \cellcolor[HTML]{FFFFC7}$\dfrac{d}{b+d}$ & \multicolumn{2}{l|}{\multirow{-2}{*}{\textbf{Accuracy =}\index{accuracy}$\dfrac{a+d}{a+b+c+d}$}} \\ \hline
		\end{tabular}
		\caption{Full Confusion matrix example}
	\end{table}
	gives in the case of our example:
	\begin{table}[H]
		\centering
		\begin{tabular}{|l|l|c|c|l|l|}
		\hline
		\multicolumn{2}{|l|}{\cellcolor[HTML]{FFFFFF}} & \multicolumn{2}{c|}{\cellcolor[HTML]{C0C0C0}\textbf{Play Golf}} & \multicolumn{2}{l|}{} \\ \cline{3-4}
		\multicolumn{2}{|l|}{\multirow{-2}{*}{\cellcolor[HTML]{FFFFFF}Confusion matrix}} & \cellcolor[HTML]{C0C0C0}Yes & \cellcolor[HTML]{C0C0C0}No & \multicolumn{2}{l|}{\multirow{-2}{*}{}} \\ \hline
		\cellcolor[HTML]{C0C0C0} & \cellcolor[HTML]{C0C0C0}Yes & $9$ & $5$ & \cellcolor[HTML]{FFFFC7}\textit{Positive Predictive Value} & \cellcolor[HTML]{FFFFC7}$0.64$ \\ \cline{2-6} 
		\multirow{-2}{*}{\cellcolor[HTML]{C0C0C0}\textbf{ZeroR}} & \cellcolor[HTML]{C0C0C0}No & $0$ & $0$ & \cellcolor[HTML]{FFFFC7}\textit{Negative Predictive Value} & \cellcolor[HTML]{FFFFC7}$0.00$ \\ \hline
		\multicolumn{2}{|l|}{} & \cellcolor[HTML]{FFFFC7}\textit{Sensitivity} & \cellcolor[HTML]{FFFFC7}\textit{Specificity} & \multicolumn{2}{l|}{} \\ \cline{3-4}
		\multicolumn{2}{|l|}{\multirow{-2}{*}{}} & \cellcolor[HTML]{FFFFC7}$1.00$ & \cellcolor[HTML]{FFFFC7}$0.00$ & \multicolumn{2}{l|}{\multirow{-2}{*}{\textbf{Accuracy =}$0.65$}} \\ \hline
		\end{tabular}
	\end{table}
	that ZeroR only predicts the majority class correctly. As mentioned before, ZeroR is only useful for determining a baseline performance for other classification methods.
	
	\paragraph{OneR (classification by induction)}\mbox{}\\\\
	"\NewTerm{OneR}\index{OneR}", short for "One Rule", is a simple, yet accurate, classification algorithm that generates one rule for each predictor in the data, then selects the rule with the smallest total error as its "one rule".  To create a rule for a predictor $p_i$ among the set of all predictor $\mathbb{P}$, we construct a frequency table for each predictor against each possible value $t_i$ of the target attribute set $\mathbb{T}$.
	
	The OneR algorithm is the following\footnote{Thanks again to Dr. Saed Sayad for having authorized us to reproduce its web page: \url{http://www.saedsayad.com/oner.htm}}:

	\begin{algorithm}[H]
	 \KwData{$p_i$ $a_{i,j}$,$r$, $T$ }
	 \KwResult{$\min\{(p_i,\varepsilon_i\}$}
	 initialization\;
	  \ForEach{predictor $p_i \in \mathcal{P}$}{%
	  	\ForEach{attribute value $a_{i,j}$ in $p_i$}{
	  	Count how often each value of target (class) $t_k\in\mathcal{T}$ appears in all rows $r$ for $p_i$\;
	  	Find the most frequent class $t_k$ for $a_{i,j}$ \;
	  	Make the rule assign that class to this value of the predictor;
	  	}
		Calculate the total error of the rules of each predictor $\varepsilon_{i}$
      }
	 \caption{OneR (one-Rule) algorithm}
	\end{algorithm}
	As companion example let us consider again the following learning table ($14$ rows):
	\begin{figure}[H]
		\centering
		\includegraphics[scale=0.7]{img/computing/oner.jpg}
		\caption{OneR Data Mining (association rule) example}
	\end{figure}
	
	Now we create the following frequency tables for each predictor and each attribute inside each predictor:
	\begin{table}[H]
		\centering
		\begin{tabular}{|l|l|c|c|}
		\hline
		\multicolumn{2}{|l|}{} & \multicolumn{2}{l|}{\cellcolor[HTML]{FFCCC9}\textbf{Play Golf}} \\ \cline{3-4} 
		\multicolumn{2}{|l|}{\multirow{-2}{*}{}} & \cellcolor[HTML]{FFCCC9}\textit{Yes} & \cellcolor[HTML]{FFCCC9}\textit{No} \\ \hline
		\cellcolor[HTML]{ECF4FF} & \cellcolor[HTML]{ECF4FF}\textit{Sunny} & $3$ & $2$ \\ \cline{2-4} 
		\cellcolor[HTML]{ECF4FF} & \cellcolor[HTML]{ECF4FF}\textit{Overcast} & $4$ & $0$ \\ \cline{2-4} 
		\multirow{-3}{*}{\cellcolor[HTML]{ECF4FF}\textbf{\phantom{x} Outlook \phantom{xx}}} & \cellcolor[HTML]{ECF4FF}\textit{Rainy} & $2$ & $3$ \\ \hline
		\end{tabular}
	\end{table}
	\begin{table}[H]
		\centering
		\begin{tabular}{|l|l|c|c|}
		\hline
		\multicolumn{2}{|l|}{} & \multicolumn{2}{l|}{\cellcolor[HTML]{FFCCC9}\textbf{Play Golf}} \\ \cline{3-4} 
		\multicolumn{2}{|l|}{\multirow{-2}{*}{}} & \cellcolor[HTML]{FFCCC9}\textit{Yes} & \cellcolor[HTML]{FFCCC9}\textit{No} \\ \hline
		\cellcolor[HTML]{ECF4FF} & \cellcolor[HTML]{ECF4FF}\textit{Hot} & $2$ & $2$ \\ \cline{2-4} 
		\cellcolor[HTML]{ECF4FF} & \cellcolor[HTML]{ECF4FF}\textit{Mild\phantom{xxxx}} & $4$ & $2$ \\ \cline{2-4} 
		\multirow{-3}{*}{\cellcolor[HTML]{ECF4FF}\textbf{Temperature}} & \cellcolor[HTML]{ECF4FF}\textit{Cold} & $3$ & $1$ \\ \hline
		\end{tabular}
	\end{table}
	
	\begin{table}[H]
		\centering
		\begin{tabular}{|l|l|c|c|}
		\hline
		\multicolumn{2}{|l|}{} & \multicolumn{2}{l|}{\cellcolor[HTML]{FFCCC9}\textbf{Play Golf}} \\ \cline{3-4} 
		\multicolumn{2}{|l|}{\multirow{-2}{*}{}} & \cellcolor[HTML]{FFCCC9}\textit{Yes} & \cellcolor[HTML]{FFCCC9}\textit{No} \\ \hline
		\cellcolor[HTML]{ECF4FF} & \cellcolor[HTML]{ECF4FF}\textit{High} & $3$ & $4$ \\ \cline{2-4} 
		\multirow{-2}{*}{\cellcolor[HTML]{ECF4FF}\textbf{\phantom{x}Humidity\phantom{xx}}} & \cellcolor[HTML]{ECF4FF}\textit{Normal\phantom{xx}} & $6$ & $1$ \\ \hline
		\end{tabular}
	\end{table}
	
	\begin{table}[H]
		\centering
		\begin{tabular}{|l|l|c|c|}
		\hline
		\multicolumn{2}{|l|}{} & \multicolumn{2}{l|}{\cellcolor[HTML]{FFCCC9}\textbf{Play Golf}} \\ \cline{3-4} 
		\multicolumn{2}{|l|}{\multirow{-2}{*}{}} & \cellcolor[HTML]{FFCCC9}\textit{Yes} & \cellcolor[HTML]{FFCCC9}\textit{No} \\ \hline
		\cellcolor[HTML]{ECF4FF} & \cellcolor[HTML]{ECF4FF}\textit{False} & $6$ & $2$ \\ \cline{2-4} 
		\multirow{-2}{*}{\cellcolor[HTML]{ECF4FF}\textbf{\phantom{xxx}Windy\phantom{xxx}}} & \cellcolor[HTML]{ECF4FF}\textit{True\phantom{xxxx}} & $3$ & $3$ \\ \hline
		\end{tabular}
	\end{table}
	Now according to the algorithm, for each predictor we keep only the attribute that has the biggest frequency (in red below\footnote{We can notice that in case of equality we take the target that occurs the more for the other attributes.}):
	\begin{table}[H]
		\centering
		\begin{tabular}{|l|l|c|c|}
		\hline
		\multicolumn{2}{|l|}{} & \multicolumn{2}{l|}{\cellcolor[HTML]{FFCCC9}\textbf{Play Golf}} \\ \cline{3-4} 
		\multicolumn{2}{|l|}{\multirow{-2}{*}{}} & \cellcolor[HTML]{FFCCC9}\textit{Yes} & \cellcolor[HTML]{FFCCC9}\textit{No} \\ \hline
		\cellcolor[HTML]{ECF4FF} & \cellcolor[HTML]{ECF4FF}\textit{Sunny} & $\color{red}{3}$ & $2$ \\ \cline{2-4} 
		\cellcolor[HTML]{ECF4FF} & \cellcolor[HTML]{ECF4FF}\textit{Overcast} & $\color{red}{4}$ & $0$ \\ \cline{2-4} 
		\multirow{-3}{*}{\cellcolor[HTML]{ECF4FF}\textbf{\phantom{x} Outlook \phantom{xx}}} & \cellcolor[HTML]{ECF4FF}\textit{Rainy} & $2$ & $\color{red}{3}$ \\ \hline
		\end{tabular}
	\end{table}
	\begin{table}[H]
		\centering
		\begin{tabular}{|l|l|c|c|}
		\hline
		\multicolumn{2}{|l|}{} & \multicolumn{2}{l|}{\cellcolor[HTML]{FFCCC9}\textbf{Play Golf}} \\ \cline{3-4} 
		\multicolumn{2}{|l|}{\multirow{-2}{*}{}} & \cellcolor[HTML]{FFCCC9}\textit{Yes} & \cellcolor[HTML]{FFCCC9}\textit{No} \\ \hline
		\cellcolor[HTML]{ECF4FF} & \cellcolor[HTML]{ECF4FF}\textit{Hot} & $\color{red}{2}$ & $2$ \\ \cline{2-4} 
		\cellcolor[HTML]{ECF4FF} & \cellcolor[HTML]{ECF4FF}\textit{Mild\phantom{xxxx}} & $\color{red}{4}$ & $2$ \\ \cline{2-4} 
		\multirow{-3}{*}{\cellcolor[HTML]{ECF4FF}\textbf{Temperature}} & \cellcolor[HTML]{ECF4FF}\textit{Cold} & $\color{red}{3}$ & $1$ \\ \hline
		\end{tabular}
	\end{table}
	
	\begin{table}[H]
		\centering
		\begin{tabular}{|l|l|c|c|}
		\hline
		\multicolumn{2}{|l|}{} & \multicolumn{2}{l|}{\cellcolor[HTML]{FFCCC9}\textbf{Play Golf}} \\ \cline{3-4} 
		\multicolumn{2}{|l|}{\multirow{-2}{*}{}} & \cellcolor[HTML]{FFCCC9}\textit{Yes} & \cellcolor[HTML]{FFCCC9}\textit{No} \\ \hline
		\cellcolor[HTML]{ECF4FF} & \cellcolor[HTML]{ECF4FF}\textit{High} & $3$ & $\color{red}{4}$ \\ \cline{2-4} 
		\multirow{-2}{*}{\cellcolor[HTML]{ECF4FF}\textbf{\phantom{x}Humidity\phantom{xx}}} & \cellcolor[HTML]{ECF4FF}\textit{Normal\phantom{xx}} & $\color{red}{6}$ & $1$ \\ \hline
		\end{tabular}
	\end{table}
	
	\begin{table}[H]
		\centering
		\begin{tabular}{|l|l|c|c|}
		\hline
		\multicolumn{2}{|l|}{} & \multicolumn{2}{l|}{\cellcolor[HTML]{FFCCC9}\textbf{Play Golf}} \\ \cline{3-4} 
		\multicolumn{2}{|l|}{\multirow{-2}{*}{}} & \cellcolor[HTML]{FFCCC9}\textit{Yes} & \cellcolor[HTML]{FFCCC9}\textit{No} \\ \hline
		\cellcolor[HTML]{ECF4FF} & \cellcolor[HTML]{ECF4FF}\textit{False} & $\color{red}{6}$ & $2$ \\ \cline{2-4} 
		\multirow{-2}{*}{\cellcolor[HTML]{ECF4FF}\textbf{\phantom{xxx}Windy\phantom{xxx}}} & \cellcolor[HTML]{ECF4FF}\textit{True\phantom{xxxx}} & $\color{red}{3}$ & $3$ \\ \hline
		\end{tabular}
	\end{table}
	This means that so far we would have the rules:
	\begin{itemize}
		\item For \textit{Outlook}:
		\begin{table}[H]
			\centering
			\begin{tabular}{l}
			\texttt{IF Outlook = Sunny THEN Play Golf = Yes}\\
			\texttt{IF Outlook = Overcast THEN Play Golf = Yes}\\
			\texttt{IF Outlook = Rainy THEN Play Golf = No}\\
			\end{tabular}
		\end{table}
	
		\item For \textit{Temperature}:
		\begin{table}[H]
			\centering
			\begin{tabular}{l}
			\texttt{IF Temperature = Hot THEN Play Golf = Yes}\\
			\texttt{IF Temperature = Mild THEN Play Golf = Yes}\\
			\texttt{IF Temperature = Cool THEN Play Golf = Yes}\\
			\end{tabular}
		\end{table}
	
		\item For \textit{Humidity}:
		\begin{table}[H]
			\centering
			\begin{tabular}{l}
			\texttt{IF Humidity = High THEN Play Golf = No}\\
			\texttt{IF Humidity = Normal THEN Play Golf = Yes}\\
			\end{tabular}
		\end{table}
	
		\item For \textit{Windy}:
		\begin{table}[H]
			\centering
			\begin{tabular}{l}
			\texttt{IF Windy = False THEN Play Golf = Yes}\\
			\texttt{IF Windy = True THEN Play Golf = Yes}\\
			\end{tabular}
		\end{table}
	\end{itemize}
	Now if we count for each attribute according to the above rules the number of error of wrongly (false) classified items, we get:
	\begin{itemize}
		\item For \textit{Outlook}: $4$ wrongly classified items ($10$ well classified)
	
		\item For \textit{Temperature}: $5$ wrongly classified items ($9$ well classified)
	
		\item For \textit{Humidity}: $4$ wrongly classified items ($10$ well classified)
	
		\item For \textit{Windy}: $5$ wrongly classified items ($9$ well classified)
	\end{itemize}
	We see that only like this it is quite not easy to select the best rule. Therefore a possible solution is to use again a confusion matrix $\mathcal{C}$ for each attribute:
	\begin{table}[H]
		\centering
		\begin{tabular}{|l|l|c|c|l|l|}
		\hline
		\multicolumn{2}{|l|}{\cellcolor[HTML]{FFFFFF}} & \multicolumn{2}{c|}{\cellcolor[HTML]{C0C0C0}\textbf{Play Golf}} & \multicolumn{2}{l|}{} \\ \cline{3-4}
		\multicolumn{2}{|l|}{\multirow{-2}{*}{\cellcolor[HTML]{FFFFFF}Confusion matrix}} & \cellcolor[HTML]{C0C0C0}Yes & \cellcolor[HTML]{C0C0C0}No & \multicolumn{2}{l|}{\multirow{-2}{*}{}} \\ \hline
		\cellcolor[HTML]{C0C0C0} & \cellcolor[HTML]{C0C0C0}Yes & $a$ & $b$ & \cellcolor[HTML]{FFFFC7}\textit{Positive Predictive Value} & \cellcolor[HTML]{FFFFC7}$\dfrac{a}{a+b}$ \\ \cline{2-6} 
		\multirow{-2}{*}{\cellcolor[HTML]{C0C0C0}\textbf{OneR}} & \cellcolor[HTML]{C0C0C0}No & $c$ & $d$ & \cellcolor[HTML]{FFFFC7}\textit{Negative Predictive Value} & \cellcolor[HTML]{FFFFC7}$\dfrac{d}{c+d}$ \\ \hline
		\multicolumn{2}{|l|}{} & \cellcolor[HTML]{FFFFC7}\textit{Sensitivity} & \cellcolor[HTML]{FFFFC7}\textit{Specificity} & \multicolumn{2}{l|}{} \\ \cline{3-4}
		\multicolumn{2}{|l|}{\multirow{-2}{*}{}} & \cellcolor[HTML]{FFFFC7}$\dfrac{a}{a+c}$ & \cellcolor[HTML]{FFFFC7}$\dfrac{d}{b+d}$ & \multicolumn{2}{l|}{\multirow{-2}{*}{\textbf{Accuracy =}$\dfrac{a+d}{a+b+c+d}$}} \\ \hline
		\end{tabular}
		\caption{Full Confusion matrix example}
	\end{table}
	Therefore after building all the confusion matrices we notice that the one with the best accuracy if for the \textit{Outlook} attribute:
	\begin{table}[H]
		\centering
		\begin{tabular}{|l|l|c|c|l|l|}
		\hline
		\multicolumn{2}{|l|}{\cellcolor[HTML]{FFFFFF}} & \multicolumn{2}{c|}{\cellcolor[HTML]{C0C0C0}\textbf{Play Golf}} & \multicolumn{2}{l|}{} \\ \cline{3-4}
		\multicolumn{2}{|l|}{\multirow{-2}{*}{\cellcolor[HTML]{FFFFFF}Confusion matrix}} & \cellcolor[HTML]{C0C0C0}Yes & \cellcolor[HTML]{C0C0C0}No & \multicolumn{2}{l|}{\multirow{-2}{*}{}} \\ \hline
		\cellcolor[HTML]{C0C0C0} & \cellcolor[HTML]{C0C0C0}Yes & $7$ & $2$ & \cellcolor[HTML]{FFFFC7}\textit{Positive Predictive Value} & \cellcolor[HTML]{FFFFC7}$0.78$ \\ \cline{2-6} 
		\multirow{-2}{*}{\cellcolor[HTML]{C0C0C0}\textbf{OneR}} & \cellcolor[HTML]{C0C0C0}No & $2$ & $3$ & \cellcolor[HTML]{FFFFC7}\textit{Negative Predictive Value} & \cellcolor[HTML]{FFFFC7}$0.60$ \\ \hline
		\multicolumn{2}{|l|}{} & \cellcolor[HTML]{FFFFC7}\textit{Sensitivity} & \cellcolor[HTML]{FFFFC7}\textit{Specificity} & \multicolumn{2}{l|}{} \\ \cline{3-4}
		\multicolumn{2}{|l|}{\multirow{-2}{*}{}} & \cellcolor[HTML]{FFFFC7}$0.78$ & \cellcolor[HTML]{FFFFC7}$0.60$ & \multicolumn{2}{l|}{\multirow{-2}{*}{\textbf{Accuracy =}$0.71$}} \\ \hline
		\end{tabular}
	\end{table}
	Hence the final chosen association rule:
	\begin{table}[H]
		\centering
		\begin{tabular}{l}
		\texttt{IF Outlook = Sunny THEN Play Golf = Yes}\\
		\texttt{IF Outlook = Overcast THEN Play Golf = Yes}\\
		\texttt{IF Outlook = Rainy THEN Play Golf = No}\\
		\end{tabular}
	\end{table}
	\StickyNote[2.5cm]{\LARGE ToDo: Create the confusion matrices for the other attributes}[6.5cm]
	
	\paragraph{Apriori}\mbox{}\\\\
	Before introducing the Apriori algorithm  which is the second most popular algorithm of the Association Rules, it is important to define some key concepts specific to these particular methods of learning.
	
	Here are some vocabulary definitions:
	\begin{itemize}
		\item A "\NewTerm{transaction}" is a basket or cart of one or more items purchased by consumers. 
		\begin{tcolorbox}[colframe=black,colback=white,sharp corners]
		\textbf{{\Large \ding{45}}Example:}\\\\
		Here is a base of five transactions or baskets:
		\begin{table}[H]
			\centering
			\begin{tabular}{|c|c|c|c|c|}
			\hline
			\rowcolor[HTML]{9B9B9B} 
			\multicolumn{1}{|l|}{\cellcolor[HTML]{9B9B9B}{\color[HTML]{333333} \textbf{Transaction Number}}} & \multicolumn{4}{c|}{\cellcolor[HTML]{9B9B9B}{\color[HTML]{333333} \textbf{Basket composition}}} \\ \hline
			$1$ & bread & juice & jam & sugar \\ \hline
			$2$ & milk & bread & sugar &  \\ \hline
			$3$ & cheese & bread & butter & milk \\ \hline
			$4$ & cheese & milk & sugar & flour \\ \hline
			$5$ & chips & bread & jam & yoghurt \\ \hline
			\end{tabular}
		\end{table}
		\end{tcolorbox}
		
		\item An "\NewTerm{item}" designates an article. Example: juice, bread, milk ...
		
		\item An "\NewTerm{itemset}" is a collection of items or articles. A itemset containing $k$ items or a "$k$-itemset" is written between brackets $\{\mathrm{item\; 1}, \mathrm{item\; 2}, \mathrm{item\; 3}, \ldots, \mathrm{item}\; k\}$.
		
		\begin{tcolorbox}[colframe=black,colback=white,sharp corners]
		\textbf{{\Large \ding{45}}Examples:}\\
		\begin{itemize}
			\item $1\text{-}\mathrm{itemset}$: $\{\mathrm{milk}\}$ or $\{\mathrm{bread}\}$
		
			\item $2\text{-}\mathrm{itemset}$: $\{\mathrm{milk},\mathrm{bread}\}$ or $\{\mathrm{bread},\mathrm{butter}\}$
		
			\item $3\text{-}\mathrm{itemset}$: $\{\mathrm{bread},\mathrm{butter},\mathrm{sugar}\}$
		
			\item ...
		\end{itemize}
		\end{tcolorbox}
		
		\item Given a $k$-itemset $P$, the "\NewTerm{support}" of $P$, is the percentage of transactions containing $P$. This is an index of reliability.
		\begin{tcolorbox}[colframe=black,colback=white,sharp corners]
		\textbf{{\Large \ding{45}}Examples:}\\\\
		E1. For $S = \{\mathrm{bread}, \mathrm{milk}\}$ then the support according to the previous set of baskets is equal to $P=2/5 = 0.4$, hence $40\%$:
		\begin{table}[H]
			\centering
			\begin{tabular}{|c|c|c|c|c|}
			\hline
			\rowcolor[HTML]{9B9B9B} 
			\multicolumn{1}{|l|}{\cellcolor[HTML]{9B9B9B}{\color[HTML]{333333} \textbf{Transaction Number}}} & \multicolumn{4}{c|}{\cellcolor[HTML]{9B9B9B}{\color[HTML]{333333} \textbf{Basket composition}}} \\ \hline
			$1$ & bread & juice & jam & sugar \\ \hline
			\cellcolor[HTML]{9AFF99}$2$ & \cellcolor[HTML]{9AFF99}milk & \cellcolor[HTML]{9AFF99}bread & sugar &  \\ \hline
			\cellcolor[HTML]{9AFF99}$3$ & cheese & \cellcolor[HTML]{9AFF99}bread & butter & \cellcolor[HTML]{9AFF99}milk \\ \hline
			$4$ & cheese & milk & sugar & flour \\ \hline
			$5$ & chips & bread & jam & yoghurt \\ \hline
			\end{tabular}
		\end{table}
		
		E2. For  $L = \{\mathrm{bread}\}$ then the support of $L$ is $P=4/5 = 0.8$, hence $80\%$ of the transactions contains $L$:
		\begin{table}[H]
			\centering
			\begin{tabular}{|c|c|c|c|c|}
			\hline
			\rowcolor[HTML]{9B9B9B} 
			\multicolumn{1}{|l|}{\cellcolor[HTML]{9B9B9B}{\color[HTML]{333333} \textbf{Transaction Number}}} & \multicolumn{4}{c|}{\cellcolor[HTML]{9B9B9B}{\color[HTML]{333333} \textbf{Basket composition}}} \\ \hline
			\cellcolor[HTML]{9AFF99}$1$ & \cellcolor[HTML]{9AFF99}bread & juice & jam & sugar \\ \hline 
			\cellcolor[HTML]{9AFF99}$2$ & milk & \cellcolor[HTML]{9AFF99}bread & sugar &  \\ \hline
			\cellcolor[HTML]{9AFF99}$3$ & cheese & \cellcolor[HTML]{9AFF99}bread & butter & milk \\ \hline
			$4$ & cheese & milk & sugar & flour \\ \hline
			\cellcolor[HTML]{9AFF99}$5$ & chips & \cellcolor[HTML]{9AFF99}bread & jam & yoghurt \\ \hline
			\end{tabular}
		\end{table}
		\end{tcolorbox}
		
		\item The "\NewTerm{minimum support}" is the minimum threshold of from which one qualifies a $k$-itemset as frequent.
		
		\item One will say of a itemset that it is a "\NewTerm{frequent $k$-itemset}" if its support is greater or equal to the minimum support.
		
		\begin{tcolorbox}[colframe=black,colback=white,sharp corners]
		\textbf{{\Large \ding{45}}Examples:}\\\\
		If the minimum support is $0.5$, we can say that $L$, from the previous example, is a frequent itemset but not $S$ whose support is only $0.4$.
		\end{tcolorbox} 
	\end{itemize}
	
	"\NewTerm{Apriori}\index{apriori}" is an algorithm that was proposed by Agrawal and Srikant in 1994  for frequent item set mining and association rule learning over transactional databases. It proceeds by identifying the frequent individual items in the database and extending them to larger and larger item sets as long as those item sets appear sufficiently often in the database. The frequent item sets determined by Apriori can be used to determine association rules which highlight general trends in the database: this has applications in domains such as market basket analysis.
	
	The "a priori" algorithm is based on the so-called a priori property which is stated as follows: \textit{If one itemset is considered as frequent then any other sub itemset arising from the items of $P$ is also frequent}.
	
	Taking again the following table as companion example:
	\begin{table}[H]
		\centering
		\begin{tabular}{|c|c|c|c|c|}
		\hline
		\rowcolor[HTML]{9B9B9B} 
		\multicolumn{1}{|l|}{\cellcolor[HTML]{9B9B9B}{\color[HTML]{333333} \textbf{Transaction Number}}} & \multicolumn{4}{c|}{\cellcolor[HTML]{9B9B9B}{\color[HTML]{333333} \textbf{Basket composition}}} \\ \hline
		$1$ & bread & juice & jam & sugar \\ \hline
		$2$ & milk & bread & sugar &  \\ \hline
		$3$ & cheese & bread & butter & milk \\ \hline
		$4$ & cheese & milk & sugar & flour \\ \hline
		$5$ & chips & bread & jam & yoghurt \\ \hline
		\end{tabular}
	\end{table}
	By setting the minimum support to $P_{\min}$ to $0.35$, the previous itemset $S = \{\mathrm{bread}, \mathrm{milk}\}$ with a support of $0.4$ is therefore referred to as frequent. The previous itemset $L = \{\mathrm{bread}\}$ is a sub-item of $L\subset S$, it is according to the apriori property also frequent (support of $L = 0.8>0.35$) and if one considers another new sub-itemset $D = \{\mathrm{milk}\}$ of $S$, one sees that the support the 1-itemset of $D$ is $3/5 = 0.6> 0.35$.
	
	If we have $n$ different items, then we have $2^n-1$ possible different itemsets (the $-1$ is to remove the empty itemset). Therefore in our case, we could create $2^{10}-1=1023$ different possible itemsets with:
	
	Thus, the number of itemsets to be analysed can be considerably reduced following the pruning of low-level non-frequent itemsets.
	
	Given a minimal support that we will denote $\delta$, the Apriori algorithm adopts an ascending iterative approach that determines all possible $1$-itemset first and then those whose support is less than $\delta$ are subject to pruning and the rest, those that are frequent are stored in an $L_1$ itemset collections.
	
	For the $5$ transactions of the previous illustration in the $10$ possible itemsets of type $1$-itemset, for a minimum support of $\delta= 0.35$, only the following itemsets will be part of $L_1$:
	\begin{table}[H]
		\centering
		\begin{tabular}{|c|c|}
		\hline
		\rowcolor[HTML]{9B9B9B} 
		\textbf{$\pmb{1}$-itemset} & \textbf{Support} \\ \hline
		$\{\mathrm{bread}\}$ & $0.8$ \\ \hline
		$\{\mathrm{sugar}\}$ & $0.6$ \\ \hline
		$\{\mathrm{milk}\}$ & $0.6$ \\ \hline
		$\{\mathrm{jam}\}$ & $0.4$ \\ \hline
		$\{\mathrm{cheese}\}$ & $0.4$ \\ \hline
		\end{tabular}
	\end{table}
	The next step consists of all possible $2$-itemsets from the items contained in $L_1$. Again the $2$-itemsets are tested, those whose support is less than $\delta$ are pruned and only the frequent $2$-itemsets are kept in a collection of $L_2$ itemsets.
	
	On the basis of the $L_1$ previously constituted, it is always possible with $\delta = 0.35$ to obtain a collection $L_2$ consisting of:
	\begin{table}[H]
		\centering
		\begin{tabular}{|c|c|}
		\hline
		\rowcolor[HTML]{9B9B9B} 
		\textbf{$\pmb{2}$-itemset} & \textbf{Support} \\ \hline
		$\{\mathrm{jam},\mathrm{bread}\}$ & $0.4$ \\ \hline
		$\{\mathrm{cheese},\mathrm{milk}\}$ & $0.6$ \\ \hline
		$\{\mathrm{sugar},\mathrm{milk}\}$ & $0.6$ \\ \hline
		$\{\mathrm{bread},\mathrm{sugar}\}$ & $0.4$ \\ \hline
		$\{\mathrm{milk},\mathrm{bread}\}$ & $0.4$ \\ \hline
		\end{tabular}
	\end{table}
	This aggregation and pruning process is repeated for a collection $L_k$ containing the possible itemsets of $k$-itemsets having validated the test of the minimum support $\delta$.
	
	So figuring the concept with transactions of a maximum of $4$-itemsets $\{\mathrm{A},\mathrm{B},\mathrm{C},\mathrm{D}\}$ then with a total of $2^4-1=15$ possible combinations, this procedure would lead for example to:
	\begin{figure}[H]
		\centering
		\includegraphics[scale=1]{img/computing/apriori.jpg}
	\end{figure}
	The algorithm stops if for a new collection $L_{k+1}$, no $(k + 1)$-itemsets validates the test of the minimum support $\delta$ (as is the case for the previous figure where one can not obtain $3$-itemsets on the basis of $L_2$). Optionally, we can define an integer $K$ which specifies either the maximum number of items that an itemset can contain or the maximum number of iterations to be performed by the algorithm.
	
	Once the collections of itemsets are formed, it is possible to transcribe this in the form of candidate rules (for example $X\Rightarrow Y$ or: $X$ implies $Y$). However, some criteria are defined to assess the relevance of an association rule:
	\begin{itemize}
		\item "Confidence" or "trust of a rule": It's the measure of certainty or confidence associated with an association rule. Formally, for a rule $X \Rightarrow Y$, it is the percentage of transaction containing both $X$ and $Y$ on the total of transactions containing $X$:
		
		\begin{tcolorbox}[colframe=black,colback=white,sharp corners]
		\textbf{{\Large \ding{45}}Example:}\\\\
		What is the confidence index of the rule:
		
		As:
		
		and:
		
		Then:
		
		thus a confidence index of $50\%$. 
		\end{tcolorbox}
		In practice, an acceptable threshold of confidence is defined. We are confident with regard to a rule when the index of confidence is large and moves away from the minimum.
		
		\item "Lift of an association rule\index{lift}\label{lift association rule}": Assuming that $X$ and $Y$ are statistically independent (\SeeChapter{see section Probabilities page \pageref{joint probability}}), the lift measures the percentage of the number of times it is more frequent to obtain $X$ and $Y$:
		
		Thus, if the lift is equal to $1$ then $X$ and $Y$ are  independent, which means that the rule is not at all relevant: a simple coincidence! While a lift greater than $1$, shows how useful the rule is and a higher value shows how strong the rule is and therefore relevant, it is then a kind of correlation measure. A lift smaller than $1$ indicates that $X$ and $Y$ appear less often together than expected, this means that the occurrence of $X$ has a negative effect on the occurrence of $Y$ or that $X$ is negatively correlated with $Y$. Thus, lift is a value between $0$ and $+\infty$.
		
		\begin{tcolorbox}[title=Remark,colframe=black,arc=10pt]
		We could also run (obviously) a complementary $\chi^2$-test of independence of a $2\times 2$ contingency table (having for entries $X$, $Y$, $\neg X$, $\neg Y$) to test if there is statistically or not independence (\SeeChapter{see section Statistics page \pageref{chi-square test of independence}}).
		\end{tcolorbox}	
	
		\begin{tcolorbox}[colframe=black,colback=white,sharp corners]
		\textbf{{\Large \ding{45}}Example:}\\\\
		What is the lift of the rule:
		
		As:
		
		and:
		
		and:
		
		Then:
		
		Of course, it is always necessary to compare these results with the values obtained with the other association rules before judging the relevance.
		\end{tcolorbox}
	\end{itemize}
	Lift and $\chi^2$ are not "null-invariant". That means that they are not good to evaluated data that contain too many few null transactions. 
	
	We limited ourselves here to the two most widely used criteria for evaluating association rules, but there are other criteria (some of them are null-invariant) such as Leverage, Conviction index, Coverage, Jaccard, Cosine, Kulcynski, etc.
	
	\begin{table}[H]
		\centering
		\begin{tabular}{|l|c|c|c|}
		\hline
		\rowcolor[HTML]{C0C0C0} 
		\multicolumn{1}{|c|}{\cellcolor[HTML]{C0C0C0}\textbf{Measure}} & \textbf{Definition} & \textbf{Range} & \textbf{Null-Invariant} \\ \hline
		$\chi^2(X,Y)$ & $\displaystyle \sum_{i,j}\dfrac{(O_{ij}-E_{ij})^2}{E_{ij}}$ & $[0,+\infty[$ & No \\ \hline
		$\text{Lift}(X,Y)$ & $\displaystyle \dfrac{S(X\cap Y)}{S(X)\cdot S(Y)}$ & $[0,+\infty[$ & No \\ \hline
		$\text{AllConf}(X,Y)$ & $\displaystyle \dfrac{S(X\cap Y)}{\max \left(S(X), S(Y)\right)}$ & $[0,1]$ & Yes \\ \hline
		$\text{Jaccard}(X,Y)$ & $\displaystyle \dfrac{S(X\cap Y)}{S(X)+S(Y)-S(X\cap Y)}$ & $[0,1]$ & Yes \\ \hline
		$\text{Cosine}(X,Y)$ & $\displaystyle \dfrac{S(X\cap Y)}{\sqrt{S(X)\cdot S(Y)}}$ & $[0,1]$ & Yes \\ \hline
		$\text{Kulczinsky}(X,Y)$ & $\displaystyle \dfrac{1}{2}\left(\dfrac{S(X\cap Y)}{S(X)}+\dfrac{S(X\cap Y)}{S(Y)}\right)$ & $[0,1]$ & Yes \\ \hline
		$\text{MaxConf}(X,Y)$ & $\displaystyle  \max\left(\dfrac{S(X)}{S(X\cap Y)},\dfrac{S(Y)}{S(X\cap Y)}\right)$ & $[0,1]$ & Yes \\ \hline
		$\ldots$ & $\ldots$ & $\ldots$ & $\ldots$ \\ \hline
		\end{tabular}
		\caption{Some support-confidence interesting measures}
	\end{table}
	
	Finally, we can say that the association rules are very easily interpretable and their implementation simple and easy to transcribe into operational rules. However, a number of points are quite problematic:
	\begin{itemize}
		\item Runtime cost: Even if the Apriori algorithm eliminates several occurrences with the minimum support rule, it is easy to imagine that with a large database, occurrences that exceed the minimum support can be large.

		\item Fallacious or parasitic associations: This effect depends in particular on the minimum support defined, but in front of a large database, to obtain a sufficient quantity of rules it is sometimes useful to consider a low minimum support.
	\end{itemize}
	
	\subsubsection{Equivalence CLAss Transformation (ECLAT)}
	The "\NewTerm{ECLAT}\index{ECLAT}" algorithm, acronym for Equivalence CLAss Transformation, is an alternative way to the previous method to build the item-set that performs better in huge databases.
	
	Let us use the companion example as before to introduce detail how ECLAT works:
	\begin{table}[H]
		\centering
		\begin{tabular}{|c|c|c|c|c|}
		\hline
		\rowcolor[HTML]{9B9B9B} 
		\multicolumn{1}{|l|}{\cellcolor[HTML]{9B9B9B}{\color[HTML]{333333} \textbf{Transaction Number}}} & \multicolumn{4}{c|}{\cellcolor[HTML]{9B9B9B}{\color[HTML]{333333} \textbf{Basket composition}}} \\ \hline
		$1$ & bread & juice & jam & sugar \\ \hline
		$2$ & milk & bread & sugar &  \\ \hline
		$3$ & cheese & bread & butter & milk \\ \hline
		$4$ & cheese & milk & sugar & flour \\ \hline
		$5$ & chips & bread & jam & yoghurt \\ \hline
		\end{tabular}
	\end{table}
	The idea of ECLAT is for each item, to store a list of transaction ID's (abbreviated "tids" as acronym from "transaction ID's" or sometimes "tidlist"). It as vertical data layout at the opposite of the above one. Indeed, the transaction table above transformed into a tids will look like following for the $1$-itemset:
	\begin{table}[H]
		\centering
		\begin{tabular}{|c|c|c|c|c|}
		\hline
		\rowcolor[HTML]{9B9B9B} 
		{\cellcolor[HTML]{9B9B9B}{\color[HTML]{333333} \textbf{Item Set}}} & {\cellcolor[HTML]{9B9B9B}{\color[HTML]{333333} \textbf{TID Set}}} \\ \hline
		bread & $1,2,3,5$ \\ \hline
		jam & $1,5$  \\ \hline
		cheese & $3,4$  \\ \hline
		juice & $1$ \\ \hline
		milk & $2,3,4$ \\ \hline
		butter & $3$ \\ \hline
		sugar & $1,2,4$ \\ \hline
		flour & $4$ \\ \hline
		chips & $5$ \\ \hline
		\end{tabular}
		\caption[]{ECLAT $1$-itemset tidlist}
	\end{table}	
	And the corresponding tidlist for $2$-itemset:
	\begin{table}[H]
		\centering
		\begin{tabular}{|c|c|c|c|c|}
		\hline
		\rowcolor[HTML]{9B9B9B} 
		{\cellcolor[HTML]{9B9B9B}{\color[HTML]{333333} \textbf{Item Set}}} & {\cellcolor[HTML]{9B9B9B}{\color[HTML]{333333} \textbf{TID Set}}} \\ \hline
		\{bread,jam\} & $1,5$ \\ \hline
		\{bread,cheese\} & $3$ \\ \hline
		\{bread,juice\} & $1$ \\ \hline
		\{bread,milk\} & $2,3$ \\ \hline
		\{bread,butter\} & $2,3$ \\ \hline
		\{bread,sugar\} & $1,2$ \\ \hline
		\{cheese,flour\} & $4$ \\ \hline
		\{breach, chips\} & $5$ \\ \hline
		\{jam, juice\} & $1$ \\ \hline
		\{jam, chips\} & $5$ \\ \hline
		\{cheese, milk\} & $4$ \\ \hline
		\{cheese, butter\} & $4$ \\ \hline
		$\ldots$ & $\ldots$ \\ \hline
		\end{tabular}
		\caption[]{ECLAT $2$-itemset tidlist}
	\end{table}	
	And the corresponding tidlist for $3$-itemset:
	\begin{table}[H]
		\centering
		\begin{tabular}{|c|c|c|c|c|}
		\hline
		\rowcolor[HTML]{9B9B9B} 
		{\cellcolor[HTML]{9B9B9B}{\color[HTML]{333333} \textbf{Item Set}}} & {\cellcolor[HTML]{9B9B9B}{\color[HTML]{333333} \textbf{TID Set}}} \\ \hline
		\{bread,jam,juice\} & $1$ \\ \hline
		\{bread,jam,chips\} & $5$ \\ \hline
		\{bread,milk,sugar\} & $2$ \\ \hline
		\{bread,cheese,milk\} & $3$ \\ \hline
		\{bread,cheese,butter\} & $3$ \\ \hline
		\{jam,juice,sugar\} & $1$ \\ \hline
		\{cheese,milk,butter\} & $3$ \\ \hline
		\{cheese,milk,sugar\} & $4$ \\ \hline
		$\ldots$ & $\ldots$ \\ \hline
		\end{tabular}
		\caption[]{ECLAT $3$-itemset tidlist}
	\end{table}
	This process repeats, with $k$ incremented by $1$ each time, until no frequent items or no candidate itemsets can be found. 

	The pros are:
	\begin{itemize}
		\item Depth-first search reduces memory requirements
		\item Usually (considerably) faster than Apriori
		\item No need to scan the database to find the support of $(k+1)$-itemsets, for $k>\geq 1$
	\end{itemize}
	and the cons:
	\begin{itemize}
		\item The TID-sets can be quite long, hence expensive to manipulate
	\end{itemize}

	
	\pagebreak
	\subsubsection{Clustering and Classification}
	In statistical data analysis, "clustering" describes empirical methods of data classification (hierarchical clustering method or data partitioning method). In other words clustering and classification are a data mining tasks for predicting the value of a categorical variable (target or class) by building a model based on one or more numerical and/or categorical variables (predictors or attributes).

	These techniques typically enable the segmentation of all customers of a company based on their demography or buying patterns, to group documents for presentations, identify new animal or plant species, to group information by individuals or by interests.
	
	\begin{table}[H]
		\begin{tabular}{|c|c|}
		\hline
		\rowcolor[HTML]{9B9B9B} 
		\textbf{Classification} & \textbf{Clustering} \\ \hline
		Use labelled as the input & Use unlabelled as the input \\ \hline
		The output is known & The output is unknown \\ \hline
		Uses supervised machine learning & Uses unsupervised machine learning \\ \hline
		\begin{tabular}[c]{@{}c@{}}A training data set is provided and used\\ to produce classifications\end{tabular} & \begin{tabular}[c]{@{}c@{}}A training data set is not provided and used\\ to produce clusters\end{tabular} \\ \hline
		\begin{tabular}[c]{@{}c@{}}Examples of algorithms: Decision-trees,\\ Bayesian classifiers and Support Vector\\ Machines (SVM)\end{tabular} & \begin{tabular}[c]{@{}c@{}}Examples of algorithms: Partition-based\\ clustering (k-means), Hierarchical clustering\\ (agglomerative \$ divisive) and DBSCAN\end{tabular} \\ \hline
		Can be more complex than clustering & Can be less complex than clustering \\ \hline
		Does not specify areas for improvement & Specify areas for improvement \\ \hline
		Known number of classes & Unknown number of classes \\ \hline
		Interpretability High & Interpretability Low \\ \hline
		\end{tabular}
		\caption{Typical Classification vs Clustering differences}
	\end{table}

	We see in practice two major families of clustering techniques (careful even experts in the field are unable to agree on a common classification ...!):
	\begin{enumerate}
		\item The "\NewTerm{non-hierarchical techniques}\index{non-hierarchical techniques}": that is to say where the number of classes (groups) final is chosen in advance.

		\item The "\NewTerm{hierarchical techniques}\index{hierarchical techniques}": that is to say where it leads to a classification by successive aggregations.
	\end{enumerate}
	Among these two families we distinguish three sub-families (this is the definition for THIS book - and that correspond almost to that of the SAS company - as they are no common accepted definition until now between specialists working in this field):
	\begin{enumerate}
		\item The techniques that make use of data whose nominal classification property is known in advance to train a prediction model: "\NewTerm{Machine Learning\footnote{Oxford Dictionary definition: The capacity of a computer to learn from experience, i.e. to modify its processing on the basis of newly acquired information.}\index{Machine Learning}}" or "\NewTerm{supervised learning}\index{supervised learning}". We find in this category the logistic regression techniques, CRT, ID3, discriminant analysis, Bayesian networks, decision lists, k-NN, neural networks, etc. 

		As there are not today a consensus in the definitions, it is important that the reader also knows that "supervised" algorithms by are algorithms that have a correction mechanism of the model parameters based on generated errors.
	
		\item The techniques that make use of data whose no nominal classification is known in advance to suggest a classification and seek a possible classification (which at the business level is often more interesting): "\NewTerm{data mining\footnote{Oxford Dictionary definition: The practice of examining large pre-existing databases in order to generate new information.}\index{data mining}}" or "\NewTerm{unsupervised learning (self guided algorithm)\index{unsupervised learning}}" or more rarely "\NewTerm{data mining}\index{data mining}". We find in this category the, HAC, $K$-means, kohonen maps, $K$-modes, $K$-medoids, etc.
		
		\item The techniques that make use of a mixture of classified and unclassified data and named "\NewTerm{semi-supervised learning}\index{semi-supervised learning}".
	\end{enumerate}
	These both families are quite well summarized by the following figure:
	\begin{figure}[H]
		\centering
		\includegraphics[scale=0.8]{img/computing/supervised_vs_unsupervised_learning.jpg}
		\caption{Supervised vs Unsupervised learning idea}
	\end{figure}
	are themselves divided into two sub-families: "\NewTerm{kernel-based}" (named also "\NewTerm{soft clustering}\index{soft clustering}" or "\NewTerm{fuzzy clustering}\index{fuzzy clustering}") or non-kernel based (named also "\NewTerm{hard clustering}\index{hard clustering}" )... (that means for the latter we don't need to make any assumptions on the statistical distribution shape of the variable of interest and that an individual can only belong to one category\footnote{Indeed, in fuzzy clustering technique like GMM (Gaussian Mixture Models) an individual has classification weights for each possible class!}).
	\begin{figure}[H]
		\centering
		\includegraphics[scale=0.48]{img/computing/data_mining_vs_machine_learning_following_sas.jpg}
		\caption[Data Mining as seen by SAS™]{Data Mining as seen by SAS™ (source: SlideShare SAS)}
	\end{figure}
	The distinction between Machine Learning and data mining has become more and more blurred, and there is a great deal of "cross-fertilization".
	
	The industrial or operational use of this knowledge in the professional world can solve very different problems, ranging from customer relationship management to preventive maintenance, through fraud detection and the optimization of websites, the supervision of financial markets, pro-active prospecting (customer consumption preferences), optimization of paths (analysis of road caps and left turns), or target selection (probability of acquiring a new given prospects) anticipate the identification of terrorists, pricing of products in comparison to equivalent one on the mark, etc.
	
	\begin{tcolorbox}[title=Remark,colframe=black,arc=10pt]
	Be careful to not to confuse strictly speaking we the concepts of classification, segmentation and association. Although the first two are often confused (classification / segmentation) because many algorithms do both at once, classification is the prediction of one or more discrete variables based on the value of other fields in the data set as the segmentation divides the data into groups of items having the most identical properties as possible. About association it is clear that many segmentation algorithms shows what is associated with what but the idea strictly speaking of association is to quantified by a scalar the degree of association between two data sets.
	\end{tcolorbox}	
	We will see here some trivial techniques that we will complete with time. We tried to order them in the ascending order of pedagogical complexity (but that concept is quite subjective obviously that's why many people would argue that $k$-nn should then presented first)...
	
	\paragraph{Naive Bayes classifiers}\mbox{}\\\\
	In Machine Learning, "\NewTerm{naive Bayes classifiers}\index{naive Bayes classifiers}" are a family of simple probabilistic classifiers based on applying Bayes' theorem with strong (naive) independence assumptions between the features (properties).

	Naive Bayes has been studied extensively since the 1950s. It was introduced under a different name into the text retrieval community in the early 1960s, and remains a popular (baseline) method for text categorization, the problem of judging documents as belonging to one category or the other (such as spam or legitimate, sports or politics, etc.) with word frequencies as the features. With appropriate pre-processing, it is competitive in this domain with more advanced methods including support vector machines. It also finds application in automatic medical diagnosis.
	
	\subparagraph{Binomial Naive Bayes classifier}\mbox{}\\\\
	The "\NewTerm{binomial naive Bayes classifier}\index{binomial naive Bayes classifier}" consists by using Bayes probabilities to classify an item in two possible outcomes.
	
	For example, the table below consisting of $4$ e-mail, having $2$ attributes and two possible classification outcomes that are $\{\mathrm{No-Spam},\mathrm{Spam}\}$:
	\begin{table}[H]
		\centering
		\begin{tabular}{|c|c|c|c|}
		\hline
		\rowcolor[HTML]{9B9B9B} 
		\multicolumn{1}{|l|}{\cellcolor[HTML]{9B9B9B}\textbf{Email}} & \multicolumn{1}{l|}{\cellcolor[HTML]{9B9B9B}\textbf{$100\%$ Guaranteed}} & \multicolumn{1}{l|}{\cellcolor[HTML]{9B9B9B}\textbf{Meeting}} & \multicolumn{1}{l|}{\cellcolor[HTML]{9B9B9B}\textbf{Classification}} \\ \hline
		$1$ & $\mathrm{No}$ & $\mathrm{Yes}$ & $\mathrm{No-Spam}$ \\ \hline
		$2$ & $\mathrm{No}$ & $\mathrm{Yes}$ & $\mathrm{No-Spam}$ \\ \hline
		$3$ & $\mathrm{Yes}$ & $\mathrm{Yes}$ & $\mathrm{Spam}$ \\ \hline
		$4$ & $\mathrm{Yes}$ & $\mathrm{No}$ & $\mathrm{No-Spam}$ \\ \hline
		\end{tabular}
		\caption{Naive Bayes spam-email companion example}
	\end{table}
	Abstractly, as already mentioned naive Bayes is a conditional probability model: given a problem instance to be classified, represented by a vector $\vec{x} =(x_{1},\dots ,x_{n})$ representing some $n$ features (independent variables), it assigns a posteriori with probabilities $P(C_k/x_1,\ldots,x_n)$ to the instance $C_k$, or also written $P(C_k/\vec{x})$, for each of $k$ possible outcomes or classes $C_k$.
	\begin{tcolorbox}[title=Remark,colframe=black,arc=10pt]
	Don't forget as we have seen in the section Probabilities (\SeeChapter{see section Probabilities page \pageref{chain rule}}) that $P(C_k/x_1,\ldots,x_n)$ is just a convenient notation to write $P(C_k/x_1\cap\ldots\cap x_n)$. And therefore $P(\vec{x})$ is also a convenient notation for $P(x_1\cap\ldots\cap x_n)$.
	\end{tcolorbox}
	With the example above it would look like this for the first e-mail (thus $n=2$):
	
	The problem with the above formulation is that if the number of features $n$ is large or if a feature can take on a large number of values, then basing such a model on probability tables is very hard. We therefore reformulate the model to make it more tractable. Using Bayes' theorem (\SeeChapter{see section Probabilities page \pageref{bayes formula}}), the conditional probability can be decomposed as:
	
	With for recall:
	\begin{itemize}
		\item $P(C_k/\vec{x})$ is the a posteriori probability to belong the class $C_k$ knowing the features $x_i$ (probability of the hypothetical class given the evidence ie. the features)
		\item $P(C_k)$ is the a priori probability (or "marginal probability") of the class $C_k$
		\item $P(\vec{x})$, the a priori probability of the feature (it is also a marginal probability)
		\item $P(\vec{x}/C_k)$, the a posteriori probability of the features $x_i$ knowing the class $C_k$, it is also the likelihood of $Y_k$ for $\vec{x}_i$ known (probability of evidence, ie. the feature, given that the hypothetical class holds).
	\end{itemize}
	In plain English, using Bayesian probability terminology, the above relation can be written for recall as:
	
	In practice, there is interest only in the numerator of that fraction (equivalent to the joint probability), because the denominator does not depend on $C_k$ and the values of the features $x_i$ are given, so that the denominator is effectively constant whatever the $k$.

	Hence the fact that we often can see in textbooks the following relation:	
	
	\begin{theorem}
	A key idea in the NB model is the following assumption:
	
	\end{theorem}
	\begin{dem}
	This equality is derived using iteratively (don't forget that we don't care about the numerator!) the relation proved in the section of Probabilities but without the denominator (\SeeChapter{see section Probabilities page \pageref{bayes formula}}):
	
	Indeed, let us rewrite this relation more explicitly:
	
	Next we rewrite the second product term using the assumption of conditional independence:
	
	Hence reinjecting in the prior-previous relation:
	
	In a condensed form the posterior probability of belonging to $C_k$ knowing $\vec{x}$ is then given by:
	
	\begin{flushright}
		$\blacksquare$  Q.E.D.
	\end{flushright}
	\end{dem}
	The idea now, when we have a training set of size $N$, is just to take for estimators:
	
	where obviously we define $C=C_k$ to be $1$ if $C=C_k$, $0$ otherwise. This estimator is simple the number of times that the label $C_k$ is seen in the training set!
	
	Similarly:
	
	This is a very natural estimate: we simply count the number of times label $C_k$ is seen in conjunction with $\vec{x}_i$ taking value $\vec{x}$ that we divide by the number of times the label $C_k$ is seen in total.
	
	\begin{tcolorbox}[colframe=black,colback=white,sharp corners]
	\textbf{{\Large \ding{45}}Example:}\\\\
	We consider the following training set:
	\begin{table}[H]
		\centering
		\begin{tabular}{|c|c|c|c|}
		\hline
		\rowcolor[HTML]{9B9B9B} 
		\multicolumn{1}{|l|}{\cellcolor[HTML]{9B9B9B}\textbf{Email}} & \multicolumn{1}{l|}{\cellcolor[HTML]{9B9B9B}\textbf{$100\%$ Guaranteed}} & \multicolumn{1}{l|}{\cellcolor[HTML]{9B9B9B}\textbf{Meeting}} & \multicolumn{1}{l|}{\cellcolor[HTML]{9B9B9B}\textbf{Classification}} \\ \hline
		$1$ & $\mathrm{No}$ & $\mathrm{Yes}$ & $\mathrm{No-Spam}$ \\ \hline
		$2$ & $\mathrm{No}$ & $\mathrm{Yes}$ & $\mathrm{No-Spam}$ \\ \hline
		$3$ & $\mathrm{Yes}$ & $\mathrm{Yes}$ & $\mathrm{Spam}$ \\ \hline
		$4$ & $\mathrm{Yes}$ & $\mathrm{No}$ & $\mathrm{No-Spam}$ \\ \hline
		\end{tabular}
	\end{table}
	A new e-mail arrives and contains the expression "$100\%$ \textit{Guaranteed}" and the expression "\textit{Meeting}". We want to predict if it is a Spam or not?
	\begin{table}[H]
		\centering
		\begin{tabular}{|c|c|c|c|}
		\hline
		\rowcolor[HTML]{9B9B9B} 
		\multicolumn{1}{|l|}{\cellcolor[HTML]{9B9B9B}\textbf{Email}} & \multicolumn{1}{l|}{\cellcolor[HTML]{9B9B9B}\textbf{$100\%$ Guaranteed}} & \multicolumn{1}{l|}{\cellcolor[HTML]{9B9B9B}\textbf{Meeting}} & \multicolumn{1}{l|}{\cellcolor[HTML]{9B9B9B}\textbf{Classification}} \\ \hline
		$5$ & $\mathrm{Yes}$ & $\mathrm{Yes}$ & ? \\ \hline
		\end{tabular}
	\end{table}
	For this purpose we will first simplify the notation by denoting the expression "$100\%$ \textit{Guaranteed}" by the letter $G$ and the expression "\textit{Meeting}" by the letter $M$.\\
	
	The posterior probabilities are in this simple case given by Bayes theorem:
	
	Hence the ratio, that eliminates (as we know) the denominator that is then useless, that must be greater that $1$ if the e-mail is a Spam:
	
	According to the assumption of the independence of conditional probabilities we can write for the first factor of the numerator and respectively for that of the denominator:
	
	So we can rewrite the ratio as:
	
	And as we have in our training set:
	\end{tcolorbox}
	\begin{tcolorbox}[colframe=black,colback=white,sharp corners]
	
	Injecting in the previous ratio, we get:
	
	That's all for this naive Bayesian example for document classification problem.
	\end{tcolorbox}
	Statisticians are somewhat disturbed by use of the NBC (which they dub Idiot's Bayes) because the naive assumption of independence is almost always invalid in the real world.
	
	However, the method has been shown to perform surprisingly well in a wide variety of contexts.
	
	\subparagraph{Gaussian Naive Bayes classifier}\mbox{}\\\\
	When dealing with continuous data, a typical assumption is that the continuous values associated with each class are distributed according to a Gaussian distribution. For example, suppose the training data contains a continuous attribute, $x$. We first segment the data by the class, and then compute the mean $\hat{\mu}$ and standard deviation $\hat{\sigma}$ of $x$ for each class.
	
	 Let $\hat{\mu} _{k}$ be the mean of the values in $x$ associated with class $C_k$, and let $\hat{\sigma}_{k}$ be the standard deviation of the values in $x$ associated with class $C_k$.
	 
	 Suppose we have collected some new observation value $x'$. Then, the probability density of $x'$ given a class $C_{k}$, $P(x=x'/C_{k})$ , can be computed by plugging $x'$ into the equation for a Normal distribution parametrized by $\hat{\mu}_{k}$ and $\hat{\sigma}_{k}^{2}$. That is:
	 
	 And after we apply again:
	 
	 
	 \begin{tcolorbox}[colframe=black,colback=white,sharp corners]
	\textbf{{\Large \ding{45}}Example:}\\\\
	We consider the following training set:
	\begin{table}[H]
		\centering
		\begin{tabular}{|c|c|c|c|}
		\hline
		\rowcolor[HTML]{9B9B9B} 
		\multicolumn{1}{|l|}{\cellcolor[HTML]{9B9B9B}\textbf{Person}} & \multicolumn{1}{l|}{\cellcolor[HTML]{9B9B9B}\textbf{Height [cm]}} & \multicolumn{1}{l|}{\cellcolor[HTML]{9B9B9B}\textbf{Weight [kg]}} & \multicolumn{1}{l|}{\cellcolor[HTML]{9B9B9B}\textbf{Foot size [cm]}} \\ \hline
		male & $182.88$ & $81.64$ & $30.48$ \\ \hline
		male & $180.44$ & $86.18$ & $27.94$ \\ \hline
		male & $170.07$ & $77.11$ & $30.48$ \\ \hline
		male & $180.44$ & $74.84$ & $25.4$ \\ \hline
		female & $152.4$ & $45.35$ & $15.24$ \\ \hline
		female & $167.64$ & $68.03$ & $20.32$ \\ \hline
		female & $165.20$ & $58.96$ & $17.78$ \\ \hline
		female & $175.26$ & $68.03$ & $22.86$ \\ \hline
		\end{tabular}
	\end{table}
	The classifier created from the training set using a Gaussian distribution assumption would be (given variances are unbiased sample variances):
	\begin{table}[H]
		\centering
		\begin{tabular}{|c|c|c|c|c|c|c|}
		\hline
		\rowcolor[HTML]{9B9B9B} 
		\multicolumn{1}{|l|}{\cellcolor[HTML]{9B9B9B}\textbf{Person}} & \multicolumn{1}{l|}{\cellcolor[HTML]{9B9B9B}\parbox{1cm}{\textbf{Mean \\Height}}} & \multicolumn{1}{l|}{\cellcolor[HTML]{9B9B9B}\parbox{1.2cm}{\textbf{Variance \\Height}}} & \multicolumn{1}{l|}{\cellcolor[HTML]{9B9B9B}\parbox{1cm}{\textbf{Mean\\ Weight}}} & \multicolumn{1}{l|}{\cellcolor[HTML]{9B9B9B}\parbox{1.2cm}{\textbf{Variance Weight}}} & \multicolumn{1}{l|}{\cellcolor[HTML]{9B9B9B}\parbox{1.3cm}{\textbf{Mean\\ Foot size}}} &\multicolumn{1}{l|}{\cellcolor[HTML]{9B9B9B}\parbox{1.3cm}{\textbf{Variance\\ Foot size}}} \\ \hline
		male & $178.45$ & $32.59$ & $79.94$ & $25.28$ & $28.58$ & $5.91$ \\ \hline
		female & $165.13$ & $90.32$ & $60.09$ & $114.88$ & $19.05$ & $10.75$ \\ \hline
		\end{tabular}
	\end{table}
	We also know from the training set that we have equiprobable classes so:
	
	Now consider a new sample to be classified as male or female:
	\begin{table}[H]
		\centering
		\begin{tabular}{|c|c|c|c|}
		\hline
		\rowcolor[HTML]{9B9B9B} 
		\multicolumn{1}{|l|}{\cellcolor[HTML]{9B9B9B}\textbf{Person}} & \multicolumn{1}{l|}{\cellcolor[HTML]{9B9B9B}\textbf{Height [cm]}} & \multicolumn{1}{l|}{\cellcolor[HTML]{9B9B9B}\textbf{Weight [kg]}} & \multicolumn{1}{l|}{\cellcolor[HTML]{9B9B9B}\textbf{Foot size [cm]}} \\ \hline
		? & $182.88$ & $58.96$ & $20.32$ \\ \hline
		\end{tabular}
	\end{table}
	We wish to determine which posterior is greater, male or female. For the classification as male the posterior is given according to our boxed relation above:
	
	For the classification as female the posterior is given by:
	
	\end{tcolorbox}
	
	\begin{tcolorbox}[colframe=black,colback=white,sharp corners]
	 We now determine the probabilities distribution for the sex of the sample:
	 \begin{itemize}
	 	\item $P(\mathrm{male})=0.5$
	 	\item $P(\mathrm{female})=0.5$
	 	\item $P(\mathrm{height}/ {\mathrm{male}})={\dfrac {1}{\sqrt {2\pi \hat{\sigma}_{H,M}^{2}}}}\exp \left({\frac {-(182.88-\hat{\mu}_{H,M} )^{2}}{2 \hat{\sigma}^{2}_{H,M}}}\right)\cong 0.0517$
	 	\item $P(\mathrm{height}/ {\mathrm{female}})={\dfrac {1}{\sqrt {2\pi \hat{\sigma}_{H,F}^{2}}}}\exp \left({\frac {-(182.88-\hat{\mu}_{H,F} )^{2}}{2 \hat{\sigma}^{2}_{H,F}}}\right)\cong 0.0073$
	 	\item $P(\mathrm{weight}/ {\mathrm{male}})={\dfrac {1}{\sqrt {2\pi \hat{\sigma}_{W,M}^{2}}}}\exp \left({\frac {-(58.96-\hat{\mu}_{W,M} )^{2}}{2 \hat{\sigma}^{2}_{W,M}}}\right)\cong 1.3142\cdot 10^{-05}$
		\item $P(\mathrm{weight}/ {\mathrm{female}})={\dfrac {1}{\sqrt {2\pi \hat{\sigma}_{W,M}^{2}}}}\exp \left({\frac {-(58.96-\hat{\mu}_{W,M} )^{2}}{2 \hat{\sigma}^{2}_{W,M}}}\right)\cong 0.0370$
		\item $P(\mathrm{foot\,size}/ {\mathrm{male}})={\dfrac {1}{\sqrt {2\pi \hat{\sigma}_{FS,M}^{2}}}}\exp \left({\frac {-(20.32-\hat{\mu}_{FS,M} )^{2}}{2 \hat{\sigma}^{2}_{FS,M}}}\right)\cong 0.0005$
		\item $P(\mathrm{foot\,size}/ {\mathrm{female}})={\dfrac {1}{\sqrt {2\pi \hat{\sigma}_{FS,F}^{2}}}}\exp \left({\frac {-(20.32-\hat{\mu}_{FS,F} )^{2}}{2 \hat{\sigma}^{2}_{FS,F}}}\right)\cong 0.1128$
	 \end{itemize}
	 Hence:
	 
	Since posterior numerator is greater in the female case, we predict the new individual is female.
	\end{tcolorbox}
	
	
	\paragraph{$k$-nearest neighbours classifier ($k$-nn)}\mbox{}\\\\
	The "\NewTerm{$k$-nn}\index{$k$-nearest neighbours}" is a quite simple supervised classification technique (do not confuse with the $K$-means that is a unsupervised clustering technique and that we will see later!). It takes a bunch of labelled points and uses them to learn how to label other points.  To label a new point, it looks at the labelled points closest to that new point (those are its nearest neighbours), and has those neighbours vote, so whichever label the most of the neighbours have is the label for the new point (the "$k$" is the number of neighbours it checks).
	
	In this framework, we have a learning database consisting of $N$ input-output pairs. To estimate the output associated with a new input $x$, the $k$ nearest neighbours method consists in taking into account (identically) the $k$ learning samples whose input is closest to the new input $x$, according to a distance to be defined.

For example, in a classification problem, we will retain the class most represented among the $k$ outputs associated with the $k$ inputs closest to the new input $x$.

	Given $D$ the dataset composed of $n$ pairs $(\vec{x}, y)$, with $\vec{x}$ the description of an observation according to $d$ quantitative variables, in the form of a real vector of dimensions $d$, and $y$ the class of this observation. Consider the case where the descriptive variables are all quantitative and where the class variable have $q$ modalities:
	
	The space of representation of the observations must be provided with a distance necessary for the construction of the neighbourhood of a new observation. The distance is a function $d:\mathbb{R}^d\times\mathbb{R}^d\mapsto\mathbb{R}^+$, respecting the three axioms of the distance that we know well (\SeeChapter{see section Topology page \pageref{distance}}).
	
	The most common distances for vectors of dimension $n$ (corresponding therefore to the number the input attributes that must all belong to $\mathbb{R}$) in a table of size $N$:
	\begin{itemize}
		\item Minkowski distance:
		
		
		\item Euclidean distance:
		
	
		\item Manhattan distance (city-block):
		
	
		\item Quadratic distance ($\sigma$ is as we know a correlation matrix):
		
		as we can see it corresponds to the square of the Mahalanobis distance.
	
		\item Correlation distance:
		
	
		\item Chi-$2$ distance:
		
	
		\item Kendall rank correlation distance:
		
		with for recall:
		
	
		\item Heterogeneous Euclidean Overlap Metric (HEOM) by focusing only on continuous attributes (case of the Tanagra software):
		
		where $\max_{i,a}$ and $\min_{i,a}$ are respectively the largest and smallest value of the attribute $i$ throughout the training sample so that the term in the parenthesis is always between $0$ and $1$ (normalized in other words...).
	\end{itemize}
	The figures below illustrate how the $k$ nearest neighbours solves a binary supervised classification task ($y_i\in \{0,1\}$) in $\mathbb{R}^2$, depending on the Euclidean distance and for $k = 3$. The first figure show (in three steps, from left to right) a new observation is attributed to class $1$, its neighbour being composed of three observations belonging to class $1$. The second figure shows how a new observation is attributed to class $1$, its vicinity being composed of two observations belonging to the class $1$ and of one observation belonging to the class $0$.
	\begin{figure}[H]
		\centering
		\includegraphics[scale=0.7]{img/computing/knn_01.jpg}
		\caption[Classification of a new observation with $k-nn$]{Classification of a new observation with $k-nn$: the three nearest neighbours belong to class $1$}
	\end{figure}
	\begin{figure}[H]
		\centering
		\includegraphics[scale=0.7]{img/computing/knn_02.jpg}
		\caption[]{Classification of a new observation with $k-nn$: $2$ of the $3$ nearest neighbours belong to class $1$}
	\end{figure}
	The distance to the $k$th nearest neighbour can also be seen as a local density estimate and thus is also a popular outlier score in anomaly detection. The larger the distance to the $k$-nn, the lower the local density, the more likely the query point is an outlier. Although quite simple, this outlier model, along with another classic data mining method, local outlier factor, works quite well also in comparison to more recent and more complex approaches, according to a large scale experimental analysis.
	
	\begin{tcolorbox}[title=Remark,colframe=black,arc=10pt]
	Obviously we may run the $k$-nn on the factorial space obtained after having run a PCA (Principal Component Analysis as seen on page \pageref{principal component analysis}) on our data. We can also even reduce the dimensions using that latter if necessary!
	\end{tcolorbox}
	
	\paragraph{Regression and classification trees}\mbox{}\\\\
	"\NewTerm{Classification and Regression Tree}\index{classification and regression tree}\index{CART}" (CART) are a set of heuristic algorithms  introduced in 1984 by Leo Breiman, Jerome Friedman, Richard Olshen and Charles Stone as an umbrella term to refer to the following types of \NewTerm{decision trees}\index{decision trees}\footnote{A "\NewTerm{decision stump}\index{decision stump}" is a Machine Learning model consisting of a one-level decision tree}": 
	\begin{itemize}
		\item "\NewTerm{Classification Trees}\index{classification trees}": where the target variable is categorical and the tree is used to identify the "class" within which a target variable would likely fall into.

		\item "\NewTerm{Regression Trees}\index{regression trees}": where the target variable is continuous and tree is used to predict it's value.
	\end{itemize}
	Decision tree are non-linear, nonparametric and nonmetric classifier. Concretely, decision trees are predictive models that proceed through a tree structure of rules or a hierarchy of tests extracted from learning or training data, to partition the basic data into homogeneous subgroups from the point of view of the variable to predict. The generated rules tree easily lends itself to the human reasoning mode and it becomes more intuitive to interpret the tree and make predictions.
	
	These trees are widely used in advanced marketing or social science to discriminate (categorize) a very large population. Obviously, these algorithms (that are part of hierarchical techniques) will never do better than a human being (at least in this first decade of the 21st century)... but also ask an employee to create groups in a population of $5$ million customers on the basis of ten explanatory variables. You will have to wait for the response a quite long time...
	
	While these automated  classification techniques are very useful in the above situations, they nevertheless have a major problem that makes we will not focus too much on this subject:
	\begin{itemize}
		\item These techniques are very sensitive to the analysed population and give very different results.

		\item The various existing  techniques of classifications give results that are completely different for the same population.
	\end{itemize}
	It is better to be cautious about the conclusions that we can draw from these models and compare the results of several methods depending on the return on experience choose the one that seems the best suited.
	
	\begin{tcolorbox}[title=Remark,colframe=black,arc=10pt]
	In good practice situations, the construction of a tree is, as always in the field of Machine Learning (...), made from a set of data named the "learning sample". Once the tree is built, it is tested for its predictive power usually on a set of data named the "test sample".
	\end{tcolorbox}

	\subparagraph{Binary classification trees with quantitative variables}\mbox{}\\\\
	For companion example, let us consider a set of categorical variables $x_1,\ldots,x_p$. The recursive partitioning has for purpose to divide the $p$ variables of the space into rectangles which do not overlap.
	
	For example, consider the variable $x_i$ and a value $s_i$ of this variable, we find that the partitioning $x_i<s_i$ and $s_i<x_i$ separates well the data into two disjoint sets. Then one of the parts is in turn divided by a value $x_i$ or by the value of another variable. We end then with three rectangles and so on...
	
	The idea is to create $n$ rectangles such that all data contained in a rectangle are homogeneous (that is to say contains only one family of points).

	To address this issue, consider the following first practical case with only quantitative variables (it is therefore a binary regression tree):
	
	A dealer would like for his city to find a way to classify the families that are able to buy a car (owners) and those that are not ready to buy a car (non-owners). A sample of $12$ owners ("$1$" in the figure below) and $12$ non-owners ("$2$" in the figure below) is selected. The two independent variables are $x_1$ (wage in kilo-dollars) and $x_2$ (area of their home). In other words, we have here a case where $x_i\in\mathbb{R}^+$ and $y_i\{1,2\}$.
	
	\begin{figure}[H]
		\centering
		\includegraphics[scale=0.9]{img/computing/cart_data_list_sample_excel.jpg}
		\caption[]{Original data list of CART application with Microsoft Excel 14.0.7172}
	\end{figure}
	We see that we have as many owners than non-owners (assumed equal appearance frequency in the whole population). Therefore the probability of belonging to a class is $50\%$.

	Or graphically:
	\begin{figure}[H]
		\centering
		\includegraphics[scale=0.8]{img/computing/cart_data_list_sample_excel_chart_0_iteration.jpg}
		\caption[]{Original  plot CART application with Microsoft Excel 14.0.7172}
	\end{figure}
	If we apply the CART algorithm on this data, we see that we must choose $x_2$ (Area) as the first choice of division with the division value of $19$ (we will justify why!). The space $(x_1,x_2)$ is now divided into two rectangles (it was easy to guess that discriminating step without even using mathematics):
	\begin{figure}[H]
		\centering
		\includegraphics[scale=0.8]{img/computing/cart_data_list_sample_excel_chart_1_iteration.jpg}
		\caption[]{First iteration of CART application with Microsoft Excel 14.0.7172}
	\end{figure}
	Notice how the division into two rectangles created two zones (splits) which are more homogeneous than the original graph! The upper rectangle contains points which are more Owners while the lower rectangle contains more Non-owners.

	To determine this division, the CART algorithm examines each variable and all possible values for each variable division in order to find the best division.

	Thus, the possible points of division for $x_1$ are (notice that this is every time the average of the two related values in the table):
	
	and that for $x_2$ are:
	
	These points are ordered by the algorithm according to the way they reduce "impurity" (heterogeneous of composition) in the rectangle that generates the "split".

	There are a large number of empirical ways to measure the impurity. But to start with our first example, let us use the easiest and most common "\NewTerm{impurity measurement criterion}\index{impurity measurement criterion}" or "\NewTerm{segmentation criteria}\index{segmentation criteria}" so far that is the use of an indicator inspired by the Gini coefficient (\SeeChapter{see section Quantitative Management Techniques page \pageref{gini index}}). Thus, if we denote the classes by $k=1,2,3,\ldots,C$ where $C$ is the total number of classes to be predicted, the "\NewTerm{Gini impurity index}\index{Gini impurity index}" for the $A$ rectangle is defined by:
	
	where $p_k$ is the fraction of observations in the rectangle $A$ which belong to the class $k$. 

	In our example, we always have only two classes: Owners / Non- Owners.

	Next, the global Gini index is defined as the weighted average of the Gini indices.

	So in our example, we have two classes, therefore $C=2$. Before the separation, we have:
	
	The separation found in $19$ (see the second figure) gives for example for the top rectangle of the first subdivision:
	
	For the inferior part:
	
	By the hazard of the choice of this example, the impurity is the same for both rectangles (top and bottom).

	The overall Gini index is then given by:
	
	Notice before continuing that if the subdivision is perfect (only have one family of points in one of the boxes), then we have:
	
	So the impurity is zero ... And if all the points appear in equal proportions in each of the rectangles (worst situation we could say), the value is then:
	
	If we generalize to $C$ classes ($C$ spatial dimensions), it comes immediately:
	
	which is the maximum impurity.

	So the impurity is always defined by a value in the range:
	
	Now, to continue with our example, even without using a computer algorithm, without even calculating the impurity, it is relatively easy to guess which will be the next discrimination: it will be $x_1=84.75$ (Income). Which will give:
	\begin{figure}[H]
		\centering
		\includegraphics[scale=0.8]{img/computing/cart_data_list_sample_excel_chart_2_iteration.jpg}
		\caption[]{Second iteration of CART application with Microsoft Excel 14.0.7172}
	\end{figure}
	What was also easy to guess even without using the calculations (try your friends, you will see that very often they can found the first two discrimination).

	The impurity will be calculated in the new discriminated area by:
	
	The overall Gini index is given by:
	
	We continue, but be aware that the result is less easy to guess. 

	The majority of individuals interviewed are wrong without using the mathematical definition of impurity and intuitively wrong to propose one or more of the following discrimination:
	\begin{figure}[H]
		\centering
		\includegraphics[scale=0.8]{img/computing/cart_data_list_sample_excel_chart_3_iteration.jpg}
		\caption[]{Third iteration of CART application with Microsoft Excel 14.0.7172}
	\end{figure}
	with (you can do the math) a total impurity of $0.2727$. When in reality, the optimum discrimination is:
	\begin{figure}[H]
		\centering
		\includegraphics[scale=0.8]{img/computing/cart_data_list_sample_excel_chart_3_iteration_real.jpg}
		\caption[]{Third real iteration of CART application with Microsoft Excel 14.0.7172}
	\end{figure}
	with a total impurity of $0.2592$. Indeed:
	
	The overall Gini index is:
	
	In the next step, we have:
	\begin{figure}[H]
		\centering
		\includegraphics[scale=0.8]{img/computing/cart_data_list_sample_excel_chart_4_iteration.jpg}
		\caption[]{Fourth iteration of CART application with Microsoft Excel 14.0.7172}
	\end{figure}
	At the next step:
	\begin{figure}[H]
		\centering
		\includegraphics[scale=0.8]{img/computing/cart_data_list_sample_excel_chart_4_iteration.jpg}
		\caption[]{Fifth iteration of CART application with Microsoft Excel 14.0.7172}
	\end{figure}
	etc. until the end:
	\begin{figure}[H]
		\centering
		\includegraphics[scale=0.8]{img/computing/cart_data_list_sample_excel_chart_final_iteration.jpg}
		\caption[]{Final iteration of CART application with Microsoft Excel 14.0.7172}
	\end{figure}
	where each rectangle is pure (only contains data that one of the two classes).

	The reason why the method is named "Classification And Regression Tree" algorithm is that each division can be represented as the division of a node into two successor nodes. The first division is shown as a branch of the tree root node. Here, for example are the first six iterations of the algorithm (only the first six one as the page size is to small to get them all):
	\begin{figure}[H]
		\centering
		\includegraphics[scale=0.8]{img/computing/cart.jpg}
		\caption{CART Final result in the traditional tree form}
	\end{figure}
	If you follow the detailed steps given in our \texttt{R} companion book you will see the corresponding result that is:
	\begin{figure}[H]
		\centering
		\includegraphics[scale=0.9]{img/computing/cart_r.jpg}
		\caption[]{CART Final result in the traditional tree form with \texttt{R} 3.0.2}
	\end{figure}
	And if you follow the detailed steps given in our MATLAB™ companion book you will see the corresponding result that is:
	\begin{figure}[H]
		\centering
		\includegraphics[scale=0.85]{img/computing/cart_matlab.jpg}
		\caption[]{CART Final result in the traditional tree form with MATLAB™ 2013a}
	\end{figure}
	We need to be careful to pick an appropriate tree depth! Indeed:
	\begin{itemize}
		\item If the tree is too deep, we may overfit
		\item If the tree is too shallow, we may underfit
	\end{itemize}
	\begin{figure}[H]
		\centering
		\includegraphics[width=0.8\textwidth]{img/computing/decision_tree_vocabulary.jpg}
		\caption{Main vocabulary of decision trees}
	\end{figure}
	Alternative strategy is to create a very deep tree, and then to prune it.
	
	\begin{tcolorbox}[title=Remark,colframe=black,arc=10pt]
	A common question for binary decision trees is: why not use logistic regression instead? The answer is quite simple! Decision trees helps you developing decision rules but in logistic regression you can not visualize a rule even though you can know which factor are more important We will also use binary decision trees when there are too many variables, especially categorical ones and you have outliers in predictor variables and relationships are non-linear. And if we want probabilities for each case, we should go for logistic regression obviously!
	\end{tcolorbox}	
	
	\subparagraph{Detection of significant splits}\mbox{}\\\\
	Quite often it is necessary to measure the significance of a split in a decision tree, especially when the information gain (see further below page \pageref{information gain}) is small.

	Let $N_{A}$ and $N_{B}$ be the number of items of class $A$ and class $B$ in the parent node. Let $N_{A L}$ represent the number of class $A$ going to the left child node, $N_{B L}$ represent the number of class B going to the left child node, $N_{A R}$ represent the number of class $B$ going to the right child node, and $N_{B R}$ represent the number of class $B$ going to the right child node.

	Let $p_{L}$ and $p_{B}$ denote the proportion of data going to the left and right node, respectively:
	
	The following measure computes the significance of a split. In other words, it measures how much the split deviates from what would be expected in the random data:
	
	where:
	
	We recognise here the shape of a $\chi^2$ random variable with $4$ degrees of freedom ($\chi^2_4$).
	
	If $K$ is small, the information gain from the split is not significant. If $K$ is big, it would suggest the information gain from the split is significant.
	
	\subparagraph{Clustering validation}\mbox{}\\\\
	There exist many different clustering methods, depending on the type of clusters sought and on the inherent data characteristics. Given the diversity of clustering algorithms and their parameters it is important to develop objective approaches to assess clustering results. Cluster validation and assessment encompasses three main tasks: "\NewTerm{clustering evaluation}" seeks to assess the goodness or quality of the clustering, "\NewTerm{clustering stability}" seeks to understand the sensitivity of the clustering result to various algorithmic parameters, for example, the number of clusters, and "\NewTerm{clustering tendency}" assesses the suitability of applying clustering in the first place, that is, whether the data has any inherent grouping structure. There are a number of validity measures and statistics that have been proposed for each of the aforementioned tasks, which can be divided into three main types:

	\begin{itemize}
		\item \textbf{External validation} measures employ criteria that are not inherent to the dataset. This can be in form of prior or expert-specified knowledge about the clusters, for example, class labels for each point.
	
		\item \textbf{Internal validation} measures employ criteria that are derived from the data itself. For instance, we can use intracluster and intercluster distances to obtain measures of cluster compactness (e.g., how similar are the points in the same cluster?) and separation (e.g., how far apart are the points in different clusters?).
	
		\item \textbf{Relative validation} measures aim to directly compare different clusterings, usually those obtained via different parameter settings for the same algorithm.
	\end{itemize}
	
	\begin{tcolorbox}[title=Remark,colframe=black,arc=10pt]
	For the reader interested to further reading on the topic of clustering validation, we strongly recommend: \textit{DATA MINING AND ANALYSIS: Fundamental Concepts and Algorithms} (see \cite{zaki2014data}), that has excellent detailed definitions with step by step manually calculated examples from page 425 to page 462 (all chapter 17).
	\end{tcolorbox}

	To see now some other segmentation external criteria but now with qualitative variables let us consider the following first companion example based on the table below:
\begin{table}[H]
	\centering
	\begin{tabular}{|c|c|c|c|}
		\hline
		\textbf{Color} & \textbf{Shape} & \textbf{Size} & \textbf{Class} \\ \hline
		red (R) & square (S) & big (B) & plus  \\ \hline
		blue (B) & square (S) & big (B) & plus  \\ \hline
		red (R) & round (R) & small (S) & minus \\ \hline
		green (G) & square (S) & small (S) & minus \\ \hline
		red (R) & round (R) & big (B) & plus  \\ \hline
		green (G) & square (S) & big (B) & minus \\ \hline
	\end{tabular}
\end{table}

	First let us do again the calculation with the Gini impurity index:
	
	where $p(i/t)$ still represents the frequency or the probability of the $i$ relatively to the partition $t$ (again, a partition is "pure" if it's Gini impurity index is equal to zero).

	As in the previous example we must afterwards calculate the overall Gini impurity index as the weighted average of the Gini impurity index of each partition following:
	
	Therefore the overall Gini index for the \textit{Color} attribute is given by (keep in mind that our dataset has $6$ rows and the numerator of each fraction represent the number of occurrences of the concerned partition!):
	
	with (keep in mind that the denominator is always the number of rows of the corresponding partition, and the numerator correspond to how many of a given \textit{Class} we have among all these rows):
	\begin{itemize}
		\item $I(B)= 1 - (0/1)^2 - (1/1)^2 = 0.0$
	    \item $I(R)= 1 - (1/3)^2 - (2/3)^2 = 0.44$
	    \item $I(G)= 1 - (2/2)^2 - (0/2)^2 = 0.0$
	\end{itemize}
	And the overall Gini index for the \textit{Shape} attribute is then given by:
	
	with:
	\begin{itemize}
		\item $I(S)= 1 - (2/4)^2 - (2/4)^2 = 0.50$
		\item $I(C)= 1 - (1/2)^2 - (1/2)^2 = 0.50$
	\end{itemize}
	And finally the overall Gini index for the \textit{Size} attribute:
	
	with:
	\begin{itemize}
		\item $I(B)= 1 - (1/4)^2 - (3/4)^2 = 0.375$
		\item $I(S)= 1 - (2/2)^2 - (0/2)^2 = 0.0$
	\end{itemize}
	As the variables with lower Gini impurity index are the one to split, then we will chose the \textit{Color} attribute as first split in our example! So we see that in classification trees, the Gini Index can be used as "\NewTerm{variable importance}\index{variable importance}" metric for classification trees!
	
	Now let us see a new segmentation criteria named the "\NewTerm{information gain}\index{information gain}\label{information gain}", used by the ID3\index{ID3}, C4.5 and C5.0 tree-generation algorithms, and based on the measurement of information's entropy from Statistical Mechanics (see page \pageref{information entropy}). It is therefore defined for the partition $t$ by the relation (named for recall "Shannon formula"):
	
	Thus, similarly to the Gini impurity index, the conditional entropy for a given attribute is given by the weighted information entropies:
	
	Candidate splits are determined by looking at each variable that makes up an object belonging in a given class. In the example above all objects can either be \textit{plus} or \textit{minus}.
	
	Now for all possible split, we determine fist the entropy before the split which is found using the classification of each object (keep in mind that our dataset has $6$ rows and the numerator of each fraction represent the number of occurrences of the concerned partition!):
	
	Now the conditional entropy for the \textit{Class} given the \textit{Color} attribute is given by : 
	
	with (keep in mind that the denominator is always the number of rows of the corresponding partition, and the numerator correspond to how many of a given \textit{Class} we have among all these rows):
	\begin{itemize}
		\item $S(B) = - (1/1)\cdot\log(1/1)$
		\item $S(R) = - (1/3)\cdot\log(1/3) - (2/3)\cdot\log(2/3)$
		\item $S(G) = - (2/2)\cdot\log(2/2)$
	\end{itemize}
	Information gain is then be defined by finding the difference in the prior entropy and the conditional entropy for the class $C$ given a specific attribute variable $a$:
	
	Therefore in our example here the information gain for colors is:
	
	And now the conditional entropy for the \textit{Class} given the \textit{Shape} attribute is then given by (we change to notation for the entropy from $H$ to $S$ in the second term to avoid having to write an ugly $S(S)$...):
	
	with:
	\begin{itemize}
		\item $S(R) = - (2/4)\cdot\log(2/4) - (2/4)\cdot\log(2/4)$
		\item $S(S)= - (1/2)\cdot\log(1/2) - (1/2)\cdot\log(1/2)$
	\end{itemize}
	Therefore the information gain for shapes is:
	
	And finally the conditional entropy for the \textit{Class} given the \textit{Size} attribute is then given by:
	
	with:
	\begin{itemize}
		\item $S(B) = - (1/4)\cdot\log(1/4) - (3/4)\cdot\log(3/4)$
		\item $S(S) = - (2/2)\cdot\log(2/2)$
	\end{itemize}
	Therefore the information gain for sizes is:
	
	As the variables with lower conditional entropy (respectively highest information gain) are the one to split, then we will chose the \textit{Color} attribute as first split in our example! So here the split choice is the same as with the Gini impurity index. So we see that in classification trees, the overall information entropy can be used as "\NewTerm{variable importance}\index{variable importance}" metric for classification trees!
	
	The problem with information gain as a measure to select the attribute for partition is that in the quest of pure partition, it may select attributes that are meaningless from the Machine Learning point of view.  This drawback is due to the inherent deficiency in the measure information gain that gives preference to the attribute that can divide the parent dataset to datasets with the least amount of entropy.
	
	\begin{tcolorbox}[title=Remark,colframe=black,arc=10pt]
	Notice that there are a lot of other segmentation criterias as the likelihood-ratio chi-squared statistic, the DKM criterion, the Twoing criterion, the Orthogonal criterion (ORT), the Kolmogorov-Smirnoff criterion, the AUC-Splitting criterion, ...
	\end{tcolorbox}

	And as mentioned above, we will stop here about our study of CART because the study of the corresponding empirical variations techniques are a full time job as they are numerous (and almost none gives the same results...).
	
	\subparagraph{Random Forests}\mbox{}\\\\
	The random forest (see figure below) takes the notion of binary decision tree to the next level by combining trees with the notion of an ensemble. Thus, in ensemble terms, the standard decision trees are weak learners and the random forest is a strong learner (in the sense of a statistical crows sourcing).
	
	In other words "\NewTerm{random forests}\index{random forests}" or "\NewTerm{random decision forests}\index{random decision forests}" are an ensemble learning method for classification, regression and other tasks, that operate by constructing a multitude of decision trees at training time and outputting the class that is the mode of the classes (classification) or mean prediction (regression) of the individual trees. Random decision forests correct for decision trees' habit of overfitting to their training set.
	\begin{figure}[H]
		\centering
		\includegraphics[width=1.0\textwidth]{img/computing/random_forest.jpg}
		\caption[Random forest illustration]{Random Forster illustration (author: ?)}
	\end{figure}
	Here is how such a system is trained; for some number of trees:
	\begin{enumerate}
		\item Sample $N$ cases at random with replacement to create a subset of the data (see top layer of figure above). The subset should be about $66\%$ of the total set.
	
		\item At each node:
		\begin{enumerate}
			\item For some number $m$, $m$ predictor variables are selected at random from all the predictor variables.
	
			\item The predictor variable that provides the best split, according to some objective function, is used to do a binary split on that node.
	
			\item At the next node, choose another $m$ variables at random from all predictor variables and do the same.
		\end{enumerate}
	\end{enumerate}
	For prediction with a new entered input into the system, it is run down all of the trees. The result may either be an average or weighted average of all of the terminal nodes that are reached, or, in the case of categorical variables, a voting majority.
	
	Note that:
	\begin{itemize}
		\item With a large number of predictors, the eligible predictor set will be quite different from node to node.
		\item The greater the inter-tree correlation, the greater the random forest error rate, so one pressure on the model is to have the trees as uncorrelated as possible.
		\item As $m$ goes down, both inter-tree correlation and the strength of individual trees go down. So some optimal value of $m$ must be discovered.
	\end{itemize}
	Random forest runtimes are quite fast, and they are able to deal with unbalanced and missing data. Random Forest weaknesses are that when used for regression they cannot predict beyond the range in the training data, and that they may over-fit data sets that are particularly noisy. Of course, the best test of any algorithm is how well it works upon your own data set.
	
	To understand why random forest works quite good you must first remember the bias-variance tradeoff relation (see earlier above page \pageref{bias-variance tradeoff}) that is given by:
	
	So first, if the trees are sufficiently deep, they have very small bias!
	
	Now let us recall the variance of the mean of identically correlated i.i.d variables  (\SeeChapter{see section Statistics \pageref{variance mean correlated variables}}):
	
	So to improve the variance over that of a single tree, we see that de-correlation gives better accuracy as a small $\rho$ decrease the first term, but we also see that the second term decrease if the number of trees $B$ increase (independently of $\rho$)! So the idea in random forests is to improve the variance reduction of bagging by reducing the correlation between the trees, without increasing the variance too much. This is achieved in the tree-growing process through random selection of the input variables.

	\paragraph{$K$-Means clustering}\mbox{}\\\\
	The "\NewTerm{$K$-means}\index{$K$-means}" algorithm or "\NewTerm{mobile centers}", also sometimes named "\NewTerm{classification method with dynamic clouds}" is in statistics and Machine Learning (specifically unsupervised learning), a data partitioning algorithm, ie a method that aims to divide observations in $K$ clusters in which each observation belongs to the partition with the nearest average (mean).

	The basic steps of the algorithm are as follows:
	\begin{enumerate}
		\item We choose a partitioning into $K$ groups

		\item We generate $K$ averages (centers $c_i$) randomly

		\item The data are assigned to the group whose center is closest to them

		\item We calculated the average of each group using the affected data (new centers)

		\item We return to step 3
	\end{enumerate}	
	\begin{figure}[H]
		\centering
		\includegraphics[width=1.0\textwidth]{img/computing/k_means_iterative_algorithm.jpg}
		\caption[$K$-means iterative algorithm]{$K$-means iterative algorithm (authors: Afshine Amidi, Shervine Amidi)}
	\end{figure}
	To make more explicit the algorithm each step will be illustrated by means of the following small dataset:
	\begin{table}[H]
		\centering
		\begin{tabular}{|c|c|c|}
		\hline
		\rowcolor[HTML]{9B9B9B} 
		\multicolumn{1}{|l|}{\cellcolor[HTML]{9B9B9B}\textbf{Object}} & \multicolumn{1}{l|}{\cellcolor[HTML]{9B9B9B}\textbf{Weight}} & \multicolumn{1}{l|}{\cellcolor[HTML]{9B9B9B}\textbf{Length}} \\ \hline
		$\mathrm{a}$ & $12$ & $10$ \\ \hline
		$\mathrm{b}$ & $15$ & $25$ \\ \hline
		$\mathrm{c}$ & $30$ & $55$ \\ \hline
		$\mathrm{d}$ & $50$ & $100$ \\ \hline
		$\mathrm{e}$ & $35$ & $70$ \\ \hline
		$\mathrm{f}$ & $45$ & $70$ \\ \hline
		$\mathrm{g}$ & $35$ & $60$ \\ \hline
		\end{tabular}
	\end{table}
	we determine $k$ temporary centers of $k$ clusters generally by randomly dragging $k$ individuals from the database. These $k$ individuals form the center of the $k$ clusters. To achieve the first clustering, we assign each individual to the cluster of which it is closest. Here the notion of proximity is understood in a measure of the Euclidean distance:
	
	For $k = 2$, we will randomly select $k$ individuals, here the object $a$ and the object $d$ for example:
	\begin{table}[H]
		\centering
		\begin{tabular}{|c|c|c|}
		\hline
		\rowcolor[HTML]{9B9B9B} 
		\textbf{Clusters} & \textbf{Objects} & \textbf{Centers} \\ \hline
		Cluster $c_1$ & $c_1\{\mathrm{a}\}$ & $(12,10)$ \\ \hline
		Cluster $c_2$ & $c_2\{\mathrm{d}\}$ & $(50,100)$ \\ \hline
		\end{tabular}
	\end{table}
	Now we must calculate the Euclidean distance of each individual with respect to the centers and assign each to the cluster that is closer to it:
	\begin{table}[H]
		\centering
		\begin{tabular}{|l|c|c|c|}
		\hline
		\rowcolor[HTML]{9B9B9B} 
		\multicolumn{1}{|c|}{\cellcolor[HTML]{9B9B9B}\textbf{Objects}} & \textbf{Distance Cluster 1} & \textbf{Distance Cluster 2} & \textbf{Affectation To} \\ \hline
		$\mathrm{a}$ & $0.00$ & $97.69$ & Cluster $c_1$ \\ \hline
		$\mathrm{b}$ & $15.30$ & $82.76$ & Cluster $c_1$ \\ \hline
		$\mathrm{c}$ & $48.47$ & $49.24$ & Cluster $c_1$ \\ \hline
		$\mathrm{d}$ & $97.69$ & $0.00$ & Cluster $c_2$ \\ \hline
		$\mathrm{e}$ & $64.26$ & $33.54$ & Cluster $c_2$ \\ \hline
		$\mathrm{f}$ & $68.48$ & $30.41$ & Cluster $c_2$ \\ \hline
		$\mathrm{g}$ & $55.04$ & $42.72$ & Cluster $c_2$ \\ \hline
		\end{tabular}
	\end{table}
	Then, we again determine the $k$ centers of the $k$ last partitions performed. The distances of each individual with respect to these centers are recalculated and each is reallocated according to the cluster that is close to it.
	
	Following the last clustering we obtain the center (point of gravity or barycenter) of each partitions like this:
	\begin{table}[H]
		\centering
		\begin{tabular}{|c|c|c|}
		\hline
		\rowcolor[HTML]{9B9B9B} 
		\textbf{Clusters} & \textbf{Objects} & \textbf{Centers} \\ \hline
		Cluster $c_1$ & $c_1\{\mathrm{a},\mathrm{b},\mathrm{c}\}$ & $(19,30)$ \\ \hline
		Cluster $c_2$ & $c_2\{\mathrm{d},\mathrm{e},\mathrm{f},\mathrm{g}\}$ & $(41.25,75)$ \\ \hline
		\end{tabular}
	\end{table}
	Again, it is necessary to calculate the Euclidean distance of each individual with respect to the centers and reassign each to the cluster that is closer to it:
	\begin{table}[H]
		\centering
		\begin{tabular}{|l|c|c|c|}
		\hline
		\rowcolor[HTML]{9B9B9B} 
		\multicolumn{1}{|c|}{\cellcolor[HTML]{9B9B9B}\textbf{Objects}} & \textbf{Distance Cluster 1} & \textbf{Distance Cluster 2} & \textbf{Affectation To} \\ \hline
		$\mathrm{a}$ & $21.19$ & $71.28$ & Cluster $c_1$ \\ \hline
		$\mathrm{b}$ & $6.40$ & $56.47$ & Cluster $c_1$ \\ \hline
		$\mathrm{c}$ & $27.31$ & $22.95$ & Cluster $c_2$ \\ \hline
		$\mathrm{d}$ & $76.56$ & $26.49$ & Cluster $c_2$ \\ \hline
		$\mathrm{e}$ & $43.08$ & $8.00$ & Cluster $c_2$ \\ \hline
		$\mathrm{f}$ & $47.71$ & $6.25$ & Cluster $c_2$ \\ \hline
		$\mathrm{g}$ & $34.00$ & $16.25$ & Cluster $c_2$ \\ \hline
		\end{tabular}
	\end{table}
	We see at this level that the object $\mathrm{c}$ change the group (centroid) he belongs to before.
	
	This last step is repeated as many times as necessary until the process is stabilized. The algorithm can be stopped either:
	\begin{itemize}
		\item Because we do not observe changes anymore in the composition of each cluster after two successive iterations
		
		\item Because an empirical control criterion is verified (number of iteration fixed for example)
	\end{itemize}

	In the previous illustration, it is rather the first reason that will decide the end of the clustering. Indeed, at the 3rd iteration there is no change in the composition of the clusters.
	
	New centers following last assignment:
	\begin{table}[H]
		\centering
		\begin{tabular}{|c|c|c|}
		\hline
		\rowcolor[HTML]{9B9B9B} 
		\textbf{Clusters} & \textbf{Objects} & \textbf{Centers} \\ \hline
		Cluster $c_1$ & $c_1\{\mathrm{a},\mathrm{b}\}$ & $(13.5,17.5)$ \\ \hline
		Cluster $c_2$ & $c_2\{\mathrm{d},\mathrm{e},\mathrm{f},\mathrm{g}\}$ & $(39,71)$ \\ \hline
		\end{tabular}
	\end{table}
	Distance calculation and reallocation:
	\begin{table}[H]
		\centering
		\begin{tabular}{|l|c|c|c|}
		\hline
		\rowcolor[HTML]{9B9B9B} 
		\multicolumn{1}{|c|}{\cellcolor[HTML]{9B9B9B}\textbf{Objects}} & \textbf{Distance Cluster 1} & \textbf{Distance Cluster 2} & \textbf{Affectation To} \\ \hline
		$\mathrm{a}$ & $7.65$ & $66.71$ & Cluster $c_1$ \\ \hline
		$\mathrm{b}$ & $7.65$ & $51.88$ & Cluster $c_1$ \\ \hline
		$\mathrm{c}$ & $40.97$ & $18.36$ & Cluster $c_2$ \\ \hline
		$\mathrm{d}$ & $90.21$ & $31.02$ & Cluster $c_2$ \\ \hline
		$\mathrm{e}$ & $56.73$ & $4.12$ & Cluster $c_2$ \\ \hline
		$\mathrm{f}$ & $61.22$ & $6.08$ & Cluster $c_2$ \\ \hline
		$\mathrm{g}$ & $47.63$ & $11.70$ & Cluster $c_2$ \\ \hline
		\end{tabular}
	\end{table}
	It is clear that the composition of the clusters does not change any more and a projection of the clusters on a chart gives:
	\begin{figure}[H]
		\centering
		\includegraphics[scale=1]{img/computing/kmeans_simple_example.jpg}
	\end{figure}
	
	So we see with this small previous example that our problem is the same as minimizing the enlarged criterion:
	
	over bother clusterings $C$ and $c_1,\ldots,c_K\in\mathbb{R}^p$. That is to say, minimize the "within-cluster sum of squares" (WCSS).
	
	\begin{tcolorbox}[title=Remark,colframe=black,arc=10pt]
	$K$-means is as we have already mentioned it a hard assignment version of the soft assignment variant of EM (see above page \pageref{EM algorithm}), with the assumptions that clusters are spherical. Here "spherical" means identical variance-covariance matrices for each cluster (assuming gaussian distribution), which is also known as model-based clustering. Furthermore notice that there is no "$K$-means algorithm". There is MacQueens algorithm for $K$-means, the Lloyd/Forgy algorithm for $K$-means, the Hartigan-Wong method, ...\\
	
	The procedure describe above is the Lloyds algorithm (the one assumed by default in most cheap books on Data Science, it is sometimes also referred to as "naive $K$-means", because there exist much faster alternatives) that consists of two step given a set of $K$-means $m_1^{(1)},\ldots,m_k^{(1)}$ and therefore of $S = \{S_1, S_2, \ldots, S_k\}$ clusters:
	\begin{itemize}
		\item the E-step (assignment step), where each object is assigned to the centroid such that it is assigned to the most likely cluster using the shortest euclidean distance):
		
		
		\item the M-step (update step), where the model (=centroids) are recomputed (= least squares optimization):
		
	\end{itemize}
	... iterating these two steps, as done by Lloyd, makes this effectively an instance of the general EM scheme.
	\end{tcolorbox}
	
	We see also well through the previous step-by-step example that the $K$-means algorithm (which therefore belongs to non-hierarchical clustering techniques) however does not necessarily converge to an optimal solution. Let us recall that a global optimization calculation is inconsistent with the data volumes used, regardless of the power of computers. Thus, the $K$-means will use iterative algorithms to reach a local optimum. It is also an algorithm that will try to find the best $K$ initial points.

	Some software gives you the ability to set the initial values and this will affect the final quality of the typology, knowing that there is ONE no good initial choice. This varies depending on the configuration of data, on the return of experience (REX) and even the chance...
	
	There are three common solutions:
	\begin{enumerate}
		\item The software determines the $K$ initial points randomly. It can perform a number of tests and he will choose the most conclusive one.

		\item We use the expert opinion that suppose someone has a fairly good knowledge of the study population for attaching to each class an ideal type. This may or not be a real individual.

		\item The software distribute the $K$ initial points not randomly but according to some empirical algorithms.
	\end{enumerate}

	Let us see a further step that is instructive to show how to implement this technique in a spreadsheet software like Microsoft Excel 14.0.6123 with genetic algorithms (see further below).
	
	For this, we will first consider the following structure for which the idea is to establish three centers (so it is a $3$-means):
	\begin{figure}[H]
		\centering
		\includegraphics[scale=0.74]{img/computing/kmeans_initial_sheet_values_and_chart_excel.jpg}
		\caption[]{Basic starting $K$-means in Microsoft Excel 14.0.6123}
	\end{figure}
	where we have on the left data from a population-based on the characteristics $X$ and $Y$ with a small table that will display the coordinates of the three centroid. We create on the same sheet the following table:
	\begin{figure}[H]
		\centering
		\includegraphics[scale=0.74]{img/computing/kmeans_initial_sheet_point_cluster_association_excel.jpg}
		\caption[]{Initial points-centroids $K$-means association in Microsoft Excel 14.0.6123}
	\end{figure}
	with trivial formulas for the three columns \texttt{N}, \texttt{O}, \texttt{P}, where we used the standard Euclidean distance:
	\begin{figure}[H]
		\centering
		\includegraphics[scale=0.5]{img/computing/kmeans_initial_sheet_point_cluster_association_excel_explicit_formulas.jpg}
		\caption[]{Initial points-centroids $K$-means association explicit formulas in Microsoft Excel 14.0.6123}
	\end{figure}
	Then we launch the Microsoft Excel 14.0.6123 Solver  with the following parameters being careful to take the Evolutionary algorithm option. Therefore we assume that at every run we could have a different results:
	\begin{figure}[H]
		\centering
		\includegraphics[scale=0.8]{img/computing/kmeans_solver_settings_excel.jpg}
		\caption[]{$K$-means solver settings in Microsoft Excel 14.0.6123}
	\end{figure}
	and therefore we get for results:
	\begin{figure}[H]
		\centering
		\includegraphics[scale=0.57]{img/computing/kmeans_final_sheet_point_cluster_association_excel.jpg}
		\caption[]{$K$-means final associations in Microsoft Excel 14.0.6123}
	\end{figure}
	A software like Minitab 15.1.1 gives us other values for the centroids. As the latter don't give any plots let us see how the associations looks like if we write manually the centroid values given by Minitab in our Microsoft Excel sheet:
	\begin{figure}[H]
		\centering
		\includegraphics[scale=0.55]{img/computing/kmeans_final_point_cluster_association_minitab.jpg}
		\caption[]{$K$-means final associations with Minitab 15.1.1 values}
	\end{figure}
	The huge difference between Microsoft Excel and any other Statistical software is quite simple to explain! Software such as Microsoft Excel 14.0.6123 minimizes the distance points to the centers but is unable simultaneously to maximize the distance between the centers. Again the statistical software have algorithms implemented for this purpose. More in details the idea is the following:
	
	The quality/assessment of a clustering\footnote{In an unsupervised learning setting, it is often hard to assess the performance of a model since we don't have the ground truth labels as was the case in the supervised learning setting.} is many times measured as following:
	\begin{itemize}
		\item We use the "\NewTerm{cluster cohesion}", also named "with sum of squares", already introduced earlier (we just generalize the notation without expliciting the type of distance/metric):
		
		In practice the purpose is to minimize it.
	
		\item We introduce the "\NewTerm{cluster separation}", also named "between sum of squares", defined by:
		
		where $n_k$ is the number of points in cluster number $k$ and $C$ is the barycenter of all clusters. In practice the purpose is to maximize it.
	\end{itemize}
	Using the properties of the barycenter (\SeeChapter{see section Geometry page \pageref{barycenter}}), we can then write the total sum of squares:
	
	That latter relation is also often written as:
	
	with the following companion illustration:
	\begin{figure}[H]
		\centering
		\includegraphics[width=1.0\textwidth]{img/computing/kmean_wss_bss_v2.jpg}
		\caption[]{$K$-means BSS and WSS}
	\end{figure}
	So in practice, the $K$-means method is fully encompassed based on the above fundamental relation. It is then quite easy to understand that if Intra-cluster inertia decreases the inter-cluster inertia increases and vice versa.
	
	Therefore with Tanagra 1.4.44 we get for example:
	\begin{figure}[H]
		\centering
		\includegraphics{img/computing/kmeans_final_point_cluster_association_tanagra.jpg}
		\caption[]{$K$-means final associations in Tanagra 1.4.44}
	\end{figure}
	and using the detailed steps given in our \texttt{R} companion book we get:
	\begin{figure}[H]
		\centering
		\includegraphics{img/computing/kmeans_final_point_cluster_association_plot1_r.jpg}
	\end{figure}
	\begin{figure}[H]
		\centering
		\includegraphics{img/computing/kmeans_final_point_cluster_association_plot2_r.jpg}
	\end{figure}
	\begin{figure}[H]
		\centering
		\includegraphics{img/computing/kmeans_final_point_cluster_association_plot3_r.jpg}
		\caption[]{$K$-means final associations in \texttt{R} 3.0.2}
	\end{figure}
	A major difficulty with the $K$-means is to choose the number of clusters. For this let us recall the fundamental relation of $K$-means:
	
	The strategy commonly chosen to choose the right $K$ is to make vary WSS as a function of $k$ and to take the last of $k$ which induces a considerable reduction of WSS. Geometrically speaking, at this optimal point, we observe a kind of "elbow" (or "knee") forming on the plot of $\text{WSS}=f(K)$:
	\begin{figure}[H]
		\centering
		\includegraphics[scale=0.7]{img/computing/kmeans_clusters_dependance_wss.jpg}
	\end{figure}
	In the case of our first $K$-means clustering, the partition in $K=2$ is more interesting, because not only does it induce a considerable gain in information (according to the figure above), thus a significant loss or reduction of WSS, but also it offers clusters having more than $2$ individuals. Beyond this, for example, for $K=3$, the gain in information is no longer significant and we will obtain a cluster with only $1$ individual (not very relevant for a group analysis...).
	
	In addition to homogeneity of the observations within each group, we also seek heterogeneity of the groups; hence, intuitively, an $F$ -like statistic as a ratio of the between-group dissimilarity to the within-group dissimilarity could be used to indicate the goodness of a given clustering. We define a "\NewTerm{pseudo $F$}":
	
	This measure is also named the "\NewTerm{Calinski-Harabasz index}\index{Calinski-Harabasz index}". The larger that ratio, the better the clustering at any fixed number of clusters.

	A practical approach to the problem is to vary $K$, as $K_{0}, K_{0}+1, \ldots$, computing $\widetilde{F}_{K}$, which will initially increase fairly rapidly, and to choose the value of $K$ as the point at which $\widetilde{F}_{K+1}-\widetilde{F}_{K}$ is relatively small. This is similar to the type of approach in a different context; that is, choosing the number of principal components.

	\begin{tcolorbox}[title=Remarks,colframe=black,arc=10pt]
	\textbf{R1.} In some applications we want each center to be one of the point itself. This is where "\NewTerm{$K$-medoids}\index{$K$-medoids}" (Partition around medoids) comes in an algorithm similar to the $K$-means algorithm, except when fitting the centers $c_1,\ldots,c_K$, we restrict our attention to the point themselves. Generally $K$-medoids obvious return a higher value of $K_{\text{means}}$. There is also on example in our \texttt{R} companion book.\\
	
	\textbf{R2.} Obviously as for the $k$-nn, we may run the $K$-means on the factorial space obtained after having run a PCA (Principal Component Analysis as seen on page \pageref{principal component analysis}) on our data. We can also even reduce the dimensions using that latter if necessary!
	\end{tcolorbox}
	A very common simple assessment metric is the "\NewTerm{silhouette coefficient}\index{silhouette coefficient}" given as graphical output in most statistical software. The idea is by noting $a$ and $b$ the mean distance between a sample and all other points in the same class, and between a sample and all other points in the next nearest cluster, the silhouette coefficient $s$ for a single sample is defined as:
	
	
	\paragraph{Support Vector Machines (SVM) classifier}\label{support vector machines}\mbox{}\\\\
	"\NewTerm{Support vector machines}\index{support vector machines}" (SVM) is learning method used for binary classification\footnote{In its most simple type, SVM doesn't support multiclass classification natively. It supports binary classification and separating data points into two classes. For multiclass classification, the same principle is utilized after breaking down the multiclassification problem into multiple binary classification problems.} that searches for so-called "\NewTerm{support vectors}" which are observations that are found to lie at the edge of an area in space which presents a boundary between one of these classes of observations (e.g., the squares) and another class of observations (e.g., the circles). SVM seems to have been introduced by Vladimir Vapnik and colleagues. The earliest mention was in (Vapnik, 1979), but the first main paper seems to be (Vapnik, 1995). In the terminology of SVM we talk about the space between these two regions as the "margin" between the classes. Each region contains observations with the same value for the target variable (i.e., the class). The support vectors, and only the support vectors, are used to identify a hyperplane (a straight line in two dimensions) that separates the classes. The maximum margin between the separable classes is sought. This then represents the model. 
	
	\begin{figure}[H]
		\centering
		\includegraphics{img/computing/svm.jpg}
		\caption[Support Vector Machine (SVM) idea]{SVM idea (source: Wikipedia)}
	\end{figure}
	It is usually quite rare that we can separate the data with a straight line (or a hyperplane when we have more than two input variables). That is, the data is not usually distributed in such a way that it is "\NewTerm{linearly separable}". When this is the case, a technique is used to combine (or remap) the data in different ways, creating new variables so that the classes are then more likely to become linearly separable by a hyperplane (i.e., so that with the new dimensional data there is a gap between observations in the two classes). We can use the model we have built to score new observations by mapping the data in the same way as when the model was built, and then decide on which side of the hyperplane the observation lies and hence the decision associated with it.
	
	Support vector machines have been found to perform well on problems that are non-linear, sparse, and high-dimensional. A disadvantage is that the algorithm is sensitive to the choice of tuning option (e.g., the type of transformations to perform), making it harder to use and time consuming to identify the best model. Another disadvantage is that the transformations performed can be computationally expensive and are performed both whilst building the model and when scoring new data.
	
	An advantage of the method is that the modelling only deals with
these support vectors rather than the whole training dataset, and so the size of the training set is not usually an issue. Also, as a consequence of only using the support vectors to build a model, the model is less affected by outliers.

	Ok let us first introduce the idea in a gently way...:

	Let $\mathcal{D}=\left\{\left(\vec{x}_{i}, y_{i}\right)\right\}_{i=1}^{n}$ be a classification dataset, with $n$ points in a $d$-dimensional space. Further, let us assume that there are only two class labels, that is, $y_{i} \in\{+1,-1\}$ denoting the positive and negative classes.
	
	A hyperplane in $d$ dimensions is given as the set of all points $\vec{x} \in \mathbb{R}^{d}$ that satisfy the equation $h(\vec{x})=0$, where $h(\vec{x})$ is the hyperplane function, defined as follows (\SeeChapter{see section Analytical Geometry page \pageref{equation of the plane}}):
	
	Here, $\vec{w}$ is a dimensional weight vector and $b$ is a scalar, named  without surprise the "\NewTerm{bias}\index{bias}". For points that lie on the hyperplane, we have:
	
	The hyperplane is thus defined as the set of all points such that $\vec{w}^{T} \vec x=-b$. To see the role played by $b$, assuming that $w_{1} \neq 0$, and setting $x_{i}=0$ for all $i>1$, we can obtain the offset where the hyperplane intersects the first axis. Therefore we have obviously:
	
	In other words, the point:
	
	lies on the hyperplane (and it's module is equal is the perpendicular distance from the hyperplane to the origin!). In a similar manner, we can obtain the offset where the hyperplane intersects each of the axes, which is given as $-b/w_{i}$ (provided $\left.w_{i} \neq 0\right)$.
	
	A hyperplane splits the original $d$ -dimensional space into two half-spaces (remember the first figure above depicting the idea!). A dataset is said to be "\NewTerm{linearly separable}" if each half-space has points only from a single class. If the input dataset is linearly separable, then we can find a separating hyperplane $h(\vec{x})=0$, such that for all points labelled $y_{i}=-1$, we have $h\left(\vec{x}_{i}\right)<0$, and for all points labelled $y_{i}=+1$, we have $h\left(\vec{x}_{i}\right)>0$. In fact, the hyperplane function $h(\vec{x})$ serves as a linear classifier or a linear discriminant, which predicts the class $y$ for any given point x, according to the decision rule:
	
	Let $\vec{a}_{1}$ and $\vec{a}_{2}$ be two arbitrary points that lie on the hyperplane. From the hyperplane equation we have:
	
	Subtracting one from the other we obtain:
	
	This means without surprise (as we also proved it in the section of Analytical Geometry) that the weight vector $\vec{w}$ is orthogonal to the hyperplane because it is orthogonal to any arbitrary vector $\left(\vec{a}_{1}-\vec{a}_{2}\right)$ on the hyperplane. In other words, the weight vector $\vec{w}$ specifies the direction that is normal to the hyperplane, which fixes the orientation of the hyperplane, whereas the bias $b$ fixes the offset of the hyperplane in the $d$-dimensional space. Because both $\vec{w}$ and $-\vec{w}$ are normal to the hyperplane, we remove this ambiguity by requiring that $h\left(\vec{x}_{i}\right)>0$ when $y_{i}=1$, and $h\left(\vec{x}_{i}\right)<0$ when $y_{i}=-1$.
	
	Ok now that we have finished with the gentle introduction let's go more in the maths with a text completely inspired from \cite{fletcher2009support}!
	
	We are given $l$ training examples $\{ \vec{x} _i , y_i\}, \ i = 1,\ldots, L \ $, where each example has $d$ inputs ($\vec{x}_i \in \mathbb{R}^d$), and a class label with one of two values ($y_i \in \{-1,1\}$).
	
	Now, all hyperplanes in $\mathbb{R}^d$ are parametrized by a vector ($\vec{w}$), and a constant ($b$), expressed in the equation:
	
	(recall again that $\vec{w}$ is in fact the vector orthogonal to the hyperplane) Given such a hyperplane ($\vec{w}$,$b$) that separates the data, this gives the
function:
	
	which correctly classifies the training data (and hopefully other "testing" data it hasn't seen yet).
	
	However, a given hyperplane represented by ($\vec{w}$,$b$) is equally expressed by all pairs $\{\lambda \vec{w}, \lambda b \}$ for $\lambda \in \mathbb{R}^+$.  So we define the "\NewTerm{canonical hyperplane}\index{canonical hyperplane}" to be that which separates the data from the hyperplane by a "distance" of at least\footnote{In fact, we require that at least one example on both sides has a distance of \textit{exactly} $1$.  Thus, for a given hyperplane, the scaling (the $\lambda$) is implicitly set.} $1$.  That is, we consider those that satisfy:
	
	or more compactly:
	
	That we often write as:
	
	All such hyperplanes have a "functional distance" $\ge 1$ (quite literally, the function's value is $\ge 1$).  This shouldn't be confused with the "geometric" or "Euclidean distance" (also known as the \text{margin}).  For a given hyperplane
($\vec{w}$,$b$), all pairs $\{\lambda \vec{w}, \lambda b \}$ define the exact same hyperplane, but each has a different functional distance to a given data point.

	To obtain the geometric distance from the hyperplane to a data point, we must normalize by the magnitude of $\vec{w}$.  This distance is simply:
	
	Intuitively, we want the hyperplane that maximizes the geometric distance to the closest data points.
	
	As simple vector geometry shows that the margin is equal to $1/\|\vec{w}\|$ and maximizing it subject to the constraint in:
	
	is equivalent to finding:
	
	Minimizing $\|\vec{w}\|$ is equivalent to minimizing $\frac{1}{2}\|\vec{w}\|^{2}$ and the use of this term makes it possible to perform Nonlinear Programming (see page \pageref{nonlinear optimization}) optimization later on. We therefore need to find:
	
	In order to cater for the constraints in this minimization, we need to allocate them Lagrange multipliers (see \pageref{Lagrange multipliers method}) $\alpha$, where\footnote{These conditions are necessary for the primal optimization problem $L_P$ to be strictly equal to the dual problem $L_D$, such that $L_P=L_D$ and not just equal or less such that $L_D\leq L_P$.} $\alpha_{i} \geq 0\;\forall_{i}$:
	
	We wish to find the $\vec{w}$ and $b$ which minimizes, and the $\vec \alpha$ which maximizes the above relation (whilst keeping $\alpha_{i} \geq 0\; \forall_{i}$). We can do this by differentiating $L_{P}$ with respect to $\vec{w}$ and $b$ and setting the derivatives to zero:
	
	Substituting these two relations into:
	
	 gives a new formulation which, being dependent on $\vec \alpha,$ we need to maximize:
	
	This new formulation $L_{D}$ is referred to as we know to as the "Dual form" of the Primary $L_{P}$. An the corresponding constraint (conditions) are named the "\NewTerm{Karush-Kuhn-Tucker conditions}\index{Karush-Kuhn-Tucker conditions}" (there exist other possible conditions but there are out of the scope of this book). It is worth noting that the Dual form requires only the dot product of each input vector $\vec x_{i}$ to be calculated, this is important for the Kernel Trick described further below.
	
	\begin{tcolorbox}[title=Remark,colframe=black,arc=10pt]
	The matrix $H$ above is often written $H_{i j} \equiv y_{i} y_{j} k(\vec{x}_{i} ,\vec{x}_{j})$ where $k(\vec{x}_{i} ,\vec{x}_{j})$ is an example of a family of functions named "\NewTerm{kernel functions}\index{kernel functions}". The special case $k(\vec{x}_{i} ,\vec{x}_{j})=\vec{x}_{i} \circ \vec{x}_{j}$ is known as a the "\NewTerm{linear kernel}".\\
	
	The reason that this kernel trick is useful is that there are many classification/regression problems that are not linearly separable/regressable in the space of the inputs $\vec{x}$, which might be in a higher dimensionality feature space given a suitable mapping $\vec{x}\mapsto \Phi(\vec{x})$. For example the "radial kernel":
	$$k\left(\vec{x}_{i}, \vec{x}_{j}\right)=e^{-\left(\frac{\left\|\vec{x}_{i}-\vec{x}_{j}\right\|^{2}}{2 \sigma^{2}}\right)}$$
	Other popular kernels (among many others) for classification and regression are the "polynomial
kernel":
	$$k\left(\vec{x}_{i}, \vec{x}_{j}\right)=\left(\vec{x}_{i} \circ \vec{x}_{j}+a\right)^{b}$$
	and the "sigmoidal Kernel":
	$$k\left(\vec{x}_{i}, \vec{x}_{j}\right)=\tanh \left(a \vec{x}_{i} \cdot \vec{x}_{j}-b\right)$$
	where $a$ and $b$ are parameters defining the kernel's behaviour.
	\end{tcolorbox}
	
	Having moved from minimizing $L_{P}$ to maximizing $L_{D},$ we need to find:
	
	This is a convex quadratic optimization problem, and we run a solver which will return $\vec \alpha$ and from:
	
	will give us $\vec{w}$. What remains is to calculate $b$.
	
	Any data point satisfying:
	
	which is a support vector denoted $\vec x_{s}$ (samples that are on the gutter) will have the form:
	
	If we inject into it the previous derived relation:
	
	we get:
	
	Where $S$ denotes the set of indices of the support vectors. $S$ is determined by finding the indices $i$ where $\alpha_{i}>0$. Multiplying through by $y_{s}$ and then using $y_{s}^{2}=1$ we have :
	
	Therefore:
	
	Instead of using an arbitrary support vector $\mathrm{x}_{s}$, it is better to take an average over all of the support vectors in $S$ :
	
	We now have the variables $\vec{w}$ and $b$ that define our separating hyperplane's optimal orientation and hence our Support Vector Machines.
	
	So to summarize...  In order to use an SVM to solve a linearly separable, binary classification problem we need to:
	\begin{itemize}
		\item Create the matrix ${H}$, where $H_{i j}=y_{i} y_{j} \vec{x}_{i} \circ \vec{x}_{j}$

		\item Find $\vec \alpha$ so that:
		
		is maximized, subject to the constraints:
		
		This is done using a nonlinear programming solver.
		
		\item Calculate:
		

		\item Determine the set of support vectors $S$ by finding the indices such that $\alpha_{i}>0$
		
		\item Calculate:
		

		\item Each new point $\vec{x}^{\prime}$ is classified by evaluating:
		
	\end{itemize}

	\begin{tcolorbox}[title=Remark,colframe=black,arc=10pt]
	We recommend that the reader interested in delving deeper into the subject to read the article \textit{Support Vector Machines Explained} from Tristan Fletcher \cite{fletcher2009support} which was used to write the above presentation. You will then see how we can extend the SVM methodology to handle data that is not fully
linearly separable or what are the maths behind regression SVM.
	\end{tcolorbox}
	
	
	\paragraph{Gaussian Mixture Model (GMM) clustering}\label{Gaussian mixture model}\mbox{}\\\\
	A more formal approach to clustering can be achieved if we assume underlying distributions of the observations. This is also known as "\NewTerm{finite mixture model clustering}\index{finite mixture model clustering}". Observations arise from a distribution that is a mixture of two or more components, or clusters. Each cluster is described by a density and has an associated probability or weight in the mixture.
	
	We can think of building a "\NewTerm{Gaussian Mixture Model}\index{Gaussian Mixture Model}" as a type of clustering algorithm. Using again the expectation-maximization algorithm seen earlier above (see page \pageref{EM algorithm}), the process and result is very similar to $K$-means clustering. The difference is that the clusters are assumed to each have an independent Gaussian distribution, each with their own mean and covariance matrix.
	
	When performing $K$-means clustering, we assign points to clusters using the straight Euclidean distance. The Euclidean distance is a poor metric, however, when the cluster contains significant covariance! The Gaussian Mixture Models approach will take cluster covariance into account when forming the clusters.

	Another important difference with $K$-means is that standard $K$-means performs a hard assignment of data points to clusters, ie each point is assigned to the closest cluster. With Gaussian Mixture Models, what we will end up is a collection of independent multivariate Gaussian distributions, and so for each data point, we will have a probability that it belongs to each of these distributions / clusters.
	
	Suppose we are given a data set $D = \{\vec{x}_1,\ldots,\vec{x}_N\}$ where $\vec{x}_i$ is a $d$-dimensional vector measurement. Let us assume that the points are generated in an independent and identically distributed way from an underlying density $P(\vec{x})$.
	
	We further assume that $P(\vec{x})$ is defined as a finite mixture model with $K$ components:
	
	where:
	\begin{itemize}
		\item The $P_k(\vec{x} | z_k, \theta_k)$ are mixture components, $1 \le k \le K$. Each is a density or distribution defined over $P(\vec{x})$, with parameters $\theta_k$. 
	
		\item $\vec{z} = (z_1,\ldots,z_K)$ is a vector of $K$ binary indicator variables that are mutually exclusive and exhaustive (i.e., one and only one of the $z_k$'s is equal to $1$, and the others are $0$). $\vec{z}$ is a $K$-ary random variable representing the identity of the mixture component that generated $\vec{x}$. It is convenient for mixture models to represent $z$ as a vector of $K$ indicator variables.
		
		\item The $\alpha_k = P(z_k)$ are the mixture weights, representing the probability that a randomly selected $\vec{x}$ was generated by component $k$, where $\sum_{k=1}^K \alpha_k = 1$.
	\end{itemize}   
	The complete set of parameters for a mixture model with $K$ components is:
	
	 We can compute the "membership weight" (ie hidden posterior) of data point $\vec{x}_i$ in cluster $k$, given  parameters $\Theta$ as:
	
	This follows from a direct application of Bayes rule often written as:
	
	The membership weights above reflect our uncertainty, given $\vec{x}_i$ and $\Theta$, about which of the $K$ components generated vector $\vec{x}_i$. Note that we are assuming in our generative mixture model that each $\vec{x}_i$ was generated by a single component so these probabilities reflect our uncertainty about which component $\vec{x}_i$ came from, not any "mixing" in the generative process.

	For $\vec{x} \in {\mathbb{R}}^d$ we can define a Gaussian mixture model by making each of the $K$ components  a Gaussian density with parameters $\vec{\mu}_k$ and $\Sigma_k$.
	
	Each component (cluster) is a multivariate Gaussian density given for recall by:
	
	with its own parameters $\theta_k = \{\vec{\mu}_k, \Sigma_k\}$.

	We use the Expectation-Maximization algorithm for Gaussian mixtures as follows: The algorithm is an iterative algorithm that starts from some initial estimate of $\Theta$ (e.g., random), and then proceeds to iteratively update $\Theta$ until convergence is detected. Each iteration consists of an E-step and an M-step.
	
	To kickstart the EM algorithm, we randomly select data points to use as the initial means, and we set the covariance matrix for each cluster to be equal to the covariance of the full training set. Also, we give each cluster equal prior probability. A cluster's prior probability is just the fraction of  the dataset that belongs to each cluster. We start by assuming the dataset is equally divided between the clusters.
	
	In other words:
	\begin{itemize}
		\item In the "Expectation" step (E-step), we will calculate the (posterior) probability that each data point $\vec{x}^{i}$ belongs to each cluster $k$ (using our current estimated mean vectors and covariance matrices) following:
				
		This seems analogous to the cluster assignment step in $K$-means. Technically this step denote the current parameter values as $\Theta$.  It computes the $w_{i,k}$ (using the equation above for membership weights) for all data points $\vec{x}_i, 1 \le i \le N$ and all mixture components $1 \le k \le K$. Note that for each data point $\vec{x}_i$ the membership weights are defined such that $\sum_{k=1}^K w_{i,k} = 1$. This yields an $N \times K$ matrix of membership weights, where each of the rows sum to $1$.

		\item In the "Maximization" step (M-step), we will re-calculate the cluster means and covariances based on the probabilities calculated in the expectation step (ie we use the membership weights and the data to calculate new parameter values). This seems analogous to the cluster movement step in $K$-means.
	\end{itemize}

	Let $N_k = \sum_{i=1}^N w_{i,k}$, i.e., the sum of the membership weights for the $k$th component---this is the effective number of data points assigned to component $k$.

	Specifically:
	
	These are the new mixture weights:
	
	The updated mean is calculated in a manner similar to how we could compute a standard empirical average, except that the $i$th data vector $\vec{x}_i$ has a fractional weight $w_{i,k}$.

	Note that this is a vector equation since $\vec{\mu}_k^{\text{new}}$ and $\vec{x}_i$ are both $d$-dimensional vectors:
	
	Again we get an equation that is similar in form to how we would normally compute an empirical covariance matrix, except that the contribution of each data point is weighted by $w_{i,k}$. Note that this is a matrix equation of dimensionality $d \times d$ on each side.

	The  equations in the M-step need to be computed in this order, i.e., first compute the $K$ new $\alpha$'s, then the $K$ new $\vec{\mu}_k$'s, and finally the $K$ new $\Sigma_k$'s.

	After we have computed all of the new parameters, the M-step is complete and we can now go back and recompute the membership weights in the E-step, then recompute the parameters again in the E-step, and continue updating the parameters in this manner. Each pair of E and M steps is considered to be one iteration.

	The EM algorithm can be started by either initializing the algorithm with a set of initial parameters and then conducting an E-step, or by starting with a set of initial weights and then doing a first M-step. The initial parameters or weights can be chosen randomly (e.g. select $K$ random data points as initial means and select the covariance matrix of the whole data set for each of the initial $K$ covariance matrices) or could be chosen via some heuristic method (such as by using the $K$-means algorithm to cluster the data first and then defining weights based on $K$-means memberships).
	\begin{figure}[H]
		\centering
		\includegraphics[width=1.0\textwidth]{img/computing/expectation_maximization_algorithm.jpg}
		\caption[Expectation-Maximization iterative algorithm]{Expectation-Maximization iterative algorithm (authors: Afshine Amidi, Shervine Amidi)}
	\end{figure} 
	Convergence is generally detected by computing the value of the log-likelihood after each iteration and halting when it appears not to be changing in a significant manner from one iteration to the next. Note that the log-likelihood (under the IID assumption) is defined as follows for this case:
	
	where $P_k(\vec{x}_i | z_k, \theta_k)$ is the Gaussian density for the $k$th mixture component. 
	
	Explicitly the complete data log-likelihood over all points $\{ ( \vec{x}_n, \vec{z}_n) \}_{n=1}^N$ is:
	
	We can now detail the Expectation-Maximization algorithm for Gaussian mixtures:
	\begin{itemize}
		\item \textbf{E-Step:}
		
		Before the E-step, we have an estimate $\theta_t$ of the parameters, and seek to compute a new lower bound on the observed log-likelihood.  Earlier, we showed that the optimal lower bound is (see page \pageref{evidence lower bound}):
	    
		where $q_{\theta_t}(z) \equiv p(\vec{z}|\vec{x},\theta_t)$ and the second term is constant with respect to $\theta$. The E-Step requires us to derive an expression for the first term. Using the relation above:
		
		the expected complete data log-likelihood is given by:
		
		where $r_{nk}:= P(z_n = k \mid x_n, \theta_t)$ is named the "responsability" that cluster $k$ takes for data point $x_n$ after step $t$.  During the E-Step, we compute these values explicitly with the relation seen earlier above:
		
		
		\item \textbf{ M-Step:}
		
		During the M-Step, we optimize our lower bound with respect to the parameters $\theta = (\vec\alpha, \vec\mu, \vec\Sigma)$.  For the mixing weights $\vec\pi$, we use Lagrange multipliers to maximize the evidence lower bound (see page \pageref{evidence lower bound}) subject to the constraint $\sum_{k=1}^K \pi_k = 1$.  The Lagrangian (see page  \pageref{Lagrange multipliers method}) is:
	    
	    with for recall:
	    
	    So let us derive the four corresponding results... In order to resolve this problem, we have to found the parameters for which the partial derivatives are null:
	    
		First, let's start with the equation with respect to $\lambda$:
		
		Ok, so we are back to the definition of the constraint. No surprise here but not very useful.
		
		Now let's move on with the equation with respect to $\alpha_k$. Note that in the following derivations, we use the fact that deriving an $\alpha_j$ with $j$ different than $j$ results in a constant (this basically means that the summation over the $K$ clusters can be ignored):
		
		Now in order to get rid of the $\lambda$, we use the definition of the constraint regarding the mixtures weights:
		 
		And this gives us the final closed-form expressions for the mixture weights:
		 
		This tells us that for each class, after the M-Step, the mixture weight will be the sum of all the individual weights for that class normalized by the sum of all the individual weights for all the classes. And this makes perfect sense, if all the observations put a little weight on a specific class compared to the other classes than this class will have a small overall weight, and vice-versa.
		
		Now let's move on the derivation of the mean vector $\mu$. This one is a little trickier because, this time, we are computing the partial derivative for vectors and matrices:
		
		This looks perfectly right. Out of the M-Step, the mean value for a specific Gaussian distribution will be the expected value of the data points with respect to the variational distribution $r_{nk}$ normalized by the sum of all the weights.
		
		Finally, let's derive the analytical update expression for the covariance matrix $\Sigma$ (we use in this development derivatives of matrices as seen in the section of Linear Algebra page \pageref{derivative of logarithm of a determinant}):
		
		And this looks pretty dam right! The covariance matrix after the M-Step is a re-weighted version of the previous covariance matrix.
	\end{itemize}

	Most algorithms searches over a range of different types of Gaussians and $n$ (named sometimes the "number of components") to find the best model by the Bayesian information criterion (BIC). The model is often referred to a three letter code nomenclature from the variance-covariance matrix describing the "shape" (relative magnitude of eigenvalues categorized into $2$ different possibilities), "volume" (absolute magnitude of eigenvalues categorized into $3$ different possibilities) and "orientation" (orientation of the eigenvectors categorized into $4$ different possibilities) of the clusters. So there is a total of $2\cdot 3\cdot 4=24$ possibilities, here is a non-exhaustive list:
	\begin{itemize}
		\item \texttt{EII}: equal volume, round shape (spherical covariance)
		\item \texttt{VII}: varying volume, round shape (spherical covariance)
		\item \texttt{EEI}: equal volume, equal shape, axis parallel orientation (diagonal covariance)
		\item \texttt{VEI}: varying volume, equal shape, axis parallel orientation (diagonal covariance)
		\item \texttt{EVI}: equal volume, varying shape, axis parallel orientation (diagonal covariance)
		\item \texttt{VVI}: varying volume, varying shape, equal orientation (diagonal covariance)
		\item \texttt{EEE}: equal volume, equal shape, equal orientation (ellipsoidal covariance)
		\item \texttt{EEV}: equal volume, equal shape, varying orientation (ellipsoidal covariance)
		\item \texttt{VEV}: varying volume, equal shape, varying orientation (ellipsoidal covariance)
		\item \texttt{VVV}: varying volume, varying shape, varying orientation (ellipsoidal covariance)
	\end{itemize}
	Some of the above combinations are illustrated below:
	\begin{figure}[H]
		\centering
		\includegraphics[width=1.0\textwidth]{img/computing/variance_covariance_matrix_structures.jpg}
		\caption{Variance-Covariance matrix typical structures}
	\end{figure}  

	
	\paragraph{Mean shift clustering}\mbox{}\\\\
	"\NewTerm{Mean shift}\index{Mean shift}" is a powerful and versatile non parametric iterative algorithm that can be used for lot of purposes like finding modes, clustering, tracking images etc. Mean Shift was introduced by Keinosuke Fukunaga and Larry D. Hostetler in 1975 and has been extended to be applicable in other fields like Computer Vision.This document will provide a discussion of Mean Shift , prove its convergence and slightly discuss its important applications.
	
	This text provides an intuitive idea of Mean shift and the later sections will expand the idea. Mean shift considers feature space as a empirical probability density function. If the input is a set of points then Mean shift considers them as sampled from the underlying probability density function. If dense regions (or clusters) are present in the feature space , then they correspond to the mode (or local maxima) of the probability density function. We can also identify clusters associated with the given mode using Mean Shift.

	For each data point, Mean shift associates it with the nearby peak of the dataset’s probability density function. For each data point, Mean shift defines a window around it and computes the mean of the data point . Then it shifts the center of the window to the mean and repeats the algorithm till it converges. After each iteration, we can consider that the window shifts to a more denser region of the dataset.
	
	At the high level, we can specify Mean Shift as follows :
	\begin{enumerate}
		\item Choose a kernel $K$
		\item Fix a window size $h$\footnote{Some algorithms have a dynamic window}
		\item Randomly segment the domain with the kernel $K$ (of size $h$)
		\item Compute the mean of data (barycenter) weighted within the window $h$ of each kernel
		\item Shift the window $h$ of each kernel $K$ to the mean (barycenter) and repeat till convergence
		\item Calculate the probability each point has to belong to one of the kernel $K$
		\item Cluster each point to the kernel $K$ have the highest probability
	\end{enumerate}
	
	Before dealing with the mathematical aspect, let us consider the graphic and playful aspect! The purpose of the Mean Shift algorithm is to find the maximum density region(s) per iteration. So consider for example the following bi-varied case with some points:
	\begin{figure}[H]
		\centering
		\includegraphics[scale=0.5]{img/computing/mean_shift_initial_points.jpg}
	\end{figure} 
	where we place randomly one or many circles (in $d$-dimensional case it will be $d$-dimensions sphere instead of simple circles) of a given radius:
	\begin{figure}[H]
		\centering
		\includegraphics[scale=0.9]{img/computing/mean_shift_initial_configuration.jpg}
		\caption[]{Initial configuration of a bi-dimensional mean shift application}
	\end{figure} 
	Then more or less empirically (there are different techniques!) we calculate the center of mass inside this circle:
	\begin{figure}[H]
		\centering
		\includegraphics[scale=0.7]{img/computing/mean_shift_barycenter_calculation.jpg}
	\end{figure} 
	Then we move the circle on the center of gravity (barycenter) to get:
	\begin{figure}[H]
		\centering
		\includegraphics[scale=0.8]{img/computing/mean_shift_barycenter_move.jpg}
	\end{figure}
	and we repeat the procedure:
	\begin{figure}[H]
		\centering
		\includegraphics[scale=0.85]{img/computing/mean_shift_barycenter_move_again.jpg}
	\end{figure}
	And so on:
	\begin{figure}[H]
		\centering
		\includegraphics[scale=0.8]{img/computing/mean_shift_barycenter_move_again_and_again.jpg}
	\end{figure}
	That is to say that if we have for example a distribution of points such as below:
	\begin{figure}[H]
		\centering
		\includegraphics{img/computing/mean_shift_3d_perspective.jpg}
	\end{figure}
	In reality we will randomly segment the domain and run the algorithm iteratively in parallel for each sphere (circle in the case of dimension 2):
	\begin{figure}[H]
		\centering
		\includegraphics{img/computing/mean_shift_random_segmentation.jpg}
	\end{figure}
	\begin{tcolorbox}[title=Remark,colframe=black,arc=10pt]
	The reader will be able to find a practical example in our \texttt{R} companion book!
	\end{tcolorbox}
	As the reader can guess intuitively, this method converges securely (it can be proved algebraically but it's quite boring!), on the other hand it converges infinitely so we must set a convergence terminal for the algorithm to stop its iterations.

	Let's move on to the computational aspect! We two different very common approaches.
	
	For the both approach, the first idea of the Mean Shift algorithm is to consider the workspace as a probability density (or multiple one!). Then each point of the workspace is identified by a coordinate vector $\vec{p}_i$ with a "mass" (corresponding mathematically to it's probability density function: the kernel!) given by the kernel:
	
	centered on $\vec{p}_0$. We have already meet such Kernels\footnote{Several types of kernel functions are commonly used: uniform, triangle, Epanechnikov, quartic (biweight), tricube, triweight, Gaussian, quadratic and cosine.} and we know what should be their properties. We won't come back here on these aspects! If the workspace contains a set of points, the latter are considered as a sample of an underlying probability density function and the dense regions (local maximum) as the modal value of the function of density of interest!
	
	Obvious a very common Kernel is the multivariate Gaussian one given for recall by (where the lowercase $k$ means that at each iteration $k$ the variance-covariance matrix is recomputed):
	
	In practice some algorithms introduce a constant $h$ such that the latter is written for a $d$-dimensional case:
	
	The first typical implementation of the Mean Shift clustering algorithm is based on the barycenter in Geometry (or center of gravity in physics). For this let us recall that we have proved in the section of Euclidean Geometry (page \pageref{barycenter}), that the barycenter was given by:
	
	This will be written in the context of Mean Shift with Gaussian kernel:
	
	The difference:
	
	is named the "\NewTerm{mean shift vector}" in the paper of Fukunaga and Hostetler. The mean-shift algorithm sets afterwards $\vec{p}_0^{[k]}\leftarrow m(\vec{p}_0)$, and repeats the estimation $k$ time $m(\vec{p}_0)^{[k]}$ converges.
	
	
	For the second typical possible implementation, consider again that $\{\vec{p}_i\}_{i=1\ldots N}$ is the position of the points and $K$ is a given kernel function (that weights the points) for a given iteration, then for each of the $N$ measurement points we have, we calculate the mean relatively to a reference point $\vec{p}_0$ (the center point of the kernel):
	
	The first step in this second method with this underlying density $f(\vec{p}_0, h)$ is to find the mode of this density! The mode is located at the zero of the gradient
$\vec{\nabla} f(\vec{p}_0, h)=\vec{0}$ and this second method is an elegant way to locate this zero!
	
	Indeed, let us compute the gradient:
	
	Now a traditional idea is to replace $\vec{p}_0$ in the last parenthesis thanks to gradient itself. Indeed, that latter leads us to:
	
	Hence after rearranging:
	
	So we get a fixed-point iteration, that brings us to write:
	
	\begin{tcolorbox}[title=Remark,colframe=black,arc=10pt]
	As the in the case of Gaussian Kernel we have $K'=K$ we then fall back on the result of the first method:
	
	\end{tcolorbox}
	What interest us is again the "\NewTerm{mean shift vector}" defined this time by:
	
	
	Whatever the method, the choice of bandwidth parameter $h$ or Kernel are critical. A large $h$ might result in incorrect clustering and might merge distinct clusters. A very small $h$ might result in too many clusters.

	Mean-shift looks very similar to $K$-Means, they both move the point closer to the cluster centroids. One may wonder: How is this different from $K$-Means? The key difference is that Mean shift does not require the user to specify the number of clusters. In some cases, it is not straightforward to guess the right number of clusters to use. In $K$-Means, the output may end up having too few clusters or too many clusters to be useful. At the cost of larger time complexity, Mean shift determines the number of clusters suitable to the dataset provided.

	Another commonly cited difference is that $K$-Means can only learn circle or ellipsoidal clusters. However, this is not true. The reason that Mean shift can learn arbitrary shapes is because the features are mapped to another higher dimensional feature space through the kernel. The arbitrary shapes are due to the algorithm finding circle or ellipsoidal clusters in higher dimensional feature space. When the features are mapped back to $1$D/$2$D/$3$D, the resulting clusters look like strange shapes. This is also the trick as used in Support Vector Machines.

	A traditional $K$-means does not use kernels, but Kernel $K$-means is available. Kernel $K$-Means is useful if:
	\begin{enumerate}
		\item The number of clusters is known or can be reasonably estimated
		
		\item Dataset needs learning non-ellipsoidal cluster shapes
	\end{enumerate}
	So, you can enjoy the better runtime complexity of K-Means and learn arbitrary clusters if you can determine the number of clusters to use.
	
	\paragraph{Hierarchical Ascendant Classification (HAC) Dendrograms}\mbox{}\\\\
	A dendrogram (from Greek dendro "tree" and gramma "drawing") is a tree diagram frequently used to illustrate the arrangement of the clusters produced by hierarchical clustering. Dendrograms are often used in computational biology to illustrate the clustering of genes or samples, sometimes on top of heatmaps.
	
	To introduce this clustering technique let us consider the following companion example based on the following list of data in Microsoft Excel:
	\begin{figure}[H]
		\centering
		\includegraphics{img/computing/dendrogram_excel_list.jpg}
		\caption[]{List of data for our study of dendrograms in Microsoft Excel 14.0.6123}
	\end{figure}
	We wish to have a hierarchical organization of likeness of individuals based on their income and their living space. 

	One possible technique is to define for this measurement a distance of similarity. For example, the Euclidean distance:
	
	is a special choice that will associate two individuals whose distance is minimal. We speak then in the area of clustering "\NewTerm{single linkage}\index{simple linkage}".
	\begin{figure}[H]
		\centering
		\includegraphics{img/computing/linkage.jpg}
		\caption{Type of links in Hierarchical Clustering}
	\end{figure}
	Formally:
	\begin{itemize}
		\item Maximum or complete linkage clustering:
		
	
		\item Minimum or single linkage clustering:
		
	
		\item Average linkage:
		
	\end{itemize}
	\begin{tcolorbox}[colframe=black,colback=white,sharp corners]
	\textbf{{\Large \ding{45}}Example:}\\\\
	How to measure the distance between two individuals considering:
	\begin{table}[H]
		\centering
		\begin{tabular}{|l|l|c|c|c|c|}
		\hline Id & Firstname & Age & Children & Shoe size & Height \\
		\hline 1 & Alain & 45 & 3 & 45 & 182 \\
		\hline 2 & Martine & 28 & 1 & 36 & 165 \\
		\hline 3 & Pierre & 22 & 0 & 43 & 172 \\
		\hline
		\end{tabular}
	\end{table}
	Who is Martine closest to? We will use the Euclidean distance.!
	\begin{itemize}
		\item Distance between Alain and Martine:
		$$
		d_{1 * 2}=\sqrt{(45-28)^{2}+(3-1)^{2}+(45-36)^{2}+(182-165)^{2}}
		$$
		
		\item Distance between Martine and Pierre:\\
		$$
		d_{2 * 3}=\sqrt{(28-22)^{2}+(1-0)^{2}+(36-43)^{2}+(165-172)^{2}}
		$$
		
		\item Distance between Alain and Pierre :
		$$
		d_{1 * 3}=\sqrt{(45-22)^{2}+(3-0)^{2}+(45-43)^{2}+(182-172)^{2}}
		$$
	\end{itemize}
	Table of distances:
	\begin{table}[H]
		\centering
		\begin{tabular}{|l|c|c|c|}
		\hline & Alain & Martine & Pierre \\
		\hline Alain & 0 & & \\
		\hline Martine & 25.74 & 0 & \\
		\hline Pierre & 25.33 & 11.61 & 0 \\
		\hline
		\end{tabular}
	\end{table}
	We notice that the two variables \textit{Size} and \textit{Children} have little weight in the distance calculation. To remedy this, we must center and reduce the data normally before computing distances!	
	\end{tcolorbox}
	So we can easily, using a spreadsheet software like Microsoft Excel 14.0.6123, create a "\NewTerm{distance matrix}\index{distance matrix}" or "\NewTerm{proximity matrix}\index{proximity matrix}" which is a symmetric matrix with zero in diagonal and that relatively to the list above will give us:
	\begin{figure}[H]
		\centering
		\includegraphics[scale=0.5]{img/computing/dendrogram_excel_distance_matrix.jpg}
		\caption[]{Dendrograms distance matrix in Microsoft Excel 14.0.6123}
	\end{figure}
	where we put in the cell \texttt{D4} the following Euclidean distance formula:
	\begin{center}
		\texttt{=SQRT((B4-D2)\string^2+(C4-D3)\string^2)}
	\end{center}
	we then drag this formula for the rest of the matrix to the cell \texttt{AA27}.

	Then we use the bottom-up method (ascendant agglomeration) where we combine the groups until there is only one group remaining (containing all data) and this is a very boring work to describe and to do manually in spreadsheet software with a table of the size given above. We will give a step by step small example later below.
	\begin{tcolorbox}[title=Remark,colframe=black,arc=10pt]
	In their implementation, hierarchical clustering algorithms can adopt one of two following methods:
	\begin{itemize}
		\item The "\NewTerm{top-down}" ("\NewTerm{descending}") or "\NewTerm{dividing method}" consists in starting with a large cluster consisting of the whole data set and then in each step, the successive splitting of the generated clusters is carried out until each cluster consists of only one individual.
	
		\item The "\NewTerm{bottom-up}" ("\NewTerm{ascending}") or "\NewTerm{agglomerative method}" reverses the previous logic, it starts with clusters each composed of an individual and then at each step, the clusters are associated in pairs to form other clusters which are in turn joined two to two until a cluster is obtained.
	\end{itemize}
	\end{tcolorbox}
	This work will give us in a tabular form following the detailed steps given in our Minitab 15.1.1.0 companion book (values are not rounded to hundredths unlike the small matrix given in the figure above):
	\begin{figure}[H]
		\centering
		\includegraphics[scale=1]{img/computing/dendrogram_minitab_summary_list.jpg}
		\caption[]{Dendrograms distance summary list in Minitab 15.1.1.0}
	\end{figure}
	where the level of similarity of the linked group $i,j$ is defined empirically by:
	
	Thus, for the first row we have for example:
	
	The preceding list is more pleasant to analyse if, as is customary, we represent is a "dendrogram" as given by Minitab 15.1.1.0 below:	
	\begin{figure}[H]
		\centering
		\includegraphics[scale=1]{img/computing/dendrogram_minitab_plot.jpg}
		\caption{Dendrograms plot in Minitab 15.1.1.0}
	\end{figure}
	If you follow the detailed steps given in our MATLAB™ 2013a companion book you will get:
	\begin{figure}[H]
		\centering
		\includegraphics[scale=1]{img/computing/dendrogram_matlab_plot.jpg}
		\caption{Dendrograms plot in MATLAB™ 2013a }
	\end{figure}
	If you follow the detailed steps given in our \texttt{R} companion book you will get:
	\begin{figure}[H]
		\centering
		\includegraphics[scale=1]{img/computing/dendrogram_r_plot.jpg}
		\caption{Dendrograms plot in \texttt{R} 3.0.2}
	\end{figure}
	and still with \texttt{R} (see the corresponding companion book) for the same data:
	\begin{figure}[H]
		\centering
		\includegraphics[scale=0.85]{img/computing/dendrogram_r_plot_heatmap.jpg}
		\caption{Dendrograms heatmap plot in \texttt{R} 3.0.2}
	\end{figure}
	Dendrograms and HAC are also used in biostatistics as show below (still with \texttt{R} but not described in the companion book yet):
	\begin{figure}[H]
		\centering
		\includegraphics[scale=0.65]{img/computing/dendrogram_r_plot_heatmap_microarray.jpg}
		\caption{Dendrograms heatmap microarray plot in \texttt{R} 3.0.2}
	\end{figure}
	Or in financial engineering to group similar times series (still with \texttt{R} and detailed steps given in the companion book):
	\begin{figure}[H]
		\centering
		\includegraphics[scale=0.65]{img/computing/dendrogram_r_plot_tsa.jpg}
		\caption{Dendrograms TSA microarray plot in \texttt{R} 3.0.2}
	\end{figure}
	Let's go now for a detailed step by step example! For this consider the following table:
	\begin{table}[H]
		\centering
		\begin{tabular}{|c|c|c|}
		\hline
		\rowcolor[HTML]{9B9B9B} 
		\multicolumn{1}{|l|}{\cellcolor[HTML]{9B9B9B}\textbf{Object}} & \multicolumn{1}{l|}{\cellcolor[HTML]{9B9B9B}\textbf{Weight}} & \multicolumn{1}{l|}{\cellcolor[HTML]{9B9B9B}\textbf{Length}} \\ \hline
		$\mathrm{a}$ & $12$ & $10$ \\ \hline
		$\mathrm{b}$ & $15$ & $25$ \\ \hline
		$\mathrm{c}$ & $30$ & $55$ \\ \hline
		$\mathrm{d}$ & $50$ & $100$ \\ \hline
		$\mathrm{e}$ & $35$ & $70$ \\ \hline
		$\mathrm{f}$ & $45$ & $70$ \\ \hline
		$\mathrm{g}$ & $35$ & $60$ \\ \hline
		\end{tabular}
	\end{table}
	All that remains is to determine the matrix of Euclidean distances. The following distance matrix is then obtained with the data set presented above:
	\begin{table}[H]
		\centering
		\begin{tabular}{|c|c|c|c|c|c|c|c|}
		\hline
		 & $\mathrm{a}$ & $\mathrm{b}$ & $\mathrm{c}$ & $\mathrm{d}$ & $\mathrm{e}$ & $\mathrm{f}$ & $\mathrm{g}$ \\ \hline
		$\mathrm{a}$ & - & $15.30$ & $48.47$ & $97.69$ & $64.26$ & $68.48$ & $55.04$ \\ \hline
		$\mathrm{b}$ & $15.30$ & - & $33.54$ & $82.76$ & $49.24$ & $54.08$ & $40.31$ \\ \hline
		$\mathrm{c}$ & $48.47$ & $33.54$ & - & $49.24$ & $15.81$ & $21.21$ & $7.07$ \\ \hline
		$\mathrm{d}$ & $97.69$ & $82.76$ & $49.24$ & - & $33.54$ & $30.41$ & $42.72$ \\ \hline
		$\mathrm{e}$ & $64.26$ & $49.24$ & $15.81$ & $33.54$ & - & $10.00$ & $10.00$ \\ \hline
		$\mathrm{f}$ & $68.48$ & $54.08$ & $21.21$ & $30.41$ & $10.00$ & - & $14.14$ \\ \hline
		$\mathrm{g}$ & $55.04$ & $40.31$ & $7.07$ & $42.72$ & $10.00$ & $14.14$ & - \\ \hline
		\end{tabular}
	\end{table}
	Obviously, the distance between a point and itself is $0$ for example $d(a, a)=0$ and the distance:
	
	For the next step, the closest individuals are detected and aggregated into a new cluster and the distance matrix is again calculated. This last one is based on the old one except that the distance between this new cluster formed and the other objects or individuals (or clusters since after $2$ iterations, it is possible that clusters already aggregate between them) are recalculated according to the defined method (Single linkage or Complete linkage, ... etc).
	
	On the basis of the previous distance matrix, it can be seen that $\mathrm{c}$ and $\mathrm{g}$ are the closest $(7.07)$, then they are aggregated. The distance between the cluster $\{\mathrm{c}, \mathrm{g}\}$ and the other objects is updated according to the Complete Linkage method (another method could have been used as we have already mention it):
	\begin{table}[H]
		\centering
		\begin{tabular}{|c|c|c|c|c|c|c|}
		\hline
		 & $\mathrm{a}$ & $\mathrm{b}$ & $\{\mathrm{c},\mathrm{g}\}$ & $\mathrm{d}$ & $\mathrm{e}$ & $\mathrm{f}$ \\ \hline
		$\mathrm{a}$ & - & $15.30$ & $\color{red}{55.04}$ & $97.69$ & $64.26$ & $68.48$ \\ \hline
		$\mathrm{b}$ & $15.30$ & - & $\color{red}{40.31}$ & $82.76$ & $49.24$ & $54.08$ \\ \hline
		$\{\mathrm{c},\mathrm{g}\}$ & $\color{red}{55.04}$ & $\color{red}{40.31}$ & - & $\color{red}{49.24}$ & $\color{red}{15.81}$ & $\color{red}{21.21}$ \\ \hline
		$\mathrm{d}$ & $97.69$ & $82.76$ & $\color{red}{49.24}$ & - & $33.54$ & $30.41$ \\ \hline
		$\mathrm{e}$ & $64.26$ & $49.24$ & $\color{red}{15.81}$ & $33.54$ & - & $10.00$ \\ \hline
		$\mathrm{f}$ & $68.48$ & $54.08$ & $\color{red}{21.21}$ & $30.41$ & $10.00$ & - \\ \hline
		\end{tabular}
	\end{table}
	As the method of the complete linkage or the maximum linkage defines the distance between cluster based on the most distant points, then one can calculate:
	
	On the basis of the updated matrix above, a close distance between $\mathrm{f}$ and $\mathrm{e}$ is detected since their distance is the smallest and equal to $10$. Again, a cluster $\{\mathrm{e}, \mathrm{f}\}$ is constructed and their distance from the others will also be readjusted according to maximum linkage:
	\begin{table}[H]
		\centering
		\begin{tabular}{|c|c|c|c|c|c|}
		\hline
		 & $\mathrm{a}$ & $\mathrm{b}$ & $\{\mathrm{c},\mathrm{g}\}$ & $\mathrm{d}$ & $\{\mathrm{e},\mathrm{f}\}$ \\ \hline
		$\mathrm{a}$ & - & $15.30$ & $55.04$ & $97.69$ & $\color{red}{68.48}$ \\ \hline
		$\mathrm{b}$ & $15.30$ & - & $40.31$ & $82.76$ & $\color{red}{54.08}$ \\ \hline
		$\{\mathrm{c},\mathrm{g}\}$ & $55.04$ & $40.31$ & - & $49.24$ & $\color{red}{21.21}$ \\ \hline
		$\mathrm{d}$ & $97.69$ & $82.76$ & $49.24$ & - & $\color{red}{33.54}$ \\ \hline
		$\{\mathrm{e},\mathrm{f}\}$ & $\color{red}{68.48}$ & $\color{red}{54.08}$ & $\color{red}{21.21}$ & $\color{red}{33.54}$ & - \\ \hline
		\end{tabular}
	\end{table}
	As the method of the Complete linkage or the maximum linkage defines the distance between cluster based on the most distant points, then one can calculate:
	
	This time it is $\mathrm{a}$ and $\mathrm{b}$ that will be clustered at a distance of $15.3$ and the matrix after updating the distances becomes this:
	\begin{table}[H]
		\centering
		\begin{tabular}{|c|c|c|c|c|}
		\hline
		 & $\{\mathrm{a},\mathrm{b}\}$ & $\{\mathrm{c},\mathrm{g}\}$ & $\mathrm{d}$ & $\{\mathrm{e},\mathrm{f}\}$ \\ \hline
		$\{\mathrm{a},\mathrm{b}\}$ & - & $\color{red}{55.04}$ & $\color{red}{97.69}$ & $\color{red}{68.48}$ \\ \hline
		$\{\mathrm{c},\mathrm{g}\}$ & $\color{red}{55.04}$ & - & $49.24$ & $21.21$ \\ \hline
		$\mathrm{d}$ & $\color{red}{97.69}$ & $49.24$ & - & $33.54$ \\ \hline
		$\{\mathrm{e},\mathrm{f}\}$ & $\color{red}{68.48}$ & $21.21$ & $33.54$ & - \\ \hline
		\end{tabular}
	\end{table}
	The following matrix will give what we have for distances after that  $\{\mathrm{c}, \mathrm{g}\}$ and $\{\mathrm{e}, \mathrm{f}\}$ are grouped to another cluster at a distance of $21.21$:
	\begin{table}[H]
		\centering
		\begin{tabular}{|c|c|c|c|}
		\hline
		 & $\{\mathrm{a},\mathrm{b}\}$ & $\left\lbrace\{\mathrm{c},\mathrm{g}\},\{\mathrm{e},\mathrm{f}\}\right\rbrace$ & $\mathrm{d}$ \\ \hline
		$\{\mathrm{a},\mathrm{b}\}$ & - & $\color{red}{55.04}$ & $97.69$ \\ \hline
		$\left\lbrace\{\mathrm{c},\mathrm{g}\},\{\mathrm{e},\mathrm{f}\}\right\rbrace$ & $\color{red}{55.04}$ & - & $\color{red}{49.24}$ \\ \hline
		$\mathrm{d}$ & $97.69$ & $\color{red}{49.24}$ & - \\ \hline
		\end{tabular}
	\end{table}
	And then it is the cluster group $\left\lbrace \{\mathrm{c}, \mathrm{g}\}, \{\mathrm{e}, \mathrm{f}\}\right\rbrace$ and $\mathrm{d}$ that will be grouped at a distance of $49.24$, updating the new distances gives:
	\begin{table}[H]
		\centering
		\begin{tabular}{|c|c|c|}
		\hline
		 & $\{\mathrm{a},\mathrm{b}\}$ & $\left\lbrace\{\mathrm{c},\mathrm{g}\},\{\mathrm{e},\mathrm{f}\},\mathrm{d}\right\rbrace$ \\ \hline
		$\{\mathrm{a},\mathrm{b}\}$ & - & $\color{red}{97.69}$ \\ \hline
		$\left\lbrace\{\mathrm{c},\mathrm{g}\},\{\mathrm{e},\mathrm{f}\},\mathrm{d}\right\rbrace$ & $\color{red}{97.69}$ & - \\ \hline
		\end{tabular}
	\end{table}
	The algorithm ends at the next iteration when $\{\mathrm{a}, \mathrm{b}\}$ and the cluster group $\left\lbrace \{\mathrm{c}, \mathrm{g}\}, \{\mathrm{e}, \mathrm{f}\},\mathrm{d}\right\rbrace$ will form the root cluster at a distance of $97.69$.
	
	This give us finally:
	\begin{figure}[H]
		\centering
		\includegraphics[scale=1]{img/computing/cluster_dendrogram.jpg}
	\end{figure}
	Hierarchical clustering offers the advantage of presenting hierarchical groups so that the resulting ramifications are easy to read. In contrast to the $K$-means, it does not require the definition of the number of clusters in advance and it is reproducible in the sense that the choice of the initial cluster centers is not random. But in return it is more expensive in time when it comes to implementing it for a large database.
	
	This is why in practice, when dealing with huge databases, before doing a HAC, we first run a $K$-Means and after we do a HAC on the resulting $K$-Means (therefore the bottom of the dendrogram tree will be obtain with a $K$-mean and the upper part with the HAC).
	
	 Obviously as for the $k$-nn and the $K$-means, we may run the HAC on the factorial space obtained after having run a PCA (Principal Component Analysis as seen on page \pageref{principal component analysis}) on our data (see our \texttt{R} companion book for an application example). We can also even reduce the dimensions using that latter if necessary!
	\begin{figure}[H]
		\centering
		\includegraphics[scale=1]{img/computing/hac_factormap.jpg}
		\caption[HAC on a factor map]{HAC on a factor map with R}
	\end{figure}
	\begin{tcolorbox}[title=Remark,colframe=black,arc=10pt]
	To do a classification on qualitative data we will use typically a Multiple Correspondence Analysis (MCA).
	\end{tcolorbox}
	
	\pagebreak
	\subsubsection{Dimensionality Reduction}
	"\NewTerm{Dimensionality reduction}\index{dimensionality reduction}" or "\NewTerm{dimension reduction}\index{dimension reduction}" is the process of reducing the number of random variables under consideration.
	
	More technically, dimensionality reduction is typically choosing a basis or mathematical representation within which you can describe most but not all of the variance within your data, thereby retaining the relevant information, while reducing the amount of information necessary to represent it.
	
	Approaches can be divided into feature selection and feature extraction.
	
	\textbf{Definitions (\#\mydef):}
	\begin{itemize}
		\item[D1.] "\NewTerm{Feature selection}\index{Feature selection}" approaches try to find a subset of the input variables (also called features or attributes). The three strategies are: the filter strategy (e.g. information gain), the wrapper strategy (e.g. search guided by accuracy), and the embedded strategy (selected features add or are removed while building the model based on prediction errors).
		Typical feature section methods are:
		\begin{itemize}
			\item Missing Value Ratio
			\item Low Variance Filter
			\item High Correlation Filter
			\item Random Forest
			\item Backward Feature Extraction
			\item Forward Feature Selection
			\item ...
		\end{itemize}
		
		\item[D2.] "\NewTerm{Feature extraction}\index{feature extraction}"  transforms the data from the high-dimensional space to a space of fewer dimensions. The data transformation may be linear, as in Principal Component Analysis (we have already study quite in detail the naive version of Principal Component Analysis in the section Statistics page \pageref{principal component analysis}), but many non-linear dimensionality reduction techniques also exist.
		
		Typical feature extraction methods are:
		\begin{itemize}
			\item Components/Factor based:
			\begin{itemize}
				\item Factor Analysis
				\item Principal Component Analysis
				\item Independent Component Analysis
				\item ...
			\end{itemize}
			\item Projection based:
			\begin{itemize}
				\item ISOMAP
				\item $T$-SNE
				\item UMAP
				\item ...
			\end{itemize}
		\end{itemize}
	\end{itemize}
	
	\paragraph{Principal Component Analysis}\mbox{}\\\\
	As already mentioned earlier, this subject is treated quite with details in the section of Statistics at page \pageref{principal component analysis}. 
	
	So let us focus on the second most common component/factor based dimensionality reduction technique:
	
	\pagebreak
	\paragraph{MultiDimensional Scaling (MDS)}\mbox{}\\\\
	The goal of "\NewTerm{multidimensional scaling}\index{multidimensional scaling}\label{multidimensional scaling}" (MDS) is to visualize a set of $N$ objects based on their similarities measured in $n$ different aspects, and the "\NewTerm{classical MDS}\index{classical MDS}" (cMDS), also known as "\NewTerm{principal coordinates analysis}\index{principal coordinates analysis}" (PCoA), is one of the methods for MDS but that adds the possibility to reduce the space dimension of the original set.
	
	Let us introduce the first aspect of MDS. Being known the level of similarity of individual cases of a dataset (without knowing their intrinsic characteristics!), the purpose of MDS is to found a way of visualizing the original intrinsic unknown characteristics. In other words, MDS is used to translate information about the pairwise distances among a set of $N$ objects or individuals into a configuration of $N$ points mapped into an abstract Cartesian space.
	
	Formally, given a pairwise similarity matrix $D_x=[d^x_{ij}]_{N\times N}$ (\SeeChapter{see section Linear Algebra page \pageref{distance matrix}}) of a set of $N$ objects ($N$ data points in an $n$-dimensional space) $X=[\vec{x}_1,\ldots,\vec{x}_N]$ but of unknown components, where the $ij$-th component $d^x_{ij}$ is some measurement of the similarity between $\vec{x}_i$ and $\vec{x}_j$ such as the Euclidean distance $d^x_{ij}=\vert\vert\vec{x}_i-\vec{x}_j\vert\vert _2$ between $\vec{x}_i$ and $\vec{x}_j$, we want to be able to construct a map $Y=[\vec{y}_1,\ldots,\vec{y}_N]$ in an $n$ dimensional space that explains well enough the original $D_x$ by reversing the original unknown components of the $X=[\vec{x}_1,\ldots,\vec{x}_N]$. When $n=2$ or $3$, the resulting dataset $Y$ can be visualized and the corresponding chart is then name a "\NewTerm{"perceptual plot}.
	
	So the idea is given the only known dissimilarity matrix $D_x=[d^x_{ij}]_{N\times N}$ to found a best estimate of the original unknown $X=[\vec{x}_1,\ldots,\vec{x}_N]$, that we will denote $Y=[\vec{y}_1,\ldots,\vec{y}_N]$. These estimates will obviously lead to a dissimilarity matrix that we will denote $D_y=[d^y_{ij}]_{N\times N}$.
	
	Our goal again is then to find $Y=\left[\vec{y}_{1}, \ldots, \vec{y}_{N}\right]$ so that its pairwise similarity matrix $D_{y}$ matches $D_{x}$ optimally, in the sense that the following objective function, named sometimes "strain" in the MDS study field, is minimized:
	
	with obviously all $d^x_{ij}$ who are given, but where the given estimated pairwise similarity of Euclidean distance:
	
	have to be determined.
	
	\begin{tcolorbox}[title=Remark,colframe=black,arc=10pt]
	This optimization problem is also often denoted:
	
	with $D_{x}^{2}$ given and $D_{\hat{x}}$ to be found when calculated as:
	
	\end{tcolorbox}
	The optimisation problem above is often written explicitly as (but there are other empirical choices as we already know that leads to different final results):
	
	otherwise it doesn't make really sense.
	
	This relation is often normalized (to have values that have non physical units) and root squared (it's more convenient to deal with distance)and a special name is given to it, the "\NewTerm{Kruskal's Stress coefficient}\index{Kruskal's Stress coefficient}" and denoted:
	
	\begin{tcolorbox}[title=Remark,colframe=black,arc=10pt]
	Some computer softwares don't normalize, nor take the square root, that's why the objective final value function may differ between softwares.
	\end{tcolorbox}
	
	\subparagraph{Classical MDS (cMDS)}\mbox{}\\\\
	Given again a pairwise similarity matrix $D_x=[d^x_{ij}]_{N\times N}$ of a set of $N$ objects ($N$ data points in an $n$ dimensional space) $X=[\vec{x}_1,\ldots,\vec{x}_N]$, where the $ij$-th component $d^x_{ij}$ is some measurement of the similarity between $\vec{x}_i$ and $\vec{x}_j$ such as the Euclidean distance $d^x_{ij}=\vert\vert\vec{x}_i-\vec{x}_j\vert\vert _2$ between $\vec{x}_i$ and $\vec{x}_j$, we can construct a map $Y=[\vec{y}_1,\ldots,\vec{y}_N]$ in an $m$ dimensional space ($m<n$) so that its pairwise distance matrix $D_y=[d^y_{ij}]_{N\times N}=\vert\vert\vec{y}_i-\vec{y}_j\vert\vert _2$ explains well enough the original $D_x$. As $m<n$, cMDS can be considered as a method for dimension reduction. 
	
	In some cases only the similarities $d^x_{ij}\;(i,j=1,\ldots,N)$ are given, while the $N$ objects in $X$ are not explicitly given, and the dimensionality $n$ may not even be known!

	We first assume the given pairwise similarity is the Euclidean distance $d_{ij}=\vert\vert\vec{x}_i-\vec{x}_j\vert\vert _2$:
	
	and we build the following similarity matrix:
	
	where we have defined the following three matrices to simplify the notations:
	
	We then introduce a centering matrix (for recall $\mathds{J}$ is a matrix of ones as already defined in the section of Linear Algebra):
	
	 which is an $N \times N$ and symmetric $\mathrm{C}_{N}^{T}=\mathrm{C}_{N}$.
	 
	 Pre-multiplying $C_{N}$ by a column vector $\vec{a}=\left[a_{1}, \ldots, a_{N}\right]^{T}$ removes the mean $\bar{a}=\sum_{i=1}^{N} a_{i} / N$ of from each component of $\vec{a}$ :
	
	Taking transpose on both sides, we see that post-multiplying $\mathrm{C}_{N}$ by a row vector $\vec{a}^{T}$ removes the mean of $\vec{a}^{T}$ from each component:
	
	We now apply double centering to $D_{x}^{2}$ by pre and post multiplying $C_{N}$ to $D_{x}^{2}$:	
	
	where $X_{r} C_{N}=\mathds{O}$ as all components of each row of $X_{r}$ are the same as their mean, and removing the mean results in a zero row vector. Similarly, $C_{N} X_{c}=\mathds{O}$, and we have also defined just above:
		
	with the mean of each row of $X$ removed, i.e., the $i$th column is:
	

	We therefore see that $\sum_{i=1}^{N} \bar{x}_{i}=0$, ie., all vectors in $\bar{X}=\left[\bar{x}_{1}, \ldots, \bar{x}_{N}\right]$ are those in $X=\left[\vec{x}_{1}, \ldots, \vec{x}_{N}\right]$ shifted by the mean vector $\bar{X}$. Now the points in $\bar{X}$ are centralized, but their pairwise similarities remain the same as the given $d_{i j}^{x}$.
	
	The relation:
	
	that we have just proved earlier can be rearranged to get:
	
	that is Gram matrix (\SeeChapter{see section Linear Algebra page \pageref{Gram matrix}}) of $\overline{X}$.
	
	Our goal is to find $Y=\left[y_{1}, \cdots, y_{N}\right]$ so that its pairwise similarity matrix $D_{y}$ matches $D_{x}$ optimally, in the sense that
the following objective function is minimized:
	
	Assuming $Y$ is also a double centered matrix as well as $X$, we get its Gram matrix:
	
	and the objective function above can be redefined as:
	
	We now carry out eigenvalue decomposition (\SeeChapter{see section Linear Algebra page \pageref{orthogonal decomposition}}) of the $N \times N$ symmetric matrix $B_{x}$ to find its real eigenvalue matrix $\Lambda$ and the orthogonal eigenvector matrix $V$ satisfying $B_{x} V= \Lambda V$, ie:
	
	To minimize the objective function $o(Y)=\left\|B_{y}-B_{x}\right\|$, we let:	
	
	Each column $\vec{y}_i$ in $Y=[\vec{y}_1,\ldots,\vec{y}_N]$ is an $n$ dimension vector. To reduce the dimensionality from $n$ to a given chosen $m$, we take the first $m$ rows of $Y$ corresponding to the $m$ greatest eigenvalues of $\lambda_1\ge\ldots\ge\lambda_m\ge\ldots\ge\lambda_N$ and their corresponding eigenvectors $\vec{v}_1,\ldots,\vec{v}_m$ as we do in Principal Component Analysis (\SeeChapter{see section Statistics page \pageref{principal component analysis}}).
	
	\begin{tcolorbox}[title=Remark,colframe=black,arc=10pt]
	In practice it happens sometimes that only $D_x^2$ is given without the original coordinates of the $N$ data points $X=[\vec{x}_1,\ldots,\vec{x}_N]$. Also the objective function may have different empirical expression and the distance may be another one than the classic Euclidean.
	\end{tcolorbox}
	
	In summary, here are the different steps:
	\begin{itemize}
		\item Given the pairwise similarity, an $N\times N$ matrix $D_x=[d_{ij}]$, construct the squared proximity matrix $D^2_x=[d^2_{ij}]$
		
		\item Apply double centering to get $B_x=-C_ND^2_xC_N$;
		
		\item Find the $m$ greatest eigenvalues $\lambda_1,\ldots,\lambda_m$ of $B_x$, and the corresponding eigenvectors $\vec{v}_1,\ldots,\vec{v}_m$;
		
		\item Get the map:

		

	\end{itemize}
	\begin{tcolorbox}[title=Remark,colframe=black,arc=10pt]
	For a practical application the reader can take a look to our \texttt{R} or MATLAB™ companion books.
	\end{tcolorbox}
	
	So let us focus on the third most common component/factor based dimensionality reduction technique:
	
	\pagebreak
	\paragraph{Factor Analysis (Exploratory Factor Analysis (EFA))}\mbox{}\\\\
	The method of "\NewTerm{Factor Analysis}\index{factor analysis}\label{factor analysis}" (FA), also named "\NewTerm{Exploratory Factor Analysis}\index{exploratory factor analysis}\label{exploratory factor analysis}" (EFA), not to be confuse with the method of Chi-square Correspondence Factor Analysis (\SeeChapter{see section Statistics page \pageref{chi-square correspondence factor analysis}}), models a set of $D$ observed manifest variables in $\vec{x}=[x_1,\ldots,x_D]^T$ as a linear combination of a set of $d<D$ unobserved hidden "\NewTerm{latent variables}" or "\NewTerm{common factors}" in $\vec{z}=[z_1,\ldots,z_d]^T$, to explain and reveal the variability and dependency among the $D$ observed variables, typically correlated, in terms of the latent variables, assumed to be independent and therefore uncorrelated. The method of FA is therefore considered as a means for dimensionality reduction\footnote{The text that follows is completely inspired from the lecture notes of the professor Ruye Wang (\url{http://fourier.eng.hmc.edu/e176/lectures/index_e176.html})}!
	
	Specifically, we assume each of the observed variables in $\vec{x}$ is a linear combination of the $d$ factors in $\vec{z}$:
	
	with $i=1,\ldots,D$ or in matrix form:
	
	where:	
	
	is a $D\times d$ factor loading matrix, and $\varepsilon_i$ is the noise associated with $x_i$.  
	\begin{tcolorbox}[title=Remark,colframe=black,arc=10pt]
	In many textbooks the reader can found the following notation:
	
	where the $l_{ij}$ are named the "loadings" and the $F_j$ the "latent factors".
	\end{tcolorbox}
	
	\begin{tcolorbox}[colframe=black,colback=white,sharp corners]
	\textbf{{\Large \ding{45}}Example:}\\\\
	Let's approach this technique by going directly with an example. For this, consider that $5$ candidates have passed $3$ exams in the field of Finance, Statistics and Standards (Norms) to validate their entry into an MBA course. Each of the corrected exams is marked on a scale ranging from $0$ to $10$ points corresponding to the variables to be explained. The data is summarized in the table below:
	\begin{table}[H]
		\centering
		\begin{tabular}{|l|l|l|l|}
		\hline Candidate n $^{\circ}$ & Finance $^{y_{1}}$ & Statistics $^{y_{2}}$ & Norms $y_{3}$ \\
		\hline 1 & 3 & 6 & 5 \\
		\hline 2 & 7 & 3 & 3 \\
		\hline 3 & 10 & 9 & 8 \\
		\hline 4 & 3 & 9 & 7 \\
		\hline 5 & 10 & 6 & 5 \\
		\hline
		\end{tabular}
	\end{table}
	It was suggested by an expert committee that the results obtained are dependent on two explanatory variables not directly observable, $f_1$ and $f_2$, which are respectively judged as being the \textit{Analytical Capacity} and the \textit{Memorization Capacity} of the candidates. It is accepted that each of the explained variables depends linearly on these two factors such as (this is a notation common in french textbooks for EFA):
	
	For information this type of model is represented in the framework of the Structural Equation Modelling (we will introduce this type of model later) in the following way (disregarding the constant and the error terms):
	\begin{figure}[H]
		\centering
		\includegraphics[scale=0.5]{img/arithmetics/efa_sem_example.jpg}
	\end{figure}
	With the following assumptions:
$$\text{E}(\varepsilon_i)=0,\text{V}(\varepsilon_i)=\sigma^2_{\varepsilon_i},\text{cov}(\varepsilon_i,\varepsilon_j)=0\; \forall i\neq j$$
	and:
	$$\text{E}(f_i)=0,\text{V}(f_i)=1,\text{cov}(f_i,f_j)=0\;  \forall i\neq j$$
	\end{tcolorbox}
	
	\begin{tcolorbox}[colframe=black,colback=white,sharp corners]
	The model is generally written in french textbooks in the following matrix form, using our particular example:
	
	\end{tcolorbox}
	Also, for simplicity and without loss of generality, we assume the dataset has a zero mean. If $W$ were available, the $d$ factors in $\vec{z}$ can be found by solving this over-determined linear equation system of $D$ equations but $d<D$ unknowns by the least-square method as seen at page \pageref{information matrix} (with minimum squared error $\vert\vert\vec{\varepsilon}\vert\vert^2$):
	
	where for recall $W^-=(W^TW)^{-1}W^T$ is the left pseudo-inverse of $W$.
	
	However, as neither $W$ nor $\vec{z}$ is available, we need to estimate both of them at the same time, based on a set of $N$ observed data points $X=[\vec{x}_1,\ldots,\vec{x}_N]$, typically $N\gg d$, by the method of expectation-maximization (EM), which in general is an iterative algorithm to find the maximum likelihood (ML) or maximum a posteriori (MAP) estimates of some model parameters.
	
	We treat both $\vec{z}$ and $\vec{\varepsilon}$ as random vectors, and make the following assumptions:
	\begin{itemize}
		\item The latent variables in $\vec{z}$ are of zero mean, independent of each other, and with unity variance:
		
		and they are Normally distributed:
		
		
		\item The noise components in $\vec{\varepsilon}$ are of zero mean, independent of each other, and with a diagonal covariance matrix:
		
		and they are also normally distributed:
		
		
		\item The latent variables and the noise are independent of each other:
		
	\end{itemize}
	The two matrices $W$ and $\Psi$ defined above are the parameters of the FA model, generally denoted by $\theta=\{W,\Psi\}$.
	
	As a linear combination of the two Gaussian vectors $\vec{z}$ and $\vec{\varepsilon}$ then:
	
	 also has a Gaussian distribution. We first find its mean $\text{E}(\vec{x})$ and variance-covariance matrix $\Sigma_x$ (see page \pageref{normal multivariate distribution}):
	
	
	\begin{tcolorbox}[title=Remark,colframe=black,arc=10pt]
	That last relation is often denoted in textbooks as:
	
	with therefore
	
	The portion of the variance of the $i$th variable contributed by the $m$ common factors is named the "\NewTerm{$i$th communality}\index{communality}". That portion of $\text{V}(x_i)=\sigma_{ii}$ due to the specific factor is often named the "\NewTerm{uniqueness}", or "\NewTerm{specific variance}\index{specific variance}". Denoting the $i$th communality by $h^2_i$, we see that:
	
	where:
	
	\end{tcolorbox}
	
	\begin{tcolorbox}[colframe=black,colback=white,sharp corners]
	\textbf{{\Large \ding{45}}Example:}\\\\
	Let us observe what are the implications of the aforementioned assumptions. To do this, let's write the general form of each of the linear relations from our previous example:
	
	If we calculate the variance, then we have using its properties:
	
	Hence in the general case:
	
	The "communality" is then the part of the variance of the variable which is explained purely by the factors of the explanatory variables of the model. The greater the communality, the better the predictive power of the factors !!

	Obviously, if the factors are perfect predictors, we have the specific variance $\sigma^2_{\varepsilon_i}$ which would then be zero.
	\end{tcolorbox}
	
	And then get the Normal distribution of $\vec{x}$ given $\vec{\theta}$:
	
	We can further find the joint distribution $P(\vec{z},\vec{x})$, also a Gaussian, which has a zero mean:
	
	and a covariance matrix $\Sigma$ composed of four submatrices:
	
	where he have immediately:
	
	Now the Normal distribution $P(\vec{z},\vec{x}\vert\vec{\theta})$ given model parameter $\vec{\theta}=\{W,\Psi\}$ can be expressed as:
	
	\begin{tcolorbox}[colframe=black,colback=white,sharp corners]
	\textbf{{\Large \ding{45}}Example:}\\\\
	Let us now calculate the covariance of a couple of variables $y_i,y_j$ from our previous example by first writing:
	
	We then have:
	
	From the definition of covariance we know that the constants will vanish. So we can focus on:
	
	We saw just previously that in the simple case where $ i = j $, we have in our particular case:
	
	Now let's deal with the case where $ i \neq j $. We then have:
	
	Since the factors are assumed to be independent random variables by assumption in the model, their covariance is zero. It is the same for the terms of errors. It remains then:
	
	We will make the assumption that the errors are random variables independent of the factors. So at the level of the covariances for the moment all the assumptions are then:
	
	So their covariance also becomes zero and it remains:
	\end{tcolorbox}
	
	\begin{tcolorbox}[colframe=black,colback=white,sharp corners]
	
	So in the general case where $i\neq j$ we finally have:
	
	We can then build the "theoretical matrix of the variances-covariances of the factorial model" in the particular case which serves as an example, we have:
	
	Notice that if we disregard the error term, it comes in our particular case:
	
	In practice, we can easily calculate the "experimental variance-covariance matrix" of the measured data. We usually note it in the case of three variables to be explained:
	
	Then we just have to put the two matrices in correspondence (well normally it is customary to denote by $b_{ij}$ the estimated values of the $\beta_{ij}$ but we will suppose that the reader will be able to easily differentiate the moment when we work with the estimators or the actual theoretical values):
	
	If we go back to our original example, then we have:
	
	\end{tcolorbox}
	
	\begin{tcolorbox}[colframe=black,colback=white,sharp corners]
	
	We will then see a little further that we can deduce all the saturations (coefficients of the model) from this is positive definite matrix!
	\end{tcolorbox}

	Now we have two typical ways to determine the parameters of interest! First an iterative way based on the Expectation-Maximization (EM) algorithm that gives us the possibility to determine $W$ and $\Psi$ and another one, named "Principal Factor Analysis Extraction", based on the eigenvalue spectral decomposition (same idea as Principal Component Analysis) to determine only $W$.
	
	\subparagraph{Expectation-Maximization (EM) algorithm solution}\mbox{}\\\\
	Based on this joint Normal distribution, we can further find the following marginal distributions (see properties of Normal distributions at page \pageref{marginal and conditional distributions of multivariate Normal distributions}):
	\begin{itemize}
		\item First:
		
		with:
		
		
		\item Secondly:
		
		with:
		
		where we have defined:
		
	\end{itemize}
	
	Note that while $P(\vec{z})={\cal N}({\vec{0}},\mathds{1})$ has zero mean and diagonal covariance, the conditional distribution $P(\vec{z}\vert\vec{x},\vec{\theta})={\cal N}(\vec{\mu}_{z\vert x},\Sigma_{z\vert x})$ has non-zero mean $\vec{\mu}_{z\vert x}$ and non-diagonal covariance $\Sigma_{z\vert x}$!
	
	The computational complexity for the inversion of the $N\times N$ matrix $WW^T+\Psi$ is $\mathcal{O}(N^3)$. However, by applying the Woodbury matrix identity (\SeeChapter{see section Linear Algebra page \pageref{Woodbury matrix identity}}):
	
	where $\Psi$ as a diagonal matrix can be easily inverted, and $\mathds{1}+W^T\Psi^{-1}W$ is an $d \times d$ matrix that can be inverted with complexity $\mathcal{O}(d^3) \ll \mathcal{O}(N^3)$.
	
	The model parameters in $\theta=\{W,\Psi\}$ can be estimated based on an observed dataset $X$ by the EM algorithm (see page \pageref{EM algorithm}), an iterative process of the following two steps:
	\begin{itemize}
		\item The E-step:
		
		Find the expectation of the log likelihood function of the model parameters in ${\theta}$, to be maximized in the following M-step.

		Assuming all samples in the observed dataset $X=[\vec{x}_1,\ldots,\vec{x}_N]$ are independent and identically distributed (i.i.d.), we can find the "complete data likelihood\index{complete data likelihood}" function of ${\theta}$ (using the Bayes' relation established in the section of Probabilities page \pageref{bayes relation for complete data likelihood}):
		
		and the log-likelihood:
		
		The second term can be dropped as $P(\vec{z}| \theta)={\cal N}({\vec 0},\mathds{1})$ is independent of the model parameters in ${\theta}$ and therefore irrelevant to the maximization of the log likelihood with respect to ${\theta}$.

		The expectation of the log-likelihood function above (aka "expected complete data log-likelihood"), denoted by $Q$, with respect to the latent variable $\vec{z}$ is:
		
		Note that $\Psi$ is diagonal and therefore also symmetric, so is its own inverse $\Psi^{-1}$, we have:
		
		Also, note that $C=-DN\ln(2\pi)/2$ is a constant independent of $\vec{\theta}$ and is therefore dropped.
		
		\item The M-step:
		
		Find the optimal model parameters in $\theta=\{W,\Psi\}$ that maximize the expectation of the log-likelihood $Q$ obtained in the E-step.

		This is done by setting to zero the derivative of $Q$ with respective each of the two parameters in $\theta=\{W,\Psi\}$:
		\begin{itemize}
			\item The first derivative is:
			
			Solving for $W$ we get:
			
			where $\text{E}_{z\vert x_n}(\vec{z})$ and $\text{E}_{z\vert x_n}(\vec{z}\vec{z}^T)$ can be found in (see previously):
			
			Therefore:
			
			The second equation is due to the fact that $\Sigma_z=\text{E}(\vec{z}\vec{z}^T)-\vec{\mu}_z\vec{\mu}_z^T$. Here $\text{E}_{z\vert x_n}(\vec{z})$ can be considered as the estimation of the $\vec{z}$, while $\text{E}_{z\vert x_n}(\vec{z}\vec{z}^T)$ the uncertainty of the estimation.
			
			\item The second derivative (we use in this development derivatives of matrices as seen in the section of Linear Algebra page \pageref{derivative of logarithm of a determinant}):
			
		Solving for $\Psi$ we get:
		
		where $\text{diag}$ represents the operation that sets all off-diagonal elements of the right-hand side to zero, so that the resulting matrix $\Psi$ is guaranteed to be diagonal as required. We further replace $W$ in front of the last term above by that in:
		
		and get
		
		\end{itemize}
	\end{itemize}
	In summary, here is the list of steps in the EM algorithm for FA:
	\begin{enumerate}
		\item Initialize parameters ${\theta}_{\text{old}}=\{W_{\text{old}},\Psi_{\text{old}}\}$
		
		\item E-step: Based on ${\theta}_{\text{old}}$, find $\vec{\mu}_{z\vert x_n}$ and $\Sigma_{z\vert x_n}$ in:
		
		and then $\text{E}_{z\vert x}(\vec{z})$ and $\text{E}_{z\vert x}(\vec{z}\vec{z}^T)$ in:
		
		
		\item M-step: Find ${\theta}_{\text{new}}=\{W_{\text{new}},\Psi_{\text{new}}\}$ by evaluating:
		
		and:
		
		
		\item Terminate if convergence criterion is satisfied, otherwise replace ${\theta}_{\text{old}}$ by ${\theta}_{\text{new}}$ and return to step 2.
	\end{enumerate}
	
	\subparagraph{Principal Factor Analysis Extraction solution}\mbox{}\\\\
	Now, if we center and reduce the starting vectors containing the measured values, the variance-covariance matrix becomes the correlation matrix as we have proved it during our study of Principal Component Analysis (PCA)! This is why the PCA is the basis of the Factorial Analysis (exploratory) and we then say that we are doing a "\NewTerm{Principal Factor Analysis Extraction}" (PFAE).
	
	To understand the idea let us recall that we can write the Factor Analysis model in the following matrix form:
	
	If the data are centered and reduced, we will write:
	
	Now let's play cleverly with an orthogonal matrix $S$:
	
	It is thus an alternative notation for the initial model but as there is an infinity of orthogonal matrices, there is an infinity of alternative models.
	
	We denoted during our study of the Principal Component Analysis the matrix of the eigenvalues arranged in decreasing order in the diagonal by $\Lambda$ and by $S$ the matrix of the eigenvectors. This then gives us using the results of our study of the Principal Component Analysis and from what has just been seen:
	
	Using now the property of the transpose seen during our study of Linear Algebra, we have:
	
	Therefore:
	
	The matrix:
	
	therefore does the trick and it is customary to name the components of the matrix $L$ the "\NewTerm{factor saturations}" and it therefore contains the coefficients of the factors of the implicit linear model when the variables to be explained are centered and reduced. This is enough to give a qualitative idea of the weight of each of the factors.
	
	Some statistical software return what is are named the "\NewTerm{loadings}" or "\NewTerm{factor score}" and which corresponds to the expression:
	
	Whatever the chosen above solution, the factor model assumes that the $D(D+1)/2$ components of its covariance-variance matrix for $\vec{x}$ can be reproduced from the $D\cdot d$ factor loading (weights) $w_{ij}$ and the $D$ specific variances $\psi_D$. Because we have proved earlier above that:
	
	then when $D=d$ any covariance matrix $\Sigma_x$ can be reproduced exactly as $WW^T$, so $\Psi$ can be the zero matrix. It is when $d$ is small relative to $D$ that factor analysis is more useful. In this case, the factor model provides a simple explanation of the covariation in $\vec{x}$ with fewer parameters than $D(D+1)/2$ in $\Sigma_x$.
	
	Unfortunately for the factor analyst it is not easy to choose the number of latent variables. If the number of common factors is not determined by a priori considerations, such as by theory or the work of other researchers, the choice of $d$ can be based on the estimated eigenvalues of the principal component analysis.
	
	Given observations $x_1, x_2, \ldots , x_D$ on $d$ generally correlated variables, factor analysis seeks to answer the question: Does the factor model (linear in our case), with a small number of factors, adequately represent the data? In essence, we tackle this statistical model-building problem by trying to verify the covariance relations:
	
	If the off-diagonal elements of the estimator of $\Sigma_x$ are small, the variables are not related, and a factor analysis will not provide any useful evidence. If $\Sigma_x$ appears to deviate significantly from a diagonal matrix, then a factor model can be entertained, and the initial problem is one of estimating the factor loadings $w_{ij}$ and specific variables $\psi_i$.
	
	It should be also noticed that there is an infinity of solution. To understand why consider the following model (we use here again the traditional notation in french textbooks):
	
	The theoretical of variance-covariance matrix of the above factorial model is then:
	
	But the following model:
	
	also has exactly the same variance-covariance matrix of the factor model as the reader can quickly verify and is therefore also a solution! The reader will be able to verify that the application:
	
	for any angle always gives the same variance-covariance matrix (reason for which there is an infinity of possible models for the same theoretical matrix of variances-covariance of the factorial model). So, in the example above, we simply took an angle of $\theta = -\pi/4$ which gives for the first line of the model:
	
	and so on line by line with each pair of coefficients and with the same rotation matrix.
	
	Although factor analysis and principal component analysis are typically labelled as data-reduction techniques, there are significant differences between these two. The objective of principal component analysis is to reduce the number of variables to a few components such that each component forms a new variable and the retained components explain the maximum amount of variance in the data. The objective of factor analysis, on the other hand, is to search or identify the underlying factor(s) or latent constructs that can explain the intercorrelation among the variables. 
	
	\pagebreak
	\paragraph{$T$-distributed Stochastic Neighbor Embedding ($T$-SNE)}\mbox{}\\\\
	"\NewTerm{$T$-distributed Stochastic Neighbor Embedding}\index{T-distributed Stochastic Neighbor Embedding}\label{tsne}" ($T$-SNE) is a machine learning algorithm for visualization developed by Laurens van der Maaten and Geoffrey Hinton. It is a non-linear dimensionality reduction technique well-suited for embedding high-dimensional data for visualization in a low-dimensional space of two or three dimensions. Specifically, it models each high-dimensional object by a two- or three-dimensional point in such a way that similar objects are modelled by nearby points and dissimilar objects are modelled by distant points with high probability.

	\begin{tcolorbox}[title=Remark,colframe=black,arc=10pt]
	It is very important that the reader keep in mind that is designed only for visualizing a dataset in a low ($2$ or $3$) dimension space. We give it all the data we want to visualize all at once. It is not a general purpose dimensionality reduction tool!
	\end{tcolorbox}
	
	The $T$-SNE algorithm comprises two main stages. First, $T$-SNE constructs a probability distribution over pairs of high-dimensional objects in such a way that similar objects have a high probability of being picked while dissimilar points have an extremely small probability of being picked. Second, $T$-SNE defines a similar probability distribution over the points in the low-dimensional map, and it minimizes the Kullback–Leibler divergence (\SeeChapter{see section Statistical Mechanics page \pageref{kullback-leibler divergence}}) between the two distributions with respect to the locations of the points in the map. Note that while the original algorithm uses the Euclidean distance between objects as the base of its similarity metric, this should be changed as appropriate.

	$T$-SNE has been used for visualization in a wide range of applications, including computer security research, music analysis, cancer research, bioinformatics, and biomedical signal processing. It is often used to visualize high-level representations learned by an artificial neural network. Indeed, as the reader may already know, deep CNN networks are basically black boxes. There is no way to really interpret what's on deeper levels in the network. A common explanation is that deeper levels contain information about more complex objects. But that's not completely true, you can interpret it like that but data itself is just a high-dimensional noise for humans. But, with the help of $T$-SNE you can create maps to display which input data seams similar for the network.

	While $T$-SNE plots often seem to display clusters, the visual clusters can be influenced strongly by the chosen parametrization and therefore a good understanding of the parameters for $T$-SNE is necessary. Such "clusters" can be shown to even appear in non-clustered data, and thus may be false findings! Interactive exploration may thus be necessary to choose parameters and validate results. It has been demonstrated that $T$-SNE is often able to recover well-separated clusters, and with special parameter choices, approximates a simple form of spectral clustering.

	Given a set of $N$ high-dimensional objects $\vec{x} _{1},\ldots ,\vec{x} _{N}$, $T$-SNE first computes probabilities $p_{ij}$ that are proportional to the similarity of objects $\vec{x}_{i}$ and $\vec{x}_{j}$, defined as follows:
	
	Note that we have obviously $\sum _{j}p_{j\mid i}=1$ for all $i$.
	
	\begin{tcolorbox}[title=Remarks,colframe=black,arc=10pt]
	It seems that in general all SNE models are written in the form:
	
	where $g$ is a given function.
	\end{tcolorbox}
	
	As Van der Maaten and Hinton explained: "The similarity of datapoint $\vec x_{j}$ to datapoint $\vec x_{i}$ is the conditional probability, $p_{j|i}$, that $\vec x_{i}$ would pick $\vec x_{j}$ as its neighbour if neighbours were picked in proportion to their probability density under a Gaussian centered at $\vec x_{i}$."
	
	Now we define:
	
	and note that $p_{ij}=p_{ji}$, $p_{ii}=0$ and $\sum _{i,j}p_{ij}=1$.
	
	The bandwidth of the Gaussian kernels $\sigma_{i}$ is set in such a way that the perplexity of the conditional distribution equals a predefined perplexity using the bisection method to match a pre-specified perplexity value (Perp). The perplexity is $\text{Perp}(P_j)=2^{H(p_j)}$, where $H(P_j)=-\sum_j p_{i|j}\log(p_{i|j})$, and $\sigma_j$ is selected so that $\text{Perp}(P_j)=\text{Perp}$. As a result, the bandwidth is adapted to the density of the data: smaller values of $\sigma_{i}$ are used in denser parts of the data space.
	
	$T$-SNE is actually very sensitive to the Perplexity value. Let us see the following value:
	\begin{figure}[H]
		\centering
		\includegraphics[width=1.0\textwidth]{img/computing/tsne_perplexity_sensitivity.jpg}
		\caption{$T$-SNE perplexity sensitivity}
	\end{figure} 
	
	\begin{tcolorbox}[title=Remark,colframe=black,arc=10pt]
	There seems to be no standard way to choose the correct perplexity aside looking at the produced reduced dimension dataset and then assessing if it is meaningful. There are some general facts, eg. distances between clusters are mostly meaningless, small perplexity values encourage small clot-like structures but that's about it. A rule of thumb is to set the perplexity to $5\%$ of the dataset size.
	\end{tcolorbox}
	
	Since the Gaussian kernel uses the Euclidean distance $\lVert \vec x_{i}-\vec x_{j}\rVert$, it is affected by the curse of dimensionality, and in high dimensional data when distances lose the ability to discriminate, the $p_{ij}$ become too similar (asymptotically, they would converge to a constant). It has been proposed to adjust the distances with a power transform, based on the intrinsic dimension of each point, to alleviate this.
	
	$T$-SNE aims to learn a $d$-dimensional map $\vec{y}_{1},\ldots ,\vec{y}_{N}$ (with $\vec{y} _{i}\in \mathbb{R}^{d}$ that reflects the similarities $p_{i\mid j}$ as well as possible. To this end, it measures similarities $q_{ij}$ between two points in the map {$\vec{y}_{i}$ and $\vec{y}_{j}$, using a very similar approach. Specifically, $q_{j\mid i}$ is defined as:
	
	Herein a heavy-tailed Student $T$-distribution (with one-degree of freedom, which is the same as a Cauchy distribution) is used to measure similarities between low-dimensional points in order to allow dissimilar objects to be modelled far apart in the map. Note that also in this case we set $q_{ii}=0$.
	
	The locations of the points $\vec{y} _{i}$ in the map are determined by minimizing the (non-symmetric) Kullback–Leibler divergence (\SeeChapter{see section Statistical Mechanics page \pageref{kullback-leibler divergence}}) of the distribution $Q$ from the distribution $P$, that is:
	
	in which $P_i$ is the conditional probability distribution over all other datapoints given data-point $\vec x_i$, and $Q_i$ represents the conditional probability distribution over all other map points given map point $\vec y_i$. 
	
	The minimization of the Kullback–Leibler divergence with respect to the points $\vec{y}_{i}$ is performed using gradient descent. The result of this optimization should be a map that reflects the similarities between the high-dimensional inputs well.

	Let us see now how to derive the gradient of the KL divergence Loss function for the standard SNE and the $T$-SNE:

	\subparagraph{KL gradient of Stochastic Neighbour Embedding (SNE)}\mbox{}\\\\
	The definition of $q_{j | i}$ for SNE is different from that of the $T$-SNE (thanks to Federico Errica for providing the detailed developments):
	
	For what follows we will treat the one-dimensional case to simplify the notations (and we will also omit the SNE exponent).
	
	Notice that $E_{ij} = E_{ji}$. The loss function is defined as:
	
	We derive with respect to a $y_i$ (notice that the first term vanish a it depends only on $x_i$ or $x_j$ but not on any $y_i$ or $y_j$. To make the derivation less cluttered, we will omit the $\partial y_i$ term at the denominator:
	
	We start with the first term, noting that the derivative are obviously non-zero if and only if each term of the sums contains at least a $i$. Therefore we can reduce the two sums to only one with two terms (by symmetry):
	
	Since $\partial E_{ij}=E_{ij}(-2(y_i-y_j))$ we have:
	
	We conclude with the second term:
	
	 Since $\sum_{l \neq j} p_{l | j}=1$ and $Z_{j}$ does not
depend on $k,$ we can write (changing variable from $l$ to $j$ to make it more similar to the already computed terms):
	
	The derivative is non-zero only when $k=i$ or $j=i$ (also, in the latter case we can move $Z_{i}$ inside the summation because constant) therefore:
	
	Combining the previous and prior-previous relations we arrive at the final result:
	
	
	\subparagraph{KL gradient of $T$-distributed Stochastic Neighbour Embedding ($T$-SNE)}\mbox{}\\\\
	Let us de now the same job for $T$-SNE and for that the reader must remember:
	
	For what follows we will treat again the one-dimensional case to simplify the notations (and we will also omit the TSNE exponent).
	
	Notice that $E_{i j}=E_{j i}$. The loss function is defined as:
	
	We derive with respect to a $y_{i}$. To make the derivation less cluttered, we will again omit the $\partial y_{i}$ term at the denominator:
	
	We start with the first term, noting that the derivative is non-zero only when $\forall j$ , $k=i$ or $l=i,$ and also remembering that $p_{j\mid i}=p_{i\mid j}$ and $E_{j i}=E_{i j}$:
	
	since $\partial E_{i j}^{-1}=E_{i j}^{-2}\left(-2\left(y_{i}-y_{j}\right)\right)$ we have:
	
	We conclude with the second term. Using the fact that $\sum_{k, l \neq k} p_{k l}=1$ and that $Z$ does not depend on $k$ or $l$:
	
	Combining we arrive at the final result:
	
	Keep in mind that $T$-SNE learns a nonparametric mapping, which means that it does not learn an explicit function that maps data from the input space to the map!!! Therefore, it is not possible to embed test points in an existing map (although we could re-run $T$-SNE on the full dataset). A potential approach to deal with this would be to train a multivariate regressor to predict the map location from the input data. But if we are trying to apply $T$-SNE to "new" data, we are probably not thinking about our problem correctly, or perhaps simply we did not understand the purpose of $T$-SNE.
	
	\pagebreak
	\subsubsection{Gradient Boosting}
	In many supervised learning problems one has an output variable $y$ and a vector of input variables $\vec{x}$ described via a joint probability distribution $P(\vec{x},y)$ (at least for the purposes of theoretical analysis!). Using a training set $\{ (x_1,y_1), \dots , (x_n,y_n) \}$ of known values of $\vec{x}$ and corresponding values of $y$, the goal is to find an approximation $\hat{F}(x)$ to a function $F(x)$ that minimizes the expected value of some specified loss function $L(y, F(x))$:
	
	The "\NewTerm{gradient boosting method}\index{gradient boosting}\label{gradient boosting}" (that personally the author of these lines like better to name "\NewTerm{incremental iterative learning}") assumes a real-valued $y$ and seeks an approximation $\hat{F}(x)$ in the form of a weighted sum of functions $h_i (x)$ from some class $\mathcal{H}$ of weak learner:
	
	Keep in mind that the underlying idea is that instead of creating a single model, boosting combines multiple simple models into a single composite model. The idea is that, as we introduce more and more simple models, the overall model becomes stronger and stronger. And for reminder, in boosting terminology, the simple models are named "weak models" or "weak learners".
	
	In accordance with the empirical risk minimization principle, the method tries to find an approximation $\hat{F}(x)$ that minimizes the average value of the loss function on the training set, i.e., minimizes the empirical risk. It does so by starting with a model, consisting of a constant function $F_0(x)$, and incrementally expands it in a greedy fashion:
	
	where $h_m \in \mathcal{H}$ is a base learner function.
	
	Unfortunately, choosing the best function $h_i$ at each step for an arbitrary loss function $L$ is a computationally infeasible optimization problem in general (as least as far as we know for the moment...). Therefore, we  restrict our approach to a simplified version of the problem.
	
	The idea is to apply a steepest descent step to this minimization problem. If we considered the continuous case, i.e. where $\mathcal{H}$ is the set of arbitrary differentiable functions on $\mathbb{R}$, we would update the model in accordance with the following equations:	
	
	where the derivatives are taken with respect to the functions $F_i$ for $i \in \{ 1,..,m \}$.  In the discrete case however, i.e. when the set $\mathcal{H}$ is finite, we choose the candidate function $h$ closest to the gradient of $L$ for which the coefficient $\gamma$ may then be calculated with the aid of line search on the above equations. Note that this approach is a heuristic and therefore doesn't yield an exact solution to the given problem, but rather an approximation.
	
	In pseudocode, the generic gradient boosting method is:
	\begin{enumerate}[label*=\arabic*.]
		\item Training set $\{(x_{i},y_{i})\}_{i=1}^{n}$, a differentiable loss function $L(y,F(x))$, number of iterations $M$.
	
		\item For $m=1$ to $M$
		\begin{enumerate}[label*=\arabic*.]
			\item Compute so-called pseudo-residuals:
			
			
			\item Fit a base learner (e.g. tree) $h_{m}(x)$ to pseudo-residuals, i.e. train it using the training set $\{(x_{i},r_{im})\}_{i=1}^{n}$.
			
			\item Compute multiplier$\gamma _{m}$ by solving the following one-dimensional optimization problem:
			
			
			\item Update the model:
			
			
			\item Output $F_M(x)$
		\end{enumerate}
	\end{enumerate}
	Here is a very nice detailed example developed by antonioACR1 (we only have his pseudonyme sadly) on \href{https://datascience.stackexchange.com/questions/9134/gradient-boosting-algorithm-example}{Stackexchange} that is categorized as a special case of gradient boosting named: "\NewTerm{least-squares boosting algorithm}\footnote{As it is based on the usual squared error loss function!}\index{least-squares boosting algorithm}" (LS Boost).

	The example aims to predict salary per month (in dollars) based on whether or not the observation has own house, own car and own family/children. Suppose we have a dataset of three observations where the first variable is \textit{have own house}, the second is \textit{have own car} and the third variable is \textit{have family/children}, and target is \textit{salary per month}. The observations are:
	\begin{itemize}
		\item \{Yes,Yes,Yes,$1000$\}
		\item \{No,No,No,$25$\}
		\item \{Yes,No,No,$5000$\}
	\end{itemize}
	Choose a number $M$ of boosting stages, say $M=1$. The first step of gradient boosting algorithm is to start with an initial model $F_0$. This model is a constant defined by $\underset{\gamma}{\min}\sum_{i=1}^3 L(y_i,\lambda)$ in our case, where $L$ is the loss function. Suppose that we are working with the usual loss function:
	
	When this is the case, this constant is equal to the mean of the outputs $y_i$, so in our case:
	
	So our initial model is $F_0(x)=5008.3$ (which maps every observation $x$ (e.g. \{No,Yes,No\}) to $5008.3$.
	
	Next we should create a new dataset, which is the previous dataset but instead of $y_i$ we take the residuals:
	
	In our case, we have for the "\NewTerm{alignment accuracy residuals}\index{alignment accuracy residuals}":
	
	So our dataset becomes:
	\begin{itemize}
		\item \{Yes,Yes,Yes,$4991.6$\}
		\item \{No,No,No,$-4983.3$\}
		\item \{Yes,No,No,$-8.3$\}
	\end{itemize}
	The next step is to fit a base learner $h$ to this new dataset. Usually the base learner is a decision tree, so we use this.
	
	Now assume that we constructed the following decision tree $h$. We constructed this tree using entropy and information gain formulas (but maybe there is some mistake as the calculations were done by hand, however for our purposes we can assume it's correct!):
	\begin{figure}[H]
		\centering
		\includegraphics{img/computing/gradient_boosting_based_learner.jpg}
		\caption[]{Decision tree base learn for gradient boosting}
	\end{figure} 
	Let's call this decision tree $h_0$. The next step is to find a constant:
	
	Therefore, we want a constant $\lambda$ minimizing:
	
	This is where gradient descent comes in handy!
	
	Suppose that we start at $P_0=0$. Choose the learning rate equal to $\eta=0.01$. We have:
	
	Then our next value $P_1$ is given by:
	
	Repeat this step $N$ times, and suppose that the last value is $P_N$. If $N$ is sufficiently large and $\eta$ is sufficiently small then $\lambda:=P_N$ should be the value where:
	
	is minimized. If this is the case, then our $\lambda_0$ will be equal to $P_N$. Just for the sake of it, suppose that $P_N=0.5$ (so that $\sum_{i=1}^{3}L(y_{i},F_{0}(x_{i})+\lambda{h_{0}(x_{i})})$) is minimized at $\lambda:=0.5$). Therefore, $\lambda_0=0.5$.
	
	The next step is to update our initial model $F_0$ by $F_1(x):=F_0(x)+\lambda_0h_0(x)$. Since our number of boosting stages is just one, then this is our final model $F_1$.
	
	Now suppose that I want to predict a new observation $x=\{\text{Yes},\text{Yes},\text{No}\}$ (so this person does have own house and own car but no children). What is the salary per month of this person? We just compute $F_1(x)=F_0(x)+\lambda_0h_0(x)=5008.3+0.5\cdot 4991.6=7504.1$. So this person earns $7504.1$ per month according to our model.
	
	Here is an interesting abstract visualization that may help to understand this incremental iterative learning process (we don't know who is the author of this figure for sure but maybe it's Terence Parr):
	\begin{figure}[H]
		\centering
		\includegraphics[width=1.0\textwidth]{img/computing/gradient_boosting_illustrated_visualization.jpg}
		\caption{Illustrate visualization of the iterative concept of Gradient Boosting}
	\end{figure}
	Or a more explicit one would be:
	\begin{figure}[H]
		\centering
		\includegraphics[width=0.9\textwidth]{img/computing/gradient_boosting_tree.jpg}
		\caption[]{Illustrated Gradient Boosting with Trees (author: Rafel Del Rio)}
	\end{figure}
	
	\begin{tcolorbox}[title=Remark,colframe=black,arc=10pt]
	It should be quite obvious at this level of the book that take a linear regression model as weak learner is a non-sense. Indeed, the minimization of the linear regression Loss function is a closed form problem whose solution is perfectly known and that gives us the best linear unbiased estimator $\vec{\beta}$ out of all possibilities. However, dropping unbiasedness may allow to do a bit better particularly under high multicollinearity.
	\end{tcolorbox}
	
	\pagebreak
	\subsubsection{Neural networks}\label{neural network}
	Neural networks made from artificial cell structures are an approach for addressing from a new angle of perspective the problems of perception, memory, learning and non-linear reasoning (in other words ... artificial intelligence, abbreviated "IA") as well as genetic algorithms (see further below). They have also proved to very promising alternatives to bypass some of the limitations of conventional numerical methods (see autodriven cars, autodriven airplanes, automatic trading systems). Thanks to their parallel processing of information and inspired mechanisms inspired of nerve cells (neurons), they infer emergent properties to solve problems once referred to be as highly complex. Sometimes they result of artificial neurons is so bluffy that even best engineers have difficulties to explain how the neural network could arrive to a given observed performance (an typical non-business well known example is the Google Deep-learning dreaming machine).
	\begin{figure}[H]
		\centering
		\begin{subfigure}{.5\textwidth}
		  \centering
		  \includegraphics[width=0.9\linewidth]{img/computing/dream_machine_google_original.jpg}
		\end{subfigure}%
		\begin{subfigure}{.5\textwidth}
		  \centering
		  \includegraphics[width=0.9\linewidth]{img/computing/dream_machine_google_result.jpg}
		\end{subfigure}
		\caption[Google Dream Machine test]{Google Dream Machine test obtained the 2016-09-27 T20:42GMT\\ (source: \url{http://psychic-vr-lab.com/deepdream})}
	\end{figure}

	However, the major problem of artificial neurons (as far as we know...) is that they are not able to self-organize, nor to self-structured intelligently by themselves, they can only change weights and change some layer parameters in a range given previously by a human. So you need at this date to proceed heuristically to find the best neural network structure adapted to a problem and this is their large current actual weakness (either using brute force via a database containing millions of models or genetic algorithms that we will see a bit further below).
	
	We will discuss here the main architectures of neural networks. The purpose is not to study them all because they are too many of them (see figure on the next page), but rather to understand the basic internal mechanisms and how and when to use them with a minimal companion example with a common spreadsheet software. We will also discuss some concepts on fuzzy sets and logic (\SeeChapter{see section Logical Systems page \pageref{fuzzy logic}}) in the idea that these later are incorporated into some neural network architectures that we will study.
	
	The human brain is said ton contains about $100$ billion neurons. These neurons enable us among others to read a text while maintaining regular breathing to oxygenate our blood, activating our heart which ensures efficient circulation of the blood to nourish our cells, etc. After a very long learning path and trial and errors they also enable us to read books, innovate, create, copy concepts and gentle with other people of the same species (but the learning path doesn't work for all human being for all humans machine as some reject respect of life and of differences...).
	\begin{figure}[H]
		\centering
	  	\includegraphics[scale=0.20]{img/computing/neural_networks.jpg}
		\caption[Neural Networks complete chart]{Neural Networks complete chart (source: \url{http://www.asimovinstitute.org}, author: Fjodor van Veen)}
	\end{figure}
	Each of these neurons is also quit complex. Essentially, it is living tissue and chemistry and application of physics law (as the rest of our body). Neuro-physicists are just beginning to understand some of their internal mechanisms and are also able to influence them since a few decades. We usually think that their different neuronal functions, including memory is stored at the connections (synapses) between neurons. It is this kind of theory that has inspired most of the artificial neural network architectures (say to be "formal neural networks"). Learning is then the process that consist either to establish new connections, or modify existing one by trial and error or by an external reference (we will focus especially on the latter option).
	
	This brings us to a fundamental question, based on our current knowledge: can we build approximate mathematical models of neurons and make them to possibly perform useful tasks? Well, the short answer is: yes, even if the networks that we develop have only a tiny fraction of the power of the human brain actually (year 2003), and that is the goal here to show how we can do it formally (because technically it is obvious that more powerful are the computer, more impressive will be the results)!
	
	Neural networks are now used in all kinds of application in various fields. For example, there are neural networks developed for: aircraft autopilots, car autopilot, automotive control systems automatic reading of bank checks, automatic reading of postal addresses, signal processing, ballistic missile autopilots, voice synthesis, personal assistant, computer vision systems, market predictions, financial risks assessments, various manufacturing processes, medical diagnosis, oil and gas exploration, robotics, telecommunications, management decisioneering systems, classification and many others. In short, neural networks today have a significant impact and there is a safe bet that their importance will continue to grow in the future.
	
	\paragraph{Neuron model}\mbox{}\\\\
	The mathematical model of an artificial neuron, or "\NewTerm{perceptron}\index{perceptron}" is shown in the figure below (also in the previous figure with the overall summary of classical neural networks). A neuron consists essentially of an "integrator" which performs the weighted sum of its inputs (such as statistical mean ponderated by the inverse number of items!). The result $n$ of that is then transformed by a transfer function $f$ that produces the output $a$ of the formal neuron.

	The $R$ neuron inputs  correspond to vector correspond traditionally denoted:
	
	while:
	
	represents the neuron's vector weight (we distinguish them to prepare ourselves to the study of multiple layers formal neurons... also named "deep learning" systems):
	\begin{figure}[H]
		\centering
		\includegraphics{img/computing/formal_neuron_network_one_layer.jpg}
		\caption{Example of one-layer formal neuron with the input vector and scalar output}
	\end{figure}
	The output of the integrator is defined (as it is an engineering technique) by the following relation (weighted sum minus the bias):
	
	that we can also write in vector form (it could also be written as tensor but ...):
	
	Such that finally:
	
	This output is obviously a weighted sum of the weights and inputs less what we name the "\NewTerm{bias $b$ of the neuron}\index{bias of a neuron}" (correction factor determined by trial and error and often null in practice). The weighted sum is named "\NewTerm{activation level of the neuron}\index{activation level of a neuron}". The bias $b$ is also named "\NewTerm{activation threshold of the neuron}\index{activation threshold of a neuron}". When the activation level reaches or exceeds the threshold $b$, then $n$, the argument of $f$, named "\NewTerm{activation function}\index{activation function}", becomes zero or positive of course. Otherwise it is negative.
	
	That latter relation is also sometimes denoted:
	
	
	\begin{tcolorbox}[title=Remark,colframe=black,arc=10pt]
	There's absolutely no reason for the weights in the linear layer (a.k.a. dense or fully-connected layer) to sum up to anything specific, such as $1.0$. They are usually initialized with small random numbers (so initial sum is unlikely to be $1.0$).
	\end{tcolorbox}

	We can draw a parallel between this mathematical model and some information that we know (or think we know) about the biological neuron. That latter has three main components: the dendrites, the cell body and axon:	
	\begin{figure}[H]
		\centering
		\includegraphics{img/computing/neuron.jpg}
		\caption{Simplified representation of the vocabulary of human neurons}
	\end{figure}
	The dendrites form a network of nerve receptors that are used to route to the body of the neuron the electrical signals from other neurons. It acts as like an integrator, accumulating electric charges. When the neuron becomes sufficiently excited (when the accumulated charge exceeds a certain threshold), by an electrochemical process, it generates an electric potential that spreads through its axon to eventually excite other neurons. The point of contact between the axon of one neuron and the dendrite of another neuron is named the "synapse". It seems that this is the spatial arrangement of neurons and their axons and also their quantity ($86,000,000,000$ neurons and $1.5\cdot 10^{14}$ synapses for the average human brain), and the quality of individual synaptic connections that determine the precise function of a biological neural network. This is based on this knowledge that the mathematical model described above has been defined (this is "Biomimicry Engineering" or "Biomimestim Engineering").
	
	A weight of an artificial neuron represents somehow the effectiveness of a synaptic connection. A negative weight inhibits in way the entrance, while a positive weight increase its effect. It is important to remember that this is a rough approximation of a real synapse resulting in fact of a quite complex chemical process and dependent on many external factors still unclear. We must understand that our artificial neuron is a pragmatic model which, as we shall see later, will help us to accomplish interesting tasks and more neurons we have more the tasks can be complex. The biological plausibility of this model is not important to us. What counts is the result that this model will allow us to achieve.
	
	Another limiting factor in the model we set ourselves is regarding its discreet nature. Indeed, in order to simulate a neural network, we will make the time discrete in our equations. In other words, we assume that all neurons are synchronous, that is to say that at each time $t$, they will simultaneously calculate the weighted sum and produce an output $a(t)=f(n(t))$. In biological networks, all neurons are actually asynchronous...

	So let us come back to our model as formulated by the previous relation and by doing a change of notation $\vec{w}^T$:
	
	This equation leads us to introduce a new more formal scheme of our formal neural network (FNW) or perceptron:
	\begin{figure}[H]
		\centering
		\includegraphics{img/computing/neural_network_one_layer_vector_simplified_form.jpg}
		\caption{Vector written example of one-layer formal neuron with the input vector and scalar output}
	\end{figure}
	We represent the $R$ inputs as a black rectangle (the number of entries is indicated below the rectangle). From this rectangle result a vector $\vec{p}$ whose dimensions are $R\times 1$ . This vector is multiplied by a vector $\vec{w}$ that contains the weights (synaptic) neuron. In the case of a single neuron, this vector has dimension $1\times R$. The result of the multiplication is the "activation level" (scalar) which is then compared to the threshold $b$ (a scalar) by subtraction. Finally, the output of the neuron is calculated by the function $f$. The output of a single neuron is then always a scalar in this special case.

	To find the components of the matrix $\vec{w}$ (neuron input weight), and the bias $b$ we use operational research techniques (simplex method, conjugate gradient method, evolutionary algorithms, etc.) on a sample of data sample of the company of size $n$ in order to "train the model of the neuron". The purpose will then be to find weights $\vec{w}$ that minimize the quadratic cost error function given by:
	
	where $a_k$ is the model output for the vector $\vec{r}_k$ and $y_k$ is the real value (the expected answer) corresponding to the $\vec{r}_k$ .
	
	After we test the result on a sample test before  putting the neural network in production for not yet existing data (we will do a detailed example with Microsoft Excel just after the presentations of the transfer functions).
		
	\pagebreak
	\paragraph{Transfer functions}\mbox{}\\\\
	So far, we have not specified the nature of the activation function $a=f(n)$ of our model. It turns out that several possibilities exist and these are empirical and must adapt to different situations (and the adaptation is sometimes also dynamic). The most common and most cited in the literature are listed in the figure below:
	\begin{table}[H]
		\begin{center}
			\definecolor{gris}{gray}{0.85}
				\begin{tabular}{|l||c|c|}
					\hline
					{\cellcolor{black!30}Activation function name $f$} & {\cellcolor{black!30}Input/Output relation} & {\cellcolor{black!30}Profile plot}  \\ \hline
					Threshold & $\begin{matrix}a=0 & \text{if} & n<0\\a=1 & \text{if} & n\geq 0 \end{matrix}$ & \parbox{1.5cm}{\includegraphics[scale=0.5]{img/computing/neural_network_treshold.jpg}}\\ \hline
					Symmetric threshold & $\begin{matrix}a=-1 & \text{if} & n<0\\a=1 & \text{if} & n\geq 0 \end{matrix}$ & \parbox{1.5cm}{\includegraphics[scale=0.5]{img/computing/neural_network_sym_treshold.jpg}}\\ \hline
					Linear & $a=n$ & \parbox{1.5cm}{\includegraphics[scale=0.5]{img/computing/neural_network_linear.jpg}}\\ \hline
					Saturated linear & $\begin{matrix}a=0 & \text{if} & n<0\\a=n & \text{if} & 0\geq n \geq 1\\ a=1 & \text{if} & n>1 \end{matrix}$ & \parbox{1.5cm}{\includegraphics[scale=0.5]{img/computing/neural_network_saturated_linear.jpg}}\\  \hline
					Symmetric saturated linear & $\begin{matrix}a=-1 & \text{if} & n<-1\\a=n & \text{if} & -1\geq n \geq 1\\ a=1 & \text{if} & n>1 \end{matrix}$ & \parbox{1.5cm}{\includegraphics[scale=0.5]{img/computing/neural_network_symetric_saturated_linear.jpg}}\\ \hline
					Rectified Linear Unit Function (RELU) & $\begin{matrix}a=0 & \text{if} & n<0\\a=n & \text{if} & n\geq 0 \end{matrix}$ & \parbox{1.5cm}{\includegraphics[scale=0.5]{img/computing/neural_network_linear_positive.jpg}}\\ \hline
					Sigmoid & $a=\dfrac{1}{1+e^{-n}}$ & \parbox{1.5cm}{\includegraphics[scale=0.5]{img/computing/neural_network_sigmoid.jpg}}\\ \hline
					Hyperbolic tangent (TANH) & $a=\dfrac{e^n-e^{-n}}{e^n+e^{-n}}$ & \parbox{1.5cm}{\includegraphics[scale=0.5]{img/computing/neural_network_hyperbolic_tangent.jpg}}\\ \hline
					Competitive & $\begin{matrix}a=1 & \text{if } n \text{ is maximum} \\ a=0 & \text{otherwise} \end{matrix}$ & \parbox{1.5cm}{\includegraphics[scale=0.5]{img/computing/neural_network_competitive.jpg}}\\ \hline
					Gompertz & $a=\alpha e^{-\beta e^{\gamma n}}$ & \parbox{1.5cm}{\includegraphics[scale=0.5]{img/computing/neural_network_sigmoid.jpg}}\\ \hline
					... & ... & ...\\ \hline
				\end{tabular}
		\end{center}
		\caption[]{Types of transfer functions for FNN}
	\end{table}
	The three most used in the field of engineering are the functions "threshold" (I) , "linear" (II) and "sigmoid" (III) as shown in details below:
	\begin{figure}[H]
		\centering
		\includegraphics{img/computing/neural_network_transfer_functions.jpg}
		\caption{Most used transfer functions for neural networks in the 20th century}
	\end{figure}
	As its name suggests it, the threshold function applies a threshold on its input. Specifically, a negative input does not pass the threshold, the function returns then a $0$ (false), while a positive or zero input exceeds the threshold, and the function returns then a $1$ (true). It is obvious that this kind of feature is to make binary decisions (this function can also be assimilated to the Heaviside function for those who know it...).
	
	The linear function is itself very simple, it directly associate its input to an output according to the relation $a=f(n)=n$. It is then evident that the output of the neuron corresponds to its activation level for which the zero value (the ordinate at the origin) occurs when $\vec{w}^T\vec{p}=b$.
	
	The sigmoid transfer function is itself defined by the mathematical relation:
	
	it looks like two threshold function, either the linear function, as we are far or near to $b$ respectively. The threshold function is very non-linear because there is a discontinuity when $\vec{w}^T\vec{p}=b$. For its part, the linear function is entirely linear. It has no change in slope. The sigmoid is an interesting compromise between the two. Finally notice that the hyperbolic tangent (TANH) function is a symmetric version of the sigmoid.
	
	It may be quite obvious to the reader that if the activation function is linear, then the neural network is a regression model! If it is a logistic function, then the neural network is a binary classification model!
	
	We have introduced during our study of logistics regressions the softmax function (see above page \pageref{softmax functiion}). The softmax function is a typical activation function for a multi-class neural network classifier. That is a neural network with multiple input but also multiple outputs. If we denote the weights $\vec{v}$ and we index each different output by the letter $j$, the such a neural network will be represented by:
	\begin{figure}[H]
		\centering
		\includegraphics[scale=0.65]{img/computing/neural_information_multiclass.jpg}
		\caption{Multi-class Softmax neural network}
	\end{figure}
	
	\pagebreak
	\paragraph{Various cost functions}\mbox{}\\\\
	Relatively to Neural Networks, as we already know, a cost function is a measure of how good a neural network did with respect to it's given training sample and the expected output. It also may depend on variables such as weights and biases.

	Remember also that a cost function is a single value (scalar), not a vector, because it rates how good the neural network did as a whole.
	
	Specifically, a cost function is of the form:
	
	where $W$ is our neural network's weights, $\vec{b}$ is our neural network's biases, $\vec{x}$ is the input of a single training sample, $\vec{y}$ is the desired output (ie expected output) of that training sample and $\phi_i$ is the activation function. 
	
	Here are some common empirical cost functions:
	\begin{itemize}
		\item The "\NewTerm{Quadratic cost}" also known as mean squared error, maximum likelihood, and sum squared error, is defined as:
		
		The gradient of this cost function with respect to the output of a neural network and is:
		
	
		\item  The "\NewTerm{Cross-entropy cost}" also known as "\NewTerm{Bernoulli negative log-likelihood}" and "\NewTerm{Binary Cross-Entropy}", is defined as:
		
		The gradient of this cost function with respect to the output of a neural network is:
		
	
		\item The "\NewTerm{Exponential cost}" requires choosing some parameter $\tau$ that you think will give you the behaviour you want, typically you'll just need to play with this until things work good, and is it is defined by:
		
		The gradient of this cost function with respect to the output of a neural network  is:
		
	
		\item The "\NewTerm{Hellinger distance}" is defined as:
		
		This needs to have positive values, and ideally values between $0$ and $1$. The same is true for the following divergences.
	
		The gradient of this cost function with respect to the output of a neural network is:
		
	
		\item The "\NewTerm{Kullback–Leibler cost}" is defined as (\SeeChapter{see section Statistical Mechanics page \pageref{kullback-leibler divergence}}):
		
	
		The gradient of this cost function with respect to the output of a neural network is:
		
	
		\item The "\NewTerm{Generalized Kullback–Leibler cost}", a special case of the "\NewTerm{Bregman divergence}\footnote{It seems that in machine learning, Bregman divergences are used to calculate the bi-tempered logistic loss, performing better than the softmax function with noisy datasets}", is defined as:
		
	
		The gradient of this cost function with respect to the output of a neural network is:
		
	
		\item The "\NewTerm{Itakura–Saito distance}", also a special case of the Bregman divergence, is defined as:
		
	
		The gradient of this cost function with respect to the output of a neural network is:
		
		
		\item And very likely many others...
	\end{itemize}
	To summarize:
	\begin{table}[H]
		\resizebox{\textwidth}{!}{\centering
		\begin{tabular}{|l|l|l|}
		\hline
		\rowcolor[HTML]{C0C0C0} 
		\textbf{Cost function name} & \textbf{Mathematical expression} & \textbf{Gradient} \\ \hline
		 Quadratic cost & $J_\text{MST}\left(W,\vec{b},\vec{y},\vec{\phi}\right)=\dfrac{1}{2}\displaystyle\sum_{i=1}^n\left(\phi_i(z)-y_i\right)^2$ & $\nabla_\phi J_\text{MST}\left(W,\vec{b},\vec{y},\vec{\phi}\right) = \displaystyle\sum_{i=1}^n (\phi_i(z)-y_i)$ \\ \hline
		 Cross-entropy cost & $J_\text{CE}\left(W,\vec{b},\vec{y},\vec{\phi}\right) =-\displaystyle\sum_{i=1}^n\left[y_i\ln(\phi(z))+(1-y_i)\ln(1-y_i)\right]$ & $\nabla_\phi  J_\text{CE}\left(W,\vec{b},\vec{y},\vec{\phi}\right) =\displaystyle\sum_{i=1}^n\dfrac{\phi_i(z)-y_i}{(1-\phi_i(z))\phi_i(z)}$ \\ \hline
		 Exponential cost & $J_\text{exp}\left(W,\vec{b},\vec{y},\vec{\phi},\tau\right) =\tau \exp\left(\dfrac{1}{\tau} \displaystyle\sum_{i=1}^n \left(\phi_i(z)-y_i\right)^2 \right)$ & $\nabla_\phi  J_\text{exp}\left(W,\vec{b},\vec{y},\vec{\phi},\tau\right) =\dfrac{2}{\tau}J_\text{exp}\left(W,\vec{b},\vec{y},\vec{\phi},\tau\right)\displaystyle\sum_{i=1}^n \phi_i(z)-y_i$ \\ \hline
		 Hellinger distance & $J_\text{HD}\left(W,\vec{b},\vec{y},\vec{\phi}\right) =\dfrac{1}{\sqrt{2}}\displaystyle\sum_{i=1}^n \left(\sqrt{\phi_i(z)}-\sqrt{y_i} \right)^2$ & $\nabla_\phi  J_\text{exp}\left(W,\vec{b},\vec{y},\vec{\phi}\right) =\dfrac{1}{\sqrt{2}}\displaystyle\sum_{i=1}^n \dfrac{\sqrt{\phi_i(z)}-\sqrt{y_i}}{\sqrt{\phi_i(z)}}$ \\ \hline
		 Kullback-Leibler (KL) cost & $J_\text{KL}\left(W,\vec{b},\vec{y},\vec{\phi}\right) =\displaystyle\sum_{i=1}^n y_i\log\left(\dfrac{y_i}{\phi_i(z)}\right)$ & $\nabla_\phi  J_\text{GKL}\left(W,\vec{b},\vec{y},\vec{\phi}\right) =\displaystyle\sum_{i=1}^n \dfrac{\phi_i(z)-y_i}{\phi_i(z)}$ \\ \hline
		 Generalized KL cost & $J_\text{GKL}\left(W,\vec{b},\vec{y},\vec{\phi}\right) =\displaystyle\sum_{i=1}^n y_i\log\left(\dfrac{y_i}{\phi_i(z)}\right)-\displaystyle\sum_{i=1}^n y_i+\displaystyle\sum_{i=1}^n \phi_i(z)$ & $\nabla_\phi  J_\text{GKL}\left(W,\vec{b},\vec{y},\vec{\phi}\right) =\displaystyle\sum_{i=1}^n \dfrac{\phi_i(z)-y_i}{\phi_i(z)}$  \\ \hline
		 Itakura-Saito distance cost & $J_\text{IS}\left(W,\vec{b},\vec{y},\vec{\phi}\right) =\displaystyle\sum_{i=1}^n \left(\dfrac{y_i}{\phi_i(z)}-\log\left(\dfrac{y_i}{\phi_i(z)}\right)-1\right)$ & $\nabla_\phi  J_\text{IS}\left(W,\vec{b},\vec{y},\vec{\phi}\right) =\displaystyle\sum_{i=1}^n \dfrac{\phi_i(z)-y_i}{\phi_i^2(z)}$ \\ \hline
		 ... & ... & ... \\ \hline
		\end{tabular}}
	\end{table}
	
	\pagebreak
	\paragraph{Network Architecture}\mbox{}\\\\
	By definition, a "\NewTerm{neural network}\index{neural network}" is a network of several neurons, usually organized in layers. To build one layer of neurons $S$, we simply need to assemble them as in the figure below ($R$ inputs with $S$ hidden neurons and $S$ outputs):
	\begin{figure}[H]
		\centering
		\includegraphics{img/computing/neural_network_one_layer_vector_form.jpg}
		\caption{Example of one-layer formal neural network with the input vector and output vector}
	\end{figure}
	The $S$ neurons of a same layer are all connected to the $R$ inputs in the figure above. We then say that the layer is "fully connected". But this is a special case and not a generality. Often the inputs of a neuron are different from those of another neuron, etc.
	
	\begin{tcolorbox}[title=Remark,colframe=black,arc=10pt]
	If a neural network that has for purpose to learn that can learn a probability distribution over its set of inputs has only one hidden layer of $S$ neurons not interconnected between them (as is the case in the figure above), that is to say independent, we speak then of "\NewTerm{restricted Boltzmann machine RBM} \index{restricted Boltzmann machine}".
	\end{tcolorbox}
	
	A weight $w_{i,j}$ is associated with each connection. We will always denote the first index by $i$ and the second by $j$. The first index (row) always means the neuron number on the layer, while the second index (column) specifies the number of the input. Thus, $w_{i,j}$ denotes the weight of the connection that connects the neuron $i$ to its input $j$. All the weights of a neuron lays thus forms a matrix $W$ of dimension $S\times R$:
	
	We must of course take into account that dimensionally we have not necessarily $S=R$ in the general case (the numbers of neurons and inputs are independent). If we consider that the $S$ neurons form a vector, then we can create the vectors:
	
	This brings us to the simplified representation illustrated below:
	\begin{figure}[H]
		\centering
		\includegraphics{img/computing/neural_network_one_layer_multiple_neurons_vector_simplified_form.jpg}
		\caption{Vector written example of one-layer formal neural network with the input vector and output vector}
	\end{figure}
	Finally, to build a neural network (or MLP for "\NewTerm{Multi-Layer Perceptron}\index{multi-layer perceptron}"), it just sufficient to combine layers as below:
	\begin{figure}[H]
		\centering
		\includegraphics[scale=0.6]{img/computing/neural_network_multiple_layer.jpg}
		\caption{Principle of construction of a Multi-Layer Perceptron}
	\end{figure}
	This example includes $R$ inputs and three layers of neurons having respectively $S^1,S^2,S^3$ neurons. In the general case, again these numbers are not necessarily equal. Each layer also has its own weight matrix $W^k$, where $k$ is the layer index. In the context of vectors and matrices relatively to one a layer, we always will use an exponent to describe this index. Thus, the vectors $\vec{b}^k$, $\vec{n}^k$, $\vec{a}^k$ are also associated with the layer $A$.
	
	\begin{tcolorbox}[title=Remark,colframe=black,arc=10pt]
	Multilayer perceptrons are also named "\NewTerm{vanilla feed-forward neural networks}\footnote{"vanilla"= the basic version.}\index{vanilla feed-forward neural networks}"\index{feed-forward neural networks}. A perceptron is always feedforward, that is, all the arrows are going in the direction of the output. Neural networks in general might have loops, and if so, are often called "\NewTerm{recurrent neural networks}\index{recurrent neural networks}" (RNN). A superposition of at least three MLP is what seems to be usually named a "\NewTerm{deep beliefs net}\index{deep beliefs net}" (DLN).
	\end{tcolorbox}
	It should be notices in this example that the layers that follows the first has as input the output of the previous layer. So we can put on as many layers as we want, at least in theory. We can set any number of neurons of each layer. In practice, we will see later however it is not desirable to use too many neurons. Note also that nothing prevents us from changing transfer function from one layer to another. Thus, in the general case we have not necessarily $f^1=f^2=f^3$.
	
	The first last layer is obviously named "\NewTerm{input layer}\index{input layer}", and the last layer the "\NewTerm{output layer}\index{output layer}". The layers between the output and input layers are commonly named "\NewTerm{hidden layers}\index{hidden layers}".
	
	Not all textbooks and softwares count the number of layers in the same way. Here is a typical example where the first layer doesn't have any activation function, but is however counted as one layer (and the notation also differs slightly):
	\begin{figure}[H]
		\centering
		\includegraphics[width=1.0\textwidth]{img/computing/explicit_multilayer_vanilla_neural_network.jpg}
		\caption{Explicit construction of a Multi-Layer Perceptron}
	\end{figure}
	Then we can relate the next layer's input to it's previous via the following relation:
	
	where:
	\begin{itemize}
		\item $f$ is the activation function
		\item $w_{jk}^i$ is the weight from the $k$-th neuron in the $(i-1)$-th layer to the $j$-th neuron in the $i$-th layer
		\item $b_j^i$ is the bias of the $j$-th neuron in the $i$-th layer
		\item $a_j^i$ represents the activation value of the $j$-th neuron in the $i$-th layer
	\end{itemize}
	
	\begin{tcolorbox}[title=Remark,colframe=black,arc=10pt]
	Multilayer neural networks are more powerful than simple single layer neural networks of course. Using two layers, provided you use a sigmoid activation function on the hidden layer we can "train" a network to produce an approximation of most functions with arbitrary precision. Except in rare cases, artificial neural networks use two or three layers.
	\end{tcolorbox}
	"\NewTerm{Train}\index{train a neural network}" a neural network means changing the value of its weight matrices and its so that he realizes the desired input/output function (I / O). We will study in detail various algorithms and methods of heuristics approach to achieve it in different contexts.
	
	\begin{figure}[H]
		\centering
		\includegraphics[scale=0.7]{img/computing/neural_information_processing.jpg}
		\caption[Neural Information Processing]{Neural Information Processing (source: Purdue University image/e-Lab)}
	\end{figure}
	
	\begin{tcolorbox}[title=Remark,colframe=black,arc=10pt]
	In the early 21st century there seems to no exist any standard nor accepted method for selecting the number of layers, and the number of nodes in each layer, in a feed-forward neural network. It's more trial and error.
	\end{tcolorbox}
	
	Let us now see a easy companion example (originally developed by Joe Breedlove) as always in this book first done with a spreadsheet software like Microsoft Excel. Afterwards we will show the same output result with \texttt{R} and MATLAB™ for which you can found the detailed procedure in the corresponding companion books.
	
	\pagebreak
	\begin{tcolorbox}[colframe=black,colback=white,sharp corners]
	\textbf{{\Large \ding{45}}Example:}\\\\
	A company has measured during $14$ weeks its actual sales (Column: \textit{Value to predict}) in function of forecast sales of five of its branches (\textit{Variable1}, \textit{Variable2}, etc.) and has reproduced them in Microsoft Excel 14.0.6123:
	\begin{figure}[H]
		\centering
		\includegraphics[scale=0.8]{img/computing/neural_network_list_training_set_microsoft_excel.jpg}
		\caption[]{Training data list for our neural network in Microsoft Excel 14.0.6123}
	\end{figure}
	Notice that there is absolutely no formula in the list above! The return on experience (especially with Microsoft Excel...) tell us it would be better to do a network architecture with two neurons, first with branches $\{1,2,3\}$ based on a sigmoid and a second with branches $\{4.5\}$ also based on a sigmoid. In addition, all should have a single bias and the both neurons should have a specific weight in comparison with the one and the other.

	We then prepare the following table:
	\begin{figure}[H]
		\centering
		\includegraphics[scale=1]{img/computing/neural_network_initial_weight_bias_and_ponderations_microsoft_excel.jpg}
		\caption[]{Weight bias and weights to determined for our neural network}
	\end{figure}
	Once the table of weights, bias and ponderations built, we write our two neurons network  with the sigmoid function, for example, right next to the training sample data (which will facilitate the comparison):
	\end{tcolorbox}
	
	\begin{tcolorbox}[colframe=black,colback=white,sharp corners]
	\begin{figure}[H]
		\centering
		\includegraphics[scale=0.6]{img/computing/neural_network_list_training_set_with_neural_network_microsoft_excel.jpg}
		\caption[]{List of sample training data with neural network cells}
	\end{figure}
	Or with the explicit formulas for the last three columns of interest:
	\begin{figure}[H]
		\centering
		\includegraphics[scale=0.6]{img/computing/neural_network_list_training_set_with_neural_network_formulas_microsoft_excel.jpg}
		\caption[]{List of sample training data with neural network cells}
	\end{figure}
	To apply operational research techniques, we need to minimize or maximize something. Therefore, we will seek to minimize the sum of squared errors by creating the following column:
	\end{tcolorbox}
	
	\pagebreak
	\begin{tcolorbox}[colframe=black,colback=white,sharp corners]
	\begin{figure}[H]
		\centering
		\includegraphics[scale=0.8]{img/computing/neural_network_list_training_set_error_minimization_microsoft_excel.jpg}
		\caption[]{List of sample training data with neural network cells and quadratic error minimization}
	\end{figure}
	Or with the explicit formulas (we see well that this corresponds indeed to the square of the difference between the measurements and model):
	\begin{figure}[H]
		\centering
		\includegraphics[scale=0.8]{img/computing/neural_network_list_training_set_error_minimization_formulas_microsoft_excel.jpg}
		\caption[]{List of sample training data with neural network cells and quadratic error minimization}
	\end{figure}
	Now with the solver of Microsoft Excel 14.0.6123 we minimize the content of the cell \texttt{L36}:
	\end{tcolorbox}
	
	\pagebreak
	\begin{tcolorbox}[colframe=black,colback=white,sharp corners]
	\begin{figure}[H]
		\centering
		\includegraphics[scale=0.8]{img/computing/neural_network_solver_excel.jpg}
		\caption[]{Neural network solver quadratic error minimization in Microsoft Excel 14.0.6123}
	\end{figure}
	we should not be too focused about the accuracy of the constraints for this case and therefore we have to play a little with this setting to get a satisfactory result:
	\begin{figure}[H]
		\centering
		\includegraphics[scale=0.65]{img/computing/neural_network_solver_settings_excel.jpg}
		\caption[]{Neural network solver settings in Microsoft Excel 14.0.6123}
	\end{figure}
	\end{tcolorbox}
	
	\begin{tcolorbox}[colframe=black,colback=white,sharp corners]
	To get a satisfactory result, it will be necessary in this case to request an accuracy of $0.001$. Which will give after the execution of the search by the solver a total square error of $0.8479$ (cell \texttt{L36}) and for the parameters of the neural network:
	\begin{figure}[H]
		\centering
		\includegraphics[scale=1]{img/computing/neural_network_solver_excel_solution.jpg}
		\caption[]{Neural network solver solution in Microsoft Excel 14.0.6123}
	\end{figure}
	Specialized software will do better with a total square error of $0.8405$ (still for cell \texttt{L36}) and for the parameters of the neural network:
	\begin{figure}[H]
		\centering
		\includegraphics[scale=1]{img/computing/neural_network_optimal_values_external_software.jpg}
		\caption[]{Neural network optimal values with specialized software}
	\end{figure}
	We can graphically compare the measurements used to train the neural network and the result of the neural network model itself. Then we have:
	\end{tcolorbox}
	
	\begin{tcolorbox}[colframe=black,colback=white,sharp corners]
	\begin{figure}[H]
		\centering
		\includegraphics[scale=0.9]{img/computing/neural_network_measurements_vs_model_plot_excel.jpg}
		\caption[]{Neural network model vs Measurements}
	\end{figure}
	Which seems not bad for a non-linear model! But once the model trained, we must always see if it applies to other data (test sample). Therefore let us consider:
	\begin{figure}[H]
		\centering
		\includegraphics[scale=0.8]{img/computing/neural_network_test_sample_excel.jpg}
		\caption[]{Neural network test data sample}
	\end{figure}
	Always with the same formulas:
	\begin{figure}[H]
		\centering
		\includegraphics[scale=0.6]{img/computing/neural_network_test_sample_formula_excel.jpg}
		\caption[]{Neural network test data sample neural network formula in Microsoft Excel 14.0.6123}
	\end{figure}
	And if we also graphically compare the real data and the modelled data, we get:
	\end{tcolorbox}
	
	\begin{tcolorbox}[colframe=black,colback=white,sharp corners]
	\begin{figure}[H]
		\centering
		\includegraphics[scale=0.9]{img/computing/neural_network_test_sample_plot_excel.jpg}
		\caption[]{Neural network test data sample neural network formula in Microsoft Excel 14.0.6123}
	\end{figure}
	and here we see that the model is significantly worse. But it is so! The predictive science is not an exact science but a heuristic...\\
	
	In the MATLAB™ companion book, working with the same data and building a similar neural network:
	\begin{figure}[H]
		\centering
		\includegraphics[scale=0.9]{img/computing/neural_network_matlab.jpg}
		\caption{Neural network in MATLAB™ 2013a}
	\end{figure}
	we get the following model fitting plot:
	\end{tcolorbox}
	
	\begin{tcolorbox}[colframe=black,colback=white,sharp corners]
	\begin{figure}[H]
		\centering
		\includegraphics[scale=0.85]{img/computing/neural_network_measurements_vs_model_plot_matlab.jpg}
		\caption[]{Neural network model vs Measurements in MATLAB™  2013a}
	\end{figure}
	and test sample plot(MATLAB™ performs better than Microsoft Excel solver in this special case):
	\begin{figure}[H]
		\centering
		\includegraphics[scale=0.85]{img/computing/neural_network_test_sample_plot_matlab.jpg}
		\caption[]{Neural network model test sample plot in MATLAB™  2013a}
	\end{figure}
	\end{tcolorbox}
	 
	 \paragraph{Backpropagation}\label{backpropagation}\mbox{}\\\\
	 Gradient descent (see earlier above page \pageref{gradient descent}) is a good technique to minimize a cost function. But if we cannot analytically compute the function local derivatives, we will instead have to compute the slope  manually by evaluating the function around a neighbourhood. Translated to the example of  descending a mountain, it is like either walking with eyes wide open (analytical way), or  being blindfolded and having to walk a few steps in each direction, each time we want to  assess the slope around us (manual way). This manual option is very time consuming,  especially if you have many parameters—which is the case with neural networks. 

	Hopefully, a technique was found to analytically compute the local derivatives in a neural  network: back-propagation. The idea is that a neural network can be used in two directions: 
	\begin{enumerate}
		\item Forward propagation where the neurons propagate information through the network by applying their activation functions to their weighted inputs and biases. 
		
		\item Backpropagation where the neurons propagate their local derivatives (that is the impact of changing weights, input values, and biases). In other words, each neuron will propagate how much a change in one of its inputs (weights, input values, bias) will impact its own output.
	\end{enumerate}
	During the training phase of a neural network, forward propagation and back-propagation will be performed in turns. Forward propagation will be used to generate predictions (as data flow through the network). Backpropagation will be used to tune the weights based on the latest errors. 

	 Therefore in the field of neural networks "\NewTerm{back-propagation}\index{back-propagation}" is a method to calculate the gradient of the loss function with respect to the weights in an artificial neural network. It is commonly used as a part of algorithms that optimize the performance of the network by adjusting the weights, for example in the gradient descent algorithm (see page \pageref{gradient descent}). It is also named "\NewTerm{backward propagation of errors}\index{backward propagation of errors}". It seems that there is no shortage of papers online that attempt to explain how back-propagation works, but few that include an example with actual numbers. The text and example below is an attempt of Matt Mazur\footnote{Reproduced from \url{https://mattmazur.com/2015/03/17/a-step-by-step-backpropagation-example/} with its authorization} to explain how it works with a concrete example that everybody can compare their own calculations to in order to ensure they understand backpropagation correctly.

	To explain backpropagation, we're going to use a neural network with two inputs, two hidden neurons, two output neurons. Additionally, the hidden and output neurons will include a bias.

	Here's the basic structure:
	\begin{figure}[H]
		\centering
		\includegraphics[scale=1]{img/computing/backpropagation_basis_structure.jpg}
		\caption[]{Basis neural network for backpropagation study}
	\end{figure}
	In order to have some numbers to work with, here are the initial weights (in red), the biases (in orange), and training inputs/outputs (in blue):
	\begin{figure}[H]
		\centering
		\includegraphics[scale=1]{img/computing/backpropagation_basis_structure_with_values.jpg}
	\end{figure}
	The goal of backpropagation is to optimize the weights so that the neural network can learn how to correctly map arbitrary inputs to outputs.

	For the rest of this tutorial we're going to work with a single training set: given inputs $0.05$ and $0.10$ and $b_1=1$ and $b_2=2$, we want the neural network to output $0.01$ and $0.99$.
	
	Let us decompose the method:
	\begin{enumerate}
		\item The Forward Pass\index{forward pass}:
		
		To begin, lets see what the neural network currently predicts given the weights and biases above and inputs of $0.05$ and $0.10$. To do this we'll feed those inputs forward though the network.

		We figure out the total net input\footnote{The "Total net input" is also referred to as just "net input" by some sources.} to each hidden layer neuron, squash the total net input using an activation function (here we use the logistic function), then repeat the process with the output layer neurons.
		
		Here is how we calculate the total net input for $h_1$:
		
		We then squash it using (arbitrarily) using the logistic function to get the output of $h_1$:
		
		Carrying out the same process for $h_2$ we get:
		
		We repeat this process for the output layer neurons, using the output from the hidden layer neurons as inputs.
		
		Here's the output for $o_1$:
		
		Hence:
		
		And carrying out the same process for $o_2$ we get:
		
		We can now calculate the error for each output neuron using the squared error function (a loss function) and sum them to get the total error (using the traditional notations in the field...), ie a cost function:
		
		Or if written in another common way:
		
		Since backpropagation uses the gradient descent method\index{gradient descent method} (see page \pageref{gradient descent}), the factor of $\frac{1}{2}$ is included to cancel the exponent when differentiating. Later, the expression will be multiplied with an arbitrary learning rate, so that it doesn't matter if a constant coefficient is introduced now.
		\begin{tcolorbox}[title=Remark,colframe=black,arc=10pt]
		Some textbooks refer to the "target" as the "ideal" and the "output" as the "actual".
		\end{tcolorbox}
		For example, the target output for $o_1$ is $0.01$ but the neural network output $0.75136507$, therefore its error is:
		
		Repeating this process for $o_2$ (remembering that the target is $0.99$) we get:
		
		The total error for the neural network is the sum of these errors:
		
		
		\item The Backwards Pass:
		Our goal with backpropagation is to update each of the weights in the network so that they cause the actual output to be closer the target output, thereby minimizing the error for each output neuron and the network as a whole.
		
		Consider $w_5$. We want to know how much a change in $w_5$ affects the total error, aka $\frac{\partial E_{\text{tot}}}{\partial w_{5}}$.
		
		By applying the chain rule twice (\SeeChapter{see section Differential and Integral Calculus page \pageref{multivariate chain rule}}) we know that:
		
		We need to figure out each piece in this equation.
		\begin{itemize}
			\item First, how much does the total error change with respect to the output?
			
			As:
			
			Therefore:
			
			
			\item Next, how much does the output of $o_1$ change with respect to its total net input?
			As:
			
			Therefore:
			
			
			\item Finally, how much does the total net input of $o_1$ change with respect to $w_5$?
			
			As:
			
			Therefore:
			
		\end{itemize}
		Putting it all together:
		
		\begin{tcolorbox}[title=Remark,colframe=black,arc=10pt]
		Some textbooks use $\alpha$ to represent the learning rate, others use $\eta$, and others even use $\varepsilon$...
		\end{tcolorbox}
		To decrease the error, we then subtract this value from the current weight, by applying the relation derived earlier above (see page \pageref{gradient descent}):
				
		 optionally multiplied by some learning rate, $\eta$, which we will set to $0.5$:
		
		\begin{tcolorbox}[title=Remark,colframe=black,arc=10pt]
		Some textbooks use $\alpha$ to represent the learning rate, others use $\eta$, and others even use $\varepsilon$...
		\end{tcolorbox}
		We can repeat this process to get the new weights $w_6$, $w_7$, and $w_8$:
		
		We perform the actual updates in the neural network after we have the new weights leading into the hidden layer neurons (ie, we use the original weights, not the updated weights, when we continue the backpropagation algorithm below)!
		
		Next, we will continue the backwards pass by calculating new values for $w_1$, $w_2$, $w_3$, and $w_4$ of the hidden layer.

		Here is what we need to figure out:
		
		We are going to use a similar process as we did for the output layer, but slightly different to account for the fact that the output of each hidden layer neuron contributes to the output (and therefore error) of multiple output neurons.
		
		We know that $\text{out}_{h_1}$ affects both $\text{out}_{o_1}$ and $\text{out}_{o_2}$ therefore the $\frac{\partial E_{\text{tot}}}{\partial \text{out}_{h_1}}$ needs to take into consideration its effect on the both output neurons:
		
		Starting with $\frac{\partial E_{o_1}}{\partial \text{out}_{h_1}}$:
		
		We can calculate $\frac{\partial E_{o_1}}{\partial \text{net}_{o_1}}$ using values we calculated earlier:
		
		And $\frac{\partial \text{net}_{o_1}}{\partial \text{out}_{h_1}}$ is equal to $w_5$. Indeed:
		
		Therefore:
		
		Plugging them in:
		
		Following the same process for $\frac{\partial E_{o_2}}{\partial \text{out}_{h_1}}$, we get:
		
		Therefore:
		
		Now that we have $\frac{\partial E_{\text{tot}}}{\partial \text{out}_{h1}}$, we need to figure out $\frac{\partial \text{out}_{h_1}}{\partial \text{net}_{h1}}$ and then $\frac{\partial \text{net}_{h_1}}{\partial w_1}$.

		As:
		
		Therefore identically as earlier above:
		
		And as:
		
		Therefore:
		
		Putting it all together (and writing all the steps explicitly):
		
		We can now update $w_1$:
		
		Repeating this for $w_2$, $w_3$, and $w_4$:
		
	\end{enumerate}
	Finally, we have updated all of our weights! When we fed forward the $0.05$ and $0.1$ inputs originally, the error on the network was $0.298371109$. After this first round of backpropagation, the total error is now down to $0.291027924$. It might not seem like much, but after repeating this process $10,000$ times, for example, the error plummets to $0.000035085$. At this point, when we feed forward $0.05$ and $0.1$, the two outputs neurons generate $0.015912196$ (vs $0.01$ target) and $0.984065734$ (vs $0.99$ target).
	
	Therefore we have seen that this optimization algorithm repeats a two phase cycle, propagation and weight update. When an input vector is presented to the network, it is propagated forward through the network, layer by layer, until it reaches the output layer. The output of the network is then compared to the desired output, using a loss function, and an error value is calculated for each of the neurons in the output layer. The error values are then propagated backwards, starting from the output, until each neuron has an associated error value which roughly represents its contribution to the original output. Backpropagation uses these error values to calculate the gradient of the loss function. In the second phase, this gradient is fed to the optimization method, which in turn uses it to update the weights, in an attempt to minimize the loss function.
	
	We have also seen that the backpropagation requires that the activation function used by the artificial neurons (or "nodes") to be differentiable.
	
	\begin{tcolorbox}[title=Remarks,colframe=black,arc=10pt]
	\textbf{R1.} Backpropagation requires a known, desired output for each input value in order to calculate the loss function gradient – it is therefore usually considered to be a supervised learning method.\\
	
	\textbf{R2.} The choice of learning rate $\eta$ is important for the method, since a high value can cause too strong a change, causing the minimum to be missed, while a too low learning rate slows the training unnecessarily. In order to avoid oscillation inside the network, such as alternating connection weights, and to improve the rate of convergence, there are refinements of this algorithm that use an adaptive learning rate.\\
	
	\textbf{R3.} Backpropagation learning does not require normalization of input vectors; however, normalization could improve performance.\\
	
	\textbf{R4.} Over-fitting is a major-concern in deep learning (and to humans also...) since large networks can have hundreds of millions of weights. In image recognition, the number of training images can be significantly increased by random jittering of the images. Another technique named \NewTerm{dropout} randomly deletes a fraction of the weights at each training iteration. Regularization (see page \pageref{regularization}) is used to assign a cost to the size of weights and many other ideas are being explored.
	\end{tcolorbox}
	For information... when the net is passed to the activation (transfer) function and the function's output is used for adjusting the weights we then speak of "\NewTerm{McCulloch–Pitts perceptron}\index{McCulloch–Pitts perceptron}" or just simply of a "\NewTerm{(feed-forward) perceptron}\index{feed-forward perceptron}". If the weights are adjusted only according to the weighted sum of the inputs (that is without passing through the activation function!), we then speak of "\NewTerm{adaptive linear neuron}\index{Adaptive Linear Neuron}\index{ADALINE}" (ADALINE) or of "\NewTerm{feed-forward adaptive linear neuron}\footnote{A multilayer network of ADALINE units is known as a MADALINE.}".
	
	Finally, don't forget (as we have seen earlier) that gradient descent with backpropagation is not guaranteed to find the global minimum of the error function, but only a local minimum!
	
	As the number of layers grew, we encounter mainly two calculatory difficulties (aside computer memory issues): "\NewTerm{exploding gradient}" and "\NewTerm{vanishing gradient}".

While the problem of exploding gradient can be treated quite well by applying simple techniques like gradient clipping and $L_1$ or $L_2$ regularization (see earlier above page \pageref{LASSO regularization}), the problem of vanishing gradient remained intractable for decades.

	What is vanishing gradient and why does it arise? To update the values of the parameters in neural networks, we have seen that the back-propagation algorithm was typically used and is based on the chain rule to calculated partial derivatives of some more or less complex functions. During gradient descent, the neural network's parameters receive an update proportional to the derivative of the cost function with respect to the current parameter in each iteration of training. The problem is that in some cases, the gradient will be vanishingly small, effectively preventing some parameters from chaining their value. In the worst case, this may completely stop the neural network from further training.

	Traditional activation functions, such as the hyperbolic tangent function i mentioned above, have gradients in the range $[0,1]$, and back-propagation computes gradients by the chain rule. That has the effect of multiplying $n$ of these small numbers to compute gradients of the earlier (leftmost) layers in an $n$-layer network, meaning that the gradient decreases exponentially with $n$. That results in the effect that the earlier layers train very slowly, if at all.
	
	\begin{tcolorbox}[title=Remark,colframe=black,arc=10pt]
	At this point the reader may have noticed that technically, logistic regression is equivalent to a single layer neural network with a sigmoid transfer function (and vice-versa).
	\end{tcolorbox}	
	
	\pagebreak
	\paragraph{Recursive Neural Networks (RNN)}\label{Recursive neural networks}\index{Recursive neural networks}\mbox{}\\\\
	Sometimes we want to remember an input for later use. There are many examples of such a situation, such as the stock market. To make a good investment judgement, we have to at least look at the stock data from a time window.

	The naive way to let neural network accept a time series data is connecting several neural networks together. Each of the neural networks handles one time step. Instead of feeding the data at each individual time step, you provide data at all time steps within a window, or a context, to the neural network.
	
	A "\NewTerm{recurrent neural network}" (RNN) is a class of artificial neural networks where connections between nodes form a directed graph along a temporal sequence. This allows it to exhibit temporal dynamic behaviour. Derived from feed-forward neural networks, RNNs can use their internal state (memory) to process variable length sequences of inputs.
	
	There are three common type of recursive neural network\footnote{Huge thanks to Shi Yan, senior software engineer at NVIDIA, for having authorized us to use and slightly modify his LSTM illustrations!} (recurrent neural network, long short term memory unit and gated recurrent unit):
	\begin{figure}[H]
		\centering
		\includegraphics[width=1.0\textwidth]{img/computing/rnn_lstm_gru.jpg}
		\caption{Three common type of recursive neural networks}
	\end{figure}
	As we will provide further below the equations of LSTM (Long-Short Term Memory) and GRU (Gated Recurrent Unit), the reader will be able to notice that GRU (introduced in 2014) are a special case of LSTM (introduced in 1997)!
	
	\begin{tcolorbox}[title=Remark,colframe=black,arc=10pt]
	The GRU is an alternative to the LSTM which is similarly difficult to justify. It seems that comparison of the GRU to the LSTM and its variants provides evidence that the GRU outperforms
the LSTM on nearly all tasks except language modelling.
	\end{tcolorbox}

	A lot of times, you need to process data that has \underline{periodic patterns} (LSTM at the day we write these lines seems to be efficient only for such patterns!). As a silly example, suppose you want to predict Christmas tree sales. This is a very seasonal thing and likely to peak only once a year. So a good strategy to predict Christmas tree sale is looking at the data from exactly a year back. For this kind of problems, you either need to have a big context to include ancient data points, or you have a good memory. You know what data is valuable to remember for later use and what needs to be forgotten when it is useless.

	Theoretically the naively connected neural network, so named "recurrent neural network":
	\begin{figure}[H]
		\centering
		\includegraphics[width=1.0\textwidth]{img/computing/vanilla_rnn.jpg}
		\caption{Recurrent neural network}
	\end{figure}
	
	can work. But in practice, it suffers from two problems: vanishing gradient and exploding gradient, which make it unusable.

	Then later,"\NewTerm{Long-Short Term memory Neural Networks}\index{long-short term memory neural network}\label{long-short term memory neural network}" (LSTM) were introduced by Hochreiter \& Schmidhuber (1997) \cite{doi:10.1162/neco.1997.9.8.1735} to solve this issue by explicitly introducing a memory unit, called the cell into the network. This is the diagram of a LSTM building block:
	\begin{figure}[H]
		\centering
		\includegraphics[width=1.0\textwidth]{img/computing/lstm_building_block.jpg}
		\caption{LSTM (long short term memory) building block}
	\end{figure} 
	Obviously in then general case most variables are vectors. But in most typical univariate forecasting techniques, most vectors (excepted $\vec{X}_t$) are one-dimensional vectors: ie scalars!
	\begin{tcolorbox}[title=Remark,colframe=black,arc=10pt]
	LSTM networks are a type of RNN that uses special units in addition to standard units. Again (!) Standard RNNs (Recurrent Neural Networks) suffer from vanishing and exploding gradient problems. LSTMs (Long Short Term Memory) deal with these problems by introducing new gates, such as input and forget gates, which allow for a better control over the gradient flow and enable better preservation of long-range dependencies.
	\end{tcolorbox}
	At a first sight, this looks intimidating. Let's ignore the internals, but only look at the inputs and outputs of the unit. The network takes three inputs. $X_t$ is the input of the current time step. $h_{t-1}$ is the output from the previous LSTM unit and $C_{t-1}$ is the "memory" of the previous unit, which may be seen as the most important input. As for outputs, $h_t$ is the output of the current network. $C_t$ is the memory of the current unit.
	
	Therefore, this single unit makes decision by considering the current input, previous output and previous memory. And it generates a new output and alters its memory.
	\begin{figure}[H]
		\centering
		\includegraphics[width=1.0\textwidth]{img/computing/lstm_overall.jpg}
		\caption{Typical LSTM overall structure}
	\end{figure}
	The way its internal memory $C_t$ changes is pretty similar to piping water through a pipe. Assuming the memory is water, it flows into a pipe. You want to change this memory flow along the way and this change is controlled by two valves:
	\begin{enumerate}
		\item The first valve is called the "\NewTerm{forget valve}". If you shut it, no old memory will be kept. If you fully open this valve, all old memory will pass through.
		\begin{figure}[H]
			\centering
			\includegraphics[scale=0.5]{img/computing/forget_valve.jpg}
		\end{figure}
	
		\item The second valve is the new "\NewTerm{memory valve}". New memory will come in through a T shaped joint like below and merge with the old memory. Exactly how much new memory should come in is controlled by the second valve.
		\begin{figure}[H]
			\centering
			\includegraphics[scale=0.5]{img/computing/memory_valve.jpg}
		\end{figure}
	\end{enumerate}
	On the LSTM diagram below the red area is the memory pipe (do not confuse it with the  "memory valve"!). The input is the old memory (a vector). The first cross $\times$ it passes through is the forget valve. It is actually an element-wise multiplication operation (ie "Hadamard product"). So if you multiply the old memory $C_{t-1}$ with a vector that is close to $0$, that means you want to forget most of the old memory. You let the old memory goes through, if your forget valve equals $1$.
	\begin{figure}[H]
		\centering
		\includegraphics[width=1.0\textwidth]{img/computing/lstm_memory_pipe.jpg}
		\caption[]{Memory pipe with memory valve and forget valve of a LSTM}
	\end{figure}
	Then the second operation the memory flow will go through is this $+$ operator. This operator means piece-wise summation. It resembles the T shape joint pipe. New memory and the old memory will merge by this operation.
	
	After these two operations, you have the old memory $C_{t-1}$ changed to the new memory $C_t$.
	
	Now let us look at the input of both valves more in details.

	We begin with the "forget valve"! It is controlled by a simple feed-forward one layer neural network. The inputs of the neural network is $h_{t-1}$, the output of the previous LSTM block, $X_t$, the input for the current LSTM block, $C_{t-1}$, the memory of the previous block and finally a bias vector $b_0$. This neural network has a sigmoid function as activation, and it's output vector is the forget valve, which will applied to the old memory $C_{t-1}$ by element-wise multiplication.
	
	\begin{figure}[H]
		\centering
		\includegraphics[width=1.0\textwidth]{img/computing/lstm_memory_pipe_forget_valve.jpg}
		\caption[]{Details of the forget valve on the memory pipe}
	\end{figure}
	Now the second valve named "memory valve"!	Again, it is a one layer simple neural network that takes the same inputs as the forget valve. This valve controls how much the new memory should influence the old memory:
	\begin{figure}[H]
		\centering
		\includegraphics[width=1.0\textwidth]{img/computing/lstm_memory_pipe_below_memory_valve.jpg}
		\caption[]{Details of the memory valve below the memory pipe}
	\end{figure}
	The output of this network will element-wise multiple the new memory valve, and add to the old memory to form the new memory:
	\begin{figure}[H]
		\centering
		\includegraphics[width=1.0\textwidth]{img/computing/lstm_add_new_a_old_memory.jpg}
	\end{figure}
	And finally, we need to generate the output for this LSTM unit. This step has an output valve that is controlled by the new memory, the previous output $h_{t-1}$, the input $X_t$ and a bias vector. This valve controls how much new memory should output to the next LSTM unit:
	\begin{figure}[H]
		\centering
		\includegraphics[width=1.0\textwidth]{img/computing/lstm_output.jpg}
	\end{figure}
	Let us see now the mathematical point of view. For this consider again the LSTM block but with more explicit notations (for recall, $\diamond$ is the vector element-wise multiplication):
	\begin{figure}[H]
		\centering
		\includegraphics[width=1.0\textwidth]{img/computing/lstm_explicit.jpg}
	\end{figure}
	And here are the corresponding relations:
	\begin{itemize}
		\item Gating variables (respectively $\vec{f}_t$ forget gate, $\vec{i}_t$ update gate, $\vec{o}_t$ output gate):
		
		
		\item Candidate (memory) cell state:
		
		
		\item Cell and hidden state:
		
	\end{itemize}
	
	\begin{tcolorbox}[title=Remark,colframe=black,arc=10pt]
	For a "\NewTerm{Gated Recurrent Unit Neural Network}\index{gated recurrent unit neural network}\label{gated recurrent unit neural network}" (GRU), only two relations change:
	
	hence the fact that GRU is a special case of LSTM!
	\end{tcolorbox}
	
	The core reason that recurrent nets are more exciting is that they allow us to operate over sequences of vectors: Sequences in the input, the output, or in the most general case both. A few examples may make this more concrete:
	 \begin{figure}[H]
		\centering
		\includegraphics[width=1.0\textwidth]{img/computing/rnn_outputs_inputs.jpg}
		\caption[]{Each rectangle is a vector and arrows represent functions (e.g. matrix multiply). Input vectors are in red, output vectors are in blue and green vectors hold the RNN's state (more on this soon). From left to right: (1) Vanilla mode of processing without RNN, from fixed-sized input to fixed-sized output (e.g. image classification). (2) Sequence output (e.g. image captioning takes an image and outputs a sentence of words). (3) Sequence input (e.g. sentiment analysis where a given sentence is classified as expressing positive or negative sentiment). (4) Sequence input and sequence output (e.g. Machine Translation: an RNN reads a sentence in English and then outputs a sentence in French). (5) Synced sequence input and output (e.g. video classification where we wish to label each frame of the video). Notice that in every case are no pre-specified constraints on the lengths sequences because the recurrent transformation (green) is fixed and can be applied as many times as we like.}
	\end{figure}
	RNNs are often used in text processing because sentences and texts are naturally sequences
of either words/punctuation marks or sequences of characters. For the same reason, recurrent
neural networks are also used in speech processing.

	In the case of a recurrent neural network, the loss function $L$ of all time:
	
	And as always we do first a forward pass (first estimation based on initial chosen values) and afterwards we correct the weights with a backward pass done at each point in time (BPTT: Back-propagation Through Time). At time-step $T$ the derivative of the loss $L_\text{RNN}$ with respect to the different weight matrices $W$ and biases is expressed as follows:
	
	Remember that during our example introducing back-propagation earlier above, we used the chain rule twice. The ideal here would be to write something similar:
	
	However this is incomplete as $h_t$ depends on previous $h_{t-\ldots}$. Therefore we should write instead:
	
	
	\begin{tcolorbox}[title=Remark,colframe=black,arc=10pt]
	A recursive network is just a generalization of a recurrent network. In a recurrent network the weights are shared (and dimensionality remains constant) along the length of the sequence because how would you deal with position-dependent weights when you encounter a sequence at test-time of different length to any you saw at train-time. In a recursive network the weights are shared (and dimensionality remains constant) at every node for the same reason.
	\end{tcolorbox}
	
	There are lots of others variant of the LSTM presented above, like "\NewTerm{Depth Gated RNNs}" by Yao, et al. (2015). There's also some completely different approach to tackling long-term dependencies, like "\NewTerm{Clockwork RNNs}" by Koutnik, et al. (2014).
	
	Which of these variants is best? Do the differences matter? Greff, et al. (2015) do a nice comparison of popular variants, finding that they're all about the same. Jozefowicz, et al. (2015) tested more than ten thousand RNN architectures, finding some that worked better than LSTMs on certain tasks \cite{jozefowicz2015empirical}.
	
	\begin{tcolorbox}[title=Remark,colframe=black,arc=10pt]
	For people interested in practical application, see our \texttt{R} companion book where there is an example with a vanilla RNN, GRU and LSTM neural network on the same meteorological dateset.
	\end{tcolorbox}
	
	\pagebreak
	\paragraph{Convolutional Neural Networks (CNN)}\label{convolutional neural network}\mbox{}\\\\
	A "\NewTerm{Convolutional Neural Network}\index{convolutional neural network}" (ConvNet/CNN) is a class of Deep Learning  feed-forward artificial neural networks algorithm which can take in an input image, assign importance (learnable weights and biases) to various aspects/objects in the image and be able to differentiate one from the other\footnote{A common misconception is to think that CNN are dedicated only to pure Computer Vision applications. However they are also used in Time Series analysis in Finance or Supply chain to detect anomalies thanks to Spectral Residual CNN (SR-CNN). Softwares like Microsoft Power BI have such tools natively available.}.
	
	The whole idea of Convolutional Neural Networks is inspired by the biology of the eye. While we as humans perceive a visual image as a detailed, coloured image of the world around us, there is actually quiet a lot of processing done in our brain to get to this point. The higher we go into the brain, the more concrete features are detected by the cells. With combination of cells that can perceive more and more complex contrast patterns, the brain is able to form cells that react to very specific visual stimulation, like cells that respond when we see cats or dogs.
	
	\begin{tcolorbox}[title=Remark,colframe=black,arc=10pt]
	CNNs are inspired by the layered structure of mammalian visual cortex based on studies of Hubel and Wiesel (1968) in which the understanding of a scenery occurs through a hierarchical construction of images at different levels of abstraction from simple primitive feature extraction in earlier layers to more complex and expressive high-level contextual features in the later layers (Aggarwal, 2018).
	\end{tcolorbox}
	
	What researchers did with Convolutional Neural Networks is exactly the same: CNN try to use this concept of combining low-level features in the image to higher and higher-level features, until we have cells that react to very specific things: Fur, eyes, cat ears etc. Then, we use a classic Neural Network to combine these features to a meaningful context: Two ears, fur and two eyes will with a high probability be a cat. You get the idea.
	
	The objective of the Convolution Operation is to extract the high-level features such as edges, from the input image. ConvNets need not be limited to only one Convolutional Layer. Conventionally, the first ConvLayer is responsible for capturing the Low-Level features such as edges, color, gradient orientation, etc. With added layers, the architecture adapts to the High-Level features as well, giving us a network which has the wholesome understanding of images in the dataset, similar to how we would.
	
	\begin{tcolorbox}[colback=red!5,borderline={1mm}{2mm}{red!5},arc=0mm,boxrule=0pt]
	\bcbombe Caution! The fields of CNN is in its nascent age. At the time we write these lines there are, as far as we know, no textbooks for "pure mathematical" lovers detailing in a purely detailed mathematical way each step of a CNN. 
	\end{tcolorbox}
	
	The reader may have notice that in images, pixels that are close to one another usually represent the same type of information: sky water, leaver, fur bricks, and so on. The exception from the rule are the edges: the parts of an image where two different objects "touch" one another.
	
	If we can train the neural network to recognize regions of the same information as well as the edges, this knowledge would allow the neural network to predict the object represented in the image. For example, if the neural network detected multiple skin regions and edges that look like parts of an oval with skin-like tone on the inside and bluish tone on the outside, then it is likely that it's a face on the sky background. If our goal is to detect people on pictures, the neural network will most likely succeed in predicting a person in this picture. 
	
	\begin{figure}[H]
		\centering
		\includegraphics[width=1.0\textwidth]{img/computing/cnn.jpg}
		\caption[]{The big picture of a convolution Neural Network}
	\end{figure}
	Or explicitly under the hood (even if the reader may not understand well the vocabulary in the image, seeing it now may help to grasp what will follow):
	\begin{figure}[H]
		\centering
		\includegraphics[width=1.0\textwidth]{img/computing/cnn_car.jpg}
	\end{figure}
	
	In the figure below, we have an RGB image which has been separated by its three color planes: Red, Green, and Blue. There are a number of such color spaces in which images exist — Grayscale, RGB, HSV, CMYK, etc.
	\begin{figure}[H]
		\centering
		\includegraphics[scale=1]{img/computing/rgb_image.jpg}
		\caption[]{$4\times 4\times 3$ RGB Image}
	\end{figure}
	The reader can imagine how computationally intensive things would get once the images reach dimensions, say $8$K ($7680\times 4320$). The role of the ConvNet is to reduce the images into a form which is easier to process, without losing features which are critical for getting a good prediction. This is important when we are to design an architecture which is not only good at learning features but also is scalable to massive datasets.
	
	Having in mind that the most important information in the image is local, we can split the image into square patches using a moving window approach\footnote{Consider this as if we looked at a dollar bill in a microscope. To see the whole bill we have to gradually move our bill from left to right and from top to bottom. At each moment in time, we see only a part of the bill of fixed dimensions. This approach is called "moving window"}. We can then train multiple smaller regression models at once, each small regression model receiving a square patch as input. The goal of each small regression model is to learn to detect a specific kind of pattern in the input patch. For example, one small regression model will learn to detect the sky; another one will detect the grass, the third one will detect edges of a building, and so on. 
	
	In the below demonstration, the image is a $5\times 5\times 1$ input image, The element involved in carrying out the convolution operation in the first part of a convolutional layer is named the "\NewTerm{Kernel}" or "\NewTerm{filter}" and is denoted $K$, represented in the by the black border square. Below we have selected $K$ as a $2\times 2 \times 1$ matrix (\SeeChapter{see section Functional Analysis page \pageref{matrix convolution}}):
	
	\begin{figure}[H]
		\centering
		\includegraphics[width=1.0\textwidth]{img/computing/convolutional_network_example.jpg}
		\caption[]{A filter convolving across an image}
	\end{figure}
	As we can see above see above, the kernel, filter shifts $9$ times because of stride length $= 1$ (non-strided), every time performing a matrix multiplication operation between $K$ and the portion $P$ of the image over which the kernel is hovering.
	
	Also notice that if we denote by $H\times W$ the dimension on the input image (Height $\times$ Width), then the output dimension after the convolution by a kernel of dimension $k_1\times k_2$ will be:
	
	And if we denote by $S$ the stride value we then have:
	
	
	The filter moves to the right with a certain stride value till it parses the complete width. Moving on, it hops down to the beginning (left) of the image with the same Stride Value and repeats the process until the entire image is traversed.
	
	\begin{figure}[H]
		\centering
		\includegraphics[width=1\textwidth]{img/computing/convolution_volume.jpg}
		\caption[]{Convolution of a volume consisting of three matrices}
	\end{figure}
	An example of a convolution of a patch of a volume consisting of depth $L=3$ is shown above. The value of the convolution, $-3$, was obtained as:
	
	Or in the most general case this will be written (the expression is slightly different from what we saw during our study of matrix convolution):
	
	What some engineer also write sometimes (...) assuming the Kernel is zero centered:
	
	or :
	
	where $C$ denotes the number of layers, $I$ the input, $K$ the kernel.
	
	\begin{tcolorbox}[colframe=black,colback=white,sharp corners]
	\textbf{{\Large \ding{45}}Example:}\\\\
	Let us consider the following notation:
	
	Let's not zero-center $I$ and compute $C[2,2]$ with a $3\times 3$ kernel:
	
	We note that a $3\times 3$ kernel that is zero-centered (i.e., its origin is in the middle of the filter) has indices running from $-1$ to $+1$. To reiterate, $-1$ is a valid index for $K$ (not padded) because we are zero-centered, but anything lower than that will just return zero.
	\end{tcolorbox}
	
	\begin{tcolorbox}[title=Remark,colframe=black,arc=10pt]
	Given an input image $I$ and a filter (kernel) $K$ of dimensions $k_1\times k_2$, the cross-correlation operation is given by:
	
	Given an input image $I$ and a filter (kernel) $K$ of dimensions $k_1\times k_2$, the convolution operation is given by:
	
	It is easy to see that convolution is the same as cross-correlation with a flipped kernel i.e: for a kernel $K$ where $K(-m,-n)=K(m,n)$.
	\end{tcolorbox}
	
	In computer vision, CNNs often get volumes as input, since an image is usually represented by three channels: R, G, and B, each channel being a monochrome picture. 
	
	Two important properties of convolution are "\NewTerm{stride}" and "\NewTerm{padding}". Stride is the step size of the moving window. In the first example above, the stride is $1$, that is the filter slides to the right and to the bottom by one cell at a time. In the figure below we can see a partial example of convolution with stride $2$. We can see that the output matrix is smaller when stride is bigger:
	\begin{figure}[H]
		\centering
		\includegraphics[width=1.0\textwidth]{img/computing/convolutional_network_stride_2.jpg}
		\caption[]{Convolution with stride $2$}
	\end{figure}
	The convolutional formula above with a spride $s$ becomes:
	
	Padding allows getting a larger output matrix (often, we want the output of a convolution to have the same size as the input\footnote{It's quite common to see convolution layers with stride of $1$, filters of size $k$, and zero padding of size $(k-1)/2$ to preserve size!}); it's the width of the square of additional cells with which we surround the image (or volume) before we convolve it with the filter. The cells added by padding usually contain zeroes. In the first example, the padding is $0$, so no additional cells are added to the image. In the figure below, on the other hand, the stride is $2$ and padding is $1$, so a square of width $1$ of additional cells are added to the image. We can see that the output matrix is bigger when padding is bigger (to save space only the first tow of the nine convolutions are shown).
	\begin{figure}[H]
		\centering
		\includegraphics[width=0.8\textwidth]{img/computing/convolutional_network_stride_2_padding_1.jpg}
		\caption[]{Convolution with stride $2$}
	\end{figure}
	An example of an image with padding $2$ is shown below. Padding is helpful with larger filters because it allows them to better scan the boundaries of the image:
	\begin{figure}[H]
		\centering
		\includegraphics[scale=0.5]{img/computing/image_padding_2.jpg}
		\caption[]{Image with padding $2$}
	\end{figure}
	\begin{tcolorbox}[title=Remark,colframe=black,arc=10pt]
	There are two types of results to the operation — one in which the convolved feature is reduced in dimensionality as compared to the input, and the other in which the dimensionality is either increased or remains the same. This is done by applying "\NewTerm{Valid Padding}" in case of the former, or "\NewTerm{Same Padding}" in the case of the latter.
	\end{tcolorbox}
	
	This section would not be complete without presenting "\NewTerm{pooling}", a technique very often used in CNNs. Pooling works in a way very similar to convolution, as a filter applied using a moving window approach. However, instead of applying a trainable filter to an input matrix or a volume, pooling layer applies a fixed operator, usually either max ("\NewTerm{Max pooling}") or average ("\NewTerm{Average pooling}"). Similarly to convolution, pooling has hyperparameters: the size of the filter, the stride and the type of pooling.
	
	Mainly the purpose of pooling is to decrease the computational power required to process the data through dimensionality reduction. Furthermore, it is useful for extracting dominant features which are rotational and positional invariant, thus maintaining the process of effectively training of the model.
	
	 An example of max pooling with filter of size $2$ and stride $2$ is shown below:
	\begin{figure}[H]
		\centering
		\includegraphics[width=1.0\textwidth]{img/computing/convolutional_network_pooling}
		\caption[]{Pooling with filter of size $2$ and stride $2$}
	\end{figure}
	The Convolutional Layer and the Pooling Layer, together form the $i$-th layer of a Convolutional Neural Network. 
	\begin{figure}[H]
		\centering
		\includegraphics[width=1.0\textwidth]{img/computing/cnn_fully_connected_layer.jpg}
	\end{figure}
	Usually, a pooling layer follows a convolution layer, and it gets the output of convolution as input. When pooling is applied to a volume, each matrix in the volume is processed independently of others. Therefore, the output of the pooling layer applied to a volume of the same depth as the input. 
	
	Depending on the complexities in the images, the number of such layers may be increased for capturing low-levels details even further, but at the cost of more computational power.
	
	We may have noticed that we create quite a lot of images by running our input image through many different filters. Even though the images size will slightly decrease with each filter in each layer, we generally generate exponentially more and more data with each convolutional layer we add. To combat this issue, we use a process called MaxPooling: After filtering an image, we will reduce its size drastically by unifying a pixel neighbourhood to one single value. Most prominently, we use MaxPooling, meaning we take the maximum pixel value of a pixel neighbourhood, but we could also use other methods like MinPolling, AvgPolling or MedianPolling. 
	
	After going through the above process, we have successfully enabled the model to understand the features. Moving on, we are going to flatten the final output and feed it to a regular Neural Network for classification purposes.
	
	In the example below depicting a complete forward pass:
	\begin{enumerate}
		\item We start with an input image of size $5\times 5$.
		
		\item We then apply convolution using $2\times 2$ kernel and stride $=1$, that produces feature map of size $4\times 4$.
		
		\item We then apply a $2\times 2$ max-pooling with stride $2$, that reduces feature map to size $2\times 2$.
		
		\item We then apply a logistic sigmoid.
		
		\item Then one fully connected layer with two neurons.
		
		\item And an output layer
	\end{enumerate}
	\begin{figure}[H]
		\centering
		\includegraphics[width=1.0\textwidth]{img/computing/cnn_simple_explicit_case.jpg}
	\end{figure}
	
	
	In the figure below we can see how CNN are influenced by the presence of a perfect gray square on a photo:
	\begin{figure}[H]
		\centering
		\includegraphics[width=1.0\textwidth]{img/computing/cnn_gray_square_text.jpg}
	\end{figure}
	
	The last level of a CNN, the most popular known one in massmedia, is sometimes names a "\NewTerm{segmentation mask}" because the input image is reduce to its simple geometrical elements:
	\begin{figure}[H]
		\centering
		\includegraphics[width=1.0\textwidth]{img/computing/cnn_segmentation_mask.jpg}
	\end{figure}
	
	Our Convolutional neural network really consists of two parts: Convolutional layers and fully connected decision layers. Both are connected using a flattening layer that converts an array of 2D images to a single 1D list of numeric values.
	
	In most cases CNNs use a cross-entropy loss on the one-hot encoded output. For a single image the cross entropy loss looks like this:
	
	where $M$ is the number of classes (i.e. $1000$ in ImageNet) and $\hat{y}_c$ is the model's prediction for that class (i.e. the output of the softmax for class $c$). Due to the fact that the labels are one-hot encoded and $y$ is a $[1000\times 1]$ vector of ones and zeroes, $y_c$ is either $1$ or $0$. Thus, out of the whole sum only one term will actually be added: the one with $y_c=1$.

	There are various architectures of CNNs available which have been key in building algorithms which power and shall power AI as a whole in the foreseeable future. Like LeNet, AlexNet, VGGNet, GoogLeNet, ResNet, ZFNet, SR-CNN:
	\begin{figure}[H]
		\centering
		\includegraphics[scale=1]{img/computing/cnn_examples.pdf}
		\caption[Various type of CNN internal structures]{Various type of CNN internal structures (author: Aqeel Anwar)}
	\end{figure}
	
	
	\pagebreak
	\paragraph{Generative Adversarial Network (GAN)}\mbox{}\\\\
	A "\NewTerm{generative adversarial network}\index{generative adversarial network}\label{generative adversarial network}" (GAN) is a class of machine learning systems invented by Ian Goodfellow and his colleagues in 2014. Two neural networks contest with each other in a game (in the sense of game theory, often but not always in the form of a zero-sum game). Given a training set, this technique learns to generate new data with the same statistics as the training set. For example, a GAN trained on photographs can generate new photographs that look at least superficially authentic to human observers, having many realistic characteristics. Though originally proposed as a form of generative model for unsupervised learning, GANs seems also proven useful for semi-supervised learning, fully supervised learning and reinforcement learning.
	\begin{figure}[H]
		\centering
		\includegraphics[width=1.0\textwidth]{img/computing/gna_principle.jpg}
		\caption{General idea of a GAN}
	\end{figure} 
	The generative network generates candidates while the discriminative network evaluates them. The contest operates in terms of data distributions. Typically, the generative network learns to map from a latent space to a data distribution of interest, while the discriminative $D$ network distinguishes candidates produced by the generator $G$ from the true data distribution. The generative network's training objective is to increase the error rate of the discriminative network (i.e., "fool" the discriminator network by producing novel candidates that the discriminator thinks are not synthesized (are part of the true data distribution)).

	A known dataset serves as the initial training data for the discriminator $D$. Training it involves presenting it with samples from the training dataset, until it achieves acceptable accuracy. The generator trains based on whether it succeeds in fooling the discriminator. Typically the generator is seeded with randomized input that is sampled from a predefined latent space (e.g. a multivariate normal distribution). Thereafter, candidates synthesized by the generator are evaluated by the discriminator. Back-propagation is applied in both networks so that the generator produces better images, while the discriminator becomes more skilled at flagging synthetic images. The generator is typically a deconvolutional neural network, and the discriminator is a convolutional neural network.
	
	Training a vanilla GAN is like a 2-players game, where:
	\begin{itemize}
		\item The discriminator is trying to maximize the cost function $L(D,G)$
		
		\item The generator is trying to minimize the cost function $L(D,G)$
	\end{itemize}
	The discriminator is the generator's opponent, and performs a mapping $D(x)\in [0,1]$. Its goal is to look at sample images (that could be real or synthetic from the generator), and determine if they are real samples ($D(x)$ closer to $1$) or synthetic samples from the generator ($D(x)$ closer to $0$). $D(x)$ can be interpreted as the probability that the image is a real training example.

	The generator, $G(z),$ has parameters $\theta^{(G)},$ and the discriminator, $D(\mathbf{x}),$ has parameters $\theta^{(D)} .$ The generator can only control $\theta^{(G)},$ while the discriminator can only control $\theta^{(D)}$.
	
	\begin{tcolorbox}[title=Remark,colframe=black,arc=10pt]
	Keep in mind that $D(x)$ is simply a function that maps to $[0,1]$. So the possible values of $D(G(z)) + D(x)$ range from $0$ to $2$.
	\end{tcolorbox}
	
	In addition to this, the discriminator and generator have different cost functions they wish to optimize:
	\begin{itemize}
		\item This ought to be intuitive as the generator and discriminator have different goals.
	
		\item We denote the discriminator's cost function as $L^{(D)}\left(\theta^{(D)}, \theta^{(G)}\right) $. For convenience, we will sometimes denote this as $L(D)$.
	
		\item We denote the generator's cost function as $L^{(G)}\left(\theta^{(D)}, \theta^{(G)}\right) $. For convenience, we will sometimes denote this as $L(G)$
	\end{itemize}
	
	What is the solution?
	\begin{itemize}
		\item The discriminator wishes to minimize $L^{(D)}$, but can only do so by changing $\theta^{(D)}$
		
		\item The generator wishes to minimize $L^{(G)}$, but can only do so by changing $\theta^{(G)}$
	\end{itemize}
	This is slightly different from the optimization problems we've described thus far, where we have one set of parameters to minimize one cost function $L$.

	Instead of treating this as an optimization problem, we treat this as a game between two players. The solution to a game is called a Nash equilibrium. For GANs, a Nash equilibrium is a tuple, $(\theta^{(D)},\theta^{(G)})$ that is:
	\begin{itemize}
		\item A local minimum of $L(D)$ with respect to $\theta^{(D)}$
		\item A local minimum of $L(G)$ with respect to $\theta^{(G)}$
	\end{itemize}

	The simplest type of game to analyse is a zero-sum game, in which the sum of the generator's loss and the discriminator's loss is always zero. In a zero-sum game, the generator's loss is:
	
	The solution for a zero-sum game is named a minimax solution, where the goal is to minimize the maximum loss. Since the game is zero-sum, we can summarize (see proof just below) the entire game by stating that the loss function is explicitly (based on binary cross-entropy as introduced at page \pageref{cross-entropy}):
	
	Often condensed in the following form:
	
	with obviously:
	
	\begin{itemize}
		\item The discriminator wants to maximize the objective (i.e., its payoff) such that $D(x)$ is close to $1$ and $D(G(z))$ is close to zero.
	
		\item The generator wants to minimize the objective (i.e., its loss) so that $D(G(z))$ is close to $1$.
	\end{itemize}
	The Nash equilibrium of this particular game is achieved at:
	
	
	\begin{dem}
	Let us recall the definition of cross-entropy:
	
	Let us rewrite it $H(y,D(x))$ where $H$ remains the cross-entropy $x$ is sampled either from $p_\text{data}$ or from $p_\text{model}$ with a probability of $50\%$. More formally:
	
	We consider $y$ to be $1$ if $x$ is sampled from the real distribution and $0$ if it is sampled from the fake one. Finally, $D(x)$ represents the probability with which $D$ thinks that $x$ belongs to $p_\text{data}$. By writing the cross-entropy formula we get:
	
	where $N$ is the size of the dataset. Since each class has $N/2$ samples we can split this sum into two parts:
	
	The first of the two terms represents the samples from the $p_\text{data}$ distribution, while the second one the samples from the $ p_\text{model}$ distribution. Since all $y_i$ are equally likely to occur, we can convert the sums into expectations:
	
	At this point, we'll ignore $1/2$ from the equations since it's constant and thus irrelevant when optimizing this equation. Now, remember that samples that were drawn from $p_\text{model}$ were actually outputs from the generator (obviously this affects only the second term). If we substitute $D(x)$,$x \sim p_\text{model}$ with $D(G(z))$,$z\sim p_z$ we'll get:
	
	This is the final form of the discriminator loss.
	\begin{flushright}
		$\blacksquare$  Q.E.D.
	\end{flushright}
	\end{dem}
	
	\begin{tcolorbox}[title=Remark,colframe=black,arc=10pt]
	This whole endeavour was to provide a mathematical formulation to training GANs. In practice there are maaaany tricks that are invoked to effectively train a GAN, that are not depicted in the above equations.
	\end{tcolorbox}
	
	For GANs, in practice this game is implemented in an iterative numerical approach. It involves two steps:
	\begin{figure}[H]
		\centering
		\includegraphics[width=1.0\textwidth]{img/computing/gna_principle_training_discriminator.jpg}
		\caption{General idea of a GAN Discriminator training}
	\end{figure} 
	Gradient ascent for the discriminator. We modify $\theta^{(D)}$ to maximize the minimax objective:
	
	
	\begin{figure}[H]
		\centering
		\includegraphics[width=1.0\textwidth]{img/computing/gna_principle_training_generator.jpg}
		\caption{General idea of a GAN Generator training}
	\end{figure} 
	Gradient descent on the discriminator. We modify $\theta^{(G)}$ to minimize the minimax objective:
	
	For the sake of intuition, consider that instead of optimizing with respect to $\theta^{(D)},$ we get to optimize $D(x)$ for every value of $x$. Further, assume that $p_{\text {data }}$ and $p_{\text {model }}$ are non-zero everywhere. What would be the optimal strategy for $D(x)$?
	
	To answer this question, we differentiate with respect to $D(x)$ and set the derivative equal to zero. In particular, we start with:
	
	Differentiating with respect to $D(x)$, we get:
	
	We can now set the derivative equal to zero to find the optimal $D(x)$. This leads to:
	
	A solution occurs when the integrands are equal for all $x$, i.e.:
	
	Rearranging terms, this gives:
	
	If the generator has high enough capacity, it will then move to set $p_{\text {model }}(x)=p_{\text {data }}(x)$ for all $x$. This results in the output:
	
	This is the Nash equilibrium.
	
	As stated, the current training paradigm has an important limitation. Note that the gradient of $\log (1-D(G(z)))$ is $-\frac{1}{1-D(G(z))}$, and thus has the following gradient as function of $D(G(z))$:
	\begin{figure}[H]
		\centering
		\includegraphics[width=0.8\textwidth]{img/computing/gan_gradient_issue.jpg}
	\end{figure}
	Therefore, if $D(G(z)) \cong 0$, as may happen early on in training when the discriminator can tell the difference between real and synthetic examples, the gradient is close to zero. This results in little learning for $\theta^{(G)}$, and thus in practice the generator cost function:
	
	is rarely ever used.
	
	Instead, we opt for a cost function that has a large gradient when $D(G(z)) \cong 0$, so that the generator is encouraged to learn much more early in training. This is the cost function:
	
	This still obtains the same overall goal of being minimized when $D(G(z)) = 1$, but now admits far more learning when the generator performs poorly.
	
	The gradient of this new cost for the generator encourages more learning when the generator performs poorly.
	\begin{figure}[H]
		\centering
		\includegraphics[width=0.8\textwidth]{img/computing/gan_gradien_correction.jpg}
	\end{figure}
	Note that by changing the cost function for the generator network, the game is no longer zero-sum. This is a heuristic change made to the game to solve the practical problem of saturating gradients when the generator isn't doing well!
	\begin{figure}[H]
		\centering
		\includegraphics[width=1.0\textwidth]{img/computing/gan_pix2pix.jpg}
	\end{figure}
	
	\begin{figure}[H]
		\centering
		\includegraphics[width=1.0\textwidth]{img/computing/gan_emotion_generator.jpg}
	\end{figure}
	
	\begin{figure}[H]
		\centering
		\includegraphics[width=1.0\textwidth]{img/computing/gan_painter_learning.jpg}
	\end{figure}
	
	\begin{tcolorbox}[title=Remark,colframe=black,arc=10pt]
	There are largely empirical good practices to train a GAN network like the fact to normalize the images between $-1$ and $+1$, and use $\tanh$ as the output of the generator model, let the prior on $z$ be Gaussian rather than uniform, avoid sparse gradients, avoid using ReLU or maxpool and use LeakyReLU instead and to downsample, increase the stride, etc.
	\end{tcolorbox}
	
	\pagebreak
	\paragraph{Objective, cost and loss function}\mbox{}\\\\
	In machine learning (ie Statistics...), people talk about objective function, cost function, loss function\footnote{Not to be confused with the "activation functions" used in neural networks!}. These are not very strict terms and they are highly related. However:
	\begin{itemize}
		\item "\NewTerm{Loss function}\index{loss function}", often denoted $L$, is usually a function defined on a \underline{single data point}, prediction and label, and measures the penalty. For example:
		\begin{itemize}
			\item Square loss (also named "quadratic loss", "$L_2$ loss" or MSE):
			
			used in linear regression.
			
			\item Mean Absolute Error (also named "$L_2$ loss" or MAE):
			
			used in linear regression.
			
			\item Cross entropy loss:
			
			used in binary classification.
	
			\item Hinge loss: 
			
			used in SVM.
			
			\item Huber loss:
			
			Typically used for regression. It's less sensitive to outliers than the MSE as it treats error as square only inside an interval. There is also a "pseudo-Huber loss" function and a "modified Huber loss".
	
			\item $0/1$ loss: 
			
			used in theoretical analysis.
			
			\item ...
		\end{itemize}
	
		\item "\NewTerm{Cost function}\index{cost function}", often denoted $J$, is usually more general. It might be a \underline{sum of loss functions} over your training set plus some model complexity penalty (regularization). For example:
		\begin{itemize}
			\item Mean Squared Error: 
			
	
			\item SVM cost function: 
			
			
			\item ...
		\end{itemize}
		Or as we have seen earlier typically for neural networks the following cost functions:
		\begin{table}[H]
			\resizebox{\textwidth}{!}{\centering
			\begin{tabular}{|l|l|l|}
			\hline
			\rowcolor[HTML]{C0C0C0} 
			\textbf{Cost function name} & \textbf{Mathematical expression} & \textbf{Gradient} \\ \hline
			 Quadratic cost & $J_\text{MST}\left(W,\vec{b},\vec{y},\vec{\phi}\right)=\dfrac{1}{2}\displaystyle\sum_{i=1}^n\left(\phi_i(z)-y_i\right)^2$ & $\nabla_\phi J_\text{MST}\left(W,\vec{b},\vec{y},\vec{\phi}\right) = \displaystyle\sum_{i=1}^n (\phi_i(z)-y_i)$ \\ \hline
			 Cross-entropy cost & $J_\text{CE}\left(W,\vec{b},\vec{y},\vec{\phi}\right) =-\displaystyle\sum_{i=1}^n\left[y_i\ln(\phi(z))+(1-y_i)\ln(1-y_i)\right]$ & $\nabla_\phi  J_\text{CE}\left(W,\vec{b},\vec{y},\vec{\phi}\right) =\displaystyle\sum_{i=1}^n\dfrac{\phi_i(z)-y_i}{(1-\phi_i(z))\phi_i(z)}$ \\ \hline
			 Exponential cost & $J_\text{exp}\left(W,\vec{b},\vec{y},\vec{\phi},\tau\right) =\tau \exp\left(\dfrac{1}{\tau} \displaystyle\sum_{i=1}^n \left(\phi_i(z)-y_i\right)^2 \right)$ & $\nabla_\phi  J_\text{exp}\left(W,\vec{b},\vec{y},\vec{\phi},\tau\right) =\dfrac{2}{\tau}J_\text{exp}\left(W,\vec{b},\vec{y},\vec{\phi},\tau\right)\displaystyle\sum_{i=1}^n \phi_i(z)-y_i$ \\ \hline
			 Hellinger distance & $J_\text{HD}\left(W,\vec{b},\vec{y},\vec{\phi}\right) =\dfrac{1}{\sqrt{2}}\displaystyle\sum_{i=1}^n \left(\sqrt{\phi_i(z)}-\sqrt{y_i} \right)^2$ & $\nabla_\phi  J_\text{exp}\left(W,\vec{b},\vec{y},\vec{\phi}\right) =\dfrac{1}{\sqrt{2}}\displaystyle\sum_{i=1}^n \dfrac{\sqrt{\phi_i(z)}-\sqrt{y_i}}{\sqrt{\phi_i(z)}}$ \\ \hline
			 Kullback-Leibler (KL) cost & $J_\text{KL}\left(W,\vec{b},\vec{y},\vec{\phi}\right) =\displaystyle\sum_{i=1}^n y_i\log\left(\dfrac{y_i}{\phi_i(z)}\right)$ & $\nabla_\phi  J_\text{GKL}\left(W,\vec{b},\vec{y},\vec{\phi}\right) =\displaystyle\sum_{i=1}^n \dfrac{\phi_i(z)-y_i}{\phi_i(z)}$ \\ \hline
			 Generalized KL cost & $J_\text{GKL}\left(W,\vec{b},\vec{y},\vec{\phi}\right) =\displaystyle\sum_{i=1}^n y_i\log\left(\dfrac{y_i}{\phi_i(z)}\right)-\displaystyle\sum_{i=1}^n y_i+\displaystyle\sum_{i=1}^n \phi_i(z)$ & $\nabla_\phi  J_\text{GKL}\left(W,\vec{b},\vec{y},\vec{\phi}\right) =\displaystyle\sum_{i=1}^n \dfrac{\phi_i(z)-y_i}{\phi_i(z)}$  \\ \hline
			 Itakura-Saito distance cost & $J_\text{IS}\left(W,\vec{b},\vec{y},\vec{\phi}\right) =\displaystyle\sum_{i=1}^n \left(\dfrac{y_i}{\phi_i(z)}-\log\left(\dfrac{y_i}{\phi_i(z)}\right)-1\right)$ & $\nabla_\phi  J_\text{IS}\left(W,\vec{b},\vec{y},\vec{\phi}\right) =\displaystyle\sum_{i=1}^n \dfrac{\phi_i(z)-y_i}{\phi_i^2(z)}$ \\ \hline
			 ... & ... & ... \\ \hline
			\end{tabular}}
		\end{table}
	
		\item "\NewTerm{Objective function}\index{objective function}" is the most general term for any function that you optimize during training. For example, a probability of generating training set in maximum likelihood approach is a well defined objective function, but it is not a loss function nor cost function (however you could define an equivalent cost function). For example:
		\begin{itemize}
			\item MLE (Maximum Likelihood Estimator) is a type of objective function (which you maximize)
	
			\item Divergence between classes can be an objective function but it is barely a cost function, unless you define something artificial
		\end{itemize}
	\end{itemize}
	Long story short, we may say that: A loss function is a part of a cost function which is a type of an objective function.
	
	\pagebreak
	\subsubsection{Genetic Algorithms}\label{genetic algorithms}
	Genetic algorithms (GAs) are iterated  stochastic optimization algorithms based on the mechanisms of natural selection and genetics belonging to the family of "\NewTerm{evolutionary algorithms}\index{evolutionary algorithms}". This is an optimization technique that has spread widely since the beginning of the 21st century through the version 14.0.6123 of Microsoft Excel wherein the solver incorporates an evolutionary algorithm by default as shown by in the screenshot below:
	\begin{figure}[H]
		\centering
		\includegraphics[scale=0.85]{img/computing/evolutionary_solver.jpg}
		\caption[]{Screenshot of the evolutionary Microsoft Excel 14.0.6123 solver}
	\end{figure}
	with the corresponding options:
	\begin{figure}[H]
		\centering
		\includegraphics[scale=0.85]{img/computing/evolutionary_solver_options.jpg}
		\caption[]{Screenshot of the evolutionary Microsoft Excel 14.0.6123 solver options}
	\end{figure}
	The process of the genetic algorithm is quite simple:
	\begin{enumerate}
		\item We start with an initial population of potential solutions (chromosomes) arbitrarily selected 

		\item We evaluate their relative performance (fitness) 

		\item Based on this performance, we create a new population of potential solutions using simple evolutionary operators: selection, crossover and mutation

		\item We start this cycle until we find a satisfactory solution
	\end{enumerate}
	GAs were originally developed by John Holland (1975). This is the book of Goldberg (1989) that we own their popularization. Their fields of application are widespread. Besides the economy (portfolio risk minimization), they are used for optimization functions in  finance, in optimal control theory (operational research), in the theory of repetitive and differentials games (namely: in evolutionary games and the prisoner's dilemma) and information retrieval (Google) and search for shortest path in graph theory (Internet routing or GPS). The reason for the large number of applications is clear: simplicity and efficiency. Of course, other stochastic exploration techniques exist, the Monte Carlo can be regarded as a similar concept.
	
	To summarize, Lerman and Ngouenet (1995) identified four main properties that make the fundamental difference between these algorithms and other methods:
	\begin{enumerate}
		\item The genetic algorithms use a coding of the input parameters, not the parameters themselves

		\item  Genetic algorithms work on a population of points, instead of a single point

		\item Genetic algorithms use only the values of the function considered, not its derivative, or other auxiliary knowledge

		\item  The algorithms use probabilistic transition rules, not deterministic one
	\end{enumerate}
	The simplicity of their mechanisms, their ease of implementation and efficiency even for complex problems led to a growing number of publication this recent years in the scientific community.
	
	\textbf{Definitions (\#\mydef):}
	\begin{enumerate}
		\item[D1.] A "\NewTerm{genetic algorithm}\index{genetic algorithm}" is defined by an individual / chromosome / sequence and a potential solution to the given problem.

		\item[D2.]  A "\NewTerm{population}\index{population (algorithm)}" is a set of chromosomes or points of the search space

		\item[D3.]  The "\NewTerm{environment}\index{environment}" is assimilated with the search space

		\item[D4.]  The function that we seek to maximize is named "\NewTerm{fitness function}\index{fitness function}"
	\end{enumerate}

	Before going further, we need to define more formally the above concepts but under the particular case of Binary coding!
	
	The organisms in competition are the "individuals". Given an alphabet $A=\{a_1,a_2,\ldots,a_n\}$, we assume that each individual can be represented by a word of fixed-length $l$ caught in the in $A^{*}$. 

	The word associated with an individual of the population will be named a "\NewTerm{chromosome}\index{chromosome}" or "\NewTerm{sequence}\index{sequence}" (the term is not quite equivalent to its biological namesake, however, it is common practice to use the term here too ) and thus given by $A$ of the length $l(A)$ with $\forall i\in[1,l]: a_i\in A=\{0,1\}$ (reason: assumption of binary coding).

	If there is no risk of confusion, we will identify the terms of "individual" and "chromosome".

	The individuals form a population $P$ of size $P$, denoted by:
	
	with $i=1\ldots N$.
	We will make another important statement, that is to say, there is a function $f$ from one sequence with positive values which we denote $f(A)$, named "\NewTerm{fitness function}\index{fitness function}" that to any $A_i$ associates real number such that for $i\neq j$:
	
	if and only if $A_i$ is better suited to the environment than $A_j$.

	Notice that the term "appropriate" is not defined. For this, we would characterize the environment in which the individuals evolve, what we will not do. In fact, since we assume the existence of such a function and we put it in equivalence to the degree of adaptation, it is automatically set by the definition of $f$.

	We will name "\NewTerm{generation}\index{generation (algorithm)}" a population at time $t$, what must be put in relation with the notion of lifetime or age. However, we place ourselves here in the particular case where each individual has a life equal to $1$, so the generation $(t + 1)$ consists of different individuals from the generation$ $t, we name them obviously the "\NewTerm{descendants}\index{descendants}". Conversely, individuals of generation $t$ are the "\NewTerm{ancestors}\index{ancestors}" of the individuals of the generation $(t + 1)$. We denote the generation at time $t$ by $P(t)$, thus the population at time $t$.

	Thus, a chromosome is seen as a bit sequence in binary code known as "\NewTerm{bit string}\index{bit string}". In the case of a non-binary coding, such as the real number encoding for example, then the sequence A contains only one point, we have then $A=\{a\}$ with $a\in\mathbb{R}$. 
	
	\begin{tcolorbox}[title=Remark,colframe=black,arc=10pt]
	The fitness (effectiveness) is given by a function with real positive values. In the case of binary encoding, we will often use a function of decoding $d$ that will gives the possibility to transform  a binary string to a real number:
	
	afterwards the fitness function is chosen such that it transforms this value into a positive value:
	
	\end{tcolorbox}	
	The purpose of a genetic algorithm is then simply to find the string that maximizes this function $f$. Of course, each individual problem will require its own functions $d$ and $f$.

	GAs are then based roughly on the following phases:
	\begin{enumerate}
		\item Initialization: an initial population of $N$ chromosomes is randomly chosen

		\item Evaluation: each chromosome is decoded and evaluated

		\item Selection: creation of a new population of $N$ chromosomes based on previous step by using an appropriate method of selection.

		\item Reproduction: possibility of crossover and mutation in the new population

		\item Return to the Evaluation phase until the stop of the algorithm
	\end{enumerate}
	Or for people that are more visual:
	\begin{figure}[H]
		\centering
		\includegraphics{img/computing/genetic_algorithm_flowchart.jpg}
	\end{figure}
	
	\paragraph{Encoding and Initial population}\mbox{}\\\\
	There exist three main types of coding:
	\begin{enumerate}
		\item Binary
		\item Gray
		\item Real
	\end{enumerate}
	We can easily move from one encoding to another. Some authors do not hesitate, moreover, to draw parallels with biology, by speaking of "genotype\index{genotype}" (\SeeChapter{see section Population Dynamics page \pageref{genotype}}) regarding the binary representation of an individual, and "phenotype\index{phenotype}" (\SeeChapter{see section Population Dynamics page \pageref{phenotype}}) with respect to its corresponding real value in the search space.

	Let us recall that the simplest transformation (decoding function $d$) of a binary string $A$ into an integer $x$ is done by the following rule (\SeeChapter{see section Numbers page \pageref{number power decomposition} or this section page \pageref{computer representation of numbers}}):
	
	where $l$ is the number of digits of the string minus $1$.	
	
	Therefore the chromosome $A=\{1,0,1,1\}$ has trivially for value:
	
	Obviously, the function needs to be adapted (by trial and error!) depending on the problem. Thus, if we seek to maximize a function $f:[0,1]\rightarrow [0,1]$ a possible method would be as follows (the size of the chromosome of course dependent on the desired accuracy):
	
	
	This can be assimilated to a harmonic series (\SeeChapter{see section Sequences and Series page \pageref{harmonic series}}). For accuracy to the fifth decimal place, we will put $l=17-1$ since:
	
	Again Another way to do would be to choose $d$ such as:
	
	Let us give an explanation of this choice:
	
	Let us put $l=n-1$:
	
	So, with $l=16$ we have $2^{17}-1=131071$ and:
	
	This last rule can be generalized. Thus, suppose that we seek to maximize ("normalize" would be a more suited term perhaps...) $f$ according to a real variable $x$. Given $D=[x_{\min},x_{\max}]$, with $D\subset \mathbb{R}$, the allowed  search space with $x_{\min}$ and $x_{\max}$ the lower and upper bounds of this space. Given $\mathrm{prec}$ the precision (decimal) with which we seek $x$. Given:
	
	the length (range) of the interval $D$. We then have to divide this interval at worst in:
	
	equal sub-intervals to meet accuracy expectations. For example, given $D=[-1,2]$ so we have $R=3$, if we wanted a precision $\mathrm{prec}=6$, then we must divide this interval in $n=3,000,000$ sub-intervals.
	
	Let $k$ denote the natural integer such that $2^k>n$, which in our example involves $k=22$ as:
	
	the transformation of a binary string $A=\{a_1,\ldots,a_l\}$ in a real number $x$ can then be run in three stages:
	\begin{enumerate}
		\item Conversion (base $2$ into base $10$):
		

		\item Normalization:
		

		\item Maximization:
		
	\end{enumerate}
	Or what remains the same directly in one step by using:
	
	Therefore for $f:[0,1]\rightarrow [0,1]$ and $\forall i,a_i=1$ we fall back well on:
	
	About the initialization phase, the procedure is quite simple. It consists of a random selection of $N$ individuals in the space of allowed individuals. In binary coding, according to the size $l$ of the string, we do for a chromosome $l$ sampling in $\{0,1\}$ with equal probability.
	
	\paragraph{Operators}\mbox{}\\\\
	Operators play a key role in the possible success of a GA. We number three main one: 
		\begin{enumerate}
			\item the selection operator
			\item the crossover operator
			\item the mutation operator
		\end{enumerate}
	If the principle of each of these operators is easy to understand, it is difficult to explain the isolated importance of each of these operators in the success of the AG. This is partly due to the fact that each of these operators acts according to various criteria that depends on its own characteristics (fitness of individuals, likelihood of activation of the operator, etc.).
		
	\pagebreak
	\subparagraph{Operator of selection}\mbox{}\\\\
	This operator may be the most important since it allows individuals in a population to survive, reproduce or die. Generally, the probability of survival of an individual will be directly connected to its relative effectiveness in the population.  The basic part of the selection process is to stochastically select from one generation to create the basis of the next generation. 

	There are several methods for reproduction.  The most known and used method is undoubtedly the biased Goldberg's (1989) lottery wheel (roulette wheel). According to this method, each chromosome is duplicated in a new population in proportion to its adaptive value. We perform in some way, as many sampling that there are individuals in the population. Thus, in the case of a binary coding, the fitness of a particular chromosome being $f (d (A))$, the probability with which it will be reintroduced into the new population of size $N$ is given by the relative fitness:
	
	Individuals with high fitness value thus have more chance of being selected by the wheel. We speak then of "\NewTerm{proportional selection}\index{proportional selection}":
	\begin{figure}[H]
		\centering
		\includegraphics{img/computing/goldberg_wheel.jpg}
		\caption{Example of Goldberg's wheel with five individuals with their respective relative fitness}
	\end{figure}
	Obviously the number of times the roulette wheel is spun is equal to the size of the new population.

	Each time the wheel stops this gives the fitter individuals the greatest chance of being selected for the next generation and subsequent: "\NewTerm{mating pool}\index{mating pool}".

	The major drawback of this method lies in the fact that an individual that is not the best may still dominate the selection (imagine the search for maxima of a function in $\mathbb{R}^2$, there may be several of them - maxima - and therefore we could get a wrong selection ...), we will speak rightly then of "\NewTerm{premature convergence}\index{premature convergence}" and this is one of the most common problems when using genetic algorithms. It can therefore also result in a loss of diversity by the domination of a super-individual. Another drawback is its poor performance towards the end when all individuals are alike.

	One solution to this problem lies not in the use of another method of selection but the use of a modified fitness function. So we can use a scaling to decrease or increase artificially the relative difference between the fitness of individuals.

	Briefly, there are other methods, the best known being that of the tournament (tournament selection) we draw two random individuals in the population and reproduce the best of both in the new population. We repeat this procedure until the new population $P$ is complete. This method gives good results. However, as important as the selection phase, it does not create new individuals in the population. This is the role of crossover and mutation operators.
	
	\subparagraph{Crossover operator}\mbox{}\\\\
	The crossover operator allows the creation of new individuals in a very simple process. It allows the exchange of information between chromosomes (individuals). First, two individuals, which then form a couple, are sample in the new population issued from the selection (or reproduction). Then one (or potentially many) crossing site is randomly draw (number between $1$ and $l-1$). Finally, according to a probability $P_c$ that the crossing is done, the end segments (in the case of a single crossing site) of both parents are then exchanged around this site:
	\begin{figure}[H]
		\centering
		\includegraphics{img/computing/crossover.jpg}
		\caption{Illustrative example of crossover}
	\end{figure}
	This operator allows the creation of two new individuals. However, an individual selected in the reproduction (selection) is not necessarily subjected to a crossover. The latter is carried out with a certain probability $P_c$. The more this probability is high and the more the population will undergo a crossover modification.

	Anyway, it is possible that the joint action of reproduction and the crossing is insufficient to ensure the success of the GA. Thus, in the case of binary encoding we have chosen so far, some information (ie the characters of the alphabet) may disappear from the population. Thus if no individual of the initial population contains a $1$ in the last position of the string and that we know a priori that this $1$ in the last position is part of the optimal string to find, all possible crosses will never show this $1$ initially unknown. In real number coding, such a situation can happen when using a simple crossover operator, it was such that the initial population was between $0$ and $40$ and that the optimal value was $50$. All possible combinations of convex digits belonging to the range $[0,40]$ will never allow to reach a the number of $50$. This is to address, among others, this problem that the mutation operator is used.
	
	As usually the crossover operation uses $2$ individuals. So, if you have $20$ individuals, we choose $10$ pairs to cross, and with a probability of $P_c=80\%$, on average we're going to cross only $8$ pairs.
	

	\subparagraph{Mutation operator}\mbox{}\\\\
	The purpose of this operator is to change randomly, with some probability., the value of a component of the individual. In the case of binary encoding, each bit $a_i\in\{0,1\}$ is replaced following a probability $P_m$ by its inverse ${a'}_i=1-a_i$. This is what is shown in the figure below. Like many crossover positions may be possible, we can very well admit that a same string can undergo several mutations.
	\begin{figure}[H]
		\centering
		\includegraphics{img/computing/mutation.jpg}
		\caption{Illustrative example of mutation}
	\end{figure}
	The mutation is traditionally considered a marginal operator even if somehow it gives to genetic algorithms the ergodic property (ie all points of the search space can be achieved). However operator is of great importance. It performs a dual role: perform a local search and / or out a global search (remote search).
	
	The operators of the genetic algorithm are guided by a number of parameters set in advance. The value of these parameters affect the success of failure of a genetic algorithm. These parameters are (the reader can compare this list with the parameters available in the Microsoft Excel solver screenshot given earlier above):
	\begin{itemize}
		\item The size of the initial population $N$, and the coding length $l$ of each individual (in the case of binary encoding). If $N$ is too large, the computation time of the algorithm can be very important, and if $N$ is too small, it may converge too quickly to the wrong chromosome.

		\item The crossover probability $P_c$, that depends on the form of the fitness function. Its choice is general heuristic (just like $P_m$). The higher it is, the more the initial population obviously undergoes significant changes. The generally accepted values are between $0.5$ and $0.9$.

		\item The probability of mutation $P_m$ is generally small since a high rate may lead to a suboptimal solution.
		
		Rather than reducing $P_m$, another way to avoid the best individuals to be altered is to use explicit report of the elite individuals in a certain proportion. So often, the top $5\%$, for example, of the population is directly reused directly, the operator of reproduction (selection) or mutation operating then only on the remaining $95\%$. This is named an "\NewTerm{elitist strategy}\index{elitist strategy}".
	\end{itemize}
	Let us now see an example of GA (example of Goldberg - 1989).
	\begin{tcolorbox}[colframe=black,colback=white,sharp corners]
	\textbf{{\Large \ding{45}}Example:}\\\\
	We want to find the maximum of the function $(f)=x$ on the interval $[0,31]$ where $x$ is an integer. The first step consist in coding the function. For example, we use a binary coding of $x$, the sequence (chromosome) containing a maximum of $5$ bits. Thus, we have $x=2\rightarrow \{0,0,0,1,0\}$, and also $x=31\rightarrow \{1,1,1,1,1\}$. We are therefore seeking maximum of a fitness function (we will choose $f(x)$ itself in this simple example) in a space of $2^5=32$ possible values of $x$.
	\begin{enumerate}
		\item Sampling and evaluation of the initial population

		We set the size of the population to $N=4$. We draw randomly $4$ chromosomes knowing that a chromosome consists of $5$ bits, and each bit has a $50\%$ probability of having a value of $0$ or $1$. The maximum, (randomly) $16$ is reached by the second sequence. Let us see how the algorithm will try to improve this result.

		First, we get the following table:
		\begin{table}[H]
		\begin{center}
			\definecolor{gris}{gray}{0.85}
				\begin{tabular}{|c|c|c|c|c|}
				\hline
				\multicolumn{1}{c}{\cellcolor{black!30}N$^\circ$} & 
	  \multicolumn{1}{c}{\cellcolor{black!30}\textbf{Chromosome}} & 
	  \multicolumn{1}{c}{\cellcolor{black!30}\textbf{Value}} & 
	  \multicolumn{1}{c}{\cellcolor{black!30}\textbf{Fitness}} & 
	  \multicolumn{1}{c}{\cellcolor{black!30}$\mathbf{P_i\%}$} \\ \hline
				\cellcolor{black!30}$1$ & $00101$ & $5$ & $5$ & $14.3$ \\ \hline	
				\cellcolor{black!30}$2$ & $10000$ & $16$ & $16$ & $45.7$ \\ \hline	
				\cellcolor{black!30}$3$ & $00010$ & $2$ & $2$ & $5.7$ \\ \hline	
				\cellcolor{black!30}$4$ & $01100$ & $12$ & $12$ & $24.2$ \\ \hhline{|=|=|=|=|=|}	
				\cellcolor{black!30}\textbf{Total} & & & $\mathbf{35}$ & $\mathbf{100}$ \\ \hline	
				\end{tabular}
		\end{center}
		\caption[]{Evolution (mutation) of chromosomes}
		\end{table}
		We turn again the Goldberg's wheel $4$ times to obtain the following sequence:
		\begin{table}[H]
		\begin{center}
			\definecolor{gris}{gray}{0.85}
				\begin{tabular}{|c|c|}
				\hline
				\multicolumn{1}{c}{\cellcolor{black!30}\textbf{Selection (sampling)}} & 
	  \multicolumn{1}{c}{\cellcolor{black!30}\textbf{Chromosome}}  \\ \hline
				\cellcolor{black!30}$1$ & $10000$  \\ \hline
				\cellcolor{black!30}$2$ & $01100$  \\ \hline
				\cellcolor{black!30}$3$ & $00101$  \\ \hline
				\cellcolor{black!30}$4$ & $00101$  \\ \hline
				\end{tabular}
		\end{center}
		\caption[]{Sampling sequence of chromosomes}
		\end{table}
		We see here well the risk that we would have to lose the Sequence N$^\circ 2$ from the start... that's the problem with this method. It can converge more slowly than others. However, the reader will notice that we have lost the sequence N$^\circ 3$.

		We now turn to the crossover part: the ancestors are randomly selected. We randomly draw a crossover location ("site" or "loci") in the sequence. The crossing then operates at this location with a probability $P_c$. The table below shows the consequences of this operator assuming chromosomes $1$ and $3$, afterwards $2$ and $4$ are paired, and each time the crossing takes place (e.g. with $P_c=1$):
	\end{enumerate}
	\end{tcolorbox}
	
	\begin{tcolorbox}[colframe=black,colback=white,sharp corners]
	\begin{table}[H]
		\centering
		\begin{tabular}{|l|l|l|}
		\hline
	    \cellcolor{black!30}& \cellcolor{black!30}$\pmb{l=2}$ & \cellcolor{black!30}$\pmb{l=3}$  \\ \hline
		\textbf{Original Sequences}\multirow{2}{*}{\cellcolor{black!30}} & $100|00$  & $01|100$  \\ 
		\cellcolor{black!30} & $001|01$ & $10|000$  \\ \hline
		\textbf{Crossed Sequences}\multirow{2}{*}{\cellcolor{black!30}} & $10001$  & $01000$  \\ 
		 \cellcolor{black!30} & $00100$ & $10100$ \\ \hline
		\end{tabular}
		\caption[]{Chromosome crossing}
	\end{table}
	We now turn to the mutation part: in this binary coding example, the mutation is the occasional random modification (low probability) of the value of a bit (bit reversal). We thus draw for each bit a random number between $0$ and $1$ and if this digit is less than $P_m$ then the mutation takes place. The table below with $P_m=0.05$ highlights this process:
	\begin{table}[H]
		\begin{center}
		\definecolor{gris}{gray}{0.85}
		\begin{tabular}{|c|c|c|c|}
		\hline
		\cellcolor{black!30}\textbf{Old chromosome} & 
		\cellcolor{black!30}\textbf{Random drawing} & \cellcolor{black!30}\textbf{New bit} & \cellcolor{black!30}\textbf{New chromosome} \\ \hline
		$10001$ & $15\;25\;36\;$ \textit{04} $\;12$ & $1$ & $10011$  \\ \hline
		$00100$ & $26\;89\;13\;48\;59$ & $-$ & $00100$  \\ \hline
		$01000$ & $32\;45\;87\;22\;65$ & $-$ & $01000$  \\ \hline
		$10100$ & $47\;$\textit{01}$\;85\;62\;35$ & $1$ & $11100$  \\ \hline
		\end{tabular}
		\end{center}
		\caption[]{Mutation of chromosomes}
	\end{table}
	Now that the new population is fully created, we can evaluate it again:
	\begin{table}[H]
		\begin{center}
		\definecolor{gris}{gray}{0.85}
		\begin{tabular}{|c|c|c|c|c|}
		\hline
		\cellcolor{black!30}N$^\circ$& 
		\cellcolor{black!30}\textbf{Chromosome} & \cellcolor{black!30}\textbf{Value} & \cellcolor{black!30}\textbf{Fitness}$f(x)$ & $\pmb{P_i\%}$ \\ \hline
		\cellcolor{black!30}$1$ & $10011$ & $19$ & $19$ & $32.2$\\ \hline
		\cellcolor{black!30}$2$ & $00100$ & $4$ & $4$ & $6.8$\\ \hline
		\cellcolor{black!30}$3$ & $01000$ & $8$ & $8$ & $13.5$\\ \hline
		\cellcolor{black!30}$4$ & $11100$ & $28$ & $28$ & $47.5$\\ \hhline{|=|=|=|=|=|}
		\cellcolor{black!30}\textbf{Total} &  & & \pmb{59} & \pmb{100} \\ \hline
		\end{tabular}
		\end{center}
		\caption[]{Evaluation of the mutation of chromosomes}
	\end{table}
	The maximum is now $28$ (N$^\circ 4$). So we went from $16$ to $28$ after a single generation. Of course, we must repeat the procedure from the selection stage until the overall maximum, $31$, is obtained, or until that a stop criterion has been satisfied.
	\end{tcolorbox}
	\begin{tcolorbox}[title=Remark,colframe=black,arc=10pt]
	It is possible to prove mathematically, what is remarkable !!!, that the portions of chromosomes that are found in the best individuals will tend to reproduce ...
	\end{tcolorbox}
	The reader interested can also take a look to the MATLAB™ companion book where we use GAs to find the optimum of the Rastriging function:
	\begin{figure}[H]
		\centering
		\includegraphics{img/computing/rastriging_function.jpg}
		\caption{Rastriging function in MATLAB™ 2013a}
	\end{figure}
	where it works quite well as show it the result below:
	\begin{figure}[H]
		\centering
		\includegraphics{img/computing/rastriging_function_matlab_optimum.jpg}
	\end{figure}
	and also for neural networks optimization.
	
	\pagebreak
	\subsubsection{Total Unduplicated Reach and Frequency Analysis (TURF)}
	"\NewTerm{TURF Analysis}\index{TURF Analysis}", an acronym for "\NewTerm{Total Unduplicated Reach and Frequency}\index{Total Unduplicated Reach and Frequency}", is a type of statistical analysis used for providing estimates of media or market potential and devising optimal communication and placement strategies given limited resources. TURF analysis identifies the number of users reached by a communication, and how often they are reached.

	Although originally used by media schedulers to maximize reach and frequency of media spending across different items (print, broadcast, etc.), TURF is also now used to provide estimates of market potential. For example, if a company plans to market a new yoghurt, they may consider launching $10$ possible flavours, but in reality, only three might be purchased in large quantities. The TURF algorithm identifies the optimal product line to maximize the total number of consumers who will purchase at least one SKU\footnote{A SKU is a distinct string of letters and numbers that helps retailers identify every product in their inventory and each product's specific traits, like its manufacturer, brand, price, style, color, and size.} (Stock Keeping Unit). Typically, when TURF is undertaken for optimizing a product range, the analysis only looks at the reach of the product range (ignoring the Frequency component of TURF).

	In order to obtain data on the items being evaluated, ratings/choices may be obtained via quantitative marketing research (such as a survey).
	
	Let us give a companion example! Consider we want to sell three new ice flavours. We did an non-representative survey on $10$ people that lead us to the following table:
	\begin{table}[H]
	\centering
	\begin{tabular}{|c|c|c|c|c|}
	\hline
	\rowcolor[HTML]{9B9B9B} 
	\textbf{Individual} & \textbf{Weight} & \textbf{Chocolate ($A$)} & \textbf{Vanilla ($B$)} & \textbf{Pistachio ($C$)} \\ \hline
	1 & 1 & 1 & 0 & 0 \\ \hline
	2 & 1 & 1 & 0 & 0 \\ \hline
	3 & 1 & 1 & 0 & 0 \\ \hline
	4 & 1 & 0 & 1 & 0 \\ \hline
	5 & 1 & 0 & 1 & 0 \\ \hline
	6 & 1 & 0 & 1 & 0 \\ \hline
	7 & 1 & 1 & 1 & 0 \\ \hline
	8 & 1 & 1 & 1 & 0 \\ \hline
	9 & 1 & 0 & 1 & 1 \\ \hline
	10 & 1 & 1 & 1 & 1 \\ \hline
	\end{tabular}
	\end{table}
	
	Three possible ice cream flavours available. After a survey, we get:
	\begin{itemize}
		\item Flavour $A$ was chosen by $60\%$ (6/10) of the people
		\begin{itemize}
			\item $50\%$ (3/6) chose $A$ exclusively
			\item $33.\bar{3}\%$ (2/6) chose also $B$
			\item $0\%$ (0/6) chose also $C$
			\item $16.\bar{6}\%$ (1/6) chose also $B$ plus $C$
		\end{itemize}
		\item Flavour $B$ was chosen by $70\%$ (7/10) of the people
		\begin{itemize}
			\item $42.85\%$ (3/7) chose $B$ exclusively
			\item $28.57\%$ (2/7) chose also $A$
			\item $14.29\%$ (1/7) chose also $C$
			\item $14.29\%$ (1/7) chose also $A$ plus $C$
		\end{itemize}
		\item Flavour $C$ was chosen by $20\%$ (2/10) of the people
		\begin{itemize}
			\item $0\%$  (0/2) chose $C$ exclusively
			\item $0\%$ (0/2) chose also $A$
			\item $50\%$ (1/2) chose also $B$
			\item $50\%$ (1/2) chose also $A$ plus $B$
		\end{itemize}
	\end{itemize}
	An obvious "macro-result" if we would have to choose only $2$ flavour, would be to choose flavours $A$ and $B$.
	
	However we must also focus on the combinations focus on all combinations of $2$ among $3$ flavours to check if it match our previous naive macro analysis (and that's may not be always the case!!!).
	
	We then have:
	
	to analyse! It's easy to do by hand and lead us to the following reaching rates:
	\begin{itemize}
		\item $100\%$ Would chose $A$ alone, $B$ alone or $A$ and $B$ together
		\item $70\%$ Would chose $A$ alone, $C$ alone or $A$ and $C$ together
		\item $70\%$ Would chose $B$ alone, $C$ alone or $B$ and $C$ together
	\end{itemize}
	So in this special case. The TURF analysis confirms the naive macro analysis!
	
	\begin{tcolorbox}[title=Remark,colframe=black,arc=10pt]
	The reader can found this example in our \texttt{R} companion book.
	\end{tcolorbox}
	
	TURF analysis allows us to see the effect of the combinations of answers that people make that we may not see in simple percentage choices. 
	
	Making decisions based on simple percentage choice can be wrong. TURF analysis then allows us to take into account the complex combinations of choices that people make when selecting products or services.
	
	Obviously, TURF analysis, has many flaws, among other the fact that  is purely deterministic. But this can be quite simply resolved using bootstrapping.
	
	\begin{flushright}
	\begin{tabular}{l c}
	\circled{60} & \pbox{20cm}{\score{3}{5} \\ {\tiny 23 votes,  58.26\%}} 
	\end{tabular} 
	\end{flushright}


	%to make section start on odd page
	\newpage
	\thispagestyle{empty}
	\mbox{}
	\section{Fractals}\label{fractals}
	\lettrine[lines=4]{\color{BrickRed}F}ractals are figures invariant by scale change (we also talk about "self-similar structures") and are the graphic representation of contractant recurrent sequences (for IFS fractals that we will see later) or not divergent (for escape-time fractals as we will see further below).
	
	The basic idea - simple and great ... at the same time - often involves taking a starting point, to build its image through a particular mathematical function, to take the image of the image and so on. The goal is to study how the successive points are allocate in the global target set of the defined function, if they are approaching a limit or if they roam between different values that can we explain, if more points in part of the set than another?
	
	The advantage of this type of questions concerns both the study of the evolution of biological populations than the future of the solar system, 3D computing (the origin being the generation of mountains for 3D landscapes) changes in stock prices or random number generation in particular fields, or even medical diagnostic (especially for brain or heart).
	
	See below a simple example image by image:
	\begin{figure}[H]
		\centering
		\includegraphics{img/geometry/fractal_mountain_1.jpg}
	\end{figure}
	\begin{figure}[H]
		\centering
		\includegraphics{img/geometry/fractal_mountain_2.jpg}
	\end{figure}
	\begin{figure}[H]
		\centering
		\includegraphics{img/geometry/fractal_mountain_3.jpg}
	\end{figure}
	\begin{figure}[H]
		\centering
		\includegraphics{img/geometry/fractal_mountain_4.jpg}
	\end{figure}
	\begin{figure}[H]
		\centering
		\includegraphics{img/geometry/fractal_mountain_5.jpg}
		\caption{Pseudo-mountains generation from a random fractal (probabilistic fractal)}
	\end{figure}
	For the average person, fractals are used to look pretty. But they have far more serious applications: for example we already saw in this book that some of these "attractive" images reproduced physical phenomena (population dynamics for the Feigenbaum's Fractal, turbulence in a fluid with the Lorentz attractor, distribution of galaxies, L-Fractals, clusters and supercluster of galaxies, ...). Fractals have also found applications in music (with software generating fractal music) and in film (3D to generate mountains, fire, grass). Finally, in the field of computer graphics, fractals are used to compress images very effectively, with consistent quality regardless of the zoom, they help to create realistic textures, and can afford to dither an image with good results. Fractals are also used to reduce the size of the receive antennas and to extend their effectiveness frequency spectrum (some of our cell phones of the early 21st century have fractal receptor of the type "Sierpinski carpet" - see below - because of all types of frequencies they need to manage!). In civil engineering fractals are used for building some sound absorbers walls. And many other things...
	
	This fractal geometry differs from the Euclidean geometry first by its definition: the figures of Euclidean geometry are generally determined by algebraic relations, while the fractal curves are defined recursively as we have already mentioned. Fractals also have fractional dimensions (we have already discussed this topic in the section of Euclidean geometry when defining the concept of dimension). On the other hand, we must not neglect their autosimilar appearance: each part of a fractal can be observed at any scale: each part is (essentially) a copy of the whole.
	
	\begin{tcolorbox}[title=Remark,colframe=black,arc=10pt]
	The developments that follow could easily have been put in the section of Sequences and Series or even of Functional Analysis or seen as a special case of the section Topology reduces to the Euclidean space (this is why you will find here also many references to the topology section). Our choice is pedagogical as well as for the section of Cryptography, in the sense that it is much more interesting for a high-school student to see an application of abstract concepts of topology in a practical framework (and furthermore aesthetics) where they are absolutely necessary for a proper understanding of the subject rather than in a framework where we can escape them very well without too suffer. The reader will find here some developments and theorems proposed elsewhere in this book and this only in the order to avoid having too "turn pages" too much.
	\end{tcolorbox}	
	
	Natural fractals are named "\NewTerm{non-deterministic objects}\index{non-deterministic objects}" because the dynamic process that allows their creation itself varies with time randomly (see the section of Population Dynamics for an excellent example). Nevertheless, we can try to model dynamic systems that lead to fractal objects under a rigorous mathematical form (that is still a good example of the way in which the mathematicians manage to make a simple and intuitive concept into a concrete abstract mathematical model and somewhat confusing as for Knot Theory).
	
	In this section, we will consider the study of two families of fractals that will be in order:
	\begin{enumerate}
		\item Deterministic fractals based on iterated function that are strictly self-similar. They are generated, as we shall see, by the recursive application of contracting functions on subsets of a metric space. The fixed point theorem will guarantee (as we shall also see!) the existence and uniqueness of a "\NewTerm{fixed subset}\index{fixed subset}" of the metric space, towards which every subset converge.
		
		\item Escape-time fractals (also known as "fractals by induction") that are not strictly self-similar: They are generated as we shall see later by recurring non-divergent sequences. The fixed point theorem serving as guarantee for the non-divergence of the function with respect to the chosen starting points.
	\end{enumerate}
	
	\subsection{IFS Fractals}
	Let us start by looking at the first family of fractals discovered by Michael Barnsley in 1987: the "\NewTerm{deterministic iterated function systems IFS}\index{deterministic iterated function systems}".
	
	Of all fractals, figures only those built by iterated function systems usually shows the self-similarity property, meaning that their complexity is invariant under change of scale.
	
	Let us start by "bounding" the thing...
	
	We take an initial geometry $E_0$ of space $E$, a function $f$ from $E$ to $E$ such that:
	
	(which requires that the initial object can not leave its own definition domain through the iteration of the function $f$) and we create the discrete dynamical system defined by:
	
	Under certain conditions we will now see, the sequence of geometric objects $(E_0)$ "tends" to a limit, which is often a fractal object (we will see further below some famous examples).
	
	Naturally, there is a rigorous mathematical framework in which the mentioned conditions and the verb "tends" have a precise definition. In particular, the objects $E_n$ are all compacts of $E$, that is to say bounded subsets (that we can include in a segment if $E$ is a straight line, in a disk if $E$ is a plane or a ball if $E$ is the three-dimensional space) and closed (every convergent sequence of $E_n$ has its limit in $E$). We place ourselves then in the compact metric space, equipped with the Hausdorff distance (see below for definition) for which we will show that it is complete when we work with compact sets the plane and  of space, and we will check that $f$ is an "\NewTerm{Hutchinson operator}\index{Hutchinson operator}", i.e. a contraction application from the space of the compact in itself for that distance. It then will then just remain to apply the fixed point theorem.
	
	Dynamic systems of this type are said to be "deterministic", and therefore named "IFS" (iterated function systems deterministic). Let us precise that the limit of the IFS is named the "\NewTerm{attractor of the IFS}\index{attractor of the IFS}". We can show that under the conditions mentioned above, this attractor does not depend on the shape of the original geometric object (we will see practical examples further below).
	
	Initially, we will limit our study to $\mathbb{R}$ (the general case is given in section Topology) knowing anyway that a generalization to the two-dimensional Euclidean space does not require too big and intellectual work and that the whole complex is isomorphic to it.
	
	\textbf{Definition (\#\mydef):}
	To enable us to define the boundaries of our fractal functions let us consider $X \subseteq \mathbb{R}$. We say that $\xi$ is the "\NewTerm{supremum}\index{supremum}" of $X$ and denote it by:
	
	if $\xi$ is the smallest "\NewTerm{upper bound}\index{upper bound}" of $X$ (an upper bound of $X$ is a number $a$ that satisfies $\forall x \in X,x\leq a$).
	
	Similarly, we say that $\xi$ is an "\NewTerm{infimum}\index{infimum}" $X$ and denote it by:
	
	if $\xi$ is the biggest "\NewTerm{lower bound}\index{lower bound}" of $X$ (a lower bound of $X$ is a number $a$ that satisfies $\forall x \in X,a\leq x$).
	
	\begin{tcolorbox}[title=Remark,colframe=black,arc=10pt]
	We often use the following characterization of the supremum:
	
	if and only if:
	
	which is almost obvious because we can approach as close as we want of $\xi$ with elements of $X$ (think with small $\varepsilon$). For information, we then also in the same idea:
	
	if and only if:
	
	\end{tcolorbox}	
	We consider as intuitive that if $X \subseteq \mathbb{R}$ has an upper bound, that is to say if there exists $a \in \mathbb{R}$ as $\forall x \in X,x \leq a$ (respectively lower bounded), then $X$ has a supremum (respectively infimum).
	
	We will see later that it is this property that will give us the possibility to prove later $\mathbb{R}$ a "\NewTerm{complete metric space}\index{complete metric space}"!
	\begin{tcolorbox}[title=Remark,colframe=black,arc=10pt]
	By the way, let us underline the importance of taking $\mathbb{R}$ as a definition for metric space for this property to be satisfied. We can in fact notice that it is not verified in the set $\mathbb{Q}$of rational numbers with the following simple example:
	
	which is majorated but has no supremum in $\mathbb{Q}$ because this supremum is in $\mathbb{R}$ as:
	
	Therefore:
	
	This is what makes $\mathbb{Q}$ is not "complete".
	\end{tcolorbox}
	
	\textbf{Definition (\#\mydef)}: We say that $X\subseteq\mathbb{R} $ is "\NewTerm{bounded}\index{bounded}" if $X$ is minorated and majorated.
	
	From the definition it follows immediately that $X$ is bounded if and only if there exists $a,b$ with such that $X\subseteq [a,b]$.
	
	Now that the concept of bound is relatively well defined, let see how a sequence can behave near from it.
	
	\textbf{Definition (\#\mydef):} We say that $(a_n)_{\mathbb{N}}$ of $\mathbb{R}$ is an "\NewTerm{increasing sequence}\index{increasing sequence}" ("decreasing" respectively) if:
	
	respectively:
	
	We say the sequence $(a_n)_{\mathbb{N}}$ is "\NewTerm{monotone}\index{monotone}" if it is increasing or decreasing as we have already seen in the section Sequences and Series.
	
	\textbf{Definition (\#\mydef):} Given $T=\left\lbrace n_0,n_1,n_2,...\right\rbrace$ infinite subset of $\mathbb{N}$ with $n_0<n_1<n_2<...$. We say the sequence $(a_{n_i})_{i\in \mathbb{N}}$  is a "\NewTerm{subsequence}\index{subsequence}" of the sequence $(a_n)_{\mathbb{N}}$.
	
	\begin{theorem}
	Let us now prove that every sequence in $\mathbb{R}$ admits a monotone subsequence (it's a bit the idea of a fractal!)
	\end{theorem}
	\begin{dem}
	We say that $a_m$ is a "\NewTerm{peak}\index{peak}" of the sequence if:
	
	Consider the set $P$ of peaks of the sequence $(a_n)_{\mathbb{N}}$.
	\begin{itemize}
		\item If $P$ is infinite then the subsequence $(a_n)_{n\in P}$ is monotone since decreasing.
		
		\item If $P$ is finite or empty:
		
		(if $P=\varnothing$ we choose any $m_1$). $a_{m_1}$ is therefore not by construction not a peak, so there exists $m_2\geq m_1$ such as $a_{m_1}\leq a_{m_2}$. In turn $a_{m_2}$ is not a peak, so there exist $m_3\geq m_2$ such as $a_{m_2}\leq a_{m_3}$ etc. We see that we define thus as an increasing subsequence.
		
		\textbf{Definition (\#\mydef):} We say that the sequence $(a_n)_\mathbb{N}$ "\NewTerm{converge}\index{convergent sequence}" into $a\in \mathbb{R}$ and we note this:
		
		if:
		
		In this case we say that $a$ is the "\NewTerm{limit of the sequence}\index{limit of the sequence}" $(a_n)_\mathbb{N}$.
	\end{itemize}
	\begin{flushright}
		$\blacksquare$  Q.E.D.
	\end{flushright}
	\end{dem}
	
	\pagebreak
	\begin{tcolorbox}[colframe=black,colback=white,sharp corners]
	\textbf{{\Large \ding{45}}Example:}\\\\
	In the example in figure below where the sequence seems to converge towards the value $1.13$ we observe that for a particular non-zero positive $\varepsilon$, there exists a particular $n$ which we will denote $N$ (which value is $17$ in the below example) from which the sequence converges.
	\begin{figure}[H]
		\centering
		\includegraphics{img/computing/convergence_sequence.jpg}
		\caption{Illustration of the principle of convergence of a sequence}
	\end{figure}
	\end{tcolorbox}
		If there is no $a$ (respectively $N$) for which the previous relation is true, then we say that the sequence "\NewTerm{diverge}\index{divergent sequence}".
	\begin{theorem}
	Let us prove now that every increasing sequence $(a_n)_\mathbb{N}$ (respectively decreasing) and majorated (resp. minorated) converges.
	
	In other words, we seek to prove that any monotonous and bounded sequence $(a_n)_\mathbb{N}$ converges (obviously ... by construction).
	\begin{tcolorbox}[title=Remark,colframe=black,arc=10pt]
	If it does not converge, we could not easily find out what is its lower bound and upper bound ... hence the fact that the need of this theorem becomes trivial.
	\end{tcolorbox}
	\end{theorem}
	\begin{dem}
	This theorem is actually quite intuitive. Consider for this an increasing sequence. We suspect that:
	
	is the limit of this sequence. Note first of all that $a=\sup\left\lbrace a_0,a_1,a_2,...\right\rbrace$ exists because $(a_n)_\mathbb{N}$ is majorated (see theorem proved previously).
	
	Given $\varepsilon>0$. It exists an $a_N$ such tat $a-\varepsilon\leq a_N \leq a$. But in this case as the sequence is increasing, we have $\forall\geq N,a-\varepsilon\leq a_n\leq a$. That is to say $|a_n-a|\leq \varepsilon$. In the case where the sequence is decreasing by proceeding in the same way we prove that $a=\inf{a_0,a_1,a_2,...}$ is the limit of this sequence.
	\begin{flushright}
		$\blacksquare$  Q.E.D.
	\end{flushright}
	\end{dem}
	\begin{theorem}
	And now the important theorem to remember after all this: Every bounded sequence of real numbers has a convergent subsequence  (that is intuitive ... but again ... when formalized it becomes sometimes less intuitive...).
	
	This is what mathematicians name the "\NewTerm{Bolzano-Weierstrass theorem}\index{Bolzano-Weierstrass theorem}" and it is extremely important in many areas of mathematics:
	\end{theorem}
	\begin{dem}
	Given $(a_n)_\mathbb{N}$ such a sequence. By a previous proposal we know there is a monotonic subsequence which we denote $(b_n)_\mathbb{N}$. $(b_n)_\mathbb{N}$ is therefore a monotone and bounded sequence and by the previous theorem, $(b_n)_\mathbb{N}$ converges.
	
	So if we are unable to determine whether the subsequence converges nor his exact limit (which in practice is often very difficult), we only need to know that the subsequence is monotone and bounded to ensure that it converges (which is most of time much easier).
	\begin{flushright}
		$\blacksquare$  Q.E.D.
	\end{flushright}
	\end{dem}
	\begin{theorem}
	Remember that we saw in the section of Sequences and Series that a Cauchy sequence is a sequence $(a_n)_\mathbb{N}$ that verifies (we restrict ourselves to the special case of Euclidean distance):
	
	The difference between the two terms of a Cauchy sequence can be made arbitrarily small provided that the indices of these terms are big enough.
	
	We have also proved (again in the section of Sequences and Series) that in the case of a distance in the general topological sense any convergent sequence is a Cauchy sequence (by cons the reciprocal is not always true at the condition that do not complete the set... otherwise the reciprocal is always true). For example, a sequence of rational numbers that converges to a real number is not a Cauchy sequence, except if we complete the set of rationals to get the set of real numbers.
	\end{theorem}
	Let us redo the proof restricted to Euclidean distance (the method is exactly the same as the reader will notice):
	\begin{dem}
	Given $\varepsilon$, we must show that there is:
	
	But $(a_n)_\mathbb{N}$ tends to $a$ therefore it exists a $N\in \mathbb{N}$ such that $n\geq N \Rightarrow |a_n-a|\leq \varepsilon/2$. For $n,m\geq N$ we therefore have:
	
	\begin{flushright}
		$\blacksquare$  Q.E.D.
	\end{flushright}
	\end{dem}
	\begin{theorem}
	Let us now prove that every Cauchy sequence is bounded (we never talked about this until now anywhere in this book therefore we need to do the proof). Since currently we have just proved that every convergent sequence is a Cauchy sequence...
	\end{theorem}
	\begin{dem}
	If $(a_n)_\mathbb{N}$ is a Cauchy sequence then particularly for $\varepsilon=1$ (randomly) we know that there exist $N\in \mathbb{N}$ such that $n,m\geq N\Rightarrow |a_n-a_m|\leq 1$. So if we fix $m$, we get:
	
	\begin{flushright}
		$\blacksquare$  Q.E.D.
	\end{flushright}
	\end{dem}
	Let us see now the fundamental theorem (it is at this level that there is a huge impact on the understanding of what is actually a fractal!) that can be deduced from the previous  lines.
	\begin{theorem}
	We will show that every Cauchy sequence of real numbers is convergent (by construction ...). We say then that the metric space $\mathbb{R}$ provided with the Euclidean distance (absolute value) is a "\NewTerm{complete space}\index{complete space}\label{complete space cauchy sequence}".
	\end{theorem}
	\begin{tcolorbox}[title=Remarks,colframe=black,arc=10pt]
	\textbf{R1.} The completeness property is related to the metric (hence this theorem could equally have its place in the section of Topology!): The same space can be complete for a given distance and incomplete for another one. It is therefore important to always specify the distance that we take when we speak of complete space.\\
	
	\textbf{R2.} Intuitively, a space is complete if it has no holes. The set of rational numbers $\mathbb{Q}$ is by example complete if the real numbers are added to it.
	\end{tcolorbox}
	\begin{theorem}
	Consider first $(a_n)_\mathbb{N}$ a Cauchy sequence. We have seen just before that  $(a_n)_\mathbb{N}$ is bounded and then that by the Bolzano-Weierstrass theorem, there exists a convergent subsequence  $(a_{n_i})_{i\in\mathbb{N}}$. Let us denote by $a$ the limit of the subsequence $(a_{n_i})_{i\in\mathbb{N}}$. We will now prove that the sequence $(a_n)_\mathbb{N}$ is convergent of limit $a$.
	\end{theorem}
	
	\begin{dem}
	Given $\varepsilon>0$, there exist a $N\in \mathbb{N}$ such as (application of the definition of convergence for a subsequence):
	
	For this same $\varepsilon$ it exists $M\geq 0$ (application of the definition of convergence for a Cauchy sequence):
	
	Given $C=N+M$. We choose $i\geq C$. We then have $|a_{n_i}-a|\leq \varepsilon/2$ and for $n,i>C$:
	
	So by the triangle inequality (\SeeChapter{see section Vector Calculus page \pageref{triangle inequality}}), for any $n\geq C$:
	
	This means precisely that $(a_n)_\mathbb{N}$ converges to $a$.
	\begin{flushright}
		$\blacksquare$  Q.E.D.
	\end{flushright}
	\end{dem}
	Basically it is an intuitive result but at the time when real numbers were not known or not rigorously defined it was a different story! In fact, it is simply enough complete any set by the real numbers to get a complete space. Moreover, some mathematicians define the set of real number saying that it is the set for which every Cauchy sequence converges...
	
	\textbf{Definition (naive \#\mydef):} An "\NewTerm{accumulation point}\index{accumulation point}" or "\NewTerm{cluster point}\index{cluster point}" is a point which we can approach as much as we want thanks to elements of a given set $X$ (we will approach it for example with a sequence). However, this accumulation point can be both inside and outside of $X$ (all items within $X$ are obviously limit points). A good image is to see a series that approach this accumulation point and define circles around it that are becoming smaller and smaller containing elements of the sequence.
	
	We can imagine as an example a sequence defined by the set $X$ of rational numbers $\mathbb{Q}$ which tends to an irrational or to a transcendental number (these two points being elements not belonging to the set of rational $\mathbb{Q}$). So in this case, the accumulation point is outside $X$ (the set of rational $\mathbb{Q}$). By cons, any accumulation point that would be a rational number for a sequence of rational number will necessarily ... in $X$ (that is to say $\mathbb{Q}$).
	
	\begin{tcolorbox}[colframe=black,colback=white,sharp corners]
	\textbf{{\Large \ding{45}}Example:}\\\\
	With respect to the usual Euclidean topology, the sequence of rational numbers:
	
	has no limit (i.e. does not converge) when $n\rightarrow \pm \infty$, but has two accumulation points (which are considered limit points here), that are $-1$ and $+1$:
	\begin{figure}[H]
		\centering
		\includegraphics[scale=0.43]{img/computing/accumulation_point.jpg}
	\end{figure}
	\end{tcolorbox}
	Therefore, comes the following definition (for more details see the section Topology):
	
	\textbf{Definition (formal \#\mydef):} Given $X \subseteq \mathbb{R}^n$. We say that $x\in \mathbb{R}^n$ is an "\NewTerm{accumulation point}\index{accumulation point}" to $X$ if for all ball $\mathcal{B}(x,r)$ of radius $r$ center on $x$ we have (for more details see the section Topology):
	
	The set of all accumulation points (of a sequence) to $X$ is the "\NewTerm{limit set}\index{limit set}" of $X$ and denoted by $\bar{X}$. We have obviously (it suffice to conceptualize it in an abstract way for all possible ball) $X \subseteq \bar{X}$.
	
	\begin{tcolorbox}[colframe=black,colback=white,sharp corners]
	\textbf{{\Large \ding{45}}Example:}\\\\
	Let us consider the interval $]0,1]$ with the ball $B(0,1)$. The intersection between the ball and the interval is not zero, we can say that $0$ is an accumulation point! But now let us take a sequence $1 / n$ for example in the interval $]0,1]$. This sequence tends to zero but $0$ is not in interval. It is a good example of $X \subseteq \bar{X}$.
	\end{tcolorbox}
	We can therefore make the proposition:
	\begin{theorem}
	Let us prove now that $x\in \mathbb{R}^n$ is adherent to $X$ if and only if there is a sequence $\left(u_n\right)_{\mathbb{N}}$ in $X$ converging to $x$ (note that the previous example shows that $x$ is not necessarily in $X$).
	
	In fact, we will instead prove (if we can say it is a proof..) that if we choose an accumulation point $x$ then we can always find a sequence in$ $X converging to $x$.
	\end{theorem}
	\begin{dem}
	If $x\in \mathbb{R}^n$ is adherent to $X$ then let us consider the following concentric balls $B\left(x,\dfrac{1}{n}\right)$ with $n\geq 1$ such as:
	
	and then there are always elements $u_n$ that satisfy:
	
	with whom we can create a sequence by the infinity of existing sequences.
	\begin{flushright}
		$\blacksquare$  Q.E.D.
	\end{flushright}
	\end{dem}	
	\textbf{Definition (\#\mydef):} We say that $X\subseteq \mathbb{R}^n$ is a "closed space" if $X=\bar{X}$.
	
	From the previous proposals it follows that in any closed space $F$, a sequence $(x_n)_{\mathbb{N}}$ that converges has its limit in $F$.
	
	We consider as if trivial that if $(F_i)_I$ is a family of closed indexed spaces on any set $I$, then $\bigcap_I F_i$ is closed.
	
	\textbf{Definition (\#\mydef):} $X\subseteq \mathbb{R}^n$ is a "\NewTerm{compact space}\index{compact space}" if $X$ is closed and bounded.
	
	The following theorem gives a characterization of the compact from the sequences:
	
	\begin{theorem}
	$X\subseteq \mathbb{R}^n$ is compact if and only if any sequence $(x_n)_\mathbb{N}$ of $X$ possess a sub-sequence that converges in $X$.
	\end{theorem}
	\begin{dem}
	First let us prove that $X$ is closed:
	
	If $X$ is compact and $(x_n)_\mathbb{N}$ is a sequence of $X$ then by the Bolzano-Weierstrass theorem, $(x_n)_\mathbb{N}$  has a convergent subsequence of limit $x\in \mathbb{R}^n$. But since $X$ is closed, we have $x\in X$. Conversely, let us assume that any sequence $(x_n)_\mathbb{N}$  of $X$ has a subsequence which converges in $X$. Then $X$ is closed because if $x\in \bar{X}$ there is a sequence $(x_n)_\mathbb{N}$ of $X$ which tends to $x$. By assumption, $(x_n)_\mathbb{N}$ has a subsequence that converges $y\in X$. The sequence $(x_n)_\mathbb{N}$ being convergent all the sub-sequences converge to the same value, therefore $x=y\in X$ (is this not great?!!). Thus $X=\bar{X}$ that is to say $X$ is closed!
	
	Let us now prove that $X$ is bounded:
	
	Let us suppose the opposite. So there exists a sequence $(x_n)_\mathbb{N}$ of $X$ such that $||x_n||\geq n$. But in this case, no subsequence of $(x_n)_\mathbb{N}$ is convergent, which is a contradiction! So $X$ is bounded. Finally, $X$ is compact!!!
	\begin{flushright}
		$\blacksquare$  Q.E.D.
	\end{flushright}
	\end{dem}
	A property of compacts is that if we consider $(A_n)_\mathbb{N}$ a decreasing sequence of non-empty compacts, that is to say $A_{n+1}\subseteq A_n$, then $\bigcap_{\in \mathbb{N}} A_n$ is a non-empty compact. We will pass trough the proof that is relatively trivial by the definition of the concept of Adherence Set that oblige that compacts are by construction non-empty...!
	\begin{tcolorbox}[colframe=black,colback=white,sharp corners]
	\textbf{{\Large \ding{45}}Example:}\\\\
	We obtain the Cantor set $C$ as follows:
	We begin by considering the closed bounded interval $C_0=[0,1]$ of $\mathbb{R}$ which is therefore a compact space (bounded and closed set ). We split $C_0$ into three equal parts and we remove the middle interval. This gives us all the set:
	
	which can be also considered also as the application of a contracting scaling factor of $1/3$ on the closed bounded interval of departure of which we translate the center of homothety.\\
	
	We start again with the two intervals $[0,1/3],[2/3,1]$ for:
	
	disjoint union of $4$ intervals. And so on... So we get a decreasing sequence $C_n$ of compact. We define:
	
	Thanks to the previous proposal, we know that $C$ is not empty and is compact which shows that the compacts are not all "trivial" as intervals. The Cantor set (because he had played by doing the drawing below starting from the bottom) is an example of  (compact) fractal:
	\end{tcolorbox}
	
	\pagebreak
	\begin{tcolorbox}[colframe=black,colback=white,sharp corners]
	\begin{figure}[H]
		\centering
		\includegraphics{img/computing/cantor_set_maple.jpg}
		\caption{Cantor set with Maple 4.00b}
	\end{figure}
	that is possible to get with the following Maple 4.00b code:\\
	
	\texttt{
	>with(plots):\\
	line := proc(a:: list, b:: list)\\
	local plotoptionen, n;\\
	if nargs > 2 then\\
	plotoptionen := seq(args[n], n=3 .. nargs)\\
	else \\
	plotoptionen := NULL\\
	fi;\\
	plot([a, b], style=line, plotoptionen);\\
	end:\\\\
	cree\_segment := (a,b,h) -> line([a,h],[b,h],color=black): \\
	f1:=x->x/3: f2:=x->(x+2)/3:\\\\
	f := s -> s union map(f1, s) union map(f2, s):\\\\
	sequence\_de\_segments := proc(l,h) \\
	local accu, i;\\
	accu := NULL;\\
	for i to nops(l) by 2 do\\
	accu := accu,cree\_segment(l[i], l[i+1], h) od;\\
	accu\\
	end:
	}
	\end{tcolorbox}
	
	\pagebreak
	\begin{tcolorbox}[colframe=black,colback=white,sharp corners]
	\texttt{
	>Cantor:= proc(n) local s, i;\\
	>option remember;\\
	>s:=sequence\_de\_segments([0,1], 1);\\
	>for i from 1 to n do\\
	>s:=sequence\_de\_segments(sort([op((f@@i)({0,1}))]), (1-i/n)), 
	>s;\\
	>od;\\
	>display({s}union{seq(textplot([[0,(i+1/2)/n, '0'], [1, (i+1/2)/n, '1']] \\
	), i=0 .. n)}, color=blue,axes=NONE,thickness=7)\\
	>end:\\\\
	>Cantor(7);
	}\\
	
	It is very interesting to notice that we converges to the Cantor fractal (in terms of geometry but also in term of values!) whatever the chosen starting compact we start from (the closed bounded interval) and also ... whatever the chosen contractor factor!\\
	
	Benoît Mandelbrot also observed this type of self similar structure in the analysis of electrical signals transmitted at his (junior) time when working for IBM on copper cables (IBM had transmission information loss problems).
	\end{tcolorbox}
	Now let us look for finish how behave compactis vis-à-vis continuous applications (we need this to prove how to determine the distance from a point to a set which we will need absolutely after to determine the properties of the Hausdorff distance).
	
	Let us recall (\SeeChapter{see section Topology page \pageref{continuity and uniform continuity}}) that an application $f:X \mapsto \mathbb{R}^m$ whatever $X \subseteq \mathbb{R}^n$  is continuous on a point $x\in X$ if:
	
	This reflects the fact that for $y$ close enough to $x$, $f (y)$ is arbitrarily close to $f (x)$. We also say that $f$ is continuous on $X$ if it is continuous at each point of $X$.
	\begin{theorem}
	Given $f:X\mapsto \mathbb{R}^m$ a continuous application on $x\in X$ and $(x_n)_\mathbb{N}$ a sequence of $X$ with:
	
	Then the sequence $f(x_n)$ converges and (this proposal is very important!):
	
	In other words, if we use as a set of starting values of a convergent sequence, then the function that take as input the values of this sequence  will converge too!
	\end{theorem}
	\begin{dem}
	Given $\varepsilon >0$. $f$ is continuous on $x$, so there exists $\delta>0$ such that:
	
	The sequence $(x_n)_\mathbb{N}$ tends to $x$ therefore it exists $N\in \mathbb{N}$ such that:
	
	If follows that $n\geq N$, we have:
	
	\begin{flushright}
		$\blacksquare$  Q.E.D.
	\end{flushright}
	\end{dem}
	If we now consider a compact $X\subseteq \mathbb{R}^n$ and $f:X\mapsto \mathbb{R}^n$ a continuous application. The application $f(X)$ is compact. In particular $\sup (f)$ and $\inf (f)$ will be reached by definition and  by construction of a compact (closed and bounded set) which is equal to its adherence.
	\begin{theorem}
		In other words, a continuous real-valued function on a compact always reaches its supremum or infimum.
	\end{theorem}
	\begin{dem}
	We will do the proof in two steps:
	\begin{itemize}
		\item Let us prove that $f(X)$ is closed.

		Indeed, given $f(x_n)$ a sequence that tends to $y\in \overline{f(X)}$ (we take theadherence to hope to prove that it is equal to the set itself) then $X$ being compact, then $(x_n)_\mathbb{N}$ has a convergent subsequence $(x_{n_i})_{i\in \mathbb{N}}$.
		Let us put:
		
		The function $f$ is continuoous, therefore:
		
		But as:
		
		we have:
		
		This prove that:
		
		and therefore that $f(X)$ is closed.
	
	\item Let us now prove that $f (X)$ is bounded.

		For this let us suppose the contrary. There is therefore a sequence $f(x_n)$ such that:
		
		for every natural integer $n$ (precisely because it is assumed unbounded). Given $(x_{n_i})_{i\in \mathbb{N}}$ a convergent subsequence of $(x_n)_\mathbb{N}$ with:
		
		Then:
		
		and it follows:
		
		but this is in contradiction with:
		
		Therefore $f(X)$ is bounded. Then $f(X)$ being closed is bounded and therefore is compact.
	\end{itemize}
	\begin{flushright}
		$\blacksquare$  Q.E.D.
	\end{flushright}
	\end{dem}
		Now let apply this result (because this is what interests us in fractal spaces) to calculate the distance from a point to a set:
	\begin{theorem}
	Given $x\in \mathbb{R}^n$, the application $f:\mathbb{R}^n \mapsto \mathbb{R}$ defined by $f (y) = d (x, y)$ is continuous.
	\end{theorem}
	\begin{dem}
	For $(y,z)\in\mathbb{R}^2$, the triangle inequality gives us:
	
	By changing the roles of $y, z$ we get:
	
	and therefore:
	
	Therefore for a given $\varepsilon >0$, $d(z,y)\leq \varepsilon$ implies:
	
	That is to say:
	
	and $f$ is therefore continuous on $y$.
	\begin{flushright}
		$\blacksquare$  Q.E.D.
	\end{flushright}
	\end{dem}
	\textbf{Definition (\#\mydef):} For $x\in \mathbb{R}^n$ and $A\subseteq \mathbb{R}^n$ we define the distance $x$ to $A$ as being the value:
	
	\begin{theorem}
	If $x\in A$ then $d(x,A)=0$ (should me almost trivial). The reciprocal is not true. Indeed in the case $x=0$ and $A=]0,1]$ we have indeed $d(x,A)=0$ but $x \in A$. So we have the following important proposal:
	
	\end{theorem}
	\begin{dem}
	The fact that $d(x,A)=0$ implies the existence of a sequence $(a_n)_{\mathbb{N}}$ of elements of $A$ such that:
	
	which means
	
	therefore $x\in \bar{A}$ (see previous developments).

	Conversely, if $x\in \bar{A}$ then for every $\varepsilon>0$ there exists $a\in A$ such as $d(x,a)\leq \varepsilon$. But $d(x,A)\leq d(x,a)\leq \varepsilon$. Thus for any $\varepsilon>0$, $d(x,A)\leq \varepsilon$. That is to say:
	
	\begin{flushright}
		$\blacksquare$  Q.E.D.
	\end{flushright}
	\end{dem}
	In general the distance from $x$ to $A$ is not reached. That is to say that there is no $a\in A$ such that $d(x,A)=d(x,a)$. To check this, it is sufficient to consider the example $x=-1$ and $A=]0,1]$. We have in this example $d(x,A)=1$ but for any $a\in A$, $d(x,a)>1$. If $A$ is compact, the situation is obviously different according to the following proposal (the most important for the Hausdorff distance in our point of view):	
	\begin{theorem}
		If $A\subseteq \mathbb{R}^n$ is compact, there exist $a\in A$ such that $d(x,A)=d(x,a)$. Therefore:
		
	\end{theorem}
	\begin{dem}
	The application $f:A\mapsto \mathbb{R}$ defined by $f(a)=d(x,a)$ is continuous as previously proved. Therefore $f(A)$ is compact (see a previous proposal). Thus, $f$ reaches its bounds, that is to say, there is $a\int A$ such that $f (a) =\inf(f (A))$. Therefore:
	
	\begin{flushright}
		$\blacksquare$  Q.E.D.
	\end{flushright}
	\end{dem}
	\begin{tcolorbox}[title=Remark,colframe=black,arc=10pt]
	This previous proposition does not say that $a$ is unique, in general in fact it is not!
	\end{tcolorbox}
	
	\subsubsection{Fractals Metric Space}\label{fractal metric space}
	Fractals are often perceived by people as pretty drawings on paper, but when we look in detail the geometry of fractal, we need a particular space to study them, much like the biologist who puts small worms on a wafer to observe the worms in detail to the microscope. We will do the same for our fractals by placing them in a place they like...
	
	This place is likely to be a subspace of $\mathbb{R}^2$ or  $\mathbb{R}^3$, since in the end it will produce drawings... And to illustrate this we often will place ourselves in  $\mathbb{R}^2$ (with the Euclidean metric), and unless otherwise stated, we always consider the case where $(X,d)$ is a complete metric space.
	
	Let us collect different items in order to construct this space:
	
	\textbf{Definition (\#\mydef):} We define $\mathcal{H}(X)$ as the space whose points are the compacts subsets of $X$ other than $X$ itself. From now, we will name "\NewTerm{fractal}\index{fractal}" any element of $\mathcal{H}(X)$.
	
	\begin{tcolorbox}[colframe=black,colback=white,sharp corners]
	\textbf{{\Large \ding{45}}Example:}\\\\
	It is immediate that if $x,y\in\mathcal{H}$, then $x\cup y\in \mathcal{H}$ but $x\cap y$ is not necessarily in $\mathcal{H}$. It is sufficient to see the figure below with the two compact sets of $\mathbb{R}^2$ (closed and bounded therefore) below. There are therefore two points of $\mathcal{H}$.. Their union is still a compact, and therefore:
	

	By cons, if the sets are disjoint (as here), $x\cap y=\varnothing$ and therefore $x$, $y$ are not a point of $\mathcal{H}(\mathbb{R}^2)$ (see previous theory).
	\begin{figure}[H]
		\centering
		\includegraphics{img/computing/fractal_compact_space.jpg}
		\caption[]{Source: IFS and L-System V. Rezzonico, C. Hebeisen}
	\end{figure}
	\end{tcolorbox}
	Another example involves taking the fractal Cantor ...
	
	\textbf{Definition (\#\mydef):} Given $x\in X$ and $B\in\mathcal{H}(X)$, we define the distance $d(x,B)$ of a point $x$ to a set $B$  by:
	
	\begin{tcolorbox}[title=Remarks,colframe=black,arc=10pt]
	\textbf{R1.} This definition is quite general and applies to any non-empty subset of $X$, by replacing $\min$ by $\inf$. But in our specific case, we are interested in taking precisely $\mathcal{H}(X)$ as a subspace.\\
	
	\textbf{R2.} This distance is well defined (is exists) as $B$ is non-empty and compact.\\
	
	\textbf{R3.} It is trivial to see that if this distance is zero, then $x\in\bar{B}$.
	\end{tcolorbox}
	\begin{tcolorbox}[colframe=black,colback=white,sharp corners]
	\textbf{{\Large \ding{45}}Example:}\\\\
	Illustration in the case where $X=\mathbb{R}^2$:
	\begin{figure}[H]
		\centering
		\includegraphics{img/computing/distance_point_subspace.jpg}
		\caption[]{Source: IFS and L-System V. Rezzonico, C. Hebeisen}
	\end{figure}
	\end{tcolorbox}
	\textbf{Definition (\#\mydef):} Given $A,B\in\mathcal{H}(X)$. We define and denote the distance from $A$ to $B$ by:
	
	\begin{tcolorbox}[title=Remarks,colframe=black,arc=10pt]
	\textbf{R1.} This definition is quite general and applies to any non-empty subset of $X$, by replacing $\min$ by $\inf$. But in our specific case, we are interested in taking precisely $\mathcal{H}(X)$ as a subspace.\\
	
	\textbf{R2.} We notice that this distance does not provide a $\mathcal{H}(X)$ metric: indeed, $d(A,B)\neq d(B,A)$ in general (take for example the Cantor fractal where some compact we have $A\subset B$ with $A\neq B$, then we have $d(A, B) = 0$ but $d(B,A)>0$).
	\end{tcolorbox}
	\textbf{Definition (\#\mydef):} Given $A,B\in\mathcal{H}(X)$. We define and denote "\NewTerm{Hausdorff distance}\index{Hausdorff distance}" between two sets $A,B\in \mathcal{H}(X)$ by:
	
	This time, by this definition, we have well a metric on $\mathcal{H}(X)$.

	Indeed, let us check that the five properties of a distance are satisifed (\SeeChapter{see section Topology page \pageref{distance}}):
	
	Given $A,B,C\in \mathcal{H}(X)$. Clearly we have without proof\footnote{But let us know as always if you want we put the proofs} (symmetry, nullity and separation on the diagonal):
	
	Moreover, since $A$ and $B$ are compact, $h(A,B)=d(A,B)$ (see on of the previous proposals) for a given $a\in A$ and a given $b\in B$. But, since $d(a,b)>0$ by definition, we have (property of positivity) finally $h(A,B)>0$ such that $a\in A$,$A\notin B$:
	
	as $B$ is closed.
	
	Finally, since $h(A,B)=d(a,b)$ (see the extension of one of previous proposals), the triangle inequality is necessarily respected and then:
	
	So $h$ is indeed a metric of $\mathcal{X}$, which makes $(\mathcal{H}(X),h)$ a metric space. This is a first step in the desired direction, we now have the tools to compare two sets belonging to $\mathcal{X}(X)$ by the Hausdorff distance between them. If the two are not "too different", so intuitively that distance should be fairly small.
	
	If we choose a strictly contracting function $f:X\mapsto X$ of constant $\lambda$ (\SeeChapter{see section Topology page \pageref{strictly contracting}}). Then, the application:
	
	defined by:
	
	is by onstruction also strictly contracting of constant $\lambda$.
	
	Given $f_i:\mathbb{R}^2\mapsto \mathbb{R}^n$, $i\leq i\leq k$ strictly contracting applications of contraction constant  $0\leq \lambda_i<1$. Then, there exists a unique compact $A\in\mathcal{H}(\mathbb{R}^n)$ such that:
	
	($A$ is the unique fixe point of $T_{\lambda_1,\ldots,\lambda_k}$) and for any compact $B$, we have:
	
	where $T_{\lambda_1,\ldots,\lambda_k}^m(B)$ is the $m$-th iteration of $B$ by $T_{\lambda_1,\ldots,\lambda_k}$.
	
	This result derives from the fixed-point theorem (\SeeChapter{see section Sequences and Series page \pageref{banach fixed point theorem}}) applied to the space $\mathcal{H}(\mathbb{R}^n)$ that is complete.

	With the same notation, we say that $f_1,\ldots,f_k$ is an IFS (Iterated Function Systems) coding of the compact $A$. Thus the functions $f_1,\ldots,f_k$ define the compact $A$. What is surprising, as we will see in the following examples, is that the $f_1,\ldots,f_k$ are usually quite simple (as homotheties of the plan) while the compact $A$ is in many cases relatively or very "complicated" visually speaking.

	If case $k=1$ is without interest, we would have $A=(x)$ where $x$ is the fixed point of $f_1$. With $k=2$ we already obtain nontrivial results.
	\begin{tcolorbox}[title=Remark,colframe=black,arc=10pt]
	When the iterative contracting functions are all homotheties we speak then of "\NewTerm{Sierpinski's fractal}\index{Sierpinski's fractal}". Thus, the Cantor fractal  belongs to the family of Sierpinski's fractals.
	\end{tcolorbox}
	A frequently used method for to generate IFS fractals (as it will be the case below with Maple 4.00b\footnote{But if some readers have reproduced all the example below in C++ they are welcome to share their code}) is to consider a point in the plane $(x_n,y_n)$ on which we can without conditions or constraints apply an affine transformation to get a new point such that:
	
	where $a$, $b$, $c$, $d$, $e$ and $f$ are any constants, and $(x_0,y_0)$ is given (chosen).
	
	We can therefore consider an application $T$ that describes our transformation, and in matrix form we can write the previous system as follows:
	
	or even:
	
	So in general, the vector $\vec{b}$ above simply describes a translation, and the matrix $A$ is the composition of a rotations and a scaling (\SeeChapter{see section Euclidean Geometry page \pageref{geometric transformations}}). Computer programs (as it will be the case in the examples below), thus often require that the knowledge of the six parameters $a$, $b$, $c$, $d$, $e$ and $f$ that can for majority be equal to zero.
	
	\pagebreak
	\subsection{Fractals Visualization}
	
	\subsubsection{Cantor's Fractal (Cantor Set)}
	Let us come back now on Cantor's Fractal but see now from the point of view of the application of two iterative contracting functions $f_1,f_2$ (thus corresponding to $k = 2$).

	So we start form the following closed bounded set (then a compacit as it is bounded):
	
	Therefore: 
	\begin{figure}[H]
		\centering
		\includegraphics{img/computing/cantor_set_01.jpg}
		\caption[]{Start set of Cantor's Fractal}
	\end{figure}
	So now by definition we split in three equal parts and we remove the middle interval. This gives us the set:
	
	Therefore:
	\begin{figure}[H]
		\centering
		\includegraphics[]{img/computing/cantor_set_03.jpg}
		\caption{First iteration of Cantor's Fractal}
	\end{figure}
	We can see that $[0,1/3]$ can be obtained by the following homothety (scaling) application of factor $1/3$ centered at $(0,0)$:
	
	and that $[2/3,1]$ can be obtained by the following homothety (scaling) of factor $1/3$ centered at $(1,0)$:
	
	and so on, and we get as we already know (see Maple 4.00b code already given above):
	\begin{figure}[H]
		\centering
		\includegraphics{img/computing/cantor_set_02.jpg}
		\caption{Cantor attractor after $6$ iterations}
	\end{figure}
	Which corresponds using the formalism seen earlier to:
	
	This process is continued ad infinitum, where the $n$-th set is
	
	with obviously $C_{0}=[0,1]$.
	
	But let us see that it works with any compact of $\mathbb{R}$ as a square for example with the following Maple 4.00b code  (we always show all the details of Maple 4.00b, because nothing says that readers have the software or that the software will still exist in 50 years for reproductibility purposes...).
	
	\begin{tcolorbox}[title=Remark,colframe=black,arc=10pt]
	It seems that there may be sometimes issues by copying and pasting the below code into Maple 4.00b. If it's the case one your computer, just rewrite it from scratch directly in the Maple console.
	\end{tcolorbox}
	
	\texttt{>transforme\textunderscore point := proc(t, p)}\\
	\texttt{   [t[1]*p[1]+t[2]*p[2]+t[5], t[3]*p[1]+t[4]*p[2]+t[6]]}\\
  	\texttt{end:}\\

  	\texttt{>IFSS := proc(n, liste\textunderscore de\textunderscore transformations,col)}\\
  	\texttt{local i, j, k, s, seq\textunderscore square:}\\
     \texttt{seq\textunderscore square :=[[0,0],[1,0],[1,1],[0,1]];} \\
     \texttt{for j to n do}\\
     \texttt{   s := NULL;}\\   
     \texttt{   for i to nops(liste\textunderscore de\textunderscore transformations) do}\\
        \texttt{       {} {} {} s := s,}\\
        \texttt{      {} {} {} seq(transform\textunderscore square(liste\textunderscore de\textunderscore transformations[i],}\\
        \texttt{      {} {} {} op(k, [seq\textunderscore square])),}\\
        \texttt{      {} {} {} k=1 .. nops([seq\textunderscore square]))}\\
      \texttt{   od;}\\
      \texttt{   seq\textunderscore square := s}\\
    \texttt{od;}\\
    \texttt{plots[polygonplot]([seq\textunderscore square], axes=none, color=col, scaling=constrained)}\\
    \texttt{end:}\\
    
	\texttt{>cantor:=[[evalf(1/3),0,0,evalf(1/3),0,0],[evalf(1/3),0,0,evalf(1/3), evalf(2/3),0]]:}\\
	
	\texttt{>IFSS(1, cantor,blue);}
	\begin{figure}[H]
		\centering
		\includegraphics{img/computing/cantor_set_square_01.jpg}
		\caption[]{First iteration on the Cantor set with squares}
	\end{figure}
	\texttt{>IFSS(2, cantor,blue);}
	\begin{figure}[H]
		\centering
		\includegraphics{img/computing/cantor_set_square_02.jpg}
		\caption[]{Second iteration on the Cantor set with squares}
	\end{figure}
	\texttt{>IFSS(3, cantor,blue);}
	\begin{figure}[H]
		\centering
		\includegraphics{img/computing/cantor_set_square_03.jpg}
		\caption[]{Third iteration on the Cantor set with squares}
	\end{figure}
	etc.
	
	So, whatever the starting set, the sequence of compact obtained by successive application of these two plane homotheties always converge (in the sense of the Hausdorff distance) to the same compact/attractor (assimilated to the fixed of the Fixed point theorem...) named Cantor's fractal or Cantor set (thus belonging to the family of Serpienski fractals). The latest figure above is a good approximation of this set.
	
	The Cantor set being self-similar, consisting
of $N=2$ congruent subsets, each when magnified by a factor of $M = 3$ yields the original set. Hence the fractal dimension (\SeeChapter{see section Euclidean Geometry page \pageref{dimensions}}) of the Cantor set is:
	
	
	\subsubsection{Triangle Sierpinski Fractal}\label{sierpinski fractal}
	To build the Sierpinski fractal (which can be found as curiosity sometimes on the seashell Cymbiola innexa REEVE), based on three iterative contracting functions $f_1,f_2,f_3$ (thus corresponding to $k = 3$), we assume for example three following points of $\mathbb{R}^2$:
	
	Which gives with Maple 4.00b:
	
	The Sierkpinski Triangle consists of $3^n$ subsets with magnification factor $2^n$. So the fractal dimension is:
	
	
	\texttt{>plots[polygonplot]( [[0, 0], [1, 0], [0.5, 1]],axes=none,color=black, scaling=constrained);}
	\begin{figure}[H]
		\centering
		\includegraphics{img/computing/sierpinski_set_01.jpg}
		\caption[]{Start set of triangle Sierpinski fractal}
	\end{figure}
	This is a triangle, but we could start from any shape and we always arrive at the same result as we will see later.
	
	We apply on each set a contracting function of factor $0.5$, this gives the triangle:
	
	and we denote that this homothety (scaling) of factor $0.5$ and center $(0,0)$ on the original triangle by:
	
	We do now on this triangle a translation of $0.5$ in the direction of the $x$-axis, which gives the triangle:
	
	which corresponds to a scaling factor of $0.5$ and center $(1.0)$ on the original triangle:
	
	We now translate $[[0,0], [0.5,0], [0.25,0.5]]$ of $0.25$ along the $x$-axis and of $0.5$ according to the $y$-axis to have:
	
	which corresponds to a scaling factor of $0.5$ of center $(0.5,0.75)$ on the original triangle:
	
	With Maple 4.00b it now gives for the three triangles:
	
	\texttt{>plots[polygonplot]([[[0,0],[0.5,0],[0.25,0.5]],[[0.5,0],[1,0],[.75,0.5]], [[0.25,0.5],[0.75,0.5],[0.5,1]]], axes=none,color=black, scaling=constrained);}
	\begin{figure}[H]
		\centering
		\includegraphics{img/computing/sierpinski_set_02.jpg}
		\caption[]{First iteration on the Sierpinksi triangle}
	\end{figure}
	and so on...:
	\begin{figure}[H]
		\centering
		\includegraphics{img/computing/sierpinski_set_03.jpg}
		\caption[]{Second iteration on the Sierpinksi triangle}
	\end{figure}
	and so on...:
	\begin{figure}[H]
		\centering
		\includegraphics{img/computing/sierpinski_set_04.jpg}
		\caption[]{Third iteration on the Sierpinksi triangle}
	\end{figure}
	and so on...:
	\begin{figure}[H]
		\centering
		\includegraphics{img/computing/sierpinski_set_05.jpg}
		\caption[]{Fourth iteration on the Sierpinksi triangle}
	\end{figure}
	and so on (the triangles begins to be quite small to see a difference without zoom)...:
	\begin{figure}[H]
		\centering
		\includegraphics{img/computing/sierpinski_set_06.jpg}
		\caption[]{Sixth iteration on the Sierpinksi triangle}
	\end{figure}
	Which corresponds by taking the formalism seen previously:
	
	We can make the same remark as when we have study the Cantor's Fractal: whatever the starting set, the sequence of compact obtained by successive application of these three plane homotheties always converge (in the sense of the Hausdorff distance) to the same compact/attractor (assimilated to the fixed of the Fixed point theorem...) named Sierpinski fractal. The latest figure above is a good approximation of this set.
	
	Let's see this with a Maple 4.00b code (once again if the copy/paste from the book in Maple 4.00b does not work, simply rewrite the code in the Maple console):
	
	\texttt{>transforme\textunderscore triangle := proc(t, triangle)}\\	
    \texttt{   local i;}\\
    \texttt{   [seq(transforme\textunderscore point(t, triangle[i]), i=1 .. 3)]}    
	\texttt{end:}\\

	\texttt{>IFS := proc(n, liste\textunderscore de\textunderscore transformations,col)}\\
     \texttt{local i, j, k, s, sequence\textunderscore de\textunderscore triangles:}\\
     \texttt{options `Copyright by Alain Schauber, 1996`;}\\
     \texttt{sequence\textunderscore de\textunderscore triangles := [[0, 0], [1, 0], [0.5, 1]];}\\
     \texttt{for j to n do}\\
     \texttt{   s := NULL;}\\   
     \texttt{   for i to nops(liste\textunderscore de\textunderscore transformations) do}\\
        \texttt{   {} {} {} s:= s,}\\
        \texttt{   {} {} {} seq(transforme\textunderscore triangle(liste\textunderscore de\textunderscore transformations[i],}\\
        \texttt{   {} {} {} op(k, [sequence\textunderscore de\textunderscore triangles])),}\\
        \texttt{   {} {} {} k=1 .. nops([sequence\textunderscore de\textunderscore triangles]))}\\
       \texttt{   {} {} {} od;}\\       
      \texttt{   {} {} {} sequence\textunderscore de\textunderscore triangles := s}\\
    \texttt{od;}\\
    \texttt{plots[polygonplot]([sequence\textunderscore de\textunderscore triangles], axes=none, color=col, scaling=constrained)}\\
  	\texttt{end:}
  	
	\texttt{>triangle\textunderscore de\textunderscore  Sierpinski:=[[0.5,0,0,0.5,0,0],[0.5,0,0,0.5,0.5,0], [0.5,0,0,0.5,0.25,0.5]]:}

	\texttt{>IFS(6, triangle\textunderscore de\textunderscore Sierpinski,blue);}
	\begin{figure}[H]
		\centering
		\includegraphics{img/computing/sierpinski_maple.jpg}
		\caption{Sierpinski triangle attractor}
	\end{figure}
	And this time we don't start from a triangle but from a square (IFS Square) with the following Maple 4.00b code:
	
	\texttt{>transforme\textunderscore square := proc(t, square)}\\	
    \texttt{   local i;}\\
    \texttt{   [seq(transforme\textunderscore point(t, square[i]), i=1 .. 4)]}    
	\texttt{end:}\\

	\texttt{>IFS := proc(n, liste\textunderscore de\textunderscore transformations,col)}\\
     \texttt{local i, j, k, s, seq\textunderscore  square:}\\
     \texttt{seq\textunderscore square := [[0, 0], [1, 0], [1, 1],[0, 1]];}\\
     \texttt{for j to n do}\\
     \texttt{   s := NULL;}\\   
     \texttt{   for i to nops(liste\textunderscore de\textunderscore transformations) do}\\
        \texttt{   {} {} {} s:= s,}\\
        \texttt{   {} {} {} seq(transform\textunderscore square(liste\textunderscore de\textunderscore transformations[i],}\\
        \texttt{   {} {} {} op(k, [seq\textunderscore square])),}\\
        \texttt{   {} {} {} k=1 .. nops([seq\textunderscore square]))}\\
       \texttt{   {} {} {} od;}\\       
      \texttt{   {} {} {} seq\textunderscore square := s}\\
    \texttt{od;}\\
    \texttt{plots[polygonplot]([seq\textunderscore square], axes=none, color=col, scaling=constrained)}\\
  	\texttt{end:}
  	  	
	\texttt{>square\textunderscore  Sierpinski:=[[0.5,0,0,0.5,0,0],[0.5,0,0,0.5,0.5,0], [0.5,0,0,0.5,0.25,0.5]]:}

	\texttt{>IFS(1, square\textunderscore  Sierpinski,green);}
	\begin{figure}[H]
		\centering
		\includegraphics{img/computing/sierpinski_square_set_01.jpg}
		\caption[]{First iteration on the Sierpinksi square}
	\end{figure}
	\texttt{>IFS(2, square\textunderscore  Sierpinski,green);}
	\begin{figure}[H]
		\centering
		\includegraphics{img/computing/sierpinski_square_set_02.jpg}
		\caption[]{Second iteration on the Sierpinksi square}
	\end{figure}
	\texttt{>IFS(3, square\textunderscore  Sierpinski,green);}
	\begin{figure}[H]
		\centering
		\includegraphics{img/computing/sierpinski_square_set_03.jpg}
		\caption[]{Third iteration on the Sierpinksi square}
	\end{figure}
	and so on until...:
	
	\texttt{>IFS(6, square\textunderscore  Sierpinski,green);}
	\begin{figure}[H]
		\centering
		\includegraphics{img/computing/sierpinski_square_set_04.jpg}
		\caption[]{Sixth iteration on the Sierpinksi square}
	\end{figure}
	Basically, the Sierpinski fractal can obviously be also seen as a triangle to which the middle of the triangle is removed and where for each of the remaining triangles, we restart the process!
	
	\subsubsection{Sierpinski carpet fractal}
	The Sierpinski carpet is the attractor of eight contracting iterative functions of homothety of ratio $1/3$ centered at the vertices and sides of a square in which can be any put any plane geometric shape.

	This time in $\mathbb{R}^2$ we consider the eight homotheties ($h$):
	
	and we start for example from the four following points:
	
	that corresponds to a filled square (but we can choose anything else!):
	\begin{figure}[H]
		\centering
		\includegraphics{img/computing/sierpinski_carpet_set_01.jpg}
		\caption[]{Start set of Sierpinski carpet fractal}
	\end{figure}
	After application of the eight homotheties functions (we leave to the reader to do the calculations manually in the same we have already do it for the triangle), we get the following form of eight squares:
	\begin{figure}[H]
		\centering
		\includegraphics{img/computing/sierpinski_carpet_set_02.jpg}
		\caption[]{Second iteration of Sierpinski carpet fractal}
	\end{figure}
	and applying again the eight homotheties (fortunately there are computers...):
	\begin{figure}[H]
		\centering
		\includegraphics{img/computing/sierpinski_carpet_set_03.jpg}
		\caption[]{Third iteration of Sierpinski carpet fractal}
	\end{figure}
	and again:
	\begin{figure}[H]
		\centering
		\includegraphics{img/computing/sierpinski_carpet_set_04.jpg}
		\caption[]{Fourth iteration of Sierpinski carpet fractal}
	\end{figure}
	etc.
	
	The resulting fixed point (attractor) is obviously named the "\NewTerm{Sierpinski carpet}\index{Sierpinski carpet}" and this is the shape of the receiving antenna of the majority of our cell phones in the early 21st century.

	The above figures can be obtained successively with the following Maple 4.00b code (if the copy/paste form the book in Maple 4.00b does not work, simply rewrite the code in the Maple console):
	
	\texttt{>transforme\textunderscore point := proc(t, p)}\\
      	\texttt{[t[1]*p[1]+t[2]*p[2]+t[5], t[3]*p[1]+t[4]*p[2]+t[6]]}\\
		\texttt{end:}

	\texttt{>transform\textunderscore square := proc(t, square) }\\
      	\texttt{local i;}\\
     	\texttt{[seq(transforme\textunderscore point(t, square[i]), i=1 .. 4)]}\\
		\texttt{end:}

		\texttt{>IFSS := proc(n, liste\textunderscore de \textunderscore transformations,col)}\\
      	\texttt{local i, j, k, s, seq\textunderscore square:}\\
      	\texttt{seq\textunderscore square :=[[0,0],[1,0],[1,1],[0,1]];}\\
      	\texttt{for j to n do}\\
      	\texttt{s := NULL;}\\
      	\texttt{for i to nops(liste\textunderscore de\textunderscore transformations) do}\\
         	\texttt{s := s,}\\
         	\texttt{seq(transform\textunderscore square(liste\textunderscore de \textunderscore transformations[i],}\\
         	\texttt{op(k, [seq\textunderscore square])),}\\
         	\texttt{k=1 .. nops([seq\textunderscore square]))}\\
       	\texttt{od;}\\
       	\texttt{seq\textunderscore square := s }\\
     	\texttt{od;}\\
     	\texttt{plots[polygonplot]([seq\textunderscore square], axes=none, color=col, scaling=constrained)}\\
   	\texttt{end:}

		\texttt{> dywan:= [[evalf(1/3),0,0,evalf(1/3),0,0],[evalf(1/3),0,0,evalf(1/3), evalf(1/3),0],[evalf(1/3),0,0,evalf(1/3),evalf(2/3),0],  [evalf(1/3),0,0,evalf(1/3),0,evalf(2/3)], [evalf(1/3),0,0,evalf(1/3),evalf(1/3), evalf(2/3)],[evalf(1/3),0,0,evalf(1/3), evalf(2/3),evalf(2/3)],	[evalf(1/3),0,0,evalf(1/3),0,evalf(1/3)], [evalf(1/3),0,0,evalf(1/3),evalf(2/3),evalf(1/3)]]:}\\

	\texttt{>IFSS(0, dywan, blue);}
	\begin{figure}[H]
		\centering
		\includegraphics{img/computing/sierpinski_carpet_set_maple_01.jpg}
		\caption[]{Start set of Sierpinski carpet fractal}
	\end{figure}
	\texttt{>IFSS(1, dywan, blue);}
	\begin{figure}[H]
		\centering
		\includegraphics{img/computing/sierpinski_carpet_set_maple_02.jpg}
		\caption[]{First iteration of Sierpinski carpet fractal}
	\end{figure}
	\texttt{>IFSS(2, dywan, blue);}
	\begin{figure}[H]
		\centering
		\includegraphics{img/computing/sierpinski_carpet_set_maple_03.jpg}
		\caption[]{Second iteration of Sierpinski carpet fractal}
	\end{figure}
	\texttt{>IFSS(3, dywan, blue);}
	\begin{figure}[H]
		\centering
		\includegraphics{img/computing/sierpinski_carpet_set_maple_04.jpg}
		\caption[]{Third iteration of Sierpinski carpet fractal}
	\end{figure}
	\texttt{>IFSS(4, dywan, blue);}
	\begin{figure}[H]
		\centering
		\includegraphics{img/computing/sierpinski_carpet_set_maple_05.jpg}
		\caption[]{Fourth iteration of Sierpinski carpet fractal}
	\end{figure}
	
	\pagebreak
	\subsubsection{Fractal spirals}
	We just saw two fractals of the Sierpinski's fractal family therefore based solely on contracting homotheties. Let us now see a fractal that combines rotation and scaling Contracting.
	
	In $\mathbb{R}^2$ we consider the two applications of homotheties ($h$) and rotations ($R$) as follows:
	
	With a triangle and always with Maple 4.00b, this gives us (if the copy/paste form the book in Maple 4.00b does not work, simply rewrite the code in the Maple console):
	
	\texttt{>transforme\textunderscore triangle := proc(t, triangle)}\\
     \texttt{local i;}\\
     \texttt{[seq(transforme\textunderscore point(t, triangle[i]), i=1 .. 3)]}\\
	\texttt{end:}\\

	\texttt{>IFS := proc(n, liste\textunderscore de\textunderscore transformations,col)}\\
     \texttt{local i, j, k, s, sequence\textunderscore de\textunderscore triangles:}\\
     \texttt{options `Copyright by Alain Schauber, 1996`;}\\
     \texttt{sequence\textunderscore de\textunderscore triangles := [[0, 0], [1, 0], [0.5, 1]];}\\
     \texttt{for j to n do}\\
     \texttt{ s := NULL;}\\
     \texttt{for i to nops(liste\textunderscore de\textunderscore transformations) do}\\
        \texttt{s := s,}\\
        \texttt{seq(transforme\textunderscore triangle(liste\textunderscore de\textunderscore transformations[i],}\\
        \texttt{op(k, [sequence\textunderscore de\textunderscore triangles])),}\\
        \texttt{k=1 .. nops([sequence\textunderscore de\textunderscore triangles]))}\\
       \texttt{od;}\\
      \texttt{sequence\textunderscore de\textunderscore triangles := s}\\
     \texttt{od;}\\
    \texttt{plots[polygonplot]([sequence\textunderscore de\textunderscore triangles], axes=none, color=col, scaling=constrained)}\\
  \texttt{end:}

  \texttt{>a:=evalf(5*Pi/6);b:=evalf(Pi/6);}\\
  \texttt{>c1x:=0.25;c1y:=0.5;c2x:=0.5;c2y:=0.5;}\\
  \texttt{>h1:=0.2;h2:=0.95;}\\
  \texttt{>spirale:=[[h1*cos(a),-h1*sin(a),h1*sin(a),h1*cos(a),(1-h1*cos(a))*c1x}\\
  \texttt{+h1*sin(a)*c1y,-h1*sin(a)*c1x+(1-h1*cos(a))*c1y],[h2*cos(b),-h2*sin(b),h2*sin(b),}\\
  \texttt{h2*cos(b),(1-h2*cos(b))*c2x+h2*sin(b)*c2y,-h2*sin(b)*c2x+(1-h2*cos(b))*c2y]]:}\\

 	\texttt{>IFS(1,spirale,blue);}\\
 	\begin{figure}[H]
		\centering
		\includegraphics{img/computing/spiral_fractal_set_01.jpg}
		\caption[]{First iteration of spiral fractal}
	\end{figure}
	and as the convergence is very slow, we will give the results by step of  $5$ iterations.	
	\texttt{>IFS(6,spirale,blue);}\\
 	\begin{figure}[H]
		\centering
		\includegraphics{img/computing/spiral_fractal_set_02.jpg}
		\caption[]{Sixth iteration of spiral fractal}
	\end{figure}
	\texttt{>IFS(11,spirale,blue);}\\
 	\begin{figure}[H]
		\centering
		\includegraphics{img/computing/spiral_fractal_set_03.jpg}
		\caption[]{Eleventh iteration of spiral fractal}
	\end{figure}
	\texttt{>IFS(16,spirale,blue);}\\
 	\begin{figure}[H]
		\centering
		\includegraphics{img/computing/spiral_fractal_set_04.jpg}
		\caption[]{Sixteenth iteration of spiral fractal}
	\end{figure}
	
	
	\subsubsection{Von Koch fractal (Koch snowflake)} 
	Still, in fractal obtained by contracting homotethies ($h$) and rotation ($R$) but to which we add now a translation ($T$), the Von Koch curve is a well known fractal, it can be obtained by the following applications (we already met this fractal in the section of Euclidean Geometry when we have introduced the concept of fractal dimension):	
	
	Here is the corresponding Maple 4.00b code (if the copy/paste form the book in Maple 4.00b does not work, simply rewrite the code in the Maple console):
	
	\texttt{>koch := proc(p:: numeric)}\\
	\texttt{local m, n, k, l, s, h, x, y, pts, t, i;}\\
     \texttt{h := 3\string^(-p);}\\
     \texttt{pts := table([]): \# [0, 0];}\\
     \texttt{pts[0]:=[0,0];}\\
     \texttt{x := 0; y := 0;}\\
     \texttt{for n from 0 to (4\string^p) do}\\
        \texttt{m := n;}\\
        \texttt{s := 0;}\\
       \texttt{ for l from 0 to p-1 do}\\
           \texttt{t := irem(m, 4);}\\
           \texttt{m := iquo(m, 4);}\\
           \texttt{s := s+irem((t+1), 3) - 1}\\
        \texttt{od;  \# end of for l}\\
        \texttt{x := evalhf(x+cos(Pi*s/3)*h);}\\
        \texttt{y := evalhf(y+sin(Pi*s/3)*h);}\\
        \texttt{pts[n+1] := [x, y];}\\
     \texttt{od;}\\
    \texttt{[seq(pts[i], i=0 .. n-1)];}\\
  \texttt{end:}\\

	\texttt{>plot(koch(0), scaling=constrained, style=LINE, axes=NONE, color=blue,thickness=2);}
	\begin{figure}[H]
		\centering
		\includegraphics{img/computing/von_koch_set_01.jpg}
		\caption[]{Start set of Von Koch fractal}
	\end{figure}
	\texttt{>plot(koch(1), scaling=constrained, style=LINE, axes=NONE, color=blue,thickness=2);}
	\begin{figure}[H]
		\centering
		\includegraphics{img/computing/von_koch_set_02.jpg}
		\caption[]{First iteration of Von Koch fractal}
	\end{figure}
	\texttt{>plot(koch(2), scaling=constrained, style=LINE, axes=NONE, color=blue,thickness=2);}
	\begin{figure}[H]
		\centering
		\includegraphics{img/computing/von_koch_set_03.jpg}
		\caption[]{Second iteration of Von Koch fractal}
	\end{figure}
	\texttt{>plot(koch(3), scaling=constrained, style=LINE, axes=NONE, color=blue,thickness=2);}
	\begin{figure}[H]
		\centering
		\includegraphics{img/computing/von_koch_set_04.jpg}
		\caption[]{Third iteration of Von Koch fractal}
	\end{figure}
	etc. etc. Until the following attractor:
	\begin{figure}[H]
		\centering
		\includegraphics{img/computing/von_koch_set_05.jpg}
		\caption[]{Fourth iteration of Von Koch fractal}
	\end{figure}
	
	The Koch snowflake (see figure below) is a variant of the Von Koch line and thab can be constructed by starting with an equilateral triangle, then recursively altering each line segment as follows:
	\begin{enumerate}
		\item Divide the line segment into three segments of equal length.
		\item Draw an equilateral triangle that has the middle segment from step 1 as its base and points outward.
		\item Remove the line segment that is the base of the triangle from step 2.
	\end{enumerate}
	\begin{figure}[H]
		\centering
		\includegraphics{img/computing/von_kock_snowflake.jpg}
		\caption[Von Koch snowflake]{Von Koch snowflake (source: Wikipedia)}
	\end{figure}
	After each iteration, the number of sides of the Koch snowflake increases by a factor of $4$, so the number of sides after n iterations is given by:
	
	If the original equilateral triangle has sides of length $s$, the length of each side of the snowflake after $n$ iterations is:
	
	the perimeter of the snowflake after $n$ iterations is therefore of:
	
	The Koch curve has an infinite length because the total length of the curve increases by one third with each iteration. Thas is to say:
	
	The funny thing is that the area is finite... when the perimeter is infinite...
	
	Indeed, in each iteration a new triangle is added on each side of the previous iteration, so the number of new triangles added in iteration $n$ is:
	
	The area of each new equilateral triangle added in an iteration is one ninth of the area of each equilateral  triangle added in the previous iteration, so the area of each equilateral  triangle added in iteration $n$ is:
	
	where $a_0$ is the area of the original equilateral  triangle. The total new area added in iteration $n$ is therefore:
	
	The total area of the snowflake after $n$ iterations is then obviously:
	
	Collapsing the geometric sum of the type $\sum x^n$ using (\SeeChapter{see section Sequences and Series page \pageref{geometric sequence}}):
	
	we get:
	
	The limit of the area is then immediate:
	
	So the area of the Koch snowflake is $8/5$ of the area of the original triangle. 
	
	What is so disturbing with Von Koch fractal is that we start from a line of finite length, to arrive at the end to a line of infinite length if we reiterate infinitly structure but it has a finished surface ... it's a "pathological" curve as the mathematicians say sometimes.
	
	The Koch curve is legendary because it was used to Mandelbrot to write an article about the problem of measuring the length of the coasts of sea coasts (because the most the basic unit of measurement taken was small, more the perimeter ot the coast was great). He proposed to consider the coasts as fractals for which it is impossible to measure the perimeter but "fractal tree" or in other words: the fractal dimension.
	
	The Koch Curve consists of $4^n$ subsets with magnification factor $3^n$. So the fractal dimension is:
	
	
	\paragraph{Coastline paradox}\mbox{}\\\\
	It is now the right moment in our point of view to speak about the "\NewTerm{coastline paradox}\index{coastline paradox}" that is the counterintuitive observation that the coastline of a landmass does not have a well-defined length. This results from the fractal-like properties of coastlines. The first recorded observation of this phenomenon was by Lewis Fry Richardson and it was expanded by Benoit Mandelbrot.

	More concretely, the length of the coastline depends on the method used to measure it. Since a landmass has features at all scales, from hundreds of kilometers in size to tiny fractions of a millimeter and below, there is no obvious size of the smallest feature that should be measured around, and hence no single well-defined perimeter to the landmass. Various approximations exist when specific assumptions are made about minimum feature size.
	\begin{figure}[H]
		\centering
		\includegraphics{img/computing/coastline_paradox.jpg}
		\caption[Coastline paradox]{Coastline paradox (source: Wikipedia)}
	\end{figure}
	The length of "true fractal" therefore always diverges to infinity, as if one were to measure a coastline with infinite, or near-infinite resolution, the length of the infinitely smaller bends of the coastline would add up to infinity. However, this figure relies on the assumption that space can be subdivided indefinitely. The truth value of this assumption - which underlies Euclidean geometry and serves as a useful model in everyday measurement - is a matter of philosophical speculation, and may or may not reflect the changing realities of 'space' and 'distance' on the atomic level (approximately the scale of a nanometer). The Planck length, many orders of magnitude smaller than an atom, is proposed as the smallest measurable unit possible in the universe.

	In reality, permanent features of the coastline of order of size $1$ cm or less do not exist, because of erosion and other action of the sea. In most places the minimum size is much larger than this. Thus the concept of an infinite fractal is not applicable to the coastline.
For practical considerations, an appropriate choice of minimum feature size is on the order of the units being used to measure. If a coastline is measured in kilometers, then small variations much smaller than one kilometer are easily ignored.

	
	
	\subsubsection{Natural fractals}
	Besides the purely mathematical aspect of fractals, we can find, via heuristics, contracting applications for fractal shapes similar to that we can find in nature. Let us see some examples with Maple 4.00b always taking first for common basis of all fractals that follow, the following procedures (if the copy/paste form the book in Maple 4.00b does not work, simply rewrite the code in the Maple console):
	
	\texttt{>transforme\textunderscore point := proc(t, p)}\\
    \texttt{[t[1]*p[1]+t[2]*p[2]+t[5], t[3]*p[1]+t[4]*p[2]+t[6]]}\\
	\texttt{end:}

	\texttt{>transforme\textunderscore triangle := proc(t, triangle)}\\
    \texttt{local i;}\\
    \texttt{[seq(transforme\textunderscore point(t, triangle[i]), i=1 .. 3)]}\\
  	\texttt{end:}

  	\texttt{>transform\textunderscore square := proc(t, square)}\\
    \texttt{local i;}\\
    \texttt{[seq(transforme\textunderscore point(t, square[i]), i=1 .. 4)]}\\
	\texttt{end:}

	\texttt{>IFS := proc(n, liste\textunderscore de\textunderscore transformations,col)}\\
     \texttt{local i, j, k, s, sequence\textunderscore de\textunderscore triangles:}\\
     \texttt{options `Copyright by Alain Schauber, 1996`;}\\
     \texttt{sequence\textunderscore de\textunderscore triangles := [[0, 0], [1, 0], [0.5, 1]];}\\
     \texttt{for j to n do}\\
     \texttt{s := NULL;}\\
     \texttt{for i to nops(liste\textunderscore de\textunderscore transformations) do}\\
        \texttt{s := s,}\\
        \texttt{seq(transforme\textunderscore triangle(liste\textunderscore de\textunderscore transformations[i],}\\
        \texttt{op(k, [sequence\textunderscore de\textunderscore triangles])),}\\
        \texttt{k=1 .. nops([sequence\textunderscore de\textunderscore triangles]))}\\
       \texttt{od;}\\
      \texttt{sequence\textunderscore de\textunderscore triangles := s}\\
     \texttt{od;}\\
    \texttt{plots[polygonplot]([sequence\textunderscore de\textunderscore triangles], axes=none, color=col, scaling=constrained)}\\
  \texttt{end:}

	\texttt{>IFSS := proc(n, liste\textunderscore de\textunderscore transformations,col)}\\
     \texttt{local i, j, k, s, seq\textunderscore square:}\\
     \texttt{seq\textunderscore square :=[[0,0],[1,0],[1,1],[0,1]];}\\
     \texttt{for j to n do}\\
     \texttt{s := NULL;}\\
     \texttt{for i to nops(liste\textunderscore de\textunderscore transformations) do}\\
        \texttt{s := s,}\\
        \texttt{seq(transform\textunderscore square(liste\textunderscore de\textunderscore transformations[i],}\\
        \texttt{op(k, [seq\textunderscore square])),}\\
        \texttt{k=1 .. nops([seq\textunderscore square]))}\\
      \texttt{od;}\\
      \texttt{seq\textunderscore square := s }\\
    \texttt{od;}\\
   \texttt{plots[polygonplot]([seq\textunderscore square], axes=none, color=col, scaling=constrained)}\\
  \texttt{end:}\\
	
	\paragraph{Branch}\mbox{}\\\\
	We start from:
	
	\texttt{>rameau:=[[.387,.430,.430,-.387,.2560,.5220], }\\
	\texttt{[.441,-.091,-.009,-.322,.4219,.5059], [-.468,.020,-.113,.015,.4,.4]]:}

	And we get:

	\texttt{> IFSS(0,rameau,green);}
	\begin{figure}[H]
		\centering
		\includegraphics{img/computing/branch_set_01.jpg}
		\caption[]{Start set for generic branch fractal}
	\end{figure}
	\texttt{> IFSS(1,rameau,green);}
	\begin{figure}[H]
		\centering
		\includegraphics{img/computing/branch_set_02.jpg}
		\caption[]{First iteration set for generic branch fractal}
	\end{figure}
	\texttt{> IFSS(2,rameau,green);}
	\begin{figure}[H]
		\centering
		\includegraphics{img/computing/branch_set_03.jpg}
		\caption[]{Second iteration set for generic branch fractal}
	\end{figure}
	\texttt{> IFSS(3,rameau,green);}
	\begin{figure}[H]
		\centering
		\includegraphics{img/computing/branch_set_04.jpg}
		\caption[]{Third iteration set for generic branch fractal}
	\end{figure}
	\texttt{> IFSS(4,rameau,green);}
	\begin{figure}[H]
		\centering
		\includegraphics{img/computing/branch_set_05.jpg}
		\caption[]{Fourth iteration set for generic branch fractal}
	\end{figure}
	\texttt{> IFSS(5,rameau,green);}
	\begin{figure}[H]
		\centering
		\includegraphics{img/computing/branch_set_06.jpg}
		\caption[]{Fifth iteration set for generic branch fractal}
	\end{figure}
	\texttt{> IFSS(6,rameau,green);}
	\begin{figure}[H]
		\centering
		\includegraphics{img/computing/branch_set_07.jpg}
		\caption[]{Sixth iteration set for generic branch fractal}
	\end{figure}
	
	\paragraph{Snowflake}\mbox{}\\\\
	We start from:
	
	\texttt{>cristal:=[[.255,0,0,.255,.3726,.6714],[.255,0,0,.255,.1146,.2232], }
	\texttt{[.255,0,0,.255,.6306,.2232],[.37,-.642,.642,.37,.6356,-.0061]]:}

	And we get:
	
	\texttt{>IFSS(0, cristal,green);}
	\begin{figure}[H]
		\centering
		\includegraphics{img/computing/snowflake_set_01.jpg}
		\caption[]{Start set for snowflake fractal}
	\end{figure}
	\texttt{> IFSS(1, cristal,green);}
	\begin{figure}[H]
		\centering
		\includegraphics{img/computing/snowflake_set_02.jpg}
		\caption[]{First iteration set for snowflake fractal}
	\end{figure}
	\texttt{> IFSS(2, cristal,green);}
	\begin{figure}[H]
		\centering
		\includegraphics{img/computing/snowflake_set_03.jpg}
		\caption[]{Second iteration set for snowflake fractal}
	\end{figure}
	\texttt{> IFSS(3, cristal,green);}
	\begin{figure}[H]
		\centering
		\includegraphics{img/computing/snowflake_set_04.jpg}
		\caption[]{Third iteration set for snowflake fractal}
	\end{figure}
	\texttt{> IFSS(4, cristal,green);}
	\begin{figure}[H]
		\centering
		\includegraphics{img/computing/snowflake_set_05.jpg}
		\caption[]{Fourth iteration set for snowflake fractal}
	\end{figure}
	\texttt{> IFSS(5, cristal,green);}
	\begin{figure}[H]
		\centering
		\includegraphics{img/computing/snowflake_set_06.jpg}
		\caption[]{Fifth iteration set for snowflake fractal}
	\end{figure}
	\texttt{> IFSS(6, cristal,green);}
	\begin{figure}[H]
		\centering
		\includegraphics{img/computing/snowflake_set_07.jpg}
		\caption[]{Sixth iteration set for snowflake fractal}
	\end{figure}
	\texttt{> IFSS(7, cristal,green);}
	\begin{figure}[H]
		\centering
		\includegraphics{img/computing/snowflake_set_08.jpg}
		\caption[]{Seventh iteration set for snowflake fractal}
	\end{figure}
	
	\pagebreak
	\paragraph{Tree}\mbox{}\\\\
	We start from:
	
	\texttt{>tree := [[-0.04, 0, -0.23, -0.65, -0.08, 0.26], [0.61, 0, 0, 0.31, 0.07, 2.5],}
	\texttt{[0.65, 0.29, -0.3, 0.48, 0.54, 0.39], [0.64, -0.3, 0.16, 0.56, -0.56, 0.4]]:}

	And we get:
	
	\texttt{>IFS(0, tree, green);}
	\begin{figure}[H]
		\centering
		\includegraphics{img/computing/tree_set_01.jpg}
		\caption[]{Start set for tree fractal}
	\end{figure}
	\texttt{> IFS(1, tree, green);}
	\begin{figure}[H]
		\centering
		\includegraphics{img/computing/tree_set_02.jpg}
		\caption[]{First iteration set for tree fractal}
	\end{figure}
	\texttt{> IFS(2, tree, green);}
	\begin{figure}[H]
		\centering
		\includegraphics{img/computing/tree_set_03.jpg}
		\caption[]{Second iteration set for tree fractal}
	\end{figure}
	\texttt{> IFS(3, tree, green);}
	\begin{figure}[H]
		\centering
		\includegraphics{img/computing/tree_set_04.jpg}
		\caption[]{Third iteration set for tree fractal}
	\end{figure}
	\texttt{> IFS(4, tree, green);}
	\begin{figure}[H]
		\centering
		\includegraphics{img/computing/tree_set_05.jpg}
		\caption[]{Fourth iteration set for tree fractal}
	\end{figure}
	\texttt{> IFS(5, tree, green);}
	\begin{figure}[H]
		\centering
		\includegraphics{img/computing/tree_set_06.jpg}
		\caption[]{Fifth iteration set for tree fractal}
	\end{figure}
	
	\pagebreak
	\paragraph{Fern}\mbox{}\\\\
	We start from:
	
	\texttt{>fougere:=[[0,0,0,0.16,0,0],[0.2,-0.26,0.23,0.22,0,1.6],}
	\texttt{[-0.15,0.28,0.26,0.24,0,0.44],[0.85,0.04,-0.04,0.85,0,1.6]]:}

	And we get:
	
	\texttt{>IFS(0, fougere, blue);}
	\begin{figure}[H]
		\centering
		\includegraphics{img/computing/fern_set_01.jpg}
		\caption[]{Start set for fern fractal}
	\end{figure}
	\texttt{>IFS(1, fougere, blue);}
	\begin{figure}[H]
		\centering
		\includegraphics{img/computing/fern_set_02.jpg}
		\caption[]{First iteration set for fern fractal}
	\end{figure}
	\texttt{>IFS(2, fougere, blue);}
	\begin{figure}[H]
		\centering
		\includegraphics{img/computing/fern_set_03.jpg}
		\caption[]{Second iteration set for fern fractal}
	\end{figure}
	\texttt{>IFS(3, fougere, blue);}
	\begin{figure}[H]
		\centering
		\includegraphics{img/computing/fern_set_04.jpg}
		\caption[]{Third iteration set for fern fractal}
	\end{figure}
	\texttt{>IFS(4, fougere, blue);}
	\begin{figure}[H]
		\centering
		\includegraphics{img/computing/fern_set_05.jpg}
		\caption[]{Fourth iteration set for fern fractal}
	\end{figure}
	\texttt{>IFS(5, fougere, blue);}
	\begin{figure}[H]
		\centering
		\includegraphics{img/computing/fern_set_06.jpg}
		\caption[]{Fifth iteration set for fern fractal}
	\end{figure}
	\texttt{>IFS(6, fougere, blue);}
	\begin{figure}[H]
		\centering
		\includegraphics{img/computing/fern_set_07.jpg}
		\caption[]{Sixth iteration set for fern fractal}
	\end{figure}
	\texttt{>IFS(7, fougere, blue);}
	\begin{figure}[H]
		\centering
		\includegraphics{img/computing/fern_set_08.jpg}
		\caption[]{Eighth iteration set for fern fractal}
	\end{figure}
	etc. Unitl we get:
	\begin{figure}[H]
		\centering
		\includegraphics{img/computing/fern_set_09.jpg}
		\caption[]{Attractor set for fern fractal}
	\end{figure}
	and we will stop here because the examples of IFS and natural fractals are uncountable...
	
	\pagebreak
	\subsection{Escape Time Algorithm Fractals}
	Several methods have been proposed to construct fractal images as we mentioned at the beginning of this section. The are generally three way of generating fractals that are well known:
	\begin{itemize}
		\item Iterated function systems (seen previously)
		\item Escape time fractals (we will see now)
		\item Random fractals
	\end{itemize}
	So we will now focus on the methods named "\NewTerm{escape time methods}\index{escape time methods}".
	
	For this, we place ourselves in the complex plane $\mathbb{C}$ consisting of the points $M (x, y)$ of affix:
	
	Afterwards we consider a complex sequence defined by:
	
	and:
	
	the function $f$ being a complex continuous function. We will assumed that $f$ is build in such a way that is has a fixed point $x_0$, that is to say that there exists $x_0$ such as:
	
	It is therefore simply the application of fixed-point theorem already mentioned several times before. Under some conditions on f and equation, we find that the following points does not diverge (which means it is not interested in points that converge but to those who do not diverge!). This method is the basis for the construction of Mandelbrot and Julia sets.
	
	Building a fractal image from such a set of defined sequences, is equivalent to study for each pair $(x, y)$ of the plane the behavior of the sequence itself. We then associates a color to each result (that is to say, each pair $(x, y)$) representing the "speed" of divergence of the sequence.
	
	To study the convergence of a series, we look at his first $n$ elements, if we detect that the divergence conditions are satisfied then we can say that this sequence diverges, otherwise this result is potentially convergent. We notice that more $n$ is big, the more accurate are the results (but the computing time will be great).
	
	The simplest algorithm for generating a representation of an escape time fractal (set) consist in repeating a calculation performed for each $x$, $y$ point in the plot area and based on the behavior of that calculation, a color is chosen for that pixel.

	The $x$ and $y$ locations of each point are used as starting values in a repeating, or iterating calculation (described in detail below). The result of each iteration is used as the starting values for the next. The values are checked during each iteration to see whether they have reached a critical "escape" condition. If that condition is reached, the calculation is stopped, the pixel is drawn, and the next $x$, $y$ point is examined. For some starting values, escape occurs quickly, after only a small number of iterations. For starting values very close to but not in the set, it may take hundreds or thousands of iterations to escape. For values within the Mandelbrot set, escape will never occur. The programmer or user must choose how much iteration, or "depth", they wish to examine. The higher the maximal number of iterations, the more detail and subtlety emerge in the final image, but the longer time it will take to calculate the fractal image.

	Escape conditions can be simple or complex. Because no complex number with a real or imaginary part greater than $2$ can be part of the set, a common bailout is to escape when either coefficient exceeds$ $2. A more computationally complex method that detects escapes sooner, is to compute distance from the origin using the Pythagorean theorem, i.e., to determine the absolute value, or modulus, of the complex number. If this value exceeds $2$, the point has reached escape. More computationally intensive rendering variations include the Buddhabrot method, which finds escaping points and plots their iterated coordinates.

	The color of each point represents how quickly the values reached the escape point. Often black is used to show values that fail to escape before the iteration limit, and gradually brighter colors are used for points that escape. This gives a visual representation of how many cycles were required before reaching the escape condition.

	To render such an image, the region of the complex plane we are considering is subdivided into a certain number of pixels. To color any such pixel, let $c$  be the midpoint of that pixel. We now iterate the critical point under the chosen function, checking at each step whether the orbit point has modulus larger than $R$ the convergence radius. When this is the case, we know that $c$  does not belong to the escape time fractal, and we color our pixel according to the number of iterations used to find out. Otherwise, we keep iterating up to a fixed number of steps, after which we decide that our parameter is "probably" in the escape time fractal, or at least very close to it, and color the pixel black.

	In pseudocode, this algorithm would look as follows. The algorithm does not use complex numbers and manually simulates complex-number operations using two real numbers, for those who do not have a complex data type. The program may be simplified if the programming language includes complex-data-type operations.
	
	\begin{algorithm}[H]
	\KwData{I,R}
	\For{\text{each pixel} $(P_x,P_y)$ \text{on the screen}}{
        $x_0 =$ scaled $x$ coordinate of pixel \;
        $y_0 =$ scaled $y$ coordinate of pixel \;
		$x=0.0$\;
		$y=0.0$\;
		iteration$=0$\;
		max\textunderscore iteration$=I$\;
		\While{$x^2 + y^2 < R^2$  AND  $i <$ max\textunderscore iteration}{
			$x=\Re(z(x,y))+x_0$\;
			$y=\Im(z(x,y))+y_0$\;
			$i:=i+1$\;
		}
		color := palette[iteration]\;
	}
	 plot $(P_x,P_y,\text{colors})$\;
	 \caption{Escape Time Fractal pseudo-code algorithm}
	\end{algorithm}
	
	\subsubsection{Mandelbrot set}
	We construct the Mandelbrot set through iterations in the complex plane (this is named also "\NewTerm{holomorphic dynamics}\index{holomorphic dynamics}"). The function is of the form:
	
	where $c$ a constant parameter such that $c\in\mathbb{C}$ (so we double the argument of the initial complex number we squared its norm!).
	
	The first term of the sequence is zero. So we have the following defined by:
	
	Why do we start with $z_0=0$?: Because zero is the critical point of $z^2+c$, that is to say the point satisfying the extremum:
	
	For each point of affix $x + \mathrm{i}y$ of the plane, we study the above sequence for $c=x+\mathrm{i}y$. If the sequence diverges, we say that the tested point does not belong to the Mandelbrot set, if the sequence converges, we say that the point belongs to the Mandelbrot set defined then by (the definition and name is due to Adrien Douady, in tribute to Benoit Mandelbrot):
	
	
	To reproduce a basic representation of the Mandelbrot fractal, we associate to $c$ complex values of the plane. It is generally considered the portion of the complex plane having as real part, values between $-2.5$ and $1.5$, and as imaginary part, values between $-1.5$ and $1.5$. This portion of the complex plane is divided so to form a grid whose elements will be associated with values of $c$. For each value of $c$, we get a result that modules can converge (bounded sequence) or diverge (non-bounded sequence).
	
	In practice, it is considered that sequence of modules converges if the first $30$ modules are less than $2$ ($R=2$ in the previous pseudo-code), that is to say $|z_{30}|\leq 30$. When the sequence of the modules converges (bounded sequence), we color in black the grid point. After considering all points of the grid, we get a set of blackened points named: "Mandelbrot set" or "Mandelbrot fractal" as we already know and denoted by $\mathcal{M}$. What constitutes a remarkably curious result!
	\begin{figure}[H]
		\centering
		\includegraphics{img/computing/elementary_mandelbrot_set_reprsentation.jpg}
		\caption{Bichromic Mandelbrot Fractal}
	\end{figure}
	We can also color the points outside the Mandelbrot set using colors that depend on the number of terms calculated before obtaining a module greater than or equal to $2$. The Module points of the same color can be interpreted as points away at the same speed of the Mandelbrot set.
	\begin{tcolorbox}[title=Remark,colframe=black,arc=10pt]
	The list of $z_i$ generated by the iteration is named the "\NewTerm{orbit}\index{orbit (fractal)}" of $z_0$.
	\end{tcolorbox}
	We can also discover the Mandelbrot fractal in-deep using the following Maple 4.00b code derived from the earlier pseudo-code (available usually in high-school). The reader just have to copy the program below on a Maple worksheet and indicate instead of -$2 .. 1 .. 1.5 -1.5$ of the last line, the range of the real and imaginary parts of $c$ he wants to discover:\\
	
	 \texttt{>restart: with(plots):\\
	>couleur:=proc(a,b)\\
	local x,y,xi,yi,n;\\
	x:=a;\\
	y:=b;\\
	for n from 0 to 30 while evalf(x\string ^2+y\string ^2) < 4 do;\\
	   xi:=evalf(x\string ^2-y\string ^2+a);\\
	   yi:=evalf(2*x*y+b);\\
	   x:=xi;\\
	   y:=yi;\\
	od;\\
	n\\
	end:}
	
	You will therefore get the following result (the equivalent code for MATLAB™ and \texttt{R} are given in the companion books about MATLAB™ and \texttt{R}):
	\begin{figure}[H]
		\centering
		\includegraphics{img/computing/mandelbrot_fractal_maple.jpg}
		\caption{Mandelbrot Fractal with Maple 4.00b}
	\end{figure}
	The Mandelbrot set is self-similar in the vicinity of points named "\NewTerm{Misiurewicz points}\index{Misiurewicz points}":
	\begin{figure}[H]
		\centering
		\includegraphics{img/computing/mandelbrot_fractal_auto_similarity.jpg}
		\caption{Self-Similarity of Mandelbrot Fractal}
	\end{figure}
	It seem also that we can prove (we still search the proof...) that the Mandelbrot fractal real axes can be put in correspondence with the logistic bifurcation diagram that we have study in the section of Population Dynamics such that:
	\begin{figure}[H]
		\centering
		\includegraphics{img/computing/mandelbrot_fractal_logistic_bifurcation.jpg}
		\caption[Mandelbrot-Logistic bifurcation correspondence]{Mandelbrot-Logistic bifurcation correspondence (source: Wikipedia)}
	\end{figure}
	The functions are obviously both quadratic. In fact, the Mandelbrot Set can be recoded into to form logistic map (and vice versa).
	
	Indeed, as the Mandelbrot Fractal is obtained by iteration:
	
	and that we know that the logistic bifurcation is obtained by the iteration of:
	
	So let us put in Mandelbrot function the (anticipated) change of variable such that $z_n$ be dependent of $x_n$ and $r$ but $c$ must as constant must also be only dependent of the constant $r$:
	
	Then:
	
	
	\subsubsection{Julia set}
	The Julia set is builded almost in the same way that the Mandelbrot set (since the Julia set is actually a subset of it after investigation!). In the Mandelbrot set, $c$ scans the plane. For the Julia set, $c$ is fixed during the computation of the image. To each $c$ corresponds a particular set that will be denoted $J(c)$. What varies this time is $z_0$, which takes the value of the point to test. It is therefore $z_0$ that scans the plan.
	
	A point of initial coordinates $(x_0, y_0)$ and affix and $x + \mathrm{i}y$ belongs to $J(c)$ if and only if the sequence defined by:
	
	converge (is bounded).
	\begin{figure}[H]
		\centering
		\includegraphics[scale=0.6]{img/computing/julia_set_making_of.jpg}
		\caption{Makingofa Julia set}
	\end{figure}
	In fact, the Mandelbrot set is the set of points $c$ such that the Julia set of parameter $c$ is connex to, that is to say we can always found a $c$ such that when starting with $z_0$ in the Mandelbrot set is equivalent after a few iterations as starting with a fixed $z_0\neq 0$ and a given $c$ (thus the Mandelbrot set generalizes all the Julia sets !!!). So the figure of the Mandelbrot set contains figures all the Julia sets, which is remarkable (but logic ...!):
	
	If again we adapt the algorithm pseudo-code given earlier, we obtain to a given scale factor given the fractal shown below (obtained through the small Maple 4.00b code below and already used earlier for the Mandelbrot fractal):
	
	\texttt{>restart; with(plots):\\
	>julia:= proc(c,x, y)local z, m;\\
	z:= evalf(x+y*I);\\
	for m from 0 to 30 while abs(z) < 3 do\\
	   z:= z\string^2 + c\\
	   od;\\
	   m\\
	end:\\\\
	>J:= proc(d)\\
	global phonyvar;\\
	phonyvar:= d;\\
	(x, y) -> julia(phonyvar, x, y)\\
	end:\\\\
	>plot3d(0, -2..2, -1.3..1.3, style=patchnogrid,orientation=[-90,0], grid=[270, 270],scaling=constrained, color=J(-1.25));\\}
	
	\begin{figure}[H]
		\centering
		\includegraphics{img/computing/julia_fractal_maple.jpg}
		\caption{Julia Fractal with Maple 4.00b}
	\end{figure}
	and to show that the Mandelbrot set contains all the Julia sets we have qualitatively:
	\begin{figure}[H]
		\centering
		\includegraphics[scale=0.8]{img/computing/mandelbrot_family.jpg}
		\caption{Illustration of "fatherhood" ... of the Mandelbrot set}
	\end{figure}
	So we must be able to write a single algorithm (see below) that achieves all of Julia fractals by simply selecting a good starting point as shown in the following figures (we can see on top right the "Douady's Rabbit Fractal" also named "dragon fractal":
	\begin{figure}[H]
		\centering
		\includegraphics{img/computing/julia_sets.jpg}
	\end{figure}
	That we get with the following Maple 4.00b  code:
	
	\texttt{>couleur:=proc(a,b)\\
	local x,y,xi,yi,n;\\
	global reel,imaginaire;\\
	x:=a;\\
	y:=b;\\
	for n from 0 to 100 while evalf(x\string^2+y\string^2)<4 do;\\
	xi:=evalf(x\string^2-y\string^2+reel);\\
	yi:=evalf(2*x*y+imaginaire);\\
	x:=xi;\\
	y:=yi;\\
	od;\\
	n;\\
	end:\\\\
	>reel:=-0.181;\\
	>imaginaire:=-0.667;\\\\
	>plot3d(0,(-13/10)..(13/10),(-13/10)..(13/10),orientation=[-90,0], style=patchnogrid,scaling=constrained,axes=framed,numpoints=20000
	,color=couleur);\\}
	
	\subsubsection{Newton set}
	Newton sets are also so named because they arise from the resolution of the problem looking for zeros of a function by the Newton's method (\SeeChapter{see section Numerical Methods page \pageref{newton method}}).

	Given $f$ a function with values in $\mathbb{C}$ and differentiable in $\mathbb{C}$ , we take $z_0$ in $\mathbb{C}$ such that:
	
	There are then two ways to proceed:
	\begin{enumerate}
		\item Either we focus on $|z_0-z_i|$ and then we do same as before.

		\item Either we wonder to which zero $r_k$ the sequence converges and we focus on $|z_i-z_k|$.
	\end{enumerate}
	If again we translate our pseudo-code algorithm, into Maple language we obtain to a given scale factor the fractal shown below obtained with Maple 4.00b:
	
	\texttt{>restart:\\
	>newton:= proc(x, y)\\
	local z, m;\\
	z:= evalf(x+y*I);\\
	for m from 0 to 50 while abs(z\string^3-1) >= 0.001 \\	do\\
	z:= z - (z\string^3-1)/(3*z\string^2)\\
	od;\\
	m\\
	end:\\\\
	>plot3d(0,-2..2,-1.5..1.5,orientation=[-90,0],grid=[250, 250], 	\\ style=patchnogrid,scaling=constrained,color=newton);\\}

	
	\begin{figure}[H]
		\centering
		\includegraphics{img/computing/newton_fractal.jpg}
		\caption{Newton set fractal with Maple 4.00b}
	\end{figure}
	
	\begin{flushright}
	\begin{tabular}{l c}
	\circled{70} & \pbox{20cm}{\score{3}{5} \\ {\tiny 20 votes,  66.00\%}} 
	\end{tabular} 
	\end{flushright}

	%to make section start on odd page
	\newpage
	\thispagestyle{empty}
	\mbox{}
	\section{Logical Systems}\label{logical systems}
	\lettrine[lines=4]{\color{BrickRed}T}he reader familiar with the purpose of this book should not expect to see here any schemes of buttons, switches, timing diagrams or MIL standard wiring diagrams. We will remain in a purely formal framework of logical systems and their tools.\\
	
	\textbf{Definitions (\#\mydef):}
	\begin{enumerate}
		\item[D1.] We speak of "\NewTerm{asynchronous logic model}\index{asynchronous logic model}" (commonly named "\NewTerm{sequential logic model}\index{sequential logic model}") when the outputs of a system depends on the chronological order in which the entries will succeed.
		
		\item[D2.] We speak of "\NewTerm{combinatorial logic model}\index{combinatorial logic model}" when the outputs of a system depend only on the combination of the input variables.
	\end{enumerate}
	\begin{tcolorbox}[title=Remark,colframe=black,arc=10pt]
	We differentiate the "\NewTerm{strict logic}\index{strict logic}" of "\NewTerm{fuzzy logic}\index{fuzzy logic}" that will both  be defined in the details later.
	\end{tcolorbox}
	
	\subsection{Strict Logic}
	Consider first a set which we will denote by $\mathcal{B}$ with two elements (more formally denoted by $\bot,\top$).
	
	\textbf{Definitions (\#\mydef):}
	\begin{enumerate}
		\item[D1.] A "\NewTerm{strict logic variable}\index{strict logic variable}" or "\NewTerm{boolean variable}\index{Boolean variable}" is an element of $\mathcal{B}$ that has only two states $0$ and $1$ (as opposed to a fuzzy variable whose value can be \underline{between} $0$ and $1$). It is represented by Latin uppercase letters or lowercase Latin letters (depending on your choice).
		
		\item[D2.] A multivariate "\NewTerm{logic function $F$}\index{logic function}" of $n$ variables applies $\mathcal{B}^n$ in $\mathcal{B}$ such that:
		
		It combines to a $n$-tuple of logical variables $(b_0,b_1,...,b_{n-1})$ a value $F(b_0,b_1,...,b_{n-1})$.
		
		\item[D3.] There are different ways of expressing a logic function ("\NewTerm{Boolean function}\index{Boolean function}"). A function of $n$ variables is fully described by stating the values of this function for the set (or the subset of definition) of the combinations of the $n$-tuple variables:
		
	\end{enumerate}
	This statement usually takes the form of a table with $n + 1$ columns and no more than $2^n$ lines, each line exposing a combination of variables and the corresponding value of the function. The following table gives the general form of a "\NewTerm{truth table}\index{truth table}" function of three variables completely defined through a function $F$ (we already saw some simple examples in the section of Proof Theory):
	\begin{table}[H]
		\begin{center}
			\definecolor{gris}{gray}{0.85}
				\begin{tabular}{|p{2cm}|p{2cm}|p{2cm}|p{2cm}|}
					\hline
					\multicolumn{1}{c}{\cellcolor{black!30}$\pmb{A}$} & 
	  \multicolumn{1}{c}{\cellcolor{black!30}$\pmb{B}$}  & \multicolumn{1}{c}{\cellcolor{black!30}$\pmb{C}$} & \multicolumn{1}{c}{\cellcolor{black!30}$\pmb{F(A,B,C)}$} \\ \hline
					\centering\arraybackslash\ $0$ & \centering\arraybackslash\ $0$ & \centering\arraybackslash\ $0$ & \centering\arraybackslash\ $F(0,0,0)$ \\ \hline
					\centering\arraybackslash\ $0$ & \centering\arraybackslash\ $0$ & \centering\arraybackslash\ $1$ & \centering\arraybackslash\ $F(0,0,1)$ \\ \hline
					\centering\arraybackslash\ $0$ & \centering\arraybackslash\ $1$ & \centering\arraybackslash\ $0$ & \centering\arraybackslash\ $F(0,1,0)$ \\ \hline
					\centering\arraybackslash\ $0$ & \centering\arraybackslash\ $1$ & \centering\arraybackslash\ $1$ & \centering\arraybackslash\ $F(0,1,1)$ \\ \hline
					\centering\arraybackslash\ $1$ & \centering\arraybackslash\ $0$ & \centering\arraybackslash\ $0$ & \centering\arraybackslash\ $F(1,0,0)$ \\ \hline
					\centering\arraybackslash\ $1$ & \centering\arraybackslash\ $0$ & \centering\arraybackslash\ $1$ & \centering\arraybackslash\ $F(1,0,1)$ \\ \hline
					\centering\arraybackslash\ $1$ & \centering\arraybackslash\ $1$ & \centering\arraybackslash\ $0$ & \centering\arraybackslash\ $F(1,1,0)$ \\ \hline
					\centering\arraybackslash\ $1$ & \centering\arraybackslash\ $1$ & \centering\arraybackslash\ $1$ & \centering\arraybackslash\ $F(1,1,1)$  \\ \hline
			\end{tabular}
		\end{center}
		\caption{Generic truth table}
		\end{table}
		The elements of input of the systems will be considered as Boolean variables on which we can build a ring structure set, that by adding a particular axiom, we can bring to an algebra (in the computational sense and the set one!) commonly named "\NewTerm{Boolean algebra}\index{Boolean algebra}" as we will see now.
		
		So Boolean algebra is an algebra on itself (with a ring structure as we will define it rigorously later) proposing to translate signals with a value of the type $0/1$ (assimilated to: True/False) in mathematical expressions. For this, we define each elementary signal by "logical variables" and their treatment by "logical functions". Methods ("truth tables") exists to define the operations that we want to achieve, and to transcribe the result into an algebraic expression. Thanks the rules we will see later, these expressions can be simplified. This will allow to represent with simple symbols a logic circuit capable of performing basic arithmetic operations, that is to say a circuit that design the core components (at logic level) regardless of the realization through transistors (physical level).
	
	\subsubsection{Boolean Algebra}\label{boolean algebra}
	Boolean algebra (or "Boolean ring" to a given axiom...) is therefore a structure which is most often used in electronic (or microelectronics/optoelectronics) this is why some people name it sometimes "\NewTerm{Switching Algebra}\index{switching algebra}" or "\NewTerm{logic gates}\index{logic gates}". Therefore, a processor is composed of transistors for performing functions on digital signals. These transistors assembled together form components for performing simple functions. From these components it is possible to create circuits performing fairly complex operations. Boolean algebra (named after the English mathematician George Boole 1815-1864) is a means to create more or less easily such circuits.
	
	\begin{tcolorbox}[title=Remark,colframe=black,arc=10pt]
	It would be better before you start reading this chapter, to read at least in diagonal the subsection about Logic in the section of Proof Theory and on algebraic structures in the section of Set Theory.
	\end{tcolorbox}
	It is necessary for a rigorous definition of a Boolean algebra to give it in terms of abstract algebra.
	
	Reminder: A "\NewTerm{Boolean algebra}\index{Boolean algebra}" $(\mathcal{B},\vee,\wedge)$ is a set containing two particular elements $\bot,\top$, (abstract forms the $0$ and $1$) and has two internal composition laws, $\vee,\wedge$ (AND and logical OR) and verifies the following axioms to form a ring structure such that $\forall a,b,c\in \mathcal{B}$:
	
	\begin{itemize}
		\item[A1.] Associativity: $(a \vee b)\vee c=a \vee (b \vee c)$ and $(a \wedge b)\wedge c=a \wedge (b\wedge c)$
		
		\item[A2.] Commutativity: $a \vee b=b \vee a$ and $a \wedge b=b\wedge a$
		
		\item[A3.] Absorption: $a \wedge (a \vee b)=a$ and $a \vee (a \wedge b)=a$
		
		\item[A4.] Distributivity: $(a \vee b)\wedge c=(a \wedge c)\vee b \wedge c$ and $(a\wedge b)\vee c=(a \vee c)\wedge(b\vee c)$
		
		\item[A5.] Idempotence: $a \vee a=a$ and $a\wedge a=a$
		
		\item[A6.] Completation (or inversion): $a$ has a complement (negation) denoted by $\neg a$ or $\bar{a}$ (NOT) such as: $a \wedge \neg a=\top$ and $a \vee \neg a=\bot$
	\end{itemize}
	\begin{tcolorbox}[title=Remark,colframe=black,arc=10pt]
	The first four axioms establish a ring structure. The fifth axiom (idempotence) added to the first four defines the concept of "Boolean algebra".
	\end{tcolorbox}
	Strictly speaking to form a Boolean algebra we required a symmetrical element (\SeeChapter{see section Set Theory page \pageref{symmetrical element}}) with one of the two fundamental operators and we cannot do this directly with the two previous operators previously defined. That is why the real operators of Boolean algebra are normally the $\wedge$ (AND) and the $\Delta$ (symmetric difference), denoted in Boolean algebra by the symbol $\oplus$, the latter one being given by the logic operation:
	
	but to simplify, in the early school grades, it is common that we do implicitly reference to it without going into details.
	
	It follows that the binary set $(\mathcal{B},0,1)$ is therefore relatively to the laws $\vee,\wedge$ an "Abelian group". (\SeeChapter{see section Set Theory page \pageref{abelian group}}) Therefore, $(\mathcal{B},\vee)$ being an Abelian group, the law  $\wedge$  being associative and distributive with respect to $\vee$, $(\mathcal{B},\vee,\wedge)$ is therefore a "commutative ring with unit" (\SeeChapter{see section Set Theory page \pageref{communtative ring}}) since $\mathcal{B}$ has a neutral element relatively to the law $\vee$.
	\begin{tcolorbox}[title=Remarks,colframe=black,arc=10pt]
	\textbf{R1.} Thus, the operations $\vee,\wedge$ admits each a neutral element such that the value $1$ is the neutral element of $\wedge$ and $0$ the neutral element of $\vee$.\\
	
	\textbf{R2.} The two operations that we usually use to form a Boolean algebra are the "inclusive OR" rigorously denoted $\vee$ but more frequently denoted by the addition sign "$+$" and the "inclusive AND" rigorously denoted by $\wedge$ but more frequently denotey by the multiplication sign "$\cdot$".
	\end{tcolorbox}
	
	The preceding axioms may, however been proved from the "\NewTerm{axioms of the definition}\index{axioms of the definition}":
	\begin{enumerate}
		\item[A1.] Negation: $\top\neg\bot$ and $\bot\neg \top$
		
		\item[A2.] Double complentation: $\neg\neg a=\bar{\bar{a}}=a$
		
		\item[A3.] Neutral element 1: $\top$ is the neutral element of $\wedge$
		\item[A4.] Neutral element 2: $\perp$ is the neutral element of $\vee$
		
		\item[A5.] De Morgan theorem: $\neg (a \vee b)=\neg a \wedge \neg b$ and $\neg (a\wedge b)=\neg a \vee \neg b$
	\end{enumerate}
	\begin{tcolorbox}[title=Remark,colframe=black,arc=10pt]
	De Morgan's theorem can be proved using a simple truth table or algebraically as we will see just a little further below.
	\end{tcolorbox}
	It thus follows the following dual expressions:
	
	
	
	
	We name these expressions "\NewTerm{dual expressions}\index{dual expressions}" because by replacing in one equation logic equation, the $0$ by the $1$, the $\cdot$ by $+$ and inversely, by this same equation remains verified.
	
	\begin{theorem}
	Let us see now what we call the "\NewTerm{constants theorem}\index{constants theorem}" that consists to prove that:
	
	\end{theorem}
	\begin{dem}
	The proof is trivial (if necessary the reader can quickly do a truth table) as it comes from the same property of the concept of "Boolean ring" and the identity element $1$ relative to the $\wedge$ and its neutral element $0$ relative to $\vee$.
	\begin{flushright}
		$\blacksquare$  Q.E.D.
	\end{flushright}
	\end{dem}
	\begin{theorem}
	We have:
	
	\end{theorem}
	\begin{dem}
	The distributivity brings us to write:
	
	and applying the complementation:
	
	applying commutativity:
	
	and finally by applying the theorem of constants:
	
	\begin{flushright}
		$\blacksquare$  Q.E.D.
	\end{flushright}
	\end{dem}
	This proof will allow us to prove  the famous "\NewTerm{consensus theorem}\index{consensus theorem}":
	\begin{theorem}
	In Boolean algebra, the consensus theorem or rule of consensus[1] is the identity:
	
	\end{theorem}
	\begin{dem}
	To verify the concesus theorem relative to logical product:
	
	we can make use of a Venn diagram where we can see trivially that the therm term $ab$ is contained in the other two:
	\begin{figure}[H]
		\centering
		\includegraphics{img/computing/consensus_theorem.jpg}
		\caption{Venn diagram of the consensus theorem}
	\end{figure}
	Or more formally with other notations:
	
	Proceeding the request of a reader we can also build a truth table:
	\begin{table}[H]
		\begin{center}
			\definecolor{gris}{gray}{0.85}
				\begin{tabular}{|p{2cm}|p{2cm}|p{2cm}|p{2cm}|p{2cm}|}
					\hline
					\multicolumn{1}{c}{\cellcolor{black!30}$\pmb{x}$} & 
	  \multicolumn{1}{c}{\cellcolor{black!30}$\pmb{y}$}  & \multicolumn{1}{c}{\cellcolor{black!30}$\pmb{z}$} & \multicolumn{1}{c}{\cellcolor{black!30}$\pmb{xy \vee \bar{x}z \vee yz}$} & \multicolumn{1}{c}{\cellcolor{black!30}$\pmb{xy \vee \bar{x}z}$} \\ \hline
					\centering\arraybackslash\ $0$ & \centering\arraybackslash\ $0$ & \centering\arraybackslash\ $0$ & \centering\arraybackslash\ $0$ & \centering\arraybackslash\ $0$ \\ \hline
					\centering\arraybackslash\ $0$ & \centering\arraybackslash\ $0$ & \centering\arraybackslash\ $1$ & \centering\arraybackslash\ $1$ & \centering\arraybackslash\ $0$ \\ \hline
					\centering\arraybackslash\ $0$ & \centering\arraybackslash\ $1$ & \centering\arraybackslash\ $0$ & \centering\arraybackslash\ $0$ & \centering\arraybackslash\ $0$ \\ \hline
					\centering\arraybackslash\ $0$ & \centering\arraybackslash\ $1$ & \centering\arraybackslash\ $1$ & \centering\arraybackslash\ $1$ & \centering\arraybackslash\ $0$ \\ \hline
					\centering\arraybackslash\ $1$ & \centering\arraybackslash\ $0$ & \centering\arraybackslash\ $0$ & \centering\arraybackslash\ $0$ & \centering\arraybackslash\ $0$ \\ \hline
					\centering\arraybackslash\ $1$ & \centering\arraybackslash\ $0$ & \centering\arraybackslash\ $1$ & \centering\arraybackslash\ $0$ & \centering\arraybackslash\ $0$ \\ \hline
					\centering\arraybackslash\ $1$ & \centering\arraybackslash\ $1$ & \centering\arraybackslash\ $0$ & \centering\arraybackslash\ $1$ & \centering\arraybackslash\ $0$ \\ \hline
					\centering\arraybackslash\ $1$ & \centering\arraybackslash\ $1$ & \centering\arraybackslash\ $1$ & \centering\arraybackslash\ $1$ & \centering\arraybackslash\ $0$  \\ \hline
			\end{tabular}
		\end{center}
		\caption{Consensus theorem truth table}
		\end{table}
		
	Proceeding also with a Venn diagram, the reader will see without problem that we also have:
	
	
	\begin{flushright}
		$\blacksquare$  Q.E.D.
	\end{flushright}
	\end{dem}
	\begin{theorem}
	And finally the very famous "\NewTerm{Shannon's theorem}\index{Shannon's theorem}" (not to be confused with the Shannon theorem in signal theory!):
	
	\end{theorem}
	\begin{dem}
	We begin with the first relation:
	
	and for the second relation:
	
	\begin{flushright}
		$\blacksquare$  Q.E.D.
	\end{flushright}
	\end{dem}
	\label{de morgan theorem}
	\begin{theorem}
	Now let us come back on the De Morgan theorems previously presented as axioms:
	
	These two relations therefore express that the inverse (or opposite) of  product (or respectively the sum) of two variables is equal to the sum (respectively the product) of the inverse of these same variables.
	\end{theorem}
	\begin{dem}
	Suppose $\overline{(a+b)}=\bar{a}\bar{b}$ is true. So under the relations $\bar{a}+a=1$ and $\bar{a}a=1$ (axiom of complementation) we must have:
	
	So we need to prove that these relations are true:
	
	and:
	
	The second De Morgan theorem can be proven in the same way (we can put the details on request).
	\begin{flushright}
		$\blacksquare$  Q.E.D.
	\end{flushright}
	\end{dem}
	\begin{tcolorbox}[title=Remark,colframe=black,arc=10pt]
	These two theorems can be extended to as many variables as we want.
	\end{tcolorbox}
	\begin{corollary}
	As immediate corollary we have:
	\begin{itemize}
		\item $ab=\overline{\bar{a}+\bar{b}}$
		\item $a+b=\overline{\bar{a}\bar{b}}$
		\item $ab=\overline{\overline{ab}}$
		\item $a+b=\overline{\overline{a+b}}$
		\item $\overline{a+b+c+...}=\bar{a}\bar{b}\bar{c}...$
		\item $\overline{abc...}=\bar{a}+\bar{b}+\bar{c}+...$
	\end{itemize}
	\end{corollary}
	The logical expressions, as we have seen it until now thanks to the previous axioms, properties and theorems, can always be writtent into two different forms (by paling also with the negations $\neg$):
	\begin{enumerate}
		\item Under the form of a sum of logical products, also named "\NewTerm{normal disjcontive form NDF}\index{normal disjcontive form}", such as for example:
		
		The constitutive terms of this polynomial are in this example the monomials $\bar{a}\bar{c}d,\bar{a}c\bar{d}$. The variables or complementary variables of the monomials are the "letters": $\bar{a},\bar{c},d,c\bar{d}$.
		\begin{tcolorbox}[title=Remark,colframe=black,arc=10pt]
		If each (all) of the product contains all the input variables in a direct or complementary form, then the form is named "\NewTerm{first canonical form}\index{first canonical form}" or "\NewTerm{disjonctive canonical form}\index{disjonctive canonical form}". Each of the products is therefore named the "\NewTerm{minterm}\index{minterm}" (therefore in the previous relation, there are $2$ minterms).\\
		
		Obviously if we consider that each term and its negation is equivalent to $0$ and $1$ if we have one variable, the first canonical form has two terms  ($2^1$), if we have two variables, the first canonical form has four terms ($2^n$) and if $n$ variables we have $2^n$ terms.
		\end{tcolorbox}
		
		\item In the form of a product of logical sum, also named "\NewTerm{normal conjunctive form NCF}\index{normal conjunctive form}":
		
		\begin{tcolorbox}[title=Remark,colframe=black,arc=10pt]
		If each (all) of the sums contains all the input variables in a direct or complementary form, the form is named "\NewTerm{second canonical form}\index{second canonical form}" or "\NewTerm{conjunctive canonical form}\index{conjunctive canonical form}". Each of the sum is therefore named the "\NewTerm{maxterm}\index{maxterm}".\\
		
		Obviously if we consider that each term and it negation is equivalent to $0$ and $1$ if we have one variable, the second canonical form has two terms  ($2^1$), if we have two variables, the second canonical form has four terms ($2^n$) and if $n$ variables we have $2^n$ terms.
		\end{tcolorbox}
		Therefore, in other words, a normal disjonctive form is either a litteral and its complementary (one letter) or a disjonction of formulas written as conjonction of litterals. A conjonctive form is either a litteral (one letter) and its complementary, or a conjonction of formulas written as disjonction of litterals.
	\end{enumerate}
	The simplification methods we will see later aim to minimize the number of letters of the expressions so as to reduce the number of inputs of our logic system logic and therefore also the number of its components.
	\begin{tcolorbox}[title=Remark,colframe=black,arc=10pt]
	The algebraic simplification of an expression is to transform it so as to minimize the number of letters by applying theorems proven previously.
	\end{tcolorbox}
	To simplify expressions (or identify them) a known technique is therefore to use the "\NewTerm{Karnaugh tables}\index{Karnaugh tables}" that we will see further below in details.
	
	\subsubsection{Logical Functions (Boolean operators)}\label{boolean operators}
	So when we talk about Boolean algebra unless other indication, we refer to the three basic Boolean operations (AND, OR, NOT) and some other logic functions arising from them for which we have the following symbols as defined in circuit theory (MIL norm representation if no error...):
	\begin{figure}[H]
		\centering
		\includegraphics{img/engineering/logic_mil_gates.jpg}
		\caption{Logic MIL Gates}
	\end{figure}
	and their respective "\NewTerm{truth tables}\index{truth tables}":
	\begin{table}[H]
	\newcolumntype{C}[1]{>{\centering\arraybackslash}m{#1}}
	\begin{center}
		\begin{tabular}{|c|C{2cm}|C{2.5cm}|}
			\hline
			\multicolumn{3}{|c|}{\cellcolor{black!30}\textbf{AND ($\wedge $) Truth Table}} \\
			\hline
			\cellcolor{black!30}\textbf{AND} & \cellcolor{black!30}\textbf{0} & \cellcolor{black!30}\textbf{1}\\
			\hline
			\cellcolor{black!30}\textbf{0} & $0$ & $0$  \\
			\hline
			\cellcolor{black!30}\textbf{1} & $0$ & $1$ \\
			\hline
		\end{tabular}
		\caption{AND ($\wedge$) Truth table}
	\end{center}
	\end{table}
	\begin{table}[H]
	\newcolumntype{C}[1]{>{\centering\arraybackslash}m{#1}}
	\begin{center}
		\begin{tabular}{|c|C{2cm}|C{2cm}|}
			\hline
			\multicolumn{3}{|c|}{\cellcolor{black!30}\textbf{OR ($\vee $) Truth Table}} \\
			\hline
			\cellcolor{black!30}\textbf{OR} & \cellcolor{black!30}\textbf{0} & \cellcolor{black!30}\textbf{1}\\
			\hline
			\cellcolor{black!30}\textbf{0} & $0$ & $1$  \\
			\hline
			\cellcolor{black!30}\textbf{1} & $1$ & $1$ \\
			\hline
		\end{tabular}
		\caption{OR ($\vee$) truth table}
	\end{center}
	\end{table}
	\begin{table}[H]
	\newcolumntype{C}[1]{>{\centering\arraybackslash}m{#1}}
	\begin{center}
		\begin{tabular}{|c|C{2cm}|}
			\hline
			\multicolumn{2}{|c|}{\cellcolor{black!30}\textbf{NOT ($\neg $) Truth Table}} \\
			\hline
			\cellcolor{black!30}\textbf{NOT} & \cellcolor{black!30}\textbf{-} \\
			\hline
			\cellcolor{black!30}\textbf{0} & $0$   \\
			\hline
			\cellcolor{black!30}\textbf{1} & $1$ \\
			\hline
		\end{tabular}
		\caption{NOT ($\neg$) truth table}
	\end{center}
	\end{table}
	All others known (common) "logic functions" can be composed of these two basic operators. Such that by definition (given with their standard definition in the first line and with their different algebraic forms under their respective truth table):
	\begin{center}
	$\neg (a \wedge b)=\overline{a\cdot b}=\bar{a}+\bar{b}$
	\end{center}
	\begin{table}[H]
	\newcolumntype{C}[1]{>{\centering\arraybackslash}m{#1}}
	\begin{center}
		\begin{tabular}{|c|C{2.3cm}|C{2.3cm}|}
			\hline
			\multicolumn{3}{|c|}{\cellcolor{black!30}\textbf{NOT-AND ($\mathrm{NAND}$): $\mathrm{NOT}$ ($a$ $\mathrm{AND}$ $b$)}} \\
			\hline
			\cellcolor{black!30}$\pmb{\mathrm{NAND}}$ & \cellcolor{black!30}\textbf{0} & \cellcolor{black!30}\textbf{1}\\
			\hline
			\cellcolor{black!30}\textbf{0} & $1$ & $0$  \\
			\hline
			\cellcolor{black!30}\textbf{1} & $1$ & $0$ \\
			\hline
		\end{tabular}
		\caption{NOT-AND truth table}
	\end{center}
	\end{table}
	\begin{center}
		$\neg (a \vee b)=\overline{a + b}=\bar{a}\cdot\bar{b}$
	\end{center}
	\begin{table}[H]
	\newcolumntype{C}[1]{>{\centering\arraybackslash}m{#1}}
	\begin{center}
		\begin{tabular}{|c|C{2cm}|C{2cm}|}
			\hline
			\multicolumn{3}{|c|}{\cellcolor{black!30}\textbf{NOT-OR ($\mathrm{NOR}$): $\mathrm{NOT}$ ($a$ $\mathrm{OR}$ $b$)}} \\
			\hline
			\cellcolor{black!30}$\pmb{\mathrm{NAND}}$ & \cellcolor{black!30}\textbf{0} & \cellcolor{black!30}\textbf{1}\\
			\hline
			\cellcolor{black!30}\textbf{0} & $1$ & $0$  \\
			\hline
			\cellcolor{black!30}\textbf{1} & $0$ & $0$ \\
			\hline
		\end{tabular}
		\caption{NOT-OR truth table}
	\end{center}
	\end{table}
	\begin{center}
		$a\oplus b=(a \vee b)\wedge \neg(a \wedge b)=(a+b)\cdot \overline{(a\cdot b)}$\\
		$a\oplus b=a\wedge \neg b+b\wedge \neg q=a\cdot\bar{b}+b\cdot \bar{a}$\\
		$a\oplus b=\neg(a\wedge b+\neg a\wedge\neg b)=\overline{a\cdot b+\bar{a}\cdot\bar{b}}$
	\end{center}
	\begin{table}[H]
	\newcolumntype{C}[1]{>{\centering\arraybackslash}m{#1}}
	\begin{center}
		\begin{tabular}{|c|C{4.1cm}|C{4.1cm}|}
			\hline
			\multicolumn{3}{|c|}{\cellcolor{black!30}\textbf{EXCLUSIVE OR ($\mathrm{XOR}$): $[a\;\mathrm{OR}\;b]\;\mathrm{AND}\;[\mathrm{NOT}\;(a\;\mathrm{AND}\;\mathrm{b}]$}} \\
			\hline
			\cellcolor{black!30}$\pmb{\mathrm{XOR}}$ & \cellcolor{black!30}\textbf{0} & \cellcolor{black!30}\textbf{1}\\
			\hline
			\cellcolor{black!30}\textbf{0} & $0$ & $1$  \\
			\hline
			\cellcolor{black!30}\textbf{1} & $1$ & $0$ \\
			\hline
		\end{tabular}
		\caption{EXCLUSIVE OR $\oplus$ Truth table}
	\end{center}
	\end{table}
	where $a$ and $b$ are, as you will have understood, variables (or "bit" of Binary Digit) that can arbitrarily take the binary values $0$ or $1$.
	\begin{tcolorbox}[title=Remark,colframe=black,arc=10pt]
	The $\mathrm{XOR}$ logic function is often denoted in the literature by the operator $\oplus$ and we will consider as obvious that the $\mathrm{XOR}$ is also a group law and thus allows to construct an abelian commutative group. This property of the $\mathrm{XOR}$ is particularly used in cryptography.
	\end{tcolorbox}
	All the tables above can be sum up in the following figure with the corresponding MIL circuit theory symbols:
	\begin{figure}[H]
		\centering
		\includegraphics[scale=0.9]{img/computing/summary_boole_de_morgan.jpg}
		\caption{Summary Boole Algebra}
	\end{figure}
	The reader can easily check by himself that with a NAND gate (or an AND gate and an INVERTER) you've got all the parts you need to build a modern computer (classical or quantum one!)\label{all gates from NAND}:
	\begin{figure}[H]
		\centering
		\includegraphics[scale=0.9]{img/computing/all_gates_from_nand_gate.jpg}
		\caption{Construction of all gates from NAND gates}
	\end{figure}
	
	\pagebreak
	\subsubsection{Karnaugh maps}
	The "\NewTerm{Karnaugh map}\index{Karnaugh map}", also known as the "\NewTerm{K-map}\index{K-map}", is a method to simplify boolean algebra expressions. Maurice Karnaugh introduced it in 1953. The Karnaugh map reduces the need for extensive calculations by taking advantage of humans' pattern-recognition capability\footnote{The reader can found easily on the Internet many K-map generators that simplifies expressions automatically!}. 
	
	The required boolean results are transferred from a truth table onto a two-dimensional grid where the cells are ordered in Gray code, and each cell position represents one combination of input conditions, while each cell value represents the corresponding output value. Optimal groups of $1$s or $0$s are identified, which represent the terms of a canonical form of the logic in the original truth table. These terms can be used to write a minimal boolean expression representing the required logic.
	
	Karnaugh maps are used to simplify real-world logic requirements so that they can be implemented using a minimum number of physical logic gates. A sum-of-products expression can always be implemented using AND gates feeding into an OR gate, and a product-of-sums expression leads to OR gates feeding an AND gate. Karnaugh maps can also be used to simplify logic expressions in software design. Boolean conditions, as used for example in conditional statements, can get very complicated, which makes the code difficult to read and to maintain. Once minimized, canonical sum-of-products and product-of-sums expressions can be implemented directly using AND and OR logic operators.
	
	Let us consider for example the function:
	
	(in disjunctive normal form) and its respective truth table:
	\begin{table}[H]
		\centering
		\begin{tabular}{|c|c|c|}
		\hline
		\rowcolor[HTML]{9B9B9B} 
		\multicolumn{1}{|l|}{\cellcolor[HTML]{9B9B9B}$\pmb{b}$} & \multicolumn{1}{l|}{\cellcolor[HTML]{9B9B9B}$\pmb{a}$} & \multicolumn{1}{l|}{\cellcolor[HTML]{9B9B9B}$\pmb{z(a,b)}$} \\ \hline
		$0$ & $0$ & $1$ \\ \hline
		$0$ & $1$ & $0$ \\ \hline
		$1$ & $0$ & $1$ \\ \hline
		$1$ & $1$ & $1$ \\ \hline
		\end{tabular}
	\end{table}
	The Karnaugh map is defined by a representation like the one below following the \texttt{karnaugh-map} package of \LaTeX{} (as it does sadly not exist at this day any international norm on how to represent Karnaugh map):
	
	\begin{center}
	\begin{tikzpicture}[thick]
		\karnaughmapcolorfield{2}{0}{teal!50}%
		\karnaughmapcolorfield{2}{1}{violet!50}%
		\karnaughmapcolorfield{2}{3}{red!50}%
		\karnaughmap[omitnegated=false,binaryidx,omitzeros=false]{1011}
	\end{tikzpicture}
	\end{center}
	
	The colored items are those that we need to keep. That is: $\bar{a}\bar{b}$, $\bar{a}b$, $ab$.
	
	The Karnaugh table of a logic function thus has as many cells as possible combinations of variables which compose it, ie four cells for a function with two variables, and $2^n$ cells for a function with $n$ variables. Each cell, which is at the intersection of a row and column of the Karnaugh table, has the state $0$ or $1$ that the function $z(a,b)$ takes for the corresponding logical product of the variables (minterms).

	In the preceding example, however, we can see something interesting, the function $z(a,b)$, as we see very well, can be simplified in two ways:
	
	or also:
	
	This possible simplification is always done with two adjacent minterms in the Karnaugh table such as for the first solution $z=b+\bar{a}\bar{b}$:
	\begin{center}
	\begin{tikzpicture}[thick]
		\karnaughmapcolorfield{2}{0}{red!50}%
		\karnaughmapcolorfield{2}{2}{red!50}%
		\karnaughmap[omitnegated=false,binaryidx,omitzeros=false]{1011}
	\end{tikzpicture}
	\end{center}
	
	and for $z=ab+\bar{a}$:
	\begin{center}
	\begin{tikzpicture}[thick]
		\karnaughmapcolorfield{2}{2}{red!50}%
		\karnaughmapcolorfield{2}{3}{red!50}%
		\karnaughmap[omitnegated=false,binaryidx,omitzeros=false]{1011}
	\end{tikzpicture}
	\end{center}
	
	We see that indeed the first grouping / simplification (horizontal) is done on the row $\bar{a}$ and the second grouping / simplification (vertical) is done on column $b$ both results of the algebraic simplification of the function as following:
	
	and:
	
	So we could make the hypothesize that the Karnaugh table has for properties:
	\begin{enumerate}
		\item[P1.] To give us the normal disjunctive form of a function.

		\item[P2.] That all adjoining cells with a value of $1$ can be simplified in the respective symbol (letter) of their union
	\end{enumerate}
	It is therefore an extremely powerful tool (algorithm) for simplifying and determining logical functions.

	Let's look at an example with three variables!
	
	First we give the truth table that can help for a better understanding:
	\begin{table}[H]
		\centering
		\begin{tabular}{|c|c|c|c|}
		\hline
		\rowcolor[HTML]{9B9B9B} 
		\multicolumn{1}{|l|}{\cellcolor[HTML]{9B9B9B}$\pmb{a}$} & \multicolumn{1}{l|}{\cellcolor[HTML]{9B9B9B}$\pmb{b}$} & \multicolumn{1}{l|}{\cellcolor[HTML]{9B9B9B}$\pmb{c}$} & \multicolumn{1}{l|}{\cellcolor[HTML]{9B9B9B}$\pmb{z(a,b,c)}$} \\ \hline
		$0$ & $0$ & $0$ & $1$ \\ \hline
		$0$ & $0$ & $1$ & $1$ \\ \hline
		$0$ & $1$ & $0$ & $1$ \\ \hline
		$0$ & $1$ & $1$ & $1$ \\ \hline
		$1$ & $1$ & $0$ & $0$ \\ \hline
		$1$ & $1$ & $1$ & $0$ \\ \hline
		$1$ & $0$ & $0$ & $1$ \\ \hline
		$1$ & $0$ & $1$ & $1$ \\ \hline
		\end{tabular}
	\end{table}
	and the corresponding Karnaugh map:
	\begin{center}
	\begin{tikzpicture}[thick]
		\karnaughmap[omitnegated=false,binaryidx,omitzeros=false]{1111 0011}
	\end{tikzpicture}
	\end{center}
	The normal disjonctive form is the given by all the cells that are equal to $1$:
	
	This corresponds to the following colored Karnaugh map:
	\begin{center}
	\begin{tikzpicture}[thick]
		\karnaughmapcolorfield{2}{0}{red!50}%
		\karnaughmapcolorfield{2}{1}{red!50}%
		\karnaughmapcolorfield{2}{2}{red!50}%
		\karnaughmapcolorfield{2}{3}{red!50}%
		\karnaughmapcolorfield{3}{6}{red!50}%
		\karnaughmapcolorfield{3}{7}{red!50}%
		\karnaughmap[omitnegated=false,binaryidx,omitzeros=false]{1111 0011}
	\end{tikzpicture}
	\end{center}
	We see quickly that the previous normal disjonctive form can be simplified as:
	
	That gives the following Karnaugh map:
	\begin{center}
	\begin{tikzpicture}[thick]
		\karnaughmapcolorfield{2}{0}{teal!50}%
		\karnaughmapcolorfield{2}{1}{teal!50}%
		\karnaughmapcolorfield{2}{2}{teal!50}%
		\karnaughmapcolorfield{2}{3}{teal!50}%
		\karnaughmapcolorfield{3}{6}{violet!50}%
		\karnaughmapcolorfield{3}{7}{violet!50}%
		\karnaughmap[omitnegated=false,binaryidx,omitzeros=false]{1111 0011}
	\end{tikzpicture}
	\end{center}
	
	But it is less obvious with a Karnaugh map to see that all previous relations can be simplified into (using binary addition rules):
	
	That gives the following Karnaugh map:
	\begin{center}
	
	\end{center}
	And this latter result can be seen with the available Karnaugh map minimizer freeware that we can found nowadays on the Internet. For example the freeware Karnaugh Map Minimizer that give us for the previous example (we also see at the same time why it is boring - as always also in other field of engineering an science - that there are no ISO norm to standardize Karnaugh map representation):
	\begin{figure}[H]
		\centering
		\includegraphics[scale=0.8]{img/computing/karnaugh_map_minimizer.jpg}
		\caption{Karnaugh Map Minimizer 0.4}
	\end{figure}
	A difficulty remains however sometimes with this technique: how to choose the best construction of the table (disposition of letters)?

	In fact, there is a specific way of associating the Boolean algebra complement rule with what we named the "Gray code".

	\textbf{Definition (\#\mydef):} The "\NewTerm{reflected binary code (RBC)}\index{reflected binary code}", also known as "\NewTerm{Gray code}\index{Gray code}" after Frank Gray, is a binary numeral system where two successive values differ in only one bit (binary digit). The reflected binary code was originally designed to prevent spurious output from electromechanical switches\footnote{The problem with natural binary codes is that physical switches are not ideal: it is very unlikely that physical switches will change states exactly in synchrony. In the transition between the two states shown above, all three switches change state.} and is useful to optimally build Karnaugh maps.

	Here is an example of the Gray code for decimal $0$ to $15$ (the reader can notice that only one bit/switch at a time change at each row):
	\begin{table}[H]
		\centering
		\begin{tabular}{|c|c|c|}
		\hline
		\rowcolor[HTML]{9B9B9B} 
		\multicolumn{1}{|l|}{\cellcolor[HTML]{9B9B9B}\textbf{Decimal}} & \multicolumn{1}{l|}{\cellcolor[HTML]{9B9B9B}\textbf{Binary}} & \multicolumn{1}{l|}{\cellcolor[HTML]{9B9B9B}\textbf{Gray}} \\ \hline
		$0$ & $0000$ & $0000$ \\ \hline
		$1$ & $0001$ & $0001$ \\ \hline
		$2$ & $0010$ & $0011$ \\ \hline
		$3$ & $0011$ & $0010$ \\ \hline
		$4$ & $0100$ & $0110$ \\ \hline
		$5$ & $0101$ & $0111$ \\ \hline
		$6$ & $0110$ & $0101$ \\ \hline
		$7$ & $0111$ & $0100$ \\ \hline
		$8$ & $1000$ & $1100$ \\ \hline
		$9$ & $1001$ & $1101$ \\ \hline
		$10$ & $1010$ & $1111$ \\ \hline
		$11$ & $1011$ & $1110$ \\ \hline
		$12$ & $1100$ & $1010$ \\ \hline
		$13$ & $1101$ & $1011$ \\ \hline
		$14$ & $1110$ & $1001$ \\ \hline
		$15$ & $1111$ & $1000$ \\ \hline
		\end{tabular}
		\caption{Gray code of $0$ to $15$}
	\end{table}
	Using Gray code we can create optimal Karnaugh tables. The reason is simple, the Gray code changes only one bit at a time at each increment as we have just seen. In practice this means that for two consecutive values, $1$ and $2$ for example, one of the two variables will be the opposite of the other one.
	
	We can see this very well with the following structures:
	\begin{center}
	\begin{tikzpicture}[thick]
		\karnaughmap{4}
	\end{tikzpicture}
	\end{center}
	but more especially with greater tables (the order of the cell is not obvious):
	\begin{center}
	\begin{tikzpicture}[thick]
		\karnaughmap{8}
	\end{tikzpicture}
	\end{center}
	or:
	\begin{center}
	\begin{tikzpicture}[thick]
		\karnaughmap{16}
	\end{tikzpicture}
	\end{center}
	or more explicitly for that latter (you can therefore compare with the above Gray code table with $4$ digits to understand from where the internal numbering comes from):
	\begin{center}
	\begin{tikzpicture}[thick]
		\karnaughmap[defaultmap=16,binaryidx,omitnegated=false]{}
	\end{tikzpicture}
	\end{center}
	\begin{tcolorbox}[colframe=black,colback=white,sharp corners]
	\textbf{{\Large \ding{45}}Example:}\\\\
	Given $01$ corresponding to $\bar{b}a$ and $11$ corresponding to $ba$, the sum (disjunctive form) would give us:
	
	 which is reduced by using the complementation rule directly to:
	
	hence the advantage to represent them next to each other in a Karnaugh table.
	\end{tcolorbox}
	All this to say that when two formulas are found side by side in a Karnaugh map, we retain the similar elements only.

	The rules are such that we can reduce therefore when (see previous concrete example):
	\begin{itemize}
		\item[R1.] Two $1$ are juxtaposed in the table (here the $\bar{c}+c$ will be simplified in the disjunctive form):
		\begin{center}
		\begin{tikzpicture}[thick]
			\karnaughmapcolorfield{2}{0}{red!50}%
			\karnaughmapcolorfield{2}{1}{red!50}%
			\karnaughmap[omitnegated=false,binaryidx,omitzeros=true]{1100 0001}
		\end{tikzpicture}
		\end{center}
		
		\item[R2.] When two $1$ are at the extremities of the table (here the $\bar{b}+b$ will be simplified in the disjunctive form:
		\begin{center}
		\begin{tikzpicture}[thick]
			\karnaughmapcolorfield{2}{0}{red!50}%
			\karnaughmapcolorfield{3}{4}{red!50}%
			\karnaughmap[omitnegated=false,binaryidx,omitzeros=true]{1000 1001}
		\end{tikzpicture}
		\end{center}
		
		\item[R3.] A whole row is full of $1$ (in this case the both variables $ab$ disappear from the disjonctive form):
		\begin{center}
		\begin{tikzpicture}[thick]
			\karnaughmapcolorfield{2}{0}{red!50}%
			\karnaughmapcolorfield{2}{2}{red!50}%
			\karnaughmapcolorfield{3}{4}{red!50}%
			\karnaughmapcolorfield{3}{6}{red!50}%
			\karnaughmap[omitnegated=false,binaryidx,omitzeros=true]{1010 1010}
		\end{tikzpicture}
		\end{center}
		
		\item[R4.] A whole column is full of $1$ (in this case the both variables $cd$ disappear from the disjonctive form):
		\begin{center}
		\begin{tikzpicture}[thick]
			\karnaughmapcolorfield{2}{2}{red!50}%
			\karnaughmapcolorfield{3}{3}{red!50}%
			\karnaughmapcolorfield{4}{4}{red!50}%
			\karnaughmapcolorfield{4}{5}{red!50}%
			\karnaughmap[omitnegated=false,binaryidx,omitzeros=true]{0000 1111 1000 0000}
		\end{tikzpicture}
		\end{center}
		
		\item[R5.] Four adjacents cells are full of $1$ (in this case $b$ and $d$ disappear):
		\begin{center}
		\begin{tikzpicture}[thick]
			\karnaughmapcolorfield{4}{4}{red!50}%
			\karnaughmapcolorfield{4}{5}{red!50}%
			\karnaughmapcolorfield{4}{0}{red!50}%
			\karnaughmapcolorfield{4}{1}{red!50}%
			\karnaughmap[omitnegated=false,binaryidx,omitzeros=true]{1100 1100 0000 0000}
		\end{tikzpicture}
		\end{center}
		
		\item[R6.] The same cells can be used for two reductions:
		\begin{center}
		\begin{tikzpicture}[thick]
			\karnaughmapcolorfield{4}{4}{red!50}%
			\karnaughmapcolorfield{4}{5}{red!50}%
			\karnaughmapcolorfield{4}{0}{red!50}%
			\karnaughmapcolorfield{4}{1}{red!50}%
			\karnaughmapcolorfield[outline,ultra thick]{4}{4}{violet}%
			\karnaughmapcolorfield[outline,ultra thick]{4}{5}{violet}%
			\karnaughmapcolorfield[outline,ultra thick]{4}{c}{violet}%
			\karnaughmapcolorfield[outline,ultra thick]{4}{d}{violet}%
			
			\karnaughmap[omitnegated=false,binaryidx,omitzeros=true]{1100 1100 0000 1100}
		\end{tikzpicture}
		\end{center}
		
		\item[R7.] The same box can be used for two reductions:
		\begin{center}
		\begin{tikzpicture}[thick]
			\karnaughmapcolorfield{4}{0}{red!50}%
			\karnaughmapcolorfield{4}{4}{red!50}%
			\karnaughmapcolorfield[outline,ultra thick]{4}{4}{violet}%
			\karnaughmapcolorfield[outline,ultra thick]{4}{c}{violet}%
			
			\karnaughmap[omitnegated=false,omitzeros=true]{1000 1000 0000 1000}
		\end{tikzpicture}
		\end{center}
	\end{itemize}
	and without errors ... that's all but it's already not bad!
	
	\pagebreak
	\subsubsection{Arithmetic Boolean (binary) operations}
	Using all the elements demonstrated and given previously, we are now able to rigorously determine the logic function allowing Boolean addition and subtraction. Let us also recall that this being done, we can construct multiplication and division using respectively addition and subtraction.

	However, we can not with formal digital systems build elements for integration and differentiation. For this we refer the reader to the section of Electrokinetics where it is shown how to use inductors and capacitors to perform such operations with signals.

	\begin{tcolorbox}[title=Remark,colframe=black,arc=10pt]
	We will work on integers but the reader must remember that rational numbers can always be increased in power to be represented in an integral way (it remains to perform the inverse operation if necessary).
	\end{tcolorbox}
	The sum of two bytes will be denoted $S$, the retention $C_s$ (outgoing retention, also often denoted $C_\text{out}$) and the deferred retention $C_e$ (inward retention, also often denoted $C_\text{in}$).

	The truth table will be build with the "tip" that the system inputs $(a,b,C_e)$ take all possible values on $3$ bits (three letters) thus $2^3=8$ rows that we have represented in the following table:
	\begin{table}[H]
		\centering
		\begin{tabular}{|c|c|c|}
		\hline
		\rowcolor[HTML]{9B9B9B} 
		\multicolumn{1}{|l|}{\cellcolor[HTML]{9B9B9B}$\pmb{C_e}$} & \multicolumn{1}{l|}{\cellcolor[HTML]{9B9B9B}$\pmb{a}$} & \multicolumn{1}{l|}{\cellcolor[HTML]{9B9B9B}$\pmb{b}$} \\ \hline
		$0$ & $0$ & $0$ \\ \hline
		$0$ & $0$ & $1$ \\ \hline
		$0$ & $1$ & $0$ \\ \hline
		$0$ & $1$ & $1$ \\ \hline
		$1$ & $0$ & $0$ \\ \hline
		$1$ & $0$ & $1$ \\ \hline
		$1$ & $1$ & $0$ \\ \hline
		$1$ & $1$ & $1$ \\ \hline
		\end{tabular}
		\caption{Identification of the reported retentions for the binary sum}
	\end{table}
	And now the idea consists in adding the column constituted by the sum:
	
	row by row (without thinking to the outgoing retention $C_s$ that we will see a little bit further below):
	\begin{table}[H]
		\centering
		\begin{tabular}{|c|c|c|c|}
		\hline
		\rowcolor[HTML]{9B9B9B} 
		\multicolumn{1}{|l|}{\cellcolor[HTML]{9B9B9B}$\pmb{C_e}$} & \multicolumn{1}{l|}{\cellcolor[HTML]{9B9B9B}$\pmb{a}$} & \multicolumn{1}{l|}{\cellcolor[HTML]{9B9B9B}$\pmb{b}$} & \multicolumn{1}{l|}{\cellcolor[HTML]{9B9B9B}$\pmb{S}$} \\ \hline
		$0$ & $0$ & $0$ & $0$ \\ \hline
		$0$ & $0$ & $1$ & $1$ \\ \hline
		$0$ & $1$ & $0$ & $1$ \\ \hline
		$0$ & $1$ & $1$ & $0$ \\ \hline
		$1$ & $0$ & $0$ & $1$ \\ \hline
		$1$ & $0$ & $1$ & $0$ \\ \hline
		$1$ & $1$ & $0$ & $0$ \\ \hline
		$1$ & $1$ & $1$ & $1$ \\ \hline
		\end{tabular}
		\caption{Sum for the binary sum}
	\end{table}
	Now, row by row, we add the outgoing retention $C_s$ (which is none other than the value that is sent to the incoming retention of the next row) of the sum $S$:
	\begin{table}[H]
		\centering
		\begin{tabular}{|c|c|c|c|c|c|}
		\hline
		\rowcolor[HTML]{9B9B9B} 
		\multicolumn{1}{|l|}{\cellcolor[HTML]{9B9B9B}$\pmb{C_e}$} & \multicolumn{1}{l|}{\cellcolor[HTML]{9B9B9B}$\pmb{a}$} & \multicolumn{1}{l|}{\cellcolor[HTML]{9B9B9B}$\pmb{b}$} & \multicolumn{1}{l|}{\cellcolor[HTML]{9B9B9B}$\pmb{S}$} & \multicolumn{1}{l|}{\cellcolor[HTML]{9B9B9B}$\pmb{C_s}$} & \multicolumn{1}{l|}{\cellcolor[HTML]{9B9B9B}\textbf{Minterms}} \\ \hline
		$0$ & $0$ & $0$ & $0$ & $0$ & $\bar{C}_e\bar{a}\bar{b}$  \\ \hline
		$0$ & $0$ & $1$ & $1$ & $0$ & $\bar{C}_e\bar{a}b$  \\ \hline
		$0$ & $1$ & $0$ & $1$ & $0$ & $\bar{C}_ea\bar{b}$ \\ \hline
		$0$ & $1$ & $1$ & $0$ & $1$ & $\bar{C}_eab$  \\ \hline
		$1$ & $0$ & $0$ & $1$ & $0$ & $C_e\bar{a}\bar{b}$  \\ \hline
		$1$ & $0$ & $1$ & $0$ & $1$ & $C_e\bar{a}b$  \\ \hline
		$1$ & $1$ & $0$ & $0$ & $1$ & $C_e a\bar{b}$   \\ \hline
		$1$ & $1$ & $1$ & $1$ & $1$ & $C_eab$  \\ \hline
		\end{tabular}
		\caption{Retentions reported for the binary sum}
	\end{table}
	Therefore we have $4$ minterms (that is, the terms for which $S$ is non-zero in the rows $2$, $3$, $5$ and $8$) such that the normal disjonctive form is written:
	
	A possible simplification is:
	
	It also comes for the outgoing retention the following minterms:
	
	So finally we have:
	
	To build physically with fundamental logic gates this addition, it is useful first to introduce an intermediary logic circuit (but it's not obliged, it's just for pedagogical reasons and by educational tradition!). 
	
	Let us for this purpose consider the truth table of the addition without incoming retention, named a "\NewTerm{half adder}\index{half adder}", and represented by the following "technical" drawing:
	\begin{figure}[H]
		\centering
		\includegraphics[scale=1]{img/computing/half_adder_schema.jpg}
		\caption{Half adder logic diagram}
	\end{figure}
	\begin{table}[H]
		\centering
		\begin{tabular}{|c|c|c|c|}
		\hline
		\rowcolor[HTML]{9B9B9B} 
		\multicolumn{1}{l|}{\cellcolor[HTML]{9B9B9B}$\pmb{a}$} & \multicolumn{1}{l|}{\cellcolor[HTML]{9B9B9B}$\pmb{b}$} & \multicolumn{1}{l|}{\cellcolor[HTML]{9B9B9B}$\pmb{S}$} & \multicolumn{1}{l|}{\cellcolor[HTML]{9B9B9B}$\pmb{C_s}$} \\ \hline
		$0$ & $0$ & $0$ & $0$  \\ \hline
		$0$ & $1$ & $1$ & $0$  \\ \hline
		$1$ & $0$ & $1$ & $0$  \\ \hline
		$1$ & $1$ & $0$ & $1$  \\ \hline
		\end{tabular}
	\end{table}
	And now we can introduce the "\NewTerm{full adder}\index{full adder}" that adds binary numbers and accounts for values carried in as well as out following the relation proved earlier above and is made of two half-adder. So here is for recall the minterms and we introduce the corresponding logical circuit:
	
	\begin{figure}[H]
		\centering
		\includegraphics[width=1.0\textwidth]{img/computing/full_adder_schema.jpg}
		\caption{Full adder logic diagram}
	\end{figure}
	\begin{table}[H]
		\centering
		\begin{tabular}{|c|c|c|c|c|}
		\hline
		\rowcolor[HTML]{9B9B9B} 
		\multicolumn{1}{|l|}{\cellcolor[HTML]{9B9B9B}$\pmb{C_e}$} & \multicolumn{1}{l|}{\cellcolor[HTML]{9B9B9B}$\pmb{a}$} & \multicolumn{1}{l|}{\cellcolor[HTML]{9B9B9B}$\pmb{b}$} & \multicolumn{1}{l|}{\cellcolor[HTML]{9B9B9B}$\pmb{S}$} & \multicolumn{1}{l|}{\cellcolor[HTML]{9B9B9B}$\pmb{C_s}$}\\ \hline
		$0$ & $0$ & $0$ & $0$ & $0$ \\ \hline
		$0$ & $0$ & $1$ & $1$ & $0$ \\ \hline
		$0$ & $1$ & $0$ & $1$ & $0$ \\ \hline
		$0$ & $1$ & $1$ & $0$ & $1$ \\ \hline
		$1$ & $0$ & $0$ & $1$ & $0$ \\ \hline
		$1$ & $0$ & $1$ & $0$ & $1$ \\ \hline
		$1$ & $1$ & $0$ & $0$ & $1$  \\ \hline
		$1$ & $1$ & $1$ & $1$ & $1$ \\ \hline
		\end{tabular}
	\end{table}
	
	\pagebreak
	The subtraction (difference) of two bytes will be denoted $D$, the borrowing $B_s$ (outgoing borrowing, also often denoted $B_\text{out}$) and the reported borrowing $B_e$ (inbound borrowing, also often denoted $B_\text{in}$). The truth table will first be build as for the addition. That is, the system inputs $(a,b,e_e)$ take all possible values on $3$-bit (three-letter) the $2^3=8$ rows. Therefore:
	\begin{table}[H]
		\centering
		\begin{tabular}{|c|c|c|}
		\hline
		\rowcolor[HTML]{9B9B9B} 
		\multicolumn{1}{|l|}{\cellcolor[HTML]{9B9B9B}$\pmb{B_e}$} & \multicolumn{1}{l|}{\cellcolor[HTML]{9B9B9B}$\pmb{a}$} & \multicolumn{1}{l|}{\cellcolor[HTML]{9B9B9B}$\pmb{b}$} \\ \hline
		$0$ & $0$ & $0$ \\ \hline
		$0$ & $0$ & $1$ \\ \hline
		$0$ & $1$ & $0$ \\ \hline
		$0$ & $1$ & $1$ \\ \hline
		$1$ & $0$ & $0$ \\ \hline
		$1$ & $0$ & $1$ \\ \hline
		$1$ & $1$ & $0$ \\ \hline
		$1$ & $1$ & $1$ \\ \hline
		\end{tabular}
		\caption{Identification of the reported borrowing for the binary subtraction}
	\end{table}
	But we will a little subtlety. Rather than bore us to calculate $D=a-b-B_e$, we will calculate $D=a+(-b)+(-B_e)$ in the purpose to be able to work with the following truth table:
	\begin{table}[H]
		\centering
		\begin{tabular}{|c|c|c|}
		\hline
		\rowcolor[HTML]{9B9B9B} 
		\multicolumn{1}{|l|}{\cellcolor[HTML]{9B9B9B}$\pmb{-B_e}$} & \multicolumn{1}{l|}{\cellcolor[HTML]{9B9B9B}$\pmb{a}$} & \multicolumn{1}{l|}{\cellcolor[HTML]{9B9B9B}$\pmb{-b}$} \\ \hline
		$0$ & $0$ & $0$ \\ \hline
		$0$ & $0$ & $1$ \\ \hline
		$0$ & $1$ & $0$ \\ \hline
		$0$ & $1$ & $1$ \\ \hline
		$1$ & $0$ & $0$ \\ \hline
		$1$ & $0$ & $1$ \\ \hline
		$1$ & $1$ & $0$ \\ \hline
		$1$ & $1$ & $1$ \\ \hline
		\end{tabular}
		\caption{Inversion of the reported borrowing for the binary subtraction}
	\end{table}
	and now the idea is to add the difference column $D=a+(-b)+(-B_e)$ row by row (without thinking about the borrowing $B_s$) which will be strictly identical to the truth table of the sum:
	\begin{table}[H]
		\centering
		\begin{tabular}{|c|c|c|c|}
		\hline
		\rowcolor[HTML]{9B9B9B} 
		\multicolumn{1}{|l|}{\cellcolor[HTML]{9B9B9B}$\pmb{-B_e}$} & \multicolumn{1}{l|}{\cellcolor[HTML]{9B9B9B}$\pmb{a}$} & \multicolumn{1}{l|}{\cellcolor[HTML]{9B9B9B}$\pmb{-b}$} & \multicolumn{1}{l|}{\cellcolor[HTML]{9B9B9B}$\pmb{D}$} \\ \hline
		$0$ & $0$ & $0$ & $0$\\ \hline
		$0$ & $0$ & $1$ & $1$\\ \hline
		$0$ & $1$ & $0$ & $1$\\ \hline
		$0$ & $1$ & $1$ & $0$\\ \hline
		$1$ & $0$ & $0$ & $1$\\ \hline
		$1$ & $0$ & $1$ & $0$\\ \hline
		$1$ & $1$ & $0$ & $0$\\ \hline
		$1$ & $1$ & $1$ & $1$\\ \hline
		\end{tabular}
		\caption{Reported borrowing for the binary subtraction}
	\end{table}
	Now, row by row, we add the outgoing borrowing $B_s$ of the difference $D=a+(-b)+(-B_e)$ which written so, then becomes a sum $S$:
	\begin{table}[H]
		\centering
		\begin{tabular}{|c|c|c|c|c|c|}
		\hline
		\rowcolor[HTML]{9B9B9B} 
		\multicolumn{1}{|l|}{\cellcolor[HTML]{9B9B9B}$\pmb{-B_e}$} & \multicolumn{1}{l|}{\cellcolor[HTML]{9B9B9B}$\pmb{a}$} & \multicolumn{1}{l|}{\cellcolor[HTML]{9B9B9B}$\pmb{-b}$} & \multicolumn{1}{l|}{\cellcolor[HTML]{9B9B9B}$\pmb{S(D)}$} & \multicolumn{1}{l|}{\cellcolor[HTML]{9B9B9B}$\pmb{B_s}$} & \multicolumn{1}{l|}{\cellcolor[HTML]{9B9B9B}\textbf{minterms}} \\ \hline
		$0$ & $0$ & $0$ & $0$ & $0$ & $\bar{B}_e\bar{a}\bar{b}$\\ \hline
		$0$ & $0$ & $1$ & $1$ & $1$ & $\bar{B}_e\bar{a}b$\\ \hline
		$0$ & $1$ & $0$ & $1$ & $0$ & $\bar{B}_ea\bar{b}$\\ \hline
		$0$ & $1$ & $1$ & $0$ & $0$ & $\bar{B}_eab$\\ \hline
		$1$ & $0$ & $0$ & $1$ & $1$ & $B_e\bar{a}\bar{b}$\\ \hline
		$1$ & $0$ & $1$ & $0$ & $1$ & $B_e\bar{a}b$\\ \hline
		$1$ & $1$ & $0$ & $0$ & $0$ & $B_ea\bar{b}$\\ \hline
		$1$ & $1$ & $1$ & $1$ & $1$ & $B_eab$\\ \hline
		\end{tabular}
		\caption{Identification of subtraction minterms}
	\end{table}
	Therefore it comes $4$ minterms (that is, the terms for which $S(D)$ is non-zero at row $2$, $3$, $5$, $8$) such that the normal disjonctive form of the subtraction can be written:
	
	A trivial possible simplification is:
	
	It also comes for the outgoing borrowing the following minterms:
	
	So finally:
	
	To build physically with fundamental logic gates this subtraction, it is useful first to introduce an intermediary logic circuit (but it's not obliged, it's just for pedagogical reasons and by educational tradition!). 
	
	Let us for this purpose consider the truth table of the subtraction without incoming borrowing, named a "\NewTerm{half subtracter}\index{half subtracter}", and represented by the following "technical" drawing (don't forget that $b$ means in fact $-b$!):
	\begin{figure}[H]
		\centering
		\includegraphics[scale=1]{img/computing/half_subtracter_schema.jpg}
		\caption{Half subtracter logic diagram}
	\end{figure}
	\begin{table}[H]
		\centering
		\begin{tabular}{|c|c|c|c|c|}
		\hline
		\rowcolor[HTML]{9B9B9B} 
		\multicolumn{1}{l|}{\cellcolor[HTML]{9B9B9B}$\pmb{a}$} & \multicolumn{1}{l|}{\cellcolor[HTML]{9B9B9B}$\pmb{-b}$} & \multicolumn{1}{l|}{\cellcolor[HTML]{9B9B9B}$\pmb{S(D)}$} & \multicolumn{1}{l|}{\cellcolor[HTML]{9B9B9B}$\pmb{B_s}$} \\ \hline
		$0$ & $0$ & $0$ & $0$ \\ \hline
		$0$ & $1$ & $1$ & $1$ \\ \hline
		$1$ & $0$ & $1$ & $0$ \\ \hline
		$1$ & $1$ & $0$ & $0$ \\ \hline
		\end{tabular}
	\end{table}
	And now we can introduce the "\NewTerm{full subtracter}\index{full subtracter}" that substracts binary numbers and accounts for values carried in as well as out following the relation proved earlier above and is made of two half-adder. So here is for recall the minterms and we introduce the corresponding logical circuit:
	
	\begin{figure}[H]
		\centering
		\includegraphics[scale=0.8]{img/computing/half_subtractor_schema.jpg}
		\caption{Full subtracter logic diagram}
	\end{figure}
	\begin{table}[H]
		\centering
		\begin{tabular}{|c|c|c|c|c|}
		\hline
		\rowcolor[HTML]{9B9B9B} 
		\multicolumn{1}{|l|}{\cellcolor[HTML]{9B9B9B}$\pmb{-B_e}$} & \multicolumn{1}{l|}{\cellcolor[HTML]{9B9B9B}$\pmb{a}$} & \multicolumn{1}{l|}{\cellcolor[HTML]{9B9B9B}$\pmb{-b}$} & \multicolumn{1}{l|}{\cellcolor[HTML]{9B9B9B}$\pmb{S(D)}$} & \multicolumn{1}{l|}{\cellcolor[HTML]{9B9B9B}$\pmb{B_s}$}\\ \hline
		$0$ & $0$ & $0$ & $0$ & $0$\\ \hline
		$0$ & $0$ & $1$ & $1$ & $1$\\ \hline
		$0$ & $1$ & $0$ & $1$ & $0$ \\ \hline
		$0$ & $1$ & $1$ & $0$ & $0$\\ \hline
		$1$ & $0$ & $0$ & $1$ & $1$\\ \hline
		$1$ & $0$ & $1$ & $0$ & $1$\\ \hline
		$1$ & $1$ & $0$ & $0$ & $0$\\ \hline
		$1$ & $1$ & $1$ & $1$ & $1$\\ \hline
		\end{tabular}
	\end{table}
	
	\pagebreak
	\subsection{Fuzzy logic}\label{fuzzy logic}
	Classical logic only permits conclusions which are either true or false. However, there are also propositions with variable answers, such as one might find when asking a group of people to identify a color. In such instances, the truth appears as the result of reasoning from inexact or partial knowledge in which the sampled answers are mapped on a spectrum.
	
	Humans and animals often operate using fuzzy evaluations in many everyday situations. In the case where someone is tossing an object into a container from a distance, the person does not compute exact values for the object weight, density, distance, direction, container height and width, and air resistance to determine the force and angle to toss the object. Instead the person instinctively applies quick "fuzzy" estimates, based upon previous experience, to determine what output values of force, direction and vertical angle to use to make the toss.
Both degrees of truth and probabilities range between 0 and 1 and hence may seem similar at first, but fuzzy logic uses degrees of truth as a mathematical model of vagueness, while probability is a mathematical model of ignorance.

	Take, for example, the concepts of "empty" and "full". The meaning of each of them can be represented by a certain fuzzy set. The concept of emptiness would be subjective and thus would depend on the observer or designer. A $100$ [ml] glass containing $30$ [ml] of water may be defined as being $0.7$ empty and $0.3$ full, but another designer might, equally well, design a set membership function where the glass would be considered full for all values down to $50$ [ml].
	
	Most of the problems encountered are certainly mathematically modelizable. But these models often require overly restrictive assumptions, making application to the real world tricky. The problems of this world must take into account imprecise, uncertain information. Let us take the example of air conditioning: if we want to obtain a cool temperature, we can ask ourselves what temperature range will be appropriate (the demand is imprecise); Furthermore the reliability of the sensors comes into play (the measurement of the ambient temperature is uncertain). We see the difficulty of interpreting the linguistic variables as fresh, hot, ... and the processing of these uncertainties.
	
	An approach was developed from 1965 by Loft. A. Zadeh, a professor at the University of California at Berkeley, based on the theory of fuzzy sets, generalizing the theory of classical sets. In the new Zadeh theory, an element can more or less belong to a certain set. Inaccuracies and uncertainties can thus be modeled, and reasonings acquire a flexibility that is not allowed by classical logic: "\NewTerm{fuzzy logic}\index{fuzzy logic}" was born. Many applications have developed in various domains, where no deterministic model exists or is practically implementable, as well as in situations where data imprecision makes control by conventional methods impossible.

	In the following, we will first develop the basics of the theory of fuzzy subsets, then we will clarify the reasoning in fuzzy logic, we will examine the methods of exploitation of the results obtained, and finally we will see an effective application (and if possible an application with MATLAB™ and/or R).

	Before turning to the formal side of the thing (mathematically speaking), it may be preferable (since it is still a technique of the engineer mainly) to briefly present the concepts of fuzzy logic in a pictorial way.
	
	\begin{tcolorbox}[title=Remark,colframe=black,arc=10pt]
	The fuzzy logic is a technique used for real in fields as varied as automatism (ABS brakes), robotics (pattern recognition), road traffic management (red lights), air traffic control, environment (meteorology, climatology, seismology), medicine (diagnostic aid), psychology, data mining, Machine Learning, and many others.
	\end{tcolorbox}
	Consider, for example, the speed of a vehicle on a national highway. The normal speed is $90\;[\text{km}\cdot\text{h}^{-1}]$. A speed can be considered as high above  $100\;[\text{km}\cdot\text{h}^{-1}]$, and as low below $80\;[\text{km}\cdot\text{h}^{-1}]$. Boolean logic would look at a thing like this:
	\begin{figure}[H]
		\centering
		\includegraphics[scale=1]{img/computing/fuzzy_logic_car_speed_example_01.jpg}
	\end{figure}
	We see above that therefore the speed is considered $100\%$ as high starting from $100\;[\text{km}\cdot\text{h}^{-1}]$, and $0\%$ below.

	Fuzzy logic, on the other hand, allows degrees of verification of the condition "Is speed high?" according to:
	\begin{figure}[H]
		\centering
		\includegraphics[scale=1]{img/computing/fuzzy_logic_car_speed_example_02.jpg}
	\end{figure}
	The situation is better here as the speed is considered as not high at all  below $80\;[\text{km}\cdot\text{h}^{-1}]$. We can therefore say that below $80\;[\text{km}\cdot\text{h}^{-1}]$, the speed is high at $0\%$. The speed is considered to be high above $100\;[\text{km}\cdot\text{h}^{-1}]$. The speed is therefore high at $100\%$ above $100\;[\text{km}\cdot\text{h}^{-1}]$. The speed is thus high at $50\%$ when at $90\;[\text{km}\cdot\text{h}^{-1}]$ and high at $25\%$ at $85\;[\text{km}\cdot\text{h}^{-1}]$.

	Similarly, the function "Is the speed low?" Will be addressed typically as following by most humans:
	\begin{figure}[H]
		\centering
		\includegraphics[scale=1]{img/computing/fuzzy_logic_car_speed_example_03.jpg}
	\end{figure}
	Therefore following the above figure the speed is considered low under $80\;[\text{km}\cdot\text{h}^{-1}]$. It is therefore $100\%$ low. The speed is considered not at all low above $100\;[\text{km}\cdot\text{h}^{-1}]$. It is therefore $0\%$ low. The speed is thus a $50\%$ (bit low) when at $90\;[\text{km}\cdot\text{h}^{-1}]$ and at $75\%$ (quite low) when at $85\;[\text{km}\cdot\text{h}^{-1}]$.
	
	We can also define a function "Is the speed average?" by:
	\begin{figure}[H]
		\centering
		\includegraphics[scale=1]{img/computing/fuzzy_logic_car_speed_example_04.jpg}
	\end{figure}
	Once the input value evaluated ("Is the speed high?"), a value can be determined for an output function. Consider the function "If the fever is strong, then administer aspirin". Such a function is named "\NewTerm{fuzzy control}". It is composed of two parts:
	\begin{enumerate}
		\item An input: "Is the fever strong?". We consider that a fever is not strong below $38^\circ$ [C], and that it is high above $40^\circ$ [C].
		
		\item An output: "Administer a given number of aspirin tablets!"
	\end{enumerate}
	These two parts are related. We can represent them together as below:
	\begin{figure}[H]
		\centering
		\includegraphics[scale=1]{img/computing/fuzzy_logic_car_speed_example_05.jpg}
	\end{figure}
	There are several empirical techniques for determining the output value (in the example: the number of aspirin tablets to be administered):

	An example consists in taking the horizontal passing through the corresponding ordinate point on the starting curve at the abscissa of the value of the input and of looking at where this horizontal section intersects the output curve. The abscissa of this point of intersection is a possible output value as shown below:
	\begin{figure}[H]
		\centering
		\includegraphics[scale=1]{img/computing/fuzzy_logic_car_speed_example_06.jpg}
	\end{figure}
	A second empirical possible choice consists in taking as output value of the center of gravity of the gray trapezoid delimited by the horizontal and the output curve as shown in the figure below:
	\begin{figure}[H]
		\centering
		\includegraphics[scale=1]{img/computing/fuzzy_logic_car_speed_example_07.jpg}
	\end{figure}
	From these two non-software oriented examples, we see that we are at the frontier of pure science and engineering frontier since there is a technical and/or statistical choice to be made in the method to be chosen.
	
	Let us now a computer aided example as engineers in practice use such stool to speed up their research and development! The case below in inspired by a textbook and many softwares take it now as basis example named the "Basic Tipping Problem". 
	\begin{tcolorbox}[colframe=black,colback=white,sharp corners]
	\textbf{{\Large \ding{45}}Example:}\\\\
	Given a number between $0$ and $10$ that represents the quality of service and food at a restaurant (where $10$ is excellent), what should the tip be? \\
	
	This problem is based on tipping as it is typically practiced in the United States. An average tip for a meal in the U.S. is $15\%$, though the actual amount may vary depending on the quality of the service provided. But because service and food is rated on a scale of $0$ to $10$, we might have the tip go linearly from $5\%$ if the service is bad to $25\%$ if the service is excellent.\\
	
	If we denote $S$ for the service and $F$ for food quality, we will chose the tip as being given by:
	
	Here is a copy/paste of the MATLAB™ 2013a script for the original url\footnote{See \url{https://ch.mathworks.com/help/fuzzy/an-introductory-example-fuzzy-versus-nonfuzzy-logic.html}}:
	\begin{figure}[H]
		\centering
		\includegraphics[scale=1]{img/computing/fuzzy_logic_tip_matlab_01_script.jpg}
	\end{figure}
	
	\end{tcolorbox}
	
	\begin{tcolorbox}[colframe=black,colback=white,sharp corners]
	In this case, the results look satisfactory, but when you look at them closely, they do not seem quite right:
	\begin{figure}[H]
		\centering
		\includegraphics[scale=0.9]{img/computing/fuzzy_logic_tip_matlab_01_plot.jpg}
	\end{figure}
	Suppose you want the service to be a more important factor than the food quality. Specify that service accounts for $80\%$ of the overall tipping grade and the food makes up the other $20\%$. Try this equation:
	
	Thus in MATLAB™ 2013a:
	\begin{figure}[H]
		\centering
		\includegraphics[scale=1]{img/computing/fuzzy_logic_tip_matlab_02_script.jpg}
	\end{figure}
	The response is still some how too uniformly linear. Suppose you want more of a flat response in the middle, i.e., you want to give a $15\%$ tip in general, but want to also specify a variation if the service is exceptionally good or bad. This factor, in turn, means that the previous linear mappings no longer apply. You can still use the linear calculation with a piecewise linear construction. Now, return to the one-dimensional problem of just considering the service. You can create a simple conditional tip assignment using logical indexing:
	\end{tcolorbox}
	
	\begin{tcolorbox}[colframe=black,colback=white,sharp corners]
	\begin{figure}[H]
		\centering
		\includegraphics[scale=1]{img/computing/fuzzy_logic_tip_matlab_03_script.jpg}
	\end{figure}
	that gives the following plot:
	\begin{figure}[H]
		\centering
		\includegraphics[scale=0.8]{img/computing/fuzzy_logic_tip_matlab_02_plot.jpg}
	\end{figure}
	Suppose you extend this to two dimensions, where we take food into account again:
	\begin{figure}[H]
		\centering
		\includegraphics[scale=1]{img/computing/fuzzy_logic_tip_matlab_04_script.jpg}
	\end{figure}
	that gives the following plot:
	\end{tcolorbox}
	
	\begin{tcolorbox}[colframe=black,colback=white,sharp corners]
	
	\begin{figure}[H]
		\centering
		\includegraphics[scale=1]{img/computing/fuzzy_logic_tip_matlab_03_plot.jpg}
	\end{figure}
	The plot looks good, but the function is surprisingly complicated. It was a little difficult to code this correctly, and it is definitely not easy to modify this code in the future. Moreover, it is even less apparent how the algorithm works to someone who did not see the original design process.
	\end{tcolorbox}
	In practice we don't use a fuzzy function like the one above as there are sharp angles (imagine a car having speed settings wit abrupt changes as above...).
	
	Therefore rather than using linear functions a input we can choose among a class of continuous and smooth functions. For example the "\NewTerm{Gaussian curve membership function}" defined by:
	
	\begin{figure}[H]
		\centering
		\includegraphics[scale=0.5]{img/computing/fuzzy_gaussmf.jpg}
	\end{figure}
	or the "\NewTerm{Trapezoidal-shaped membership function}" defined by:
	
	The parameters $a$ and $d$ locate the "feet" of the trapezoid and the parameters $b$ and $c$ locate the "shoulders."
	\begin{figure}[H]
		\centering
		\includegraphics[scale=0.5]{img/computing/fuzzy_trapmf.jpg}
	\end{figure}
	Or also the "\NewTerm{Triangular-shaped membership function}" defined by:
	
	The parameters $a$ and $c$ locate the "feet" of the triangle and the parameter $b$ locates the peak:
	\begin{figure}[H]
		\centering
		\includegraphics[scale=0.5]{img/computing/fuzzy_trimf.jpg}
	\end{figure}
	Let us see with \texttt{R} the tipper problem managed with the fuzzy functions above (for more details see the \texttt{R} companion book!).
	
	\begin{tcolorbox}[colframe=black,colback=white,sharp corners]
	Here is a copy/paste of the \texttt{R} script  given in \cite{wagner2011fuzzy} with detailed explanations.
	\begin{figure}[H]
		\centering
		\includegraphics[scale=0.8]{img/computing/fuzzy_tipping_problem_r.jpg}
	\end{figure}
	That gives:
	\begin{figure}[H]
		\centering
		\includegraphics[scale=0.5]{img/computing/fuzzy_tipping_problem_r_plot.jpg}
	\end{figure}
	\end{tcolorbox}
	
	\subsubsection{Fuzzy set}
	\textbf{Definition (\#\mydef):} Given $X$ a set. A "\NewTerm{fuzzy subset}" $A$ of $X$ is defined by a belonging function $f_A$ on $X$ with values in the interval $[0,1]$.
	\begin{tcolorbox}[title=Remark,colframe=black,arc=10pt]
	The belonging function $f_A$ can be set arbitrarily. A practical application problem is for the engineer to define these functions (we usually use statistical data or the opinion of an expert to make the least worst choice...).
	\end{tcolorbox}
	The notion of fuzzy subset encompasses that of classical subset for which $f_A$ is the indicator function given for recall by:
	
	\textbf{Definition (\#\mydef):} If $A$ and $B$ are two sets, such that $A$ is included in $B$ (ie $A\subset B$), we name "\NewTerm{indicator function}\index{indicator function}" of $A$ (relatively to $B$), the function $1_A$ defined in $\{0,1\}$, and such that:
	
	This is for a classical set. Obviously such indicator functions are often very practical technical intermediaries!
	
	But for fuzzy logic we have scnerio such like this:
	\begin{tcolorbox}[colframe=black,colback=white,sharp corners]
	\textbf{{\Large \ding{45}}Example:}\\\\
	A possible characteristic function to define the fuzzy subset $A$ "to be twenty years" on the set $X$ of the real positive numbers:
	\begin{figure}[H]
		\centering
		\includegraphics[scale=1]{img/computing/fuzzy_logic_centered_linear_fuzzy_function_simple.jpg}
		\caption{Centered linear fuzzy function}
	\end{figure}
	\end{tcolorbox}
	The following concepts are characteristic of $A$ in the field of fuzzy sets:
	
	\pagebreak
	\textbf{Definitions (\#\mydef):}
	\begin{enumerate}
		\item[D1.] Support of $A$:
		

		\item[D2.] Height of $A$:
		

		\item[D3.] A is said to be normalized if $h(A)=1$
		\begin{tcolorbox}[colframe=black,colback=white,sharp corners]
		\textbf{{\Large \ding{45}}Example:}\\\\
		A possible characteristic function to define the fuzzy subset $A$ "to be twenty years" on the set $X$ of the real positive numbers:
		\begin{figure}[H]
			\centering
			\includegraphics[scale=1]{img/computing/fuzzy_logic_centered_linear_fuzzy_function.jpg}
			\caption{Centered linear fuzzy function}
		\end{figure}
		\end{tcolorbox}
		\begin{tcolorbox}[title=Remark,colframe=black,arc=10pt]
		The fuzzy subsets considered will all be assumed normalized, in extenso of height equal to $1$.
		\end{tcolorbox}
		
		\item[D4.] The kernel of $A$:
		
	
		\item[D5.] Cardinality of $A$:
		
		
		\item[D6.] $A$ is "more specific" than $B$ if:
		
		
		\item[D7.] $A$ is "more precise" than $B$ if:
		
		
		\item[D8.] There is equality between two fuzzy subsets if and only if:
		
		\item[D9.] There is equality between two fuzzy subsets if and only if:
		
		
		\item[D10.] There is inclusion between two fuzzy subsets if and only if:
		
		
		\item[D11.] The intersection $A\cap B$ is defined by:
		
		
		\item[D12.] The union $A\cup B$ is defined by:
		
	\end{enumerate}
	\begin{tcolorbox}[colframe=black,colback=white,sharp corners]
	\textbf{{\Large \ding{45}}Example:}\\\\
	Let us return to the case already envisaged. We consider the "twenty-year-old" people and those with "being in age" (ie, those who are in age to vote/drink alcohol) (dotted in the figure: we consider it as a non-fuzzy sub-set!):
	\begin{figure}[H]
		\centering
		\includegraphics[scale=1]{img/computing/fuzzy_logic_example_01.jpg}
	\end{figure}
	According on the definitions of the intersection ("logical AND" or logical multiplication according to the Boolean algebra) and the union ("logical OR" or logical addition according to Boolean algebra), we can characterize the subsets (first figure below), as well as those "being in their twenties or being in age" (second figure below):
	\end{tcolorbox}
	\begin{tcolorbox}[colframe=black,colback=white,sharp corners]
	\begin{figure}[H]
		\centering
		\includegraphics[scale=1]{img/computing/fuzzy_logic_example_02.jpg}
	\end{figure}
	\begin{figure}[H]
		\centering
		\includegraphics[scale=1]{img/computing/fuzzy_logic_example_03.jpg}
	\end{figure}
	\end{tcolorbox}
	
	\begin{flushright}
	\begin{tabular}{l c}
	\circled{80} & \pbox{20cm}{\score{3}{5} \\ {\tiny 10 votes,  66.00\%}} 
	\end{tabular} 
	\end{flushright}


	%to make section start on odd page
	\newpage
	\thispagestyle{empty}
	\mbox{}
	\section{Error-Correcting Codes}
	\lettrine[lines=4]{\color{BrickRed}I}f the first half of the 20th century was that of the analogue revolution by radio and television, the second half of this century is that of the digital revolution and the systematic use of algebra in the data transmission. It is also the emergence automated error handling, where an "\NewTerm{Error}\index{error}" is a condition when the output information does not match with the input information. During transmission, digital signals suffer from noise that can introduce errors in the binary bits travelling from one system to other. \\
	\begin{figure}[H]
		\centering
		\includegraphics[scale=1]{img/computing/error_transmission.jpg}
	\end{figure}
	The "\NewTerm{error correction codes}\index{error correction codes}" (ECC) are used to add redundancy to the data to make it tolerant to the transmission errors (at least to a given degree). Basically, the idea is to encode one way or another an information sequence and add encoding the original data as a control data integrity. So even if some of the information is corrupted, but not too much... redundancy will identify the incorrect parts of the message.
	
	\begin{tcolorbox}[title=Remark,colframe=black,arc=10pt]
	ECC are not only used for data transmission but also to give unique identifiers to bills in some countries and also in some factories to build a nomenclature system (naming) of pieces and to control afterwards the input of employees in computer systems! For example, bar-codes and QR-codes contains ECC.
	\end{tcolorbox}
	
	Thus, after the appearance of audio CD in the early 1980s, we must take into account the development of broadcasting satellites  and new means of communication such as the fax, the Minitel, Internet or digital phone using any correcting codes errors (CCE). Even photography, radio and books are becoming digital (and will probably only be in this format in a hundred years).
	
	Image or sound reproduction techniques are related to the transmission and to the correct reading of many digital messages, also known as "words". A message consists of words themselves made up of symbols (a particular example being the "bit" which for reminder is contraction of "BInary digiT"), taken from an alphabet. If the alphabet is binary so each symbol will be a bit.
	
	Let us take the message $00101$ formed of $5$ bits each worth $0$ or $1$. If we send the message as is, a transmission error or of reading can take place and make the message unintelligible (or the bill/piece unique identifier). Let us decide to repeat the message three times and send it:
	
	If the received message contains an error, this error can be corrected. If there are two errors, the receiver is able to detect that there was a mistake but can not always recover the original message. Finally, if it occurs more than two errors during transmission, the receiver can not detect them.
	
	We have seen just now a first example of ECC, named "\NewTerm{repeatedly code}\index{repeatedly code}". This code, which corrects errors and detects two, was used in some Audio CD player having three heads. The signal $0$ or $1$ is read independently by each of these three heads to give a word of three digits, and a reading error can be sometimes corrected.
	
	Note that it is natural to extend a message to protect it. Let us consider the words of a language. They are usually very far from each other, two words differs according to their lengths and in the letters and syllables used. Thus it will be difficult to confound the words "library" and "cabinet" although these words are mispronounced or misheard and we will naturally reconstitute the message in a conversation even when some letters would be deleted or distorted. The military meanwhile spell some information by saying "Alpha Zulu" for "AZ" to avoid errors...
	
	A second example widely used in computing science of error detection is the addition of a "\NewTerm{parity bit}\index{parity bit}". Let $00101$ be the original message and add it a last bit obtained by adding five bits of departure modulo $2$. The message is $001010$ and we can therefore detect errors but can not correct it.... For this, we make the sum of all the bits to obtain $0$ if there is no error, and $1$ otherwise. This code named "\NewTerm{parity code}\index{parity code}" is used everywhere: in the social security numbers where we add a key, in those of bank accounts or in bar-code of supermarkets where it is the 13th digit that is the control key (the space probe Voyager II is one of the many users of parity codes to communicate almost in a reliable way and also the $8$th bit in the ASCII system which is used as a parity bit).
	
	For many years, the DRAM sockets managed words with one bit only; it was then necessary to put the $8$ memory card sockets for working with bytes ($8$ bits). But at this time many cards included not $8$ but $9$ sockets! The ninth socket was designed to store a parity bit at each start of a memory byte. When reading a byte, we checked, if between the time of writing and that of reading, parity had not been changed (due to a parasite, for example).
	\begin{figure}[H]
		\centering
		\includegraphics[scale=0.8]{img/computing/ecc_ram.jpg}
		\caption{ECC RAM vs Non-ECC RAM}
	\end{figure}
	\begin{tcolorbox}[title=Remark,colframe=black,arc=10pt]
	We would like to point out that ECC memory is considered server-grade. They are sold to a market which pays much closer attention to reliability than the consumer market. As a result, ECC RAM is usually subjected to much stricter testing and validation before shipping. This is a big factor in why ECC modules see much lower failure rates.
	\end{tcolorbox}
	Finally, let us see a third example used on some computer servers that use  in RAID4 or RAID6 parallel disks, this latter using Hamming codes that we will see later.
	
	Suppose we have $3$ hard drives, and the content of the first byte of each disc is the following:
	
	So it is sufficient take each column and count the number $p$ of $1$ in the column. The value of the parity bit is then $p$ modulo $2$. We have for the first column in the example above $p$ which is $2$. Therefore, the parity bit is equal to $0$, etc. Then we have on the control disc (CD):
	
	These three examples are fundamental to the basic coding theory and show that we can control the appearance of error by deliberately lengthening the message before transmission or reading. More sophisticated algebraic techniques are then used to improve the performance of coding, thanks to:
	
	\begin{enumerate}
		\item Know if any errors occurred (detection problems)
		
		\item find the initial correct message from the message received (correction of problem)
		
		\item correct the most  possible mistakes while using the least possible additional bits (the problem of performance encoding)
	\end{enumerate}
	From the mathematical point of view, one of the interests of coding theory is to show that algebra applies once again well in our everyday life when we listen to music, or we settle in front of our televisions, and that such abstract notions as those of vector spaces or polynomials over finite fields allow us to read messages, listen to music or watch movies in optimum conditions!
	
	We distinguish the following two classes of ECC: 
	\begin{enumerate}
		\item Block codes
		\item Treillis codes
	\end{enumerate}
	The figure below provides a simple overview of the error correcting codes family\footnote{For example "Quantum Error Correction" is not indicated}. In the first class (right in the figure), we have the most popular codes such as BCH codes, Reed-Solomon and Goppa, Golay and Hamming. The second class (left in the figure) is less rich in variety but has a lot more flexibility, especially in the choice of parameters and decoding algorithms available. Let us cite for example the binary systematic recursive convolutional codes widely used in coded modulation and the parallel concatenated codes (Turbo Codes).
	\begin{figure}[H]
		\centering
		\includegraphics{img/computing/error_correcting_codes.jpg}
		\caption{Non-exhaustive orgchart of correcting codes}
	\end{figure}
	\begin{tcolorbox}[title=Remark,colframe=black,arc=10pt]
	To introduce the foundations of the theory of error correcting codes, we recommend strongly the reader to have read through first the section of Set Theory, after this of  Statistical Mechanics (where the information theory is) after of Numerical Systems and finally of Topology.
	\end{tcolorbox}
	
	When possible we will show the reader how to put quickly in practice ECC using native MATLAB™ Communication Toolbox functions.
	
	\subsection{CheckSum}\label{checksum}
	Before starting the part of pure mathematics, we would make a small introduction to the "checksum" (control sum) that is a tool frequently used in the business when exchanging files over several Giga Bytes between two computers or when downloading on the Internet.
	
	The checksum, also sometimes named "fingerprint" is a basic concept of coding theory used for correcting codes. It corresponds to a particular case of "redundancy control code". It is widely used in computer and digital telecommunications as already said.
	
	One of basic technique (among a dozens more or less sophisticated) is to take the sum of a given length of bits (byte, word, or other ...) and calculate the modulo 255 (FF in hexadecimal). For example, if we take two words and we rely on their hexadecimal ASCII code (you can find ASCII tables almost everywhere on the internet):
	\begin{figure}[H]
		\centering
		\includegraphics[scale=0.9]{img/computing/checksum.jpg}
		\caption[Basic Checksum principle]{Basic Checksum principle (source: Wikipedia)}
	\end{figure}
	Some tools also use the MD5 algorithm (Message Digest 5) or SHA (Secure Hash Algorithm) to have an checksum of a message (\SeeChapter{see section Cryptography page \pageref{md5} and page \pageref{sha 1}}).
	
	\subsubsection{Luhn algorithm}
	The Luhn algorithm or Luhn formula, also known as the "modulus 10" or "mod 10" algorithm, is a simple checksum formula used to validate a variety of identification numbers, such as Credit card numbers, IMEI numbers, National Provider Identifier numbers in the US, and Canadian Social Insurance Numbers (see examples further below). It was created by IBM scientist Hans Peter Luhn and described in U.S. Patent No. 2,950,048, filed on January 6, 1954, and granted on August 23, 1960.

	The algorithm is in the public domain and is in wide use today. It is specified in ISO/IEC 7812-1 and is completely described in the following figure:
	\begin{figure}[H]
		\centering
		\includegraphics[scale=0.8]{img/computing/luhn_algorithm.jpg}
		\caption{Luhn algorithm for checksum principle}
	\end{figure}
	The Luhn algorithm will detect any single-digit error, as well as almost all transpositions of adjacent digits. It will not, however, detect transposition of the two-digit sequence $09$ to $90$ (or vice versa). It will detect $7$ of the $10$ possible twin errors (it will not detect $22 \leftrightarrow 55$, $33 \leftrightarrow 66$ or $44 \leftrightarrow 77$).
	
	A picture is worth a thousand words let us see how this apply to VISA cards:
	\begin{figure}[H]
		\centering
		\includegraphics[scale=0.6]{img/computing/luhn_algorithm_visa.jpg}
	\end{figure}

	Other, more complex check-digit algorithms (such as the Verhoeff algorithm and the Damm algorithm) can detect more transcription errors.
	
	\pagebreak
	\subsection{Check Digit}
	Also before starting the part of pure mathematics, we would make a small introduction to some common check digits (in fact any professionally driven government or company should have all objects or individuals identifiers with at least one check digit). We will only focus on example that have been asked to me by my students or people that contacted me on the Internet (otherwise I can dedicate a whole book only for examples on check digits...)!
	
	\textbf{Definition (\#\mydef):} A "\NewTerm{check digit}\index{check digit}" is a form of redundancy check used for error detection on identification numbers, such as bank account numbers, which are used in an application where they will at least sometimes be input manually. It is analogous to a binary parity bit used to check for errors in computer-generated data. It consists of one or more digits computed by an algorithm from the other digits (or letters) in the sequence input.
	
	\subsubsection{European Article Numbering (EAN-13)}
	Let us see first the check digit of EAN-13 bar codes (Cyclic Redundancy Check\footnote{type of checksum, specifically a position dependent checksum algorithm} (CRC) ECC type):
	\begin{figure}[H]
		\centering
		\includegraphics[scale=0.6]{img/computing/ean_13_barcod.jpg}
		\caption{EAN-13 bar-code example}
	\end{figure}
	There are various different errors that can occur when numbers are written, printed or transferred in any manner. Different methods of assigning check digits are better at detecting certain kinds of errors than others. The most common types of errors that occur in practice and their frequencies, according to one study, are as follows:
	\begin{table}[H]
	\begin{center}
		\definecolor{gris}{gray}{0.85}
			\begin{tabular}{|l|c|l|}
				\hline
				\multicolumn{1}{c}{\cellcolor{black!30}\textbf{Error type}} & 
  \multicolumn{1}{c}{\cellcolor{black!30}\textbf{Form}} & 
  \multicolumn{1}{c}{\cellcolor{black!30}\textbf{Relative frequency}} \\ \hline
				 Single error & $a$ replaced by $b$ & $79.1\%$ \\ \hline
				 Transposition of adjacent digits & $ab$ replaced by $ba$ & $10.2\%$ \\ \hline
				 Jump transposition & $abc$ replaced by $cba$ & $0.8\%$ \\ \hline
				 Twin error & $aa$ replaced by $bb$ & $0.5\%$\\ \hline
				 Phonetic error & $a0$ swapped with $1a$, $a = 2,\ldots,9$ & $0.5\%$ \\ \hline
				 Jump twin error & $aca$ replaced by $bcb$ & $0.3\%$ \\ \hline
		\end{tabular}
	\end{center}
	\caption[]{Various costs}
	\end{table}
	Another common type of error not mentioned here is accidental insertion or deletion of characters. In the cases we will consider, the number will have a fixed length, so insertions and deletions will be automatically detected.
	
	The EAN-13 format uses a modulus $10$ scheme, with check digit ($a_ c$) defined by
	
	For example, if we start with the number $1234567$ in the EAN-8 scheme, then our check digit is:
	
	which makes the full bar code number $12345670$.
	
	When dealing we ECC must worry about how effective a scheme is in detecting errors!
	
	Let us consider only two cases:
	\begin{enumerate}
		\item Single error detection rate:

		If a digit $d$ whose weight is $1$ is changed to $c$, the weighted sum will change by $d-c$. The error will go undetected only if $d-c=0 \mod 10$. But this happens only when $d=c$, in which case there has not been an error after all, so all errors of this kind are caught.

		What if the weight were $3$? Then the error would be undetected if $3(d-c)=0 \mod 10$. But again, this cannot happen unless $d=c$. Thus, this method has a $100\%$ single error detection rate (SEDR).

		\item Transposition of adjacent digits detection rate:

		Suppose two adjacent digits, $cd$, are transposed to $dc$. If $c$’s weight is $3$ (hence $d$’s weight is $1$), the weighted sum is changed by:
		
		which will be detected unless $2(c-d)=0 \mod 10$, which can happen only if $c$ and $d$ differ by $5$. The same would have applied if $c$ had been weighted by $1$ and $d$ by $3$.

		As a result, the transpositions that will go undetected must involve $0 \leftrightarrow 5, 1\leftrightarrow 6, 2 \leftrightarrow 7, 3 \leftrightarrow 8$ and $4 \leftrightarrow 9$. So, $10$ transpositions are undetectable.

		There are 100 possibilities for each pairing, and the transposition of 90 of these would result in an error. Therefore the detection rate is $ 80/90 = 88.9\% $ transposition error detection rate (TEDR).
	\end{enumerate}
	
	\subsubsection{Swiss Post Payment slip}
	Another famous example for my country is the Swiss Post payment slip where numerous numbers have a check digit.
	
	Here is a sample preview of such a payment slip:
	\begin{figure}[H]
		\centering
		\includegraphics[scale=0.5]{img/computing/swiss_payment_slip.jpg}
		\caption{Swiss payment slip}
	\end{figure}
	and here technical description of the bottom right number with in red the check digit on which we will focus here:
	\begin{figure}[H]
		\centering
		\includegraphics[scale=0.6]{img/computing/swiss_payment_slip_check_digit.jpg}
		\caption{Swiss payment slip check digit for Reference Number}
	\end{figure}
	All the digit check in this Swiss payment slip are based on a recursive modulo $10$ computation.
	
	Let us give a small example on how to calculate the check digit based on the Swiss method. For this let us consider first the following matrix:
	\begin{figure}[H]
		\centering
		\includegraphics[scale=0.55]{img/computing/swiss_payment_slip_matrix.jpg}
	\end{figure}
	So what would be the check digit of the number 70004152 (it's a typical 8 digit Swiss Post account)?
	
	The process to determine that latter is simply the following:
	\begin{figure}[H]
		\centering
		\includegraphics[scale=0.74]{img/computing/swiss_payment_slip_check_digit_calculation_procedure.jpg}
	\end{figure}
	
	\subsubsection{International Bank Account Number (IBAN)}
	The International Bank Account Number (IBAN) is an internationally agreed system of identifying bank accounts across national borders to facilitate the communication and processing of cross border transactions with a reduced risk of transcription errors. It was originally adopted by the European Committee for Banking Standards (ECBS), and later as an international standard under ISO 13616.

	The IBAN consists of up to 34 alphanumeric characters comprising: a country code; two check digits; and a number that includes the domestic bank account number, branch identifier, and potential routing information. The check digits enable a sanity check of the bank account number to confirm its integrity before submitting a transaction.

	Before IBAN errors of transcription were not detectable and it was not possible for a sending bank to validate the routing information prior to submitting the payment. Routing errors caused delayed payments and incurred extra costs to the sending and receiving banks and often to intermediate routing banks.

	The IBAN should not contain spaces when transmitted electronically. When printed it is expressed in groups of four characters separated by a single space, the last group being of variable length. Here is an example of how typically Switzerland write IBAN numbers:
	\begin{center}
		\texttt{CH93 0076 2011 6238 5295 7}
	\end{center}
	Permitted IBAN characters are the digits $0$ to $9$ and the $26$ upper-case Latin alphabetic characters $A$ to $Z$. This applies even in countries (e.g., Thailand) where these characters are not used in the national language.
	
	An IBAN is validated by converting it into an integer and performing a basic modulo $97$ operation (as described in ISO 7064: \textit{Security techniques -Check character systems}) on it. If the IBAN is valid, the remainder equals $1$.

	The procedure is a follows:
	\begin{itemize}
		\item Check that the total IBAN length is correct as per the country. If not, the IBAN is invalid;

		\item Move the four initial characters to the end of the string;

		\item Replace each letter in the string with two digits, thereby expanding the string, where A $= 10$, B $= 11$, ..., Z $= 35$;
	
		\item Interpret the string as a decimal integer and compute the remainder of that number on division by $97$.
	\end{itemize}
	If the remainder is $1$, the check digit test is passed and the IBAN might be valid.
	\begin{tcolorbox}[colframe=black,colback=white,sharp corners]
	\textbf{{\Large \ding{45}}Example:}\\\\
	Given a fictitious United Kingdom bank, sort code \texttt{12-34-56}, account number \texttt{98765432}. For the sanity check we follow the above procedure:
	\begin{itemize}
		\item The IBAN is therefore:
		\begin{center}
			\texttt{GB82 WEST12345698765432}
		\end{center}

		\item We rearrange:
		\begin{center}
			\texttt{WEST12345698765432GB82}
		\end{center}

		\item We convert to integer:
		\begin{center}
			\texttt{3214282912345698765432161182}
		\end{center}

		\item We compute the modulo $97$:
		\begin{center}
			\texttt{3214282912345698765432161182} $\mod 97=1$
		\end{center}
	\end{itemize}
	\end{tcolorbox}
	
	\subsubsection{UIC wagon numbers}
	Wagon numbers (or coach numbers) are key data for railway operations. They enable a railway wagon or coach to be positively identified and form a common language between railway operators, infrastructure companies and the state authorities. The system of wagon numbering has been laid down by the International Union of Railways (Union internationale des chemins de fer or UIC) and is similar to that used for the locomotives and multiple units.
	\begin{figure}[H]
		\centering
		\begin{subfigure}{0.4\textwidth}
			\includegraphics[width=\textwidth]{img/computing/uic_wagon_slovakia.jpg}
			\caption{Slovak UIC wagon number}
		\end{subfigure}
		\begin{subfigure}{0.4\textwidth}
			\includegraphics[width=\textwidth]{img/computing/uic_wagon_switzerland.jpg}
			\caption{Swiss UIC wagon number}
		\end{subfigure}				
	\end{figure}
	The complete wagon number comprises 12 digits. The individual digits have the following meaning:
	\begin{itemize}
		\item Digit 1-2: Type of vehicle and indication of the interoperability capacity

		\item Digit 3-4: Country Code (Switzerland (CH) = $85$)

		\item Digit 5-8: Vehicle type information

		\item Digit 9-11: Individual running number (serial number)
		
		\item Digit 12: Self-check digit
	\end{itemize}
	The digits are multiplied individually from right to left alternately by $2$ and $1$, and digit summed. The difference between this sum and the next multiple of ten is the check digit, placed after the eleventh digit, separated by a dash.
	\begin{tcolorbox}[colframe=black,colback=white,sharp corners]
	\textbf{{\Large \ding{45}}Example:}\\\\
	Given a fictitious Wagon UIC \texttt{21-81-24 7121 7}. For the sanity check we follow the above procedure:
	\begin{itemize}
		\item The UIC is therefore:
		\begin{center}
			 \texttt{2 1 8 1 2 4 7 1 2 1 7}
		\end{center}

		\item We multiply:
		\begin{center}
			 \texttt{2} $\cdot 2\quad$ \texttt{1}$\cdot 1\quad$ \texttt{8}$\cdot 2\quad$ \texttt{1}$\cdot 1\quad$ \texttt{2}$\cdot 2\quad$ \texttt{4}$\cdot 1\quad$ \texttt{7}$\cdot 2\quad$ \texttt{1}$\cdot 1\quad$ \texttt{2}$\cdot 2\quad$ \texttt{1}$\cdot 1\quad$ \texttt{7}$\cdot 2$
		\end{center}
		It gives:
		\begin{center}
			$4 \; 1\; 16 \; 1\; 4\; 4\; 14\; 1\; 4\; 1\; 14$
		\end{center}

		\item We sum up the digits using:
		
		
		\item We take the next multiple of $10$ that $40$ so the check digit is equal to three $3$.
		
		\item Finally we get:
		\begin{center}
			\texttt{21-81-24 7121 7-3}
		\end{center}
	\end{itemize}
	\end{tcolorbox}
	\begin{tcolorbox}[title=Remark,colframe=black,arc=10pt]
	The swiss "Unternehmens-Identifikationsnummer" (company identification number) also have a check digit and so on...
	\end{tcolorbox}
	
	
	\pagebreak
	\subsection{Permutations}
	Also before starting the part of pure mathematics, we would make a small introduction to the check digit using permutations in VISA card numbers.
	
	Credit cards use an error-detecting scheme that was developed by IBM. It uses the permutation:
	
	In other words (for more details see the section of Set Algebra page \pageref{set algebra}) $\sigma (0)=0$, $\sigma (1)=2$, $\sigma (2)=4$, etc.
	
	Notice also that:
	

	In a $16$ digit credit card number, the final digit is the check digit. Let the credit card number be $(a_1, a_2,\ldots , a_{15}, a_{16})$, with $a_{16}$ being the check digit. Then
	
	Note that in this example the permutation was applied to $a_ i$ where $i$ is odd ($a_1, a_3$, etc), because there is an odd number of digits excluding the check digit. Had this scheme been used on a number with an even number of digits excluding the check digit, the permutation would have been applied to $a_ i$ where $i$ was even.
	
	As we know form the previous example, dealing we ECC must worry about how effective a scheme is in detecting errors!
	
	Let us consider again only two cases:
	\begin{enumerate}
		\item Single error detection rate:

		This scheme catch all single-digit errors ($100\%$ SEDR). Indeed, for example if digit $a_ i$ is changed from $c$ to $d$, and $i$ is even, the remainder will change by $c-d$, which is non-zero (and is, of course, smaller than the modulus $N=10$). If $i$ is odd, it will change by $\sigma (c)-\sigma (d)$. This is again non-zero: $\sigma (c)$ cannot be equal to $\sigma (d)$ if $\sigma $ is a permutation.

		\item Transposition of adjacent digits detection rate:
		
		If two adjacent digits $c$ and $d$ are transposed, one of them must have the permutation applied - say $c$. The remainder will be unchanged only if $\sigma (c)+d = \sigma (d)+c$. Since $\sigma (x) = 2x \mod 9$, this happens only when $c = d \mod 9$, that is, only when $c$ and $d$ are $0$ and $9$ (in either order).
		
		Therefore, for each pair of adjacent digits, of the $90$ possible transposition errors, two will be undetectable. So the detection rate for transpositions is $88/90 = 97.8\% $ (TEDR).
	\end{enumerate}
	
	\subsection{Encoders}
	Given $Q$ a finite set of $q$ elements (bits, alphabets). Given $k$ and $n$ two  non-zero integers with $k\leq n$. The set of messages will be a part $E$ of $Q^k$ and we introduce a bijective application (at least that is the goal):
	
	named "\NewTerm{encoding application}\index{encoding application}" or "\NewTerm{encoder}\index{encoder}". The message or word is an element  of $E$ that is to say $Q^k$. It is modified to provide the word: 
	
	It is the word $c$ that will be transmitted and read by any system to give a message received $x=(x_1,...,x_n)$ possibly flawed.
	
	Let us now denote $C=f(E)$ the image of $f$. Since $f$ is subjective by definition it is also injective, $f$ realized a bijection of $E$ on $C$ and $C$ can be considered as the set of all possible error coding messages. $C$ is named the "\NewTerm{code of length $n$}\index{lengths of code}", and the elements of $C$ are named the "\NewTerm{words}\index{words}" of the code. The cardinal of the code is by definition that of $C$ that is to say $\text{Card}(C)$. To measure the degree of difference between two words $x$ and $y$ of $Q^n$, we use the "\NewTerm{Hamming distance}\index{Hamming distance}" $d_H$ defined by:
	
	
	\begin{theorem}
	On any set $Q$, we therefore define the application $d:Q^n\times Q^n \rightarrow \mathbb{R}$ by:
	
	If we denote by $\delta_x$ the characteristic function of $x$:
	
	then:
	
	is a distance.
	\end{theorem}
	Let us now prove that following the topological axioms of a distance (\SeeChapter{see section Topology page \pageref{distance}}) that this is really a distance:
	\begin{dem}
	Ok let us prove the five axioms of a distance:
	\begin{enumerate}
		\item We will suppose that for the reader:
		
		is obvious (if not send us a request).
		
		\item We will also suppose that:
		
		is obvious (if not send us a request).
		
		\item We will also suppose that:
		
		is obvious (if not send us a request).
		
		\item We will also suppose that:
		
		where $d_H(x,y)=0$ mean that $x_i=y_i$ for $i=1...n$ and therefore that $x=y$.
		
		\item Finally:
		
		Indeed:
		
		but as:
		
		as $(1-\delta_{x_i}(z_i))$ is equal to $1$ if $x_i\neq z_i$ and $0$ otherwise, then:
		
	\end{enumerate}
	\begin{flushright}
		$\blacksquare$  Q.E.D.
	\end{flushright}
	\end{dem}
	\begin{tcolorbox}[colframe=black,colback=white,sharp corners]
	\textbf{{\Large \ding{45}}Example:}\\\\
	The Hamming distance between the word "\textbf{ramer}" and "\textbf{cases}" or between "\textbf{0100}" and "\textbf{1001}" is equal to $3$.
	\end{tcolorbox}
	
	\begin{tcolorbox}[colback=red!5,borderline={1mm}{2mm}{red!5},arc=0mm,boxrule=0pt]
	\bcbombe Caution!!! Vectors will be denoted without the arrow in respect for tradition for this study field.
	\end{tcolorbox}
	
	The Hamming distance $d_H$ is therefore indeed a metric (\SeeChapter{see section Topology page \pageref{metric}}) as we just proved above and then we name "\NewTerm{Hamming space}\index{Hamming space}" on $Q$ the set $Q^n$ equipped with the metric $d_H$.
		
	\textbf{Definition (\#\mydef):} If $Q$ is a group, the "\NewTerm{Hamming weight}\index{Hamming weight}" $w_H(x)$ of a word $x\in Q^n$ is the number of non-zero components:
	
	where $0$ is the word (vector) of $Q^n$ with all its components equal to the neutral element of $Q$. Furthermore, we have the following trivial property:
	
	\begin{tcolorbox}[title=Remark,colframe=black,arc=10pt]
	When $Q=\{0,1\}$ we will talk about "\NewTerm{binary code}\index{binary code}" (we'll see soon later an another form of writing for this binary set) of dimension $n$ equal to $2$.
	\end{tcolorbox}
	The "\NewTerm{minimum distance}\index{minimum distance}" of the code $C$ is the minimum distance between two distinct words of this code. We denote that integer by $d(C)$ or simply $d$ and therefore:
	
	or using the property of Hamming weight $d_H(x,y)=w(x-y)=w(x)$:
	
	\begin{tcolorbox}[colframe=black,colback=white,sharp corners]
	\textbf{{\Large \ding{45}}Example:}\\\\
	Let us consider the following redundant code denoted $(5, 4)$ for $4$ coded words of length $5$:
	\begin{table}[H]
	\centering
		\begin{tabular}{|c|c|c|c|}
		\hline
		\rowcolor[HTML]{9B9B9B} 
		\textbf{Original Word} & \textbf{Coded Word} & \textbf{Weight} & \textbf{Code Identifier} \\ \hline
		$00$ & $\pmb{00}000$ & $0$ & $C_1$ \\ \hline
		$01$ & $\pmb{01}110$ & $3$ & $C_2$ \\ \hline
		$10$ & $\pmb{10}011$ & $3$ & $C_3$ \\ \hline
		$11$ & $\pmb{11}101$ & $4$ & $C_4$ \\ \hline
		\end{tabular}
	\end{table}
	The Hamming distance of each of the pairs of code are:
	
	The smallest non-zero minimum weight is therefore $3$ and the smallest Hamming distance is $3$.
	\end{tcolorbox}
	
	\pagebreak
	\textbf{Definitions (\#\mydef):}
	\begin{itemize}
		\item[D1.] A code $C$ of length $n$, of cardinal $M$ and of minimum distance $d$ is named a "\NewTerm{$(n, M, d)$ code}". The numbers $n$, $M$, $d$ are the "code parameters". Thus, the code $(7, 4, 3)$ is a code of length seven, that is to say that the receiver receives seven bits, of length four, that is to say that once decoded, the message contains four symbols (letters) and the minimum distance between each codeword is three.

		\item[D2.] We name "\NewTerm{minimum weight}" of a $C$ code the integer:
		
	\end{itemize}
	The parameter $d$ plays an important role because it is closely related to the number of errors that can be corrected. Suppose that the encoded message is $c=(c_1,\ldots,c_n)$ and that there were at least $e$ errors of transmission or reading. The resulting message obtained $x=(x_1,\ldots,x_n)$ satisfies $d(x,c)\le e$. We can fall back on $c$ from $x$ if, and only if, there exists a single code word located at a distance of $x$ less than or equal to $e$ (ie the center-to-center distance between two balls is equal to $2e$). In other words, it is necessary and sufficient that the closed balls of radius $e$ and centered on the elements of the code $C$ are disjoint. A code will correct $e$ errors if this condition is true.

	Therefore, a code $C$ of minimum distance $d$ corrects at most:
	
	where $[]$ represents the integer part of a real number.

	Indeed, if a message of the code is at $d/2$ we will not be able to know to which message of the code (center of ball) it belongs since being (in a pictorial way) at the tangent of two balls. This is the reason why we will take $\dfrac{d-1}{2}$ which is then the "\NewTerm{safe distance}\index{safe distance}" to correct as much as possible an erroneous message of the code. Moreover, since the number of errors is an integer, it comes the previous notation with the square brackets.
	
	It is clear that the code can detect at most $d-1$ errors. Indeed, however how to distinguish an incorrect code from a correct code (code = coded message)? Apart from the fact that each element of the code is different (injective application of the set of messages in the set of encoded messages), it is also necessary to be able to differentiate among them those which are erroneous codes from those which do not Are not. Hence the $d-1$!!!
	\begin{tcolorbox}[colframe=black,colback=white,sharp corners]
	\textbf{{\Large \ding{45}}Example:}\\\\
	Let us consider the following redundant code denoted $(5, 4)$ for $4$ coded words of length $5$ whose minimal distance was therefore $d=3$:
	\begin{table}[H]
	\centering
		\begin{tabular}{|c|c|c|c|}
		\hline
		\rowcolor[HTML]{9B9B9B} 
		\textbf{Original Word} & \textbf{Coded Word} & \textbf{Weight} & \textbf{Code Identifier} \\ \hline
		$00$ & $\pmb{00}000$ & $0$ & $C_1$ \\ \hline
		$01$ & $\pmb{01}110$ & $3$ & $C_2$ \\ \hline
		$10$ & $\pmb{10}011$ & $3$ & $C_3$ \\ \hline
		$11$ & $\pmb{11}101$ & $4$ & $C_4$ \\ \hline
		\end{tabular}
	\end{table}
	This code therefore allows and make it possible to detect at most:
	
	errors and to correct at most:
	
	of them.
	\end{tcolorbox}
	
	\subsubsection{Block code}
	In coding theory, a block code is any member of the large and important family of error-correcting codes that encode data in blocks. There is a vast number of examples for block codes, many of which have a wide range of practical applications. Block codes are conceptually useful because they allow coding theorists, mathematicians, and computer scientists to study the limitations of all block codes in a unified way. Such limitations often take the form of bounds that relate different parameters of the block code to each other, such as its rate and its ability to detect and correct errors.

	\textbf{Definition (\#\mydef):} A "\NewTerm{block code}\index{block code}" of size $M$ and length $n$ defined on an alphabet of $q$ symbols ($1$ and $0$ for the binary language for example) is a set of $M$ vectors named the "\NewTerm{code words}\index{code word}". The idea is that each information word composed of $k$ symbols is associated with a single codeword composed of $n$ symbols. The vectors are therefore of length $n\ge k$ and their components are $q$-ary (thus "$2$-ary" in the case of the binary language).
	
	Examples of block codes are Reed–Solomon codes, Hamming codes, Hadamard codes, Expander codes, Golay codes, and Reed–Muller codes. These examples also belong to the class of linear codes, and hence they are named "linear block codes". More particularly, these codes are known as algebraic block codes, or cyclic block codes, because they can be generated using boolean polynomials.
	
	\textbf{Definition (\#\mydef):} A "\NewTerm{linear coding}\index{linear code}" means a set of code in which any linear combination (modular 2 sum most of time) of two codes within the set results in a code which also belong to the original set. Let's assume that you have a set of codes as in the figure below:
	\begin{figure}[H]
		\centering
		\includegraphics{img/computing/linear_code.jpg}
		\caption{Principle of a Linear code}
	\end{figure}
	Take out any two codes from the set and take modular $2$ sum of them. The result is also a member of the set as shown below. Take any other two codes and try yourself.
	
	\begin{tcolorbox}[title=Remark,colframe=black,arc=10pt]
	The linearity of the block codes also mean that the $n$ symbols of the code word are obtained by a linear combination of the $k$ symbols of the information word.
	\end{tcolorbox}
	Let us see an example!
	\begin{tcolorbox}[colframe=black,colback=white,sharp corners]
	\textbf{{\Large \ding{45}}Example:}\\\\
	Let us start from the following $M$ vectors based on $q = 3$ binary symbols (hence $k = 2$). In this case, we have $M=q^k$:
	\begin{table}[H]
		\centering
		\begin{tabular}{|c|}
		\hline
		\rowcolor[HTML]{9B9B9B} 
		\multicolumn{1}{|l|}{\cellcolor[HTML]{9B9B9B}\textbf{Vectors}} \\ \hline
		$000$ \\ \hline
		$100$ \\ \hline
		$010$ \\ \hline
		$110$ \\ \hline
		$001$ \\ \hline
		$101$ \\ \hline
		$011$ \\ \hline
		$111$ \\ \hline
		\end{tabular}
	\end{table}
	We then choose $n$ as equal to $6$ and we define a bijective mapping such that:
	% Please add the following required packages to your document preamble:
% \usepackage[table,xcdraw]{xcolor}
% If you use beamer only pass "xcolor=table" option, i.e. \documentclass[xcolor=table]{beamer}
	\begin{table}[H]
		\centering
		\begin{tabular}{|c|c|}
		\hline
		\rowcolor[HTML]{9B9B9B} 
		\multicolumn{1}{|l|}{\cellcolor[HTML]{9B9B9B}\textbf{Vectors}} & \textbf{Code Vectors} \\ \hline
		$000$ & $000\pmb{000}$ \\ \hline
		$100$ & $110\pmb{100}$ \\ \hline
		$010$ & $011\pmb{010}$ \\ \hline
		$110$ & $101\pmb{110}$ \\ \hline
		$001$ & $101\pmb{001}$ \\ \hline
		$101$ & $011\pmb{101}$ \\ \hline
		$011$ & $110\pmb{011}$ \\ \hline
		$111$ & $000\pmb{111}$ \\ \hline
		\end{tabular}
	\end{table}
	As shows this example, of the particular block code conventionally denoted $(n, k)_q = (6, 3)_2$, the code has no particular structure. The decoding operation involves making an exhaustive comparison of the word received at the output of the channel with all the code modes before determining the most likely code word. This simple and stupid approach explains why many times this error correcting code is faster than many others...
	\end{tcolorbox}
	Thus, according to the above definition, a block code $C$ is the result of a injective application which associates with each vector formed by $k$ $q$-ary symbols ($k$ information symbols), an image vector of length $n$ with components in the same alphabet ($n$ encoded symbols):
	
	 The encoding adds to the initial information $n-k$ additional symbols. The quantity:
	
	is named the "\NewTerm{rate of $C$}\index{rate of error correcting code}", or "\NewTerm{coding rate}\index{cording rate}". The block encoding operation is "without memory", in extenso the blocks are coded independently without any correlation between two consecutive blocks.
	
	Now it is convenient to come back a little bit on Boole's Algebras (\SeeChapter{see section Logical Systems page \pageref{boolean algebra}}). To the $5$ axioms which define a Boolean algebra let us add a sixth one which gives it a structure of a field:
	\begin{enumerate}
		\item[A6.] The Boolean algebra (extension of a unitary ring by an axiom) with the law $*$ (or $\wedge$) is a field.
	\end{enumerate}
	\begin{tcolorbox}[title=Remark,colframe=black,arc=10pt]
	Let us recall that a field is a non-zero ring in which every non-zero element is invertible.
	\end{tcolorbox}
	If we take the Boolean algebra formed by the $q=2$ elementary elements $\{0,1\}$ forming a binary set (alphabet), we actually have $1$ which is invertible since there exists $x$ such that:
	
	which is $1$ itself!

	This field is denoted $\mathbb{q}=\mathbb{F}_2$. In the area of error correcting codes, we often work in $\mathbb{F}_2$ (single field with two elements) where for recall the addition is defined by:
	
	The multiplication being defined by:
	
	
	To return to our theory of codes: the set of messages $E=\mathbb{F}_q^k$ can be equipped with a vector space structure of dimension $k$ on $\mathbb{F}_q$ (\SeeChapter{see section Set Theory page \pageref{vector space}}). Indeed, it suffices for this that $(E, +)$ to an Abelian group and $*$ an external law defined by $\mathbb{F}_q^k\times \mathbb{F}_q\mapsto \mathbb{F}_q^k$. If we decide to use only encoders that are linear (applications), the code $C=f(\mathbb{F}_q^k)$ becomes a vector subspace of $\mathbb{F}_q^n$ (because even if the application is bijective, since the body of the coded messages is finite, we necessarily have vector subspace of the vector space of all possible encoded messages).
	
	\textbf{Definition (\#\mydef):} A "\NewTerm{linear code}\index{linear code}" of dimension $k$ and length $n$ is a vector subspace of dimension $k$ of $\mathbb{F}_q^n$ (it is the way this is said..). If the minimum distance of $C$ is $d$, we say that $C$ is a "\NewTerm{$[n, k, d]_q$ code}" or more simply "\NewTerm{code $[n,k]$}".
	
	\begin{tcolorbox}[title=Remark,colframe=black,arc=10pt]
	Linear codes are therefore a special case of block codes as shown in the hierarchical scheme at the beginning of this section.
	\end{tcolorbox}
	The addition of the linearity constraint could undermine the quality of the code sought, but fortunately the performance study shows that the linear codes are very close to the best Block codes. Thus, linearity facilitates the study of block codes and allows the use of very powerful algebraic tools without reducing the class of linear blocks to an inefficient class.
	
	Let us denote by $G$ the matrix of the linear application $f:\mathbb{F}_q^k \mapsto \mathbb{F}_q^n$. $G$ is a matrix of obviously dimension $n\times k$ type and every word $c$ of $C$ is obtained from every word $x$ of $E$ by:
	
	where $c=(c_1,\ldots,c_n)\in\mathbb{F}_q^n$ and $x=(x_1,\ldots,x_k)\in\mathbb{F}_q^k$ are line vectors with always $n\ge k$. Thus $f(\mathbb{F}_q^k)=C$.
	\begin{tcolorbox}[title=Remark,colframe=black,arc=10pt]
	The bases of $\mathbb{F}_2^k$, $\mathbb{F}_2^n$ are the common canonical bases (those we have often used in the section of Vector Calculus).
	\end{tcolorbox}
	
	\textbf{Definition (\#\mydef):} Let $C$ be a linear code $[n, k]$ and given $\{g_1,g_2,\ldots,g_k\}$ the basis of $C$. A "\NewTerm{generating matrix}\index{generating matrix}" $G$ of $C$ is therefore a matrix $n\times k$ whose columns are formed by the vectors $g_i$ of the basis (see the example further below).

	Given $u=(u_1,\ldots,u_k)$ the information word, in extenso the vector containing the $k$ information symbols. Then we can write the matrix relation linking the code word $c$ and the information word $u$ by:
	
	
	\textbf{Definition (\#\mydef):} Let $C$ be a block code $[n, k]$. This code is named "\NewTerm{systematic code}\index{systematic code}", if the set of code words contains the $k$ unmodified symbols of the original information (we will return on this type of code further below). The remaining $n-k$ symbols are named "\NewTerm{parity symbols}\index{parity symbols}".
	\begin{tcolorbox}[title=Remark,colframe=black,arc=10pt]
	The "Hamming Code" is such a code! In addition, systematic codes are special cases of block codes and we will return to their study further below.
	\end{tcolorbox}
	
	\textbf{Definition (\#\mydef):} Let $H$ be an $(n-k)\times n$ matrix with elements in $\mathbb{F}_q$, which satisfies $Hc=0$ for every word $c$ of a linear code $C$ (in other words: whose kernel is $C$). Then, $H$ is named the "\NewTerm{control matrix}\index{control matrix}" of the code $C$. Conversely, $c$ belongs to the code if and only if $Hc=0$. Otherwise there is a mistake!
	\begin{tcolorbox}[title=Remark,colframe=black,arc=10pt]
	It is easy to find $H$ because it is "orthogonal" to $G$ since the above definition implies:
	
	of course we must not take $H = 0$ in practice...
	\end{tcolorbox}
	Let's see a companion example of all this with the Hamming code which is a systematic block code (caution!! it seem that there exist several definitions of a "Hamming code!"):

	This method consists of doubling the information, sending as many parity bits as data bits. A first matrix for this purpose is:
	
	The coding matrix $G$ above is of dimension $n\times k$, where $n$ is the number of bits received per packet, and $k$ is the number of bits per message containing the information (here $n=8$ and $k=4$). It automatically generates the parity bits specific to a message. For example, in order to send the message $1101$, in order to respect the matrix multiplication rule, consider this quartet as a column vector:
	
	So by multiplying, we get:
	
	We will therefore send a byte $11010110$, whose the first four bits form the message $u$ and the last four bits the parity bits, which are used to check the veracity / integrity of the message.

	The corresponding control matrix $H$ is:
	
	Thus, when the receiver receives the byte $11000110$ instead of $11010110$, the decoding gives as "syndrome":
	
	The resulting column vector is therefore not zero. So there is an error! With the control matrix, the theory (see proof below) makes it possible to assert that as the vector obtained is the same as that which is in fourth position in the decoding matrix, the error is due to the fourth bit. As we are in base $2$, it is enough to change the $0$ into a $1$. This coding of the information is expensive, because it occupies twice as much bandwidth. However this is one of the most effective ways to secure information.

	\begin{theorem}
	The syndrome $s$ of a Hamming code corresponds to one of the column of the control matrix $H$.
	\end{theorem}
	\begin{dem}
	To show that the syndrome of a Hamming code corresponds to one of the columns of the control matrix, we denote by $e_i$ the vectors-columns of the canonical basis on $\mathbb{F}_2^n$, $e_i=(0,\ldots,1,0,1,\ldots,0)$ with $1$ in the $i$-th place. Given $c$ a code word. We thus have by the definition of $H$: $Hc=0$. Let us suppose that the received word, which we will denote by $\tilde{c}$, is tainted by a single error and that this error is on the $j$-th bit. Therefore we have:
	
	and:
	
	Therefore it comes that:
	
	but $He_j$ is the $j$-th column vector of the matrix $H$.
	
	This shows us that when we $\tilde{c}$ and we compute $H\tilde{c}$ we get the column vector of the matrix $H$ located exactly at the location of the error (in this case $j$).
	\begin{flushright}
		$\blacksquare$  Q.E.D.
	\end{flushright}
	\end{dem}
	\begin{tcolorbox}[title=Remark,colframe=black,arc=10pt]
	A null syndrome does not mean the absence of error(s). There are therefore undetectable error configurations!!!
	\end{tcolorbox}
	Let us now write:
	
	Then we will notice that the matrices $G$ and $H$ of our companion example above are formed by the blocks $\mathds{1}_4$ and $A$ in the following way:
	
	named the "\NewTerm{(canonical) generator matrix of a linear $(n,k)$ code}", and:
	
	The  "\NewTerm{parity-check matrix}\index{parity-check matrix}".
	
	Therefore:
	
	Because $1 + 1 = 0$ in $\mathbb{F}_2$.

	In general, if we work with the alphabet $\mathbb{F}_2$ and if $G=\begin{pmatrix}\mathds{1}_k\\ A\end{pmatrix}$ where $A$ is $(n-k)\times k$ matrix then $H=(A\quad -\mathds{1}_k$ is also a control matrix because again:
	
	\begin{tcolorbox}[title=Remark,colframe=black,arc=10pt]
	In $\mathbb{F}_2$, we have $\mathds{1}_k=-\mathds{1}_k$, since $1=-1$ this is why we wrote $H=(A\quad -\mathds{1}_k)$ in the previous companion example.
	\end{tcolorbox}
	
	\subsubsection{Systematic codes}
	As we have mention it earlier above, let us come back on "systematic codes" that we have already defined.
	
	Constructing a systematic code consists as we already know in adding to each word $x=(x_1,\ldots,x_k)$ of the message $n-k$ symbols $(c_{k+1},\ldots,c_{k+n})$ linearly depending of the $x_i$ to get the code word $c=f(x)$.

	We know already that symbols are named "\NewTerm{control bits}\index{control bits}" and (we will see another example just below):
	
	where, for recall, $(\mathds{1}_k|A)$ denotes the matrix $n\times k$ obtained by writing one below the other, the identity matrix $\mathds{1}_k$ of size $k$ and any matrix $A$.
	
	We will say that a code $C$ is "systematic" if it has a generating matrix of the form $G=(\mathds{1}_k|A)$.
	\begin{tcolorbox}[colframe=black,colback=white,sharp corners]
	\textbf{{\Large \ding{45}}Example:}\\\\
	We propose to construct a systematic linear code with $n = k = 3$ as example. We will denote by $a_1,a_2,a_3$ the information bits. The control bits $a_4,a_5,a_6$ will be defined by:
	
	The generating matrix $G$ is such that its upper part is the identity matrix of dimension $3$ (we had the same thing for the Hamming code). The first line $(110)$ of the matrix $A$ corresponds to the expression of the control bit $a_4$:
	
	etc. For each control bit.\\

	The generating matrix $G$ is then written:
	
	By multiplying this matrix by the $2^3=8$ possible vectors (the words consisting of three bits of information), we get the following code words:
	\begin{table}[H]
		\centering
		\begin{tabular}{|c|c|c|c|c|c|}
		\hline
		\rowcolor[HTML]{9B9B9B} 
		$\pmb{a_1}$ & $\pmb{a_2}$ & $\pmb{a_3}$ & $\pmb{a_4}$ & $\pmb{a_5}$ & $\pmb{a_6}$ \\ \hline
		 $0$ & $0$ & $0$ & $0$ & $0$ & $0$ \\ \hline
		 $0$ & $0$ & $1$ & $0$ & $1$ & $1$ \\ \hline
		 $0$ & $1$ & $0$ & $1$ & $1$ & $0$ \\ \hline
		 $1$ & $0$ & $0$ & $1$ & $0$ & $1$ \\ \hline
		 $1$ & $0$ & $1$ & $1$ & $1$ & $0$ \\ \hline
		 $1$ & $1$ & $0$ & $0$ & $1$ & $1$ \\ \hline
		 $1$ & $1$ & $1$ & $0$ & $0$ & $0$ \\ \hline
		\end{tabular}
	\end{table}
	We thus find that the minimum weight of the code words is $3$. Therefore the code detects $3-1 = 2$ errors and can correct of them $\left[\dfrac{3-1}{2}\right]=1$.
	\end{tcolorbox}
	The reader interested can refer to out MATLAB™ companion book to see how to handle or generate error correcting codes.

	\begin{flushright}
	\begin{tabular}{l c}
	\circled{60} & \pbox{20cm}{\score{3}{5} \\ {\tiny 11 votes,  58.18\%}} 
	\end{tabular} 
	\end{flushright}

	%to make section start on odd page
	\newpage
	\thispagestyle{empty}
	\mbox{}
	\section{Automata Theory}\label{automata theory}
	\lettrine[lines=4]{\color{BrickRed}T}he purpose of this section is to study the theoretical aspect of the computer/machine concept. We are located here at the level of mathematics and logic, regardless of any reference to a real specific computer/machine (or software). We will look at how this theoretical machine will take make acquisition of digital data, of whatever nature, to do treatment on it or to solve a general problem. We will then be taken to see that, from this point of view, any theoretical machine is reducible in its operating principle, to an ideal machine. Thus, we can say that all computers, or all programs are equivalent to each other, as the purpose of computer, in its theoretical definition, is universal, that is to say the capacity to treat all actually treatable problems.
	
	Modern computing is the result of research undertaken in the early 20th century by Bertrand Russell and Alfred North Whitehead to constitute a formal mathematical system where any proposal could be proved by a logical calculation \SeeChapter{see section Proof Theory page \pageref{proof theory}}. David Hilbert and Kurt Gödel accomplished decisive steps in the exploration of this program. In 1931 Gödel proved that (recall):
	\begin{enumerate}
		\item It may be that in some cases we can prove one thing and its opposite (inconsistency).
		
		\item In any formal mathematical system there are mathematical truths that can not be proved (incompleteness)
	\end{enumerate}
	Gödel's theorem thus ruin the dream to make from mathematics a perfectly coherent deductive system, but from the intellectual activity around the Principia project of Russel and Whitehead will be released the founding ideas of computer science. This brings Alan Turing in 1936, after Gödel, to tackle the problem of decidability.
	
	\textbf{Definition (\#\mydef):} A system is named "\NewTerm{decidable system}\index{decidable system}" if there is an effective procedure for distinguishing provable proposals of others. To define more rigorously the concept of effective procedure, Alan Turing developed the concept of "automata", hereinafter named "\NewTerm{Turing machine}\index{Turing machine}" (see example below), which allows him to clarify the concept of implementation of an "\NewTerm{algorithm}\index{algorithm}".
	
	\textbf{Definition (\#\mydef):} A "\NewTerm{Turing Complete language}\index{Turing Complete language}" is a language with at least a conditional and a while-loop construct - such a language can be used to implement a Turing machine that is "powerful" enough to perform any realizable algorithm 
	
	Inventing effective procedures (algorithms) is to determine a sequence of elementary operations that perform the calculations/treatments necessary to solve problems for which there are computable solutions (there are unsolved problems and incalculable solutions as we have seen during our study of complexity in the section of Theoretical Computing). Turing also proved that its calculation model is universal, that is to say that all Turing machines are equivalent (we will prove this below). It makes the assumption that any algorithm can be computed by a Turing machine. These ideas underpin the theory of computer programming.
	
	\begin{tcolorbox}[title=Remark,colframe=black,arc=10pt]
	This section would have normally be placed at the first position of this Chapter but it seemed wiser to do first hand on concrete examples of theoretical computer before moving to the abstract formalism of their executions. This is one of the reasons why we will return here briefly on the concepts of algorithms, complexity, formal logical systems, proof theory and information (see all sections with the corresponding name). Furthermore, for this section, an experience in the development of computer software is a big plus for understanding certain concepts (or to imagine practical applications).
	\end{tcolorbox}
	Before we begin, it should be useful for the reader to have a non-exhaustive overview of the applications of Language and Automata Theory: specification of programming languages, compilation, pattern matching (in a text, in a database on the web, in the genes .. .), text compression, program verification, electronic computers, encoding for transmission, encryption, decoding of the genome, language, cognitive science, etc.
	
	The modern computing science (in the mathematical point of view only) was born from research undertaken in the early 20th century by Bertrand Russell and Alfred North Whitehead to constitute the mathematical formal system where any proposal could be proved by a logical calculation (\SeeChapter{see section Proof Theory page \pageref{proof theory}}). David Hilbert and Kurt Gödel accomplished decisive steps in the exploration of this program. In 1931 Gödel proved that (recall):
	\begin{enumerate}
		\item It may be that in some cases we can prove one thing and its opposite (inconsistency).
	
		\item In any formal mathematical system there are mathematical truths that can not be proven (incompleteness)
	\end{enumerate}
	Gödel's theorem then  ruin the dream of mathematicians to bring mathematics into a perfectly coherent deductive system, but the intellectual ferment around the \textit{Principia} project of Russel and Whitehead will release the founding ideas of theoretical computing. That brings in 1936 Alan Turing, after Gödel, to tackle the problem of decidability.
	
	\textbf{Definition (\#\mydef):} A system is named "\NewTerm{decidable system}\index{decidable system}" if there exist an effective procedure to distinguishing demonstrable proposals of the others. 

	To define more rigorously the notion of effective procedure, Turing developed the concept of "\NewTerm{Automata}\index{automata}", hereinafter named "\NewTerm{Turing machine}" (see example below), which allows him to clarify the concept of running an "\NewTerm{algorithm}" (\SeeChapter{see section Numerical Methods page \pageref{algorithm}}).
	
	Invent effective procedures (algorithms) consists to determine a sequence of elementary operations that perform the calculations necessary to solve problems for which there are computable solutions (there are unsolved problems and incalculable solutions as we seen during our study of complexity in the section of Theoretical Computing). Turing also proved that its calculation model is universal, that is to say, all Turing machines are equivalent (we will prove it later below). He made the assumption that any algorithm can be computed by a Turing machine. These ideas underlying the theory of computer programming and was one important factor of winning the second World War thanks to the machine developed by Turing to uncrypt the Nazi Enigma cypher machine (a romance movie has been made about this subject in 2015 with Alan Turing as major role).

	\subsection{Von Neumann machine}
	We due to John von Neumann to conceive in 1945 the general architecture of the concrete apparatuses which will carry out the calculations according to the Turing model, an architecture so efficient and elegant that the computers of today are still constructed on these principles.
	
	\begin{tcolorbox}[title=Remark,colframe=black,arc=10pt]
	In a way, we can say that this decade between 1936 and 1945 (corresponding well to the Second Worl War) saw the birth of computer science, which went from the mathematical and logical intellectual construction stage to the application of these ideas to the realization of concrete physical systems.
	\end{tcolorbox}
	Here is the diagram of the of a von Neumann architecture:
	\begin{figure}[H]
		\centering
		\includegraphics{img/computing/von_neumann_machine.jpg}
		\caption{Principle of the von Neumann machine}
	\end{figure}
	The Control Units, Arithmetic-Logic Unit (ALU), and Primary Memory constitute all three the "Central Unit", or "Processor", of the computer. The processor consists of electronic circuits that can perform actions. The set of actions wired in the processor constitutes the instruction set of the processor and determines the basic language of its use, named "\NewTerm{machine language}".

	The role of the Control Unit is to enable the desired action (instruction) to be triggered at the desired moment. This instruction can belong to the Arithmetic-Logic Unit, to the Memory Unit or to the Control Unit itself. An instruction can also consult the contents of the Primary Memory unit (the "read") or modify the contents of that memory (the "write"). In general, an action consists in either consulting or modifying the state of the memory or one of the registers (which are special memory elements incorporated in the central processing unit), or triggering an input-output operation (communication with the outside world and in particular a human user).
	
	The memory is made up of elements that can take states. A basic element of the memory can take two distinct states and can be used to represent an elementary data item, or "bit" (\SeeChapter{see section Logical Systems \pageref{bit}}). This representation of a data element by a memory element is named a "\NewTerm{code}". A memory with many bits allows the coding of complex data, within the limit of the size of the memory.

	The way in which the central unit, memory and I/O devices (input/output) communicate is generically named a "\NewTerm{bus}" (it is, in a way, the highway where data flows from one point to another Using addresses). In a somewhat formal way, a bus is a complete connected graph (\SeeChapter{see section Graph Theory page \pageref{complete graph}}), which means in common language that all the elements connected to the bus can communicate with one another.
	\begin{tcolorbox}[title=Remark,colframe=black,arc=10pt]
	The "code" makes bits and groups of symbols match together. The simplest symbols are numbers and letters. To represent complex data, you can define methods, rules for grouping symbols, and associate a data element with a symbol group constructed according to the rules.
	\end{tcolorbox}
	\textbf{Definition (\#\mydef):} We will name "\NewTerm{language}\index{language}" a set of symbols or groups of symbols, constructed according to certain rules, and which are the "\NewTerm{words}\index{word}" of that language. A "\NewTerm{language syntax}\index{language syntax}" is the set of construction rules for language words.
	
	The memory of the computer (this is the basic idea of von Neumann) contains information of two types: Programs and data. The programs and the data are represented with the same symbols, only the semantics allows to interpret their respective texts. Moreover, the text of a program can sometimes be considered as data for another program, for example a program of translation from one language to another.
	
	\begin{fquote}[John von Neumann]Computers are like humans - they do everything except think.
 	\end{fquote}
	
	\subsection{Turing machine}
	It is important to be convinced (it will perhaps not be done in one day...) that all the programs we can write in different languages are equivalent!!! The Turing machine is a model of an automata whose description is very low-level (before going on to a much more formal definition). The von Neumann architecture, designed to efficiently perform the processes described by a Turing machine, generates imperative languages (see definition in remark R1 below). Any program, functional or imperative, intended to be executed, will be translated into an imperative language, the "machine language" of the computer used. The coherence of computer science and the semantic equivalence of programs written in various languages which ensure the validity of this operation are the result not of chance but of a common original theoretical conception. Gödel, Church, von Neumann and Turing were all in Princeton in 1936...!
	\begin{tcolorbox}[title=Remarks,colframe=black,arc=10pt]
	\textbf{R1.} The first evolved languages that have appeared are so-named "\NewTerm{imperative languages}", based on the notion of the state of memory (it is the "\NewTerm{assembly language}\index{assembly}" by the way!). These languages, inspired by John von Neumann's model, include, like machine language, instructions that produce changes in memory (assignment instruction). The writing of a program in imperative language consists in writing the sequence of instructions which will cause the successive states by which the memory will have to pass so that, starting from an initial state allowing the initialization of the program, it arrives in a state providing final results.
	\begin{figure}[H]
		\centering
		\includegraphics[scale=0.91]{img/computing/assembly_demo.jpg}
		\caption{Assembly language "Hello World" demo}
	\end{figure}
	\textbf{R2.} In addition to "computational languages", we distinguish in computer science "sequential languages", "interpreted languages", "description languages" and "compiled languages".
	\end{tcolorbox}
	A formal model for an effective procedure (to describe an algorithm) must possess certain properties. First, each procedure must receive a finite definition. Secondly, the procedure must consist of separate steps, each of which must be capable of being accomplished mechanically. In its simplicity, the Turing machine composed of the following elements answers to this program:
	\begin{enumerate}
		\item An infinite memory represented by a ribbon divided into boxes. Each square of the ribbon may be given a symbol of the alphabet defined for the machine;
	
		\item A reading head capable of traversing the tape in both directions;
	
		\item A finite set of states among which we distinguish an initial state and the other states, named "\NewTerm{accepting states}"
	
		\item A transition function which, for each state of the machine and each symbol under the read head, specifies: the next state, the character that will be written on the ribbon instead of the one that was under the head The direction of the next playback of the playback head.
	\end{enumerate}
	One can equip his Turing Machine with the finite alphabet of his choice. His ribbon can be infinite in either direction or in one. It may even have several ribbons. It is shown that these various machines are equivalent.
	\begin{figure}[H]
		\centering
		\includegraphics[scale=0.7]{img/computing/turing_machine.jpg}
		\caption{Turing Machine principle}
	\end{figure}
	We are then led to the following simplistic definition:
	
	\textbf{Definition (\#\mydef):}
	A "\NewTerm{finite automate}" is a mathematical model of systems having a finite number of states and that actions (external or internal) can move from one state to another. The external actions are represented by the symbols of an alphabet $\mathcal{A}$; The internal actions (invisible, silent, or spontaneous) are represented by a symbol not belonging to the aforementioned alphabet.
	
	\begin{tcolorbox}[colback=red!5,borderline={1mm}{2mm}{red!5},arc=0mm,boxrule=0pt]
	\bcbombe Caution! For example, any real computer is finite so it isn't an abstract model like a Turing Machine. The finite version of the Turing Machine has a confusing name: "\NewTerm{Linear Bounded Automaton}". This is basically what real computers are. However, LBA are not seen or used in discussion nearly as much as Turing Machine.
	\end{tcolorbox}

	An automate is represented by a graph (\SeeChapter{see section Graph Theory page \pageref{graph theory}}) whose vertices are states and with each arc is associated the recognition of one or more letters.

	Finite automata are used to model and control finite-state systems and to solve common problems: lexical analysis, search of patterns in text, genome analysis, etc.
	
	\begin{tcolorbox}[colframe=black,colback=white,sharp corners]
	\textbf{{\Large \ding{45}}Examples:}\\\\
	E1. A finite and deterministic automate which recognizes all integers whose writing is normalized (regular language), that is to say not starting with $0$ (the numbers in the circles are just there to describe the order in which the controller performs the operation):
	\begin{figure}[H]
		\centering
		\includegraphics[scale=0.91]{img/computing/finite_automate_example_01.jpg}
		\caption{First example of a finite automata}
	\end{figure}
	Description: The automata receives an integer in input \circledtext{1}, it looks at whether this number starts with a $0$ or it is a number between $1$ and $9$. If the number starts with zero, the controller exits and stop at \circledtext{3}. Otherwise, the controller goes to \circledtext{2} and analyses the numbers one after the other until it reaches the end and then stops and goes out at \circledtext{3}.\\
	
	E2. A finite and deterministic automata which recognizes a numerical input in a regular language spreadsheet (for example: $+12,3$ or $08$ or $-15$ or $5\text{E}12$ or $14\text{E}-3$):
	\begin{figure}[H]
		\centering
		\includegraphics[scale=0.91]{img/computing/finite_automate_example_02.jpg}
		\caption{Second example of a finite automata}
	\end{figure}
	\end{tcolorbox}
	
	\begin{tcolorbox}[colframe=black,colback=white,sharp corners]
	In other words, it is enough to recognize a language of the form:
		
	which is indeed regular, where $\varepsilon$ is the empty word, $\mathcal{A}$ is the alphabet $\{0,1, \ldots, 9\}$ and equation the set of words (in extenso of numbers) that can be written with $\mathcal{A}$.\\
	
	E3.  A finite and deterministic automata recognizing all the multiples of $3$, regular language type (in other words if such a multiple is found, the automata gives an output, otherwise nothing):
	\begin{figure}[H]
		\centering
		\includegraphics[scale=1]{img/computing/finite_automate_example_03.jpg}
		\caption{Third example of a finite automata}
	\end{figure}
	\end{tcolorbox}
	I strongly recommend any reader that want the be familiar with complete Turing Machines to practice the challenge that Google made with the Turing Doodle (it is probably "complete" but not sure):
	\begin{figure}[H]
		\centering
		\includegraphics[scale=1]{img/computing/turing_doodle.jpg}
		\caption[]{Turing Doodle (source: Google)}
	\end{figure}
	available here:
	\begin{center}
		\url{http://www.sciences.ch/htmlen/turing_doodle/}
	\end{center}
	The purpose of that Doodle is to change the algorithm so that when it is executed and the machine stops the content of the tape is compared to the content of the display on the right top corner. When the comparison is successful the player will go to the next level with another algorithm to change... and so on...
	
	Some passionate of Turing Machines have build real life small and beautiful electronic Turing machines as the one visible in the picture below (perhaps they build as a hobby or sell it to schools for education purposes???):
	\begin{figure}[H]
		\centering
		\includegraphics[scale=0.7]{img/computing/turing_machine_photo.jpg}
		\caption[Turing Machine]{Turing Machine (source: ?)}
	\end{figure}
	\begin{tcolorbox}[title=Remarks,colframe=black,arc=10pt]
	\textbf{R1.} Relative to a common question: Yes! Artificial intelligence are Turing Machines. However at the day we write these lines (early 21st century), we still have no evidence based likelihood to claim that Homo Sapiens brain is a Linear Bounded Automaton or not (we let this work to future generations!).\\
	
	\textbf{R2.} Markov chains can be represented by finite states machines. The main thing to keep in mind is that the transitions in a Markov chain are probabilistic rather than deterministic. which means that you can't always say with perfect certainty what will happen a time $t+1$. The brain is an more near than a Markov chain than a Turing machine.
	\end{tcolorbox}
	
	\pagebreak
	\subsection{Chomsky hierarchy}
	The "\NewTerm{Chomsky hierarchy}\index{Chomsky hierarchy}" is a classification of the languages described by the formal grammars proposed in 1956 by the linguist Noam Chomsky. It is widely used today in computing, especially for the design of interpreters or compilers, or for the analysis of natural languages.
	
	It is necessary before to define certain concepts!
	
	\subsubsection{Formal language}
	\textbf{Definition (naive version \#\mydef):} In a broad range of contexts (scientific, legal, etc.) we designate naively by "\NewTerm{formal language $\mathcal{L}$}\index{formal language}" a more formalized and more precise form of expression than the everyday natural language (the two do not necessarily go together).
	
	In mathematics, logic and computer science, a formal language is formed by:
	\begin{enumerate}
	 	\item A set of words obeying to strict logical rules (formal grammar or syntax).

		\item Possibly of an underlying semantics (the strength of formal languages is to be able to disregard such semantics, which makes theories reusable in several models).
	\end{enumerate}
	\begin{tcolorbox}[title=Remark,colframe=black,arc=10pt]
	Thus, while a particular payroll or inverse matrix calculation will always remain a payroll or inverse matrix calculation, a group theorem will apply both to the set of integers and to the transformations of the Rubik cube.
	\end{tcolorbox}
	The formal language of a scientific discipline is therefore actually a language obeying strict formal syntax and which will serve to expose statements precisely, if possible concisely and without ambiguity, and is in opposition to natural language.

	Formal language has the advantage of making easy the manipulation and transformations these statements. Indeed, we will generally have precise transformation rules (development of logical formulas, normal forms, contrapositions, commutativity, associativity, etc.) that can be applied without even knowing the meaning of the statements to be transformed or the meaning of the transformation. It is therefore a powerful exploration tool, and it is the only language that allows machines to "do mathematics".

	The disadvantage is obvious: not knowing the meaning of the statement prevents us from knowing which are the relevant transformations and hurts the intuition of the reasoning. Thus, it is good to know how to quickly read a statement in formal language and to translate it just as quickly into one or more natural language statements.
	
	This is where the limit of what we call "proof-aid software" lies: of course, the computer has (for now...) no intuition. The skill of the designer of such a program is to find ways for the computer to understand.

	Giving relevant meaning to a programming language in order to run its programs is relatively easy, because these formal languages have been designed to mean sequences of elementary actions of the machine. To prove a program (to prove that the algorithm ends in a finite number of times) or a mathematical theorem (which is almost the same thing), there is, on the other hand, no infallible method, the correction of a program being an undecidable decision problem. Thus, the prover must simply apply certain heuristics (a technique consisting in learning little by little taking into account what has been done beforehand) and often calling for help to the human user (same as humans do in fact...!). However, thanks to its heuristics and computing power, the computer explores thousands of ways that the human user would not have been able to test in several years, thus accelerating the work of the mathematician, physicist or engineer.
	
	\textbf{Definition (\#\mydef):} As an object of study, a "\NewTerm{formal language $\mathcal{L}$}\index{formal language}" is defined as a set $\mathcal{W}$ of finite-length words $w_i$ (ie strings) deduced from a certain finite alphabet $\mathcal{A}$, that is to say a free monoid (the set of words formed on an alphabet, provided with the internal law of concatenation - which is a law of composition - is a monoid which we name "free monoid", which empty word is the neutral element) on this alphabet.
	
	\begin{tcolorbox}[title=Remark,colframe=black,arc=10pt]
	Thus, while a particular payroll or inverse matrix calculation will always remain a payroll or inverse matrix calculation, a group theorem will apply both to the set of integers and to the transformations of the Rubik cube.
	\end{tcolorbox}
	
	\subsubsection{Syntax}
	\textbf{Definition (\#\mydef):} The "\NewTerm{syntax}\index{syntax}" is the branch of linguistics that studies the way in which "free morphemes" (words) combine to form "syntagmas" (nominal or verbal) that can lead to propositions that can combine in turn to form statements.
	
	\begin{tcolorbox}[colframe=black,colback=white,sharp corners]
	\textbf{{\Large \ding{45}}Example:}\\\\
	The syntagma (sentence): \textit{a modest house of red bricks} is encompassed in the upper syntagma, that is, the complete sentence. But this same sentence \textit{a modest house of red bricks} includes among its elements, the lower syntagma \textit{of red bricks}, complement of the name house.
	\end{tcolorbox}
	
	\textbf{Definitions (\#\mydef):} 
	\begin{enumerate}
		\item[D1.] In grammar school, a "\NewTerm{proposition}\index{proposition}" is a syntagma articulated around a verb. This notion is mainly used in language learning.

		\item[D2.] A "\NewTerm{statement}\index{statement}" in linguistics is everything that is pronounced by a speaker between two breaks. Syntactically, the statement can therefore extend from the simple word to the length of a sentence (even to a discourse), through the syntagma.
	\end{enumerate}
	The term "syntax" is also used in computer science, where its definition is similar, modulo a different terminology... Thus the syntax is the respect, or the non-respect, of the formal grammar of a computer language, that is to say of the rules of arrangement of the lexemes (which in computer science are only lexical entities) in more complex terms, often: "programs". In the theory of formal languages, what plays the role of lexeme is usually named "\NewTerm{letter}\index{letter}" or "\NewTerm{symbol}\index{symbol}", and the product terms are named "\NewTerm{words}\index{words}".
	
	From a purely grammatical point of view, the study of syntax concerns three kinds of units:
	\begin{itemize}
		\item The "\NewTerm{sentence}\index{sentence}", which is the upper limit of the syntax.

		\item The "\NewTerm{word}\index{word}", which is its basic constituent, sometimes named "\NewTerm{terminal element}\index{terminal elements}"

		\item The "\NewTerm{syntagma}\index{syntagma}", which is its intermediate unit
	\end{itemize}
	The syntactic relations between these different units can be of two kinds:
	\begin{itemize}
		\item The "\NewTerm{coordination}" when the elements are of the same status

		\item The "\NewTerm{subordination}" in the opposite case (when there is subordination, the subordinate element fulfils a syntactic function determined with respect to the higher level unit)
	\end{itemize}
	The study of syntax will take into account, in particular, the nature (or category or species) of the words, their form (morphology) and their function. It is why we speak more generally of "\NewTerm{morphosyntactic relations}".
	
	\subsubsection{Grammar}
	\textbf{Definition (naive version \#\mydef):} A "\NewTerm{formal grammar $\mathcal{G}$}\index{formal grammar}" is a formalism used to define a syntax and therefore a formal language $\mathcal{L}$, that is to say a set $\mathcal{W}$ of words $w_i$ on a given alphabet $\mathcal{A}$.
	
	The concept of formal grammar is particularly used in the following fields:
	\begin{itemize}
		\item Programs compilation (syntactic analysis)

		\item The analysis and processing of natural languages

		\item Calculation models (automata, circuits, Turing machines, etc.)
	\end{itemize}
	To define a grammar, we need (see the example below to understand):
	\begin{enumerate}
		\item An alphabet of non-terminal items

		\item An alphabet of terminals item;

		\item An initial symbol (the axiom) taken among the non-terminals items;

		\item A set of production rules.
	\end{enumerate}
	\begin{tcolorbox}[colframe=black,colback=white,sharp corners]
	\textbf{{\Large \ding{45}}Examples:}\\\\
	E1. We can define arithmetic expressions in the following way (writings that we often find in the Proof theory) where $|$ is the symbol commonly use for the logical "OR":
	\begin{center}
		\texttt{exp}$\rightarrow$ \texttt{exp+exp}$|$\texttt{exp*exp}$|$\texttt{(exp)}$|$\texttt{num}
	\end{center}
	where \texttt{exp} means "expression" or:
	\begin{center}
		\texttt{num}$\rightarrow 0$\texttt{num}$|1$\texttt{num}$|2$\texttt{num}$|\ldots|9$\texttt{num}$|1|2|\ldots|9$
	\end{center}
	The non-terminals here are explicitly \texttt{exp} and \texttt{num}, the terminals are \texttt{+}, \texttt{*}, (\texttt{,}) and the digits. The axiom is \texttt{exp}.\\
	
	E2. The syntax of classical propositional logic can be defined in the following way (\SeeChapter{see section Proof Theory page \pageref{grammar}}):
	
	\end{tcolorbox}
	The most commonly used types of grammars are:
	\begin{enumerate}
		\item Left linear grammars that produce the same languages as regular expressions (this is what we are interested to in this section)

		\item Context-free grammar (example above)

		\item Contextual grammars (this type of grammar requires a mathematical formalism and can not be defined without it)
	\end{enumerate}
	A language is therefore a set of words $\mathcal{W}$, which are simply sequences of symbols chosen from a finite alphabet set $\mathcal{A}$. The languages of the Chomsky's hierarchy consist of words that respect a particular formal grammar. What distinguishes them within the framework of classification is the nature of the grammar.
	\begin{tcolorbox}[title=Remark,colframe=black,arc=10pt]
	Most often, the symbols that are considered are formed of several characters, so that they correspond rather to what "words" in the current language. When there is ambiguity, for example in lexical analysis (vocabulary) and syntactic analysis (part of the grammar that deals with the function and the disposition of words and propositions in the sentence), we speak of "characters" for the symbols of the alphabet used to encode the information, and of "lexemes" for the symbols of the abstract alphabet, which are the basic units of the language. Similarly, the "words" of the language correspond rather to "sentences" or "texts".
	\end{tcolorbox}
	The Chomsky hierarchy consists of the following $4$ levels, from the most restrictive to the broadest one:
	\begin{enumerate}
		\item[L1.] The "\NewTerm{languages of type 3}" or "\NewTerm{regular language}\index{regular language}": these are the languages defined by a regular grammar or a regular expression, or the languages recognizable by a finite-state automata.

		\item[L2.]  The "\NewTerm{languages of type 2}" or "\NewTerm{algebraic languages}\index{algebraic language}" also named "\NewTerm{context-free languages}\index{context-free languages}": these are the languages defined by a context-free grammar, or the languages recognizable by a non-deterministic stack automata. In this category are for example the computer programming languages.

		\item[L3.]  The "\NewTerm{languages of type 1}"  or "\NewTerm{context sensitive languages}\index{context sensitive languages}": these are the languages defined by a contextual grammar, or the languages recognizable by a non-deterministic Turing machine with a length bounded by a fixed multiple of the word length's (these types of languages require a mathematical formalism and can not be defined without it).

		\item[L4.]  The "\NewTerm{languages of type 0}", or "\NewTerm{recursively enumerable languages}\index{recursively enumerable languages}": This set includes all languages defined by a formal grammar. It is also the set of languages acceptable by a Turing machine (which is allowed to loop on a word that is not of the language).
	\end{enumerate}	
	\begin{figure}[H]
		\centering
		\includegraphics[scale=1]{img/computing/chomsky_hierarchy.jpg}
		\caption{Chomsky hierarchy}
	\end{figure}
	\begin{tcolorbox}[title=Remarks,colframe=black,arc=10pt]
	\textbf{R1.} In addition to the $4$ types of the Chomsky hierarchy, there are remarkable intermediate classes! For example between types $3$ and $2$: deterministic non-contextual languages, recognizable by a deterministic stack automata and languages between levels $1$ and $0$: recursive languages, that is to say, recognizable by a Turing machine (the latter must refuse words which are not in the language).\\
	
	\textbf{R2.} The $4$ types of languages and $2$ of intermediate languages above are strictly included in each other.
	\end{tcolorbox}
	A parser for a formal language is a computer program that decides whether a given input word belongs or not to the language, and possibly constructs a derivation of it.

	There are systematic methods for writing type $2$ or $3$ language analysis programs (parsers). Interpreters or compilers almost always include a phase of lexical analysis, which consists of recognizing type $3$ languages, followed by a phase of syntactic analysis that is a in fact a type $2$ language analysis.

	We can now finally in a vulgarize way (always with the aim of paving the way) define what an automata is in the Chomsky hierarchy.
	
	\subsubsection{Associated automata}
	\textbf{Definition (naive version \#\mydef):}  In the field of theoretical computing, an "\NewTerm{automata}\index{automata}" is a machine to process information by a formal model (a Turing machine) on a given language. So:
	\begin{itemize}
		\item On a "\NewTerm{finite language}\index{finite language}" (language containing a finite number of words), the associated automata is a machine comparing a text with that which is stored in a read-only memory. The grammar associated with a finite language is a list of the words of the language.
		
		\item On a "\NewTerm{regular language}\index{regular language}" (language where the syntactic correction is verified by storing only a finite number of information), the associated automata is the "deterministic finite automaton" (that is, for each word entered, there is only one possible path of the graph) or the "non-deterministic finite automata". The grammar associated with a regular language is a left linear grammar.
		
		\item On an "\NewTerm{algebraic language}" (language where the main syntactic constraint are the parenthesis), the associated automata is the "pushdown (stack) non-deterministic automata". The associated grammar is the algebraic grammar.
		
		\item On a "\NewTerm{bounded language}" (description requiring a mathematical formalism), the associated automata is the  "linearly bounded automata". The associated grammar is the contextual grammar.
		
		\item On a "\NewTerm{decidable language}" (an intelligent being manages to find a process to know whether or not one he is in the language), the associated automata is a Turing machine that stops on all data. There is no grammar associated to it.
		
		\item On a "\NewTerm{semi-decidable language}" (an intelligent being manages to find a process to know whether or not one he is in the language), the associated  automata is the Turing machine (thus contains conditional structures and loops). The associated grammars are the "semi-Turing grammar", the "de Vangarden grammar" or the "affixed grammars".
	\end{itemize}
	
	\pagebreak
	\subsection{Terminology}
	The automata therefore work mainly on letters, words, sentences and languages. In order to construct rigorous and optimal analysis methods and treatment on the subject it is interesting to formalize the treated objects. This is what we will do initially by defining these latter and their mathematical properties (which are very intuitive).
	
	\subsubsection{Words}
	\textbf{Definitions (\#\mydef):}
	\begin{enumerate}
		\item[D1.] An "\NewTerm{alphabet $\mathcal{A}$}\index{alphabet}" is a set whose elements are the "\NewTerm{letter $\ell$}". The alphabets are always supposed to be finished.
		
		\item[D2.] A "\NewTerm{word $w_i$}\index{word}" is a finite sequence of "letters" which we denote by juxtaposition:
		
	
		\item[D3.] The "\NewTerm{empty word}", denoted $\varepsilon$, is the only word composed of no letters.
	
		\item[D4.] The "\NewTerm{length}" of a word $w$ is the number of letters that compose it, and is denoted $|w|$ (the empty word $\varepsilon$ is the only word of length $0$).
	
		The "\NewTerm{concatenation product}\index{concatenation product}" of two words $w_1=a_1a_2\ldots a_n$ and $w_i=b_1b_2\ldots b_m$ is the word $w_1w_2$ obtained by juxtaposition (concatenation):
		
		Of course (trivial), we have:
		
		We denote by $\mathcal{A}^{*}$ the set of words on $\mathcal{A}$.
		
		\begin{tcolorbox}[colframe=black,colback=white,sharp corners]
		\textbf{{\Large \ding{45}}Example:}\\\\
		Genes are words on the ACGT alphabet, proteins are words on a $20$-letter alphabet. The natural integers, written in base $10$, are words on the alphabet of the decimal digits...
		\end{tcolorbox}
		Let $\mathcal{A}$ be an alphabet. Let $\mathcal{B}$ be a subset of $\mathcal{A}$ . For any word $w\in \mathcal{A}$, the length in  $\mathcal{B}$ of $w$ is the number of occurrences of letters of $\mathcal{B}$ in the word $w$. This number will be denoted $|w|_{\mathcal{B}}$.
		\begin{tcolorbox}[title=Remark,colframe=black,arc=10pt]
		In particular, we have trivially $|w|=|w|_{\mathcal{A}}$.
		\end{tcolorbox}
		For every letter $\ell\in\mathcal{A}$, $|w|_{\ell}$ is the number of occurrences of $\ell$ in $w$. We have:
		
		
		\item[D5.] Given $w=\ell_1\ell_2\ldots\ell_n$, with $\ell_1\ell_2\ldots\ell_n\in\mathcal{A}$. The "\NewTerm{mirror word}\index{mirror word}" of $w$ is the word denoted $\widetilde{w}$ defined by:
		
		Obviously:
		
		
		\item[D6.] A word $u$ is a "\NewTerm{prefix}" or "\NewTerm{left factor}" of a word $v$ if there is a word $x$ such that $ux=v$. The word $u$ is moreover a "\NewTerm{strict prefix}" or "\NewTerm{eigen-prefix}" if $u\neq v$. Symmetrically, $u$ is a "\NewTerm{suffix}" or "\NewTerm{right factor}" of $v$ if $xu=v$ a word $x$. If $u\neq v$, then $u$ is "\NewTerm{strict suffix}" or "\NewTerm{eigen-suffix}". The number of prefixes of a non-empty word $v$ is $1+|v|$ (the empty word always being a prefix, we always have any non-empty word that has at least the empty word as a prefix).
		\begin{tcolorbox}[colframe=black,colback=white,sharp corners]
		\textbf{{\Large \ding{45}}Example:}\\\\
		The word $w=aabab$ on $\mathcal{A}=\{a,b\}$ has $12$ different possible factors:
		
		\end{tcolorbox}
	\end{enumerate}
	\begin{lemma}[Levy's lemma]
	Given $x$, $y$, $z$, $t$ be words such as $xy=zt$. Then there exists a word $w$ such that:
	
	with obviously:
	
	or:
	
	with also by extension:
		
	\end{lemma}
	It results logically in particular that if $|x|=|y|$, the word $w$ is empty, and therefore $x=z$ and $y=t$. In other words:
	\begin{theorem}
	A free monoid (see the reminder below) can be simplified on the left and on the right.
	\end{theorem}
	\begin{dem}
	Let us put:
	
	with $a_i\in\mathcal{A}$, similarly:
	
	with $b_i\in\mathcal{A}$.
	
	As:
	
	We have:
	
	(but not necessarily $n=p$) and:
	
	for $i=1,\ldots,m$ so that:
	
	If $|z|=p\le n=|x|$, let us put $w=x_{p+1}\ldots x_n$. Therefore:
	
	If $|z|>|x|$, let us put $w=x_{n+1}\ldots x_p$. Therefore:
	
	\begin{flushright}
		$\blacksquare$  Q.E.D.
	\end{flushright}
	\end{dem}
	Let us just do a recall of a parallel of the section of Set Theory... In the framework of the study automata a "\NewTerm{free monoid}\index{free monoid}" is a set $\mathcal{A}$ (the alphabet), whose elements are the letters $\ell_i$. Therefore for the composition law denoted $\cdot$:
	
	In the Set Theory section, we were simply talking about the concept of "monoid". The monoid $(\mathcal{A},\cdot)=(\mathcal{A},\text{concatenation})$.
	
	\pagebreak
	\subsubsection{Languages}
	The subsets of $\mathcal{A}$ are named "\NewTerm{formal languages}\index{formal language}". For example, for $\mathcal{A}=\{a,b\}$, the set $\mathcal{A}^{+}=\{a^nb^n|n\ge 0\}$ is a language.

	We define on the languages several operations. The set operations are the union, intersection, complementarity and the resulting difference (\SeeChapter{see section Set Theory page \pageref{set operations}}). If $X$ and $Y$ are two parts of $\mathcal{A}^{*}$ then for recall each of this operation is defined by:
	\begin{itemize}
		\item Union:
		

		\item Intersection:
		

		\item Complementarity:
		

		\item Difference:
		
	\end{itemize}
	The also have for operation the product (of concatenation) of two languages $X$ and $Y$ is the language:
	
	and we have for recall:
	
	and also the operation of left quotient of $Y$ by $X$:
	
	\begin{tcolorbox}[colframe=black,colback=white,sharp corners]
	\textbf{{\Large \ding{45}}Example:}\\\\
	Let us consider three languages:
	
	Then we have for the union:
	
	for the concatenation:
	
	\end{tcolorbox}
	
	\begin{tcolorbox}[colframe=black,colback=white,sharp corners]
	for a given quotient:
	
	another quotient:
	
	and a last quotient:
	
	and a stupid difference example:
	
	Following the request of a reader we will detail:
	
	by recalling the definition:
	
	We then have explicitly:
	
	and in the concatenation product of the two languages $X$ and $Y$, the only words where we find the elements $\{0,1\}$ of the language $Z$ as a prefix are:
	
	and as by definition $Z^{-1}(XY)$ is the unique set of terms $w$ which follow the prefixes that constitute the terms of $Z$, then there remains only:
	
	\end{tcolorbox}
	We have the following properties:
	\begin{enumerate}
		\item[P1.] Obviously:
		

		\item[P2.] Less obvious:
		
		where the inclusion is generally strict. To conceptualize this property, we must not forget that $X$ is a set of words and that $Y$, $Z$ do not necessarily have words of the same length!
	\end{enumerate}	
	The powers of $X$ are defined by
	
	for $n\geq 1$

	In particular, if $\mathcal{A}$ is an alphabet, $\mathcal{A}^n$ is the set of words of length $n$.
	
	\textbf{Definitions (\#\mydef):}
	\begin{enumerate}
		\item[D1.] The "\NewTerm{Kleene star}\index{}" (or "\NewTerm{Kleene operator}" or "\NewTerm{Kleene closure}") of $X$ is the set:
		
		\begin{tcolorbox}[colframe=black,colback=white,sharp corners]
		\textbf{{\Large \ding{45}}Example:}\\\\
		Given $X=\{a,ba\}$. The words of $X^{*}$, classified by length are:
		\begin{table}[H]
		\begin{tabular}{cc}
		$0$      & $\varepsilon$              \\
		$1$      & $a$                        \\
		$2$      & $aa,ba$                    \\
		$3$      & $aaa,aba,baa$              \\
		$4$      & $aaaa,aaba,abaa,baaa,baba$ \\
		$\ldots$ &                           
		\end{tabular}
		\end{table}
		\end{tcolorbox}
	
		\item[D2.] The operator "$+$" is defined similarly:
		
	\end{enumerate}
	
	
	\subsubsection{Equations}
	First let us see a little something we will need later: given $u$ and $v$ two non-empty words. The three following conditions are equivalent (without proof because quite trivial):
	\begin{enumerate}
		\item[C1.] $uv=vu$
		\item[C2.] $\exists\, n,m>1:\quad u^n=v^m$
		\item[C3.] $\exists\, w\neq\varepsilon, k,l\ge 1:\quad u=w^k,v=w^l$ 
	\end{enumerate}
	\begin{tcolorbox}[title=Remark,colframe=black,arc=10pt]
	Let us recall again that we do not necessarily have $|u|=|v|$ but that we can very well have $|u|>|v|$.
	\end{tcolorbox}
	Let us now turn to interesting things (some fuzzy points of the section of Proof Theory can be clarified here sometimes...)!
	
	\textbf{Definitions (\#\mydef):}
	\begin{enumerate}
		\item[D1.] Let $\mathcal{V}$ and $\mathcal{A}$ be two disjoint alphabets (you can imagine them as the set of variables and respectively of the constants for example...). An "\NewTerm{equation in words}" with constants on $\mathcal{A}$ is a couple $e=(\alpha,\beta)$ of words $(\mathcal{V}\cup \mathcal{A})^{*}$. Such an equation is represented by $\alpha=\beta$. It is therefore necessary to see the two chosen words as the left and right members respectively of an equation.
		
		\item[D2.]  An equation is say to be "\NewTerm{non-trivial equation}" if $\alpha\neq \beta$.
		
		\begin{tcolorbox}[colframe=black,colback=white,sharp corners]
		\textbf{{\Large \ding{45}}Example:}\\\\
		Given $\mathcal{V}=\{x\}$ and $\mathcal{A}=\{a\}$ and let us define:
		
		then we have the following equation in words:
		
		\end{tcolorbox}

		\item[D3.] An equation $e$ is say to be an "\NewTerm{equation without constant}" if $\alpha,\beta\in\mathcal{V}^{*}$.

		\item[D4.] A "\NewTerm{solution}" of the equation $e$ is a monoid homomorphism (\SeeChapter{see section Set Theory page \pageref{homomorphism of monoid}}):
		
		invariant (because every letter on $\mathcal{A}$ is sent on $\mathcal{A}^{*}$ and therefore every word on $\mathcal{A}$ is sent on $\mathcal{A}$) on $\mathcal{A}$ such that:
		
		\begin{tcolorbox}[title=Remark,colframe=black,arc=10pt]
		Let us recall that the definition of the homomorphism is such that if $\alpha=xy$ then:
		
		\end{tcolorbox}
		\begin{tcolorbox}[colframe=black,colback=white,sharp corners]
		\textbf{{\Large \ding{45}}Example:}\\\\
		Given $\mathcal{A}=\{a,b\}$ and $\mathcal{V}=\{x,y\}$. Let us wow consider the following words:
		
		let us define $h$ such that it sends $x$ on $b$, $y$ on $a$, $a$ on $a$, $b$ on $b$. Therefore we have well:
		
		and we will always have for every couple:
		
		\end{tcolorbox}
		
		\item[D5.] A solution $h$ is say to be a "\NewTerm{cyclic solution}" if there exists a word $w$ (belonging to $\mathcal{A}$) such that $h(x)\in w^{E}$ (considering the word itself as an alphabet therefore) for any variable $x$.
	\end{enumerate}
	
	
	\subsubsection{Codes}
	\textbf{Definition (\#\mydef):} We name "\NewTerm{code}\index{code}" any part $\mathcal{C}$ of a free monoid $\mathcal{A}^{*}$ that satisfies the following condition for any (word) $x_1,\ldots,x_n,y_1,\ldots,y_m\in \mathcal{C}$:
	
	In other words, $\mathcal{C}C$ is a code if every word of $\mathcal{C}^{*}$ (word composed of words) \underline{uniquely factorize} into a product words of $\mathcal{C}$. When a set is not a code, general we see it quite easily.
	\begin{tcolorbox}[colframe=black,colback=white,sharp corners]
	\textbf{{\Large \ding{45}}Examples:}\\\\	
	E1. The set (language) $\{a,ab,ba\}$ is not a code since the word $aba$ can written both as product $a\cdot ba$ and as product $ab\cdot a$.\\
	\begin{tcolorbox}[title=Remark,colframe=black,arc=10pt]
	The simplest codes are the "\NewTerm{uniform codes}\index{uniform codes}". These are sets whose all words have the same length (which means that since each word is different, the combination of words can hardly differ).
	\end{tcolorbox}
	\phantom \\
	E2. The set $\mathcal{A}^{*}$ of the words of length $n$ is a code, if $n\ge 1$. The ASCII code that associates to some characters binary words of length $7$ (see ASCII table) with some characters is an example of uniform code.
	\end{tcolorbox}
	
	\pagebreak
	\paragraph{Prefix codes}\mbox{}\\\\
	\textbf{Definition (\#\mydef):} A "\NewTerm{prefix code}\index{prefix code}" is a type of code system (typically a variable-length code) distinguished by its possession of the "prefix property", which requires that there is no whole code word in the system that is a prefix (initial segment) of any other code word in the system. 
	
	Prefix codes are also known as "\NewTerm{prefix-free codes}", "\NewTerm{prefix condition codes}" and "\NewTerm{instantaneous codes}". Although Huffman coding is just one of many algorithms for deriving prefix codes, prefix codes are also widely referred to as "Huffman codes" (see further below), even when the code was not produced by a Huffman algorithm. 
	
	Using prefix codes, a message can be transmitted as a sequence of concatenated code words, without any out-of-band markers or (alternatively) special markers between words to frame the words in the message. The recipient can decode the message unambiguously, by repeatedly finding and removing sequences that form valid code words. This is not generally possible with codes that lack the prefix property
	\begin{tcolorbox}[title=Remarks,colframe=black,arc=10pt]
	\textbf{R1.} The variable-length Huffman codes, country calling codes, the country and publisher parts of ISBNs, the Secondary Synchronization Codes used in the UMTS W-CDMA 3G Wireless Standard, and the instruction sets (machine language) of most computer microarchitectures are prefix codes.\\
	
	\textbf{R2.} Prefix codes are not error-correcting codes. In practice, a message might first be compressed with a prefix code, and then encoded again with channel coding (including error correction) before transmission.
	\end{tcolorbox}
	The Morse codes encodes the letter \texttt{E}, the most frequent, with a '.' And the letter \texttt{Y}, more rare, by '-.--': this is an example of a variable length code, which makes it possible to represent the most frequent letters or words by shorter words. 
	\begin{figure}[H]
		\centering
		\includegraphics[scale=1]{img/computing/morse_code.jpg}
		\caption{'Hello World' in Morse code}
	\end{figure}
	An important property is the uniqueness of the decoding (injective application), a problem that does not arise for codes of constant length. It can be solved, but too costly, when a special symbol separates two successive words of the code (the "blank" in the case of the Morse code). We can therefore not use such a non-constant length code if no word is the prefix of another code word. And as the reader has probably understand it now, a code with this property is named a "prefix code".
	
	\begin{tcolorbox}[colframe=black,colback=white,sharp corners]
	\textbf{{\Large \ding{45}}Example:}\\\\	
	Suppose we decide on a variable-size code convention, which matches, among other things, the following values:
	\begin{table}[H]
		\centering
		\begin{tabular}{|l|l|}
		\hline
		\rowcolor[HTML]{9B9B9B} 
		\textbf{Character} & \textbf{Code} \\ \hline
		$0$ & $11$ \\ \hline
		$2$ & $11010$ \\ \hline
		$12$ & $00$ \\ \hline
		$127$ & $0111100$ \\ \hline
		$255$ & $0100$ \\ \hline
		\end{tabular}
	\end{table}
	Let us suppose that we have to decode the sequence: 
	\begin{center}
		$1101000111100$
	\end{center}

	Several interpretations (factorization) are then possible:
	\begin{center}
		$1101000111100 = 11\; 0100\; 0111100 = 0\; 255\; 127$
	\end{center}
	or:
	\begin{center}
		$1101000111100 = 11010\; 00\; 11\; 11\; 00 = 2\; 12\; 0\; 0\; 12$
	\end{center}
	And now we are very embarrassed! With several equivalent possibilities between which one can not decide, one is incapable of retranscribing the initial code.\\

	The problem that has arisen here is that some codes are the beginning of other codes. Here, "$11$" is the code of the number $0$, but it is also the beginning of "$11010$", code of the number $2$. Hence the ambiguity!
	\end{tcolorbox}
	We then better understand the purpose and the name of "prefix codes". Thus, in order for there to be no ambiguity at the time of the decoding, we must absolutely have a prefix code if the code is not of constant length.
	\begin{tcolorbox}[colframe=black,colback=white,sharp corners]
	\textbf{{\Large \ding{45}}Example:}\\\\	
	The set:
	
	is a code. Here, knowledge of the beginning of a possible code $abababa$ does not yet make it possible to know whether the decomposition begins by $ab\cdot ab\cdot ab\cdot a$ or by $ababa\cdot ba$. It is only after reading the next letter (not indicated in this example) that we know if the decomposition starts with $ab$ (if the letter is $b$) or by $ababa$ (if the letter is $a$)
	\end{tcolorbox}
	
	\pagebreak
	\textbf{Definition (\#\mydef):} A code is with "\NewTerm{finite decryption delay}" if there exists an integer $d$ such that, whenever a message $w$ begins with $p=x_1x_2\ldots x_{d+1}$ with $x_1,x_2,\ldots, x_{d+1}\in X$ then the complete factorization of $w$ begins with $x_d$. It is therefore after a "delay" of $d$ code words that we can affirm that the first word found is the right one (as in the example above that is a $8$ finite decryption delay code).
	\begin{tcolorbox}[colframe=black,colback=white,sharp corners]
	\textbf{{\Large \ding{45}}Examples:}\\\\	
	E1. The code:
	
	has therefore a delay $d=0$.\\

	E2. The code:
	
	has therefore a delay $d=2$.
	\end{tcolorbox}
	
	\subsection{Linguistic algorithms}
	Let us put into practice what has already been seen so far in order to support us a little on "useful" concrete stuff!
	
	\subsubsection{Huffman algorithm}\label{huffman algorithm}
	Let us first recall that in computing, we decide to encode an integer that has a value between $0$ and $255$ by a sequence of $8$ binary digits (or "bits" in English, valued $0$ or $1$), also called byte (whose maximum value is equal to $2^8$).
	
	Even if there is a mathematical logic in the way of associating an $8$-bit binary number to an integer between $0$ and $255$ (\SeeChapter{see section Numerical Methods page \pageref{computer representation of numbers}}) we can imagine any coding of the type:
	\begin{table}[H]
		\centering
		\begin{tabular}{|l|l|}
		\hline
		\rowcolor[HTML]{9B9B9B} 
		\multicolumn{1}{|c|}{\cellcolor[HTML]{9B9B9B}Integer} & \multicolumn{1}{c|}{\cellcolor[HTML]{9B9B9B}Corresponding unique byte} \\ \hline
		$\ldots$ & $\ldots$ \\ \hline
		$138$ & $10001010$ \\ \hline
		$139$ & $10001011$ \\ \hline
		$\ldots$ & $\ldots$ \\ \hline
		\end{tabular}
	\end{table}
	in fact the correspondence can be any one, dictated by our imagination, as long as each integer between $0$ and $255$ is assigned to a fixed length binary code and only one. Once a correspondence is fixed, it is enough to take it as a convention.
	
	The byte defined according to this convention is the basic unit of data storage. Any computer file is a sequence of bytes arranged in a defined order. The size of the file is simply the number of bytes that constitute it. The kilobyte (KB) corresponds to $1024$ (not $1000$) bytes, the megabyte (MB) to $1024\times 1024$ bytes.

	It should be noticed that this representation in base $2$ is only a convention! Other conventions are possible, which would be equally appropriate if everyone agrees to use the same convention.

	The problem we are asking ourselves is: would there be another way of coding the numbers, perhaps less logical but more judicious, in such a way that the size of the same file rewritten according to the new convention would be smaller?

	The binary encoding convention is ultimately very democratic: whether you are a $0$ or a $255$, we allocate you $8$ bits anyway to be able to code you. In other words, each possible input (a number between $0$ and $255$) is encoded on $8$ bits. This is a fixed size encoding.

	From the point of view of our problem (data compression), it would not matter if each of the possible values ($0$...$255$) were represented as frequently as the others. But in general this is not the case.

	For example, see below the parsing of Wordpad.exe file (see your Accessories folder in your Microsoft Windows operating system\footnote{This plot date from 2001... so its rendering quality is very bad. We will do it again when we will have the time}...). In this plot, on the abscissa as on the ordinate are the possible values of a given octet (hence $0$ ... $255$). On the diagonal, at the abscissa and corresponding ordinate, the size of the circle is proportional to the number of bytes having this value in the file:
	\begin{figure}[H]
		\centering
		\includegraphics{img/computing/wordpad_byte_parsing.jpg}
		\caption[]{Analysis of the byte distribution of a the Wordpad.exe file}
	\end{figure}
	We see clearly that the values $0$, $128$ and $255$ are much more frequent than the others!

	As an indication, here are some values:
	\begin{table}[H]
		\centering
		\begin{tabular}{|l|l|l|}
		\hline
		\rowcolor[HTML]{9B9B9B} 
		\textbf{Value} & \textbf{Count} & \textbf{Frequency} \\ \hline
		$0$ & $71891$ & $34.4\%$ \\ \hline
		$2$ & $1119$ & $0.53\%$ \\ \hline
		$128$ & $1968$ & $0.94\%$ \\ \hline
		$130$ & $79$ & $0.038\%$ \\ \hline
		$255$ & $10422$ & $4.99\%$ \\ \hline
		\end{tabular}
	\end{table}
	We will now decide of a variable-sized coding convention, which represents a value that is frequent by a small number of bits, and an uncommon value by a large number of bits.
	
	We will now decide on a variable-sized coding convention, which represents a value that is frequent by a small number of bits, and an uncommon value by a large number of bits.

	For example, $0$ will now be represented by the sequence "$11$" (when before it was "$00000000$"), $128$ by "$1011010$" (when before it was "$10000000$"), $255$ by "$0100$" (when before it was "$11111111$"), etc.

	Given that $0$ represents almost one third of the file, we have gained a considerable place by coding it on two bits instead of eight! And same for the other frequent values...

	Therefore "\NewTerm{Huffman algorithm}\index{Huffman algorithm}" is a recipe for generating a variable-length prefix code from the frequency table of a sequence of values. So, if you have followed the theory so far, it is a solution to our problem.

	Suppose that our file is extremely simple, consisting of a single french word (\textit{unconstitutionally}):
	\begin{center}
		\texttt{anticonstitutionnellement}
	\end{center}
	There are $25$ characters in this file. Each character being encoded by an $8$-bit octet (ASCII encoding), this means $25$ bytes, or $200$ bits! Let's see what we can do with that.

	First let us render the table of frequencies:
	\begin{table}[H]
		\centering
		\begin{tabular}{|c|c|}
		\hline
		\rowcolor[HTML]{9B9B9B} 
		\multicolumn{1}{|l|}{\cellcolor[HTML]{9B9B9B}\textbf{Letter}} & \multicolumn{1}{l|}{\cellcolor[HTML]{9B9B9B}\textbf{Count}} \\ \hline
		\texttt{a} & $1$ \\ \hline
		\texttt{c} & $1$ \\ \hline
		\texttt{s} & $1$ \\ \hline
		\texttt{u} & $1$ \\ \hline
		\texttt{m} & $1$ \\ \hline
		\texttt{o} & $2$ \\ \hline
		\texttt{l} & $2$ \\ \hline
		\texttt{i} & $3$ \\ \hline
		\texttt{e} & $3$ \\ \hline
		\texttt{n} & $5$ \\ \hline
		\texttt{t} & $5$ \\ \hline
		\end{tabular}
	\end{table}
	All other bytes (strings) have a null frequency: they are not represented in the file.

	Now we create a "terminal node" for each entry of the array:
	\begin{figure}[H]
		\centering
		\includegraphics[scale=0.9]{img/computing/huffman_step_0.jpg}
	\end{figure}
	What makes for us now $11$ trees containing only one knot each.

	We now start an iteration: each time we delete the two trees on the left and replace them with a "sum tree". The new tree is inserted in the list in ascending order, and is repeated until there is only one tree left. Therefore we get:
	\begin{itemize}
		\item First Iteration:
		\begin{figure}[H]
			\centering
			\includegraphics[scale=1]{img/computing/huffman_step_1.jpg}
		\end{figure}
		
		\item Second iteration:
		\begin{figure}[H]
			\centering
			\includegraphics[scale=0.9]{img/computing/huffman_step_2.jpg}
		\end{figure}
		
		\item Third iteration:
		\begin{figure}[H]
			\centering
			\includegraphics[scale=1]{img/computing/huffman_step_3.jpg}
		\end{figure}
		
		\item ....
	\end{itemize}
	...and the final tree is:
	\begin{figure}[H]
		\centering
		\includegraphics[scale=0.75]{img/computing/huffman_step_final.jpg}
	\end{figure}
	And that's it!

	Now, the associated code to each letter is none other than the path to the corresponding terminal node from the root, noting $0$ for each left branch and $1$ for each right branch.

	Finally:
	\begin{table}[H]
		\centering
		\begin{tabular}{|c|c|}
		\hline
		\rowcolor[HTML]{9B9B9B} 
		\multicolumn{1}{|l|}{\cellcolor[HTML]{9B9B9B}\textbf{Letter}} & \multicolumn{1}{l|}{\cellcolor[HTML]{9B9B9B}\textbf{Huffman binary code}} \\ \hline
		\texttt{n} & $00$ \\ \hline
		\texttt{t} & $01$ \\ \hline
		\texttt{i} & $100$ \\ \hline
		\texttt{e} & $101$ \\ \hline
		\texttt{a} & $11000$ \\ \hline
		\texttt{c} & $11001$ \\ \hline
		\texttt{o} & $1101$ \\ \hline
		\texttt{l} & $1110$ \\ \hline
		\texttt{m} & $11110$ \\ \hline
		\texttt{s} & $111110$ \\ \hline
		\texttt{u} & $111111$ \\ \hline
		\end{tabular}
	\end{table}
	And here is now, transcribed with our new code, the starting word:
	\begin{center}
	{\small \texttt{110000001100110011101001111100110001111111011001101000010111101110101111101010001}}
	\end{center}
	which makes $81$ bits, instead of $200$ at the beginning! This corresponds to a compression ratio of almost $60\%$.
	
	The fact of having generated code using a binary tree ensures that no code can be the prefix of another one. You can verify that using the encoding table, there is no ambiguity possible to decode our compressed word! This is why the Huffman algorithm is in many compressed formats (for example the famous MP3 for audio encoding).
	
	\subsubsection{Sardinas and Patterson algorithm}
	When we have to deal with long codes, the difficulty is to check whether the code is really one... In order to do this, we can use the Sardinas and Patterson algorithm (the proof of this algorithm will be done during the next update of this section of the book).

	In coding theory, the Sardinas–Patterson algorithm is a classical algorithm for determining in polynomial time whether a given variable-length code is uniquely decodable, named after August Albert Sardinas and George W. Patterson, who published it in 1953.

	To do this check, we construct a graph $G(X)=(P,U)$, where $P$ is the set of non-empty prefixes (according to the definition of "prefixes" given earlier above) of words of $X$, and $U$ the set of pairs $(u, v)$ such that one of the following possibilities is met:
	\begin{enumerate}
		\item[P1.] $uv\in X$ (by eliminating duplicate pairs if necessary)

		\item[P2.] $v\notin X$ and it exist $x\in X$ such that $ux=v$
	\end{enumerate}
	\begin{tcolorbox}[colframe=black,colback=white,sharp corners]
	\textbf{{\Large \ding{45}}Example:}\\\\	
	For $X=\{a,bb,abbba,babab\}$, the set $P$ contains, in addition to $X$ (which are prefixes of $\varepsilon$), the words $\{b,ab,abb,abbb,b,ba,bab,baba\}$ (respectively prefixes of $\{b,bba,ba,a,abab,bab,ab,b\}$).\\

	First we see immediately that the set of $X=\{a,bb,abbba,babab\}$ is not a code because:
	
	Now the pairs of $U$ are for the first possibility P1:
	
	and for the second possibility P2:
	\end{tcolorbox}
	
	\begin{tcolorbox}[colframe=black,colback=white,sharp corners]
	
	for which the $x$ which is used to form the $v$ is respectively $bb$, $bb$, $a$, $a$.\\
	
	Sardinas and Patterson will therefore be:
	\begin{figure}[H]
		\centering
		\includegraphics{img/computing/sardinas_patterson_example_graph.jpg}
		\caption{Sardinas and Patterson example graph}
	\end{figure}
	where the vertices corresponding to the words of $X$ are doubly circled. The label of each arc is a word of the set $X$. The "crossed arcs" are traced in dotted lines and the "front arcs" in solid lines. If the arc $(u,v)$ is crossed, then the label is $uv$, otherwise it is the word $x$ such that $ux = v$.
	\end{tcolorbox}
	In our example above, there is a path from $a$ to $a$. By virtue of the Sardinas and Patterson theorem (which we will proved in the next update of this section), the set $X$ is not a code (a single and unique path between any two vertices of $X$ is enough).
	
	The set $X$ is a code if and only if there is no non trivial path in $G(x)$ of a vertex of $X$ to a vertex of $X$ (in other words, a set $X$ is a code if and only if there is only the one and only trivial path leading from one vertex of $X$ to another one - this vertex may be the same as in the previous example).
	
	\begin{flushright}
	\begin{tabular}{l c}
	\circled{60} & \pbox{20cm}{\score{2}{5} \\ {\tiny 10 votes,  46.00\%}} 
	\end{tabular} 
	\end{flushright}

	%to make section start on odd page
	\newpage
	\thispagestyle{empty}
	\mbox{}
	\section{Cryptography}\label{cryptography}
	\lettrine[lines=4]{\color{BrickRed}C}ryptography is one of the disciplines of cryptology endeavouring to protect messages (ensuring confidentiality and / or authenticity) that two people wish to share through an insecure channel often thanks to secrets or keys.\\
	
	The history of cryptography is already long and exciting (since it is a kind of "game"). We report its first use in Egypt 4,000 years ago. However, for centuries, the methods used were often remained very primitive. Moreover, its implementation was limited to the needs of the army and diplomacy. Thus, encryption methods and cryptanalysis (code breaking) experienced an important development during the Second World War and had a profound influence on the course of it.
	
	At the end of the 20th century (especially!), with the proliferation of computers and electronic communications media, it became increasingly important to use secret codes for transmitting data between the military or private organizations. Thus, engineers have had to look at this same time solid numerical methods whose implementation and use was within reach of almost everyone (nation, enterprise and individual) while ensuring that external attacks required tools out of reach of an individual or group of individuals equipped with standard and high-performance IT tools (in computing power). Engineers and researchers then plunged into the mathematical tools to search for satisfying these specifications and the most known systems, mathematical theories which were adopted had over 200 years old (apart quantum cryptography).
	
	The growth of cryptographic technology has raised a number of legal issues in the information age. Cryptography's potential for use as a tool for espionage and sedition has led many governments to classify it as a weapon and to limit or even prohibit its use and export. In some jurisdictions where the use of cryptography is legal, laws permit investigators to compel the disclosure of encryption keys for documents relevant to an investigation Cryptography also plays a major role in digital rights management and piracy of digital media.
	
	Steganography techniques (art of concealing a message in another one or in an image) however must be preserved because nothing tells us that computing power will still be available in times of war. It should be noted also that the steganography deployed wealth of imagination. Note for example: the permutations of letters, special and subtle formatting of characters, use of synonyms, hidden messages in text or comma behind a stamp, inside shots chess (hence the fact that these games were banned by the USA for some years after the attack on Pearl Harbor), in pictures / drawings, musical scores, etc. All of these techniques make that during the Second World War, the office of censorship in the United States occupied 10,000 full-time employees that analysed the mail of citizens, classified ads, radio text, etc.
	
	\begin{tcolorbox}[title=Remarks,colframe=black,arc=10pt]
	\textbf{R1.} To address the foundations of the theory of cryptography, we advise the reader to have read at first and at least in diagonal the section on Number Theory, on Set Theory, on Numerical Methods (especially the subsection on computational complexity), on Numerical Systems , on Statistical Mechanics (where information theory can be found) and for the part about quantum cryptography: the section of on Quantum Computing.\\
	
	\textbf{R2.} We must remain aware that cryptography is more an engineer science than physicist science (except with quantum cryptography) and we must then not be so surprised to see some algorithms like fallen just from nowhere and adopted by industry because they just work almost well... Furthermore, it is also certain that only a few years after writing this text it will already be considered as obsolete (that is the art of engineering ... planned obsolescence).
	\end{tcolorbox}
	
	\subsection{Cryptographic systems}
	
	\textbf{Definitions (\#\mydef):}
	
	A "\NewTerm{cryptographic system}\index{cryptographic system}" is composed of:
	\begin{enumerate}
		\item[D1.] A finite set $P$ named "\NewTerm{space of clear texts}\index{space of clear texts}".
		
		\item[D2.] A finite set $C$ named "\NewTerm{space of encrypted texts}\index{space of encrypted texts}".
		
		\item[D3.] A finite set $K$ named  the "\NewTerm{space of keys}\index{space of keys}".
	\end{enumerate}
	For each key $k$, we seek an encryption function $e_k$:
	
	and a deciphering (decryption) function $d_k$:
	
	such as (\SeeChapter{see section Set Theory page \pageref{identity application}}):
	
	In other words, these two functions must be injective!
	
	To achieve this, two types of cryptographic techniques are mainly distinguished, encompassing almost all known modern encryption methods of the 20th century (for mathematical details see below):
	\begin{enumerate}
		\item The first concern cryptosystems with "\NewTerm{symmetrical secret key}\index{symmetrical secret key}".
		\begin{tcolorbox}[title=Remark,colframe=black,arc=10pt]
		Public keys often refer to the DES protocol (see below) for: Data Encryption System.
		\end{tcolorbox}
		
		\item The second concerning encryption systems with "\NewTerm{asymmetric public key}\index{asymmetric public key}".
		\begin{tcolorbox}[title=Remark,colframe=black,arc=10pt]
		This type of key often refers for example to the RSA protocol, the names of those to whom we awarded the Development: Rivest, Shamir and Adleman. They are widely used thanks to their rapid time encryption and decryption as well as their high entropy (see definition below).
		\end{tcolorbox}
	\end{enumerate}
	By nature, these two types of keys are very different. Let us try to understand the reasons:
		
		A symmetric encryption means a system where the key used in the encryption operation is that used in the deciphering operation. In this case, during a secure exchange (assumed to be as), both sides of the correspondence must share a same secret: the used key or "\NewTerm{session key}\index{session key}".
		
		An asymmetric encryption designates an encryption system in which the key used for encryption (private key of the sender) differs from that used for decryption (recipient's private key). The only exchange that exists between members of the group is the public key, which allows each member to adjust its encryption based on the private key of the other members (among the many who have asymmetric systems been proposed, one of the most widespread in the early 21st century is the RSA).
		
		The symmetric key ciphers are traditionally classified into two groups:  "\NewTerm{stream ciphers}\index{stream ciphers}" and "\NewTerm{block ciphers}\index{block ciphers}".
		
		\textbf{Definitions (\#\mydef):}
		\begin{enumerate}
			\item[D1.] A "\NewTerm{block cipher}\index{block cipher}" is an encryption algorithm that encrypts a fixed size of $n$-bits of data - known as a block - at one time. The usual sizes of each block are $64$ bits, $128$ bits, and $256$ bits. So for example, a $64$-bit block cipher will take in $64$ bits of plaintext and encrypt it into $64$ bits of ciphertext. In cases where bits of plaintext is shorter than the block size, padding schemes are called into play. Majority of the symmetric ciphers used today are actually block ciphers. DES, Triple DES, AES, IDEA, and Blowfish are some of the commonly used encryption algorithms that fall under this group.  
	
				\item[D2.] A "\NewTerm{stream cipher}\index{stream cipher}" is an encryption algorithm that encrypts $1$ bit or byte of plaintext at a time. It uses an infinite stream of pseudorandom bits as the key. For a stream cipher implementation to remain secure, its pseudorandom generator should be unpredictable and the key should never be reused. Stream ciphers are designed to approximate an idealized cipher, known as the One-Time Pad. RC4, which stands for Rivest Cipher 4, is the most widely used of all stream ciphers, particularly in software. The cypher Engima machine of the second World Was is also a famous application of stream cipher.
		\end{enumerate}
		
	\begin{tcolorbox}[colframe=black,colback=white,sharp corners]
	\textbf{{\Large \ding{45}}Example:}\\\\
	Without going in the mechanical and electrical description of Enignma, the reader hast just to know that second version of the Enigma\label{enigma} cypher machine had first a box $5$ rotors with $26$ start positions! The user had to choose $3$ of theses $5$ rotors and put them in a given position (position order has an importance!) in the machine.

	So we have:
	
	combinations to put $3$ rotors in a given order choosing among $5$ rotors ($5$ rotors for the first  position, multiplied the $4$ remaining rotors for the second position and so on...).
	
	After the Enigma user had to choose among on of the $26$ position of each rotor. Then the number of starting positions possibilities is equal to:
	
	Finally the business and military version of the Enigma machine had something extra: plugboard.
	
	This plug-board has $10$ wires that connect two letter together among 26 letters (there $2\cdot 20$ letters combine together). The combination is therefore:
	
	Indeed, we have $26!$ combinations of letters, but as there are $10$ cables for therefore $20$ letters we don't care about the combinations of $6$ of them. Hence the division by $6!$. We divide by $10!$ as for all the remaining combination of letters we can use only $10!$ of them. Since we don't care about the direction of the cables (from $A$ to $B$ or $B$ to $C$) and that we have $10$ cables, we must divide $10$ times by $2$ and this is equivalent as dividing by $2^{10}$.
	So finally the total is:
	
	This is the total number of ways you can set the Enigma machine...
	\end{tcolorbox}
		
		\begin{tcolorbox}[title=Remarks,colframe=black,arc=10pt]
		\textbf{R1.} In 2001, Microsoft Internet Explorer (Microsoft's web browser in this time) worked with a 1024-bit asynchronous system certified by a synchronous system and Adobe Acrobat (PDF) in 2004 with an AES (Advanced Encryption System) of 128 bits for the low protection as well in the years 2010-2015 the iPhone 4S and 5.\\
		
		\textbf{R2.} Microsoft Windows Enterprise and its E.F.S. system (Encrypting File System) uses a symmetric key (to encrypt the file) named  "File Encryption Key" and asymmetric cryptography to encrypt the symmetric key in the file header as shown below (the key being updated regularly via Windows Update root certificates):
		\begin{figure}[H]
			\centering
			\includegraphics{img/computing/cryptography_FEK.jpg}
			\caption[Principle of F.E.K. in Microsoft Windows O.S.]{Principle of F.E.K. in Microsoft Windows O.S. (source: Wikipedia)}
		\end{figure}
		However, the cryptography keys for EFS are in practice protected by the user account password, and are therefore susceptible to most password attacks. In other words, encryption of files is only as strong as the password to unlock the decryption key.
		\end{tcolorbox}
		These methods are still decipherable, provided that the interceptor has enough time and paper/money (excepted at this date for quantum encryption).
		
		Here is a small summary table of broken keys and their respective size for both conventional systems:
		\begin{table}[H]
	\begin{center}
		\begin{tabular}{|c|c|}
			\hline
			\multicolumn{2}{|c|}{\cellcolor{black!30}\textbf{Secret key (symmetric system)}} \\
			\multicolumn{2}{|c|}{\cellcolor{black!30}exhaustive search} \\
			\hline
			\cellcolor{black!30}Number of bits & \cellcolor{black!30}Year \\
			\hline
			$40$ & Broken in 1995 \\
			\hline
			$56$ & Broken in 1998 \\
			\hline
			$64$ & Brokable \\
			\hline
			$128$ & Brokable in $\sim$2100 \\
			\hline
			$256$ & ? \\
			\hline
		\end{tabular}
		\begin{tabular}{|c|c|}
			\hline
			\multicolumn{2}{|c|}{\cellcolor{black!30}\textbf{RSA public key (asymmetric system)}} \\
			\multicolumn{2}{|c|}{\cellcolor{black!30}exhaustive search} \\
			\hline
			\cellcolor{black!30}Number of bits & \cellcolor{black!30}Year \\
			\hline
			$256$ & Broken in 1985 \\
			\hline
			$512$ & Broken in 1999 \\
			\hline
			$1024$ & Broken in 2010 \\
			\hline
			$2048$ & Brokable in $\sim$2100 \\
			\hline
			$4096$ & ? \\
			\hline
		\end{tabular}
		\caption{Key systems and recent broke}
	\end{center}
	\end{table}
	
	\subsubsection{Kerckhoffs' principle}
	The primary function of cryptography is therefore to ensure the confidentiality of information exchange. Two parts of a confidential exchange will first agree on a secret convention to write their messages, and if they have carefully chosen, no one else should be able to enter their exchange.
	
	If the secrecy of such agreements is possible from a few isolated individuals for a limited period, it is inconceivable at  large scale and for a fairly long period. This is what Auguste Kerckhoffs understood when establishing the basic principles of practical cryptography which requires a fundamental principle encryption system "that does not require secrecy, and which can conveniently fall into the hands the enemy".
	
	The six postulates of Kerchoffs are:
	\begin{enumerate}
		\item The system must be practically, if not mathematically, indecipherable;
		\item It should not require secrecy, and it should not be a problem if it falls into enemy hands;
		\item It must be possible to communicate and remember the key without using written notes, and correspondents must be able to change or modify it at will;
		\item It must be applicable to telegraph communications;
		\item It must be portable, and should not require several persons to handle or operate;
		\item Lastly, given the circumstances in which it is to be used, the system must be easy to use and should not be stressful to use or require its users to know and comply with a long list of rules.
	\end{enumerate}
	The second postulate, known today as the "\NewTerm{Kerckhoffs principle}\index{Kerckhoffs principle}"  states that the security of an encryption system is not based on the secrecy of the procedure, but only on one parameter used when its implemented: the key. This key is the only secret of the Exchange Agreement.
	
	This principle, however, was reformulated by Claude Shannon: "the enemy knows the system". This formulation is known as the "Shannon's maxim". This is the principle usually adopted by cryptologists, as opposed to the security through obscurity.
	
	\subsection{Traps}
	Sometimes there are what we call the "trap doors" in public and secret keys. This is because when generating the key, which hast to be done randomly within certain predefined theoretical constraints, the random generator may have an issue (the issue is sometimes voluntary on the part of the supplier of the material for spy purpose...).
	
	In the secret keys, the traps are located at the level of the "\NewTerm{key's entropy}\index{key's entropy}" (\SeeChapter{see section Statistical Mechanics page \pageref{entropy}}), directly linked to the entropy of the random generator. We can simplistically define the entropy of a key generator by the average optimal binary questions (that is to say giving rise to the type of answers Yes / No) that we need to ask someone knowing a key produced by this generator to determine it. More the entropy of a key generator is high, higher is the number of questions we need to determine the key. Conversely, the smaller is the entropy, the lower are the questions, so that the search of a key is facilitated.
	
	The introduction of traps in the asymmetric key systems is much more difficult, since this type of key already has intrinsic mathematical structure: their construction is not due to chance but is the result of mathematical rules. Chance is here in the choice of large prime numbers used. The fact that asymmetric systems can be easily calculated, but they are difficult to reverse are sometimes named "trapdoor functions".

	\begin{tcolorbox}[colframe=black,colback=white,sharp corners]
	\textbf{{\Large \ding{45}}Example:}\\\\
	An example of a simple mathematical trapdoor is "$6895601$ is the product of two prime numbers. What are those numbers?" A typical solution would be to try dividing $6895601$ by several prime numbers until finding the answer. However, if one is told that $1931$ is one of the numbers, one can find the answer by entering "$6895601\div 1931$" into any calculator. This example is not a sturdy trapdoor function – modern computers can guess all of the possible answers within a second – but this sample problem could be improved by using the product of two much larger primes.\\
	
	Therefore if a random generator that generates prime numbers is biased (\SeeChapter{see section Statistics page \pageref{likelihood estimators}}), this bias will facilitate the research of a trapdoor.
	\end{tcolorbox}
	
	\textbf{Definition (\#\mydef):} A "\NewTerm{trapdoor function}\index{trapdoor function}" is a function that is easy to compute in one direction, yet difficult to compute in the opposite direction (finding its inverse) without special information.
	
	\begin{tcolorbox}[title=Remarks,colframe=black,arc=10pt]
	\textbf{R1.} Functions related to the hardness of the discrete logarithm problem (either modulo a prime or in a group defined over an elliptic curve) are not known to be trapdoor functions, because there is no known "trapdoor" information about the group that enables the efficient computation of discrete logarithms.\\
	
	\textbf{R2.} Trapdoor must not to be confused with a "\NewTerm{backdoor}\index{backdoor}" as this latter is a deliberate mechanism that is added to a cryptographic algorithm or operating system, for example, that permits one or more unauthorized parties to bypass or subvert the security of the system in some fashion.
	\end{tcolorbox}
	
	\subsection{Secret-key encryption system}
	\textbf{Definition (\#\mydef):} The "\NewTerm{single-use encryption}\index{single-use encryption}" is a secret key encryption algorithm proved unconditionally secure. Properly used (and that's an important point), it provides an unbreakable encryption in reasonable time.
	
	The theoretical basis of this encryption system are:
	
	Given a message $M$ in binary form to be transmitted between people $A$ (creator and originator of the message $M$) and $B$ (reader and receiver). We generate a large amount of bits if possible "truly randomly" forming a secret key $K$ of same size as the message to be transmitted (computer programs, deterministic by nature, can not generate truly random bits).
	
	This key will be sent to $B$ by a channel supposedly safe ... A given time after the transmission of this key $A$ will encode his message into $C$ by performing the operation:
	
	where $\star$ is an operator that must satisfy to a group law (\SeeChapter{see section Set Theory page \pageref{group law}}) on a finite set (that contains a limited number of items or "letters").
	
	The idea in computing science is to use the XOR law (exclusive OR) denoted $\oplus$ for what will follow (\SeeChapter{see section Logical Systems page \pageref{boolean operators}}) as it is enough as a group law (remember that a group is the smallest structure having an opposite - and is associative and having neutral element - that gives therefore the possibility to reverse the encoding process). Therefore:
	
	Finally, the sender $A$ transmits the encrypted version of his message $C$ by a route not necessarily secure. $B$ can read the original message $M$ by using the inverse operator $\oplus^{-1}$ (the XOR operator is its own inverse as we have proved it thanks to its the truth table in the section of Logical Systems!!!!). So receiver $B$ will do the following:
	
	Provided that the $K$ has been generated totally randomly and that each bit of it has been used only once to encrypt the message, an interceptor gets no information about the clear message $M$ if he intercepts $C$. Indeed, in these conditions, we can not establish any correlation between $M$ and $C$ without the knowledge of $K$.
	
	Even with future ultra-powerful quantum computers, the problem is insoluble, because nothing connects the information which is available and the problem to solve. Consequently, the "single-use encoding" is an encryption algorithm "unconditionally secure". The proof of its security does not rely on unproven mathematical conjectures and decryption attempts of an interceptor with infinite computing power are futile.
	
	However, each stage of encryption is a source of possible errors. Indeed, the key $K$ may have been poorly developed. The slightest statistical deviation of $K$ compared to the "real" random provides information on the clear message $M$ from its encrypted version. This is why the $K$ bits are to be used only once if possible.
	
	Indeed, suppose that same key is used to encrypt messages of French language $M_1$ and $M_2$ an attacker manages to intercept the two corresponding encrypted messages. From $C_1$ and $C_2$ the interceptor and can easily obtain information about $K$ and this because of language peculiarities (same for English). Indeed, since:
	
	then the interceptor knows a simple result that involves $M_1$ and $M_2$ without the key $K$:
	
	as:
	
	(if necessary make the truth table to be convinced of this relation). Now, if $M_1$ and $M_2$ are in the same language, we will know, usually due to language redundancies (e.g. the letter "e" often appears in French), found from $C_1\oplus C_2$ each of the two original messages (the work is though laborious without statistical automated tools).
	
	\begin{tcolorbox}[colframe=black,colback=white,sharp corners]
	\textbf{{\Large \ding{45}}Example:}\\\\
	Imagine that we want to send a little message $M$ binary coded by $1101$ and we generated a random key $K$ that gave $0101$.\\
	
	Then we have:
	
	and therefore:
	
	\end{tcolorbox}
	Obviously in this kind of small situations we can guess $M$ without much difficulty just by having $C$ if there such like here only a single encryption step. This is why there are encoding patterns as we shall see now.
	
	The main problem with this technique is the creation of a key as random as possible. To overcome this, mathematicians do pass the key through a series of nested functions, the result after many iterations, becomes "pseudo-random".
	
	Building a pseudo-random iteration is one thing, building a pseudo-random bijection is yet another!!! Indeed, we need to decrypt the message later, which is why we absolutely need a bijective system (which has everything arrival element - encrypted message - matches a single starting element - decrypted message - and vice versa).
	
	\subsubsection{Feistel Schemes}
	Even if encryption algorithms in this late 20th century and early 21st century consider sufficient with a key having a finite number of bits, the goal remains the development that from a message $M$ and a random sequence of digits, or at least that looks like, to build a key $K$ to send an encrypted message $C$ that can be decipher easily only by people knowing the key. Specifically, this target application is to construct or identify a function which, firstly, do correspond to each digit of $M$ a digit $C$ that seems to look random (but whose value depends in reality on the deterministic key) and, secondly, authorizing the reverse path (inverse function by the property of bijection), that is to say that from a digit $C$, we can uniquely trace back to the corresponding digit of $M$. We therefore would like to find a pseudo-random bijection function.
	
	In the years 1950s, the mathematician Horst Feistel has shown that a pseudo-random function transformed itself, by a relatively simple method, in bijection function. Today, the "\NewTerm{Feistel cypher}\index{Feistel cypher}" is most commonly used in the secret key encryption systems and is also the basis of the DES (Data Encryption System). How does it work?
	
	Here is the principle:
	
	The initial message to be encrypted has a size of $2n$ bits. The split the original message $M$ into two blocks (thus the Feistel Schemes belongs ton the family of "\NewTerm{block cipher}\index{block cipher}"), $G$ and $D$, of equal length ($G$ includes the first $n$ bits and $D$ the following) and we build the transformation $\varphi$ that associates to $G$ and $D$ the numbers $T$ and $S$ such as:
	
	where for reminder the $\oplus$ still represents the bit by bit XOR operation and where $f_1$ is any function, non-necessarily bijective, from $n$ bits to $n$ bit using the secret key $K$.
	
	The transformation $\varphi(G,D)=(S,T)$ is indeed bijective, as we can go back in a unambiguous way starting form $S$ and $T$ to $G$ and $D$ by the operations:
	
	Obviously we must not stop here, since the right side of the message, $D$, has not been encrypted, it is simply passed to the left. However, as $\varphi$ is bijective, we can repeat the process. A Feistel scheme where we apply $n$ times the function $\varphi$ is named a "\NewTerm{$n$-step pattern}\index{$n$-step pattern}".
	
	\begin{tcolorbox}[colframe=black,colback=white,sharp corners]
	\textbf{{\Large \ding{45}}Example:}\\\\
	We will encrypt by the a two-step Feistel cypher a message consisting of $4$ bits (thus $2^4=16$ possibilities of messages), what is equivalent to building a bijection from $4$ bits to $4$bits from two functions $f_1,f_2$ of two bits to two bits . The functions $f_1,f_2$ have in input both: the message to encrypt and the secret key. We will assume that for some input key, these functions are:
	\begin{table}[H]
	\centering
		\begin{tabular}{|c|c|c|l|c|c|c|}
		\cline{1-3} \cline{5-7}
		\multicolumn{1}{|l|}{\cellcolor[HTML]{9B9B9B}{\color[HTML]{333333} \textbf{Input}}} & \multicolumn{1}{l|}{\cellcolor[HTML]{9B9B9B}{\color[HTML]{333333} \textbf{$f_1$}}} & \multicolumn{1}{l|}{\cellcolor[HTML]{9B9B9B}{\color[HTML]{333333} \textbf{Output}}} &  & \multicolumn{1}{l|}{\cellcolor[HTML]{9B9B9B}{\color[HTML]{333333} \textbf{Input}}} & \multicolumn{1}{l|}{\cellcolor[HTML]{9B9B9B}{\color[HTML]{333333} \textbf{$f_2$}}} & \multicolumn{1}{l|}{\cellcolor[HTML]{9B9B9B}{\color[HTML]{333333} \textbf{Output}}} \\ \cline{1-3} \cline{5-7} 
		$00$ & $\rightarrow$ & $01$ &  & $00$ & $\rightarrow$ & $11$ \\ \cline{1-3} \cline{5-7} 
		$01$ & $\rightarrow$ & $11$ &  & $01$ & $\rightarrow$ & $00$ \\ \cline{1-3} \cline{5-7} 
		$10$ & $\rightarrow$ & $10$ &  & $10$ & $\rightarrow$ & $00$ \\ \cline{1-3} \cline{5-7} 
		$11$ & $\rightarrow$ & $01$ &  & $11$ & $\rightarrow$ & $01$ \\ \cline{1-3} \cline{5-7} 
		\end{tabular}
		\caption{Input/Output key matches by functions}
	\end{table}
	Let us notice that neither $f_1$ nor $f_2$ are bijections ($f_1(00)=f_1(11)=01$,$f_2(01)=f_2(10)=00$). For example, encrypt the message 1101. $G$ designates the left part of the message to be encrypted, $D$ the right part:
	\begin{figure}[H]
		\centering
		\includegraphics{img/computing/feistel_encryption_simple_example.jpg}
		\caption{Encryption of $1101$ using the Feistel method}
	\end{figure}
	The result is $0010$. We will compute the image of the other $15$ other possible messages and verify that there is an unambiguous correspondence between each message and its image by the Feistel scheme: we have constructed a bijection from two functions that are not bijective.
	\end{tcolorbox}

	\pagebreak
	Quite complex theoretical results guarantee the cryptographic security of Feistel schemes starting from $4$ steps when $n$ is large enough and when the functions $f_i$ are indistinguishable from truly random functions. In practice, rather than using $4$ steps and functions $f_i$ that look like random, it is generally preferred to use more steps and more simple functions $f_i$. After a few steps, the obtained bijection often becomes very difficult to distinguish from random bijections. And for parameters well chosen, we no longer know at all how to distinguish them from truly random bijections!!!

	Most of the secret key encryption algorithms currently in this end of 20th century used in the civilian world are Feistel schemas. In particular, the DES (Data Encryption System) algorithm  which is a $16$-step Feistel scheme as shown in the figure below and the Triple DES (TDES) algorithm which is a $48$-step Feistal scheme and the Blowfish algorithm that will not be discussed here).
	
	\begin{tcolorbox}[title=Remark,colframe=black,arc=10pt]
	For example, there are, in some bank cards (at least in the beginning of this 21st century...), a DES key (or TDES since October 2001) which provides proof of the legitimacy of the card between the bank's control center and the merchant's terminal in addition to Public part of an RSA key to make sure the user code is entered (control done by an internal chip on the card, which must then be manufactured in very secure premises).
	\end{tcolorbox}
	Rigorously the Feistel scheme is a bit different because it involves keys, which we did not use in the example presented before. Here is a more detailed figure in what this Feistel scheme consists of (see figures below).

	Principle of the diagram: A message to be encrypted is divided into blocks of $64$ bits, each of which is divided into two $32$-bit sub-blocks, the left block ($G$) and the right-hand block ($D$). At each iteration, the old right block becomes the new left block and the new right block results from the XOR operation of the old right block, whose bits are mixed by a confusion function, and of the previous left block. The iteration is repeated $16$ times.
	\begin{figure}[H]
		\centering
		\includegraphics{img/computing/feistel_encryption_more_realistic.jpg}
		\caption[Feistel's scheme]{Feistel's scheme a little more realistic (source: ?)}
	\end{figure}
	The confusion function (\textbf{f}), which acts on the $32$-bit blocks, mixes the bits according to the following processes (see the figure below):
	\begin{itemize}
		\item First, it transforms the $32$-bit block into a $48$-bit block by duplicating certain bits ("expansion"). 

		\item Then, it adds to this block a $48$-bit ("token key") subkey extracted from the $56$-bit secret key 

		\item And then transforms each $6$-bit set into $4$ bits by local transformations (\textbf{S} transform)
	\end{itemize}
	The result is a $32$-bit block which is finally mixed according to a fixed permutation.
	
	\begin{figure}[H]
		\centering
		\includegraphics{img/computing/feistel_confusion_function_diagram.jpg}
		\caption[Feistel's scheme confusion function diagram]{Feistel's scheme confusion function diagram (source: ?)}
	\end{figure}
	
	\pagebreak
	\subsection{Public key encryption}
	In 1975, W. Diffie and M. E. Hellman revolutionized the science of cryptography by proving the existence of a protocol that could not be deciphered by an interceptor unless the interceptor had large computer resources. The most fascinating in their method - the principle of which is still in use in this early 21st century - is that the code used does not require to hide the method used and can be used repeatedly without any modification (Kerckhoffs principle). At their time, they simply created the concept of "public-key cryptography", or "asymmetric cryptography" (which we mentioned earlier in this section), an invention that sparked the emergence of a dynamic academic and industrial community.
	\begin{tcolorbox}[title=Remark,colframe=black,arc=10pt]
	Contrary to what one might think, public key cryptography has not relegated secret key cryptography to oblivion, on the contrary: these two types of cryptography are most often used in hybrid cryptosystems where the authentication of published keys is performed by a "certification authority".
	\end{tcolorbox}
	Before describing the Diffie-Hellman protocol  in detail, let us recall that the protocol of exchange of the "secret keys" was not reliable at that time (and is still not today) as it was transiting between the interlocutors, the element making it possible to encrypt and therefore decrypt the messages. In addition, even if only one key were to travel, anyone with sufficient computing power could break the code. Hence the need to change (misfortune more!) Periodically the keys (cryptoperiod). At least two solution are therefore available to us:
	\begin{enumerate}
		\item Do not exchange any key (it is possible but it is quite long as we will see in the figure below)

		\item Exchange a secret key using a non-invertible mathematical function or at least very difficult to inverse (this is the Diffie-Hellman protocol that we will also see in a figure below).
	\end{enumerate}
	Therefore Public key cryptography systems often rely on cryptographic algorithms based on mathematical problems that currently admit no efficient solution—particularly those inherent in certain integer factorization, discrete logarithm, and elliptic curve relationships. Public key algorithms, unlike symmetric key algorithms, do not require a secure channel for the initial exchange of one (or more) secret keys between the parties.

	Because of the computational complexity of asymmetric encryption, it is usually used only for small blocks of data, typically the transfer of a symmetric encryption key (e.g. a session key). This symmetric key is then used to encrypt the rest of the potentially long message sequence. The symmetric encryption/decryption is based on simpler algorithms and is much faster.
	
	Let us see what the first solution is and its blatant disadvantage:
	\begin{figure}[H]
		\centering
		\includegraphics{img/computing/public_key_principle.jpg}
		\caption[Principle of public key encryption]{Principle of public key encryption (source: ?)}
	\end{figure}
	Explanation: Alice and Bernard want to transmit a message on an unsecured line and without exchanging keys. To do this, Alice puts her letter in a chest that she closes with her key and sends it to Bernard. The latter returns the chest to Alice where he added his own padlock which he closed with his own key. When Alice receives the chest, she takes off her padlock and sends Bernard a chest that no longer includes Bernard's padlock closed with Bernard's key. The latter then only has to open the chest to read the letter. This operation is safe and does not require exchange of keys. On the other hand, it requires several paths (the process is represented by the first $4$ transactions of the figure above).
	
	The principle of the public key must allow secure exchanges, without a secret key, in a single path. Bernard distributes widely copies of his public padlock. Alice gets one, but anyone could do the same. Alice places the message in the trunk and closes it with the Bernard code lock, then sends it the trunk (represented by the $5$ transaction in the figure above). On receiving the trunk, Bernard can open the trunk, since he alone holds the key that opens this lock. The transfer is safe in one trip. In cryptography, the public key is equivalent to the code lock, which is available for example in directories, while the key that opens this lock is the private key, owned solely by their owner and never disclosed. The private and public keys (the so-named "key trousseau") are constructed from a supposed "one-way" mathematical function.

	Let's now see the second solution making use of public key according to the Diffie-Hellman protocol:
	
	\subsubsection{Diffie-Hellman protocol}
	As the name implies, a one-way function gives easily a result, but the reverse operation is very difficult. Finding such functions in the mathematical world seemed very arduous to mathematicians. How to imagine a function that is one-way for everyone, except for its creator who can reverse it through the knowledge of a particular information. Thus, W. Diffie and Hellman were the first to publicly propose a one-way function to solve the problem of agreeing on a common secret. The basic idea is to calculate values of the type:
	
	where $\alpha$ and $a$ are imposed as being integers and $p$ is a prime number.
	
	Mathematicians call this kind of operation a "\NewTerm{modular exponentiation}\index{modular exponentiation}" or "\NewTerm{discrete exponential}" and it is customary to denote the finite field of integers modulo $p$ (where $p$ is a prime number) by $\text{GF}(p)$ in honour of Évariste Galois.
	
	To explicate such a calculation (as a reminder of what was seen in the section of Number Theory ...), we raise a number $\alpha$ to the power of $a$, and then divide the result by a large prime number $p$ and we keep finally the remainder of this division (operation modulo $p$). If this remainder is denoted $r$ then we write this:
	
	Modular exponentiation similar to the one described above are considered easy to compute, even when the numbers involved are enormous. On the other hand, computing the "\NewTerm{discrete logarithm}\index{discrete logarithm}" – that is, the task of finding the exponent $a$ when given $\alpha$, $p$, and $r$ (ie $\alpha^a \mod p$) – is believed to be difficult. This one-way function behaviour makes modular exponentiation a candidate for use in cryptographic algorithms! In addition, one-way functions such as the one above from the modular arithmetic behave very irregularly as is shown in the table with the particular example below:
	\begin{table}[H]
		\centering
		\begin{tabular}{|c|c|c|}
		\hline
		\rowcolor[HTML]{9B9B9B} 
		\multicolumn{1}{|l|}{\cellcolor[HTML]{9B9B9B}$\pmb{a}$} & \multicolumn{1}{l|}{\cellcolor[HTML]{9B9B9B}$\pmb{\alpha^a=3^a}$} & \multicolumn{1}{l|}{\cellcolor[HTML]{9B9B9B}$\pmb{\alpha^3 \mod p=3^a \mod 7}$} \\ \hline
		$0$ & $1$ & $1$ \\ \hline
		$1$ & $3$ & $3$ \\ \hline
		$2$ & $9$ & $2$ \\ \hline
		$3$ & $27$ & $6$ \\ \hline
		$4$ & $81$ & $4$ \\ \hline
		$5$ & $243$ & $5$ \\ \hline
		$6$ & $729$ & $1$ \\ \hline
		$7$ & $217$ & $3$ \\ \hline
		$8$ & $6561$ & $2$ \\ \hline
		\end{tabular}
		\caption{Examples of modular exponentiation applications}
	\end{table}
	So even if it is easy to compute a discrete exponential, it is almost impossible to find the starting number $a$ from the result, especially when this modular function is applied to very large primes $p$.

	The reader can check this by playing with Maple 4.00b that can calculate the discrete logarithm as following:
	
	\texttt{>with(numtheory):\\
	>mlog(r,alpha,p);}
	
	Ok we know how to crypt message... But now how can we communicate messages to someone that should be able to uncrypt them? This is named the "\NewTerm{Diffie–Hellman key exchange}\index{Diffie–Hellman key exchange}" that is a specific method of securely exchanging cryptographic keys over a public channel and was one of the first public-key protocols as originally conceptualized by Ralph Merkle and named after Whitfield Diffie and Martin Hellman. D–H is one of the earliest practical examples of public key exchange implemented within the field of cryptography. Here is the idea of the protocol:
	\begin{table}[H]
		\centering
		\begin{tabular}{|l|l|l|}
		\hline
		\cellcolor[HTML]{FFCCC9}\textbf{ALICE} & \cellcolor[HTML]{EFEFEF}\textbf{Public (Internet)} & \cellcolor[HTML]{34FF34}\textbf{BERNARD} \\ \hline
		\multicolumn{3}{|l|}{\parbox{13cm}{We choose an arbitrary common prime number $p=419$ and a common random number smaller than $p$: $\alpha=7$. These two values are assumed to be secret.}} \\ \hline
		\parbox{5cm}{Alice chooses a secret\\ random number: $a=178$} &  & \parbox{5cm}{Bernard chooses a secret\\ random number: $b=344$} \\ \hline
		\parbox{5cm}{With the number $a$ Alice generates the public element:\\
		$k_a=\alpha^a \mod(p)=181$} &  &  \parbox{5cm}{With the number $b$ Bob generates the public element:\\
		$k_b=\alpha^b \mod(p)=351$}\\ \hline
		\parbox{5cm}{The result is sent to Bernard:\\ $k_a=181$} & $\xrightarrow{\makebox[2cm]{}}$ & $k_a=181$ \\ \hline
		$k_b=351$  & $\xleftarrow{\makebox[2cm]{}}$ &  \parbox{5cm}{The result is sent to Alice:\\ $k_b=351$} \\ \hline
		\parbox{5cm}{The shared secret is then:\\ $K=(k_b)^a \mod(p)=493$} &  & \parbox{5cm}{The shared secret is then:\\ $K=(k_a)^b \mod(p)=493$} \\ \hline
		 \multicolumn{3}{|c|}{$\leftarrow$ The exchanges are then encrypted with the secret key $K$ $\rightarrow$} \\ \hline
		\end{tabular}
		 \caption{Example of key exchange following Diffie-Hellmann protocol}
	\end{table}
	The security of this protocol is computational. It is based on the assumption that with limited computing power and time, an opponent (spy) can not reverse the modular exponential function (by making use of the properties of the logarithms with the exponential functions as we saw in the section of Functional Analysis) and therefore can not find the secret $a$ from the exchanged elements. This computational difficulty is due to the fact that the computation time necessary for the inversion of a one-way function does not have an algorithmic complexity (\SeeChapter{see section of Numerical Methods page \pageref{algorithm complexity}}) polynomial but exponential with $p$.
	
	Alice and Bernard have calculated the same common secret: $493$. Then $493$ is used to encrypt the exchanged data (in practice, much larger numbers are used). The spy is supposed to be able to intervene only after the exchange of the common choice of $p$ and $\alpha$ (no man-in-the-middle attack!).

	\begin{tcolorbox}[title=Remark,colframe=black,arc=10pt]
	This protocol is vulnerable to a "man-in-the-middle attack" that is an attack where the attacker secretly relays and possibly alters the communication between two parties who believe they are directly communicating with each other. One example of man-in-the-middle attacks is active eavesdropping, in which the attacker makes independent connections with the victims and relays messages between them to make them believe they are talking directly to each other over a private connection, when in fact the entire conversation is controlled by the attacker. The attacker must be able to intercept all relevant messages passing between the two victims and inject new ones.
	\end{tcolorbox}
	The key $K$ is obtained by the fact that the power operation is compatible with the relation of equivalence modulo $p$ (\SeeChapter{see section Number Theory page \pageref{congruence}}) such that:
	
	\begin{tcolorbox}[colframe=black,colback=white,sharp corners]
	\textbf{{\Large \ding{45}}Example:}\\\\
	We have:
	
	when with $w=2$ we have:
	
	but
	
	If it is not clear let us write it differently:
	
	Indeed, for the first one: $5-2=3$ can be divided by $3$ and for the second one $25-4=21$ can be divided by $3$.
	\end{tcolorbox}
	Thus, since $x<p$, the second modulo below has no meaning, so we can write:
	
	identically:
	
	and therefore:
	
	Diffie-Hellman is a cornerstone of modern cryptography used for VPNs, HTTPS websites, email, and many other protocols. Bad implementation choices combined with advances in number theory mean real-world users of Diffie-Hellman are likely vulnerable to state-level attackers. 
	
	Despite these precautions, experts established a record at the beginning of the 21st century using a new algorithm, they succeeded in reversing the modular exponential function for a $p$-number of 120 digits (about $400$ bits), using a Computer with four $525$ [MHz] processors. This record shows that the security of the protocol depends greatly on the constant progress made in the field of algorithmic complexity. Researchers estimate that breaking a single, common $1024$-bit prime would allow NSA (USA National Security Agency) to passively decrypt connections to two-thirds of VPNs and a quarter of all SSH servers globally. Breaking a second 1024-bit prime would allow passive eavesdropping on connections to nearly $20\%$ of the top million HTTPS websites. In other words, a one-time colossal investment in power-lifting computation would make it possible to eavesdrop on trillions of encrypted connections.
	
	The clever schema of Diffie-Hellman remains a schema of principle. Its main disadvantage is that it does not make it possible to provide the traditional security services: authentication of the two interveners, control of the integrity of the key and anti-replay (verification that information already transmitted is not re-transmitted ). It follows that an attacker can, for example, impersonate Alice by replacing the public element of Alice with her own public element. To overcome this disadvantage, secure versions of this generic protocol have been published, for example a protocol named "STS" (Station To Station), which uses, in particular, the electronic signature to ensure the authentication of the interveners (see below). This policy is the basis of the secured Internet connection (IPSec).
	
	The Diffie-Hellman protocol paved the way for a whole series of algorithms, that of "public key encryption" being the first. The idea was to break the symmetry of encryption and decryption by using one-way functions.
	
	\pagebreak
	\subsubsection{R.S.A system}
	Curiously, the first "\NewTerm{R.S.A. encryption system}\index{RSA encryption system}" is conceptually quite different from the Diffie-Hellman protocol: it does not use the discrete exponential, but the factorization of large numbers. This public key system was invented in 1977 by Ron Rivest, Adi Shamir, and Leonard Adleman (hence the abbreviation "R.S.A."). Having quickly become an international standard, the R.S.A. technique has been marketed by more than $400$ companies and we estimate that more than 400 million software use it. It is implemented in web browsers, such as Netscape Navigator, Microsoft Internet Explorer, or some bank smart cards, such as VISA cards.

	The R.S.A. system is based on the difficulty of factorizing large numbers and the one-way function used is a "power" function. The R.S.A. encryption protocol is divided into three phases:
	\begin{enumerate}
		\item Creation of keys (public and private)

		\item Encryption using the recipient's public key

		\item Decryption using the private key
	\end{enumerate}
	Its concept is based on a famous theorem named "\NewTerm{Euler's theorem}\index{Euler's theorem}" (nothing to do with the theorem of the same name seen in the section of Graph Theory or in the section of Geometric Shapes). Let's see what it is (be careful it is relatively long!).
	
	\paragraph{Euler's theorem}\mbox{}\\\\
	Before we see what Euler's theorem consists of, we must define two elements that are included in it. Apart the concept of congruence which we have already studied in the section of Number Theory, there remains a special function named the "\NewTerm{Euler indicator}\index{Euler indicator}\label{euler indicator function cryptography}" or also named "\NewTerm{totient function}\index{totient function}" and defined in general by:
	
	In other words, the function $\phi$ of the integer $m$ results in a number $n$ strictly less than $m$, given by the number of elements between $1$ and $m$ whose greatest common divisor (\SeeChapter{see section Number Theory page \pageref{greatest common divisor}}) with $m$ is $1$. We have already given a practical example of the utility of this indicator function in the section of Number Theory in the framework of the reduced systems of residues and which are at the center of the proof of Euler's theorem.
	
	This can be formulated in the following form: the indicator $\phi$ of the integer $m$ is defined as the number of positive integers less than or equal to $m$ and prime with $m$.

	This function therefore has the remarkable property of counting the number of positive integers smaller than $m$ and "relatively prime" (ie, having greater common divider equal to $1$) with $m$.

	Here are some values of $\phi(m)$ for $m$ that range from $0$ to $19$:
	\begin{table}[]
		\centering
		\begin{tabular}{|l|l|c|c|c|c|c|c|c|c|c|}
		\hline
		\rowcolor[HTML]{C0C0C0} 
		$\pmb{\phi(m)}$ & $\pmb{0}$ & \multicolumn{1}{l|}{\cellcolor[HTML]{C0C0C0}$\pmb{1}$} & \multicolumn{1}{l|}{\cellcolor[HTML]{C0C0C0}$\pmb{2}$} & \multicolumn{1}{l|}{\cellcolor[HTML]{C0C0C0}$\pmb{3}$} & \multicolumn{1}{l|}{\cellcolor[HTML]{C0C0C0}$\pmb{4}$} & \multicolumn{1}{l|}{\cellcolor[HTML]{C0C0C0}$\pmb{5}$} & \multicolumn{1}{l|}{\cellcolor[HTML]{C0C0C0}$\pmb{6}$} & \multicolumn{1}{l|}{\cellcolor[HTML]{C0C0C0}$\pmb{7}$} & \multicolumn{1}{l|}{\cellcolor[HTML]{C0C0C0}$\pmb{8}$} & \multicolumn{1}{l|}{\cellcolor[HTML]{C0C0C0}$\pmb{9}$} \\ \hline
		\cellcolor[HTML]{C0C0C0}$\pmb{0+}$ &  & $1$ & $1$ & $2$ & $2$ & $4$ & $2$ & $6$ & $4$ & $6$ \\ \hline
		\cellcolor[HTML]{C0C0C0}$\pmb{10+}$ & \multicolumn{1}{c|}{$4$} & $10$ & $4$ & $12$ & $6$ & $8$ & $8$ & $16$ & $6$ & $18$ \\ \hline
		\end{tabular}
		\caption{Some values of the Euler indicator $\phi(m)$}
	\end{table}
	Let us now introduce two properties of $\phi(m)$:
	\begin{enumerate}
		\item[P1.] We notice the (trivial) property of this function when we denote any prime number (remember that $1$ is not a prime number!) by the letter $p$ then:
		
		as it is highlighted by the table above.
	
		\item[P2.] The Euler indicator can also be written in the following form if $p$ and $q$ are relatively prime (this is the padlock of the R.S.A system which is more complicated than the simple multiplication of $p$ and $q$):
		
		this last relation can easily be verified (without proof) by taking some values from the preceding table (if we do it like Ramanujan...).
	\end{enumerate}
	\begin{theorem}
	This done, given $(a,m)=1$ (the greatest common divisor of $a$ and $m$, ie $a$ and $m$ are relatively prime), the "\NewTerm{Euler's theorem}\index{Euler's theorem}" says that if $m$ is a natural number and $a$ is relatively prime with $m$ then we have:
	
	in which we see the Euler indicator defined above. It is a rather surprising relation. Let's see if it works with $7$ and $2$ which are relatively prime between them:
	
	the remainder being indeed therefore equal to $1$ when we compute $64$ modulo $7$.
	\end{theorem}
	\begin{dem}
	Let us first recall (\SeeChapter{see section Number Theory page \pageref{system of reduced residue}}) that a reduced system of residuals modulo $m$ is a set of integers equation that satisfy the three properties:
	\begin{enumerate}
		\item[P1.] The remainder $r_i$ and $m$ are relatively prime, ie $(r_i,m)=1$

		\item[P2.] $r_i$ is not congruate $r_j$ modulo $m$ when $i\neq j$

		\item[P3.] Each integer $x$ relatively prime with $m$ is congruent to some $r_i$ modulo $m$
	\end{enumerate}
	\begin{tcolorbox}[colframe=black,colback=white,sharp corners]
	\textbf{{\Large \ding{45}}Example:}\\\\
	For example, the set $\{1,5\}$ is a reduced system of residuals modulo $6$ or another example, $\{1,2,3,4,5,6\}$ is a reduced system of residuals modulo $7$. \\

	We also check for the first example that $1$ is not congruent $5$ modulo $6$ (indeed, $6$ does not divide $(5-1)$) and that $5$ which is relatively prime to $6$ is congruent to itself.\\

	For the second set, we notice that the cardinal of the set of residuals corresponds to the value of the Euler indicator for the number $7$.
	\end{tcolorbox}
	\begin{lemma}
	Thus, given $\{r_1,r_2,\ldots,r_{\phi(m)}\}$ be a reduced system of residuals modulo $m$. We need for the proof of Euler's theorem, to prove beforehand the lemma that $\{ar_1,ar_2,\ldots,ar_{\phi(m)}\}$ is also a reduced system of residuals modulo $m$.
	\end{lemma}
	
	\begin{tcolorbox}[title=Remark,colframe=black,arc=10pt]
	As we have already mentioned in the previous example, you can observe that the cardinality of the set of residuals corresponds, for a given prime modulo $m$, to the result defined by the property P1 of the Euler indicator function $\phi(m)$. This property is to this day only a "conjecture", that is to say, an assumption based on probabilities (because it seems it has not be proven so far!).
	\end{tcolorbox}
	For this, let us recall that by the property of a reduced system:
	
	and that by hypothesis:
	
	then we want the lemma that:
	
	is also satisfied.
	
	Let us put for this $d=1$ (by tradition ...). We then have since $(r_i,m)=d$ that $d|r_i$ and $d|m$ and identically for $(a,m)=d$ that $d|a$ and $d|m$. Now if $d$ divides well $a$ or $r_i$ in this case we have $d|r_i(a)$ or (equivalently) $d|a(r_i)$. Therefore $d|ar_i$ and $d|m$ which allows us to write:
	
	Let us return to our Euler's theorem... if still follow ... We have just proved that there is bijection between the two sets of residues. That is to say that for each residue $r_i$ of the reduced system modulo $m$, we will have a residue $ar_i$ of the reduced system modulo $m$ according to the fundamental property of the congruence which we recall says that: we can multiply the two members of a congruence by the same integer number and it will remain congruent modulo $m$ and modulo $m$ multiplied by this integer number.
	
	\begin{tcolorbox}[colframe=black,colback=white,sharp corners]
	\textbf{{\Large \ding{45}}Example:}\\\\
	Let us take:
	
	indeed:
	
	because the remainder of the division of $30$ by $6$ is indeed equal to zero. If we take for example:
	
	then we also:
	
	and the remainder is also zero...
	\end{tcolorbox}
	Let us make a recall on bijection (\SeeChapter{see section Set Theory page \pageref{bijection}}): We say that we have a bijection, if to each element of a starting set corresponds one and only one element in the arrival set (if there was for every man on Earth only one woman - in equal proportions therefore - there would be a bijection between the set of Men and Women).

	In short, since there is bijection between the two sets of residues, we can write:
	
	\begin{tcolorbox}[colframe=black,colback=white,sharp corners]
	\textbf{{\Large \ding{45}}Example:}\\\\
	The set $\{1,5\}$ is a reduced system of residuals modulo $6$ as we have already seen. So we have:
	
	We then:
	
	If we take an $a$ such that $(a,m)=1$, for example $a=7$ because indeed $(7,6)=1$, then:
	
	because $6|(35-5=30)$. Indeed, $6$ divides well $30$ with a remainder equal to $0$.
	\end{tcolorbox}
	So let us return to our bijection, which can be written by the elementary rules of algebra:
	
	Since:
	
	(you can verify, but this is the very definition of a set of residues!), we are then obliged to conclude that:
	
	and anyway, even if it does not seem obvious to you, you just need to multiply each of the members of the equality of the congruence by:
	
	as permit us one of the intrinsic properties of congruence previously proved.
	\begin{flushright}
		$\blacksquare$  Q.E.D.
	\end{flushright}
	\end{dem}
	This interlude theory being done, let us consider a number $N$ of which we wish to decide whether it is prime number or not.

	We know from the Euler theorem and of the property P1 of the Euler indicator that if $N$ is a prime number and if $a\in\mathbb{N}$, where $a<N$, then:
	
	which is named the "\NewTerm{Fermat's little theorem}\index{Fermat's little theorem}".
	\begin{tcolorbox}[title=Remark,colframe=black,arc=10pt]
	This relation follows from the properties we have presented in our proof of Euler's theorem:
	
	and of the property P1 of the function $\phi(m)$ for a prime number $p$:
	
	\end{tcolorbox}
	The Fermat's little theorem is however, also valid for some numbers $N$ which are not prime. But the numbers which check this without being prime are rare, and it is worthwhile to look for a more sophisticated algorithm to know if $N$ is really prime or not (we say that in this case, $N$ is a good candidate for primality and is then named "\NewTerm{pseudo-prime number}"). To test whether the number non-prime number $N$ is "sufficiently prime", we try with an algorithm to test the Fermat's little theorem a maximum number $a\in\mathbb{N}$ with $a<N$.
	
	According to the property of congruence (see above), we also have:
	
	We can apply this last theorem to a number $N$ on which we would like to know at best whether it is prime or not.

	There are a large number of other non-optimal methods for determining whether $N$ is prime; including preliminary division trials by $2$, $3$, $5$, $7$, $11$, $\ldots$ and small prime numbers up to $p\leq\sqrt{N}$ according to the method of the Eratosthenes screening which is best known method in high schools.
	\begin{tcolorbox}[title=Remark,colframe=black,arc=10pt]
	In fact, with the help of a fairly powerful computer, we can decide whether a natural number of the order of $10^{300}$ ($10$ followed by $300$ zeros) is first or not within a few minutes or seconds. What is important to know is that, given a natural number $N$, one can decide in relatively short time whether it is prime or not, without knowing however its prime factors!!
	\end{tcolorbox}
	However, according to the fundamental theorem of arithmetic we have that:
	\begin{theorem}
	Any natural number $N$ can be written as a product of prime numbers, and this representation is unique, apart from the order in which the prime factors are arranged.
	\end{theorem}
	The proof is already in the section of Number Theory but exceptionally we will reproduce it here as it is quite a short proof:
	\begin{dem}
	The proof uses Euclid's lemma (\SeeChapter{see section Number Theory page \pageref{euclid lemma}}): if a prime $p$ divides the product of two natural numbers $a$ and $b$, then either $p$ divides $a$ or $p$ divides $b$ (or both).
	
	If $N$ is prime, and therefore product of a unique prime integer, namely itself, the result is true and the proof is complete (say that a prime number is product of itself is obviously a misnomer! ). Suppose that $n$ is not prime and therefore strictly greater than $1$ and consider the set:
	
	So, $D\subset \mathbb{N}$ and since $N$ is composite, we have that $D\neq \varnothing$. According to the principle of good order, $D$ has a smaller element $p_1$ that is prime, otherwise the minimum choice of $p_1$ is contradicted. We can the write $N=p_1N_1$. If $n_1$ is prime, then the proof is complete. If $n_1$ is also composite, then we repeat the same argument as before and we deduce the existence of a prime number $p_2$ and of an integer $N_2<N_1$, such as $N=p_1p_2N_2$. By continuing we come inevitably to the conclusion that $N_k$ will be prime.
	
	So finally we well show that any number can be decomposed into prime numbers factors with the principle of good order.
	\begin{flushright}
		$\blacksquare$  Q.E.D.
	\end{flushright}
	\end{dem}
	So finally we have proved that any number is decomposable into prime factors using the principle of good order. There exist in the set of natural numbers $\mathbb{N}$, some which can be expressed by (or only by) two prime factors traditionally denoted $p$ and $q$. These are the numbers we use in public key cryptography according to the R.S.A. protocol.
	\begin{tcolorbox}[title=Remark,colframe=black,arc=10pt]
	We do not know to this day a law that makes it possible to easily and quickly calculate the $i$-th prime factor $p_i$ of a number. In fact, even with the most powerful computers we have now in year $2002$ when we write these lines, it would take several years to find the two prime factors $p$ and $q$ of a "\NewTerm{RAS number}\index{RAS number}" $N=pq$ where $p$ and $q$ are of the order of $10^{100}$ each. And it seems unlikely that we will discover in the near future an algorithm sufficiently effective to improve appreciably this computing time. Note that it is possible to determine in less than $5$ minutes (in year $2002$) whether a number of $200 $digits is prime or not. However, to factorize a number of $200$ digits into two prime numbers, it would take at least $100$ years. Wonderful thing: the theories that allow these exploits are very deep and were developed partly long ago in a very different setting.\\
	
	Now in year $2009$ a RSA number of $232$ digits (ie $768$ bit RSA number) was factorized ($7$ years after we wrote the lines above) in half a year one eighty 2.2 GHz AMD Opteron processors...
	\end{tcolorbox}
	The fact that it is much more difficult to find the prime factors of a number $N$ than to find out if $N$ is prime or compound is precisely what made it possible to develop this very ingenious method of encoding and decoding messages according to the RSA protocol.
	\begin{tcolorbox}[colframe=black,colback=white,sharp corners]
	\textbf{{\Large \ding{45}}Example:}\\\\
	Let us consider now a group of individuals who regularly transmit messages by e-mail and for which it is important that the messages are known only to the sender and the recipient. Then, the group member (here Alice) who wants to receive encrypted information, choose two very large prime numbers $p$ and $q$ of the order of $10^{100}$. To find such prime numbers, we randomly choose a number of $100$ digits and we check by one of the known algorithms whether it is prime or not and we repeat the experiment until we get a prime number. Once this is done with these two prime numbers, we compute the expression:
	
	named the "\NewTerm{modulus}".\\
	\end{tcolorbox}
	
	\begin{tcolorbox}[colframe=black,colback=white,sharp corners]
	Then, Alice (who is the only one in possession of the number $N$ for the moment) who wants to receive the encrypted informations chooses a positive integer $a$ such (p.g.c.d.) that:
	
	So $a$ (often denoted $e$ in the literature) is a prime integer with $\phi(N)$ sometimes named the "\NewTerm{generator}".\\
	
	And as:
	
	Suppose a Alice wants to receive a message from Bob, one of her friends.\\

	Alice has therefore the "\NewTerm{public key}\index{public RSA key}", defined by the couple:
	
	to Bob.\\
	
	Bob receives the public key and wishes to send the french message: \textit{déclencher l'opération rouge}\footnote{In English: \textit{trigger the red operation}}. To do this, Bob first transforms the message into numbers by using the convention that each letter is replaced by its corresponding position in the alphabet starting counting from $01$ (the character "space" will be encrypted "$27$").\\

	Thus the clear message denoted $M$ afterwards becomes:
	
	\begin{tcolorbox}[title=Remark,colframe=black,arc=10pt]
	For technical reason, $M$ and $N$ must have no common divisor other than $1$ (otherwise, a possible spy could reduce the problem of two very large numbers difficult to manipulate to that of smaller numbers, easier to manipulate). Otherwise, at the end of $M$, we add numbers without value, such as $01$ (for example), to finally have $M$ and $N$ without common divisor other than $1$.
	\end{tcolorbox}
	We can also break $M$ into pieces $M_i$ whose number of digits does not exceed $99$ (remember that we set a lower limit of a power of $100$ for $p$ and $q$ and that it would therefore suffice that one of the two prime numbers to be $1$ and the other exactly a number with an exponent $100$ to be at the limit of the number $N$ then comprising at worst 100 digits, even if this extreme example is quite bad for technical reasons as more easy to crack), in which case one will always have:
	
	\end{tcolorbox}
	
	\begin{tcolorbox}[colframe=black,colback=white,sharp corners]
	We cut $M$ into pieces, each being smaller than $N$:
	
	and we work successively with each piece $M_1,M_2,\ldots,M_{12}$ of the message.\\
	
	We consider the power $a$ of $M_1$, that is, $M_1â$. We replace $M_1$ by the number $\bar{M}_1$, which is the remainder of the division by $N$ of the number $M_1^a$. The same procedure is followed for all other $M_i$ pieces such as:
	
	Then Bob then sends the encoded message to Alice:
	
	An interceptor of the encoded message and of the public key, knowing the encryption algorithm, would therefore have to solve the problem of one equation with two unknowns (equation obtained simply from the mathematical expression of the encryption rule):
	
	Obviously unspecified problem!
	\end{tcolorbox}
	To see how the receiver decrypts the message, we need an additional mathematical tool.

	Let us recall that the receiver chooses $a$ such that $(a,\phi(N))=1$, which implies, according to the Bézout's theorem\index{Bézout's theorem} (\SeeChapter{see section Number Theory page \pageref{bezout theorem}}), that if $a$ and $\phi(N)$ are relatively prime (that is to say for recall that their greatest common divisor is $1$) there exist integers $x$ and $y$ such that (we can assume that $x>0$, in which case $y<0$):
	
	or otherwise written:
	
	This is how we will determine the value of $x$ (we must use algorithms to find the solution $x$ to this equation).
	
	Which means:
	\begin{enumerate}
		\item If $a$ is prime with $\phi(N)$ then by the properties of congruence it is also prime with $p-1$ and $q-1$.
	
		\item That $a$ is invertible modulo $\phi(N)$
	
		Indeed, because:
		
		And according to the definition of congruence ($m|(a-b)$) we have:
		
		since $\phi(N)$ divides the right-hand side of $ax-1=\phi(N)y$ and therefore by the equality, the left-hand member. Therefore:
		
	\end{enumerate}
	Only the receiver of the message, can easily calculate the number $x\le a$ used above. In order to do this, it is necessary to be able to calculate the value of $\phi(N)$ and thus know $p$ and $q$.

	If $M_i$ is the original message (its numerical value) and $\bar{M}_i$ is the received encoded message (its numeric value), then we have the following relation:
	
	This is completely logical since the difference $M_i^a-\bar{M}_i$, where for recall, $\bar{M}_i$ is the remainder of the division of $M_i^a$ by $N$, can therefore only be divisible by $N$.
	
	Alice thus receives the coded message $\bar{M}$ and raise to the power of $x$ the numbers $\bar{M}_i$ and thus obtains the initial message.

	Indeed, she will apply for each $\bar{M}_i$ the following mathematical property of congruence:
	
	The "\NewTerm{private key}\index{private RSA Key}" (allowing to decrypt the message and which can be easily known only by the Alice) is thus defined by the couple:
	
	Let us give more indeed explanations about what we have stated just above! We have showed that:
	
	and from the property of symmetry of congruence (\SeeChapter{see section Numbers page \pageref{congruence}}), we can write:
	
	Now we can write:
	
	according to the second principal property of congruence, which says for recall that the two members of a congruence can be elevated to the same power! That latter relation can also be written (application of Bézout's theorem):
	
	Remains to prove that:
	
	where we can write $M_i^{1-\phi(N)y}$ under the form:
	
	Now, remember that we have proved Euler's theorem:
	
	and that one of the properties of congruence gives us the right to elevate to any power the two members of the congruence such as:
	
	But as 1 raised to any power makes $1$, we have:
	
	This last relation allows us to verify that we can authorize ourselves to write:
	
	since the two left members are well modulos $N$. 
	
	So if we sum up all this, Alice receives a piece $\bar{M}_i$ and raises it automatically to the power $x$ to obtain a number which according to her should be the true $M_i$. To be sure, it applies the verification:
	
	It is easy to see that any interceptor can not decode and in addition verify if the decoding is indeed the right one, because for this it should know the value of $x$, which in turn depends on $\phi(N)$, that it does not know either, because he does not know the prime factors of $N$ that are $p$ and $q$.

	It is customary to say that the RSA system uses the numbers $p$, $q$ (secrets), $N$ (public), $a$ (public) and $x$ (secret). The whole being summed up by the triplet $\{n, a, x\}$ denoted sometimes in the literature $\{n, e, d\}$.
	\begin{figure}[H]
		\centering
		\includegraphics[scale=1]{img/arithmetics/rsa_detailed_cyphering.jpg}
		\caption{Principle of RSA public key encryption}
	\end{figure}
	And here is a small practical application with Maple 4.00b:
	
	\texttt{> \#Initialization of the Maple 4.00b random generator\\
	> randomize():\\
	> \#definition of the desired size for N (this is an even number)\\
	> t:=30:\\
	> \#Generation of two integers of t/2 bits size\\
	> x:=rand(2\string^(t/2-1)..2\string^(t/2))();\\
	> y:=rand(2\string^(t/2-1)..2\string^(t/2))();\\
	> \#Calculation of the following prime numbers\\
	> p:=nextprime(x);\\
	> q:=nextprime(y);\\
	> \#Generation of the RSA key\\
	> n:=p*q;\\
	> phi:=(p-1)*(q-1);\\
	> \#We choose "a" empirically\\
	> a:=65537;\\
	> \#We check that it is prime with phi\\
	> igcd(a,phi);\\
	> \#we calculate the inverse of "a" modulo phi\\
	> x:=1/a mod phi;\\
	> \#we choose a message a being "1234"\\
	> m:=1234;\\
	> \#we cypher\\
	> c:=m\&\string^a mod n;\\
	> \#we decode\\
	> c\&\string^x mod n;\\
	}

	Following the request of a reader here is a literal summary of what we have seen so far for the first steps of the algorithm above with practical value and a given message:
	\begin{tcolorbox}[colframe=black,colback=white,sharp corners]
	\textbf{{\Large \ding{45}}Example:}\\\\
	We want to cypher the message $M=314158$.

	\begin{enumerate} 
		\item We choose $p$ and $q$ prime and sufficiently large:
		
		we then have:
		
	
		\item We compute the Euler indicator:
		
	
		\item We choose the generator $a$ such that:
		
		and for this we will take $a=5$. The pair $(a,N)$ is the public key (can be distributed to everyone for a specified time).
	
		\item Then we calculate:
		
		So the pair $(x, a)$ is the private key (to be kept secret).
		
		\item Now we cypher with:
		
		Therefore:
		
		
		\item Now to decipher (the exponent calculation cannot be done with sample spreadsheet softwares or simple online scientific calculator):
		
	\end{enumerate}
	\end{tcolorbox}
	For security reasons, public key cryptography is used in conjunction with secret key cryptography. For example, at the time of writing these lines, the SSL protocol for Internet pages uses the RSA to exchange a secret key (symmetric system) and then encrypts the data using a conventional symmetric algorithm.

	Let us conclude this brief presentation of the messages cyphering by informing the reader that the American government (and not only...!) closely monitors the activities of mathematicians who work on the factorization of large numbers. Indeed, if one of them could find an algorithm allowing to factorize in a short time a number of two hundred digits (greater than $524$ bits unsigned), this would jeopardize the secret nature of several communications of a military order. In fact, this surveillance has raised a protest by the scientists in the United States, who see their professional freedom undermined (Notices of American Mathematical Society, January 1983).

	For technical information, the software PGP (Pretty Good Privacy) published my the MIT (Massachusetts Institute of Technology), uses an RSA encryption system.
	
	\pagebreak

	
	\pagebreak
	\subsection{Hash functions}
	A "\NewTerm{hash function}\index{hash function}" is a function that associates to a big set a much smaller set (of the order of a few hundred bits) that is characteristic of the starting. This property makes it very used in computing, in particular for quick access to data thanks to "hash tables" or to check the result of huge data transmission (downloads). Indeed, a hash function makes it possible to associate a particular integer with a string. Thus, if we know the fingerprint of the stored character strings, we can quickly check whether a string is in this table (in $\mathcal{O}(1)$ if the hash function is good enough). Hash functions are also extremely useful in cryptography to speed up encryption.

	The two most commonly used condensation algorithms at the beginning of the 21st century are the "Secure Hash Algorithm (SHA)", which calculates a $160$-bit summary, and the MD5 (Message Digest 5 - Run Rivest 1992), which calculates a $128$-bit summary called "Message Digest".
	
	\subsubsection{MD5 message digest condensation function}\label{md5}
	This "\NewTerm{Message Digest MD5}\index{message Digest MD5}" algorithm is (was) used mainly for digital signatures (notion used, when validating certificates of authenticity as we will see later) but as it has been found to suffer from extensive vulnerabilities\footnote{In 2004 it was shown that MD5 is not collision-resistant. As such, MD5 is not suitable for applications like SSL certificates or digital signatures that rely on this property for digital security.}. It can still be used as a checksum to verify data integrity, but only against unintentional corruption.
	\begin{figure}[H]
		\centering
		\includegraphics[scale=1]{img/computing/md5.jpg}
		\caption{Illustrated result of MD5 algorithm}
	\end{figure}
	Here are the different stages of its operation:
	\begin{enumerate}
		\item Completion:
		
		The message consists of $b$ bits. The message is completed with a $1$, and sufficiently enough $0$ for the extended message to have a multiple length of $512$ bits. After this initial processing, the input text is manipulated in blocks of $512$ bits divided into $16$ sub-blocks \texttt{M[i]} of $32$ bits.
		
		\item Initialization:
		
		We define the $32$-bit "chaining variables" \texttt{A}, \texttt{B}, \texttt{C} and \texttt{D} initialized as follows (the digits are hexadecimal):
		\begin{center}
			\texttt{A=01234567}, \texttt{B=89ABCDEF}, \texttt{C=FEDCBA98}, \texttt{D=76543210}
		\end{center}
		We also define four non-linear functions \texttt{F}, \texttt{G}, \texttt{H} and \texttt{I} which take arguments coded on $32$ bits, and return a value on $32$ bits, the operations taking place bit by bit.
	
		\texttt{F(X,Y,Z) = (X AND Y) OR (NOT (X) AND Z)}\\
		\texttt{G(X,Y,Z) = (X AND Z) OR (Y AND NOT (Z))}\\
		\texttt{H(X,Y,Z) = X XOR Y XOR Z}\\
		\texttt{I(X,Y,Z) = Y XOR (X OR NOT (Z))}
	
		What is important with these four functions is that if the bits of their arguments \texttt{X}, \texttt{Y} and \texttt{Z} are independent, the resulting bits are also independent.
		
		\item Iterative calculation:
		
		The main loop has $4$ rounds (see figure below) which each use a different non-linear function (hence the fact that there are $4$ rounds). Each round therefore consists of $16$ executions of an operation (because there are $16$ sub-blocks).

		Each operation calculates a non-linear function of three of the variables \texttt{A}, \texttt{B}, \texttt{C} and \texttt{D}, adds to it a sub-block $M[i]$ of the text to be encrypted, a predefined constant $s$ (encoded on $32$ bits) and to a circular shift on the left of a variable number of bits $n$. Here is the example for \texttt{A}:
		\begin{itemize}
			\item \texttt{A = B + A + F(B,C,D) + M[i] + s} circularly offseted from $n$ bits to the left
			\item \texttt{A = B + A + G(B,C,D) + M[i] + s} circularly offseted from $n$ bits to the left
			\item \texttt{A = B + A + H(B,C,D) + M[i] + s} circularly offseted from $n$ bits to the left
			\item \texttt{A = B + A + I(B,C,D) + M[i] + s} circularly offseted from $n$ bits to the left
		\end{itemize}
		This new value of \texttt{A} is then summed with the old one.
		
		\item Writing of the summary (fingerprint):
		
		The $128$-bit summary is obtained by putting end-to-end the four $32$-bit chaining variables \texttt{A}, \texttt{B}, \texttt{C}, \texttt{D} obtained at the end of the iteration.
	\end{enumerate}
	\begin{figure}[H]
		\centering
		\includegraphics[scale=1]{img/computing/md5_algorithm_flow.jpg}
		\caption[Illustrated MD5 algorithm flow]{Illustrated MD5 algorithm flow (source: Wikipedia, author: Dake)}
	\end{figure}
	\begin{tcolorbox}[title=Remark,colframe=black,arc=10pt]
	Normally we would put the MD5 algorithm pseudocode but... as it is a pain in the a.. to write it in LaTeX with the \texttt{algorithm2e} package we will for the moment not do it....
	\end{tcolorbox}
	The MD5 function as we have already mention it is not safe and not unique (two different inputs can give the same signature: we talk then of "collision"). However, the MD5 function is still widely used as a verification tool during downloads and the user can validate the integrity of the downloaded version thanks to the fingerprint. This can be done with a program for example named \texttt{md5sum} for MD5 and \texttt{sha1sum} for SHA-1 (see just below).
	
	Here is the fingerprint (abusively sometimes named "signature") obtained on a sentence\footnote{Made with the online tool \url{http://www.md5hashgenerator.com}} (which we took without accents):
	\begin{center}
	MD5("Wikipedia, the free encyclopedia") = f8aa0d3b1dae3f41d67c200688723c1b
	\end{center}
	By modifying a character, this impression drastically changes:
	\begin{center}
	MD5("Wikipedia, the free encyclopediA") = f9829ad9d4c2713140973520cad9206c
	\end{center}
	Specifically, the MD5 fingerprint or can be performed as follows: when downloading a program, we write (copy) the character set indicated on the download page. When this download is complete, we launch one of the aforementioned software on the downloaded file.
	\begin{figure}[H]
		\centering
		\includegraphics[scale=0.87]{img/computing/md5_cisco_hash.jpg}
		\caption{Illustrated MD5 download hash from CISCO}
	\end{figure}
	It must also be noticed why that main reason why using symmetric (or asymmetric) encryption is not advisable for protecting passwords is: key management. When using encryption, you must protect the encryption key (or the entropies from which the key is derived). And protecting the key is a very difficult task to solve. Hashing (with SHA, MD5, or any other algorithm) solves the problem of key protection, because you don't need to keep any secret value (other than salt, but salt is significantly less sensitive than encryption key; you can store salt in plain text). So if you only keep passwords for authentication purposes (performed by your app), there is absolutely no reason to use encryption; hashing would do just fine. 
	
	\pagebreak
	\subsubsection{SHA-1 Secure Hash Algorithm condensation function}\label{sha 1}
	The "\NewTerm{SHA-1}\index{SHA-1}" or "\NewTerm{Secure Hash Algorithm-1}\index{Secure Hash Algorithm-1}" is used in competition with the MD5 for Digital Signature Algorithm as specified by the Digital Signature Standard (DSS). It was designed by the United States National Security Agency and is a U.S. Federal Information Processing Standard published by the United States (INST).

	SHA-1 is no longer considered secure against well-funded opponents. In 2005, cryptanalysts found attacks on SHA-1 suggesting that the algorithm might not be secure enough for ongoing use, and since 2010 many organizations have recommended its replacement by SHA-2 or SHA-3. Microsoft, Google, Apple and Mozilla have all announced that their respective browsers will stop accepting SHA-1 SSL certificates by 2017.

	On February 23, 2017 CWI Amsterdam and Google announced they had performed a collision attack against SHA-1, publishing two dissimilar PDF files which produce the same SHA-1 hash as proof of concept (\url{https://shattered.io}).

 	For a message of length less than $2^{64}$, the SHA-1 generates a $160$-bit digest of the message named also "hash" or "fingerprint". Again, identically to the MD5, a tiny modification of the original message must have a big impact on the condensed message and there must not be an identical Message Digest for two messages of different origin.

	As for the MD5, we work on messages whose length is a multiple of $512$ bits.
	\begin{enumerate}
		\item Completion:
		
		If the message does not have a length of $512$ bits, we add as many $1$ as necessary at the end of the message. The last $64$ bits of the $512$-bit block are used to set the original length of the message. The $512$-bit block is then transformed into sub-blocks \texttt{M[i]} of $32$ bits each expressed in hexadecimal ($0\ge i\ge 15$).
		
		\item Initialization:
		
		As for the MD5, this time we define $80$ chaining variables of $32$ bits $K[i]$ initialized as following (the digits are hexadecimal):
		\begin{itemize}
			\item \texttt{K[t]=01234567} for $0\ge t\ge 19$
			\item \texttt{K[t]=89ABCDEF} for $20\ge t\ge 39$
			\item \texttt{K[t]=FEDCBA98} for $40\ge t\ge 59$
			\item \texttt{K[t]=76543210} for $60\ge t\ge 79$
		\end{itemize}
		We also define $80$ non-linear functions \texttt{F[0]}, \texttt{F[1]}, \texttt{F[2]}, ..., \texttt{F[79]} which take $32$-bit arguments and return a $32$-bit value, the operation being done bit by bit:
		\begin{itemize}
			\item \texttt{F[t](X,Y,Z) = (X AND Y) OR (NOT(X) AND Z)} for $0\ge t\ge 19$
			\item \texttt{F[t](X,Y,Z) = (X XOR Y) XOR D } for $20\ge t\ge 39$
			\item \texttt{F[t](X,Y,Z) = (X AND Y) OR (X AND Z) OR (Y AND Z)} for $40\ge t\ge 59$
			\item \texttt{F[t](X,Y,Z) = X XOR Y XOR Z} for $60\ge t\ge 79$
		\end{itemize}
		What is important with these $80$ functions is that if the bits of their arguments \texttt{X}, \texttt{Y} and \texttt{Z} are independent, the bits of the result are also independent.
	
		\item Iterative calculation:
		
		The iteration uses two buffers, each consisting of the use of $5$ chaining variables. The chaining variables of the first buffer are denoted \texttt{A}, \texttt{B}, \texttt{C}, \texttt{D}, \texttt{E}. The second buffer contains the chaining variables \texttt{H[0]}, \texttt{H[1]}, \texttt{H[2]}, \texttt{H[3]}, \texttt{H[4]}.
	\end{enumerate}

	Moreover, let \texttt{S\string^n} denote the circular shift of \texttt{n} bits to the left, here is the SHA-1 algorithm (if we have the time in the future we will write it properly with the correct LaTeX package...):
	
	\begin{verbatim}
	For t = 16 to 79 Do
     M[t] = S^1(M[t-16] XOR M[t-15] XOR M [t-14] XOR M [t-13]);
	End For
	A = H[0];
	B = H[1]; 
	C = H[2];
	D = H[3]; 
	E = H[4]
	For t = 0 to 79 Do
	     TEMP = S^5(A) + F[t](B,C,D) + E + M[t] + K[t]
	     E = D; 
	     D = C; 
	     C = S^30(B); 
	     B = A; 
	     A = TEMP;
	End For
	H[0] = H[0] + A;
	H[1] = H[1] + B; 
	H[2] = H[2] + C; 
	H[3] = H[3] + D; 
	H[4] = H[4] + E;
	\end{verbatim}
	\begin{tcolorbox}[title=Remark,colframe=black,arc=10pt]
	Since we have written this text on SHA-1, new versions, sometimes significantly different of SHA-1 have been released. The SHA-0 was released in 1998, the SHA-1 presented above is a minor correction of the SHA-0, the SHA-2 was released in 2001 and finally the SHA-3 in 2012.
	\end{tcolorbox}
		
	To sum up a bit, it must be clear that MD5 (Message-Digest algorithm 5) is a cryptographic hash function, while Advanced Encryption Standard (AES) or RSA are symmetric-key encryption algorithms, so they are used for different purposes. A hash, like MD5 or SHA are used to verify passwords because they are hard to invert, that is, to obtain the password from the hash-string. An AES or RSA encryption, on the other hand, are invertible, the original message can be obtained if we know the key. 
	
	
	\subsection{Certificate based authentication}
	We saw during our study of public key and secret key cryptography that there was an issue in the system of transmission of the keys at the beginning of the communication.

	Thus, in both systems, the issue lies in the fact that a malicious person ("man-in-the-middle" attack) can replace the real interlocutor and send either a false secret key or a false public key (depending on the case).

	Thus, a certificate of authenticity makes it possible to associate a key with an entity (a person, a machine, etc.) in order to ensure its validity (association with the "real person"). The certificate is in a way the identity card of the key or the "\NewTerm{digital signature}\index{digital signature}", issued by an organization named "\NewTerm{certification authority}\index{certification authority}".

	The technologies using digital signatures are part of a larger set known as "\NewTerm{Public Key Infrastructure (PKI)}\index{public key infrastructure}". The whole takes place by means of certificates which you can obtain from a Certification Authority (see example below). When you request your certificate, your computer creates the key pair consisting of a private key (the yellow on the schema) and a public key (the black one). Your private key is secret and it is only you who have access to it while the public key is freely available for everyone. Your public key will be attached to your certificate that you will get from the certification authority to whom you have submitted your certificate request.

	The PKI (on which the IPSec connection is based) essentially targets $4$ important points:
	\begin{enumerate}
		\item The authentication (the recipient of your email must be able to verify that it is you who sent the object and not another individual).

		\item Integrity (ensure that the content has not been changed along the way).

		\item Confidentiality (ensuring that the content is readable only by the recipient).

		\item Non-repudiation (arising from the first 3 points)
	\end{enumerate}
	The certification authority is responsible for issuing the certificates, assigning them a validity date, and possibly revoking certificates before that date if the key is compromised.
	
	Certificates are small files divided into two parts:
	\begin{itemize}
		\item The part containing the information
		
		\item The part containing the signature of the certification authority (see Microsoft Internet Explorer browser for an example)
	\end{itemize}

	The certificate structure is standardized by the International Telecommunication Unification (ITU) standard X.509, which defines the information contained in the certificate:
	\begin{itemize}
		\item The name of the certification authority (VeriSign for example)

		\item The name of the owner of the certificate (the UBS bank for example)

		\item The date of validity of the certificate ($X$ day from the current date)

		\item The encryption algorithm used (MD5RSA)

		\item The owner's public key
	\end{itemize}
	Here is a quite good schematic example:
	
	To sign the message you are sending (point \circledtext{5} in the figure below), it is sufficient to apply a hash function (point \circledtext{1}) which produces a summary (hash code) of the message (using MD5 or any version of SHA). The summary (fingerprint) obtained is (almost...) specific to each message, like the image of a fingerprint we know that a hash algorithm ensures that if a single bit of the original text is modified and a new hash is made, the latter will with a very high probability radically different from the first one, and the hashed code can then be encrypted using your private key (\circledtext{2} and \circledtext{3}), this results constitutes the "digital signature". The recipient of the message (point \circledtext{6}) can then verify that you are the sender by encrypting the digital signature (point \circledtext{7}), by means of your public key (point \circledtext{8}), that you transmitted to it automatically with the mail (point \circledtext{4}), to get the hashed code (point \circledtext{9}). The recipient then applies the same hash function to the received message (point \circledtext{10} in the diagram). If the two codes (points \circledtext{11} and \circledtext{12} on the schema) are identical, you are the sender of the message (authentication) and the message has not been altered (integrity).
	\begin{figure}[H]
		\centering
		\includegraphics[scale=0.9]{img/computing/principle_of_digital_signatures.jpg}
		\caption{Principle of digital signatures}
	\end{figure}
	All this looks very complicated, but in practice, depending on the software, you just have to click on one or three buttons on the screen to start the whole process.

	Otherwise let's see another figure involving now a Certificate authority:
	\begin{figure}[H]
		\centering
		\includegraphics[scale=1]{img/computing/certificate_authority.jpg}
		\caption[Principle of certificate authority]{Principle of certificate authority (source: Pour la Science)}
	\end{figure}
	Where we have:
	\begin{enumerate}
		\item Alice uses a private key ($a$) as well as a public key ($b$) received from a certificate authority that has typically transmitted the private and public keys to Alice in a smart card containing a digital certificate ($c$). This certificate also includes the signature of the certificate authority, which can be verified by any person (or software) who knows or has access to the public key of this organization.
		\begin{figure}[H]
		\centering
		\includegraphics[scale=1]{img/computing/quovadis_suisse_id.jpg}
		\caption[]{Example of digital certificate smart card used by the author of this chapter}
	\end{figure}
		
		\item The public key ($d$) of the certificate authority is provided to those who need it, for example Bob. This key can be included in the web browsers programs and in other software used for secure computer communications.
		
		\item Alice digitally signs the message she sends to Bob. First, she creates a digest of the message by applying a hash function to it. The digest thus created is then encrypted using the secret key of Alice which gives the digital signature of the message ($e$). This signature is sent to Bob at the same time as the encrypted message ($f$) and the public key.
		
		\item Bob uses the public key of the certificate authority to verify that the official digital signature on the certificate is authentic and that the accompanying public key is that of Alice. He then uses this key to decrypt Alice's digital signature and gets the digest of the message. Finally, Bernard applies the hash function to the message sent by Alice and thus gets a digest of the message. If this digest is identical to that obtained by Alice's numerical encryption, Bob is sure that the message comes from Alice and has not been altered by a third person.
	\end{enumerate}
	
	\pagebreak
	\subsection{Quantum cryptography}\label{quantum cryptography}
	"\NewTerm{Quantum cryptography}\index{quantum cryptography}" is a marketing expression, but somewhat misleading: it is not a question of encrypting a message using quantum physics, but of using quantum physics to ensure that the transmission of the key has not been spied. Currently used popular public-key encryption and signature schemes (RSA) can be broken by quantum adversaries. The advantage of quantum cryptography lies in the fact that it allows the completion of various cryptographic tasks that are proven or conjectured to be impossible using only classical (i.e. non-quantum) communication (see below for examples). For example, it is impossible to copy data encoded in a quantum state and the very act of reading data encoded in a quantum state changes the state. This is used to detect eavesdropping in quantum key distribution.

	Indeed, as we have already explained it in the section of Quantum Computing, the transmission of a message, encrypted or not, can be done using the two orthogonal linear polarization states of a photon, for example $|x\rangle$, $|y\rangle$. We can decide to assign by convention the value $1$ to the polarization $|x\rangle$ and the value $0$ to the polarization $|y\rangle$: each photon therefore carries one bit of information. Any encrypted or unencrypted message can then be written in binary language, such as a sequence of $0$ and $1$, and the message $1001110$ will be encoded by Alice thanks to the sequence of photons $|x\rangle |y\rangle |y\rangle |x\rangle |x\rangle |x\rangle |y\rangle$, which she will send to Bob for example by an optical fiber. Using a birefringent plate, Bob separates the photons with vertical and horizontal polarization and two detectors placed behind the slide allow him to decide whether the photon was polarized horizontally or vertically:
	\begin{figure}[H]
		\centering
		\includegraphics{img/computing/photon_polarization_experiment.jpg}
		\caption{Thought experiment for polarization measurement}
	\end{figure}
	
	The whole process and protocol can be summarized by the excellent following figure:
	\begin{figure}[H]
		\centering
		\includegraphics[scale=0.65]{img/computing/quantum_cryptography.jpg}
		\caption{Quantum key distribution}
	\end{figure}
	\begin{tcolorbox}[title=Remark,colframe=black,arc=10pt]
	The protocol described above is named BB84, named after its inventors Bennett and Brassard.
	\end{tcolorbox}
	Let us now turn to the formal part (we strongly recommend the reader to first take a look the section of Quantum Computing!).

	The states of the quantum system are the states of polarization of a photon: the measurements (of the observable) will also have its polarization states. Possible measures will include:
	
	we will denote the corresponding states $|0\rangle$ and $|1\rangle$ (orthonormal basis of the space of the states of polarization): it is the base H/V (Horizontal/Vertical).

	Let us consider several cases:
	\begin{enumerate}
		\item[C1.] Given a photon in the state $|\Psi\rangle=|0\rangle$ then as we have seen in the section Quantum Computing, we will have:
		

		\item[C2.] Or a photon in the state:
		
	\end{enumerate}
	And it customary to write the sequence of the key as following:
	
	\begin{tcolorbox}[title=Remarks,colframe=black,arc=10pt]
	\textbf{R1.} Let us recall that this (famous) value is chosen for normalization purposes such as that $\langle \Psi|\Psi\rangle$!!! Many people ask the question of where the square root comes from in Quantum Computing? The answer is simply for normalization as we have detailed it in the section of Quantum Computing\\

	\textbf{R2.} Let us also recall that the photons $|\Psi_{01}\rangle$ and $|\Psi_{11}\rangle$ are not polarized in the direction "$|0\rangle+|1\rangle$" (ie in the oblique direction) but are in a quantum superposition of these two polarizations!
	\end{tcolorbox}
	Then for example (we apply as we saw in the section of Quantum Computing, the test $|0\rangle$ to the state $|\Psi_{01}\rangle$):
	
	and:
	
	\begin{tcolorbox}[title=Remark,colframe=black,arc=10pt]
	Let us recall that in this book, we write in Quantum Physics the module of a complex number and the norm, indistinctly by the symbol $||{}\|$ therefore caution to the possible confusions!
	\end{tcolorbox}
	
	\subsection{Alternative cryptography}
	Mathematicians sometimes venture out of the beaten path of Number Theory: they invent cryptosystems based on braids or networks (see the corresponding sections of Knot Theory or Graph Theory). Physicists are not left behind and offer methods of encryption that use the theory of chaos or quantum physics. The latter would provide a definitive solution to the delicate problem of key exchange and jeopardize cryptosystems based on factorization.

	Most of these methods are outside the scope of this book for the moment but we can give however a non-exhaustive list:
	\begin{itemize}
		\item The LLL algorithm based on the mesh structure of sets of numbers and based on the Minkowski theorem ensuring that the content of a disc of given radius at a point contains at least one other point of the network

		\item The ultravariable cryptography in which the data pass through systems of superimposed quadratic equations.

		\item Optical hyperchaos, obtained by the passage of a LASER in a IKEDA ring in which a non-linear wavelength material is integrated.

		\item ...
	\end{itemize}
	The future will tell us the rest!
	
	
	


	

	\begin{flushright}
	\begin{tabular}{l c}
	\circled{50} & \pbox{20cm}{\score{2}{5} \\ {\tiny 12 votes,  50.00\%}} 
	\end{tabular} 
	\end{flushright}

	%to make section start on odd page
	\newpage
	\thispagestyle{empty}
	\mbox{}
	\section{Quantum Computing}\label{quantum computing}
	\lettrine[lines=4]{\color{BrickRed}Q}uantum computing (we should rather speak of "\NewTerm{quantum calculation}\index{quantum calculation}" because we are currently very far from an input / output system) is a beautiful example of the use of specific theoretical models of quantum physics for treatment and the transmission of information.\\
	
	However it must also be remembered that the behaviour of transistors etched on the chip in your computer could not be imagined in 1947 by Bardeen, Brattain and Shockley that from their knowledge of quantum physics. So all of our electronic devices already operating on the basis of semiconductors operate with developments achieved through quantum physics.
	
	The big news, since the early 1980s, is the ability of physicists to manipulate and observe individual elementary quantum objects: photons, atoms, ions, etc. It is this ability to manipulate and observe basic quantum objects that is the cause of quantum information, where these elementary quantum objects will physically build the "\NewTerm{qubits}\index{qubits}" (for "Quantum Bit"). That said, no fundamentally new concept has been introduced since the 1930s and the founding fathers of quantum physics (Heisenberg, Schrödinger, Dirac, Planck, Einstein, etc.), if they revived today, would not be surprised by quantum computing, even if they would surely be surprised by the prowess of the experimenters who now realize experiences qualified in their era of "gedanken experiment" (imaginary experiment that was impossible to do in laboratory).
	
	It is also interesting to notice that the increasing miniaturization of electronics will find its limits because of quantum effects, which will become essential below the nanometer. Thus, we believe that Moore's Law (which assumption that the computing power of machines doubles roughly every $18$ months) may not be correct anymore by the years 2015-2020.
	
	It is likely that the trend of the study of quantum physics and its application to quantum information (and quantum electronics and quantum telecommunications) will explode in the coming decades (especially towards the end of the 21st century). Thus, engineering schools will integrate in all study field Quantum Physics in school curricula. What physicists studying for soon already almost 100 years in their curriculum.
	
	Before moving to the formal side, we felt, however, interesting to make a small popularized passage because we noticed that it helps to understand the calculations that will be made thereafter.
	
	In the 70 and 80, the first quantum computers are born from the minds of physicists such as Richard Feynman, Paul Benioff, David Deutsch and Charles Bennett. Feynman's idea was that instead of complaining that the simulation of quantum phenomena demand enormous powers to our to days computers today, that we use the power of quantum phenomena to make the computers faster than classic computers.
	
	During long time physicists doubted that quantum computers can be used, and even that we can do something viable if they existed. But:
	\begin{itemize}
		\item In 1994, Peter Shor, a scientist of AT\&T shows it is possible to factor large numbers in a reasonable time using a quantum computer. This discovery unlocks suddenly credits for quantum computers research.
		
		\item In 1996, Lov Grover, invented an algorithm based on quantum computers to find an entry in an unsorted database.
		
		\item In 1998, IBM was the first to present a $2$-qubit quantum calculator.
		
		\item In 1999, the IBM R\&D team used the Grover algorithm for fast quantum search on a database (quantum database search) on a calculator with $3$-qubits and beat their record the following year with a $5$-qubit computer.
		
		\item In 2001, IBM created a quantum computer with $7$-qubits and factored the number $15$... thanks to the Shor algorithm (\SeeChapter{see section Numerical Methods page \pageref{quantum computing}}). Calculators with $7$-qubits are built around chloroform molecules and their useful life is no more than a few minutes.
		
		\item In 2007, the Canadian company D-Wave during a demonstration presented a quantum computer with $16$-qubits.
		
		\item In May 2013, Google announced that it was launching the Quantum Artificial Intelligence Lab, hosted by NASA's Ames Research Center, with a $512$-qubit D-Wave quantum computer. 
		
		\item In October 2015 researchers at University of New South Wales built a quantum logic gate in silicon for the first time.
	\end{itemize}
	\begin{tcolorbox}[title=Remark,colframe=black,arc=10pt]
	As a $1$ fermion spin qubit is equivalent to $2$ bits, we have then that $N$ fermion spin qubits are equal to $2^N$ classical bits. Therefore if one day we reach the $100$ fermion spin qubits computer we will have a potential memory of $2^{100}\cong 1\cdot 10^{15}$ Petabytes...
	\end{tcolorbox}
	The memory of a classical computer is made of bits (\SeeChapter{see section Numerical Methods page \pageref{bit}}). Each bit carries either a $1$ or a $0$ (bipolar mode). The machine computes by manipulating those bits. A quantum computer is working on a set of qubits. A qubit can wear either a $1$ or a $0$, or a superposition of a $1$ and a $0$ (or, more accurately, he wears a phase distribution). The quantum computer computes by manipulating those distributions as discussed in detail below.
	
	Query a qubit whose phase angle is not $0^{\circ}$ or $90^{\circ}$ ($\pi/2$) is not very useful: we will get the answer $0$ with a given probability and $1$ with another probability and ... it is possible to construct random generators much cheaper! However, if we manage to create an algorithm that systematically leads it to a phase $0^{\circ}$ or $90^{\circ}$, we get a deterministic result. But is also necessary that it corresponds to a sought response.
	
	A quantum computer could be implemented from any particles that can have two states. It can be built from photons, or from any particle or atom/molecule having a spin.
	
	As we know, a classical computer with one bit can only store numbers with one digit composed of a one or a zero (\SeeChapter{see section Numerical Methods page \pageref{bit}}) for a total of $2^1=2$ states that it must treat separately. At one point, he could contain the bits $1$ for example.
	
	A quantum computer with one qubit can actually store $3$ states as it can have a project state corresponding to $1$ or to $0$ and any superposition of the states $0$ and $1$ with a given probability (the third state!) before being observed (\SeeChapter{see section Wave Quantum Physics page \pageref{quantum superposition}}). When the calculator make the measurement the state superposition is cancelled as we know and therefore the third state can only be used (at least as far as we know) for intermediate and temporary calculations (as they cannot be read with our actual knowledge). This is why today qubits are specialized for given algorithms and we have "quantum calculators" but not "quantum computers".
	
	Before we study the mathematical aspects let us start by the study of one of the most famous cat in the world to better understand (we hope so):	
	
	\subsection{Schrödinger's Cat superposition}
	As we have study it in details in the section of Quantum Sections, unlike in classical mechanics, essentially different states can mix in
quantum mechanics at least until a specific property is measured!

	Now let us recall this famous thought experiment:
	
	\begin{enumerate}
		\item Put a (living) cat in box.
		\item Add a container with deadly poison, that can be remotely released.
		\item Close and seal the box.
		\item Connect the remote to a quantum randomness source (e.g. nuclear decay).
	\end{enumerate}
	In what state" is the cat? Is it alive or dead? Who knows?
	\begin{figure}[H]
		\centering
		\includegraphics{img/computing/schrodinger_cat_experiment.jpg}
		\caption[Schrödinger Cat experiment]{Schrödinger Cat experiment (source: Wikipedia)}
	\end{figure}
	As long as no one checks (!!!), it a sensible way to think about this to
consider the cat being in an intermediate state:
	
	As long as no one checks (!!!) the cat is in a superposition of two
states that are classically impossible to consider at the same time.

	Now let us start by understanding the underlying concepts of quantum theory and of quantum computing with the study of the polarization of the photon.
	
	Now if we open the box obviously, we will see ("measure") either a dead or alive cat.
	
	By "measuring" the system, we put it back in a classical, pure state! As we knot it already: Measuring affects the system!
	
	The Cat set of states will be written:
	
	Take a second cat, same arrangement and consider a 2-cat system:
	
	That is to say with two cats:
	
	Therefore as we can see the central idea of quantum computing is manipulating the $N$-qubit system to change at the same time ALL $2^N$ pure states probabilities but when the $N$-qubit system will be measured,
only a single pure state will be picked out!

	A quantum algorithm must therefore modify the probability distribution of the quantum system, such that the correct "result" state has an (almost) $100\% $ probability.
	
	\subsection{Photon polarization}
	Since Albert Einstein, we know that light is composed of photons, or particles of light, and that it has a dual wave-particle appearance (\SeeChapter{see section Wave Optics page \pageref{young interference experiment}}). If we reduce the light intensity of a beam of photons, we should be able to study the polarization of individual photons, that we know perfectly how to detect using photomultipliers. Suppose that the experiment detects $N$ photons. When $N\rightarrow +\infty$, we must fall back on the results of wave optics (see section of the same name page \pageref{wave optics}).
	
	Let us perform by example the following experience:
	\begin{figure}[H]
		\centering
		\includegraphics{img/computing/photon_polarization_experiment.jpg}
		\caption{Thought experiment for polarization measurement}
	\end{figure}
	A birefringent plate separates a light beam whose polarization is makes an angle $\theta$ with O$x$ in a subsequent beam polarized  following O$x$ and another polarized following O$y$, the intensities being respectively $I\cos^2(\theta)$ and $I\sin^2(\theta)$ (according to the proof of Malus Law made in the section of Wave Optics).
	
	Let us reduce the intensity so that the photons arrive one by one, and let us place two photodetectors $D_x,D_y$ behind the blade. The experiment shows that $D_x,D_y$ never click together at the same time (except in cases of "dark count" when a counter is triggered spontaneously because of background noise): a photon comes entirely either on $D_x$, either on $D_y$, a photon therefore can not divide itself. On the other hand, the experiment shows that the probability $P_x$ (respectively $P_y$) of detecting a photon by $D_x$ (respectively $D_y$) is equal to $\cos^2(\theta)$ (respectively $\sin^2(\theta)$). So if the experiment detects $N$ photons, we will have $N_x$ (respectively $N_y$) photons detected by $D_x$ (respectively $D_y$):
	
	where the $\cong$ takes int account the statistical fluctuations. As the light intensity is proportional to the number of photons, we fall indeed back on the Malus law the limit $N\rightarrow +\infty$.
	
	However, we notice two problems:
	\begin{enumerate}
		\item Can we predict, for a given photon, it will trigger $D_x$ or $D_y$? The answer of quantum theory is: NO, statement has deeply shocked Albert Einstein (God does not play dice!). Some physicists (including Albert Einstein) have been tempted to assume that quantum theory was incomplete, and that there were "hidden variables" whose knowledge would provide the individual destiny of each photon. Under very reasonable assumptions on which we will come back, we now know that such hidden variables are excluded. The probabilities of quantum theory are, as we know (\SeeChapter{see section Wave Quantum Physics page \pageref{wave quantum physics}}), intrinsic! They are not related to an imperfect knowledge of the physical situation, as is the case for example in the game of heads or tails.

		\item If we combine the two beams of the first birefringent plate, using a second blade that is symmetrical to the first:
		\begin{figure}[H]
			\centering
			\includegraphics{img/computing/photon_polarization_experiment_recombining_photons.jpg}
			\caption[]{Imaginary experience recombining the two beams}
		\end{figure}
		and if we seek the probability that a photon passes through the analyser, a photon can choose the path $x$ with a probability $\cos(\theta)^2$, then it has a probability $\cos^2(\alpha)$ to cross the analyser thus a total probability of $\cos^2(\theta)\cos^2(\alpha)$. If it chooses the path $y$, they will have a probability $\sin^2(\theta)\sin^2(\alpha)$ to cross the analyser. The total probability is thus obtained by summing the probabilities of the two possible options:
		
		This result is FALSE! Indeed, classical optics tells us that the intensity is $I\cos^2(\theta-\alpha)$ (\SeeChapter{see section Wave Optics page \pageref{malus law}}) and the correct result is confirmed by experience:
		
		Which is not the same at all!
		
		In fact, to fall back on the results of optics wave, it must be remembered that the probability in quantum physics is obtained through the norm to the square of the probability amplitude (\SeeChapter{see section Wave Quantum Physics page \pageref{de broglie normalization condition}}). Therefore:
		
		and we must add the amplitudes for indistinguishable paths and using basic trigonometric relations, we get:
		
		which gives well:
		
	\end{enumerate}
	Suppose we have a way of knowing whether the photon takes the path $x$ or the path $y$ (impossible in our case, but analogue experiments answering to the question "What path?" were realized with atoms). We could then divide the photons into two classes, those who "chose" the path $x$ and those who "chose" the path $y$.
	
	For photons having chosen the path $x$, we may block the path $y$ by a cache without changing anything, and vice versa for photons having chosen the path $y$ we could block the path $x$. Obviously, the result can then only be ${P'}_\text{tot}$. If we can discriminate between the paths, the result is not the same anymore, the paths are no longer indistinguishable. Under the experimental conditions where it is impossible in principle to distinguish between the paths, we can say either:
	\begin{enumerate}
		\item Either the photon use the both paths at once (...)

		\item Or it make no sense to ask the question "Which path?", Since the experimental conditions do not allow to answer.
	\end{enumerate}
	It must be notice that if the experience allows to decide between two paths, the result is ${P'}_\text{tot}$, even if we decide not to observe them. It is just enough that the experimental conditions allow, in principle, to distinguish between the two paths.
	
	\subsection{Qubit}
	Ultimately, the idea of quantum computing is therefore to connect
quantum gates in a suitable fashion while protecting the superposition
between the $N$ qubits from any external influence.

We can use the polarization of photons to transmit information, for example by an optical fiber. We decide quite arbitrarily, to assign the value of $1$ bit to a photon polarized along O$x$ and $0$ to a photon polarized along O$y$.

	To investigate the theory, it has become traditional to imagine that the two people who exchange information are conventionally named Alice (A) and Bob (B)... Alice sends to Bob for example the following sequence of polarized photons:
	
	Bob analyses the polarization of these photons with a birefringent plate and derives the message from Alice:
	
	This is obviously not a very efficient way to exchange messages, but it is the basis of quantum cryptography (\SeeChapter{see section Cryptography page \pageref{quantum cryptography}}). However, the interesting question now is: what is the value of the bit that we can attribute for example to a photon polarized at $45^\circ$...? According to the above results, a photon polarized at $45^\circ$. is a linear superposition of photon polarized along O$x$ and a photon polarized along O$y$. A qubit is therefore an entity much richer than a regular bit, which cannot take in the strict sense only the values $0$ and $1$.
	
	In a sense, a qubit can take all values between $0$ and $1$ and therefore contains an infinite amount of information! However, this optimistic statement was immediately denied when we realize that the measurement of qubit can give only the result $0$ or $1$, regardless of the chosen base. However, we can ask ourselves the question of this "hidden information" in the linear superposition and we will see that we can exploit it under certain conditions.
	
	To take into account for the possibility of linear superposition, it is natural to introduce for the mathematical description of the polarization a complex vector space (cause: phasers) in two dimensions corresponding to the polarization plane as we saw it in the section of Wave Optics. We denote this vector space $\mathcal{H}$ (we take again the notation of Hilbert spaces) and name the "\NewTerm{Hilbert space of polarization states}\index{Hilbert space of polarization states}".

	We may well decompose the vector corresponding to linear polarizations O$x$ and O$y$ in two ket vectors equation and equation such that any polarization state (whether linear, circular, or other) may decompose according to this basis:
	
	Thus, a linear polarization is described by real coefficients $\lambda,\mu$ but the description of a circular or elliptical polarization obviously require to use complex coefficients!

	The probability amplitudes will correspond to a scalar product on this space. So given two vectors corresponding to two different polarizations:
	
	The Hermitian dot product (\SeeChapter{see section Vector Calculus page \pageref{hermitian inner product}}) will be:
	
	Now a linear polarization state (\SeeChapter{see section Wave Optics page \pageref{linear polarization}}) following $\theta$ will be given logically  by (if we restrict ourselves to the linear case so!):
	
	where $|x\rangle$, $|y\rangle$ are vectors having unit norms. This is consistent with the mental representation:
	\begin{figure}[H]
		\centering
		\includegraphics{img/computing/recall_of_field_decomposition.jpg}
		\caption[]{Reminder of field decomposition principle}
	\end{figure}
	where the amplitude is normalized to the unit.

	The probability amplitude for a polarized photon following $\theta$ goes through an analyser oriented following $\alpha$ can now be written:
	
	and the probability of passing through the analyser will always be given by the squared norm of this amplitude as we have prove it earlier above:
	
	In general, we define probability amplitudes, where $|\Phi\rangle$, $|\Psi\rangle$ are polarization states:
	
	and the corresponding probability will be:
	
	We are now ready to tackle the crucial issue of the measure as part of this quantum experiment. Let take again the polariser / analyser experiment, assuming that the analyser is oriented along O$x$. If the polariser is oriented along O$x$, an outgoing photon passes through the polariser analyser with a probability of $100\%$; if the polariser is oriented along O$y$, the probability is $0\%$. The analyser performs a (polarization) test, and the test result is $1$ or $0$. The test allow us to determine the polarization state of the photon.

	But this is not the general case!
	
	Let us assume that the polariser is oriented following the general direction $\theta$ or the orthogonal direction $\theta_{\perp}$ (there is a rotation of $\pi/2$). We then use the properties of the unitary trigonometric circle:
	
	and therefore if the polariser prepares for example the photon in the state $|\theta\rangle$ and the analyser is oriented along O$x$, the probability of success of the test will always be $\cos^2(\theta)$ whatever the type of polarization!! Let us recall that in this example, after the passing through the analyser, the polarization state of the photon is no longer $|\theta\rangle$, but $|x\rangle$. The measurement therefore changes the polarization state.
	
	\begin{tcolorbox}[title=Remark,colframe=black,arc=10pt]
	Of course, another way of seeing that the two vectors above are orthogonal is to make a scalar product and to see that the result is immediately equal to zero.
	\end{tcolorbox}
	We see a difference of principle between measurement in classical physics and measurement in quantum physics. In classical physics, the physical quantity to be measured predates the measurement: if a radar measures the speed of your car at $180$ [km$\cdot$h$^{-1}$] on the highway, this speed pre-existed to its measure by the policeman. On the contrary, in the measurement of polarization of a photon $|\theta\rangle$ by an analyser oriented along O$x$, the fact that the test gives a polarization according to O$x$ does not make it possible to conclude that the photon tested had previously its polarization following O$x$.

	So we have a device preparing the quantum system in the state $|\Phi\rangle$ and a second one capable of "preparing" it in the state $|\Psi\rangle$ that we will use as an analyser. After the test, the quantum system will therefore be in the state $|\Psi\rangle$, which means mathematically that we realize an orthogonal projection on $|\Psi\rangle$.

	Let $\mathcal{P}_\Psi$ this projector, then the vectorial orthogonal projection (\SeeChapter{see section Vector Calculus page \pageref{dot product}}) is given by:
	
	which consists (for recall) of a simple scalar product (scalar orthogonal projection) multiplied by the vector $|\Psi\rangle$. This is easily seen by judiciously placing the parentheses:
	
	and therefore:
	
	The projection of the state vector is named, as we have already seen it (in the section of Wave Quantum Physics), in the Copenhagen interpretation of quantum physics "state vector reduction", or, for historical reasons, "\NewTerm{reduction of the wave packet}\index{reduction of the wave packet}". This reduction of the state vector is a convenient fiction of the Copenhagen interpretation, which avoids having to ask questions about the measurement process ...

	The reader accustomed to Linear Algebra (see section of the same name page \pageref{linear algebra}) will have probably notice trivially that we can manipulate the notation convention of the projector as a matrix (linear mapping) such as in two simple particular cases (those of interest to us):
	
	A reader has asked us to explicit this matrix approach. Let us see how we arrive at this matrix aspect of the orthogonal projection with a particular example of two vectors in a real two-dimensional space. For this let us consider:
	
	and therefore (the procedure is the same for $y$):
	
	\begin{tcolorbox}[title=Remark,colframe=black,arc=10pt]
	The matrix notation of the orthogonal projector is often presented as a definition of a mathematical tool named "\NewTerm{outer product}\index{outer product}".
	\end{tcolorbox}
	So finally, to get back to our previous subject, we have:
	
	and same for the other component.
	
	As:
	
	We then have:
	
	We will notice that the identity operator can be written as the sum of the two projectors $\mathcal{P}_x$,$\mathcal{P}_y$:
	
	relation named "\NewTerm{closure relation}", which can be generalized to an orthonormal basis of a Hilbert space $\mathcal{H}$ of dimension $N$:
	
	Moreover, the projectors $\mathcal{P}_x$,  $\mathcal{P}_y$ commute (trivial verification):
	
	Thus, the tests $|x\rangle$, $|y\rangle$ are compatible (whatever the direction of the measurement the result is independent). In contrast, the projectors $\mathcal{P}_\theta$, $\mathcal{P}_{\theta,\perp}$:
	
	which satisfy (trivial verification) to:
	
	as well as (trivial verification):
	
	Do not commute with $\mathcal{P}_x$,$\mathcal{P}_y$:
	
	and therefore the tests $|x\rangle$ and $|\theta\rangle$ are incompatible.

	For the next developments, it will be useful to notice that the knowledge of the probabilities of success of a $T$ test makes it possible to define an average value (expected mean):
	
	In analogy with the context, we can read this as following: the expected mean of the test is equal to the representative value of the photon oriented according to O$x$ (corresponding arbitrarily  to the value $1$) multiplied by the probability of passing through the analyser oriented also according to O$x$ (therefore test concluding at $100\%$) summed with the representative value of the photon oriented according to O$y$ (corresponding arbitrarily to the value $0$) multiplied by the probability of passing through the analyser always oriented along O$x$ (therefore $0\%$ of the photons O$y$ will pass the O$x$ test).

	For example, if the test $T$ is represented by the procedure $|\Phi\rangle$ and we apply it to a state $|\Phi\rangle$ (containing as we have seen above the  representative values of the linearly or non-linearly polarized photon...) then the test corresponds to a scalar product:
	
	
	And as we have seen it in the section of Wave Quantum Physics, we know that in fact:
	
	is the mean value of an operator $M$ in the state $|\Phi\rangle$. Thus, to the test $T$ to which a procedure $|\Psi\rangle$ is associated, we can associate the projector $\mathcal{P}_{\Phi}$ whose mean value in the state $|\Phi$ gives the probability of success of the test.

	The generalization of this observation makes it possible to construct the physical properties of a quantum system. Let us give an example by returning to the case of polarization. Suppose that we construct (quite arbitrarily) a property $\mathcal{M}$ of a photon as follows:
	\begin{itemize}
		\item $\mathcal{M}$ is equal to $+1$ if the photon is polarized following O$x$

		\item $\mathcal{M}$ is equal to $-1$ if the photon is polarized according to O$y$
	\end{itemize}
	We can associate with the physical property $\mathcal{M}$ the Hermitian operator:
	
	Which satisfies (trivial) the relation between operator, eigenvalue and vector:
	
	The mean value (expected mean) of $M$ then being (by definition):
	
	Let us assume the photon in the linear polarization state of angle $\theta$, then the mean value $\rangle M \langle_\theta$ in the state $|\theta\rangle$ is (trivial):
	
	Before seeing, how can we construct such an operator $M$ with another object than the photon and with the same properties, let us introduce a mathematical tool generalizing the conditions and the configuration of any polarized wave:
	
	\subsubsection{Bloch sphere}
	The Bloch sphere is, as we will see, a geometrical representation of the pure states of the qubits (two-level quantum mechanical system) as points of the surface of a sphere, named after the physicist Felix Bloch

	A given number of elementary operations done in quantum computing can be carried out with this sphere under the choice of a suitable projector .

	We will see that a state of an arbitrary qubit (vector in the complex plane) can be written:
	
	where $\gamma\in\mathbb{R}$ and $0\le \phi\leq 2\pi$, define a point on the three-dimensional Bloch sphere and where we have the two basic vectors:
	
	for which sometimes the definition is reversed (but it does not matter as long as it forms an orthogonal basis!).

	The qubits represented by arbitrary values $\gamma$ (global gauge invariance according to $U(1)$ as seen in the section of Set Algebra page \pageref{set algebra}) are all represented by the same point on the Bloch sphere because we will show that the factor has no observable effect and that we can then write without losing in generality:
	
	
	which is represented as we will justify later by the figure below:
	\begin{figure}[H]
		\centering
		\includegraphics[scale=1]{img/computing/bloch_sphere.jpg}
		\caption{Bloch-sphere (two dimensional one)}	
	\end{figure}
	
	\begin{tcolorbox}[title=Remark,colframe=black,arc=10pt]
	Let us assume that we have a Quantum System with $k$ qubits:
	
	The qubit will be in a superimposed state:
	
	where $\alpha_i$ is a complex vector with the property $\sum \alpha_i =1$.
	\end{tcolorbox}	
	
	The Bloch sphere is a generalization of the representation of a complex number $z$ (\SeeChapter{see section Numbers page \pageref{complex numbers}}) with $|z|^2=1$ as point of the circle in the (Gauss-) plane.

	We also saw in this same section that a complex number could be represented by a complex exponential such that:
	
	and if the circle were unitary:
	
	Let us notice that the constraint $|z|^2=1$ eliminates a degree of freedom.

	We will now notice the decomposition of a state of polarization in the form:
	
	And this one in a more traditional form in quantum computing (logic when we see the bases ...):
	
	where $\alpha,\beta\in\mathbb{C}$ (yes! indeed we are not anymore in the simple case of a wave linearly polarized now ...!) without forgetting the condition of normalization:
	
	We can therefore write the qubit in the form:
	
	
	\begin{tcolorbox}[colback=red!5,borderline={1mm}{2mm}{red!5},arc=0mm,boxrule=0pt]
	\bcbombe Caution! It is very important to have read the part dealing with the polarization of light in the section of Wave Optics to understand that it does not come from nowhere!!! With the difference that we do not work here with phasors because the solution of Schrödinger's equation of evolution involves complex exponentials as we saw in the context of the resolution of the latter for the eigen-mode of free particle.
	\end{tcolorbox}
	Adding an overall phase factor should have no influence on the coefficients $\alpha$, $\beta$ because:
	
	and similarly for $|\beta|^2$. Thus, we are free to multiply our polarized and standardized qubit:
	
	by the global phase $e^{-\mathrm{i}\phi_a}$ which gives:
	
	In addition, we always have the condition of normalization $\langle \Psi' | \Psi\rangle$ to respect (impose).
	
	Returning to the Cartesian coordinates, we have:
	
	And the normalization constraint then gives:
	
	what is the equation of a unit sphere in $\mathbb{R}^2$ space with the Cartesian coordinates $(x,y,r_\alpha)$. Hence the quantum origin of the Bloch sphere!
	
	We know (\SeeChapter{see section Vector Calculus page \pageref{spherical coordinates}}) that the Cartesian coordinates are connected to the spherical coordinates by the relations:
	
	therefore by putting $z:=r_\alpha$ and remembering that $r=1$, we can write:
	
	We now have only two useful parameters to know to define our point on the unit sphere (and always to the arbitrariness of phase).
	
	The reader will notice that unlike the linearly polarized qubit, the general case above adds a complex term and that inversely by removing this additional term, we fall back on the relation of the linearly polarized wave seen at the beginning of this section. However, notice (for general culture) that if we add the same complex term to the first term, then we also have the very common following representation of the linearly polarized wave whose writing is named "\NewTerm{Jones vector}\index{Jones vectors}":
	
	But we have not finished yet!

	Let us return to (we remove the apostrophe for the state of polarization):
	
	and notice that:
	
	and:
	
	without forgetting that $\phi\in\mathbb{R}$ and that in this case the exponential factor in front of the $|1\rangle$ is a global phase change without influence.

	All this suggests that $0\le \theta \le \pi/2 $ is sufficient to describe any state of polarization and hence all points of the Bloch sphere.

	On the other hand, we can see that in the system $(r,\theta,\phi)$ the point of coordinates $(1,\pi-\theta,\phi+\pi)$ is the point opposite to that of coordinates $(1,\theta,\pi)$:
	
	We also have (this is quite immediate but we can detail on request):
	
	and therefore opposite points on the Bloch's sphere correspond orthogonal qubits (states of polarization)!

	Thus we can consider only the upper hemisphere of the Bloch sphere since the opposite points differ only from a phase factor $-1$ and are therefore equivalent in the representation of the Bloch sphere.

	Thus, the relation:
	
	is sufficient to describe the whole Bloch sphere in a complex space of dimension $2$.

	By construction, each point given by the previous relation of dimension $2$ contains a double representation of a rotation in the real space of dimension $3$.

	We have also seen in the chapter of Spinor Calculus that a rotation could be written in the form:
	
	With for recall the Pauli Matrices (\SeeChapter{see section Relativistic Quantum Physics page \pageref{pauli matrices}}):
	
	Either with the usual (traditional) writing of the field of Quantum Computing:
	
	Either after simplifying the last matrix:
	
	Now let us consider the rotation of our polarization state vector (due to a projector):
	
	To get a coefficient of $|0\rangle$ that is real (in order to have an observable in the projection of an axis), we multiply by a phase factor $e^{\mathrm{i}\alpha/2}$ giving:
	
	So to get a rotation around the $z$-axis it enough to change $\phi\rightarrow \phi+\alpha$.

	So if we come back to:
	
	it can be shown in the same way that in a general framework a unitary qubit operator can be written in the observable form:
	
	It is then necessary to choose the angles and the axis of rotation to define the operator completely.
	
	\paragraph{Qubit of polarization}\mbox{}\\\\
	We will come back here on the case of polarization of the photon but this time we will be able to generalize thanks to the formulation of the Bloch's sphere to any type of polarization.

	Let us consider for this a polariser that only passes photons polarized vertically followed by a photodetector, which do a "click" if a photon is detected and nothing else. This device allows us to detect vertically polarized photons.

	Let us translate this in the language of quantum mechanics: The states of the system are therefore the states of polarization of a photon. The measurements of the observable will also have its states of polarization.

	The possible measures are:
	
	We will denote the corresponding states $|x\rangle$, $|y\rangle$. In our configuration, it is then obvious that the couple $(\lambda_0,\lambda_1)$ represents the eigenvalues and $|x\rangle$, $|y\rangle$ the eigenvectors of an operator (which we do not know for know) and that we will therefore denote $\mathcal{P}$.

	As we know, $|x\rangle$, $|y\rangle$ is an orthonormal basis of the space of states (of polarization). This is the base named "H/V base" for "Horizontal/Vertical" and is denoted normally:
	
	Let us now take several cases:
	\begin{enumerate}
		\item Given a photon in the sate:
		
		then:
		
	
		\item Let a photon be in the semi-vertical / horizontal state, that is to say oblique (which can be assimilated to the quantum superposition of its two polarizations):
		
		where the root is just there to ensure the normalization condition:  $\langle \Psi|\Psi\rangle=1$. Indeed:
		
		Then since there is superposition (ie half of each in the total wave):
		
		
		\item Let us now take any polarization (and this is what we did not have before!):
		
		which is well-normalized as we know it. So:
		
		The sum of the two probabilities giving indeed $1$!
	\end{enumerate}
	Now, let's imagine that we turn the polariser of $\pi/4$. We will denote the new basis of this polariser $|0'\rangle$, $|1'\rangle$ determined by a rotation of angle $\pi/4$ with by construction:
	
	where the first base therefore corresponds to the diagonal polarization and the second is named "antidiagonal polarization". It is therefore the "\NewTerm{D/A base}" (Diagonal / Antidiagonal). In the form of the Jones vectors the last two relations are written:
	
	If we imagine that we have two polarisers that follow one another. The first having the D/A base and the second the H/V base, the first will prepare the photon polarized generally in a particular state (polarization) which will be by construction be the oblique state for the base H/V. Thus our second polariser will have only situations of the type:
	
	Thus, this shows that any measure obviously disturbs the state of polarization of the photon and thus disrupts the state of the system. This last result is used in quantum cryptography!

	We notice by the way that by using the spin operator introduced during our study of the chapter on Wave Quantum Physics we have:
	
	which is of the same form as the following relation (relation linking eigenvector and eigenvalue) obtained in the in the section of Wave Quantum Physics:
	
	And that indicates indeed that the state:
	
	seems to be necessarily associated with a $1/2$ spin particle. We also notice that this state is an eigenvector of the operator $S_x$ associated with the eigenvalue $\hbar/2$. Indeed:
	
	Since in the present case with the operator $S_x$ we have immediately the eigenvalue and the associated eigenvector without making any intermediate calculations, it comes that the probability of measuring this eigenvalue (refer for recall to the 4th postulate of Wave Quantum Physics which associates operator to a measurement through the eigenvalue) is easy to calculate because the eigenvector is equal to the state vector in this particular case. Therefore:
	
	
	\paragraph{$1/2$ spin Qubit}\mbox{}\\\\
	We will see here how to build a qubit based on a particle with a $1/2$ spin.

	In the study of the Bloch sphere, we have examined a qubit at a given instant and we have seen that in a Hilbert space $H$, this qubit is described (by choice) by a unit vector:
	
	decomposed into an orthonormal basis $(|0\rangle, |1\rangle)$.

	Let us consider the most general and minimum initial state corresponding to an arbitrary orientation of a spin:
	
	Which corresponds, as we know, to two opposite (and superposed) states on the Bloch sphere.
	
	Let us notice that we have indeed a normalized probability of the form:
	
	We have also seen that the projection along $z$ by the operator $R_z(\alpha)$ of the state $\Psi$ is given to the phase arbitrariness by:
	
	Now let u recall our example:
	
	where $M$ was a Hermitian operator. Now the Pauli matrices are simple Hermitian operators. Moreover, as we have demonstrated in the section of Spinor Calculus, the Hermitian operator $\sigma_z$ (assimilated to $M$) has by chance the same eigenvalues and eigenvectors corresponding to the two relations above. But it is then written in a traditional way as we saw it in the section of Spinor Calculus:
	
	or even more condensed:
	
	Moreover, this operator also satisfies the relation:
	
	And what is the physical property associated with $\sigma_z$? Well this is the spin (!) and we will come back to it a bit further below because that means we can use the spin $1/2$ as a qubit.

	Now let us introduce the evolution of the system on this projection because it is not static (but this would not change this particular case to consider it static).

	We have seen in the section of Wave Quantum Physics that this operation consisted in a simple case (as here) of introducing a phase term dependent on time of the type:
	
	Therefore:
	
	What is written more soberly:
	
	Let us recall that the state of a $1/2$ spin particle is two-dimensional and described by the state matrix (\SeeChapter{see section Relativistic Quantum Physics page \pageref{emerging electron spin value}}):
	
	with:
	
	If we want to calculate the expected mean of the observable (physical property) along each axis, then we use the $5$th postulate in the case of the $x$-axis:
	
	Follow the $y$-axis:
	
	and finally following the $z$-axis:
	
	For thus fall back well for the component $z$ (because it is the only one that interests us here), the result which was imposed above in the form:
	
	with an angle difference which is a matter of substitution and an amplitude which makes it possible to match the particularity of the configuration of the system. We therefore have mathematical relations similar in all respects to the manipulation of qubits of oriented spins or qubits of polarized photons.

	The question we can ask ourselves about the spin is how to prepare it in the state $|\Psi\rangle$ In fact, we can do this with a magnetic field by copying the Stern-Gerlach experiment, which allows us to separate a beam of spin $1/2$ particles into two distinct beam.

	Since we now know all the eigenvalues and eigenvectors of the spin operator $S$, we can then determine the general form of the spin operator in any orientation (!):
	
	We have also:
	
	We have the chosen $|\Psi\rangle$ which is, in addition to being a state vector, an eigenvector of $S_n$ with the eigenvalue $\hbar/2$. Therefore the probability of measuring this particular state vector is equal to $1$.

	Now we know by using the $5$th postulate that the probability of finding the eigenvalue $\hbar/2$(of the operator $S_n$), in a measurement of the property $S$ along the $z$-axis performed at time $t$ on the quantum system prepared in the state $\Psi$, is given by the square of the module of the projection of the function or state vector $\Psi$ on the vector or eigenvector $\varphi$ associated with the eigenvalue $\hbar/2$ (and its operator along this axis).

	But the operator according to $z$ is:
	
	For the same operator, we have seen in the section of Spinor Calculus that it had as eigenvectors:
	
	Let us take the first eigenvector oriented according to $z$-axis on the Bloch sphere. We have then:
	
	And the probability according to the other eigenvector would give the same expression but with a sinus. The sum of the two probabilities would then give us well $1$!

	We thus notice that the relations are very similar between the photon and the spin in our case of study. This is normal since both are two-level systems, resulting in similar results.
	
	\subsection{Entangled qubit system}
	A $2$ qubit system is a superposition of all possible $2$ qubit states. A $2$ qubit system and can be represented as:
	
	Measuring the $2$ qubit system, as earlier, results in the collapse of the superposition and the result is one of 4 qubit states. The probability of the measure state is the square of the amplitude $|\alpha_{ij}|^2$.
	
	More specifically a 2 qubit system in which we have:
	
	the tensor product of the $2$ qubits is:
	
	where (notice that we have still a set of orthonormal vectors):
	
	
	\subsection{Quantum logic gates}
	In quantum computing and specifically the quantum circuit model of computation, a quantum gate (or quantum logic gate) is a basic quantum circuit operating on a small number of qubits. They are the building blocks of quantum circuits, like classical logic gates are for conventional digital circuits.
	
	Let us recall that we have stated above that the construction of a quantum qubit was done on the vector basis:
	
	or in extenso, by linearity (and this is all elementary trick!), any transformation that acts on these basic vectors, will therefore act on all qubit of the complex plane.

	Let us consider the particular case for this introduction the logic gate which modifies the state of the $1$-qubit, that is to say qubit which are collinear to one of the basis vectors, into the opposite state. That is to say the "\NewTerm{quantum inverter}":
	
	We quickly guess that the matrix that satisfies this relation (caution! the reader will notice that it is not a special case of the rotation matrix in the plane as seen in the section of Euclidean Geometry!):
	
	And which is sometimes written by specialists:
	
	And we see that this logical gate of negation of the $1$-qubit is nothing else than the first matrix of Pauli. So we can also write:
	
	Negation which is sometimes named "\NewTerm{$X$-Pauli quantum gate}" or "\NewTerm{NOT quantum gate}\index{NOT quantum gate}" and represented by the following symbol in "\NewTerm{quantum circuits}\index{quantum circuits}":
	\begin{figure}[H]
		\centering
		\includegraphics[scale=1]{img/computing/quantum_gate_not.jpg}	
		\caption{NOT Quantum gate}
	\end{figure}
	Now let us look for the logic gate making the following orthogonal clockwise transformation:
	
	We guess quite quick that the matrix which satisfies this relation (caution! the reader will notice that this time is a special case of the rotation matrix in the plane as seen in the section of Euclidean Geometry!):
	
	And which is sometimes written by specialists (in a somewhat unfortunate way):
	
	
	And we see that this logical transformation gate in the clockwise direction of the $1$-qubit involves the second Pauli matrix. So we can also write:
	
	Negation which is sometimes named "\NewTerm{$Y$-Pauli quantum gate}\index{$Y$-Pauli gate}" and which has no classical equivalent.
	
	Now let us seek for the logical gate making the following strictly counter-clockwise orthogonal transformation:
	
	We guess quite quick that the matrix that satisfies this relation (caution! the reader will notice that it is not a special case of the rotation matrix in the plane seen in the section of Euclidean Geometry!):
	
	And which is sometimes written by specialists (in a somewhat unfortunate way):
	
	And we see that this logical transformation gate in the clockwise direction of $1$-qubit involves the second Pauli matrix. So we can also write:
	
	Negation then we sometimes name "\NewTerm{$Z$-Pauli quantum gate}\index{$Z$-Pauli quantum gate}".
	
	When describing a quantum gate on an individual qubit, any dynamical operation, $G$, is a member of the unitary group $\text{U}(2)$ (\SeeChapter{see section Set Algebra page \pageref{set algebra}}), which consists of all $2\times 2$ matrices where $G^\dagger=G^{-1}$. Up to a global (and unphysical) phase factor, any single qubit operation can be expressed as a linear combination of the generator of $\text{SU}(2)$ that are the Pauli matrices!
	
	Now let us return to the following transformation which we have deal with earlier:
	
	In other words, it is the transformation (diagonal for recall) that makes to qubit equiprobable\footnote{As the sum of square of the amplitudes is indeed equal to: $(1/\sqrt{2})^2+(1/\sqrt{2})^2=0.5+0.5=1$}:
	
	We quickly notice that the corresponding matrix is then:
	
	which is named "\NewTerm{Hadamard quantum gate}\index{Hadamard quantum gate}\label{hadamard quantum gate}" and thus corresponds to an counter-clockwise rotation of $\pi/4$. This gate is represented by the following symbol in quantum circuits:
	\begin{figure}[H]
		\centering
		\includegraphics[scale=1]{img/computing/quantum_gate_hadamard.jpg}	
		\caption{Hadamard Quantum gate}
	\end{figure}
	And we can continue like this for a long time to create empirical quantum logic gates ... so we will stop here.
	
	Now remember that as we have seen in the section of Boolean logic (see page \pageref{all gates from NAND}), with a NAND gate (or an AND gate and an INVERTER) we've got all the parts we need to build a modern computer!

	We have already seen the quantum inverter earlier above and its corresponding matrix. We need now and AND gate, however that latter has two input, ie entangles quantum bits. 

	We built a "\NewTerm{quantum AND gate}" as following:
	\begin{figure}[H]
		\centering
		\includegraphics[width=1.0\textwidth]{img/computing/quantum_nand_gate.jpg}	
		\caption{NAND Quantum gate}
	\end{figure}

	\begin{flushright}
	\begin{tabular}{l c}
	\circled{60} & \pbox{20cm}{\score{2}{5} \\ {\tiny 22 votes,  53.64\%}} 
	\end{tabular} 
	\end{flushright}
	
	
\chapter{Social Sciences}

	\textit{\textbf{Social mathematics are the analysis and formal modeling tools of the behavior and management of a population and its trade in goods in all its activities, in interaction with its environment.}} (Sciences.ch)
	\minitoc
	\pagebreak
	%to force start on odd page
	\newpage
	\thispagestyle{empty}
	\mbox{}
	\section{Population Dynamics}\label{population dynamics}
	\lettrine[lines=4]{\color{BrickRed}I}t is said that Einstein at the end of his life, used to evoke the three explosions that would soon threaten humanity: the explosion of nuclear bombs, the explosion of our knowledge (robotics), the explosion in the number of humans on Earth. The role of the scientific at this level is not only to improve their knowledge but to share and especially to disseminate the awareness they have gained on the consequences they can discern of the explosion of the number of humans on Earth. Einstein was keen to play this role. Beyond the immediate vicissitudes, he knew to see the long-term developments. Among these developments, the number of humans appeared to him clearly as that of the greatest dangers (already introduced by the initiators of dynamic population science in the early 18th century!!!). Today, at the beginning of the 21st century, we can see their insight. Humanity is actually taken by the throat by the increase of its actual effective without forcing it down (because politicians favor the infinite growth and participation to social helps which do impose an endless increase of the population by mathematical construction!) as human do with the regulation of the number of animal in some protected areas. The major (naive) problems for politicians being to ensure full employment (avoid "technological unemployment" at some level) and the fact that the increase in population naturally enriches the owners and impoverishes workers (among many other problems...).
	
	Until the politicians of all major countries understand and show the example by limiting the population of their respective country (and therefore the number of births) and by proposing the same constraints for the whole World and the mentality of the population is open to type of proposal (it is it that elects politicians...), the mathematical theoretical study of the subject can help open some perspectives... regarding the renewal and the cycle of the resources of our unique planet... (because we only have one... and for the moment there is almost not good plan $B$).
	
	To introduce social mathematics, we must first determine the characteristics that describe the dynamics of the number of individuals before formalizing the unique properties that characterize them.
	
	\subsection{Birth rate and mortality tables (biometric features)}
	No one knows the day nor the hour of his death (excepted for some special cases...). This individual trivial fact is no longer relevant if we are interested not only in one person, but a numerous collectivity (a population in other words). Then we have a play between those who succumb from premature accidents and those who escape almost miraculously to the worst dangers. We can then describe how the way that is paid globally the tribute to death by considering a large number of children born the same year and specifying, thanks to the civil status data (when the exist and are robust), how their numbers gradually decrease for a day cancel completely.
	
	Such a set of people is named by demographers a "\NewTerm{cohort}". So consider the cohort of French males born in $1900$. Each year, we can, by gathering the indications of the government, calculate the number of those who died in the cohort.
	
	Let us represent by $d_a$ the ratio the number of deaths between birthdays $a$ and $a + 1$ by the initial size of the cohort.
	\begin{tcolorbox}[title=Remark,colframe=black,arc=10pt]
	We will in what follows define the basis of probabilities required for actuarial calculation. Indeed, actuarial mathematics merge the financial calculation and the calculation of probabilities. The payment of a capital is not certain and depends, for example on the survival of a person (see the section of Quantitative Management Techniques for Actuarial Calculation).
	\end{tcolorbox}
	The sequence of these numbers contains all the necessary information to study mortality in this cohort.
	
	We can infer the proportion of them surviving at age $a$:
	
	We can also characterize the intensity of the mortality at each age by dividing the number of deaths between ages $a$ and $a + 1$ by the number of survivors. This number is then named the "\NewTerm{mortality rate}":
	
	The list age by age of these three parameters $d, S$ and $q$ is the "\NewTerm{mortality table}" or "\NewTerm{life table}\label{life table}" of the studied cohort. The table below provides a summary for ages multiple of $5$.
	\begin{table}[H]
		\centering
		\renewcommand{\arraystretch}{1.2}
		\small
		\begin{tabular}{cccc}\hline
		Age &  Survivals & Death from age $a$ to $a+5$ & Probability of death  \\[-3pt]
$a$ & $S_a$ & $S_a-S_{a+5} $ & $q_a=\dfrac{S_a-S_{a+5} }{S_a}$ \\ \hline % ne pas enlever les espaces vides entre les lignes!!!
		0 & 1 & 0.228 & 0.228 \\

		5 & 0.772 & 0.013 & 0.017 \\

		10 & 0.759 & 0.009 & 0.012 \\

		15 & 0.750 & 0.023 & 0.031 \\

		20 & 0.727 & 0.023 & 0.032\\

		25 & 0.704 & 0.020 & 0.028\\

		30 & 0.684 & 0.021 & 0.031 \\

		35 & 0.663 & 0.025 & 0.038 \\

		40 & 0.638 & 0.029 & 0.045 \\ 
		
		45 & 0.609 & 0.026 & 0.043 \\ 
		
		50 & 0.583 & 0.036 & 0.062 \\ 
		
		55 & 0.547 & 0.047 & 0.086 \\ 
		
		60 & 0.500 & 0.062 & 0.124 \\
		
		65 & 0.438 & 0.078 & 0.178 \\ 
		
		70 & 0.360 & 0.360 &  \\  
		\hline
		\end{tabular}
		\caption{Mortality table example}
	\end{table}
	
	\begin{tcolorbox}[title=Remark,colframe=black,arc=10pt]
	Follow a real generation of individuals throughout its existence or a specific period of time is named "\NewTerm{longitudinal analysis}" in contrast to "\NewTerm{cross-sectional analysis}", which is to study the characteristics of a population at a given time . The table above is thus an example of a longitudinal analysis.
	\end{tcolorbox}
	
	Let us randomly select in the table the name of an unknown child in the list of births in the year $1900$. The big question at this birth was: how many years will he live? Today we are able to retroactively answer this question, at least evoking probabilities because we know the mortality table of his cohort.
	
	If the only information we have about this child today is the fact that he was born in 1900, we can state that the probability that he was still alive at the age of $5$ years is equal to $0.772$, at the age of $50$ years of $0.583$... In other words the probability that he died before $5$ years is equal to $0.228$.
	
	If we now choose today an unknown individual included on the list of those who were included in the cohort during the year 1920 at the age of $20$, we can still calculate the probabilities of the various periods of his life but we have additional information: he was still alive at age $20$, he avoided the risk of death before this age. The probability that it then reaches the age of $50$ years has become:
	
	Therefore the probability for a person of age $a$ to to be alive at age $a + n$ is equal to\label{life probability}:
	
	It is of course possible to make this calculation also for many individuals simultaneously. Indeed, in the field of insurance sector it is common for insurance to be a couple (husband and wife) where the age of each differs. Therefore, the joint probability of survival will logically be given by (\SeeChapter{see section Probabilities page \pageref{joint probability}}):
	
	Or if we are interested in the probability that one of the people is still alive (two incompatible events not necessarily disjoint as see in the section of Probabilities):
	
	The probability of a person of age to die  between age $a$ and $a + n$ is logically given by:
	
	Thus at each age, we can give the law of the variable "time to stay alive". This law can be summarized by indicating its expected mean (\SeeChapter{see section Statistics page \pageref{expected mean continuous variable}}). An immediate calculation makes it possible to give the value for each age in function of the mortality table.
	
	This can be exciting for a historian, but it gives an answer to a question asked long ago and since forgotten. What interests us is the present. This child that is just born, what is his life expectancy? To answer, we must know the mortality table of his generation, but the gift of premonition does not exist (as far as we know...). The probabilistic approach circumvents this difficulty as long as you specify the underlying assumptions.
	
	So, to answer the question: what is the life expectancy of newborns in 1990, then we assume, completely free, they will encounter at every age, in the future, the conditions that 'encountered in 1990 people of these ages: in 2000 they will suffer the same mortality than that experienced in 1990 by those born in 1980. Of course, no one would imagine that the reality will be consistent with this hypothesis, but the calculation it allows provides a synthetic picture of current conditions of the struggle against death.
	
	It is this assumption that life expectancy sometimes rejuvenates the old people (the life expectancy tends to increase as they become older thanks to advances in science...).
	
	Let us take for the calculation of life expectancy, the mortality table for men in Switzerland in 1983-1993:
	\begin{table}[H]
		\centering
		\renewcommand{\arraystretch}{1.2}
		\small
		\begin{tabular}{cccccc}\hline
		Age &  Survivals & Death Life Expectancy & Age &  Survivals & Death Life Expectancy \\[-3pt]
$n$ & $S_a$ & $\text{E}(a)$ & $n$ & $S_a$ & $\text{E}(a)$  \\ \hline % ne pas enlever les espaces vides entre les lignes!!!
	1 & 1.000000 & 73.688230 & 55 & 0.902240 & 22.810930 \\ 
	  2 & 0.992460 & 73.248060 & 56 & 0.895810 & 21.974660 \\ 
	  3 & 0.991830 & 72.294590 & 57 & 0.888750 & 21.149220 \\ 
	  4 & 0.991480 & 71.320110 & 58 & 0.880990 & 20.335510 \\ 
	  5 & 0.991170 & 70.342410 & 59 & 0.872470 & 19.534090 \\ 
	  6 & 0.990900 & 69.361580 & 60 & 0.863120 & 18.745700 \\ 
	  7 & 0.990660 & 68.378380 & 61 & 0.852880 & 17.970770 \\ 
	  8 & 0.990440 & 67.393570 & 62 & 0.841690 & 17.209690 \\ 
	  9 & 0.990220 & 66.408550 & 63 & 0.829480 & 16.463010 \\ 
	  10 & 0.990010 & 65.422630 & 64 & 0.816180 & 15.731280 \\ 
	  11 & 0.989710 & 64.442460 & 65 & 0.801740 & 15.014620 \\ 
	  12 & 0.989600 & 63.449630 & 66 & 0.786090 & 14.313540 \\ 
	  13 & 0.989380 & 62.463730 & 67 & 0.769180 & 13.628210 \\ 
	  14 & 0.989150 & 61.478260 & 68 & 0.750960 & 12.958870 \\ 
	  15 & 0.988890 & 60.494420 & 69 & 0.731380 & 12.305790 \\ 
	  16 & 0.988600 & 59.512170 & 70 & 0.710400 & 11.669210 \\ 
	  17 & 0.988230 & 58.534450 & 71 & 0.687980 & 11.049490 \\ 
	  18 & 0.987700 & 57.565860 & 72 & 0.664090 & 10.446990 \\ 
	  19 & 0.986920 & 56.611360 & 73 & 0.638730 & 9.861773 \\ 
	  20 & 0.985810 & 55.675100 & 74 & 0.611900 & 9.294182 \\ 
	  21 & 0.984390 & 54.755410 & 75 & 0.583620 & 8.744543 \\ 
	  22 & 0.982850 & 53.841210 & 76 & 0.553930 & 8.213240 \\ 
	  23 & 0.981310 & 52.925700 & 77 & 0.522910 & 7.700465 \\ 
	   \hline
		\end{tabular}
		\caption[]{Mortality table in Switzerland in 1983-1993 (part 1)}
	\end{table}
	\begin{table}[H]
		\centering
		\renewcommand{\arraystretch}{1.2}
		\small
		\begin{tabular}{cccccc}\hline
		Age &  Survivals & Death Life Expectancy & Age &  Survivals & Death Life Expectancy \\[-3pt]
$n$ & $S_a$ & $\text{E}(a)$ & $n$ & $S_a$ & $\text{E}(a)$  \\ \hline % ne pas enlever les espaces vides entre les lignes!!!
	  24 & 0.979750 & 52.009970 & 78 & 0.490660 & 7.206599 \\ 
	  25 & 0.978150 & 51.095050 & 79 & 0.457330 & 6.731813 \\ 
	  26 & 0.976530 & 50.179810 & 80 & 0.423090 & 6.276608 \\ 
	  27 & 0.974900 & 49.263710 & 81 & 0.388190 & 5.840903 \\ 
	  28 & 0.973280 & 48.345710 & 82 & 0.352880 & 5.425357 \\ 
	  29 & 0.971660 & 47.426310 & 83 & 0.317480 & 5.030301 \\ 
	  30 & 0.970070 & 46.504050 & 84 & 0.282340 & 4.656372 \\ 
	  31 & 0.968500 & 45.579430 & 85 & 0.247850 & 4.304337 \\ 
	  32 & 0.966940 & 44.652970 & 86 & 0.214460 & 3.974494 \\ 
	  33 & 0.965410 & 43.723730 & 87 & 0.182630 & 3.667196 \\ 
	  34 & 0.963880 & 42.793140 & 88 & 0.152810 & 3.382828 \\ 
	  35 & 0.962360 & 41.860730 & 89 & 0.125460 & 3.120277 \\ 
	  36 & 0.960820 & 40.927820 & 90 & 0.100910 & 2.879397 \\ 
	  37 & 0.959270 & 39.993950 & 91 & 0.079420 & 2.658524 \\ 
	  38 & 0.957680 & 39.060350 & 92 & 0.061110 & 2.455081 \\ 
	  39 & 0.956040 & 38.127360 & 93 & 0.045930 & 2.266492 \\ 
	  40 & 0.954340 & 37.195280 & 94 & 0.033730 & 2.086273 \\ 
	  41 & 0.952570 & 36.264390 & 95 & 0.024140 & 1.915079 \\ 
	  42 & 0.950710 & 35.335340 & 96 & 0.016800 & 1.751786 \\ 
	  43 & 0.948740 & 34.408710 & 97 & 0.011340 & 1.595238 \\ 
	  44 & 0.946650 & 33.484680 & 98 & 0.007390 & 1.447903 \\ 
	  45 & 0.944410 & 32.564100 & 99 & 0.004640 & 1.306034 \\ 
	  46 & 0.942010 & 31.647060 & 100 & 0.002790 & 1.172043 \\ 
	  47 & 0.939420 & 30.734310 & 101 & 0.001600 & 1.043750 \\ 
	  48 & 0.936620 & 29.826190 & 102 & 0.000870 & 0.919540 \\ 
	  49 & 0.933560 & 28.923960 & 103 & 0.000450 & 0.777778 \\ 
	  50 & 0.930230 & 28.027500 & 104 & 0.000210 & 0.666667 \\ 
	  51 & 0.926590 & 27.137600 & 105 & 0.000090 & 0.555556 \\ 
	  52 & 0.922590 & 26.255260 & 106 & 0.000040 & 0.250000 \\ 
	  53 & 0.918210 & 25.380500 & 107 & 0.000010 & 0 \\ 
	  54 & 0.913380 & 24.514710 & 108 & 0 &  \\ 
	  55 & 0.908080 & 23.657790 &  &  &  \\ 
	   \hline
		\end{tabular}
		\caption[]{Mortality table in Switzerland in 1983-1993 (part 2)}
	\end{table}	
	\pagebreak	
	Whose corresponding graph, named "\NewTerm{the order of living}" is:
	\begin{figure}[H]
		\centering
		\includegraphics[scale=0.75]{img/economy/order_of_livings.jpg}
		\caption{Cumulative probability of survival as a function of age}
	\end{figure}
	Let us see how to calculate life expectancy. For this, consider a human alive when he his $a$-th birthday. The number of years left to live is a random variable which we can calculate the expected value (\SeeChapter{see section Statistics page \pageref{expected mean discrete variable}}). Neglecting fractional years, that expected mean can be written:
	
	If $a$ is taken as zero, demographers speak of LEAB (Life Expectancy At Birth).
	
	Here is for information the increasing life expectancy for men (source: French National Institute for Demographic Studies) from 1996 to 2006:
	\begin{table}[H]
		\centering
		\begin{tabular}{|c|c|}
		\hline 
		$1996$ & $74.1$ \\ 
		\hline 
		$1997$ & $74.5$ \\ 
		\hline 
		$1998$ & $74.8$ \\ 
		\hline 
		$1999$ & $75.0$ \\ 
		\hline 
		$2000$ & $75.3$ \\ 
		\hline 
		$2001$ & $75.5$ \\ 
		\hline 
		$2002$ & $75.7$ \\ 
		\hline 
		$2003$ & $75.9$ \\ 
		\hline 
		$2004$ & $76.7$ \\ 
		\hline 
		$2005$ & $76.8$ \\ 
		\hline 
		$2006$ & $77.2$ \\ 
		\hline 
		\end{tabular} 
		\caption{Increased life expectancy at birth in FRA 1996-2006}
	\end{table}
	We can therefore observed using the above table that life expectancy increases by a little over one year every four years for more than 50 years (and until when ...???).
	\begin{figure}[H]
		\centering
		\includegraphics[scale=0.55]{img/economy/world_life_expectency.jpg}
		\caption[Life expectancy in the early 21st century]{Life expectancy in the early 21st century (source: Wikipedia)}
	\end{figure}
	Or the life expectancy evolution in France since 1740:
	\begin{figure}[H]
		\centering
		\includegraphics[scale=1]{img/economy/life_expectency_evolution_in_france.jpg}
		\caption[Life expectancy evolution in France since 1740]{Life expectancy evolution in France since 1740 (source: INED)}
	\end{figure}
	Finally, it must be know that actuaries define conditional life expectancy as the sum of:
	
	at the age $a$ for the gender $s$. These calculations will be useful to us later for calculating pension insurance premiums in the section of Quantitative Management.
	
	Let's move to another topic. We know that to date, to make a child, we need most of time  two people of opposite sex, certainly demographers know it too. But this double source of each of the newborn poses such problems in the description and analysis of fertility that in general they prefer to ignore it. Their attitude is justified by the fact that only the design requires the intervention of two people. At the time of birth, the mother acts alone. But demography is not interested in the designs, inaccessible to observation, but only to births.
	
	We have seen how we can track a sample of men or women born in a given year and record the successive deaths, allowing us to establish this cohort mortality table. Similarly, we can note the numbers of children they give birth every year. We obtain then the "\NewTerm{fertility table}".
	
	We just need to follow a cohort of women between $15$ to $50$ years to have an almost complete description of its reproductive behavior. Data given by the government gives us the possibility to calculate each year the number of children that women have given birth in this cohort, grouped by age or age groups. Dividing by the rate of women surviving at this age, we get the "\NewTerm{fertility rate}" $F_a$.
	
	If we add all these rate, we get the number of children that would have, on average, the women of this cohort if their mortality was zero. That was obviously not the case! To characterize how they ensured the renewal of their generation, we have to sum the real average of births, product of rate $F_a$ by the survival rate $S_a$.
	
	The set of these data is presented in a fertility table where we have an example where below where $n(a,a+5)$ is the number of births among survivos women survivors in this age group and where $F_a$ is the number of births of $1,000$ women of this age group.
	
	French women born around 1830:
	\begin{table}[H]
		\centering
		\renewcommand{\arraystretch}{1.2}
		\small
		\begin{tabular}{cccccc}\hline
		Age &  Survival rate & Number of births & Fecondity rate \\[-3pt]
$n$ & $S_a$ & $n(a,a+5)$ & $F_a$  \\ \hline % ne pas enlever les espaces vides entre les lignes!!!
	  15 & 0.672 & 91 & 135  \\ 
	  20 & 0.645 & 464 & 720 \\ 
	  25 & 0.616 & 589 & 955 \\ 
	  30 & 0.587 & 475 & 810  \\ 
	  35 & 0.558 & 328 & 588 \\ 
	  40 & 0.528 & 153 & 290 \\ \hline
	  Total &  & 2,100 & 3,498 \\  
	   \hline
		\end{tabular}
		\caption[]{Fertility table in France in 1830}
	\end{table}
	For example, on $1,000$ girls born in 1830, $645$ have reached the age of $20$ years and had $464$ children between the age of $20$ and $25$ years. The intensity of fertility is measured by the number of births that would have $1,000$ women of this age:
	
	And the average of births:
	\begin{table}[H]
		\centering
		\renewcommand{\arraystretch}{1.2}
		\small
		\begin{tabular}{cccccc}\hline
		Age &  Survivals & Number of births & Fecondity rate \\[-3pt]
$n$ & & $n(a,a+5)$  \\ \hline % ne pas enlever les espaces vides entre les lignes!!!
	  15 & 672 & 91  \\ 
	  20 & 645 & 464 \\ 
	  25 & 616 & 589 \\ 
	  30 & 587 & 475  \\ 
	  35 & 558 & 328 \\ 
	  40 & 528 & 153 \\ \hline
	  Total &  & 2,100 \\  
	   \hline
		\end{tabular}
		\caption[]{Average of births in 1830 in France}
	\end{table}
	We can observe that in 1830, from $20$ to $30$ years old women gave birth to an average of:
	
	this value being assimilated by the general public as the "\NewTerm{fertility rate}" (take care not to confuse with $F_a$).
	\begin{figure}[H]
		\centering
		\includegraphics{img/economy/world_fertility.jpg}
		\caption[Fertility rates in the early 21st century]{Fertility rates in the early 21st century (source: Wikiped}
	\end{figure}
	
	\subsubsection{Population Renevewal}
	The essential question for a set of human (or animals) in permanent renewal due to inflows and outflows that are births and deaths is: is the effective increasing or decreasing?
	
	The female birth table gives the possibility to answer this question, thanks to the ratio of boys at birth to the number of girls. Thus, in the early 21st century:$105$ boys born for every $100$ girls in the world. Thus, the proportion of girls is therefore:
	
	And the sex ration is therefore:
	
	Around forty years ($49$ in France according to INSEE), the balance is reversed and the number of women generally outweighs the number of men, despite significant regional disparities.
	
	\begin{figure}[H]
		\centering
		\includegraphics[scale=0.8]{img/economy/sex_ratio.jpg}
		\caption[Sex ratio trend in the early 21st century]{Sex ratio trend in the early 21st century (source: chartsbin.com}
	\end{figure}
	Assuming the proportion was the same at the time, the fertility table in 1830 therefore shows that on average a woman in the 1830 cohort spawned:
	
	girl. That is to say and increase which we will denote $k$ (in analogy with the exponential population model that we will see later) of $3\%$. Some politicians argue therefore that the value $2.1$ is the necessary fertility rate ... it takes to ensure the renewal of generations and we have just seen that is not entirely accurate.
	
	The female effective was therefore increasing. Therefore the population could ensure its renewal (unless the ratio is greater than $1$).
	
	The resulting number is the "\NewTerm{net reproduction rate (NRR)}". This rate is normally constituted by the ratio between the number of girls born for $100$ women, this ratio is adjusted by the planned mortality between the birth of these girls and the average age of reproduction, because some of the girls don't reach not the age of reproduction, given the deaths among those between birth and age at childbirth. The average age at reproduction is given by the average age of mothers at birth.
	
	An NRR of $1$ means that each generation of mothers is having exactly enough daughters to replace themselves in the population. If the NRR is less than one, the reproductive performance of the population is below replacement level.

	\pagebreak
	\subsection{Population Models}
	Population dynamics is the branch of life sciences that studies the size and age composition of populations as dynamic systems, and the biological and environmental processes driving them (such as birth and death rates, and by immigration and emigration).
	
	There are many mathematical models for studying the growth of a population as there all not all yet merged together in a single beautiful theory. Obviously the term "population" is used herein in the broadest sense - it can be a population of humans, animals, plants, individuals infected with a virus, etc.
	
	As we know, to build a mathematical model, it is necessary to make assumptions. These assumptions play two roles: to preserve some essential features of reality and simplify this reality sufficiently so it can be studied by mathematics.
	
	\subsubsection{Exponential model}
	In this population dynamics model (one of the simplest), the event will be: the rate of change of the population is proportional, at any time $t$, to the population $P(t)$ at time $t$.
	
	We can think, a priori, that this assumption is reasonable for a lot of situations. For example, more the human population is big, the greater the rate of change of this population will be, expressed in number of people add (or subtracted) per unit time, will be great. Similarly, the more people infected with a virus and more in the coming weeks, there will be new cases of people infected (if we don't take into consideration the limitness of the population and the methods to cure of a virus).
	
	Mathematically, this assumption can be translated using the differential equation (\SeeChapter{see section Differential and Integral Calculus page \pageref{first order differential equations}}):
	
	So this differential equation is a mathematical model representing a situation where the growth rate of the population is proportional of a factor $k$ to the size of the population at any time $t$. In this case, $k$ is a constant named "\NewTerm{growth rate}" and we will see later how we can determine it. In some situations, the value of $k$ is negative indicating that the population decreases with time rather than increasing. It is clear that a solution to this differential equation (\SeeChapter{see section Differential and Integral Calculus page \pageref{first order differential equations}}) is:
	
	First we will determine the value of the constant $k$ from demographic data for the year 1965. At that time, there were almost $3$ billion people on planet Earth. Moreover, at that time, the population grew by almost $54$ million per year.
	
	Therefore, in 1965:
	
	This give us:
	
	The differential equation is then:
	
	To determine the multiplicative constant, we simply have to put $t=0$ and choose accordingly (since it corresponds to the initial condition). Thus, in 1965 we had:
	
	with $t$ being the number of years after 1965.
	
	If this mathematical model is consistent with reality, the solution found should allows us to estimate the population of humans on Earth to later times than 1965. Here is the graph of the function:
	
	\texttt{>plot(3*10\string^9*exp(0.018*t),t=1..50,labels=[years,population],\\
	title='PopulationGrowth');}
	\begin{figure}[H]
		\centering
		\includegraphics{img/economy/population_exponential_model_growth_maple.jpg}
		\caption{Evolution of the world population since 1965 with Maple 4.00b}
	\end{figure}
	If we evaluate $P (37)$, this will provide us with the prediction for the population in 2002. We find $5.84$ billion which is pretty close to reality.
	
	The readers can also easily check the following table according to the values of $k$:
	\begin{table}[H]
	\begin{center}
		\definecolor{gris}{gray}{0.85}
			\begin{tabular}{|c|c|}
				\hline
				\multicolumn{1}{c}{\cellcolor{black!30}\textbf{Growing rate $k$ per year}} & 
  \multicolumn{1}{c}{\cellcolor{black!30}\textbf{Doubling time in years}} \\ \hline
				 $0.5\%$ & $139$ \\ \hline
				 $1\%$ & $69$ \\ \hline
				 $1.5\%$ & $46$ \\ \hline
				 $2\%$ & $35$ \\ \hline
				 $2.5\%$ & $28$ \\ \hline
				 $3\%$ & $23$ \\ \hline
				 $3.5\%$ & $20$ \\ \hline
				 $4\%$ & $17$ \\ \hline
		\end{tabular}
	\end{center}
	\caption[]{Various costs}
	\end{table}
	We then see immediately, that according to the exponential model, a growth that may seem slow, on the order of $3\%$ per year, is truly made an explosive results since doubling every $23$ years. That is to say a multiplication over $17$ in a century. Or a century passed quickly (at least for physicists because politicians seems to not be able to think on more than 50 years)!
	
	This simple observation shows how in this model are false the arguments that make us seek the solution of an economic or social problem in growth, and even, as it is often admitted in sustainable growth!!! It is clear that no growth can truly last, it is only a transitory episode, necessarily followed by a plateau or even a decrease. Resolve an economic issue by the growth or its stability is thereof transfer the problem to later, at a period at which it will be necessary to find ways both: to stop the growth or to resolve it in another way to solve a problem hidden by politicians not listening to scientists since to long time (more than 150 years!!!).
	
	The growth model we have just see right now may obviously not be valid over very long periods of time. Indeed, if we calculated, using the above equation, the population in 7centuries the result would be that on every square meter of land, excluding water, there would, on average, be ten human! Similarly, a population of people infected with a virus can also not really be described by such a model.
	
	These results tell us that if we want to develop a model of the growth of a population, that is more in line with reality, we have to change our initial assumptions. This is what we will see after with deterministic logistic models.
	
	But first let us give a nice example where an exponential explosion also occurs.
\
	Every individual normally has on Earth, as we know, two parents, four grandparents and eight great-grandparents and so on in every generation, if we go back in time, the number of ancestors doubles. But it can not double indefinitely and exponential growth must stop at some point or another! If the number of ancestors doubles with each generation and that we go back only 2,000 years ago, we would individually have about ancestors $1023$ considering only $25$ years for a generation ... Two thousand years before, which is not so distant, all the ancestor of all our actual population would then represent roughly the equivalent to the total mass of the Earth if this mass was made only of human bodies! The answer to this apparent paradox is of course that there are a lot (really lot!) of repetitions in each family tree (parents with several children ... and fathers with several women... and vice versa) and so we are all in a relatively short period cousins...
	
	We also have the following data on the evolution of the world population:
	\begin{table}[H]
    \begin{minipage}{.5\linewidth}
      \centering
        \begin{tabular}{|l|l|}
        	\hline
            Year & World Population\\\hline
            $-100000$ & $0.5$ million\\\hline
            $-10000$ & $1$ to $10$ million\\\hline
            $-6500$ & $5$ to $10$ million\\\hline
            $-5000$ & $5$ to $10$ million\\\hline
            $-200$ & $150$ to $20$ million\\\hline
            $1$ & $170$ to $400$ million\\\hline
            $200$ & $190$ to $206$ million\\\hline
            $400$ & $190$ to $206$ million\\\hline
            $500$ & $190$ to $206$ million\\\hline
            $600$ & $200$ to $206$ million\\\hline
            $700$ & $207$ to $210$ million\\\hline
            $800$ & $220$ to $224$ million\\\hline
            $900$ & $226$ to $240$ million\\\hline
            $1000$ & $254$ to $345$ million\\\hline
            $1100$ & $301$ to $320$ million\\\hline
            $1200$ & $360$ to $450$ million\\\hline
            $1250$ & $400$ to $416$ million\\\hline
            $1300$ & $360$ to $432$ million\\\hline
            $1400$ & $350$ to $374$ million\\\hline
            $1500$ & $425$ to $540$ million\\\hline
            $1600$ & $545$ to $579$ million\\\hline
            $1650$ & $470$ to $545$ million\\\hline
            $1700$ & $600$ to $679$ million\\\hline
            $1800$ & $0.813$ to $1.125$ million\\\hline
            $1850$ & $1.128$ to $1.402$ million\\\hline
            $1900$ & $1.550$ to $1.762$ million\\\hline
        \end{tabular}
    \end{minipage}%
    \begin{minipage}{.5\linewidth}
      \centering
        \begin{tabular}{|l|l|}
        	\hline
            Year & World Population\\\hline
            $1910$ & $1.750$ billion\\\hline
            $1920$ & $1.860$ billion\\\hline
            $1930$ & $2.07$ billion\\\hline
            $1940$ & $2.3$ billion\\\hline
            $1950$ & $2.519$ billion\\\hline
            $1960$ & $3.023$ billion\\\hline
            $1965$ & $3.337$ billion\\\hline
            $1970$ & $3.696$ billion\\\hline
            $1975$ & $4.073$ billion\\\hline
            $1980$ & $4.442$ billion\\\hline
            $1985$ & $4.843$ billion\\\hline
            $1990$ & $5.279$ billion\\\hline
            $1995$ & $5.692$ billion\\\hline
            $2000$ & $6.085$ billion\\\hline
            $2005$ & $6.5$ billion\\\hline
        \end{tabular}
    \end{minipage} 
      \caption[Evolution of the world population]{Evolution of the world population (source: Wikipedia)}
	\end{table}
	We see well an exponential growth for now ... but still not with the values corresponding to the small example calculation of our family tree even if we take the cumulative value!
	
	\subsubsection{Deterministic Logistic Model (Verlhust)}
	Now we will focus on a different type of model than the exponential one (where the population explodes) which has the advantage of having an asymptotic behavior rather than a divering one.
	
	This type of behavior is interesting because resources are usually limited and there is a competition between individuals. The logistic model, also named "\NewTerm{Verhulst model}" allows to account for this relatively well.
	
	Let $N (t)$ be the population at time $t$. Let us write:
	
	where $r$ is the rate of increase that this time will not be constant and is defined as being:
	
	where the idea is that $K$ is the maximum capacity of the medium. We see that if $K$ is infinite we fall immediately on the exponential model and if $N (t)$ equals $K$ then $r$ is zero and therefore the model is asymptotic. The constant $r_0$ is determined by adjusting the model to real datas (likelihood method or conjugate gradient method like seen in the section of Numerical Methods).
	
	Finallly we have:
	
	which gives after rearrangement:
	
	Or mathematically:
	
	with as initial condition that $y(0)=N_0$.
	
	We have now to solve this differential equation (\SeeChapter{See Differential and Integral Calculus}).
	
	Therefore we have to solve the differential equation:
	
	We put $y=1/f(x):=1/f$. The differential equation then becomes:
	
	which simplifies to:
	
	which gives after rearrangement:
	
	This differential equation is solved as usual using conventional techniques studied in the Differential and Integral Calculus section. We first write the homogeneous equation:
	
	And immediate particular solution is:
	
	where $A$ is a constant. By injecting this solution in the initial differential equation this gives:
	
	We see immediately that for equality to be satisfied our particular solution becomes the general solution if we write it in the form:
	
	The solution of the differential equation is therefore:
	
	The initial condition:
	
	Takes us to:
	
	Therefore:
	
	This gives us:
	
	Either in traditional form in the field of population dynamics:
	
	which is therefore the final expression of logistic deterministic model. We see easily that:
	\begin{enumerate}
		\item If $t$ tends to infinity then the horizontal asymptote is $K$.
		
		\item At time $t=0$, the initial population is equal to $N(0)=N_0$.
	\end{enumerate}
	\begin{figure}[H]
		\centering
		\includegraphics[scale=0.75]{img/economy/logistic_model.jpg}
		\caption{Logistic model VS Exponential model}
	\end{figure}
	To finish by returning to the previous form of our development:
	
	If we rewrite with a small change of variable we get:
	
	we obtain what is commonly named by its form a "\NewTerm{logistic function}".
	
	\pagebreak
	\subsubsection{Chaotic Logistic Model}\label{chaotic logistic model}
	We will still assume now that each generation is proportional to the previous one. We will suppose from one period to the next that the evolution of the population can then be translated into a sequence of the type:
	
	As the previous model this suppose obviously that the initial population size we not zero...
	
	where $x_t$ is the population at time $t$, $x_{t+1}$ the population at time $t+1$ and $K$ the reproduction ratio.
	
	But we quickly realize that such a model is unrealistic: the earth would have since long time be submerged by a human tide. Indeed, a power model such this one ignores important factors like the fact that a priori resources are not unlimited.
	
	To establish a more satisfactory model we must take into account (or by prudence have the pessimistic point of view) that the available resources are limited and that, for any country, there is a maximum population beyond which the population decreases, regardless of the species. To find a function that translates more realistically the evolution of a population, let us enumerate, simplifying a bit, the properties that must satisfied this evolution:
	
	\begin{enumerate}
		\item This is an iterative phenomenon: If $x_{t+1}$ is the size of the population of the period $t+1$, it depends on that of the previous period $t$.
		
		\item A population can not grow indefinitely in a defined territory: there is a maximum after which it must decrease. We must plan a retroactive factor limiting the population increase when its density becomes too high. To simplify the calculations, $x$ will not represent anymore the population in absolute number but a percentage of that maximum value corresponding to a given territory. The idea is than that $x_t$ can fluctuate between only between $0$ and $1$.
		
		\item $K$ remain the actual growth rate from one period to the next.
	\end{enumerate}	
	The size of the population of the period $t+1$ will be the value $x_{t+1}$, expressed as a percentage of the maximum population that can accommodate the given territory, and will be obtained from the size of the population $x_t$ of the previous period $t$ by the supposed relation:
	
	the factor $(1-x_t)$ represents obviously the retroactive effect. This suite is often named "logistic sequence" and we also find it in many other areas of nonlinear physics.
	
	Indeed, when the population density is high, close to saturation, the $x_t$ is close to $1$ and therefore $1-x_t$ will be close to $0$. So this retroactive factor will tend to minimize the population $x_{t+1}$.
	
	Thus, for certain animal species, it is normal that population change regularly (oscillates), while for others it is normal to move towards an equilibrium situation. This variety, "chaos" in some cases, is even linked to the mathematical properties of the previous relation.
	
	We will see that depending on the value of its actual growth rate $k$, an animal population (including therefore humans), may tend toward a state of equilibrium, or fluctuate between two or four or eight values, or change completely pseudo-randomly.
	
	The complexity becomes the rule and not the exception, and what appears "chaotic" follows very specific properties from a very specific closed form function.
	
	It is these results that we will see. For this, we will study the evolution of animal populations (this includes obviously therefore humans) whose numbers growth rate $K$ varies from $1$ to $4$. We will study these variations by the calculation and by studying the graph of the logistic function.
	
	Let us see now some special cases:
	\begin{enumerate}
		\item If the initial value $x_0$ is equal to $0.1$ and $K=1$ the evolution of the population for the first forty periods is represented by the graph below:
		\begin{figure}[H]
			\centering
			\includegraphics{img/economy/logistic_population_01.jpg}
			\caption{Plot of the logistic equation with $x_0=0.1$ and $K=1$}
		\end{figure}
		We note that for this growth rate and this initial condition, the population will decrease and approach zero.
		
		\item If the initial value $x_0$ is equal to $0.1$ and $K=2$ the evolution of the population for the first forty periods is represented by the graph below:
		\begin{figure}[H]
			\centering
			\includegraphics{img/economy/logistic_population_02.jpg}
			\caption{Plot of the logistic equation with $x_0=0.1$ and $k=2$}
		\end{figure}
		We note that for this growth rate, the population will stabilize around a number corresponding to half of the population that the territory could accommodate (remember that $1=100\%$ on the $y$-axis).
		
		\item If the initial value $x_0$ is equal to $0.1$ and $K=3$ the evolution of the population for the first forty periods is represented by the graph below:
		\begin{figure}[H]
			\centering
			\includegraphics{img/economy/logistic_population_03.jpg}
			\caption{Plot of the logistic equation with $x_0=0.1$ and $k=3$}
		\end{figure}
		With a growing rate of $3$, the number of individuals of this population starts to oscillate between two values that can be calculated using the computer. We then obtain the values $0.64$ and $0.68$.
		
		\item If the initial value $x_0$ is equal to $0.1$ and $K=3.5$ the evolution of the population for the first forty periods is represented by the graph below:
		\begin{figure}[H]
			\centering
			\includegraphics{img/economy/logistic_population_04.jpg}
			\caption{Plot of the logistic equation with $x_0=0.1$ and $K=3.5$}
		\end{figure}
		With an effective growth rate of $3.5$ the populatio oscillates between four values: $0.39$ and $0.83$, then $0.49$, and finally $0.87$. The evolution of a population that has such actual growth rate is clearly cyclical.
		\item If the initial value $x_0$ is equal to $0.1$ and $K=4.5$ the evolution of the population for the first forty periods is represented by the graph below:
		\begin{figure}[H]
			\centering
			\includegraphics{img/economy/logistic_population_05.jpg}
			\caption{Plot of the logistic equation with $x_0=0.1$ and $K=4.5$}
		\end{figure}
		We notice that when the growth rate is is equal to $4.5$, the number of individuals of  seems to oscillate irregularly in a "chaotic" way between two extremes: the saturation as $x$ approaches $1$ and extinction as $x$ approaches $0$.
	\end{enumerate}
	The question now is what is the behavior of this function for other initial values of population?
	
	To find out why, we take the above calculations with an initial population representing $0.8$ ($80\%$) of the maximum population of $1$ ($100\%$) for a given area and compare the results.
	
	The population growth is always given by the equation:
	
	\begin{enumerate}
		\item Depending on to the initial population is $x_0=0.1$ or $x_0=0.8$ we get for $K=1$:
		\begin{figure}[H]
			\centering
			\includegraphics{img/economy/logistic_population_06.jpg}
			\caption{Comparison plots of logistic equations for small $k$}
		\end{figure}
		
		\item Depending on to the initial population is $x_0=0.1$ or $x_0=0.8$ we get for $K=2$:
		\begin{figure}[H]
			\centering
			\includegraphics{img/economy/logistic_population_07.jpg}
			\caption{Comparison plots of logistic equations for medium $k$}
		\end{figure}
		We see that regardless of the initial population, it tends to a stabilization (equal to $0.5$) after a few iterations named an "\NewTerm{attractor}\index{attractor}". This is even more visible with the following plot:
		\begin{figure}[H]
			\centering
			\includegraphics[scale=0.5]{img/economy/logistic_attractor_k_2.jpg}
		\end{figure}
		or with $K=1.6$ we see that the attractor is at almost $0.38$:
		\begin{figure}[H]
			\centering
			\includegraphics[scale=0.5]{img/economy/logistic_attractor_k_1_6.jpg}
		\end{figure}
		or with $K=2.8$ we see that the attractor is at almost $0.64$:
		\begin{figure}[H]
			\centering
			\includegraphics[scale=0.5]{img/economy/logistic_attractor_k_2_8.jpg}
		\end{figure}
		
		\item Depending on to the initial population is $x_0=0.1$ or $x_0=0.8$ we get for $k=3.8$:
		\begin{figure}[H]
			\centering
			\includegraphics{img/economy/logistic_population_08.jpg}
			\caption{Comparison plots of logistic equations for large $k$}
		\end{figure}
		We can notice here that the evolution of populations is very different. We took remote initial populations. If we had very similar initial populations, what would have been the evolution of the two populations?
		
		\item Depending on to the initial population is $x_0=0.8$ or $x_0=0.8001$ we get for $k=3.8$:
		\begin{figure}[H]
			\centering
			\includegraphics{img/economy/logistic_population_09.jpg}
			\caption{Comparison plots of logistic equations for one large $k$ and small $\delta k$}
		\end{figure}
		We can notice here that evolution of the populations are very different even though they were very close initially and even if the evolution function of these populations is a very simple expression. If we had taken as initial population $x_0=8.00001$ we could have also found a very different evolution of population trends.
		
		This is oscillation is even more visible with the following plot:
		\begin{figure}[H]
			\centering
			\includegraphics[scale=0.5]{img/economy/logistic_attractor_k_3_1.jpg}
		\end{figure}
		We see above that the points oscillates between two bounds. This is not an attractor with a unique point, but an attractor with two points. The attractor is then named a "\NewTerm{periodic attractor}\index{periodic attractor}".
		
		With $K=3.5$ there are $4$ periodic attractors:
		\begin{figure}[H]
			\centering
			\includegraphics[scale=0.5]{img/economy/logistic_attractor_k_3_5.jpg}
		\end{figure}
		With $K=3.56$ there are $8$ periodic attractors:
		\begin{figure}[H]
			\centering
			\includegraphics[scale=0.5]{img/economy/logistic_attractor_k_3_56.jpg}
		\end{figure}
		
	\end{enumerate}
	Now if we plot the attractors value in function of the iterations for difference values of the $K$ above we get the following dot point plot:
	\begin{figure}[H]
		\centering
		\includegraphics[scale=0.5]{img/economy/logistic_attractors_plot.jpg}
	\end{figure}
	The conclusion of this model is that outside of the chaotic phase, the initial value is not important, but in this chaotic phase, otherwise, the smallest variation in initial value change completely the following values. This is named the "\NewTerm{butterfly effect}" (for more details about the Butterfly Effect see the mathematical developments of the section of Marine \& Weather engineering).
	
	\paragraph{Feigenbaum's Bifurcation Diagram}\mbox{}\\\\
	To understand the evolution of a population with logistic model, we have previously showed the evolution over time, forty periods, of a population corresponding to a specific initial value and a constant $k$ determined. We have seen, by taking some particular values of $k$, that for these different values, the population had a different pattern. So we will study now the behavior of the logistics by taking $k$ as variable.
	
	Giving to $k$ values between $0$ and $4$ with a step of $0.02$, we will, for each of these values, calculate what will be the population for each period between the $30$th and $130$th.
	
	With Maple 4.00b we get (in the R companion book the equivalent script is also given in R language):
	
	\texttt{
	>with(plots): with(plottools);\\
	>feigenbaum:=proc(start,end,step) local k,iteration,a,b,s;\\
	s:={}; a:=stat;\\
	while a<=end do iteration:=0.1;\\
  	for k to 50 do iteration:=a*iteration*(1-iteration) od;\\
    for k to 100 do iteration:=a*iteration*(1-iteration);\\
      s:=s union {[a,evalf(iteration,4)]};\\
    od;\\
    a:=a+step\\
	od;\\
	plot([op(s)],'a'=start..end,style=POINT,symbol=POINT)\\
	end:\\
	>feigenbaum(1,4,0.01);	}
	
	\begin{figure}[H]
		\centering
		\includegraphics{img/computing/feigenbaum_maple.jpg}
		\caption{Feigenbaum diagram with Maple 4.00b}
	\end{figure}
	This diagram (where we can observe bifurcations), is named "\NewTerm{Feigenbaum diagram}", named after the physicist Mitchell Feigenbaum who studied it in depth and showed that we found him in many natural phenomena.
	
	We can identify in the diagam above the different attractors hat we have seen just earlier:
	\begin{figure}[H]
		\centering
		\includegraphics[scale=0.5]{img/economy/logistic_attractors_full_plot.jpg}
	\end{figure}
	
	We will now have an qualitative approach of two remarkable properties of this diagram: the doubling period and its fractal dimension!!
	
	Going back to our example where the logistics function gives the evolution of a population according to its actual growth rate $k$, the Feigenbaum diagram indicates that, when the rate is less than $3$, the system tends towards a stable final state . This generally corresponds to our intuition influenced by our unconscious desire for order and simplicity.
	
	But starting from $3$, the things become littre bit more complicated: the number of individuals per generation starts oscillating between $2$ and $4$, then $8$ values ... to finally get into a "chaotic" area where all values seem possible. There is already something fascinating. But if you observe more closely this doubling of interval before the chaotic area, we will discover even more curious results.
	
	So we first take the sequence of the three first three duplications with $2$, then $4$, then $8$ branches, for an effective growth rate $k$ on an interval $k=[3,3.564407]$ and we get the following diagram:
	\begin{figure}[H]
		\centering
		\includegraphics{img/economy/feigenbaum_bifurcation.jpg}
		\caption{Bifurcations in the Feigenbaum diagram}
	\end{figure}
	It is remarkable to observe in the figure above that there are two point that are totally stable (the both just a little bit after $3.3$)!
	
	If we take the interval $k=[3.5,3.569891]$ we get the following Feigenbaum diagram:
	\begin{figure}[H]
		\centering
		\includegraphics{img/economy/feigenbaum_bifurcation_multiple.jpg}
		\caption[]{Multiple bifurcations in the Feigenbaum diagram}
	\end{figure}
	We can notice in the figure above that the bifurcation multiplies from points that are always more and more closer and on intervals always shorter. The physicist Feigenbaum has therefore proven (by trial and error) two strange results:
	\begin{enumerate}
		\item The bifurcations will multiply to infinity (!) on a range which does not exceed the point $3.5699456$ of the $x$-axis named the "\NewTerm{Feigenbaum point}" or "\NewTerm{gateway to chaos}", because, after this point, the system becomes chaotic. It begins to fluctuate between unpredictable values and becomes extremely sensitive to initial conditions.
		
		\item The lengths of intervals specific to different classes of bifurcations $(2, 4, 8, 16, ...)$ decreases in a constant ratio of $4.6692...$ named, off course, the "\NewTerm{Feigenbaum constant}".
	\end{enumerate}
	\begin{tcolorbox}[title=Remark,colframe=black,arc=10pt]
	Heinz-Otto Peitgen that these bifurcations, Feigenbaum points and constant, is not only found in the case of the logistic function studied by Robert May (see below), but in many physical phenomena such as hydrodynamics, electronics, LASERs or acoustics.
	\end{tcolorbox}	
	Finally, we can notice a curious phenomenon this of stable windows. If we examine more closely the chaotic area between the Feigenbaum point and $k=4$, we see that there are areas where the logistic function starts to oscillate between a finite number of bifurcations before plunging into chaos again:
	\begin{figure}[H]
		\centering
		\includegraphics{img/economy/feigenbaum_bifurcation_stable_windows.jpg}
		\caption[]{Viewing the background changes and the stabilization windows}
	\end{figure}
	We can notice that in this windows there are bifurcations that look like the Feigenbaum diagram itself if we zoom on it.
	
	We could also take the same calculations between $k=3.848$ and $3.850$ and we would have found we find that we fall back on the same figure, and if we zoom on it to infinity, we always found the same figure ever and ever (!) in an auto-similarity process:
	\begin{figure}[H]
		\centering
		\includegraphics{img/economy/feigenbaum_autosimilarity.jpg}
		\caption[]{Autom-similarity of the Feigenbaum diagram}
	\end{figure}
	We have therefore just discovered a fractal figure (see the section Fractals for more details): named also "\NewTerm{strange attractor of the logistics function}" also named "\NewTerm{bifurcation diagram of the logistic equation}" or simply "\NewTerm{Feigenbaum diagram}".
	
	We will see (without proof) in the section Fractals that the Feigenbaum diagram above is closely related to the Mandelbrot fractal (without proof as we never used this result for engineering applications).
	
	Thanks to the mathematics of chaos theory applied to population dynamics, ecology received a decisive stimulation. Until the early 1960s, the debate on population dynamics opposed the proponents of a deterministic theory, seeing regular changes in populations undergoing exceptionally sudden changes, to those believing these developments as purely random. But many facts remain poorly explained. Especially cyclic explosions of some populations and their strange periodicity did not fall into either of the two explanations. By showing that deterministic models can give rise to random behavior, Robert May, reconciles those views from a deeper theory. What appears to a given level of apprehension as a general instability can appear to another level as a stable chaos. A chaotic mathematical situation can be stable ecologically.
	
	If, because of the increasing power of computers, chaos theory has led to major advances in different disciplines, it raises a major debate, that of determinism. How does science can predict the future? For some, the results thanks to chaos theory prove the importance of initial conditions. For these people, deterministic equations have limited scope and the future remains unpredictable. For others however, the results show that we can find order and laws in what may seem chaotic. These laws are simply more complex.
	
	It is an open and lively debate that goes far beyond the scientific community.

	\subsubsection{Malthusian Growth Law}
	We have already seen some deterministic and chaotic models using growth rates for the simulation. Now introduce another type of model using the number of children, women and their respective fertility.
	
	For this model, we state the following hypothesis (assumptions):
	\begin{enumerate}
		\item[H1.] At the beginning, $N$ people where $N/2$ are females.

		\item[H2.] The growing rate $r$ is supposed to be constant because of a fecondity rate $f$ that is also constant.
		
		\item[H3.] The population is growing.
	\end{enumerate}
	So we have for data: the population at time $t$ denoted $N (t)$, the female population denoted $N_F(t)$ and $N_E(t)$ the number of children.
	
	With the following relations:
	
	and therefore:
	
	then the growth rate is:
	
	Finally we get the Malthus law:
	
	this model is therefore continuous, takes into account the fecundity but diverge.

	\subsubsection{Leslie model}
	The Leslie model is a bit more advanced than the others deterministcs models but just as empirical  (it is possible, as in any field of science to construct valid theoretical models always more complete and complex).
	
	Besides the fertility and mortality rates, this model can take into account the ages groups of the population and some of their properties in relation to the two aforementioned factors (fertility, mortality). The disadvantage of this model is, however, the too many parameters to be determined for all age groups...
	
	The system is based on the cutting of the population in age groups such as for example:
	
	\begin{itemize}
		\item $N_1(t)$: Is the number of people of age $1$
		
		\item $N_2(t)$: Is the number of people of age $2$

		\item ...

		\item $N_i(t)$: Is the number of individuals of age $i$.

		\item ...

	 	\item $N_T(t)$: Is the number of individuals of age $T$.
	\end{itemize}
	Then the idea is that the evolution of an age group depends on the other age groups. For example, births are given by by the reproductiion rate $r$ summed over all age groups (of course for some of them the rate is zero...) such as:
	
	It is customary and reasonable for humans being to accept that normally only age groups for which $i\geq 11$ and $T=55$ are taken into account regarding to the reproduction which implies that normally the $r_i$ are non-zero only for this interval.
	
	Then, aging and mortality $m$ will be taken into account by the relations (we take again all the age groups of the entire population because obviously there is not only the individuals between $11$ and $55$ that die ...):
	
	It is relatively easy to see that these equations can be put in a matrix form (\SeeChapter{see section Linear Algebra page \pageref{linear algebra}}) as follows:
	
	where the matrix containing the coefficients of mortality and reproduction is named "\NewTerm{Leslie matrix}".
	
	More compactly it is written:
	
	or when starting for the initial population:
	
	It is possible to make interesting analyzes on this model with respect to the age of having children and the consequences relating thereto. This model seems to be relatively widely used in marine biology.
	
	\pagebreak
	\subsubsection{SIR Model for Spread of Disease}
	The "theory of epidemics" provides many intuitive systems of differential equations or with partial derivatives in discreet time in which intervene contamination phenomena, diffusion phenomena, etc.
	
	Here we will take the example of the spread of rabies in the fox population, and present a fairly simple model where we consider that individuals (foxes) can be in only three different states:
	\begin{enumerate}
		\item There are healthy foxes in quantity $S$ (susceptible)

		\item There are infected foxes in quantity $I$

		\item There are death foxes in quantity $R$
	\end{enumerate}
	This assumption of classification into three categories make that this model of spread of epidemics is often named the "\NewTerm{SIR model}".
	
	For more optimistic people, we can also use the model that assumes that some individuals are immune and can therefore not be infected. Finally, with three different states, we can link the three proper equation of developments such that:
	
	with:
	 
	where the $S,I,R$ are given in \% of the whole population.
	
	In this model, the first equation corresponds to the phenomenon of contamination: when coexist healthy individuals and of infected individuals, a number of healthy individuals will be infected. It is also natural to consider this term as proportional to the product $IS$. Indeed, the quantity of microbes in the defined medium (thus the factor $r$ for a healthy individual to be infected in the presence of an infected individual) is proportional to $I$ according to the factor. We then have to multiply this probability by the number of healthy individuals, that is to say, by $S$. The negative sign is  present to indicate the reduction of the healthy population.
	
	Regarding the variations of $I$, the first term corresponds to the infected individuals (which increase $I$). The second term in $-aI$, consists of individuals who die.
	
	The plot below shows a calculation of the evolution of the populations $S$, $I$, $R$ as a function of time (one second is considered as a loop of the algorithm). According to the values of parameters $a$, $r$ the behaviors are completely different: if $a$ is too high, for example, infected individuals die almost immediately and do not have time to infect many others (...) . Here is a plot of the evolution of populations over time, performed with the following algorithm with $r=4,a=2$:
	\begin{figure}[H]
		\centering
		\includegraphics[scale=0.75]{img/economy/sir_model.jpg}
		\caption{Evolution of healthy, infected and death foxes in function of time}
	\end{figure}
	The implementation algorithm looks like this in pseudo code (for a more formal approach see my MATLAB™  book):
	
	In pseudo-code (non-unique and not optimized):\\\\
	\begin{algorithm}[H]
	 \KwData{$r$,$a$, $\max_t$, $\Delta_t$ }
	 \KwResult{$\bar{x}$}
	 initialization\;
	\For{$i=1$ \KwTo $E(\max_t/ \Delta_t)$}{
        $\mathrm{d}S=-rS(i)I(i)$\;
		$\mathrm{d}I=rS(i)I(i)-aI(i)$\;
		$\mathrm{d}R=aI(i)$\;
		$\mathrm{d}R=aI(i)$\;
		$S(i+1)=S(i)+\mathrm{d}t\mathrm{d}S$\;
		$I(i+1)=I(i)+\mathrm{d}t\mathrm{d}I$\;
		$R(i+1)=R(i)+\mathrm{d}t\mathrm{d}R$\;
	}
	 plot $(I,t),(S,t),(R,t)$\;
	 \caption{SIR Model pseudo-code algorithm}
	\end{algorithm}
	We can of course come to many variations of this model depending on the assumptions we make about the disease under study. It will often be useful to define more populations categories. Thus, for S.T.M. (Sexually Transmitted Diseases), we will differenciate the populations according to sex (we should also take the age into account), which double the number of possibilities. To define the underlying differential system, it becomes better sometimes to model the process by a diagram.
	
	Thus, to study naively an infection like AIDS (Acquired Immunodeficiency Syndrome), we can make such a model in which we consider four population groups: healthy individuals $X$, infected individuals $Y$, the healthy carriers $Z$ and finally the sick individuals $A$. In the diagram below, the arrows between the different states represent possible transitions for a given individual with the corresponding probabilities:
	\begin{figure}[H]
		\centering
		\includegraphics{img/economy/epidemia_propagation.jpg}
		\caption{Markov chain of deadly disease}
	\end{figure}
	Indeed, from the perspective of the study of AIDS, a given individual, whatever his condition, may die of something else than AIDS, this is what we put into the category "natural death", We consider that there is a fixed mortality rate, denoted $n$, for all populations considered (healthy, infected, sick or healthy carriers). On the other hand, a healthy individual $X$ may also be contaminated, in which case it becomes an infected individual $Y$. An infected individual can change into to two possible stages: sick $A$ or healthy carrier $Z$.
	
	The number $s$ is the speed of evolution of the disease: bigger is this rate, the more the infected individuals evolve into one of the next stages. The number $p$, which value is between $0$ and $1$, represents the probability that this change is done into the direction of the sick state. Of course, if $p$ decreases, the amount of healthy carriers, immunized against the disease increases. The disease is then globally less dangerous. Finally, a patient may die of his disease, with a mortality rate $d$ considered as fixed, while a healthy carrier, immunized, can not change anymore of the point of view of this disease. All these assumptions lead to put  a system of equations of the form:
	
	where:
	
	
	\pagebreak
	\subsubsection{Lotka–Volterra predator–prey model}\label{lotka volterra model}
	This predator-prey interaction model was proposed by Volterra after the First World War. The purpose was to explain the dynamics of populations of sardines and sharks in the Adriatic (and hop! a bit of Oceanography in this book...!). Especially to explain in why the quantities of sardines fished after the interruption due to the war were not as important as before (which may seem counter-intuitive) and why at the resumption of fishing the observed proportion of sharks increased.
	
	Taking again what we saw at the beginning of this section by writing $N(t)$ the number of preys and $P(t)$ the number of predators (that is to say implicitly a growth following a power law of the number of preys, in absence of fishing):
	
	
	and in the absence of preys (in the case of fishing), the number of predators decrease also following a power law, we have respectively:
	
	At this point of the speech, we must consider two species (preys and predators) that are obviously not isolated but in interaction. To quantify the contribution of the interaction between species, we will consider only predation, by assuming that is value or intensity (of interaction) is a function of the probability of prey-predators to meet together that will be supposed proportional to the product $N\cdot P$ of the percentages of the two populations.
	
	These meetings do not have the same effects on both species. First, of course, each prey eaten by a predator is a net gain for the population of the latter and a net loss for the first. Thus, if the effect of interactions is accepted as being proportional to $N\cdot P$, the signs of the influence of interaction differ for the two species so that our equations with interaction this time become:
	\begin{equation}
	  	\addtolength{\fboxsep}{5pt}
	   	\boxed{
	   	\begin{gathered}
	   		\begin{aligned}
			&\dfrac{\mathrm{d}N}{\mathrm{d}t}=rN-cNP\\
			&\dfrac{\mathrm{d}P}{\mathrm{d}t}=-mP+bNP\\
	   		\end{aligned}
	   	\end{gathered}
	   	}
	\end{equation}
	This system of equation is named the "\NewTerm{Lotka-Volterra model}\index{Lotka-Volterra model}".
	
	Before going further, let us seek for the values for which the derivatives vanish (which will give us in fact the equilibrium point of the system):
	
	where $N'$ and $P'$ are the values that cancel the derivatives (not to be confused this time with the notation sometimes used to condense the writing of a derivative !!!).
	
	Thus:
	
	A trivial solution is the "\NewTerm{extinction solution}" (also sometimes named "{critical solution}") given by:
	
	Otherwise, we also have as a possible solution:
	
	Now we normalize these equations by writing (so they are dimensionless):
	
	with this normalization, the model can be rewritten:
	
	By rearranging the coefficients, the system can finally be written (excluding the solution of inexistance):
	
	for which derivatives vanish at the point $(1,1)$.
	
	The discreet plot of this system of equations (in which we recognize a logistic term as seen in the section of Populations Dynamics) gives us with $m=r=b=c=1$ and initial conditions $(x_0,y_0)=(4,1)$:
	\begin{figure}[H]
		\centering
		\includegraphics{img/economy/cycle_offer_demand.jpg}
		\caption{Theoretical fluctuation of population}
	\end{figure}
	In comparison here is a practical (real) example of prey-predators measurements (Hares-Lynx) by the Hudson Bay-Company (when we seek for data corresponding to a model we will finish soon or late to found data fitting the model...):
	\begin{figure}[H]
		\centering
		\includegraphics{img/economy/lotka_volterra_offer_demand_hudson.jpg}
		\caption[]{Hudson Bay-Company real Lotka-Volterra measurements}
	\end{figure}
	The two previous figures represent the variables $x$ and $y$ as a function of time. However, what can be interesting for a scientist (or an economist) is the representation of $y$ in terms of $x$ and vice versa. Thus, we obtain for the same initial conditions $m, r, b, c$ and for the various initial values of $(x_0,y_0)$ (the Predators is in ordinate and the Preys on the abscissa):
	\begin{figure}[H]
		\centering
		\includegraphics{img/economy/offer_demand_space_phase.jpg}
		\caption{Representation in phase space prey/predator cycle}
	\end{figure}
	The above figure is very interesting to interpret when you follow a path in the counter-clockwise direction.
	
	Thus we see (in the phase space representation) that for fixed initial conditions, the system is periodic and has trivially an equilibrium point at:
	
	named "\NewTerm{fixed points}" that correspond to the points where:
	
	Finally, we have two pairs of equilibrium points (that is almost trivial, looking at the system of equations):
	
	The question that will naturally arise is the "real" direction of rotation (representation) of the phase plane. Thus, representing the directions using a vector field, we get the representation which is effectively counter-clockwise (more interesting representation to understand what happened after 1st World War):
	\begin{figure}[H]
		\centering
		\includegraphics{img/economy/offer_demand_vector_field.jpg}
		\caption{Rotation direction offer/demand phase space}
	\end{figure}
	
	To know in which direction we are going in the phase space at a given time, it suffices to know the derivative $\mathrm{d}y /\mathrm{d}x$ (or vice versa $\mathrm{d}x /\mathrm{d}y$). Therefore we have:
	
	That said, we see well on the phase diagram with vector field as vectors that it comes a time in the cycle of this model where the offer is very high for low demand. So the mathematical model (theoretical) clearly explains what can be a priori counter-intuitive to many humans (offer creates the demands).
	
	However, we can (must) ask ourselves the question of what happens after a small perturbation around the equilibrium point.
	
	So we have the following Lotka-Volterra system in equilibrium:
	
	By putting an infinitely small perturbation, it is will be written:
	
	Neglecting the quadratic $xy$ terms, we obtain:
	
	\begin{enumerate}
	\item We focus first on the study near the point of extinction $(0,0)$, this is why we can neglect the quadratic terms $xy$ but the expression remain too complicated. So, always considering we are near the point of extinction $(0,0)$, we consider a bit unfairly but cleverly (otherwise we still could not solve the problem analytically) the following approximations:
	
	Therefore:
	
	This shows us for the system in equilibrium, close to the inexistence point, the prays decreases exponentially (power law) as predators increases exponentially:
	
	This shows us for the equilibrium system, near the extinction point, that the number of individuals predators decreases exponentially while the prey are increasing exponentially. This has a biological sense: when there are few individuals predators, then prey multiply and at the same time the number of prey increases, predators multiply and concentrate increasingly on their prey; respectively when there are few prey only, the number of predator decreases as the number of prey increases (ahhh nature...).
	
	\begin{tcolorbox}[title=Remark,colframe=black,arc=10pt]
	In the literature we find sometimes the "-" sign in the top or bottom in the previous equations. In reality, the position of the "-" sign is not important because it is just the starting choice in the system dynamics.
	\end{tcolorbox}
	
	\item Close to the equilibrium point $(1,1)$ we start from:
			
		And we put $x:=1+x,y:=1+y$ where $x,y$ are now small perturbations near the point $(1,1)$. Therefore we have:
		
		 (hop! we change the position of the "-" sign on purpose to show that this is only a starting choice !!!):
		
		To solve this system, let us differentiate the first equation once again:
		
		and by injecting in it $\mathrm{d}y/\mathrm{d}t$:
		
		So we get a small second order differential equation (\SeeChapter{see section Differential and Integral Calculus page \pageref{second order differential equations}}). Whose typical simple solution is:
		
		By injecting this solution into the differential equation, we obtain after simplification of exponentials a simple polynomial of the second degree (\SeeChapter{see section Calculus page \pageref{polynomial}}):
		
		Whose solution is trivial:
		
		Thus, the general solution of the differential equation is the linear combination of the two special solutions we get:
		
		But we therefore have:
		
		Therefore, knowing $x(t)$ we obtain easily:
		
		Now let us use the Euler formula (\SeeChapter{see section Numbers page \pageref{euler formula}}):
		
		Thus we have:
		
		and as (\SeeChapter{see section Trigonometry page \pageref{remarkable angles}}) $\cos(x)=\cos(-x),-\sin(x)=\sin(-x)$ then we have:
		
		and similarly, we obtain:
		
		Thus, around the equilibrium point $(1,1)$ with sufficiently small perturbations to validate linearization the system oscillate as ellipses (or circles) whose axes are defined by the two equations above.
	\end{enumerate}
	We can get the graphs above with Maple 4.00b (this is a nice example of application of this software):
	
	\texttt{>restart: with(plots): with(DEtools):}\\
 	\texttt{>rate\_eqn1:= diff(h(t),t)=(0.1)*h-(0.005)*h*(1/60)*u;}\\
 	\texttt{rate\_eqn2:=diff(u(t),t)=(0.00004)*h*u-(0.04)*u;vars:= [h(t), u(t)];}\\
 	\texttt{>init1:=[h(0)=2000,u(0)=600]; init2:=[h(0)=2000,u(0)=1200]; init3:=[h(0)=2000, u(0)=3000];domain := 0 .. 320;}\\
 	\texttt{>L:= DEplot({rate\_eqn1, rate\_eqn2}, vars, domain,{init1}, stepsize=0.5, scene=[t, u], arrows=NONE):}\\
  	\texttt{>H:= DEplot({rate\_eqn1, rate\_eqn2}, vars, domain,{init1 }, stepsize=0.5, scene=[t, h], arrows=NONE):}\\
  	
  	\begin{figure}[H]
		\centering
		\includegraphics{img/economy/lotka_volterra_offer_demand_maple.jpg}
		\caption{Lotka-Volterra offer/demand model periodic plot in Maple 4.00b}
	\end{figure}
  	
  	\texttt{>DEplot({rate\_eqn1, rate\_eqn2}, vars, t= 0 .. 160, {init1, init2, init3}, stepsize=0.5, scene=[h,u], title='Demande vs. 60 * Offer for t = 0 .. 160', arrows=slim);}\\
  	
  	\begin{figure}[H]
		\centering
		\includegraphics{img/economy/lotka_volterra_offer_demand_phase_space_maple.jpg}
		\caption{Lotka-Volterra offer/demand phase space model plot in Maple 4.00b}
	\end{figure}
	
	\pagebreak
	\subsection{Schaefer's Optimal capture model}
	A useful principle for the management of fishing, hunting, wild collection quotas or exploitation of raw materials is the "\NewTerm{maximum sustainable yield}" which is directly linked to the good management of the exploited dynamics of the concerned population.
	
	\textbf{Definition (\#\mydef):} The "\NewTerm{maximum sustainable yield}" is the maximum amount of fish, deer, birds or fungi, etc. which can be killed / harvested in a population without jeopardizing its survival.
	
	Determine this maximum sustainable yield give the possibility the get to the goal of fishing, hunting, gathering as much as possible, without falling into the overconsumption that potentially lead to the disappearance of the population.
	
	The distinction between a balanced consumption of natural resources and an exaggerated one is often hard to do. Mathematical and statistical models are then always useful and necessary to establish that frontier.
	
	The calculations are involved in the calculations of quotas even still in the early 21st century. The simplistic model that we will present is used by some large States (but it will in time disappear to make way for a more sophisticated model with probabilistic parameters).
	
	We are therefore interested in the biomass of a population of a consumable. The model put forward by Schaefer in 1954 involves only two parameters:
	\begin{enumerate}
		\item The maximum biomass $M$ that can live constantly in a system

		\item The reproduction rate $r$
	\end{enumerate}
	We start from the assumption (to check in practice because of the Lotka-Volterra model) that at the year zero (before to the consumption of the resource), the corresponding biomass is:
	
	The empirical model is:
	
	To understand this model, let assume first that the population is not exploited, that is to say that $C=0$. At the  year $n$, the population then increases of:
	
	Thus, in the absence of fishing, if $X_0$ is smaller than $M$, the population will grow slowly until it reaches its maximum capacity $M$.
	
	In the extreme case, where consumption $C$ is equal to $M$, we have:
	
	which makes sense ... (there is a total extinction!).
	
	Now rewrite the Schaefer model as follows:
	
	We would like to know for what value of $X_n$, we would have the biomass which no longer varies. That is to say:
	
	Therefore:
	
	So it is a simple equation of the second degree! It has a real solution only if the discriminant (\SeeChapter{see section Calculus page \pageref{discriminant}}) is positive or zero. Therefore, knowing that $r$ is positive or zero by construction:
	
	Thus, the maximum possible value of $C$ is $rM/4$. When we replace $C$ by this value in the original equation:
	
	we get:
	
	Therefore:
	
	Therefore the solution is:
	
	Thus, there exists a time $n$ (infinite) for which the population tends asymptotically to $M/2$ at the condition that $X_0\geq \dfrac{M}{2}$.
	
	Still its simplicity and its deterministic aspect  the Schaefer model is still used in the early 21st century in fisheries management.
	
	\subsection{Hardy-Weinberg model}
	The Hardy-Weinberg model (developed in 1908) describes the diversity (rather than evolution) of living in the point of view of population genetics. However, this model still allows to understand the impact of a mutation agent on a population of individuals and therefore to solve problems related to the refusal of the Natural selection model  (evolution) of Darwin. Indeed, Darwin's work focused on characters that varied continuously. But he could not explain how individuals transmit these changes (this is actually quite annoying). The Mendel's genetics, meanwhile imposed that only discontinuous traits were hereditary.
	
	So there was  a conflict between the continuous model (Darwin) and  the discontinuous one (Mendel) and it was genetics that in the twentieth century helped to establish that fact that it is the mix of genes that creates diversity and verbatim natural selection.
	
	\begin{tcolorbox}[title=Remark,colframe=black,arc=10pt]
	We will prove in the section of Numerical Methods (chapter of Theoretical Computing) as part of the study of genetic algorithms that the chromosome having the biggest function value of "fitness" will be the one most likely to reproduce. This is part of the theorem named "The Schema Theorem" ...
	\end{tcolorbox}
	
	Conclusion: The evolution is not made at the level of the individual but of the population (this is a quite good lesson to remember...)
	
	\begin{enumerate}
		\item[D1.] A gene or "\NewTerm{genotype}\label{genotype}" is composed of "\NewTerm{alleles}" (usually $2$, so we will consider individuals as being "$diploid$")

		\item[D2.] A "\NewTerm{gene pool}" is a set of genes present in a population.

		\item[D3.] A "\NewTerm{micro-evolution}" is a change in the frequency of alleles of a gene pool of a population.

		\item[D4.] An allele is named "\NewTerm{fixed allele}" when all members of a population have two identical alleles. 

		\item[D6.] A "\NewTerm{speciation}" is a long-term occurrence of a micro-evolution, making appear a new species.

		\item[D7.] The "\NewTerm{mutation}" is an evolutionary agent that alters the gene pool of a population by producing new genes (we therefore consider that there is no evolutionary agents at work in the population in this model).
	\end{enumerate}
	
	\begin{tcolorbox}[title=Remark,colframe=black,arc=10pt]
	Mutation is a rare event and usually harmful. Its quantitative effects are greater in short generation time organisms (bacteria and viruses), the effect is much less pronounced among organisms with long generation time (animals and plants).
	\end{tcolorbox}
	
	Let us now consider a population consisting of $13$ individuals (1 gene, each consisting of two alleles selected among 3):
	\begin{gather*}
	P=\left\{
	\begin{tabular}{@{}l@{}}
	    X1X1,X2X1,X3X2,X1X1,X1X1,X1X1,X2X3, \\
	    X1X3,X3X2,X1X1,X1X3,X3X3,X3X3
	\end{tabular}
	\right\}
	\end{gather*}
	The frequency of allele $X1$ is in this particular case:
	
	The frequency of allele $X2$ is:
	
	The frequency of allele $X3$ is:
	
	And obviously:
	
	To summarize, if we now consider a population of individuals possessing genes made of only two types of alleles, we have therefore (more concrete case relatively to the actual living biology on Earth):
	
	So we have for the relative frequency of the number of alleles $B$:
	
	and for the relative frequency of allele b:
	
	With obviously:
	
	We can make a few observations about the way of constructing this model. Indeed, the frequency of alleles (genetic equilibrium) remains in a population approximately constant generation after generation if:
	\begin{enumerate}
		\item The population is very large (a tiny disturbance changes only a little bit the frequency of the types of genotypes).

		\item No emigration or immigration (no new types of alleles in the population).

		\item No mutations altering the alleles (this contains the point number 2).

		\item Random couplings (not influenced by the type of allele studied).

		\item No natural selection (this contains the point number 4).
	\end{enumerate}
	It may be relevant (for general knowledge) to highlight the fact that we distinguish two types of non-random accouplement:
	\begin{enumerate}
		\item Inbreeding: crossing between individuals of the neighborhood, thus having family links; in species dispersing only a little bit. This apply especially to vegetables.

		\item Homogamy: free choose of partners who have some similarities for given physical characters. This apply especially among animals.
	\end{enumerate}
	The inbreeding and homogamy accouplement is not a direct cause of microevolution since they do not alter the gene pool of a population. However, if some individuals coming from this non-random accouplement are more likely to have accouplement with others (natural selection), it will follow a change in the allele frequency in the downward population and therefore a microevolution. As a result, non-random accouplement are a potential factor of microevolution.
	
	Finally, to return to our model, the probability that an individual has one of three possible genotypes (relatively to a genotype) in the population (relative to our example) is:
	\begin{enumerate}
		\item Probability that an individual is $BB$:
		

		\item Probability that an individual is $bb$:
		

		\item Probability that an individual is $Bb$ or $bB$:
		
	\end{enumerate}
	Finally, we get the "\NewTerm{Hardy-Weinberg equation}":
	
	That we can represent the graph:
	\begin{figure}[H]
		\centering
		\includegraphics{img/economy/hardy_weinberg.jpg}
		\caption{Plot of the Hardy-Weinberg equation}
	\end{figure}
	After this study, we can finally (re)defined the "\NewTerm{Darwin's law}" so it give us the possibility to eliminate the problem of the big question (problem) of continuous or discrete changes:
	
	\textbf{Definition (\#\mydef):} Natural selection is an evolutionary agent that alters the gene pool of a population by increasing the frequency of alleles producing phenotypes\label{phenotype} (genotypes defines the behavior of an individual) best suited to the medium (environment). Thus, the individuals that are the best adapted to their environment reproduce more than others and contribute more to the genetic heritage of the descendants. The frequency of good genes gradually increases in the population from generation to generation. Thus, natural selection directs the adaptation of a population to its environment by accumulating genotypes that promote survival in the medium (environement).
	
	\begin{tcolorbox}[title=Remarks,colframe=black,arc=10pt]
	\textbf{R1.} The Hardy-Weinberg model does not exist in nature. Indeed, it is impossible that all the necessary assumptions are present simultaneously. However, this is a theoretical model that evaluates if there is a microevolution in a population. Thus, we measure the frequency of alleles in the parent population and then we apply the Hardy-Weinberg equation. If the genetic structure of the descendant population differs from that predicted by the law, then we know that at least one agent of evolution is at work.\\
	
	\textbf{R2.} This theoretical model can help to predict a punctual estimator of the social costs of genetic diseases. Indeed, if we know the percentage of individuals with recessive disease in a population, we can apply the Hardy-Weinberg equation and evaluate the number of healthy transmitters and then we can predict the likelihood of the disease in future generations. If it is large (this reasoning is valid only in a liberal and capitalist framework), we may decide to invest in research or providing funds to treat the sick people.
	\end{tcolorbox}	
	\begin{tcolorbox}[colframe=black,colback=white,sharp corners]
	\textbf{{\Large \ding{45}}Examples:}\\\\
	E1. Given a genotype of possible sequences $\{BB,bb,Bb\}$ and let consider that $p_B=0.6,p_b=0.4$. We then have:
	
	Therefore, in a population of $1'000'000$ individuals, we should have:
	

	E2. If in Quebec ($6$ million inhabitant), one in five has blue eyes ($bb\Rightarrow q^2=1/5$), how many people are of genotype $BB$ and how many of type $Bb$?
	
	Therefore:
	
	Therefore we have:
	
	
	E3. We introduce 1,000 spotted frogs in a pond ( homozygous for this characteristic: $TT$) and 250 frogs without spots (homozygous for this trait: $tt$). If we allow the frogs to reproduce for several years, assuming that the population remains stable, what number of frogs $TT, Tt, tt$ should we then observe?\\
	
	So we have a population of $1,250$ individuals in which we $1,000 TT$ ($2,000$ allele $T$) and $250 tt$ ($500$ alleles $t$). So:
	
	If the population is of $1,250$ people, we have:
	
	\end{tcolorbox}


	\subsection{Mendel's law}
	During methodical hybridization of many species and varieties of various plants, it has been observed, in general, the uniformity and character of the first hybrid generation ($G1$) intermediate between the parents. So the inheritance is totally taken formthe dominant character in the first generation (father or mother but of the mix). At the second generation ($G2$) occurs polymorphism showing various combinations of characters of the initial generation ($G0$). Thus we find in the second generation $25\%$ of inidividuals with non-dominant character of the initial generation and $75\%$ with the dominant character. The character that was masked inthe first generation (G1) is then named "\NewTerm{recessive character}".
	
	This fact can be summarized with the following structure:
	\begin{figure}[H]
		\centering
		\includegraphics[scale=0.75]{img/economy/mendel.jpg}
	\end{figure}
	Mendel then made the assumption that the gametes of hybrids of the first generation ($G1$) must be in equal numerical proportions, of the type of one or other of the parents. This is the "\NewTerm{law of purity of gametes}".
	
	It has been established that human blood groups also obey to the rules of Mendelian inheritance. Thus, $A$ and $B$ boold groups are dominant and can appear in children if and only if they are present by the parents.
	
	\pagebreak
	\subsection{Growth rate with temperature}
	For many species (mammals, fish, micro-organisms), it is reasonable to consider as a first approximation, the relation of the population growth rate with the temperature of the environment is a second degree polynomial:
	
	where the values $a,b,c$ will depend of the considerated species.
	
	We know also that there is for most individuals a given optimum growth rate at a given temperature denoted $T_{\text{opt}}$, as well as minimum $T_{\min}$ and maximum $T_{\max}$ temperatures below and above which there is no more growth.

	Therefore, the parameters $a, b, c$ must satisfy the following relations:
	
	This implies for the three temperatures $T_{\text{opt}}, T_{\min}, T_{\max}$ to respect the relations proved in the context of the study of polynomials of the second degree in the section Calculus:
	
	So if we know for a living entity $T_{\text{opt}}$ and $T_{\min}$ it is easy to deduce theoretically an approximation of $T_{\max}$ in the context of the main assumption...
	
	\begin{flushright}
	\begin{tabular}{l c}
	\circled{90} & \pbox{20cm}{\score{2}{5} \\ {\tiny 57 votes,  60.00\%}} 
	\end{tabular} 
	\end{flushright}
	
	%to make section start on odd page
	\newpage
	\thispagestyle{empty}
	\mbox{}
	\section{Game and Decision Theory}\label{game and decision theory}
	\lettrine[lines=4]{\color{BrickRed}T}he decision and game  theory goes far beyond the narrow confines of social games, even if they constituted the first object of study and gave it its name in most commercially available books. Furthermore, the two theories are very close to each other from which the fact that they are very often non-differentiated in the literature.
	
\textbf{Definitions (\#\mydef):} 

\begin{enumerate}
	\item[D1.] The "\NewTerm{Game theory}" is the study of decision making models in uncertain future with unknown probabilities.
	
	\item[D2.] The "\NewTerm{Decision theory}" (sometimes also called "decision analysis") is the study of decision-making models in an uncertain probabilistic (objectively or subjectively) future.
	
	\item[D3.] The "\NewTerm{Data drive decision making}" is the practice of basing decisions on the analysis of data ("\NewTerm{Machine Learning"} or "\NewTerm{Data Mining"}), rather than purely on intuition (\SeeChapter{see section Numerical Methods page \pageref{data mining}}).
\end{enumerate}
Each of the methods of analysis of these two theories is mainly made in tabular (table) form or as a vertical or horizontal tree.

Here is a fairly familiar pattern for project manager who sums up the whole situation (data driver decision is not included):
\begin{figure}[H]
\centering
\includegraphics[scale=0.75]{img/economy/decision_classification_techniques.eps}
\caption{Elementary classification of decision techniques}
\end{figure}
These tools aim to try to formalize to decide which configuration or decision is better than another? We will look for this purpose to find the optimum of certain parameters that quantify the quality of a strategic situation. We must also determine which conditions lead to a configuration that is considered optimal.

Game theory and the decision is now quite common and used in academic circles, not only in economy (particularly corporate finance), but also by a whole class of other science in which the study on conflictual situations is relevant: sociology, biology, evolution, computer (video games), marketing, etc.

	\begin{tcolorbox}[title=Remark,colframe=black,arc=10pt]
In the world of corporations, and especially in Europe, decisions techniques are unknown to almost all of the leaders whose choices are often more qualitative and instinctive than scientific and thus less accurate...
	\end{tcolorbox}	
	
We will try, as always in this book, to minimize as possible the number of definitions and concepts in order not to drown the rigour of mathematical analysis in the chaos of useless vocabulary and not necessary for such analysis (and in the field of game theory it is otherwise a bit like in graph theory... a real nightmare of definitions!).

\textbf{Definition (naive \#\mydef):} A "\NewTerm{game}" is a situation where the players are driven to make strategic choices among a number of possible actions, within the framework defined in advance by the "\NewTerm{rules of the game}", the result of these choices constituting an "\NewTerm{outcome of the game}", which is associated with a "\NewTerm{gain}" (or payment), positive or negative, for each player.

	\begin{tcolorbox}[title=Remark,colframe=black,arc=10pt]
A player may be a person, a group of people, a society, a region, a political party, a country or even mother Nature ...
	\end{tcolorbox}	

Assumptions (we will find them again in the Economety section):
\begin{itemize}
	\item[A1.] The market is governed by competition and cooperation
	\item[A2.] The behavior of economic agents are rational (...)
	\item[A3.] It is possible to formalize competitive behavior
	\item[A4.] All competitive phenomena have a utilitarian dimension
\end{itemize}

We differentiate and define four types of situations (which we formalize below).

\textbf{Definitions (\#\mydef):}
\begin{itemize}
	\item[D1.] The "\NewTerm{cooperative or non-cooperative games}": a game is said to be cooperative when players can freely communicate and make agreements (e.g. in the form of a contract). They thus form a coalition and seek the general interest followed by a partition of earnings between all players. In a non-cooperative game, players (who do not communicate or can not communicate with one another) act according to the principle of economic rationality: each seeks to make the best decisions for themselves (i.e. seeks to maximize selfishly its individual earnings). This type of game involves probabilities.
	
	A first possible approach (without using math at first) of this game spirit is accessible to young children (without they know it!). Indeed, let us consider the following example:
	
	\begin{tcolorbox}[colframe=black,colback=white,sharp corners]
	\textbf{{\Large \ding{45}}Example:}\\\\
	Imagine two children, both gourmands in the presence of a homogeneous cake, perfectly divisible (and very good ...). If mom does two parts, there will inevitably disputes, each one finding the part of the other bigger. The only way (expected order) to avoid any argument is for the mother to impose the following rule: one of the children makes the parts, and the other one chooses first the part. The one who cuts can therefore not reason anymore taking into account only his own preferences, which would push him to cut a big part. He knows that the other will choose it. If then he cut a larger part than the other, it could find the latter in the neighbor's plate. He will therefore try to cut as evenly as possible in his point of view. Thus, whatever the choice of the other, he will think that the game is strategically equilibrated. It is this anticipation of the choice of the other decision which makes the originality of the theory of decision and also of cooperation!
	\end{tcolorbox}
	
	\item[D2.] The "\NewTerm{zero-sum or non-zero game}\label{zero sum or non zero sum game}": a game is said to be "zero sum" when the sum of the earnings of players is constant (or by the subtle choice of a utility function may be constant...) or in other words: what one wins is necessarily lost by another (chess, poker and some say that the stock market is a zero sum game but in reality it is false when you think about it globally ...). The social games are often zero-sum games but the real situations are often better described by the non-cooperative games with non-zero sum because some issues are profitable for all, or harmful for all (political, business situations...).

	\begin{tcolorbox}[title=Remark,colframe=black,arc=10pt]
	\textbf{R1.} Some theorists criticize zero-sum games, at least in the area of economic situation, on the basis that economic exchange is in principle mutually beneficial and that the zero-sum games would be totally unrealistic.\\
	
	\textbf{R2.} Zero sum games are sometimes named "\NewTerm{antagonistic games}".\\
	
	\textbf{R3.} Since the invention of the atomic bomb, the balance of terror is  based on the doctrine of offensive deterrence. Opposed by the reciprocal ability to inflict massive damage, the respective nuclear arsenals self-canceling in a zero-sum game, by a principle of mutually assured destruction.
	\end{tcolorbox}	
	
	\item[D3.] The "\NewTerm{with and without balance games}": a cooperative non zero sum game is said with "\NewTerm{Nash equilibrium}" if there are a couple of strategies (in the case of a two-player game) such that no player has the interest to unilaterally change his strategy to ensure the maximum minimum (the "\NewTerm{maximin}") earnings.
	
	\item[D4.] The "\NewTerm{competitive or non-competitive games}": a non-competitive game is the opposite of a competitive game such as by definition, when any pair of strategies (in the case of a two-player game) is such that it is win or lose all players simultaneously a given gain (when I lose something, you lose something, when I win something you also win something).
\end{itemize}

	\pagebreak
	\subsection{Behavorial decision bias (cognitive bias)}\label{cognitive bias}
	Before starting with technical and mathematics technics it is important for the manager, top manager, CEO/CFO and any other CXO(or even the student in Economy) to understand why it is more important to rely in mathematical results rather than to trust on human intuition and to try not forget what follows as most of time I see huge error of evaluation in Fortune $500$ companies!
	
	First, for information, the study of human intuition in economy is named "comportemental economy". And in this stud field "\NewTerm{cognitive biases}\index{cognitive biases}" are tendencies by humans to think in certain ways that can lead to systematic deviations from a standard of rationality or good judgment, and are often studied in psychology and behavioral economics.
	
	In the early 1970s, Amos Tversky and Daniel Kahneman introduced the term "cognitive bias" to describe people's systematic but purportedly flawed patterns of responses to judgment and decision problems (managers, students, economists, and so on...). Indeed, they have showed that people routinely employ heuristics - rules of thumb, or mental shortcuts - to simplify and, worse, oversimplify decisions under uncertainty. Moreover, they have shown time and again that our choices are frequently skewed by an array of cognitive biases.
	
	Most work before the 1970 were done in behavioral finance and focused on asset pricing and the behavior of investors. But increasingly, attention is being paid to decision-making in the corporate realm. Because of their training and experience, managers might be presumed to be less likely to use mental shortcuts, and less vulnerable to cognitive biases. True or not, consultants in decision analysis have made a good living by showing managers how they fall into decision traps, and professors have delighted in showing their executive MBA students just how flawed their judgment can be.
	
	On the way it is also important to distinguish between "cognitive biases" and "logical fallacies". A logical fallacy is an error in logical argumentation (e.g. ad hominem attacks, slippery slopes, circular arguments, appeal to force, etc.). A cognitive bias, on the other hand, is a genuine deficiency or limitation in our thinking — a flaw in judgment that arises from errors of memory, social attribution, and miscalculations (such as statistical errors or a false sense of probability\footnote{These latter being often the most difficult point during debates when people don't understand that when they use arguments, the have the \underline{\textbf{burden of proof}}, and even more difficult when they don't understand the concept of "proof"!!!!}).
	
	Although the reality of these biases is confirmed by replicable research, there are often controversies about how to classify these biases or how to explain them. Some are effects of information-processing rules (i.e., mental shortcuts), called heuristics, that the brain uses to produce decisions or judgments. Such effects are called cognitive biases. Biases have a variety of forms and appear as cognitive ("cold") bias, such as mental noise, or motivational ("hot") bias, such as when beliefs are distorted by wishful thinking. Both effects can be present at the same time.
	
	Over time, the behaviorists have compiled a long list of biases and heuristics. No one can say with certainty which of these inflict the most harm, but financial managers would do well to watch out for five: anchoring and adjustment, framing, optimism, overconfidence, and self-serving bias. One the next page the reader can see a very good summary illustration (the original author is sadly unknown...) of the most common well known cognitive bias:
	\begin{figure}[H]
		\centering
		\includegraphics[scale=0.45]{img/economy/cognitive_bias_summary.jpg}
	\end{figure}
	Also here is another useful and nice complementary poster that will be appreciated by any scientist or any person that has for purpose to reason following some good practices:
	\begin{figure}[H]
		\centering
		\includegraphics[width=1.0\textwidth]{img/economy/logical_fallacies.jpg}
		\caption{Some famous logicial fallacies}
	\end{figure}
	And apart from the fact that you can detect logical fallacies used by people, you can also categorize them quite well using the Myers-Briggs classification model:
	\begin{figure}[H]
		\centering
		\includegraphics[width=1.0\textwidth]{img/economy/myers_briggs_types.jpg}
		\caption{Myers-Briggs personality type classification}
	\end{figure}
	\begin{tcolorbox}[title=Remark,colframe=black,arc=10pt]
	And keep in mind.... the fact that you (as reader of this book) or other people are unable to grasp science and maths is obviously not a valid argument against it.
	\end{tcolorbox}
	Here we will focus only on two important bias for business: the "sunk cost" bias and the "anchor bias".
	
	\subsubsection{Sunk Cost}
	A sunk cost is a cost that has already been incurred and thus cannot be recovered. A sunk cost differs from future costs that a business may face, such as decisions about inventory purchase costs or product pricing. Sunk costs (past costs) are excluded from future business decisions, because the cost will be the same regardless of the outcome of a decision.
	
	The sunk cost fallacy is in game theory sometimes known as the "Concorde Fallacy", referring to the fact that the British and French governments continued to fund the joint development of Concorde even after it became apparent that there was no longer an economic case for the aircraft. The project was regarded privately by the British government as a "commercial disaster" which should never have been started and was almost canceled, but political and legal issues had ultimately made it impossible for either government to pull out.
	
	The sunk cost fallacy is the theory that continuing to put money into a project or other investment that is failing is worthwhile because of the expense that has already been spent on that investment. This is a problem because each injection of capital or investment money should be judged in terms of the likely returns on that money, not on costs previously accrued that may already be lost.

	
	The sunk cost dilemma with its sequence of good decisions should not be confused with the sunk cost fallacy, where a misconception of sunk costs can lead to bad decisions.[10] Sunk-cost fallacy occurs when people make decisions about a current situation based on what they have previously invested in the situation. For example, spending \$100 on a concert and on the day you find that it's cold and rainy. You feel that if you don't go you would've wasted the money and the time you spent in line to get that ticket and feel obligated to follow through even if you don't want to.
	
	When making business decisions, organizations consider relevant costs, which include the future costs and revenue of one choice compared with another. To make an informed decision, a business only considers the costs and revenue that will change as a result of the decision; sunk costs that do not change are not considered.
	
	\begin{tcolorbox}[colframe=black,colback=white,sharp corners]
	\textbf{{\Large \ding{45}}Example:}\\\\
	In a study of $96$ business students in 1976\footnote{Staw, Barry; Blumer, Catherine (1976). "Knee Deep in the Big Muddy". Organizational Behavior and Human Decision Process. 35: 124–140. doi:10.1016/0749-5978(85)90049-4} (sample size that seems to me far too slow for such a study after having done the sample size calculation by hand...) had a choice between making an R\&D investment either in an underperforming company department, or in other sections of the hypothetical company. The participants were divided into two groups: a low responsibility condition and a high responsibility condition. In the high responsibility condition, the participants were told that they, as manager, had made an earlier, disappointing R\&D investment. In the low responsibility condition, subjects were told that a former manager had made a previous R\&D investment in the underperforming division and were given the same profit data as the other group. In both cases subjects were then asked to make a new \$ $20$ million investment. There was a significant interaction between assumed responsibility and average investment, with the high responsibility condition averaging \$ $12.97$ million and the low condition averaging \$ $9.43$ million.\\

	Similar results seems to have been obtained in earlier studies by Staw, Arkes and Blumer (1985\footnote{ Arkes, Hal; Blumer, Catherine (1985). "The Psychology of Sunk Cost". Organizational Behavior and Human Decision Process. 35: 124–140. doi:10.1016/0749-5978(85)90049-4}) and Whyte (1986\footnote{Whyte, Glen (1986). "Escalating Commitment to a Course of Action: A Reinterpretation". The Academy of Management Review. 11 (2): 311. doi:10.2307/258462. ISSN 0363-7425}) but as it quite difficult to get the source data and scientific protocol it difficult to trust these results.
	\end{tcolorbox}
	Many people have strong misgivings about "wasting" resources (loss aversion). In the above example involving a non-refundable project, many people, for example, would feel obliged to go to continue the project despite not really wanting to, because doing otherwise would be wasting the project price; they feel they've passed the point of no return. This is sometimes referred to as the "sunk cost fallacy". Economists would label this behavior "irrational": it is inefficient because it misallocates resources by depending on information that is irrelevant to the decision being made.
	
	\subsubsection{Anchoring Bias}
	Anchoring or focalism is a cognitive bias that describes the common human tendency to rely too heavily on the first piece of information offered (the "anchor") when making decisions. 
	
	When people are trying to make a decision, they often use an anchor or focal point as a reference or starting point. Psychologists have found that people have a tendency to rely too heavily on the very first piece of information they learn, which can have a serious impact on the decision they end up making. In psychology, this type of cognitive bias is known as the anchoring bias or anchoring effect.
	
	For example, the initial price offered for a used car sets the standard for the rest of the negotiations, so that prices lower than the initial price seem more reasonable even if they are still higher than what the car is really worth.
	
	Studies have shown that anchoring is very difficult to avoid. For example, in one study students were given anchors that were obviously wrong. They were asked whether Mahatma Gandhi died before or after age 9, or before or after age 140. Clearly neither of these anchors are correct, but the two groups still guessed significantly differently (choosing an average age of 50 vs. an average age of 67).
	
	\begin{tcolorbox}[colframe=black,colback=white,sharp corners]
	\textbf{{\Large \ding{45}}Example:}\\\\
	The anchoring and adjustment heuristic seems was first to "theorized" by Amos Tversky and Daniel Kahneman. In one of their first studies, participants (in a unknown sample size to us so quite difficult to say if the result is accurate or not) were asked to compute, within 5 seconds, the product of the numbers one through eight, either as $1 \times 2 \times 3 \times 4 \times 5 \times 6 \times 7 \times 8$ or reversed as $8 \times 7 \times 6 \times 5 \times 4 \times 3 \times 2 \times 1$. Because participants did not have enough time to calculate the full answer, they had to make an estimate after their first few multiplications. When these first multiplications gave a small answer – because the sequence started with small numbers – the median estimate was $512$; when the sequence started with the larger numbers, the median estimate was $2,250$. (The correct answer was $40,320$).\\ 
	
	In another study by Tversky and Kahneman, participants (again in a unknown sample size to us...) observed a roulette wheel that was predetermined to stop on either $10$ or $65$. Participants were then asked to guess the percentage of the United Nations that were African nations. Participants whose wheel stopped on $10$ guessed lower values ($25\%$ on average) than participants whose wheel stopped at $65$ ($45\%$ on average). The pattern has held in other experiments for a wide variety of different subjects of estimation\footnote{Tversky, A.; Kahneman, D. (1974). "Judgment under Uncertainty: Heuristics and Biases" (PDF). Science 185 (4157): 1124–1131. doi:10.1126/science.185.4157.1124. PMID 17835457}.
	\end{tcolorbox}
	This is why in business whoever makes that first offer has the edge since the anchoring effect will essentially make that number the starting point for all further negotiations. Not only that, it will bias those negotiations in your favor. That first offer helps establish a range of acceptable counteroffers, and any future offers will use that initial number as an anchor or focal point. One study even found that starting with an overly high salary request actually resulted in higher resulting salary offers.

	\subsubsection{Wisdom of Crowds}
	Way back in 1906, the English polymath Francis Galton visited a country fair in which $800$ people took part in a contest to guess the weight of a slaughtered ox. After the fair, he collected the guesses and calculated their average which turned out to be $547$ [kg]. To Galton's surprise, this was within $1\%$ of the true weight of $543$ [kg].
	
	This is one of the earliest examples of a phenomenon that has come to be known as the "\NewTerm{Wisdom of the crowd}\index{wisdom of crowds}" (even if it is anecdotal, not scientific!). The idea is that the collective opinion of a group of individuals can be better (outperform) than a single expert opinion.

	The popular interpretation of the Wisdom of Crowds phenomenon is that each participant brings a certain amount of information to the result, and a certain amount of noise. Over a large enough sample size, the noise (divergent) cancels itself out if the estimates are independent and non-emotional (there is no influence between the people), while the information converges on a value which, in the absence of systematic bias, should be proximate to the true value. This works quite well on very simple questions that doesn't involve mathematical competences.
	
	This is result is related to the  central limit theorem and the law of large numbers that are cousins. Basically, the main difference is that CLT is concerned with the distribution of the empirical mean and the LoLN is concerned with how close the empirical mean approaches the expected value:
	\begin{itemize}
		 \item Law of large numbers (LoLN): As the number of samples grows large, the empirical mean approaches the expected value with probability $1$, for recall (\SeeChapter{see section Statistics page \pageref{weak law of large numbers}}):
		
		\item Central limit theorem: As the number of samples grows large, the empirical mean becomes normally distributed. The mean of the normal distribution is that specified by the LoLN. Also, the standard deviation shrinks as the number of samples grows so, for a very large number of samples, the normal distribution becomes eventually a single number and the two results become equivalent.
		
	\end{itemize}

	 Obviously there as statistical situations where the crowd produces very bad judgment, and argues that in these types of situations their cognition or cooperation failed because (in one way or another) the members of the crowd were too conscious of the opinions of others and began to emulate each other and conform rather than think differently. 
	 
	 The most common application is the prediction market or project management task duration estimation.
	 
	 To make it work well the following practical assumptions need to be meet:
	 \begin{itemize}
	 	\item The individuals must be isolated to not be influenced by other people
	 	\item The individuals must come from different localizations to avoid cultural bias
	 	\item The individuals must have a given degree of expertise in the field related to the question
	 	\item The reasoning methods of individuals must be checked (to avoid false experts and emotion bias)
	 	\item The individuals must have a reward if they are near from the real value (otherwise some will troll the results)
	 	\item Individuals that may have a benefit for one reason or another that their estimated value is below or above the real one must be eliminated (risk of conflict of interest)
	 	\item Extreme values must be eliminated (as sadly there a still people like to play to trolls)
	 \end{itemize}
	 and a good practice (as always in management!) is to ask for a three point estimates: optimistic, modal (most likely), pessimistic.
	\begin{figure}[H]
		\centering
		\includegraphics{img/economy/sources.jpg}
	\end{figure}
	
	
	\pagebreak
	\subsection{Utility}
	As we will see more in details in the section Economy, "\NewTerm{utility}" is a measure of preferences over some set of goods and services. The concept is an important underpinning of rational choice in economics and game theory, because it represents satisfaction experienced by the consumer of a good. A good is something that satisfies human wants. Since one cannot directly measure benefit, satisfaction or happiness from a good or service, economists instead have devised ways of representing and measuring utility in terms of economic choices that can be measured. Economists have attempted to perfect highly abstract methods of comparing utilities by observing and calculating economic choices. In the simplest sense, economists consider utility to be revealed in people's willingness to pay different amounts for different goods.	
	
	\textbf{Definition (\#\mydef):} A "\NewTerm{utility function}" or "\NewTerm{payoff function}" is a function of all earnings (gains) of the set of $n$-players to $\mathbb{R}^n$ (most of time corresponding to monetary value) that associates utilities withdrawn by each player at the issue of the game. Formally if a game has a set of strategies $S_i$ for a given player $i$ we denote it by:
	
	If $U$ is a utility function, we will denote $U_i$ the function of the set of all outgoings of a game to $\mathbb{R}$ of player $i$. Such a function will be said to be "\NewTerm{representative of the preference $\succcurlyeq$}" if for any outgoing $x,y\in I$ (any outgoing of the set of outgoings $I$) we have:
	
	The theory of utility that game theory use axiomatize the fact that only this concept of preference should be important. Briefly, we will say that only the preference order of the utility of the outgoings of a game is important, the values of the gains relatively to each outgoing being without importance.
	
	\subsubsection{Pareto Optimum}\label{pareto optimum}
	A first criterion that comes to mind when we speak about Utility, and that is due to the Italian sociologist Vilfredo Pareto, is the optimality of the same name\footnote{Not to be confused with the "Pareto" completely empirical concept in economics that the most distributions are in the ratio $20/80\%$ (\SeeChapter{see section Quantitative Management page \pageref{pareto analysis}}).}.
	
	If in a game, a couple of strategies (tactics/choices) is such that it is impossible to improve the score of one of the two players without decreasing the score of the other, we say that these outcome is "\NewTerm{Pareto-optimal}" or "\NewTerm{Pareto-efficient}".
	
	More formally:
	
	Let us consider two issues (choices) $x$ and $y$, both belonging to the set $I$ of issues, and suppose that for each individual $i$, we have the following situation:
	
	In other words, no individual would be a priori prejudiced (compared to the other) if we have substituted for each state $y$ by the state $x$. Let us assume moreover, that there is at least one individual $j$ who prefers strictly $y$ to $x$ as:
	
	In these circumstances, we do not really see why player should choose $y$ rather than $x$.
	
	\textbf{Definition (\#\mydef):} A feasible outcome $i$ that admits no possible improvement is named a "\NewTerm{Pareto optimum}" and is strictly formally defined as:
	
	
	The existence of Pareto optimality must be understood as a prerequisite, an "minimum minimorum" without which the concept of a cooperative game solution that we seek to develop should automatically be rejected!!!

	\begin{tcolorbox}[title=Remark,colframe=black,arc=10pt]
	This result form what we have already written earlier in this section. That is to say that if in a game, a couple of outcomes is such that it is impossible to improve the score of one of the two players without decreasing the score of the other, we say that these outcomes are "\NewTerm{Pareto-optimal}" or "\NewTerm{Pareto-efficient}".
	\end{tcolorbox}
	
	\subsubsection{Nash Equilibrium}
	\textbf{Definition (\#\mydef):} A "\NewTerm{Nash equilibrium}" (or "equilibrium" to make short) thus described game in issue in which no player has interest to change its strategy unilaterally, given the strategies of other players. In other words, if each player has chosen a strategy and no player can benefit by changing strategies while the other players keep theirs unchanged, then the current set of strategy choices and the corresponding payoffs constitutes a Nash equilibrium. 
	
	Game theorists use the Nash equilibrium concept to analyze the outcome of the strategic interaction of several decision makers. In other words, it provides a way of predicting what will happen if several people or several institutions are making decisions at the same time, and if the outcome depends on the decisions of the others. The simple insight underlying John Nash's idea is that one cannot predict the result of the choices of multiple decision makers if one analyzes those decisions in isolation. Instead, one must ask what each player would do, taking into account the decision-making of the others.
	
	Given $G$ a game with $n$ players, and:
	
	a combination of strategic choices of these $n$ players is the best strategic choice of player $i$ with $s_i^{*}\in S_i$, the set of all feasible strategies by player $i$.

	Given $U(s_1^{*},\ldots,s_n^{*})$ the utility of player $i$ when $s^{*}$ is selected. A combination of strategic choices is a Nash equilibrium if and only if:
	
	for any $s_i$ in $S_i$ and any $i$.
	
	Interpretation: No player may make a profit of a deviation of $s_i^{*}$, whatever the strategy he chooses in the set $S_i$. Thus, no player has an interest to deviate, and $s_i^{*}$ participates to an equilibrium.
	\begin{tcolorbox}[title=Remark,colframe=black,arc=10pt]
	It can happen that a Pareto optimum is not distinguishable form Nash equilibrium but this is not always the case (therefore a Nash equilibrium is not always Pareto optimal)!
	\end{tcolorbox}
	\textbf{Definition (\#\mydef):} When the strategy of a player is the best response to all possible strategies of the rivals, we speak then of "\NewTerm{dominant strategy}" (this strategy dominates all other strategies of the player). The equilibrium of this game is the named "\NewTerm{equilibrium in dominant strategy}".
	
	Verbatim, a strategy is "\NewTerm{dominated}" if it offers the player always lower earnings than those associated with at least one other of its strategies.
	\begin{tcolorbox}[title=Remark,colframe=black,arc=10pt]
	We can ask ourselves if in a non-cooperative game a Nash equilibrium (if it exists) is not such that it cause anyway like an implicit cooperation? In fact, this is not the case (and it's a very important result) as we will discussed it later in the study of the famous "prisoner's dilemma", that is a game where the Nash equilibrium is ensured by such individualistic and rational choices they are uncooperative !!! So it will be an extremely important example in the context of the market economy.
	\end{tcolorbox}
	Method: One way of determining the equilibrium of the game is to first eliminate all dominated strategies and to seek the equilibrium in the reduced corresponding game.
	\begin{tcolorbox}[colframe=black,colback=white,sharp corners]
	\textbf{{\Large \ding{45}}Example:}\\\\
	Consider the following game in a table form:
	\begin{table}[H]
	\centering
		\begin{tabular}{|l|*{3}{c|}}
			\hline
			{\cellcolor{black!30}}\backslashbox{\textbf{$J_1$}}{\textbf{$J_2$}}& {\cellcolor{black!30}}\textbf{$S_1$} & {\cellcolor{black!30}}\textbf{$S_2$} & {\cellcolor{black!30}}\textbf{$S_3$}\\
			\hline
			{\cellcolor{black!30}}\textbf{$S_1$} & $5,2$ & $4,4$ & $6,4$ \\ \hline
			{\cellcolor{black!30}}\textbf{$S_2$} & $3,1$ & $2,0$ & $5,2$ \\ \hline
		\end{tabular}
		\caption{Payoff matrix with Nash Equilibrium}
	\end{table}	
	But the next game at the opposite, does not include a Nash equilibrium. Indeed, whatever the couple of planned strategies, one player always gets  more by modifying his choice:
	\end{tcolorbox}
	
	\begin{tcolorbox}[colframe=black,colback=white,sharp corners]
	\begin{table}[H]
	\centering
		\begin{tabular}{|l|*{2}{c|}}
			\hline
			{\cellcolor{black!30}}\backslashbox{\textbf{$J_1$}}{\textbf{$J_2$}}& {\cellcolor{black!30}}\textbf{$S_1$} & {\cellcolor{black!30}}\textbf{$S_2$}\\
			\hline
			{\cellcolor{black!30}}\textbf{$S_1$} & $1,0$ & $0,1$ \\ \hline
			{\cellcolor{black!30}}\textbf{$S_2$} & $0,1$ & $1,0$ \\ \hline
		\end{tabular}
		\caption{Payoff matrix without Nash Equilibrium}
	\end{table}	
	\end{tcolorbox}
	However, currently it seems at least premature to prescribe to the players the choice of an equilibrium. Indeed if it is chosen, the situation is relatively stable, but there are three problems:
	\begin{enumerate}
		\item We are not sure of the existence of a couple of tactical equilibrium (conjunction of prudent tactics)

		\item Even in case of existence, we are not sure of the uniqueness of a couple of tactics in equilibrium

		\item Even in case of existence and uniqueness, we can prescribe another choice (!!!!)
	\end{enumerate}
	
	\pagebreak
	\subsection{Games Representations}

There are different ways to formalize Games and Decision theory and especially depending the type of situations in question. Thus, we distinguish:
	\begin{enumerate}
		\item The "\NewTerm{extensive forms}" which are synoptic forms  (tree, branch, leaf) useful to a simple understanding of the possible strategies and where the outcome of a game is represented by a leaf in which we find the vector of gains (or "earnings") of the respective players. This kind of representation becomes complicated (hard to draw) with repetitive games.
		When an extensive form use probabilities, we then refer to a "\NewTerm{decision tree}" because, as we mentioned at the beginning, which says known probabilities says complete apart theory: decision theory.
		\item 	The "\NewTerm{normal forms}" that can significantly reduce the size and time of graphical representation of a game as a table (matrix) of gains (or "earnings") but are inappropriate for repetitive games.\\\\
		At least two main sub-categories can be distinguished:
		\begin{itemize}
			\item The "\NewTerm{normal forms of zero-sum games}" (strictly competitive games) where according to an appropriate choice, it is possible to simplify the matrix representation (or "bimatrix") in a half-matrix since earnings are equal and opposite for players for each particular strategy.
			
			\item The "\NewTerm{normal forms of non-zero-sum games}" (competitive games).
		\end{itemize}
	\begin{tcolorbox}[title=Remark,colframe=black,arc=10pt]
Each cell of the table/matrix therefore contains a "vector" whose components are the respective gains of the players. If the game is a zero-sum one each cell contains a single value as what is earned by a player is lost by the other. We will see many such examples soon.
	\end{tcolorbox}		
		\item The "\NewTerm{set-forms}" that have a probability set-oriented approach that will allow us to study the last form below.
		
		\item The "\NewTerm{graphic forms}" that are pleasant to watch and that we will introduce as a complementary approach as using operations research (\SeeChapter{see section Numerical Methods page \pageref{operational research}}).
	\end{enumerate}
	
	Formally a game specifies:
	\begin{enumerate}
		\item Players: A set of players (agents who play the game) $N = \left\lbrace 1,..., n\right\rbrace$ with typical element (player) $i\in N$.
		
		\item Strategies: For each player $i$ a nonempty set of feasible strategies $S_i$ with typical element is $s_i\in S_i$.
		
		\item  Payoffs: For each player $i\in N$ a payoff (utility) function:
		
	\end{enumerate}
	A formal way to write down the normal-form of a game is:
	

	\subsubsection{Extensive representation of a Game}

The rules of a game of strategy and associated payoffs with it can thus be represented in a more extensive form commonly named by specialists "\NewTerm{Kuhn tree}".

As example consider two computers firms named $MBI$ and $Poire$ that have to do the choice of an operating system $CAM$ or $MAC$. The compatibility between the systems would be socially preferable, but for reasons related to the history of the two firms, each would prefer that it be the other who make the effort to adapt. If both firms choose $CAM$, $MBI$ (abbreviated $J_1$) wins \$600 million and Pear (abbreviated $J_2$) wins \$200 million. If they choose $MAC$, Pear wins \$600 millions and MBI \$200 million. If they are not compatible, then each earn \$100 million.

\begin{tcolorbox}[title=Remark,colframe=black,arc=10pt]
We call this type of game, a "\NewTerm{coordination game}". For example, the choice of television standards or Mac and PC drive match this type of games. Every manufacturer wants to impose its own standard but in case of disagreement, consumers may refuse to buy the product.
	\end{tcolorbox}	

Firms sequentially play thus the game can be represented as a decision tree:

\begin{figure}[H]
\centering
\includegraphics[scale=0.75]{img/economy/kuhn_tree.eps}
\caption{Sequential game in the form of a decision tree (or Kuhn's tree)}
\end{figure}

	\begin{tcolorbox}[title=Remarks,colframe=black,arc=10pt]
	\textbf{R1.} The information structure highlighted above refers to the information available to each player at each node making the game.\\
	
	\textbf{R2.} $MAC/CAM$ is a "\NewTerm{perfect information game}\label{perfect information game}" in the sense that players know exactly their range of strategies and those of their opponent and the precise consequences of these strategies. Thus, each node of the extensive form is visible by the players (we will define the concept of perfect information formally a little bit later).
	\end{tcolorbox}	
A simple analysis of the best strategy in the context of a game is to jump to the normal form as we shall see a little further (but this normal form is not suitable for an extensive form of a decision).

	\subsubsection{Extensive representation of a Decision}\label{extensive representation of a decision}

As we mentioned at the beginning of this section, the theories of games and decision are differentiated by the fact that the data of the first are totally deterministic universe while for the second, they are completely probabilistic. This last case is so important in the industry that there are, as we will see later, well-know softwares (@RISK, Isograph, TreeAge) specialized for the management of extensive forms (the latter being methods included in the risk Management ISO~31010 norm).

	The simplest case of decision in the industry decision is the "\NewTerm{event tree analysis} which uses (when the probabilities are fixed) only the basic axioms of probability (\SeeChapter{see section Probabilities page \pageref{kolmogorov axioms}}). If the probabilities are not fixed (which is common in reality), it will be necessary to use softwares incorporating the Monte Carlo methods (\SeeChapter{see section Numerical Methods page \pageref{monte carlo simulations}}).
	
	The event tree analysis is a graphical technique for representing sequences of mutually exclusive events following an initiating event depending on the operational/non-operational state of various systems designed to limit its consequences.
	
	Below you can see a simple example of event tree whose calculations are made automatically with the Microsoft Office Visio software (but same can be done with Microsoft Office Excel spreadsheet software):
	\begin{figure}[H]
		\centering
		\includegraphics[scale=0.85]{img/economy/event_tree.eps}
		\caption{Event tree analysis with fixed probabilities in Microsoft Office Visio}
	\end{figure}

	The annual frequency in the column located at the extreme right of the table (the "leafs" as say practitioners or "resulting frequencies") is simply equal to the estimated annual frequency of the initiating event multiplied by the product of the probabilities of a branch such that:
	
and we must of course be careful that the sum of the probabilities in each column is equal to 100\%.

Another typical example of extensive form are "\NewTerm{decision trees}".

To see what it is let us imagine an IT company $B$ in potential competition with another company $A$ (the latter can be seen as a set of competitors too!) for an international IT migration for a firm $X$.

Simplifying a little bit, however without being being out of reality, consider that two options are open to $B$: target "high prices" or target "low prices".

Suppose we also know that in the past $B$ has submitted a proposal for each contract of this type, while $A$ has done it only in 60\% of cases (no probability distribution function in our scenario yet but only punctual estimations!).

We also know that:
	\begin{enumerate}
		\item If $B$ submits a high price and is the only one to submit a proposal, its expected profit is 22 million.
		\item If $B$ submits a higher price but is in competition with $A$, it will get the contract on the price level requested by the group $A$. In this case, he knows he will get an average of 1 million.
		\item If $B$ submits a low price, it is sure to get the contract and make a profit of 10 million.
	\end{enumerate}
So in the case where the project is chosen only by its price (at the detriment of quality as often in reality...) the questions are then the following:

	\begin{enumerate}
		\item[Q1.] What should do $B$, if no further information can be obtained?\\\\
		This is a situation of the type: "\NewTerm{decision without information}".
		\item[Q2.] Assuming that a spy within the group $A$ can inform $B$ if the group $A$ will submit an offer or not, what would be the price of this information for $B$?\\\\
		This is a situation of the type: "\NewTerm{decision with perfect information}"
		\item[Q3.] A consulting firm may advise us, but his expertise, expensive, amounts to 1 million per study. To consider the use of its services, we know that in the past, of the 30 times that Group $A$ had in fact submitted an offer, the consulting firm had anticipated it 24 times. And, of the 20 times he had not submitted an offer, the consulting firm had anticipated it 17 times. Should $B$ order a study (anyway in reality this kind of information is almost very hard to obtain...)?\\\\
		This is a situation of the type: "\NewTerm{decision with imperfect information}".
	\end{enumerate}
Let us see now the solution (S) for each question (Q):
	\begin{enumerate}
		\item[S1.] To answer the first question (Q1), we first represent the problem to solve in the graphic form of a decision tree (which is for the moment quite simple to build also as a table) with the TreeAge software for example:
	\begin{figure}[H]
	\centering
	\includegraphics[scale=0.9]{img/economy/tree_without_information.eps}
	\caption{Decision tree without information in TreeAge}
	\end{figure}
		\begin{tcolorbox}[title=Remark,colframe=black,arc=10pt]
We would like to indicate that it is possible to make the same type of trees with Microsoft Office Visio but Monte Carlo modeling is not incorporated in this software and creating formulas consumes a lot of time (count a time factor of 10 to 20 compared to TreeAge, Isograph or @RISK).
	\end{tcolorbox}
Then, launching the calculation of the expected mean at each branch, named "\NewTerm{expected monetary value E.M.V.}"  TreeAge gives us just (this software has an option to make Monte Carlo modeling but the example here being with fixed probabilities this option is not necessary at this level):
	\begin{figure}[H]
	\centering
	\includegraphics[scale=0.9]{img/economy/tree_without_information_expected_mean.eps}
	\caption{Decision tree without information and expected mean in TreeAge}
	\end{figure}
Thus, the answer to the first question is that the strategy giving the greatest expected mean is the "low price" strategy because there is an expected gain of 10 million.

With the first decision (high price) we would have an expected mean of:
	
		\begin{tcolorbox}[title=Remark,colframe=black,arc=10pt]
In decision trees a basic check rule is to have all probabilities of a given branch that sum up to 1!
	\end{tcolorbox}
\item[S2.] To answer the second question (Q2) which is to know the financial value of the information given by the spy, we must first build the tree (in facts the example is so simple here that it is not really necessary but...) of a so-called competitive situation with "perfect information" (as the spy can give us an information completely sure).

The tree is with this small example very easy to build. If the spy tells us that the competitor A will make an offer, then we will have to propose the cheapest offer. Otherwise, we will propose the highest offer. The scenario is as follows:
	\begin{figure}[H]
	\centering
	\includegraphics[scale=0.9]{img/economy/tree_with_perfect_information.eps}
	\caption{Decision tree with perfect information in TreeAge}
	\end{figure}
The probability that there is competition is 60\% and 40\% that there is none. So the "\NewTerm{expected mean of monetary value of perfect information (E.M.V.P.I.)}" is in a situation of perfect information:
	
Therefore compared to the best previous situation we have a positive difference of 4.8 million. So this is the price value of the perfect information given by the spy.
	\item[S3.]About the third question (Q3) of determining the value of imperfect information provided by the consulting firm, the only certainty we have is that this information can't have a value greater than that of the perfect information. Thus it will have a value between 0 and 4.8 million.
	
To start, remember that according to the statement, we believe that for the current request for proposal, there is 60\% probability that there is competition and the consulting company in the past had 80\% of the time right  (24 times out of 30) when it said there would be competition (and thus 20\% of other times it was wrong...).

Respectively, we believe for the current request proposal that there is 40\% probability that there is no competition ($1-60\%$) and the consulting firm in the past was right 85\% of the time (17 times out of 20) when it said there would be no competition (and thus 15\% of other times wrong...).

What can be resume in the form of a table:
	
	\begin{table}[H]
	\begin{center}
		\begin{tabular}{|l|*{4}{c|}}
			\hline
			{\cellcolor{black!30}}\backslashbox{\textbf{Reality}}{\textbf{Previsions}}& {\cellcolor{black!30}}\textbf{Probability} & {\cellcolor{black!30}}\textbf{With Competition}			 & {\cellcolor{black!30}}\textbf{Without Competition}\\
			\hline
			{\cellcolor{black!30}}\textbf{With Competition} & 60\% & 80\% & 20\%\\ \hline
			{\cellcolor{black!30}}\textbf{Without Competition} & 40\% & 15\% &  85\% \\ \hline
		\end{tabular}
		\caption{Decision with imperfect information (original form)}
	\end{center}
	\end{table}

We would like now:
	\begin{enumerate}
		\item Calculate the probability that there is really competition AND that the consulting firm has planned a scenario with competition.
		\item Calculate the probability that there is really no competition AND that the consulting firm has planned a scenario with competition.
		\item Calculate the probability that there is really no competition AND that the consulting firm has planned a scenario without competition.
		\item Calculate the probability that there is really competition AND that the consulting firm has planned a scenario without competition.
	\end{enumerate}
To calculate these probabilities, we will use Bayes' formula (\SeeChapter{see section Probabilities page \pageref{bayes formula}}). As a reminder, the posterior and a priori probabilities are given by:
	
thus (we know a posteriori probabilities in our example):
	
We can now:
	\begin{enumerate}
		\item Calculate the probability that there is really competition AND that the consulting firm has planned a scenario with competition. We then use the previous table:
			
		where in this situation $B$ is the event "there was really competition" and $A$ is the event "competition scenario planned by consulting firm". $B/A$ is thus the a posteriori event "there was really competition when competition planned by consulting firm ".
		\item  Calculate the probability that there is really no competition AND that the consulting firm has planned a scenario with competition. We then use the previous table:
			
where in this situation $B$ is the event "there was really no competition" and $A$ is the event "competition scenario planned by consulting firm". $B/A$ is thus the a posteriori event "there was really no competition when competition planned by consulting firm".
		\item Calculate the probability that there is really no competition AND that the consulting firm has planned a scenario without competition. We then use the previous table:
			
where in this situation $B$ is the event "there was really no competition" and $A$ is the event "no competition scenario planned by the consulting firm". $B/A$ is thus the a posteriori event "there was really no competition when no competition was planned by consulting firm".
		\item Calculate the probability that there is really competition AND that the consulting firm has planned a scenario without competition. We then use the previous table:
			
where in this situation $B$ is the event "there was really competition" and $A$ is the event "no competition scenario planned by the consulting firm". $B/A$ is thus the a posteriori event "there was really  competition when no competition was planned by consulting firm".
	\end{enumerate}
We then have the following table which summarizes in a more or less usable way all possible scenarios:

	\begin{table}[H]
	\begin{center}
		\begin{tabular}{|l|*{4}{c|}}
			\hline
			{\cellcolor{black!30}}\backslashbox{\textbf{Reality}}{\textbf{Previsions}.}& {\cellcolor{black!30}}\textbf{With Competition} & {\cellcolor{black!30}}\textbf{Without Competition}			 \\
			\hline
			{\cellcolor{black!30}}\textbf{With Competition} & 48\% & 12\%\\ \hline
			{\cellcolor{black!30}}\textbf{Without Competition} & 6\% & 34\% \\ \hline
			{\cellcolor{black!30}}\textbf{Total} & 54\% & 46\% \\ \hline
		\end{tabular}
		\caption{Decision with imperfect information (tabular form)}
	\end{center}
	\end{table}
	
Using this table, we can easily check that the sum of columns equal 100\% (that is to say all eventualities) and thus that the calculations are correct.

So we see that 54\% of the time the consultancy firm plans a competition scenario (whatever will be the reality) and 46\% of the time no competition (whatever will be the reality).

	We therefore get the following decision tree with the TreeAge software:

	\begin{figure}[H]
	\centering
	\includegraphics[scale=0.8]{img/economy/tree_with_imperfect_information.eps}
	\caption{Decision tree with imperfect information in TreeAge}
	\end{figure}
	
Giving after calculations (always in TreeAge):

	\begin{figure}[H]
	\centering
	\includegraphics[scale=0.8]{img/economy/tree_with_imperfect_information_expected_mean.eps}
	\end{figure}

We find ourselves with an expected gain of 13 million (visible on the root of the above tree) minus the 1 million for the payment for the consulting firm, resulting finally in a net gain of 12 million.

Thus, the imperfect information the value is 2 million (remember that with low price strategy we are sure to get the contract and have a profit of 10 million) and must be compared to 4.8 million of the case with perfect information. This result is quite logic.
	\end{enumerate}
	
	\begin{tcolorbox}[title=Remarks,colframe=black,arc=10pt]
	\textbf{R1.} This type of tree is often used in pharmaco-economics (Cost–Benefit analysis). Thus, the root of the tree is an infection $X$ for which there are several antibiotics (branches $A$/$B$) and each has two issues (successful/failed treatment) with two possible outcomes (side effects yes/no). A probability is associated with each node and for the terminal nodes the treatment costs. Thus, by calculating the expectation, it is possible to determine the best economic antibiotic choice for the Medical Institute (obviously this is ethically useful when the success rate of both antibiotics $A$/$B$ is close and that the rate of side effects yes/no is also close... otherwise this method would be a scandal!).\\
	
	\textbf{R2.} In industry, many companies use these trees with probability distributions defined on each node. They then make a Monte Carlo simulation on the whole and make a sensitivity analysis (Tornado graph) with business intelligence softwares such as @RISK from Palisade.
	\end{tcolorbox}	
	
	\pagebreak
	\paragraph{Real Options}\mbox{}\\\\
	When using trees for investment decisions in projects (we speak then about "\NewTerm{investment analysis with real options}") we must take into account that each series of parallel branches is a project period (month, quarter, semester or year). We must therefore not forget then to update the values to the risk free rate of return of the market (\SeeChapter{see section Economy page \pageref{risk-free rate of return}}) for each period. Also such a tree give the possibility  calculate the investment option value, which is required by the executive board in high level companies.

	Consider as illustration case the following situation with a single time period:
	
	\begin{figure}[H]
		\centering
		\includegraphics[scale=0.75]{img/economy/investment_tree.jpg}
		\caption{Complete investment Tree (unreduced)}
	\end{figure}	
	In a software like Palisade @Risk this will give typically (unfortunately the options costs are always hidden in the total of the branches in specialized software, which is why I prefer most of time to draw my own trees in Microsoft Excel):		
	\begin{figure}[H]
		\centering
		\includegraphics[scale=0.75]{img/economy/investment_tree_palisade.eps}
		\caption{Previous tree builded in the software Precision Tree 6.3 of Palisade}
	\end{figure}

	Thus we have obviously:
	\thickmuskip=0mu
	\medmuskip=0mu
	
	\thickmuskip=3mu
	\medmuskip=3mu
	
	Therefore the tree can be reduced to:
	\begin{figure}[H]
		\centering
		\includegraphics[scale=0.75]{img/economy/investment_tree_reduced_optimistic.jpg}
		\caption{Optimistic investment reduced tree}
	\end{figure}
	But what is the expected net present value (eNPV)? Well this type of softwares allows us easy to integrate this type of calculation with or without a Monte Carlo simulation. Thus, if the market risk free rate is 10\% over a year, then we have (see the section on Economy for the detailed study of Net Present Value):
	
	Now it might be interesting (even requested by executive board!) to calculate the "\NewTerm{price of the option to invest}". In the pessimistic configuration  our tree becomes:
	\begin{figure}[H]
		\centering
		\includegraphics[scale=0.75]{img/economy/investment_tree_reduced_optimistic_and_pessimistic.jpg}
		\caption{Optimistic investment reduced tree}
	\end{figure}
	We then for this configuration:
	
	The price of real investment option (expected Option Value) is therefore:
	
	
	\subsubsection{Normal representation of a Game}
	To move to the "\NewTerm{normal representation}\index{normal representation}" or "\NewTerm{strategic representation}\index{strategic representation}", we define a strategy as a comprehensive action plan for each player, which specifies a choice for each node of the tree and therefore for every situation that may occur during the game. the "\NewTerm{payoff matrix}\index{payoff matrix}" represents the strategic situation of the players and the payoffs they receive for each strategy.
	
	We take again the previous example MAC/CAM:
	\begin{figure}[H]
	\centering
	\includegraphics[scale=0.75]{img/economy/kuhn_tree.eps}
	\caption{Sequential game in the form of a decision tree (or Kuhn's tree)}
	\end{figure}
	And we get for the strategic representation:
	\begin{table}[H]
	\centering
		\begin{tabular}{|l|*{3}{c|}}
			\hline
			{\cellcolor{black!30}}\backslashbox{\textbf{$J_1$}}{\textbf{$J_2$}}& {\cellcolor{black!30}}\textbf{CAM} & {\cellcolor{black!30}}\textbf{MAC}			 \\
			\hline
			{\cellcolor{black!30}}\textbf{CAM} & 600 , 200 & 100 , 100 \\ \hline
			{\cellcolor{black!30}}\textbf{MAC} & 100 , 100 & 200 , 600 \\ \hline
		\end{tabular}
		\caption{Payoff matrix of a non-zero sum game}
	\end{table}
	So this is a simple tabular representation of a game.

	\begin{tcolorbox}[title=Remarks,colframe=black,arc=10pt]
\textbf{R1.} We see in this matrix that the interests of the two companies are not completely opposed, they progress each time in the same direction when the strategies are opposed (if one lose, the other lose too and vice versa). Thus, the MAC/CAM game is a game whose payoffs are not growing in opposite directions (strategies). We speak then of "\NewTerm{not strictly competitive game}" (we will define this concept formally a little bit further).\\\\
\textbf{R2.} We also see that regardless of the strategy chosen by one player, each possible choice by the other player always bring to equivalent payoff. Therefore, when we say that it is a "\NewTerm{no prudent tactical game}".
	\end{tcolorbox}
	
	\textbf{Definition (\#\mydef):} A strategy is said to be a "\NewTerm{prudent tactic}" (that is the choice of the row number for the row player or the number of the column for the column player) when the gain of one of the players is such that when compared to a selected strategy, all the choice of its competitor provides a maximum gain to the latter. The minimum insured gain of (for example) $J_1$ is named the "\NewTerm{security level}" of $J_1$.
	
	\begin{table}[H]
	\centering
		\begin{tabular}{|l|*{5}{c|}}
			\hline
			{\cellcolor{black!30}}\backslashbox{\textbf{$J_1 (A)$}}{\textbf{$J_2 (B)$}}& {\cellcolor{black!30}}\textbf{$b_1$} & {\cellcolor{black!30}}\textbf{$b_2$} & {\cellcolor{black!30}}\textbf{$b_3$} & {\cellcolor{black!30}}\textbf{$b_4$}			 \\
			\hline
			{\cellcolor{black!30}}\textbf{$a_1$} & 5 , 5 & 6 , 4 & 0 , 10 & 4 , 6 \\ \hline
			{\cellcolor{black!30}}\textbf{$a_2$} & 1 , 9 & 7 , 3 & 5 , 5 & 6 , 4 \\ \hline
			{\cellcolor{black!30}}\textbf{$a_3$} & 6 , 4 & 7 , 3 & 7 , 3 & 8 , 1 \\ \hline
			{\cellcolor{black!30}}\textbf{$a_4$} & 4 , 6 & 8 , 1 & 0 , 10 & 2 , 8 \\ \hline
			{\cellcolor{black!30}}\textbf{$a_5$} & 3 , 7 & 5 , 5 & 9 , 0 & 0 , 10 \\ \hline
		\end{tabular}
		\caption{Payoff of a game with potential prudent tactic}
	\end{table}
	Player $A (J_1)$ may think that player $B (J_2)$ is very insightful, or very lucky, and so is able to choose the best possible response to any tactic of $A$.
	
	Therefore:
	\begin{itemize}
		\item If $A$ choose $a_1$, $B$ guess that and choose $b_3$ and $A$ would won $0$ (while $B$ will won $10$)
		\item If $A$ choose $a_2$, $B$ guess that and choose $b_1$ and $A$ would won $1$ (while $B$ will won $9$)
		\item If $A$ choose $a_3$, $B$ guess that and choose $b_1$ and $A$ would won $6$ (while $B$ will won $4$)
		\item If $A$ choose $a_4$, $B$ guess that and choose $b_3$ and $A$ would won $0$ (while $B$ will won $10$)
		\item If $A$ choose $a_5$, $B$ guess that and choose $b_4$ and $A$ would won $0$ (while $B$ will won $10$)
	\end{itemize}
	The careful choice of player $A$ ($J_1$) is $a_3$, which ensures him to earn at least $6$ (maximin). This minimum guaranteed gain is the security level. By doing the same the player $B$ ($J_2$), if he fears the extreme insight of the player $A$ ($J_1$), will choose $b_1$. This tactic ensures a gain of $4$ (maximin), which is also its security level.
	
	\begin{tcolorbox}[title=Remark,colframe=black,arc=10pt]
The response (strategy) of a player $i$ to the responses (strategies) of all other player is denoted by $r_i(\sigma_{-i})$ and therefore the best response to maximize the utility (payoff) $U$ of player $i$ is denoted by:
	
	Therefore in the example above $(a_3,b_1)$ is the best response of each player to the othere one and therefore we write this:
	
	\end{tcolorbox}
	
	If we study the game MAC/CAM by its first payoff matrix:
	\begin{table}[H]
	\centering
		\begin{tabular}{|l|*{3}{c|}}
			\hline
			{\cellcolor{black!30}}\backslashbox{\textbf{$J_1$}}{\textbf{$J_2$}}& {\cellcolor{black!30}}\textbf{CAM} & {\cellcolor{black!30}}\textbf{MAC}\\
			\hline
			{\cellcolor{black!30}}\textbf{CAM} & 600 , 200 & 100 , 100 \\ \hline
			{\cellcolor{black!30}}\textbf{MAC} & 100 , 100 & 200 , 600 \\ \hline
		\end{tabular}
		\caption{Payoff matrix of a non-zero sum game}
	\end{table}	
	 we can realize that there are two outstanding issues where the gain of both companies is maximum compared to other strategies (if we exclude the case taking into account probabilities and therefore "mixed strategies" that we will see much further below). These two issues are of interest for many reasons!
	
	Indeed, the two companies have no regrets about their choice of strategy. If they consider the strategy of their opponent as inevitable, their own strategy of choice is the best possible. Remember that we say therefore that the two issues are a "\NewTerm{Nash equilibrium}". The Nash equilibrium somehow characterizes individual rationality!
	
	The MAC/CAM game table above having two Nash equilibrium we are therefore not able, without any further information, to predict exactly what will be the outcome of the game. The two results are also equally likely.
	
	This is how game theory reveals the most favorable social strategy to both players: that is to say both players have at least to adopt the same strategy. All the problem that remains is to know which strategy will be chosen... therefore the game will have to be a cooperative game if possible.

	In the previous MAC/CAM game, the conjunction of tactics $(a_3, b_1)$ is also a Nash equilibrium in the way that each player has no reason to unilaterally change his strategy if he wants to preserve the minimum gain. This is due to a peculiarity of this specific game! In other words there are games:
	\begin{enumerate}
		\item That do not have equilibrium.
		\item That have equilibrium not corresponding to the conjunction of prudent tactics.
	\end{enumerate}

	
	\begin{tcolorbox}[colframe=black,colback=white,sharp corners]
	\textbf{{\Large \ding{45}}Example:}\\\\
	In this game, two players $J_1,J_2$ compete to stone, scissors, paper (SCP game...). In general, stone beats scissors (by blunting them), scissors beat paper (by cutting them), paper beats stone (by wrapping them). So every strategy beats another strategy, a draw against the second (equivalent) and is beaten by the third (so there is no winning strategy and there is verbatim no Nash equilibrium).
	\begin{figure}[H]
	\centering
	\includegraphics[scale=0.75]{img/economy/stone_scissors_paper_game.jpg}
	\caption{Extensive form of the stone-scissors-paper game (Kuhn tree)}
	\end{figure}
	
	To reveal the simultaneity of the game on the above representation, we surrounded the sets of information. $J_1$ knows that $J_2$ has chosen one of the three items, but he does not know which one, so he does not know the exact node where his own choice will intervene, and therefore he is unable to determine the outcome of the game. The game is therefore "\NewTerm{imperfect information}".
	\end{tcolorbox}

	\begin{tcolorbox}[colframe=black,colback=white,sharp corners]
	In a normal representation form, we have:
	\begin{table}[H]
	\centering
		\begin{tabular}{|l|*{3}{c|}}
			\hline
			{\cellcolor{black!30}}\backslashbox{\textbf{$J_1$}}{\textbf{$J_2$}}& {\cellcolor{black!30}}\textbf{Stone} & {\cellcolor{black!30}}\textbf{Scissors} & {\cellcolor{black!30}}\textbf{Paper}\\
			\hline
			{\cellcolor{black!30}}\textbf{Stone} & 0 , 0 & 1 , -1 & -1 , 1 \\ \hline
			{\cellcolor{black!30}}\textbf{Scissors} & -1 , 1 & 0 , 0 & 1 , -1 \\ \hline
			{\cellcolor{black!30}}\textbf{Paper} & 1 , -1 & -1 , 1 & 0 , 0\\ \hline
		\end{tabular}
		\caption{Payoff matrix of the stone-scissors-paper zero sum game}
	\end{table}	
	As we know this game is a "zero sum game" in the sense that whatever is gained by one player is lost by the other. In other words, we have already seen that we could therefore speak about a "strictly competitive game".\\
	
	Because in a  "zero sum game" whatever is gained by one player is lost by the other. we obviously see that we can represent the payoff matrix by a semi-matrix (therefore relatively to only one player):
	\begin{table}[H]
	\centering
		\begin{tabular}{|l|*{3}{c|}}
			\hline
			{\cellcolor{black!30}}\backslashbox{\textbf{$J_1$}}{\textbf{$J_2$}}& {\cellcolor{black!30}}\textbf{Stone} & {\cellcolor{black!30}}\textbf{Scissors} & {\cellcolor{black!30}}\textbf{Paper}\\
			\hline
			{\cellcolor{black!30}}\textbf{Stone} & 0 & 1 & -1 \\ \hline
			{\cellcolor{black!30}}\textbf{Scissors} & -1 & 0 & 1 \\ \hline
			{\cellcolor{black!30}}\textbf{Paper} & 1 & -1 & 0\\ \hline
		\end{tabular}
		\caption{Payoff semi-matrix of the stone-scissors-paper zero sum game}
	\end{table}	
	\end{tcolorbox}
	If necessary, if the respective game gains and losses  does not have the same "delta", we can just define a suitable utility function for the other player such that it is always possible for any strictly competitive game  where the gains are not equal and opposite to be put in the form of a half-matrix. We will prove later that there is such an utility function.
	
	\begin{tcolorbox}[title=Remark,colframe=black,arc=10pt]
	On the half-matrix of a zero-sum game, it is very easy to recognize if there is a Nash equilibrium or not. For example:
	\begin{table}[H]
	\centering
	\begin{tabular}{|>{\centering\arraybackslash}p{2cm}|>{\centering\arraybackslash}p{2cm}|}
	\hline
	2 & 0 \\\hline
	1 & 3 \\\hline
	\end{tabular}
	\end{table}
	In this game, the second row is the prudent tactic of line player and column player player will choose the first column as prudent tactic in which the greatest loss is not. Therefore, the line player will have interest to move to the first row and therefore tactics are not a balance and also there is no Nash equilibrium!!!
	\end{tcolorbox}
	
	\pagebreak
	In the tactical duel thus defined, the expected gain of the line player is the maximum of the minimum of the rows, that is to say the "\NewTerm{maximin}" while the expected gain of the column player is the minimum of the maximum of that is to say the "\NewTerm{minimax}".
	
	\textbf{Définitions:}
	\begin{itemize}
		\item[D1.] The "\NewTerm{maximin}", also sometimes named "\NewTerm{Wald criterion}" is a pessimistic criterion. It is indeed the purpose of this criterion to maximize the minimum result. That is to say formally in a 2-person game:
		
		where $D$ denotes the decision space, $S(d)$ denotes the set of states associated with decision $d$ and $f(d,s) $ denotes the payoff (outcome) associated with decision $d$ and state $s$.
		
		To implement it, it is convenient in a zero sum game and with perfect information:
		\begin{enumerate}
			\item For each decision (or strategy), to retain the minimum expected result.
			\item Among the worst results, choose the higher of the worst results of the various strategies.
		\end{enumerate}
		\item[D2.] The "\NewTerm{maximax}" according to the same logic as the previous criterion is to retain the best results of the possible strategies, therefore it is an optimistic criterion:
		
		To implement it, it is convenient in a zero sum game and with perfect information:
		\begin{enumerate}
			\item For each decision (or strategy), to retain the most highest expected result.
			\item Among the best results, choose the highest of the best results of the various strategies.
		\end{enumerate}
		\item[D3.] The "\NewTerm{minimax}" criterion is also named the "\NewTerm{Savage Criterion}" and only sometimes "\NewTerm{Von Neumann criterion}" and the idea is to retain the smaller of the best results of the possible strategies:
		
		To implement it, it is convenient in a zero sum game with perfect information:
		\begin{enumerate}
			\item For each decision (or strategy), to retain the lowest expected result.
			\item Among the best results, choose the highest of the best results of the various strategies.
		\end{enumerate}
	\end{itemize}
	If and only if the maximin equals minimax, their common value, which is the common expected gain of both opponents (players), is named the "\NewTerm{game value}" (we will prove this just a few lines below), and any couple formed by a such prudent tactic of the row player and prudent tactic of the column player define an equilibrium (for this reason the above example has not such equilibrium).
	
	\begin{tcolorbox}[colframe=black,colback=white,sharp corners]
	\textbf{{\Large \ding{45}}Example:}\\\\
	Consider the following semi-matrix of a zero-sum game with a Nash equilibrium:
	\begin{table}[H]
		\centering
		\begin{tabular}{|>{\centering\arraybackslash}p{2cm}|>{\centering\arraybackslash}p{2cm}|}
		\hline
		\cellcolor{black!30}2 & 4 \\\hline
		1 & 3 \\\hline
		0 & 4 \\\hline
		\end{tabular}
	\end{table}
	In this game, the line $1$ is the best prudent tactic of the row player and the column player player will choose column $1$ as prudent tactic in which the smallest losses are. Therefore, the top left cell corresponds to joint prudent tactics and corresponds as we can see it to a Nash equilibrium.
	\end{tcolorbox}
	
	\textbf{Definition (\#\mydef):} In a zero-sum game we name "\NewTerm{top}" the utility (seen in the point of view of the gain or loss), which is both a minimum in its row and a maximum in its  in its column (this is the case in the previous example where the equilibrium is a "top").
	
	\begin{theorem}
	Let us prove now that in any zero-sum game, if and only if the security levels of both players are opposed (the minimax equals the maximin), the combination of prudent tactics is always an equilibrium.
	\end{theorem}
	\begin{dem}
	Let us take again definition of a couple formed:
	\begin{itemize}
		\item by a prudent tactic $a_p$ for the player $A$, assuring him of gain at least of $v$ (maximin).
		
		\item by a prudent tactic $b_p$ for the player $B$, assuring him of gain at least of $-w$ (maximin).
	\end{itemize}
	In the case of a zero-sum game, we can always redefine the utility function of one of the players to obtain $v=w$ as we have seen before in order to write the half-matrix game form. Therefore, let us observe what happens (remembering that in such a game, the gain is equivalent to the loss, therefore by extension when the gain is minimal loss for the first player then the loss  is minimal for the other player):
	
	The couple $(a_p,b_p)$, like any couple that contains $a_p$ ensures $A$ to earn at least $v$ and verbatim $B$ to earn at least $-v$ (since $v=w$).
	
	$A$ hast therefore has no interest to deviate unilaterally from the strategy $a_p$, since $B$ has guaranteed himself to lose at maximum $v$ in the strategy of $A$. Similarly, $B$ has no interest to unilaterally deviate from the strategy $b_p$ since $A$ has guaranteed himself to earn at least $v$.
	
	Therefore, in the case where security levels of the two players are equal and opposite, the combination is a prudent tactical balance.
	
	We have already seen previously an example where the levels were not exactly opposed.
	\begin{flushright}
		$\square$  Q.E.D.
	\end{flushright}
	\end{dem}
	
	Let us now see a famous example that summarize some concepts see so far and that has some important implication in Economy named the "\NewTerm{prisoner's dilemna}\index{prisoner's dilemna}".
	\begin{tcolorbox}[colframe=black,colback=white,sharp corners]
	\textbf{{\Large \ding{45}}Example:}\\\\
	The prisoner's dilemma is a standard example of a game that shows why two completely "rational" individuals might not cooperate, even if it appears that it is in their best interests to do so. The prisoner’s dilemma is therefore a scenario in which the gains from cooperation are larger than the rewards from pursuing self-interest. It applies well to concept of "oligopoly" that we will introduce in the section of Economy (see page \pageref{oligopoly}).\\
	
	The story behind the prisoner's dilemma goes like this:\\

	Two co-conspiratorial criminals are arrested. When they are taken to the police station, they refuse to say anything and are put in separate interrogation rooms. Eventually, a police officer enters the room where Prisoner $A$ is being held and says: \textit{You know what? Your partner in the other room is confessing. So your partner is going to get a light prison sentence of just one year, and because you're remaining silent, the judge is going to stick you with $8$ years in prison. Why don't you get smart? If you confess, too, we'll cut your jail time down to $5$ years, and your partner will get $5$ years, also}. Over in the next room, another police officer is giving exactly the same speech to Prisoner $B$. What the police officers do not say is that if both prisoners remain silent, the evidence against them is not especially strong, and the prisoners will end up with only $2$ years in jail each.\\

	To understand the dilemma, first consider the choices from Prisoner $A$'s point of view. If $A$ believes that $B$ will confess, then $A$ ought to confess, too, so as to not get stuck with the $8$ years in prison. But if $A$ believes that $B$ will not confess, then $A$ will be tempted to act selfishly and confess, so as to serve only $1$ year. The key point is that $A$ has an incentive to confess regardless of what choice $B$ makes! $B$ faces the same set of choices, and thus will have an incentive to confess regardless of what choice $A$ makes. Confess is considered the dominant strategy or the strategy an individual (or firm) will pursue regardless of the other individual's (or firm's) decision. The result is that if prisoners pursue their own self-interest, both are likely to confess, and end up doing a total of $10$ years of jail time between them.\\

	It is implied that the prisoners will have no opportunity to reward or punish their partner other than the prison sentences they get, and that their decision will not affect their reputation in the future and they have no loyalty to each other.\\
	
	Because betraying a partner offers a greater reward than cooperating with them, all purely rational self-interested prisoners would betray the other, and so the only possible outcome for two purely rational prisoners is for them to betray each other. The interesting part of this result is that pursuing individual reward logically leads both of the prisoners to betray, when they would get a better reward if they both kept silent.
	\end{tcolorbox}
	For summary, the prisoner's dilemma is a two-player non-zero-sum game (so cooperation may be worthwhile), but does not have perfect information (so cooperation is tricky). It has a Nash equilibrium (mutual defection) that is not Pareto optimal. Defection is always a dominant strategy.
	
	What is important to remember in this game is that the members of an oligopoly (\SeeChapter{see section Economy page \pageref{oligopoly}}) can face a prisoner's dilemma, also! If each of the oligopolists cooperates then high monopoly profits are possible!!!
	
	
	\paragraph{Repetitive Games}\mbox{}\\\\ 
	Suppose that a man $J_a$ and a woman $J_1$ go see a movie. Once there, they must choose between going to see a documentary or a comedy. One of them prefers documentaries and the other comedies, but both prefer to see a movie together rather than separately: it's... the war of the sexes (WoS...).
	
	The strategies available to each player, considering that they make their choices simultaneously (which is unlikely in a real case, gallantry forcing to desynchronize the game in favor of women \Winkey) are then:
	
	\begin{enumerate}
		\item Go see a documentary, what that we will denote by $Doc$
		
		\item Go see a comedy, what we will denote by $Com$
	\end{enumerate}
	The payoff matrix will then be:
	\begin{table}[H]
	\centering
		\begin{tabular}{|l|*{2}{c|}}
			\hline
			{\cellcolor{black!30}}\backslashbox{\textbf{$J_1$}}{\textbf{$J_2$}}& {\cellcolor{black!30}}$Doc$ & {\cellcolor{black!30}}$Com$ \\
			\hline
			{\cellcolor{black!30}}$Doc$ & $2,3$ & $1,1$  \\ \hline
			{\cellcolor{black!30}}$Com$ & $1,1$ & $3,2$  \\ \hline
		\end{tabular}
		\caption{Payoff matrix of the WoS repetitive game}
	\end{table}
	That can be rewritten in the following form:
	\begin{table}[H]
	\centering
		\begin{tabular}{|l|*{2}{c|}}
			\hline
			{\cellcolor{black!30}}\backslashbox{\textbf{$J_1$}}{\textbf{$J_2$}}& {\cellcolor{black!30}}$Doc$ & {\cellcolor{black!30}}$Com$ \\
			\hline
			{\cellcolor{black!30}}$Doc$ & $1,2$ & $0,0$  \\ \hline
			{\cellcolor{black!30}}$Com$ & $0,0$ & $2,1$  \\ \hline
		\end{tabular}
		\caption{Payoff equilibrated matrix of the WoS repetitive game}
	\end{table}
	First point, we can notice that the WoF is not a strictly competitive game (therefore we don't need to try to represent it in the form of a half-matrix) and it is a coordination game. Second point, we note that the two issues with the maximum gain are Nash equilibrium (so we can not predict the outcome of the game). Therefore it is a game with multiple Nash equilibrium.
	
	Suppose now that that the couple returns to the cinema the following week, and they must make this choice again. We can represent this new situation with a game, which is actually a repetition of WoS, let us denote it by WoS2.
	
	If we consider that during the second stage each player knows what the other chosed in the first step, the available strategies are now conditional strategies: the new strategy can take into account of moves played by the opponent during the preceding game!!!
	
	The description of these strategies follows the following pattern: if we play  $S_1$ the first time, then if the other chose the documentary $Dom$ at the first movie, then we play $S_{2A}$, otherwise $S_{2B}$ with $S_1, S_{2A}, S_{2B}$ taking their value in the set $\left\lbrace Doc, Com \right\rbrace$. We denote these strategies by:
	
	We can read this notation as follows: we play $S_1$, then if we find ourselves in the situation $[S_1,Doc]$, then we play $S_{2A}$, or if we find ourselves in the situation $[S_2,Com]$, the we play equation $S_{2B}$. In this case WoS2 we have obviously $8$ strategies:
	
	\begin{enumerate}
		\item $\left(Doc,Doc|\left[Doc,Doc\right];Doc|\left[Doc,Com\right]\right)$: We always choose the documentary (aggressive strategy).
		
		\item $\left(Doc,Doc|\left[Doc,Doc\right];Com|\left[Doc,Com\right]\right)$: We always choose the documentary, excepted if the first time we found ourselves alone (then to be nice we choose the comedy).
		
		\item  $\left(Doc,Com|\left[Doc,Doc\right];Doc|\left[Doc,Com\right]\right)$: We choose always the documentary, except if the first time we've both have chosen the documentary (logic strategy).
		
		\item $\left(Doc,Com|\left[Doc,Doc\right];Com|\left[Doc,Com\right]\right)$: The first time we choose the documentary and the second time the comedy (win-win strategy).
		
		\item $\left(Com,Doc|\left[Com,Doc\right];Doc|\left[Com,Com\right]\right)$: The first time we choose the comedy and the second time the documentary (win-win strategy).
		
		\item $\left(Com,Doc|\left[Com,Doc\right];Com|\left[Com,Com\right]\right)$: We always choose the comedy, excepted if the first time we found ourselves alone (then to be nice we choose the documentary).
		
		\item $\left(Com,Com|\left[Com,Doc\right];Doc|\left[Com,Com\right]\right)$: we choose the comedy unless the first time, we both chose the comedy.
		\item $\left(Com,Com|\left[Com,Doc\right];Com|\left[Com,Com\right]\right)$: We always choose the comedy (aggressive strategy).
	\end{enumerate}
	For each end of WoS2, utility vectors are determined by the sum of vectors obtained for each step considered as end of WoS. We say that WoS is a "\NewTerm{super game}" which WoS is the "\NewTerm{constitutive game}".
	
	\textbf{Definition (\#\mydef):} A "\NewTerm{subgame perfect equilibrium}" corresponds to a strategic combination of actions such that every subgame are Nash equilibrium.
	
	\begin{tcolorbox}[title=Remark,colframe=black,arc=10pt]
	Obviously a "subgame" is simply a subtree of the game tree.
	\end{tcolorbox}
	
	Let us now see all these concepts in ensemblist (Set Theory) way (hang on well!!!).
	
	\pagebreak
	\subsubsection{Set representation of a Game}
	So far we have seen that there are a given number of elements that make up a game: the players, actions and strategies of the players, the sequences and steps of the game, game results and information available by the players at each choice of action.
	
	So far we have seen that there are a number of elements that make up a game: the players, the actions and strategies of the players, the sequences and stages of the game, the game results and the information available to the players each time they choice an action.
	\begin{enumerate}
		\item[D1.] The rules of a game indicate or must indicate:
			\begin{itemize}
				\item The succession of stages of the game, and the order in which the players are involved.
				
				\item The actions that are allowed at each step.
				
				\item The information available to the player everytime it needs to make a decision.
			\end{itemize}
		We have seen so far that there are two forms of possible representations for a game. One of them uses a tree (an extensive form) and the other a table (normal form). Under a formal expression it gives:
		
		\item[D2.] We have already defined previously the Set point of view of a normal form of a finite (non-probabilistic) game\index{finite game}. Remember it was defined formally by (we change the notation a little bit):
			
			with for recall:
			\begin{enumerate}
				\item Players: A set of players (agents who play the game) $J = \left\lbrace 1,..., n\right\rbrace$ with typical element (player) $i\in J$.
				
				\item Strategies: For each player $i$ a nonempty set of feasible strategies $S_i$ with typical element is $s_i\in S_i$.
				
				\item  Payoffs: For each player $i\in J$ a payoff (utility) function:
				
			\end{enumerate}
			Then for a finite game $G^N$ of length $K$ (the number of finite steps so that the game is surely finished for one player) and players $n=2$, a strategy for player 1 is a function:
			
			a strategy for player 2 is function:
			
			Notice that it is Player 1 turn to move if and only if the sequence of preceding moves is even, and Player 2 turn if and only if it is odd.
			
			\begin{tcolorbox}[title=Remarks,colframe=black,arc=10pt]
			If it may help, a "\NewTerm{finite game}\index{finite game}":
			\begin{itemize}
				\item Has a beginning and a end
				\item Is played with the goal of winning (so there is always a winner)
				\item Rules exist to ensure the game is finite
			\end{itemize}
			and a "\NewTerm{infinite game}\index{infinite game}":
			\begin{itemize}
				\item Do not necessarly have a knowable end point (no definite winner)
				\item Repeated interaction continue for an infinite amount of time
				\item The rational long term behavior is affected by endless moves of the players
				\item The players are and the rules are not fixed
			\end{itemize}
			\end{tcolorbox}
		
		\item[D3.] Formally a game tree (extensive form game) can be defined in multiples ways as there are not yet a convention between all practitioners. In our point of view the most complete one for finite and non-probabilistic games requires first the following definitions:
			\begin{enumerate}
				\item A finite set of nodes $N$.
				\item A unique initial node $n^0 \in N$.
				\item A set of terminal nodes $T \subset N$.
				\item A set of decision nodes such as $D= N \setminus T$ that is to say $D\cap N=\varnothing$.
				\item An immediate predecessor function $p$  such that $p: N \mapsto D$.
				\item A set of players (agents who play the game) $J = \left\lbrace 1,..., n\right\rbrace$ with typical element (player) $i\in J$.
				
				\item For each player $i$ a nonempty set of feasible strategies $S_i$ with typical element is $s_i\in S_i$.
				
				\item  For each player $i\in J$ a payoff (utility) function $U_i$.
			\end{enumerate}
			And therefore we can define a game tree as:
			
			
			\item[D4.] A "\NewTerm{pure strategy}" $s$ for a player is an application of the whole set of strategies $S_i$ of this player  to the set $S_i$ of all decisions of the game such that:
			
			Simply said, a pure strategy is a strategy that avoid any form of hazard, and is therefore completely deterministic.
			
			A pure strategy therefore provides a complete definition of how a player will play a game. In particular, it determines the move a player will make for any situation he or she could face. A player's strategy set is the set of pure strategies available to that player.
			
			\item[D5.] A "\NewTerm{mixed strategy}" for a player is a probability distribution $(P_1,P_2,...,P_n)$ with:
			
			 on the set of all pure strategies $S_i=(s_1,s_2,...,s_n)$ of player $j=i$.
		\begin{tcolorbox}[colframe=black,colback=white,sharp corners]
		\textbf{{\Large \ding{45}}Example:}\\\\
		Penalty kicks are a form of mixed-strategy game. Indeed, the goalkeeper has to anticipate the shot and can almost not analyze it quick enough. He must randomly choose if it will remain in the middle, or go to the left or right. Same thing for the attacker (usually the guardian should start just when the attacker kick) that can not know where the goalkeeper will jump will kick randomly.
		\end{tcolorbox}
		
		Obviously when a player no dominant strategy, he should consider playing a mixed strategy. We will see further below an application of such a situation involving probabilities (be patient!).
		
	\begin{tcolorbox}[title=Remarks,colframe=black,arc=10pt]
	\textbf{R1.} A pure strategy can obviously be regarded as a strategy that gives the probability $1$ at the strategy $s_i$ and $0$ to all others.\\
	
	\textbf{R2.} In our definition of the set of strategies $S_i$ of player $j=i$, there is a finite number of strategies for each agent but in economics, sets of strategies are often continuous and contain an infinite number of possible strategies (choice of quantity, price, etc.).
	\end{tcolorbox}
	Naturally, the result obtained by the player can therefore not be guarantee anymore with certainty, since the selection process of the decision involves probabilities. We can only get an expected mean!
	
	A pure strategy is thus a strategy by choosing among of all mixed strategies and that will be used thereof for all the duration of the game. A player using a mixed strategy against a player using a pure strategy will use (or will be forced) therefore to use also a pure strategy, but will not always use the same pure strategy every time they meet on the same game.
	
	\item[D6.] A "\NewTerm{strategy profile}" (sometimes named a "\NewTerm{strategy combination}") is a set of strategies for all players which fully specifies all actions in a game. A strategy profile must include one and only one strategy for every player.
	
	While a mixed strategy assigns a probability distribution over pure strategies, a behavior strategy assigns at each information set a probability distribution over the set of possible actions. While the two concepts are very closely related in the context of normal form games, they have very different implications for extensive form games. Roughly, a mixed strategy randomly chooses a deterministic path through the game tree, while a behavior strategy can be seen as a stochastic path.
		\begin{enumerate}
			\item A game is a "\NewTerm{purely concurential}" or "\NewTerm{strictly competitive}" if:
			
			Therefore a game is strictly competitive if for a set of pairs of outgoings, the gains of at least one of the player decrease whatever the strategy. If the both player have for a pair of outgoing, their respective gains that increase or decrease, then we have:
			
			
			\item A strictly competitive game is a "\NewTerm{zero sum game}" if:
			
			As we already know, we are dealing with a zero sum when the players' interests are diametrically opposed. In a two-player zero-sum game, for example, what is gained by one is lost by the other. This term has for original parlor games like poker where a player who wants to make money should do it at the expense of the others. Chess is also a zero sum game.
		\end{enumerate}
		
		\item[D8.] A "\NewTerm{super game}" is defined by:
		
		and is made by a constitutive game $G^N(J,\{S_i\}_{i\in J},\{U_i\}_{i\in N})$, and a number of iterations $T$ and of the vector $\Omega=(\omega_1,\ldots,\omega_n)$ of the discount rate of utility, $\omega_i$ being the discount rate of the player $i$ (most of times taken as being equal to $1$).
		
		Thus, as we have already mentioned it during our study of the WoS2 repetitive game, we consider that at a stage $t$ the choice dictated by a strategic combination $s$ to the player $n$ is denoted by $s_{n,t}$ and that the utility, for this same player, obtained at this stage of the game, that is to say the utility corresponding to the constituting game is denoted by $U_n(s_{n,t})$, then the utility associated with the outcome of the super game is given by:
	
	\end{enumerate}
	
	\pagebreak
	\subsubsection{Graphical representation of a Game}
	We now have cumulated enough evidence to have a probabilistic and operational approach relatively to simple zero-sum games.
	
	As it is still relatively difficult not to be too theoretical for this field of mathematics, to remain understandable we will study the theory through a unique "big" example.
	
	Let us consider two companies that we will name respectively $S1$ and $S2$ which are specialized in large-scale sale of a certain product and form a bilateral oligopoly in perfect competition (\SeeChapter{see section Economy page \pageref{oligopoly}}). The company $S1$ decided to invest in a new market, consisting of a set of areas of comparable importance.
	
	The penetration in different regions is done through the installation of a stand in chains stores $C1$ or $C2$ in each region. To better motivate its retailers, the company $S1$ will choose at the final stage only one distribution channel ($C1$ or $C2$) by region to sell its products.
	
	Following a market survey (we must get at least some number to do some math ...), the company $S1$ learn that its gains relatively to that of its competitor would be those as shown in the table below:
	\begin{table}[H]
	\centering
		\begin{tabular}{|l|*{3}{c|}}
			\hline
			{\cellcolor{black!30}}\backslashbox{\textbf{$S1$}}{\textbf{$S2$}}& {\cellcolor{black!30}}\textbf{C1} & {\cellcolor{black!30}}\textbf{C2} & {\cellcolor{black!30}}\textbf{NP}\\
			\hline
			{\cellcolor{black!30}}\textbf{C1} & $0$ & $2$ & $4$ \\ \hline
			{\cellcolor{black!30}}\textbf{C2} & $6$ & $-3$ & $8$ \\ \hline
			{\cellcolor{black!30}}\textbf{NP} & $-3$ & $-5$ & $0$\\ \hline
		\end{tabular}
		\caption[]{Game preparation for graphical analysis}
	\end{table}	
	The company $S2$ arrives at the same result following a market study (we simplify thanks to this hypothesis the analysis of the problem).
	\begin{tcolorbox}[title=Remarks,colframe=black,arc=10pt]
		\textbf{R1.} Since everything that wins a competitor would be lost by the other, the game is zero sum game (hence the fact that there is only one value in each cell)!\\
		
		\textbf{R2.} We will assume that the two companies can not and do not want to communicate with each other, in other words it is a non-cooperative game.
	\end{tcolorbox}
	Let us start by analyzing what are the strategies that have no interest for the both companies.
	
	For that, let us look if there is a strategy that will never be chosen by $S1$ whatever the strategy of $S2$:
	\begin{enumerate}
		\item If $S2$ choose $C1$ then $S1$ will have for best interest to choose $C2$.
	
		\item If $S2$ chooses $C2$ then $S1$ will have best interest to choose $C1$.
	
		\item If $S2$ chooses $NP$ then $S1$ will have for best interest to choose $C2$.
	\end{enumerate}
	We see here that whatever is the choice of $S2$, the company $S1$ will never choose $NP$. Therefore the strategy $NP$ for $ S1$ is totally dominated and can be therefore eliminated.
	
	Similarly, let us look if there is a strategy that will never be chosen by $S2$ whatever the strategy $S1$:
	\begin{enumerate}
		\item If $S2$ choose $C1$ then $S1$ will have for best interest to choose $C2$.
	
		\item If $S2$ chooses $C2$ then $S1$ will have best interest to choose $C1$.
	
		\item If $S2$ chooses $NP$ then $S1$ will have for best interest to choose $C2$.
	\end{enumerate}
	We see here that whatever is the choice of $S1$, the company $S2$ will never choose $NP$. Therefore the strategy $NP$ for $ S2$ is totally dominated and can be therefore eliminated.
	
	The table can therefore be simplified in the following way:
	\begin{table}[H]
	\centering
		\begin{tabular}{|l|*{2}{c|}}
			\hline
			{\cellcolor{black!30}}\backslashbox{\textbf{$S1$}}{\textbf{$S2$}}& {\cellcolor{black!30}}\textbf{C1} & {\cellcolor{black!30}}\textbf{C2}\\
			\hline
			{\cellcolor{black!30}}\textbf{C1} & $0$ & $2$\\ \hline
			{\cellcolor{black!30}}\textbf{C2} & $6$ & $-3$\\ \hline
		\end{tabular}
		\caption[]{Simplification of the game for graphical analysis}
	\end{table}
	Moreover, this game contains no Nash equilibrium (thus no pure strategy is advantageous). This game is therefore without equilibrium. Indeed, if $S1$ choose $S1$ , the $S2$ has better for interest to choose $C1$. But then $S1$ has better interest to play $C2$. But $S2$ has now better interest to choose rather $C2$. Which gives $S1$ again the desire to choose $C1$...
	
	For this purpose let us denote by $p$ and $q$ the frequencies with which the companies $S1$ and $S2$ choose the store chain $C1$:
	\begin{table}[H]
	\centering
		\begin{tabular}{|l|*{3}{c|}}
			\hline
			{\cellcolor{black!30}}\backslashbox{\textbf{$S1$}}{\textbf{$S2$}}& {\cellcolor{black!30}}\textbf{C1} & {\cellcolor{black!30}}\textbf{C2} & \\
			\hline
			{\cellcolor{black!30}}\textbf{C1} & $0$ & $2$ & $p$ \\ \hline
			{\cellcolor{black!30}}\textbf{C2} & $6$ & $-3$ & $1-p$ \\ \hline
			 & $q$ & $1-q$ & \\ \hline
		\end{tabular}
		\caption[]{Parametric form of the game for graphical analysis}
	\end{table}	
	These probabilities must be interpreted as follows:
	\begin{enumerate}
		\item If $p$ and $q$ are equal to $1$ then for all regions, it will be the store $C1$ that will get the business market.

		\item If $p$ and $q$ are equals for example to $9/11$ and $5/11$ it would mean that the company $S1$ will give the right to sales to the store chain $C1$ in $9$ regions on $11$ (the remaining $2$ being for the store chain $C2$) and respectively the company $S2$ will give the right to sell the store chain $C1$ in $5$ regions on $11$ (the remaining $6$ being for $C2$).
	\end{enumerate}
	So let us begin our study\label{graphical strategy with probabilities}. We will put us in an analysis point of view in which the company $S1$ is looging for a mixed strategy to maximize its gain (or utility) that we will denote by $v$ and to know the mixed strategy of the company $S2$ so that it minimizes its loss $v$ (since it is a zero sum game all what one wins another one will loose it).
	
	The system of equations will therefore be naturally for the company $S1$:
	
	and for the company $S2$:
	
	Now, we fall back here on a remarkable situation. Indeed, we have only here two standard forms of linear programming (\SeeChapter{see section Numerical Methods page \pageref{linear programming}}). We saw in our study of the latter that when there is only one unknown by system then it is possible, to pass a to graphical resolution without using the simplex algorithm (graphical resolution is easy to use in $2$D or $3$D situations).
	
	After simplification we get:
	\begin{table}[H]
		\centering
		\begin{tabular}{|c|c|}
		\hline
		\cellcolor{black!30}$S1$ & \cellcolor{black!30}$S2$ \\\hline
		$\max(v)$ & $\min(v)$ \\
		$v\leq -6p+6$ & $v\geq -2q+2$ \\
		$v\leq 5p-3$ & $v\geq 9q-3$ \\
		$0\leq p \leq 1$ & $0\leq 1 \leq 1$ \\\hline
		\end{tabular}
	\end{table}
	and the corresponding graphical representations of $v$ as a function of $p$ and respectively $q$:
	\begin{figure}[H]
		\centering
		\includegraphics{img/economy/game_graphic_representation.jpg}
		\caption{Correspondence of the inequalities of the optimization problem}
	\end{figure}
	By solving the simplex algorithm (\SeeChapter{see section Numerical Methods page \pageref{simplex algorithm}}) we have for optimal values for the two respective systems (it is also possible to read the approximate value on the graphs above but nevermind):
	\begin{table}[H]
		\centering
		\begin{tabular}{|c|c|}
		\hline
		\cellcolor{black!30}$S1$ & \cellcolor{black!30}$S2$ \\\hline
		$v=12/11$ & $v=12/11$ \\
		$p=9/11$ & $q=5/11$ \\ \hline
		\end{tabular}
	\end{table}
	The company $S1$ can therefore guarantee an average gain $v$ (we should talk about "mean" rather to be rigorous) of $12/11$. Indeed:
	
	and the probability $p$ is giving in fact the distribution between the store chain $C1$ that will have $9/11$ of the market of $C2$ the rest $2/11$ (the sum being of course equal to $1$).
	
	The company $S2$ may therefore also ensure an average gain $v$ of $12/11$. Indeed:
	
	and the probability $q$ is giving in fact the distribution between the store chain $C1$ that will have $5/11$ of the market of $C2$ the rest $6/11$ (the sum being of course always equal to $1$).
	
	\subsection{Expected Utility}
	To introduce expected utility that is used a lot by undegraduate project managers let us consider the following null-sum non-cooperative game:
	\begin{table}[H]
	\centering
		\begin{tabular}{|l|*{2}{c|}}
			\hline
			{\cellcolor{black!30}}\backslashbox{\textbf{$J_1$}}{\textbf{$J_2$}}& {\cellcolor{black!30}}\textbf{$S_1$} & {\cellcolor{black!30}}\textbf{$S_2$}\\
			\hline
			{\cellcolor{black!30}}\textbf{$S_1$} & $0$ & $2$ \\ \hline
			{\cellcolor{black!30}}\textbf{$S_2$} & $3$ & $1$ \\ \hline
		\end{tabular}
		\caption{Payoff matrix without Nash Equilibrium}
	\end{table}
	that does not have any equlibrium as we have seen earlier above. In this kind of game, any recommendation to a player to choose one tactic over another can harm him, since the opponent is informed, or can guess that recommendation.
	
	Indeed, if $J_1$ think that $J_2$ will choose the strategy $1$, he has better to chose the strategy $2$ (utility $3$ against $0$). But then, if $J_2$ think that $J_1$ 2 will choose the strategy $2$, he has better to chose the strategy $2$ (loss of $1$ instead of $3$). Then, if $J_1$ think that $J_2$ will choose the strategy $2$, he has better to chose the strategy $1$ (utility $2$ against $1$). But then, if $J_2$ think that $J_1$ will choose the strategy $1$, he hst better to to chose the strategy $1$ (loss of $0$ instead of $3$). And the circle is completed...
	
	Ultimately, the thing that matters the most in a non-cooperative game is that the strategy of a player can not be guessed by his opponent. Like any reasoning eventually be anticipated, the opponents being perfectly rational and informed, the only conceivable solution is to rely on a specific process, relied on probabilities assigned to various possible tactics. Thus, as we have defined it earlier above, this game has a "\NewTerm{mixed strategy}".
	
	Naturally, the result obtained by the player can not be guaranteed as the process of choice of the decision involves probabilities. Compare results therefore is like comparing lotteries. We imagine the situation of an admiral that has to answer to a military court for the loss of a ship, and explaining that he made his decision in playing dice (assuming a non-cooperative battle without Nash equilibrium): even if it is full compliant with the requirements of game theory, this explanation will be low efficient argument!
	
	\subsubsection{Hurwitz Criteria}\label{hurwitz criteria}
	So we need to introduce a probabilitic utility that we also sometimes name the "\NewTerm{Hurwitz criterion}". Let us consider a two-strategies $S_1,S_2$ game   and let us note the respective utility by:
	
	which allows to get $S_1$ with a probability $P$ and $S_2$ with a probability $1-P$. This relation is written with obvious notation (\SeeChapter{see section Statistics page \pageref{expected mean discrete variable}}):
	
	where E is obviously the "\NewTerm{expected utility}" (in similarity with the concept of expected mean studied in the section Statistics) or "\NewTerm{expected early gain}".
	
	We can already be notice that, if there is such an utility (expected), there are an infinite to a given arbitration, obtained from $U$ by an affine transformation strictly increasing, that is to say, a relation of the form:
	
	with $a>0$. Indeed, the prior-previous relation gives for $a>0$:
	
	which, added term by term to the obvious relation (required):
	
	leads to (do not confuse with the notatio of the variance!):
	
	This proves among other that we announced earlier: we can always choose a utility function (even in a perspective of pure strategy where $P=0$ or $P=1$) such that the delta of gains of the players in the zero-sum games are equal and opposite.
	\begin{tcolorbox}[title=Remark,colframe=black,arc=10pt]
	The expected utility (or "Hurwitz criterion") coincides with the criterion of the maximin when $P=0$ and maximax when $P=1$ (see below).
	\end{tcolorbox}
	Let us see first an academic example by considering the following zero-sum game:
	\begin{table}[H]
	\centering
		\begin{tabular}{|l|*{2}{c|}}
			\hline
			{\cellcolor{black!30}}\backslashbox{\textbf{$J_1$}}{\textbf{$J_2$}}& {\cellcolor{black!30}}\textbf{$b_1$} & {\cellcolor{black!30}}\textbf{$b_2$}\\
			\hline
			{\cellcolor{black!30}}\textbf{$a_1$} & $5$ & $2$ \\ \hline
			{\cellcolor{black!30}}\textbf{$a_2$} & $3$ & $4$ \\ \hline
		\end{tabular}
	\end{table}
	We see in this game that there is no Nash equilibrium. Indeed, if $J_2$ think that $J_1$ will decide $a_1$, he has better to chose $b_2$ (loss of $2$ instead of $5$). But $J_1$ understanding this will change for $a_2$ (gain of $4$ instead of $2$). But $J_2$ guessing this will change to $b_1$ (loss of $3$ instead of $4$), and $J_1$ that has understand everything will go back to $a_1$ (gain of $5$ instead of $3$).
	
	Let us now consider that the player $J_1$ will choose a number $x$ between $0$ and $1$, and thus take the decision $a_1$ with the probability $x$ and $a_2$ with the probability $1-x$. Similarly, the player $J_2$ will choose a number $y$ between $0$ and $1$ and htus take the decision $b_1$ with the probability $y$ and $b_2$ with the probability $1-y$.

	The results of these joint decisions are then:
	\begin{itemize}
		\item The gain of $5$, resulting from the combination of the decisions $(a_1,b_1)$, is obtained with the probability $xy$ (decisions are independent for both players!).

		\item The gain of $2$, resulting from the combination of the decisions $(a_1,b_2)$, is obtained with the probability $x(1-y)$ (decisions are still independent for both players!).

		\item The gain of $3$, resulting from the combination of the decisions $(a_2,b_1)$, is obtained with the probability $(1-x)y$ (decisions are still independent for both players!).
		
		\item The gain of $4$, resulting from the combination of the decisions $(a_2,b_2)$, is obtained with the probability $(1-x)(1-y)$ (decisions are still independent for both players!).
	\end{itemize}	
	The expected gain (or utility) of the player $J_1$ is therefore:
	
	\begin{tcolorbox}[title=Remark,colframe=black,arc=10pt]
	We see well that if $x = 0$ (and $y = 1$) then we fall on the criterion of Minimax (the maximum gain of the more pessimistic strategies) that is to say $\text{E}(J_1)$ is equal to $3$. Similarly if $x = 1$ (and $y = 1$) then we fall on the criterion of Maximax (the maximum gain the more optimistic strategies).
	\end{tcolorbox}
	If there is an equlibrium between the probabilistic strategies. $J_1$ will have no reason to change the value of $x$ in the hope of increasing $\text{E}(J_1)$. Therefore, the derivative with respect to $x$ must be zero as (maxima):
	
	In these conditions:
	
	To examine what is offer to $J_2$ as opportunities, whose expected mean, let us recall it, will be in a zero-sum game necessarily opposed to that of $J_1$, we write:
	
	Applying the same reasoning (but implicitly for the minima):
	
	In this case:
	
	Thus, we have determined the probabilities of the strategies that maximize the expected gains of this non-cooperative game! By adopting them $J_1$ is certain of an expected gain of at $7/2$ (since $J_2$ has nothing to gain to change its strategy) and $J_2$ is certain go get a gain of at least$|-7/2|$. The value $7/2$ (absolute value) is the "\NewTerm{value of the game}".
	
	\textbf{Definition (\#\mydef):} If the value of a non-cooperative game with mixed strategy is the same for both players, then we say that this is a "\NewTerm{mixed strategy equilibrium}" (neither player interest to deviate unilaterally).
	
	This result is certainly the most remarkable so far in out study of decision theory, as the non-cooperative games are the largest on the market.
	
	\subsubsection{Laplace Criteria}
	The Laplace criterion is a criterion that assigns the same probability in the absence of information for each decision (equal probability). The idea is then also tol calculate an expected gain for each decision given the assigned probability.

	In other words, the Laplace criterion purpose is to determine for each strategy the expected value by assigning the same probability to each state of nature and keeping the one whose expected utility is the highest.

	Let s see on an example by considering again the next zero-sum game:
	\begin{table}[H]
	\centering
		\begin{tabular}{|l|*{2}{c|}}
			\hline
			{\cellcolor{black!30}}\backslashbox{\textbf{$J_1$}}{\textbf{$J_2$}}& {\cellcolor{black!30}}\textbf{$b_1$} & {\cellcolor{black!30}}\textbf{$b_2$}\\
			\hline
			{\cellcolor{black!30}}\textbf{$a_1$} & $5$ & $2$ \\ \hline
			{\cellcolor{black!30}}\textbf{$a_2$} & $3$ & $4$ \\ \hline
		\end{tabular}
	\end{table}
	By applying the equiprobability, we have the following table:
	\begin{table}[H]
	\centering
		\begin{tabular}{|l|*{2}{c|}}
			\hline
			{\cellcolor{black!30}}\backslashbox{\textbf{$J_1$}}{\textbf{$J_2$}}& {\cellcolor{black!30}}\textbf{$\text{E}(b_1)$} & {\cellcolor{black!30}}\textbf{$\text{E}(b_2)$}\\
			\hline
			{\cellcolor{black!30}}\textbf{$\text{E}(a_1)$} & $5\dfrac{1}{2}+2\dfrac{1}{2}=3.5,5\dfrac{1}{2}+3\dfrac{1}{2}=4$ & $5\dfrac{1}{2}+2\dfrac{1}{2}=3.5,2\dfrac{1}{2}+4\dfrac{1}{2}=3$  \\ \hline
			{\cellcolor{black!30}}\textbf{$\text{E}(a_2)$} &  $3\dfrac{1}{2}+4\dfrac{1}{2}=3.5,5\dfrac{1}{2}+3\dfrac{1}{2}=4$  & $3\dfrac{1}{2}+4\dfrac{1}{2}=3.5,2\dfrac{1}{2}+4\dfrac{1}{2}=3$  \\ \hline
		\end{tabular}
		\caption[]{Detailed application of Laplace criterion}
	\end{table}
	The game then becomes:
	\begin{table}[H]
	\centering
		\begin{tabular}{|l|*{2}{c|}}
			\hline
			{\cellcolor{black!30}}\backslashbox{\textbf{$J_1$}}{\textbf{$J_2$}}& {\cellcolor{black!30}}\textbf{$\text{E}(b_1)$} & {\cellcolor{black!30}}\textbf{$\text{E}(b_2)$}\\
			\hline
			{\cellcolor{black!30}}\textbf{$\text{E}(a_1)$} & $3.5,4$ & $3.5,3$  \\ \hline
			{\cellcolor{black!30}}\textbf{$\text{E}(a_2)$} &  $3.5,4$  & $3.5,3$  \\ \hline
		\end{tabular}
		\caption[]{Laplace criterion simplified table}
	\end{table}
	In this example, where the expected utilities is always equal for the player $J_1$ whatever his strategy, the player $J_2$ will choose the strategy where his loss is the lowest either $b_2$. So here we have a Nash equilibrium (without Pareto optimality).
	\begin{tcolorbox}[title=Remark,colframe=black,arc=10pt]
	In the AHP, each element in the hierarchy is considered to be independent of all the others at the opposite of the ANP (Analytic Network Process). 
	\end{tcolorbox}
	
	\subsection{Evolutionary Game}
	The strategies of biological evolution, as we have mentioned it earlier in this chapter, can be modeled using game theory. In this context, the biologist is bring to define remarkable relations defining a given development strategy (dominance, stagnation, suicide or other).
	
	\textbf{Definition (\#\mydef):} An "\NewTerm{evolutionary steady strategy (ESS)}" is a strategy adopted by the majority and preventing a population is invaded by a mutant that would use a different strategy.

	This strategy is written in the form of a stable condition such that given two strategies $S_1,S_2$ of two players we have (explanations are given just after):
	
	or (if it does not occur) by the simultaneity of the two strategies of non-selection and suicide:
	
	\begin{enumerate}
		\item The first relation means that in no case an individual has to change from strategy to defend against a mutant evolution with the same strategy than him, because any other strategy would be unfavorable.

		\item The second relation means that whatever the strategy adopted against a mutant strategy, there will be stagnation.

		\item The third relation means that against $S_2$ any strategy different of $S_2$ is preferable to counteract $S_2$ itself. In other words, apply a strategy $S_2$ different from $S_1$ is suicidal (the opposite is therefore not!).
	\end{enumerate}
	To introduce such decision situation let use again a long companion example: the Hawk (bad) against Doves (good) game.
	 
	The $H$/$D$ game has for purpose to model the relations between individuals competing for a rare resource (food, water, area, companion, contract, etc.) which degree of adaptation will be changed both by obtaining this resource and the violence they suffer or inflict to them to get it.
	
	In their competitive interactions, biological organismes use two types of strategies (behaviors): the strategy of the hawk or that of the dove (as you can notice the "neutral" one is not present as it a suicide strategy). 

	The falcon intensify the conflict until he is injured or until the other run away. The dove withdrew after a first fight if the opponent chooses to escalate the violence of the conflict. When two hawks meet, one is injured and the other carries the resource. If a hawk and a dove fight, the hawk seized the safe resource to be injured and the dove gets no advantage or damage (at least on the short term and if they is not implicit trap). Finally, two doves share equally the resource.
	
	We also put the following hypothesis (assumptions):
	\begin{enumerate}
		\item[H1.] Fight take place one by one
		
		\item[H2.] The population can be considerate as infinite
		
		\item[H3.] The encounters are random
		
		\item[H4.] The battles are balanced (in the sense that neither age nor size nor experience influence the outcome of the battle)
		
		\item[H5.] It is impossible to know before the start of a conflict which strategy will be adopted by one of the player
	\end{enumerate}
	Based on these rules of interaction (where some are quite far from reality ...), it is possible to construct the tabular form of the game that allow us to calculate the advantages or disadvantages of the various strategies depending on the circumstances.

	Therefore the tabular form of the game is (explanations of the variables are given after):
	\renewcommand{\arraystretch}{2}
	\begin{table}[H]
	\centering
		\begin{tabular}{|l|*{2}{c|}}
			\hline
			{\cellcolor{black!30}}\backslashbox{\textbf{$J_1$}}{\textbf{$J_2$}}& {\cellcolor{black!30}}\textbf{$H$ (Hawk)} & {\cellcolor{black!30}}\textbf{$D$ (Dove)}\\
			\hline
			{\cellcolor{black!30}}\textbf{$H$ (Hawk)} & $\dfrac{V-C}{2},\dfrac{V-C}{2}$ & $V,0$  \\ \hline
			{\cellcolor{black!30}}\textbf{$D$ (Dove)} &  $0,V$  & $\dfrac{V}{2},\dfrac{V}{2}$  \\ \hline
		\end{tabular}
		\caption[]{Hawk vs Dove evolutionary game matrix}
	\end{table}
	As we can see the above game is a zero sum game, thus we can simplify the above table:
	\renewcommand{\arraystretch}{2}
	\begin{table}[H]
	\centering
		\begin{tabular}{|l|*{2}{c|}}
			\hline
			{\cellcolor{black!30}}\backslashbox{\textbf{$J_1$}}{\textbf{$J_2$}}& {\cellcolor{black!30}}\textbf{$H$ (Hawk)} & {\cellcolor{black!30}}\textbf{$D$ (Dove)}\\
			\hline
			{\cellcolor{black!30}}\textbf{$H$ (Hawk)} & $\dfrac{V-C}{2}$ & $V$  \\ \hline
			{\cellcolor{black!30}}\textbf{$D$ (Dove)} &  $0$  & $\dfrac{V}{2}$  \\ \hline
		\end{tabular}
		\caption{Simplified Hawk vs Dove evolutionary game matrix}
	\end{table}
	\renewcommand{\arraystretch}{1}
	We denote here by $V$ the advantage that a player withdraws from obtaining the resource. $V$ does not designate the resource itself, but the increased degree of adaptation that it provides to the player that gets it. $C$ represents the cost paid, endangerment or injury, to acquire the resource.

	First let us make explicit how this table should be read technically:
	\begin{enumerate}
		\item For the strategy $(D, D)$ - everyone is nice to everyone - the total gain of the two individuals is:
		
		The population will therefore remain constant (it's the stagnation). In short, by their behavior doves amicably share the resource value.

		\item For the strategies $(H,D),(D,H)$ the "doves" $D$ always lose (they do not progress in their evolution). Their gain is zero while the "hawks" have eliminated a number $V$ of doves (hence the gain).

		\item For the $(H,H)$ strategy the "hawks" support a loss $(V-C)/2$  where $C$ is a constant and such that the sum of gains of the hawks is normally less than or equal to $V$. In other words:
	
	In short, when a hawk fight another hawk, he gets on average $50\%$ of the time the value of the ressource reduced of the cost price price to get it.
	\end{enumerate}
	In summary to simplify:
	\begin{enumerate}
		\item Hawks (aggressive):
		\begin{itemize}
			\item Always attack other individuals, taking th e resource if the win
			\item If a hawk encounters another hawk, it will win only half the time
			\item When it loses, it will suffer and injury cost
		\end{itemize}
			
		\item Doves (non-aggressive):
		\begin{itemize}
			\item When a dove encounters an opponent, it may put on an aggressive display, but it does not fight
			\item When a dove encounters a dove, it will win the resource one halft the time
			\item When a dove encounters a hawk, the hawk will always win the resource.
			\item But since they don't fight, they don't incur any injury cost
		\end{itemize}
	\end{enumerate}
	Or in the form of a table:
	\begin{table}[H]
	\centering
		\begin{tabular}{|l|*{2}{c|}}
			\hline
			{\cellcolor{black!30}}\backslashbox{\textbf{$J_1$}}{\textbf{$J_2$}}& {\cellcolor{black!30}}\textbf{$H$ (Hawk)} & {\cellcolor{black!30}}\textbf{$D$ (Dove)}\\
			\hline
			{\cellcolor{black!30}}\textbf{$H$ (Hawk)} & \parbox{4.5cm}{Hawk wins $50\%$ of fights;\\ is injured in $50\%$ of fights.} & \parbox{4.5cm}{Hawk always wins;\\ dove flees.}  \\ \hline
			{\cellcolor{black!30}}\textbf{$D$ (Dove)} &  \parbox{4.5cm}{Dove never wins;\\is never injured.}  & \parbox{4.5cm}{Dove wins $50\%$ of fights;\\is never injured;wates times.}   \\ \hline
		\end{tabular}
		\caption{Hawk vs Dove evolutionary game matrix explicit example}
	\end{table}
	\begin{tcolorbox}[title=Remark,colframe=black,arc=10pt]
	This game can be seen as a war game between two players ... the interpretation of this results is therefore more than relevant. It is also an excellent comparison example between the competition that may exist between government offices (dowes) and private offices (hawks) providing the same type of product/service.  
	\end{tcolorbox}
	We must now consider two strategies:
	\begin{enumerate}
		\item The study of the game using a pure strategic game (without probabilities)

		\item The study of the game using a mixed strategy (involving probabilities)
	\end{enumerate}
	
	\subsubsection{Dove \& Hawk game in pure strategy (without probabilities)}	
	For this type of strategy we will consider $3$ cases:
	
	\begin{itemize}
		\item If $V>C$ (higher benefit than the cost of the strategy), choosing $(H,H)$ (which is therefore the strict Nash equilibrium of the game) we can see that the game will be a stable evolutionary strategy (SES). Indeed, we fall back on the relation:
		
		corresponding well to:
		
		and we can therefore also observe that there exist is also a weakly dominant strategy (no natural selection) in:
		
		corresponding well to:
		
		But that latter will not be adopted as weaker than the Nash equilibrium.

		\item If $V=C$ (benefit equal to the cost), the game is also of the SES kind. Indeed, $(H, H)$ becomes a weakly dominant strategy:
		
		corresponding well to:
		
		 and there is therefore no evolution and we can also observe that there is also:
		
		corresponding to:
		
		Since we have simultaneously:
		
		when $V=C$ the game is an SES.

		\item If $V<C$ (benefit less than the cost), nor $H$, nor $D$ are the dominant strategies and we do not have an SES:
		
		and:
		
		the latter two relations corresponding both to:
		
		It is rather annoying... it's a kind of collective suicide...
	\end{itemize}
	\begin{tcolorbox}[title=Remark,colframe=black,arc=10pt]
	The last two relations lead us to observe that the hawks will not want to necessarily reveal to others their strategy of doves predators, since: any strategy is better to be countered by another strategy rather than by itself. They may prefer to discuss among themselves which leads to the fact that the game therefore cooperative. 
	\end{tcolorbox}
	
	\subsubsection{Dove \& Hawk game in stable evolutionary strategy (with probabilities)}\label{dove hawk game probabilities}
	Let us now using the study in mixed strategy what we could do to bring the previous configuration to an SES (thus relatively to the last configuration $V<C$ to see more closely what we can do to avoid this) :

	We consider a population of individuals who therefore play a mixed strategy which we denote for each:
	
	with $P=P(H)$ and:
	
	If $P=1$ (pure strategy), we will therefore have:
	
	Let us now study the three cases:
	
	\begin{itemize}
		\item If $V>C$ with $\sigma_H$ and $1>P\geq 0$ we always have:
		
		in other words, the strategy will be always of the kind stable evolutionary (SES) if $\sigma_H$ is a pure strategy and this even if $\sigma$ can vary and tend to $\sigma_H$.

		\item If $V=C$ with $\sigma_H$ and $1>P\geq 0$, we always have:
		
		in other words: the strategy will always be of the non-selective type if $\sigma_H$ is a pure strategy even if $\sigma$ can vary and tend to $\sigma_H$.

		\item If $V<C$ and $1>P\geq 0$, we drop $\sigma_H$ to focus only to the generalization $\sigma(P,1-P)$. Then we have:
		
		Indeed:
		
		Also:
		
		Indeed:
		
		and we would to arrive at a SES with a mixed strategy. Is this possible?

		As part of a mixed strategy, we have proved during our study of a zero-sum game that mixed equilibrium was given by:
		
		It is therefore quite obvious that for a game that is not a zero-sum game, we have the mixed equilibrium given by:
		
		Therefore we seek the relation between $P$, $C$ and $V$ such that this equilibrium is reached. So in cascade we have:
		
		that can be written:
		
		Which lead us to write:
		
		Therefore:
		
		Knowing the utilities:
		
		from which we derive that the equilibrium in mixed strategy is given by:
		
		and therefore that the equilibrium is given by the mixed strategy:
		
		To what will this strategy lead to in practice? Well, simply in the suicidal case this mixed strategy is the best answer against itself (this is what is best done in what is worst) because it leads to the two conditions that satisfy a SES.
	\end{itemize}	
	
	\pagebreak
	\subsection{Cournot Competition}
	Cournot competition is an economic model used to describe an industry structure in which companies compete on the amount of output they will produce, which they decide on independently of each other and at the same time. 
	
	Let us imagine two companies $M$ and $N$, and two sources whose qualities are identical and that are positioned to concurrently supply the same market so that the total quantity delivered to shops consists of the sum of the quantities $m$, $n$ delivered by each company at a price that is necessarily the same for each of them as there is no reason to prefer one source to another. This price is determined when the sum of the quantities $m$, $n$ is completely known, because of the link between offer and demand. Let us suppose the company $N$ has set arbitrarily, without regard to the price, quantity he $n$ he intends to deliver, then the company $M$ will indirectly fix the sale price on the market, that is to say the total avalaible production (consisting of the sum of the quantities $m$ and $n$ for recall...), that is to say, its own production quantity $m$ to obtain the greatest possible income.
	
	In practice, a series of trial and error will bring the two companies in their equilibrium position, and the theory shows that this equilibrium is stable: that is to say that if one or the other owners depart momentarily from this equilibrium, he will be taken back by a series of oscillations of the kind that had originally led to constitute the balance.
	
	\textbf{Definition (\#\mydef):} A "\NewTerm{Cournot equilibrium}" is therefore related to the fact that when a firm raises price,the others will stay put. If a firm reduces price,the other firms will do the same. 
	
	\begin{tcolorbox}[title=Remark,colframe=black,arc=10pt]
	The Nash equilibrium is the generalization of the Cournot equilibrium. A Nash equilibrium can be interpreted as a pair of expectations about each player's choice such that, when the other person's choice is revealed, neither individual wants to change his behavior. 
	\end{tcolorbox}
	OK! Let us now focus on the maths part! But for this let us first gives the hypothesis (assumptions) of the mode:
	\begin{enumerate}
		\item[H1.] There is more than one firm and all firms produce a homogeneous product, i.e. there is no product differentiation;
		\item[H2.] Firms do not cooperate, i.e. there is no collusion;
		\item[H3.] Firms have market power, i.e. each firm's output decision affects the good's price;
The number of firms is fixed;
		\item[H4.] Firms compete in quantities, and choose quantities simultaneously;
		\item[H5.]  The firms are economically rational and act strategically, usually seeking to maximize profit given their competitors' decisions.
	\end{enumerate}
	We will put ourselves in the situation of a game with two plays. We will put that the price per unit $P$ of the product is a linear function of the total quantity produced $Q$ such that:
	
	where $\alpha$ a normalization constant of units: [price $\cdot$ piece$^{-2}$]. As it represents a price per unit divided be a global quantity.
	\begin{tcolorbox}[title=Remark,colframe=black,arc=10pt]
	We will see that this choice we will lead ob absurd result as in reality the price per unit decrease in function of the quantity and don't increase.
	\end{tcolorbox}
	We assume the marginal production costs equal and fixed and represents by the variable $C$, and the fixed costs as zero such that the production cost will be written respectively for the both companies:
	
	To recognize a game in normal form, it remains for us to identify the benefit of each opponent for any pair $(m,n)$ of strategies such that, if desired, to build the matrix of gains.

	Let us denote $\pi_i$ the margin of competitor $i$. Then we have respectively:
	
	The search for a Nash equilibrium leads each company to choose a production quantity that maximize profits and minimize its storage costs (see the Wilson's model in the section of Quantitative Management Techniques), the production of his concurrent being assumed to be known.
	
	For this purpose, we cancels the derivative of the previous two functions:
	
	System whose resolution brings leads us easily leads to the obvious solution that:
	
	It remains to verify that these solutions are maximums, by controlling the second-order derivatives, and not minimums. 

	The equilibrium position of the duopoly intervenes therefore when each of the two companies produces the same quantity... (not surprising result). The is then the Nash equilibrium production quantity.
	
	The sale price unit under a Nash equilibrium would be:
	
	Either a value below the marginal cost $C$ which obviously sounds not good as the margin is then negative. Indeed:
	Therefore the profit is given by:
		
	So it would be better that the price is not an affine function of the quantity...
	
	These calculations are to be compared with the purely economic reasoning, for which each company would like to be alone on the market (monopoly). The profit of the company $M$ in monopoly position would then be in the assumptions above given by:
	
	
	So we see obviously that:
	
	and therefore that the quantity produced in case of monopoly is greater in case of duopoly (logical!) and the profit and the prices can then be higher.
	
	The sale price unit under a monopoly would then be:
	
	and therefore:
	
	Therefore the benefit is still negative with this affine model (...) and even worst... the benefit in case of monopoly is small that in case of duopole...
	
	The idea would be now, if we go back to our two companies, that they establish and agreement (case named "\NewTerm{oligopoly agreement}" against the competition ... which is forbidden by the commercial laws in most countries!), that makes them share a potential majorited profit. The perfect symmetry of the situations would lead to the division by half of the market. But the difficulty is that the decision to produce:
	
	is not the best answer because it encourages to betray the agreement with the other. Thus, the best balance is that of imposing Nash equilibrium already seen above.
	
	During the development of an agreement or a cartel, one can distinguish several levels of quantities that depend on the degree of precision of the rules defined by the companies belonging to this cartel.

	The first case is that we can name a "\NewTerm{perfect agreement}"; This is the agreement that maximizes the total profit of all the companies at the same time. 

	The maximization of the total margin (profit) a set of companies will obviousyl be given by:
	
	where for recall we have the conditions:
	
	This profit is maximized when obviously all the partial derivatives of order $1$ are zero and maximum, that is:
	
	Where the term:
	
	expresses the change in total revenue caused by a small change in the quantity $q_i$ produced by the manufacturer $i$, and: 
	
	
	expresses the change in cost caused by the same variation of $q_i$ (producer $i$ marginal cost ). 

	The marginal revenue caused by a given change in output $Q$ is the same, regardless of the manufacturer which changed its production. Indeed, the influence of an additional production on the total supply and the price is the same, that this additional production comes from a producer or another one.
	
	\pagebreak
	\subsection{Markov Decision Processes (MDPs)}\label{markov decision process}
	As we mentioned in the section Probability, Markov chains are also used in the field of decision making techniques. We speak then of "\NewTerm{Markov decision processes (MDPs)}" (denomination that can be found in the ISO 31010 risk management standard) that provide a mathematical framework for modeling decision making in situations where outcomes are partly random and partly under the control of a decision maker. They are used in a wide area of disciplines, including robotics, automated control, economics, and manufacturing.

	More precisely, a Markov Decision Process is a discrete time stochastic control process. At each time step, the process is in some state $s$, and the hazard may choose any action $a$ a that is available in state $s$. The process responds at the next time step by randomly moving into a new state $s'$. The finally issue led to the possibility to help to take better decisions.
	
	\begin{tcolorbox}[title=Remark,colframe=black,arc=10pt]
	Strictly speaking, Markov decision processes are an extension of Markov chains; the difference is the addition of actions (allowing choice) and rewards (giving motivation). Conversely, if only one action exists for each state (e.g. "wait") and all rewards are the same (e.g. "zero"), a Markov decision process reduces to a Markov chain.
	\end{tcolorbox}
	For example, the following simple Markov chain (state diagram) on a particular medical symptom, has a step say to "\NewTerm{absorbing state}" known by all that it is not possible to survive to this day...:
	\begin{figure}[H]
		\centering
		\includegraphics[scale=1]{img/economy/markov_chain_medical.jpg}
		\caption{Medical state diagram}
	\end{figure} 
	And here is the same Markov chain decomposed over $20$ cycles as traditionally presented in the medical field (this assumes that the probabilities do not change over time ...)
	\begin{figure}[H]
		\centering
		\includegraphics[scale=1]{img/economy/markov_chain_medical_detailed.jpg}
	\end{figure} 
	We can also consider the following example applied to finance:
	\begin{figure}[H]
		\centering
		\includegraphics[scale=0.6]{img/economy/markov_chain_finance.jpg}
		\caption[Financial state diagram]{Financial state diagram (source: Wikipedia, author: Gareth Jones)}
	\end{figure} 
	where a bull week\footnote{"\NewTerm{Bearish}\index{bearish}" and "\NewTerm{Bullish}\index{bullish}" are simply terms used to characterize trends in the currency, commodity or stock markets. If prices tend to be moving upward, it is a bull market. If prices are moving downward, it is a bear market.} is followed by another bull week $90\%$ of the time, a bear week $7.5\%$ of the time, and a stagnant week the other $2.5\%$ of the time. Labelling the state space {1 = bull, 2 = bear, 3 = stagnant} the transition matrix (\SeeChapter{see section Probabilities page \pageref{transition matrix}}) for this example is:
	
	The distribution over states can be written as a stochastic row vector $p$ with the relation:
	
	So if at time $n$ the system is in state $p(n)$, then three time periods later, at time $p + 3$ the distribution is:
	
	In particular, if at time n the system is in state 2 (bear), then at time $n + 3$ the distribution is:
	
	Using the transition matrix it is possible to calculate, for example, the long-term fraction of weeks during which the market is stagnant, or the average number of weeks it will take to go from a stagnant to a bull market. Using the transition probabilities, the steady-state probabilities indicate that $62.5\%$ of weeks will be in a bull market, $31.25\%$ of weeks will be in a bear market and $6.25\%$ of weeks will be stagnant, since:
	
	Obviously it is possible to perform the same calculation directly in matrix form... again  the transition matrix (stochastic matrix) is easy to identify. It is (\SeeChapter{see section Probabilities page \pageref{transition matrix}}):
	
	The initial probability vector $p(0)$ is obviously in the most common cases of disease ...:
	
	Every time we multiply the transpose of the transition matrix by the vector of initial probabilities, we then get the probability of being in a given state at a given cycle !:
	
	With the Microsoft Excel 14.0.7173 the modeling is quite simple to reproduce:
	\begin{figure}[H]
		\centering
		\includegraphics[scale=0.8]{img/economy/markov_chain_medical_excel_datas.jpg}
		\caption[]{Medical Markov decision process example of vector probabilities calculations in Microsoft Excel 14.0.7173}
	\end{figure}
	Explicity for the first probability vectors:
	\begin{figure}[H]
		\centering
		\includegraphics[scale=0.75]{img/economy/markov_chain_medical_excel_formulas.jpg}
		\caption[]{Medical Markov decision process explicit vector probabilities calculations in Microsoft Excel 14.0.7173}
	\end{figure}
	If we continue up to the $20$th cycle:
	\begin{figure}[H]
		\centering
		\includegraphics[scale=0.75]{img/economy/markov_chain_medical_excel_datas_final_cycles.jpg}
		\caption[]{Medical Markov decision process far time states in Microsoft Excel 14.0.7173}
	\end{figure}
	It is very common in companies to synthesize the evolution graphically (more meaningful to management...):
	\begin{figure}[H]
		\centering
		\includegraphics[scale=0.85]{img/economy/markov_chain_medical_excel_plot.jpg}
		\caption[]{Medical Markov decision process far time states in Microsoft Excel 14.0.7173}
	\end{figure}
	So we see that the state "death" is an absorbing state because all probabilities converges to it (unfortunately ...). We also see that the total life expectancy if of $4.99$ cycles. If we assimilate a cycle to one year, then life expectancy is $4.99$ years. We immediately see that the stationary measure (\SeeChapter{see section Probabilities page \pageref{stationary measure}}) of the chain is:
	
	
	\pagebreak
	\subsection{Multi-Criteria Decision Making (MCDM)}
	 the field of multi-criteria decision making (MCDM) studies (among other topics) the problem of how to select the best alternative among a number of competing alternatives. This is an important task in decision making. In such a setting each alternative is described in terms of a set of evaluative criteria. These criteria are associated with weights of importance. Intuitively, one may think that the larger the weight for a criterion is, the more critical that criterion should be. However, this may not be the case. It is important to distinguish here the notion of criticality with that of importance. By critical, we mean that a criterion with small change (as a percentage) in its weight, may cause a significant change of the final solution. It is possible criteria with rather small weights of importance (i.e., ones that are not so important in that respect) to be much more critical in a given situation than ones with larger weights.
	 
	 In our daily lives, we usually weigh multiple criteria implicitly and we may be comfortable with the consequences of such decisions that are made based on only intuition. On the other hand, when stakes are high, it is important to properly structure the problem and explicitly evaluate multiple criteria. In making the decision of whether to build a nuclear power plant or not, and where to build it, there are not only very complex issues involving multiple criteria, but there are also multiple parties who are deeply affected from the consequences.
	 
	 The are dozens of MCDM models but we will here focus only on the most widely used one and also available in specialized decision softwares: the AHP.
	 
	 \subsubsection{Analytic Hierarchy Process (AHP)}
	 The analytic hierarchy process (AHP) is a MCDM structured technique for organizing and analyzing complex decisions, based on mathematics and psychology. It was developed by Thomas L. Saaty in the 1970s and has been extensively studied and refined since then. It is used almost exclusively in large US multinationals and in governments where politician know well the issue of human decision bias to make complex deterministic decisions.
	 
	 In the large majority of books, AHP is wrongly introduced (also in Wikipedia still in 2015...). So let us see exactly what are the mathematical concepts behind (you can check and apply this model with a free software like R - see my book on this software that have more details).
	 
	 \begin{tcolorbox}[title=Remark,colframe=black,arc=10pt]
	In the AHP, each element in the hierarchy is considered to be independent of all the others at the opposite of the ANP (Analytic Network Process). 
	\end{tcolorbox}
	We know that traditional voting or score / rank systems can not be used in multi-criteria and multi-level complex decisions because they do not trivialally take into account the relative importance of components. Moreover, even if a weight was affected the problem lies in the choice of an empirical value of that latter which makes the approach totally unscientific and opens the door to endless debates in practice.

	To introduce the technique of the AHP it seems pertinent to study this technique with a companion example of naive analysis of a sum of scores:	
	\begin{table}[H]
	\centering
		\begin{tabular}{|l|*{4}{c|}}
			\hline
			{\cellcolor{black!30}}\backslashbox{\textbf{Criteria}}{\textbf{Altnernative}}& {\cellcolor{black!30}}\textbf{Choice $X$} & {\cellcolor{black!30}}\textbf{Choice $Y$} & {\cellcolor{black!30}}\textbf{Choice $Z$} & {\cellcolor{black!30}}\textbf{Range}\\
			\hline
			{\cellcolor{black!30}}\textbf{Factor $A$} & $1$ & $4$ & $5$ & $0$ to $+5$  \\ \hline
			{\cellcolor{black!30}}\textbf{Factor $B$} & $20$ & $70$ & $50$ & $+1$ to $+100$  \\ \hline
			{\cellcolor{black!30}}\textbf{Factor $C$} & $-2$ & $0$ & $1$ & $-2$ to $+2$  \\ \hline
			{\cellcolor{black!30}}\textbf{Factor $D$} & $0.4$ & $0.75$ & $0.4$ & $0$ to $+1$  \\ \hline
			\hhline{|=|=|=|=|=|}
			{\cellcolor{black!30}}\textbf{Sum} &  $19.4$ & $74.75$	& $56.4$ &  \\ \hline
			{\cellcolor{black!30}}\textbf{Normalized score} &  $12.9\%$ & $49.7\%$ & $37.5\%$ & \\ \hline
		
		\end{tabular}
		\caption{Naive ranged scoring decision analysis approach}
	\end{table}
	However, the reader will notice that the range of values for each factor of the product / project  is not the same. It is then quite unfair to summarize all the values of several criteria and compare the result. It is clear that factor $B$ is dominant because the range has a higher value. To be fair, we can propose at least two trivial solutions:
	\begin{itemize}
		\item Instead of using arbitrary values for each factor, we position just the choice for each factor: the smaller rank value is more preferable than the higher rank value (this is psychology in relation to the daily use of The majority of humans).

		\item  We transform the score value of each factor as a function of the range value so that each factor will have the same extent (normalization of the scores).
	\end{itemize}
	Let us try the two solutions, starting with the first:
	\begin{table}[H]
	\centering
		\begin{tabular}{|l|*{3}{c|}}
			\hline
			{\cellcolor{black!30}}\backslashbox{\textbf{Criteria}}{\textbf{Altnernative}}& {\cellcolor{black!30}}\textbf{Choice $X$} & {\cellcolor{black!30}}\textbf{Choice $Y$} & {\cellcolor{black!30}}\textbf{Choice $Z$}\\
			\hline
			{\cellcolor{black!30}}\textbf{Factor $A$} & $3$ & $2$ & $1$ \\ \hline
			{\cellcolor{black!30}}\textbf{Factor $B$} & $3$ & $1$ & $2$ \\ \hline
			{\cellcolor{black!30}}\textbf{Factor $C$} & $3$ & $2$ & $1$  \\ \hline
			{\cellcolor{black!30}}\textbf{Factor $D$} & $2$ & $1$ & $2$  \\ \hline
			\hhline{|=|=|=|=|}
			{\cellcolor{black!30}}\textbf{Sum} & $11$ & $6$	& $6$ \\ \hline
			{\cellcolor{black!30}}\textbf{Normalized score} &  $26.09\%$ & $36.96\%$ & $36.96\%$\\ \hline
		\end{tabular}
		\caption{Naive ranged scoring decision analysis approach}
	\end{table}
	Where since the smallest rank value is the best, we have renormalized the codes in the following logical way:
	\begin{enumerate}
		\item Since the sum of all ranks is always equal, subtracting from the total of the ranks $R_T$ the sum of the ranks of a column automatically has the effect of giving it the inverse importance (indeed if the total sum of the ranks is equal to $23$ as above, then $23-6 = 17$ whereas $23-11 = 12$ so we have the desired effect). We then have the following sum of ranks ($SR$) for each criterion ($C$):
		
 
		\item We want a percentage, so it is natural to write:
		
 
		\item The problem of this last calculation is that it gives systematically, summed for all the columns, the value of $200\%$. It must then be multiplied by $0.5$ such that in the end:
		
	\end{enumerate}
 	Now it is important to notice that working with the ranks no longer gives the same result as in the antecedent table. Indeed, the choices $Y$ and $Z$ become indifferent and so we lost information!

	Reason why it is recommended to work on scores rather than ranks because not only working on the ranks can lead to equalities that did not need have to exist to be but in some cases it even reverses the final rank of some elements of a choice (we speak then of "\NewTerm{inversion effect of rank by the scores}\index{inversion effect of rank by the scores}").

	Now let us see the effect of normalizing the scores of the first table between $0$ and $100\%$ by simply using a rule of three (after shifting scores). The following table (same as first one above):
	\begin{table}[H]
	\centering
		\begin{tabular}{|l|*{4}{c|}}
			\hline
			{\cellcolor{black!30}}\backslashbox{\textbf{Criteria}}{\textbf{Altnernative}}& {\cellcolor{black!30}}\textbf{Choice $X$} & {\cellcolor{black!30}}\textbf{Choice $Y$} & {\cellcolor{black!30}}\textbf{Choice $Z$} & {\cellcolor{black!30}}\textbf{Range}\\
			\hline
			{\cellcolor{black!30}}\textbf{Factor $A$} & $1$ & $4$ & $5$ & $0$ to $+5$  \\ \hline
			{\cellcolor{black!30}}\textbf{Factor $B$} & $20$ & $70$ & $50$ & $+1$ to $+100$  \\ \hline
			{\cellcolor{black!30}}\textbf{Factor $C$} & $-2$ & $0$ & $1$ & $-2$ to $+2$  \\ \hline
			{\cellcolor{black!30}}\textbf{Factor $D$} & $0.4$ & $0.75$ & $0.4$ & $0$ to $+1$  \\ \hline
			\hhline{|=|=|=|=|=|}
			{\cellcolor{black!30}}\textbf{Sum} &  $19.4$ & $74.75$	& $56.4$ &  \\ \hline
			{\cellcolor{black!30}}\textbf{Normalized score} &  $12.9\%$ & $49.7\%$ & $37.5\%$ & \\ \hline
		
		\end{tabular}
	\end{table}
	Therefore becomes:
	\begin{table}[H]
	\centering
		\begin{tabular}{|l|*{3}{c|}}
			\hline
			{\cellcolor{black!30}}\backslashbox{\textbf{Criteria}}{\textbf{Altnernative}}& {\cellcolor{black!30}}\textbf{Choice $X$} & {\cellcolor{black!30}}\textbf{Choice $Y$} & {\cellcolor{black!30}}\textbf{Choice $Z$}\\
			\hline
			{\cellcolor{black!30}}\textbf{Factor $A$} & $0.109$ &	$0.436$ & $0.545$  \\ \hline
			{\cellcolor{black!30}}\textbf{Factor $B$} & $0.035$ &	$0.127$ & $0.090$  \\ \hline
			{\cellcolor{black!30}}\textbf{Factor $C$} & $0.000$ &	$0.091$ & $0.136$  \\ \hline
			{\cellcolor{black!30}}\textbf{Factor $D$} & $0.036$ &	$0.068$ & $0.036$  \\ \hline
		\end{tabular}
	\end{table}
	Now let us see a naturally more advanced version inspired by financial portfolios including weights to criteria. Obviously the result of the example that follows will no longer be comparable to the previous ones.

	So consider the following table of weights:
	\begin{table}[H]
	\centering
		\begin{tabular}{|l|c|c|c|c|c|}
	         \cellcolor[HTML]{C0C0C0} & \cellcolor[HTML]{C0C0C0} \textbf{Factor $A$} & \cellcolor[HTML]{C0C0C0}\textbf{Factor $B$ }& \cellcolor[HTML]{C0C0C0}\textbf{ Factor $C$} & \cellcolor[HTML]{C0C0C0}\textbf{Factor $D$} & \cellcolor[HTML]{C0C0C0}\textbf{Sum} \\
	\cellcolor[HTML]{C0C0C0}\textbf{Weight} & $6$  & $2$ & $2$ & $1$ & $11$ \\ \hline
	\cellcolor[HTML]{C0C0C0}\textbf{Relative Weight} & $54.5\%$ & $18.2\%$ & $18.2\%$ & $9.1\%$ & $100\%$ \\ \hline                  
		\end{tabular}
	\end{table}
	Our table of normalized scores therefore becomes:
	\begin{table}[H]
	\centering
		\begin{tabular}{|l|*{4}{c|}}
			\hline
			{\cellcolor{black!30}}\backslashbox{\textbf{Criteria}}{\textbf{Altnernative}}& {\cellcolor{black!30}}\textbf{Weight} & {\cellcolor{black!30}}\textbf{Choice $X$} & {\cellcolor{black!30}}\textbf{Choice $Y$} & {\cellcolor{black!30}}\textbf{Choice $Z$}\\
			\hline
			{\cellcolor{black!30}}\textbf{Factor $A$} & $54.5\%$	& $0.109$ & $0.436$ &	$0.545$ \\ \hline
			{\cellcolor{black!30}}\textbf{Factor $B$} & $18.2\%$	& $0.035$ &	$0.127$ &$0.090$ \\ \hline
			{\cellcolor{black!30}}\textbf{Factor $C$} & $18.2\%$	$0.000$ & $0.091$ & $0.136$  \\ \hline
			{\cellcolor{black!30}}\textbf{Factor $D$} & $9.1\%$ &	$0.036$ & $0.068$ &	$0.036$  \\ \hline
			\hhline{|=|=|=|=|=|}
			{\cellcolor{black!30}}\textbf{Sum} & $100\%$ & $0.180$ &	 $0.722$ & $0.808$ \\ \hline
			{\cellcolor{black!30}}\textbf{Normalized score} &  & $10.5\%$ & $42.2\%$ & $47.2\%$ \\ \hline
		\end{tabular}
	\end{table}
	Well now that these basic decisional techniques have been presented let us now strictly speaking introduce the AHP which has a different complementary approach!
	\begin{tcolorbox}[title=Remark,colframe=black,arc=10pt]
	Notice that the author of this method (Thomas Saaty) has edited a software that allows to automate complex decision-making procedures based on his method and which can be found at \url{http: /www.superdecisions.com}. The reader interested can also see a real application example in our R companion book.
	\end{tcolorbox}
	The AHP approach is obviously quite more mathematical that previous  methods. Thus, we can already say that the normalized scores are actually the components of a weight vector of preference, whose sum of component are equal to $100\%$:
	
	therefore with $\sum_i w_i=100\%$.

	Then the second idea is to compare via a matrix for a given factor the choices that are available (obviously to have an approach using the power of Linear Algebra). For example, by taking our tables above and focusing on Factor $A$, we would have:
	\begin{table}[H]
	\centering
		\begin{tabular}{|l|c|c|c|}
	         \cellcolor[HTML]{C0C0C0}\textbf{Factor $A$} & \cellcolor[HTML]{C0C0C0} \textbf{Choice $A$} & \cellcolor[HTML]{C0C0C0}\textbf{Choice $B$ }& \cellcolor[HTML]{C0C0C0}\textbf{ Choice $C$} \\
	\cellcolor[HTML]{C0C0C0}\textbf{Choice $A$} & $1$  & $?$ & $?$ \\ \hline
	\cellcolor[HTML]{C0C0C0}\textbf{Choice $B$} & $?$ & $1$ & $?$ \\ \hline 
	\cellcolor[HTML]{C0C0C0}\textbf{Choice $C$} & $?$ & $?$ & $1$  \\ \hline                  
		\end{tabular}
	\end{table}
	The values (scores) of the diagonal are obvious because each choice can not be preferred to itself than in a relation other than a strict unitary equality.

	Already two questions can be arised at this level:
	\begin{enumerate}
		\item[Q1.] Knowing that the scores (sometimes named "stimuli") must have a form of symmetry, how many scores according to the number of choices $n$ will have to make a "judge".

		\item[Q2.] Which scale of the score ("stimuli") do we have to choose and what relation of symmetry (for example between Choice $B$-Choice $A$ and Choice $A$-Choice $B$ because if we have to choose one, the other must follow automatically)
	\end{enumerate}
	To answer the first question we can construct a simple table (it is therefore simply the number of cases above or respectively below the diagonal of the matrix of preference):
	\begin{table}[H]
		\centering
		\begin{tabular}{|l|c|c|c|c|c|c|c|c|c|}
		 \hline
		 \cellcolor[HTML]{C0C0C0}\textbf{Number of choices} & $1$  & $2$ & $3$ & $4$ & $5$ & $6$ & $7$ & $\ldots$ & $n$ \\ \hline
		\cellcolor[HTML]{C0C0C0}\textbf{Number of comparisons} & $0$ & $1$ & $3$ & $6$ & $10$ & $15$ & $\ldots$ & $21$ & $\dfrac{n(n-1)}{2}$  \\ \hline                  
			\end{tabular}
	\end{table}
	To answer the second question it is better to put ourselves in the shoes of a mathematician, engineer or statistician. First, knowing that equality is logically equal to "$1$" it would be inappropriate to take the set of reals number $\mathbb{R}$ with negative values for the reciprocal knowing that in this case the equality should be zero and not $1$. Taking a matrix of preference whose diagonal would be zeros, would consist to use techniques specific to Data Mining (\SeeChapter{see section Numerical Methods page \pageref{data mining}}) since this is equivalent to have a matrix of distances. But the problem of matrices of distances in the context that interests us here is that the components are symmetrical and in the case of preferences this is a non-sense!

	So let us denote the square matrix of the scores in the following way:
	
	A possible way would be to say that since we have a vector of weights $\vec{w}$, one would have to operate between the matrix of preferences and this vector of weights in one way or another. Now, thinking of the fact that in the domain of Statistics we often find the concept of eigenvector and eigenvalue (\SeeChapter{see section Linear Algebra page \pageref{eigenvector}}) then, the choice of multiplication is quite natural:
	
	Now, having taken the multiplication gives us an idea on the choice of the components of the matrix of preferences to make. Indeed, if we take:
	
 	we have a relation between an eigenvector and an eigenvalue via a matrix (linear application) whose components are preferences. Notice also that the sum of the values of a column $i$ is equal to $1/w_i$. And if we observe well we have in fact:
 	
 	This is equivalent to have a problem with eigenvectors and eigenvalues (\SeeChapter{see section Linear Algebra page \pageref{eigenvector}}):
	
 	Given the following characteristic polynomial (\SeeChapter{see section Linear Algebra page \pageref{characteristic polynomial determinant}}):
	
 	or also the following one (\SeeChapter{see section Linear Algebra page \pageref{determinant of three by three matrix}}):
 	
	Knowing the construction of the preference matrix, we see that:
	
 	and therefore there is only one eigenvalue, it is an integer and is well equal to $\lambda=n=2$. We then say that every square matrix of dimensions $2$ is "consistent matrix\index{consistent matrix}".

	Let us move on to a higher dimension:
	
 	We say in this case that any square matrix of dimensions $3$ is "not necessarily consistent\index{not necessarily consistent matrix}". For this to happen, it would be necessary to have:

	and only in this case will we have a unique eigenvalue $\lambda=n=3$. Notice that if we write this last relation in the form:
	
	Then if:
	
 	the matrix is consistent! We will see later the practical impact of this last equality.

	It is obviously necessary, if necessary, to normalize the eigenvector so that the sum of the preference weights is unitary such that:
	
 	So this leads us to have if we judge our preference on a scale of the following type for our three choices of the Factor $A$:
	
	and by taking care to put the direct preferences in the upper part of the matrix and the inverse in the lower part (otherwise the eigenvector will not be the same and in extenso the weights too!):
	\begin{table}[H]
	\centering
		\begin{tabular}{|l|c|c|c|}
	         \cellcolor[HTML]{C0C0C0}\textbf{Factor $A$} & \cellcolor[HTML]{C0C0C0} \textbf{Choice $A$} & \cellcolor[HTML]{C0C0C0}\textbf{Choice $B$ }& \cellcolor[HTML]{C0C0C0}\textbf{ Choice $C$} \\
	\cellcolor[HTML]{C0C0C0}\textbf{Choice $A$} & $1$  & $\textbf{\textcolor{red}{1/3}}$ & $\textbf{\textcolor{red}{5}}$ \\ \hline
	\cellcolor[HTML]{C0C0C0}\textbf{Choice $B$} & $3$ & $1$ & $\textbf{\textcolor{red}{7}}$ \\ \hline 
	\cellcolor[HTML]{C0C0C0}\textbf{Choice $C$} & $1/5$ & $1/7$ & $1$  \\ \hline                  
		\end{tabular}
	\end{table}
	In the above example the preferences are consistent because if $B$ is preferred to $A$ ($B\succ A$) and $A$ is preferred to $C$ ($A\succ C$) then we should logically have $B\succ C$ what is indeed the case here. If not, we say that: the choices are inconsistent. This consistency is summarized by the relation seen above where in the ideal case we should have:
	
 	So to be consistent we will be interested in the combination involving the choices $C$ and $B$, that is to  say:
	
	So to be totally consistent we should have for $a_{BC}$ the value of $15$ but that is not something we can practically request strictly speaking... However, the deviation to the perfect consistency will generate additional eigenvalues other than the unique eigenvalue which is $n$ when the consistency is perfect. Hence, a possible measure of inconsistency, denoted IC for "\NewTerm{index of inconsistency}\index{index of inconsistency}", is the average of the eigenvalues other than the principal one:
	
 	which in the case of perfect consistency is therefore zero.

	To simplify the calculation of IC, we will use the fact that we have proved in the section of Linear Algebra that the trace of a matrix is equal to the sum of these eigenvalues. So we can write:
	
 	Hence a possible choice to measure the inconsistency:
	
 	But by tradition we take the negative value of this relation as a definition of the "\NewTerm{criterion of consistency}\index{criterion of consistency}":
	
	Thus above, for example, the choice $B$ has been noted $3$ with respect to the choice $A$ (therefore $B$ is preferred to $A$). Determine the weight of the preferences is thus equivalent to a problem of eigenvectors and eigenvalues:
	
	where we must not forget that multiplying a matrix with a scalar is equivalent to multiply the eigenvalues by this same scalar (and not directly the components of the eigenvectors!). This is why were seeking the eigenvectors of the matrix of preferences or of the matrix of preferences normalized by $1/n$ since the eigenvectors will be the same!

	So by calculating by hand (\SeeChapter{see section Linear Algebra page \pageref{linear algebra}}) or by computer, we get:
	
	Therefore:
	
 	But the three eigenvalues are not null because we have:
	
	And we know that this is due to the low non-consistency of our preferences matrix. We then have as a criterion of consistency:
	
 	OK it's close to $0$ so a priori it's a good consistency. But to have a scientific judgment on this value the best still remains to make thousands of simulations of matrices with random judgments consistent or not and calculate their average consistency. Thus, according to the original article, with $50,000$ simulations we obtain for each of the matrices of preference of size $n$ a "\NewTerm{random index of consistency}\index{random index of consistency}" (denoted RI for "random index"):
 	\begin{table}[H]
		\centering
		\begin{tabular}{|l|c|c|c|c|c|c|c|c|c|c|}
		 \hline
		 \cellcolor[HTML]{C0C0C0}\textbf{$n$} & $1$  & $2$ & $3$ & $4$ & $5$ & $6$ & $7$ & $8$ & $9$ & $10$ \\ \hline
		\cellcolor[HTML]{C0C0C0}\textbf{RI} & $-$ & $0$ & $0.52$ & $0.89$ & $1.11$ & $1.25$ & $1.35$ & $1.40$ & $1.45$ & $1.49$ \\ \hline                  
			\end{tabular}
	\end{table}
	\begin{tcolorbox}[title=Remark,colframe=black,arc=10pt]
	In practice it is advised to not to exceed $n=7$ because beyond the human mind has trouble to conceptualize quickly at the opposite of artificial intelligence.
	\end{tcolorbox}
 	So in the case of our example, we have the consistency ratio:
	
 	And of course the smaller is the ratio (in the sense: smaller than the unit!) the better it is. It is common practice to say that if this ratio is less than $0.1$ then the inconsistency is acceptable (and therefore the consistency is good).

	Let us now consider a multi-level case of the AHP.
	\begin{tcolorbox}[colframe=black,colback=white,sharp corners]
	\textbf{{\Large \ding{45}}Example:}\\\\
	Let us consider that we must choose from $4$ cars on the basis of $3$ criteria: 
	\begin{figure}[H]
		\centering
		\includegraphics[scale=1]{img/economy/ahp_initial_decision_tree.jpg}
		\caption{AHP starting decision tree example}
	\end{figure}
	Let us suppose that the matrix of preferences is the following based on three technical criterias:
	\begin{table}[H]
	\centering
		\begin{tabular}{|l|c|c|c|}
	         \cellcolor[HTML]{C0C0C0}\textbf{Criterias} & \cellcolor[HTML]{C0C0C0} \textbf{Style} & \cellcolor[HTML]{C0C0C0}\textbf{Reliability}& \cellcolor[HTML]{C0C0C0}\textbf{Consumption} \\
	\cellcolor[HTML]{C0C0C0}\textbf{Style} & $1$  & $\textbf{\textcolor{red}{1/2}}$ & $3$ \\ \hline
	\cellcolor[HTML]{C0C0C0}\textbf{Reliability} & $2$ & $1$ & $4$ \\ \hline 
	\cellcolor[HTML]{C0C0C0}\textbf{Consumption} & $1/3$ & $1/4$ & $1$  \\ \hline                  
		\end{tabular}
	\end{table}
	\end{tcolorbox}
	\begin{tcolorbox}[colframe=black,colback=white,sharp corners]
	By doing the same calculations as above or with the help of a software, we get:
	\begin{figure}[H]
		\centering
		\includegraphics[scale=1]{img/economy/ahp_initial_decision_tree_weights.jpg}
		\caption{AHP decision tree example with weights}
	\end{figure}
	Now we do the second level, that is to say the matrix of preferences of the cars for the style (comparison by pair of styles therefore ...):
	\begin{table}[H]
	\centering
		\begin{tabular}{|l|c|c|c|c|}
	         \cellcolor[HTML]{C0C0C0}\textbf{Car Style} & \cellcolor[HTML]{C0C0C0} \textbf{Civic} & \cellcolor[HTML]{C0C0C0}\textbf{Saturn}& \cellcolor[HTML]{C0C0C0}\textbf{Escort} & \cellcolor[HTML]{C0C0C0}\textbf{Clio}\\
	\cellcolor[HTML]{C0C0C0}\textbf{Civic} & $1$  & $1/4$ & $4$ & $1/6$ \\ \hline
	\cellcolor[HTML]{C0C0C0}\textbf{Saturn} & $4$  & $1$ & $4$ & $1/4$ \\ \hline 
	\cellcolor[HTML]{C0C0C0}\textbf{Escort} & $1/4$  & $1/4$ & $1$ & $1/5$  \\ \hline                  
	\cellcolor[HTML]{C0C0C0}\textbf{Clio} & $6$  & $4$ & $5$ & $1$  \\ \hline                  
		\end{tabular}
	\end{table}
	And we do the same for the other two criterias to get:
	\begin{figure}[H]
		\centering
		\includegraphics[scale=1]{img/economy/ahp_initial_decision_tree_weights_cars.jpg}
	\end{figure}
	\end{tcolorbox}
	\begin{tcolorbox}[colframe=black,colback=white,sharp corners]
	It is now a question of combining these two levels! For this it seems intuitive to make the following matrix multiplication (weighting):
	
 	So the Clio is the winner of this AHP!
	\end{tcolorbox}
	
	\begin{flushright}
	\begin{tabular}{l c}
	\circled{95} & \pbox{20cm}{\score{4}{5} \\ {\tiny 73.79 votes,  60.00\%}} 
	\end{tabular} 
	\end{flushright}
	
	%to force start on odd page
	\newpage
	\thispagestyle{empty}
	\mbox{}
	\section{Economy}\label{economy}
	
	\lettrine[lines=4]{\color{BrickRed}E}conomics is the study of how humans make decisions in the face of scarcity. These can be individual decisions, family decisions, business decisions or societal decisions. If you look around carefully, you will see that scarcity is a fact of life. Scarcity means that human wants for goods, services and resources exceed what is available. Resources,
such as labor, tools, land, and raw materials are necessary to produce the goods and services we want but they exist in limited supply. Of course, the ultimate scarce resource is time - everyone, rich or poor, has just $24$ hours in the day to try to acquire the goods they want. At any point in time, there is only a finite amount of resources available.

	Think about it this way: In 2015 the labor force in the United States contained over $158.6$ million workers, according to the U.S. Bureau of Labor Statistics. Similarly, the total area of the United States is $3,794,101$ square miles. These are large numbers for such crucial resources, however, they are limited. Because these resources are limited, so are the numbers of goods and services we produce with them. Combine this with the fact that human wants seem to be virtually infinite, and you can see why scarcity is a problem.


	The "Economy", that we assimilate in this book to the domain encompassing the "\NewTerm{theory of goods}", the "\NewTerm{financial mathematics}", the "\NewTerm{financial analysis}", the "\NewTerm{portfolio theories}", aims to reach a settlement, model and determine the origins, dynamics and optimum prices for exchange goods or values "of economic agents" (market actors) in rational competition (...) according to theoretical statistics models (simplified and idealized ...) of markets.

	The mathematical tools related to Economy are responsible for the implementation of concepts that are an integral part to political decisions, financial modeling, forecasting, analysis of investments that without the support of mathematics would be very difficult to propose, evaluate and benchmark. 

	\subsection{Concepts}
	
An economic agent to live will have to satisfy two types of needs that may require to obtain them one or more exchanges:
\begin{enumerate}
	\item  The set $B_p$ of "\NewTerm{primary needs}" or physiological needs (finite and countable) 
	
	\item The set $B_s$ of "\NewTerm{secondary needs}" (which are not vital and not necessarily finite and countable) and are subjectively unique to every individual (and not only human either!)
	
	\item The "\NewTerm{opportunity cost}" is what must be given up to obtain something that is desired.
\end{enumerate}

	\begin{tcolorbox}[title=Remark,colframe=black,arc=10pt]
Secondary needs are very difficult to define and measure, but if we reason in terms of set theory, we can simply say that "secondary needs" $B_s$ are anything that is excluded from all primary needs $B_p$ so that $B_s=\bar{B}_s$.
	\end{tcolorbox}

\textbf{Definitions (\#\mydef):}

\begin{enumerate}
	\item[D1.] We say that need is an "\NewTerm{economic need}" when it concerns a "\NewTerm{rare good}" whose obtention requires several exchanges. They are oppose to "free goods" that are goods available in abundance to all, no work (typically...) and no exchanges being assumed necessary to benefit of them.
	
	The large quantity of goods requires us to propose a classification like the following one:
	
	\begin{enumerate}
		\item[C1.] "\NewTerm{Material goods}" that have a physical reality, palpable and can that can be stored.
		\item[C2.] "\NewTerm{Intermediate goods}" or "\NewTerm{services}" whose production and consumption are considered as simultaneous.

		\item[C3.] "\NewTerm{Virtual goods}" which have a mathematical existence and often limited in time.
	\end{enumerate}
		\begin{tcolorbox}[title=Remark,colframe=black,arc=10pt]
The number of goods in most sophisticated macro models is equal to five: commodities, labour, bonds, money and foreign exchange.
	\end{tcolorbox}
	\item[D2.] A "\NewTerm{market}" is a system consisting of the meet between supply and demand which relates to a particular good.
	
	We are thus led to state the following assumptions:
	\begin{enumerate}
		\item[P1.] The market is supposed to be an isolated and isotropic system.
		\item[P2.] Any active agent is rational and in competition
		\item[P3.] Any agent respects market rules
	\end{enumerate}
	
	\item[D3.] "\NewTerm{Microeconomics}" is the branch of economics that analyses the economic behavior at the level of individual entities such as consumer or firms. Consumers are regarded as labor suppliers and goods seekers. Firms are, in turn, seeking for work and goods suppliers.
	
	\item[D4.] "\NewTerm{Macroeconomics}" is the theoretical approach that studies the economy through the relations between the major economic aggregates: income, investment, consumption, unemployment, inflation, etc.
\end{enumerate}

	\pagebreak
	\subsubsection{Microeconomics}
\textbf{Definition (\#\mydef):} The "\NewTerm{exchange value}" of a specific product specify for each good the amount of other goods that are equivalent to it. Usually, we consider that the "\NewTerm{price}" (or "\NewTerm{currency}") $P$ is the monetary form of exchange value (we will come back later on the concept of money).

	\begin{tcolorbox}[title=Remark,colframe=black,arc=10pt]
The "price" is a parameter which is of interest in the economy. Any tangible good or human resource and a given currency have a price whose (relative) value must be determined either empirically or with more or less complex mathematical statistics models.
	\end{tcolorbox}	
	
There are different types of prices for which we give here a sample in the order of a classic economic process (the definitions are specific to this book!):

\textbf{Definitions (\#\mydef):}
\begin{enumerate}
	\item[D1.] The "\NewTerm{price of manufacturing $P_F$}" of a good at time $t$ is determined by the sum of direct manufacturing costs (not necessarily constant!) $C_D$ at the same time $t$ (salaries, raw materials, machinery, patents, licenses,...):
		
	\item[D2.] The "\NewTerm{factory price}" of a good at time $t$ is the sum of manufacturing costs plus indirect costs $C_I$ at the same time $t$ (taxes, administrative costs, storage costs, advertising, etc.). To be able to model a little bit this theoretically price, we will assume that the market is "lean" or balanced if you prefer (we will see later that it is implicit in the "Say's law"). In other words, the goods are produced directly in line with demand, no storage and no time gap between the availabilit on the market and sales (that is a rough approximation, but we are obliged). Since then:
		
	\item[D3.] The "\NewTerm{net selling price $P_{VN}$}" of a good at time $t$ (or seen by the side of buyer: the "\NewTerm{net buying price}" $P_{AN}$) is factory price at the same time $t$ plus the "\NewTerm{safety margin}" (called also "\NewTerm{gross profit}" or "\NewTerm{profit margin}") $B_b$ of the factory such as:
		
	\begin{tcolorbox}[title=Remark,colframe=black,arc=10pt]
The gross profit will be invested in many areas by the manufacturer (research and development, redistribution to investors, etc.) and the balance is to help protect themselves against various direct market fluctuations, that is to say: wages, taxes, commodity prices, unanticipated events.
	\end{tcolorbox}
We can then consider at least two trivial cases:
	\begin{enumerate}
		\item The gross profit is greater than the sum of general expenses and unanticipated expenses (so there will be a net benefit).
		\item The gross profit is smaller than the general expenses (so there will be a deficit or a net loss).
	\end{enumerate}
	\item[D4.] The "\NewTerm{net income $B_n$}" is given by the portion of the safety margin which was scheduled for a period and that ultimately was not used by unanticipated expenses $C_\Delta$ during this period as:
		
	\begin{tcolorbox}[title=Remark,colframe=black,arc=10pt]
If sales are higher than expected and that the general and unforeseen expenses were addressed to customers, we talk then for this unexpected additional  about "\NewTerm{overactivity bonus}" which obviously increases the expected net income. Otherwise, we speak of "\NewTerm{cost of partial inactivity}", which obviously reduces the net income expected. These are very important concepts within the enterprise project estimate.
	\end{tcolorbox}
		\item[D5.] The "\NewTerm{call price $P_{AP}$}" at time $t$ is the factory price multiplied by a sentimental and artistic factor $F_{SA}$ (fashion, gossip, subjective reasons, etc.) also sometimes named "\NewTerm{B.F.I. index}" (for "Big Fake Index...). This factor can be quantified statistically from the uniqueness of the property, the lifetime of the good, the number of potential buyers and this as nobody operates so as to change the original after the manufacturing. We have therefore:
		
		\item[D6.] The "\NewTerm{gross selling price $P_{VB}$}" at time t or seen by the buyer as the "\NewTerm{gross purchase price}" $P_{AB}$ is the call price augmented by the profit margin of the seller (intermediate between the factory and the purchaser) plus the general sales expenses $F_G$. The seller margin may be included initially in direct expenses but overheads are not deterministic apart in a lean market where there are none overheads and as we have hypothesized a lean market, we thus have:
		
		\begin{tcolorbox}[title=Remark,colframe=black,arc=10pt]
The gross purchase price is also sometimes named "\NewTerm{catalog price}".
		\end{tcolorbox}
		\item[D7.] The "\NewTerm{cost price $P_R$ }" at time $t$ is the gross sales price (or purchase price depending from the point of view...) minus various possible or obligatory deductions $D$ (being a negative value) made by the seller such that:
		
\end{enumerate}

	\pagebreak
	The agents  of goods market exchange sometimes accept a discount on the catalog price. Reductions exist in two main know forms:
	\begin{enumerate}
		\item The "\NewTerm{remission $R$}" which is a price bonification (negative value) given to an agent who buys goods in large quantities $N$ or to a retailer which is charged a branded item sale price imposed by the manufacturer (strategic factor business, promotion, etc.). The discount depends on the time $t$ and the amount $n$ bought or ordered.
		\item The "\NewTerm{prompt payment discount $E$}" which is a deduction at time $t$ granted to the buyer agent for cash payment or prepayment or for payment for an agreed period (we will formally come back on this topic later below during our study of simple interest in actuarial calculation).
	\end{enumerate}
	
	In the most general case we talk at a given time $t$ of  "\NewTerm{exercise price}" (or "\NewTerm{invoice price}") to which the good can be bought or sold as:
		
	\begin{tcolorbox}[title=Remark,colframe=black,arc=10pt]
	All the terms of these relations usually take their values in $\mathbb{R}$...
	\end{tcolorbox}

	The factors to be considered when developing a pricing policy are synthesized without being exhaustive in the following diagram:

	\begin{figure}[H]
		\centering
		\includegraphics[scale=0.8]{img/economy/price_policy.eps}
		\caption{Simplified schematic summary to develop a pricing policy}
	\end{figure}

\textbf{Definitions (\#\mydef):}
\begin{enumerate}
	\item[D1.]	The "\NewTerm{propensity to consume $P_C$}" equation is the part of the income $R$ of an agent that is devoted to consumption of an amount $C$ (primary and secondary consumption):
	
	\item[D2.] The difference between consumption expenditure and the income is defined as "\NewTerm{savings $E$}" while cotisations and revenues on services represent "\NewTerm{social transfers}":
	
	\item[D3.] The "\NewTerm{income elasticity}" is the ratio of the change in consumption on the change in income:
	
	The concept of income elasticity propose to classify goods in supplementary categories:
	\begin{enumerate}
		\item[1.] "\NewTerm{Inferior goods}" are consumer goods whose income elasticity is negative and therefore whose consumption decreases with increasing income as $E_r<0$ (bread, flour, etc.)
		\item[2.] "\NewTerm{Superior goods}" which are luxury consumer goods whose income elasticity is positive and therefore of which the consumption increases with an increase in income as $E_r>0$ (health, leisure, etc.)
		\item[3.] "\NewTerm{Normal goods}": which are goods that are neutral and with a coefficient of elasticity with respect to income is a little different from 0 such that  $E_r\cong 0$.
	\end{enumerate}
	\item[D4.] The "\NewTerm{price elasticity}" is the ratio of the change in the amount of demand for a good on the change in its price and is given by:
	
		\begin{tcolorbox}[title=Remark,colframe=black,arc=10pt]
A demand is named "\NewTerm{price sensitive}" when the percentage change in quantity demanded is greater than the percentage change in price. If we talk about "\NewTerm{rigid price}" demand.
	\end{tcolorbox}
	\item[D5.] An "\NewTerm{investment $I$}" is the operation performed by an economic agent whose objective is to obtain produced goods in exchange.
	\item[D6.] A "\NewTerm{transaction $T$}" is the exchange of a quantity of goods at a specified price between a "seller" and "buyer". It concludes on the market whose form is determined by the number of economic agents who are involved which determines the "\NewTerm{concurrence}".
	
	The table below shows the different types of contracts:
	\begin{table}[H]
	\begin{center}
		\begin{tabular}{|c|c|c|c|}
			\hline
			{} & \multicolumn{3}{|c|}{\cellcolor{black!30}\textbf{Suppliers}} \\
			\hline
			\cellcolor{black!30}\textbf{Seekers} & \cellcolor{black!30}multitude & \cellcolor{black!30}some & \cellcolor{black!30}only one \\
			\hline
			\cellcolor{black!30}multitude & perfect competition & oligopoly\label{oligopoly} & monopoly \\
			\hline
			\cellcolor{black!30}some & oligopsony & bilateral oligopoly & upset monopoly \\
			\hline
			\cellcolor{black!30}only one & monopsony & upset monopsony & bilateral monopoly \\
			\hline
		\end{tabular}
		\caption{Different forms of market competition}
	\end{center}
	\end{table}
	When the actors of an ogilopoly work together to fix prices, we speak then of "\NewTerm{cartel}\index{cartel}".
	\begin{tcolorbox}[colframe=black,colback=white,sharp corners]
	\textbf{{\Large \ding{45}}Examples:}\\\\
	E1. Lysine, a \$$600$ million-a-year industry, is an amino acid used by farmers as a feed additive to ensure the proper growth of swine and poultry. The primary U.S. producer of lysine is Archer Daniels Midland (ADM), but several other large European and Japanese firms are also in this market. For a time in the first half of the 1990s, the world’s major lysine producers met together in hotel conference rooms and decided exactly how much each firm would sell and what it would charge. The U.S. Federal Bureau of Investigation (FBI), however, had learned of the cartel and placed wire taps on a number of their phone calls and meetings.	From FBI surveillance tapes, following is a comment that Terry Wilson, president of the corn processing division at ADM, made to the other lysine producers at a 1994 meeting in Mona, Hawaii:\\
	
	\textit{I wanna go back and I wanna say something very simple. If we're going to trust each other, okay, and if I'm assured that I'm gonna get 67,000 tons by the year's end, we're gonna sell it at the prices we agreed to... The only thing we need to talk about there because we are gonna get manipulated by these [expletive] buyers-they can be smarter than us if we let them be smarter. ...They [the customers] are not your friend. They are not my friend. And we gotta have 'em, but they are not my friends. You are my friend. I wanna be closer to you than I am to any customer. Cause you can make us ... money. ... And all I wanna tell you again is let's - let's put the prices on the board. Let's all agree that's what we're gonna do and then walk out of here and do it.}\\

	The price of lysine doubled while the cartel was in effect. Confronted by the FBI tapes, Archer Daniels Midland pled guilty in 1996 and paid a fine of \$$100$ million. A number of top executives, both at ADM and other firms, later paid fines of up to \$$350,000$ and were sentenced to $24$-$30$ months in prison. \\
	
	In another one of the FBI recordings, the president of ADM told  stated the slogan of its company this way\textit{: Our competitors are our friends. Our customers are the enemy}.
	\end{tcolorbox}
	
	\begin{tcolorbox}[colframe=black,colback=white,sharp corners]
	E2. French antitrust authorities in 2014 issued two fines of a cumulative amount of some $951$ million euros against $13$ of the main manufacturers of the maintenance sector, hygiene and beauty for having coordinated their marketing policy with the mass retailers and to have agreed on the price increases between 2003 and 2006. The first fine of concerns Colgate-Palmolive, Henkel, Unilever, Procter \& Gamble, the second one concerns the same + Gillette and L'Oréal.\\

	This case was revealed $3$ years after another important decision of the Authority on the laundry sector, already containing some of the same companies. The four major manufacturers of laundry detergents - Unilever, Procter \& Gamble, Colgate-Palmolive and Henkel - secretly agreed during $6$ years on their prices and promotions in France, resulting in a $4\%$ surcharge of $6\%$ for consumers. These manufacturers were then sentenced to pay only $361$ million euros in fines...
	\end{tcolorbox}
	A competition is qualified of "\NewTerm{pure competition}" (P.P.C.: pure and perfect competition) or "\NewTerm{perfect competition}" if it meets the next five following hypothesis:
	\begin{enumerate}
		\item[H1.] \NewTerm{Atomicity}: Buyers and sellers are numerous to the point that no one alone can influence the prices.
		
		\item[H2.] \NewTerm{Homogeneity}: The goods traded are identical and interchangeable with each other. They can satisfy the same need.
		
		\item[H3.] \NewTerm{Free entry}: There is no barrier to entry and exit of new economic agents.
		
		\item[H4.] \NewTerm{Free movement}: Economic agents can move freely (from a market to the other one).
		
		\item[H5.] \NewTerm{Perfect information}: Everyone knows at the same time and for free all quantities supplied and demanded by all the agents at different prices (\SeeChapter{see section Game and Decision Theory page \pageref{perfect information game}}).
	\end{enumerate}
	\item[D7.] The "\NewTerm{intermediate management balances}" are parts of the overall result of market activity period that are significant to the financial analyst. There are multiple definitions which arise from elementary algebraic operations on concepts defined above as (non exhaustive list!):
	\begin{itemize}
		\item The "\NewTerm{gross margin}" is the difference between the proceeds of sales of goods and the purchase cost of goods sold (the sales margin is specific to trading activities, that is to say to companies with a distribution activity ).
		\item The "\NewTerm{production of the fiscal period}" which is the sum of production sold, stored and immobilized (production of exercise is specific to production activities, that is to say companies with an industrial activity).
		\item The "\NewTerm{gross margin}" is the difference between the product withdrawn from sale for the fiscal period and purchases of raw materials consumed.
		\item The "\NewTerm{turnover}", which is the sum of revenue from sales of goods and sales of goods and services.
		\item The "\NewTerm{added value}" which is defined as the difference between production for the fiscal period and the intermediate consumption by the economic agents (the manager considers it created wealth resulting from the real business of the company and the added value also of national importance because it is an aggregate).
		\item The "\NewTerm{gross operating surplus G.O.S.}" is the result of the current activity of the company and is defined as:
		
	\end{itemize}
\end{enumerate}

\pagebreak
\paragraph{Average \& Marginal Cost/Revenue}\mbox{}\\\\
Let us introduce the subject directly by an example:

Suppose an amateur cook (and economist) invites his friends to his table and proposes to do their a tomato salad. It evaluates the work he has to do this job and figure it in monetary value. It considers that a minute of his time spent in the preparation of the salad corresponds to an expense of 1.-.

So the data are:
	\begin{enumerate}
		\item Each minute of work value is 1.-
		\item Each tomato is 1.-/unit
		\item Prepare the salad takes 15 minutes
	\end{enumerate}
If each of his friends is satisfied with a single tomato, prepare dinner for five friends (the cook does not eat) will cost in total:
	
	Mathematically, the "\NewTerm{total cost function}" will looks like following if we write $q$ as the number of friends (in practice it is often a strictly increasing polynomial function):
	
	\textbf{Definition (\#\mydef):}  The "\NewTerm{average cost}" or "\NewTerm{average cost function}" for each guest is of 20.- divided by $5$ thus 4.-. This corresponds to:
	
	which is not necessarily an increasing function!
	
	\begin{theorem}
	We can see that as $q$ tends to infinity, under the assumption of constant production effort, the average cost $C_m$ tends to the unit cost $C_U$. That is so to say:
	
	\end{theorem}
	\begin{dem}
	 Indeed, if we denote $C_F$ the assumed fixed cost and $C_U$ the unit cost, we have:
	
	and when $q\rightarrow+\infty$ then as $C_Uq\gg C_F$ we can write:
	
	\begin{flushright}
		$\square$  Q.E.D.
	\end{flushright}
	\end{dem}
	
	\begin{tcolorbox}[title=Remark,colframe=black,arc=10pt]
	According to the books (and authors), the average cost is denoted by a lowercase $m$ or uppercase $M$ and the marginal cost (see further below) too. 
	\end{tcolorbox}

	If our cook invites a sixth friends, the total cost will be 21.-. In fact the preparation time will remain, at least we assume ... constant! In this case, the marginal cost of the sixth guest is:
	
while the average cost for all guests is now:
	
	We notice in this situation that the average cost goes down of 0.5.- indefinitely as the number of guests increases. This quantity of 0.5.- is the marginal revenue!

	This example illustrates the return scale and shows that we often have an incentive to increase production to reduce the average cost of production (but this does not mean that our amateur cook will appreciate the reasoning when reaching its physical limit...).

Moreover, it this not however a general rule! While the bowl of our economist can contain only 6 tomatoes, the 7th guest will force him to prepare a second bowl. In this case, the marginal variation will be greater than the previous one.

We must also be aware that the total cost function is in fact rarely a continuous function (because you have to buy machinery or engage staff in stepwise quantities) and even more it is the market that dictates the quantities to a factory and not the inverse.

\textbf{Definition (\#\mydef):} Mathematically, the "\NewTerm{marginal cost}\index{marginal cost}" is defined by the variation of the total cost function $C_T(q)$, relative to the quantity produced $q$:
	
	or if the total cost function is differentiable (anyway in economics, even if it is not continuous we makes abstraction of this fact...):
	
	The marginal cost corresponds thus to the cost of producing one more or less unit (in the example above it equal to 1.-). In practice, we are more interested in the cost of an additional series.

	In addition, the reader will notice that as the total cost is a strictly increasing function, the marginal cost is also a constant or increasing function.

	\begin{tcolorbox}[title=Remark,colframe=black,arc=10pt]
	If the marginal cost increases as the quantity $q$ increases, we say that the returns are diminishing. In contrast, they are increasing if the marginal cost is decreasing as the amount increases.
	\end{tcolorbox}
	The marginal cost corresponds thus to the cost of producing one more or less unit. In practice, we are more interested in the cost of an additional series.

	In addition, the reader will notice that as the total cost is a strictly increasing function, the marginal cost is also a constant or increasing function.

\begin{theorem}
Let us show now that if the average cost goes through an extremal value, the marginal cost is equal to it at that point. We name this particular situation a "technical optimum".
\end{theorem}

Recall previously that if a continuous and differentiable function $f(x)$ has a minimum (or maximum), its derivative at that point is equal to zero (\SeeChapter{see section Differential and Integral Calculus page \pageref{turning point}}).

\begin{dem}
Apply this property to the average cost function. But recall that previously (\SeeChapter{see section Differential and Integral Calculus page \pageref{usual derivatives}}):
	
So we have if the derivative of the average cost is zero:
	
So this is the derivative of the ratio of two functions. therefore:
	
Hence we deduce:
	
Thus:
	
	\begin{flushright}
		$\square$  Q.E.D.
	\end{flushright}
\end{dem}
This result tell us that in this situation, the marginal cost is equal to average cost (we find the definition of each cost on the to left or on the right respectively of equality).

In other words, where the average cost reaches an optimum (minimum in our case), then the marginal cost equals average cost.

	\textbf{Definition (\#\mydef):} Mathematically, the "\NewTerm{marginal revenue}\index{marginal revenue}" is defined by the additional margin relative to the quantity produced $q$ (it is the additional revenue that will be generated by increasing product sales by one unit):
	
	\begin{theorem}
	As $q\rightarrow +\infty$ the marginal revenue $C_R$ tends to zero. In other words, more we produce a given product the smaller is the margin variation.
	\end{theorem}
	\begin{dem}
	As we have:
	
	Then when $q\rightarrow +\infty$ we have:
	
	and therefore:
	
	\begin{flushright}
		$\square$  Q.E.D.
	\end{flushright}
	\end{dem}
	For our above example the tabulated value gives in a spreadsheet software:
	\begin{figure}[H]
		\centering
		\includegraphics[scale=0.8]{img/economy/total_cost_marginal_cost_marginal_revenue.jpg}
	\end{figure}
	

	\begin{tcolorbox}[colframe=black,colback=white,sharp corners]
\textbf{{\Large \ding{45}}Example:}\\\\
A factory has calculated that the cost of producing q units of a product can be modeled by the following function within the yield capacity limits of a machine and manpower it has at its disposal:
	
	Let us calculate the cost, the average cost and marginal cost of the production of $1,000$, $2,000$ and $3,000$ units. We then have the following information:
	
Then we have:

	\begin{table}[H]
	\begin{center}
		\definecolor{gris}{gray}{0.85}
			\begin{tabular}{|c|c|c|c|}
				\hline
				\multicolumn{1}{c}{\cellcolor{black!30}\textbf{Quantity}} & 
  \multicolumn{1}{c}{\cellcolor{black!30}\textbf{Total Cost}} & 
  \multicolumn{1}{c}{\cellcolor{black!30}\textbf{Average Cost}} & 
  \multicolumn{1}{c}{\cellcolor{black!30}\textbf{Marginal Cost}} \\ \hline
				 1,000 & 5,600 & 5.60 & 4.00 \\ \hline
				 2,000 & 10,600 & 5.30 & 6.00 \\ \hline
				 3,000 & 17,600 & 5.87 & 8.00 \\ \hline
		\end{tabular}
	\end{center}
	\caption[]{Various costs}
	\end{table}	
And for the average cost to be as low as possible, it is necessary that the marginal cost is equal to average cost:
	
We conclude:
	
By injecting this value in the total cost function, we deduce that the minimum average cost is $5.22.-$.
	\end{tcolorbox}

In reality, we make Monte Carlo simulations that gives not only the possibility to work in an uncertain and quantities, but that an also take into account non-continuous functions! For more information on Monte Carlo modeling, the reader can refer to section of Numerical Methods.

	\pagebreak
	\subsubsection{Macroeconomics}
"\NewTerm{Aggregates}" are synthetic variables developed by nations for their national accounting and measuring the result of their entire economy. Some of the main aggregates are defined by:

\textbf{Definitions (\#\mydef):}
	\begin{enumerate}
		\item[D1.] The "\NewTerm{gross national product (G.N.P.)}" whose purpose is to measure domestic production (considered as isolated), that is to say all the monetary values of goods and services produced in a given period (the term "Gross" indicates that the value of G.N.P. is not deducted from different existing taxes on production).
		
		\item[D2.] The "\NewTerm{gross domestic product (G.D.P.)}" whose role is to measure national production (such as G.N.P.) and take into account the income of the rest of the world. In other words, G.D.P. is the G.N.P. which we sum the capital coming from outside and which is subtracted from capital going outside.
		
		\item[D3.] The "\NewTerm{gross fixed capital formation (G.F.C.F.)}" measures the value of acquisitions of new or existing fixed assets by the business sector, governments and "pure" households (excluding their unincorporated enterprises) less disposals of fixed assets and it is a component of the expenditure on gross domestic product (G.D.P.), and thus shows something about how much of the new value added in the economy is invested rather than consumed.
		
		\item[D4.] The "\NewTerm{national income (N.I.)}" whose role is to measure all the incomes received by economic agents.
		
		\item[D5.] The "\NewTerm{Consumption}"  whose role is to represent the value of goods and services used for the direct needs of economic agents.
		
		\item[D6.] The "\NewTerm{consumer price index (C.P.I.)}" measures changes in prices of goods and services representative of the consumption of private households. It indicates how much consumers have to increase or decrease their expenses to maintain the same volume of consumption.
		
		\item[D7.] The "\NewTerm{inflation I}"  is defined as the increase in prices or the purchasing power of money the most common way to calculate the inflation rate is by recording the prices of goods and services over the years, take a base year and then determine the percentage rate changes of those prices over the years. For example with C.P.I. we get:
		
		
		\item[D8.] The "\NewTerm{Purchasing power standard (P.P.S.)"} is an artificial common reference currency unit used in the European Union which eliminates the differences of price levels between countries. So, a P.P.S. allows to buy the same volume of goods and services in all the countries. This unit allows significant comparisons in volume of economic indicators between countries.
		
		\item[D9.] The "\NewTerm{Purchasing power parity (P.P.P.)"} is the exchange rate that equalizes the prices of internationally traded goods across countries. A group of economists at the International Comparison Program, rub by the World Bank, have calculated the PPP exchange rate for all countries, based on detailed studies of the prices and quantities of internationally tradable goods. The PPP is often used in any international comparison involving prices or amounts of money!!!
		
		\item[D10.] The "\NewTerm{added value (A.V.)}" of a company whose role is to represent the difference between the value of goods and services produced by it with the value of goods and services used to produce these goods and services.
	\end{enumerate}
and so on...

	\begin{tcolorbox}[title=Remark,colframe=black,arc=10pt]
 	Never forget that when you read macro-economics indicators in papers that first there is a margin error on them (even if it is not indicated) and second that these indicator are given with a time lag of months or years sometimes. So when you see a good macroeconomic indicator this year it is in fact the macroeconomics wealth of previous year and furthermore because sells of major Fortune 500 are done on customers orders of prior-previous year (typical for machines and some food stuff) it is in fact with two years late!
	\end{tcolorbox}

\paragraph{Cobb-Douglas Model}\mbox{}\\\\\
In 1928, Charles Cobb and Paul Douglas published a study in which appeared the modeling of the growth of the US economy between 1899 and 1922. They had adopted a simplified view of the economy in that the quantity produced of goods is based that the amount of work done and the amount of capital invested.

Although many other factors affect economic performance, their model has shown to be remarkably accurate. The function they used to model production was of the form:
	
where $P$ is the total production (the monetary value of all goods produced in one year), $L$ the amount of work (the total number of hours worked in a year) and $K$ the amount of investment (the monetary value of all machinery, equipment and buildings). The factor $b$ is a "collector" constant that represents technology, efficiency, and other things not accounted for by $K$ and $L$.

Although this model has been applied primarily to economy, we can found it in many scientific articles in biology or human resources. It seemed appropriate to show in detail (since it is the purpose of this book to see things in details!) where does this strange function comes from because the approach is relatively great.

First, if we have a function of two variables $P(L,K)$, the partial derivative $\dfrac{\partial P}{\partial L}$ indicates the rate of change in production with respect to the amount of labor only. This is what economists name the "\NewTerm{marginal production in relation to labor}" or "\NewTerm{marginal productivity of labor}". Similarly, the partial derivative $\dfrac{\partial P}{\partial K}$ indicates the rate of change in production relative to capital and is named the "\NewTerm{marginal productivity of capital}".

If we increase the number of machines $K$, will we be able to produce more cookies? Of course! What if we increase the amount of labor $L$? Same thing. We can observe this mathematically as well:
	
In these terms, the hypothesis (assumptions) of Cobb and Douglas (to confront to historical measures and adapt to these measures in individual cases) are as follows:

	\begin{enumerate}
		\item[H1.] Without labor $L$ or without capital $K$ no production.
		\item[H2.] The marginal productivity of labor is proportional to the amount produced $P$ per unit of labor $L$.
		\item[H3.] The marginal productivity of capital is proportional to the amount produced $P$ per unit of capital $K$
	\end{enumerate}
Given that the production per work unit is the $\dfrac{P}{L}$, hypothesis 2 can be written as:
	
for a given constant $\alpha$. 

If the case where the invested capital K is constant ($K=K_0$), the partial differential equation becomes an ordinary differential equation:
	
This gives us:
	
So solving this differential equation gives:
	
Therefore:
	
because as we fixed K it is obvious that the constant depends on $K_0$.

By doing exactly the same for the third hypothesis:
	
Therefore:
	
So we have at the end:
	
and identifying term by term, we get immediately:
	
Let us observe the implications of this last relation if capital and labor are both multiplied by a constant factor $m$:
	
If we impose:
		
then the production is multiplied by the same factor $m$. And we are in a proportionality assumption of production which seemed was consistent Cobb and Douglas assumption. since then:
	
The economic data used by Cobb and Douglas are those of the following table published by the U.S.A. Government:

	\begin{table}[H]
	\begin{center}
		\definecolor{gris}{gray}{0.85}
			\begin{tabular}{|c|c|c|c|c|}
				\hline
				\multicolumn{1}{c}{\cellcolor{black!30}\textbf{Year}} & 
\multicolumn{1}{c}{\cellcolor{black!30}\textbf{$P$}} & \multicolumn{1}{c}{\cellcolor{black!30}\textbf{$L$}} & \multicolumn{1}{c}{\cellcolor{black!30}\textbf{$K$}} \\ \hline
		1899 & $100$ & $100$ & $100$ \\ \hline
		1900 & $101$ & $105$ & $107$ \\ \hline
		1901 & $112$ & $110$ & $114$ \\ \hline
		1902 & $122$ & $117$ & $122$ \\ \hline
		1903 & $124$ & $122$ & $131$ \\ \hline
		1904 & $122$ & $121$ & $138$ \\ \hline
		1905 & $143$ & $125$ & $149$ \\ \hline
		1906 & $152$ & $134$ & $163$ \\ \hline
		1907 & $151$ & $140$ & $176$ \\ \hline
		1908 & $126$ & $123$ & $176$ \\ \hline
		1909 & $155$ & $143$ & $198$ \\ \hline
		1910 & $159$ & $147$ & $208$ \\ \hline
		1911 & $153$ & $148$ & $216$ \\ \hline
		1912 & $177$ & $155$ & $226$ \\ \hline
		1913 & $184$ & $156$ & $236$ \\ \hline
		1914 & $169$ & $152$ & $244$ \\ \hline
		1915 & $189$ & $156$ & $236$ \\ \hline
		1916 & $225$ & $183$ & $298$ \\ \hline
		1917 & $227$ & $198$ & $335$ \\ \hline
		1918 & $223$ & $201$ & $366$ \\ \hline
		1919 & $218$ & $196$ & $387$ \\ \hline
		1920 & $231$ & $194$ & $407$ \\ \hline
		1921 & $179$ & $146$ & $417$ \\ \hline
		1922 & $240$ & $161$ & $431$ \\ \hline
	\end{tabular}
	\end{center}
	\caption[]{U.S. data in \% used by Cobb and Douglas}
	\end{table}	

They deliberately took 1899 as a base, that is to say, they attributed the level $100$ to each factor and expressed the values of other years as a percentage of that year.

To determine the Cobb-Douglas coefficients, we just have to make a linear regression by the least squares method (\SeeChapter{see section Numerical Methods page \pageref{least squares method}}). If we take the logarithm:
	
and after rearranging, still using the properties of logarithms (\SeeChapter{see section Functional Analysis page \pageref{logarithms}}):
	
We put now to simplify:
	
Therefore we just end up with a linear function:
	
We then just adapt the table to inject it into a spreadsheet or statistical software to obtain (\SeeChapter{see section Numerical Methods page \pageref{least squares method}}):
	
Therefore:
	
By testing compared to the data table we get:
	
	So we see that the model... is a model... with some errors as every model. For example a typical "evolutionary" error (that a lot of economists still ignore and also politicians) of this model can illustrated by the following assumption: 

	If we keep increasing the number of machines $K$ and hold labor
$L$ constant, economists say that we will produce less and less cookies per machine added. Same goes for labor if you hold capital constant. Why? Diminishing marginal returns! If we have too many machines but not enough workers, the machines will be idle. If we have too many workers for the machines, they overcrowd. Simple as that. But economists and politicians forget that now machine are "intelligent" and can manage and learn by themselves!

	How can we prove mathematically this previous fact (that marginal returns diminish when increasing of the variable)?
	
	Mathematically, we say that $Y$ is increasing in $K$, but at a declining rate. $Y$ is also increasing $L$, but at a declining rate. Indeed here is a plot of this:
	\begin{figure}[H]
		\centering
		\includegraphics[scale=0.7]{img/economy/cobb_douglas_declining_rate.jpg}
		\caption{$Y$ Increasing in $K$ and $L$ at a declining rate}
	\end{figure}
	This require us to fin the second derivatives:
	
	as $(\alpha-1)$ is negative, since we stipulated that $0<\alpha<1$. 
	
	The assumptions H2 and H3 implies what economists name the "\NewTerm{constant return to scale}". In other words, we are taking our original $K$ and $L$, and multiplying them both by the same proportion $c^{te}$. Then we investigate what happens to output.
	
	So originally, we have:
	
	After increases in input by a constant $c^{te}$, we have:
	
	We have just shown that if we increase both $K$ and $L$ by a same constant, then $Y$ gets increased by the same factor. In fact we know it is much more interesting to invest in machines rather than in humans. So this model and through the assumptions H2 and H3 was valid only at the time computers (and defacto machines) were not able to do the work of humans.

	\pagebreak
	\subsection{Monetary Model}
	
It is funny for an engineer to see at least once in his life how an idealistic macroeconomic Monetary Model can be build. We can see that the whole is just a summation of obvious assumptions that has for only purpose to try to have the most complicated mathematical formula at the end to makes it look like a pure science...

To build a macroeconomic monetary model in a pure and perfect competition market (remember above the five assumptions of such a market) we will assume that the "\NewTerm{monetary utility}" can be defined a priori by three properties:
		\begin{enumerate}
			\item[P1.] It is a counting unit
			\item[P2.] It is a medium of payment (via trade)
			\item[P3.] It is a reserve value
			\item[P4.] It is in relative value a conservative quantity
		\end{enumerate}
This description approach is however insufficient to mathematical analysis: we need a comprehensive explanatory system because here we only formalized facts. We must therefore establish the link between money and monetary theory.

Apart from the value that represents currency, it derives its utility of the property it provides exchanges of goods. This is what we name the "\NewTerm{derived utility}".

Let us note $O_u$ the available money on the market. It therefore depends on the existing  total quantity of money $Q_u$ minus the "cash" $e$ kept by economic agents (who exchanged goods against money). We can then write the following relation named "\NewTerm{money supply according to Walras}":
	
The cash $e$ is also in some way that of households and is a real demand for goods, which can be expressed necessarily in monetary form.

	\begin{tcolorbox}[title=Remark,colframe=black,arc=10pt]
If all people really knew how modern money creation process works (the one used since the majority of states around the world allow banks to create "debt money" - that is to say, to create the money from a debt repayment promise) the system would probably not last long. Moreover, the contemporary system is relatively fragile and vicious in reality as we will see later (it will probably crash down soon or later).
	\end{tcolorbox}
	
	Sales agents, when they sale their goods, a priori desire a given sum of monetary value against the sale of these goods, value named "\NewTerm{desired monetary amount}" and denoted by $D_u$. The problem is that $D_u$ can be expressed as working hours, gold, air, animals, vegetables, etc. This is why it is important to convert this value into common artificial price measure of the monetary value.
	
	We express this desired monetary amount in "cash" and for this we introduce a price to monetary value (because there exist different way to pay a cash value of a good having a given desired monetary amount). The desired monetary amount is then written over all market balances. The cash serves to express the relative prices for the overall balance. There is a cash desired by the agents for the realization of general equilibrium. These are actually real goods in monetary form:
	
	where $p_u$ is the "\NewTerm{monetary cash price}" (factor variable over time and that brings in a market that is not "flow" to speculate). In a flow-market $p_u$ will always assumed to be equal to unity. We can then write for different currencies in a global heterogeneous market:
	
	where $p_i$ is the \underline{relative} monetary cash price of currency $i$ relatively to $p_u$
	\begin{tcolorbox}[title=Remarks,colframe=black,arc=10pt]
		\textbf{R1.} In an isotropic single currency market this relations would not need to be written.\\
		
		\textbf{R2.} "\NewTerm{Parity}" is the term used when we look for equivalence of foreign currencies exchange rate and it is dependent (among others) of time and it is important to consider its variations in the context of goods market where the currency is not unique and payments not immediate.
	\end{tcolorbox}
	The desired monetary amount by agents can then be define using the relation:
	
	Returning back to $O_u$ but this time in the point of view of the companies (firms). They need money to make payments and to operate (wages, investments, etc.) and the desired monetary amount $D_u^{'}$ of all these companies at a given moment in time is in an ideal case necessarily equal to the total money available on the market such that:
	
	since companies (firms) sell goods on the quantity of money $Q_u$ of agents of the economic market minus the cash margins of these agent $e$.
	
	The last relation implies that:
		\begin{enumerate}
			\item The aggregate demand necessarily creates an equal quantity of aggregate production ("\NewTerm{Say Law}").
			\item The selling price of goods tends to be equal to their cost price (this is for simplification purposes and also a condition for what will follow!).
		\end{enumerate}
	
	\begin{tcolorbox}[title=Remark,colframe=black,arc=10pt]
		This last relation also means that the whole offer is satisfied only by the demand from the economic agents and the aforementioned amount of monetary value consists only of goods outside the companies (producers/firms/manufactures) and that they therefore have no stocks!
	\end{tcolorbox}
	
	This also corresponds to a certain quantity of goods since the idea is to propose goods to get money (seen of the point of view of companies/firms/manufactures!). Therefore we can write:
	
	But as the goods already in possession by the economic agents will also have to be renewed, the companies (firms) have finally as potential total amount of money available in the market:
	
	The sum in brackets corresponds to the total money available in the market in the form of household assets and potential cash in the restriction of goods with a common currency cash price $p_u$. This is named the "\NewTerm{money supply}\index{money supply}". It is restrictive as a model but sufficient in the context of determining the price of a given type good.
	
	We write then by definition for the money supply according to Walras:
	
	This last relation implies that economic agents may not collectively choose to increase the amount of money $e$ they hold...
	
	In view of the foregoing, the reader will have noticed that this model considers that money is neutral in the sense that the total amount of money in circulation exerts no influence on the relative prices of products from each to others, nor on the level of supply and demand for products. The currency is not desired for itself...
	
	\subsubsection{Monetary base conservation}
	In economics, the "\NewTerm{monetary base}\index{monetary base}" in a country is defined as the portion of a commercial bank's reserves that consist of the commercial bank's accounts with its central bank plus the total currency circulating in the public, plus the currency, also known as vault cash, that is physically held in the bank's vault.
	\begin{tcolorbox}[title=Remark,colframe=black,arc=10pt]
	The monetary base should not be confused with the "money supply" which consists of the total currency circulating in the public plus the non-bank deposits with commercial banks.
	\end{tcolorbox}
	The previous relations that we have obtained assume therefore that the money supply is constant. What is interesting to know then is that when a central bank print out money at time $t$, since it keep this money inside the bank and don't tell anyone it print it out, the relative value of money will remain constant such that:
	
	But if the central bank put the new quantity of money on the market, then by conservation of the value $p_i$ should be re-evaluated. That means that all the prices, wages, etc. in the country where the bank has printed out the money should be multiplied by a factor:
	
	So there is no "law of conservation of money". or of "law of monetary base conservation" on the long time term at work but.... rather a "law of relative money price value to the monetary base"!
	
	However it is funny to notice that some Governments, hands in the hands with the Central bank ("\NewTerm{monetary policy}\index{monetary policy}") don't such corrections depending on the situations and therefore violates the monetary base conservation principle. For example in Switzerland, the monetary base in 2012 had increased by a factor $6$, from $100$ billion to $600$ billions (we speak the of "\NewTerm{money dilution}\index{money dilution}"), without saying to the banks, shops and individuals that everything should have therefore been multiplied by a factor $6$ (the amount in the bank accounts, the wages, GDP, and so on...) since the value of the Swiss money therefore lost a value of a factor $6$ in an international point of view (deflation). 
	\begin{figure}[H]
		\centering
		\includegraphics[scale=0.64]{img/economy/swiss_national_bank_adjustement.jpg}
		\caption[Swiss National Banks reserve]{Swiss National Banks reserve (source: Bloomberg)}
	\end{figure}
	\pagebreak
	But in this case the Swiss Central Bank as argue during a conference (using random bullshit language...) that this was 
	\begin{enumerate}
		\item Because the Swiss money was over-evaluated of the point of view of the international market.
		
		\item The wages, and individual bank account are not included in the measurement of the monetary base (this last statement being given without surprise with no mathematical proof...)
	\end{enumerate}
	It is common in Governments to split the monetary base in three "\NewTerm{monetary aggregates}\index{monetary aggregates}" (some split it in four):
	\begin{enumerate}
		\item[M1.] Money (banknotes and coins)

		\item[M2.] Household savings (private saving accounts, deposits $<2$ years)

		\item[M3.] Corporates savings (debts $<2$ years, bonds, shares)
	\end{enumerate}
	And we have (the $M_1$ and $M_2$ as previously mentioned are sometimes ignored and the argument - when the mathematical proof shows the opposite... - is that they can be neglected):
	
	Here are some plots showing a monetary policy for a country (still Switzerland) for $M_1$, $M_2$ and $M_3$:
	\begin{figure}[H]
		\centering
		\includegraphics[scale=1]{img/economy/monetary_aggregates_absolute_values.jpg}
		\caption[Evolution of monetary base in Switzerland]{Evolution of monetary base in Switzerland (source: \url{www.les-crises.fr}, author: Olivier Berruyer)}
	\end{figure}
	and in relative value:
	\begin{figure}[H]
		\centering
		\includegraphics[scale=1]{img/economy/monetary_aggregates_relative_values.jpg}
		\caption[Relative evolution of monetary base in Switzerland]{Relative evolution of monetary base in Switzerland (source: \url{www.les-crises.fr}, author: Olivier Berruyer)}
	\end{figure}
	
	\subsubsection{Walras' law}
	Walras' Law is a principle in general equilibrium theory asserting that budget constraints imply that the values of excess market demands (or, conversely, excess market supplies) must sum to zero in a pure and perfect competition market where economic agents do not increase the amount of money they hold and where the value of money is equal...
	
	Moving from the particular to the general, the money value of what the $j$-th agent transactor plans to purchase (Demand) can be written symbolically as:
	
	where $p_1, ... , p_n$ are the prices of the $n$ goods, and $D_{1j}, ... , D_{nj}$ are the quantities of those goods that the $j$-th agent plans to purchase.
	
	Similarly, the money value of what the $j$'th individual plans to Sell can be written symbolically as:
	
	where $p_1, ... , p_n$ are the prices of the $n$ goods, and $S_{1j}, ... , S_{nj}$ are the quantities of those goods that the $j$'th agent plans to sell.
	
	Now let us adopt Say's Law that is to say that the money value of all the goods that the $j$-th individual plans to buy must always be equal to the money value of all the goods that individual plans to sell, we may write:
	
	This condition is obviously written as an identity (remember that an identity is a proposition which is true by definition and is usually written with a $\equiv$ rather than an $=$ sign in it), since we assume that
no individual transactor in our model will be so misguided as to suppose that he or she can acquire something for nothing. (Tacitly, our model assumes that transactors are not thieves, extortioners, embezzlers - or philanthropists!).

	\begin{tcolorbox}[title=Remark,colframe=black,arc=10pt]
		The changes in the currency prices do not affect the actual balance if and only if all these relative prices change in the same proportion (and thus excess demand should no increase or decrease).
	\end{tcolorbox}

Granted that each individual's planned market transactions in goods satisfy condition the previous identity, it follows as a matter of simple arithmetic that the money value of the quantities demanded by all individuals must be equal to the money value of the quantities offered for sale by all
individuals. Therefore, summing previous identity over all individuals (as supposed to be independent!) we obtain:
	
	when we assume that there are $m$ individual transactors.
	
	Factoring out the price variables from this expression yields:
	
	
	However, the expression in parentheses on the left-hand side is simply the total market demand for the $i$-th good, since it is the sum of the individual transactors' demands for that good. We will write this total market demand as $D_i$. Similarly, the expression in parentheses on the right-hand side is simply the total market supply of the $i$'th good, since it is the sum of the individual transactors' supplies of that good. We will write this total market supply as $S_i$.
	
	Thus, we arrive at the conclusion:
	
	
	Which is a proposition known as "\NewTerm{Walras' Identity}". It states that the money value of all planned market purchases are identically equal to the money value of all planned market sales and it is valid whether or not market prices equate demand with supply for each individual type of good.
	
	We can found also sometimes the identity with the following rearrangement:
	
	
	But this result is trivial and is not what interest in reality! What interest us is two very important and obvious implications that can be written in a mathematical way (do not forget that it is the main purpose of economists even if the level is a pre-school one). One implication relates to the generality of equilibrium. The other refers to states of dis-equilibrium. We will deal with each in turn using maths (even if it is trivial by the Say's Law).
	
	\begin{theorem}
		We will proof that if all but one of the markets in an economy are in equilibrium (+some other assumptions already communicated before!), then that other market also must be in equilibrium.
	\end{theorem}
	\begin{dem}
	Assume that a set of prices has been established which will equate demand with supply in every market except the $n$-th market. Since all $n - 1$ markets are in equilibrium, then if the market is big enough we can always find and indexation of goods such that:
	
	Next, multiply through by the set of prices that put these $n - 1$ markets in equilibrium. Then:
	
	And we apply Say's law we get:
	
	If this is now subtracted from Walras' Identity we obtain:
	
	Which implies immediately that the $n$-th market is also in equilibrium. It also follows in an isotropic single currency market that:
	
	To recapitulate verbally, we have shown that if all but one of the markets in an economy are in equilibrium (+some other assumptions already communicated before!), then that other market also must be in equilibrium. This result is known as "\NewTerm{Walras' law}":
	
	\begin{flushright}
		$\square$  Q.E.D.
	\end{flushright}
	\end{dem}
	We now look at some implications of Walras' Identity for dis-equilibrium.
	
	\begin{corollary}
	Assume that one market (the $n$-th market) is in dis-equilibrium. This may take the form of (positive) excess demand (where $p_nD_n > p_nS_n$) or excess supply, also known as "\NewTerm{negative excess demand}" (where $p_nD_n < p_nS_n$).
	\end{corollary}
	\begin{dem}
	It is an implication of Walras' Identity that for all markets taken as a whole there can be neither excess supply nor excess demand when we sum over all markets. We can see this by rearranging Walras's Identity as follows (without forgetting assumptions!):
	
	In order for this condition to be satisfied in the presence of disequilibrium in the $n$-th market, it must be the case that there is on off-setting dis-equilibrium in at least one other market. Form this result emerge the basic idea of "\NewTerm{demand \& supply equilibrium}".
	\begin{flushright}
		$\square$  Q.E.D.
	\end{flushright}
	\end{dem}
		
	To resume this pseudo-proof, Walras' and Say's laws are like the conservation energy in physics in a closed thermodynamic system but where mass are goods and the constant normalizing factor to conversion in energy is the money... 
	
	\begin{corollary}
	Let us see now another pseudo-proof that shows that equilibrium (Say's Law) is not affected by the change in price if we suppose that the demand and supply are proportional (or inversely proportional) to the price. That is to say that the demand/supply is a homogeneous function of degree 1 of the price such that:
	\end{corollary}
	\begin{dem}
	
	Study this fact is interesting only to have once again at look on how economists try to formalize idealistic concepts using scientific words and mathematical methods. For this purpose we start from Walras's identity:
	
	That we rewrite to indicate that $D_i$ and $P_i$ depends on prices $p_1,...,p_n$ of their own components:
	
	Now consider a change $\alpha$ on prices of components:
	
	Using the assume property of homogeneity we can write this:
	
	This result for the moment just means that a change in prices of the components of a goods is impacted on the global price of the good (trivial!).
	Therefore:
	
	And obviously we can eliminate $\alpha$ from the two side of the equality (identity).
	This finish the proof.
	\begin{flushright}
		$\square$  Q.E.D.
	\end{flushright}
	\end{dem}
	Therefore the equilibrium is not affected by movements in prices under the assumptions given before. We can therefore understand why in such a simple market wages that are the monetary price of work decrease in implicit value in proportion to the increase of supplied goods (at least in theory...) because everything should be in equilibrium and work being a good like any other... (if something increase in monetary value, something must decrease in monetary value).
	
	Here, relations are based on equations. Walras, however, distinguish two procedures to ensure balance between supply and demand:
	\begin{enumerate}
		\item A theoretical algebraic method. But... we can not easily determine individual needs in advance to know when there will be demand and prepare to build the offer. This system only works if and only if the economic agents are reasonable and agree to wait or that we have a huge database with all information about all products and all economic agents supply and offers.
		\item An empirical method that searches the solution by tests/errors operations. There is the presence of a kind of market secretary, "auctioneer". This last announce the prizes for each type of goods that could exist. The economic agents react to this prices: they ask and they offer based on that prices. For the good $i$, there is a price $p_i$, we therefore have $D_i,S_i$. We then compare the supply and demand. In case of equality, the price is an equilibrium price. In case of non-equilibrium, the auctioneer starts the procedure again and so on until there is equilibrium. That is basically the procedure that is used in the stock market and is often named the "\NewTerm{invisible hand of the market}" !!!
	\end{enumerate}
	However, the equations show us that we need the price of currency to measure supply and demand and it should be remembered that we have considered money as a given quantity fixed merchandise because the system is in equilibrium between supply and demand. But precisely, agents can not indefinitely divide the total amount of money if the number of goods is increasing. Therefore, for the process of offer and demand to be possible, we must be prepared to inject money in the market (otherwise it could become immobile which is probably not good on the long term...). We must of course also be prepared to withdraw also money and this is where the government intervenes in the economy to regulate the quantity of money in every possible way (through taxes for example).	
	
	Thus, according to the Walras model, the quantity of money available in the market is only depending on the number of economic agents. But should we therefore develop a new model for a more general framework of money demand?
	
	In fact, it is not necessary. We know that if there is a general equilibrium for $n$ goods, there is general equilibrium for $n + 1$ goods (and recursively for $n-1$ as well); the last market being simle that of the currency. The Walras model explains therefore why for a given level of quantity of currency there is a corresponding monetary value to goods!
	
	\subsection{Price Index and GDP}
	Many people and also sometimes junior politicians when speaking about retirement pension amount forget to take into account that the value of money is not the same across time. Indeed (!) 1.- today is not equal to 1.- tomorrow because of inflation (variation of price of goods). The same occur with many project manager that in their calculation of the Net Present Value (\SeeChapter{see section Quantitative Management page \pageref{net present value}}) forget to take this into account for big project whose development range on 10-20 years or sometimes even more (30 years!).

	So a useful (but not unique) way to take this into account is the use of price index that is most of times a normalized average (typically a weighted average) of price relatives for a given class of goods or services in a given region, during a given interval of time.
	
	Also when as consultant and trainer for business analysts of Fortune 500 companies that make market analysis (of consumption, of counterfeits or others) or appoint auditing companies like KPMG or Ernst \& Young in the purpose to compare year by year indicators I see always the common undergraduate error that they don't take into account the national price index or GDP to make things really comparable (among other awful errors)... 

	\textbf{Definition (\#\mydef):} A "\NewTerm{price index}" is a statistic designed to help to compare how the prices relatives, taken as a whole, differ between time periods or geographical locations. The index can be said to measure the economy's general price level or a cost of living. More narrow price indices help producers with business plans and pricing and are always useful in helping to guide investment.
	
	A naive approach is to consider a $\mathcal{C}$ of goods and servies, the total market value of transactions in $\mathcal{C}$ in some period $T$ would be:
	
	where $p_{c,T}$ represents the prevailing price of $c$ in period $T$ and $q_{c,T}$ represents the quantity of $c$ sold in period $T$.

	If, across two periods $T_{0}$ and $T_{n}$, the same quantities of each good or service were sold, but under different prices, then:
	
	and:
	
	would be a reasonable measure of the price of the set in one period relative to that in the other, and would provide an index measuring relative prices overall, weighted by quantities sold.
	
	Of course, for any practical purpose, quantities purchased are rarely if ever identical across any two periods. As such, this is not a very practical index formula.
	
	One might be tempted to modify the formula slightly to:
	
	
	This new index, however, doesn't do anything to distinguish growth or reduction in quantities sold from price changes. To see that this is so, consider what happens if all the prices double between $T_{0}$ and $T_{n} $while quantities stay the same: $P$ will double. Now consider what happens if all the quantities double between $T_{0}$ and $T_{n}$ while all the prices stay the same: $P$ will double. In either case the change in $P$ is identical. Various indices have therefore been constructed in an attempt to compensate for this difficulty.

	Of course, for any practical purpose, quantities purchased are rarely if ever identical across any two periods. As such, this is not a very practical index formula.
	
	\subsubsection{Paasche and Laspeyres price indices}
	The two most basic formulae used to calculate price indices are the "\NewTerm{Paasche index}" (after the economist Hermann Paasche) and the "\NewTerm{Laspeyres index}" (after the economist Etienne Laspeyres).

	The Paasche index is computed as:
		
	while the Laspeyres index is computed as:
	
	When applied to bundles of individual consumers, a Laspeyres index equal to $1$ would state that an agent in the current period can afford to buy the same bundle as he consumed in the previous period, \underline{given that income has not changed}; a Paasche index of $1$ would state that an agent could have consumed the same bundle in the base period as he is consuming in the current period under the same assumption!!!
	
	The Laspeyres index tends to overstate inflation (in a cost of living framework), while the Paasche index tends to understate it, because the indices do not account for the fact that consumers typically react to price changes by changing the quantities that they buy. 
	
	As can be seen from the definitions above, if one already has price and quantity data (or, alternatively, price and expenditure data) for the base period, then calculating the Laspeyres index for a new period requires only new price data. In contrast, calculating many other indices (e.g., the Paasche index) for a new period requires both new price data and new quantity data (or, alternatively, both new price data and new expenditure data) for each new period. Collecting only new price data is often easier than collecting both new price data and new quantity data, so calculating the Laspeyres index for a new period tends to require less time and effort than calculating these other indices for a new period.

	In practice, price indexes regularly compiled and released by national statistical agencies are of the Laspeyres type, due to the above-mentioned difficulties in obtaining current-period quantity or expenditure data.
	
	\subsubsection{Fisher index and Marshall–Edgeworth index}
	A third index, the "\NewTerm{Marshall–Edgeworth index}" (named for economists Alfred Marshall and Francis Ysidro Edgeworth), tries to overcome these problems of under- and overstatement by using the arithmetic means of the quantities:
	
	A fourth, the Fisher index (after the American economist Irving Fisher and not the statisician!!!), is calculated as the geometric mean of $P_\text{P}$ and $P_\text{L}$:
	
	However, there is no guarantee with either the Marshall–Edgeworth index or the Fisher index that the overstatement and understatement will exactly cancel the other.

	While these indices were introduced to provide overall measurement of relative prices, there is ultimately no way - as far as we know - of measuring the imperfections of any of these indices (Paasche, Laspeyres, Fisher, or Marshall–Edgeworth) against reality.
	
	The Fisher index is used by the Canada statistics since 2001 instead of the Laspeyres formula to calculate the estimate of real GDP in terms expenditure. The Switzerland OFS use the Laspeyres index (but they do not say since how many years ...).
	
	\pagebreak
	\subsubsection{Gross domestic product (GDP)}
	\textbf{Definition (\#\mydef):} The "\NewTerm{Gross domestic product (GDP)}" is a monetary measure of the market value of all final goods and services produced in a period (quarterly or yearly). Nominal GDP estimates are commonly used to determine the economic performance of a whole country or region, and to make international comparisons. Nominal GDP, however, does not reflect differences in the cost of living and the inflation rates of the countries; therefore using a "\NewTerm{gross domestic product (at purchasing power parity) per capita (GDP PPP)}". Such calculations are prepared by various organizations, including the International Monetary Fund and the World Bank. As estimates and assumptions have to be made, the results produced by different organizations for the same country are not hard facts and tend to differ, sometimes substantially, so they should be used with caution (some companies use the McDonald burger index\footnote{ published by The Economist as an informal way of measuring the purchasing power parity} as element of comparison!).
	\begin{figure}[H]
		\centering
		\includegraphics{img/economy/gdp_country.jpg}
		\caption[Countries by GDP (PPP) Per Capita in 2015]{Countries by GDP (PPP) Per Capita in 2015 (source: Wikipedia)}
	\end{figure}
	The basic formul for caculating GDP is:
	
	This formula is almost self-evident (if you take time to think about it)!

	GDP is a measure of all the goods and services produced domestically. Therefore, to calculate the GDP, one only needs to add together the various components of the economy that are a measure of all the goods and services produced.

	Many of the goods and services produced are purchased by consumers. So, what consumers spend on them ($C$) is a measure of that component. 

	The next component is the somewhat mysterious quantity "$I$," or investment made by industry. However, this quantity is mysterious only because investment does not have its ordinary meaning. When calculating the GDP, investment does NOT mean what we normally think of in the case of individuals. It does not mean buying stocks and bonds or putting money in a savings account ($S$ in the diagram). When calculating the GDP, investment means the purchases made by industry in new productive facilities, or, the process of "buying new capital and putting it to use". This includes, for example, buying a new truck, building a new factory, or purchasing new software. This is indicated in the diagram by an arrow pointing from one factory (enterprise) to another. In essence, it shows the factory "reproducing itself" by buying new goods and services that will produce still more new goods and services. 

	The next component is $E$, or the difference between the value of all exports and the value of all imports. If Exports exceeds imports, it adds to the GDP. If not, it subtracts from the GDP. Thus, even if a nation's people work very hard to produce products for exports, but still import more than they export, the nation's GDP will be negatively impacted. This is one of the reasons trade deficits are frequently a political target. Because the balance of trade can be either positive or negative, we can rewrite the equation, showing the components of $E$, using $X$ for Exports and $M$ for Imports:
	
	\begin{figure}[H]
		\centering
		\includegraphics{img/economy/gdp_calculation_synoptic.jpg}
	\end{figure}
	The practitioner mus obviously never that all these fours variables are associates with an estimation error that must be taken into account in any micro or macroeconomics model (for example Switzerland say that the error is only about $\pm 0.5\%$... but this statement should be check by a private independent company as some areas in the country say they have an error of $\pm 2.5\%$).
	
	\pagebreak
	\subsection{Supply and Demand Theory}
	The theory of supply and demand as presented below requires a major revision because highly incomplete. Therefore the ideas presented below are to this day to be taken lightly and only as a simple introduction to the underlying concepts of prices adjustment of goods. 

	In our human society where there is a exchange (reference) money and goods, one major problems remains that is the of determining the monetary value of a good (material or immaterial). To determine this, we need at least to know the evolution of supply and demand. This is on what we will focus now by starting with simplistic models and complicating them increasingly:
	
	\begin{tcolorbox}[title=Remark,colframe=black,arc=10pt]
	In our point of view and based on our consulting experience the determination of a product price belongs more to the field of corporate financial engineering than to Economy or Econometrics. Indeed, determining the product of an insurance or Call option doesn't use at all the same techniques level as determining the price of a manufacturing price.
	\end{tcolorbox}
	
	\subsubsection{Expected utility theory}
	Before going trough the lecture of a highly complex model on supply and demand, it is necessary to identify what motivates economic agents in their consumption choices and to model their behavior based on the fundamental principle of rationality.
	
	The economic agent will be perceived as a unique individual with a choice that he seeks to maximize the satisfaction. His tastes are subjective even though they depend on certain objective characteristics such as age, the level of culture, his fortune, etc. The level of satisfaction will be defined from a utility function that we will see the basic principles and the maximization under constraints.
	
	Several principles based utility of goods and lead to the concept of "\NewTerm{marginal utility}", a central concept in the theory of the preference of the economic agent. According to Aristotle (being at the origin of the concept of value-utility), the utility of goods derives from the satisfaction of needs. Étienne Bonnot de Condillac states: "the value of things is based on their usage". This idea of value based on utility, fundamental for marginalist economists, opposes therefore to the current theoretical labor-value based on the amount of work, direct and indirect, incorporated in the manufacture of the good (Adam Smith, Karl Marx).
	
	However we should, preferably, consider an important assumption in this model: There is a certain satiation of needs, but it is almost never total.
	
	Thus, for a given good marginal utility of the last unit consumed is lower than that of previous units but not null and always positive. This is the "\NewTerm{principle of diminishing marginal utility}" on the additional unit consumed. Or in other words this law states that as additional units of a good ar consumed, while holding then consumption of all other goods constant, the resulting increments in utility will diminish.
	
	Thus, under the multiple consumption $n \in \mathbb{N}$ of a single good of given nominal value, the total utility $U_T$ (sum of marginal utilities $U_M(n)$) is a curve of the type:
	
	\begin{figure}[H]
		\centering
		\includegraphics{img/economy/asymptotic_utility.jpg}
		\caption{Asymptotic utility}
	\end{figure}
	and therefore the marginal utility is of the type:
	\begin{figure}[H]
		\centering
		\includegraphics{img/economy/asymptotic_utility.jpg}
		\caption{Asymptotic marginal utility}
	\end{figure}
	So, faced with a price for each good, the economic agent compares this price with the marginal utilities that he could successively use from their consumption. He buys them as long as their value exceeds the price (surplus related to their purchase) and ceases to buy as soon as the marginal utility falls below the price of the good. His interest is therefore to buy other products for which there is a positive surplus (marginal utility that is higher to their prices).
	
	This example, relative to one good, should now be extended to a basket of goods to determine the overall utility of this basket.
	
	Consider for this an economic agent $i$ in an economy that has $I \in \mathbb{N}$ goods. It can therefore buy at most $I$ goods. A basket of consumption can therefore corresponds to a vector $\vec{n}_i=(n_1,n_2,...,n_k)$ of goods: where the $n_j$ may represent null quantities purchased by the consumer. The utility $U(\vec{n}_i)$ of this basket will be assumed as additive such that:
	
	that is to say the sum of total of utilities relative to the quantities consumed of each good.
	Let us now consider a basket with two goods, we can without much error hypothesize that these goods can be divided into small fractions $x_i$ as small as we want from other goods/components. So roughly, we no longer work $\mathbb{N}$ but in $\mathbb{N}_+$.
	Therefore if we consider a basket $\mathbb{R}^2$ of an economic agent, we will assume that it is such that its exact total differential (\SeeChapter{see section Differential and Integral Calculus page \pageref{total exact differential}}) is zero such that:
	
	The ratio:
	
	is defined as the "\NewTerm{marginal rate of substitution}" (MRS) between the two elementary goods $1,2$. That is to say the additional quantity of good $1$ we need to provide to the economic agents to accurately compensate for a decrease of one unit of good $j$ under the hypothesis that they are always compensated (fact of the null total exact differential).
	
	The behavior attributed to the economic agent is to classify all the baskets of possible  goods (vectors) using a scale of utility without it necessarily corresponds to a quantified assessment. This classification capability corresponds to the concept of "\NewTerm{ordinal utility}" (which can - by its name - therefore be ordered...) and to the use of a preference relation, denoted by $\succeq $ (preferred or indifferent to) that satisfies the following trivial properties:
	\begin{enumerate}
		\item[P1.] Reflexitivity: For every $\vec{n}_1$, For every $\vec{n}_1 \succeq \vec{n}_1$.
		
		\item[P2.] Completeness: For every $\vec{n}_1$ and $\vec{n}_1$ either $\vec{n}_1 \succeq \vec{n}_2$ or $\vec{n}_1 \preceq \vec{n}_2$.

		\item[P3.] Transitivity: $\vec{n}_1 \succeq \vec{n}_2 \wedge \vec{n}_2 \succeq \vec{n}_3 \Leftrightarrow \vec{n}_1 \succeq \vec{n}_3$ (consistency of successive rankings).
		
		\item[P4.] Continuity: Let $\vec{n}_1$, $\vec{n}_2$ and $\vec{n}_3$ be such that $\vec{n}_1 \succeq \vec{n}_2 \succeq \vec{n}_3$ then there exists a proportion/fraction $p \in [0,1]$ (that can be assimilated to a probability) such that $\vec{n}_2$ is equally good as $\vec{n}_2 \sim p \vec{n}_1+(1-p)\vec{n}_3$.
		
	\end{enumerate}
	This order relation in the mathematical sense, is used in most of today's presentations of preference theory. This order is complete if it permits to compare two baskets of goods in the set $I$ of goods.
	
	Such a complete order relations permits to define an equivalence relation on the set of goods and a strict order, and to represent preferences from utility functions:
	
	If the function $U$ is defined by a number, it does not reflect anymore an assessment of the utility, but since only the possibility to compare the order of the utilities, relatives to baskets with any goods.
	
	The ability to order different baskets of goods of $\mathbb{R}^I$ permits to define surfaces with isolines (\SeeChapter{see section Functional Analysis page \pageref{isoline}}) where the utility is constant, named "\NewTerm{indifference curves}\index{indifference curve}" or "\NewTerm{iso-utility curves}\index{iso-utility curve}". 
	
	The following figures give a good representation of these curves in $\mathbb{R}^2$ (basket of two type of goods) and their main properties that will help us latter to better understand the indifference curved used to build the Capital Asset Pricing Model (CAPM) for the study of modern portfolio theory (see further below).
	
	Thus, two baskets such that $\vec{A} \sim \vec{B}$ in $\mathbb{R}^2$ is graphically traduced in a family of curves, where each is constantly decreasing such that:
	\begin{figure}[H]
		\centering
		\includegraphics{img/economy/iso_utility.jpg}
		\caption{Example of iso-utility (indifference) curve}
	\end{figure}
	Why do not we have straight lines plots or other things? The reason is relatively simple and the following chart explains that trivially. Consider the iso-utility (indifference) curve below that is a projection in the plane of the $U_{\vec{B}}(x_1,x_2): \mathbb{R}^2 \mapsto \mathbb{R}^3$:
	\begin{figure}[H]
		\centering
		\includegraphics{img/economy/counter_example_of_iso_utility.jpg}
		\caption{Counter example of iso-utility (indifference) curve}
	\end{figure}
	
	Above $\vec{B}$ dominates $\vec{A}$ but its basket of goods has more $x_1$ AND $x_2$ than $\vec{A}$ and then $U_{\vec{B}}(x_1,x_2)>U_{\vec{A}}(x_1,x_2)$. These two points can therefore not be on the same indifference curve (iso-line) and impose an indifference curve to be decreasing i.e. to be convex!
	
	The assumption that $\vec{A} \sim \vec{B}$ also implies directly that the total exact differential is zero.
	
	The iso-utility curves (indifference curves) can not cut themselves (may not intersect). Even if it is obvious if you did forget that $U_{\vec{B}}(x_1,x_2): \mathbb{R}^2 \mapsto \mathbb{R}^3$, consider the two iso-utility curves (indifference curves) below:
	\begin{figure}[H]
		\centering
		\includegraphics{img/economy/iso_utility_non_intersect.jpg}
		\caption{Iso-utility intersection impossibility}
	\end{figure}
	The baskets $\vec{A}$ and $\vec{B}$ are located on the same iso-utility curve $U_1$ while $\vec{B}$ and $\vec{C}$ are themselves located on the same curve $U_2$. Thus, we can write $\vec{A} \sim  \vec{B}$ and that $\vec{B} \sim  \vec{C}$. According the property of transitivity of the utilities, we should then have $\vec{A} \sim  \vec{C}$. These two baskets are not equivalent since $\vec{A}$ and $\vec{C}$ are not located on the same curve. Therefore two indifference curves can not intersect.
	
	We are therefore aware that there are special relations between the goods that will change our consumption attitudes. This is particularly true for complementary and substitutions goods:
	
	\textbf{Definitions (\#\mydef):}
	\begin{enumerate}
		\item[D1.] Two goods are said to be "\NewTerm{complementary goods}" if the possession of one and the other provides superior satisfaction to the total satisfaction of both goods if taken isolated ("\NewTerm{super-additive}"). Thus, there is complementarity between skis and a package on the ski lifts, between a car and gasoline. This can be interpreted by the following indifference curve:
	\begin{figure}[H]
		\centering
		\includegraphics{img/economy/utility_super_additivity.jpg}
		\caption{Utility super-additivity visual representation}
	\end{figure}
	Indeed, for the couple $\left\lbrace \text{car}, \text{petrol} \right\rbrace$ each curve has respectively a minimum below which we can not go down so that the couple maintains its consumer interest as such (it is not worthwhile to buy a car if the utility of gasoline tends to zero).
	\item[D2.] Two goods are said to be "\NewTerm{substitutable goods}" if we can easily replace one by the other, for example in the case of or prices increasing or deacreasing. Tea and coffee are substitutable goods because without one, we often defer to the other one. This is even more true for two brands of the same drink (Pepsi\circledR{} and Coca-Cola\circledR). The mad cow crisis is also a good indicator of the substitutability of meat products, with a postponing consumption on poultry and lamb. This can be interpreted by the following indifference curve:
	\begin{figure}[H]
		\centering
		\includegraphics{img/economy/utility_substituables_goods.jpg}
		\caption{Visual representation of substitutable goods utility}
	\end{figure}
	\end{enumerate}
	Indeed, the intersection with the respective axes indicates precisely the total possible substitution of one good by the other in the basket.
	
	\begin{tcolorbox}[colframe=black,colback=white,sharp corners]
	\textbf{{\Large \ding{45}}Example:}\\\\
	Consider we want to calculate the MRS along the indifference curve of $U=x_1x_2=100$ (this is a hyperbolic function). We have proved that MRS was given by under a given assumption that:
	
	Therefore, for the three points $A, B, C$ of respective coordinates:
	
	we found the respective MRS values:
	
	These values express the equivalences between the goods $2$ and $1$ for marginal changes in the quantities of these goods. Thus at point $A$, to keep the utility level of $100$, the consumer is willing to abandon the good 2 for 1 to increase its consumption of good $1$ by a factor of $4$. At point $B$ the equivalence between the two goods is in a ratio of $1$ to $1$, etc.
	\end{tcolorbox}
	\begin{tcolorbox}[title=Remark,colframe=black,arc=10pt]
The concept of indifference curves was developed by Vilfredo Pareto and others in the first part of the 20th century. The use of this concept has allowed the economic analysis to use this concept of preferences in determining the choice rather than the concept of utility that has the problem of not being able to be measured objectively.
	\end{tcolorbox}

	\pagebreak
	\subsection{Net gain/loss opposite feedback model}
	Let us now consider, regardless of the preference theory, a monopoly model with  perfect information for a primary need. We denote by  $D(t)$ the demand in the market and $O(t)$ the offer. We assume then a infinitely small change in demand depending on the time proportional to the demand (in absence of offer) that is to say the demand follows an exponential law:
	
	and for the offer, in the absence of demand (!), is a decreasing exponential law:
	
	Customers (demand) and suppliers (offers) are interacting. To quantify the contribution between groups, we consider the offer on the assumption that its value or intensity is a function of probability of meeting between the consumer and the supplier and will be proportional to the multiplication of the percentage of supply and demand $O\cdot D$.
	
	The fact of discovering a new product does not have the same effects on both groups (consumers/suppliers). First, of course, each offer acquired by a supplier is a net gain for the second (supplier) and will be assumed as a net loss for the first (consumer). Therefore, if the effect of interactions is assumed as being proportional to $O\cdot D$ the signs of influence of interaction differ by:
	
	As we can see this model is strongly inspired by the famous "Lotka-Voltera model" (\SeeChapter{see section Population Dynamics page \pageref{lotka volterra model}}).
	
	Therefore this model could be used to explain economics crash or the fact that some trends product have a cyclic demand (for sure with periods that need to be determined) or that we can see in supply chain management small periodic fluctuations.
	
	Before going further, let us look to the values for which the derivatives vanish (which will give us in fact the equilibrium point between supply and demand because the two relations can therefore be written as equal):
	
	Thus:
	
	A trivial solution is the "\NewTerm{solution of inexistance}" (also sometimes named "{critical solution}") given by:
	
	Otherwise, we also have as a possible solution:
	
	Now we normalize these equations by writing (so they are dimensionless):
	
	with this normalization, the model can be rewritten:
	
	By rearranging the coefficients, the system can finally be written (excluding the solution of inexistance):
	
	for which derivatives vanish at the point $(1,1)$, which can be assimilated to the "Say equilibrium".
	
	The discreet plot of this system of equations (in which we recognize a logistic term as seen in the section of Populations Dynamics) gives us with $m=r=b=c=1$ and initial conditions $(x_0,y_0)=(4,1)$:
	\begin{figure}[H]
		\centering
		\includegraphics{img/economy/cycle_offer_demand.jpg}
		\caption{Cycle of offer and demand}
	\end{figure}
	We see as the real market seems show to us, the cycles of supply/demand (some outdated products become fashion again) for which, using statistics, we must determine the initial conditions in order to know the period of time. We note also on that figure that offers is always a little behind demand in this model (which is not always true because we know that many times that offer created demands). Unfortunately this model does not have a damping or multiplier factor suggesting a potential for improvement.
	
	The previous figure represent the variables $x$ and $y$ as a function of time. However, what can be interesting for a scientist (or an economist) is the representation of $y$ in terms of $x$ and vice versa. Thus, we obtain for the same initial conditions $m, r, b, c$ and for the various initial values of $(x_0,y_0)$ (the Offer is in ordinate and the Demand on the abscissa):
	\begin{figure}[H]
		\centering
		\includegraphics{img/economy/offer_demand_space_phase.jpg}
		\caption{Representation in phase space offer/demand cycle}
	\end{figure}
	The above figure is very interesting to interpret when you follow a path in the counter-clockwise direction.
	
	Thus we see (in the phase space representation) that for fixed initial conditions, the system is periodic and has trivially an equilibrium point at:
	
	that correspond to the points where:
	
	Finally, we have two pairs of equilibrium points (that is almost trivial, looking at the system of equations):
	
	The question that will naturally arise is the "real" direction of rotation (representation) of the phase plane. Thus, representing the directions using a vector field, we get the representation which is effectively counter-clockwise:
	\begin{figure}[H]
		\centering
		\includegraphics{img/economy/offer_demand_vector_field.jpg}
		\caption{Rotation direction offer/demand phase space}
	\end{figure}
	
	To know in which direction we are going in the phase space at a given time, it suffices to know the derivative $\mathrm{d}y /\mathrm{d}x$ (or vice versa $\mathrm{d}x /\mathrm{d}y$). Therefore we have:
	
	That said, we see well on the phase diagram with vector field as vectors that it comes a time in the cycle of this model where the offer is very high for low demand. So the mathematical model (theoretical) clearly explains what can be a priori counter-intuitive to many humans (offer creates the demands).
	
	However, we can (must) ask ourselves the question of what happens after a small perturbation around the equilibrium point (which is of utmost importance in economy).
	
	So we have the following Lotka-Volterra system in equilibrium:
	
	By putting an infinitely small perturbation, it is will be written:
	
	Neglecting the quadratic $xy$ terms, we obtain:
	
	\begin{enumerate}
	\item We focus first on the study near the point of extinction $(0,0)$, this is why we can neglect the quadratic terms $xy$ but the expression remain too complicated. So, always considering we are near the point of extinction $(0,0)$, we consider a bit unfairly but cleverly (otherwise we still could not solve the problem analytically) the following approximations:
	
	Therefore:
	
	This shows us for the system in equilibrium, close to the inexistence point, the offer (supply) decreases exponentially (power law) as demand increases exponentially:
	
	This makes economic sense: when there is little offer (respectively demand), the demand (respectively offer) is growing when as far as demand increases, the offer grows and converge increasingly near to the demand (nature...).
	
	\begin{tcolorbox}[title=Remark,colframe=black,arc=10pt]
	In the literature we find sometimes the "-" sign in the top or bottom in the previous equations. In reality, the position of the "-" sign is not important because it is just the starting choice in the system dynamics.
	\end{tcolorbox}
	
	\item Close to the equilibrium point $(1,1)$ we start from:
			
		And we put $x:=1+x,y:=1+y$ where $x,y$ are now small perturbations near the point $(1,1)$. Therefore we have:
		
		 (hop! we change the position of the "-" sign on purpose to show that this is only a starting choice !!!):
		
		To solve this system, let us differentiate the first equation once again:
		
		and by injecting in it $\mathrm{d}y/\mathrm{d}t$:
		
		So we get a small second order differential equation (\SeeChapter{see section Differential and Integral Calculus page \pageref{second order differential equations}}). Whose typical simple solution is:
		
		By injecting this solution into the differential equation, we obtain after simplification of exponentials a simple polynomial of the second degree (\SeeChapter{see section Calculus page \pageref{polynomial}}):
		
		Whose solution is trivial:
		
		Thus, the general solution of the differential equation is the linear combination of the two special solutions we get:
		
		But we therefore have:
		
		Therefore, knowing $x(t)$ we obtain easily:
		
		Now let us use the Euler formula (\SeeChapter{see section Numbers page \pageref{euler formula}}):
		
		Thus we have:
		
		and as (\SeeChapter{see section Trigonometry page \pageref{remarkable angles}}) $\cos(x)=\cos(-x),-\sin(x)=\sin(-x)$ then we have:
		
		and similarly, we obtain:
		
		Thus, around the equilibrium point $(1,1)$ with sufficiently small perturbations to validate linearization the system oscillate as ellipses (or circles) whose axes are defined by the two equations above.
	\end{enumerate}
	We can get the graphs above with Maple 4.00b (this is a nice example of application of this software):
	
	\texttt{>restart: with(plots): with(DEtools):}\\
 	\texttt{>rate\_eqn1:= diff(h(t),t)=(0.1)*h-(0.005)*h*(1/60)*u;}\\
 	\texttt{rate\_eqn2:=diff(u(t),t)=(0.00004)*h*u-(0.04)*u;vars:= [h(t), u(t)];}\\
 	\texttt{>init1:=[h(0)=2000,u(0)=600]; init2:=[h(0)=2000,u(0)=1200]; init3:=[h(0)=2000, u(0)=3000];domain := 0 .. 320;}\\
 	\texttt{>L:= DEplot({rate\_eqn1, rate\_eqn2}, vars, domain,{init1}, stepsize=0.5, scene=[t, u], arrows=NONE):}\\
  	\texttt{>H:= DEplot({rate\_eqn1, rate\_eqn2}, vars, domain,{init1 }, stepsize=0.5, scene=[t, h], arrows=NONE):}\\
  	
  	\begin{figure}[H]
		\centering
		\includegraphics{img/economy/lotka_volterra_offer_demand_maple.jpg}
		\caption{Lotka-Volterra offer/demand model periodic plot in Maple 4.00b}
	\end{figure}
  	
  	\texttt{>DEplot({rate\_eqn1, rate\_eqn2}, vars, t= 0 .. 160, {init1, init2, init3}, stepsize=0.5, scene=[h,u], title='Demande vs. 60 * Offer for t = 0 .. 160', arrows=slim);}\\
  	
  	\begin{figure}[H]
		\centering
		\includegraphics{img/economy/lotka_volterra_offer_demand_phase_space_maple.jpg}
		\caption{Lotka-Volterra offer/demand phase space model plot in Maple 4.00b}
	\end{figure}
	
	This model, however is imperfect because it takes into account only a disgruntled monopoly with net loss and perfect information. The fact of considering the population constant is not too bothersome but strictly speaking we should add a logistic term in the initial equations. There is still work to do...
	
	\pagebreak
	\subsection{Capitalization and Actuarial}
	\textbf{Definition (\#\mydef):} The "\NewTerm{capitalization}" is the field of financial mathematics which calculates future values based on present values, while the "\NewTerm{actuarial calculation}" consist to determine how much should be loan to get a given amount fixed in advance (the opposite of capitalization).
	
	In a dynamic market, economic agents can lend or borrow capital in return for which they receive or respectively pay a periodic interest. This interest is justified by the risk-taking that takes the creditor (the lender's of capital) in case of non-repayment of all or a share of the initial capital that must repay the debtor (the one who must repay the borrowed capital) . Alternatively, in the point of view of economic market, loans allow some economic agents to set up goods by betting on the fact that either they will create the offer or that the offer will come by itself but wishing be ahead of the concurrence.
	
	\begin{tcolorbox}[title=Remark,colframe=black,arc=10pt]
	When credit is contracted with an economic agent that is non solvable financial analysts speak about "NINJA" loan for "no income, no job and no assets". 
	\end{tcolorbox}
	
	When a capital is lent (or borrowed, it depends from the point of view ...) in order to increase market dynamics (the amount of goods in circulation on a given duration) then we talk of "\NewTerm{financial asset}", this to make clear that capital participates in the activities of the economy.
	
	\textbf{Definitions (\#\mydef):}
	\begin{enumerate}
		\item[D1.] We name "\NewTerm{yield of a loaned financial asset}" the relative progress ratio given by:
		
		\item [D2.] We name "\NewTerm{arithmetic return on investment}" the relative progress ratio:
		
		where $V_i$ is the initial value of the investment and $V_f$ its final value.
	\end{enumerate}
	It follows from this latter definition that if an investment has reported $5\%$ the first year and carried a net loss of $2\%$ the second year, the "\NewTerm{\underline{arithmetic} return on investment}" is then:
	
	But it is wrong to use the arithmetic average for this type of situation because the final sum obtained after two years would be mathematically:
	
	which then gives the previous example:
	
	So the real average return is by definition the "\NewTerm{geometric return on investment}\index{geometric return on investment}" as:
	
	that is to say, it is simply a geometric mean (\SeeChapter{see section Statistics page \pageref{geometric mean}}) \label{geometric mean for finance}. It comes then:
	
	which is obviously much different from the ROI arithmetic average obtained above!
	\begin{tcolorbox}[title=Remark,colframe=black,arc=10pt]
	We say of an asset that has a "risk-free return\index{risk-free return}" or "risk-free rate of return\index{risk-free rate of return}\label{risk-free rate of return}" if its future value (respectively it's corresponding rate) is perfectly known. 
	\end{tcolorbox}
	Given an asset that can have the future optimistic return $r_{f,1}$ with probability $p_1$ and the pessimistic value $r_{f,2}$  with a probability $p_2$ or other values $r_{fn}$ with probabilities $p_n$ then expected mean of return is obviously given by:
	
	that the monetary sum is of the type active/passive, the types of applicable yields can be identical or variable. There exists, however, classic cases that are well know and non-stochastic. For their study, let us define some variables:
	\begin{itemize}
		\item $C_0$ represents the initial capital or more technically the "\NewTerm{actual value AV}" or "present value PV".
		\item $C_n$ represents the final capital or "\NewTerm{capitalized value CV}" or "\NewTerm{future value FV}" after $n$ time periods.
		\item $t\%$ is the rate most technically named "\NewTerm{effective rate}".
		\item $I_n$ represents the "\NewTerm{interest}" obtained after $n$ periods (horizon) by the current value
	\end{itemize}
	Let us add also the relation:
	
	named "\NewTerm{capitalization factor}".
	
	\textbf{Definition (\#\mydef):} We define the "\NewTerm{interest}" as the remuneration of a capital (the amount of money) loaned or invested for a given time. The interest can be paid at once or periodically if the duration of the loan or investment is for a long time. The interest may be payable in advance (\NewTerm{praenumerando}) or at the end of the period (\NewTerm{postnumerando}). The interest is based on the duration of the loan (or investment), of the borrowed (or paid) capital that we also name the "principal" and also from the "rate" of interest charged. The period over which the interest is calculated is in general the year, but it can be shorter: semester, quarter, month, day, hour, minute, seconds, etc.
	
	\begin{tcolorbox}[title=Remark,colframe=black,arc=10pt]
	In a text, the interest rate\index{interest rate} is normally expressed in percentage \% but in financial calculations, it is customary to calculate or write it in decimal form.\\
	
	The reader should also be aware the in the finance industry, we use many times the "\NewTerm{basis points}\index{basis points}" (BPS) as common unit of measure for interest rates and other percentages. One basis point is equal to $1/100$th of $1\%$, or $0.01\%$ ($0.0001$), and is used to denote the percentage change in a financial instrument. The relation between percentage changes and basis points can be summarized as follows: $1\%$ change = $100$ basis points, and $0.01\% = 1$ basis point.
	\begin{table}[H]
		\centering
		\begin{tabular}{|c|c|}
		\hline
		\rowcolor[HTML]{C0C0C0} 
		\textbf{Basis Points} & \textbf{Percentage Terms} \\ \hline
		$1$ & $0.01\%$ \\ \hline
		$10$ & $0.1\%$ \\ \hline
		$50$ & $0.5\%$ \\ \hline
		$100$ & $1\%$ \\ \hline
		$1000$ & $10\%$ \\ \hline
		$10,000$ & $100\%$ \\ \hline
		\end{tabular}
		\caption{Basis points (BPS) and percentage equivalence}
	\end{table}
	\end{tcolorbox}
	
	\subsubsection{Dates Intervals}
	In determining the amount of interest of a loan (or investment), it is first necessary to know the duration of the latter or the dates defining the periods of payment of an obligation (\NewTerm{settlement date}).
	
	The calculation of dates and durations is the first step to learn in actuarial mathematics. If some software used in the calculation of the period the calendar with the civil year (365 days according to the Gregorian calendar), others are based on the commercial year (360 days), which was the case of most European banks (it is to their advantage financially to make the choice of this latter...) before the arrival of the calendar for the Eurozone.
	
	\begin{tcolorbox}[title=Remarks,colframe=black,arc=10pt]
	\text{R1.} In financial markets, there is as far as we know only one single convention for determining a date range to calculate a duration: the first day (departure date) is included in the period. The last day (end date or settlement date) is excluded from the period. Thus the period from 15 to 25 June has 10 days.\\
	
	\textbf{R2.} In the context of this book, who wish to have the more rigorous possible approach on treated topics, we will not dwell on the aberrant German, European or American 30/360, methods (we can then do every country on the planet then... and refer to the calculations of dates functions of a spreadsheet software like Microsoft Excel...) to focus only on 365 days method (the exact base system) that is, and remains, the most natural counting system to use as it includes the months with 28, 29, 30 or 31 days.\\
	
	\textbf{R3.} Note that in regard to savings accounts, banks are based on a system of "fortnight" ( (two weeks/half a month), and they therefore estimate that there are 24 fortnights a year.
	\end{tcolorbox}
	
	We must therefore in this system of the exact basis know how to calculate the number of days between two dates $D_1,D_2$ given by the calculation $D_2-D_1$ each such date being encrypted in standardized format $yyyy-mm-dd$ (year-month-day) or under non-standardized $d.m.y$ format (day.month.year).
	
	\textbf{Definitions (\#\mydef):}
	
	\begin{enumerate}
		\item[D1.] The Gregorian Calendar is defined as having 12 months.
		\item[D2.] The months:
			
			are 31-day months and the months of:
			
			have $30$ days.
		\item[D3.] February is a special case to correct the fact that the civil year calendar of 365 days does not correspond quite the orbital period of the Earth around the Sun that is about $365.25$ days... Thus, all the years that are multiples of $4$ or $400$ are leap years (February has $29$ days instead of $28$) but the years that are divisible by $100$ are not leap years.
	\end{enumerate}
	\begin{tcolorbox}[colframe=black,colback=white,sharp corners]
\textbf{{\Large \ding{45}}Examples:}\\\\
	E1. 1992, 1996, 2004, 2008 are leap years.\\
	
	E2. The years 1900, 2100, 2200, 2300 are not leap years (because divisible by $100$).\\
	
	E3. The years 1600, 2000, 2400, 2800 are leap years because even if they are divisible by $100$, they are multiples of $400$.
	\end{tcolorbox}
	
	These definitions and examples being given, imagine now date given following the above rules. The number of days since year $1$ is therefore (as seen in the section of Numerical Methods $[x]$ will be the integer par of $x$):
	
	This relation is logically deduced as follows for the dates where $m \leq 2$ (that is to say only for the first two months of the year including the problematic of February):
	\begin{enumerate}
		\item We have $365 (y-1)$ because given $a$ the number of calendar days since the year $1$ is $365$ times the number of years $y$ has decreased by one unit as the current year is not over.
		
		\item The same remark applies for months with $31(m-1)$: considering months of 31 days, the current month is not taken into account (this is why we have $m-1$). The correction for the months with $28$ or $30$ days is taken into account in the case where $m>2$ that we will see later below.
		
		\item Logically, we add $d$ (which contains all the information as to know whether the year  $y$ is leap year or not) to the sum of the previous two terms.
		
		\item The terms $\left[\dfrac{y-1}{4}\right]-\left[\dfrac{y-1}{100}\right]+\left[\dfrac{y-1}{400}\right]$ equation give the number of 29th February between year $1$ and taking into account all leap years that occur all the multiples of $4$ and $400$ years except for years that are multiples of $100$.	
	\end{enumerate}
	
	If $m>2$, we must use the following relation:
	
	
	This relation is deducted in the same way as the previous one except that certain terms in the numerator are not reduced by one unit because having $m>2$, it is necessary, for these terms, to consider the current year in the calculation.
	
	The last term $\left[2 + 0.5 + 0.42M\right]$ is here to fix the fact that every month does not have $31$ days. To get it, we build the following table (the third column gives the offset in days compared to the case where all the months have $31$ days):
	
	\begin{table}[H]
	\begin{center}
		\definecolor{gris}{gray}{0.85}
			\begin{tabular}{|c|c|c|}
				\hline
\multicolumn{1}{c}{\cellcolor{black!30}\textbf{Month}} & \multicolumn{1}{c}{\cellcolor{black!30}\textbf{Month number $n$}} & \multicolumn{1}{c}{\cellcolor{black!30}\textbf{Offset $\Delta$}} \\ \hline
		march & 3 & 3\\ \hline
		april & 4 & 4\\ \hline
		may & 5 & 4\\ \hline
		june & 6 & 5 \\ \hline
		july & 7 & 5\\ \hline
		august & 8 & 5\\ \hline
		september & 9 & 6\\ \hline
		october & 10 & 6\\ \hline
		november & 11 & 7\\ \hline
		december & 12 & 7\\ \hline
	\end{tabular}
	\end{center}
	\caption[]{Monthly offset $\Delta$ in days}
	\end{table}
	A simple linear regression (\SeeChapter{see section Numerical Methods page \pageref{least squares method}}) gives:
	
	By taking the integer value and verifying that the selected function is correct, we finally get (taking an accuracy of two decimal places):
	\begin{table}[H]
	\begin{center}
		\definecolor{gris}{gray}{0.85}
			\begin{tabular}{|c|c|c|c|c|}
				\hline
\multicolumn{1}{c}{\cellcolor{black!30}\textbf{Month}} & \multicolumn{1}{c}{\cellcolor{black!30}\textbf{Month number $n$}} & \multicolumn{1}{c}{\cellcolor{black!30}\textbf{Offset $\Delta$}}  & \multicolumn{1}{c}{\cellcolor{black!30}\textbf{$\Delta(n)$}}  & \multicolumn{1}{c}{\cellcolor{black!30}\textbf{Offset $\left[\Delta(n)\right]$}} \\ \hline
		march & 3 & 3 & 3.26 & 3 \\ \hline
		april & 4 & 4  & 3.68 & 4 \\ \hline
		may & 5 & 4  & 4.1 & 4\\ \hline
		june & 6 & 5  & 4.52 & 5 \\ \hline
		july & 7 & 5  & 4.94 & 5\\ \hline
		august & 8 & 5  & 5.36 & 5 \\ \hline
		september & 9 & 6  & 5.78 & 6\\ \hline
		october & 10 & 6  & 6.2 & 6 \\ \hline
		november & 11 & 7  & 6.62 & 7\\ \hline
		december & 12 & 7  & 7.04 & 7 \\ \hline
	\end{tabular}
	\end{center}
	\caption[]{Comparison of the regression function with the target}
	\end{table}

	 \subsubsection{Rates Equivalence}
	 Let us now briefly focus on rates calculation before directly addressing the various calculations of interests types.

	\textbf{Definition (\#\mydef):} The "\NewTerm{proportional rate} $t_p\%$" is bringing to a same capital, during the same period, the same "\NewTerm{simple interest}" (see the definition of the simple interest further below), and given by the relation:	
	
	If the proportional rate is calculated on the basis of a year, then we speak of "\NewTerm{annualized yield rate}"; if calculated on the basis of a month, then we speak of "\NewTerm{monthly-rate of return}".
	\begin{tcolorbox}[colframe=black,colback=white,sharp corners]
\textbf{{\Large \ding{45}}Example:}\\\\
	Calculate the monthly proportional rate $t_p\%$ (i.e.: the monthly rate of return) of an annual rate $t\%$ of $12\%$:
	
	\end{tcolorbox}
	\textbf{Definition (\#\mydef):} The "\NewTerm{equivalent rate $t_e\%$}" is is bringing to a same capital, during the same period, the same "\NewTerm{compound interest}" (see definition of compound interest below), and given by the relation:
	
	and vice versa:
	
	\begin{tcolorbox}[colframe=black,colback=white,sharp corners]
\textbf{{\Large \ding{45}}Example:}\\\\
	Monthly equivalent rate $t_e\%$ to an annual rate of $12\%$ (result truncated at ten-thousandth):
	
	the reverse procedure would therefore be to calculate the annualized rate and then we see that a monthly rate of $1\%$ annualized would be not exactly $12\%$!!
	\end{tcolorbox}
	
	\subsubsection{Simple Interest}
	\textbf{Definition (\#\mydef):} The "\NewTerm{simple interest}" is defined by the relations (see above for the definition of notations):
	
	implying a capitalization (final value):
	
	It is simply the interest that is calculated each period on the sole basis of the capital loaned or borrowed originally.
	\begin{tcolorbox}[title=Remarks,colframe=black,arc=10pt]
	\textbf{R1.} It is very easy from the knowledge of three of the four parameters of the above relation to find the fourth one. As this is a simple equation of the first degree, we will not dwell on this kind of elementary algebra exercise.\\
	
	\textbf{R2.} A specificity of the simple interest is to be proportional to the duration of the investment. If interest e.g. over a year is $12\%$, the "equivalent rate $t_e\%$" of an identical placement for $12$ months will be $1\%$ per month. This property is not true for compound interest as we will see afterwards.\\
	
	\textbf{R3.} For savings accounts, we have already mentioned that some financial institutions use as fifteen time period (24 periods in the year consisting of fictive 30-day months). So to calculate the annual interest at each fortnight, they take the lowest balance on the account during the fortnight and calculate simple interest on a reported rate of $24$ fortnights a year and defer the result obtained in annual closure of the account at the end of the year (they are not crazy...).
	\end{tcolorbox}
	Moreover, if more simple interests investments are made simultaneously for different periods and at different rates, we may calculate the average $T$ of all of these investments.
	
	If we note $C_t$  the investment number $k$, $I_k$ the interest rate of investment number number $k$, $n_k$ the duration (number of periods) of the investment and $N$ the number of investments, we obtain the following weighted arithmetic mean (\SeeChapter{see section Statistics page \pageref{weighted average by the classes frequencies}}) as follows:
	
	\begin{tcolorbox}[colframe=black,colback=white,sharp corners]
\textbf{{\Large \ding{45}}Example:}\\\\
	We seek the average annual rate of three placements $\left\lbrace1000.-; 90 \text{ days}; 3\%\right\rbrace$, $\left\lbrace2000.-; 120 \text{ days}; 4\%\right\rbrace$, $\left\lbrace3000.-; 170 \text{ days}; 5\%\right\rbrace$. Then we have:
	
	\end{tcolorbox}
	
	\pagebreak
	\paragraph{Discounts}\mbox{}\\\\
	Still regarding to the simple interest, we can come back to a concept we talked about at the beginning of this section: the discount. This concept is used both in the traditional trade as in the financial markets for Treasury bills (see further below). 

	Let us recall that the discount is a rebating granted to a buyer by a seller in order to induce him to pay quickly before $n$ units (periods) of time (that is the number of units $n$ that matters!). A buyer should in principle take advantage of this discount (which can be seen as a credit). Otherwise, it is as if the purchaser implicitly borrowed from a bank for a given period ($n$ units). 

	Historically, the discount is the withholding that does a banker on a promise of payment that has a merchant (promise signed by a buyer for a certain date), that the banker pays before maturity (thus as an advance). 

	Let us see this:
	
	Let us note $C_0$ the present value discount included, $C_n$ the amount without discount named "\NewTerm{nominal value}", $n$ the duration relative to the time scale of the discount rate, $t\%$ the discount rate and $I$ the implicit simple equivalent interest in case of renunciation of the discount.
	
	We now have the following trivial relations:
	
	And nothing prevents us from writing that the amount without discount can be obtained by simple interest of the form:
	
	from the value with discount. The simple implicit interest on the present value being then trivially:
	
	Therefore, it comes by substitution:
	
	We notice then that it is sufficient to know the discount rate $t\%$ granted (often annualized...) and the renunciation period $n$ to determine the simple equivalent rate of the granted credit. 

	If the rate $t\%$ communicated from the start is not the rate corresponding to the entire duration of $n$ (as is the case above) but the rate of the corresponding time unit, then we have the following relation:
	
	\begin{tcolorbox}[colframe=black,colback=white,sharp corners]
\textbf{{\Large \ding{45}}Example:}\\\\
	Let us calculate the implicit simple rate $i$ of a discount of $1\%$ to $10$ days or net at $30$ days:
	
	Thus, the discount if it is not considered, can be seen as being a credit to of $0.0505\%$ daily during 20 days on the sum with discount! This corresponds on a month of $30$ days to a simple interest rate of $1.515\%$.
	\end{tcolorbox}
	This calculation method is named "\NewTerm{commercial discount}" because the calculations are based on the nominal value and not of the current value. However, historically, the calculations were rather similar to the following:
	\begin{tcolorbox}[colframe=black,colback=white,sharp corners]
\textbf{{\Large \ding{45}}Example:}\\\\
	On March 19, a banker discount of $t\%=6\%$ (annualized) a buyer contract to a trader of $240'000.-$ payable on May 31 (so there are $73$ calendar days between the two dates). The discount value is then (on an annual basis of 360 days):
	
	So the present value of the contract will be 237,080. The simple daily implicit rate will be given by:
	
	Therefore:
	
	\end{tcolorbox}
	
	\subsubsection{Compound Interest}
	\textbf{Definition (\#\mydef):} The "\NewTerm{compound interest}\index{compound interest}\label{compound interest}" is defined by the relation:
	
	and implies that the final value is given by:
	
	We say that the interest rate is "compound" when at the end of each period the interest is added to capital for the calculation of the next period. This relation obviously implies that there is no withdrawal or deposit in all periods.
	
	We also deduce the following trivial relations (\SeeChapter{see section Calculus page \pageref{calculus} and Function Analysis page \pageref{logarithms}}):
	
	where the first relation gives the initial value if the final value is known and that there is no withdrawal or deposit during all periods.
	
	\begin{tcolorbox}[title=Remark,colframe=black,arc=10pt]
	The equivalent function in Microsoft Office Excel 11.8346 are respectively to find $C_n,C_0,n,i$ the function \texttt{FV(), PV (), NPER (), RATE()} where the abbreviation \texttt{NPER} means "Number of PERiodic payments".	
	\end{tcolorbox}
	
	In the context of cumulative interest (compounds), two important concepts are the "\NewTerm{present value PV}" and "\NewTerm{future value FV}" of an acquired a capital.
	
	By answering the question: «\textit{What capital $C_n$ do we get after some time today placing an amount $C_0$ in a savings account?}» we make a research of future value of capital. We speak then of "\NewTerm{capitalization process}".
	
	By cons, if we ask, «\textit{What capital $C_0$ should we place today on a savings account to get after some time a capital $C_n$?}», We make a research of present value of a capital. We speak then about "\NewTerm{discount operation}" (that is characteristic of "\NewTerm{actuarial calculations}")
	
	\textbf{Definition (\#\mydef):} We name the $f_c$ the "\NewTerm{capitalization factor}" and $f_e$  the "\NewTerm{discount factor}\index{discount factor}\label{discount factor}" defined by the relations:
	
	also sometimes denoted by $\text{DF}$.
	
	This brings us also to have:
	
	Starting from the compound capitalization relation give above but with the condensed notation we get:
	
	The initial capital (present value) can be therefore expressed with the discount factor:
	
	This then makes very simple the calculation of the actualization or capitalization as it is to multiply the final or initial capital by the discount factor or capitalization factor to the $n$-th power. So as we already know we talk about calculate the "\NewTerm{future value FV}" or "\NewTerm{present value PV}" when we invest a certain amount for a certain time at a certain rate and we speak of "present value" when we calculate what should be the initial value investing to get a given amount after a given time at a given rate.
	
	Let us now recall the relation that we have obtained in our initial presentation of equivalent rates:
	
	Often in order to simplify the calculations, the person seeking the equivalent rate will simply ask (normalized) $n=1$. This led him to write:
	
	Then comes a small trick that use credit analysts in their sales processes that are the concepts of "\NewTerm{effective rate}" (already see above!) and "\NewTerm{nominal rate}". The nominal rate is always lower than the actual rate, it allows the issuer of the debt to show a lower rate than what it actually is (incidentally this practice is prohibited by law in some countries!).
	
	\begin{tcolorbox}[colframe=black,colback=white,sharp corners]
\textbf{{\Large \ding{45}}Example:}\\\\
	Suppose that the conditions for a loan are as follows: annual interest $t_n=12\%$ (nominal rate written in small in the contract or in the add) payable by monthly increments of $t_e\%=t_n\%/12=1\%$. A careful individual realizes that paying 1\% monthly interest in a compound system does not give an annual interest of 12\% but:
	
	which is the effective rate $t_e\%$!
	\end{tcolorbox}
	\begin{tcolorbox}[title=Remark,colframe=black,arc=10pt]
	Warning! Never forget that the nominal rate includes the rate of inflation $t_i\%$. So the "\NewTerm{real rate}" is equal in reality to:
		
	\end{tcolorbox}
	Now, if more investments are made simultaneously for different durations and at different rates, we may calculate the average compound rate $T$ of all of these investments.
	
	If we denote by $C_t$ the investment number $t$, $f_{c,t}$ the capitalization factor of investment umber $t$, $n_t$ the duration of investment number $t$, $k$ the number of investments and finally $T$ the average compound rate of all investments we can with the help of algebra rule until fourth degree (\SeeChapter{see section Calculus}) or using numerical method (taking the solver or target tool of Microsoft Excel for example), solve the following equation:
	
	If we make a change of variables $x=1+T$ we then solve the unknown equation in $x$ (all other terms are normally known in the problem statement):
	
	
	\subsubsection{Continuous Interest}
	Let us recall that the compound interest is defined by (with a different common notation for $f_c$):
	
	For $n$ periods on a given time basis. Now consider an equivalent rate splitted in smaller periods $p$ such that: 
	
	We can now ask ourselves what would happen to the rate $t_n\%$ if interest was paid not monthly, not daily, but continuously, in an instantaneous way (or almost instantaneous). We write therefore (\SeeChapter{see section Functional Analysis page \pageref{Euler number}}):
	
	Thus, in case of continuous capitalization, the function of capitalization can be finally written:
	
	And as:
	
	We then have (relatively important relation in finance markets):
	
	Be with this approximation! In practice we can use this approximation for large sums of money if the rate is relatively low (under $10\%$) and that the number of investment period not exceed the ten...
	
	A useful property and widely used in trading of the continuous interest is that it is  "\NewTerm{time consistent}". To see what it is, consider the following three values of any portfolio of equity invested:
	\begin{table}[H]
	\begin{center}
		\definecolor{gris}{gray}{0.85}
			\begin{tabular}{|p{2cm}|r|p{4.5cm}|p{4cm}|}
				\hline
\multicolumn{1}{c}{\cellcolor{black!30}\textbf{Capital}} & \multicolumn{1}{c}{\cellcolor{black!30}\textbf{Value}} & \multicolumn{1}{c}{\cellcolor{black!30}\textbf{Simple return rate}} & \multicolumn{1}{c}{\cellcolor{black!30}\textbf{Continuous return rate}} \\ \hline
		$C_{n-2}$ & 10.00.- & - & -  \\ \hline
		$C_{n-1}$  & 14.00.- & $= (14-10)/10 = 40\%$ & $\ln\left(\dfrac{14}{10}\right)\cong 33.65\%$\\ \hline
		$C_n$ & 9.00.- & $= (9-14)/14 = -35.71\%$ & $\ln\left(\dfrac{9}{14}\right)\cong -44.18\%$\\ \hline
		Sum: &  & $= 4.29\%$ & $=-10.54\%$ \\ \hline
		Comparison: & & $(9-10)/10 = 10\%$ &  $\ln\left(\dfrac{9}{10}\right)\cong 10.54\%$\\ \hline
	\end{tabular}
	\end{center}
	\caption{Comparison between exact rate and approximated continuous}
	\end{table}
	So as we can see, unlike the simple interest rate, the continuous rate of return is time-consistant in time because we can sum the different rates for the total variation. This result simply comes from the following property:
	
	
	\subsection{Progressive interest (annuities)}
	\textbf{Definition (\#\mydef):} A "\NewTerm{pension}" or "\NewTerm{annuity}" is a series of periodic payments at regular time intervals and for a period fixed in advance with compound Interest (typical of the second or third pillar in Switzerland).
	
	Then it is enough to apply the relation (see above proof) $C_0=C_nf_e^n$ to each annuity term paid if we wish to know the present value of the annuity.
	
	By cons, if we wish to obtain the final value of an annuity, we will apply to each term the following relation (see above the proof):
	
	\textbf{Definition (\#\mydef):} If the annuity is payable at the period end, it is named "\NewTerm{postnumerando annuity}". By cons, if payable a the beginning of period, it is named  "\NewTerm{praenumerando annuity}".
	\begin{tcolorbox}[title=Remarks,colframe=black,arc=10pt]
	\textbf{R1.} Annuities that are always paid are named  "\NewTerm{certain annuities}" and when the duration is predetermined, we speak of "\NewTerm{temporary annuities}".\\
	
	\textbf{R2.} Annuities based on the life of an individual are named "\NewTerm{life annuities}".
	\end{tcolorbox}
	Since the terms are often assumed to be constant, we have the habit of basing the calculations on the value of a currency unit (1\$). Thus, we note (the adopted notation is that we find in the literature and as they are not very practical, they are original and pretty to look at ...):
	
	\begin{itemize}
		\item $a_{\lcroof{n}}$ the present value of an annuity of a monetary unit payable postnumerando (at deadline) for a duration of $n$ periods
		\item $\ddot{a}_{\lcroof{n}}$ the present value of an annuity of a monetary unit payable praenumerando (in advance) for a duration of $n$ periods
		\item $s_{\lcroof{n}}$ the future value of an annuity of a monetary unit payable postnumerando (at deadline) for a duration of $n$ periods
		\item $\ddot{s}_{\lcroof{n}}$ the future value of an annuity of a monetary unit payable praenumerando (in advance) for a duration of $n$ periods
	\end{itemize}
	
	The relations that determine these values make use of the properties of geometric series and their partial sum (\SeeChapter{see section Sequences and Series page \pageref{geometric sequence} and page \pageref{partial sum}}).
	
	\subsubsection{Postnumerando annuities}
	At constant term, to calculate the final value of an annuity at maturity/postnumerando,  we can therefore only work on the capitalization factor multiplied by the final annuity amount.
	
	\begin{tcolorbox}[colframe=black,colback=white,sharp corners]
\textbf{{\Large \ding{45}}Example:}\\\\
	We want to calculate the final value of a postnumerando annuity of $3,500.-$ paid during $10$ periods and calculated with the periodic effective interest rate of $6\%$.\\
	
	Payments held on dates $1, 2,$ and $10$. The payment $C_0=3500$ of date $1$ has for acquired value on the date $10$: $C_0(1+t\%)^9$. Similarly, the payment at date $2$ earns interest for $8$ years. Its value gained at date $10$ is therefore : $C_0(1+t\%)^8$, etc. Finally, at the end, the payment at date $10$ (which we have just put in the bank) has the value $C_0$. The acquired value of the $10$ payments is therefore, setting $f_c=1+t\%$ (we will prove the simplifcations right after):
	
	In  Microsoft Excel 14.0.7106 we simply write:
	\begin{center}
		\texttt{=3500*FV(6\%,10,-1,0,0)=46,132.80.-}
	\end{center}
	\end{tcolorbox}
	Therefore the postnumerando annuity is a payment with constant terms and at constant rates over a number of given periods leading to a simple geometric sequence.
	
	Let us recall (again!) that:
	
	For the postnumerando annuity with constant terms, we then have the general form\label{power sum in finance}:
	
	Which is written:
	
	Therefore:
	
	In fact, we have a geometric series of reason $q$ (\SeeChapter{see section Sequences And Series page \pageref{geometric sequence}}). Since then:
	
	Therefore:
	
	Finally:
	
	\begin{tcolorbox}[colframe=black,colback=white,sharp corners]
\textbf{{\Large \ding{45}}Example:}\\\\
	So we have for your example ten periods (and therefore ten terms with $n=10$):
	
	This capital corresponds then to the sum gained after ten periods.
	\end{tcolorbox}
	The method of calculating the "present value of an annuity at maturity/postnumerando" works on the same principle but in reverse according to the relation proved above:
	
	So if the terms (amounts paid) are constant we can write:
	
	Therefore:
	
	Or:
	
	Therefore:
	
	Finally:
	
	\begin{tcolorbox}[title=Remark,colframe=black,arc=10pt]
	The value $C_0=C_n a_{\lcroof{n}}$ correspond therefore to the amount we would need to put on a savings account at the rate of $t\%$to be able to make a constant periodic removal $C_n$ during $n$ periods and thereby settle the account. 
	\end{tcolorbox}
	\begin{tcolorbox}[colframe=black,colback=white,sharp corners]
\textbf{{\Large \ding{45}}Example:}\\\\
	Let us calculate the present value of a postnumerando  annuity of $3,500.-$ paid during $10$ years at the rate of $6\%$ interest:
	
	Or in Microsoft Excel 14.0.1706:
	\begin{center}
		\texttt{=3500*PV(6\%,10,-1,0)=25,760.30.-}
	\end{center}
	\end{tcolorbox}
	We also have the following relations between actual and final postnumerando rents:	
	
	We also have the following chain rule operations:
	
	\begin{tcolorbox}[title=Remark,colframe=black,arc=10pt]
	It is clear that since known $a_{\lcroof{n}},f_e^n,t\%$ that $n=\dfrac{\ln(1-t\% a_{\lcroof{n}})}{\ln(f_e)}$ and so on for all other types of annuity.
	\end{tcolorbox}
	
	\subsubsection{Praenumerando annuities}\label{praenumerando annuities}
	The method of calculating the "\NewTerm{present value of an annuity in advance/praenumerando}" works on the same principle as the last one still using the relation $C_0=C_nf_e^n$, but this time the terms of the geometric sequence change since the payment is done in advance:
	
	Therefore:
	
	But:
	
	Therefore:
	
	Finally:
	
	\begin{tcolorbox}[title=Remark,colframe=black,arc=10pt]
	For the same number of periods and the same interest rate, we obviously have $\ddot{a}_{\lcroof{n}} \geq {a}_{\lcroof{n}}$ as  $\ddot{a}_{\lcroof{n}} =f_e{a}_{\lcroof{n}}$.
	\end{tcolorbox}
	\begin{tcolorbox}[colframe=black,colback=white,sharp corners]
\textbf{{\Large \ding{45}}Example:}\\\\
	Let us calculate the present value of a preanumerando annuity of $3,500.-$ paid during $10$ years at the rate of $6\%$ interest:
	
	Or in Microsoft Excel 14.0.1706:
	\begin{center}
		\texttt{=3500*PV(6\%,10,-1,0,1)=27,305.92.-}
	\end{center}
	\end{tcolorbox}
	The method of calculating the "final value of a rent in advance/praenumerando" works on the same principle as the last one using the relation $C_n=C_0f_c^n$ but this time the terms of the geometric sequence change since the payment is done in advance:
	
	Therefore:
	
	As we have proved in the section Sequences and Series (page \pageref{geometric series}) during our study of geometric series that:
	
	Therefore:
	
	finally noting $d=\dfrac{t\%}{1-t\%}$ we have:
	
	\begin{tcolorbox}[title=Remarks,colframe=black,arc=10pt]
	\textbf{R1.} With the same notation we have also the prensent value of the praenumerando annuity that is written $\ddot{a}_{\lcroof{n}}=\dfrac{1-f_e^n}{d}$.\\
	
	\textbf{R2.} For the same number of periods and the same interest rate, we obviously have $\ddot{s}_{\lcroof{n}} \geq {s}_{\lcroof{n}}$ as  $\ddot{s}_{\lcroof{n}} =f_c{s}_{\lcroof{n}}$.
	\end{tcolorbox}
	\begin{tcolorbox}[colframe=black,colback=white,sharp corners]
\textbf{{\Large \ding{45}}Example:}\\\\
	Let us calculate the final value of a preanumerando annuity of $3,500.-$ paid during $10$ years at the rate of $6\%$ interest:
	
	Or in Microsoft Excel 14.0.1706:
	\begin{center}
		\texttt{=3500*FV(6\%,10,-1,0,1)=48,900.75.-}
	\end{center}
	\end{tcolorbox}
	
	\subsection{Rounding}
	Rounding a numerical value means replacing it by another value that is approximately equal but has a shorter, simpler, or more explicit representation. Rounding is often done to obtain a value that is easier to report and communicate than the original. Rounding can also be important to avoid misleadingly precise reporting of a computed number, measurement or estimate; for example, a quantity that was computed as 123,456 but is known to be accurate only to within a few hundred units is better stated as "about 123,500".
	
	On the other hand, rounding of exact numbers will introduce some round-off error in the reported result. Rounding is almost unavoidable when reporting many computations — especially when dividing two numbers in integer or fixed-point arithmetic; when computing mathematical functions such as square roots, logarithms, and sines; or when using a floating point representation with a fixed number of significant digits. In a sequence of calculations, these rounding errors generally accumulate, and in certain ill-conditioned cases they may make the result meaningless.
	
	To round a number $x$ to the nearest multiple of $1 / n$ the relation to be used is as follows:
	
	The proof is almost intuitive. Just imagine the real axis and cut it into $1/n$ small intervals. Then $x$ is a given number, the number of these intervals will in $x$ given by:
	
	Finally, to know the number strictly less than the desired multiple, we take the integer value of the last relation and multiply it by $1 / n$ such that:
	
	If, however, we wish to have the number rounded to the nearest multiple, then we see with a linear regression that it is necessary to add $0.5$ as:
	
	To round to the upper multiple we add one such that:
	
	OK... this seems simple but there's a lot more to rounding than might at first meet the eye...:
	\begin{itemize}
		\item Round-Toward-Nearest
		\item Round-Half-Up (Arithmetic Rounding)
		\item Round-Half-Down
		\item Round-Half-Even (Banker's Rounding)
		\item Round-Half-Odd
		\item Round-Ceiling (Positive Infinity)
		\item Round-Floor (Negative Infinity)
		\item Round-Toward-Zero
		\item Round-Away From-Zero
		\item Round-Up
		\item Round-Down
		\item Truncation (Chopping)
		\item Round-Alternate
		\item Round-Random (Stochastic Rounding)
		\item Sign-Magnitude Binary Values
		\item Rounding Signed Binary Values
	\end{itemize} 
	For business training purpose, among this list, let us just introduce the "\NewTerm{Banker's rounding}":

	If values are always rounded in the same direction (for example, if $+5.5$ rounds up to $+6$ and $+6.5$ rounds up to $+7$, as is the case with the "\NewTerm{Round-Half-Up}" algorithm presented earlier), the result can be a bias that grows as more and more rounding operations are performed. One solution toward minimizing this bias is to sometimes round up and sometimes round down.

	In the case of the round-half-even algorithm (which is often referred to as "Bankers Rounding" because it is commonly used in financial calculations), "half-way" values are rounded toward the nearest even number. Thus, $+5.5$ will round up to $+6$ and $+6.5$ will round down to $+6$.

	This algorithm is, by definition, symmetric for positive and negative values, so both $-5.5$ and $-6.5$ will round to the nearest even value, which is $-6$.

	In the case of data sets that feature a relatively large number of "half-way" values (financial records often provide a good example of this), the round-half-even algorithm performs significantly better than the Round-Half-Up scheme in terms of total bias (it is for this reason that the use of round-half-even is a legal requirement for many financial calculations around the world!).
	
	As surprising as it can be, a spreadsheet software like Microsoft Excel seems to not implement by default a simple function to do such a rounding... (when we know that this a legal requirement).

	
	\pagebreak
	\subsection{Loans Amortization/Repayments}
	Individuals and companies often use loans (credit) as a financial way to found money. Here we will define the main types of loans encountered in practice and the mathematical relations that characterize them.
	
	\textbf{Definitions (\#\mydef):}
	
	\begin{enumerate}
		\item[D1.] In finance, a "\NewTerm{loan}" is a debt provided by an entity (organization or individual) to another entity at an interest rate, and evidenced by a note which specifies, among other things, the principal amount, interest rate, and date of repayment. A loan entails the reallocation of the subject asset(s) for a period of time, between the lender and the borrower.
	
		\item[D2.] We name "\NewTerm{indivis loan}", a loan with only one lender, usually a financial institution.
		
		\item[D3.] As for equity management (see further below) a "\NewTerm{guarantor}" is the party exposed to the loan risk, but he don not present credit risk itself. We thus avoid the problem of two-sided credit risk.
	\end{enumerate}
	\begin{tcolorbox}[title=Remark,colframe=black,arc=10pt]
	When talking in financial terms, a loan is usually followed by an interest on the amount unlike credit, which might or might not be followed by an interest. As they seems not to be a consensus on the difference between loan and credit some people say that credit grants them the ability to borrow, while a loan is actually the act of borrowing. If a bank gives me a $500.-$ line of credit, I can, at some date in the future when I want to, borrow money from them. If a bank gives me a $500.-$ loan, I take possession of them money today.
	\end{tcolorbox}	

	In a loan, the borrower initially receives or borrows an amount of money, named the "principal", from the lender, and is obligated to pay back or repay an equal amount of money to the lender at a later time.

	The major variables in a mortgage calculation include loan principal, balance, periodic compound interest rate, number of payments per year, total number of payments and the regular payment amount. More complex calculators can take into account other costs associated with a mortgage, such as local and state taxes, and insurance.
	
	When purchasing a new home, most buyers choose to finance a portion of the purchase price via the use of a mortgage. Prior to the wide availability of mortgage calculators, those wishing to understand the financial implications of changes to the five main variables in a mortgage transaction were forced to use compound interest rate tables. These tables generally required a working understanding of compound interest mathematics for proper use. In contrast, mortgage calculators make answers to questions regarding the impact of changes in mortgage variables available to everyone.

	Mortgage calculators can be used to answer such questions as:

	If one borrows $250,000.-$ at a $7\%$ annual interest rate and pays the loan back over thirty years, with $3,000.-$ annual property tax payment, $1,500.-$ annual property insurance cost and $0.5\%$ annual private mortgage insurance payment, what will the monthly payment be? The answer is $2,142.42.-$.
	
	Mainly we consider two type of loans:
	\begin{enumerate}
		\item[D1.] A "\NewTerm{secured loan}" is a loan in which the borrower pledges some asset (e.g. a car or property) as collateral.
		\item[D2.] An "\NewTerm{unsecured loan}" is monetary loan that is not secured against the borrower's assets. These may be available from financial institutions under many different guises or marketing packages (credit card, peer-to-peer landing, etc.)
	\end{enumerate}
	The main points of interest on loans are: 
	\begin{itemize}
		\item Know the state of the debt at any time
		\item Know the amount repayable in each period
		\item Know the interest due each period
	\end{itemize} 
	
	\textbf{Definition (\#\mydef):} We name  "\NewTerm{annuities}", payments made in the context of the clearance of a loan. An annuity includes a part of refund $R$ also named "\NewTerm{financial depreciation}" and part $I$ of interest according to the relation:
	
	The decomposition of the annuity in a depreciation part and interest is an important concept not only in finance but also in accounting. Indeed, the part of financial amortization corresponds to a debt repayment at the difference of the interest that is a financial burden.
	
	We will study here three types of loans:
	\begin{enumerate}
		\item Loan with fixed maturity
		\item Loans with constant repayments
		\item Loans with constant annuities (the most practiced)
	\end{enumerate}
	
	\begin{tcolorbox}[title=Remark,colframe=black,arc=10pt]
	\textbf{R1.} We consider here periodic loans. The transition from one temporal period to another and the calculation of an equivalent rate will be based on relation already proved before.\\
	
	\textbf{R2.} Stable monthly annuities are named obviously "monthly payments".
	\end{tcolorbox}
	
	\subsubsection{Fixed-Term Loan}
	\textbf{Definition (\#\mydef):} We speak about "fixed-term loan" or "in fine loan" when at each period the annuity includes only the part of interest! The last period the annuity includes interest and all (!) the repayment of the loan.
	
	\begin{tcolorbox}[title=Remark,colframe=black,arc=10pt]
	This model of amortization is particularly used in bonds that we will study later.
	\end{tcolorbox}
	The following relations make possible to establish any element of the "amortization table".
	
	Thus, the status of debt (loan) $C$ on $n$ years is at the beginning of the year $k$ is always:
	
	The epayment (amortization) $R_k$ performed at the en of the year $k$ is equal to the accumulated depreciation $S_k$ at the end of the year $k$ and as it is only made last year $n$ such that:
	
	The paid $I_k$ will be constant throughout the loan term according to an interest rate $t\%$ on borrowed capital $C$ as (simple interest):
	
	The annuity becomes therefore:
	

	\begin{tcolorbox}[colframe=black,colback=white,sharp corners]
\textbf{{\Large \ding{45}}Example:}\\\\
	Let us see the amortization table of a loan of $1,000.-$ at $10\%$ year repaid at maturity after $4$ years. The corresponding amortization table will be:
	\begin{table}[H]
	\begin{center}
		\definecolor{gris}{gray}{0.85}
			\begin{tabular}{|p{1.4cm}|p{1.4cm}|p{1.8cm}|p{1.9cm}|p{1.5cm}|p{2cm}|}
				\hline
				\multicolumn{1}{c}{\cellcolor{black!30}\textbf{Period $k$}} & 
  \multicolumn{1}{c}{\cellcolor{black!30}\textbf{Debt $C_k$}}  & 
  \multicolumn{1}{c}{\cellcolor{black!30}\textbf{Amort. $R_k$}}  & 
  \multicolumn{1}{c}{\cellcolor{black!30}\textbf{Cumulated Amort. $S_k$}}  & 
  \multicolumn{1}{c}{\cellcolor{black!30}\textbf{Interest $I_k$}}  & 
  \multicolumn{1}{c}{\cellcolor{black!30}\textbf{Annuity $A_k$}}    \\ \hline
				\centering\arraybackslash\ $1$ & \centering\arraybackslash\ $1,000$ & \centering\arraybackslash\ $0$ & \centering\arraybackslash\ $0$ & \centering\arraybackslash\ $100$ & \centering\arraybackslash\ $100$ \\ \hline
				\centering\arraybackslash\ $2$ & \centering\arraybackslash\ $1,000$ & \centering\arraybackslash\ $0$ & \centering\arraybackslash\ $0$ & \centering\arraybackslash\ $100$ & \centering\arraybackslash\ $100$ \\ \hline
				\centering\arraybackslash\ $3$ & \centering\arraybackslash\ $1,000$ & \centering\arraybackslash\ $0$ & \centering\arraybackslash\ $0$ & \centering\arraybackslash\ $100$ & \centering\arraybackslash\ $100$ \\ \hline
				\centering\arraybackslash\ $4$ & \centering\arraybackslash\ $1.000$ & \centering\arraybackslash\ $1,000$ & \centering\arraybackslash\ $1,000$ & \centering\arraybackslash\ $100$ & \centering\arraybackslash\ $1,100$ \\ \hline
		\end{tabular}
	\end{center}
	\caption{Fixed-Term Loan amortization table}
	\end{table}		
	The cost of credit is the sum of interest that is $400.-$.
	\end{tcolorbox}
	
	\subsubsection{Loan with constant amortization}
	\textbf{Definition (\#\mydef):} We speak of a "\NewTerm{constant amortization loan}" when the annual amount repaid (amortization) is constant, that is to say identical from year to year (intuitive case).
	
	The following relation make possible to establish any element of this amortization table (no more details are needed because it is very trivial):
	
	\begin{tcolorbox}[colframe=black,colback=white,sharp corners]
\textbf{{\Large \ding{45}}Example:}\\\\
	A loan of $1,000.-$ at $10\%$ per year is reimbursed by constant amortization in $4$ years. The corresponding amortization table will be:
	\begin{table}[H]
	\begin{center}
		\definecolor{gris}{gray}{0.85}
			\begin{tabular}{|p{1.4cm}|p{1.4cm}|p{1.8cm}|p{1.9cm}|p{1.5cm}|p{2cm}|}
				\hline
				\multicolumn{1}{c}{\cellcolor{black!30}\textbf{Period $k$}} & 
  \multicolumn{1}{c}{\cellcolor{black!30}\textbf{Debt $C_k$}}  & 
  \multicolumn{1}{c}{\cellcolor{black!30}\textbf{Amort. $R_k$}}  & 
  \multicolumn{1}{c}{\cellcolor{black!30}\textbf{Cumulated Amort. $S_k$}}  & 
  \multicolumn{1}{c}{\cellcolor{black!30}\textbf{Interest $I_k$}}  & 
  \multicolumn{1}{c}{\cellcolor{black!30}\textbf{Annuity $A_k$}}    \\ \hline
				\centering\arraybackslash\ $1$ & \centering\arraybackslash\ $1000$ & \centering\arraybackslash\ $250$ & \centering\arraybackslash\ $250$ & \centering\arraybackslash\ $100$ & \centering\arraybackslash\ $350$ \\ \hline
				\centering\arraybackslash\ $2$ & \centering\arraybackslash\ $750$ & \centering\arraybackslash\ $250$ & \centering\arraybackslash\ $500$ & \centering\arraybackslash\ $75$ & \centering\arraybackslash\ $325$ \\ \hline
				\centering\arraybackslash\ $3$ & \centering\arraybackslash\ $500$ & \centering\arraybackslash\ $250$ & \centering\arraybackslash\ $750$ & \centering\arraybackslash\ $50$ & \centering\arraybackslash\ $300$ \\ \hline
				\centering\arraybackslash\ $4$ & \centering\arraybackslash\ $250$ & \centering\arraybackslash\ $250$ & \centering\arraybackslash\ $1,000$ & \centering\arraybackslash\ $25$ & \centering\arraybackslash\ $275$ \\ \hline
		\end{tabular}
	\end{center}
	\caption{Constant amortization loan table}
	\end{table}		
	The cost of credit is the sum of interest that is $250.-$. So we pay less than with the previous system.
	\end{tcolorbox}
	
	\subsubsection{Loan with constant annuity}
	This is the most common case (the definition is in the title itself). It is used by most small credit and leasing institutions. The borrower knows in advance the amount he will have to pay each year. In other words, it is as if it were a capital $C$ that must be settle each period by a constant withdrawal $A$: which consists to determine the present value of an postnumerando annuity such that:
	
	The following relations are then used to establish any element of the amortization table:
	
	and because $C=A a_{\lcroof{n}}$, then:
	
	therefore, when $k=1$, we have in accordance with what we expect $C_1=A a_{\lcroof{n}}$.
	
	And therefore the amortization is equal to:
	
	
	The cumulated depreciation is a little less obvious to find with common sense, let us take for for the proof a depreciation $A$ with a rate $t\%$ on $n$ periods. Then we have by definition:
	
	with $k=2$ and $n=3$ we get:
	
	Therefore:
	
	Therefore:
	
	and also:
	
	\begin{tcolorbox}[colframe=black,colback=white,sharp corners]
\textbf{{\Large \ding{45}}Example:}\\\\
	A loan of $1,000.-$ at $10\%$ per year is reimbursed by constant annuity in $4$ years. The corresponding amortization table is:
	\begin{table}[H]
	\begin{center}
		\definecolor{gris}{gray}{0.85}
			\begin{tabular}{|p{1.4cm}|p{1.4cm}|p{1.8cm}|p{1.9cm}|p{1.5cm}|p{2cm}|}
				\hline
				\multicolumn{1}{c}{\cellcolor{black!30}\textbf{Period $k$}} & 
  \multicolumn{1}{c}{\cellcolor{black!30}\textbf{Debt $C_k$}}  & 
  \multicolumn{1}{c}{\cellcolor{black!30}\textbf{Amort. $R_k$}}  & 
  \multicolumn{1}{c}{\cellcolor{black!30}\textbf{Cumulated Amort. $S_k$}}  & 
  \multicolumn{1}{c}{\cellcolor{black!30}\textbf{Interest $I_k$}}  & 
  \multicolumn{1}{c}{\cellcolor{black!30}\textbf{Annuity $A_k$}}    \\ \hline
				\centering\arraybackslash\ $1$ & \centering\arraybackslash\ $1000$ & \centering\arraybackslash\ $215$ & \centering\arraybackslash\ $215$ & \centering\arraybackslash\ $100$ & \centering\arraybackslash\ $315$ \\ \hline
				\centering\arraybackslash\ $2$ & \centering\arraybackslash\ $785$ & \centering\arraybackslash\ $237$ & \centering\arraybackslash\ $452$ & \centering\arraybackslash\ $78$ & \centering\arraybackslash\ $325$ \\ \hline
				\centering\arraybackslash\ $3$ & \centering\arraybackslash\ $548$ & \centering\arraybackslash\ $261$ & \centering\arraybackslash\ $713$ & \centering\arraybackslash\ $55$ & \centering\arraybackslash\ $325$ \\ \hline
				\centering\arraybackslash\ $4$ & \centering\arraybackslash\ $287$ & \centering\arraybackslash\ $287$ & \centering\arraybackslash\ $1,000$ & \centering\arraybackslash\ $29$ & \centering\arraybackslash\ $325$ \\ \hline
		\end{tabular}
	\end{center}
	\caption{Constant annuity loan table}
	\end{table}		
	The cost of credit is the sum of interest that is $262.-$. This result could be obtained by $nA-C$.
	\end{tcolorbox}
	\begin{tcolorbox}[title=Remark,colframe=black,arc=10pt]
	Financial institutions add different charges to credit as filing fees, insurance costs, gurantee fees, etc. Adding these fees has the effect of increasing the interest rate whose final real value (all inclusive) is named "\NewTerm{global effective rate}" (GER). Because each financial institution made its small recipes or that each country imposes a particular calculation method (the obligation to communicate the GER and the associated methodology is specified in many countries by legislation to ensure that consumers are not deceived) we did not want to develop the subject in this book. However, the interested reader may refer to Econometrics exercises available on the associated website where the reader will be able to found a concrect example... but particular!
	\end{tcolorbox}
	
	\pagebreak
	\subsection{Modern Portfolio Theory}
	Warning! Topics related to modern portfolio theory are relatively to the number of definitions almost as worst as Thermodynamics. It is the same for the number of assumptions of the theoretical models used therein. Let us also tell you that they are often several vocabulary terms (and also mathematical notations) to describe the same thing in the trade finance (we have no knowledge of an ISO standard in finance vocabulary although one exists to standardize the symbols of financial instruments) and that happens quite regularly that the banks themselves may correct mathematical definitions in their brochures and softwares to align with the majority opinion of the moment... (sign convention, symbols convention, convention of what should be divided by what, convention of what should be subtracted from what, etc.). We have also, in consistence with the other chapters of this book, tried to simplify to the maximum the mathematical proofs (and therefore limiting ourselves to the simplest theoretical bachelor level model...) even taking sometimes shortcuts that will make grind some teeth.... and we have also tried to give each time the different vocabulary terms of use and existing multiple mathematical notations for the same theoretical concept. So good read and good luck!
	
	The theory of stock market also named "\NewTerm{modern portfolio theory}" is the mathematical theory that deals with the prices, selection, management and exchange transactions, loans and capital through time (in fact it is more related to engineering as this theories are build by trial and error and should be named  "\NewTerm{financial engineering}" and be but in the corresponding chapter!). This theory is strongly makes a intensive uage of to statistical models and so it is important to have read and understood the sections of Probabilites and Statistics of this book before. We will see that the majority of models are based on a probabilistic representation and the problem becomes a determination of a price at a future date and this is equivalent to determine the probability distribution of asset prices, these latter being seen as random variables.
	
	It should however be known that in practice in the private or public financial institutions, only a tiny minority of market players knows, understands and applies mathematical models and for the others who obtained certification or continuing education qualifications (CFA, FRM , CAIA, etc.), the mathematical level is often very very low. Indeed the vast majority of traders (market operators) are limited to the graphical analysis of candles stock type diagrams using analysis of moving averages type at the limit based on control charts (\SeeChapter{see section Industrial Engineering page \pageref{quality control charts}}) with Bollinger bands or other USL/LSL, linear regressions and almost other similar techniques (the funniest being empirical index with scientific names like "stochastic index", which are actually just relative variations... ). Financial management is therefore ultimately often only the application of common sense (at least at the end of 20th century...) on the variation of prices based on quantities... analyzing graphs and knowing when to sell or buy at the right moment.
	
	This situation can be easily explained: the current theoretical models of the 20th century and early 21st century are unable to date to take into account the real complexity and interaction of our modern world. You will see in this section that actually the majority of mathematical models studied in major universities involve simplified and idealized cases (independence, unimodal distributions, finite variance, choice of empirical statistical indicators, etc.). So in the current situation it is often better to listen to a trader who is informed of Government and Corporates strategies than to a mathematician to which the world is resume as a fairytale (but nevertheless allows him to sell consulting hours at a very high rate and reassure his customers who are desperate to predictions that have virtually almost no sense).
	
	The reader must also be aware that a number of people are convinced that everything is written somewhere, than a kind of abstract reality exists outside our physical world and that if we were smart enough, we could mathematically formalize and provide future developments in the long term. In fact the scientific knows that in this kind of field we have to deal with the deterministic chaos of the market and that the only way to manage this is to correct with thumbs with a vague idea what will happen. In economics some specialists speak about the "\NewTerm{markter dictatorship}", but this is is recognized in a sense that we know we can predict almost nothing! Obviously some in the complexity business sell to bankers and others the idea that they will predict the fluctuations of the stock market... but it is sufficient to observe the past to see that no modern model could have predicted it. The only thing that math can do in financial management is to analyze the behavior of an ideal financial asset in a idealized context and this is already not bad and forces a little common sense as decision support system (D.S.S.)... (for those who can do math which is very far from the case of 99\% of people working in finance).
	
	In finance, mathematical models therefore are used mainly to quantify and minimize the investment risk. As such, they play the role of decision support system for managers, investors and regulators. But, with rare exceptions, a bank or an investment fund does not base a major investment decision on a mathematical equation. The decision, for investment banks, is often motivated by the search for ever greater returns and for that they are not based on mathematical models. Moreover, the ruling people of private banks or government are often people from the world of politics, law or business management with little skill to really understand how markets work (see the Master Graduate program trading in many countries, for example). Also beware of companies - especially multinationals - seeking financial specialists mastering Microsoft Excel. VBA, Microsoft Access, C\#... Because it means that they use non-professional tools to do a job which ought to be with the appropriate tools (and Microsoft Excel or Microsoft Access are not and C++ is much more faster than C\#)!!! So in terms of internal organization, you can ensure that these companies organize and analyze anything, anyhow, with no suitable tools and therefore that there is internally a general mess.
	
	\begin{tcolorbox}[title=Remark,colframe=black,arc=10pt]
	Many of the techniques presented below are available as complete code in the C++ book \cite{oliveira2015practical} that we strongly recommend to the reader.
	\end{tcolorbox}
	
	This did not stop however, who those that consider mathematics as an art (which I do), that financial theoretical models have a certain elegance and allow to thoroughly understand the working mechanism of financial engineering and that these models have sometimes interesting connections with other fields of Applied Mathematics.
	
	Furthermore, increasingly used by financial institutions, automated trading bots account in 2016 for nearly $60\%$ of global daily transactions, and almost $80\%$ just on Wall Street. The high frequencies trading bot do transaction in $5$ microseconds and this is so quick that even visualization tools as used by humans are not adapted anymore as the human brain needs at most a $5$ seconds zoom out to be able to understand and analyze what happens. What happened is finance was quite easily predictable and probably the same will happen in a near of far future to all qualified jobs that can be quantified (Management, Project Management, Quality Management etc.).
	
	\pagebreak
	It might also be useful to make liable this business by implementing at least the following measures:
	
	\begin{itemize}
		\item Require the make public the technical documentation of components (underlyings), algorithms and mathematical models of financial products.
		
		\item Require a dynamical level of own funds (fractional reserve) to financial institutions based on their market positions and risk classes of assets under management.
		
		\item Forcing actors of the market to operate in an area of activity with a given predefined level of connectivity with the class of managed assets.
		
		\item Limit the number of transactions for a financial asset not limited in time and also impose for this same family of asst a minimum time of non-transaction.
		
		\item Avoid front running or spoofing - form of insider trading\footnote{The FINRA (Financial Industry Regulatory Authority) has a website tool named "BrokerCheck" where investors can obtain more information about any FINRA-registered broker or brokerage firm that have been barred from the securities industry for insider trading!} consisting in pretending buying or selling a security with the intention to cancel the transaction at the last minute - by requiring automated transactions (bypassing humans (brocker / trader )) and if necessary prohibit practicing for life to people cheating and withdraw their academic degrees.
		
		\item Forcing people active in this business to take high level exams regularly to check that they understand what they do and the assumptions relating thereto (as for actuaries in certain insurance)
		
		\item Prohibit deposit banks to do also trading.
		
		\item Closely monitor all certified "risk manager" coming out of training centers for the past 30 years because given the events during those years, it would appear that they do not do their job or if they do, that their recommendations are not implemented at the highest level of management of the banks (and thus their work is totally counterproductive).
	\end{itemize}
	There would be a number of proposals to be implemented to restore its original purpose and ... ethics to the field of financial management.
	
	\textbf{Definitions (\#\mydef):} 
	
	\begin{enumerate}
		\item[D1.] "\NewTerm{Stock Exchange}" is the public market organized and in theory... regulated... .where securities are exchanged (shares, bonds, options, etc.) whose value will fluctuate in relatively to the "\NewTerm{fundamental value}" (basic value calculated using theoretical models) depending on the offer ("ask") and demand ("bid"). When a security is asked too much, the price goes up, and vice versa, when nobody wants it. The difference between the bid and the ask named the "\NewTerm{spread}".
	
	The stock exchange is a structure that allows:
	\begin{enumerate}
		\item For companies that want to invest (and therefore increase their capital) to get funds to meet potential or proven demand.
		
		\item Make the economy more stable by making it more dynamic and fluid as possible (but still under control ...) in order that it regulates itself.
	\end{enumerate}
	The above mentioned system is working if and only if it is transparent, rational, efficient, self-regulating and balanced!
	
	\begin{tcolorbox}[title=Remark,colframe=black,arc=10pt]
	We talk about "\NewTerm{speculative bubble}" when prices observed on a stock market deviate too much from the fundamental value of traded goods.
	\end{tcolorbox}
	
	\item[D2.] We name "\NewTerm{over-the-counter market OTC}"a transaction between two parties that are free (directly between the seller and the buyer) to contract and normally out of the Stock Exchange and that is therefore private, non-organized and unregulated (or in a very flexible way...). Sometimes a broker acts as an intermediary in OTC, but this latter is not a counterpart: it will not intervene in the settlement rules of the transaction. Sometimes a bank offers itself this type of transaction and ensures itself the counterpart, but often by covering their risk in another existing market.
	
	For example, the ForEx currency market (Foreign Exchange) or CDS (Credit Default Swap) are essentially OTC market (daily turnover of several thousand billion!). For example in the context of currencies, a company or a bank that wishes to make a foreign exchange transaction will be put in direct contact with another bank. However, there exist an organized market of currencies.
	
	\item[D3.] An investment is named "\NewTerm{liquid investment}" if the investment involves financial instruments that one can buy or sell at any time respectively. Verbatim a portfolio is named "\NewTerm{liquid portfolio}" if it contains a majority of liquid instruments.
	
	\item[D4.] In a financial institution we speak of "\NewTerm{front office}" when we mean traders who execute transactions, take positions, etc. We talk about "middle office" when we refer to risk managers that follow the risks taken by the front office. Finally, we talk about "\NewTerm{back office}" for any remaining administrative part (management of entries, accounting, administration, etc.).
	
	\item[D5.] Investors are heterogeneous in terms of fortune. There are generally small savers ("\NewTerm{retail}), customers of private banks ("\NewTerm{private banking}" and "\NewTerm{family office}") and very fortunated investors ("\NewTerm{high net worth}").
	\end{enumerate}	
	
	Before starting to tackle pure and hard mathematics, we will need once again to give many definitions to be fluent with the vocabulary in use by financial analysts and financial engineers (caution! the vocabulary list is relatively long...).
	
	\subsubsection{No Arbitrage Opportunity (N.A.O.)}
	One of the fundamental assumptions of the usual financial models is that there is no financial strategy to, for a zero initial cost, acquire a certain fortune in a future date. This hypothesis is named "\NewTerm{no arbitrage opportunities (NAO)}". It is theoretically justified by the uniqueness of prices that characterize a market in perfect competition (\SeeChapter{see section Game Theory page \pageref{perfect information game}}).
	
	In practice, there are arbitrage opportunities but that disappear very quickly due to the existence of arbitrage-seeker, these latter are market agents whose role is to detect this type of opportunities and use them. They then create a force that tends to change the price of the asset to its no-arbitrage price (the "real" price).
	
	Thus, if multiple assets of same risk offer different returns, investors seeking new opportunities will logically turn their purchases to those whose performance is the highest, so this behavior leads to a decline in the performance of these assets (because they become more expensive).
	
	The financial mathematics based on the N.A.O. let these people make money and neglects these opportunities appearances which in any case will always exist but assumed on a very short time (this type of strategy is leveraged today with computers that can give orders sales and purchasing order at the millisecond and even faster).
	
	A remarkable example to illustrate this is to use a simplified version of the model developed by Cox, Ross and Rubinstein as it explicitly reflects the concept of the N.A.O. and the importance of probabilistic models.
	
	The example is based on the assumption that the market is composed of one risky asset and a (non-risky) constant investment rate $r$. For example, a sum of $1.-$ today placed at rate $r$, generates a guaranteed income $1 + r$ at time $1$ regardless of the future market developments in chosen example (by definition of riskless/risk-free investment rate...).
	
	We first study this market over a single period of time such that the initial time will be noted $t=0$ and the final instant $t=1$ (we name this situation a "monoperiodic market"). We assume also that we have a perfect knowledge of the market at the initial time. In our context this means that the price of the risky asset is $S_0>0$ and fixed and the non-risky asset is determined by its performance $r>0$.
	
	And about the risky asset, its value at time $t=1$ is not known in advance. To decrease the field of possibilities, we will assume that the performance of this asset can take only two values $b$ (bottom) and $h$ (high) with:
	
	Thus, the risky asset at time $t=1$ can take only two positive values. The low value:
	
	other the high value:
	
	hence the name "\NewTerm{(naive) binomial model}"...
	
	An investor can then buy a quantity $\delta$ of risky asset and place (lend or borrow!) an - unlimited - amount $n_0$ of money at the assumed risk-free rate $r$. The value $V_0$ at time $t=0$ of the portfolio of composition $(\delta,n_0)$ is therefore:
	
	At the final moment, we will have:
	
	As we explained above, $S_1$ can take two values, it is therefore the same for $V_1$. This means that the income of the portfolio is also uncertain.
	
	Now, to show the concept of N.O.A. let us do a numerical example considering the special situation where it is more advantageous, and without fail, to invest in the risky asset rather that the riskless (risk-free) one.
	
	Imagine for this purpose that we borrowed $100.-$ to a bank to a risk-free rate of $5\%$ and that the risky asset unit we want to acquire with this money is traded today at $S_0=100.-$ and that this latter can take two future values:
	
	We then have for our portfolio at the initial time:
	
	and at the final time two possible cases:
	
	We then see trivially that if $r<b$ there is an arbitrage opportunity as it becomes possible to make surely money without spending any! To avoid a N.O.A. in this situation, it is necessary that the market is balanced and that there is:
	
	Conversely, if it is surely more lucrative to invest in risk-free asset rather than in the risky asset, the market must organized itslef to avoid any arbitrage opportunity to ensure that:
	
	Thus, in both cases, we must avoid at all times that in the (naive) binomial market we have a N.O.A. and this will be possible if:
	
	The absence of arbitrage opportunity to two major simple involvement (there are less simple one as we shall see later...). Consider the case where we have two assets of performance (yield/return) respectively $i$ and $j$ and that we know that the performance of asset $i$ will not default in the payment of dividends. Instead the asset of performance $j$ may fail with probability $1-p$. So, by the N.O.A. we must have:
	
	So we can infer the probability of non-default of the asset of performance $j$ (given the two performances at the same time):
	
	and therefore the probability of default:
	
	This type of reasoning thus also allows to require a performance $k$ of an issuer of assets (typically bonds type) knowing by expertise/audit the probability of default of payment/repayment relatively to a $100\%$ safe assets of performance $i$.
	
	Such a reasoning can help to avoid misunderstanding and bad long term strategies investments. Indeed, here are four assets, real data (no information here about time, but it's the same scale for the four of them):
	\begin{figure}[H]
		\centering
		\includegraphics{img/economy/arbitrage_opportunity_four_assets_start_point.jpg}
	\end{figure}
	A little survey on Twitter was made by Arthur Charpentier on this chart to if you could invest in one, and only one, asset which one will you pick ?
	
	The answers were: $15\%$ blue one, $58\%$ black one, $21\%$ red one, $6\%$ green one.
	
	The assets were here, respectively:
	\begin{itemize}
		\item green = U.S. Treasury Bills
		\item red = Stock Market (SP500)
		\item black = Fairfield Sentry
		\item blue = Pfizer (single stock)
	\end{itemize}
	The black curve was for a firm that had among the largest exposures to the Bernard Madoff fraud. And here are the curves (again, real data!).

	The more complete graph after the cut is:
	 \begin{figure}[H]
		\centering
		\includegraphics{img/economy/arbitrage_opportunity_four_assets_after_start.jpg}
	\end{figure}
	
	Now, if you think about it, it is kind of odd that almost $60\%$ people pick the black one. In a real market, how can we get at the same time the black and the green curve after what we just studied??? They both had no risk, and very different return. The black curve has the same return as the SP500 index, without any risk. That should be seen as odd…. Of course, it’s easy to comment, once you see the data, but still. 
	
	\pagebreak
	\subsubsection{Portfolios}
	\textbf{Definitions (\#\mydef):}
	\begin{enumerate}
		\item[D1.] "\NewTerm{Portfolio management}" is the art and science of making decisions about investment mix and policy, matching investments to objectives, asset allocation for individuals and institutions, and balancing risk against performance. Portfolio management is all about determining strengths, weaknesses, opportunities and threats in the choice of debt vs. equity, domestic vs. international, growth vs. safety, and many other trade-offs encountered in the attempt to maximize return at a given appetite for risk.

		\item[D2.] An "\NewTerm{Investment fund}" or more simply "\NewTerm{portfolio}" is an investment vehicle (portfolio of securities, shares or bonds, for example) that financial institutions offer their customers and that is structured and maintained to match the investment objectives stated in its prospectus. n theory, if the stock market is failing, the investment fund should not loose money.
		
		Even if an investment funds contains various financial assets, the customers can buy the issued units to a quite low price compared to buying individual assets. Each Unit theoretically contains a proportion of each asset being in the fund. They guarantee a right to participate in the overall fund's assets without giving the rights on the companies included in the fund.
		
		Some people estimate in the early 21st century that there is nearly $30,000$ investment funds worldwide.

		\item[D3.] A "\NewTerm{mutual fund}" is an investment vehicle that is made up of a pool of funds collected from many investors for the purpose of investing in securities such as stocks, bonds, money market instruments and similar assets. Mutual funds are operated by money managers, who invest the fund's capital and attempt to produce capital gains and income for the fund's investors. A mutual fund's portfolio is structured and maintained to match the investment objectives stated in its prospectus.

		One of the main advantages of mutual funds is that they give small investors access to professionally managed, diversified portfolios of equities, bonds and other securities, which would be quite difficult (if not impossible) to create with a small amount of capital. Each shareholder participates proportionally in the gain or loss of the fund. Mutual funds then offers to small investors the possibility to enjoy of a not too bad risk diversification to and also to get discount prices on transactions made by the fund managers.
		
		\item[D4.] A "\NewTerm{hedge fund}" is an investment fund that pools capital from a limited number of accredited sophisticated individual (the criteria for becoming one are lengthy and restrictive) or institutional  investors and invests in a variety of assets, often with complex portfolio construction and risk management techniques. It is administered by a professional management firm, and often structured as a limited partnership, limited liability company, or similar vehicle. Hedge funds are generally distinct from mutual funds as their use of leverage is not capped by regulators and distinct from private equity funds as the majority of hedge funds invest in relatively liquid assets such as options and to do aggressive investment (short sell). This will typically increase the leverage - and thus the risk - of the fund. This also means that it's possible for hedge funds to make money when the market is falling. Mutual funds, on the other hand, are not permitted to take these highly leveraged positions and are typically safer as a result.
	\end{enumerate}
	
	The majority of stock transactions concern the content of "\NewTerm{security portfolio}", which are the set of titles that a market player may hold. Manage a portfolio is therefore (most typically...) for a manager to seek to maximize a return on investment (ROI) for the customer while minimizing the risks (at least on long term investments...).
	
	\begin{tcolorbox}[title=Remark,colframe=black,arc=10pt]
	The ROI is also sometimes named "\NewTerm{yield}" or "\NewTerm{performance}" or "\NewTerm{profit rate}" and thus refers to a financial ratio that measures the amount of money gained or lost relative to the amount originally invested (often on the basis of an annual or semi-annula period). Typically, this ratio is expressed as a percentage rather than as a decimal.
	\end{tcolorbox}
	
	The "\NewTerm{securities}" (financial securities) are derived in the form of shares, bonds, foreign exchange options and commodities all more generally all named  "\NewTerm{financial products}" or "\NewTerm{financial assets}" and whose definitions (non exhaustive) are given below. When some of these products are mixed, we then speak of "\NewTerm{structured products}" (typically the combination of an underlying with an option).
	
	Issuing securities to the public isn't free, and the costs of different methods are important determinants of which is used. These costs associated with floating a new issue are generically named "\NewTerm{flotation costs}".
	\begin{tcolorbox}[colframe=black,colback=white,sharp corners]
	\textbf{{\Large \ding{45}}Example:}\\\\
	On March 17, 2011, Cornerstone , the software company, went public via an IPO (Initial Public Offering). Cornerstone issued $7.5$ million shares of stock at a price of $\$13$ each. The lead underwriters on the IPO were Goldman, Sachs \& Co. and Barclays Capital, assisted by a syndicate made up of $4$ other investment banks.\\
	
	Even though the IPO raised a gross sum of $\$97.5$ million, Cornerstone got to keep only $\$90,538,500$ after expenses. The biggest expense was the $7.14\%$ underwriter spread, which is ordinary for an offering of this size. Cornerstone sold each of the $7.5$ million shares to the underwriters for $\$12.0718$, and the underwriters in turn sold the shares to the public for $\$13.00$ each.	But wait, there's more! Cornerstone spent $\$15,421$ in SEC registration fees, $\$13,783$ in other filing fees, and $\$150,000$ to be listed on the NASDAQ Global Market. The company also spent $\$2.76$ million in legal fees, $\$1,012,700$ on accounting to obtain the necessary audits, $\$9,350$ for a transfer agent to physically transfer the shares and maintain a list of shareholders, $\$130,000$ for printing and engraving expenses, $\$5,000$  in certain legal compliance fees and expenses, and finally, $\$3,746$ in miscellaneous expenses.\\
	
	As this example show, an IPO can be a costly undertaking! In the end, expenses totaled $\$11,061,500$, of which $\$6,961,500$ went to the underwriters and $\$4,100,000$ went to other parties. All told, the total cost  was $12.2\%$ of the issue proceeds.
	\end{tcolorbox}

	\textbf{Definitions (\#\mydef):}
	\begin{enumerate}
		\item[D1.] To measure the overall evolution of the stock market, we use "\NewTerm{market indices}" reflecting the arithmetic mean or the weighted average prices (values) of a number of representative assets. This allows economic agents to know the overall performance a reference basket for a group of technologies or for a country. 
		
		 A price index (there exist also volatility indices) try to provides a summary of the overall price market by tracking some of the top stocks within that market\footnote{In fact they are all bad indicators as they do not incorporate implicitly the market volatility and history trend} or small, medium, and large companies, while other indexes represent only the largest companies. Some indexes also tend to track companies within a certain sector, like technology, while other indices are more broad.
		
	\begin{figure}[H]
		\centering
		\includegraphics[scale=0.39]{img/economy/bloomberg_dowjones_index.jpg}
		\caption{Down Jones Industrial Index Monitoring in Bloomberg\textsuperscript{TM} Terminal}
	\end{figure}
	Here are a few of the most known worldwide market indices:
	\begin{itemize}
		\item The "\NewTerm{S\&P 500}" tracks $500$ large U.S. companies (selected by committee) across a wide span of industries and sectors. The stocks in the S\&P 500 represent roughly $70\%$ percent of all the stocks that are publicly traded.
		
		Standard \& Poor is a company that doles out financial information and analysis and that was founded in 1860 by Henry Varnum Poor. In 1941 Poor's Publishing (Henry Varnum Poor's original company) merged with Standard Statistics (founded in 1906 as the Standard Statistics Bureau) and therein assumed the name Standard and Poor's Corporation.
		
		In order to keep the S\&P 500 Index consistent over time, it is adjusted to capture corporate actions which affect market capitalization, such as additional share issuance, dividends and restructuring events such as mergers or spin-offs. Additionally, to remain indicative of the U.S. stock market, the constituent stocks are changed from time to time. Between 2005 and 2014, $188$ index components were replaced by other components.
		
		To calculate the value of the S\&P 500 Index, the sum of the adjusted market capitalization of all $500$ stocks is divided by a factor, usually referred to as the Divisor $d$. For example, if the total adjusted market cap of the $500$ component stocks is US$\$13$ trillion and the Divisor is set at $8.933$ billion, then the S\&P 500 Index value would be $1,455.28$.
		
		The formula to calculate the S\&P 500 Index value is:
		
		where $p_i$ is the price of each stock $i$ in the index and $n_i$ is the number of shares publicly available for each stock and $d$ the "\NewTerm{S\&P 500 Divisor}".
		
		Companies can be listed in more than one index. Some of the largest companies within the S\&P 500 are also in the Dow Jones Industrial Average.
		
		\item The "\NewTerm{Dow Jones Industrial Average}" index (create in 	
1896 by Charles Dow and Edward Jones) tracks the $30$ largest U.S. companies. This means it represents "large-cap" companies, which is the industry term for "very big companies".
		
		Although the companies within the Dow Jones represent only about $25$ percent of all stocks, the DJIA is widely accepted as the leading indicator of market health.
		
		To calculate the DJIA, the sum of the prices of all 30 stocks is divided by a divisor, the Dow Divisor. The divisor is adjusted in case of stock splits, spinoffs or similar structural changes, to ensure that such events do not in themselves alter the numerical value of the DJIA. Early on, the initial divisor was composed of the original number of component companies; which made the DJIA at first, a simple arithmetic average. The present divisor, after many adjustments, is less than one (meaning the index is larger than the sum of the prices of the components). That is:
	
	where $p_i$ are the prices of the component stocks and $d$ is the "\NewTerm{Dow Divisor}".

	Events such as stock splits or changes in the list of the companies composing the index alter the sum of the component prices. In these cases, in order to avoid discontinuity in the index, the Dow Divisor is updated so that the quotations right before and after the event coincide:
	
	The Dow Divisor was $0.14602128057775$ on December 24, 2015 such that every $\$1$ change in price in a particular stock $j$ within the average, equates to a movement of:
		
		that is to say a $6.87$ point movement.
		
		\item The "\NewTerm{NASDAQ}" market index (for "National Association of Securities Dealers Automated Quotations"), which is known as the "\NewTerm{NASDAQ Composite}", tracks the roughly $3,000$ companies that are traded on the NASDAQ Exchange and began in 1971 at a Base Value of $100.00$. This is unusual, because no other exchange has its own popular index. The nightly news doesn't read stats from the "New York Stock Exchange Composite".

		The NASDAQ Composite has grown popular because it's commonly accepted as a shorthand indicator of how tech-sector and innovative companies – both big and small – are faring.
		
		The NASDAQ assigns every company on the exchange a "share weight" based on the company's market capitalization, or the total value of its shares. The larger the company, the greater the weight and the greater the effect a change in its share price will have on the NASDAQ Composite Index. Every company's share weight is multiplied by its share price, and all companies' results are added together. That sum is then divided by a number called, appropriately enough, the divisor. The purpose of the divisor is the same as for previous defined indexes.
		
		Therefore:		
		
		where $p_i$ is the price of each stock $i$ in the index and $n_i$ is the number of shares publicly available for each stock and $d$ the "\NewTerm{NASDAQ Divisor}".
		
		\item and many other index per country... (like SMI fort Swiss Market Index in Switzerland or CAC40 in France)
	\end{itemize}
	\begin{figure}[H]
		\centering
		\includegraphics[scale=0.87]{img/economy/stock_market_indices.jpg}
		\caption[Comparison of most well known market indices]{Comparison of most well known market indices (source: StockCharts.com)}
	\end{figure}
	Investment funds monitored on the basis of market indices are named "\NewTerm{ETF}" for "\NewTerm{exchange traded funds}" and that's just funds that replicate the performance of well known indices and that do not try (most of time) to outperform them.
	
		\item[D2.] A "\NewTerm{derivative}" is a product / financial instrument, which is also bought and sold, and is still built on the basis of a financial security/asset. This latter is then named "\NewTerm{underlying}" or "\NewTerm{support}" of the derivative. These underlying can therefore be stocks, bonds, currencies and ... even derivatives... The danger with derivatives is that when nobody stop superimposing them that nobod'y knows anymore exactly what were underlying, which is why they are sometimes referred to as "massive damage weapons" in finance (some leaders have eliminated derivatives from their portfolios but this can the opposite effect: increase the risk since the derivatives have been created at the origin to hedge risks... in short... it is not easy to find to good mix!).
		
		\item[D3.] "\NewTerm{Private equity}" is equity capital (remember that in accounting: Equity\index{equity}\label{equity} = Assets - Liabilities) that is not quoted on a public exchange. Private equity consists of investors and funds that make investments directly into private companies.
		
		\item[D4.] The "\NewTerm{volatility}" measure the amplification of the variation of a trade. In other words, a financial security/asset whose volatility is high sees its trade vary greatly, even sometimes exaggerated over time. Conversely, an asset whose volatility is low will see its trade vary slightly and/or fairly consistently. The volatility is often expressed as a percentage in simple mathematical models (because there are several definitions which we will see later). Thus, the volatility of an asset over a specified period is basically defined by:
		
		
		\item[D5.] The "\NewTerm{ticker}" is the abbreviated name on the markets of any financial instrument (or company name).
	\end{enumerate}
	In more complex situations, volatility is often assimilated to the standard deviation (i.e. variance) or semivariance and we will see that later.
	
	We will try here to give an exhaustive view of the following marketable assets\label{marketable assets}
	\begin{figure}[H]
		\centering
		\includegraphics[scale=0.8]{img/economy/marketable_securities.jpg}
		\caption{Marketable Securities (non-exhaustive)}
	\end{figure}
	Fixed Income Securities market is a large part of the financial industry, and it presents unique challenges and opportunities for its practitionnes. A large amount of money managed by pension funds and other institution funds is allocated to fixed income investments. Because fixed income has a quite predictable income stream, conservative money managers view it as a safer investment option when compared to socks and more exotic derivatives. This is ture because, for equity investments (stocks), for example, it is practically impossible to determine how much money a company will make in a few years from now. With a fixed income investment such as a bond, however, we have a contract that guarantees the return on the investment for a specified period of time.
	
	Clearly, there are also risks in such fixed income investments. A well-known risk is that of default on which we will come back later with mathematical tools. The second big risk, which is frequently overlooked by investors, is that the rate of return will not be able to cope with inflation during the period of investment. This all shows that analyzing fixed income investments is not as easy as it initially sound and software are typical tools that help investors in this purpose.
	
	But we will not study non-marketable assets:
	\begin{figure}[H]
		\centering
		\includegraphics{img/economy/non_marketable_securities.jpg}
		\caption{Non-Marketable Securities (non-exhaustive)}
	\end{figure}
	
	We also differentiate a set of various portfolio or funds management strategies:
	
	\textbf{Definitions (\#\mydef):}
	\begin{enumerate}
		\item[D1.] "\NewTerm{Passive management}" also named "\NewTerm{Index management}" is an investing strategy that tracks a market-weighted index or portfolio. The most popular method is to mimic the performance of an externally specified index and doing this by buying one or more index funds and this is what we name "physical replication". When doing index replication through derivatives we speak about "synthetic replication".

		\item[D2.] "\NewTerm{Active management}" refers to a portfolio management strategy where the manager makes specific investments with the goal of outperforming an investment benchmark index. 
		
		\item[D3.] \NewTerm{Diversified management}" refers to a portfolio management strategy where the manager mix obligations and shares in various weights and the latter having different risks profiles in the purpose to maximize the return on investment.

		\item[D4.] \NewTerm{Alternative management}", that is most of time associated hedge funds, is a management strategy without constraints using therefore not long only investments and using stronly leverage effects.
		
		\item[D5.] "\NewTerm{Overlay management}" is an overarching canopy of quantitative analyses and systems that operates an array of investments to achieve a desired net effect for an investor. Overlay management uses software to track an investor's combined position from the separate accounts. Any possible portfolio adjustments will be analyzed by the overlay system, which ensures the overall portfolio will remain in balance and prevent any inefficient transactions from occurring.
		
		\item[D6.] "\NewTerm{Structured management}" is a technique of portofolio management based on portfolio insurance techniques. Many of these portfolios are build on thanks to a unique set of formulas.
	\end{enumerate}
	
	\pagebreak
	\paragraph{Stocks (shares of stocks)/Equities}\mbox{}\\\\
	Owning shares of company profits is one of the most common ways to invest and generate wealth. A large number of people who have made a fortune have achieved it by creating or buying an equity stake in a successful corporation.
	\textbf{Definitions (\#\mydef):} 
	\begin{enumerate}
		\item[D1.] "\NewTerm{Stocks}" are values papers acknowledging by contract property rights in the capital of an entity known as "\NewTerm{public limited-liability company}" (PLLC). This contract has a price and is exchanged on the market.
		
		\item[D2.] The "\NewTerm{capital stock}" of a corporation constitutes the equity stake of its owners. It represents the residual assets of the company that would be due to stockholders after discharge of all senior claims such as secured and unsecured debt. Stockholders' equity cannot be withdrawn from the company in a way that is intended to be detrimental to the company's creditors.
	\end{enumerate}
		
	Therefore the stock of a corporation is partitioned into shares, the total of which are stated at the time of business formation. Additional shares may subsequently be authorized by the existing shareholders and issued by the company. In some jurisdictions, each share of stock has a certain declared par value, which is a nominal accounting value used to represent the equity on the balance sheet of the corporation. In other jurisdictions, however, shares of stock may be issued without associated par value.

Shares represent a fraction of ownership in a business. A business may declare different types (classes) of shares, each having distinctive ownership rules, privileges, or share values. Ownership of shares may be documented by issuance of a stock certificate. A stock certificate is a legal document that specifies the amount of shares owned by the shareholder, and other specifics of the shares, such as the par value, if any, or the class of the shares.

	\begin{tcolorbox}[title=Remark,colframe=black,arc=10pt]
	"\NewTerm{Rule 144 Stock}" is a common name given to shares of stock subject to SEC Rule 144: Selling Restricted and Control Securities. Under Rule 144, restricted and controlled securities are acquired in unregistered form. Investors either purchase or take ownership of these securities through private sales from the issuing company or from an affiliate of the issuer. Investors wishing to sell these securities are subject to different rules than those selling traditional common stock. These individuals will only be allowed to liquidate their securities after meeting the specific conditions set forth by SEC Rule 144.
	\end{tcolorbox}
	
	We distinguish in general the following stock shares:
	\begin{enumerate}
		\item The "\NewTerm{classic share}" or "\NewTerm{common share}" that gives a vote right at general meetings, the right to information (...) and the right to dividends.
		\begin{figure}[H]
			\centering
			\includegraphics{img/economy/stock_share_common.jpg}
			\caption{Example of an old French common stock share paper}
		\end{figure}
		
		\item The "\NewTerm{privileged share"} that provides a privilege that can be a priority when voting at general meetings or a priority in the distribution of the dividend.
		
		\item The "\NewTerm{preferred share}" that only gives privileged access to dividends and no voting rights!.
		
		\item The "\NewTerm{share warrant}" which are shares that entitle the holder to subscribe for new shares at a given date.
	\end{enumerate}
	\begin{tcolorbox}[title=Remark,colframe=black,arc=10pt]
	 We also differentiate "\NewTerm{bearer stocks}" negotiable without restrictions on the stock exchange with "\NewTerm{registered stocks}" whose value must be negotiated with more or less complex legal restrictions as there contains the name of the stockholder that must be registered to the stockholders register (most of times because of government taxes).\\
	\end{tcolorbox}
	
	Something very tricky with shares is the "\NewTerm{dilution effect}". Indeed, When a company issues additional shares (and therefore its stock-capital), this reduces an existing investor's proportional ownership in that company. This is a risk of investing in stocks that investors must be aware of. The new stocks will be (normally... at the opposite of what happened for example by Facebook with Eduardo Saverin) offered to the stockholders of the company at a fixed rate and in proportion of stocks they already own ("\NewTerm{subscription right}") in order not to penalize against a discount of the value because of the new issued stocks. This will allow them to maintain the percentage of their share capital, and the weight of their voting rights. Dilution can also happen when there is a warrant plan (abusively named "stock-options plan" as we will see further below) as a company therefore grant to employees a large number of optionable securities and these latter exercise the options, common shareholders may be significantly diluted. 
	
	\begin{tcolorbox}[colframe=black,colback=white,sharp corners]
	\textbf{{\Large \ding{45}}Example:}\\\\
	Assume that a simple business has $10$ shareholders, and that each shareholder owns one share, or $10\%$ of the company. If each investor receives voting rights for company decisions based on share ownership, every shareholder has $10\%$ control.\\

	Suppose that the company then issues $10$ new shares and that a single investor buys them all up. There are now $20$ total shares outstanding, and the new investor owns $50\%$ of the company.\\

	Meanwhile, each original investor now owns just $5\%$ of the company ($1$ share out of $20$ outstanding), because their ownership has been diluted by the new shares.
	\end{tcolorbox}
	Because the earnings power of every share is reduced when convertible shares are executed, investors may want to know what the value of their shares would be if all convertible securities were executed.
	
	\textbf{Definition (\#\mydef):} The "\NewTerm{diluted earnings per share (EpS)}" is calculated by firms and reported in their financial statements and is the value of earnings per share if equity warrants (see definition further below), stock options (see definition further below) and convertible bonds were converted to common shares. In a given point of the EpS is like a pessimistic value of a share.
	
	\subparagraph{Dividend Yield}\mbox{}\\\\
	\textbf{Definitions (\#\mydef):} There are several types of "\NewTerm{dividend yield}" based on the context that are intuitive for some and for others quite complex. Here are three of the most common (we will see other more complex one later...) to our knowledge when discussing for the first time on financial mathematics:
	
	\begin{itemize}
		\item[D1.] There is the ratio, expressed as a percentage, named "\NewTerm{yield}" or "\NewTerm{rate of return}" (not to be confused with "\NewTerm{yield to maturity}" that we will see further below), between the dividend per share distributed by a company and the share price on the stock markets of the company at the time the dividend is paid (some practitioners sometimes take the arithmetic average of the dividends paid over several periods):
		
		
		\item[D2.] There is the ratio, expressed in cash value by year and named "\NewTerm{share stock}", of the difference between the sale trading value (bid) of the stock with its dividends values added and the buy trading value (ask) of stock on the number of periods (mainly expressed in years):
		
		Obviously if we divide the result of this latter annual cash yield by the initial capital invested, we get the return in percentage.
		
		\item[D3.] There is the "\NewTerm{Implicit Rate of Return (IRR)}" which is the ratio between the net profit of the company and its stock market capitalization value (the implicit rate or return abbreviation can be confusing with the "internal rate of return" that we have seen in the section of Quantitative  Management Techniques this is why we avoid to use its abbreviation in practice):
		
		The IRR is a kind of performance yield between the potential income of a stock and its price (caution!: it would a performance yield only if we took into account the ration dividends/stock price that can be very different). We assume, rightly or wrongly, that the company has a better long-term use in making profits than distribute them as dividends...
	
		\item[D4.] The "\NewTerm{Price Earnings Ratio (PER)}\label{price earning ratio}" is simply define as at a time $t$ by: 
		
		Therefore the PER also gives the number of years needed to pay back the share only with the dividends (obviously under the assumption of constant dividends). So strictly speaking, the ratio is measured in years, since the price is measured in dollars and earnings are measured in dollars per year.
		
		The PER is generally quoted in the press is the one calculated on the last published annual profit of a company. However analysts are often based their judgment on the expected profit for the current year.
		
		Here is a small table that can illustrated a possible interpretation of the PER:
		\begin{table}[H]
		\begin{center}
			\definecolor{gris}{gray}{0.85}
				\begin{tabular}{|p{1.5cm}|p{10cm}|}
					\hline
					\multicolumn{1}{c}{\cellcolor{black!30}\textbf{PER}} & 
	  \multicolumn{1}{c}{\cellcolor{black!30}\textbf{Quality}} \\ \hline
					\centering\arraybackslash\ $0$ to $10$ &  The share is undervalued or the profits of the company are expected to soon to be in decline \\ \hline
					\centering\arraybackslash\ $10$ to $17$ & For most companies, a ratio that fall within that range is considered good \\ \hline
					\centering\arraybackslash\ $17$ to $25$ & The share is overvalued or there is growth in profits for the latest announcements  \\ \hline
					\centering\arraybackslash\ $>25$ & It is likely that high profits are expected in the future, or the share is the subject of a speculative bubble.  \\ \hline
			\end{tabular}
		\end{center}
		\caption{Qualitative judgments of PERs commonly accepted}
		\end{table}
	\end{itemize}
	\begin{figure}[H]
			\centering
			\includegraphics[scale=1]{img/economy/per_sp_plot.jpg}
			\caption[Parallel graphs of the S\&P index and PER for the same period]{Parallel graphs of the S\&P index and PER for the same period (source: Wikipedia)}
		\end{figure}
	\begin{tcolorbox}[title=Remark,colframe=black,arc=10pt]
	Once a customer asked me a list of common financial indicators. In fact you can have an empirical choice with a poor scientific point of view of such a list by the UBS (Swiss Banks Union) named: \textit{The $100$ most important financial performance indicators} that you can order on the web.
	\end{tcolorbox}
	
	\begin{tcolorbox}[colframe=black,colback=white,sharp corners]
	\textbf{{\Large \ding{45}}Examples:}\\\\
	E1. Consider a repayable stock purchased six years before for the price of $10.-$ (invested capital). The investor sell it the $12.50.-$. The investor received before three times a dividend: three times $2.20.-$ and one time $1.-$. Both yields then give respectively:
	
	and if we divide this latter by the initial invested capital ($10.-$) we get $16.8\%$.\\
	
	E2. The maximum price (including expenses) we can put up to buy a stock of $500.-$ with a relating dividend of $12\%$ in comparison to a similar investment of the same price with a yield of $5\%$ with same periodicity is $1'200.-$. Indeed, the latter stock would bring $60.-$ dividend at each period ($12\%$ of $500.-$). The sum that should be set to have the same interest at a rate of $5\%$ is $1'200.-$. Indeed $5\%$ of $1'200.-$ being equal also to $60.-$.\\

	E3. The stock issued by a company (whose capital consists of $10$ million shares) is valuated at $10.-$, bringing the stock market value of the company at $1$ billion. The planned net profit is $50$ million for the current fiscal year, that is to say $5.-$ per stock (under the assumption that all profits are distributed into dividends...). The division of the net profit by the stock market value gives us a price earning ratio (PER) of $20$.
	\end{tcolorbox}
	
	\subparagraph{Shares Benchmark Indices}\mbox{}\\\\
	We consider an index composed of $n$ shares. Given $P_i(t)$ the price of the share $i$ and $R_i(t)$ the corresponding yield for the period $[t-1,t]$:
	
	The value of the index (or "benchmark") $B(t)$ is defined for $t\geq 1$ by:
	
	with $B(0)=1$, where $w_i(t)$ is the weight of the share $i$ in the index such that $\sum_i w_i(t)=1$. The value of $B(t)$ is generally calculated on market closure.
	
	The majority of market indices use obviously a weight based on the traded capitalization:
	
	where $N_i(t)$ is the number of shares $i$ issued (some indices as the Nikkei use weight based on the prices and not on the quantity.
	
	\subparagraph{Durand Model}\mbox{}\\\\
	To conclude this small introduction on the stocks, note that a theoretical model of current evaluation ("current" meaning "at this date") of the stock is named the "\NewTerm{Durand Model}". The behind idea is quite simple: the price of a stock $P_0$ today is equal to the sum of the cash flows (and therefore of its dividends $D_t$) actualized using market geometric rates $t_i\%$ (or expected/required by the stockholder) of the corresponding period $i$ paid at each period $k$ (therefore we assumed we are in a non-probabilistic scenario...) as well as its future resale price $P_T$ also actualized.
	
	That is to say formally:
	
	If we make the usual high-school level assumption that the rate is always constant (if not, we will use a geometric average of the rates on the totality of the time or better... use a Monte Carlo simulation), we then have:
	
	this relation is know under the name "\NewTerm{Irving Fisher fundamental value}".
	
	But the resale price at time $T$ will be equal to this same relation and so on to infinity, because a stock is not intended to be repaid! Then we have:
	
	and as:
	
	Then it remains:
	
	that is also sometimes written:
	
	and is named the "\NewTerm{Discount Cash Flows Discount Model }".
	
	Now let us consider a simplification of the Durand model named the "\NewTerm{Gordon-Shapiro model}\index{Gordon-Shapiro model}\label{gordon shapiro model}" or "\NewTerm{Dividend Discount Model DDM}\index{divident discount model}". This latter model considers that at each period, dividends are growing at a same rate denoted $t_{cd}\%$ by:
	
	By applying what we have just proved in the Durand's model:
	
	Let us put:
	
	Then we have:
	
	We have proved in the section Sequences and Series that for any geometric sequence of reason $x$ was given for recall by:
	
	Then if $n$ approaches infinity and under the assumption that:
	
	that is to say, the rate of return expected by stockholders exceeds the dividend growth rate, then we have:
	
	Therefore:
	
	Hence in the end:
	
	The dividend growing rate $t_{cd}\%$ is determined either from historical data of the stock, either from the forecasts of analysts on future dividends.
	
	Note that we find often this latter relation in the following forms in the literature:
	
	Practitioners sometimes use the last expression in the form:
	
	to compare the yield $t\%$ of various stock knowing their dividend, their current price and forecasted expected dividend growth rate.
	
	In the last relation the $\mathcal{E}$ in subscript of the $r$ stands for "Equity" (\SeeChapter{see section Quantitative Management page \pageref{equity}}).

	Because $r_\mathcal{E}$ is the return that the shareholders require on the stock, it can be interpreted as the "\NewTerm{firm's cost of equity capital}".
	
	In the case where $t_{cd}\%$, that is to say in the case of "\NewTerm{preferred stocks}\index{preferred stocks}", that are stocks whose holders do not benefit from increases in the firms earnings and they generally cannot vote in corporate elections, the latter relation reduces obviously to:
	
	
	
	\begin{tcolorbox}[colframe=black,colback=white,sharp corners]
	\textbf{{\Large \ding{45}}Example:}\\\\
	We want to valorize a stock that pays a first dividend of $5.-$ and whose supposed growth is assumed constant (to infinity...) with a yield (geometric mean one) of $14.87\%$ compared to an average geometric return of the market of $20\%$. We then have the price of this action that can be estimated by:
	
	\end{tcolorbox}
	
	Let us note to close this valuation on stocks an important property, that also applies to obligations that we will see just after below, and who is the "\NewTerm{risk of credit default}". This risk, which must be quantified at best in terms of probabilities, simply consist in the fact that the issuer of such stocks may not meet its obligations and pay the money that we are entitled to expect of him simply because it is bankrupt or that the economic system does not lend itself (major economic crisis for example). The Governements normally guarantiee only a fraction of the expected gain (guarantee that must also mathematically be actualized). Therefore, you should know that:
	
	\begin{enumerate}
		\item The real actualized price is still slightly lower than the simple model above, since it is related to an event with probability not equal to 100% (but less!).
		
		\item The actualized price is in reality a weighted mean by the probabilities of the economic scenarios.
		
		\item All algebraic economic models are idealized and you should never forget that (they do not into account macro-economic phenomena or when they do they actuall do it bad most of time)!
	\end{enumerate}
	
	\pagebreak
	\paragraph{Obligations (Bonds)}\mbox{}\\\\
	Unlike the individual loan (indivis loan), the loan named "\NewTerm{bond issue}" involves many lenders, named "\NewTerm{subscribers}", who receive in exchange for money lent, securities named "\NewTerm{obligations}".
	
	\textbf{Definition (\#\mydef):} "\NewTerm{Obligations}", also named "\NewTerm{bonds}" or "\NewTerm{"financial commitment"}, are values papers (debt securities of an issuer) establishing through a contract debt securities (capital loaned) to an investor and that pay a fixed interest (usually in the form of annual coupons like dividends to) to the holder during a defined period of time(the initial amount invested is repaid to a deadline specified in the contract). This contract has a price (depending on the date!), It is most of time redeemable on the market and the debtor is obliged to pay interest (remuneration by coupons). Moreover, if the obligation is a "\NewTerm{convertible bond}" it entitles the creditor to obtain either a refund of the duty, or conversion into stocks, following procedures set  in advance in the contract.
	
	Stocks and Bonds are very different in what they represent. While the stock (auction) means a title related to the share capital of a company and vote right, coted or not (negotiable or not) without expiration and depending on company performance (long term security), the bond is a debt security (also negotiable or not depending on contract terms) limited in time and whose interest is considered a cost doing business and therefore tax deductible (at the opposite of stocks). The bond is based on the debt of a company, state, or local community while the stock (auction) is part of the equity of a corporation. Any company composed of stocks is not necessarily coted on the stock market, and any company is not necessarily a stock company. The issuance of a bond allows the issuer to diversify its sources of loans, and for a stock, to diversify its sources of funding. Also in case of bankruptcy of the issuer the holder of an obligation has priority over stockholders and can make a legal recourse in case if payments are missed (stockholders cannot make legal recourse!)  then risk is less.
	
	\begin{tcolorbox}[title=Remark,colframe=black,arc=10pt]
	Some bonds have their coupons which are not disclosed in cash but are based on the level of economic indices as we will see later.
	\end{tcolorbox}
	
	We distinguish three main types of bonds (actually we practitioners differentiate more than 20 types of bonds but that are most of times special cases of the three main one such as some that we alread mentioned above such as: convertible bonds, bearer bonds, exchangeable bonds, perpetual bonds, ):
	
	\begin{enumerate}
		\item "\NewTerm{Fixed rate bonds}" or "\NewTerm{common obligation}" which is the traditional bonds (it represents in the early 2000s about $85\%$ of the bond market). It provides a cash-flow definitely fixed of interest when issued (coupons) in a predefined intervals until maturity (which is safe) and whose mathematical interest rate corresponding to this flow is named "\NewTerm{yield to maturiy}". Financial practitioners often refer to as "\NewTerm{plain vanilla bond}". When the duration of the flow can be considered infinite, we speak of "perpetual obligation" .This is however not a safe investment as we will see in a simple example below.
 		\begin{figure}[H]
			\centering
			\includegraphics[scale=0.8]{img/economy/bond_coupons.jpg}
			\caption{Example of bond with fixed rate with fixed number of coupons (at the time of paper...)}
		\end{figure}
		When the duration of the cash-flow can be considered infinite or really is unlimited, we speak of "\NewTerm{perpetual obligation}". This is however obviously not a safe investment that we need to quantify as we will see in a simple example further below.
		\begin{figure}[H]
			\centering
			\includegraphics[scale=0.55]{img/economy/bond_without_coupons.jpg}
			\caption{Example of perpetual bond with fixed rate (at the time of paper...)}
		\end{figure}
		and in computer versions with the Bloomberg\textsuperscript{TM} interface for an OCT bond (Obligation Comparable to a Treasury Bond):
		\begin{figure}[H]
			\centering
			\includegraphics{img/economy/bloomberg_bond_oct.jpg}
			\caption{Example of fixed rate OCT with coupons on Bloomberg\textsuperscript{TM} Terminal (at the time of the computers...)}
		\end{figure}
		The fixed rate bond is typically quoted in prices or rates. The fixed rate bond is valued by actualizing future cash flows it delivers \footnote{Unlike the NYSE where many stocks are trader, there is no particular physical locations where bonds are traded. Instead, they are traded electronically. The Financial Industry Regulatory Authority (FINRA) has made an effort to make bond prices more widely available. Their website, \url{www.finra.org/marketdata}, allows anyone to search for the most recent trades and quotes for bonds.}.

		\item "\NewTerm{Floating rate bonds}" whose interest cash-flows, but not the redemption price, are indexed to a benchmark rate sucha as the policy rate of a central bank, the results of a company, or any other. The risk associated with such variable rate is named "\NewTerm{interest rate risk}". For the holder of a bond portfolio that wishes to protect its capital, it is sufficient to immunize the portfolio against fluctuations in the rate. This is what we name "duration hedging" and that we will detailed further below.
		
		\item "\NewTerm{Zero-coupon bonds}" (also sometimes named "\NewTerm{discount bonds}") that has only two financial flows: an initial flow (purchase) and a final flow (the unique coupon and the Nominal), without any interim payment (hence its name because it does not pay coupon meantime or coupons to $0\%$ ... and only nominal is paid at maturity) and therefore the valuation calculation thereof require to know only the simple interest. This is most of time the least risky of all obligations since it pays one coupon and therefore its effective yield is equal to its original actuarial rate (since there can be reinvested in the meantime). The buyer purchased the bond at a value less than the face value, price which is paid at maturity of the contract. The zero-coupon is usually indexed on the inflation index (in this special case it belongs to the floating rate bonds family).
	\end{enumerate}
	\begin{tcolorbox}[title=Remark,colframe=black,arc=10pt]
	A short-term bond with a maturity of less than one year is named "\NewTerm{treasury bond}" ("T-bill" in the U.S.A. or "BTF" in France). On the same model as the zero-coupon bonds the T-Bill do not pay interest before maturity but instead are sold at a discount to their face value which allows the subscriber to obtain a benefit at maturity. Most of time T-Bill are reserved to bonds emitted by governments. 
	\end{tcolorbox}
	
	Bonds are characterized by several properties:
	\begin{itemize}
		\item[P1.] The name of the issuer with its abbreviated name (ticker).
		
		\item[P2.] Their "\NewTerm{basis currency}" that could also fluctuate on a global market.
		
		\item[P3.] Their "\NewTerm{deadline}" or "\NewTerm{maturity date}" that will give the possibility, depending on their date of issue and the type of calendar, to know the actualized estimated value of the obligation at any time.
		
		\item[P4.] Their "\NewTerm{face value}", named also the "\NewTerm{pair}", designate the value used to calculate the interests (but most of time it is implicit as all calculations are normalised to a face value of $100.-$.
		
		\item[P5.] Their "\NewTerm{coupon rate}" or "\NewTerm{nominal rate}" associated with their "\NewTerm{frequency}" (often annual or semi-annual) that gives the possibility to define the interest named the "\NewTerm{coupon}" or "\NewTerm{dividend coupon}" applied on the nominal value of a bond that will be paid to the subscriber to the date named "\NewTerm{date of record}". Normally the calculation method of the interest rate must be reported.
		
		\item[P6.] Their "\NewTerm{issue price}" or "\NewTerm{subscription price}" or just "\NewTerm{price}" or even "\NewTerm{"last price}" is the amount to pay today by the subscriber (investor) for owning a bond. The issue of bonds is therefore "at par" if the nominal value is equal to the amount requested for acquisition. The subscription price is below par if the requested amount is less than the nominal (most common), but the price can also sometimes be above par! The difference between nominal value and subscription price is named the "\NewTerm{share premium}".
		
		The issue price is not always equal to the nominal value in order to make investor to want buy without necessarily offer him a nominal interest rates too high. Thus, the issue of a bond is a subtle game for the issuer between the value of nominal interest rate and the issue price.
		
		\item[P7.] Their "\NewTerm{redemption price}" or "\NewTerm{redemption value}" is the amount really paid to the borrower on repayment of the bond at maturity. The reimbursement may be made at par or sometimes above when at maturity (in fine), in installments or never (perpetual bonds). The difference between the redemption value and the nominal is named the "\NewTerm{redemption premium}".
	\end{itemize}
	
	Bond investors prefer in general short maturities because the rate of return reflects more accurately the investor enrichment if he can reinvest each coupon at the same rate and retains the obligation upon maturity. This scenario is obviously more difficult to guarantee on the long term. In addition there is the "\NewTerm{risk of default}", that is to say, the risk of bankruptcy of the issuer.
	
	Conversely and obviously, issuers have usually a preference for longer maturities, which allow them to spread their debts over time. The divergence between demand (investors) and supply (issuers) results in yields generally lower on the short-term than for medium and long term (this is trivial as the risk is lower for short-term bonds). The "\NewTerm{yield curve}" which is a function at a given date and for each maturity on the $x$-axis shows the level of interest rates associated with the orderly financial futures and therefore typically has a growing form.
	\begin{figure}[H]
		\centering
		\includegraphics{img/economy/bond_characteristics.jpg}
		\caption{Typical Bond details on Bloomberg\textsuperscript{TM} Terminal}
	\end{figure}
	For example the zero-coupon rate curve of U.S.A. bonds as to the 5 September 2001 below comping from zero-coupon U.S.A. Treasury bonds (as the reader can see it!) where there is no interpolation - or "stripping" as say sometimes financial practitioners... - but only a simple straight line between points while some programs offer various stripping based on polynomials or splines or more complicate models what we will see later):
	\begin{figure}[H]
		\centering
		\includegraphics{img/economy/bloomberg_zero_coupon_yield_curve.jpg}
		\caption{Example of yield curve for zero-coupon USA bonds (T-Bills) in Bloomber\textsuperscript{TM} Terminal}
	\end{figure}
	At a given date, in a given country or in an unified economic zone, there is a multitude of interest rate curves: When the yield curve is flat, we speak then logically of "\NewTerm{flat curve}" when it is growing of "\NewTerm{upward sloping curve}" and when it is decreasing of "\NewTerm{downward sloping curve}".
	
	Note that the "\NewTerm{periodic zero yield}", also sometimes named "\NewTerm{zero coupon yield to maturity}" (or "\NewTerm{yield to maturity}" as we have already mentioned) at maturity of $n$ units of period of time for a zero-coupon bond with nominal $C$ an issue price $P$ can be easily obtained from the relation of compound interest in the deterministic case:
	
	Relation that will we frequently find in the literature in the following forms:
	
	with PV meaning as we already know "Present Value" and FV "Future Value" and where the rate is denoted $y$ to mean "yield"... (obviously when we reverse the parenthesis we talk as usual of "actualization factor" or " discount factor" but this is not new for us as we saw in our study of annuities and loans earlier).
	
	Therefore:
	
	What we sometimes find in the literature rather in the following traditional form (long live the lack of standards for mathematical notations in finance...!):
	
	
	\begin{tcolorbox}[colframe=black,colback=white,sharp corners]
	\textbf{{\Large \ding{45}}Example:}\\\\
	The zero-coupon yield to $4$ years .... corresponding to a nominal zero-coupon of $100.-$ of issue price $90.-$ (under par price) is:
	
	\end{tcolorbox}
	\begin{tcolorbox}[title=Remark,colframe=black,arc=10pt]
	Among the bonds, only fixed-rate zero-coupon bonds allow only to eliminate almost surely the risk rate between two dates (but not risk of bankruptcy). A conventional fixed rate obligation actually generates as much additional rate risk that it has intermediate financial cash-flows: the reinvestment rate of each of the coupons from the date of payment and the final repayment date is in fact unknown even if it is implicit in the price of the bond. 
	\end{tcolorbox}
	There is another current and very convenient way to write the current value of a fixed rate zero-coupon bond. Recall that we have proved much higher under certain conditions that:
	
	Therefore:
	
	\begin{tcolorbox}[colframe=black,colback=white,sharp corners]
	\textbf{{\Large \ding{45}}Example:}\\\\
	The zero-coupon yield to $10$ years .... corresponding to a nominal zero-coupon of $100.-$ of issue price $55.3895.-$ (under par price) is:
	
	In continuing interest, it gives (as proved already done earlier above) the "\NewTerm{continuous actuarial rate zero coupon}":
	
	\end{tcolorbox}
	\begin{tcolorbox}[title=Remark,colframe=black,arc=10pt]
	Investors should pay particular attention to the indication "\NewTerm{contingent}" on its bond paper, which means that in case of bankruptcy of the debtor (likened to the "\NewTerm{issuer risk}"), the holder of the obligation will be repaid only after all other creditors... The issuer risk can be avoided by choosing (very) safe bonds as government bonds or renowned companies. The downside is then the low rates offered that we must furthermore put in contrast with inflation (on a rate of $3\%$ over $10$ years of a governement bond that undergoes an inflation of $2\%$ there remains only $1\%$ of remuneration for example).
	\end{tcolorbox}
	Let us indicate that multiple coupons bonds are trendy thanks to the tax system of the States and also by the cost of paperwork they generate. Hence the fact that the zero-coupon are preferred.
	\begin{tcolorbox}[colframe=black,colback=white,sharp corners]
	\textbf{{\Large \ding{45}}Examples:}\\\\
	E1. Consider a bond of $3,000,000.-$ divided into $300$ bonds each of nominal $10,000.-$  issued in June 2004 for a period of $10$ years. Subscription: $99.5\%$ of the par value. Redemption at par at maturity. Fixed annual interest: $4.5\%$.\\
	
	The values defined above are then expressed as follows:\\
	
	The nominal value $C$ of the obligation is to $10,000.-$. The number $N$ of bonds is $300$. The duration $n$ à of the loan is $10$ years and the nominal rate is $4.5 \%$ per year. The issue price is $99.5\%$ of $10,000.-$ that is to say $E = 9,950.-$ (below par!). The coupon therefore has a value $c=450.-$ and the repayment $R$ is made at par and so is equal to $10,000-4,500 = 5,500.-$.\\
	
	E2. Given fixed-rate bond, issued at a price of $1,000.-$, and paying an annual coupon of $100.-$. The rate used is $100 / 1,000 = 10\%$.\\
	
	Suppose the market rates go to $15\%$. This means that a new obligation, which is issued for a price of $1,000.-$, serves a coupon of $150.-$ (because $150 / 1,000 = 15\%$).\\
	
	The new bond is more interesting than the old one, and everyone will want to sell the old to buy new. That is why the price of the old bond will implicitly decrease, until it matches that of a financial product providing $15\%$ or here $666.-$. Then, we will have as expected $100/666 = 15\%$.\\
	
	Similarly, if market rates fall to $5\%$, this means that a new obligation, which is issued to a price of $1,000.-$ price, will serve a coupon of $50.-$ (because $50 / 1,000 = 5\%$).\\
	
	The new bond is the less interesting than the ole one, and almost no one will buy it. That is why the price of the old obligation will implicitly go up until it match that of a financial product providing $5\%$ rate that is to say here $2,000.-$. So, we will have as expected $100 / 2,000 = 5\%$.
	\end{tcolorbox}
	Thus, the price of a fixed rate bond decreases when market interest rates rise and rises when market interest rates fall. This is why an investment in bonds is not without risks: one can lose part of the capital. In fact, the only (almost) safe strategy is to buy bonds at the time of issuance, and keep them until maturity.
	
	At any time, the market value of a fixed rate bond or variable rate one must therefore be equal to the sum of the present values of the coupons and of the repayment which it gives the right. The present value is calculated at the bond market rate for bonds of the same type and same duration.
	
	Thus, the present value of a fixed-rate bond (the case being easily generalized to a variable rate) should be seen as an initial capital of which is removed during $n$ remaining periods a certain fixed sum corresponding to the amount of the price of the coupon (facial interest rate multiplied by the nominal):
	
	with $C$ the nominal value of the obligation and total cumulated being periodically subject to market interest rate $t_M\%$  (thus the "rate to maturity") that is considered as constant in the context of a certain (deterministic) future.
	
	Thus, the present value of a fixed-rate bond is initially formed as the present value of the remaining future coupons (often also names  "cash-flows") during $n$ periods such that:
	
	relation that we found many times in the specialized literature in the following form:
	
	This part of the price of the value of the bond corresponds to the total amount required as we can settle the amount $P_c$ after removing $n$ times (the number of periods remaining) the value $c$ at an interest rate $t_M\%$.
	\begin{tcolorbox}[title=Remark,colframe=black,arc=10pt]
	Let us recall that after we have proved in the Section Sequences and Series, that if $t_M\%$ is less than $1$ and the sum tends to infinity, then:
	
	\end{tcolorbox}
	Then the bond is also made of the value of the reimbursement $R$. Although it is repaid at maturity, it can be seen as a savings capital at a rate equal to that of the market $t_M\%$ as:
	
	The present value of the bond concerning the reimbursement is then:
	
	which is the capital for repayment $R$ remaining after the $n$ remaining periods.
	
	Thus, the total price of a fixed-rate bond, also named "\NewTerm{non-abritrage bond price}" is:
	
	that is to say, the present value of future coupons and also the present value of the  in fine reimbursement. This relation is importance in deterministic quantitative finance, it is necessary to remember it!! In the context above, be aware that the rate equation is often named "\NewTerm{yield to maturity at par}".
	
	The value of a bond, in the point of view of its trading price, may differ from its nominal value set at the issue if interest rates change in the market hence the purpose to calculate its current value.
	\begin{tcolorbox}[colframe=black,colback=white,sharp corners]
	\textbf{{\Large \ding{45}}Example:}\\\\
	Consider we want to calculate the current price of a bond, with annual coupon of $450.-$, with a reimbursement at par in $5$ years for $10,000.-$.\\
	
	The present value $P$ for a market rate between $0\%$ and $100\%$ has the following characteristics:
	\begin{figure}[H]
		\centering
		\includegraphics{img/economy/bond_present_value.jpg}
		\caption{Current value of a bond based on the market rates (yield)}
	\end{figure}
	\end{tcolorbox}
	Thus, a fundamental property of the price of a fixed rate bond is that this is a strictly decreasing function of market rate of return. The finance practitioners (and mathematicians) say than the function is convex: that is to say, when market rates fall, the price accelerates upwards and vice versa when the market interest rates rise, the price is decelerating downwards.
		
	We also guess that at the view of the relation of the non-arbitrage bond price $P$, that bond prices variations increase with the increase in maturity and with the value of coupons.
	\begin{tcolorbox}[colframe=black,colback=white,sharp corners]
	\textbf{{\Large \ding{45}}Example:}\\\\
	Today, we buy a fixed rate bon with of $3$ years maturity for a nominal price $100.-$, the coupon rates are of $5\%$ and the fixed market rate $10\%$. The cash-flows   collected are then of $5.-, 5.-$ and $105.-$ after respectively $1$ year, $2$ years and $3$ years. The price of this bond is then equal to:
	
	That can get directly with Microsoft Excel 14.0.6129:
	\begin{center}
	\texttt{=PRICE("2013-01-01","2017-01-01",5\%,10\%,105,1,3)=87.577}
	\end{center}
	\end{tcolorbox}
	Evaluate a bond is therefore  equivalent to finding what should its value in principle in the actual conditions of market, that is to say its potential trading value, by a mathematical operation named "operation of actualization" determining its theoretical present value. It is therefore, as we already know, an actuarial calculation.
	
	The bond investor will of course have for aim to seek the market rate that allows to make the investment a profitable action. Thus, we define the "\NewTerm{yield to maturity YTM}" $x$ as the interest of the market that can satisfy the following relations, depending on the remaining time periods $n$ of the bond among $N$ periods.
	
	Thus, when the bond is issue:
	
	or on any date $n<N$:
	
	The yield to maturity of a bond is then the rate $x$ that cancels the difference between the value of the issue price $E$  and the present value of future cash flows it generates. This rate is calculated at the date of settlement and must appear in the emission brochures. For the buyer of the obligation, the actuarial rate represents the rate of return achieved if keeping the bond until its repayment and reinvesting the interests at the same actuarial interest rate.
	
	Let us see some other useful definitions of bonds:
	
	\textbf{Definitions (\#\mydef):}
	\begin{enumerate}
		\item[D1.] The "\NewTerm{expired coupon} (EC)" (also named "\NewTerm{net coupon}") of a bond is paid to the owner this latter after deducting anticipated taxes $\text{AT}\%$ ($\text{AT}$ being the abbreviation for: Anticipated Taxes). Thus the calculation of the annual net coupon of bonds traded at $X.-$ (monetary value) with a yield of $Y\%$ yield is trivially given by:
		
		
		\item[D2.] The "\NewTerm{accrued interest (AI)}" is the amount of interest (fraction) that has been accumulated since the last date of payment of interest, but which is not yet due. It is won by a bond since its last term and is determined on a sale or inventory. Or in other words it is method when interest that is either payable or receivable has been recognized, but not yet paid or received. Accrued interest occurs as a result of the difference in timing of cash flows and the measurement of these cash flows.		
		
		Its calculation is trivially given by an application of rules of simple interest such as:
		
		where $\Delta d$ is the time since last coupon payment and $\Delta T$ the period of coupons (both must be obviously in the same units).
		
		So to get the effective value of a bond (the "present value" as we know), we add to its traded value, named "\NewTerm{clean price}", the accrued interest (in green in the figure below) since the last coupon payment.
		\begin{figure}[H]
			\centering
			\includegraphics{img/economy/bond_accrued_interest.jpg}
			\caption{Schematic principle behind the bond accrued interest}
		\end{figure}
		
		\item[D3.] By extension, if we seek to calculate the net value of $n$ (parallel) coupons with a same yield of $ t_0\%$ of a bond which nominal value is $P$ with an anticipated tax of $\text{AI}\%$, we then can calculate the "\NewTerm{net annual coupon at maturity (NEC}" by the following trivial relation:
		
	\end{enumerate}
	At the opposite of to the calculation of accrued interest of the coupons of a bond, the calculation of the accrued dividend of a stock is veeeery difficult to estimate. The stock price, however, is influenced by the more or less near the date of the dividend payment. 
	
	\begin{tcolorbox}[title=Remark,colframe=black,arc=10pt]
	Let us indicate that the market in which the issuers selling their bonds (new) by auction, syndication or by direct placement to investors is named "\NewTerm{primary market}". The market in which investors exchange among themselves bonds (sometimes the same obligations which are offered at different prices) already in circulation (second hand market) is named "\NewTerm{secondary market}".\\
	
	Below an example of announcement of bond issuance by syndication on Bloomberg\textsuperscript{TM} through a new:
	\begin{figure}[H]
		\centering
		\includegraphics{img/economy/bond_syndication_news.jpg}
		\caption{Current value of a bond based on the market rates (yield)}
	\end{figure}
	On the secondary market bonds are quoted as a percentage of their face value (i.e. when they are traded on $100\%$ of the nominal we, as we alread know, that they are "au pair") and the quoted price market (traded price) is sometimes named as we also alread know the "clean price", to get the overall price ("\NewTerm{dirty price}")  of the transaction (purchase or sale) we must added the accrued interest of the coupon and other taxes and trading costs.
	\end{tcolorbox}
	
	\pagebreak
	\paragraph{Warrants}\mbox{}\\\\
	\textbf{Definition (\#\mydef):} Basically a "\NewTerm{warrant}" or "\NewTerm{option warrant"} ("\NewTerm{stock-warrant}" or "\NewTerm{equity warrant}" or "\NewTerm{warrant}"... as their are many type of warrants it is important to precise!) is the privilege sold by one party to another, that gives the buyer the right, but not the obligation, to buy (call) or sell (put) at an agreed-upon price (often an average of stock prices before the issuance of the warrants) within a certain period or on a specific date another financial underlying security \underline{that has to be issue} (share, bond or any other good...). As these both financial instrument are linked to an underlying they therefore belong both to the category of derivatives instruments.
	\begin{figure}[H]
		\centering
		\includegraphics{img/economy/warrant.jpg}
		\caption{Warrant old style paper version}
	\end{figure}
	\begin{tcolorbox}[colframe=black,colback=white,sharp corners]
	\textbf{{\Large \ding{45}}Example:}\\\\
	The warrant of a given company X gives to possibility to subscribe for one share of the company at a price of $500.-$ until April 30, 2004. If the share of X exceeds the level of $525.-$, the warrant that allows in this situation the right to purchase one share at a lower cost to the stock market proves a winning placement. If the action is so for example traded at $525.-$ in April 30, 2004, the gain will be worth $25.-$.
	\end{tcolorbox}	
	
	In addition, Warrants (also Call Options), are financial assets with low risk since there is no obligation to apply them and they were offered... it could also be noted that many companies cancel employee Warrants when they leave the company too early...
	
	Even if we don't have seen yet what are Call Options (see further below)  it seems to us important to make already the \underline{DEFAULT} difference between Warrants and Call Options as they are very similar. For instance, many warrants confer the same rights as equity options and warrants often can be traded in secondary markets like options. However, there also are several key differences between warrants and stock options (or more generally equity options):
	\begin{itemize}
		\item The main default difference is straightforward and we already know it. It's the "\NewTerm{dilutive effect}" (see definition during our study of shares). Indeed an option is the right to purchase an existing share of a company's stock from the company at a specific price (usually the fair market value of that share on the date of issue), whereas a warrant gives the holder the right to purchase a share that will need to be issued (that is, "created") in the future. Therefore there is a boosting int the total share count.

	 \item Warrants are issued by private parties, typically the corporation on which a warrant is based, rather than a public options exchange.
	 
	 \item Warrants are considered over the counter instruments and thus are usually only traded by financial institutions with the capacity to settle and clear these types of transactions.
	 
	 \item A warrant's lifetime is measured in years (as long as 15 years), while options are typically measured in months. Even LEAPS (long-term equity anticipation securities), the longest stock options available, tend to expire in two or three years.
	 
	 \item Warrants are not standardized like exchange-listed options. 
	 
	 \item The tax rules governing options and warrants are completely different.
	\end{itemize}
	
	\begin{tcolorbox}[title=Remark,colframe=black,arc=10pt]
	The big problem with warrants is that because the supply is limited, brokers can do all sorts of nasty things to manipulate the price of the warrants.  Because anyone can issue an option, there are fewer nasty things that you can to do manipulate their price.
	\end{tcolorbox}
	
	The warrant (as Call Options) allows to be interested in the rise or fall of a security without having to devote the same amount of money than buying shares directly. Therefore, on acquisition, if the underlying security has a higher value than on the warrant, the buyer will make a profit named "\NewTerm{surplus value acquisition}". After the buyer that now owns the underlying securities may well sell them when the price is higher than when it made the acquisition and then it creates a (pseudo) second benefit named  "\NewTerm{cession plusvalue gain}".
	
	\begin{tcolorbox}[title=Remark,colframe=black,arc=10pt]
	One warrant can be so attached to the issue of a share or obligation. Therefore, depending on the case, we speak about "shares warrants to purchase shares (SWPS)"  or "bonds warrants to purchases shares (BWPS)", but also of "bonds warrants to purchase bonds (BWPB)" or about "shares warrants to purchase bonds (SWPB)".\\
	
	Since the issuance of these composed instruments, all split into parts: the stocks or the bonds become again classic titles and warrants has its own life. They are traded separately after the issuance.
	\end{tcolorbox}
	
	For the mathematical point of view the pricing future value of warrants is not easy as we can not apply the Black \& Scholes equation that we will prove later as:
	
	\begin{itemize}
		\item A warrant has a long life (2-4 years) and makes it difficult to accepte the consistency of the hypothesis of constant interest rates used by the Black \& Scholes model.
		
		\item Any operation of the issuer that changes the underlying security value affects the value of the warrant. Indeed, companies have the legal right to to issue a new contract for the Warrants and to change their value and the time period of their validity!
		
		\item If the underlying warrant is a bond, the price of this latter changing over time and knowing that the more a bond approaches maturity, the more its value tends toward its redemption price. Its volatility is gradually reduced, which renders inapplicable the Black \& Scholes model which postulates the constancy of volatility in time!
	\end{itemize}
	Therefore if Black \& Scholes assumptation are not valid the only one simple thing we know so fare is tobuild some naive mathematical model to computer their actual value.
	
	In order to see how this works we need some simple notation (all values are those actual time $t$):
	\begin{enumerate}
		\item $E$ is the value of the firm's equity at time $t$.
		\item $S$ is the common share price of the outstanding shares at time $t$.
		\item $N$ is the number of old common shares outstanding prior to any exercise of the warrants.
		\item $W$ is the value of the warrant at time $t$.
		\item $M$ is the actual number of warrants.
		\item $X$ is the exercise price of the warrants.
	\end{enumerate}
	
	The choice of whether to exercise a warrant at maturity can be based on the following criteria. We exercise if
	
	That is to say that we exercise the warrant if the average value per share of the company is greater than the actual value of the warrant (obvious!).
	
	If the stock price after all the warrants are exercised is greater than the warrant's exercise price, all holders of the warrants will exercise. Hence the payoff of a warrant at maturity is:
	
	This is quite different than the payoff of a regular Call Option as we will see later in that the payoff depends on the number of warrants outstanding.
	
	\begin{tcolorbox}[title=Remark,colframe=black,arc=10pt]
	The development of liquidity in the equity and bond markets has prompted financial institutions to issue warrants to acquire securities (underlying) existing independently of the financial operations of the concerned company. These warrants relate only to investors and are issued only by banks between them and do not allow the business financing (so this is pure speculation!). These warrants (also traded) are named "\NewTerm{covered warrants}" because on issue, the financial institution covered itself by purchasing shares on the market.
	\end{tcolorbox}
	
	\pagebreak
	\paragraph{Futures \& Forwards}\mbox{}\\\\
	When we think about selling and buying commodities/currencies we target three types of markets:
	\begin{enumerate}
		\item An immediate physical cash settlement market also named "spot market" involving producers and consumers.
		
		\item A physical market at maturity also named "Forward/Future market" involving producers and that have for desire to hedge price risks consumers and speculators that want to play with the trend of the underlying to make money.
	\end{enumerate}
	\textbf{Definition (\#\mydef):} A "\NewTerm{Forward contract}" (or "\NewTerm{commodity Forward}") and a "\NewTerm{Future}" are both contracts to buy or sale of a financial product/underlying (therefore they belong both to the category of derivatives instruments), passed between two counterparties, all of whose characteristics are fixed in advance: settlement date Forward prices, etc. The price entered is named "\NewTerm{Forward price}" or respectively "\NewTerm{Future price}" or simply "\NewTerm{fair value}" for the both, and the exchange and also the payment will be mandatory at that price regardless of the current market price underlying (also named "\NewTerm{spot price}") on the date of delivery (i.e. maturity date)!
	
	Fundamentally, Forwards and Futures contracts have the same function: both types of contracts allow people to buy or sell a specific type of asset at a specific time at a given price. The purpose of Futures and Forwards for stakeholders is obviously to freeze the prices over the future: it is in this case a hedge (hedging) in the point of view of the seller.

	However, it is in the specific details that these contracts differ.
	\begin{enumerate}
		\item Futures contracts are exchange-traded and, therefore, are standardized contracts. Forward contracts, on the other hand, are private agreements between two parties and are not as rigid in their stated terms and conditions. Because Forward contracts are private agreements, there is always a chance that a party may default on its side of the agreement. Futures contracts have clearing houses that guarantee the transactions, which drastically lowers the probability of default to almost never.
		
		\item Specific details concerning settlement and delivery are quite distinct. For Forward contracts, settlement of the contract occurs at the end of the contract. Futures contracts are marked-to-market daily, which means that daily changes are settled day by day until the end of the contract. Furthermore, settlement for Future contracts can occur over a range of dates. Forward contracts, on the other hand, only possess one settlement date.
		
		\item Lastly, because Future contracts are quite frequently employed by speculators, who bet on the direction in which an asset's price will move, they are usually closed out prior to maturity and delivery usually never happens (closed out prior is not authorized with Forwards). On the other hand, Forward contracts are mostly used by hedgers that want to eliminate the volatility of an asset's price, and delivery of the asset or cash settlement will usually take place.  Today, around $97\%$ of Futures trading is done by speculators.
		
		\item Forward don't have brokers margins. Indeed, margin are set by the futures exchanges to lower the risk exposure. Therefore the higher is a future margin the higher is the volatility of the underlying. As distinction is made between:
		\begin{itemize}
			\item \NewTerm{Initial  Margin}: That is the amount of money that is required to open a buy or sell position on a Future contract.
			
			\item \NewTerm{Margin Maintenance}: That is the amount of money where a loss on your Futures position requires you to allocate more funds to bring the margin back to the initial margin level.
			
			\item \NewTerm{Margin Calls}: A margin call on Future contracts is triggered when the value of your account drops below the maintenance level. The investor has then the option of taking and opposite positions to eliminate the margin call to pay the difference. In any case if the buyer don't pay the broker bears the cost. If the broker then defaults the clearing house bears the cost!
		\end{itemize}
	\end{enumerate}		
	For summary:
	
	\begin{minipage}[t]{0.5\linewidth}
	    \textbf{FORWARD}:
	    \begin{itemize}[leftmargin=10pt]
	    	\item Contracts executed by banks
		    \item Private contract (unregulated)
		    \item Not standardized contract
		    \item High counterparty risk
		    \item No initial payment required
		    \item Normally one specified delivery date
		    \item Settle and the end of maturity
		    \item Traded $24$ hours a day
		    \item No cash exchange prior to maturity
		    \item More than $90\%$ of contract settled by actual delivery of assets
		Delivery or final cash settlement usually takes place
	    \end{itemize}
	    \end{minipage}%
	    \begin{minipage}[t]{0.5\linewidth}
	    \textbf{FUTURES}:
	    \begin{itemize}[leftmargin=15pt]
	    	\item Contracts executed by brokerage houses
		    \item Traded on organized exchanges
		    \item Standardized contract (regulated)
		    \item Low counterparty risk
		    \item Initial margin payment required
		    \item Range of delivery date
		    \item Daily settled
		    \item Traded $4-8$ hours a day
		    \item Profit/Loss are paid in cash
		    \item Not more than $5\%$ of contract settled by actual delivery of assets
		    \item Contract normally closed out prior to the delivery
	    \end{itemize}
	\end{minipage}\par\bigskip
	
	\begin{figure}[H]
		\centering
		\includegraphics[scale=0.5]{img/economy/bloomberg_future.jpg}
		\caption{Future Equity description in Bloomberg\textsuperscript{TM} Terminal}
	\end{figure}
	
	In 2003, there would have been $2,848$ million Forward contracts exchanged on markets respectively $2.06\%$ for exchange rates, $17.08\%$ for raw materials ($43.9\%$ for food, $10\%$ for metals, $45.8\%$ for fuels and $0.3\%$ other), $25.48\%$ for shares and bonds and $55.37\%$ on interest rates.
	\begin{tcolorbox}[colframe=black,colback=white,sharp corners]
	\textbf{{\Large \ding{45}}Example:}\\\\
	A Swiss industrialist knows that he must receive an important amount of money in euros in six months. To hedge against a falling of euro (...), he buys a Forward contract of sale, with maturity six months on euro, in swiss franc (if financial operators also know that the euro will fall, the contract Forward sales will have a high enough cost so that the purchase contract is not interesting). Note that this currency hedging operation may be adversely to him if in six months, the contract gives less than the future spot exchange rate.
	\end{tcolorbox}
	
	As we have already mention it, the major difference between Futures and Forwards is that the gains and losses in relation to underlying fluctuations are paid to the counterparty on a daily basis for Futures ("\NewTerm{daily mark to market margin}" or "\NewTerm{daily mark to market settlement}" ) !!! This means that the gain or loss is already almost entirely / disbursed the day of the settlement date (only remains to adjust the difference of the last day).
	
	\begin{tcolorbox}[colframe=black,colback=white,sharp corners]
	\textbf{{\Large \ding{45}}Example:}\\\\
	Consider a contract about $1,000$ tons of apples for delivery on a given settlement date. A future contract is available for a future price of $200.-$ by $1,000$ tons. The initial margin is of $20.-$ and the maintenance margin of $10$. \\
	
	Let's say that on first trading day the traded value of the future value is $195.-$. Obviously the buyer of the $200.-$ contract will feel sad and the seller will feel smart! But the principle behind the mark to market is the margin will help the seller to avoid that the buyer close the position. Indeed, as there is $20.-$ margin, the $5.-$ loss can be taken on this margin and theoretically nothing is lost and the $200.-$ Future will be reset to $195.-$ because remember that if the seller don't do that the buyer will close is positions. Then it remains $15.-$ margin.\\
	
	Let's say that on second trading day the traded value of the Future value is $188.-$. Obviously the buyer of the now $195.-$ contract will feel once again sad and the seller will feel smart! But the principle behind the mark to market is the margin will help again the seller to avoid that the buyer close the position. Indeed, as there is $15.-$ margin remaining, the $7.-$ loss can be taken on this margin and theoretically nothing is lost and the $195.-$ Future will be reset to $188.-$ because remember that if the seller don't do that the buyer will close is positions. Then it remains $8.-$ margin. But now as the remaining margin is less than the maintenance margin of $10.-$, through a margin call, the buyer has the option to put $12.-$ again to be at the initial value of $20.-$ or to close the position. \\
	
	 If the buyer refuses to add the extra $15.-$ dollar. The contract is unbound. So the seller can sell his contract to someone else. The loss of the buyer is $20.-$. In a "good" contract those $20.-$ aren't lossed by the buying or selling party. The margin is just there as an assurance to both parties that ensure they will keep their position until the end of the deal (or lose $20.-$).\\
	
	At the opposite if the spot price goes up, then the seller has to transfer that difference by which the price increased to the buyer in order to cancel the changes in price. The seller also has a margin not only the buyer!\\
	
	It must be understood that if the price of the contract goes down to $100.-$, then the buyer only pays $100.-$, but then adds on an extra $100.-$ in the margin account. If the final value of the contract is $200.-$ then the seller and the buyer thakes back the $20.-$ initial margin.\\
	
	As we can not not cancel ("opt-out") a Future contract, "closing out" a Future contract means that you enter into a contract that is opposite to the one you initially entered. In this example, the losing party (the buyer) would sell a contract for delivery at $185.-$. \\
	
	Thanks to this example we can see that the margins reduce volatility by eliminating counter-party risk, that is to say the risk that the exchange will end up buying the product because the original buyer disappears without having paid anything.
	\end{tcolorbox}
	
	We see with the above example that the Future contract is something very safety for the seller of the commodity (farmer typically), assimilated to the "hedger", as he will have anyways at least his $200.-$ if the market price decrease (but if the market price raise he will virtually loose money as he will not get the positive difference, therefore pricing Future correctly is important to be at the best right price!!).
	
	But how do speculators make money with Futures because of this mark to market margin????? Simple! Imagine they buy at day $d$ the Future contract for $200.-$ plus the margin hoping that the price of commodity will increase. If $n$ days after, the price of the commodity increase of $\delta$ (the seller has to give us the $\delta$ day by day thanks to the mark to market margin) so will do the old contracts and new investors coming in the market will have the choice to buy a completely new Future contract (issue by the farmer) with the price $200+\delta$ and a given margin or an old contract in possession of a speculator  who's spot price is also equal to $200+\delta$ plus the margin. And therefore the speculator has make a profit of $\delta$!!! In fact the speculator wins and keeps the day to day cumulative margin!!!!!!!
	 
	Two types of executions may occur:
	\begin{enumerate}
		\item The "\NewTerm{physical settlement}" the underlying is really exchanged (which is rare in the virtual economy, but in the real economy should be a reality...).
		
		\item The "\NewTerm{cash settlement}": if the price of the underlying is below the set price, the buyer supplied himself on the market he wants and pays the difference to the seller and vice versa.
	\end{enumerate}
	In practice, speculators, named "\NewTerm{index managers}", for which the raw material representing the underlying matter is of no interest, have the mission to reposition their portfolios before settlement to avoid physical delivery. This procedure, named "\NewTerm{roll-over}". The roll-over is normally done on the last day or last week of the month preceding the start of the period of delivery period.
	
	A Future or Forward contract must refer to the role of each participant (buyer or seller), an official reference market - underlying - (stocks, indexes, bonds, currency market, interest rates, etc.)  - whose spot price at time $t$ will be denoted $S_t$, the settlement date reference denoted $T$, the nominal price $K$, the notional amount $N$ to which the transaction of the Futures of Forwards is targeted (which is therefore a quantity for physical settlement and that is mathematically almost always reduced to unity), payment instructions (physical or cash settlement) and for Futures the various requires margins $M_i$ (initial margin), and $M_m$ (maintenance margin).
	
	At maturity $T$, the gain or "payoff" of the contract is in both cases as we will see in the table below (whether for a Forward or Future):
	 
	This value\footnote{Future prices are sometimes denoted $\mathcal{F}_t$ in textbooks} simply refers to losses or realized gains (we will reverse the sign depending on whether the trader is a buyer or seller). Indeed, it can be understood better by seeing the cash flow amount in the table below (this also helps to explain why the losses or gains are daily for Futures and that if in the case of gains we can reinvest them quickly on the market):
	\begin{table}[H]
	\begin{center}
		\definecolor{gris}{gray}{0.85}
			\begin{tabular}{|c|c|c|}
				\hline
				\multicolumn{1}{c}{\cellcolor{black!30}\textbf{Time}} & 
  \multicolumn{1}{c}{\cellcolor{black!30}\textbf{Forward}}  & 
  \multicolumn{1}{c}{\cellcolor{black!30}\textbf{Future}}  \\ \hline
				$0$ & $0$ & $0$\\ \hline
				$1$ & $0$ & $S_1-K$\\ \hline
				$2$ & $0$ & $S_2-(K-(K-S_1))=S_2-S_1$\\ \hline
				$3$ & $0$ & $S_3-(S_2-(S_2-S_1))=S_3-S_2$\\ \hline
				$4$ & $0$ & $S_4-S_3$\\ \hline
				$5$ & $0$ & $S_5-S_4$\\ \hline
				
				$\cdots$ & $0$ & $\cdots$ \\ \hline
				$\cdots$ & $0$ & $\cdots$ \\ \hline
				$\cdots$ & $0$ & $\cdots$ \\ \hline
				$T-1$ & $0$ & $S_{T-1}-S_{T-2}$\\ \hline
				$T$ & $S_T-K$ & $S_T-S_{T-1}$\\ \hhline{|=|=|=|}
				Total: & $S_T-K$ & $S_T-K$\\ \hline
		\end{tabular}
	\end{center}
	\caption{Cash-Flow of Forward and Future of the point of view of a buyer}
	\end{table}
	
	\subparagraph{Futures \& Forwards naive pricing}\mbox{}\\\\
	The pricing of Forward or Future contracts as part of a naive approach is quite simple if the underlying yield is deterministic and the average geometric return of the market as often assumed to be known .... But we must be honestly agree on one thing ...: the final price of a Forward contract or its Future equivalent (at least in the conditions seen before ..) must be constituted at final only of the risk premium  that will take upon him the seller of the contract.
	
	For the pricing let start with the first element: there is a first information needed from the perspective of contract vendor, it will already actualize $K$ (fixed term value of the contract) to the free risk rate of the to already know the lower limit of price that it will ask today to get $K$ when the time comes (if the seller does not ... other vendors will offer lower prices to the buyer - competition does its job - or the buyer will do the calculation work himself). That is to say:
	
	And of course if the sale of the contract is done later than the date of issue, we have:
	
	What will be written in the continuous case (and adapting traditional notation that of continuous rate of return):
	
	Therefore this is the lower limit to apply at any time $t$ for a potential contract buyer (waiting the maturity of the this amount of money will be placed in an investment  considered as safe). However, this is not strictly speaking the risk premium and thus,, the contract price in the sense we usually hear it. Indeed, the seller of the contract will have to buy the underlying  of contract. If for example the sale of the contract is made just at its issue (that is to say at time $T$ before maturity), then we have the actualized  price of the assumed (projected) value of the underlying at maturity knowing it will have himself intrinsic yield (that the seller will take in the worst case: that is to say a positive rate) which we will denote $y$ (the "spot-rate") and in non-stochastic framework we can do this calculation only based on something known which is the underlying spot price:
	
	That is to say for any time $t$:
	
	We then have the actualized payoff of a Forward or Future corresponding to the risk premium which is given in any time under the assumptions underlying the previous relations by the difference (obviously buyer point of view!):
	
	which is written by tradition in the purpose to make appear the Future or Forward price at each day $t$:
	
	The factor on the left is the risk free actualization factor. Therefore we guess that we can focus only on the parenthesis: $(S_te^{-y\tau}e^{r\tau}-K)$ that is the non-actualized risk premium. Fro this we conclude that:
	
	and therefore:
	
	It is interesting to notice that $y$ is subtracted of the risk free rate $r$ and therefore make the contract price lower. This is obviously valid if be in possessions of underlying can make us win money, but if this make us loose money (for example the cost of storage can be very expensive), then $y$ will be negative and will increase the Forward or Future price.
	
	If the underlying pays dividends, then obviously the value above is overestimated. We have logically then to actualize dividends and to subtract it to have the value of the Future or Forward equal to:
	
	
	\textbf{Definitions (\#\mydef):}
	\begin{enumerate}
		\item[D1.] As for any derivative, we say that the market is "\NewTerm{contango}" if the option price (in our case: the Future or derivative) is greater than the spot price of the underlying.
		
		\item[D2.]  As for any derivative, we say that the market is "\NewTerm{backwardation}" if the option price (in our case: the Future or derivative) is lower than the spot price of the underlying.
	\end{enumerate}
	
	Therefore, more explicitly, contango is a situation where the Futures price (or Forward price) of a commodity is higher than the expected spot price. In a contango situation, hedgers (commodity producers and commodity users) or arbitrageurs/speculators (non-commercial investors), are "willing to pay more (now) for a commodity at some point in the future than the actual expected price of the commodity (at that future point). This may be due to people's desire to pay a premium to have the commodity in the future rather than paying the costs of storage and carry costs of buying the commodity today". And a market is "in backwardation" when the Futures price is below the expected future spot price for a particular commodity. This is favorable for investors who have long positions since they want the futures price to rise.
	
	A contango is normal for a non-perishable commodity that has a cost of carry. Such costs include warehousing fees and interest forgone on money tied up (or the time-value-of money, etc.), less income from leasing out the commodity if possible (e.g. gold).
	
	The figure below depicts how the price of a single Forward contract will behave through time in relation to the expected Future price at any point time:
	\begin{figure}[H]
		\centering
		\includegraphics[scale=0.8]{img/economy/contago_backwardation.jpg}
		\caption{Typical spot price and Future/Forward price dynamics}
	\end{figure}
	
	\begin{tcolorbox}[colframe=black,colback=white,sharp corners]
	\textbf{{\Large \ding{45}}Example:}\\\\
	A customer wants to buy to us in $6$ months US dollars with Euros (or in other words it pays in euros in $6$ months a machine that he bought today and we are working in US dollars). We would then protect us already against changes in the raise of the monetary value (bid/ask) of the US dollar (because if the US dollar rises then we lose verbatim money since the Euro is then less so for a $1$ US dollar). However, there is available on the market only Futures issued already $6$ months before with maturity (cash settlement) in $12$ months. Knowing that the market risk-free yield is $3\%$ (in continuous yield rate...), that the dollar rate is assumed at worst at $+1.5\%$ (also in continuous yield rate...) and that we want to know the contract price in cash in monetary unit and that contracts $6$ months ago with maturity in $6$ months had an exercise price of $1.00.-$\euro{} of $1.50.-$\$:
	
	dollars per unit of currency in desired euros (so we pay $11\%$ as a risk premium which is a typical value of what is practiced empirically on markets: added to $5\%$ to $15\%$). That sounds like a lot but it limits our risk and transfers it on the seller of the contract. Moreover, buying the machine we could also see first the risk premiums and adapt price negotiations accordingly.\\
	
	The spot price of the Future contract is therefore obviously:
	
	that is to say $1.612.-$\$ for $1.00.-$ \euro{} (which corresponds to the result returned by the GUIDE R package). So we cannot obviously get $1.50.-$\$ for $1.00.-$\euro{} and this difference is explained by risk taken by the seller of the Future contract.\\
	
	So the total price of the Future is finally $1.612.- + 0.11.-$\$ while the spot price is $1.60.-$\$. The difference, omitting the risk premium, is of $-0.012$ and is named the "\NewTerm{basis}" or "\NewTerm{cost of carry}" and it indicates to a future buyer that market agents are planning an increase if the basis is negative ("under") and a decrease if positive ("over").
	\end{tcolorbox}
	\begin{tcolorbox}[title=Remark,colframe=black,arc=10pt]
	In practice, as with many financial instruments, the risk premium is included in a broader terminology fees we call "\NewTerm{margin money}" which includes transaction costs, risk premium, insurance costs for bankruptcy of the buyer of the Future/Forward contract, etc. Some institutions, however, use the term "\NewTerm{cost of carry}" for all types of costs and then the risk premium alone is named "\NewTerm{net cost of carry}". 
	\end{tcolorbox}
	We will not look now the mathematical tools for the pricing of Futures/Forward contracts using stochastic return, because as they are "only" a special case of the pricing of options that we will see later it would be redundant to present every theorems twice while there is only a small change to make for in the Black \& Scholes model to get one from the other.
	
	Indeed, as we will see an option is a derivative that does not necessarily result in an execution due to its optional character in relation to a Future/Forward contract where the settlement date is fixed in advance.
	
	\subparagraph{Cox-Ingersoll-Ross equality of the Future/Forward price}\mbox{}\\\\
	Under a specific condition and a given strategy it is possible to demonstrate that the price of a Future is equal to that of a Forward at maturity (we own this proof to J.C. Cox, J.E. Ingersoll and S.A. Ross in their article of 1981).

	Let us suppose that the market risk-free rate is $r$ and that the latter is constant and that each day $i$ we apply the very special strategy of selling Futures in a quantity $e^{-ri}$ in the so-called "\NewTerm{CIR conditions}" (these are the initials of Name of the authors of the strategy).
	
	So in the beginning we sell Forward but we have to actualize the price, so the actualized price is then:
	
	On the second day, we sell contracts that must also be actualized:
	
	And we quickly see that we will have in total:
	
	This is simplified immediately to:
	
	We thus fall back on the actualized of the pay-off of the forward according to the table seen earlier above. Obviously, this equality is valid only if the return of the risk-free market is constant and in the strategy mentioned above, which is not the case (realistic) the majority of time!
	
	As far as we knew, even if this theoretical result is interesting, practical implications, however, of this observation are minimal.
	
	\subparagraph{Futures \& Forwards commodity hedging}\mbox{}\\\\
	A company will prefer Future/Forwards to options if its purpose is to know its costs in advance so that it is $100\%$ certain to consume the underlying. While the purchase of an option requires the payment of a risk premium generates a cost of opportunity. In the case where the company is not sure to use the underlying, it will prefer the option that can be executed or not, which means that the management reserves a good result never mind what will happen but they will pay in counterpart the option risk premium.
	
	Let us study now an interesting case of risk hedging of raw materials. Imagine that a company need to buy for a total amount $S$ a raw material but for which there is sadly no Forwards or Futures contracts available (and no partner agree to make an OTC contract). This company wish to hedge against an increase in the price of the raw material she is interested for and therefore at least two typical strategies are available to it:
	\begin{enumerate}
		\item Found Future or Forward contracts of a commodity positively correlated to the raw material of interest. Therefore, hoping that this commodity replicates well the behavior of the raw material of interest and that when its price will increase the other one will also probably do same (and if possible in the same proportions)... and the idea is to have a buyer (long) position of contract to buy at maturity the commodity to a value less than the spot price and hope to found someone to sell the latter to people interesting to buy quickly and use the payoff to buy the raw material of interest with a minimum loss (an in the ideal case with a loss equal to zero or even a positive remainder).
		
		\item Found Future or Forward contracts of a commodity negatively correlated to the raw material of interest. Therefore, hoping that this commodity replicates well the behavior of the raw material of interest and that when its price will decrease the other one will probably increase (and if possible in the same proportions)... and the idea is to have a seller (short) position of contract to sell at maturity the commodity to a value more than the spot price and use the payoff to buy the raw material of interest with a minimum loss (an in the ideal case with a loss equal to zero or even a positive remainder).
	\end{enumerate}
	
	Mathematically the strategy consists to write in the botch case (we will see later what sign corresponds to which strategy):
	
	where $\Delta S$ is the variation of the raw material price on the market and $\Delta F$ is the payoff of the contracts that replicates positively or negatively the variation (increase) of the price $S$. We have $\Delta V$ that is the difference of the good or bad replication and that must in the ideal case tends to zero. Another way to see the context, more common as it encompass the point that makes problem (that is to say: the volatility!!!), is to write this strategy with the information available on the market such that we take the variance of the previous relation:
	
	and to search the value $N$ that minimize the variance $\Delta V$ in function of the correlation of the both commodities (that we will have to make appear) and of their volatility (containing implicitly the quantity as we will see). We haven then by derivating relatively to $N$:
	
	Which gives using the relation between the covariance and the linear correlation coefficient proved in the section Statistics:
	
	As the standard deviations are positive and $N$ must be positive (and be rounded to the nearest integer to be a realistic physical solution), it is obvious that the correlation coefficient should always be of opposite sign to the type of the chosen strategy. This is why we have finally for tradition to take(as it still remains the same) the positive sign such as:
	
	and always take the correlation coefficient as positive (absolute value). Therefore $N^{*}$ is the number of Futures or Forward contracts that whatever the strategy, will minimize the total variance.
	\begin{figure}[H]
		\centering
		\includegraphics{img/economy/bloomberg_correlation_matrix_forex}
		\caption{Correlation matrix for various currencies in Bloomberg\textsuperscript{TM} Terminal}
	\end{figure}
	However, it is customary to bring this expression such that we have not the variance of the price change, but in the change in rates especially in the purpose to make appear the quantity!!! Thus, as we have:
	
	Therefore it comes:
	
	Hence:
	
	That it is customary to write in the following condensed form:
	
	
	The variance V therefore becomes using this optimum (we take again the double sign that will be useful to us):
	The variance V therefore becomes using this optimum (we take again the double sign that will be useful to us):
	
	Obviously, the only one sign that seems to have a physical meaning is the "$-$" sign that will decrease the global variance (as it is the purpose of the both strategies). Then we have:
	
	Thus Finally:
	
	\begin{tcolorbox}[colframe=black,colback=white,sharp corners]
	\textbf{{\Large \ding{45}}Example:}\\\\
	Let us consider that an airline company needs $10,000$ tons of fuel (which the current price is $277.-$ / ton) for its aircraft fleet in $3$ months and wants to hedge against possible price increases of this transformed raw material and that there are no direct Future contracts directly fuel.. The airline company then wants to hedge risk by using another specific commodity for which there exists Future contracts which underlying quantity is $42,000$ tonnes ($0.6903 .-$ / ton). We would like to calculate the decrease of the fuel variance  price by taking a purchase position (Call) on this Futures (so we do a down bet for the replicating commodity) knowing that the variance at  $3$ months of fuel is estimated to be $21.17\%$ and that of the replicating commodity being of $18.59\%$ over obviously  the same period and that their correlation is in absolute value of equal to $0.8242$.\\
	
	Then we have:
	
	As we have for the volatility of the fuel in money:
	
	And once covered by the replicating commodity, it becomes:
	
	The hedging by Future contracts reduce the volatility of cash of a factor around $43.38\%$.
	\end{tcolorbox}
	\begin{tcolorbox}[title=Remark,colframe=black,arc=10pt]
	Some practitioners use the following empirical indicator named "\NewTerm{Ederington efficiency}" as hedging quality:
	
	Which is nothing more than the ratio of the difference of the non-hedged variance with the hedged one by the non-hedged variance. In the example above its value is $67.49\%$. More this ratio is close to unity, the better it is! 
	\end{tcolorbox}
	When the relation:
	
	is applied not with a numerator that corresponds to a raw material but to stock portfolio, this technique is named "\NewTerm{stock portfolio hedging by replication}" or more simply "\NewTerm{equity portfolio hedging}" and is then rather denoted as follows:
	
	where $V$ stands for "value" (absolutely no relation with the notation of the variance!!!). Or even in the following form ($P$ being the value of the portfolio and $F$ that of Futures):
	
	\begin{tcolorbox}[colframe=black,colback=white,sharp corners]
	\textbf{{\Large \ding{45}}Example:}\\\\
	A trader has a portfolio of  $2,000,000.-$ of IBM stocks. He would like to hedge this portfolio thinking that the market will go down on the S\&P 500. Therefore he sells ("short") Futures for an amount of $225,000.-$ and he know that the beta correlation between IBM and S\&P 500 is equal to $1.1$ (a beta less than $1$ indicates that the investment is less volatile than the market, while a beta more than $1$ indicates that the investment is more volatile than the market). The number of Futures to sale is then (therefore if the portfolio value goes down with the same magnitude as the S\&P 500 the trader will lose nor gain anything):
	
	If the trader has purchased Futures to protect himself from the decrease he expected to happen (bet down price and therefore "short" strategy for recall...) then if IBM shares lost $10\%$, but the index S\&P 500 index lost more than what the beta indicated such for example $15\%$, so even if the IBM portfolio had lost $200,000.-$ of its value, the fact that the S\&P 500 has lost $15\%$, thanks to the Futures, the trader will have made an indirect gain of:
	
	So ultimately a gain for the trader of $137,500.-$ (like what... sometimes... bet down is good).
	\end{tcolorbox}
	
	\pagebreak
	\subparagraph{Options}\mbox{}\\\\
	Options are "\NewTerm{contingent asset}" (also named "\NewTerm{contingent claim}"), that is to say, a particular form of a title (contract) giving the holder in counterpart of payment the right, but not the obligation to buy or sell a given amount of a financial asset (stock, bond or foreign exchange), at a given date or up to a given date (expiry or maturity) to a price fixed in advance.
	
	It is primarily as Futures and Forwards also a derivative that permits to hedge the risk of market fluctuations and to speculate. For example, if Boeing sells aircraft in euros but produced in the dollar zone. The sale price is fixed today, but the sale is made on delivery! Boeing must therefore protect itself against the risk of the exchange rate (which can sometimes be of a magnitude $100\%$ in a few years). Generally, most big companies protect themselves from these risks by purchasing in banks derivatives such as options.
	
	In 2003 there would have been $5,210$ million of options traded on the markets and respectively $0.98\%$ for raw materials, $0.28\%$ for exchange rates, $5.80\%$ for the interest rates and $92.94\%$ in securities and bonds.
	
	\begin{tcolorbox}[title=Remarks,colframe=black,arc=10pt]
	\textbf{R1.} We will return further below in detail on the mathematics of the options that are important products as also many other derivatives relies on the same mathematical properties.\\
	
	\textbf{R2.} Options involve zero-sum game (\SeeChapter{see section Game Theory page \pageref{zero sum or non zero sum game}}) in the sense that for every seller there is a buyer and a seller and what one wins, the other lose it (and vice versa).\\
	
	\textbf{R3.} The main difference between options and Futures lies in the fact that options represent a right to buy or sell at the contract maturity while Future represent the obligation to exercise the contract at maturity.
	\end{tcolorbox}
	
	\textbf{Definitions (\#\mydef):}
	
	\begin{enumerate}
		\item[D1.] An "\NewTerm{option}" is a derivative product that gives the right, but not the obligation, to buy ( "\NewTerm{call option}", also just "\NewTerm{Call}") or sell ("\NewTerm{put option}", also just "\NewTerm{Put}") a specified amount of an underlying asset $S$ (share, bond, stock index, currency, row material, another derivative product, etc.) to a predetermined price named "\NewTerm{strike price}" denoted by $K$ or "\NewTerm{exercise  price}"and therefore denoted by $E$ and during (until to) a given time named "\NewTerm{maturity}" denoted by $T$ in exchange of a "\NewTerm{risk premium}" depending (denoted by $C$ or $P$ for the Call to Put) on the intrinsic value at maturity of the option name "\NewTerm{flow}" or more often "\NewTerm{terminal payoff}" (and by some practitionners: "\NewTerm{stochastic target}"). The determination of the premium with mathematical models is what we name the "\NewTerm{pricing of the option}". An option is therefore a kind of "insurance policy" to protect against variations in costs by transferring the risk to a person willing to sell the option against a risk premium ... this latter being also often named "\NewTerm{insurance premium}".
		
		\item[D2.] We talk about "\NewTerm{spot price}" or just "\NewTerm{spot}" to designate the current market price $S$ of the underlying asset at the time of an immediate transaction of the option (call or put). If the underlying is a foreign currency exchange rate, then we speak of "\NewTerm{cross over}" or simply "\NewTerm{cross}".
		
		\item[D3.] We speak of "\NewTerm{Forward rate}" or simply "\NewTerm{Forward}" to describe the rate that will be applied to the underlying asset $S$ in a transaction at the maturity option (call or put). We then fall back on the definition of a Future contract as seen above (+ some given conditions that we will see further below).
		
		\item[D4.] If the intrinsic value of an option is positive with respect to the spot price, it is said to be "\NewTerm{in-the-money}". In the case of the purchase of a Call for example, it means that the exercise price of the Call is below the spot price. It is possible therefore to buy cheaper than the traded value at the execution date (maturity most of time) of the option (which is a date between the option issue date and maturity date of the option for recall!!!).
		
		\item[D5.] If the intrinsic value of an option is not advantageous compared to the spot price, it is said to be "\NewTerm{out-the-money}". In this case, the exercise price is above the current spot price for a Call for example. It would therefore be unwise to exercise the Call at the due date (either before), because that would be more expensive to apply the Call than to buy at the spot price on that date!
		
		\item[D6.] If the intrinsic value at this date of the underlying  is equal to its current spot at maturity, the intrinsic value is zero and the value of the option is said to be "\NewTerm{at-the-money}" and abbreviated in many softwares obviously: ATM (we will see later the mathematical implications of this).
		
		\item[D7.] If the intrinsic value of an option is very far form the stock price are referred to as "\NewTerm{deep in-the-money}" or "\NewTerm{deep out-the-money}".
	\end{enumerate}
	
	Given the above definitions and the example about Boeing the options can be naively summarized by the following figure:
	\begin{figure}[H]
		\centering
		\includegraphics[scale=0.55]{img/economy/options_summary.jpg}
		\caption{Options graphical concept summary}
	\end{figure}
	\begin{tcolorbox}[title=Remarks,colframe=black,arc=10pt]
	\textbf{R1.} The usefulness of the existence of options are the same as for the Futures and Forwards: they can be viewed as an artificial way to create assets (in this case: derivatives) to increase financial volatility (standard deviation or "loss / gain deviation") of market and thus guarantee his equilibrium.\\
	
	\textbf{R2.} The holder or buyer of an option contract is said to be in a "\NewTerm{long position}" while its counterpart, the issuer or seller of the contract is said to be in a "\NewTerm{short position}".\\
	
	\textbf{R3.} If the option can be exercised at any time prior to maturity, we talk about "\NewTerm{American option}", if the option can be exercised at maturity, we talk about "\NewTerm{European option}". One unexercised option is considered as "abandoned" (lost). These two families of options in their Cal/Put version are in in practice grouped under the term "\NewTerm{plain vanilla options}" as these are the most common (such as ice cream with vanilla... that are anything but exotic...).
	\end{tcolorbox}
	Alongside the classic options (vanilla options), appear since the years 1990s, on the markets, options named "\NewTerm{exotic options}" sometimes characterized by the name of the place where they were created and the manner of determining the exercise price at maturity (so there exists at most as many mathematical models as there are types of options...) as the:
	\begin{itemize}
		\item "\NewTerm{Asian options}" whose exercise price is based on the average of the underlying price during the life of the option
		
		\item "\NewTerm{Parisian options}" that can be activated or canceled if the underlying traded price remains above a given threshold value during a given time range
		
		\item the "\NewTerm{Russian options}" which are perpetual American options that guarantee to get the maximum margin observed between the issue and the exercise date
		
		\item "\NewTerm{Binary/Digital options}" that gives the right to the amount fixed in advance if the underlying at maturity exceeds the exercise price
		
		\item "\NewTerm{Lookback options}" that give the right at maturity to the difference between the value of the traded underlying at maturity and the value of the min or max during the life of the option
		
		\item "\NewTerm{Quantile options}" that are a refinement of lookback options because the rule of the game is based on a quantile, not on the max or min
		
		\item "\NewTerm{Barrier options (knock out, knock in)}" whose exercise is permitted if the traded value of the underlying cross or not crossed a certain threshold
		
		\item "\NewTerm{Quanto options}" who are on foreign underlying, but paid in local currency (then you should take into account the exchange rate)
		
		\item "\NewTerm{Ratchet options}" that gives the buyer the right to block its gains realized on the underlying during determined intervals during the life of the option
		
		\item "\NewTerm{Compound options}" or "\NewTerm{chooser options}" that are options on options, etc.
	\end{itemize}
	
	\pagebreak
	A small summary before of all derivative that we have seen until we continue is perhaps a good thing:
	
	\begin{minipage}[t]{0.31\linewidth}
	    \textbf{FORWARD}:
	    \begin{itemize}[leftmargin=10pt]
	    	\item Contracts executed by banks
		    \item Private contract (unregulated)
		    \item Not standardized contract
		    \item High counterparty risk
			\item Unlimited profit for a buyer with limited loss, unlimited loss for a seller with limited profit
		    \item No initial payment required
		    \item Normally one specified delivery date
		    \item Settle and the end of maturity
		    \item Traded $24$ hours a day
		    \item No cash exchange prior to maturity
		    \item More than $90\%$ of contract settled by actual delivery of assets
			\item Delivery or final cash settlement usually takes place
	    \end{itemize}
	\end{minipage}%
	\begin{minipage}[t]{0.31\linewidth}
	    \textbf{FUTURES}:
	    \begin{itemize}[leftmargin=15pt]
	    	\item Contracts executed by brokerage houses
		    \item Traded on organized exchanges
		    \item Standardized contract (regulated)
		    \item Low counterparty shared risk
		    \item Unlimited profit for a buyer with limited loss, unlimited loss for a seller with limited profit
		    \item Initial margin payment required
		    \item Range of delivery date
		    \item Daily settled
		    \item Traded $4-8$ hours a day
		    \item Profit/Loss are paid in cash
		    \item Not more than $5\%$ of contract settled by actual delivery of assets
		    \item Contract normally closed out prior to the delivery
		    \item Contract seller has obligation to sell/buy if buyer exercise right
		    \item Preferred by speculators or arbitrageurs
	    \end{itemize}
	\end{minipage}
	\begin{minipage}[t]{0.31\linewidth}
	    \textbf{OPTIONS}:
	    \begin{itemize}[leftmargin=15pt]
	    	\item Contracts executed by brokerage houses or private ("dealer options")
		    \item Traded on organized exchanges or private
		    \item Standardized or non-standardized contract 
		    \item High counterparty risk borne by one party only
		    \item Unlimited profit for a buyer with limited loss, unlimited loss for a seller with limited profit
		    \item Initial margin payment required (but less than Futures)
		    \item Execution date fixed or variable
		    \item Settled maturity of range of date
		    \item Traded $24$ hours a day
		    \item Profit/Loss are paid in cash
		    \item Contract normally closed at maturity
		    \item Contract seller has obligation to sell/buy if buyer exercise right
		    \item Preferred by hedgers
	    \end{itemize}
	\end{minipage}\par\bigskip
	
	\pagebreak
	Here is a summary table between Call and Put:
	\begin{table}[H]
	\begin{center}
		\definecolor{gris}{gray}{0.85}
			\begin{tabular}{|c|c|c|c|c|}
				\hline
				\multicolumn{2}{|c|}{\cellcolor{black!30}\textbf{Call}} &  {} & 
  \multicolumn{2}{|c|}{\cellcolor{black!30}\textbf{Put}}  \\ \hline \hline
				\parbox{3cm}{\centering Buyer of a Call \\ (\NewTerm{Long Call})}  & \parbox{3.5cm}{\centering  Seller of a Call \\ (\NewTerm{Short Call})} & {} & \parbox{3.5cm}{\centering  Buyer of a Put \\ (\NewTerm{Long Put})} & \parbox{3.5cm}{\centering  Seller of a Put \\ (\NewTerm{Short Put})}\\ \hline
				\parbox{3.5cm}{\centering  Has the right, but not the obligation, to buy the underlying stock at the predetermined price until the due date.} & \parbox{3.5cm}{\centering  Has the obligation to sell the underlying asset at the predetermined price if the Call is exercised the due date.} & & \parbox{3.5cm}{\centering  Has the right, but not the obligation, to buy the underlying stock at the predetermined price until the due date.} & \parbox{3.5cm}{\centering  Has the obligation to buy the underlying stock at the predetermined price if the put is exercised} \\ \hline
		\end{tabular}
	\end{center}
	\caption{Differences between Put and Call}
	\end{table}
	So there is obviously an enormous important mathematical difference between options having stocks/bonds. Indeed, the options have an exercise date, their price dynamic can be statistically predictable and this even better when we are close to their date of exercise and this is not applicable for the stocks/bonds because we never know at the strategic level when will be sold or respectively purchased.

	While the naive theoretical models for valuation of options (such as Black \& Scholes) assume a constant historical volatility whatever the price of the underlying in reality the graphical representation of the price of an option based on their underlying price is not a horizontal line but a curve that is usually named a "volatility smile". We will come back later in detail on this concept for which we have an illustration below:
	\begin{figure}[H]
		\centering
		\includegraphics[scale=0.55]{img/economy/bloomberg_volatility_smile.jpg}
		\caption{Typical volatility smile}
	\end{figure}
	\begin{tcolorbox}[title=Remark,colframe=black,arc=10pt]
	As we will prove it further below, some options are named "\NewTerm{non-path-dependent options}" (NPDO) have their price which depends (among others...) only of the final price at maturity of the underlying $K$, while others options named "\NewTerm{path-dependent options}" (PDO) have their price which depends on all the values of the underlying during the term of the option contract.
	\end{tcolorbox}
	
	Let us formalize things now! But without going into too much detail at first time (we keep the details for the study of the Black \& Scholes model later that consists to determine the premium price of an option according to some hypothesis). We consider first for simplicity options on a single underlying not paying dividends.
	
	We will denote the price of the underlying asset of the concerned option by $S_t$ (trade/exchange rate value) at time $t$ and maturity $T$ and will disregard the difference between continental Put and Calls (American and European).
	
	Let us imagine a Call, which gives the holder the right (but not the obligation as we know) to buy the underlying asset at any time between today $t=0$ and the $t=T$ at the (maturity) exercise price $K$ set in advance. Let us take the practical common case of a Call option that protects a business against the rise of the euro/dollar exchange rate for example. Acquired today by the firm, it will therefore give it the right (but not the obligation!) to buy $\$1$ in exchange of $K$ euros (the strike price at maturity $K$ is a fixed characteristic of the contract) to the future date $T$ (maturity date).
	
	If the exchange rate has for value $S_t$ at time $t$ (that is to say today $\$1= S_t$ EUR), this insurance comes from the perspective of the firm to save an amount (a payoff Call from the perspective of buyer):
	
	also denoted by practitioners:
	
	euros at maturity $T$ and denoted (the traded price/rate being a maturity denoted by $S_T$ obviously...). At any time, two major cases occur once for our Call buyer:
	\begin{enumerate}
		\item $S_t<K$: in this case, the Call gives the right to buy the underlying at the price $K$ that we could buy cheaper on the market. In the case of our Call for the dollar/euro it would be better for the buyer to apply the exchange market rates rather than exercise him Call since otherwise he will have less euros for one dollar (that's why a Call has a zero intrinsic value in this situation at maturity). But we will lose the risk premium we paid to purchase Call and this risk premium has also to be taken into account before not exercising the Call.

	The reader most not forget that for European options for we only exercise at maturity. Therefore we don't write the condition $S_t<K$ but $S_T<K$.

		\item $S_t>K$: in this situation, the Call gives us the possibility to purchase the underlying cheaper than the market price/rate. We probably will exercise that opportunity (the profit being the difference between the two prices + take into condieration the paid risk premium). In the case of our Call for the dollar/euro exchange we will exercise our right at the lower rates guaranteed by the option contract ($\$1 = K$ EUR).
	\end{enumerate}
	
	From the perspective of the counterparty (seller of the Call), in the situation (1) it does not pay anything to the buyer, and all is forgotten (the contract expires, any contractual relationship between the two parties disappear). In case (2), the vendor is assigned, it must sell to his counterparty the underlying at price $K$. If it does not hold this underlying, it must first buy it at a most expensive price on market (at the price $S_t>K$). However, in both cases the counterparty (seller of the Call) has get the cash of the risk premium per unit of Call (cash that will perhaps avoid it to loose to much money).
	
	Thus, in the first case, the buyer and seller do not pay anything. In the second case, it is as if the buyer of the Call buy the underlying on the market and received at the same time the amount $S_t-K$ (for the seller it is obviously the reverse). So with these derivatives is the seller of the Call (or Put) which assumes almost all the market risk and of course the interest is quite high to neutralize this risk by using a mathematical formalism to determine the risk premium price (using typically the Black \& Scholes model).
	
	Let us see now an example of the investment perspective (risk-taking is evident in this example):
	\begin{tcolorbox}[colframe=black,colback=white,sharp corners]
	\textbf{{\Large \ding{45}}Example:}\\\\
	Imagine the case of a share which is currently worth $1000.-$ (regardless of the currency) and it is expected to increase by $12\%$ in a year (typically what people expect on stock options plans).\\
	
	Let us imagine also that an investor has the alternative of buying the stock at $1,000.-$ or buy the corresponding Call option at exercise (strike) price of $1000.-$ (therefore assumed to be equal to the actual price of the share, which is not necessarily always the case in practice obviously...) for a premium of $40.-$ (remember that we will see later how to calculate risk premiums). Obviously, the investor can then for $1'000.-$ by $25$ Call options rather than a single stock option if he wants but after he will have to find the money to but the $25$ underlyings... (or found someone else that will buy them).\\
	
	The question is to find the most interesting investment: Thus, an increase of $120.-$ on one year in the case of buying a share represents a return on investment of $12\%$ per year, while the purchase of a Call option will have a return on investment $80.-$ ($120.-$ gains on the sale price less the premium paid of $40.-$) hence $200\%$ of return!!!
	\end{tcolorbox}
	
	It is clear from this example that the purchase of a Call with a same investment has a profitability significantly higher than the purchase of the underlying at least since the option risk premium does not exceed a certain threshold.
	
	Now lets discuss in detail and example another major concept that we have already implicitly presented in the preceding paragraphs and that requires our attention because it is often mentioned by analysts. This is "\NewTerm{leverage effect}" of options that is a double-edged weapon or a portfolio mass destruction weapon (if leverage is not calculated in advance and even approximately).
	
	When we talk about the options, we often retain only the right to buy or sell a financial instrument (at a fixed price and during a given time), neglecting the obligation by the seller of the option. However, the leverage that characterizes these financial instruments can make this a devastating obligation for the seller.
	
	Ok let us see more in detail what we speaking about:
	\begin{itemize}
		\item First with the case of a Call:
		
		The buyer of a Call on an share (for example) limit its risk to the option risk premium price for unlimited earning potential. The sellet of Call is in exactly the opposite position: he is collecting the option risk premium price amount, but takes unlimited risks.
		
		As as theoretical introduction... let us take the example of a share traded $350.-$ at the mid-October. An investor bets on the rising of the share and buys it $12.50.-$ a Call option risk premium with a maturity to next year in January at the exercise price of $380.-$. A graphical representation allows to easily show the relation between the evolution of the share price (on the $x$-axis) and its effect on the buyer or seller of the Call option.
		\begin{itemize}
			\item Let us consider first the buyer of the Call:
			
			As long as the share price remains below $380.-$ ("\NewTerm{leverage value}"), exercise price, the buyer of the Call will have no interest in exercising his option, which is said as we alread know "out of the money" (OTM). By cons, if the share price increased and exceeds the exercise price, the option is said to be as we also know "in the money" (ITM) and it becomes interesting to exercise the option. When the option exercise price is equal to the underlying price of the stock exchange, we say that the option is "at the money" (ATM) as we also alread know. As soon as the share price exceeds $392.50.-$, hence the sum of the exercise price and the option risk premium at the mid-October ($380 + 12.50$), the holder of the Call begins to earn money on his initial investment. If the share price rises suddenly to $500$, an increase of just over $30\%$, the gain will be much more than proportional: for $12.50.-$ invested, the purchaser will realize a profit of $107.50.-$ a gain of $860\%$: this is the famous "leverage effect".
			\begin{figure}[H]
				\centering
				\includegraphics{img/economy/payoff_simple_call_buyer.jpg}
				\caption{Leverage effect for the buyer of a Call (simple Call payoff diagram)}
			\end{figure}
			We can see above that the payoff of a Call from the buyer's point of view is a convex\label{finance convex function} function (\SeeChapter{see section Functional Analysis  page \pageref{convex function}}).
			
			\item Let us consider now the seller of the Call:
		
			As the long as the share remains below $380.-$ ("leverage value" for recall...), the Call seller makes a profit of $12.50.-$ representing the option risk premium. Starting from $380.-$, the seller may be obliged to deliver the share at the strike price, that is to say $380.-$. Starting from $392.50.-$, it starts to lose money on the transaction, since the share that he will undoubtedly have to deliver will worth be more expensive than the sum of the initial exercise price and of the option risk premium received. If unfortunately for him the share actually goes up to $500.-$ and he does not have the underlying, he will have to go buy it on stock market to honor the exercise request from the holder of the Call, loosing at the same time $107.50.-$ on the deal, thus more than eight times the premium received!!!!! With the leverage effect, it is possible to lose more than the risk premium received. We then say that we are experiencing a "\NewTerm{margin call}" and we better understand the importance to find an approximate martingale to evaluate the risk premium at its best.
			\begin{figure}[H]
				\centering
				\includegraphics{img/economy/payoff_simple_call_seller.jpg}
				\caption{Leverage effect for the seller of a Call (simple Call payoff diagram)}
			\end{figure}
			We can see above that the payoff of a Call from the seller's point of view is a concave function (\SeeChapter{see section Functional Analysis page \pageref{concave function}}).
			
			Thus, if the underlying price increases of $3.29\%$, the seller of the Call loose $100\%$. We speak then of a leverage effect of $3.29$ for $100$ or more simply of a leverage corresponding to a ration $100 / 3.29$ thus a leverage of about $30.39$ (it is the same of course for the gain!).
			\end{itemize}
			
		\item Second (and last) with the case of a Put:
		\begin{itemize}
			\item Let us consider the buyer of the Put:
			The buyer of a Put limits its exposure to that of the cost of the option premium for a potential gain far more important. In front of him, the seller of the Put is in exactly the opposite position: he takes the option premium but takes a much greater risk. If we take the same share $X$ traded at  $350.-$ at mid-October, we are now with an investor that bets on the fall of the share price. So he buys for $49.50.-$ ("premium" price) the Put of maturity in December at an exercise price of $390.-$.
			
			The buyer of the Put begins to make a profit if the share price falls below $340.50.-$, that is to say the strike price minus the option price ($390-49.50$). Between $340.50.-$ and $390.-$ the option is not profitable but permits to reduce the loss. Above the at-the-money price ($390.-$) the exercise of the Put really offers no interest anymore and we say when then as we know that the Put option is out-the-money.
			\begin{figure}[H]
				\centering
				\includegraphics{img/economy/payoff_simple_put_buyer.jpg}
				\caption{Leverage effect for the buyer of a Put (simple Call payoff diagram)}
			\end{figure}
			
			\item Let us consider now the seller of the Put:
			
			The seller of the Put first get the cash of option premium that is to say $49.50.-$. As long as the traded price remains above $390.-$ he is a winner. If the traded price of the option is between $340.50.-$ and $390.-$ and it loses some of its risk premium cash but he is still winning. Below $340.50.-$ the seller of the Put will be forced upon maturity to to pay $390.-$ to the buyer of the Put (by selling the underlying and by paying the difference in one way or another) . Obviously if the underlying price falls to zero, the seller of the put may lose up $340.50.-$ of capital.
			\begin{figure}[H]
				\centering
				\includegraphics{img/economy/payoff_simple_put_seller.jpg}
				\caption{Leverage effect for the seller of a Put (simple Call payoff diagram)}
			\end{figure}
		\end{itemize}
	\end{itemize}
	At the vocabulary level here below is an excellent summary diagram (top buyer of Call, down buy of a Put) knowing that the "\NewTerm{moneyness}" is defined as the ratio between the price of the underlying $S$ and the strike price $K$:
	\begin{figure}[H]
		\centering
		\includegraphics{img/economy/vocabulary_payoff_diagram_moneyness.jpg}
		\caption[Basic vocabulary of a situation of option payoff]{Basic vocabulary of a situation of option payoff\\(source:Fast Calibration in the Heston Model, BAUER Rudolf, p. 14)}
	\end{figure}
	
	If we think for a moment, a possible strategy is to buy Put and Call of the same underlying with the same strike price and the same expiration date (maturity). This type of strategy is named a "straddle" or "straddle purchase" or "bottom straddle". We will study options strategies mixtures much further below in details.
	
	Thus, if the traded price of the underlying is near the strike price $K$ at the expiration date $T$ then the straddle generates a net loss (as we will not use either the Put or Call and we will have lost the amount equivalent to the options premium). However, if the underlying price is quite far from the strike at maturity the profit can be very important!
	
	This is pure speculation of investment banks derivatives stratgies and this should be objectively prohibited... but let see anyway the principle through an example...
	\begin{tcolorbox}[colframe=black,colback=white,sharp corners]
	\textbf{{\Large \ding{45}}Example:}\\\\
	Let us consider an underlying that is traded actually $69.-$ will shift violently within three months. We can set up a straddle strategy by buying both a Call and a Put with strike of $70.-$ and maturity in $3$ months. Let us suppose that the  Call cost $4.-$ and the Put $3.-$.\\

	If at maturity, the underlying value is that of strike $K$, for the speculator it is the worst situation as it will have lost the sum of the two risk premiums, that is to say $7.-$ because he did not have any interest in exercising the Call using its Put (since there is no difference between the both).
	\end{tcolorbox}
	
	\begin{tcolorbox}[colframe=black,colback=white,sharp corners]
	If the underlying traded value remains at $69.-$ we lose $6.-$ because the Call has no value anymore at maturity (at least as it makes lose money it would be unwise for a pure speculator to exercise it) and the Put is only worth $1.-$ (as we will be able to sell at $70.-$ something that has a value of $69.-$ and it will make a small gain of $1.-$). Therefore:
		
	If the price of the underlying rises to $90.-$ (or down to $50.-$), we will realize a profit of $13.-$ thanks to the Put as:
	
	\end{tcolorbox}
	It is hoped that the speculator to it will bet correctly on a strong variation ignored by most of the market players, otherwise the option risk premiums will be obviously high enough that its gain will be anyway equal to zere. Graphically ... the Straddle strategy is often represented as follows (the verticle axis is the gain, the horizontal axis is the underlying price):
	\begin{figure}[H]
		\centering
		\includegraphics{img/economy/straddle_strategy.jpg}
		\caption{Pay-off diagram of a straddle strategy}
	\end{figure}
	We can summarize these properties under the form of the following table:
	\setlength\extrarowheight{10pt}
	\begin{table}[H]
		\begin{center}
			\definecolor{gris}{gray}{0.85}
				\begin{tabular}{|c|c|c|c|c|}
					\hline
					\multicolumn{1}{c}{\cellcolor{black!30}\textbf{Strategy}} & 
	  \multicolumn{1}{c}{\cellcolor{black!30}\textbf{Trend anticipation}}  & 
	  \multicolumn{1}{c}{\cellcolor{black!30}\textbf{Potential Gain}} & 
	  \multicolumn{1}{c}{\cellcolor{black!30}\textbf{Potential Loss}} & 
	  \multicolumn{1}{c}{\cellcolor{black!30}\textbf{Profit curve}}  \\ \hline
					Buy a Call & \parbox{2.5cm}{upward trend\\(bullish trend)}& "Illimited" & Limited & \centering\arraybackslash\ \pbox{5cm}{\includegraphics{img/economy/mini_call_bullish.jpg}}\\ \hline
					Buy a Put & \parbox{2.5cm}{falling trend\\(bearish trend)} & Limited & Limited & \centering\arraybackslash\ \pbox{5cm}{\includegraphics{img/economy/mini_put_bearish.jpg}}\\ \hline 
					Sell a Call & Stable or weakly bearish & Limited & "Illimited" & \centering\arraybackslash\ \pbox{5cm}{\includegraphics{img/economy/mini_call_bearish.jpg}} \\ \hline
					Sell a Put & Stable or weakly bullish & Limited & Limited & \centering\arraybackslash\ \pbox{5cm}{\includegraphics{img/economy/mini_put_bullish.jpg}}\\ \hline
			\end{tabular}
		\end{center}
		\caption{Different elementary Call-Put strategies}
	\end{table}
	\setlength\extrarowheight{0pt}
	
	\pagebreak
	The "\NewTerm{option plans}", better known as the "\NewTerm{stock option plans}" or "\NewTerm{employees stock option plan (ESOP)}" are warrants (or Call Options) issue by packages (nominatives) for good employees of a company and have for purpose to strengthen the partnership of development between the same company and its employees. Thus, the latter on the acquisition of securities will be full-fledged stockholders receiving dividends and may participate in stockholders meetings. What is supposed to increase the motivation of the employees (...). This motivation is mainly done by the fact that the Warrants (also Call Options) that are given to employees will be very interesting to exercise if the company performs with them, and therefore the underlying price far exceeds the Exercise price of the Warrant. Thus, employees will exercise their Warrant and selling the underlying to get a profit (it is a common practice in US start-up with limited financial resources in the beginning to hire specialists and that make some employees - 10,000 in the case of Microsoft - became millionaires after the exercise of the Call they owned).
	
	As we will see a little bit further below there is a huge difference between stock options and warrants relatively to the property of "dilution effect" and then we say that the warrant is not a "\NewTerm{compensatory vehicles}"!!! 
	
	\paragraph{Returns and Investments rates}\mbox{}\\\\
	To define the objective pursued by the owner of financial assets, we will refer to the economic motivation of any act of investing. This practice is to make currently an expense for by expecting an increase of the asset in the future.

	Of two or more investment strategies, the best at the individual level is the one that maximizes the final capital invested.

	Then there exist different types of return rates on investment following the object of study. Thus, we differentiate in finance (before seeing the details):

	\begin{enumerate}
		\item The financial asset returns on a given economic horizon (return on investment) and their yield (rate of return, continuous or not, nominal or not).

		\item The returns on investment compared to an average geometric rate of the market and the limit of the corresponding rate of return (internal rate of return).
	\end{enumerate}

	Then we have to consider other rate of return approaches. Besides the two above-mentioned two other classics are portfolio management (before we see the details further below):
	\begin{enumerate}
		\item The "money weighted rate or return (M.W.R.R.)" for invested capital which has the advantage over the internal rate of return to  take into account investments made outside the classical periods of time.

		\item  The "time of weighted rate of return (T.W.R.R.)" which is a useful tool for measuring the performance of fund managers because it does not take into account the flows (withdrawals or investment) of investors that are uncontrollable most of time.
	\end{enumerate}
	\begin{tcolorbox}[title=Remark,colframe=black,arc=10pt]
	We strongly recommend the reader to come back on the Capitalization and Actualization subsection where we treat in details of: interest rates, nominal rates effective rates.
	\end{tcolorbox}
	Let us see a bit all this in details:
	
	\subparagraph{Return on Investment}\mbox{}\\\\
	In practice, we will define the objective of the investor as of maximizing the growth of its initial fortune, whatever the modalities of this increase. This increase is named commonly "\NewTerm{return on investment (ROI)}"  or, more briefly, just "\NewTerm{return $R$}" at time $t$ and is defined by the (logical) following logical relation in the field of asset management by:
	
	where $R_t$ is the return of the financial asset for the period (ending time) $t$, $P_t$ the market price at time $t$ of the financial asset and $C_t$ the cash income attached to the holding of the financial assets during that same period (typically dividends). Obviously the practitioner must take care to take into account the various government taxes or trading fees when calculating this return, this is why when fees and taxes are not remove we speak rater of "\NewTerm{Total return}"!

	The income $C_t$ is assumed to be received at time $t$, or, if it is perceived between time $t-1$ and $t$ it is not supposed to be reinvested before time $t$ (otherwise we must calculate it's future value). The market price $P$ at time $t-1$ is a value "ex coupon" that is to say a value recorded immediately after the perception of the dividends. Empirically, the assumption of non-reinvestment until the elementary time period used is short (maximum one month) to avoid excessive statistical distortions in the treatment of historical data.

	To facilitate comparisons between investment, we use a measure expressed in relative terms (as always) the "\NewTerm{(short) rate of return}" logically defined by:
	
	where $r_t$ is the rate of return for period $t$.
	
	The latter relation can also be found in the following form:
	
	name the "\NewTerm{Total Stock Return}" where $P_1$ is the initial stock price, $P_0$ the ending stock price (period 1), and $D$ the dividends.
	
	Never forget that we proved in the section of Statistics that:
	
	That is that the average long term period of return over-estimated the "real" average return given by the geometric mean (\SeeChapter{see section Statistics page \pageref{geometric mean}}).

	We will come back during our study of the mathematical model of asset pricing on these tools.
	
	\subparagraph{Internal Rate of Return}\mbox{}\\\\
	The use of a financial capital to enable the realization of real economic transactions (that is to say, devoting directly or indirectly, the financial capital for the acquisition or to build production tools, in the broadest sense of the term) can therefore produce over time positive or negative cash-flows.

	The actuarial calculus gives us the possibility to build a decision criterion to compare investments with different returns. Indeed, we define (logically but without being completely realistic) the risk-taking by the "\NewTerm{Net Present Vale}" as proved in the section of Quantitative Management:
	
	where $t\%$ is supposed to the compared risk free rate!
	
	Obviously if there is no residual value this reduce to
	
	where the $C_i$ can sometimes be negative (intermediate investments). So the three decision criteria are for recall:
	\begin{itemize}
		\item $\text{NPV}<0$ corresponds to an investment that should be probably  avoided
		\item $\text{NPV}=0$ then we are in front of an undecidable investment
		\item $\text{NPV}>0$ corresponds then probably to a good investment
	\end{itemize}
	
	Most of time the investor is interested to know in comparison to all its periodic investments and withdrawals (gains) what is the rate that cancel the NPV and after to compare this rate to that of the market risk free rate, some other portfolios or hedge funds rates. We know from our study on the section of Quantitative Management that this rate named "\NewTerm{Internal Rate of Return (IRR)}" is the average geometric rate (discount rate) $t\%$ such that it is the smallest positive root of the polynomial that satisfied:
	
	So if we consider that the smallest positive root of this polynomial equation is denoted IRR than for an initial investment $V_0$ and some wanted cash-flows $C_i$, then if a banks offers us a given TSR investment strategy (Total Stock Return) such that $\text{TSR}\geq\text{IRR}$ the we should probably consider this offer seriously.
	
	We will see further below during our study of the Black \& Scholes model much more sophisticated ways to consider investment strategies that are not purely deterministic as before but stochastic.
	
	A firm's required rate of return on investments is often referred to as its "\NewTerm{hurdle rate}\index{hurdle rate}" because all projects must earn a rate of return high enough to clear this rate. Otherwise, a project will not cover its cost of financing, thereby reducing shareholder wealth.
	
	\subparagraph{Money Weighted Rate of Return (M.W.R.R.)}\mbox{}\\\\
	We will now introduce a kind of internal rate of return different the one related to NPV and that applies best to portfolio management (dynamics) than  the internal rate of return see just before (for which we recall that one of the assumption is that cash flows are at periodic intervals).

	Let us consider a fund $F$ and the following information:
	\begin{enumerate}
		\item The value of the fund $F_0\in \mathbb{R}^{+}$ just before time $0$.

		\item The value of the fund $F_1\in\mathbb{R}^{+}$ just after the time $1$.

		\item A total net cash value $N\in\mathbb{R}$ invested during the period $[0,1]$ paid in two halves at the beginning and at end of period (to simplify the example ...).
	\end{enumerate}
	The data that will interest us are:
	\begin{enumerate}
		\item The value $F_0+N/2$ that represents the total value of the fund and part of the investment at time $0$.

		\item The value $N/2(1+t\%)^{-1}$ that represents the capital that we should have been gathering at time $0$ to arrive at end of period to the value $N/2$ when the market free-risk rate is at $t\%$.

		\item The value $(1+t\%)^{-1}F_1$ that represents at time $0$ the value of the fund we would like to reach so we have at the end of period the value $F_1$ when the market free-risk rate is at $t\%$.
	\end{enumerate}
	The difference:
	
	gives the value that should have been capitalized for the get the sum $\left(F_1-\dfrac{N}{2}\right)$.

	What is is trivially interesting for an investor is then to know the rate $t\%$ such as the first investment equalizes its annual target. That is to say:
	
	Hence:
	
	relationship is named "\NewTerm{Hardy's relation}".
	
	If this relationship holds for $F_0,N$ known and given $F_1$, an investor will have nothing to gain or to lose by investing in the fund or to capitalize at the market rate $t\%$.

	If Hardy's relation is not equal to zero but positive then te investment in the fund is probably not interesting. If it is negative, then it is probably better to invest in the fund.

	Using elementary algebra, leads us to the relation:
	
	with:
	
	Indeed:
	
	The rate $t\%$ is often named in fund management the "\NewTerm{Money Weighted Rate of Return (M.W.R.R.)}" or "\NewTerm{weighted return rate by invested capital (W.R.R.I.C)}" and therefore represents the performance of a fund (portfolio) with consideration of inputs and outputs during the evaluation period.
	
	Therefore relation of money weighted rate of return that we get above:
	
	Can obviously also be written:
	
	And by changing the notation, as in some textbooks, this is rewritten:
	
	where $F$ represents the net cash transactions recorded during the related time period $(t_0, T)$; $V(t_0)$ the initial value of the fund; $V (T)$ the value of the fund at the end of the period of observation; $\bar{V}(t_0,T)$ the average capital the client has invested. 
	\begin{tcolorbox}[colframe=black,colback=white,sharp corners]
	\textbf{{\Large \ding{45}}Example:}\\\\
	A fund had the following income during 2006:
	\begin{itemize}
		\item Value at 1 January 2006: $30$ M\$

		\item Investment in the fund during the year: $18$ M\$

		\item Withdrawals from the fund: $30$ M\$

		\item Value of the fund at 31 December 2006: $21$ M\$
	\end{itemize}
	and we want to know what is the effective rate (M.W.R.R.) of this fund in 2006?

	We then as initial data $F_0=30,F_1=21,N=18-30=-12$ which gives if we accept the assumptions about $N$:
	
	and then:
	
	and therefore:
	
	Let us consider now that we know that investments took place the $3/8$th part of the year and the withdrawals the $3/4$th part of the year.\\

	The M.W.R.R. is then the rate of the following cash flows sequence ($C_i=(t,V_i)$):
	
	We must then find $t\%$ as:
	
	Solving this equation with Maple 4.00b gives:
	
	With a spreadsheet software like Microsoft Excel 14.0.7173 this is given by:
	\begin{center}
		\texttt{=XIRR({-30,-18,30,21},{0,136.88,273.75,365})}
	\end{center}
	We see that in considering the cash flow and the fractional times when they take place (thus more accurate and rigorous analysis) the M.W.R.R. is smaller. Moreover, the last calculation is more rigorous than the first, it is the latter that the investor want to know at the end of the year.
	\end{tcolorbox}
	We see that from the above example in the more accurate M.W.R.R. example in comes that finally the M.W.R.R. is given by the $t\%$ that satisfies:
	
	This rate is therefore an effective measure of the increase funds rate, giving the impact of cash flow weight on the fund's value. It is therefore just a generalization of the I.R.R. (Internal Rate of Return).
	\begin{tcolorbox}[title=Remark,colframe=black,arc=10pt]
	In Switzerland, some insurance foundations of the LPP (Law of retirement provisions) communicate the annual or quarterly M.W.R.R. on the $5$ to $10$ past years. 
	\end{tcolorbox}
	
	\subparagraph{Time Weighted Rate of Return (M.W.R.R.)}\mbox{}\\\\
	We will now discuss another financial tool for portfolio management also used to judge the performance of an investment.

	Let us consider a fund such as:
	\begin{table}[H]
	\begin{center}
		\definecolor{gris}{gray}{0.85}
			\begin{tabular}{|l|c|c|c|c|c|}
				\hline
\multicolumn{1}{c}{\cellcolor{black!30}} & \multicolumn{1}{c}{\cellcolor{black!30}\textbf{December}} & \multicolumn{1}{c}{\cellcolor{black!30}\textbf{T1}} & \multicolumn{1}{c}{\cellcolor{black!30}\textbf{T2}}  & \multicolumn{1}{c}{\cellcolor{black!30}\textbf{T3}}  & \multicolumn{1}{c}{\cellcolor{black!30}\textbf{T4}}\\
		\multicolumn{1}{c}{\cellcolor{black!30}} & \multicolumn{1}{c}{\cellcolor{black!30}\textbf{31 2000}} & \multicolumn{1}{c}{\cellcolor{black!30}\textbf{2001}} & \multicolumn{1}{c}{\cellcolor{black!30}\textbf{2001}}  & \multicolumn{1}{c}{\cellcolor{black!30}\textbf{2001}}  & \multicolumn{1}{c}{\cellcolor{black!30}\textbf{2001}}\\ \hline

		 \multicolumn{1}{l}{\cellcolor{black!30}\textbf{Starting value of the fund}}  & & $1,000.-$ & $370.-$ & $81.-$ & $7.8.-$ \\ \hline
		 \multicolumn{1}{l}{\cellcolor{black!30}\textbf{Gain or (loss) for the quarter in $\%$}}  & & $10\%$ & $3\%$ & $-4\%$ & $6\%$ \\ \hline
		 \multicolumn{1}{l}{\cellcolor{black!30}\textbf{Gain or (loss) for the quarter (in $.-$)}}  & & $100.-$ & $11.1.-$ & $-3.2.-$ & $0.5.-$ \\ \hline
		 \multicolumn{1}{l}{\cellcolor{black!30}\textbf{Quarterly cash flow Inflows/Outflows}}  & & $-730.-$ & $-300.-$ & $-70$ & $0.-$ \\ \hline
		 \multicolumn{1}{l}{\cellcolor{black!30}\textbf{Fund value}}  & $1'000.-$ & $370.-$ & $81.1$ & $7.8.-$ & $8.3.-$ \\ \hline
	\end{tabular}
	\end{center}
	\caption{Time Weighted Rate Of Return}
	\end{table}
	The 31 December 2000, the fund has a value of $1,000.-$ value. During the first quarter T1 of 2001 it has a $+10\%$ return but we imagine that this value is far from what was expected, then the investor withdraws $730.-$ of the fund (portfolio based on the funds). During the second quarter T2, the fund gained $+3\%$ and an additional amount of $300.-$ were withdrawn of the fund by the investor. In the third quarter T3 the fund lost $-4\%$ and $70.-$ were removed from the fund. The last quarter T4, the fund gained $+6\%$ and no funds have been withdrawn.
	
	We then have the overall  increased (return) on the whole period (year) that is given by:
	
	We see well that this value is independent of the cash flows of the investor's portfolio. We named the value of $15.3\%$ the "\NewTerm{Time Weighted Rate of Return (T.W.R.R)}" on a quarterly basis.
	
	In other words, the T.W.R.R. simply indicates the performance of a portfolio without taking into account the inflow and outflows during the evaluation period.

	This specific case can be written generally as:
	
	It should be remembered that if we wanted to calculate the average fund performance by quarter, we would have simply used the geometric mean (\SeeChapter{see section Statistics page \pageref{geometric mean}})!

	The T.W.R.R. is a practical tool to measure the performance of fund managers because it does not take into investor cash flows that are most of time uncontrollable. Thus, we have a measure of the quality of the dynamics of the fund independent of the choice of the investors that could consider withdrawals or investments and that would be used to calculate an I.R.R. which would more or less be meaningless compared to the dynamics of the fund.
	
	\begin{tcolorbox}[title=Remark,colframe=black,arc=10pt]
	In Switzerland, some insurance foundations of the LPP (Law of retirement provisions) communicate the annual or quarterly M.W.R.R. on the $5$ to $10$ past years. 
	\end{tcolorbox}
	
	\pagebreak
	\paragraph{Theory of Speculation}\label{theory of speculation}\index{theory of speculation}\index{speculation theory}\index{Bachelier model}\mbox{}\\\\
	After these many contextual definitions, the goal now is to introduce basic stochastic speculative mathematical techniques used in finance. Indeed, finance has become over time a domain increasingly competitive, the margins on standard products tend to be reduced, the premium is on innovation and automation. This has led to an increasing sophistication of financial products, thus making use of advanced mathematical concepts, primarily based on probability models introduced by Louis Bachelier in his "Theory of Speculation" (1900) but really used massively in practice only since 1973 through different work of Black \& Scholes and Merton (who have earned the Nobel Prize in economics).

	Let us look to start what are the developments proposed by Louis Bachelier in his thesis to determine the predictive mathematical expectation and the estimated standard deviation of a financial asset (result that we will use as part of the study of the evaluation model Black \& Scholes).

	Let us denote by $p_{x,t}$ the probability density function of the price of an asset is traded $x$ at time $t$. Therefore, the elementary cumulative probability that the current value lies within the elementary interval $[x, x + \mathrm{d}x]$ at time $t$ is of the form:
	
	which integral on the whole domain of definition will have obviously to be equal to $1$.
	
	Under the fourth axiom of Probabilities (see section of the same name page \pageref{kolmogorov axioms}), the elementary cumulative probability that the price moves of a certain value to another one (temporal Markov chain in continuous time), will be equal to the product of the density probability that value is traded in a given interval $x$ the time $t_1$ that is to say:
	
	multiplied by the elementary cumulative probability that the price being treated at the value $x$ at the time $t_1$, is traded $z$ during a time interval $t_1+t_1$, that is to say, multiplied by:
	
	The desired elementary cumulative probability is then obviously:
	
	This writing supposes that trading values are independent random variables...
	
	As the traded value can be found at time $t_1$ in all the intervals $\mathrm{d}x$ included between $\pm \infty$, the cumulative probability that the traded value is $z$ at the time $t_1+t_2$ will be (the integration is relatively to $\mathrm{d}x$):
	
	As the probability that the value $z$ in the time interval $t_1+t_2$ has for expression:
	
	Then we have:
	
	or after simplification:
	
	such is the equation that must satisfy the probability distribution function $p$ that we seek. This equation is true, as we will see it, under some strong assumptions, by the function:
	
	But do not forget that this is a special solution (this why this model is sometimes named "\NewTerm{Bachelor's Gaussian model}") ... and nothing guarantee us that two independent random variables follow the same probability distribution...

	The both assumptions to build the model seen so far (independence and identical distribution) are often indicated in finance under the name of "\NewTerm{independence and stationary assumptions}" and this is why the time variable does not appear anymore!

	That said, we have then of course require (probability axioms oblige!):
	
	We recognize in the second term above the Gauss Integral that has for value (\SeeChapter{see section Statistics page \pageref{Gauss integral}}):
	
	
	we must therefore necessarily have to have for the normalization:
	
	It results:
	
	By putting $x=0$, we get $A=p_0$, that is to say that $A$ equals the actual traded price. 

	We must now check that relation:
	
	Satisfy well:
	
	under the given assumptions!!!!
	
	Given the quantities $p_1,p_2$ the quantities corresponding to $p_0$ and relative to times $t_1,t_2$ then we must prove that the expression:
	
	Can bet put under the form $Ae^{-B^2z^2}$ where $A,B$ depends of  the time!
	
	This integral becomes by noting that $z$ is not an integration variable (we assume that it is independent of $x$ as you have understand from the beginning!):
	
	Now we will change the shape of the integral (we also change the exponential notation otherwise it becomes unreadable):
	
	and let us put:
	
	Then we will have:
	
	The integral:
	
	having for value $1$ we get finally:
	
	This expression has the desired shape since:
	
	This relation expresses the fact that the total cumulative probability that the random variable $z$ can take any value is equal to unity.

	We must conclude that the probability that the asset is traded $z$ at time $t_1+t_2$ is expressed by the relation:
	
	That can be written:
	
	and this prove our initial hypothesis.
	
	We see that the probability is then still governed by a Normal centered reduced distribution type (\SeeChapter{see section Statistics page \pageref{gauss distribution}})! This is a remarkable result obtained by Louis Bachelier in 1900 and which had already been speculated by Jules Regnault in the mid 19th century!
	
	Indeed, Jules Regnault compared speculation to a heads or tail game in which the two sides of the coin correspond to the two possibilities, increase or decrase. Assuming that at any time whatsoever, there are never more chance for benefits than for loose. In other words, at each quote value, price has $50\%$ chance of increasing and $50\%$ of decreasing. But every trader has their opinion on the issue... Without this diversity of opinions, there would therefore no trades or price changes. So operators are divided into two groups (bullish, bearish) that make subjective assessments of the future value of the course which necessarily include a margin of error. However, to Jules Regnault, errors speculators are not any, they follow a Normal distribution. Indeed, as shown by Pierre Simon de Laplace, if the probability of error is small and they numerous independent errors then the results follow a Normal distribution (\SeeChapter{see section Statistics page \pageref{gauss distribution}}).
	\begin{figure}[H]
		\centering
		\includegraphics{img/economy/bloomberg_gaussian_shape_of_returns.jpg}
		\caption{Yield distribution analysis in the Bloomberg\textsuperscript{TM} terminal}
	\end{figure}
	The prior previous relations shows that the parameters $p_0=f(t)$ satisfy the functional relation:
	
	Let us differentiate relatively to $t_1$ and after relatively to $t_2$. The first member having the same shape in both cases, we get:
	
	So after simplification:
	
	Which finally gives:
	
	This relation being verified regardless the values of $t_1,t_2$, the common value of the two ratio is constant and then we have:
	
	A function that satisfies this relation exists and is:
	
	Where $H$ denotes a constant or an independent function of time.

	Verification:
	
	therefore:
	
	thus we have the final expression for the probability density function of the current traded value $x$ under the above assumptions:
	
	with $x$ that is for recall greater than or equal to $0$.
	
	The reader will have noticed perhaps that for a given value of $H$ and $t$ fixed we always have here as a centered Normal distribution (\SeeChapter{see section Statistics page \pageref{gauss distribution}})!! Financial analysts then say that we are dealing with a "wise chance," this implied that the changes are small and regular.
	
	Let us now obviously calculate the mean and variance of this important distribution function (notice that these two moments should be dependent of time!).

	As the course (traded value) can not be negative, we restrict the calculation to the positive mean as being (\SeeChapter{see section Statistics page \pageref{expected mean continuous variable}}):
	
	by writing:
	
	the "\NewTerm{the coefficient of instability}" (of which we know nothing), so finally we therefore the positive mean of the course that is:
	
	the expected mean of the course is therefore proportional to the square root of time like is the Brownian motion that we studied in the section of Statistical Mechanics !!

	It also follows immediately from this result that the average deviation of the current value at two consecutive times is also proportional to the square root of time between the two times!

	We also notice that at the instant where $t = 0$, the positive expected mean of the gain is zero, because the value is known with certitude (it is how it should be interpreted).
	
	Let us also calculate the positive variance:
	
	We put:
	
	Therefore:
	
	with:
	
	Therefore it comes:
	
	But in the section Statistics, we have proved by integration by parts (\SeeChapter{see section Differential and Integral Calculus page \pageref{integration by parts}}) that:
	
	Thus:
	
	So if we put:
	
	We have finally:
	
	Thus the standard deviation is also proportional to the square root of time (and the result would be the same if we had calculated the global standard deviation)!
	\begin{tcolorbox}[title=Remark,colframe=black,arc=10pt]
	The Brownian motion (and its variations) is heavily used by professionals, since many annualized volatility calculations (in $\%$ / year) which we find the results in any financial page of the daily press, are only the square root conversions time of periodic volatility calculations (in $\%$ / month or $\%$ / week) used as the basis of estimation.\\
	
	Multiplying monthly Standard Deviation by the $\sqrt{12}$ is an industry standard method of approximating annualized Standard Deviations of Monthly Returns.
	\end{tcolorbox}
	Therefore by the stability of the Normal law (\SeeChapter{see section Statistics page \pageref{stability of the sum in statistics}}) we have:
	
	It follows immediately that:
	
	and:
	
	and:
	
	So the traded price variations of a financial asset between two successive times have then obviously a probability distribution described by a Normal centered  distribution (thus resulting by the stability property of this law) that is characterized by a positive expected mean and a positive  standard deviation proportional to the square root of time!
	
	This mathematically proved result was measured by Regnault fifty years ago before Bachelier ($\sim$1850) by observing that the standard deviation of French bonds was proportional to the square root of time.
	
	That said we have to accept the limitations of this approach. Let us take for example of the daily returns of the Dow Jones in 2008 and 2009. According to experts having the detailed of this data, it would follow rather follow a Student's $t$-distribution of with $3$ defrees of freedom rather than a Normal distribution...!
	
	To give a blatant comparison of the limits of these approach let us remind (\SeeChapter{see section Statistics page \pageref{gauss distribution}}) that the cumulative probability of a random variable following a Normal distribution beyond $4\sigma$ is $1-99.99366\%$, that is $0.00634\%$ . This means, if the stock market has $252$ open workdays, we have certainty of having a large deviation every:
	
	we therefore consider (\SeeChapter{see section Probability page \pageref{disjoint events}}) pairwise disjoint events.
	
	Finally three major results are to be retain here under the strong assumptions of centered Normality and independence:
	\begin{enumerate}
		\item That the probability distribution function that the price of a financial asset to be $x$ at time $t$ given follows a Mormal centered distribution... !!

		\item That positive expected mean and the positive standard-deviation of the value of a financial asset are proportional to the square root of time with a factor about which we don't know anything!!

		\item That the expected mean gain is generally zero (which makes it unrealistic model but already provides a working basis...).
	\end{enumerate}
	Much later below we will compare the Bachelier Call pricing Model and the Black \& Scholes Call pricing model. But for this we need to know an assumption that did Bachelier crystallized in his famous dictum:
	
	\begin{center}
	\textit{L'ésperance mathématique du spéculateur est nul}
	\end{center}
	i.e. "\textit{the mathematical expectation of a speculator is zero}". Therefore as we assume a Normally distribution of the price and the standard deviation and the mean are proportional to the square root of time but that the mean is assumed as zero, we get:
	
	and this is the relation we will have to use much later below.
	
	
	Finally, it should be noted that this is a theoretical model...! It must therefore always be compared to real practical values to see if it is valid or not. In this case the observation of financial markets shows that it is only outside of the periods of speculative bubbles that the variations can be modeled by a Brownian motion. We must therefore look for more powerful models and we will see one day (...) that the Brownian movement (also named "Brownian process") that is smooth (continuous) and therefore without sudden jumps (therefore unable to model some sudden event markets like the crash or corrections) is a special case of "Lévy processes".
	
	\pagebreak	
	\paragraph{Portfolio efficient diversification models}\label{portfolio efficient diversification models}\mbox{}\\\\
	Diversification is a well known way of reduce common known investment risks under the assumptions that the risk must be minimized and a given portfolio construction target is reached.
	
	The most common efficient diversification refers to the organizing principle of portfolio theory, which attempts to maximize portfolio expected return for a given level of portfolio risk. In other words, having enough diversification in an investment portfolio to earn money, while still keeping risks within reasonable bounds.
	
	Here is the definition given by the \hyperref[NASDAQ]{http://marketbusinessnews.com/financial-glossary/efficient-diversification/} (it is not the best definition as we will see later with the various existing models that are not based only on highest return):
	
	\textbf{Definition (\#\mydef):} "\NewTerm{Efficient diversification}" is the organizing principle of portfolio theory, which maintains that any risk-averse investor will search for the highest expected return for any particular level of portfolio risk.
	
	There is quite a large number of other empirical models of portfolios diversification models. Among the best known (we tried to put them in the order in which they are normally learned at University:
	\begin{itemize}
		\item Overall minimum variance portfolio (PVGM) also named Markowitz portfolio
		\item Tangent portfolio (TP)
		\item Minimum overall robust variance portfolio (MORVP)
		\item Minimum overall robust semivariance portfolio (MORSVP)
		\item Maximum Sharpe ratio portfolio (MSRP)
		\item Minimum Tracking error portfolio (MTEP)
		\item Identical weight Portfolio (P1/N)
		\item Zero index portfolio (PIN)
		\item- Neutral-dollar Portfolio (sum of weight zero) (NDP)
		\item Single average portfolio (SAP)
		\item Zero beta portfolio (ZBP)
		\item Maximum utility portfolio (MUP)
		\item Maximum diversification Portfolio (MDP)
		\item Black-Litterman Portfolio (BLP)
		\item Draw-down portfolio (DDP)
		\item Minimum VaR portfolio (MVaRP)
	\end{itemize}
	A number of practical and theoretical problems have limited the use of portfolio diversification models by investment managers. For example, it is often difficult to obtain sufficient high-quality historical data for thorough analysis. In addition, the efficient frontier where optimal portfolios lie tends to shift over time, quickly making these portfolios suboptimal.
	
	For each of the model above we will start with the theoretical aspect and give afterwards a detailed example with Microsoft Excel, R or MATLAB™  depending of what software is the more suited to the corresponding.
	
	\subparagraph{Overall minimum variance portfolio (Markowitz portfolio)}\label{markowitz overall minimum variance portfolio}\mbox{}\\\\
	They are different approaches for introducing the Markowitz portfolio diversification model. We will start with the one using quadratic optimization.

	Given $R_p$ the performance of a portfolio of $n$ assets characterized by their respective performance (return) $R_1,R_2,\ldots,R_n$. We put, as constraint, that each asset $i$ enters the portfolio $P$ in a proportion $X_i$ such that:
	
	\begin{tcolorbox}[title=Remark,colframe=black,arc=10pt]
	A part $X_i$ of an asset may also be negative ... Hold a negative share of an asset is what is already know as being "short-selling" (short sale). This technique consists for recall to borrow a lot of assets (assumed overvalued on the market) to a bank, sell them to make the price of the asset go lower, and making a profit by buying them cheaper to make given them back to the bank (roughly as it can be quite complex and subtil in the reality...).
	\end{tcolorbox}
	Therefore the variance of the portfolio is given by (\SeeChapter{see section Statistics page \pageref{covariance}}) using the standard variance (we will see later that there is a model using "semivariance"):
	
	Before going further, let us precise (because it is important in practice) that we can also write the last relation in matrix form (the reader can easily verify by taking such two assets that both writing give an identical result) if we denote by $\vec{X}$ the vector or the portions of assets and $\vec{X}^T$ the correspond transposed vector:
	
	and finally $c_{ij}$ is the covariance matrix (remember, that there exist a simple method to move from the covariance matrix to the correlation matrix and vice versa as seen in the section Statistics):
	
	Matrix that as we know (\SeeChapter{see section Statistics page \pageref{variance covariance matrix}}) can be simplified directly in:
	
	we finally get the relation of the variance in condensed matrix form:
	
	as we often see it in the literature.
	
	To come back to the algebraic form of the model, since the covariance is symmetric (\SeeChapter{see section Statistics page \pageref{variance covariance matrix}})
	
	and that:
	
	We can simplify and write the variance:
	
	as following:
	
	as we often see it in the literature of a few decades old...

	Select a Markowitz portfolio is therefore equivalent as solving the following constraints under maximization problem:
	
	using quadratic programming (\SeeChapter{see section Numerical Methods page \pageref{quadratic programming}}).

	In practice, we seek not seek one but all portfolios that for a given expected return minimizes the variance. We then obtain a function of the mean return based on the variance for the optimal portfolios. If we plot it on a graph (see below). This function is often assimilated by financial practitioners (rightly!) to a frontier as specified in the following definition.
	
	\textbf{Definition (\#\mydef):} The frontier that characterizes the polygon or the of curve constraints in this situation is named the "\NewTerm{(Markowitz) efficient frontier}" and in the polygon / curve occur all portfolios to reject that are so named "\NewTerm{dominated portfolios}". Another way of stating this is to say that the combinations of (performance, risk) of this frontier form an optimum set of Pareto (\SeeChapter{see section Game and Decision Theory page \pageref{pareto optimum}}), that is to say, if one of the elements increases, the other must also increase.
	
	Now let us formalize the optimization in the way that was done in the days when people had to further develop the algorithms themselves ...

	Given $Z$ the aforementioned economic function:
	
	which should be maximized under the constraint $\sum_{i=1}^n X_i=1$ that and where $\Phi$ is a parameter that represents the degree of risk aversion of investors (in the standardize the relation units...).
	
	The problem of maximization under constrained s to determine the maximum of the economic function $Z$ defined by:
	
	This function of $n + 1$ variables $(X_1,X_2,\ldots,X_n,\lambda)$ is maximized if its derivative (partial derivative) from each of these variables is zero, which is to put the following system:
	
	Let us put:
	
	Then we can write:
	
	or in matrix form:
	
	And we will put now:
	
	In this case, the system of equations to solve can be summarized in matrix form as:
	
	Therefore:
	
	Determining the weight of each of the $n$ assets that enter into the composition of a portfolio thus requires inversion of a square matrix of $n + 1$ rows and $n + 1$ columns having $(n^2-n)/2$ covariances (the diagonal having variances only and the matrix being symmetrical!). What is relatively long to calculate for large portfolios.

	However, even once the weighting of assets is completed, the problem itself is not completely finished. Indeed, we can find the efficient frontier but the customer (investor) will impose a logic constraint relatively to the zero risk level of its portfolio and of the radio return/maximum risk.

	Given the heavy calculations required for the inversion of the matrix $A$, William F. has proposed a simplified model that we will see after a practical example of the Markowitz model.
	\begin{tcolorbox}[colframe=black,colback=white,sharp corners]
	\textbf{{\Large \ding{45}}Example:}\\\\
	Let us consider three assets entering a portfolio composition at the same time in equal proportions and $n$ of their return observations (without missing values) $R_{ij}$ entered in Microsoft Excel 14.0.7173 (the component $j$ can be seen as a time period):
	\begin{figure}[H]
		\centering
		\includegraphics{img/economy/markowitz_observed_returns_excel_list.jpg}
		\caption[]{Initial observed values of our portfolio assets}
	\end{figure}
	The goal will be now using the native tools of this spreadsheet software to determine the Markowitz  portfolio efficiency frontier model and the C.M.L. (see definition and mathematical details further below) and the corresponding weightsofassets (optimal weights) that minimizes the variance for a given expected mean of the return knowing the risk-free market rate $R_f$ is equal to $0.22$.
	\end{tcolorbox}
	
	\begin{tcolorbox}[colframe=black,colback=white,sharp corners]
	Below the table given above we will create in Microsoft Excel the list containing the proportions $X_i$ of the assets (which we assume equidistributed at the beginning such that each weight is equal to $1/3$), we will display the average short return $\mu_i$ of each asset calculated using the estimator\footnote{Obviously we know that this is not the best way as the arithmetic mean overestimated the real mean compared to geometric mean as $\mu_a>\mu_g$}:
	
	and the variance $\sigma_i^2$ calculated for each asset of the estimator (we could also take the semivariance but this is another subject...):
	
	This gives us the following list in Microsoft Excel 14.07.7173:
	\begin{figure}[H]
		\centering
		\includegraphics[]{img/economy/markowitz_observed_mean_variance_excel_list.jpg}
		\caption[]{Proportions and expected mean/variance estimators of $3$ assets}
	\end{figure}
	Thus explicitly:
	\begin{figure}[H]
		\centering
		\includegraphics[scale=0.8]{img/economy/markowitz_observed_mean_variance_excel_list_explicit.jpg}
		\caption[]{Explicit Proportions and expected mean/variance estimators of $3$ assets in Microsoft Excel 14.07.7173}
	\end{figure}
	We must now calculate the average return of the portfolio as given by:
	
	\end{tcolorbox}
	
	\begin{tcolorbox}[colframe=black,colback=white,sharp corners]
	This relation is a bit long to enter in a spreadhseet software as it is, and will be even more if we have a much larger number of assets.\\

	In our case, it is to the sum of the products term by term of two cell ranges ($X_i$ and $\hat{\mu}_i$) having the same dimensions (same number of rows and the same number of columns). We can then use the following function in the in Microsoft Excel 14.0.7173:
	\begin{center}
		\texttt{=SOMMEPROD(B14:D14,B15:D15)}
	\end{center}
	For the portfolio variance is a little bit more complicated since we have to calculate:
	
	The relation developed in our particular case gives:
	
	The trick to apply this in a spreadsheet like Microsoft Excel is to use linear algebra and write this relationship in matrix form as we have proved it earlier above:
	
	Which is equivalent to write in Microsoft Excel 14.0.:
	\begin{center}
		\texttt{=SUMPROD(MMULT(B14:D14,G14:I16),B14:D14)}
	\end{center}
	Either in explicit matrix form:
	
	Based on the above tables, it is simple in Microsoft Excel to obtain the covariance matrix (in practice it is often considered as difficult to get a robust estimate of the covariance matrix and there are various techniques for this purpose):
	\begin{figure}[H]
		\centering
		\includegraphics{img/economy/markowitz_covariance_matrix_excel_list.jpg}
		\caption[]{Covariance matrix of our $3$ assets}
	\end{figure}
	\end{tcolorbox}
	
	\begin{tcolorbox}[colframe=black,colback=white,sharp corners]
	Thus explicitly (the covariance matrix is symmetrical for recall... (\SeeChapter{see section Statistics page \pageref{variance covariance matrix}}):
	\begin{figure}[H]
		\centering
		\includegraphics[scale=0.8]{img/economy/markowitz_covariance_matrix_excel_list_formulas.jpg}
		\caption[]{Explicit Covariance matrix of our $3$ assets in Microsoft Excel 14.07.7173}
	\end{figure}
	And the expected mean and the portfolio variance we have the following table:
	\begin{figure}[H]
		\centering
		\includegraphics{img/economy/markowitz_portfolio_return_mean_and_variance_excel_list.jpg}
		\caption[]{Portefolio Mean and Variance estimators}
	\end{figure}
	thus applying the above relations:
	\begin{figure}[H]
		\centering
		\includegraphics{img/economy/markowitz_portfolio_return_mean_and_variance_excel_formulas.jpg}
		\caption[]{Explicit Portefolio Mean and Variance estimators in Microsoft Excel 14.07.7173}
	\end{figure}
	The problem now is to determine a fixed return of the portfolio (cell B20), the different proportions $X_i$ of the asset that minimize the risk.\\

	After having add the both cells B24 (expected return of the portfolio) and B25 (total number of proportions of assets):
	\begin{figure}[H]
		\centering
		\includegraphics{img/economy/markowitz_constraints_cells.jpg}
	\end{figure}
	We must now solve the nonlinear optimization problem:
	
	A globe view of our sheet gives so far:
	\end{tcolorbox}
	
	\begin{tcolorbox}[colframe=black,colback=white,sharp corners]
	\begin{figure}[H]
		\centering
		\includegraphics[scale=0.8]{img/economy/markowitz_sheet_excel_overview.jpg}
		\caption{Markowitz manual resolution configuration in Microsoft Excel 14.07.7173}
	\end{figure}
	And now we launch the solver:
	\begin{figure}[H]
		\centering
		\includegraphics[scale=0.8]{img/economy/markowitz_excel_solver_settings.jpg}
		\caption[]{Markowitz Solver settings in Microsoft Excel 14.07.7173}
	\end{figure}
	\end{tcolorbox}
	
	\begin{tcolorbox}[colframe=black,colback=white,sharp corners]
	This give straight for solution (the same as the one given by R 3.0.2):
	\begin{figure}[H]
		\centering
		\includegraphics[scale=0.76]{img/economy/markowitz_solver_solution.jpg}
	\end{figure}
	Now to determiner the efficient Markowitz frontier we report manually (or with VBA) for different values of the target mean, the corresponding minimum variance that gives us the following list:
	\begin{figure}[H]
		\centering
		\includegraphics[scale=0.78]{img/economy/markowitz_efficient_frontier_values_list.jpg}
		\caption[]{Markowitz efficient frontier values samples}
	\end{figure}
	This gives us the following $\text{E}(R_P)=f(\sigma_{R_p})$ plot:
	\begin{figure}[H]
		\centering
		\includegraphics[scale=0.78]{img/economy/markowitz_efficient_frontier_excel_plot.jpg}
		\caption[]{Markowitz efficient frontier plot in Microsoft Excel 14.07.7173}
	\end{figure}
	\end{tcolorbox}
	
	\begin{tcolorbox}[colframe=black,colback=white,sharp corners]
	If the reader refer to our R companions book the equivalent result is for the efficient frontier (EWP: Equally Weighted Portfiol):
	\begin{figure}[H]
		\centering
		\includegraphics[scale=0.78]{img/economy/markowitz_efficient_frontier_r_plot.jpg}
		\caption[]{Markowitz efficient frontier plot in R 3.0.2}
	\end{figure}
	Or still with R always by following the step given in the companion book we can plot the evolution of the return and risk in function of the weights:
	\begin{figure}[H]
		\centering
		\includegraphics[scale=0.80]{img/economy/markowitz_assets_weight_plot_r.jpg}
		\caption[]{Markowitz asset weight plot in R 3.0.2}
	\end{figure}
	We will not this time give the MATLAB™ results as the results are plotted in quite an awful way...
	\end{tcolorbox}
	So this a nice little application example in a software accessible to almost everyone but that is relatively false in reality (but at least it is educational)!

	However remember two construction assumptions of this model are that its use is limited ....:
	\begin{enumerate}
		\item[H1.] The mean-variance optimization model is equivalent to model to the assets of the portfolio by random variables for which these two moments exist. But in reality, nothing says that this is the case.

		\item[H2.] The model is build on a posteriori information without taking into account for the future could be. So an a priori information of the market shoul complete this model.

		\item[H3.]  The model is based on average return estimated for an allocation period that may change in the short-term. But a very small change in average return can have disproportionate impact on the proportions.
	\end{enumerate}
	Also in practice we must be careful to one thing also.... Rebalancing a portfolio following an optimization calculation can be very expensive in transaction fees. A case often cited in school is that of a US pension fund consisting of more than $2,000$ different types of assets whose rebalencement in October 2000 implied $40$ portfolio managers, $500$ million of assets for a total of $17.5$ billion dollars. It is said that cost of this rebalancing in terms of transactions cost was not far from $120$ million dollars...!
	
	\subparagraph{Overall minimum Sharpe portfolio}\mbox{}\\\\
	The use of the Markowitz model, as proposed in his 1959 book, raised many problems when it came to use algorithms for a basic list with a large number of values. These problems were of two orders:
	\begin{enumerate}
		\item The sizeofmatrices required at that time a huge computer capacity and a fairly long period of calculation!

		\item The use of the base model required that we know in its entirety the covariance matrix. The main problem about it is in the number of estimates to provide to build it and also the difficulty of making accurate and consistent estimates.
	\end{enumerate}
	If we want the approach proposed by Markowitz can enter the field of application, we must obviously find a way to significantly simplify the procedure while losing as little as possible of the rigor of the method.

	In 1963, William Sharpe proposed a solution whose essential characteristic consists in the assumption that the returns of the various values are only linked by their common relation with an underlying basic factor (stock index typically) which allows determining a coefficient named the "beta" (correlation like factor between the return of a security and that of a market portfolio).
	
	This purely empirical hypothesis named "\NewTerm{one index model}" (or "\NewTerm{unifactor model}", "\NewTerm{monofactorial model}") has later take considerable importance because it was, as we shall see in subsequent developments, at the basis of the theory of pricing for some financial assets in an uncertain Universe.

	\begin{tcolorbox}[title=Remark,colframe=black,arc=10pt]
	Again, the developments that follow should look abstract but ... we'll see how to solve the previous above example with Microsoft Excel for the Markowitz model but applying the Sharpe's model and we would then even be able visually to compare two methods.
	\end{tcolorbox}
	The term "uniffactor model" name therefore comes from the fact that the basic purpose of the Sharpe model is to define the performance of a financial investment in terms of its non-diversifiable risk, assimilated to that of themarket risk (or "\NewTerm{systematic risk}" ) given by a number number named the "beta" as previously mentioned.
	
	Investors and managers distinguish three main kinds of risks:
	\begin{enumerate}
		\item The "\NewTerm{specific risk}" relative (implicit) to the asset himself (its variance) also named "\NewTerm{unsystematic risk}" or "\NewTerm{idiosyncratic risk}".

		\item The "\NewTerm{systematic/nondiversifiable risk}" relatively to the economy / market in the broadest sense of the term  (variance of the market reference portfolio).

		\item The "\NewTerm{global risk}" which is somehow the sum of both (it's a little more subtle than a simple sum if fact...).
	\end{enumerate}
	As you have probably guessed, the risk factor is difficult to quantify. The element that will help determine the variation is the performance of the asset compared to the change of market performance as a whole. Financial assets whose prices change frequently and whose volatility is high presente therefore a high risk if the market is almost stable.
	
	Here are for example distributions of two investment funds: 
	\begin{figure}[H]
		\centering
		\includegraphics[scale=0.53]{img/economy/portfolio_diversification_influence.jpg}
		\caption[Multi-Asset moving average strategy VS One index portfolio]{Multi-Asset moving average strategy VS One index portfolio (source: Faber (2009), Butler|Philbrick \& Associates (2010))}
	\end{figure}
	Here, is another example of return distributions of the second retirement fund (LPP) in Switzerland depending on the portfolio diversification strategy (the values after the abbreviation LPP represent the $\%$ of shares in the portfolio):
	\begin{figure}[H]
		\centering
		\includegraphics{img/economy/portfolio_diversification_influence_lpp.jpg}
		\caption{Swiss LPP various diversification returns}
	\end{figure}
	
	\textbf{Definition (naive version \#\mydef):} The "\NewTerm{beta}" measure the dependence of the performance of a portfolio or of a financial asset and the performance of a benchmark index and is the slope of a line named "\NewTerm{security characteristic line (SCL)}":
	
	This ratio is of course especially accurate when the past data are on the long run and that the sampling frequency is smaller. This ratio is also sometimes named "\NewTerm{relative volatility}".
	\begin{figure}[H]
		\centering
		\includegraphics[scale=1]{img/economy/beta_concept.jpg}
		\caption{Idea of the linear regression behind the $\beta$ coefficient}
	\end{figure}
	\begin{tcolorbox}[title=Remark,colframe=black,arc=10pt]
	The benchmark index is chosen in the most relevant possible way with what implies ... If possible when the return of the reference index is zero, the change in value of the portfolio or asset should be zero.
	\end{tcolorbox}
	Or a more realistic example (some informations visible in this screenshot will be discussed later):
	\begin{figure}[H]
		\centering
		\includegraphics[scale=1]{img/economy/beta_bloomberg.jpg}
		\caption{Linear regression principle of beta (and alpha) in Bloomberg\textsuperscript{TM} Terminal}
	\end{figure}
	\textbf{Definitions (\#\mydef):} The beta of a stock can be presented as either an Adjusted Beta or a Raw Beta:
	\begin{enumerate}
		\item[D1.] A "\NewTerm{Raw Beta}" is obtained from the linear regression of a stock's historical data. Raw Beta, also known as "\NewTerm{Historical Beta}", is based on the observed relationship between the security's return and the returns of an index.

		\item[D2.] The "\NewTerm{Adjusted Beta}" is an estimate of a security's future Beta. Adjusted Beta is initially derived from historical data, but modified by the assumption that a security's true Beta will move towards the market average, of 1, over time. The formula used to adjust Beta is according to Blume's Method (used by Bloomberg Terminal):
		
	\end{enumerate} 

	A simple analysis of the graph above (ie elementary functional analysis) therefore shows that a beta equal to $1$ for a given asset means that an increase (respectively decrease) of $10\%$  of the market reference index for a certain period will result in an increase (respectively decrease) of $10\%$ on average performance of this asset. So the volatility of the asset is equal to that of the index.

	A beta greater than $1$ means that the evolution of the return of the financial asset is more volatile (or rather, was volatile, as this coefficient usually refers to a past period) than the return of the market reference index, while a beta less than $1$ indicates the opposite. Thus, a fund with a beta of $1.15$ is $15\%$ more volatile than the index. Conversely, a fund with a beta of $0.70$ is $30\%$ less volatile than the index.

So to summarize:
	\begin{itemize}
		\item An investment with no risk (relatively to the benchmark) therefore display a beta of $1$.

		\item A beta of less than $1$ indicates that if the market (benchmark) is down, the investment will likely to decline less than the market.

		\item A beta greater than $1$ indicate that if the market (benchmark) is on the rise, the investment will likely to follow the slower upward trend.
	\end{itemize}
	The concept of beta having been naively  introduced, let us go now to the theory of the model that aims to simplify the Markowitz using this famous coefficient.

	First, by choice, we take the overall beta of a portfolio as being determined from the respective weighted betas of each of the asset or underlying betas that it make up such that:
	
	with $\beta_P$ being the beta of the overall portfolio, $X_i$ the proportion of security $i$ in portfolio $P$, $\beta_i$ the beta of asset $i$ and $n$ is obviously the number of financial assets in the portfolio (the beta is then additive when weighted).

	Sharpe suggested that the return $R_i$ of each asset $i$ at time $t$ is given by the linear regression (\SeeChapter{see section Numerical Methods page \pageref{least squares method}}) determining the "\NewTerm{security characteristic line}" view above (so it is a linear regression model  with one explanatory factor but in practice there are obviously multiple linear or nonlinear models):
		
	where:
	\begin{itemize}
		\item $I$ is the performance of a given reference economic index ... (stock index, index of gross national product, price levels or even portfolio market performance itself...) at time $t$ and is the dependent variable of the univariate regression (in the terminology used in the section of Numerical Methods) considered as a random variable\footnote{we know that is this case we should rather use the Deming regression seen also in the section of Numerical Methods} (...).
		
		\item $\alpha_i,\beta_i$ are unbiased estimators (\SeeChapter{see section Statistics page \pageref{likelihood estimators}}) of the parameters specific to that value. The first term in finance is named "\NewTerm{alpha coefficient}" ans is simply the intercept of the regression (the return of the assets when the performance of the benchmark is zero or when the market has zero returns often also named "\NewTerm{residual return}") and the second parameter is simply to recall the "beta" of the risky asset $i$. The alpha is not a source of risk, it is often ignored in the calculations.
		
		\item $\varepsilon$ is a random variable assumed to be characterized by a zero mean, and variance equal to a constant (\SeeChapter{see section Numerical Methods page \pageref{univariate linear regression gaussian model}}) and the different $\varepsilon_i$ are assumed uncorrelated between them (zero covariance).
	\end{itemize}
	About the level of the index $I$, it will be characterized by the relation (to simplify later some developments):
	
	where $\alpha_{n+1}$ is an additional unbiased  parameter to characterize index $I$ and $v_{n+1}$ a random variable characterized by a zero mean and variance equal to a constant.
	
	Notice first that (by adopting on the way the traditional notation for the performance of the market index) for an asset:
	
	This is easily generalized to a sum of weighted assets for a portfolio and where we get also as a standard deviation of the portfolio is the sum of two terms: systematic risk and specific risk (diversifiable):
	
	 Of course if the number of assets approaches infinity and thus the weight approaches zero, then the total risk also tends to zero and this is the trick of multi-asset portfolio management.
	 
	 Notice that in econometric the above relation is rewritten as:
	
	And firm specific covariance is sometimes assumed zero. So we can write:
	
	
	To summarize the main points, the simple linear regression model of financial asset yield is based on the following key assumptions:
	\begin{enumerate}
		\item[H1.] The return model is written in general (univariate case):
		
		assuming that we do not any error on the linear assumption of the model, nor the list of regressors (see other models below taking into account other factors).

		\item[H2.] We assume that the perturbation $\varepsilon$ satisfies the same assumption as the usual general linear models (Gaussian or others) as the zero mean, zero covariance, fixed variance over time which has to be said is a quite dangerous practice but useful assumption to be usable (and understandable) by practitioners of finance...
		
		\item[H3.] For any sample of size $n$, we use the maximum likelihood estimators (\SeeChapter{see section Statistics page 	\pageref{likelihood estimators}}) for the mean and variance of returns on financial assets of the reference portfolio (index):
	
	\end{enumerate}
	These assumptions now states, we also use the results obtained in the section of Numerical Methods on the linear regression to get the beta. We have proved that there are several ways of doing a linear regression, one is to use the covariance and mean. By adopting the notation of econometrics, the slope of the regression (beta) can then be written:
	
	giving the rigorous definition of the beta coefficient according to the Sharpe model where $R_i$ is the return of the financial asset and $R_I$ that of the market reference index (or market reference portfolio).
	
	Now, considering the same assumption as in the Markowitz model, the return $R_P$ of portfolio is defined again quite logically by:
	
	where for recall $X_i$ is the proportion of the asset $i$ in the multi-asset portfolio $P$.

	If the return are not explicitly known in practice, then we use the linear model:
	
	Therefore by using the properties of mean:
	
	Let us put to simplify the notation that:
	
	In this case, by the two of the various hypothesis of the model ($\text{E}(\varepsilon_i)=0$  and $\text{E}(\nu_{n+1})=0$):
	In this case, as by hypothesis $\text{E}(\varepsilon_i)=0$ and $\text{E}(\nu_{n+1})=0$:
	
	Finally:
	
	If the returns are explicitly given and therefore known, the expected mean is calculated with:
	
	This relation is often named "\NewTerm{weighted average return}" or "\NewTerm{weighted average yield}".

	As the customer (investor) will often seek to maximize the expected mean return while minimizing the variance (risk) it remains to determine the latter. Since now we assume explicitly known the returns of the financial assets and the market portfolio returns (index) we have:
	
	Hypothesis: If the index $I$ is selected correctly, when $I=0$ we must have $R_i=0$ implying $\alpha_i=0$ (this is a strong assumption that leads to have an approximation!).

	Therefore:
	
	Finally:
	
	So what would give for portfolio with two assets:
	
	We can condense the writing of the variance using as always the matrix formulation by first noting respectively the transposed vector and the vector-column of the weights of the portfolio by:
	
	and by defining the array of betas:
	
	Which finally gives us:
	
	For a portfolio of two assets this last relationship is reduced to:
	
	We thus find much the same as the algebraic form!

	If we do not know explicitly the returns, the study of the variance is a little trickier. Then we must use the linear model such as:
	
	And always under the assumption already specified above as that if the index $I$ is properly chosen it implies $\alpha_i=0$.

	Also, let us put:
	
	Moreover, we know that:
	
	Therefore ignoring the variance of the intercept:
	
	because:
	
	Finally:
	
	In this context the problem comes down to maximize the economic function $Z$:
	
	and the difference that now it will be written:
	
	The calculation of each of partial derivatives then gives:
	
	or in matrix form:
	
	The resolution of this system then goes through the inversion of a matrix simpler than that of Markowitz model but still requires relatively restrictive assumptions.
	
	Finally, notice that the finance practitioner often use the performance indicators weighted  by the risk, the most internationally known being the "\NewTerm{Sharpe ratio}". It is determined by the ratio (to be more exact it is its expecte mean) between the differential of the returns of an investment (assets) supposed without risk $R_f$ (this difference being named by the practitioners "\NewTerm{active return}") and the performance of our portfolio and standard deviation of this portfolio at the market rates (we will rigorously determine the origin of this relation further in our study of the CAPM):
	
	This relation expresses the level of pure return by volatility unit (or per unit of risk). For simplicity, it is a  (marginal)  profitability indicator obtained per unit of risk taken in an multi-assent portofolio management strategy. It helps answer the question: the portfolio manager is he able to get a higher return to the reference investment but with more risk?
	\begin{itemize}
		\item If the Sharpe ratio is negative, the portfolio performed less than the benchmakr and the situation is very bad.

		\item If the Sharpe ratio is between $0$ and $0.5$, the invested portfolio outperformed the benchmark in a way that is considered as too excessive risk taking. Or, the risk taken is too high for the achieved return performance.

		\item If the Sharpe ratio is greater than $0.5$, the invested portfolio returns outperformed the benchmark by taking ad hoc risk. In other words, the outperformance is not done at the cost of too high risks.
	\end{itemize}
	The Sharpe Ratio can be developed to give a more explicit form relatively to what we have seen just before:
	
	Let us also notice another common indicator that is the "\NewTerm{tracking error}\index{tracking error}" defined as the standard deviation of the difference in performance between the portfolio and the benchmark. The higher the tracking error, the more the investment looks like the benchmark in terms of risk:
	
	\begin{tcolorbox}[title=Remark,colframe=black,arc=10pt]
	Some practitionner prefer to use the "CALMAR ratio" instead of the Sharpe ratio for comparing investment performant. Conceptually, as we have just seen, the Sharpe Ratio divides the average return of an investment by the standard deviation of its returns.  The standard deviation is taken as a measure of the investment's risk.  A higher Sharpe Ratio suggests more returns at lower risk.\\

	But the standard deviation includes variations above the average returns. Most people like those and only worry about the below average returns. The CALMAR ratio uses maximum drawdown (see our study of Time Series further below for the definition) instead of standard deviation as a measure of risk (but it's empirical because we could use also the semivariance and create a bunch of other indicators like the Omega ratio, Sterling ratio, MAR ratio, etc.). So the "\NewTerm{CALMAR ratio}\index{CALMAR ratio}" (acronym of: \textit{\textbf{CAL}ifornia \textbf{M}anaged \textbf{A}ccounts \textbf{R}eports} where it was published for the first time) is an investment's average return (usually for a $3$ year period, ie $36$ months) divided by its maximum drawdown in the same period (analyzed on batches of $36$ months):
	
	A higher CALMAR Ratio suggests more returns at lower risk.\\
	
	It should be known that later versions of the Calmar ratio introduce the risk free rate into the numerator to create a Sharpe type ratio.
	\end{tcolorbox}
	
	These models are relatively complex. This is why some years later, Sharpe and Lintner have created a new model that earned them the Nobel Prize in economics and that we will study afterwards a practical example of what we just saw.
	\begin{tcolorbox}[colframe=black,colback=white,sharp corners]
	\textbf{{\Large \ding{45}}Example:}\\\\
	Let us consider three assets in equal proportions in an invested portfolio and $n$ equal, non-missing, observations of their return $R_{ij}$ entered in the spreadsheet software Microsoft Excel 14.0.7173 as visible below. These returns will be compared to a benchmark that $I$ will be the performance of a benchmark market portfolio $P_M$:
	\begin{figure}[H]
		\centering
		\includegraphics[scale=1]{img/economy/sharpe_model_initial_values.jpg}
		\caption[]{Three assets with returns compared to a reference market portfolio}
	\end{figure}
	The goal will be to determine the portfolio efficiency frontier with Sharpe's model.\\

	In detail under graphical form let us first give the betas (return on assets based on the performance of the market portfolio / benchmark) with Microsoft Excel 14.0.7173 elementary tools:
	\begin{figure}[H]
		\centering
		\includegraphics[scale=1]{img/economy/sharpe_model_linear_regression_for_betas_in_excel.jpg}
		\caption{Linear regression to calculate the betas of the assets returns}
	\end{figure}
	and the following construction list for calculating the betas, variance and the expected mean return of the market portfolio and the different assets:
	\end{tcolorbox}
	
	\begin{tcolorbox}[colframe=black,colback=white,sharp corners]
	\begin{figure}[H]
		\centering
		\includegraphics[scale=1]{img/economy/sharpe_model_initial_values_with_beta_mean_return_and_variance_in_excel.jpg}
		\caption[]{Key indicators of the benchmark portfolio and the $3$ assets}
	\end{figure}
	Explicitly:
	\begin{figure}[H]
		\centering
		\includegraphics[scale=0.7]{img/economy/sharpe_model_initial_values_with_beta_mean_return_and_variance_formulas_in_excel.jpg}
		\caption[]{Key indicators of the benchmark portfolio and the $3$ assets with explicit formulas in Microsoft Excel 14.0.7173}
	\end{figure}
	The expected portfolio return consisting of the three assets is easy to calculate because we have their returns. So:
	
	This gives in Microsoft Excel 14.0.7174:
	\begin{figure}[H]
		\centering
		\includegraphics[scale=1]{img/economy/sharpe_model_invested_portfolio_return_calculation_in_excel.jpg}
		\caption[]{Expected mean return of each asset and of the overall portfolio}
	\end{figure}
	Or explicitly:
	\end{tcolorbox}
	
	\begin{tcolorbox}[colframe=black,colback=white,sharp corners]
	\begin{figure}[H]
		\centering
		\includegraphics[scale=0.8]{img/economy/sharpe_model_invested_portfolio_return_calculation_with_formulas_in_excel.jpg}
		\caption[]{Expected mean return of each asset and of the overall portfolio explicit formulas in Microsoft Excel 14.0.7174}
	\end{figure}
	Now we must calculate the variance using the relation proved in the theoretical part of the preceding paragraphs:
	
	with for reminder in our particular case:
	
	with in our example with $\sigma_I^2=0.039$ (cell B13).

	Or in a expanded form for our example:
	
	Which gives in Microsoft Excel for our array of betas $B_{ij}$:
	\begin{figure}[H]
		\centering
		\includegraphics[scale=1]{img/economy/sharpe_model_beta_matrix_in_excel.jpg}
		\caption[]{Matrix of betas $B_{ij}$}
	\end{figure}
	\end{tcolorbox}
	
	\begin{tcolorbox}[colframe=black,colback=white,sharp corners]
	Or explicitly:
	\begin{figure}[H]
		\centering
		\includegraphics[scale=1]{img/economy/sharpe_model_beta_matrix_with_formulas_in_excel.jpg}
		\caption[]{Matrix of betas $B_{ij}$ explicit formulas in Microsoft Excel 14.0.7174}
	\end{figure}
	And finally the couple mean/variance of the portfolio is given by (cells E16 and E17):
	\begin{figure}[H]
		\centering
		\includegraphics[scale=0.8]{img/economy/sharpe_model_beta_mean_variance_couple_and_all_others.jpg}
		\caption[]{Couple mean / variance of the invested portfolio and global overview of all previous calculations}
	\end{figure}
	Explicitly of the part that interest us:
	\begin{figure}[H]
		\centering
		\includegraphics[scale=1]{img/economy/sharpe_model_beta_mean_variance_couple_formulas_in_excel.jpg}
		\caption[]{Couple mean / variance of the invested portfolio explicit formulas in Microsoft Excel 14.0.7174}
	\end{figure}
	\end{tcolorbox}
	
	\begin{tcolorbox}[colframe=black,colback=white,sharp corners]
	Once this done, we proceed as for the Markowitz method. We use the solver to minimize the variance while imposing an expected mean return and a constraint as what the sum of weight of financial assets must be equal to unity:
	\begin{figure}[H]
		\centering
		\includegraphics[scale=0.85]{img/economy/sharpe_model_beta_excel_solver_settings.jpg}
		\caption[]{Solver setting to find the desired portfolio in Microsoft Excel 14.0.7174}
	\end{figure}
	That gives:
	\begin{figure}[H]
		\centering
		\includegraphics[scale=0.85]{img/economy/sharpe_model_beta_excel_solver_result_desired_portfolio.jpg}
		\caption[]{Desired portfolio results with Microsoft Excel 14.0.7174}
	\end{figure}
	To build the Sharpe efficient frontier we proceed in the same way as for the Markowitz efficient frontier:
	\end{tcolorbox}
	
	\begin{tcolorbox}[colframe=black,colback=white,sharp corners]
	\begin{figure}[H]
		\centering
		\includegraphics[scale=0.85]{img/economy/sharpe_model_beta_efficinet_frontier_list_in_excel.jpg}
		\caption[]{Sharpe efficient frontier values samples}
	\end{figure}
	This gives us the following $\text{E}(R_P)=f(\sigma_{R_p})$ plot (we have put the Sharpe and Markowitz efficient frontier together so that the reader can compare them):
	\begin{figure}[H]
		\centering
		\includegraphics[scale=1]{img/economy/sharpe_model_beta_efficient_frontier_plot_in_excel.jpg}
		\caption[]{Sharpe \& Markowitz efficient frontier plot in Microsoft Excel 14.07.7173}
	\end{figure}
	\end{tcolorbox}
	
	\pagebreak
	\paragraph{Capital Asset Pricing Model (CAPM)}\label{capital asset pricing model}\mbox{}\\\\
	As we have seen before, Markowitz (1959) developed the theory of optimal choice of a portfolio by an individual on the basis of the variance and the expected return. Later (1963), Sharpe developed the theory of optimal choice of a portfolio by an individual on the basis of risk indices such as beta coefficients to simplified the heavy calculations (but since the computer are almost all powerful enough to run the Markowitz model without difficulties).

	Sharpe, Lintner and Mossin (1965) then studied the consequences of these theories to develop an extremely simple theory to evaluate the beta coefficients, expected returns and variances of financial assets portfolio from data statistics on the global market and the specificity of the composition of a portfolio.
	
	This theory based again on the mean-variance problem is named the "\NewTerm{capital asset pricing model (CAPM)}" or "\NewTerm{capital asset pricing model at equilibrium (CAPME)}". This is a frequently used model, both by practitioners than academics to evaluate the anticipated returns of equilibrium of any risky assets in the market.

	To begin, let recall that we have already seen in our study of the return that the short periodic rate of return (daily, weekly, monthly, yearly) of an asset is calculated as follows:
	
	with $P_t$ that is the price of an asset at the period $t$, $P_{t-1}$ the price of the same asset at the previous asset $t-1$ and finally $C_t$ the cash-flow paid by the asset during the period running from $t-1$ to $t$.
	
	This is used as we know to calculate the "realized return" of an asset while in fact it is the "expected return" return that interest a particular investor.

	At the time of taking the decision, the return that will achieve an investor by holding a given asset is uncertain this is why we speak of "expected returns": this is a return that we seeks to evaluate and we hope to receive during the next investment period.

	To calculate the expected return, as we have seen, it should be allocated to each possible value of the return a probability of realization, then calculate a weighted average of the possible values using the probabilities $p_i$ as weights:
	
	However, it is clear that in a given economy, investors will be tempted to hold more that one financial assets for obvious reasons and therefore seek to compose portfolios. The expected mean return of a portfolio may be calculated using the known relation:
	
	where for recall $n$ is the number of assets included in the portfolio, $\text{E}(R_i)$ the return of asset $i$ in the portfolio and $X_i$ the proportion of this asset $i$ in the invested portfolio.

	The expected return rate is however as we know insufficient to characterize an investment opportunity and we must also take into account the risk, ie the variability of return on that investment over financial assets. The variance is as we have already seen used as simple measurement of risk and given for a financial asset by:
	
	Therefore:
	
	The calculation of the risk of a portfolio thus involves without surprise two important concepts: the variability of returns of each asset, measured by the variances of these, as well as the relationships between the various assets composing the portfolio.

	The dependency between two assets is often measured, as we have already mentioned in our previous study of the Sharpe and Markowitz model, by the covariance or the correlation coefficient.

	The covariance between two assets $i$ and $j$ is calculated as follows for recall:
	
	As we know the probabilities are equally likely (and if we do not work on a sample otherwise we we should have $n -1$ instead of $n$ in the denominator for the product factor of the sum):
	
	The covariance between the returns of two assets may be positive or negative and its value has no economic significance - in the sens of causality - as we know (\SeeChapter{see section Statistics page \pageref{linear correlation coefficient}}).
	\begin{tcolorbox}[title=Remark,colframe=black,arc=10pt]
	Let us once again recall that we have seen in the section Statistics that when the returns of two active assets vary in the same direction (respectively in opposite direction) their covariance will be positive (respectively negative).
	\end{tcolorbox}
	The correlation coefficient between two assets $i$ and $j$  meanwhile is calculated as follows (\SeeChapter{see section Statistics page \pageref{linear correlation coefficient}}):
	
	Once the variance and covariance of the various assets calculated, we will be able to calculate the variance of the return of whole portfolio with $n$ sets. This variance is given by the following relation as already proved in the section Statistics and the previous models above:
	
	or written differently:
	
	The above relation of the variance of portfolio return clearly shows that even if the returns to different assets in the portfolio are completely uncorrelated, the variance of the latter can be further reduced by adding more assets.

	To understand this, we notice that for $n$ uncorrelated assets, the variance is reduced to (since the covariance is then zero):
	
	Simplifying even more, if all variances are assumed equal and if all the assets are held in the same proportions ($1 / n$), we have obviously:
	
	Thus, when $n$ approaches infinity, the portfolio variance approaches zero. So if uncorrelated risks are combined portfolio, total risk can be eliminated by diversification, we speak then as we know of "diversifiable risk." In the cases where the risks are correlated, diversification eliminate only the risks specific to the assets while the market risk will continue to exist. 
	
	Notice that the risk reduction is greater when the various assets held are negatively correlated!!! In this situation, it is possible to combine risky assets to form a completely riskless portfolio!!!!

	From the foregoing, it is clear that any investor wishing to build a portfolio will seek to hold a set of risky assets that will allow him to receive some return with minimum risk. In other words, he will seek to minimize the variance for an expected level of performance while respecting some budget constraints. We know that the expected return and the performance variance of a portfolio containing $n$ risky assets are written as follows:
	
	Moreover, we know that from these $n$ assets that it is possible to construct an infinite number of portfolios by varying the weights $X_i$. But the most interesting portfolios for a particular investor portfolios are those that minimize the risk that he must endure to achieve a given level of performance. These portfolios are the result of the following minimization problem is a nonlinear optimization problem (\SeeChapter{see section Numerical Methods page \pageref{nonlinear optimization}}):
	
	that we had already seen in our study of the Markowitz model.
	
	It is therefore possible to create an infinite number of portfolios by varying the proportions invested in each asset. The next step is to select from among all available portfolios, a given portfolio. To do this, one must consider the individual preferences of the investor.
	
	A rational investor should therefore consider only the portfolios lying on the efficient frontier for its investment choices. Its optimal portfolio will be at the point of tangency between the efficient frontier and its highest indifference curve that it would be capable of achieving.

	Let us explicit that a little bit...! Consider for this the following MATLAB™ plot:
	\begin{figure}[H]
		\centering
		\includegraphics[scale=0.9]{img/economy/efficient_frontier_and_indifference_curve.jpg}
		\caption[]{Efficient frontier and indifference curve in MATLAB™ 2013a}
	\end{figure}
	The indifference curve above is quit obvious to understand: the average investor agree to take more risk as the expected returns increase and expect a zero risk when the return is the minimum of the mark risk free rate. This indifference curve is for most people obviously with a convex shape (the opposite would be quite hard to explain...).
	
	By doing so, each investor maximizes his expected utility. In the presence of an economy containing only risky assets, the composition of the portfolio of risky assets varies from one individual to another.
	
	To determine the correct (best) tangent to indifference curve and efficient frontier we can use an information partially available on the market: the risk free rate. Indeed, almost all investors have the possibility to invest in risk free financial assets. We will then seek for the portfolio that is the prolongation of the risk free investment and therefore corresponding to the tangent of a given indifference curve as visible in the figures above and below:
	\begin{figure}[H]
		\centering
		\includegraphics[scale=0.95]{img/economy/efficient_frontier_and_indifference_curve_and_sml.jpg}
		\caption[]{Efficient frontier and SML in MATLAB™ 2013a}
	\end{figure}
	Then let us consider a portfolio that is a combination of risk-free assets (or risk free sub-portfolio) and the at risk asset (a market non-risk free portfolio). Then we have:
	
	where $X_M$ is the fraction of the portfolio invested in the market portfolio ($M$) of return $R_M$ and $R_f$ is as always the risk-free rate return (or "certain return").
	
	We then have (remember that the expected mean of a constant is equal to that constant as proved in the section Statistics):
	
	and therefore:
	
	Either in condensed form:
	
	The derivative of expected return relatively to $X_M$ gives us:
	
	The derivative of the standard deviation relatively to $X_M$ gives us (don't forget that the standard deviation of a constant is equal to zero!):
	
	Putting these results together (ratio), we have:
	
	and since:
	
	It comes then:
	
	And since the main target in finance is to represent graphically as for the Markowitz and Sharpe efficient frontier plot:
	
	So, it is traditional to write the prior-previous relation like an affine function:
	
	This relation is the equation of a line named "\NewTerm{capital market line (C.M.L.)}". The intercept is obviously $R_f$ and its slope is a variable function of the expected mean and standard deviation of the market portfolio and the factor find in front of the standard deviation of the $R_P$ is a coefficient named "Sharpe ratio" (or "Sharpe ratio"):
	
	that we have talked about earlier above, but without prooving the origin. Notice that in a well builded portfolio we should have $S_P>0$ therefore the slope of the C.M.L. should be always positive!

	This is the central relation of the C.A.P.M. model. Any portfolio measured on the market which is outside this line of this model is considered (at least in theory) as not being in balance.

	By construction, this line equation therefore associates with each level of risk, the highest expected return at the intersection of the efficient frontier. Thus, given the performance of a risk-free asset it becomes easy from this equation to determine the point of tangency with the efficiency frontier of Markowitz or Sharpe to get the most efficient portfolio based on the risk-free performance and even without knowing the indifference curve!!

	Let us now to determine an equation for the expected return of any individual asset.

	Consider a new portfolio of return $R_P$ which is a combination of any unique asset $A$ and the reference market portfolio, where $X_A$ is the fraction of the portfolio invested in the asset $A$.

	What we would like to do is to evaluate the slope of the curve of the combinations expected mean/standard deviation of the returns when we combine the market portfolio with the asset $A$.

	We wish to evaluate the value of the slope of the tangent to the efficient frontier such that the weighting of the asset $A$ is zero.
	
	We have:
	
	We get immediately:
	
	and (\SeeChapter{see section Statistics page \pageref{properties of the variance} and page \pageref{covariance}}):
	
	Let us derivative the expected return of the portfolio relatively to $X_A$, we get then for the expected mean:
	
	Doing same for the standard deviation, we get:
	
	The contribution of Sharpe and Lintner was that we must evaluate these derivatives at the point where $X_A=0$ that is to say where the weighting of asset $A$ in the new portfolio is zero.

	In doing so, we get the following expression for the standard deviation of the new portfolio (of course, the expression for the expected return does not change):
	
	which gives after simplification:
	
	With the both derivatives above, we can obtain an expression for the curve of the combinations means/standard deviation for the new portfolio. Then we have:
	
	This slope must be equal to that of the C.M.L. By equalizing, we get:
	
	After a trivial simplification we get:
	
	And therefore after rearranging:
	
	Hence:
	
	By putting what we have already seen during our study of the Sharpe model, that is to say the non-diversifiable risk as beta factor:
	
	So it is the volatility of the return of the non-risky asset compared to that to the market portfolio.

	Then we have:
	
	This expression allows to express the excess return of risk-free asset as the product of the excess return of the market portfolio and the beta factor of the risk-free asset.

	The excess return of an asset does not directly depend of its variance, which is often an intuitive measure of risk of an asset. What counts is its beta factor, which depends on its covariance with the market portfolio.

	More typically, the last relation is used graphically as a straight line equation:
	
	This relation is named the "\NewTerm{security market line (SML)}", and more rarely "\NewTerm{market asset line}", it is extremely important in finance, as it therefore gives the mean expected return of an asset $A$ in function of the beta, of the market return and the risk free rate. This model is say to be "monofactorial" in that it does distinguish only  single explanatory factor of the risk of an asset.

	There are obviously more complex models taking into account other additional factors such as the Fama-French model that use three explanatory variables.
	\begin{tcolorbox}[colframe=black,colback=white,sharp corners]
	\textbf{{\Large \ding{45}}Example:}\\\\
	The risk free rate is $5\%$ and the expected market return is $8\%$. An asset $A$ is twice more sensible to market movement than the benchmark, verbatim when the index recorded an increase of $1\%$, the asset $A$ increase of $2\%$. According to S.M.L, the expected return is therefore by appyling:
		
	\end{tcolorbox}
	\begin{tcolorbox}[title=Remark,colframe=black,arc=10pt]
	The reader will have easily notice that in the above relation, if the beta is equal to unity, the expected return of asset $A$ is then equal to the expected return of the market. The CAPM is then only a tool for determining excess return that we can expect wait when one invests in a portfolio or a company (company whose return is estimated on the basis of its dividend). 
	\end{tcolorbox}
	The latter relation can also be found in the litterature under the form:
	
	with $\lambda$ that is named the "\NewTerm{premium per unit of risk}" or just "\NewTerm{risk premium}" (excess of return required by investors when they put their money in the market benchmark portfolio rather than any  other asset) and the ordinate is obviously the risk free interest rate (usually based on government bonds).

	Finally, let indicate that in theory (in the cas of no arbitrage opportunity) we should have:
	
	but in practice with the numerical values of the market the result can be different from zero. This "anomaly" in the market is therefore named the "\NewTerm{Jensen alpha}":
	
	If Jensen's alpha is greater than $0$, this obviously means that the portfolio asset beats its reference market portfolio. If it is less than $0$, the asset portfolio underperformed the market portfolio.
	
	The CAPM therefore stipulates that the expected rate of return (or that should require a rational investor having a risk aversion) of risky assets must be equal the rate of return of risk-free asset, plus a risk premium named sometimes "\NewTerm{credit spread}". In this case, the relation between systematic risk and expected return remains linear.

	It may be interesting to clarify the mathematically assumptions on which were based developments we have made so far. So these are the assumptions of the CAPM, some of which seem difficult to accept. However the reader should not forget that the validity of a model does not depend on the realism of its assumptions but the conformity of its implications with reality.

	We have recall stated explicitly or implicitly the following assumptions:
	\begin{enumerate}
		\item[H1.] Investors make up their portfolios being concerned exclusively by the expected return and the variance of these
		
		\item[H2.] Investors are risk averse: they do not like risk
		
		\item[H3.] There is no transaction cost (which is a joke in the case of delta-hedging\footnote{most modern software can take transaction costs into account}...)
		
		\item[H4.] The assets are perfectly divisible
		
		\item[H5.] Neither dividends nor capital gains are taxed
		
		\item[H6.] Many buyers and sellers are involved in the market and none of them can have influence on prices (perfect information)
		
		\item[H7.] All investors can lend or borrow the amount they wish at the risk free rate and that latter is know by all
		
		\item[H8.] The anticipation of the different investors are homogeneous
		
		\item[H9.] The investment period is the same for all investors
	\end{enumerate}
	Other more modern and complex models have been developed since obviously! Taking into account for example the asymmetrical distribution of returns (non-normal) or based on the semivariance rather than the variance, base on a priori estimates of the market behavior, or based not only on a single factor, etc.
	
	\begin{tcolorbox}[colframe=black,colback=white,sharp corners]
	\textbf{{\Large \ding{45}}Example:}\\\\
	For the example let us return to our example of the Markowitz model for which we had for recall for the assets:
	\begin{figure}[H]
		\centering
		\includegraphics[scale=1]{img/economy/markowitz_observed_returns_excel_list.jpg}
	\end{figure}
	with some elementary indicators:
	\begin{figure}[H]
		\centering
		\includegraphics[scale=1]{img/economy/markowitz_observed_mean_variance_excel_list.jpg}
	\end{figure}
	And for the covariance matrix:
	\begin{figure}[H]
		\centering
		\includegraphics[scale=1]{img/economy/markowitz_covariance_matrix_excel_list.jpg}
	\end{figure}
	Now we will determine the capital market line for this example which is for recall the line formed by all portfolios composed of assets without risk, on one hand, and the market portfolio, on the other. By construction, it associates at each level of risk, the highest expected return.\\

	To determine this line with Microsoft Excel we set first an initial rate of return without risk $R_f$ which that we arbitrarily take as worth $0.22$ (in Switzerland in 2011, the risk-free yield was estimated at $1\%$ for example...) and to simplify the analysis with the spreadsheet software we will take the naive approach consting be interpolate the efficiency frontier by the equation of a parabola such that:
	\end{tcolorbox}
	
	\begin{tcolorbox}[colframe=black,colback=white,sharp corners]
	\begin{figure}[H]
		\centering
		\includegraphics[scale=0.8]{img/economy/capm_markowitz_parabola.jpg}
		\caption{Approximated parabola to Markowitz efficiency frontier in Microsoft Excel 14.0.7173}
	\end{figure}
	So we have the approximate Markowitz equation curve:
	
	By the construction of the CML we can already draw approximately:
	\begin{figure}[H]
		\centering
		\includegraphics[scale=0.8]{img/economy/capm_markowitz_parabola_cml.jpg}
		\caption{Approximated CML on Markowitz efficiency frontier in Microsoft Excel 14.0.7173}
	\end{figure}
	and let us write the CML equation by:
	\end{tcolorbox}
	
	\begin{tcolorbox}[colframe=black,colback=white,sharp corners]
	
	with the condition (see on the plot above):
	
	We then have two equations with two unknowns to solve this problem (the intersection of the line and parabola for the first equation and the equality o the slope of the line and the parabola at the intersection for the second equation):
	
	The second equation gives us:
	
	Injected into the first equation:
	
	If we solve this polynomial of the second degree we have two real solutions (Microsoft Excel can not determine the roots of this directly but with Maple 4.00b it's very simple):
	
	Following a request from a user, here are the corresponding Maple lines:\\

	\texttt{>a:=18.795;b:=-8.3892;c:=0.9384;\\
		>f:=a*(((-d/0.22)-b)/(2*a))\string^2+b*(((-d/0.22)-b)/(2*a))+c\\
		=(-d/0.22)*(((-d/0.22)-b)/(2*a))+d;\\
		>solve(f,d);
	}
	
	Solution $2$ is to eliminate (we know it by trying to take it as a solution). So we have finally:
	
	Which gives graphically (obviously the CML gives as expected the highest return for a given variance so it was impossible for the line to be left oriented):
	\end{tcolorbox}
	
	\begin{tcolorbox}[colframe=black,colback=white,sharp corners]
	\begin{figure}[H]
		\centering
		\includegraphics[scale=0.8]{img/economy/capm_markowitz_parabola_cml_real_tradition.jpg}
		\caption{Approximated parabola to Markowitz efficiency frontier with CAPM and traditional orientation Microsoft Excel 14.0.7173}
	\end{figure}
	It also comes immediately:
	
	Thus, by reusing the solver as above but with this new value for the expected mean we get for an optimal market portfolio with a risk-free asset having a return $R_f$ of 0.22, an overall return of $0.2314276$... the following asset weights:
	
	with:
	
	So it is by construction the portfolio which, among all portfolios having only risky assets, maximizes performance while it behaves such as having a risk-free asset while minimizing the risk !!! Again, remember that this portfolio is named the "\NewTerm{tangent portfolio}".\\
	
	If the reader refer to our MALTAB companion book he can found the script that leads to the following results:
	\end{tcolorbox}
	
	\begin{tcolorbox}[colframe=black,colback=white,sharp corners]
	\begin{figure}[H]
		\centering
		\includegraphics[scale=0.8]{img/economy/capm_markowitz_cml_matlab.jpg}
		\caption{Markowitz efficiency frontier with CAPM and CML in MATLAB™ 2013a}
	\end{figure}
	and the corresponding numerical values of interest:
	\begin{figure}[H]
		\centering
		\includegraphics[scale=0.8]{img/economy/capm_markowitz_cml_values_matlab.jpg}
	\end{figure}
	Therefore we can see without surprise a significant difference between our rough Microsoft Excel approach and the results given by MATLAB™.
	\end{tcolorbox}
	
	\pagebreak
	\paragraph{Black \& Scholes option pricing models}\mbox{}\\\\
	It is to the genius of three famous mathematicians that the derivatives market (which it is usual to classify into three large families: Derivatives on securities, Derivatives on rates, Derivatives on currencies) owes its success, thanks to Black \& Scholes equation established in the years 1970 (and published in 1973) that theoretically determines the exact premium (under several assumptions) that a customer must pay to acquire a Call or a Put and the strategy to be followed by the seller of these Options to hedge risk (to cite only the most well known example). Obviously this model works only if the temporal periods considered are relatively short (on the order of the week or a few months at best). Beyond the use of this particular theoretical model is quite a joke!
	\begin{tcolorbox}[title=Remark,colframe=black,arc=10pt]
	The dynamics of the Black \& Scholes model had already been well dusted before by Louis Bachelier and Edward O. Thorp, which is why we sometimes refer to the "Bachelor-Thorpe" model. 
	\end{tcolorbox}
	Fischer Black,  Myron Scholes and Robert C. Merton are the ancestors of a generation of sophisticated derivative products, giving a lexicon of terms as exotic as Butterflies, Rainbows, Knock-in, Knock-Out, Barriers, Swaps, Calls, Puts, Sneakers, Swings, Caplets. This model is also considered this said as one of the main factors of the stock market crash of 1987 by some specialists ...

	Evidently, however, there are many more sophisticated empirical models, such as:
	\begin{itemize}
		\item The Black \& Scholes \& Barenblatt model
		\item The Merton Jump Scattering odel
		\item The Heston model
		\item and so on...
	\end{itemize}
	What we would like in what follows is to determine the theoretical value of the premium of an option from the following five data:
	\begin{itemize}
		\item The present value of the underlying financial asset of the option (determined by market speculation)

		\item The time remaining to the option before its maturity (chosen by the issuing company)

		\item The strike price fixed by the issuer subjectively or after modeling.

		\item The risk-free interest rate (assumed to be the expected rate of return of the underlying asset).

		\item The volatility (standard deviation) of the underlying price of the option (measured on the market).
	\end{itemize}
	The premium for the option thus determined should be unique and equitable for both parties. Indeed, the system of options would make it possible to charge an option price (premium) higher than the market forecasts and thus to generate a certain profit and from nothing, but the many players in the market will compete to be fair and attract the customer on their options rather than on those of the competition.

	The modeling of options prices (Black \& Scholes) is based on the use of stochastic differential calculus. Thus, the Black \&  Scholes approach assumes that the evolution of the asset defines a geometric Brownian motion (in the sense that the possible movements of the price tend towards infinity) and that its output defines a Wiener process (a concept that we will define later on).
	
	\subparagraph{Put-Call parity equation}\mbox{}\\\\
	Before attacking somewhat arduous stochastic calculations, it is useful to establish beforehand a so-named "\NewTerm{Put-Call parity equation}\index{Put-Call parity equation}" which will serve as a kind of conservation law to verify the validity of the results that we will establish further for the evaluation of options prices.

	The goal will be to answer the following question: What amount of money $M$ must we pay now to receive a guaranteed sum corresponding to the price "\NewTerm{strike price}\index{strike price}" denoted $J$ to a future time $T$?

	To answer this question, let us recall that during our study of the calculation of interest we saw that in considering a capital $C$ and a constant interest $r$ we had in a case of continuous capitalization:
	
	Therefore, if we but $C=K$ and $C_0=M$ we have then in a universe without risk, that is to say in delta-neutral (the case with risk will be the subject of the Black \& Scholes that we will see later):
	
	Therefore:
	
	But this relationship is not quite right. Indeed, we must have $M = K$ that is guarantee at time $T = t$. Hence we are naturally led to put:
	
	This small elementary reminder being made, we will now assume for what will follow that the Call and the Put possess the following properties:
	\begin{enumerate}
		\item[P1.] Same support that is valued $S$ at time $t$ (spot price\index{spot price}).

		\item[P2.] Same maturity $T$

		\item[P3.] Same strike price $K$
	\end{enumerate}
	And the following assumptions (which are the basis of the Black \& Scholes model that we will describe below):
	\begin{enumerate}
		\item[H1.] There are no transaction costs (which is a joke in delta-hedging strategies...)

		\item[H2.] Delta-hedging eliminates all risk

		\item[H3.] The support is not a term instrument (it is possible to make short sales)

		\item[H4.] The support does not pay dividends over the life of the option (ie between $[0, T]$).

		\item[H5.] The options are European


		\item[H6.] The risk-free market rate is assumed to be known in advance and considered as constant

		\item[H7.] There is no arbitrage opportunities (but well in practice ... there is almost only this)

		\item[H8.] The price of options does not affect the price of the underlying (mouarf!)

		\item[H9.] Yields are not autocorrelated
	\end{enumerate}
	By asking us again now the initial question: How much do we have to pay now for a portfolio in order to receive a guaranteed sum $K$ (strike price) at a future time $T$?

	Since the portfolio can be considered as a black box, there is nothing to prevent us by denoting by $S$ the actual price of the underlying, $P$ that of the Put and $C$ that of the Call:
	
	which is nothing else than the "\NewTerm{Put-Call parity equation}\index{Put-Call parity equation}" or otherwise written if we put the beginning of time at zero, then:
	
	Strictly speaking, it would be better to write this last relation in the following form:
	
	Whatever the rearrangement of the terms in this last equality, there must always be equality, otherwise there is an arbitrage opportunity . This relation also shows that the value of a European Call with an exercise price $K$ and maturity $T$ can be deduced from that of a European Put with the same exercise price $K$ and the same maturity $T$.
	
	Let us see more in detail following request of a reader:
	\begin{theorem}
	The Put-Call parity equation arises from the absence of arbitrage opportunity.
	\end{theorem}
	\begin{dem}
	Let us consider the frequent and natural configuration of a first economic agent with a portfolio $P_A$ containing a European Call option ($C$) and an amount $Ke^{-rT}$ invested in a risk-free investment. The initial value of this portfolio will therefore be:
	
	Let us consider the second configuration naturally opposed to the first portfolio containing a European Put of price $P$ and the underlying of price $S$. The initial value of this portfolio will therefore be:
	
	We then have two scenarios:
	\begin{enumerate}
		\item[S1.] At maturity (maturity) $T$, if the underlying has a value less than or equal to the strike price (ie $S_T\leq K$), the portfolio $P_A$ will have for value $K$ because the Call option will most likely not be exercised and therefore its financial value becomes zero but the risk-free asset (placed at the risk-free rate) will have give a return equal $K$. The portfolio $P_B$ will be him equal to $(K-S_T)+S_T=K$ where the term in parentheses represents the gain generated by the Put.

		\item[S2.] At maturity (maturity) $T$, if the underlying has a value higher than the strike price (ie $S_T>K$), we are therefore in a situation contrary to that of the first scenario. The portfolio $P_A$ will then have for value $(S_T-K)+K=S_T$. The portfolio $P_B$ will be him equal to $0+S_T=S_T$ because the sale will not be made and the Put option will be considered as having a zero value.
	\end{enumerate}
	To summarize:
	\begin{table}[H]
		\centering
		\begin{tabular}{|
>{\columncolor[HTML]{C0C0C0}}c |c|c|}
\hline
 & \cellcolor[HTML]{C0C0C0}\textbf{If $\pmb{S_T\leq K}$} & \cellcolor[HTML]{C0C0C0}\textbf{If $\pmb{S_T>K}$} \\ \hline
\textbf{$\pmb{P_A}$} & $0+K=K$ & $(S_T-K)+K=S_T$ \\ \hline
\textbf{$\pmb{P_B}$} & $(K-S_T)+S_T$=K & $0+S_T=S_T$ \\ \hline
		\hhline{|=|=|=|}
\textbf{Conclusion:} & $P_A=P_B$ & $P_A=P_B$ \\ \hline
		\end{tabular}
		\caption[]{Summary of lack of absence of arbitrage opportunity}
	\end{table}
	So the two portfolios are always equal at maturity otherwise there is an arbitrage opportunity. What we can write (the index $T$ for the price of the underlying at maturity is often omitted):
	
	\begin{flushright}
		$\square$  Q.E.D.
	\end{flushright}
	\end{dem}
	
	\pagebreak
	\subparagraph{Efficient Market Hypothesis}\mbox{}\\\\
	The Black \& Scholes model (and many other financial models) is based on the premise that the market is "efficient".

	\textbf{Definition (\#\mydef):} An "\NewTerm{efficient market}\index{efficient market}\index{efficient market hypothesis}" is a market where prices fully reflect all available information. Thus, if the market is efficient, it is not possible to make abnormal profits.

	We can distinguish three types of efficient markets that depend on the type of information available:
	\begin{enumerate}
		\item The efficient market hypothesis in "weak form", which explicitly states that prices reflect all the information contained in the historical price series.

		\item The efficient market hypothesis in "semi-strong form" establishes that prices reflect all available public information.

		\item The "strong form" market assumption that all known public and private information is reflected in market prices.
	\end{enumerate}
	Several studies have attempted to test the hypothesis of asset market efficiency. To test the weak form of the hypothesis, the analysis of the time series (see below) was used by specifically testing the hypothesis of a random walk (a "Brownian motion" - we will return to this later). More specifically, these tests have tried to test whether price increases are independent of past increases. If the random walk hypothesis is rejected, then the market is not efficient, since past price increases could help to anticipate future asset prices. The empirical evidence supports the efficient market hypothesis in weak form. In order to test the semi-strong form of the hypothesis, the rate of adjustment of market prices to the arrival of new information has been evaluated. The evidence for a rapid adjustment of market prices is dominant. The strong form of the market efficiency hypothesis is to test whether it is possible to take advantage of privileged information (information accessible to a small group of economic agents). Since non-public information can not be identified, a strong form test type considers reviewing the investment performance of individuals or groups who may have private information. Edwin J. Elton and  Martin J. Gruber (1984) report that the analysis of the performance of mutual funds, after deducting costs, supports the strong form of efficiency.

	This implies the following assumptions (to summarize):
	\begin{enumerate}
		\item[H1.] The past history of the asset price is completely reflected in the present price which does not contain any other information about the asset.

		\item[H2.] The market immediately incorporates any new information into the price of an asset.
	\end{enumerate}
	The paradox of the Hypothesis of Market Efficiency is that if every investor truly believed that the market was perfectly efficient, then nobody would study companies, their balance sheets, and so on. It would suffice to buy the index. Indeed, efficient markets rely on individuals who are active on the market because they believe (and their believes are true when there is an opportunity arbitrage!) that this market is "inefficient" and that they can do better than the market!
	
	This hypothesis is the source of much debate in the field ...
	
	\begin{tcolorbox}[title=Remark,colframe=black,arc=10pt]
	With the two assumptions previously stated, any unanticipated change in the asset price is named a "\NewTerm{Markov process}\index{Markov process}\footnote{A Markov process is a process whose future evolution depends on its past only through its state at the present moment. The course of an action is probably not a Markov process (the "memory" of the process is probably longer - for example, a seasonal trend).}".
	\end{tcolorbox}

	\subparagraph{Wiener Process}\mbox{}\\\\
	Let $\Delta z$ be the variation of the non-trend value of an asset over a small time interval denoted $\Delta t$.

	Using the knowledge of the two major results of the Bachelier model, we therefore have for the variations of the value of the asset a positive expected mean dependent in a manner proportional to the square root of time according to:
	
	Where we put as hypothesis (acceptable ... because we work on small variations for recall!) that the coefficient of instability is a function:
	
	where $\mathcal{N}(0,1)$ is as we know the notation of the reduced normal Normal law as we have established it in the Statistics section.
	\begin{tcolorbox}[title=Remark,colframe=black,arc=10pt]
	Often in the field of economics, we write it $W\mathcal{N}$ instead of $\mathcal{N}$ in tribute to Wiener.
	\end{tcolorbox}
	That said, the antecedent relation is often denoted in a generalized way:
	
	and defined as a "\NewTerm{standard Brownian motion}\index{standard Brownian motion}\label{standard brownian motion}" with "\NewTerm{white noise}\index{white noise}" (Normal marginal law), or "\NewTerm{arithmetic Brownian motion}\index{arithmetic Brownian motion}", where the $W$ is there for homage to Wiener! It is interesting to notice that the Brownian motion is supposed indefinitely divisible (which means that the temporal period taken does not influence the law of probability which remains always the same ... it is a fractal property of the Brownian motion which has been study by Benoît Mandelbrot too!).

	Do not forget for a proof that we will do further below than we can deduce that:
	
	It is possible to produce a plot of this Brownian motion with Microsoft Excel (here the version 11.8346) with in the column \texttt{A} the time with a typical equation step of $0.01$ [s] (column which will be there only for the sake of comfort of reading and by tradition) and in cell \texttt{B2} the following formula:
	\begin{center}
	\texttt{=B1+NORMSINV(RAND())*SQRT(0.01)}
	\end{center}
	where \texttt{B1} contains the value $0$.

	We then get the following variations of values for $4$ columns of the same type:
	\begin{figure}[H]
		\centering
		\includegraphics[scale=1]{img/economy/brownian_arithmetic_plot_excel.jpg}
		\caption{Example of $4$ standard Brownian motions with Microsoft Excel 11.8346}
	\end{figure}
	Standard Brownian movements have some remarkable properties as we can see: the trajectory tends to alternate above and below the abscissa axis. This comes from the fact that the Normal law considered has zero mean, in other words that there is no general tendency to increase or decrease the variations (to check it made at least $30,000$ points in Microsoft Excel and you will see...).

	It is easily possible to characterize $\Delta z$ with its mean:
	
	Indeed, let us recall that for the Normal reduced centered law, we have:
	
	So we could expect this result of total absence of general trend (it was quasi-intuitive!).

	We can also characterize $\Delta z$ using its variance:
	
	hence:
	
	Indeed, let us recall that for the Normal reduced centered law , we have:
	
	Finally (in fact this result immediately follows from the linearity property of the Normal law):
	
	So when we have:
	
	it is customary to speak of the "\NewTerm{Gauss-Wiener process}\index{Gauss-Wiener process}" or more simply of "\NewTerm{Wiener process}\index{Wiener process}".

	To summarize a little bit all this...:
	\begin{enumerate}
		\item We knew with Bachelier's model that the positive expected mean and the positive standard deviation of the value are proportional to the square root of time. We used these two results here.

		\item We now know (under the well-defined hypothesis of a Normal type coefficient) that the variations have zero expected mean (trend) and a standard deviation proportional to the square root of the temporal variation.
	\end{enumerate}
	The property which has just been established remains valid for a large time interval denoted $T$ corresponding to $n$ small intervals $\Delta T$! In other words:
	
	In this context, it is appropriate to replace large variations by $\Delta x$ such that:
	
	But without any trend, we have for the moment the equality:
	
	As in the case of a change in the price over a short period of time, it is possible to characterize $X(T)-x(0)$ using its expected mean and its standard deviation:
	
	which makes sense ...

	We then fall back, for a large time interval $T$:
	
	Which we can also write in the following form using the properties proven in the section Statistics of the Normal law:
	
	Result we could reasonably expect with the aforementioned assumptions ...

	This last result is written in the following explicit form in spreadsheets softwares:
	
	and we see that this is unrealistic because it would mean that any financial asset follows the same law (regardless of its volatility ...) and would have no general downward or upward trend. We will see how to improve this approach.

	To close this approach, notice that if $\Delta t$ tends to $0$ (which amounts to considering a subdivision of time $T$ into extremely small intervals), the course undergoes an infinitely large number of variations over period $T$. In other words, the process of evolution of the price of the asset is continuous, which leads to replace $\Delta t$ by $\mathrm{d}t$, $\Delta x$ by $\mathrm{d}x$ and $\Delta z$ by $\mathrm{d}z$.

	In this case, we get:
	
	which defines a "\NewTerm{Wiener process}\index{Wiener process}" (we will come back to this when we will have established the stochastic differential equation).

	But obviously this does not really conform to the reality as we have already mentioned ... We prefer then to add a constant shift in time which gives the Brownian motion that we will see now.
	
	\pagebreak
	\subparagraph{Generalized Brownian motion}\mbox{}\\\\
	In this case (generalization a little more realistic), the evolution of the course depends not only on a standard Brownian random process (second term below on the right of the equality), but also on a parameter of central tendency, or "drift" (first term below to the right of the equality):
	
	With always:
	
	and:
	
	We thus have a "\NewTerm{generalized Brownian motion}\index{generalized Brownian motion}", consisting of a standard Brownian motion ($\mathrm{d}z$ represented by a Normal law of zero expected mean and variance $\mathrm{d}t$ as we saw above) and a drift. In this scenario, $a$ and $b$ are imposed as constant contrary to the more general case that we shall see later.

	The antecedent relation is often represented in the literature in the following differential form:
	
	So graphically this gives, by adding this drift and taking a positive and non-zero value for $a$, a standard Brownian motion that will tend to alternate above and below the drift:
	\begin{figure}[H]
		\centering
		\includegraphics[scale=1]{img/economy/brownian_arithmetic_with_drift_plot.jpg}
		\caption{Example of standard Brownian motion with drift (and decomposition)}
	\end{figure}
	On a small time interval $\Delta t$, the process, in discrete time is obviously written:
	
	In this case, we have:
	
	since only $\Delta z$ has a random component.

	Therefore:
	
	Finally:
	
	By subdividing a period $T$ into $n$ time intervals $\Delta t$ (ie $T=n\Delta t$), the price change becomes over this period $T$:
	
	Since then:
	
	Finally:
	
	Therefore:
	
	or also:
	
	It is then easy to understand why we say that the Normal law governs the random variable obtained by stopping a Brownian process at a given instant: it is an instantaneous picture of the Brownian motion simple or generalized!

	By choosing:
	
	We then have the prior-previous relation which is traditionally written in the following explicit form in spreadsheets softwares:
	
	where $\mu$ is the average return in $\%$ of the financial asset and $\sigma$ the volatility of the return in $\%$. Or written otherwise:
	
	In practice, the risk-free rate is chosen to represent the underlying returns because we assume that we always work in a neutral risk probability (absence of arbitrage opportunity).

	It is then interesting for the financial analyst to visualize the expected mean as a function of $t$ and the value $x(t)$ corresponds to a cumulative probability of $2.5\%$ and $97.5\%$ on a plot to have an idea of the evolution of the confidence interval at $95\%$ of its $x(0)$. This is very easy to get in a spreadsheet software like Microsoft Excel and we can typically falls by playing with several types of charts on the same type of plot as shown below for a portfolio of $500$ million USD with $5\%$ yield and standard deviation of $20\%$:
	\begin{figure}[H]
		\centering
		\includegraphics[scale=0.9]{img/economy/brownian_arithmetic_with_drift_plot_excel.jpg}
		\caption[]{Standard Brownian motion with confidence interval plot in Microsoft Excel 11.8346}
	\end{figure}
	With the following table:
	\begin{figure}[H]
		\centering
		\includegraphics[scale=0.8]{img/economy/brownian_arithmetic_with_drift_plot_data_excel.jpg}
		\caption[]{Data table corresponding to the previous plot in Microsoft Excel 11.8346}
	\end{figure}
	Using the relations introduced above:
	\begin{figure}[H]
		\centering
		\includegraphics[scale=0.5]{img/economy/brownian_arithmetic_with_drift_plot_data_excel_explicit_formulas.jpg}
		\caption[]{Data table corresponding to the previous plot in Microsoft Excel 11.8346 with explicit formulas}
	\end{figure}
	Obviously, in practice it is possible to make this type of graph with any data comprising a linear drift and whose standard deviation is known.
	
	\pagebreak
	\subparagraph{Brownian bridge}\mbox{}\\\\
	Let us now consider a case of application of the Brownian motion which will be useful to us later! By revising the developments seen above, it comes immediately that:
	
	Indeed, let us check this by choosing:
	
	We have:
	
	And therefore for $t=1$:
	
	And similarly for $t=2$:
	
	This is indeed equivalent to:
	
	For $\Delta t=2$.
	
	Well, this being said, we would like to build a Brownian motion which starts at the time $t=0$ from zero (which imposes $x(0)=0$) fluctuates during a given period $T$ and returns to zero as shown in the two figures below:
	\begin{figure}[H]
		\centering
		\includegraphics[scale=1]{img/economy/browian_bridges_examples.jpg}
		\caption{Two examples of Brownian bridges}
	\end{figure}
	As it is usual, let us choose the following notation:
	
	If we want the Brownian motion to vanish at time $T$, it is sufficient to make a simple subtraction of the following type:
	
	where BB means "\NewTerm{Brownian bridge}\index{Brownian bridge}". If we stop here, we have indeed:
	
	But we have a problem with time $t=0$:
	
	and to make sure that the difference is zero, the natural trick is to write:
	
	Thus, we have:
	
	But then comes another problem. Indeed:
	
	So the final solution is to write:
	
	and therefore:
	
	and we have:
	
	The Brownian Bridge:
	
	works for all the Wiener's processes, reason why you will sometimes see in the literature sometimes bridges with:
	
	or for example with:
	
	which is only one particular case of the preceding with $W_0=1$.
	\begin{tcolorbox}[colframe=black,colback=white,sharp corners]
	\textbf{{\Large \ding{45}}Example:}\\\\
	To build such a process in Microsoft Excel 11.8346 for example we build:
	\begin{figure}[H]
		\centering
		\includegraphics[scale=0.6]{img/economy/brownian_bridge_explicit_formulas_excel.jpg}
		\caption{Explicit formulas in Microsoft Excel 11.8346 to build a Brownien bridge}
	\end{figure}
	as the time interval is $\Delta t=0.01$ then $T=1/0.01=100$ and we get visually:
	\begin{figure}[H]
		\centering
		\includegraphics[scale=0.6]{img/economy/brownian_bridge_plot_excel.jpg}
		\caption{Brownian bride in Microsoft Excel 11.8346}
	\end{figure}
	\end{tcolorbox}
	
	
	\subparagraph{Itô process}\label{ito process}\mbox{}\\\\
	 Let us consider now a Brownian process corresponding to a variation of $x$ in continuous time defined by:
	
	$a$ and $b$ then being functions of the two variables $x$ and $t$. This consideration is what we name an "\NewTerm{Itô process}", which is named after Kiyosi Itô. It is therefore a generalization of the previous case where $a$ and $b$ are no longer constant.

	It is possible to calculate the expected mean and the variance of $\mathrm{d}x$ exactly in the same way as for the Wiener process and we obtain very easily by analogy:
	
	Therefore, we can write:
	
	Where $a(x, t)$ corresponds to the "instantaneous drift" and $b (x, t)$ to the "instantaneous variance".
	\begin{tcolorbox}[title=Remark,colframe=black,arc=10pt]
	It seems that the general case where $a$ and $b$ depend on time and of a parameter $x$ (which, as we shall see later, corresponds to the strike price) is part of a model class which is named "\NewTerm{Heat-Jarrrow-Mortong models}\index{Heat-Jarrrow-Mortong models}". In the case where only $b$ would no longer be dependent on time, it seems that than it is named "\NewTerm{Ho \& Lee model}\index{Ho \& Lee model}". In reality there is a plethora of empirical models that adapt more or less well according to the situations and to study the whole, a life is no longer enough as for each model you can found a book between $200$ and $400$ pages...
	\end{tcolorbox}
	The "\NewTerm{geometric Brownian motion}\index{geometric Brownian motion}" which allows to theoretically define a possible prediction of the evolution of the yield of an asset is a special case of Itô processes (among so many other models ...) where we assume that:
	
	So the variance and the drift is independent of time! The advantage of assuming this is that unlike the Brownian arithmetic motion we will see that the yield then follows a log-normal law and therefore can not be negative. However, the two processes have two common weaknesses: the price of the associated asset can tend towards infinity in an infinite time (which corrects the Vasicek process) and both consider volatility as constant in time...

	Then we can write the expression of the geometric Brownian motion of the value of the asset as following:
	
	often represented in the literature also in the following form:
	
	Or even more explicitly:
	 
	\begin{tcolorbox}[title=Remark,colframe=black,arc=10pt]
	The "\NewTerm{Chan-Karolyi-Longstaff-Sanders model}\index{Chan-Karolyi-Longstaff-Sanders model}" is an attempt to generalize many stochastic process models use in finance:
	
	Here are some of the best known cases used in finance:
	\begin{table}[H]
	\centering
	\label{my-label}
	\begin{tabular}{|l|l|}
	\hline
	\rowcolor[HTML]{9B9B9B} 
	\multicolumn{1}{|c|}{\cellcolor[HTML]{9B9B9B}{\color[HTML]{333333} \textbf{Reference}}} & \multicolumn{1}{c|}{\cellcolor[HTML]{9B9B9B}{\color[HTML]{333333} \textbf{Model}}} \\ \hline
	\rowcolor[HTML]{FFFFFF} 
	Merton (1973) &  $\mathrm{d}X_t=\alpha\mathrm{d}t+\sigma\mathrm{d}W_t$ \\ \hline
	\rowcolor[HTML]{FFFFFF} 
	Vasiceck (1977) &  $\mathrm{d}X_t=(\alpha+\mu X_t)\mathrm{d}t+\sigma\mathrm{d}W_t$ \\ \hline
	\rowcolor[HTML]{FFFFFF} 
	Cox-Ingersoll-Ross (1985) & $\mathrm{d}X_t=\alpha\mathrm{d}t+\sigma\sqrt{X_t}\mathrm{d}W_t$ \\ \hline
	\rowcolor[HTML]{FFFFFF} 
	Dothan (1978) &  $\mathrm{d}X_t=\sigma\mathrm{d}W_t$ \\ \hline
	\rowcolor[HTML]{FFFFFF} 
	\parbox{5.5cm}{Rendlemann-Bartter Geometric\\ brownian motion (1905)}  &  $\mathrm{d}X_t=\mu X_t\mathrm{d}t+\sigma X_t\mathrm{d}W_t$ \\ \hline
	\rowcolor[HTML]{FFFFFF} 
	Schwartz and Brennan (1980) & $\mathrm{d}X_t=(\alpha+\mu X_t)\mathrm{d}t+\sigma X_t\mathrm{d}W_t$ \\ \hline
	\rowcolor[HTML]{FFFFFF} 
	Cox and al. (1980) &  $\mathrm{d}X_t=\sigma X_t^{3/2}\mathrm{d}W_t$ \\ \hline
	\rowcolor[HTML]{FFFFFF} 
	Variance with constant elasticity & $\mathrm{d}X_t=\mu X_t \mathrm{d}t+\sigma x_t^\gamma \mathrm{d}W_t$ \\ \hline
	\rowcolor[HTML]{FFFFFF} 
	CKLS (1992) &  $\mathrm{d}X_t=(\alpha+\mu X_t)\mathrm{d}t+\sigma X_t^\gamma \mathrm{d}W_t$ \\ \hline
	\rowcolor[HTML]{FFFFFF} 
	Heston (1993) & \parbox{6cm}{$\mathrm{d}X_t=\mu X_t \mathrm{d}t+\sigma_t X_t\mathrm{d}W_t^X$\\ $\mathrm{d}\sigma_t=\kappa(\theta-\sigma_t)\mathrm{d}t+\xi\sqrt{\sigma_t}\mathrm{d}W_t^\sigma$}\\ \hline
	\rowcolor[HTML]{FFFFFF} 
	$\ldots$ & $\ldots$  \\ \hline
	\end{tabular}
	\caption{Some Chan-Karolyi-Longstaff-Sanders derived stochastic processes}
	\end{table}
	\end{tcolorbox}
	Considering an asset returns and its variance as constants in time are choices based on the empirical observation of markets when assets are stocks. However, some more sophisticated models have asset returns and variances that depend on the asset value in time.

	The financial interpretation of the relation:
	
	Becomes apparent when we divide the two members by $x$:
	
	Which corresponds to the rate of return of the asset over an infinitesimal period $\mathrm{d}t$.

	The geometric Brownian motion is therefore a priori a good candidate to model the evolution of the price of certain financial assets from its rate of return.

	In the specialized literature, the return is also sometimes denoted (justified notation) in the form of the following stochastic differential equation (S.D.E.):
	
	where $S_t$ is obviously the price of the underlying named as we know already the "\NewTerm{stock price}\index{stock price}" at time $t$, $\mu$ is named the "\NewTerm{drift}\index{drift}" (often assimilated to yield) and $\sigma$ the "\NewTerm{volatility}\index{volatility}" (yield volatility). It is the notation and the vocabulary that we will adopt for the rest of the development. Let us notice that we see better than this assumes that the drift is independent of the value of the underlying (which is a verified observation for some assets).

	In the case of a continuous dividend payment, given $q$ the dividend rate, the dynamics of the price can be written intuitively:
	
	Notice that since we have:

	We can also write (we continue with the case without dividends!):
	
	In the case where $\sigma=0$ (Wiener process, ie the price of the underlying is perfectly known at a given time and without risks), we find ourselves with a differential equation (well known in the finance domain) that we can solve immediately:
	
	This is therefore an exponential (as the continuous interest we saw at the beginning of this section). This relation is valid only if the time interval is therefore very small and return not too big.

	We will now see thanks to the "\NewTerm{Itô lemma}\index{Itô lemma}" that it is possible (which is not the  unique possibility!) to establish that such a process can define a log-normal law (\SeeChapter{see section Statistics page \pageref{log normal distribution}}).

	The Itô lemma is derived from the Taylor expansion (\SeeChapter{see section Sequences and Series page \pageref{taylor series}}) with $2$ variables $x$ and $t$ and given by:
	
	with $F(0)=0$ at the origin of the Brownian motion.

	Considering $\Delta t\ll\Delta x$, and taking the terms only up to the second order (formal perilous approximation but numerically not mandatory thanks to computing power of modern computers), we have:
	
	Let us now return to:
	
	Let us raise to the square, we get:
	
	But:
	
	And as we have proved it in the Statistics section:
	
	Then we have:
	
	Therefore:
	
	By the way:
	
	which both tend to $0$ when $\Delta t$ tends to $0$.

	We also have by considering a subdivision of time in extremely small intervals $\mathrm{d}t$ (approximation which will be useful to us a little later):
	
	which implies in that latter case:
	
	Thus taking place in continuous time (thus a continuous model with extremely small time intervals), the application of the discrete Taylor development seen above:
	
	Can then be written:
	
	Therefore:
	
	This final result is the "Itô's lemma" also named "\NewTerm{Itô-Doeblin theorem}\index{Itô-Doeblin theorem}".

	Therefore as the reader can read it on Wikipedia: Itô's lemma is an identity used in Itô calculus to find the differential of a time-dependent function of a stochastic process. It serves as the stochastic calculus counterpart of the chain rule. It can be heuristically derived by forming the Taylor series expansion of the function up to its second derivatives and retaining terms up to first order in the time increment and second order in the Wiener process increment. The lemma is widely employed in mathematical finance, and its best known application is in the derivation of the Black–Scholes equation for option values.
	\begin{tcolorbox}[title=Remarks,colframe=black,arc=10pt]
	\textbf{R1.}  Compare the form of the last equality with the relation: $\mathrm{d}x=a(x,t)\mathrm{d}t+b(x,t)\mathrm{d}z$\\
	
	\textbf{R2.} The expression in parentheses above (factor of $\mathrm{d}t$) is sometimes referred to as the "\NewTerm{Kolmogorov backward equation}\index{Kolmogorov backward equation}".
	\end{tcolorbox}
	If we take:
	
	Therefore:
	
	In that case:
	
	Returning to the geometric Brownian motion hypothesis, we know that we must consider that:
	
	Therefore we have:
	
	and we finally get the following stochastic differential equation with constant coefficients:
	
	Either by taking the notation of the beginning in explicit form:
	
	Or in another even more explicit form:
	
	Sometimes named the "\NewTerm{integral equation of price dynamics}\index{integral equation of price dynamics}".

	A trivial approach to the resolution of this integral is (we always consider in this model that volatility is independent of time...):
	
	Therefore:
	
	Hence:
	
	\begin{tcolorbox}[title=Remarks,colframe=black,arc=10pt]
	\textbf{R1.} Remember that we started from the relation $\mathrm{d}F=\mu x\mathrm{d}t+\sigma x\mathrm{d}z$\\
	
	\textbf{R2.} Brownian movements were successively discovered from the hypothesis of Normality in the 1960s and then from the stability hypothesis in the 1980s. With these two hypotheses, mathematicians rank them in the special category of "\NewTerm{$2$-stable Levy processes}\index{Levy process}".
	\end{tcolorbox}
	$\mathrm{d}F$ defines a geometrical Brownian motion with particular drift whose parameters that we can now measure (this is what we wanted to achieve). Therefore, the results we obtained for the Brownian motion can be recovered and allow us to write in the end:
	
	Which means that $\mathrm{d}F$ follows a log-normal law (\SeeChapter{see section Statistics page \pageref{log normal distribution}}) of parameters:
	
	Or otherwise written:
	
	
	A small reminder with what was seen in the Statistics section but applied to finance of the log-normal law finance may be welcome. Consider the present value of a given asset (compound interest) whose rate varies for each period:
	
	So if we take the logarithm:
	
	Thus, if the rate of return is a variable random and independent from period to period and identically distributed, then we have an expression which will give a Normal distribution according to the central limit theorem (\SeeChapter{see section Statistics page \pageref{central limit theorem}}). Therefore, a random variable is log-normalized when its logarithm is Normally distributed!
	\begin{tcolorbox}[colframe=black,colback=white,sharp corners]
	\textbf{{\Large \ding{45}}Examples:}\\\\
	E1. Let us consider a stock with a current price of $40$.-, with an assumed average geometric yield of $16\%$ per year and an annual volatility of $20\%$. In $6$ months ($T = 0.5$) we then have:
	
	We can then build a $95\%$ confidence interval (\SeeChapter{see section Statistics page \pageref{confidence interval}}):
	
	And therefore:
	
	So there is $95\%$ cumulative probability that in $6$ months the value of the security is between $32.55.-$ and $56.56.-$.\\
	
	E2. Knowing that:
	
	and that a continuous rate:
	
	We then have for performance the possibility of applying the same type of calculations as before:
	
	\end{tcolorbox}
	Let's go a step further by integrating the differential element. So we have:
	
	Let us integrate this last relation:
	
	The first primitive is simple:
	
	The second primitive is also quite simple (no integration constant because at time zero the expectation of gain is zero):
	
	The third primitive is equal to (no integration constant because at time zero the value of the gain is perfectly known as being equal to zero):
	
	\begin{tcolorbox}[title=Remark,colframe=black,arc=10pt]
	Caution to the abuse of writing here!!! In the root it is implicitly a $ \mathrm{d}t$ (therefore the analysis time step!) and not simply a $t$. Remember that for the remaining developments below!!!
	\end{tcolorbox}
	We therefore have:
	
	And finally (we put the "$-$" sign in the constant):
	
	To find the meaning of the first factor, it suffices to state the initial condition:
	
	We then have immediately for the final expression of the geometric Brownian which is for recall a random variable:
	
	that was obtained by P. Samuelson in 1965 and which is sometimes named the "\NewTerm{Bachelier-Samuelson model}\index{Bachelier-Samuelson model}" or "\NewTerm{log-normal representation of the value of the asset}\index{log-normal representation of the value of the asset}" or "\NewTerm{Itô's solution}\index{Itô's solution}".

	This last relation is often written in the following form:
	
	Notice that if we write\label{jensen inequality finance} (due to the Jensen's inequality proven in the Statistics section page \pageref{jensen inequality}):
	
	this gives us the expectation mean of positive deviations from the strike price. And if we update this value in the idea that this sum of money can be invested in a risk-free return asset, we then get the Monte Carlo version of the evaluation of a Call type option:
	
	And in general the same form of relation written for a Put will be named the "\NewTerm{fundamental relation of option pricing}\index{fundamental relation of option pricing}". In reality we must take care to the fact that here the word "fundamental" means that we must be careful, however, to choose the right stochastic process, which is why in the literature we have more often the following expression that generalizes the previous relationship:
	
	where $f$ is a function of a given stochastic process (since there are several functions depending on the type of underlying), and the function $D$ is named a "\NewTerm{discount factor}\index{discount factor}". For this last relation, we find almost as many different writings as there are textbooks on the subject...

	We can classify the dynamics of this function in three particular cases:
	\begin{itemize}
		\item If $\mu>\dfrac{1}{2}\sigma^2$, $X_t\rightarrow +\infty$ when $t\rightarrow +\infty$
		\item If $\mu<\dfrac{1}{2}\sigma^2$, $X_t\rightarrow 0$ when $t\rightarrow +\infty$
		\item If $\mu=\dfrac{1}{2}\sigma^2$, $X_t$ will randomly fluctuate between high and small values when $t\rightarrow +\infty$
	\end{itemize}
	Following the request of a reader, although we have seen that the Brownian geometric motion was built higher from a log-Normal law of which we know the expected mean, let us do the inverse calculation, that is to say to calculate the expected mean of the Brownian geometric movement to evidently in the end fall back on our feet:
	
	So the term in parentheses follows by definition (or "by construction" if you prefer) a log-Normal law. The expected mean of the latter is then:
	
	To price an asset according to this model over a horizon, we can use the following Monte-Carlo VBA function (the reader can also refer to our R or MATLAB™ companion book for the same calculation):
	\begin{figure}[H]
		\centering
		\includegraphics[scale=1]{img/economy/pricing_stock_vba.jpg}
	\end{figure}
	We finally have a formulation (in the form of a probabilistic distribution function) of a temporal variation and of the intrinsic return of an asset that can be used for investment decision-making purposes on a forecast. But this model is still too smooth to model stock market crashes or bubbles (it is the same for recall with the standard Brownian motion) that can occur in the long term. This is why some more recent models that we will not study here add a Poisson process (discrete and with rare events by construction) to that of Wiener.

	There are other models than the log-Normal but this one by is by the fact that it is the most easy to understand the most widespread. However, other more general methods must be encouraged!

	To conclude this section, we summarize by a comparison the standard Brownian motion and the geometrical Brownian motion that govern the price dynamics when the parameters (instantaneous yield and volatility) are given in $\%$:
	
	and let us recall that the advantage of the geometrical Brownian motion is that it eliminates (thanks to the exponential) the negative values of the course which we could obtain with the standard Brownian motion of Bachelier.

	It is then interesting for the financial analyst to visualize the expected mean as a function of $T$ and the value $x(T)$ corresponding to a cumulative probability of $2.5\%$ and $97.5\%$ on a plot to have an idea of the evolution of the confidence interval at $95\%$ of its $x(0)$. This is very easy to get in a spreadsheet software like Microsoft Excel and we can get by playing with several types of plots in the same chart on something like this ($400$ MUSD portfolio with $5\%$ yield and $20\%$ standard deviation):
	\begin{figure}[H]
		\centering
		\includegraphics[scale=1]{img/economy/brownian_geometric_plot_excel.jpg}
		\caption{Geometric Brownian motion plot with $95\%$ C.I. in Microsoft Excel  11.8346}
	\end{figure}
	With the following structure in Microsoft Excel:
	\begin{figure}[H]
		\centering
		\includegraphics[scale=0.7]{img/economy/brownian_geometric_with_drift_plot_data_excel.jpg}
		\caption{Data table corresponding to the previous plot in Microsoft Excel 11.8346}
	\end{figure}
	with the explicit corresponding formulas to Microsoft Excel 11.8346 (don't hesitate to zoom in to see better):
	\begin{figure}[H]
		\centering
		\includegraphics[scale=0.45]{img/economy/brownian_geometric_with_drift_plot_data_excel_explicit_formulas.jpg}
		\caption{Data table corresponding to the previous plot in Microsoft Excel 11.8346 with explicit formulas}
	\end{figure}
	It is interesting to compare the evolution of this portfolio with the same parameters (volatility and yield) over the same period of time but with the standard brownian motion. This graphically gives:
	\begin{figure}[H]
		\centering
		\includegraphics[scale=1]{img/economy/brownian_arithmetic_geometric_comparison_plot_excel.jpg}
		\caption{Comparison of geometric and standard brownian motion}
	\end{figure}
	Finally, the reader will notice that we can generalize the writing of the two Brownian process (taking the logarithm of Neperian of the Brownian geometric process):
	
	by writing them in a form proposed by Benoît Mandelbrot in 1962:
	
	where $\delta$ and $c$ are respectively the localization parameters (average return) and dispersion (non-Gaussian volatility) of the process, and where $L_t^\alpha$ designates the standard $\alpha$-stable Lévy process.

	The problem with this model is the loss of existence, for certain laws of probability that work very well, of the second moment (the variance) so important in terms of communication and images for professionals in the 1970s because it served them as the only measure of risk. The absence of finite variance was probably one of the most powerful causes of rejection (brain inertia not of customers but of board of directors...).
	
	\pagebreak
	\subparagraph{Black \& Scholes Equation}\mbox{}\\\\
	We have obtained during the preceding developments the following differential equation for the random walk of the value of an asset under the constraint of a log-normal law and a Brownian motion:
	
	Either with the good notations in finance:
	
	If we now construct a neutral (not risky for recall) delta portfolio consisting of one option in short position and a number $-\Delta$ of underlying securities (often also denoted $-\Delta_t$ in the literature) in a long position (or conversely it does not matter).

	The value of the portfolio $\Pi$ is then expressed by:
	
	The idea of the equality above is the same as when we have a portfolio with a cash amount $S$ of a security (ie the total cash amount of the securities in the securities portfolio) and we protect ourselves, for example, from a downward trend with the acquisition of a number of Put but with the difference that here the total cash amount of options that is set at $F$ (ie the total cash amount of the options in the options portfolio) and that we look for the number of underlying to buy (hence the negative sign) in the case of a supposed bearish trend.

	The time differential of the portfolio is then written (which assumes that we make instantaneous delta hedging without costs, what in practice is quite difficult to achieve...!):
	
	You will notice that we assume constant the quantity $\Delta$ during the time differential...
	
	\begin{tcolorbox}[title=Remark,colframe=black,arc=10pt]
	Caution!!! Some authors choose, by convention, to invert the signs of the terms that are on the right of the equality of the two preceding relations. This does not change the final result of the solution of the differential equation but it must be known that this can be done since it is only a protection choice of bullish or bearish trend.
	\end{tcolorbox}
	By combining the previous relations and (here we adopt the traditional notation used in the finance field) the equation of the risky asset given by:
	
	we get:
	
	where we have in the right-hand square brackets the geometrical Brownian motion.

	This gives, after rearrangement of terms, the differential equation of the portfolio:
	
	Consider now that $\Delta$ is bound by the speculative dependence relation (of which we take the integer value) which eliminates from the previous relation the randomly risky part of the portfolio (it is the $\mathrm{d}z$ generates the risk randomly for recall!) such that (we will see again this expression of the delta during our study of the "Greeks" further below):
	
	We can then write the infinitesimal variation without risk of our portfolio:
	
	But, we also have for a portfolio placed in a risk-free investment (since we have just eliminated the random risk we can write it like this):
	
	That is sometimes also written in the literature:
	
	The attentive reader will have noticed that the two relations above:
	
	Are therefore a situation of non-arbitrage.
	
	Now substituting the four relations:
	
	into:
	
	We get:
	
	which is nothing else than the "\NewTerm{Black \& Scholes linear parabolic partial differential equation (without second member) of the second order}\index{Black \& Scholes differential equation}" (with non-constant coefficient $S$). This relation is therefore valid only under the assumption that the return $r$ is the risk-free one on the market and therefore that the Brownian motion is a Brownian motion at neutral risk! Similarly, this last relation also assumes that the variance is independent of time!

	The fact that the drift $\mu$ is equal to the risk-free return $r$ also has a significant impact on the interpretation of this differential equation. This indeed assumes indirectly that two options with different drifts have their implied return which is equal and that is taken as the one without risk of the market !!! This is the reason why in finance if we have two options whose drift differs and the variance and maturity are identical, the Black \& Scholes equation gives the same pricing by construction !!!

	This last equation is more often written in the form relating to the implicit reference of a Call such as:
	
	Or in a slightly more condensed form (depending on the textbooks...):
	
	and even more condensed... (but here it becomes frankly an abused notation...):
	
	The reader will have noticed that the parameter $\mu$ (derivation) is absent from this equation! In other words, the value of an option is independent of the rate of change in the values of the underlying securities! The only parameter that affects the price of the option is the volatility $\sigma$ of the underlying. A consequence of this is that two people with divergent opinions about the value of $\mu$ are still in agreement on the value of the option.

	The objective is obviously to solve this differential equation in order to determine the return $F(S, t)$. This can not be solved in two lines...

	Before working on this hard painful task some definitions and practical preconditions concerning certain parameters are useful and necessary (we will determine their explicit form after the resolution of the differential equation):
	
	

	\pagebreak
	\subparagraph{Self-financing portfolio on underlying}\mbox{}\\\\
	A "\NewTerm{self-financing portfolio}\index{self-financing portfolio}" strategy is a dynamic strategy of buying or selling securities and loans or borrowing from the bank, the value of which is not modified by the addition or withdrawal of cash (we could have introduced this subject at the beginning of this section but we thought it more appropriate to do so only now). In other words: a portfolio is self-financing if there is no exogenous infusion or withdrawal of money; the purchase of a new asset must be financed by the sale of an old one.

	We will assume here for the example that we can only invest in a single security (risky investment), and in cash (supposedly non-risky investment), that is to say for that latter by placing or borrowing money at a bank.

	We denote by $S_t$ the price on the date $t$ of the security, by $r_t$ the interest rate for an investment between $[t,t+\mathrm{d}t]$ at the bank.

	Given $V_t$ the market value, or the net asset value, or even the "Mark to Market value (MtM)" of the portfolio at date $t$. After renegotiation, the number of shares $\delta_t$ in the portfolio is constant until the next management expiry date. For the sake of simplicity, we assume for the moment that the manager takes into account the value of the underlying price in his decision rule only at the time of renegotiation.

	In a very short time, the change in value of the portfolio is due only to the change in the value of the underlying and to the interest paid by the bank on the cash, ie, since the amount invested in the cash is:
	
	we have the "\NewTerm{self-financing equation}\index{self-financing equation}":
	
	Thus, for a Call seller (for example ...), the purpose is to find the initial cost $V_0$ and strategy $\delta_t$ that make it possible to get (the finance practitioner talk about "realizing the financial asset"):
	
	in all market scenarios. If there is such a hedging strategy, then we say that we are dealing with a "\NewTerm{complete market}\index{complete market}".
	
	\pagebreak
	\subparagraph{Greeks and others...}\mbox{}\\\\
	In the reality these theoretical models are very beautiful to read (for the honor of the human spirit) but quite unrealistic. Reason why traders use (as in engineering) simple and comprehensible indicators named the "Greek" to make their sales or purchase decisions (or to make the information understable by the Director Board...). However, with a little creativity, the same indicators can be used in many other fields (manufacturing, project management, chemistry, etc.).

	Here is the list of the best known one (we will calculate their explicit expression for some models that we will develop further below):

	\textbf{Definitions (\#\mydef):}
	\begin{itemize}
		\item[D1.] The "\NewTerm{delta}\index{delta}" of an option, which is important to understand (or know), is given by:
		
		and represents the rate of change in the value of the portfolio's options, depending on the values of the underlying securities $S$ (mathematically, therefore, it is the first derivative of the option premium on the price of the underlying). This term is fundamental in theory (it is a hypothesis in the construction of the Black \& Scholes model as we saw above) and in practice and we will use it frequently. It is therefore a measure of the correlation between the movement of the option (or other financial assets and derivatives) and the underlying assets.
		\begin{tcolorbox}[colframe=black,colback=white,sharp corners]
		\textbf{{\Large \ding{45}}Example:}\\\\
		Let us consider for example, that a Call on the $ABC$ share has a delta of $0.25$ with a price of the underlying (spot) of $90$.- and a premium at $5$.-. This mean that when the price of the share $ABC$ changes from $90$.- to $91$.-, the premium of the option will then increase by $1$ delta, and then becomes $5.25.-$. When the price of the $ABC$ share goes from $90$.- to $88$.-, the premium of the option will decrease by $2$ times delta, and becomes $4.50.-$. This variation in terms of delta (integer) is then denoted $\delta_t$.\\
		
		\begin{tcolorbox}[title=Remark,colframe=black,arc=10pt]
		Many practitioners prefer to use the "\NewTerm{delta cash}\index{delta cash}" defined by the product of the delta and the spot price. So in our example above the delta cash is worth $90$ times $0.25$ which is worth $22.5$.-. If the share price increases from $90$.- to $91$.- ie $1.09\%$ increase, the variation of the Call is the product of the delta cash by the variation in $\%$ ie $22.5$.- times $1.09\%$ which gives a positive variation of $0.25$.- and then we fall back on our $5 + 0.25 = 5.25$.-.
		\end{tcolorbox}
		\end{tcolorbox}
		The delta is therefore an important parameter for a practitioner who wants to cover himself against the risk. Indeed, in order to obtain the global delta of a position, it is sufficient to multiply the value of the delta of each option by its position. Then we add up all these deltas.

		For example, if on the same underlying we are a seller of $5$ calls $C_1$ and a buyer of $7$ Call $C_2$ then our global delta will be equal to:
		
		The value of this parameter informs us about the quantity of underlyings to be bought or sold (opposite position on the underlying) in order to immunize the valuation of our portfolio to changes in the price of this underlying. We then say that it is a "\NewTerm{delta-neutral strategy}\index{delta-neutral strategy}" or a "\NewTerm{delta-hedging strategy}\index{delta-hedging strategy}". More precisely, it is a dynamic strategy (dynamic hedging\index{dynamice hedging}) because it is necessary to maintain a delta-neutral portfolio throughout the life of the option (in theory instantly for recall!) Contrary to statistical strategies (static hedging\index{static hedging}). For example, a static strategy to hedge the sale of a Call of strike $125$.- out of $1,000$ shares (spot price $100$.-), may consist of buying immediately $1,000$ shares on the market for $100,000$.-, and wait for the due date. In this situation, if the buyer exercises the Call the seller is able to deliver the $1,000$ shares for $125,000$.-. But if he does not exercise it, the seller stays with $1,000$ shares... Similarly, if the final price is set at eg $80$.-, the seller records a loss of $20,000$.-. Delta-hedging eliminates this risk.
		
		However, the strategy of dynamic hedging has a vicious side because if the majority of the market players buy underlying to hedge against an increase in the prices, then this will increase the value of the underlying even more Supply and Demand rule...) and thus they will buy even more underlyings, which will have the effect of increasing even more its value and so on and then we go in circles until reaching a potential Crash of the markets (the 1987 crash would be originally due to this type of perverse dynamics, same for some automated trading bots). The effect will be the same (but opposite) to protect against a decrease... It is for this reason that it is strongly recommended to make delta hedging with other diversified financial assets that the underlying itself!!!!!
		
		Thus, the managers, will between the date on which they received the premium (by having sold an option contract) and its maturity $T$, naturally manage in delta-neutral over time a self-funded portfolio made up of $\delta_t$ underlying assets $S$ (or equivalent replicated one) at every instant $t$, in order to dispose of the stochastic target at certainty (thus without risk) at maturity. We are also talking about "\NewTerm{portfolio hedging}\index{portfolio hedging}". However this strategy masks a terrible fault at the global level, indeed if a market player applies this strategy by selling massively (for example to protect itself from a bearish trend), then all the other actors will also perhaps sell then increasing by extension the delta and thus amplifying the volatility and so on until the crash.
		
		\item[D2.] The "\NewTerm{theta}\index{theta}" of an option gives the sensitivity of the option price to its maturity and is given by:
		
		Applied to our imaginary portfolio, the theta simply gives us the value lost or gained after one day passed for example.
		
		\pagebreak
		\item[D3.] The "\NewTerm{rho}\index{rho}" calculates the sensitivity of the option price to the geometric interest market rate without risk and is given by:
		
		This indicator seems to be relatively little used by professionals.
		
		\item[D4.] The "\NewTerm{vega}\index{vega}", represented by the lowercase greek letter nu because the name vega is not itself a Greek letter name...., measures the sensitivity of the option to volatility and is given by:
		
		The "\NewTerm{gamma}\index{gamma}" corresponds to the derivative of the delta and is therefore given by:
		
		A possible reading of the gamma is the direction of evolution of the delta as a function of the price of the underlying. A positive gamma indicates that price of the underlying and the delta move in the same direction, whereas a negative gamma shows the opposite.

		In order to avoid adjusting the Delta in delta-neutral strategy, and thus to pay the transaction costs but also the inherent spreads, a good way is to have the delta as stable as possible. Having the most stable delta means little or no change in its initial value, thus having a "\NewTerm{Gamma-Neutral strategy}\index{Gamma Neutral strategy}" in addition to having a delta-neutral strategy. However, since the delta-neutral strategy means that we already play with the underlying to make the delta constant, we can not use the same underlying to make the gamma neutral. The gamma hedging principle of a portfolio containing at least one option is then to find another option on the market in proportions that compensate for the gamma of the original position.
		\begin{tcolorbox}[title=Remark,colframe=black,arc=10pt]
		As we will see later during ou study of risk analysis in Finance using Taylor series developments, the delta does in fact allow for hedging only in the first order (that is, for small variations of the underlying). By supplementing with gamma-hedging we eliminate the effects of the second order. So a good strategy is to ensure the instantaneous values of delta and gamma to be zero.
		\end{tcolorbox}
		\begin{tcolorbox}[colframe=black,colback=white,sharp corners]
		\textbf{{\Large \ding{45}}Example:}\\\\
		Let us consider as summary of all these Greeks a Call with an underlying with a spot price of $100$.-, a strike of $100$.-, a volatility return of $24\%$ and an geometric average market interest rate without risk of $5\%$. Consider that we have the following values:
		
		The value of the delta thus tells us that if the price of the underlying increase of $1$.-, the value of the option will increase by $0.62$.-. The gamma tells us that if the underlying price increase of $1$.-, the delta of the option will go from $0.62$.- to $0.64.-$. The value of the theta tells us at the end of the day that the value of the option will have decreased by $0.02$.-. The value of the vega tells us that if the volatility of the underlying increases from $24\%$ to $25\%$, the value of the option will increase by $0.37$. Finally, the value of the rho indicates that if the value of the risk-free geometric average market rate increases from $5\%$ to $5.5\%$, the value of the option will increase by $0.25$.-.
		\end{tcolorbox}
		
		\item[D5.] The "\NewTerm{Black \& Scholes linear differential operator}\index{Black \& Scholes linear differential operator}" $L_\text{BS}$ is given by:
		
		seems to have a financial interpretation as a measure of the difference between the return of an option (the first two terms) and a set of a risk-free return portfolio containing that option (the last two terms).
	\end{itemize}
	In short, this being said, we can therefore have the following technical writing of the Black \& Scholes PDE (with non-constant coefficient $S$) using the Greeks:
	
	Here is a quite good visual summary of all Greeks for a special case:
	\begin{figure}[H]
		\centering
		\includegraphics[scale=0.45]{img/economy/greeks_summary.jpg}
		\caption{Greeks summary}
	\end{figure}
	
	\pagebreak
	\subparagraph{Solving the Black \& Scholes equation}\label{solving black and scholes}\mbox{}\\\\
	Before solving the Black \& Scholes PDE let us first already give the solutions with a reminder of the terms (this will allow to have a preliminary idea of the concepts used during the developments and I do not risk to write these before some years due to lack of time so it's better to have them already while waiting...):
	
	Let $F(S, t)$ denote the value of a Call $C(S, t)$ or Put $P(S, t)$ option, $\sigma$ the volatility of the underlying, E the strike, T the date Of expiration and r the interest and S the price of the underlying. We will solve the differential equation of BS (with non-constant coefficient S) written the following form:
	
	The trick (which led to a Nobel Prize in Economics for recall) is to see that it is a differential equation with partial differentials and more particularly with differentials of the first order in $t$ and second order in $S$. There exists a differential equation of this type which is very well known in physics: the heat equation (\SeeChapter{see section Thermodynamics page \pageref{heat equation}}). The idea is then by changes of variables to get back to it. The difficulty was to find the right changes of variable and our predecessors took care of this error and trials... for us (thanks to them!). We then start by put (this choice also allows to shift to initial conditions similar to those of the equation of heat):
	
	and therefore explicitly:
	
	We will use further below the fact that:
	
	For practical purposes, let us write the inverse transformation (which will be useful to us much later) that follows:
	
	A part of the underlying idea of this choice of change of variables is to transform the fact that in the differential equation of Black \& Scholes we know the final conditions (since at maturity the price of the option is perfectly Determined) whereas with the heat equation we need initial and not final conditions! We will see immediately that this choice of change of variables allows this. Indeed, remember that at maturity we know that:
	
	Therefore:
	
	As:
	
	We then have:
	
	And the fact what allows us to have an initial condition is that:
	
	To transform our differential equation to partial derivatives (with non-constant coefficient $S$):
	
	With these new variables we have to calculate some partial derivatives of $C$. We then have:
	
	and:
	
	And also that:
	
	By injecting this into the differential equation, it comes:
	
	We simplify the $S$:
	
	We multiply on both sides by:
	
	To get:
	
	Therefore:
	
	Let us put as it is usual:
	
	Then there comes the differential equation (with constant coefficient this time!):
	
	With for recall the initial condition:
	
	\begin{tcolorbox}[title=Remark,colframe=black,arc=10pt]
	In the prior-previous relation, the sum $\dfrac{\partial u}{\partial \tau}+\left(\dfrac{1}{2}-\kappa\right)\dfrac{\partial u}{\partial x}$ is sometimes named the "\NewTerm{advection part}", $ku$ the "\NewTerm{source term}" and $\dfrac{1}{2}\dfrac{\partial^2 u}{\partial x^2}$ the "\NewTerm{diffusion term}", all this in analogy to fluid mechanics and thermodynamics.
	\end{tcolorbox}
	What follows also very clever and tricky  consists in making another change of variable by putting:
	
	We will determine the two parameters $\alpha,\beta$ which allow to write the partial differential equation with constant coefficients in the form of a heat equation!

	We shall proceed first as above by determining the equivalence of the different partial derivatives of $u$. We have then:
	
	and:	
	
	And also that:
	
	Then comes a trick very difficult to guess (at least a priori because I did not personally find something simpler ...). Indeed before substituting these last three relations in the P.D.E., Let us recall that:
	
	So the last three relations can be written (finding a mixture function of $u$ and $v$ is very counter-intuitive):
	
	and now only let us do the (partial!) substitution in the P.D.E.! We then move from:
	
	to:
	
	Let us write this last equality a little differently by putting factoring on the left and right what contains $u$:
	
	and then by putting:
	
	The differential equation is then simplified as:
	
	The exponential can be simplified and it remains:
	
	That can be rewritten in the form:
	
	And if we choose:
	
	We then have:
	
	Or otherwise written (in the physicist way of life...):
	
	We thus recognize the heat equation (or "diffusion equation") proved in the section Thermodynamics but where the coefficient of thermal expansion is equal to $1/2$.

	It follows from these developments that:
	
	Either by injecting the second equality into the first and after some elementary simplifications:
	
	So we see that finally:
	
	Now we have to transform the condition into $u$:
	
	In an initial condition in $v$ using:
	
	So it comes immediately:
	
	The remaining part (the resolution of the differential equation) will come in the near future ... except accident (i no have time with my job to continue to write it... very sorry!).
	\StickyNote[2.5cm]{\LARGE To finish translate and write before year 2020}[6.5cm]
	
	
	
	
	
	
	
	
	
	
	
	
	
	
	
	
	
	
	
	
	
	
	
	
	
	
	
	
	
	
	
	
	
	
	
	
	
	
	
	
	
	
	
	
	
	
	
	
	
	
	
	
	
	
	
	
	
	
	
	
	
	
	
	
	
	
	
	
	
	
	
	\begin{table}[H]
	\begin{center}
	\begin{tabular}{c|l}
	$u$ stuff on left	   &  $u$ stuff on right\\
	\hline 
	$-\alpha u$ & or  $\frac{1}{2}\beta^2u-\beta\left(\kappa-\frac{1}{2}\right)u-\kappa u$\\ 
	$-\alpha u$ & $\left(\frac{1}{2}\beta^2-\beta\left(\kappa-\frac{1}{2}\right)-\kappa\right)u$\\
	\end{tabular}
	\end{center}
	\end{table} 
	
	
	
	
	      
	
	
	
	
	
	
	
	
	
	\begin{enumerate}
	\item[(1)] $-\alpha=\left(\frac{1}{2}\beta^2-\beta\left(\kappa-\frac{1}{2}\right)-\kappa\right)$
	\item[(2)] $\beta =\left(\kappa-\frac{1}{2}\right)$
	\end{enumerate}
	
	
	
	
	
	
	
	
	 
	
	
	
	
	
	
	 
	
	
	
	
	
	
	
	
	
	
	
	
	
	
	
	
	
	
	
	
	
	
	
	
	
	
	
	
	
	
	
	
	
	
	
	
	
	
	
	
	
	
	
	
	
	
	
	
	
	
	
	
	
	
	
	
	
	
	
	
	
	
	
	
	
	
	
	
	
	
	
	
	
	
	
	
	
	
	
	
	
	
	
	
	
	
	
	
	
	
	
	
	
	
	
	
	
	
	
	
	
	
	
	
	
	
	
	
	
	
	
	
	
	
	
	
	
	
	
	
	
	
	
	
	So $\mathrm{d}2=\mathrm{d}1-\sigma\sqrt{(T-t)}$
	
	\begin{itemize}
		\item For the European Call (the value of the purchase option with maturity $T$ and strike $K$), the solution (whose demonstration still needs to be written in this section) is for a non-dividend-paying share-based underlying and for a geometric Brownian motion:
		
		where for recall (\SeeChapter{see section Statistics page \pageref{gauss distribution}}):
		
		The centered reduced Normal cumulative distribution function with:
		
		and:
		
		\begin{tcolorbox}[title=Remarks,colframe=black,arc=10pt]
		\textbf{R1.} If we have a Call or Put at the money (for reminder it means that $S = K$) then the two coefficients above simplify since the Neperian logarithm is then null.\\
		
		\textbf{R2.} Sometimes we write:
		
		where $M(t)$ is the moneyness of which we have already spoken several times.
		\end{tcolorbox}
		From the relation:
		
		We obviously deduce that a Call option can not have a value greater than the underlying (otherwise there would be an arbitrage opportunity to buy the underlying and sell the Call option - at least if we find a rather stupid buyer ... - and in extenso it would make us win every time).
		
		\item For the European Put (the value of the sell option of maturity $T$ and strike $K$) for a non-dividend-paying share-based underlying and for a geometric Brownian motion:
		
		From this last relation we conclude that a Put option can never have a value greater than the strike $K$ otherwise there would also be an arbitrage opportunity by selling the Put and buying the underlying.
	\end{itemize}
	If necessary for the curious reader here is an example of valuing a vanilla Call option on a Bloomberg terminal:
	\begin{figure}[H]
		\centering
		\includegraphics[scale=1]{img/economy/bloomberg_call_option_valuation.jpg}
		\caption{Valuation of a vanilla Call option on $1$ underlying of the S\&P 500}
	\end{figure}
	
	It is relatively easy to verify that the solutions above satisfy the Put-Call parity equation:
	
	where we recall that $K$ is the strike price of the two options $P$ and $C$.
	
	Now, as promised earlier, we can prove the explicit relations of the Greeks for the closed relations of the Calls (the principle being the same for the Put), besides the relations demonstrated above, notice that:
	
	Indeed:
	
	and also that:
	
	We have immediately:
	
	So now let's go!
	\begin{itemize}
		\item The Delta of the Call:
		
		Since the derivative is always positive, we can conclude that the price of Call only increases if the underlying increases in value (this is obviously the reverse for Put).
		
		The "delta of the Put" is given by (it should also be quite immediate):
		
		which is therefore always a negative value.
		
		\item The Gamma of the Call:
		
		
		\item The Vega of the Call:
		
		We thus find here a variation of the price of the option which is proportional to the square root of the elapsed time (the origin of this property being well known to us!).
		
		\item The Theta of the Call:
		
		
		\item The rho of the Call:
		
		So we can easily understand why the price of a Call is as high as $r$ is large (at least in the standard values of $r$ such that we can observe them in markets).
	\end{itemize}
	A common question from students (and practitioners) is to know whether the price of a Call increases or decreases depending on the distance at maturity. It is difficult to conclude in just a few seconds by deriving relatively to $T$ so it is better to go through computer tools and hence it is quite easy to verify that the intuition that the price of Call increases with time at maturity is correct. Moreover, we can also confirm another intuition that the price of Call tends towards the value of the strike, the more $T$ is distant.
	
	The same goes for the price of a Call depending on the strike. There it is easier to do it by hand:
	
	So the higher the price of the strike, the less the price of the Call will be.

	Here are the commands built into the Microsoft Excel 14.0.6129 to do the calculation with arbitrary input values:
	\begin{figure}[H]
		\centering
		\includegraphics[scale=0.8]{img/economy/black_scholes_pricing_excel_explicit_formulas.jpg}
		\caption{Application of Black \& Scholes pricing version of Microsoft Excel 14.0.6129}
	\end{figure}
	\begin{tcolorbox}[title=Remark,colframe=black,arc=10pt]
	It is clear that the Black \& Scholes have allowed the developments of options markets, allowing for more secure speculation. This remains speculation (the players speculate relative to each other on the volatility of equities), but this speculation remains almost secured by the coverage equation, which avoids that the losses are too large. There are nevertheless disadvantages to their use. The most important is surely the effect of runaway that they cause. For example, suppose that you are the seller of an option on the stock of a company $A$. Imaging that this company announces results slightly lower than expected. Its price falls, and that's normal. The Black \& Scholes coverage equation then recommends that you reduce the number of shares of that company in your portfolio, which you do (at least if you did the error of not replicate). But almost all market players will do the same reasoning, causing a further decline in the share price. The coverage equation of Black \& Scholes recommends that you sell more shares, etc. This can trigger a real runaway on the market, both downward and upward. This is accentuated by the fact that, often, purchase or sales orders are new automated, implemented directly in softwares, and no longer require human intervention. On the other hand, the coverage equation of Black \& Scholes is efficient for small price fluctuations, but not for brutal and important "unscrewing". Thus, just a year after receiving their Nobel Prize in Economics, Robert Merton and Myron Scholes were involved in the collapse of the US investment fund LTCM in autumn 1998, following the severe Russian crisis of the 1998 summer.
	\end{tcolorbox}
	In fact, option prices are not calculated using the Black \& Scholes formula (especially since the return is assumed to be constant ... hence the use of Monte-Carlo techniques and binomial trees that we will see further below). Most of the time, these prices are simply the result of the law of supply and demand, which prevails in most markets (for which reason, in markets, Calls for an underlying of a given spot and same strike and maturity do not have the same price!). When we then look at the non-reduced expression:
	
	It is therefore an equation with one unknown, since the only value then lacking in market practice is the volatility that we then name "\NewTerm{implied volatility}\index{implied volatility}" (as opposed to the volatility used when we value options and which we classically named "\NewTerm{empirical (observed) volatility}") and which should (...) in theory be equal in the markets for a Call and a Put having the same maturity and the same exercise price.

	In practice, the calculation of implied volatility is made, for a given underlying, for several $K$ and $T$ values at the spot and yield given by the market at present or in the future. In the end, instead of having only one point, we obtain a tablecloth like the one below:
	\begin{figure}[H]
		\centering
		\includegraphics[scale=1]{img/economy/black_sholes_typical_profile_implied_volatility.jpg}
		\caption{Typical profile of an implied volatility surface}
	\end{figure}
	where the "moneyness\index{moneyness}" represents the ratio between $S$ and $K$. If the Black \& Scholes equation were respected, we should have above a plan (so something completely flat)...
	\begin{figure}[H]
		\centering
		\includegraphics[scale=1]{img/economy/bloomberg_implied_volatility_surface.jpg}
		\caption{Typical profile of implied volatility in the Bloomberg terminal}
	\end{figure}
	On the markets, practitioners (especially traders) constantly look at the surfaces of implied volatilities (of which the interpretations can be multiple!). The concern is that these surfaces are not static: they change every day. The knowledge (in fact rather the approximation...) of the dynamics of the volatility sheets is therefore a crucial and subjectively complex subject (whole textbooks are dedicated only to this subject!)...

	A typical approach is to make assumptions about the behavior of the underlying. Sometimes there are closed formulas (as in the Black \& Scholes model), but often it is necessary to use numerical methods to obtain the price of the corresponding option and then to reverse the problem to find the local volatility.
	
	\textbf{Definitions (\#\mydef):}
	\begin{enumerate}
		\item[D1.] We name "\NewTerm{historical volatility}\index{historical volatility}" or "\NewTerm{statistical volatility}" or "\NewTerm{realized volatility}", the past volatility of a financial instrument underlying an option.

		\item[D2.] We name "\NewTerm{implied volatility}\index{implied volatility}" or "\NewTerm{Forward volatility}", the volatility injected into the Black \& Scholes model which is an estimate of the average of the future projection of volatility over a given period.

		\item[D3.] We name "\NewTerm{local volatility}" the volatility of an option calculated a posteriori option from the Dupire model (inversion of the Black \& Scholes relation) in a way.

		\item[D4.] We name "\NewTerm{observed volatility}" the true volatility of an option as observed in financial markets.
	\end{enumerate}
	Now let us talk about the case of the underlying that pay dividends! If the underlying whose spot price we inject into the option valuation equation is $S$ it is supposed as we know to be the present value estimated by the economic agents. If the underlying pays dividends then this price is actually not its actual face value since it implicitly contains the present values of the dividends! To obtain its true value, it is then necessary to subtract from $S$ the updated values of the dividends before injecting the value of the evaluation relations.
	
	\begin{tcolorbox}[colframe=black,colback=white,sharp corners]
	\textbf{{\Large \ding{45}}Example:}\\\\
	Let us consider a Call on $1$ year of a BMW share with a strike price of $40$.-. Suppose the current spot price of the BMW share is $35$.-, the instantaneous annual risk-free market interest rate is $5\%$, the volatility is $20\%$ per year, and there are two dividends during the period of one year of respectively $1$.- at two months and $0.5$.- to eight months.\\

	Therefore, the present value of the dividends is:
	\begin{gather*}
		1\cdot e^{-\dfrac{2}{12}5\%}+5\cdot e^{-\dfrac{8}{12}5\%}
	\end{gather*}
	and not $35$.-.
	And thus this deterministic amount is implicitly included in the current spot price of the market. Therefore, the spot value $S$ that we have to inject into the evaluation of the Call is:
	\begin{gather*}
		35-1.4753 = 33.5247
	\end{gather*}
	and not $35$.-.
	\end{tcolorbox}
	
	In the case where the underlying pays dividends which in constant continuous rate $q$ then its true face value discounted on the time $T$ will be given using the continuous rate (if necesssary review the construction of the concept of continuous rate!):
	
	By injecting this last relation into the evaluation of the options, we have for example for the Call:	
	
	It is quite obvious that this last equality is a generalization of the options on the underlying type of assets paying no dividends. We could, by the way, here again calculate the corresponding Greeks ...
	
	\pagebreak
	\paragraph{Binomial option pricing model (CRR model)}\mbox{}\\\\
	There is a more old style approach to the option pricing model that is not used anymore in practice in the simple form we will see here (CRR model) but that has inspired quite a lot of much more modern and complicated models that are still sometimes used today. 	
	
	We will see that the asymptotic limit of the binomial CRR model we will see now take us back to the Black \& Scholes formula. But before doing this let us recall that:
	\begin{enumerate}
		\item A portfolio is a "replicating portfolio" of an option if the portfolio and the option have exactly the same payoff
in each state of future.

		\item By using no arbitrage argument, the cost or price of the replication portfolio is the same as the value of the option.
	\end{enumerate}
	\begin{tcolorbox}[title=Remark,colframe=black,arc=10pt]
	The finance literature has revealed no fewer than $11$ alternative version of the binomial option pricing model for options on log-normally assets (CRR with or without drift, Jarrow-Rudd model, Tian model, Leisen-Reimer model, ...). These models are derived under a variety of assumption and in some cases require information that is ordinarily unnecessary to value options \cite{chance2007synthesis}. 
	\end{tcolorbox}
	The basic building element for the Black \& Scholes formula is the assumption that over one instant, the stock price can only move up or down (this is named a "\NewTerm{binomial process}"). So you must first understand how to work in such a world. Over two instants, the stock price can move up twice, move up once and move down once, or move down twice. Use the letter $u$ to describe the stock price multiplier when an up move occurs, and d to describe the stock price multiplier when a down move occurs. You can represent the stock price process with a binomial tree - where one branch represents a price-up movement and the other a price-down movement. For example if $d = 0.96$ (which means that on a down move, the stock price declines by $4\%$) and $u = 1.05$ (the stock price increases by $5\%$), the stock price is as follows:
	\begin{figure}[H]
		\centering
		\includegraphics[scale=0.5]{img/economy/binomial_tree_01.jpg}
	\end{figure}
	Note that at Instant $2$, the middle outcome occurs on two possible paths, while the two extreme outcomes occur only on one path each; $u\cdot d\cdot S_0$ can come about if there is one $u$ followed by one $d$, or if there is one $d$ followed by one $u$. This is already a statistical distribution that shares with a bell-shaped (Normal) distribution the feature that middle outcomes are more likely than extreme outcomes (with many bell shaped distributions and binomial trees we indeed end up with a continuous distribution that looks a lot like a bell-shaped curve).
	
	If we know that our underlying stock follows this binomial process, and we know the values of $u$ and $d$, can we price a Call option with a strike price of \$ $50$? On inspection of the tree, realize that the Call option pays \$ $0$ if the stock moves down twice, \$ $0.40$ if the stock price moves up once and down once (or vice versa), and \$ $5.125$ if the stock price moves up twice:
	\begin{figure}[H]
		\centering
		\includegraphics[scale=0.5]{img/economy/binomial_tree_02.jpg}
	\end{figure}
	As we know the purpose of the seller of the Call is to calculate the risk premium $C_0$ of the Call and this should be made such that  he can hedge the option risk. This should be made using arbitrage opportunity with an another investment on the market. So the risk premium should be calculated at each instant to what would happen (what will be the cost) if we hedge the Call option by buying buy immediately a portfolio $\Delta$ with a quantity $\delta$ of the underlying stocks and (assumed) a given quantity $b$ risk-free bonds.
	
	First let us place ourselves into the position where the stock price has moved down once already, that is, where the stock price stands at \$ $48.00$:
	\begin{figure}[H]
		\centering
		\includegraphics[scale=0.5]{img/economy/binomial_tree_03.jpg}
	\end{figure}
	So the equation is: what would be the minimum cost (and therefore the composition) of a replication portfolio such that we receive \$ $0$ if the underlying stocks of our Call moves down and if we receive \$ $0.40$ if the stocks moves up. That is assuming a risk-free rate of (for example) $0.1\%$ and that we know the underling is traded at \$ $48.00$ the following system of two linear equations with two unknowns:
	
	That is with our numerical values:
	
	The solution is given after elementary simplifications by:
	
	That is if we purchase a replication portfolio of $0.0926$ underlying (which costs $0.0926\cdot \$48\cong \$4.444$) and borrow $\$4.262$ for a net outlay of $4.444-4.262=\$0.182$, then in the next period, this replication portfolio will pay off $\$ 0$ in the downstate and $\$ 0.40$ in the upstate. Because this is exactly the same as the payoff on the Call option, the $C_1^d$ Call option should also be worth $\$ 0.182$.

	Now to continue we must consider still at Instant 1:
	\begin{figure}[H]
		\centering
		\includegraphics[scale=0.5]{img/economy/binomial_tree_04.jpg}
	\end{figure}
	Now we repeat the same reasoning where the stock price stands at $\$52.50$ and such that at the next instant we can end up with either $\$0.40$ in the downstate or $\$5.125$ in the upstate. In this case, we simply have to solve:
	
	and the solutions are:
	
	That is if we purchase a replication portfolio of $1.00$ underlying (which costs $1.00\cdot \$52.50= \$52.50$) and borrow $\$49.95$ for a net outlay of $52.50-49.95=\$2.550$, then in the next period, this replication portfolio will pay off $\$ 5.125$ in the upstate and $\$ 0.40$ in the downstate. Because this is exactly the same as the payoff on the Call option, the $C_1^u$ Call option should also be worth $\$2.550$.
	
	And finally:
	\begin{figure}[H]
		\centering
		\includegraphics[scale=0.5]{img/economy/binomial_tree_05.jpg}
	\end{figure}
	To determine the value of the call $C_0$ at the outset we have to find the price of the underlying that will be worth $\$0.182$ if the stock moves from $\$50$ to $\$48$, and worth $\$2.55$ if the stock moves from $\$50$ to $\$52.50$:
	
	and the solutions are:
	
	So the trader has to purchase $0.5262$ shares (cost today: $\$26.31$), and borrow $\$25.05$ dollars. The replication portfolio's total net outlay is then $\$26.31-\$25.05 \cong \$1.26$. Therefore, it follows that, by arbitrage, the price of the call option $C_0$ must be about $\$1.26$ today.
	
	In real life, the stock price can move many more times than just twice. You need a tree with many more levels, so you need to generalize this binomial process to  more levels. For example, if there are $10$ instants, what would be the worst possible outcome? Ten instant down movements mean that the stock price would be:
	
	The second-worst outcome would be one instant of up movement, and nine instants of down movement:
	
	Although the worst scenario can only occur if there are exactly $10$ down movements, there are $10$ different ways to fall into the second-worst scenario, ranging from $duuuuuuuuu$, $uduuuuuuuu$, ... , to $uuuuuuuuud$. This should bring to the memories of combinatorics (\SeeChapter{see section Probabilities page \pageref{simple combinations without repetitions}}). These are the $10$ possible combinations, better written as:
	
	Thus in general:
	
	Therefore, with $N$ levels in the tree, the stock price will be:
	
	in $\begin{pmatrix}N\\i\end{pmatrix}$ paths. 
	
	Now let us introduce probabilities! If we have a constant probability $p$ of going up, then we have a probability $(1-p)$ of going down then we can write obviously:
	
	that gives the expected value we can get after $N$ paths for $i$ given downside movements (and $N-i$ upside) with a probability $p$ whatever the order of the movements!
	
	We recognize here a well know pattern!!!! Something of the type:
	
	So we could think to a geometric distribution (\SeeChapter{see section Statistics page \pageref{geometric distribution}}) but this would be a mistake as all possible identical patterns interest us! So we must multiply that relation by the number of way we can get a same pattern. That is:
	
	that gives the expected \underline{total} we can get after $N$ paths for $i$ given downside movements (and $N-i$ upside) with a probability $p$ whatever the order of the movements!
	
	And if we want to take into account ALL possible movements, then we are just writing the expected mean and it comes:
	
	But... but... If we consider that we are considering that we are pricing a Call option, we are not interested to all path where $S_0<K$ (since the Call we not be used!). Then for this case we should write:
	
	But we must not forget that we are looking also to the price compared to a risk-free return investment. So we should write:
	
	This relation is know under the name "\NewTerm{Cox-Ross-Rubinstein model}\index{Cox-Ross-Rubinstein model}" (or simply "CRR model") of option pricing as first proposed in 1979 (almost $6$ years after the publication of Black \& Scholes model).
	
	Now to continue we will see why some companies prefer to use binomial pricing rather than Black \& Scholes model (as that latter overestimate prices). To fall back on the Black \& Scholes equations starting from CCR model we must first impose that:
	
	that is to say that the original stock price will be retained after an equal number of upward and downward price movements in any up-down sequences (quite strong assumption!).
	
	Now let $S*$ be the random price of the underlying at the end of the $N$ periods. For $n$ upward price movements over the $N$ periods since:
	
	we get:
	
	Here, $\ln(S^*/S)$, is the natural logarithm of one-plus-return for holding the stock over the $N$ periods;
it is equivalent to a continuously compounded return over the $N$ periods.

	For a binomial distribution with the probability of each upward movement being $p$, the expected value and the variance of $n$ are (\SeeChapter{see section Statistics page \pageref{binomial distribution}}):
	
	Then we have immediately involving:
	
	and:
	
	We can capture the above expressions more succinctly by defining (they are both the factors of the $N$):
	
	as being the fraction time expected and volatility instantaneous return.

	And using our previous assumption ($u=1/d$):
	
	For algebraic convenience that will soon be clear, let us put:
	
	where the parameter $b$ has yet to be determined. Accordingly, we can write:
	
	and:
	
	If we express $u$ as an exponential function of the form $u = e^h$ the both relations above reduce
to:
	
	Now we suppose also that the convergence of the Black \& Scholes and Cox-Ross-Rubinstein models requires that $\hat{\mu}N/T$ and $\hat{\sigma}N/T$ converge to the parameters $\mu$ and $\sigma^2$, respectively as $N$ approaches infinity!
	
	Now we must found a value for $b$ and $h$ such that when $N\rightarrow +\infty$ we get $\hat{\mu}=\mu$ and $\hat{\sigma}=\sigma$. There are various combinations for this but we must found one where none of the values diverge! One such proposition is after trial and errors:
	
	It follows that:
	
	Therefore we have indeed:
	

	It also follows that:
	
	and here we see that if $N\rightarrow +\infty$ we get:
	
	Therefore:
	
	Finally we have (what some authors put as a definition):
	
	and:
	
	that we name the "\NewTerm{physical probability}\index{physical probability (CRR binomial model)}" for obvious reasons we will see further below.
	
	Oufff!!!! This is done! Now let us come back to the CRR model:
	
	When the final stock price is great than the strike price, that is, when:
	
	Or:
	
	Since:
	
	The conditions becomes:
	
	or:
	
	The CRR model can therefore we written:
	
	where:
	
	For convenience, we put:
	
	So that:
	
	Let us consider first $U_2$ as it is well known (\SeeChapter{see section Statistics page \pageref{binomial distribution}}); a binomial distribution approaches a Normal distribution as the number of trials approaches infinity according to the central limit theorem (\SeeChapter{see section Statistics page \pageref{central limit theorem}}). Specifically, when there are $N$ trials and $p$ is the probability of success, the probability distribution of the number of successes is approximately Normal with mean $Np$ and standard deviation $\sqrt{Np(1-p)}$ (\SeeChapter{see section Statistics page \pageref{gauss distribution}}). The variable $U_2$ above is the probability of the number of success being more than $\alpha$. From the properties of the Normal distribution, it follows that, for large $N$:
	
	Substituting for $\alpha$, we get:
	
	From the following result we get previously:
	
	we see that as $N$ tends to infinity:
	
	and therefore:
	
	and here we see a big problem! The use of physical probability bring us to nothing we expected. We should be able to have an expression with $\mu$, $r_f$ and $\sigma$ as we know it with the Black \& Scholes model since these three parameters appear in the $\Phi$ of that latter.

	So the idea now is the following: The risk free return $r$ in a non-arbitrage assumption should be equivalent to the expected mean return of the asset. That means under this assumption we can write:
	
	which implies after rearrangement:
	
	denoted $p_r$ to make the difference with the "physical probability" introduced earlier before and named the "\NewTerm{risk-neutral probability}\index{risk-neutral probability}"!
	
	With a continuous rate this is immediately written:
	
	An equivalent way to introduce this probability is just tow write that in a non-arbitrage universe, the risk neutral expected value of $S$ equal to the binomial expected value of $S$:
	
	Solving for $p$ yields immediately to:
	
	So finally in the CRR universe:
	
	CRR recognize that the only condition required ton prevent arbitrage is the equivalent of this set of relations!
	
	So this done let us come back to:
	
	As:
	
	By expanding the exponential functions in a Maclaurin series (\SeeChapter{see section Sequences and Series page \pageref{usual maclaurin developments}}) to the first order:
	
	we therefore have:
	
	Therefore:
	
	As $N$ tends to infinity:
	
	And second order Taylor development would lead exactly to the same asymptotic value!
	
	For the remaining part to fall back on the Black \& Scholes model we need second order Maclaurin development. This is quite boring algebra so let us do it with Maple 4.00b:
	
	\texttt{
	>p:=(taylor(exp(x),x=0,3)-taylor(exp(-y),y=0,3))/(taylor(exp(y),y=0,3)\\-taylor(exp(-y),y=0,3));\\
	>x:=r*T/N;y:=-sigma*sqrt(T/N);\\
	>limit(sqrt(N)*(p-1/2),N=infinity);
	}
	
	and this gives:
	
	
	Therefore:
	
	Ok this is finish about $U_2$. We now move on to evaluate $U_1$ given for recall by:
	
	We put:
	
	It then follows that:
	
	and we can then write $U_1$ as following:
	
	Since:
	
	we have:
	
	This shows that $U_1$ also involves a binomial distribution where the probability on an up movement is $p^*$ rather than $p$. Approximating the binomial distribution with a Normal distribution, we get, similarly to $U_2$:
	
	and substituting for $\alpha$, gives as with $U_2$:
	
	Substituting for $u$ and $d$, and the risk-neutral probability in $p^*$ we get:
	
	By expanding the exponential functions in a Maclaurin series of order $2$ we see that, as $N$ tend to infinity that (we can detail the developments on readers requests):
	
	and:
	
	with the result that:
	
	So finally:
	
	where:
	
	and:
	
	So we see again now that for a risk-neutral option valuation, $p$ and $\mu$ are unnecessary, exactly as provided by the Black \& Scholes pricing model. The fact that the probability $p$ doesn't appear in this formula means that even if investors have different subjective probabilities for the up and down movements, they do agree on the relation between Call, $S$ and $r$. So the option's value doesn't depend on the investor's attitude toward risk.
	
	
	
	
	So for recall the assumptions (hypothesis) of that model are:
	\begin{enumerate}
		\item[H1.] We are in a non-arbitrage context with fixed probability $p_r$

		\item[H2.] The price of the asset today is $S_0$ and known

		\item[H3.] The assets do not pay dividends

		\item[H4.] The risk-free rate $r$ is positive and $d<r<u$
	\end{enumerate}
	However Binomial options pricing model \underline{approach} is widely used since it is able to handle a variety of conditions for which other models cannot easily be applied. This is largely because the binomial option pricing model is based on the description of an underlying instrument over a period of time rather than a single point. As a consequence, it is used to value American options that are exercisable at any time in a given interval as well as Bermudan options that are exercisable at specific instances of time. Being relatively simple, the model is readily implementable in computer software (including a spreadsheet).

	Although computationally slower than the Black \& Scholes formula, it is more accurate, particularly for longer-dated options on securities with dividend payments. For these reasons, various versions of the binomial model are widely used by practitioners in the options markets.
	
	\pagebreak
	\paragraph{Bachelier option pricing model}\mbox{}\\\\
	We want to illustrate now Louis Bachelier's efforts to obtain applicable formulas for option pricing in pre-computer time. Bachelier's model yields good short-time approximations of prices and volatilities and some banks still use that model in the beginning of the 21st century.

	So let us first recall that we get during our study of Bachelier's speculation theory that:
	
	
	Let us fix a strike price $K$, a horizon $T$ and consider the European Call $C$, whose pay-off at time $T$ is modeled by the random variable:
	
	Applying Bachelier's fundamental principle (zero mean and standard deviation proportional to the square root of time) we get for the price of the option at time $t=0$ considering the point of view of Bachelier that if we write the price of underlying ase $S_T=S_0+x$ then the Call price is therefore given by:
	
	and as according to Bachelier assumptions we have the amount of profit that follow:
	
	and that therefore the integration bounds are $K-S_0$ (indeed, this is the limit above which $S_0+x-K\geq 0$) to $+\infty$:
	
	as for recall (\SeeChapter{see section Statistics page \pageref{gauss distribution}}):
	
	Let us do now the change of variable:
	
	Therefore:
	
	Let now $u=-y^2$. It follows that $\mathrm{d}u=-2y\mathrm{d}y$. So then
	
	So:
	
	where $\phi(x)$ denotes for recall the density of the standard normal distribution.
	
	Finally:
	

	
	Interestingly, Bachelier explicitly wrote down the formula:
	
	but did not bother to spell out formula:
	
	The main reason seems to be that at his time option prices – at least in Paris – were quoted the other way around: while today the strike prices K is fixed and the option price fluctuates according to supply and demand, at Bachelier's times the option prices were fixed (at $10$, $20$ and $50$ Centimes for a "rente", i.e., a perpetual bond with par value of $100$ Francs) and therefore the strike prices $K$ fluctuated. What Bachelier really needed was the inverse version of the above relation between the option price $C_B^0$ and the strike
price $K$.
	
	\pagebreak
	\paragraph{Black Model (Future options pricing model)}\mbox{}\\\\
	The "\NewTerm{Black model}\index{Black model}" (sometimes known as the Black-76 model) is a variant of the Black–Scholes option pricing model. Its primary applications are for pricing options on Future contracts, bond options, Interest rate cap and floors, and swaptions. It was first presented in a paper written by Fischer Black in 1976.
	
	Black's model can be generalized into a class of models known as log-normal Forward models, also referred to as LIBOR market model.
	
	The Black formula is  therefore similar to the Black-Scholes formula for valuing stock options except that the spot price of the underlying is replaced by a discounted Futures price $F$.

	Let us recall that in the case of a Call on an underlying type of share we have its price at any time that is given by:
	
	While for a Future (corresponding to a Forward for the conditions we had demonstrated earlier above) we have:
	
	But we have proved in our very brief introduction to Forward and Future that:
	
	So since we solved the Black \& Scholes equation just before for $S_t$ it is enough to use the previous relation which gives us:
	
	Therefore, we have:
	
	So a call option on a future can be seen as a Call on an share underlying type that does not pay dividends but in quantity $e^{(r-y)(T-t)}$ and with strike instead $K^{-(r-y)(T-t)}$ instead of $K$.
	
	If we write the Call function of an underlying that does not pay dividends in the following way (the order of the arguments is chosen to be the same as in MATLAB™ and as most software does not manage other calculations than those based explicitly on $t=0$ and $y=0$):
	
	So in view of the above we see immediately that a Call on an underlying of type Future type can be obtained from a Call on an underlying of Share type by writing:

	This last implicit expression (and also the explicit version) constitutes what we named the "\NewTerm{Black model}" or also says "\NewTerm{Black-76 model}" for a Call (the idea for a Put is the same!).

	This equivalence can easily be checked for example with MATLAB™ (so you do not have to waste time with Microsoft Excel):
	\begin{figure}[H]
		\centering
		\includegraphics[scale=1]{img/economy/black_model_matlab.jpg}
		\caption{Black model with MATLAB™ 2016a}
	\end{figure}
	Now let us recall the explicit Black \& Scholes relation for a Call paying no dividends:
	
	Therefore the Black model is explicitly:
	
	
	\pagebreak
	\paragraph{VIX volatility index}\mbox{}\\\\
	VIX plays a very important role in the in financial derivatives pricing, trading, risk control strategy. It could be said it would not be a financial market without the financial market volatility. If it is the lack of risk management tools and the market volatility is too large, the investors may be worried about the risk and give up trading, then the market is less  attractive.
	
	VIX plays a more and more important role in the in financial derivatives pricing, trading, risk control strategy. After the global stock market crash in 1987, it is to stabilize the stock market and protect the investors, the New York Stock Exchange (NYSE) in 1990 introduced a circuit breaker mechanism (Circuit-breakers). When the stock price changes unusually, it occurs a temporary suspension of trading, and it is helpful to try to reduce the market volatility in order to restore the investor confidence on the stock market. However, due to the introduction of circuit breaker mechanism, there is many new insights for how to measure market volatility, and it is gradually produced a dynamic display of market volatility requirements. Therefore, not long after the New York Stock Exchange (NYSE) used Circuit-Breakers to solve the problem of excessive volatility in the market, the Chicago Board Options Exchange began to introduce the CBOE market volatility index (VIX) in 1993,which is used to measure market volatility implied by at the money S\&P 100 Index (OEX) option prices.

When the stock option transactions began in April 1973, Chicago Board Options Exchange(CBOE) has envisaged that the market volatility index can be constructed by the option price, which could be shown that the expectation of the future volatility in the option market. Since
then there were gradually many various calculation methods proposed by some scholars, Whaley (1993) proposed a calculation approach which is the preparation of market volatility index as a measure of future stock market price volatility. In the same year, Chicago Board Options Exchange (CBOE) started to research the compilation of the CBOE Volatility Index (VIX), which is based on the implied volatility of the S\&P 100 Index options, and at the same time also calculate implied volatility of call option and put option in order to take into account
the right of traders to buy or sell option under the preferences.

After ten years of development and improvement, the VIX index gradually was agreed by the stock market, CBOE calculated several other volatility indexes including, in 2001 NASDAQ 100 index as the underlying volatility index (NASDAQ Volatility Index, VXN), in 2003 the VIX index based on the S\&P 500 Index, which is much closer to the actual stock market than the S\&P 100 Index, CBOE DJIA volatility Index (VXD), CBOE Russell 2000 Volatility Index (RVX),in 2004 the first volatility futures (Volatility Index Futures) VIX Futures, and at the same year a second volatility commercialization futures, that is the variance futures (Variance Futures), subject to threemonth the S\&P 500 Index of realized Variance (Realized Variance). In 2006, the VIX Index options began to trade in the Chicago Board Options Exchange. In 2008, CBOE pioneered the used of the VIX methodology to estimate the expected volatility of some commodities and foreign currencies. There are many developments. For example, in India, VIX was launched in April, 2008 by National stock exchange (NSE). The VIX index of India is based on the Nifty 50 Index Option prices. The methodology of calculating the VIX index is same as that for CBOE VIX index. The current focus on the VIX is due to its inherent property of negative correlation with the underlying price index, and its usefulness for predicting the direction of the price index. And in HongKong, Hong Kong got its own volatility index for financial products that allow investors to hedge against excessive market movements. Hang Seng Indexes Company, the company which owns and manages the benchmark indexes in Hong Kong including the Hang Seng index (HSI), launched the HSI Volatility index or "VHSI" on Feb. 21. The index is modeled on the lines of the Chicago Board of Exchanges VIX index .VIX in that it measures the 30-calendar-day expected volatility of the Hang Seng index using prices of options traded on the index.
	
	
	
	
	
	
	
	
	
	
	
	
	
	
	
	
	
	
	
	
	
	
	
	
	
	
	
	
	
	
	
	
	
	
	
	
	
	
	
	
	
	
	
	
	
	
	
	
	
	
	
	
	
	
	
	
	
	
	
	
	
	
	
	
	
	
	
	
	
	
	
	
	
	
	
	
	
	
	
	
	
	\pagebreak
	\paragraph{Value at Risk}\label{value at risk}\mbox{}\\\\
	The risk measurements have evolved since Markowitz advanced his famous theory of portfolio diversification in the late 1950s, a theory that had to revolutionize modern portfolio management. The risk of a portfolio was then related to the matrix of covariances-variances, as we have demonstrated theoretically and by the examples above.

	In the 1960s, Sharpe proposed the unifactory model of valuation of financial assets where the beta is the primary explanatory factor of a portfolio's risk via the beta matrix.

	In the early 1990s, a new measure of risk was introduced (it seems that JP Morgan is at the origin of that one). Indeed, the limits of traditional risk measures were increasingly recognized. Quantitative indicators of the risk of value loss of assets had then to be found. To do this, indicator had to be found that were more closely related to the overall cash flow distribution of a portfolio. It is in this context that a nominal risk measure has been proposed: the "\NewTerm{Value at Risk (VaR)}\index{value at risk}" (denote sometimes "V@R" in project management). The idea is basically to say that we have $x\%$ cumulative probability of not losing more than a given amount $y$ in cash in the next $N$ periods of time (the period can be: days, months, etc.).
	
	This new measure was first used to quantify the market risk to which portfolios are subject. Indeed, the Basel Accords recommended banks since 1997 to hold a regulatory capital amount to preserve their stability against standard market risks. However, this capital has since been calculated from the VaR and has become increasingly popular for assessing the risk of institutional or individual portfolios (and not only for this!). However, there is no single measure of VaR. Indeed, it is based on the concept of volatility, which is essentially latent. This is why banks need to use several VaR models in order to define the range of their possible losses. These calculations are all the more complex since the distribution of the yields of the securities measured at high frequency is significantly different from the Normal distribution (therefore for long intervals of time we approach rather often a Normal distribution).
	
	\textbf{Definition (\#\mydef):} The "\NewTerm{Value at Risk (VaR)}\index{Value at Risk}" is the theoretical maximum loss that a portfolio manager (whose value is necessarily implicitly variable) can experience and for a certain period of time with a given cumulative probability (the use of VaR is not limited to financial instruments, it is used in many other areas of risk management in general).
	
	\begin{tcolorbox}[title=Remark,colframe=black,arc=10pt]
	The VaR is not really relevant if it is not presented with other risk indicators such as the Sharpe ratio, the Treynor ratio or the Greek coefficients (such as the beta). Finally, let us say that in practice the VaR is sometimes indicated in $\%$ of the total value of a portfolio.
	\end{tcolorbox}
	
	\pagebreak
	\subparagraph{Relative Value at Risk}\mbox{}\\\\
	In the classical model of the "\NewTerm{relative VaR}\index{relative VaR}\label{parametric VaR}" (also sometimes named "\NewTerm{parametric VaR}\index{parametric VaR}"), we will assume that the statistical distribution of the results of a portfolio obeys at every moment a Normal distribution... that we will write for what will follow:
	
	The next idea is that the random variable $X$ can therefore be rewritten with a random centered reduced variable Normal (\SeeChapter{see section Statistics page \pageref{normal reduced centered distribution}}) by putting:
	
	such that (use of the basic properties of the Normal distribution):
	
	and this writing is therefore used in many other fields besides finance (project management, quality assurance, supply chain, etc.).

	Let $\alpha$ be the critical threshold (in quantiles) associated with the cumulative probability targeted. We can then write:
	
	which is an interesting notation because it deferred the analysis of risk and variability to the estimate of the standard deviation only (what the finance practitioners appreciate ...)! This notation is however a very common abuse of the following rigorous writing (which indicates more clearly that we use the quantile of $\alpha$):
	
	What would be written using the same notation as in the Statistics section when the distribution is Normal:
	
	This type of notation can easily be verified with the spreadsheet software Microsoft Excel 11.8346 for the skeptics ... Consider a portfolio $P$ with an annual standard deviation of $10\%$ (that will have to be expressed in cash) and assume that we have $1,000$.- in assets of this portfolio (on average). We then have the first year:
	\begin{center}
		\texttt{=NORMINV(99\%;1000;10\%*1000)=1000+NORMSINV(99\%)*10\%*1000
=1000+2.326*10\%*1000 =1,232.6}
	\end{center}
	That is to say $99\%$ of cumulative probability of having a portfolio that has a value between $0$ and $1,232.6$.- at any time (we consider as negligible the cumulative probability that the portfolio has a negative value with this notation).

	But what interests the manager is not to hedge the risk of the mean (because it is zero by construction here!) but of the volatility! In the previous case its value is $100$.- ($10\%$ of $1,000$.-) and follows a reduced center Normal distribution. Hence the reason for defining the VaR formally as the mathematical relation which gives a confidence interval (or a quantile according to the point of view) of the standard deviation:
	
	Thus, for a cumulative probability of $99\%$ softwares gives us in absolute value (see the treatment of the confidence intervals in the section of Statistics):
	\begin{center}
		$\alpha$\texttt{=NORMSINV(99\%)=ABS(NORMSINV(1\%))=2.326}
	\end{center}
	where by tradition the finance practitioner take the $\alpha$ (and thus the VaR) as being positive. Hence the fact that they speak of risk hedged to $99\%$ (implicitly for a given period!) whereas in reality it is to cover again  risk that has $1\%$ cumulative probability of taking place (but strictly speaking it is the same thing just that the first one is easier to make understand a customer... !!!). This is why we sometimes find the VaR in the following form:
	
	It should be noticed that the result of the calculation may also be referred to as a "\NewTerm{fractional reserve}\index{fractional reserve}" (although this has nothing to do with the way in which States impose a fractional reserve to banks) or "\NewTerm{mathematical reserve}\index{mathematical reserve}" (although hist has nothing to do with the way in which insurance calculates the mathematical reserve).

	If this is not very clear, let us recall the following diagram seen in the section of Statistics:
	\begin{figure}[H]
		\centering
		\pgfplotsset{compat=1.7}
		\pgfmathdeclarefunction{gauss}{2}{\pgfmathparse{1/(#2*sqrt(2*pi))*exp(-((x-#1)^2)/(2*#2^2))}%
		}
		\begin{tikzpicture}
		\begin{axis}[no markers, domain=0:10, samples=100,
		axis lines*=left, xlabel=Standard deviations, ylabel=Frequency,,
		height=6cm, width=10cm,
		xtick={-3, -2, -1, 0, 1, 2, 3}, ytick=\empty,
		enlargelimits=false, clip=false, axis on top,
		grid = major]
		\addplot [fill=cyan!20, draw=none, domain=-3:3] {gauss(0,1)} \closedcycle;
		\addplot [fill=orange!20, draw=none, domain=-3:-2] {gauss(0,1)} \closedcycle;
		\addplot [fill=orange!20, draw=none, domain=2:3] {gauss(0,1)} \closedcycle;
		\addplot [fill=blue!20, draw=none, domain=-2:-1] {gauss(0,1)} \closedcycle;
		\addplot [fill=blue!20, draw=none, domain=1:2] {gauss(0,1)} \closedcycle;
		\addplot[] coordinates {(-1,0.4) (1,0.4)};
		\addplot[] coordinates {(-2,0.3) (2,0.3)};
		\addplot[] coordinates {(-3,0.2) (3,0.2)};
		\node[coordinate, pin={68.2\%}] at (axis cs: 0, 0.4){};
		\node[coordinate, pin={95\%}] at (axis cs: 0, 0.3){};
		\node[coordinate, pin={99.7\%}] at (axis cs: 0, 0.2){};
		\node[coordinate, pin={34.1\%}] at (axis cs: -0.5, 0){};
		\node[coordinate, pin={34.1\%}] at (axis cs: 0.5, 0){};
		\node[coordinate, pin={13.6\%}] at (axis cs: 1.5, 0){};
		\node[coordinate, pin={13.6\%}] at (axis cs: -1.5, 0){};
		\node[coordinate, pin={2.1\%}] at (axis cs: 2.5, 0){};
		\node[coordinate, pin={2.1\%}] at (axis cs: -2.5, 0){};
		\end{axis}
		\end{tikzpicture}
		\caption{Sigma intervals for the Normal distribution}
	\end{figure}
	which was presented into the following table for some selected values commonly used in the section Engineering (we will see some of these typical values in finance further below):
	\begin{table}[H]
		\centering
		\begin{tabular}{|c|c|c|}
		\hline
		\rowcolor[HTML]{C0C0C0} 
		\textbf{Sigma Quality Level} & \textbf{Left/Right Cumulated $\%$} & \textbf{One Sided Cumulative $\%$} \\ \hline
		$1\sigma$ & $68.26894$ & $68.26894/2=34.13447$ \\ \hline
		$2\sigma$ & $95.4499$ & $95.4499/2=47.72495$ \\ \hline
		$3\sigma$ & $99.73002$ & $99.73002/2=49.86501$ \\ \hline
		$4\sigma$ & $99.99366$ & $99.99366/2=49.99683$ \\ \hline
		$5\sigma$ & $99.999943$ & $99.999943/2=49.9999715$ \\ \hline
		$6\sigma$ & $99.9999998$ & $99.9999998/2=49.9999999$ \\ \hline
		\end{tabular}
	\end{table}
	An interesting information: If we cover the risk at $2\sigma$ for example on a daily basis (which a very poor coverage...) this means that in the one-sided case, we therefore assume about $5\%$ cumulative probability per day, that the adverse event occurs. In extenso if the undesirable events are independent then we can use the geometric law which we proved in the Statistics section that the expected mean of the first occurrence of the event (undesirable in the present case...) was equal to the inverse of the probability. So with the chosen range, we have the event in question that will appear on average about once every $20$ days ($1/5\%$).
	\begin{tcolorbox}[title=Remark,colframe=black,arc=10pt]
	The RiskMetrics model of J.P. Morgan/Reuters proposed in 1996 to take an $\alpha$ quantile of 1.65 corresponding to a cumulative probability of $95\%$. Some practitioners strongly recommend, in view of the strong and simplifying hypothesis of Normality ..., to multiply the VaR by a factor $2$ or $3$ (we will demonstrate why).
	\end{tcolorbox}
	\begin{tcolorbox}[colframe=black,colback=white,sharp corners]
	\textbf{{\Large \ding{45}}Example:}\\\\
	A portfolio $P$ of value $1,000$.- has an annual volatility $\sigma_A$ of $10\%$. The daily (instantaneous) volatility of the yield is then (we use here the property of the standard Brownian motion where, for recall, the returns are assumed to be independent from one day to another ...):
	
	where $252$ is the number of trading days in a given year for a given country. Either in cash:
	
	The VaR relative to the $99\%$ threshold at one day is then:
	
	Also over one year we would have the following relative annual VaR at the following $99\%$ threshold:
	
	Or a relative VaR of $10\%\cdot 2.326=23.26\%$ (just to give it in percentages as it is customary in the financial field).

	Thus, with respect to the relative annual VaR, we then have $99\%$ cumulative probability of winning $232.60$.- but also to lose them! Indeed we have $1\%$ cumulative probability of having an annual loss of:
	\begin{center}
		\texttt{=NORMSINV(1\%*10\%*1000)=-232.6}
	\end{center}
	So it would require at least a risk capital (equity) of $232.6$.- to cover $99\%$ of the risks (cover this cumulative probability of $1\%$ of being in a bad year respectively). We can also say that we have $99\%$ cumulative probability of not losing more that $232.6$.-. We thus find the same numerical result as with the previous example.

	The reader will notice that we thus have in the field of the Stock Exchange (this follows from the standard Brownian motion) to change from a daily time horizon to an annual one:
	
	\end{tcolorbox}
	Finance practitioners name this property of Brownian motion in the context of the use of VaR:
	
	the "\NewTerm{scaling law}\index{scaling law}". It is authorized by the Basel Accords in 1996 which presuppose a Normal distribution and advise a time horizon of $10$ to $30$ days. We have seen, however, in our proof of the standard Brownian motion model, that we underestimate the real risk under this hypothesis and that this reflex of change of scale via the square root is highly criticized by some specialists.
	
	Therefore the VaR, while being an amount putted in provisions for liabilities and charges in order to face a violent but temporary breakdown of the market, is not sufficient to protect an institution from a strong and sustainable return of the market. .. Reason why banks often provision the minimum minimorum...

	In the case where the returns are not independent but are described by an autoregressive process of order $1$ (see the Time Series study later in this section) such that:
	
	We then have considering the residual variance as negligible and that the yields have an equal variability from one day to the next (that is to say $\text{V}(R_t)=\text{V}(R_{t-1})$:
	
	We therefore have:
	
	If the correlation $\rho$ is zero we fall back on the square root time scale factor. It is interesting to notice also that in the general case (see the study of AR($1$) further below) where the return has a tendency to increase, the AR($1$)-VaR is always higher than the classic VaR.
	\begin{tcolorbox}[title=Remarks,colframe=black,arc=10pt]
	\textbf{R1.} Personally I will advocate covering up by the Six Sigma method at $99.9996\%$ over a time horizon corresponding to the at least to the median position time. But it's personal and very simplistic ...\\

	\textbf{R2.} There is a simplistic and traditional version of the VaR, named "\NewTerm{nonparametric VaR}\index{nonparametric VaR}", which consists in taking a given percentile of the distribution of historical or simulated losses. Thus, if we have at our disposal $N$ data of loss, the VaR parameter at the threshold of $99\%$ then consists in taking the $99$th percentile of the series of values. So nothing special to report at this level ...
	\end{tcolorbox}
	The goal now is to prove why the Basel Accords recommends to multiply the VaR by a factor of $3.03$.

	Remember first that in the section of Statistics a variant of writing of the inequality of Bienaymé-Chebyshev is:
	
	Let us now make the strong hypothesis that the underlying distribution is symmetrical. Hence, it is almost immediate that we are led to write:
	
	Suppose we impose to this deviation to the mean to have a cumulative probability of $1\%$. This is therefore equivalent to write:
	
	Therefore the deviation from the mean at the cumulative probability threshold of $1\%$ can then be written:
	
	Now, let us recall that in the case of a Normal distribution, we have just seen above that:
	
	There is therefore a ratio of:
	
	We thus fall back on the multiplier proposed in the first Basel Accords concerning to the best practices about the VaR. Obviously, it is quite justified that the reader wonders about big difference between the two approaches (almost $300\%$ !!!). It is simple to remember that the inequality of BT is an inequality valid for any symmetric distribution function and the value of $7.071$ is therefore the upper limit of the worst (max) for the general case whereas the value of $2.326$ is only for the Normal distribution which is a nice distribution (with for recall the probability which decreases exponentially whereas for some symmetric laws the decay is just of power type as it is the case for the Pareto distribution function).
	
	
	\subparagraph{Absolute Value at Risk}\mbox{}\\\\
	The VaR measure we have just given is a relative measure because it does not take into account the average of future losses and gains.

	If the volatility is $100$.- in the example just given, the relative VaR is therefore $232.6$.- But since the average profit is generally non-zero over a long period of time, we must use most of the time the "absolute measure of VaR" (over a very short period, profit being considered sometimes as zero, the calculation of the relative VaR is used).

	Let us first recall that following our study of the Bachelier model we have proved that the positive expected mean of the value (or yield) as well as the positive standard deviation of a portfolio are proportional to the square root of time.

	Suppose that the observation period $t$ is in months. The expected monthly return for the portfolio of initial value $S$ is then of $\mu$ (its expected mean therefore!!) and the monthly variance of its return is $\sigma^2$.

	Its relative VaR to the confidence threshold $\alpha$ is therefore after $t$ months given by (you can verify that the relation is indeed homogeneous!):
	
	as we were able to verify in the previous example (so far nothing new ...). The square root of time comes, as a reminder, from the Bachelier model (standard Brownian motion).

	In Monte-Carlo version with VBA (because the version above is named "punctual") this gives:
	\begin{figure}[H]
		\centering
		\includegraphics[scale=1]{img/economy/monte_carlo_relative_value_at_risk_vba.jpg}
	\end{figure}
	\begin{tcolorbox}[title=Remark,colframe=black,arc=10pt]
	Contrary to what we saw in our study of thresholds/confidence intervals in the section Statistics, we do not divide by $2$ the \texttt{NORMALSINV()} function argument of Microsoft Excel 11.8346 to get the $\alpha$ in the above situation because what interests us is only one side of the Normal reduced centered distribution (the "pessimistic" side) and not both.
	\end{tcolorbox}
	If we take the same example as before (portfolio of $1,000$.- with $10\%$ annual volatility). The relative VaR is therefore on a $30$-day projection of:
	
	But this last relation does not take into account the expected average return portfolio equation over time. The "\NewTerm{absolute VaR}\index{absolute value at risk}" is thus obtained by subtracting this yield from the relative VaR over the same time period, that is to say:
	
	where we make the particular assumption that the return is thus linearly time-dependent... (in accordance with the semi-empirical construction of standard Brownian motion). The absolute VaR is therefore obviously less than the relative VaR of this amount.

	It should be noted that the calculation of the absolute VaR may be considered vicious or of little interest because it assumes that the gain obtained from the return will be placed in the fractional reserve to finance the relative VaR. In most cases, however, the earnings will be re-affected in the Market.
	
	Let us come back to our usual example under this assumption (portfolio of $1,000$.- with $10\%$ annual volatility) with an annual return rate of $15\%$. We have then:
	
	Concretely, if we finance the VaR with the gains, then on one year it is enough to have $82.6$.- of fractional reserve. In practice it may be interesting to after how much time the earnings cover the entire VaR. In this case it is a simple equation of the second degree such that:
	
	and we would find in our example $2.4$ years. Concretely after $2.4$ years the gains will have covered all the risks according to the assumptions of the construction of the VaR...

	\subparagraph{Delta-Normal Value at Risk}\mbox{}\\\\
	We have just seen that when we work with an asset and the standard deviation is expressed in cash (or in $\%$ then converted into cash), the cash value of the relative VaR is therefore given under the assumption of Normality by:
	
	But what about the VaR of a derivative (option typically!) knowing that it depends indirectly on the value of the underlying and that only the latter is measurable on the market as a whole (in order to get a statistical distribution)? Well for this we will re-use the Taylor development used earlier in our study of Itô processes:
	
	Therefore, by taking only the first order, we have the following affine approximation (thus a coarse approximation valid only in the linear domain):
	
	and leaving aside the temporal variation (...) and taking an infinitely small variation (which, however, makes it possible to make the approximation a little more realistic):
	
	And by writing with the traditional notation of the underlying as used during our pricing studies of options and using the notation of the Greeks as defined earlier:

	Thus, the variation of a derivative assumed to depend only on the changes in the first-order of the underlying value is directly proportional to the marginal variation in the price of the option multiplied by the change in the underlying. Thus, if we assume the underlying price distribution of the underlying as Normal -  reason for which we speak of "\NewTerm{delta-Normal VaR}\index{delta-Normal Value at Risk}", then a coarse approach naturally consists in writing whether the volatility of the underlying is given in $\%$ :
	
	and if the volatility of the underlying is given directly in cash:
	

	\subparagraph{Historical Value at Risk}\mbox{}\\\\
	A third pragmatic by the way non-parametric way of calculating the relative VaR is based on historical data. This is the easiest way to do the calculation with the ease of use of existing spreadsheet softwares.

	Let us assume for the example that we have the last $100$ daily performances of a portfolio. The $10$ worst daily performances are given in ascending order:
	\begin{table}[H]
		\centering
		\begin{tabular}{|c|}
		\hline
		\rowcolor[HTML]{C0C0C0} 
		\multicolumn{1}{|l|}{\cellcolor[HTML]{C0C0C0}\textbf{Historical Data}} \\ \hline
		$-19,000$.- \\ \hline
		$-16,450$.- \\ \hline
		$-15,000$.- \\ \hline
		$-12,500$.- \\ \hline
		$-11,950$.- \\ \hline
		$-11.250$.- \\ \hline
		$-11,050$.- \\ \hline
		$-10,600$.- \\ \hline
		$-10,500$.- \\ \hline
		$-10,250$.- \\ \hline
		$\ldots$ \\ \hline
		\end{tabular}
	\end{table}
	The $95\%$ historical VaR for $1$ day then consist to determine the $5$th percentile. Since we have $100$ samples, it is easy to determine that this is the $5$th value in ascending order of values. Therefore in our example above:
	
	As already mentioned in the section Statistics, spreadsheet software use the \texttt{CENTILE( )} function which is not necessarily calculated in the same way from one software to another...
	
	\subparagraph{Credit Value at Risk (CVaR)}\mbox{}\\\\
	The "\NewTerm{Credit Value at Risk (CVaR)}\index{credit value at risk}" of a portfolio is the worst loss expected to be suffered due to counterparty default over a given period of time with a given probability. The abbreviation CVaR should not be confused with the Conditional Value at Risk  we will see further below!
	
	To introduce the credit VaR ludicrously let us consider a companion example of three portfolios $A$, $B$ and $C$, whose credit default probabilities follow a Bernoulli distribution (ie, a binary event) of respective values of $5\%$, $10\%$ and $20\%$ and the monetary values are in millions of dollars of $25$.-, $30$.- and $45$.-.

	For the following example, we will make the assumptions (which are often implicit even in all we have seen so far) that the risk exposure is constant, that the portfolios are independent and that in case of credit default we lose everything!!!

	As the three portfolios are therefore independent, the expected mean of loss (denoted "$L$" for Loss) is easily computable using the linearity property of the mean:
	
	Since the loss of credit is considered a Bernoulli random variable (and the portfolios are always independent), we always have in millions of dollars (the reader can refer to the calculation of the variance of a Bernoulli random variable in the section Statistics):
	
	Well this being done for pleasure ... How can we determine as finely as possible the distribution of losses to have the credit VaR at a given threshold? Well, for that we will have to construct the table of all possible outcomes (or build in the state of the art a Kernel Smoothing distribution function but manually this is quite boring to do).
	
	If there are $N$ credits, we have $2^N$ possible scenarios therefore:
	\begin{table}[H]
		\centering
		\begin{tabular}{|c|c|c|c|}
		\hline
		\rowcolor[HTML]{C0C0C0} 
		\textbf{Loss Scenario} & \textbf{\parbox{2cm}{\centering Associated\\ \centering Loss}} & \textbf{Probability} & \textbf{\parbox{2cm}{\centering Cumulated \\ Probability}} \\ \hline
		None & $0$.- & $(1-0.05)\cdot(1-0.1)\cdot(1-0.2)=68.40\%$ & $68.40\%$ \\ \hline
		$A$ & $25$.- & $0.05\cdot(1-0.1)\cdot(1-0.2)=3.60\%$ & $72.00\%$ \\ \hline
		$B$ & $30$.- & $(1-0.05)\cdot 0.1 \cdot(1-0.2)=7.60\%$ & $79.60\%$ \\ \hline
		$C$ & $45$.- & $(1-0.05)\cdot(1-0.1)\cdot 0.2=17.10\%$ & $96.70\%$ \\ \hline
		$A, B$ & $55$.- & $0.05\cdot 0.1 \cdot (1-0.2)=0.40\%$ & $97.10\%$ \\ \hline
		$A, C$ & $70$.- & $0.05\cdot (1-0.1)\cdot 0.2=0.90\%$ & $98.00\%$ \\ \hline
		$B, C$ & $75$.- & $(1-0.05)\cdot 0.1\cdot 0.2=1.90\%$ & $99.00\%$ \\ \hline
		$A, B, C$ & $100$.- & $0.05\cdot 0.1 \cdot 0.2=0.1\%$ & $100\%$ \\ \hline
		\end{tabular}
	\end{table}
	Thus, the credit VaR closest to $95\%$ is $45$ million USD. Since we can expect a loss of $13.25$ million USD, the unexpected portion will be considered as $31.75$ million USD (that is to say the difference).
	
	\pagebreak
	\subparagraph{Operational Value at Risk}\mbox{}\\\\
	We have already encountered the concept of convolution several times in the section Statistics but for continuous laws explicitly known in their algebraic form .... In practice it is quite different! Thus, actuaries have often at their disposition for a given period (often over a year), an experimental distribution of the frequency of events of interest (accidents, deaths or other) and an experimental distribution of the costs of events of interest. For example, consider the very simplified (in size!) following case:
	\begin{table}[H]
		\centering
		\begin{tabular}{|c|c|l|c|c|}
		\cline{1-2} \cline{4-5}
		\multicolumn{2}{|l|}{\cellcolor[HTML]{656565}{\color[HTML]{FFFFFF} \textbf{Frequency Distribution (losses)}}} &  & \multicolumn{2}{l|}{\cellcolor[HTML]{656565}{\color[HTML]{FFFFFF} \textbf{Distribution of Loss Costs}}} \\ \cline{1-2} \cline{4-5} 
		\cellcolor[HTML]{C0C0C0}\textbf{Probability} & \cellcolor[HTML]{C0C0C0}\textbf{Frequency} &  & \cellcolor[HTML]{C0C0C0}\textbf{Probability} & \cellcolor[HTML]{C0C0C0}\textbf{Frequency} \\ \cline{1-2} \cline{4-5} 
		$0.6$ & $0$ &  & $0.5$ & $1,000$.- \\ \cline{1-2} \cline{4-5} 
		$0.3$ & $1$ &  & $0.3$ & $10,000$.- \\ \cline{1-2} \cline{4-5} 
		$0.1$ & $2$ &  & $0.2$ & $100,000$.- \\ \cline{1-2} \cline{4-5} \hhline{|=|=|=|=|=|}
		\multicolumn{1}{|l|}{\cellcolor[HTML]{C0C0C0}\textbf{Expected mean:}} & $0.5$ &  & \cellcolor[HTML]{C0C0C0}\textbf{Expected mean:} & $23,500$.- \\ \cline{1-2} \cline{4-5} 
		\end{tabular}
	\end{table}
	If we want from these experimental and discrete values to determine the empirical distribution of costs, we will have to make a discrete convolution whose principle is the following:
	\begin{table}[H]
		\centering
		\begin{tabular}{|c|c|c|c|c|}
		\hline
		\rowcolor[HTML]{9B9B9B} 
		\textbf{Number of losses} & \textbf{First loss} & \textbf{Second loss} & \textbf{Total Loss} & \textbf{Probability} \\ \hline
		$0$ & $0$ & $0$ & $0$ & $=0$ \\ \hline
		$1$ & $1,000$.- & $0$ & $1,000$.- & $=0.3\cdot 0.6=0.150$ \\ \hline
		$1$ & $10,000$.- & $0$ & $10,000$.- & $=0.3\cdot 0.3=0.090$ \\ \hline
		$2$ & $100,000$.- & $0$ & $100,000$.- & $=0.3\cdot 0.2=0.060$ \\ \hline
		$2$ & $1,000$.- & $1,000$.- & $2,000$.- & $=0.1\cdot 0.5\cdot 0.5=0.025$ \\ \hline
		$2$ & $1,000$.- & $10,000$.- & $11,000$.- & $=0.1\cdot 0.5\cdot 0.3=0.015$ \\ \hline
		$2$ & $1,000$.- & $100,000$.- & $101,000$.- & $=0.1\cdot 0.5\cdot 0.2=0.010$ \\ \hline
		$2$ & $10,000$.- & $1,000$.- & $11,000$.- & $=0.1\cdot 0.3\cdot 0.5=0.015$ \\ \hline
		$2$ & $10,000$.- & $10,000$.- & $20,000$.- & $=0.1\cdot 0.3\cdot 0.3=0.009$ \\ \hline
		$2$ & $10,000$.- & $100,000$.- & $110,000$.- & $=0.1\cdot 0.3\cdot 0.2=0.006$ \\ \hline
		$2$ & $100,000$.- & $1,000$.- & $101,000$.- & $=0.1\cdot 0.2\cdot 0.5=0.010$ \\ \hline
		$2$ & $100,000$.- & $10,000$.- & $110,000$.- & $=0.1\cdot 0.2\cdot 0.3=0.006$ \\ \hline
		$2$ & $100,000$.- & $100,000$.- & $200,000$.- & $=0.1\cdot 0.2\cdot 0.2=0.004$ \\ \hline
		\end{tabular}
	\end{table}
	The loss expectated mean is then equal to $11,750$.-, which is also equivalent to simply calculate (the expectated mean of the product of two independent random variables being equal to the product of the expectated man as we have proved it in the section Statistics):
	
	Then we combine and sort the losses and probabilities:
	\begin{table}[H]
		\centering
		\begin{tabular}{|c|c|}
		\hline
		\rowcolor[HTML]{9B9B9B} 
		\textbf{Loss} & \textbf{Probability} \\ \hline
		$0$.- & $60\%$ \\ \hline
		$1,000$.- & $75\%$ \\ \hline
		$2,000$.- & $77.5\%$ \\ \hline
		$10,000$.- & $86.5\%$ \\ \hline
		$11,000$.- & $89.5\%$ \\ \hline
		$20,000$.- & $90.4\%$ \\ \hline
		$100,000$.- & $96.4\%$ \\ \hline
		$101,000$.- & $98.4\%$ \\ \hline
		$110,000$.- & $99.6\%$ \\ \hline
		$110,000$.- & $100\%$ \\ \hline
		\end{tabular}
	\end{table}
	On the side of the operational VaR, if we set it at $95\%$ (the Basel II agreements propose instead to take $99.9\%$), the nearest value is then:
	
	Therefore, the reserve of unexpected loss is:
	
	Graphically our tables, manipulations and calculations can be summarized as follows:
	\begin{figure}[H]
		\centering
		\includegraphics[scale=1]{img/economy/operational_var.jpg}
	\end{figure}
	It is interesting to notice how two simple tables produce a quantity of computations in convolution relatively huge. This is why, in real practical cases, the use of computers is necessary.
	
	\subparagraph{Surplus Value at Risk (Pension VaR)}\mbox{}\\\\
	Risks should be seen in a more general context in the case of a portfolio or an accounting system including asset management  and liabilities.

	Thus, in the case of a portfolio or an accounting, the risk is that the asset values no longer cover the liabilities of the portfolio.
	
	\textbf{Definition (\#\mydef):}  We name "\NewTerm{surplus}\index{surplus}", and denote it by $S$, the difference between the amount of assets $\mathcal{A}$ and liabilities $\mathcal{L}$ (liabilities). But in a totally financial context aspect we will focus mainly on the difference of the mean of the returns between the two such as:
	
	\begin{tcolorbox}[colframe=black,colback=white,sharp corners]
	\textbf{{\Large \ding{45}}Example:}\\\\
	Consider a cash portfolio with $\mathcal{A}_t=120$ and $\mathcal{L}=100$. Therefore:
	
	Suppose we have:
	
	Consider a cash portfolio with $\mathcal{A}_t=120$ and $\mathcal{L}=100$. Therefore:
	
	Suppose we have:
	
	With the correlation:
	
	Then in this case over the next period it is very likely to have in cash:
	
	So in extenso, we have:
	
	So the expected surplus for the next time period will be:
	\end{tcolorbox}
	\begin{tcolorbox}[colframe=black,colback=white,sharp corners]
	
	And for the variance of the surplus, we need it to calculate the SVaR, that is the "Surplus Value at Risk\index{surplus Value at Risk}", then it comes:
	
	Therefore:
	
	Now, let us recall that under certain hypotheses necessarily realized, we have demonstrated that the absolute VaR was given in the case of a Normal underlying distribution by:
	
	We have then in our example the surplus VaR at one unit time horizon unit and at the $95\%$ threshold:
	
	\end{tcolorbox}
	So generally the example above give us explicitly:
	

	\subparagraph{Variance-Covariance Value at Risk}\mbox{}\\\\
	The variance-covariance VaR  is based on a more realistic case of calculating the VaR for a portfolio composed of several financial assets correlated or not (unlike the previous cases where we had only one asset).

	In order to introduce this concept, let us consider a companion example of a portfolio $P_1$ of $5,000,000$.- and daily volatility of $2\%$ (ie $100,000$.- / day) and a second portfolio $P_2$ of $7,000,000$.- and daily volatility of $1\%$ (ie $70,000$.-/day).

	Our measurements show that their Pearson correlation coefficient $\rho$ is $0.6$. The daily global standard deviation is then (\SeeChapter{see section Statistics page \pageref{covariance}}):
	
	where as usual we see that if the correlation is negative this decreases the overall volatility.

	Thus, the relative variance-covariance daily VaR at $99\%$ for the global portfolio is (no scaling law to apply here since the standard deviation is daily and we want the daily relative VaR):
	
	It is interesting to compare the daily relative VaR with the sum of the relative VaRs of the two portfolios:
	
	We then have:
	
	This is due to the diversification gain!
	\begin{tcolorbox}[title=Remark,colframe=black,arc=10pt]
	In finance when the merging of risks is less than the sum of the risks, we then technically say that the risk measure is "\NewTerm{subadditive}".
	\end{tcolorbox}
	Many practitioners do the same calculation with yield rather than with cash values. Thus we get the VaR expressed in terms of yield.
	\begin{tcolorbox}[title=Remark,colframe=black,arc=10pt]
	A first trap in calculating the relative variance-covariance VaR  would have been to calculate the global standard deviation in $\%$ and then apply to it  the relation of the calculation of the global VaR. The result would have been wrong! A second trap is to use the same percentile of the VaR for merging risks as for individual risks because if we can accept individual ruins in a highly competitive market we can not accept it with the same level of risk For the same large structures (we should apply Bonferroni correction or Fisher $p$-value calculation as seen in the section Statistics).
	\end{tcolorbox}
	If we had a single portfolio consisting of several financial instruments such that the total proportion is equal to unity, then we should in accordance with what we saw in our study of the Markowitz model work with the following relation if we work with returns:
	
	Thus, if we consider the first portfolio in our companion example above in a proportion of $75\%$ (and therefore in extenso the second in a proportion of $25\%$). We then have:
	
	Thus a VaR return of (always under the assumption of Normality):
	
	That therefore represents in $\%$, the amount we could lose our portfolio.

	Therefore the general "\NewTerm{variance-covariance (relative) Value at Risk}\index{variance-covariance (relative) Value at Risk}", "\NewTerm{Portfolio Value at Risk}\index{portfolio Value at Risk}", also named,  relation is given by:
	
	Obviously (\SeeChapter{see section Statistics page \pageref{covariance}}) we could write it in matrix form if we want...
	
	\subparagraph{Marginal and Component Value at Risk}\mbox{}\\\\
	In our study of Markowitz's efficient diversification model, we showed that the variance of a portfolio composed of several investments in given proportions $X_i$ was:
	
	The variation of this variance with respect to a variation of one of the proportions $X_i$ alone gives:
	
	Let us recall the property of bilinearity of the covariance proved in the section of Statistics:
	
	By putting $Y=X=R_i$, $a=X_i$, we have:
	
	Let us put $b=1$, $Z=\sum_{j\neq i} X_jR_j$, we have:
	
	Therefore we get:
	
	So finally:
	
	But we also have:
	
	So we deduce:
	
	The reader will notice that we fall back on the beta coefficient. Therefore, we can write (on the way I prefer personally to write the weights according to the most usual notation $w_i$ instead of $X_i$):
	
	Thus, remembering that the relative VaR is given by:
	
	The "\NewTerm{marginal Value at Risk}\index{marginal Value at Risk}" is then logically given by:
	
	This gives for a given asset having a given weight $w_i$ in a given portfolio the percentage increase in its contribution to the portfolio's risk when increases its weight by $1\%$.
	
	We define the "\NewTerm{component Value at Risk}\index{component Value at Risk}" as:
	
	Let us recall for close this subject what is the purpose of the VaR? First of all, it is of great utility since it is measured in nominal terms. Once a financial institution has calculated its aggregate VaR, that is, the maximum loss that it can incur over the entire balance sheet for a predetermined probability, it is entitled to use that amount o determine the minimum capital (equity) it must hold in order not to become bankrupt. If it holds less capital and the maximum probability loss occurs, its equity will be negative and it may have to file for bankruptcy. Many studies have proven that this tool is statistically much better that any known qualitative approach encompassing all well know human cognitive bias!

	The VaR is therefore very useful for a financial institution because it enables it to determine the level of capital it must maintain in order to survive with a given probability. When VaR is used for this purpose, we refer to it more commonly as "\NewTerm{Capital at Risk}\index{Capital at Risk}", which means that the capital that a financial institution must hold is calculated or assessed according to the risks to which it is exposed . The greater the risk, the greater the need to maintain capital. This seems reasonable, because the capital held by a financial institution is first and foremost a safety net. For a bank, it aims to protect deposits in its liabilities. VaR is therefore an appropriate measure to define the regulatory capital that a financial institution must hold. For this reason, the Basel Committee, under the umbrella of the Bank for International Settlements, used this measure to calculate the regulatory capital of a depository institution in 1995 and became effective in January 1998. They now have to calculate their exposure Risk by using VaR and testing its accuracy by performing stress tests (as opposed to calculations to extreme variations) as well as to back testing by verifying that large deviations (outside the range of confidence) do not occur more than $5$ times per stock market year (thus compared to historical data).
	
	\subparagraph{Conditional Value at Risk (CVaR)}\mbox{}\\\\
	A forecast of financial losses often used in the academic world of finance is the "\NewTerm{conditional loss}\index{conditional loss}", often referred to "\NewTerm{expected shortfall}\index{expected shortfall}" or "\NewTerm{expected loss}\index{expected loss}" or "\NewTerm{expected tail loss}\index{expected tail loss}" (ETL).
	
	For the sake of simplification, we often limit ourselves to the calculation of the conditional loss of the VaR in the case that it follows a Normal law (we speak then in the field of finance of an "expected shortfall for Gaussian loss distribution") and this is what we are going to do here. The reader will find, however, a more realistic case based on a Pareto type I law in the Probability section when we introduced the concept of conditional expectation.

	Let us consider therefore the expected shortfall at a threshold $\alpha$ corresponding to the percentile such as we have seen earlier above. Consistent with what we have seen in the Probability section, the conditional expectation of such a threshold will therefore be given by:
	
	For the denominator, it is immediate (it is simply the cumulative probability at the level of the quantile threshold of the selected VaR):
	
	As in finance we can always center and reduce a random variable, let us focus on the case of a centered reduced Normal law to treat the numerator (hence $\text{VaR}_\alpha$ the quantile of the centered reduced VaR to the threshold $\alpha$). We then have:
	
	What is computed with a spreadsheet like Microsoft Excel 14.0.7184:
	\begin{center}
	\texttt{=-NORMDIST(NORMINV(alpha),FALSE)/(1-alpha)}
	\end{center}
	Or in an equivalent manner:
	\begin{center}
	\texttt{=NORMDIST(NORMINV(1-alpha),FALSE)/(1-alpha)}
	\end{center}
	\begin{figure}[H]
		\centering
		\includegraphics[width=\textwidth]{img/economy/cvar.jpg}
		\caption[Graphical representation of VaR, VaR Deviation, CVaR, CVaR Deviation, Max Loss, and Max Loss Deviation]{Graphical representation of VaR, VaR Deviation, CVaR, CVaR Deviation, Max Loss, and Max Loss Deviation (source: Value-at-Risk vs. Conditional Value-at-Risk in Risk Management and Optimization}
	\end{figure}
	There is a close correspondence between CVaR and VaR: with the same confidence level, VaR is a lower bound for CVaR. The problem of the choice between VaR and CVaR, especially in financial risk management, has been quite popular in academic literature. Reasons affecting the choice between VaR and CVaR are based on the differences in mathematical properties, stability of statistical estimation, simplicity of optimization procedures, acceptance by regulators, etc
	
	\begin{tcolorbox}[colframe=black,colback=white,sharp corners]
	\textbf{{\Large \ding{45}}Example:}\\\\
	Given a VaR following a Normal law that has been centered and reduced. The conditional loss expectation above $99\%$ (ie at the threshold of $\alpha=1\%$) is then given by:
	\begin{center}
	\texttt{=-NORM.S.DIST(NORM.S.INV(99\%),FALSE)/(1-99\%)=2.67}
	\end{center}
	or:
	\begin{center}
	\texttt{=-NORM.S.DIST(2.326,FALSE)/1\%=2.67}
	\end{center}
	Which is still $15\%$ more than the quantile of $2.326$ !!!\\
	
	This corresponds in comparison to the relative VaR to an equivalent of:
	\begin{center}
	\texttt{=1-NORM.S.INV(2.67;1)=99.6\%}
	\end{center}
	\end{tcolorbox}
	So through this example we see that there is no "problem" in working with a centered reduced conditional expected loss since we can always reduce the conditional expectation of a loss to a given relative VaR (that latter being already not accurate...).
	
	After the choice of reserves is purely from the internal policy to the financial institutes.
	
	\subparagraph{Back-Testing Value at Risk}\mbox{}\\\\
	We will see here a naive back-testing model of the VaR (there are many) but when the reader goes further through the predictive models (moving average, exponential smoothing, ARIMA, etc.) we will see that we will compare these models with the same values using error measurements (ME, MAD, MDS, MPE). These comparisons are then considered as a form of back-testing.

	So to go back to the VaR, it is relatively obvious that when we have just put in place an annual VaR model and we have little history it is better to bring it back to a daily VaR and do back-testing on the basis of a smaller unit.

	Let us recall the example of the beginning where we had a relative daily VaR and annual respectively at the threshold of $1\%$ (thus $99\%$ coverage) of:
	
	Suppose we do a back-testing of this model by comparing it to the $600$-day data we have in our possession. We observe $9$ values above $14.65$.-. What can we conclude ???

	In fact we must first know that if the $N$ days are independent and we denote by $p$ the probability that one of the days is out of VaR (thus the associated value equal to $1\%$ in our example!), then we are dealing to an experiment of $N$ Bernoulli's tests. It is therefore a Binomial law whose associated cumulative probability for recall (\SeeChapter{see section Statistics page \pageref{binomial distribution}}):
	
	Hence, the expected mean is given by (it's nice to calculate it but in reality it is not very useful):
	
	However, to know the cumulative probability associated with $9$ events, we calculate for example with Microsoft Excel 14.0.6129:
	\begin{center}
		\texttt{=BINOM.DIST(9;600;1\%;TRUE)=BINOM.DIST(9;600;1\%;1)=91.71\%}
	\end{center}
	That is, a $p$-value of $8.29\%$. So whether it is unilateral or bilateral, this hypothesis test leads us not to reject the VaR model. It should not be forgotten that if $N$ is very large and the probability is small, the binomial law tends towards a Poisson law (\SeeChapter{see section Statistics page \pageref{poisson distribution}}) which is easier to manipulate (this is the recommendation of the CreditRisk+ standard).
	
	
	\pagebreak
	\subsection{Exchange rate}
	In finance, an "\NewTerm{exchange rate}\index{exchange rate}" (also known as a "\NewTerm{foreign-exchange rate}\index{foreign-exchange rate}", "\NewTerm{forex rate}\index{forex rate}") between two currencies is the rate at which one currency will be exchanged for another. It is also regarded as the value of one country's currency in relation to another currency. For example, an interbank exchange rate of $119$ Japanese yen (JPY, ¥) to the United States dollar (US\$) means that ¥$119$ will be exchanged for each US\$1 or that US\$1 will be exchanged for each ¥$119$. In this case it is said that the price of a dollar in relation to yen is ¥$119$, or equivalently that the price of a yen in relation to dollars is \$$1/119$.
	
	Exchange rates are determined in the foreign exchange market, which is open to a wide range of different types of buyers and sellers, and where currency trading is continuous: $24$ hours a day except weekends, i.e. trading from 20:15 GMT on Sunday until 22:00 GMT Friday. The spot exchange rate refers to the current exchange rate. The forward exchange rate refers to an exchange rate that is quoted and traded today but for delivery and payment on a specific future date.

	In the retail currency exchange market, different buying and selling rates will be quoted by money dealers. Most trades are to or from the local currency. The buying rate is the rate at which money dealers will buy foreign currency, and the selling rate is the rate at which they will sell that currency. The quoted rates will incorporate an allowance for a dealer's margin (or profit) in trading, or else the margin may be recovered in the form of a commission or in some other way. 
	\begin{figure}[H]
		\centering
		\includegraphics[scale=0.8]{img/economy/exchange_rate.jpg}
	\end{figure}
	Different rates may also be quoted for cash (usually notes only), a documentary form (such as traveler's cheques) or electronically (such as a credit card purchase). The higher rate on documentary transactions has been justified as compensating for the additional time and cost of clearing the document. On the other hand, cash is available for resale immediately, but brings security, storage, and transportation costs, and the cost of tying up capital in a stock of banknotes (bills).
	
	\begin{tcolorbox}[title=Remark,colframe=black,arc=10pt]
	The "\NewTerm{real exchange rate (RER)}\index{real exchange rate}" is the purchasing power of a currency relative to another at current exchange rates and prices. It is the ratio of the number of units of a given country's currency necessary to buy a market basket of goods in the other country, after acquiring the other country's currency in the foreign exchange market, to the number of units of the given country's currency that would be necessary to buy that market basket directly in the given country. There are various ways to measure RER.
	\end{tcolorbox}
	
	The world has over $150$ different currencies, from the Afghanistan afghani and the Albanian lek all the way through the alphabet to the Zambian kwacha and the Zimbabwean dollar. For international economic transactions, households or firms will wish to exchange one currency for another. Perhaps the need for exchanging currencies will come from a German firm that exports products to Russia, but then wishes to exchange the Russian rubles it has earned for euros, so that the firm can pay its workers and suppliers in Germany. Perhaps it will be a South African firm that wishes to purchase a mining operation in Angola, but to make the purchase it must convert South African rand to Angolankwanza. Perhaps it will be an American tourist visiting China, who wishes to convert U.S. dollars to Chinese yuan to pay the hotel bill.

	Exchange rates can sometimes change very swiftly. For example, in the United Kingdom the pound was worth \$$2$ in U.S. currency in spring 2008, but was worth only \$$1.40$ in U.S. currency six months later. For firms engaged in international buying, selling, lending, and borrowing, these swings in exchange rates can have an enormous effect on profits.

	The quantities traded in foreign exchange markets are breathtaking. A survey done in April, 2013 by the Bank of International Settlements, an international organization for banks and the financial industry, found that \$$5.3$ trillion per day was traded on foreign exchange markets, which makes the foreign exchange market the largest market in the world economy. In contrast, 2013 U.S. real GDP was \$ $15.8$ trillion per year.
	\begin{table}[H]
		\centering
		\begin{tabular}{|l|c|}
		\hline
		\rowcolor[HTML]{9B9B9B} 
		\multicolumn{1}{|c|}{\cellcolor[HTML]{9B9B9B}\textbf{Currency}} & \textbf{\% Daily share} \\ \hline
		U.S. dollar & $87.0\%$ \\ \hline
		Euro & $33.4\%$ \\ \hline
		Japanese yen & $23.0\%$ \\ \hline
		British pound & $11.8\%$ \\ \hline
		Australian dollar & $8.6\%$ \\ \hline
		Swiss franc & $5.2\%$ \\ \hline
		Canadian dollar & $4.6\%$ \\ \hline
		Mexican peso & $2.5\%$ \\ \hline
		Chinese yuan & $2,2\%$ \\ \hline
		\end{tabular}
		\caption[Currencies traded most on Foreign Exchange Markets as of April, 2013]{Currencies traded most on Foreign Exchange Markets as of April, 2013 (source: http://www.bis.org/publ/rpfx13fx.pdf)}
	\end{table}
	Portfolio investment is often linked to expectations about how exchange rates will shift. Look at a U.S. financial investor who is considering purchasing bonds issued in the United Kingdom. For simplicity, ignore any interest paid by the bond (which will be small in the short run anyway) and focus on exchange rates. Say that a British pound is currently worth \$$1.50$ in U.S. currency. However, the investor believes that in a month, the British pound will be worth \$$1.60 $in U.S. currency. Thus this investor would change \$$24,000$ for $16,000$ British pounds. In a month, if the pound is indeed worth \$$1.60$, then the portfolio investor can trade back to U.S. dollars at the new exchange rate, and have \$ $25,600$... a nice profit! A portfolio investor who believes that the foreign exchange rate for the pound will work in the opposite direction can also invest accordingly. Say that an investor expects that the pound, now worth \$$1.50$ in U.S. currency, will decline to \$$1.40$. Then that investor could start off with £$20,000$ in British currency (borrowing the money if necessary), convert it to \$$30,000$ in U.S. currency, wait a month, and then convert back to approximately £$21,429$ in British currency making again making a nice profit! Of course, this kind of investing comes without guarantees, and an investor will suffer losses if the exchange rates do not move as predicted.

	Many portfolio investment decisions are not as simple as betting that the value of the currency will change in one direction or the other. Instead, they involve firms trying to protect themselves from movements in exchange rates. Imagine you are running a U.S. firm that is exporting to France. You have signed a contract to deliver certain products and will receive $1$ million euros a year from now. But you do not know how much this contract will be worth in U.S. dollars, because the dollar/euro exchange rate can fluctuate in the next year. Let's say you want to know for sure what the contract will be worth, and not take a risk that the euro will be worth less in U.S. dollars than it currently is. You can hedge, which means using a financial transaction to protect yourself against a risk from one of your investments (in this case, currency risk from the contract). Specifically, you can sign a financial contract and pay a fee that guarantees you a certain exchange rate one year from now regardless of what the market exchange rate is at that time. Now, it is possible that the euro will be worth more in dollars a year from now, so your hedging contract will be unnecessary, and you will have paid a fee for nothing. But if the value of the euro in dollars declines, then you are protected by the hedge. Financial contracts like hedging, where parties wish to be protected against exchange rate movements, also commonly lead to a series of portfolio investments by the firm that is receiving a fee to provide the hedge.
	
	With the price of a typical good or service, it is clear that higher prices benefit sellers and hurt buyers, while lower
prices benefit buyers and hurt sellers. In the case of exchange rates, where the buyers and sellers are not always intuitively obvious, it is useful to trace through how different participants in the market will be affected by a stronger or weaker currency. Consider, for example, the impact of a stronger U.S. dollar on six different groups of economic actors, as shown in the table below:
	\begin{table}[H]
		\centering
		\begin{tabular}{|l|c|c|}
		\hline
		\rowcolor[HTML]{9B9B9B} 
		\textbf{Economic actors} & \multicolumn{1}{l|}{\cellcolor[HTML]{9B9B9B}\textbf{A stronger U.S. dollar}} & \multicolumn{1}{l|}{\cellcolor[HTML]{9B9B9B}\textbf{A weaker U.S. dollar}} \\ \hline
		A U.S. exporting firm & {\LARGE\Sadey} & {\LARGE \Smiley} \\ \hline
		A foreign firm exporting to the U.S. & {\LARGE \Smiley} & {\LARGE \Sadey} \\ \hline
		A U.S. tourist abroad & {\LARGE \Smiley} & {\LARGE \Sadey} \\ \hline
		A foreign tourist in the U.S. & {\LARGE \Sadey} & {\LARGE \Smiley} \\ \hline
		A U.S. investor abroad & {\LARGE \Sadey} & {\LARGE \Smiley} \\ \hline
		A foreign investor in the U.S. & {\LARGE \Smiley} & {\LARGE \Sadey} \\ \hline
		\end{tabular}
	\end{table}
	However, the foreign exchange value is not always determined by market forces, indeed, we distinguish $4$ types of foreign exchange policies:
	\begin{enumerate}
		\item When the FOREX is completely determined by market forces, we speak of "\NewTerm{floating exchange rates}\index{floating exchange rates}"

		\item When the FOREX is usually determined by market, but central bank sometimes intervenes, we speak then of "\NewTerm{soft exchange rate pegs}\index{soft exchange rate pegs}"

		\item When the FOREX is fixed at a certain level by in central bank we speak then of "\NewTerm{hard exchange rate pegs}\index{hard exchange rate pegs}"

		\item When the currency of a country is made identiacal to the currency of another country we speak then of "\NewTerm{merging currencies}\index{merging currencies}"
	\end{enumerate}
	
	
	\pagebreak
	\subsection{Credit default Risk}
	Credit risk is the risk involved in all transactions of borrowing and lending. If you lend money, what are the probabilities that the borrower will make the repayment?

	In case the borrower is likely to promptly return the money, then you have low credit risk, and in case there is a high probability of default than you have a high credit risk. So each borrowing and lending has its own element of risk involved.
	
	Credit risk can be classified in the following way:
	\begin{itemize}
		\item "\NewTerm{Credit default risk}\index{credit default risk}": The risk of loss arising from a debtor being unlikely to pay its loan obligations in full or the debtor is more than $90$ days past due on any material credit obligation; default risk may impact all credit-sensitive transactions, including loans, securities and derivatives.
	
		\item "\NewTerm{Concentration risk}\index{concentration risk}": The risk associated with any single exposure or group of exposures with the potential to produce large enough losses to threaten a bank's core operations. It may arise in the form of single name concentration or industry concentration.
	
		\item "\NewTerm{Credit spread risk}\index{credit spread risk}": The risk occurring due to volatility in the difference between investments' interest rates and the risk free return rate.
	
		\item "\NewTerm{Country risk}\index{country risk}": The risk of loss arising from a sovereign state freezing foreign currency payment (transfer/conversion risk) or when it defaults on its obligations (sovereign risk).
	
		\item "\NewTerm{Country spread risk}\index{country spread risk}": The risk occurring due to volatility in the difference between investments’ interest rates and the risk free return rate.
	\end{itemize}
	\begin{figure}[H]
		\centering
		\includegraphics[width=\textwidth]{img/economy/sovereign_credit_default_risk.jpg}
		\caption{Some countries spread risk}
	\end{figure}
	If we have large enough databases on customer profiles or companies profiles it is possible to set up mathematical models allowing to assign a certain probability of credit default to an economic agent (the most common technique used in the 20th century being the logistic regression demonstrated in the section of Numerical Methods).

	\subsubsection{Credit spread risk}
	Financial institutions that do not have such databases or the analytical capacity or that just consider that others to do better refer to most of times to Standard \& Poor's ratings for which a cumulative probability of default over several years $i$ is associated with each rating:
	\begin{table}[H]
		\centering
		\begin{tabular}{|c|c|c|c|c|c|c|c|}
		\hline
		\rowcolor[HTML]{C0C0C0} 
		\textbf{Year/Ranking} & \textbf{AAA} & \textbf{AA} & \textbf{A} & \textbf{BBB} & \textbf{BB} & \textbf{B} & \textbf{CCC/C} \\ \hline
		$1$ & $0.00$ & $0.01$ & $0.04$ & $0.29$ & $1.28$ & $6.24$ & $32.35$ \\ \hline
		$2$ & $0.00$ & $0.03$ & $0.13$ & $0.86$ & $3.96$ & $14.33$ & $42.35$ \\ \hline
		$3$ & $0.04$ & $0.08$ & $0.26$ & $1.48$ & $7.32$ & $21.57$ & $48.66$ \\ \hline
		$4$ & $0.07$ & $0.16$ & $0.43$ & $2.37$ & $10.51$ & $27.47$ & $53.65$ \\ \hline
		$5$ & $0.12$ & $0.26$ & $0.66$ & $3.25$ & $13.36$ & $31.81$ & $59.49$ \\ \hline
		$6$ & $0.21$ & $0.40$ & $0.90$ & $4.15$ & $16.32$ & $35.47$ & $62.19$ \\ \hline
		$7$ & $0.31$ & $0.56$ & $1.16$ & $4.88$ & $18.84$ & $38.71$ & $63.37$ \\ \hline
		$8$ & $0.48$ & $0.71$ & $1.41$ & $5.60$ & $21.11$ & $41.69$ & $64.10$ \\ \hline
		$9$ & $0.54$ & $0.83$ & $1.71$ & $6.21$ & $23.22$ & $43.92$ & $67.78$ \\ \hline
		$10$ & $0.62$ & $0.97$ & $2.01$ & $6.95$ & $24.84$ & $46.27$ & $70.8$ \\ \hline
		\end{tabular}
		\caption{Standard \& Poor rating table between 1981 and 2004}
	\end{table}
	From the cumulated probability of being in bankrupt at year $i$, we can determine, as we shall see, the conditional probability of bankrupt at year $i$, knowing that we survived until year $i$ (that is, the proportion who are insolvent in year $i$ among those who are still "alive" at the beginning of year $i$).
	
	If $c_n$ is the cumulative probability of going bankrupt between the year $0$ and the year $n$, then we have that $1-c_n$ which is of course the cumulative probability of having survived between year $0$ and $n$. It comes then quite intuitively (we guess by extension that we can apply the biometric function techniques seen in the section of Dynamics of Populations to the study of Credit Risk default):
	
	Thus we immediately see that it is easy to obtain the conditional probability $d_n$ of insolvency in year $n$ among those who are still "alive" at the beginning of year $n$:
	
	The trap here would be to make use not of conditional probabilities but to calculate the difference:
	
	which corresponds to the proportion of insolvent firms in year $n$ among all (those that have been insolvent and will be insolvent).
	
	Moreover we can notice that this error is dangerous since from what we obtained above it follows immediately that:
	
	So as I always say: in any case all the risks that one calculates - that the mathematical model is estimated as correct or not - always underestimate the real risks because there exists a quantity of risk that one will not have thought or that one has rightly or wrongly considered it unlikely.
	
	\begin{tcolorbox}[colframe=black,colback=white,sharp corners]
	\textbf{{\Large \ding{45}}Example:}\\\\
	E1. In the second year, for a company (or a country) rated BBB, we then have the conditional probability of default which is:
	
	
	E2. A borrower is rated BBB and we know that in case of default we are to be assured to be able to recover at $37\%$ of the sum invested (it is necessary to know that in practice this rate is also a statistical distribution ...). We can for example for $5$ years maturity bonds and a total invested amount of $100$ million and by taking in the previous rating table $c_5=3.25\%$ calculate that the estimated credit loss is then:
	
	\end{tcolorbox}
	There is a nice thing to notice when using this type of rating: indeed, the cumulative probability of default sometimes used by asset sellers can be estimated indirectly in comparison to risk-free returns and assuming the guaranteed $\%$ income in case of default. Let us see this with a simple case already partially known.
	
	Suppose we want to update a zero coupon bond at the risk-free market rate. We then write as usual:
	
	where $N$ is the nominal (often taken in practice as equal to $100.-$ as we already know).
	
	For a zero coupon bond that has the cumulative probability $c_n$ of default and denoting by $f$ the guaranteed fraction of the amount in the case of default, a neutral risk market would naturally requires that (the expression on the right of the equality is only an expected mean):
	
	where $r_f$ is the free risk market rate.
	
	Therefore, by simplifying the nominal disappears:
	
	Thus we can play according to what we know to determine different parameters as for example for the most common and interesting cases:
	
	where we notice for the both extreme cases that in accordance with intuition in the first case where the investment is $100\%$ safe and in the second case where the investment is $100\%$ risky:
	
	So it makes sense that the higher our risk, the higher the required return.

	We can also deduce the following very interesting equality (since it makes it possible to judge the reliability of a borrower):
	
	With the frequent case  $f=0$ where this obviously reduces to:
	
	It seems that some practitioner refer to the prior-previous relation as the "\NewTerm{estimate reduced model of the implicit credit default}".
	
	\subsubsection{Value At Risk equity coverage rating}
	The "\NewTerm{VaR equity coverage rating}\index{VaR equity coverage rating}" is based on the empirical basis that if an organization wishes to be rated in a certain way by Moody's then according to the Normal loss distribution model (but there are other far more pessimistic models that go so far to a multiple factor from $2$ to $4$) it can calculate the reserves it must have in order to enter a certain rating class.

	Consider, for example, the following Moody's table:
	\begin{table}[H]
		\centering
		\begin{tabular}{|l|c|c|}
		\hline
		\rowcolor[HTML]{C0C0C0} 
		\multicolumn{1}{|c|}{\cellcolor[HTML]{C0C0C0}\textbf{\begin{tabular}[c]{@{}c@{}}Desired Rating \\ (Moody's)\end{tabular}}} & \textbf{Probability of default to $\pmb{1}$ year ($\pmb{\alpha}$)} & \textbf{Corresponding $\pmb{Z}$} \\ \hline
		Aaa & $0.01\%$ & $3.72$ \\ \hline
		Aa1 & $0.02\%$ & $3.54$ \\ \hline
		Aa2 & $0.02\%$ & $3.54$ \\ \hline
		Aa3 & $0.03\%$ & $3.43$ \\ \hline
		A1 & $0.05\%$ & $3.29$ \\ \hline
		A2 & $0.06\%$ & $3.24$ \\ \hline
		A3 & $0.07\%$ & $3.19$ \\ \hline
		Baa1 & $0.13\%$ & $3.01$ \\ \hline
		Baa2 & $0.16\%$ & $2.95$ \\ \hline
		Baa3 & $0.70\%$ & $2.46$ \\ \hline
		Ba1 & $1.25\%$ & $2.24$ \\ \hline
		Ba2 & $1.79\%$ & $2.10$ \\ \hline
		Ba3 & $3.96\%$ & $1.76$ \\ \hline
		B1 & $6.14\%$ & $1.54$ \\ \hline
		B2 & $8.31\%$ & $1.38$ \\ \hline
		B3 & $15.08\%$ & $1.03$ \\ \hline
		\end{tabular}
		\caption{Moody's one year risk default}
	\end{table}
	
	Suppose now that an organization has its profit and loss account that historically has a standard deviation of $70,000$.-. Then the fractional reserves that it should have to be rated Aa2 are:
	
	For the people interested here is a grid with the equivalence of different Risk Metrics:
	\begin{figure}[H]
		\centering
		\includegraphics[scale=1]{img/economy/riskmetrics.jpg}
		\caption{Moody's, S\&P and Fitch Riskmetrics}
	\end{figure}
	
	\pagebreak
	\subsection{Time Series Analysis}\label{time series analysis}
	The goal of "\NewTerm{time series analysis (TSA)}" is not to link variables together, but to focus on the dynamics of a unique variable in time to discover some regularities in order to extrapolate or to forecast under the assumption that we can connect an observation with those that preceded it. 
	
	Therefore the difference between times series and econometric models differs from the fact that times series models do not start with a conceptual framework that defines the relationship between the economic variable and economic theory. Time series models use a variable's own history values to estimate the future values of the same variable as a hypothetical description process.	
	
	With a fine analysis, it is even possible to make "robust" forecasting vis-à-vis sudden breaks and non-anticipated changes.
	
	\begin{tcolorbox}[title=Remark,colframe=black,arc=10pt]
	This study domain is often named "\NewTerm{Business Intelligence}" in companies (this is a subfamily in fact limited to descriptive elementary statistics that should be named "Business Facts" instead... as we have described during our study of Data Mining at page \pageref{data mining}). A variable analyzed using TSA tools will be named a "\NewTerm{Key Performance Indicator (IPC)}" and a set of IPC a "\NewTerm{dashboard}". 
	\begin{figure}[H]
		\centering
		\includegraphics[scale=0.6]{img/economy/bloomberg_dashboard.jpg}
		\caption{Bloomberg\textsuperscript{TM} terminal KPI's dashboard}
	\end{figure}
	\end{tcolorbox}
	\textbf{Definition (\#\mydef):} A "\NewTerm{discrete time series}" or "\NewTerm{discrete chronological series}" is a series of observations $Y_t$ of a variable $y$ at different dates $t$. Usually the basis space $t$ is countable, so that at $t=1\ldots T$. The whole being denoted:
	
	and sometimes the capital $Y$ is denoted by $X$ (anyway it does not really matter ...).
	\begin{tcolorbox}[title=Remark,colframe=black,arc=10pt]
	We differentiates  significantly the models that predict real values (values in $\mathbb{R}$) of those who must predict purely integer values (count data belonging to $\mathbb{N}$). To this date, the content of this book deals only with models especially suitable for continuous variables TSA. 
	\end{tcolorbox}
	A time series is thus every sequence of observations corresponding to the same variable: it can comes to macroeconomic data (GDP of a country, inflation, exports), microeconomics (sales of a given company, number of employees, the income of an individual, the production quantity...), financial (the CAC 40, S\&P 500, the price of a call option or put option, the course of stock), weather (rainfall, number of sunny days per year,...), political (the number of voters, of votes received by a candidate ...), demographic (the average size of population, age, the propagation of an epidemy...).
	
	In practice, all that is quantifiable and vary over time can be analyzed relatively effectively as TSA since the person handling the model knows what she does (which is quite rare). But the reader must keep in mind that Economics and finance ask many questions to which the answers are more difficult and complex than those often pondered in mathematics and physics. For example, economic agents can react to economic forecasts, which makes predicting the stock market even harder than predicting the weather. Imagine how much more difficult it would be for meteorologists to forecast if the weather could read its own forecast and then change its behavior because it read the weather forecast!
	
	Therefore in finance we use these models to forecast returns, volatility (standard deviation), resupply stocks, demand, VaR and so on. In short, there really are no limits to the application of TSA.
	\begin{figure}[H]
		\centering
		\includegraphics[scale=0.9]{img/economy/sample_temporal_series_without_forecasting.jpg}
		\caption{Some time series without forecasting}
	\end{figure}
	The time dimension is important here because it is the analysis of a historical chronicle: the variations of the same variable over time in the purpose to understand its dynamics. We generally represent 2D time series on value charts (ordinate) versus time (abscissa). Such observation is an essential tool that allows the modeler with little experience to immediately realize the main dynamic properties to see what statistical test practice (at leat in the univariate and bivariate case...). The previous figure shows different time series as examples.
	
	A time series is divided mainly into four basic components:
	\begin{itemize}
		\item The "\NewTerm{secular long-term trend}" or simply "\NewTerm{trend}". It corresponds to a non-seasonal cyclical movement that reflects the long-term evolution of the measured variable (can be linear or not).
		
		\item The "\NewTerm{seasonal variations}" that are periodic fluctuations that occur regularly (periodically: days, months, quarters, years or more...).
		
		\item The "\NewTerm{hazards}" or "\NewTerm{residual/incidental variations}" or "\NewTerm{irregularities}" are random type fluctuations, generally of low amplitude.
	\end{itemize}
	\textbf{Definitions (\#\mydef):}
	\begin{enumerate}
		\item[D1.] The series which oscillate around their average are named "\NewTerm{stationary series}" (we'll see a rigorous definition of this futher below).

		\item[D2.] The series that seem to grow (bullish) or fall (bearish) on all of the observed sample are named "\NewTerm{trend series}" or "\NewTerm{short history}" and their averages are not constant.
	
		\item[D3.] The series that are neither stationary nor bullish or bearish on the long term trend are named "\NewTerm{non-stationary series"} or "\NewTerm{random series}" or "\NewTerm{stochastic series}". Any process whose time evolution can be analyzed in terms of probability is named as we know "\NewTerm{stochastic process}\label{stochastic process}".
	
		\item[D4.] The series which have a regular  frequency fluctuation are named "\NewTerm{seasonal series}" Regularly, stationary series are trend series with a periodic component.
	
		\item[D5.] The series that seem to be seasonal or appear to have a seasonal component but whose frequency is variable, are named "\NewTerm{cyclic series}".
		
		\item[D6.] A times series that has a significant jump (variation) at one moment in time is say to have a  "\NewTerm{level shift}".
		
		\item[D7.] A times series that has a significant jump (variation) at variosu moment in time and fall back to zero after every jump is "\NewTerm{intermittent series}".
	\end{enumerate}
	Given $y_t$ a value of the time series, $t_t$ its trend component, $s_t$ its seasonal component and $\varepsilon_t$ its random part, we define:
	\begin{enumerate}
		\item[D1.] An "\NewTerm{additive model} by an expression like:
		
	
		\item[D2.] A "\NewTerm{simple multiplicative model}" by an expression like:
		
		where we have the seasonal component that is then related to the trend component (as opposed to the additive model where the seasonal component is the same from multiple periods to another).
	
		\item[D3.] A "\NewTerm{complete multiplicative model}" by an expression like:
		
		
		\item[D4.] A "\NewTerm{mixed model}" by an expression like:
		
	\end{enumerate}
	The characteristics of these graphs are all modeled and analyzed in the context of time series analysis. For this there exists are more or less complex tools, some of which can not be doubted and others of which are heuristics and need to be handle and use with caution.

	In the text that follows, we will look to the elementary models available without a heavy mathematical artillery.

	Humility in the forecaster's job is however set to accept the mistake repeatedly, while seeking to improve the quality of its forecasts (sales, market volatility, correlation of assets, etc.) and identification of nonrandom components phenomena. It is important to remember the following 4 criteria:
	\begin{enumerate}
		\item The forecasts are generally not accurate. We need a robust ecosystem to react quickly to the unexpected events (that we can not predict, does not mean that we can not take advantage of the unpredictable) and this especially when we know that some multinationals and governments (which I will not give the name ... request projections over 20 months or sometimes 15 years...!!!).

		\item A good forecast is more than a numerical value!!Because forecasts are generally not accurate, a good forecast should ALWAYS include a measurement (or plot) of expected error for prediction (confidence interval). It is obvious that most of time the confidence interval (prediction errors) grows more you go in the future!!!!

		\item Aggregate forecasts are in general more accurate (yes we know this from the properties of the variance of an average). In general, we notice that the error made to the sales forecast of a line of products is generally less than the error made in the sales forecast of a single element.

		\item Long-term forecasts are less accurate. There is not much to say because it's intuitive....

		\item Forecasts can be sensitive to initial conditions (think of the chaotic patterns in meteorology or population dynamics ...)
	\end{enumerate}
	\begin{tcolorbox}[title=Remark,colframe=black,arc=10pt]
	Examples of where large multinational executives or governments have, for lack of statistical knowledge or misguided by non-statisticians managers / non-economists made projections of earnings or sales over $10$ to $20$ years (this is an aberration !! ) without confidence interval  are legion! Obviously the values communicated and the associated development strategy led some of these multinationals and governments to near catastrophe or simply to bankrupt. 
	\end{tcolorbox}
	It should be noticed that as some forecasts using sometimes past data that are based on sampling, obviously almost all models that will follow take this into account but this is not a reason for hiding on 2D or 3D plots the error bars of past data (I have seen many Fortune 500 companies that hide them to not show that their data or not usable... in reality!).
	
	Let us notice that quantitative forecasting techniques also use mainly the following tools (it is of course possible to combine them) for which we have already given examples in the respective mentioned chapters in parenthesis:
	\begin{itemize}
		\item Simple random walk (\SeeChapter{see section Thermodynamics page \pageref{brownian motion}})
		
		\item Simple linear regression (\SeeChapter{see section Numerical Methods page \pageref{simple linear regression}})
		
		\item Multiple linear regression (\SeeChapter{see section Numerical Methods page \pageref{multiple linear regression gaussian model}})
		
		\item General linear models (\SeeChapter{see section Numerical Methods})
		
		\item Logistic regression (\SeeChapter{see section Numerical Methods page \pageref{logistic regression logit}})
		
		\item Polynomial interpolation (\SeeChapter{see section Numerical Methods page \pageref{lagrange polynomial interpolation method}})
		
		\item Logarithmic and exponential linearization (\SeeChapter{see section Numerical Methods page \pageref{logarithmic and exponential linearization}})
		
		\item Modelling by Brownian motion (\SeeChapter{see section Mechanics page \pageref{standard brownian motion}})
		
		\item Moving Average (\SeeChapter{see section Statistics page \pageref{simple moving average}})
		
		\item Correlation analysis (\SeeChapter{see section Statistics page \pageref{coefficient of correlation}})
		
		\item Control Charts (\SeeChapter{see section Industrial Engineering page \pageref{quality control charts}})
		
		\item Jacknife or bootstrapping regression (\SeeChapter{see section Numerical Methods page \pageref{jacknife resampling} page \pageref{bootstrap}})
		
		\item Neural networks (\SeeChapter{see section Numerical Methods page \pageref{neural network}})
		
		\item Fourier (\SeeChapter{see section Sequences an Series page \pageref{fourier transform}})
	\end{itemize}
	and they can be grouped into two main families:
	 \begin{enumerate}
		\item Causes \& effects models based on several variables

		\item The time-series models whose only explanatory variable is time

		\item Permutations simulations

		\item Machines Learning modelization

		\item Spectral analysis
	\end{enumerate}
	and it is always difficult for forecasters to choose which is the best suited model to his working framework and the level of understanding of his superiors and colleagues. The interesting part of the job is that once or the model(s) selected (quite hard to choose from the hundreds of published empirical models...), we can compare them with past statistical data of the company and see if predictions are in agreement with what had really happened.
	\begin{tcolorbox}[title=Remark,colframe=black,arc=10pt]
	It is preferable to use statistical based models based rather than deterministic models that are of poor quality (as the simple moving average or double exponential smoothing).
	\end{tcolorbox}
	
	\subsubsection{Type of Errors}
	Before building predictive models we must be able to estimate the quality thereof (for comparison purposes between models) and it is customary to use several empirical indicators:
	\begin{itemize}
		\item The "\NewTerm{forecast error}":
		
		which is the difference between the relation value $Y_t$ and the value given by the model $\hat{Y}_t$.
	
		\item The "\NewTerm{mean error}" or "\NewTerm{forecasting bias}":
		
		that will only tell us if there is a positive or negative bias between the forecast and the real value (because if the value ME is not null then the model is biased). But as this value is often very small (negative values balancing the positive in some models) we prefer the following measurement error:
			
		\item The "\NewTerm{Average Absolute Deviation}" (named also "\NewTerm{Mean Absolute Deviation}" or "\NewTerm{Mean Absolute Error}" or "\NewTerm{Mean Forecast Error}"):
		
		which is the mean of the absolute errors made by the forecasting model on the duration of it, notwithstanding the fact that the error is overestimate or underestimate.
		
		\item The "\NewTerm{Average Squared Residuals}" (also named "\NewTerm{Mean Square Deviation}" or "\NewTerm{Mean Square Errors}"):
		
		allowing to penalize the large errors more than the small errors since each is multiplied by itself, giving a greater weight to large errors than to small errors.
		
		The problem with the previous indicators is that the magnitude of the error depends on the amplitudes of the analyzed values. To avoid this, it is more interesting to work with a percentage (relative value). Thus, we define the "\NewTerm{relative error}" or "\NewTerm{percentage error}" indicated in percent  by:
		
		which is the difference in percent between the real values and modeling one. It follows then the "\NewTerm{Mean Relative Error}", also named "\NewTerm{Relative Bias of Prediction}" or "\NewTerm{Mean Percentage Error}", always indicated in percents:
		
		For the same reasons that for the ME we prefer the following indicator error:
		
		\item The "\NewTerm{Relative Absolute Average Deviation}" also named "\NewTerm{Mean Absolute Percentage Error}"always indicated in percents:	
		
		The only problem with the relative indicators is when the analyzed time series contains zeros...
		
		\item The "\NewTerm{time momentum}" are the basic position and dispersion parameters that we know in statistics that are the mean and standard deviation but  just taken over a time period (widely used in finance!). Thus, the average and standard deviation of a measured variable of a time-window $[t-p, t]$ is obviously given by:
		
		
		\item The "\NewTerm{hit score}" (widely used in finance markets), the idea is very simple: it is simply the ratio of the number of good periodicals forecasts (the concept of "good" being quite empirical because there are dozens of choices) on the total number of past periods.
	\end{itemize}
	Finally, to end this section on error indicators, notice that in the field of forecasting we must also checked on the long term if a theoretical model is viable or not to replace it with another one. The basic techniques for this are to make use for example of an Autocorrelated controls chart (\SeeChapter{see section Industrial Engineering page \pageref{autocorrelated measurement control charts}}) where the measurement are simply the deviations (without taking their absolute value!) between the projected values and what really happened.
	
	\begin{tcolorbox}[title=Remark,colframe=black,arc=10pt]
	Let us notice also the usage in practice (especically banks) of the "\NewTerm{drawdown}\index{drawdown}" that is the measure of the decline from a historical peak in some variable (typically the cumulative profit or total open equity of a financial trading strategy).\\

	Somewhat more formally, if $X=(X(t),t\geq 0)$ is a random process with $X(0) = 0$, the drawdown at time $T$, denoted $D(T)$, is defined as:
	
	The "\NewTerm{maximum drawdown (MDD)}\index{maximum drawdown}" up to time $T$ is the maximum of the Drawdown over the history of the variable. More formally:
	
	\begin{figure}[H]
		\centering
		\includegraphics{img/economy/drawdown.jpg}
		\caption[]{Maximum Drawdown}
	\end{figure}
	\end{tcolorbox}
		
	\subsubsection{Decompositions}
	Some models using time series are adequate forecasting tools provided that the variations have shown some a stable pattern over time and that the conditions in which the pattern appeared always apply and that the past measurements don't have a to big intrinsic error.

	Sometimes a pattern is not apparent in the analysis of raw data; these can however be decomposed so as to reveal patterns that facilitate the projection data in the future (we will see the spectral decomposition by Fourier transform further below). The $4$ generally recognized components (already mentioned above) for time series out spectral analysis are: trend $T_t$ (linear, exponential, sigmoid, etc.), seasonality $S_t$, cyclicity $C_t$, random (irregular) $I_t$.
	
	Hence a time series using an additive model can be thought of as:
	
	While a multiplicative model would be:
	
	An additive model would be used when the variations around the trend does not vary with the level of the time series where as a multiplicative model would be appropriate if the trend is proportional to the level of the time series.
	
	Let us see this with a small companion practical example using (for simplicity) the  Microsoft Excel 14.0.6123. 

	Let us create a random component based on the absolute value of a Normal distribution of mean of $10$ and standard deviation of $20$ and over a period of $32$ months with the following formula in column \texttt{B} (\texttt{B1} to \texttt{B32}):
	\begin{center}
	=\texttt{ABS(NORM.INV(RAND(),10,20))}
	\end{center}
	where column \texttt{A} will contain the numbering of the periods ranging from $1$ to $32$ (\texttt{A1} to \texttt{A32}).

	Which will give a simulation of a random graph for which we will fix the values. For example:
	\begin{figure}[H]
		\centering
		\includegraphics{img/economy/tsa_decomposition_random.jpg}
		\caption[]{Plot of an example of the random function with Microsoft Excel 14.0.6123}
	\end{figure}
	Then we add a linear trend component (drift) linear of slope $4$ and of ordinate at the origin $20$ to the random component so as to have this time in column \texttt{B} the following formula:
	\begin{center}
	\texttt{=ABS(NORM.INV(RAND(),10,20))+(4*A1+20)}
	\end{center}
	The sum gives graphically:
	\begin{figure}[H]
		\centering
		\includegraphics{img/economy/tsa_decomposition_random_and_trend.jpg}
		\caption[]{Plot of the sum of the random function and the trend with Microsoft Excel 14.0.6123}
	\end{figure}
	Now let add a seasonal component of the sinusoidal type with a range of $50$ and $12$ months and without phase:
	\begin{figure}[H]
		\centering
		\includegraphics{img/economy/tsa_decomposition_seasonal_only.jpg}
		\caption[]{Plot of the seasonal part only with Microsoft Excel 14.0.6123}
	\end{figure}
	The sum of the three components will give in column \texttt{B}:
	\begin{center}
	\texttt{=ABS(NORM.INV(RAND(),10,20))+(4*A1+20)+50*SIN(A1*2*PI()*1/12)}
	\end{center}
	and graphically it gives:
	\begin{figure}[H]
		\centering
		\includegraphics{img/economy/tsa_decomposition_random_and_trend_and_seasonality.jpg}
		\caption[]{Plot of the sum of the random function, the trend and seasonality with Microsoft Excel 14.0.6123}
	\end{figure}
	and finally let us add a cyclical component which will again be for example a sinusoidal function of amplitude $20$ and period $64$:
	\begin{figure}[H]
		\centering
		\includegraphics{img/economy/tsa_decomposition_cyclic.jpg}
		\caption[]{Plot of the cyclic component with Microsoft Excel 14.0.6123}
	\end{figure}
	The sum of the four components gives in column \texttt{B}:
	\begin{center}
	\texttt{=ABS(NORM.INV(RAND(),10,20))+(4*A1+20)+50*sin(A1*2*PI()*1/12)+20*SIN(A1*2*PI( )*1/64)}
	\end{center}
	and graphically it gives:
	\begin{figure}[H]
		\centering
		\includegraphics{img/economy/tsa_decomposition_full.jpg}
		\caption[]{Sum of all components with Microsoft Excel 14.0.6123}
	\end{figure}
	Now, as we have introduced in detail in the section of Sequences And Series, let us make a fast Fourier transform (spectral analysis) - abreviated FFT - with the integrated Data Analysis Tools in Microsoft Excel 14.0.6123. This gives us then:
	\begin{figure}[H]
		\centering
		\includegraphics{img/economy/tsa_decomposition_fft_excel_list.jpg}
		\caption[]{Table of the FFT with the Data Analysis Toolpack of Microsoft Excel 14.0.6123}
	\end{figure}
	But unlike the section of Sequences and Series we will this time plot the frequency spectrum. For this, we first create a column with numbers from $0$ to $31$ and just next we put the formula visible in the screenshot below:
	\begin{figure}[H]
		\centering
		\includegraphics{img/economy/tsa_decomposition_fft_excel_list_frequencies.jpg}
		\caption[]{Formulas to get the frequencies in Microsoft Excel 14.0.6123}
	\end{figure}
	which gives:
	\begin{figure}[H]
		\centering
		\includegraphics{img/economy/tsa_decomposition_fft_excel_list_frequencies_and_periods.jpg}
		\caption[]{Values of the frequencies and periods of the FFT in Microsoft Excel 14.0.6123}
	\end{figure}
	Or graphically for the first 16 points (the other being identical by symmetry):
	\begin{figure}[H]
		\centering
		\includegraphics{img/economy/tsa_decomposition_fft_spectral_plot.jpg}
		\caption[]{Plot of the frequency peaks of the FFT with Microsoft Excel 14.0.6123}
	\end{figure}
	So we see that the transformation highlights a constant component (zero frequency), which is very important. We can abusively assmilate it with the linear component.

	We then see then that three points stand out with the respective periods (inverse of frequency): $31$ months, $15.50$, $10.33$. But we will only consider the first two values to associate them to the cyclicity (having the greatest period by definition) and seasonality.

	Thus, the Fourier transform leads us to consider a consistent trend, a cyclical phenomenon with a period of $31$ months (while we simulated a cycle with a period of $64$ months ...) and a seasonal phenomenon having a period $15$ months ( while we had a simulation with a period of $12$ months).

	So we see that this type of analysis has limitations but it can be useful as a tool for decision support.
	
	\subsubsection{Types of Forecasting Models}
	Time series analysis has two main approaches:
	\begin{enumerate}
		\item The deterministic approach that no use at all of statistical tools

		\item The statistical approach that allows to infer on forecasts\footnote{Also named "\NewTerm{predictive analytics}\index{predictive analytics}"}
	\end{enumerate}
	It goes without saying that the first method is easier than the second, but the second includes the first and is scientifically more acceptable since it will be possible to define confidence intervals (it fact it should be the only acceptable one!).

	These general concepts being presented we'll move on to the presentation of some regression mathematical models of time series. You should know about it that:
	\begin{enumerate}
		\item There are (to my knowledge) almost $60$ different mathematical well known models

		\item  Some models are deterministic, other stochastic (the latter being the best in the scientific point of view)

		\item They are (to my knowledge) all empirical and no one is perfect

		\item An spreadsheet software like Microsoft Excel is enough to model the majority of them in school classes
	\end{enumerate}
	Obviously there are a lot of forecasting models used by some (mainly) multinational companies. Most of them are presented further below in details but as requested by a reader here is a list of the most known one in increasing order of ascending complexity:
	\begin{itemize}
		\item Extrapolation (linear regression prediction)
		\item Simple Moving average
		\item Linear model with seasonal coefficient (LMSC)
		\item Simple Exponential smoothing (EWMA)
		\item Double exponential smoothing with one parameter (Brown Method)
		\item Holt's double exponential smoothing with $2$ parameters
		\item Holt's and Winter triple exponential smoothing with $3$ paramaters
		\item Logistic Model
		\item Kalman filtering
		\item Autoregressive processes (AR)
		\item Moving average autoregressive processes (MA)
		\item Autoregressive moving average (ARMA)
		\item Autoregressive integrated moving average (ARIMA)
		\item Vecteur autoregressive models
		\item Fractional ARIMA (FARIMA)
		\item Seasonal ARIMA (SARIMA)
		\item Generalized linear model (GLM)
		\item Generalized AutoRegressive Conditional Heteroskedasticity GARCH (or sGARCH)
		\item rich univariate GARCH, integrated GARCH (iGARCH)
		\item Neural networks
		\item ...
	\end{itemize}

	We will see below the proof of those that require a mathematical approach and not already studied in previous chapters (like the Generalized Linear Models). The others being only toolboxes, they don't have their place in such a book.
	
	Typical time series patterns can be seen in the figure below:
	\begin{figure}[H]
		\centering
		\includegraphics[scale=0.8]{img/economy/exhibit_times_series_patterns.jpg}
		\caption{Exhibit of easily recognizable time series patterns}
	\end{figure}
	and a common classification of existing model can visually be given by:
	\begin{figure}[H]
		\centering
		\includegraphics[scale=0.75]{img/economy/classification_time_series_models.jpg}
		\caption{Classification of common time series models}
	\end{figure}
	We will in what follows obviously focus first on classic undergraduate family of models.
	
	\begin{tcolorbox}[title=Remark,colframe=black,arc=10pt]
	Using a single variable to analyze a complex situation is a good way to reach bad conclusions (never forget this when you make time forecasting based only as the time for explanation variable...!).
	\end{tcolorbox}
	
	\pagebreak
	\paragraph{Simple Moving Average (moving average smoothing)}\label{simple moving average}\mbox{}\\\\
	The simple moving average (MA) forecast technique of order $k$ or is certainly the empirical technique the simplest and also the worst in terms of predictive power. Based on the definition of the moving average (\SeeChapter{see section Statistics page \pageref{moving average}}) it is defined by:
	
	and is therefore a deterministic forecast on only a single subsequent period.
	
	Let us see an example with Microsoft Excel 11.8346 with a forecast on the horizon of a month with moving average smoothing models in columns \texttt{D} and \texttt{E}:
	\begin{figure}[H]
		\centering
		\includegraphics{img/economy/forecast_moving_average_list.jpg}
		\caption[]{Base list for the example of moving average smoothing in Microsoft Excel 11.8346}
	\end{figure}
	Explicitly:
	\begin{figure}[H]
		\centering
		\includegraphics{img/economy/forecast_moving_average_list_explicit_formulas.jpg}
		\caption[]{Base list for the MA example with explicit formulas in Microsoft Excel 11.8346}
	\end{figure}
	and the corresponding chart:
	\begin{figure}[H]
		\centering
		\includegraphics{img/economy/forecast_moving_average_plot.jpg}
		\caption{MA($3$) and MA($5$) plot with Microsoft Excel 11.8346}
	\end{figure}
	And elementary calculations give (we can provide the details with Microsoft Excel on readers request):
	
	We have not used the Analysis Toolpack of Microsoft Excel 11.8346 (the latter incorporating a tool automatically making the calculations of the moving average) only because we wanted show the reader the explicit application of the mathematical relationship given above.
	\begin{figure}[H]
		\centering
		\includegraphics{img/economy/forecast_moving_average_excel_analysis_tookpack.jpg}
		\caption{Microsoft Excel Moving Average choice in the Analysis Toolpack}
	\end{figure}
	The choice of the number of periods $k$ to be used in calculating the simple moving average depends very much on expected variations in the data. This can be illustrated by two characteristics of the forecast:
	\begin{itemize}
		\item "\NewTerm{Stability}": By averaging several periods, we attenuate the random fluctuations also that the prediction is more stable. Stability is the property of the forecast to not fluctuate randomly. Win in stability is an advantage if there is a lot of random fluctuations in the data. A moving average will win in stability if a larger number of periods is used in the calculation of the average. A winn in stability is desirable only to the point where random fluctuations are mitigated. If the number of periods used in the calculation of the average is too large, the average will be so stable that it will respond too slowly to non-random changes.

		\item "\NewTerm{Reactivity}": Reactivity is the property that the forecast quickly adjust to a change in the actual average level. The use of a reactive forecast is appropriate in the case where random fluctuations are small. The smaller the number of periods used in the calculation of the moving average is, the more the forecast model will be reactive.
	\end{itemize}
	\begin{tcolorbox}[title=Remark,colframe=black,arc=10pt]
	Let us notice that some forecasting techniques do moving averages of moving averages ... In the literature, this is named "\NewTerm{double moving average}" ... There are also models that apply empirical weight to past periods, this is named "\NewTerm{weighted moving average}" and this is in my opinion out of interest. 
	\end{tcolorbox}

	\pagebreak
	\paragraph{Linear Model With Seasonal Coefficients (LMSC)}\mbox{}\\\\
	The linear predictive technique with seasonal factors is simple to implement and to understand for most people. It combines a simple linear regression (\SeeChapter{see section Numerical Methods page \pageref{simple linear regression}}) and seasonal factors calculated on the basis of past values relatively to the linear model. This technique allows to identify trends by smoothing the time series and showing significant variations in time. 

	Let us see a companion example with Microsoft Excel always 11.8346:
	\begin{figure}[H]
		\centering
		\includegraphics{img/economy/forecast_linear_model_seasonal_coefficients_list_excel.jpg}
		\caption[]{Companion LMSC example starting values in Microsoft Excel 11.8346}
	\end{figure}
	Which looks like:
	\begin{figure}[H]
		\centering
		\includegraphics{img/economy/forecast_linear_model_seasonal_coefficients_list_excel_plot_sales.jpg}
		\caption[]{Corresponding Sales plot in Microsoft Excel 11.8346}
	\end{figure}
	The first thing to do is to determine the equation of the regression line (which allows verbatim also to make inference!). What is simple with the least squares method proved in the section of Numerical Methods and extremely simple to obtain with automatic tools built integrated in Microsoft Excel 11.8346 (Analysis Toolpack again... or the functions already seen in the section of Numerical Methods):
	\begin{figure}[H]
		\centering
		\includegraphics{img/economy/forecast_linear_model_seasonal_coefficients_list_excel_plot_sales_and_linear_regression.jpg}
		\caption[]{Linear regression equation for Sales in Microsoft Excel 11.8346}
	\end{figure}
	This allows us to have three new columns in our Microsoft Excel list. One with the values of the linear model and another with the ratio between the actual and the linear model (which corresponds to the seasonal coefficients!):
	\begin{figure}[H]
		\centering
		\includegraphics{img/economy/forecast_linear_model_seasonal_coefficients_list_excel_plot_sales_with_coefficients.jpg}
		\caption[]{LMSC full analysis starting values in Microsoft Excel 11.8346}
	\end{figure}
	That is to say visually:
	\begin{figure}[H]
		\centering
		\includegraphics{img/economy/forecast_linear_model_seasonal_coefficients_excel_plot.jpg}
		\caption[]{LMSC full plot in Microsoft Excel 11.8346}
	\end{figure}
	We can then make a prediction based on the linear regression multiplied by the average of the seasonal coefficients of each month. What we write in this case:
	
	So this is a deterministic method to make forecasts on as many future periods as desired with the ability to make statistical inference if necessary as there is an underlying linear Gaussian model (this is the only deterministic model on which the inference is directly applicable as far as we know).
	
	So for example:
	\begin{figure}[H]
		\centering
		\includegraphics{img/economy/forecast_linear_model_seasonal_coefficients_list_excel_plot_sales_with_coefficients_projection.jpg}
		\caption[]{LMSC list with projection in Microsoft Excel 11.8346}
	\end{figure}
	Explicitly:
	\begin{figure}[H]
		\centering
		\includegraphics{img/economy/forecast_linear_model_seasonal_coefficients_list_excel_plot_sales_with_coefficients_projection_explicit.jpg}
		\caption[]{LMSC projection with formulas in Microsoft Excel 11.8346}
	\end{figure}
	That is to say visually:
	\begin{figure}[H]
		\centering
		\includegraphics{img/economy/forecast_linear_model_seasonal_coefficients_list_excel_plot_sales_with_coefficients_and_projection.jpg}
		\caption[]{LMSC plot with projection in Microsoft Excel 11.8346}
	\end{figure}
	And elementary calculations (we can provide the details with Microsoft Excel on readers request):
	
	
	\paragraph{Simple Exponential Smoothing (EWMA)}\mbox{}\\\\
	The prediction $F_t$ provided by the exponential smoothing method (also named "\NewTerm{Exponential Weighted Moving Average (EWMA)}" or "\NewTerm{Single Exponential Smoothing (SES)}"), also named "\NewTerm{exponential weighted moving average}", with "\NewTerm{smoothing constant $\beta \in ]0,1[$}" is given as we will prove it later by:
	
	This definition is based on a simple idea (yet totally empirical!): we assume that the observations influence less the forecast as they are far from the time $t$ for which we forecast and that this influence decreases exponentially (this model remains even in the early 21st century is still very used to model the variance of financial instruments). We then see easily at first glance that:
	\begin{itemize}
		\item Mote the smoothing constant $\beta$ is close to $1$, the more the influence of past observations goes back further in time and the forecast is rigid, that is to say not very sensitive to early economic fluctuations.

		\item On the contrary, more the smoothing constant is close to zero, more the forecast is influenced by recent observations (flexible forecasting).
	\end{itemize}
	Let us prove first that the previous definition can be written in another form less technique and that is most common in statistical books:
	
	and we also have verbatim:
	
	Therefore if we reindex the sum we find a well-known writing in the finance world (volatility measurement of the RiskMetrics model):
	
	The forecast $F_{t+1}$ appears as the weighted average between the forecast made at time $t$ and the last observation $Y_t$, the weight given to the latter observation is even stronger than the smoothing constant is small.
	
	We will now demonstrate why software forecast the EWMA models with a mean equal to the last measurement:
	
	In practice, the choice of the smoothing constant is often based on subjective criteria of rigidity or flexibility of the prediction. But there is a more objective method of determining the constant using operational research tools (\SeeChapter{see section Numerical Methods page \pageref{operational research}}) where the constant minimizes the sum of squared forecast errors.

	Caution!!! In the majority of contemporary statistics books, the smoothing constant is defined as:
	
	It comes then:
	
	writing that leads us to consider the simple exponential smoothing as a simple weighted moving average.

	One and the other writing are equivalent of course but we with some softwares to be careful to know which of the two constants of smoothing we are dealing with... We will personally go with the latter form for the rest of our developments.

	Now, let us start again from the above relation to write it in another form using recursion:
	
	Let us not prove that the sum of weights is equal to the unit to see that it indeed a weighted system. So we have:
	
	We have proved in the section of Sequences and Series that:
	
	Then it comes into our example:
	
	Therefore:
	
	Finally, let us talk about the origin of the name of this technique. So we have:
	
	So see in there the coefficients:
	
	and as $1-\lambda$ is strictly less than unity, the factors 
$(1-\lambda)^2$ decreases exponentially (just plot the function to see it) as:
	
	with obviously $n\in\mathbb{N}$.
	
	This being done, let us talk now about the problem of the initial values. So we have:
	
	It is then agreement to take:
	
	Which gives us:
	
	This is why a program like Microsoft Excel 11.8346 returns for $F_1$ the error \texttt{\#N/A}.
	
	For the rest it is necessary that we study the origin of the expression of the simple exponential smoothing. The idea at the base of mathematicians was to reduce the error between the real value $Y_{t-j}$ and the prediction considered as constant a:
	
	but the goal was also to weaken the errors smaller than the unit and amplify those that were larger by squaring deviations such as:
	
	and finally to weight those that were remote in time from the prediction by a factor such that we have minimize finally:
	
	Taking the derivative of $E$ with respect to $a$, and by canceling it comes:
	
	We then get for $\beta \neq 1$:
	We then get for $\beta\neq 1$:
	
	and thus finally:
	
	and for large $t$ we get:
	
	Notice that it is not obvious that the desired constant that minimizes the error can be used as prediction function. However it is the usage that is made by many practitioners... Some experts prefered to write the latter result as (do not try to compare this too much this notation with that of the previous relation):
	
	where $S$ refers to the "S" there is in the word "eStimated". There is even statistical software offering to our astonishment to do forecasts a time $t+n$ based on this method ... which seems to me crazy... (to be continued...).
	\begin{tcolorbox}[title=Remark,colframe=black,arc=10pt]
	The simple exponential smoothing seems that it was first suggested by Charles C. Holt in 1957, but the formulation (the form of the predictive relation) is attributed to Brown. This is a technique that can be found implemented in almost all statistical software. 
	\end{tcolorbox}
	
	
	\begin{tcolorbox}[colframe=black,colback=white,sharp corners]
	\textbf{{\Large \ding{45}}Example:}\\\\
	So we want to apply:
	
	Let us consider the following table (the column C already contains the mathematical model which we will detail the formulas just after):
	\begin{figure}[H]
		\centering
		\includegraphics[scale=0.55]{img/economy/forecast_ses_list_excel_original.jpg}
		\caption[]{List of data of Exponential Smoothing in Microsoft Excel 14.0.6123}
	\end{figure}
	\end{tcolorbox}
	
	\begin{tcolorbox}[colframe=black,colback=white,sharp corners]
	We will not use the Analysis Toolpack of analysis of Microsoft Excel (the latter incorporating a tool doing automatically the calculations of the simple exponential smoothing) only to show the reader the explicit application of the previous proven mathematical relation. Furthermore, the Analysis Toolpack of Microsoft Excel 14.0.6123 (and previous releases) suffers from a major flaw: it requires the value of the smoothing constant rather than calculate an optimum value automatically.\\

	There are only three columns that have formulas, the remainder being static entries. We have:
	\begin{figure}[H]
		\centering
		\includegraphics[scale=0.65]{img/economy/forecast_ses_list_excel_original_explicit_formulas.jpg}
		\caption[]{Explicit list of data of SES in Microsoft Excel 14.0.6123}
	\end{figure}
	where the attentive reader will have noticed the forecast at the line 28 over a period of one month.\\

	Little bit more far in the worksheet we have put values essential to the optimization of the model thanks to the solver that we will immediately see:
	\begin{figure}[H]
		\centering
		\includegraphics[scale=0.65]{img/economy/forecast_ses_excel_parameters.jpg}
		\caption[]{Parameters cells and SES Forecast errors in Microsoft Excel 14.0.6123}
	\end{figure}
	Now we will use the solver to minimize either MAPE, MAD or MSD. We then for example by minimizing the MAPE error:
	\end{tcolorbox}
	
	\begin{tcolorbox}[colframe=black,colback=white,sharp corners]
	\begin{figure}[H]
		\centering
		\includegraphics[scale=0.65]{img/economy/forecast_ses_excel_solver_parameters.jpg}
		\caption[]{Solver parameters for SES Forecast in Microsoft Excel 14.0.6123}
	\end{figure}
	This gives the following result:
	\begin{figure}[H]
		\centering
		\includegraphics[scale=0.65]{img/economy/forecast_ses_excel_solver_results.jpg}
		\caption[]{Solver results for SES Forecast in Microsoft Excel 14.0.6123}
	\end{figure}
	and graphically:
	\begin{figure}[H]
		\centering
		\includegraphics{img/economy/forecast_ses_excel_plot.jpg}
		\caption[]{SES Forecast plot in Microsoft Excel 14.0.6123}
	\end{figure}
	and therefore the table of values becomes:
	\end{tcolorbox}
	
	\begin{tcolorbox}[colframe=black,colback=white,sharp corners]
	\begin{figure}[H]
		\centering
		\includegraphics{img/economy/forecast_ses_list_excel_original_after_computation.jpg}
		\caption[]{SES Forecast final values in Microsoft Excel 14.0.6123}
	\end{figure}
	so we see that the forecast on line 28 of the table is:
	
	With for recall:
	
	\end{tcolorbox}
	To conclude, let us indicate that if we make the strong assumption that for any time t:
	
	Then in the case of an infinite series (or supposed as such ...) with independent values:
	
	However, as $\beta\in]0,1[$ and if $t$ is large enough, we get:
	
	and let us consider the error:
	
	We have under the assumption that the two subtracted terms are independent ...:
	
	It follows under the assumption of Normality (which is the case since the beginning of this development) and that $F_t$ is the mean of $Y_t$:
	
	and under the assumption that the variance is well known, we get:
	
	But if the standard deviation is only estimated, it is obviously necessary to use the Student law (\SeeChapter{see section Statistics page \pageref{student distribution}}):
	
	And that will be all so far regarding to the simple exponential smoothing model.
	
	\paragraph{Double Exponential Smoothing with One Parameter (Brown Method)}\mbox{}\\\\
	The idea of the double exponential Brown smoothing is again to reduce the total error but this time using not a constant but a straight linear the neighborhood of $t$ such that:
	
	what is written traditionally (the $-$ sign is absorbed into the parameter $a$ and the fact that the parameters of the straight line are recalculated at each $t$ is omitted in the way to not make any confusion with the index of summation...):
	
	To minimize we differentiate the error $E$:
	
	Therefore:
	
	Let us simplify all this by first replacing when $t$ is large:
	
	For the others, under certain conditions that are met in our case, we can derivate one term to one term a convergent series to get derivative thereof. Thus, given that:
	
	and as in in the above conditions:
	
	we then have immediately:
	
	Multiplying by $\beta$ on the left and right we get to finally:
	
	which is the second desired simplification we were looking form.

	For the third one we simply derivate once again the previous equality. We then get:
	
	And we end by multiplying left and right by $\beta$:
	
	We then get:
	
	What we can write under the form:
	
	Let us define now the smoothed series (the reader will notice that this is in fact the simple exponential smoothing):
	
	and the double smoothes series (hence the technical name of the method we are studying now):
	
	In practice will just use:
	
	
	After the change of variable $i+j=k\Leftrightarrow-i-j=-k$ we get:
	
	The index of the second sum is explained by the fact that $k$ is the new indexation variable (so it must necessarily be in the index!). And when $i$ is zero we have:
	
	For the suffix of the sum it is explaine by the fact that for the maximum value we have $i=t-j-1$ and therefore as $i+j=k$ it comes if we substitute in it $i$:
	
	Then, the reader can check by itself... on paper... that we can rearrange the terms as follows:
	
	Then we have:
	
	Substituting these results in:
	
	we get:
	
	Then it comes after successive simplifications:
	
	and therefore also after successive simplifications:
	
	Results that it is customary to write:
	
	Therefore:
	
	is the straight line constituted by a double exponential smoothing that approximates at best (the estimator) at a time $t$ the real values that minimizes the total error. Unlike the simple exponential smoothing, we can by now against making predictions (however it remains an abusive use of this mathematical result) thanks to the presence of the variable $j$.
	
	Explicitly, this straight line will be written by changing a little bit the notations (do not forget that at the beginning we had implicitly injected a negative sign in the term $a$):
	
	It comes then that our forecast will be written:
	
	And obviously, if we want ${F''}_{t+1}$, we will have to choose $j=-1$. Therefore:
	
	To start the forecast with this model, we need an estimate of $S_1(0)$ and $S_2(0)$. The first forecasts will be greatly affected by these estimates. The practitioner should also always check what its software as method for the determination for these first parameters in the purpose to specify it in his reports.
	\begin{tcolorbox}[title=Remark,colframe=black,arc=10pt]
	Brown's exponential smoothing is rarely integrated in statistical software. We prefers to him the Holt's model that we will see immediately see further below and that contains the Brown model implicitly. 
	\end{tcolorbox}
	\begin{tcolorbox}[colframe=black,colback=white,sharp corners]
	\textbf{{\Large \ding{45}}Example:}\\\\
	Let us consider the following list in Microsoft Excel 14.0.6123 that contains the observed data in column \texttt{B}, and the model in columns \texttt{C} through \texttt{F} for which we will detail the explicit formulas just after:
	\begin{figure}[H]
		\centering
		\includegraphics[scale=0.7]{img/economy/forecast_double_ses_brown_list_excel.jpg}
		\caption[]{Starting data for the Brown's double exponential smoothing model with Microsoft Excel 14.0.6123}
	\end{figure}
	\end{tcolorbox}
	
	\begin{tcolorbox}[colframe=black,colback=white,sharp corners]
	Let us give now the formulas for C, D, E, F, G (C and D are the same formulas as for the simple exponential smoothing). It comes then:
	\begin{figure}[H]
		\centering
		\includegraphics[scale=0.45]{img/economy/forecast_double_ses_brown_list_excel_formulas.jpg}
		\caption[]{Explicit formulas for the Brown's double exponential smoothing model with Microsoft Excel 14.0.6123}
	\end{figure}
	With a little bit more far in the worksheet the values essential to the optimization model (and standard forecast errors) with the solver that we will immediately see:
	\begin{figure}[H]
		\centering
		\includegraphics[scale=0.7]{img/economy/forecast_double_ses_brown_list_excel_initial_parameters.jpg}
		\caption[]{Initial double SES parameters in Microsoft Excel 14.0.6123}
	\end{figure}
	After optimization with the solver by minimizing MAPE we get:
	\begin{figure}[H]
		\centering
		\includegraphics[scale=0.7]{img/economy/forecast_double_ses_brown_list_excel_final_parameters.jpg}
		\caption[]{Final double SES parameters in Microsoft Excel 14.0.6123}
	\end{figure}
	So finally the whole table gives:
	\end{tcolorbox}
	
	\begin{tcolorbox}[colframe=black,colback=white,sharp corners]
	\begin{figure}[H]
		\centering
		\includegraphics[scale=0.7]{img/economy/forecast_double_ses_brown_list_excel_final_list.jpg}
		\caption[]{Final double SES values in Microsoft Excel 14.0.6123}
	\end{figure}
	The result gives graphically (and also in comparison with the simple exponential smoothing) still for $j=-1$:
	\begin{figure}[H]
		\centering
		\includegraphics[scale=0.7]{img/economy/forecast_double_ses_brown_list_full_plot.jpg}
		\caption[]{Final double SES values in Microsoft Excel 14.0.6123}
	\end{figure}
	What is interesting and special with the double exponential smoothing is to play with the value of $j$. Thus, by putting $j=-1$, we will have that for each value value in column \texttt{G} its projection in the corresponding period plus $|j|$ period.\\

	Let us make a little visual example. By putting $j=-6$ (thus a projection to $6$ months ...) then we have:
	\end{tcolorbox}
	
	\begin{tcolorbox}[colframe=black,colback=white,sharp corners]
	\begin{figure}[H]
		\centering
		\includegraphics[scale=0.8]{img/economy/forecast_double_ses_brown_list_excel_forecast_j_6.jpg}
		\caption[]{Projection $j=-6$ with the DSES in Microsoft Excel 14.0.6123}
	\end{figure}
	Then, what is interesting is to remember is that if we put $j=-6$, the $20$th period, then projects for the $26$th period, the $21$th on the $27$th, etc. We get so graphically:
	\begin{figure}[H]
		\centering
		\includegraphics[scale=0.8]{img/economy/forecast_double_ses_brown_list_full_plot_forecast_j_6.jpg}
		\caption[]{Projection over a period of one semester using DEBS with Microsoft Excel 14.0.6123}
	\end{figure}
	\end{tcolorbox}
	Obviously, like any model, the practitioner must be very careful with projections to $6$ months (or less or more) because the model does not take into account the factors of the stock market, natural disasters, pandemics, wars, etc. So long as there is no statistical inference, it is not possible to give a confidence interval for the forecasting what is scientifically speaking not acceptable.
	
	\paragraph{Holt's Double Exponential Smoothing with $2$ Parameters (Additive Method)}\mbox{}\\\\
	We will discuss now a technique which construction is empirical and which is due to Charles C. Holt (1957). Even if in the original article of C. Holt there is no rigorous proof of the construction of this model... so it is more mathematical engineering that pure mathematics itself.

	The idea is as with the previous model, to have a model of the form:
	
	but with an empirical approach that requires:
	\begin{itemize}
		\item The slope $a_t$ of the straight line of model which origin of the frame is in $t$ will be obtained by a simple exponential smoothing of all the slopes of the previous straight lines.

		\item The intercept $b_t$ of the straight line on $t$ will be obtained by a simple exponential smoothing all intercepts also of all the previous straight lines.
	\end{itemize}
	Let us first recall that we  have obtain for the simple exponential smoothing:
	
	It is clear that the slope of each line of the estimate is given by (see figure below):
	
	and for the last estimate, more specific, we will have:
	
	\begin{figure}[H]
		\centering
		\includegraphics{img/economy/holt_additive_model_construction_principle.jpg}
		\caption[]{Idea diagram of the building blocs of Holt's model}
	\end{figure}
	and the idea is to make an exponential smoothing of the slope of all straight lines whose slope is calculated in the same manner, that is (compared to the expression of simple exponential smoothing)
	
	The latter expression is just the estimated slope at time $t$.

	Now, for the intercept, we see in the figure that $b_t=S_t$. Therefore, it comes at first as:
	
	Once this fact seen, we choose a model where the intercept is an exponential smoothing of all intercepts with his own smoothing constant such as:
	
	What may seem sometimes confusing here is the last parenthesis. In fact, technically, we should rather write:
	
	but as the $\Delta t$ is always equal to $1$, we omit to write it. So the last parenthesis represents the estimated at a one previous time unit to which we add the increase of one time unit of the estimated of slope as estimated at the same time $t$.

	The expression:
	
	is somewhat unfortunately named in practice the "estimated level series at period $t$".

	In the end, we work with these three relations:
	
	To start the forecast with this model, we need an estimate of the settings $S_0$ and $a_0$. The first forecast values will be greatly affected by these estimates. The practitioner should also always check what his software uses for a method of determination for these settings to specify it in his reports.
	\begin{tcolorbox}[colframe=black,colback=white,sharp corners]
	\textbf{{\Large \ding{45}}Example:}\\\\
	Let us consider the following list and model settings area in Microsoft Excel that 14.0.6123 that contains the observed data  in column \texttt{B}, and the model in columns \texttt{C} through \texttt{E} for which we will detail the formulas just after:	
	\begin{figure}[H]
		\centering
		\includegraphics[scale=0.7]{img/economy/forecast_double_ses_holt_initial_list_and_parameters_excel.jpg}
		\caption[]{Initial values list and area model settings for Holt's model with Microsoft Excel 14.0.6123}
	\end{figure} 
	\end{tcolorbox}
	
	\begin{tcolorbox}[colframe=black,colback=white,sharp corners]
	Let us indicate explicitly all formulas:
	\begin{figure}[H]
		\centering
		\includegraphics[scale=0.38]{img/economy/forecast_double_ses_holt_explicit_initial_list_and_parameters_excel.jpg}
		\caption[]{Explicit formulas and area model settings for Holt's model with Microsoft Excel 14.0.6123}
	\end{figure} 
	As for all the models seen previous, we start the solver with the idea to minimize the MAPE:
	\begin{figure}[H]
		\centering
		\includegraphics[scale=0.75]{img/economy/forecast_double_ses_holt_explicit_final_list_and_parameters_excel.jpg}
		\caption[]{Final values and projections of Holt's model with Microsoft Excel 14.0.6123}
	\end{figure} 
	\end{tcolorbox}
	
	\begin{tcolorbox}[colframe=black,colback=white,sharp corners]
	Which gives graphically (and also in comparison with the single and double smoothing):
	\begin{figure}[H]
		\centering
		\includegraphics[scale=0.75]{img/economy/forecast_double_ses_holt_full_plot_excel.jpg}
		\caption[]{Full Holt's model plot with Microsoft Excel 14.0.6123}
	\end{figure} 
	Holt's smoothing has a MAPE of $7.49\%$ while the Brown model has for recall a MAPE of $7.45\%$ (this comparison is a form of "back-testing"). However the advantage of Holt's model relatively to previous models is that we can make projections on more than one period.\\

	It is of course interesting to compare the projections to $6$ months between the additive method of Holt and Brown's smoothing:
	\begin{table}[H]
	\begin{center}
		\definecolor{gris}{gray}{0.85}
			\begin{tabular}{|c|c|c|c|c|}
				\hline
  \multicolumn{1}{c}{\cellcolor{black!30}\textbf{Month}}  & 
  \multicolumn{1}{c}{\cellcolor{black!30}\textbf{Brown's DSES}} & 
  \multicolumn{1}{c}{\cellcolor{black!30}\textbf{Holt's additive DSES}} \\ \hline
				$25$ & $261.50$ & $263.30$ \\ \hline
				$26$ & $268.19$ & $270.00$ \\ \hline
				$27$ & $264.55$ & $276.69$ \\ \hline
				$28$ & $280.53$ & $283.39$ \\ \hline
				$29$ & $238.20$ & $290.08$ \\ \hline
				$30$ & $311.19$ & $296.78$ \\ \hline
		\end{tabular}
	\end{center}
	\caption[]{Comparison between Holt's and Brown's DSES}
	\end{table}
	So we see that Holt's model almost always have forecasting values greater than the Brown model.\\

	It is the job of the practitioner to judge  these values and compare them to past observations and also to future one...
	\end{tcolorbox}
	
	\pagebreak
	\paragraph{Holt's and Winter Triple Exponential Smoothing with $3$ Parameters (Multiplicative Method) }\mbox{}\\\\
	We will introduces here a technique which construction is also as empirical as the previous model and also strongly inspired from that latter. I never found a rigorous proof of this model. It is therefore probably more mathematical engineering that pure mathematics strictly speaking, but as it is a classic in statistical softwares, we will focus on it a little bit.

	Peter Winters (Charles C. Holt's student) has based it's work in the years 1960 on the method of his teacher to define a model that takes into account the seasonal component. The idea is to use three smoothing equations: one for the level of demand (without seasonality), one for the trend and one for seasonality.
	\begin{tcolorbox}[title=Remark,colframe=black,arc=10pt]
	There are two methods Winter's method according to that seasonality is additive or multiplicative. We present here only the multiplicative model because it is the one most commonly used. And we can not present all the existing models as they are all empirical and there are probably hundreds of them! 
	\end{tcolorbox}
	Let $N$ be the number of periods in each seasonal cycle (cycle assumed to be constant!). Three exponential smoothing equations are used in each period to update the estimates of series (the part without seasonilty), seasonal factors and trend. These equations  have different smoothing constants traditionally denoted $\alpha$, $\beta$ and $\gamma$:
	\begin{enumerate}
		\item The first equation is defined quite naturally by:
	
	where $Y_t$ is then the current level of the series without the seasonality effect (the we speak sometimes of  "adjusted series") thanks to the appropriate seasonal factor $C_{t-N}$ (otherwise we fall back on the Holt's method when $C_{t-N}=1$).

		\item The second equation is naturally defined by:
		
		where we we see again the slope of the trend identical to Holt's model.

		\item The third equation is completely new, this does not prevent prevent that it is quite intuitive and is given by:
		
		and finally the forecast is seasonlized again (it is the form of the forecast that makes that we speak of "multiplicative model"):
		
		This equation assumes that $j<N$. If $N<j<2N$, the appropriate seasonal factor. If equation, the equation would be $C_{t+j-2N}$. If $2N<j<3N$, the appropriate seasonal would be $C_{t+j-3N}$ and so on.
	\end{enumerate}
	So in the end, we have the system:
	
	Also sometimes denoted:
	
	We can found also the following variation of the previous model in some books:
	
	Inspired by the additive Holt's model, the Holt \& Winter additive model is given by:
	
	The difference between the multiplicative and additives methods is in the nature of the seasonal component. The additive method is preferred when the variation of seasonal component is almost constant through the series. Seasonal component is expressed in absolute terms and sum up to zero. While multiplicative is used when seasonal variations changes proportionally to the level of the series. 
	
	To start the forecast with the multiplicative model, we also need to estimated the initial parameters. Again, first forecasted values will be greatly affected by these estimates. The practitioner should as always check that what method its software for the determination of the parameters to specify it in his reports.
	
	\pagebreak
	\begin{tcolorbox}[colframe=black,colback=white,sharp corners]
	\textbf{{\Large \ding{45}}Example:}\\\\
	 Let us consider the following list in Microsoft Excel  14.0.6123 that contains the observed data in column \texttt{D} and the model in the columns \texttt{E} to \texttt{H} which we will detail the formulas just after (here we have the $N$ of our above system that is then equal to zero). Notice that we have $6$ years with each $4$ quarters (so there will be verbatim $4$ seasonal factors that are the first $4$ rows of column G):	 
	 \begin{figure}[H]
		\centering
		\includegraphics[scale=0.55]{img/economy/forecast_holt_winter_list_excel_and_parameters.jpg}
		\caption[]{Original data and parameters for Holt \& Winter's model with Microsoft Excel 14.0.6123}
	\end{figure}
	and explicitly for the equations and errors:
	\begin{figure}[H]
		\centering
		\includegraphics[scale=0.4]{img/economy/forecast_holt_winter_list_excel_list_formulas.jpg}
		\caption[]{Explicit formulas for data and error for Holt \& Winter's model with Microsoft Excel 14.0.6123}
	\end{figure}
	and for the parameters:
	\end{tcolorbox}
	
	\begin{tcolorbox}[colframe=black,colback=white,sharp corners]
	\begin{figure}[H]
		\centering
		\includegraphics[scale=0.8]{img/economy/forecast_holt_winter_list_excel_list_formulas_parameters.jpg}
		\caption[]{Explicit formulas for parameters and error indicators model with Microsoft Excel 14.0.6123}
	\end{figure}
	As for all the models seen previous, we start the solver with the idea to minimize the MAPE:
	\begin{figure}[H]
		\centering
		\includegraphics[scale=0.75]{img/economy/forecast_holt_winter_list_excel_list_final_values_parameters_and_errors.jpg}
		\caption[]{Holt \& Winter's final parameters and error values with Microsoft Excel 14.0.6123}
	\end{figure} 
	This gives for values:
	\begin{figure}[H]
		\centering
		\includegraphics[scale=0.7]{img/economy/forecast_holt_winter_list_excel_list_final_values_forecasts_errors.jpg}
		\caption[]{Holt \& Winter's final values, forecasts and errors with Microsoft Excel 14.0.6123}
	\end{figure}
	\end{tcolorbox}
	
	\begin{tcolorbox}[colframe=black,colback=white,sharp corners]
	Graphically this gives if we compare all:
	\begin{figure}[H]
		\centering
		\includegraphics{img/economy/forecast_holt_winter_final_plot.jpg}
		\caption{Holt \& Winter's full plot and comparison with other models with Microsoft Excel 14.0.6123}
	\end{figure}
	Holt's smoothing therefore had a MAPE of $7.49\%$ and the Brown's one a MAPE of $7.45\%$. The Holt \& Winter's seasonal smoothing has a MAPE of $9.61\%$, that is to say the fitting is worst that the other models. However the advantage of Holt \& Winter's model is that we can make projections over more than one period AND with cycles as we can see on the chart above.\\
	
	It is of course interesting to compare the $6$ months forecast values between the Brown's, Holt's additive and Holt \& Winter's model:
	\begin{table}[H]
	\begin{center}
		\definecolor{gris}{gray}{0.85}
			\begin{tabular}{|c|c|c|c|c|}
				\hline
  \multicolumn{1}{c}{\cellcolor{black!30}\textbf{Month}}  & 
  \multicolumn{1}{c}{\cellcolor{black!30}\textbf{Brown's DSES}} & 
  \multicolumn{1}{c}{\cellcolor{black!30}\textbf{Holt's DSES additive}} & 
  \multicolumn{1}{c}{\cellcolor{black!30}\textbf{Holt \& Winter's multiplicative}}\\ \hline
				$25$ & $261.50$ & $263.30$ & $249.42$ \\ \hline
				$26$ & $268.19$ & $270.00$ & $270.85$ \\ \hline
				$27$ & $264.55$ & $276.69$ & $292.96$ \\ \hline
				$28$ & $280.53$ & $283.39$ & $258.17$\\ \hline
				$29$ & $238.20$ & $290.08$ & $265.60$ \\ \hline
				$30$ & $311.19$ & $296.78$ & $265.60$\\ \hline
		\end{tabular}
	\end{center}
	\caption{Comparison between Holt's and Brown's DSES + Holt \& Winter's multiplicative TSES}
	\end{table}
	So we see again that depending on the chosen technique, the differences are relatively important.
	\end{tcolorbox}
	Don't forget that there is also an additive Holt \& Winter's model. Furthermore we have see once that there is also a "Holt \& Winter's multiple model" that is not the same as the "Holt \& Winter's multiplicative model"!
	
	\pagebreak
	\paragraph{Logistic Model}\mbox{}\\\\
	It always happens in the corporations that in the analysis of a product or service over a given period of time that the quantities of sales of the latter increase, then go through an inflection point (second derivative equal to zero as seen at page \pageref{inflection point}) and then go towards an asymptote to decrease again later with a more or less similar characteristic.

	The logistic model adapted to simulate such behavior as part of the field of time series analysis (and that must not be confused with the logistic regression as seen in the section of Numerical Methods) is defined as:
	
	The logistic model is inspired from many models that we find in physics and $\hat{Y}_{\max}$ is the estimated saturation threshold (horizontal asymptote) which can be approximated (with a tolerance interval) by an audit of the market.
	\begin{tcolorbox}[title=Remark,colframe=black,arc=10pt]
We often speak about 3M (Mathematics Method for Management) or even sometimes "scientific management" to describe the set of mathematical tools applied to management (there exists also a training curriculum on the subject of fifty days...). Some people also use the abbreviation "SciProM" for "Scientific Project Management). A common English term that became fashionable in Europe to describe this area of application is also the "\NewTerm{Decisioneering}" referring to the fact that these are tools decision support for engineers. 
	\end{tcolorbox}
	It mus be known that in such a market dynamics this model is obviously much better than that used by the Exponential smoothing\footnote{i.e. simple exponential smoothing} option included in the Microsoft Excel 14.0.6123 Data Analysis Toolpack (obviously as we have seen it mathematically previously, they absolutely do not work on the same principles). A simple comparative observation of the results is enough to realize it.
	
	The parameters $b$ and $r$ are two parameters of the model such as:
	
	the inflection point is always given the cumulated $50\%$ of saturation value $\hat{Y}_{\max}$. The result is then a S-curve of the following type (we will see where it comes from further below):
	\begin{figure}[H]
		\centering
		\includegraphics{img/economy/forecast_logistic_s_curve_characteristic.jpg}
		\caption{Logistic S-curve characteristics with Microsoft Excel 14.0.6123}
	\end{figure}
	where in blue has been represented the actual data of a product and in blue the associated estimated theoretical model with its forecast and asymptotic value.

	To determine the equation of the logistic model, we can directly use the solver of some softwares. But they may need to have the initial data close to the theoretical value. We will first show how these starting values can be determined with an example.

	Let us consider the following first $24$ rows of list in Microsoft Excel 11.8346 (the sales are in units of hundreds of thousands):
	\begin{figure}[H]
		\centering
		\includegraphics[scale=0.8]{img/economy/forecast_logistic_model_initial_values.jpg}
		\caption[]{Logisttic initial values in Microsoft Excel 14.0.6123}
	\end{figure}
	and the associated graph:
	\begin{figure}[H]
		\centering
		\includegraphics{img/economy/forecast_logistic_model_initial_plot.jpg}
		\caption[]{Logistic initial values plot with Microsoft Excel 14.0.6123}
	\end{figure}
	which could be considered as linear (locally) following the time at which begins the descriptive analysis of the sales.

	To determine the theoretical model, we will linearize the logistic equation using a hypothetical threshold (market sales targets) of $\hat{Y}_{\max}=800$.

	So:
	
	That is to say we have the new following variable to explain:
	
	And for which we will write the corresponding linear model:
	
	with therefore:
	
	Finally injecting in the initial expression we get:
	
	So we see here that we fall back typically on an expression similar to that of the Logistic regression as seen in the section of Numerical Methods.

	In our example the linear regression (\SeeChapter{see section Numerical Methods page \pageref{univariate linear regression gaussian model}}) gives:
	
	That is:
	
	Then we have immediately:
	
	That gives graphically:
	\begin{figure}[H]
		\centering
		\includegraphics{img/economy/forecast_logistic_forecast_values_and_analytical_forecast.jpg}
		\caption[]{Logistic initial values and analytical forecast plot with Microsoft Excel 14.0.6123}
	\end{figure}
	with this analytical approach, we have a sum of squared error (SSE) between the real values and the model equal to:
	
	Now let enter that data into Microsoft Excel 14.0.6123 as follows:
	\begin{figure}[H]
		\centering
		\includegraphics[scale=0.6]{img/economy/forecast_logistic_list_excel_real_analytical_model_solver_model_initial_values.jpg}
		\caption[]{Logistic initial values and analytical/solver forecast with Microsoft Excel 14.0.6123}
	\end{figure}
	Thus explicitly:
	\begin{figure}[H]
		\centering
		\includegraphics[scale=0.6]{img/economy/forecast_logistic_list_excel_real_analytical_model_solver_model_explicit_formulas.jpg}
		\caption[]{Logistic initial values and analytical/solver forecast explicit formulas with Microsoft Excel 14.0.6123}
	\end{figure}
	If we run the Microsoft Excel 14.0.6123 Solver  with the following parameters (we have put $0.0001$ as lowest value since the solver does not offer a strict order relation):
	\begin{figure}[H]
		\centering
		\includegraphics[scale=0.8]{img/economy/forecast_logistic_parameters_optimization_solver_excel.jpg}
		\caption[]{Logistic parameters optimization with Microsoft Excel 14.0.6123}
	\end{figure}
	Which gives:
	\begin{figure}[H]
		\centering
		\includegraphics{img/economy/forecast_logistic_optimized_parameters_with_excel_solver.jpg}
		\caption[]{Optimized logistic parameters with Microsoft Excel 14.0.6123}
	\end{figure}
	Therefore:
	
	with:
	
	What is quite good with the solver approach is that we get $\hat{Y}_{\max}$ and this in a very important information for trendy product type of marketing and business analysts!!!

	As we can see with the solver approach the SSE is quite much lower that the method using linear regression method. Indeed, let us see the following list:
	\begin{figure}[H]
		\centering
		\includegraphics[scale=0.8]{img/economy/forecast_logistic_analytical_vs_solver_values.jpg}
		\caption[]{Logistic Analytical vs Solver values with Microsoft Excel 14.0.6123}
	\end{figure}
	Which gives visually:
	\begin{figure}[H]
		\centering
		\includegraphics{img/economy/forecast_logistic_model_final_plot_values_analytical_model_solver_model_excel.jpg}
		\caption{Logistic Analytical vs Solver plot with Microsoft Excel 14.0.6123}
	\end{figure}
	We see clearly that the solver using the solver (numerical model) is better than the analytical model given indirectly by the linear regression.

	To end this part on deterministic forecasting models, remember that they must be monitored regularly to ensure that the model and parameters used are still appropriate. As the reader may have guessed, a forecasting model must not have bias. The sum of the errors must be near zero, with sometimes overestimates and underestimates values. When a model tends to always overestimate (or underestimate), the latter has a bias and it must be revised (to monitor this we can use the control chart technics as study in the section of Industrial Engineering).
	
	It is also necessary for the conclusion to remember that there is sometimes a significant difference between the observed values (actual values) and the observed that would really occur if the supply (offer) was infinite. Indeed, observed sales do not take into account missed sales  because lack of stock in one or many of the thousands of subsidiaries that have your company, and therefore the possible option that a customer went buy at competitor product. Thus, sales observed are most of times a "minimum" because of missed opportunities!
	
	\begin{tcolorbox}[title=Remark,colframe=black,arc=10pt]
	Let us indicate that it is trivially possible to make the difference of all subsequent points in time of a time series and then to make a histogram to determine the probability distribution of changes, allowing to make statistical inference with all necessary precautions (analysis that should always be done in practice!). 
	\end{tcolorbox}
	
	\subsubsection{Autoregressive Models}
	In statistics and signal processing, an autoregressive (AR) model is a representation of a type of random process; as such, it describes certain time-varying processes in nature, economics, etc. The autoregressive model specifies that the output variable depends linearly on its own previous values and on a stochastic term (an imperfectly predictable term); thus the model is in the form of a stochastic difference equation.

	Together with the moving-average (MA) model, it is a special case and key component of the more general ARMA and ARIMA models of time series (see further below), which have a more complicated stochastic structure; it is also a special case of the vector autoregressive model (VAR) as we will see further below, which consists of a system of more than one stochastic difference equation.

	In the study of a time series, it is natural to think that the series value at time $t$ may depend in part on the values taken on the previous dates such that:
	
	which of course is for now a limited approach to only discrete univariate series (we will study multivariate series further below)  and for which the sampling frequencies constant (at the opposite of what is is done in some areas of financial transactions).
	
	Autoregressive models have become such common that they are now integrated in Microsoft Excel 2016:
	\begin{figure}[H]
		\centering
		\includegraphics{img/economy/forecast_excel_2016_native_tool.jpg}
		\caption[]{Microsoft Excel 2016 SARIMA forecast native tool button}
	\end{figure}
	that typically gives for result:
	\begin{figure}[H]
		\centering
		\includegraphics[scale=0.85]{img/economy/forecast_excel_2016_native_tool_example.jpg}
		\caption{Microsoft Excel 2016 forecast native tool demo}
	\end{figure}

	Before focusing on AR models remember that we saw in the section Probabilities that when we focus only a dependency at time $t-1$, then we speak of "Markov chain", or more rigorously of "order $1$ Markov chain" and... let us open a small parenthesis on the interest of Markov chains (recall oriented for the finance field)... 
	
	Let us consider that we have sampled stock prices at a fixed frequency (eg daily or hourly ... whatever!). It is then very easy to calculate the proportion of increases ($A$) versus that of Decreases ($D$) of the values of interest. Thus, taking into account the whole set of history, we will typically have a table of the following type (values derived from the S\& P500 indices between 1947 and 2007):
	\begin{table}[H]
		\begin{center}
		\definecolor{gris}{gray}{0.85}
		\begin{tabular}{|c|c|c|}
		\hline
		{\cellcolor{black!30}\parbox{4.5cm}{\textbf{Proportion Increases}\\ \centering \textbf{$(A)$}}} & 
		{\cellcolor{black!30}\parbox{4.5cm}{\textbf{Proportion Decreases}\\ \centering \textbf{$(D)$}}} & 
		{\cellcolor{black!30}\textbf{Total}}  \\ \hline
		$0.474$ & $0.526$ & $1$ \\ \hline
		\end{tabular}
		\end{center}
		\caption{Proportion of $A$/$D$ variations on a history set}
	\end{table}
	But this is not quite useful excepted perhaps to run a hypothesis test on the both proportion ... If we complicate a little bit the analysis by asking us how much we have times we have two successive increases ($AA$), successive decreases ($DD$) or alternating events ($DA$) and ($AD$) throughout history this doubles the work and could give us a typical table of the following type:
	\begin{table}[H]
		\begin{center}
		\definecolor{gris}{gray}{0.85}
		\begin{tabular}{|c|c|c|c|}
		\hline
		{\cellcolor{black!30}$X_i=$} & 
		{\cellcolor{black!30}$(A)$} & {\cellcolor{black!30}$(B)$} & 
		{\cellcolor{black!30}\textbf{Total}}  \\ \hline
		{\cellcolor{black!30}$X_{i-1}=(D)$} & $0.519$ & $0.481$ & $1$ \\ \hline
		{\cellcolor{black!30}$X_{i-1}=(A)$} & $0.4.33$ & $0.567$ & $1$ \\ \hline
		\end{tabular}
		\end{center}
		\caption[]{Proportion of $A$/$D$ variations relatively to previous period}
	\end{table}
	So the above values can obviously be seen as conditional probabilities:
	
	And if now we do an analysis not only on the previous period but the last two, it means that we calculate the proportions of the sequences $(DDD)$, $(DDA)$, $(DAD)$, $(DAA)$, $(ADD)$, $( ADA)$, $(AAD)$, $(AAA)$ and it doubles again once the number of calculations:
	\begin{table}[H]
	\begin{center}
		\begin{tabular}{|c|c|c|c|c|}
			\hline
			\cellcolor{black!30}{} & \cellcolor{black!30}{} & \multicolumn{2}{|c|}{\cellcolor{black!30}$X_i=0$} & \cellcolor{black!30}{} \\
			\hline
			\cellcolor{black!30}$X_{i-1}$ & \cellcolor{black!30}$X_{i-1}$ & \cellcolor{black!30}$(A)$ & \cellcolor{black!30}$(D)$ & \cellcolor{black!30}\textbf{Total}\\
			\hline
			\cellcolor{black!30}$(D)$ & \cellcolor{black!30}$(D)$ & $0.501$ & $0.499$ & $1$ \\ \hline
			\cellcolor{black!30}$(D)$ & \cellcolor{black!30}$(A)$ & $0.412$ & $0.588$ & $1$ \\ \hline
			\cellcolor{black!30}$(A)$ & \cellcolor{black!30}$(D)$ & $0.539$ & $0.461$ & $1$ \\ \hline
			\cellcolor{black!30}$(A)$ & \cellcolor{black!30}$(A)$ & $0.449$ & $0.551$ & $1$ \\ \hline
		\end{tabular}
		\caption{Proportion of $A$/$D$ variations relatively to $2$ previous periods}
	\end{center}
	\end{table}
	The above values must obviously be seen as conditional probabilities:
	
	We see that each time we add an additional period to the analysis, we double the number of calculations. Thus, taking the last $20$ periods, this leads us to nearly $1$ million conditional probabilities (proportions). So we understand then the interest to restrict ourselves to a small number of previous values or a single value as do the Markov chains. 

	Coming back to our example, we have then that the probability that the next $4$ days the sequence is $(ADAA)$ will be without simplifying assumption:
	
	but will be reduced in the hypothesis of a Markov chain (that is to say a dependency only with the previous period) to:
	
	Whether the approach of total conditional probabilities or the simplified Markov chain approach is the best is a broad debate and should be treated case by case. The hypothesis of random walk (implicitly: the Markov model) is defended by a majority of economists ... but this hypothesis obviously also a number of opponents.
	
	It is therefore not usually necessary to take into account all the past values of series or only the last event and we often limit ourselves to $p$ values:
	
	where $\varepsilon_t$ is a white noise $\mathcal{N}(0,\sigma)$  or Wiener process sometimes denoted WN. More rigorously, and for reasons of application of statistical techniques, the white noise is defined such that:
	
	The first equality is obvious. The third one is required a constraint. The second has not been proved in the section Statistics. So let us prove this now!
	
	So remember that in the section Statistics we have proved the Huyghens theorem:
	
	Solving for the needed quantity gives:
	
	But four our case $\text{E}(X)=0$, therefore:
	
	Hence:
	
	The only difference relatively to the standard Brownian motion is that as we will see further below, there is the presence here of an inertia factor traditionally denoted  $\alpha$ in the simple cases, that strongly influences the dynamics of the process such as:
	
	Indeed, as it is very easy to do in Microsoft Excel in accordance with the procedure indicated in our study of Wiener process earlier in this section.

	Here are various plots of the time series according to some values of the inertia factor:
	\begin{figure}[H]
		\centering
		\includegraphics{img/economy/white_noise.jpg}
		\caption{Autoregressive process of order $1$}
	\end{figure}
	
	If we consider $X_t$ as a specific random variable (with a given density function) that we would denote by $X$ and $X_{t-h}$ as another specific random variable (with a given density function) that we would denote by $Y$, then nothing prevents us being known the density functions of each of these variables to compute their covariance denoted most of times $\gamma_t(h)$ in the field of time series analysis:
	
	For example, in practice we often know the mean $\mu$ of the two random variables at the two different times and also some of the values of their underlying distributions (random issues). So, it becomes easy to calculate their covariance. But this is not really a useful indicator. Nothing prevents us assuming a linear relation to use the linear correlation coefficient:
	
	But that is written traditionally and trivially  in the field of time series analysis:
	
	and is named "\NewTerm{autocorrelation coefficient}" often abbreviated in AFC in statistical software.
	\begin{tcolorbox}[title=Remark,colframe=black,arc=10pt]
	There are different variations of this relation in some statistical software. Therefore we get sometimes small numerical differences with respect to the use of the above ACF analysis.
	\end{tcolorbox}
	Personally I prefer to write the above relation in the following explicit form (the covariance is explicit):
	
	The reader should know that when we do the analysis of time series in practice, we often compare a series with itself but with a time lag $h$. The autocorrelation of order $h$ is then the correlation between the series and itself delayed by $h$ lags.

	Thus, as both series are dependent and have the same statistical characteristics, we write in this special case say to be a "\NewTerm{second order stationary process}" the following relation named the "\NewTerm{empirical autocorrelation coefficient}":
	
	where obviously we must not forget that in practice the mean and the standard deviation are only estimators although this is not explicit in the above definition since the in the field of time series analysis we indicate rarely the estimators with hat symbol as we do in Statistics or Reliability analysis.

	Obviously in the case of the analysis of time series where we suspect cyclicality, the largest value $\rho$ for a given $h$ indicates the underlying cyclicality frequency (respectively the periodicity) of the series which can help by for example the choice of the type of moving average (see further below).
	
	\textbf{Definition (\#\mydef):} We name "\NewTerm{correlogram}", the diagram representing the autocorrelation coefficients of order $1, 2, 3, \ldots, h,\ldots$. series.
	
	Below are given for example a family of time series:
	\begin{figure}[H]
		\centering
		\includegraphics{img/economy/times_series_for_afc_example.jpg}
		\caption{Some examples of time series}
	\end{figure}
	with their corresponding correlogram for different values of $h$ on the x-axis and the values of $\rho$ on the $y$-axis:
	\begin{figure}[H]
		\centering
		\includegraphics{img/economy/times_series_afc_examples.jpg}
		\caption{Corresponding AFC correlograms of the time series}
	\end{figure}
	
	Let us make a practical example of calculation of correlogram with Microsoft Excel 14.0.6123 and fictitious data:
	\begin{tcolorbox}[colframe=black,colback=white,sharp corners]
	\textbf{{\Large \ding{45}}Example:}\\\\
	Consider the following list of data:
	\begin{figure}[H]
		\centering
		\includegraphics[scale=0.8]{img/economy/afc_list_data_excel.jpg}
		\caption[]{Data list for AFC example and comparison with Minitab/R in Microsoft Excel 14.0.6123}
	\end{figure}
	with the following matrix formula in the cells D3 and below (sorry in did not found until now a rother way to do it):
	\begin{figure}[H]
		\centering
		\includegraphics[scale=0.65]{img/economy/afc_explicit_formula_excel.jpg}
		\caption[]{Autcorrelation formula in Microsoft Excel 14.0.6123}
	\end{figure}
	and visually this gives the following correlogram:
	\begin{figure}[H]
		\centering
		\includegraphics[scale=0.8]{img/economy/afc_correlogram_excel_plot.jpg}
		\caption[]{AFC Correlogram in Microsoft Excel 14.0.6123}
	\end{figure}
	\end{tcolorbox}
	\textbf{Definitions (\#\mydef):}
	\begin{enumerate}
		\item[D1.] A time series $X_t$ is said to be "\NewTerm{strictly stationary}" when the probability density function of the collection of the random variables $X_t$ is the same for any $t$.  The strict stationarity property restricts the probabilistic properties of any collection of random variables to be invariant to time shifts, which implies that the probabilistic behavior of each random variable is the same across time.
	
		\item[D2.] A time series is said to be "\NewTerm{stationnary in the weak sense}" or just "\NewTerm{weak stationary process}" or also "\NewTerm{second-order or covariance stationary}, if the first (mean) and second (variance) moments exist, are finite and constant over time (therefore not time-dependent), that is $\forall i\in 1\ldots T$:
		

		The weak stationarity property restricts the mean and variance of the time series to be finite and invariant in time (the first condition of constancy of the mean eliminates any possible trend) and the autocorrelation $\gamma_t(h)$ to be independent of time $t$ but only depending from the lag $h$.. 
		
		If the underlying density function of each $X_t$ is a Normal distribution, then we speak of "\NewTerm{Gaussian process}".
		
	Second order stationarity is weaker than strict stationarity. Second order stationarity requires that first and second order moments (mean, variance and covariances) are constant throughout time and, hence, do not depend on the time at which the process is observed. 
	
	Second order stationarity series are not necessarily strictly stationary because the mean and autocovariance are not, in general, enough to determine the distribution. 

	\item[D3.] If we have $N$ non-stationary time series and that their weighted sum (the sum of the weights $w_i$ being equal as usual unit) is such that  $\forall i\in 1\ldots T$:
		
	\end{enumerate}
	gives a weak stationary process, then we say that the time series are "\NewTerm{co-integrated}". This is useful in finance so that with some non-stationary financial instruments we can produce a portfolio whose time series is weakly stationary (for mathematical processing reasons). 
	
	Let us consider an important case in many companies on some more or less long periods. Given:
	
	the time average. If $\bar{X}_t$ converges in probability to $\mu$ when $T \rightarrow +\infty$ we say that the process is "\NewTerm{ergodic for the average}".
	
	\paragraph{AR($p$) Autoregressive processes}\mbox{}\\\\
	Let us introduce the concept of autoregressive process by introducing the particular case of an autoregressive process of order $1$, denoted by AR($1$) and defined by the relation (up to an additive constant):
	
	It is therefore a Markov chain of order $1$ (we can always verify in practice if the resulting model is a posteriori with or without gaussian residues!).

	We say that the autoregressive AR($1$) is stationary if we have for any time $t$:
	
	More generally, we have:
	
	and therefore the stationary necessarily implies that (remember that the additive constant was put as zero in the above definition, otherwise it changes the conclusions below):
	
	Then stationary implies the two useful results:
	
	It is interesting to notice that the variance is independent of the time and therefore it does not diverge. However, we must not necessarily consider see this a positive fact because it is not very consistent in practice that the volatility is constant (homoscedastic process). This is why we will address more complex models whose volatility is time-dependent.

	Notice also that for the variance to be always positive (by definition of the variance) we must have $\varphi<1$.

	Let's see what the hypothesis of stationarity involves and for this, we'll first need the following equality:
	
	Therefore:
	
	And let us recall the Huygens theorem (\SeeChapter{see section Statistics page \pageref{huygens relation}}):
	
	Therefore it comes:
	
	So the autocorrelation is indeed dependent only of the lag $h$ and given by:
	
	Since the stationarity implies that $\varphi \ll 1$, the ACF correlograms of AR($p$) processes will have a decreasing profile: the process somehow forgets past values.

	We also have for an AR($1$) stationary process with constant (offset):
	
	the following mean:
	
	Therefore:
	
	So if the constant is equal to zero we fall back well on a mean equal to zere as proved earlier above. The variance will have the same expression as already the one proved earlier above, that is to say:
	
	since the addition of a constant does not change the variance of the series. So we got above the mean and (unconditional) variance of an AR($1$) process.
	
	If $\varphi=1$ then we have the AR($1$) process which is written:
	
	which is then by construction not stationary anymore. Nonstationarity is common in practice and bring to analysis issues. There is therefore a trick to make some stationary some non-stationary series, do manipulations on them and then to make them again nonstationary by the reverse procedure.
	
	Thus, about the AR($1$) with $\varphi=1$, the idea is to use what we name the "\NewTerm{integration property}" of a time series. Thus, AR($1$) is an integrated series of order $1$, designated by I($1$) because if it differs it from a time unit, we get a stationary series:

	Which is written with the "\NewTerm{lag operator $L$}":
	
	And so a series is "integrated of order $2$" if it needs to be "differentiate" twice to make it stationary.
	
	An AR($2$) process is an auto-regressive process that verifies a relation of the following (up to an additive constant):
	
	So in such a model, the influence of the past is manifested by a linear regression on the two previous values. Depending on the values of $\varphi_1$ and $\varphi_2$ it is not always possible to find a stationary process satisfying this last relation.

	In general, an AR($p$) process is a process that depends linearly on the $p$ previous values (up to an additive constant):
	
	
	\paragraph{MA($q$) Moving Average stationary process}\mbox{}\\\\
	First a disclaimer!: Do not confuse this Moving Average stationary process with deterministic Moving Averages. Indeed, the term is not very appropriate because even if there is indeed mobility (which is the minimum for a process), it is quite wrong to speak of average...

	We are in fact in the context of times series forecasts that show autocorrelated errors but nothing else (in fact it will serve for a mix a little further below).

	Let us introduce the concept of autoregressive moving average by studying first the special case of autoregressive moving average of order $1$, denoted MA($1$) and defined by the:
	
	For such a process, we have by definition of a white noise:
	
	So the two moments are both time independent and we are dealing with a weak stationary process. However, we must not necessarily see this as a positive fact because it is not very consistent in practice that the volatility is constant (homoscedastic process). This is why we will address more complex models whose volatility is time-dependent further below.
	
	Regarding the autocorrelation, we have first (do not forget to review the three conditions which define the white noise and especially the third one!):
	
	The order $1$ autocorrelation coefficient is therefore equal to:
	
	For $h>1$, we have immediately by doing the same development as:
	
	Therefore the autocorrelation coefficients of order greater than $1$ are all zero!

	A MA($2$) process is an AR process that verifies a relation of the form:
	
	And we can repeat the same developments as for MA($1$) with enthusiasm ... to arrive also to the conclusion that the autocorrelation (ACF) is zero for $\forall h>2$.

	In general, a MA($q$) processes a process which depends linearly on the previous $q$ values:
	
	and thus for such a process it can be accepted without the general proof that:
	
	and that the correlation is zero for $\forall h>q$. Thus, one can easily recognize that a MA($q$) process since its ACF has a "cut-off" from $q$.
	
	\pagebreak
	\paragraph{ARMA($p,q$) Autoregressive non-seasonal moving average processes}\mbox{}\\\\
	We can obviously consider combining the two previous models (MA($q$) and AR($p$)) by introducing:
	\begin{itemize}
		\item A dependency of the process vis-à-vis its past with an AR($p$)

		\item A delay the random errors with a MA($q$) process
	\end{itemize}
	Such a model, named "\NewTerm{non-seasonal autoregressive mobile average}" or "\NewTerm{non-seasonal ARMA}", is characterized by the parameter $p$ of the autoregressive part and the parameter $q$ of the moving average part.

	Thus, an non-seasonnal ARMA($p, q$) process verifies the relation:
	
	The main advantage of non-seasonal ARMA process is that they allow to provide long-term forecasts (at least more distant in time as the next date...). Indeed, recall that for AR (1) process, we obtained:
	
	Then we have:
	
	Therefore:
	
	The forecasts are then obtained recursively:
	
	And therefore in a more general way:
	
	and so when $k$ becomes very large, then we have:
	
	In other words, for a high order $k$, the AR($1$) model provides as conditional forecasting mean... the mathematics average (history) of the process. When the conditional mean is zero, then we say in the field of time series analysis that the "\NewTerm{mean is orthogonal to any past}".
	
	So as we can see it, the ARMA process has a mean that has an exponentially decreasing type of memory because of the factor $\varphi$ that is at the power of $k$ in prior-previous relation. While an MA  process type that has a function that cut the temporal memory to a value $q$ (cut-off of $q$ periods). These two processes are therefore considered as short memory process.
	
	\paragraph{ARIMA($p,d,q$) Autoregressive non-seasonal integrated moving average processes}\mbox{}\\\\
	The time series models view previously are in reality just special cases of a more general empirical model that is named "\NewTerm{AutoRegressive Integrated Moving Average (ARIMA)}" also known as "\NewTerm{Box-Jenkins models}". 

	This model assumes that the time series is generated by a random process (Brownian or other) and that future values are related to past errors.
	
	However, this model has an important limitation, and the fact that we find through this generalization the exponential smoothing means that they also have the same limitation: the time series must be stationary. So to recap, this means that moments like mean or variance and correlation must be constant over time.
	\begin{tcolorbox}[title=Remark,colframe=black,arc=10pt]
	In 1970 George Box and Gwilym Jenkins popularized ARIMA (Autoregressive Integrated Moving Average) models in their seminal textbook, \textit{Time Series Analysis: Forecasting and Control}. Technically, the forecasting technique described in the text is an ARIMA model, however many forecasters (including the author) use the phrases "ARIMA models" and "Box-Jenkins models" interchangeably. 
	\end{tcolorbox}
	An ARIMA model is often classified under the notation:
	
	where $p$ is the number of autoregressive terms, $d$ the number of non-saesonals differences (order of intergration) and $q$ the number of terms of errors (moving average) in the prediction equation. 
	
	Specifically for a model as $\text{ARIMA}(1, 0, 12)$ the trick to remember the parameters is to see the content of the parenthesis as:
	
	Here are some examples:
	\begin{itemize}
		\item Average: ARIMA$(0,0,0)$ with constant
							
		\item Drift: ARIMA$(0,1,0)$ with constant
			
		\item ARMA: ARIMA$(1,0,1)$
		\item Single Exponential Smoothing: ARIMA$(0,1,1)$
		\item Double Exponential Smoothing Holt additive: ARIMA $(0,2,2)$
			
		\item Holt \& Winters Exponential Smoothing (multiplicative): no ARIMA equivalent to our knowledge 
		\item Holt \& Winters Exponential Smoothing (additive) : SARIMA$(0.1, m + 1)(0,1,0)$
	\end{itemize}
	Let us come back on projection techniques (forecasting) of ARIMA models. Let's start with a simple special case (that is in our opinion more educational) considering the process denoted AR$(1)$ and defined for recall by the relation (sum up over a given additive constant):
	
	When we make a punctual estimate of a projection of this model, on $l$ subsequent units of time, it is customary to write the punctual estimate value by different usual notations:
	
	where $\varphi$ is obviously supposed to have been determined with known historical values. However, the projected point estimate is an precisely an... estimate... !!!! We would have to know its mean. The simplest naive assumption is to use the conditional expectation of the forecast value, who led us for recall (see just above) to the result:
	
	which tends to $0$ as $l$ approaches infinity as we have already proved it. But remember that this AR$(1)$ is defined up to an additive constant (which is often the arithmetic mean or the median of known values of the time series), so in the general case it tends towards a constant!!!
	
	On the way, notice that we explicitly the forecast error that is given by:
	
	And let us recall that we have proved earlier above that for AR$(1)$:
	
	And let us consider $h$ (respectively $l$) big and $|\varphi|<1$, then we have:
	
	Adapted to our the notation of our book, this gives:
	
	Given an MA$(\infty)$ process. Then:
	
	We notice that the forecast error $e_t(l)$ is a process of the type process MA$(l-1)$. Then it comes immediately form this last development:
	
	where the sum of the square has been simplified using a result proved in the section Sequences And Series. Thus we see that the variance of the error  forecast does not diverge (at least in the case of an AR (1)) and tends to a finite value (which is not the case of all processes!) when $l$ approaches infinity provided that $\varphi<1$. The difficulty in practice is to select a technique to determine $\sigma^2$ (there are a quantity of such techniques: adequacy, Monte Carlo, bootstrapping, Bayesian inference, etc.). This is why the practitioner should never limit himself to the use of a single statistical software to make projections, but at least three.
	
	Why we speak again of that subject will you think maybe ??? For the simple reason that we always need a statistical prediction interval!!! Thus, given that we have a point estimator $X_{t+1}$ which is the realization of the random variable of the forecasted value and its associated mean value $\hat{X}_{t+1}$, we can write:
	
	and assume from the previous developments that the forecast error follows a Normal centered distribution:
	
	We then have a given confidence level if $\sigma^2$ is completely determined:
	
	and if we have only an estimator of $\hat{\sigma}^2$ then we will take (\SeeChapter{see section Statistics page \pageref{student confidence interval of the mean}}):
	
	
	The asymptotic shape of the error term is quite well know by practitioner as it is of the form with Maple 4.00b:
	
	\texttt{>plot([sqrt((1-0.9\string^(2*l))/(1-0.9\string^2)),sqrt((1-0.2\string^(2*l))/(1-0.2\string^2))],l=0..20);}
	\begin{figure}[H]
		\centering
		\includegraphics[scale=0.5]{img/economy/arima_error_convergence_maple.jpg}
	\end{figure}
	Practicing these forecasting model show they tend to underestimate the risk (variance). So rather than taking a $95\%$ interval it is perhaps better to take a higher range depending on the return on experience.
	
	Here are some example for a same set of data of the different forecast we can have depending on the chosen model:
	\begin{figure}[H]
		\centering
		\includegraphics[scale=0.8]{img/economy/sample_temporal_series_with_forecasting.jpg}
		\caption{Some time series with forecasting (GARCH, ARFIMA, Holt-Winter, ARIMA, etc.) with R}
	\end{figure}
	
	\subsubsection{Durbin-Watson autocorrelation test}
	The "\NewTerm{Durbin-Watson test}\index{Durbin-Watson test}" also named "\NewTerm{Durbin-Watson statistics}\index{Durbin-Watson statistics}" consists in testing the null hypothesis that the autocorrelation between the residuals of a regression $e_i$ is zero (against the alternative hypothesis that this same correlation is non-zero). That is to say:
		
	Even if this test is used a lot in linear or non-linear regression to check if the model assumption are satisfied (so it could have its place in the section of Numerical Methods), it is used much more in time series analysis where the practitioner is most often more skilled than people doing regression in corporations and therefore more apt to read the result of the test and perform the necessary corrective actions.

	The idea of this test is to write (read well until the end before thinking that the choice of the ratio is empirical!):	
	
	For the latter term:
	
	notice that we can write it in the following form since the expected mean of the residues are assumed to be zero (therefore this writing is the same!) such as $\text{E}(e)=0$. Therefore:
	
	We thus fall back on the general expression of the autocorrelation seen earlier above. It then comes:
	
 	The approximation (valid only if $n$ is sufficiently large) that we have done above justifies the fact that some software offers an approximate or exact version.

	So far as I know there is not exact distribution know law that this statistics follows. Therefore it is tabulated with Monte Carlo methods. Simulations gives that when:
	\begin{itemize}
		\item $\rho$ is close to $1$, the DW statistic is close to $0$
		\item $\rho$ is close to $0$, the DW statistic is close to $2$
		\item $\rho$ is close to -1, the DW statistic is close to $4$
	\end{itemize}
	Upper and lower critical values, denoted $\text{DW}_U$ (or sometimes $Q_U$) and respectively $\text{DW}_L$ (or sometimes $Q_L$), have been tabulated for different values of $k$ (the number of explanatory variables also sometimes denoted $\Lambda$) and of $n$ (also sometimes denoted $N$) and the rules are:
	\begin{itemize}
		\item If $\text{DW}<\text{DW}_L$ we reject $H_0$
		\item If $\text{DW}>\text{DW}_U$ we cannot reject $H_0$
		\item If $\text{DW}_L<\text{DW}<\text{DW}_U$ the test is inconclusive
	\end{itemize}
	\begin{tcolorbox}[title=Remark,colframe=black,arc=10pt]
	In the R software (see the corresponding companion book) there are actually two functions to run this test. One written \texttt{durbinWatsonTest( )} from the \texttt{car} package and one written \texttt{dwtext( )} from the \texttt{lmtest} package. Even if they give veeeery close $p$-values and DW values for small values of $n$, they diverge quite significantly when $n$ becomes big. This is because of the chose algorithm. Indeed, the help associated with the  \texttt{dwtext( )} function gives us: \textit{The $p$-value is computed using a Fortran version of the Applied Statistics Algorithm AS 153 by Farebrother (1980, 1984). This algorithm is called "pan" or "gradsol". For large sample sizes the algorithm might fail to compute the $p$ -value; in that case a warning is printed and an approximate $p$-value will be given; this $p$ value is computed using a Normal approximation with mean and variance of the Durbin-Watson test statistic}. While the help associated with the \texttt{durbinWatsonTest( )} function gives us: \textit{simulate: if 'TRUE' p-values will be estimated by bootstrapping}.
	\end{tcolorbox}
	
	\pagebreak
	For the people that may not have access to a spreadsheet or statistical software here are some useful tables:
	\begin{center}
		BOUNDS FOR CRITICAL VALUES OF THE \\
		DURBIN-WATSON STATISTIC \\
		\ \\
		\begin{tabular}{rr@{\ }rr@{\ }rr@{\ }rr@{\ }rr@{\ }r}
		\multicolumn{11}{c}{1\% SIGNIFICANCE POINTS OF $Q_L$ AND $Q_U$} \\
		\ \\
		&\multicolumn{2}{c}{$\Lambda=2$}&\multicolumn{2}{c}{$\Lambda=3$}
		&\multicolumn{2}{c}{$\Lambda=4$}&\multicolumn{2}{c}{$\Lambda=5$}
		&\multicolumn{2}{c}{$\Lambda=6$}\\
		$N$ & 
		$Q_L$ & $Q_U$ & $Q_L$ & $Q_U$ & $Q_L$ & $Q_U$ & $Q_L$ & $Q_U$ & $Q_L$ & $Q_U$ \\
		\ \\
		 15&0.811&1.069&0.700&1.252&0.591&1.465&0.487&1.705&0.390&1.967\\
		 16&0.844&1.087&0.738&1.253&0.633&1.447&0.532&1.664&0.437&1.901\\
		 17&0.873&1.102&0.773&1.255&0.672&1.432&0.574&1.631&0.481&1.847\\
		 18&0.902&1.118&0.805&1.259&0.708&1.422&0.614&1.604&0.522&1.803\\
		 19&0.928&1.133&0.835&1.264&0.742&1.416&0.650&1.583&0.561&1.767\\
		 20&0.952&1.147&0.862&1.270&0.774&1.410&0.684&1.567&0.598&1.736\\
		 21&0.975&1.161&0.889&1.276&0.803&1.408&0.718&1.554&0.634&1.712\\
		 22&0.997&1.174&0.915&1.284&0.832&1.407&0.748&1.543&0.666&1.691\\
		 23&1.017&1.186&0.938&1.290&0.858&1.407&0.777&1.535&0.699&1.674\\
		 24&1.037&1.199&0.959&1.298&0.881&1.407&0.805&1.527&0.728&1.659\\
		 25&1.055&1.210&0.981&1.305&0.906&1.408&0.832&1.521&0.756&1.645\\
		 26&1.072&1.222&1.000&1.311&0.928&1.410&0.855&1.517&0.782&1.635\\
		 27&1.088&1.232&1.019&1.318&0.948&1.413&0.878&1.514&0.808&1.625\\
		 28&1.104&1.244&1.036&1.325&0.969&1.414&0.901&1.512&0.832&1.618\\
		 29&1.119&1.254&1.053&1.332&0.988&1.418&0.921&1.511&0.855&1.611\\
		 30&1.134&1.264&1.070&1.339&1.006&1.421&0.941&1.510&0.877&1.606\\
		 31&1.147&1.274&1.085&1.345&1.022&1.425&0.960&1.509&0.897&1.601\\
		 32&1.160&1.283&1.100&1.351&1.039&1.428&0.978&1.509&0.917&1.597\\
		 33&1.171&1.291&1.114&1.358&1.055&1.432&0.995&1.510&0.935&1.594\\
		 34&1.184&1.298&1.128&1.364&1.070&1.436&1.012&1.511&0.954&1.591\\
		 35&1.195&1.307&1.141&1.370&1.085&1.439&1.028&1.512&0.971&1.589\\
		 36&1.205&1.315&1.153&1.376&1.098&1.442&1.043&1.513&0.987&1.587\\
		 37&1.217&1.322&1.164&1.383&1.112&1.446&1.058&1.514&1.004&1.585\\
		 38&1.227&1.330&1.176&1.388&1.124&1.449&1.072&1.515&1.019&1.584\\
		 39&1.237&1.337&1.187&1.392&1.137&1.452&1.085&1.517&1.033&1.583\\
		 40&1.246&1.344&1.197&1.398&1.149&1.456&1.098&1.518&1.047&1.583\\
		 45&1.288&1.376&1.245&1.424&1.201&1.474&1.156&1.528&1.111&1.583\\
		 50&1.324&1.403&1.285&1.445&1.245&1.491&1.206&1.537&1.164&1.587\\
		 55&1.356&1.428&1.320&1.466&1.284&1.505&1.246&1.548&1.209&1.592\\
		 60&1.382&1.449&1.351&1.484&1.317&1.520&1.283&1.559&1.248&1.598\\
		 65&1.407&1.467&1.377&1.500&1.346&1.534&1.314&1.568&1.283&1.604\\
		 70&1.429&1.485&1.400&1.514&1.372&1.546&1.343&1.577&1.313&1.611\\
		 75&1.448&1.501&1.422&1.529&1.395&1.557&1.368&1.586&1.340&1.617\\
		 80&1.465&1.514&1.440&1.541&1.416&1.568&1.390&1.595&1.364&1.624\\
		 85&1.481&1.529&1.458&1.553&1.434&1.577&1.411&1.603&1.386&1.630\\
		 90&1.496&1.541&1.474&1.563&1.452&1.587&1.429&1.611&1.406&1.636\\
		 95&1.510&1.552&1.489&1.573&1.468&1.596&1.446&1.618&1.425&1.641\\
		100&1.522&1.562&1.502&1.582&1.482&1.604&1.461&1.625&1.441&1.647
		\end{tabular}
		\end{center}
		
		\newpage
		
		\begin{center}
		BOUNDS FOR CRITICAL VALUES OF THE \\
		DURBIN-WATSON STATISTIC \\
		\ \\
		\begin{tabular}{rr@{\ }rr@{\ }rr@{\ }rr@{\ }rr@{\ }r}
		\multicolumn{11}{c}{2.5\% SIGNIFICANCE POINTS OF $Q_L$ AND $Q_U$} \\
		\ \\
		&\multicolumn{2}{c}{$\Lambda=2$}&\multicolumn{2}{c}{$\Lambda=3$}
		&\multicolumn{2}{c}{$\Lambda=4$}&\multicolumn{2}{c}{$\Lambda=5$}
		&\multicolumn{2}{c}{$\Lambda=6$}\\
		$N$ & 
		$Q_L$ & $Q_U$ & $Q_L$ & $Q_U$ & $Q_L$ & $Q_U$ & $Q_L$ & $Q_U$ & $Q_L$ & $Q_U$ \\
		\ \\
		 15&0.949&1.222&0.827&1.405&0.706&1.615&0.588&1.848&0.478&2.099\\
		 16&0.980&1.235&0.864&1.403&0.748&1.594&0.636&1.806&0.527&2.035\\
		 17&1.009&1.248&0.899&1.403&0.788&1.578&0.680&1.773&0.574&1.983\\
		 18&1.035&1.261&0.930&1.405&0.825&1.567&0.720&1.746&0.619&1.939\\
		 19&1.060&1.274&0.959&1.407&0.859&1.558&0.758&1.724&0.660&1.902\\
		 20&1.082&1.286&0.988&1.410&0.890&1.551&0.794&1.705&0.699&1.871\\
		 21&1.104&1.297&1.012&1.415&0.920&1.546&0.826&1.691&0.734&1.845\\
		 22&1.124&1.308&1.036&1.419&0.947&1.543&0.858&1.678&0.769&1.823\\
		 23&1.144&1.319&1.059&1.424&0.973&1.541&0.887&1.668&0.801&1.804\\
		 24&1.161&1.329&1.080&1.429&0.997&1.539&0.914&1.659&0.830&1.787\\
		 25&1.178&1.339&1.099&1.435&1.019&1.539&0.939&1.652&0.859&1.773\\
		 26&1.194&1.348&1.118&1.439&1.041&1.538&0.964&1.646&0.886&1.761\\
		 27&1.208&1.358&1.135&1.445&1.061&1.539&0.986&1.641&0.911&1.751\\
		 28&1.222&1.367&1.153&1.450&1.080&1.540&1.007&1.637&0.934&1.742\\
		 29&1.236&1.375&1.168&1.455&1.098&1.541&1.028&1.634&0.958&1.734\\
		 30&1.249&1.383&1.183&1.460&1.115&1.542&1.047&1.632&0.978&1.727\\
		 31&1.261&1.391&1.197&1.465&1.132&1.544&1.066&1.630&0.999&1.721\\
		 32&1.273&1.399&1.211&1.469&1.147&1.546&1.083&1.628&1.018&1.715\\
		 33&1.284&1.406&1.224&1.474&1.163&1.548&1.099&1.627&1.037&1.711\\
		 34&1.294&1.413&1.236&1.479&1.176&1.550&1.115&1.626&1.054&1.707\\
		 35&1.305&1.420&1.248&1.484&1.190&1.553&1.131&1.626&1.071&1.704\\
		 36&1.315&1.426&1.259&1.488&1.203&1.555&1.145&1.625&1.087&1.701\\
		 37&1.324&1.433&1.270&1.493&1.215&1.557&1.159&1.625&1.102&1.698\\
		 38&1.333&1.439&1.281&1.497&1.227&1.560&1.173&1.625&1.117&1.695\\
		 39&1.342&1.445&1.291&1.501&1.238&1.562&1.185&1.626&1.131&1.693\\
		 40&1.350&1.451&1.300&1.506&1.249&1.564&1.197&1.626&1.144&1.692\\
		 45&1.388&1.477&1.343&1.525&1.298&1.576&1.252&1.630&1.204&1.687\\
		 50&1.420&1.500&1.380&1.543&1.338&1.588&1.297&1.636&1.255&1.685\\
		 55&1.447&1.520&1.411&1.559&1.373&1.600&1.335&1.642&1.297&1.686\\
		 60&1.471&1.538&1.438&1.573&1.404&1.610&1.369&1.649&1.333&1.688\\
		 65&1.492&1.554&1.461&1.587&1.430&1.620&1.398&1.655&1.365&1.691\\
		 70&1.511&1.568&1.482&1.599&1.453&1.630&1.424&1.662&1.393&1.695\\
		 75&1.528&1.582&1.501&1.610&1.474&1.638&1.446&1.668&1.418&1.699\\
		 80&1.543&1.594&1.518&1.619&1.493&1.647&1.467&1.674&1.441&1.703\\
		 85&1.557&1.605&1.534&1.629&1.510&1.654&1.485&1.680&1.461&1.707\\
		 90&1.570&1.614&1.548&1.638&1.525&1.662&1.502&1.686&1.479&1.711\\
		 95&1.582&1.624&1.560&1.646&1.539&1.668&1.517&1.691&1.495&1.715\\
		100&1.593&1.633&1.573&1.654&1.552&1.675&1.532&1.696&1.511&1.718
		\end{tabular}
		\end{center}
		
		\newpage
		
		\begin{center}
		BOUNDS FOR CRITICAL VALUES OF THE \\
		DURBIN-WATSON STATISTIC \\
		\ \\
		\begin{tabular}{rr@{\ }rr@{\ }rr@{\ }rr@{\ }rr@{\ }r}
		\multicolumn{11}{c}{5\% SIGNIFICANCE POINTS OF $Q_L$ AND $Q_U$} \\
		\ \\
		&\multicolumn{2}{c}{$\Lambda=2$}&\multicolumn{2}{c}{$\Lambda=3$}
		&\multicolumn{2}{c}{$\Lambda=4$}&\multicolumn{2}{c}{$\Lambda=5$}
		&\multicolumn{2}{c}{$\Lambda=6$}\\
		$N$ & 
		$Q_L$ & $Q_U$ & $Q_L$ & $Q_U$ & $Q_L$ & $Q_U$ & $Q_L$ & $Q_U$ & $Q_L$ & $Q_U$ \\
		\ \\
		 15&1.077&1.361&0.945&1.543&0.814&1.750&0.685&1.977&0.562&2.220\\
		 16&1.106&1.371&0.982&1.539&0.857&1.728&0.734&1.935&0.615&2.157\\
		 17&1.133&1.381&1.015&1.536&0.897&1.710&0.779&1.900&0.664&2.104\\
		 18&1.158&1.392&1.046&1.535&0.933&1.696&0.820&1.872&0.710&2.060\\
		 19&1.180&1.401&1.075&1.535&0.967&1.685&0.859&1.848&0.752&2.022\\
		 20&1.201&1.411&1.100&1.537&0.998&1.676&0.894&1.828&0.792&1.991\\
		 21&1.221&1.420&1.125&1.538&1.026&1.669&0.927&1.812&0.828&1.964\\
		 22&1.240&1.429&1.147&1.541&1.053&1.664&0.958&1.797&0.863&1.940\\
		 23&1.257&1.437&1.168&1.543&1.078&1.660&0.986&1.786&0.895&1.919\\
		 24&1.273&1.446&1.188&1.546&1.101&1.657&1.013&1.775&0.925&1.902\\
		 25&1.288&1.454&1.206&1.550&1.123&1.654&1.038&1.767&0.953&1.886\\
		 26&1.302&1.461&1.224&1.553&1.143&1.652&1.062&1.759&0.979&1.873\\
		 27&1.316&1.468&1.240&1.556&1.162&1.651&1.083&1.753&1.004&1.861\\
		 28&1.328&1.476&1.255&1.560&1.181&1.650&1.104&1.747&1.028&1.850\\
		 29&1.341&1.483&1.270&1.563&1.198&1.650&1.124&1.743&1.050&1.841\\
		 30&1.352&1.489&1.284&1.567&1.214&1.650&1.143&1.739&1.070&1.833\\
		 31&1.363&1.496&1.297&1.570&1.229&1.650&1.160&1.735&1.090&1.825\\
		 32&1.373&1.502&1.309&1.573&1.244&1.650&1.177&1.732&1.109&1.819\\
		 33&1.383&1.508&1.321&1.577&1.258&1.651&1.193&1.730&1.127&1.813\\
		 34&1.393&1.514&1.332&1.580&1.271&1.652&1.208&1.728&1.144&1.807\\
		 35&1.402&1.519&1.343&1.584&1.283&1.653&1.222&1.726&1.160&1.803\\
		 36&1.411&1.524&1.354&1.587&1.295&1.654&1.236&1.725&1.175&1.799\\
		 37&1.419&1.530&1.364&1.590&1.307&1.655&1.249&1.723&1.190&1.795\\
		 38&1.427&1.535&1.373&1.594&1.317&1.656&1.261&1.723&1.204&1.792\\
		 39&1.435&1.540&1.382&1.597&1.328&1.658&1.273&1.722&1.218&1.789\\
		 40&1.442&1.544&1.391&1.600&1.338&1.659&1.285&1.721&1.231&1.786\\
		 45&1.475&1.566&1.430&1.615&1.383&1.666&1.336&1.720&1.287&1.776\\
		 50&1.503&1.585&1.462&1.628&1.421&1.674&1.378&1.721&1.334&1.771\\
		 55&1.527&1.601&1.490&1.640&1.452&1.681&1.414&1.724&1.374&1.768\\
		 60&1.549&1.616&1.514&1.652&1.480&1.689&1.444&1.727&1.408&1.767\\
		 65&1.567&1.629&1.536&1.662&1.503&1.696&1.471&1.731&1.438&1.767\\
		 70&1.583&1.641&1.554&1.671&1.524&1.703&1.494&1.735&1.464&1.768\\
		 75&1.598&1.652&1.571&1.680&1.543&1.709&1.515&1.739&1.486&1.770\\
		 80&1.611&1.662&1.586&1.688&1.560&1.715&1.534&1.743&1.507&1.772\\
		 85&1.624&1.671&1.600&1.696&1.575&1.721&1.551&1.747&1.525&1.774\\
		 90&1.635&1.679&1.612&1.703&1.589&1.726&1.566&1.751&1.542&1.776\\
		 95&1.645&1.687&1.623&1.709&1.602&1.732&1.579&1.755&1.557&1.778\\
		100&1.654&1.694&1.634&1.715&1.613&1.736&1.592&1.758&1.571&1.780
		\end{tabular}
		\end{center}
	
	
	\begin{flushright}
	\begin{tabular}{l c}
	\circled{70} & \pbox{20cm}{\score{3}{5} \\ {\tiny 71 votes,  64.21\%}} 
	\end{tabular} 
	\end{flushright}
	
	%to force start on odd page
	\newpage
	\thispagestyle{empty}
	\mbox{}
	\section{Quantitative Management}
	\lettrine[lines=4]{\color{BrickRed}T}he aim of this chapter is to introduce the main mathematical techniques of production and management, maintenance and quality whose use has become essential for engineers, managers and executives of modern companies (those who can make science, the others make methodology ...) and which are the minimum-minimorum of business knowledge in general management. These are techniques that date back to the early 20th century and were first used for the majority by the U.S. military and then  gradually implemented in the largest global companies.

	Moreover, it is an excellent section for the general culture of the 	physicist or mathematician ... and an achievement for the engineer! And as high level practitioners say: a management model without maths is incomplete model!

	Many techniques are however not presented in this section because the have already been proved in other chapters of the book. Indeed, complex, critical and cutting edge management techniques make a huge usage of statistics, probabilities, decision theory, graph theory (Markov chains or not, complete graphs for the management of communication, etc.), financial analysis and optimization algorithms. Because whole chapters in this book are already dedicated to those enumerated subjects, it would be redundant to return on them!

	\begin{tcolorbox}[title=Remark,colframe=black,arc=10pt]
We often speak about 3M (Mathematics Method for Management) or even sometimes about "scientific management" to describe the set of mathematical tools applied to management (there exists also a training curriculum on the subject of fifty days...). Some people also use the abbreviation "SciProM" for "Scientific Project Management) when it is project oriented and more generally of "Evidence-Based Management". A common English term that became fashionable in Europe to describe this area of application is also the "\NewTerm{Decisioneering}" referring to the fact that these are tools decision support for engineers. 
	\end{tcolorbox}

	Before starting let us indicate that despite their intrinsic importance, quantitative management techniques (also including operational research techniques, Monte Carlo simulation and Bootstrapping as studied in the section of Numerical Methods and also Six Sigma as studied in the section of Industrial Engineering and the Decision Theory) are still little used in industry, either because of lack of training of policy makers (except in the financial sector), or by the lack of relevance of the tool or the difficulty to implement it. The main concerns raised by decision makers in the application of mathematical models in the companies are:
\begin{enumerate}
	\item \textbf{The tools are too abstract}: In such situations, I always remember that... the concrete is abstract which we are accustomed!
	\item \textbf{A limited consideration of external and internal factors}: For strategic reasons, the pure and perfect answer to a mathematical solution seems rarely applicable de facto. Although quantitative techniques incorporate many factors, though some aspects are relatively easy to model in the mathematical sense of the term (cost, profitability, distance, time, pace... for example), other elements, however, are more difficult to model: legal constraints, commercial motivation to block a competitor, importance of networking, social climate, etc. The weight of these elements in the decision is however important, sometimes decisive!
	\item \textbf{A significant investment}: The mathematical tool itself requires a high level of mathematical knowledge, a good ability to model problems and to describe factors. These constraints are consuming a lot of time and a certain amount of money (whether through internal development, which consumes resources - or by external development, which consumes money). It is then necessary to find a balance between the investment required and expected returns.
	\item\textbf{ For infrequent events}: The company does not have the opportunity the use the  "effect of experience" and therefore a time to another, the problem is with a different service or managers that have changed between two studies or do not communicate between them. It is therefore difficult to maintain the skills within the company. The decision-maker will have to take these aspects into account when deciding whether or not to implement quantitative techniques in his business.
\end{enumerate}
We can also argue that the fact that scientific management techniques (SMT) or the mathematical methods of management (3M) are not generally is that:
\begin{enumerate}
	\item Depending on the context, the extra power offered by SMT/3M does not necessarily justify the efforts necessary to their learning.
	\item The SMT/3M are demanding in reasoning and rigorous, they may not be suitable for people who know neither one nor the other ...
\end{enumerate}

	\begin{tcolorbox}[title=Remark,colframe=black,arc=10pt]
	Beware of companies - especially multinationals - who are looking specialists or graduate in project management or supply chain mastering Microsoft Excel, Microsoft Access or VBA. Because it means that they use non-professional tools to do a job which ought to be done with the appropriate tools (and Microsoft Excel or Microsoft Access are not professional tools !!!). So in terms of internal organization, you can ensure that these companies organize and analyse anything, anyhow, with unsuitable tools and therefore that there is a general mess internally. 
	\end{tcolorbox}
We have chosen in this section to present the more conventional quantitative techniques in an ascending order of difficulty (technical level). This seems to us to this date the most relevant pedagogical choice...

	Let us give a schematic summary of the scientific method that we studied in the Introduction chapter and that applies pretty well to the business world regarding scientific management:

	\begin{figure}[H]
		\centering
		\includegraphics{img/economy/process.eps}
		\caption{Typical simple process}
	\end{figure}

	The simple subjects below are given without practical examples while those that exceed the high-school level are accompanied by practical examples more or less detailed.
	
	One last word to close this introduction... the quantitative manager or quantitative analyst has to know something... Sure you want to write verbose explantations of your analysis but never forget that Executives (at least actually) are looking only for what they name WIIFM (What's In It For Me). So you must understand that you are paid most of times to produce headlines and not to write novels...

\pagebreak
	\subsection{Corporate and Government Finance Management}
	Corporate and Government finance is the area of finance dealing with the sources of funding and the capital structure of corporations and governments and the actions that managers and politicians should take to increase the value of the firm/administration to the shareholders, the competitiveness/attractiveness of their business/nation as well as the tools and analysis used to allocate financial resources in a certain or probabilistic World (ie Universe). The primary goal of corporate and governments finance is to maximize or increase shareholder/citizen value through various adjustment variables and anticipate market trends and risks.
	
	Investment analysis (or capital budgeting) is concerned with the setting of criteria about which value-adding projects should receive investment funding, and whether to finance that investment with equity or debt capital.
	
	As many subject in Corporate and Government Finance are related to investment banking topics we will not repeat in this section the study of Actuarial, Interests calculations, Loan Amortization/Repayment, Portfolio Theory and Forecasting that we have already study in detail in the section Economy (page \pageref{economy}).
	
	Also the reader has to know that Corporate and Government Finance is not only related to investment but also to benchmarking and optimization and as we have already study these subject in the sections of Game and Decision Theory and Numerical Methods/Analysis we will not come back on these latter.
	\begin{figure}[H]
		\centering
		\includegraphics{img/economy/corporate_finance.jpg}
	\end{figure}
	Corporate Finance management is a every day challenge for every good CXO (CEO, CFO, CTO, CRO, etc.) and costs optimization should be his everyday way of life by always taking care of his own cognitive biases, market uncertainties (nothing is "sure" in Economy/Finance\footnote{Never forget the adage: "The problem in the world is that intelligent people are full of doubts, while the idiots are full of confidence"}) and having always a reserve of alternative strategies in case of economic conjuncture variations (that in the probabilistic point of view may always have the possibility to occur!).
	
	All CXO have always to face risks (especially "\NewTerm{risk–benefit ratio}\footnote{It is the ratio of the risk of an action to its potential benefits. For example, driving an automobile is a risk most people take daily, also since it is mitigated by the controlling factor of their perception of their individual ability to manage the risk-creating situation.}\index{risk-benefit ratio}") and uncertainties hence the fact that any good manager associate arguments with probabilities or with precautions (some bad managers avoid risks subjectively but the main purpose of a real entrepreneur is to take risks to innovate and perform always better!):
	\begin{figure}[H]
		\centering
		\includegraphics[scale=0.07]{img/economy/rum.jpg}
		\caption{Risk-Uncertainty Matrix CXO knowledge}
	\end{figure}
	The list of possible risks is huge and we can found some exhaustive lists in good project management books, but here is a quite good summary of most important financial related type of risks:
	\begin{figure}[H]
		\centering
		\includegraphics[width=0.8\textwidth]{img/economy/types_of_risks.jpg}
		\caption{Typical type of financial risks}
	\end{figure}
	
	
	\pagebreak
	\subsubsection{Basic Accounting Equation}
	The "\NewTerm{basic accounting equation}, also named the "\NewTerm{balance sheet equation}, represents the relationship between the assets, liabilities, and owner's equity of a business. It is the foundation for the double-entry bookkeeping system. For each transaction, the total debits equal the total credits. It can be expressed as:
	 
	or more formally:
	 
	Since every business transaction affects at least two of a company's accounts, the accounting equation will always be "in balance", meaning the left side should always equal the right side. Thus, the accounting formula essentially shows that what the firm owns (its assets) is purchased by either what it owes (its liabilities) or by what its owners invest (its shareholders equity or capital).
	\begin{tcolorbox}[colframe=black,colback=white,sharp corners]
	\textbf{{\Large \ding{45}}Example:}\\\\
	 A student buys a computer for $945.-$. This student borrowed $500.-$ from his friend and spent another $445.-$ earned from his part-time job. Now his assets are worth $945.-$, liabilities are $500.-$, and equity $445.-$.	 
	\end{tcolorbox}
	We have typically:
	
	\begin{minipage}[t]{0.32\linewidth}
	    \textbf{ASSETS}:
	    \begin{itemize}[leftmargin=10pt]
	    	\item Cash
			\item Temporary Investments
			\item Accounts Receivable
			\item Merchandise Inventory
			\item Office Supplies
			\item Buildings
			\item Property \& Equipment
			\item Patents
			\item Goodwill
	    \end{itemize}
	\end{minipage}%
	\begin{minipage}[t]{0.32\linewidth}
	    \textbf{LIABILITIES}:
	    \begin{itemize}[leftmargin=15pt]
	    	\item Accounts Payable
			\item Rent Payable
			\item Salaries Payable
			\item Estimated Warranty Liability
			\item Dividends Payable
			\item Long Term Liabilities
			\item Discount on Bonds Payable
			\item Tax Payable
	    \end{itemize}
	\end{minipage}
	\begin{minipage}[t]{0.32\linewidth}
	    \textbf{EQUITIES}:
	    \begin{itemize}[leftmargin=15pt]
	    	\item Owner's Equity
			\item Owner's Withdrawals
			\item Common Shares
			\item Preffered Shares
			\item Retained Earnings
			\item Dividends
	    \end{itemize}
	\end{minipage}\par\bigskip
	
	\pagebreak
	\subsubsection{Ratio Analysis}
	A problem involved when comparing companies (or organizations) of different sizes is to calculate and compare  ratios that can provide very useful managerial information about efficiency. Using ratios eliminates as we know the size problem because the size effectively divides out. We are then left with percentages, multiples, or time periods. 
 
	There is a problem in discussing ratios. Because a ratio is simply one number divided by another they are incapable of accommodating multiple inputs and outputs when accurate objective relative weights for inputs and outputs are not known, and because there are so many accounting numbers out there, we could  examine a huge number of possible ratios (remember the reference on the UBS catalog on financial indicators that we have mention in the section Economy\footnote{reader can also refer to book as \textit{Financial Ratios for Executives, How to Assess Company Strength, Fix Problems, and Make Better Decisions} of  Michael Rist and Albert J. Pizzica}).
	\begin{tcolorbox}[title=Remark,colframe=black,arc=10pt]
	Sadly as far as we know and still in 2016 the are no ISO standard for corporate  indicators.
	\end{tcolorbox}
	Everybody has a favorite indicator. We will restrict ourselves to a representative sampling.  In this section, we only want to introduce you to some commonly used financial ratios and as you will see it is quick boring as the variables are never robustly well defined (furthermore most teachers and practitioners of corporate finance do not know how to write a technical paper or book properly...). The chosen indicators are not necessarily the ones we think are the best. In fact, some of them may strike you as illogical or not as useful as some alternatives. If they do, don't be concerned. As an engineer, you can always decide how to compute your own ratios. 
	
	What you do need to worry about is the fact that different people and different sources  seldom compute these ratios in exactly the same way, and this leads to much confusion.  The specific definitions we use here may or may not be the same as ones you have seen or will see elsewhere. If you are ever using ratios as a tool for analysis, you should be careful  to document how you calculate each one. And if you are comparing your numbers to numbers from another source, be sure you know how those numbers are computed. 
	
	We will defer much of our discussion of how ratios are used and some problems that come up with using them. For now, for each of the ratios we discuss, we consider several questions: 
	\begin{enumerate}
		\item How is it computed? 
		\item What is it intended to measure, and why might we be interested? 
		\item What is the unit of measurement? 
		\item What might a high or low value tell us? How might such values be misleading? 
		\item How could this measure be improved?
	\end{enumerate}
	
	\pagebreak 
	Financial ratios are traditionally grouped into the following categories:
	\begin{enumerate}
		\item Short-term solvency, or liquidity, ratios. 
		\item Long-term solvency, or financial leverage, ratios. 
		\item Asset management, or turnover, ratios. 
		\item Profitability ratios. 
		\item Market value ratios. 
	\end{enumerate} 
	
	\paragraph{Short term solvency or liquidity measure}\mbox{}\\\\
	As the name suggests, "\NewTerm{short-term solvency ratios}" as a group are intended to provide information about a firm’s liquidity, and these ratios are sometimes namede "\NewTerm{liquidity measures}"

 	The primary concern is the firm's ability to pay its bills over the short run without undue stress. Consequently, these ratios focus on current assets and current liabilities.

	For obvious reasons, liquidity ratios are particularly interesting to short-term creditors. Because financial managers work constantly with banks and other short-term lenders, an understanding of these ratios is essential
	One of the best known and most widely used ratios is the "\NewTerm{current ratio}". As you might guess, the current ratio at time $t$ is defined  as follows:
	
	Because current assets and liabilities are, in principle, converted to cash over the following $12$ months, the current ratio is a measure of short-term liquidity.

	To a creditor (particularly a short-term creditor such as a supplier) the higher the current ratio, the better. To the firm, a high current ratio indicates liquidity, but it also may indicate an ineffi cient use of cash and other short-term assets. Absent some extraordinary circumstances, we would expect to see a current ratio of at least $1$ because a current ratio of less than $1$ would mean that net working capital (current assets less current liabilities)
is negative. This would be unusual in a healthy firm, at least for most types of businesses.
\\
Inventory is often the least liquid current asset. It’s also
the one for which the book values are least reliable as measures of market value because the quality of the inventory isn’t considered. Some of the inventory may later turn out to be damaged, obsolete, or lost. 

	More to the point, relatively large inventories are often a sign of short-term trouble. The fi rm may have overestimated sales and overbought or overproduced as a result. In this case,
the firm may have a substantial portion of its liquidity tied up in slow-moving inventory.

	To further evaluate liquidity, the "\NewTerm{quick ratio}", or "\NewTerm{acid-test}", is computed just like the current ratio (CR), except inventory $\mathcal{I}$ is omitted at a given time $t$:
	
	Notice that using cash to buy inventory does not affect the current ratio (CR), but it reduces the quick ratio (QR). Again, the idea is that inventory is relatively illiquid compared to cash.
	
	We briefly mention three other measures of liquidity. A very short-term creditor might be interested in the \NewTerm{"cash ratio (CR)}":
	
		The "\NewTerm{interval measure (IM)}" is also useful. It give us how long the company can operate until it needs another round of financing. The average daily operating cost for companies is often named the "\NewTerm{burn rate}" meaning the rate at which cash is burned in the race to become profitable:
	
	where $\bar{C}_d$ is the average daily cost.

	\paragraph{Long term solvency or Liquidity measure}\mbox{}\\\\
	Long-term solvency ratios are intended to address the firm’s long-term ability to meet its obligations, or, more generally, its financial leverage. These are sometimes named "\NewTerm{financial leverage ratios}" or just "\NewTerm{leverage ratios}. 
	The "\NewTerm{total debt ratio} takes into account all debts of all maturities to all creditors. It can be defined in several ways. One possible way is at a given time $t$:
	
	We can define two useful variations on the total debt ratio: the "\NewTerm{debt–equity ratio}" and the "\NewTerm{equity multiplier}:
	
	where $\mathcal{D}$ is the total debt at time $t$ and:
	
	To complicate matters, different people (and different books) mean different things by the term debt ratio . Some mean a ratio of total debt, some mean a ratio of long-term debt only, and, unfortunately, a substantial number are simply vague about which one they mean.
	
	Another common measure of long-term solvency is the "\NewTerm{times interest earned ratio}" . Once again, there are several possible (and common) definitions. One is given by:
	
	where EBIT is the Earnings Before Interest And Taxes and $I$ the interests paid.
	
	Obviously as the name suggests, this ratio measures how well a company has its interest obligations covered.
	
	\pagebreak
	\paragraph{Profability Measures}\mbox{}\\\\
	The best known and most widely used of all financial ratios. In one form or another, they are intended to measure how efficiently a firm uses its assets and manages its operations. The focus in this group is on the
bottom line, net income.

	Companies pay a great deal of attention to their "\NewTerm{profit margins}":
	
	and obviously this give the amount of profit for every currency unit in sales.

	All other things being equal, a relatively high profit margin is obviously desirable. This situation corresponds to low expense ratios relative to sales.
	
	The "\NewTerm{return on assets}" is a measure of profit per currency unit of assets. It can be defined several ways, but the most common is this:
	
	The "\NewTerm{Return on equity}" is a measure of how the stockholders fared during the year. Because benefiting shareholders is our goal, RoE  is, in an accounting sense, the true bottom-line measure of performance. RoE is usually measured as follows:
	
	The fact that RoE exceeds RoA reflects typically the use of financial leverage (debts).
	
	As we mentioned in discussing RoA and RoE, the difference between these two profitability measures is a reflection of the use of debt financing, or financial leverage. We illustrate the relationship between these measures in this section by investigating a famous way of decomposing RoE into its component parts.
	
	Let us recall the definition of RoE:
	
	If we were so inclined, we could multiply this ratio by Assets/Assets without changing anything:
	
	Notice that we have expressed the RoE as the product of two other ratios: RoA and an equity multiplier:
	
	We can further decompose ROE by multiplying the top and bottom by total sales:
	
	If we rearrange things a bit, RoE looks like this:
	
	What we have now done is to partition RoA into its two component parts, profi t margin and total asset turnover. The last expression of the preceding equation is named the "\NewTerm{DuPont identity}, after the DuPont Corporation, which popularized its use.
	\begin{tcolorbox}[colframe=black,colback=white,sharp corners]
	\textbf{{\Large \ding{45}}Example:}\\\\
	General Motors provides a good example of how DuPont analysis can be very useful and also illustrates why care must be taken in interpreting RoE values. In 1989, GM had an RoE of $12.1\%$. By 1993, its RoE had improved to $44.1\%$, a dramatic improvement. On closer inspection, however, we find that over the same period GM's profit margin had declined from $3.4$ to $1.8\%$, and RoA had declined from $2.4$ to $1.3\%$ percent. The decline in RoA was moderated only slightly by an increase in total asset turnover from .71 to .73 over the period.\\

	Given this information, how is it possible for GM's RoE to have climbed so sharply? From our understanding of the DuPont identity, it must be the case that GM's equity multiplier increased substantially. In fact, what happened was that GM's book equity value was almost wiped out overnight in 1992 by changes in the accounting treatment of pension liabilities. If a company's equity value declines sharply, its equity multiplier rises. In GM's case, the multiplier went from $4.95$ in 1989 to $33.62$ in 1993. In sum, the dramatic "improvement" in GM's RoE was almost entirely due to an accounting change that affected the equity multiplier and didn't really represent an improvement in financial performance at all.
	\end{tcolorbox}
	 
	 \paragraph{Growth Rate}\mbox{}\\\\
	 The first growth rate of interest is the maximum growth rate
that can be achieved with no external financing of any kind. We will named this the "\NewTerm{internal growth rate}" because this is the rate the fi rm can maintain with internal financing only.

	Given that there is therefore either no Equities or Debts only the RoA will be useful to create this rate. Considering that a part of the RoA will be use for internal expenses, or taxes, we multiply the RoA (remember that it is a ratio, therefore without units) by a factor $p$ named the "\NewTerm{plowback}" factor to get the net PoA!
	
	Afterwards the most naive high-school level model make the assumption that the growing rate is a linear function of time (forge any ARMA/ARIMA, GARCH forecasting model in Corporate Finance trainings even at a graduate level!) such that:
	
	and therefore after making the assumption of small variations we get:
	
	Hence:
	
	But it is quite naive and very academic relation. In reality no serious practitioners should use such a method! But instead stochastic related forecasting techniques!!
	
	\paragraph{Asset management or turnovers measures}\mbox{}\\\\
	The measures in this section are sometimes named "\NewTerm{asset utilization ratios}". The specific ratios we discuss can all be interpreted as measures of turnover. What they are intended to describe is how efficiently or intensively a firm uses its assets to generate sales.
	
	The "\NewTerm{inventory turnover ratio}" is defined as:
	
	the higher this ratio is, the more efficiently we are managing inventory.

	Our inventory measures give some indication of how fast we can sell product. We now look at how fast we collect on those sales. The "\NewTerm{receivables turnover ratio} is defined much like the inventory turnover ratio:
	
	
	\paragraph{Market Value Measures}\mbox{}\\\\
	Our final group of measures is based, in part, on information not necessarily contained in financial statements: the market price per share of stock. Obviously, these measures can be calculated directly only for publicly traded companies.
	
	We begin with on that we already defined in the section Economy. The earning per share:
	
	We also have the "\NewTerm{price earning ratio}" defined by:
	
	the  ratio measures how much investors are willing to pay per dollar of current earnings, higher PERs are often taken to mean the firm has significant prospects for future growth. Of course, if a firm had no or almost no earnings, its PE would probably be quite large. So, as always, care is needed in interpreting this ratio.
	
	Another commonly quoted market value measure is the "\NewTerm{market to 
book ratio}":
	
	The book value per share is total equity (not just common stock) divided by the number of issued shares!!
	
	In a loose sense, the market-to-book ratio therefore compares the market value of the firm's investments to their cost. A value less than $1$ could mean that the firm has not been successful overall in creating value for its stockholders.
	
	This completes our definitions of some common one-to-one ratios. We could tell you about more of them, but these are enough for now and its already quite boring enough in absence of ISO norms. So now let us deal with multiple-to-multiple ratios performance analysis:
	
	\paragraph{Data Envelopment Analysis}\mbox{}\\\\
 	"\NewTerm{Data envelopment analysis}\index{data envelopment analysis}" (DEA) is a operational research methodology (\SeeChapter{see section Numerical methods page \pageref{nonlinear optimization}}) to evaluate the relative efficiency score $\eta$\footnote{When we focus on service organizations we generally cannot determine what the engineered, optimum or absolute efficient output-to-input ratio is. Consequently we cannot determine whether a service unit is absolutely efficient. We can, however, compare several service unit output-to-input ratios and determine that one unit is more or less efficient than another - benchmarking. The difference in efficiency will be due to the technology or production process used, how well that process is managed, and/or the scale or size of the unit.} of multiple decision-making units (DMUs or "comparable production units") when the production process presents a structure of multiple inputs $x_i$ and outputs $y_i$. It is a quite powerful service management and benchmarking technique originally developed by Chames, Cooper and Rhodes (1978) to evaluate nonprofit and public sector organizations\footnote{It has been developed in fact to analyze the efficiency of the US resources allocation  strategy to the schools and have afterwards be applied to all other government services. The India government also use it for the same purposes.}. DEA has since been proven to locate ways to improve service not visible with other techniques. Yet there is an anomaly surrounding this developing methodology. One of the largest US banks located over $\$100$ million of excess annual personnel and operating costs, enough to affect their earnings per share and these savings were not identifiable with other techniques in use. While other banks have also realized improved profits through initiatives driven by DEA, we could not locate more than 10 banks in this category. While businesses have no obligation to report their internal methods, DEA has not been widely adopted by banks. Why is DEA, a method that can generate new paths to improved profits not used when other less powerful techniques continue in use? We believe that greater adoption of DEA will only be possible when it is more accessible to non-technical skills (as most managers and CXO are soft-skills).
 	
 	\begin{tcolorbox}[title=Remark,colframe=black,arc=10pt]
	In the literature, the two most widely used variants of the DEA method are: the CCR model (Charnes, Cooper and Rhodes, 1978) which assumes constant scale returns and the BCC model (Banker, Charnes et al. Cooper, 1984) which assumes variable scale yields. In the case of constant scale returns, it is assumed that an increase in the amount of inputs consumed will lead to a proportional increase in the quantity of outputs produced. On the other hand, in the case of returns of variable scales (increasing or decreasing), the quantity of outputs produced is considered to increase more or less proportionally than the increase in inputs.
	\end{tcolorbox}
 	Managers have not widely adopted DEA to improve organization performance, in part, because most DEA publications are in academic journals or books requiring the ability to understand operational research and supporting mathematical notation. In fact, some 
managers trying to use DEA based on their understanding of academic publications have misunderstood the way to apply DEA. They erroneously attribute weak results to the technique when the problem is often due to the misapplication of DEA. Another problem (maybe it is the biggest), DEA will be more used when CXO will stop to fear the fact to listen that their organization is non-efficient.

	We will explains here what DEA does, how DEA evaluates efficiency, how DEA identifies paths to improve efficiency, limitations of DEA, and how to use DEA. This will enable managers or governments to explore and assess the value of their organization using DEA in their service operations.
	
	What does DEA do?
	\begin{enumerate}
		\item DEA compares service units considering all resources used and services provided, and identifies the most efficient units or best practice units (branches, departments, individuals) and the inefficient units in which real efficiency improvements are possible. This is achieved by comparing the mix and volume of services provided and the resources used by each unit compared with those of all the other units. In short, DEA is a very powerful benchmarking technique.
		
		\item DEA calculates the amount and type of cost and resource savings that can be achieved by making each inefficient unit as efficient as the most efficient - best practice - units.
		
		\item Specific changes in the inefficient service units are identified, which management can implement to achieve potential savings located with DEA. These changes would make the efficient units performance approach the best practice unit performance. In addition, DEA estimates the amount of additional service an inefficient unit can provide
without the need to use additional resources.
		
		\item Management receives information about performance of service units that can be used to help transfer system and managerial expertise from better-managed, relatively
efficient units to the inefficient ones. This has resulted in improving the productivity of the inefficient units, reducing operating costs and increasing profitability.
	\end{enumerate}
	The above four types of DEA information prove extremely valuable because they identify relationships not identifiable with alternative techniques that are commonly used in service organizations. As a result, improvements to operations extend beyond any performance improvements management may have achieved using other techniques.
	
	It is considered that the non-efficiency of an organization is mainly due to two main factors:
	\begin{itemize}
		\item The organization may be non-efficient because its management may perform better (not to say deficient...).
		
		\item The organization may be non-efficient because it hasn't reached it's optimal size
	\end{itemize}
	
 	\begin{tcolorbox}[title=Remark,colframe=black,arc=10pt]
	Some managers pretend that their organization that performs badly according to a DEA analysis because they cannot be compared with other organization because of special other factors. This may be true but then... the \textit{burden of proof} lies on them (as we have proved it in the section Statistics page \pageref{nhst}).
	\end{tcolorbox}
	Obviously the efficiency score must be considered as an order of amplitude, a result of binary judgment that helps manager to know if there is something to change or not. We could obviously used combinatorics methods on the inputs and outputs to get an interval rather than a punctual value.
	
	Let us introduce now the corresponding mathematical framework! The operational research technique is used to find the set of coefficients ($u$'s and $v$'s) that will give the highest possible efficiency ratio of outputs to inputs for the service unit being evaluated.
	
	In the model:
	\begin{itemize}
		\item $j$ is the number of service units (SU) being ocmapred in the DEA analysis
		\item $\text{SU}_j$ is the service unit number $j$
		\item $\eta_j$ is the efficiency of the service unit $j$ being evaluated by DEA
		\item $y_{rj}$ is the amount of output $r$ used by service unit $j$
		\item $x_{ij}$ is the amount of input $i$ used by the service unit $j$
		\item $i$ number of inputs used by the SUs
		\item $r$ number of outputs generated by the SUs
		\item $u_r$ coefficent or weight assigend by DEA to output $r$
		\item $v_i$ coefficient or weight assigend by DEA to input $i$
	\end{itemize}
	The data required to apply DEA are the actual observed outputs produced $y_{rj}$ and the actual inputs used $x_{ij}$, during one time period for each service unit in the set of units being evaluated. Hence, $x_{ij}$ is the observed amount of the $i$th input used by the $j$th service unit, and $y_{rj}$ is the amount of $r$th output produced by the $j$th service unit.
	
	If the value of $\theta$ for the service unit being evaluated is less than $100\%$, then that unit is inefficient, and there is the potential for that unit to produce the same level of outputs with fewer inputs.
	
	The objective function is given by:
	
	This is subject to the constraint that when the same set of $u$ and $v$ coefficients is applied to all other service units being compared, no service unit (SU) will be more than $100\%$ efficient as follows:
	
	Finally all this compacted into the classical form of optimization problem will be written:
	
	\begin{tcolorbox}[title=Remark,colframe=black,arc=10pt]
	It is recommended that the number of inputs and outputs attributes to not be bigger that then number of organizations divided by $3$.
	\end{tcolorbox}
	Note that optimal multipliers calculated by DEA are objectively determined weights that may not correspond to relative values that an analyst would assign to outputs and inputs and this is a good point as therefore there a are not subjective (or biased) human choice to do! Furthermore the weights can be different for each SU and this is more realistic! This is actually a strength and is not a weakness of DEA. A unit located as inefficient using DEA is so identified only after all possible weights have been considered to give that branch the highest rating possible consistent with the constraint that no branch in the data set can be more than $100\%$ efficient. Hence, any other set of weights applied to all branches would only make an inefficient branch appear equally or less efficient; that is, DEA gives the benefit of the doubt to each branch or service
unit in calculating the efficiency value. In addition DEA will not erroneously locate an efficient unit as inefficient. Hence, DEA gives the "benefit of the doubt" to each unit being evaluated trying to make it look as efficient as possible in comparison with the other units. This bias makes this a tool that managers can use with confidence.
	\begin{figure}[H]
		\centering
		\includegraphics[width=1.0\textwidth]{img/economy/dea.jpg}
		\caption{Data Envelopment Analysis}
	\end{figure}
	To appreciate the power and limitations of DEA in improving efficiency,let us see an example with Microsoft Excel (the reader can also found the corresponding example in the R companion book):
	\begin{tcolorbox}[colframe=black,colback=white,sharp corners]
	\textbf{{\Large \ding{45}}Example:}\\
	Consider we have five administrations (office) that delivers birth and marriage acts in various quantities during the same period of time and that each office has one employee only. We want to run a DEA analysis on the following situation summarized in the table below:
	\begin{table}[H]
		\centering
		\begin{tabular}{|l|c|c|c|}
		\hline
		\rowcolor[HTML]{C0C0C0} 
		\textbf{Government Office} & \textbf{\begin{tabular}[c]{@{}c@{}}Input 1\\ Employees\end{tabular}} & \textbf{\begin{tabular}[c]{@{}c@{}}Output 1\\ Birth acts\end{tabular}} & \textbf{\begin{tabular}[c]{@{}c@{}}Output 2\\ Marriage acts\end{tabular}} \\ \hline
		Office A & $1$ & $1$ & $6$ \\ \hline
		Office B & $1$ & $3$ & $8$ \\ \hline
		Office C & $1$ & $4$ & $3$ \\ \hline
		Office D & $1$ & $5$ & $6$ \\ \hline
		Office E & $1$ & $6$ & $2$ \\ \hline
		\end{tabular}
	\end{table}
	For this purpose in Microsoft Excel 14.0.7190 we build the following structure:
	\end{tcolorbox}
	
	\begin{tcolorbox}[colframe=black,colback=white,sharp corners]
	\begin{figure}[H]
		\centering
		\includegraphics[width=0.8\textwidth]{img/economy/dea_ms_excel_01.jpg}
		\caption[]{Data Envelopment Analysis example in Microsoft Excel 14.0.7190}
	\end{figure}
	Or explicitly:
	\begin{figure}[H]
		\centering
		\includegraphics[width=0.9\textwidth]{img/economy/dea_ms_excel_02.jpg}
		\caption[]{Data Envelopment Analysis formulas in Microsoft Excel 14.0.7190}
	\end{figure}
	Once this done we run a first time the solver with the following settings:
	\begin{figure}[H]
		\centering
		\includegraphics[width=0.8\textwidth]{img/economy/dea_ms_excel_03.jpg}
		\caption[]{Data Envelopment Analysis solver in Microsoft Excel 14.0.7190}
	\end{figure}
	\end{tcolorbox}
	
	\begin{tcolorbox}[colframe=black,colback=white,sharp corners]
	Then we get for Office A:
	\begin{figure}[H]
		\centering
		\includegraphics[width=0.6\textwidth]{img/economy/dea_ms_excel_04.jpg}
	\end{figure}
	Doing the same for Office B but with the following settings:
	\begin{figure}[H]
		\centering
		\includegraphics[width=0.7\textwidth]{img/economy/dea_ms_excel_05.jpg}
	\end{figure}
	we get:
	\begin{figure}[H]
		\centering
		\includegraphics[width=0.6\textwidth]{img/economy/dea_ms_excel_06.jpg}
	\end{figure}
	\end{tcolorbox}
	
	\begin{tcolorbox}[colframe=black,colback=white,sharp corners]
	And proceeding same for Office C, we get:
	\begin{figure}[H]
		\centering
		\includegraphics[width=0.7\textwidth]{img/economy/dea_ms_excel_07.jpg}
	\end{figure}
	and for Office D:
	\begin{figure}[H]
		\centering
		\includegraphics[width=0.7\textwidth]{img/economy/dea_ms_excel_08.jpg}
	\end{figure}
	and finally for Office E:
	\begin{figure}[H]
		\centering
		\includegraphics[width=0.7\textwidth]{img/economy/dea_ms_excel_09.jpg}
	\end{figure}
	We see that all five efficiencies $\{0.7500,1.0000,0.7308,1.0000,\}$ matches perfectly with the results we get with R in the corresponding companion book.
	\end{tcolorbox}
	
	First, as we know, efficiency can be simply defined as the ratio of output to input.
More output per unit of input reflects relatively greater efficiency. If the greatest possible output per unit of input is achieved, a state of absolute or optimum efficiency has been achieved and it is not possible to become more efficient without new technology or other changes in the production process.
 	 	
 	Some of the advantages of DEA are:
	\begin{itemize}
		\item no need to explicitly specify a mathematical form for the production function
		\item proven to be useful in uncovering relationships that remain hidden for other methodologies
		\item capable of handling multiple inputs and outputs
		\item capable of being used with any input-output measurement
		\item the sources of inefficiency can be analyzed and quantified for every evaluated unit
	\end{itemize}
	Some of the disadvantages of DEA are:
	\begin{itemize}
		\item results are sensitive to the selection of inputs and outputs
		\item you cannot test for the best specification
		\item the number of efficient firms tends to increase with the number of inputs and output variables
	\end{itemize}
	DEA is commonly applied in the electric utilities sector. For instance a government authority can choose Data Envelopment Analysis as their measuring tool to design an individualized regulatory rate for each firm based on their comparative efficiency. The input components would include man-hours, losses, capital (lines and transformers only), and goods and services. The output variables would include number of customers, energy delivered, length of lines, and degree of coastal exposure (Berg 2010). DEA is also regularly used to assess the efficiency of public and not-for-profit organizations, e.g. hospitals (Kuntz, Scholtes \& Vera 2007; Kuntz \& Vera 2007; Vera \& Kuntz 2007) or police forces (Thanassoulis 1995; Sun 2002; Aristovnik et al. 2013, 2014).
	
	DEA has been recognized as a valuable analytical research instrument and a practical decision support tool. DEA has been credited for not requiring a complete specification for the functional form of the production frontier nor the distribution of inefficient deviations from the frontier. Rather, DEA requires general production and distribution assumptions only. However, if those assumptions are too weak, inefficiency levels may be systematically underestimated in small samples. In addition, erroneous assumptions may cause inconsistency with a bias over the frontier. Therefore, the ability to alter, test and select production assumptions is essential in conducting DEA-based research. However, the DEA models currently available offer a limited variety of alternative production assumptions only.
	
	We have described the basic DEA model, the insights it provides in a simple example where similar insight could be gained from observation and where the operational research programming calculations could easily be done with a simple spreadsheet software.
	
	The reader should know however that in the specialized literature about the subjects he can also found a lot of other DEA models. Respectively (with their respective abreviations):
	\begin{itemize}
		\item FDH: Free disposability hull, no convexity assumption
		\item VRS: Variable returns to scale, convexity and free disposability
		\item DRS: Decreasing returns to scale, convexity, down-scaling and free disposability
		\item CRS: Constant returns to scale, convexity and free disposability
		\item IRS: Increasing returns to scale, (up-scaling, but not down-scaling), convexity and free disposability
		\item IRS2: Increasing returns to scale (up-scaling, but not down-scaling), additivity, and free disposability
		\item ADD: Additivity (scaling up and down, but only with integers), and free disposability; also known af replicability and free disposability, the free disposability and replicability hull (frh) – no convexity assumption
		\item FDH+: A combination of free disposability and restricted or local constant return to scale
	\end{itemize}		
	The model that we have seen above is named the "\NewTerm{CRS model}\index{CRS model}" for "Constant Returns to Scale".

	Each model exists also for information in two variations (or "orientations") that so far I'm not very a fan of (so I will not introduce them in this book):
	\begin{itemize}
		\item "Input oriented": the DEA model minimize the inputs for a given level of outputs. In other words, it indicates of how much an organization can increase its inputs by still producing the same level of outputs. Then the performance margin is obviously on the input (ie it must decrease\footnote{This is typically the case in organizations that has become too big and must apply a cost compression optimization plan}).
		
		\item "Output oriented": the DEA model maximize the outputs for a given level of inputs. In other words, it indicates of how much an organization can increase its outputs by still having the same level of inputs. Then the performance margin is obviously on the output (ie it must increase).
	\end{itemize}
	The orientation of the model must be chosen according to the variables (inputs or outputs) over which decision-makers exercise the greatest management power. For example, a public school principal probably has more management power over the teaching staff (input) than the number of students received or the student results (outputs). In this case, an input orientation is more appropriate.

	In the public sector, but sometimes also in the private sector, a certain level of resources is allocated and guaranteed to organizations. In such a case, the decision makers seek to maximize the benefits provided, and therefore choose an output orientation. Otherwise, if the objective of the decision-makers is to produce a certain level of outputs (for example an imposed quota), the latter seek to minimize the consumption of resources. They therefore opt for an input orientation.

If no constraints are imposed on the decision-makers and if they exercise a management power over both the inputs and the outputs, the orientation of the model depends on the objectives set for the organizations. Is the goal to reduce costs (input orientation) or to maximize production (orientation output)?
	
	\pagebreak
	\subsubsection{Weighted Average Cost of Capital (WACC)} 
	When a firm issues debt, equity, or other securities to fund a new investment, there are many potential consequences of its decision. By far the most important question for a CFO is whether difference choices will affect the return of the firm and thus the amount of capital it can raise. Companies can use also the WACC to see if the investment projects available to them are worthwhile to undertake.
	
	To answer this question we will assume a "perfect capital market" world in which:
	\begin{enumerate}
		\item Securities are fairly priced: Investors and firms can trade the same set of securities at competitive market prices equal to the present value of their future cash flows.
		
		\item There are no tax consequences or transactions cost: There are no tax consequences, transaction cost, or other issuance costs associated with financing decisions or security trading.
		
		\item Investment cash flows are independent of financing choices.  A firm's financing decisions do not change the cash flows generated by its investments, nor do they reveal new information about those cash flows.
	\end{enumerate}
	The are many ways to introduce the WACC that are more or less complex. In fact you know that in this book we like the easiest one.
	\begin{tcolorbox}[title=Remark,colframe=black,arc=10pt]
	The absence of debts means the absence of financial leverage (in fact there is a leverage but not so fast and big as with debts). Equity in a firm with no debt is therefore named "\NewTerm{unlevered equity}". Equity in a firm that also has debt is named \NewTerm{levered equity}".
	\end{tcolorbox}
	
	Consider a firm that has an amount of dept $\mathcal{D}$ (share of stocks) with a return $r_\mathcal{D}=(1+t_\mathcal{D}\%)$ and an amount of equities $\mathcal{E}$ (bonds) with a return $r_\mathcal{E}=(1+t_\mathcal{E}\%)$
	
	By application of simple interest rate composition we write:
	
	and therefore:
	
	As obviously:
	
	The prior previous relation can be written with weights notation:
	

	Another complication is the effect of taxes $T$, and the fact that interest payments (dividends) paid to bondholders are tax-deductible by the firm. Therefore the investors
	
	where $r_{\mathcal{E}}$ is the required rate of return on equity, estimated from the CAPM (\SeeChapter{see section Economy page \pageref{capital asset pricing model}}) and the history of the firm's share price (hence the fact the WACC use the same assumptions) and $r_{D}$ is the rate of return (yield) on the company's bonds, (also named for recall the "borrowing rate").
	
	This $r$ is the rate of discount applied to risky future cash flows, as in chapter 4. The assumption is that the new project will be financed with the same mixture of debt and equity as already exists in the firm, and this assumption is correct if the money comes from retained earnings. But sometimes, the firm will issue new equity to finance an expansion or project; and sometimes, the project will be riskless enough that it can be
financed completely with new debt. All of these complications, and the added problem of accounting for taxes, are best left for another course.

	Note that the cost of debt is not adjusted for taxes, because we are assuming perfect capital market and thus are ignoring taxes. When we compute the weighted average cost of capital without taxes we refer to it as the firm's pretax WACC.

	In general, the WACC can be calculated with the following relation:
	
	where $N$ is the number of sources of capital (securities, types of liabilities), $r_i$ is the required rate of return for security $i$ and $\text{MV}_i$ is the market value of all outstanding securities $i$.
	
	And as seen previously, in the case where the company is financed with only equity and debt, the average cost of capital is computed as follows:
	
	\begin{figure}[H]
		\centering
		\includegraphics{img/economy/bloomberg_wacc.jpg}
		\caption{WACC of Pepsi company in Bloomberg\textsuperscript{TM} Terminal}
	\end{figure}
	
	\pagebreak
	\subsubsection{Break-even Point Analysis (BEPA)}

	Break-even point analysis, also named "\NewTerm{neutral point analysis}", is the set of calculations to determine the minimum activity level at which an organization becomes profitable for itself by its economies of scale, that is to say, it stop to lose money on this activity. Obviously, this technique also allows you to compare several strategies together.

	The "\NewTerm{neutral point}" or "\NewTerm{break-even point}" is intuitively enough the intersection point between the curve of Turnover (because turnover is normally a function of several variables) and curve of Spending (also a function of several variables) necessary to produce that revenue. In other words, this is the corresponding coordinates where the two functions are equal  (in practice it is rarely represented graphically because they are too many variables).

	Formally... given a vector $\vec{x}\in \mathbb{R}^n$ having for components all economic variable describing the problem. We consider a function of total costs $T_c(\vec{x}): \mathbb{R}^n \mapsto \mathbb{R}$ and another function or revenues $T_r(\vec{x}): \mathbb{R}^n \mapsto \mathbb{R}$. The break even analysis consists obviously to found the vector $\vec{x}$ such that
	
	or equivalently:
	
	and the solutions can be non-unique.

	Normally this technique should allow a manager or governement to determine information such as:
	\begin{itemize}
		\item Predict sales volume to cover just the costs (Neutral Point)
		
		\item What should be the price per unit to cover just the costs (Neutral Point)
		
		\item What is the level of fixed costs that balances just gains (Neutral Point)
		
		\item How prices affect the turnover volume corresponding to equilibrium (Neutral Point)
		
		\item What is the tax discount that should be given to a company that relocates depending of the number of new jobs it will create
	\end{itemize}
	The neutral point analysis is based on the assumption that all costs relating to the manufacture of a property can be divided into two categories: variable costs and fixed costs.
	
	The total cost is defined in a linear model by the relation:
		
		where $F$ represents fixed costs, $c$ the cost per unit (assumed to be independent of the quantity) of produced goods and $Q$ the corresponding quantity.

	The assumption that the cost per unit is independent of the quantity is used to simplify the calculation for the total cost which is a simple linear function, but this assumption is not really necessary today with spreadsheets softwares.

	If we assume that all units produced are sold, the total revenue $T_r$ is defined by:
	
	where $p$ is the revenue per unit sold.

	The break-even point is then the equality between the two costs, which means:
	
	So we could then for example after some elementary algebraic operations, determine the quantity that gives the break-even point:
	
	Thus, knowing $F$ (fixed costs), $p$ (revenue per produced unit) and $c$ (costs per produced unit), it is possible to determine the quantity $Q$ for sale to get the break-even point.

	Note that a subject of great interest to managers is the study of the variation of some parameters depending on other variables. We name this the "\NewTerm{sensitivity analysis}" in the multivariate case (analysis based on the Pearson correlation coefficient) and "\NewTerm{double entry analysis}" in the bivariate case. Obviously in the linear case, this analysis is quite trivial.

	\begin{tcolorbox}[title=Remark,colframe=black,arc=10pt]
Since the early 1980s, there are very simple softwares to use such as @Risk of Palisade  to do Monte-Carlo simulations to determine the probabilistic break-even point and even solvers that solve nonlinear systems and do high level analysis of the break-even point. Otherwise for an high-school level analysis, spreadsheets softwares are enough...
	\end{tcolorbox}
	
	\pagebreak
	\subsubsection{Investment Strategies}
	By definition, an investment is the economic motivation to do the acquisition or development of a property (regardless of its form, material or not) by a corporation, community or an individual. This practice is to make currently an expense for an entity and expected a return on investment in the future.
	
	In finance, an investment strategy is more general a set of rules, behaviors or procedures, designed to guide an investor's selection of an investment. Individuals have different profit objectives, and their individual skills make different tactics and strategies appropriate. Some choices involve a tradeoff between risk and return or simply love for a product or a company. Most investors fall somewhere in between, accepting some risk for the expectation of higher returns or to maintain a the manufacturing of product they like to see exist (obviously from two or more investment strategies, the best at the individual level is the one that maximizes the final investor capital at equal risk...).
	
	In most academics teachers put apart the love for a product to highlight and focus only on the quantitative part. Indeed, measuring the utility for the love of a product is a quite difficult challenge...

	Then there exist different types of return on investment following the object of study. Thus, we differentiate in finance mainly (before seeing the details):
	\begin{itemize}
		\item The returns of financial asset on a given economic horizon ("\NewTerm{return on investment}") and their corresponding yield ("\NewTerm{rate of return}").

		\item The returns on investment on a project compared to an average (neutral risk) geometric rate of the market (stock exchange, OTc, etc.) that limits the interest to invest in a project ("\NewTerm{internal rate of return}").
	\end{itemize}
	
	We will focus here and the second and therefore the reader should know that corporate investments involves in an economical framework:
	\begin{enumerate}
		\item An immediate expense

		\item Future entries or outputs, named "\NewTerm{cash flow}" 

		\item A residual value	
	\end{enumerate}
	There are several criteria and techniques for making and investment choice in corporate finance, which we will present below, that permits to opt for an investment $A$ or $B$: that of the "net present value (NPV)", that of "internal rate return (IRR)" or that of "payback time (PT)" also nalled "recovery time".
	\begin{tcolorbox}[title=Remark,colframe=black,arc=10pt]
	We must not forget also that mathematical decision techniques (\SeeChapter{see section Game and Decision Theory page \pageref{game and decision theory}}) have a huge importance when we considered that very height (or significant) amount of money are involved in an investment.
	\end{tcolorbox}
	
	\paragraph{Net Present Value}\label{net present value}\mbox{}\\\\
	In finance, the "\NewTerm{net present value\footnote{Net present value as a valuation methodology dates at least to the 19th century. Karl Marx refers to NPV as fictitious capital, and the calculation as "capitalising".} (NPV)}" or "\NewTerm{net present worth (NPW)}" is a measurement of the profitability of an undertaking that is calculated by subtracting the "\NewTerm{present values (PV)}" of cash outflows (including initial cost) from the present values of cash inflows over a period of time. Incoming and outgoing cash flows can also be described as benefit and cost cash flows, respectively.

	The idea behind net present value is determined by calculating the costs (negative cash flows) and benefits (positive cash flows) for each period of an investment. Many people therefore say that an equivalent statement is that:
	
	\begin{center}
	\includegraphics{img/economy/npv.jpg}\\
	{\Large \calligra Money \textbf{now} is more valuable than money \textbf{later on}.}
	\end{center}
	Why? Because you can use money to make more money! Indeed, you could run a business (assuming its without risk...), or buy something now and sell it later for more (assuming its not defective and there is no devaluation...), or simply put the money in the bank to earn interest (assuming its also without risk...).
	
	An we already mention it investment involves in a simple economic framework:
	\begin{enumerate}
		\item An immediate expense payable in one or more times denoted $V_0$ in the NPV framework

		\item Future entries, named "cash flow" $C_i$ in the NPV framework

		\item A residual value $V_n$ in the NPV framework
	\end{enumerate}
	in more accurate models we have to take into account the uncertaintity and the we don't speak of NPV anymore but of "\NewTerm{expected Net Presente Values (eNPV)}" or "\NewTerm{risk-adjusted net present value (rNPV)}".
	
	The period of calculation of the NPV or eNPV typically one year, but could be measured in quarter-years, half-years or months. After the cash flow for each period is calculated, the present value of each one is achieved by discounting its future value (see below a recall of actuarial calculations we proved in the section Economy) at a periodic rate of return (the rate of return dictated by the market). NPV is the sum of all the discounted future cash flows. Because of its simplicity, NPV is a useful tool to determine whether a project or investment will result in a net profit or a loss. A positive NPV results in profit, while a negative NPV results in a loss. The NPV measures the excess or shortfall of cash flows, in present value terms, above the cost of funds. NPV is a central tool in "\NewTerm{discounted cash flow (DCF)}\index{discounted cash flow analysis}" analysis and is a standard method for using the time value of money to appraise long-term projects. It is as we already know now after our study of the section Economy widely used throughout economics, finance, and accounting.

	In the case when all future cash flows are positive, or incoming (such as the principal and coupon payment of a bond) the only outflow of cash is the purchase price, the NPV is simply the PV of future cash flows minus the purchase price (which is its own PV). NPV can be described as the "difference amount" between the sums of discounted cash inflows and cash outflows. It compares the present value of money today to the present value of money in the future, taking inflation and returns into account.

	The NPV of a sequence of cash flows takes as input the cash flows and a discount rate or discount curve and outputs a price. The converse process in DCF analysis - taking a sequence of cash flows and a price as input and inferring as output a discount rate (the discount rate which would yield the given price as NPV) - is named the "yield" and is more widely used in bond trading.
	
	What obviously interest and investor is that in present value, an investment more profitable than what is or would be spent.
	
	So let us see now a companion example to formalized all that!
	
	Consider the following standard situation: A company wishes to acquire a new machine worth $6,000$.-, which should lead to lower production costs of $1,000$.- during 5 years. We believe that in $5$ years, the residual value of this machine will be $3,000$.-. Should we buy this machine if this investment can be financed by a loan at $10\%$ (or in other words: if we have the option to invest in a fund having a $10\%$ annual effective return yield)?
	What information do we have here?
	\begin{itemize}
		\item The immediate expense $V_0=6000$

		\item The final or residual value of the equipment well after $5$ years $V_n=3000$

		\item The annually cash flows $C_k=1000$ (which are constant and positive over the entire period in this special example)

		\item The corresponding interest rate (geometric average market rate) of borrowing or investment corresponding $t\%=10\%$
	\end{itemize}
	What information, or interesting questions, financially speaking, can we ask from the data above?:
	\begin{enumerate}
		\item What should be the initial capital at a given market rate that would permit us to withdraw $1,000$.- per year during $5$ years (until the account balance is zero) ?

		From our study of actuarial calculations (see preceding section), we get:
		
		Which will be written in the special case where $C_i=C$ (we fall back the relation of certain postnumerando annuity proved in the section of Economy during our study of acturia calculations), that is to say a simple present value calculation:
		
		An numerical application with our example  (after a little calculation) gives about $3,790$.-.

	In other words, it would be enough for us to save $3,790$.- during the same $5$ years, to earn $1,000$.- per year to settle the account. So for now, an investment of $3,790$.- to save (earn) $1,000$.- per year seems much more favorable than spend $6,000$.- for the same return over the same period...
	
	Already here, we can say that the purchase of the machine is unfavorable at least... in certain future projection....

	But we must not forget also a second factor ... the residual value of our machine!!!

		\item What would be the initial capital that at the geometric average market rate would give us a value equivalent to the residual value of our machine (it is a capital asset just like a savings, so we can look at what would happen if this money came from savings)?

	So once again we use our knowledge of actuarial calculus by writing again the present value:
	
	In our special example, this quantity is equal to approximately $1,862$.-
	\end{enumerate}
	In other words, it would suffice us to save $1,862$.- during the same $5$ years to get an amount equal to the residual value of our machine. So what?	

	Well, the conclusion is quite simple .... The sum (in certain future and without intermediate taxes...):
	
	represents the return on the basis of an initial saving to get, relatively to the information of residual value and cash flow in certain future and without yearly taxes, the final sum equivalent to the final purchase of our machine. However, in this example, this quantity gives us about $5,652$.-
	
	This result is important because it is to be compared with the investment we wanted to do initially. Two options are therefore available to us:
	\begin{itemize}
		\item Buy $6,000$.- the machine with the cash flows and residual values that we know

		\item Save $5,652$.- during the same period, with the same cash flows to end up with a final savings that should be equivalent to the residual value of our machine.
	\end{itemize}
	Now, what can we conclude from this example? Well simply that we are in a worst case if:
	
	Thus explicitly:
	
	In Microsoft Excel 14.0.7106, to get the result we just write (the writing is not obvious but when we thing about it, finally it's logic!):
	
	\begin{center}
	\texttt{=-6000+NVP(10\%,1000,1000,1000,1000,4000)=-346.60}
	\end{center}
	
	In other words, for the financial analyst, business analyst, or project manager, the calculation interesting to do is in certain future, without intermediate taxes:
	
	where the $C_i$ can sometimes be negative (intermediate investments). So the three decision criteria are:
	\begin{itemize}
		\item $\text{NPV}<0$ corresponds to an investment that should be probably  avoided
		\item $\text{NPV}=0$ the we in front of an undecidable investment
		\item $\text{NPV}Y0$ positive is probably a good investment
	\end{itemize}
	In most high-school level books and undergraduate project management books, the relation of NPV above is simplified, because they assume no residual values. Thus we have:
	
	And let us recall that:
	
	is often named "discount factor" of the $i$th period.
	
	Some people still ask after these mathematical developments why is the NPV bigger when the interest rate is lower? Because the interest rate is like the team you are playing against, play an easy team (like a $6\%$ interest rate) and you look good, a tougher team (like a $10\%$ interest) and you don't look so good!
	
	Thus, the NPV is used as a decision criterion in large cutting edge companies or governments to quantify the specific contribution of a project to the finance of the company, taking into account the cost of capital via the discount factor.

	So in the case of a choice between several investments, we will choose the one whose NPV is highest. If the cash flows are not deterministic we will then have to calculate the mean and variance of the expected-NPV. 
	\begin{tcolorbox}[title=Remark,colframe=black,arc=10pt]
	NPV is also often named "\NewTerm{updated quasi-rent}". We also find often the name "\NewTerm{method of discounted cash flows}".
	\end{tcolorbox}
	If we consider in certain future and without taxes a $10,000$.- initial investment with a constant discount rate of $10\%$ and a perfectly periodical cash flow of $3,250$.-, $3,750$.-, $4,250$.- and finally $4'750$.- we have also traditionally the following tabular representation:
	\begin{table}[H]
	\begin{center}
		\definecolor{gris}{gray}{0.85}
			\begin{tabular}{|c|c|c|c|c|}
				\hline
				\multicolumn{1}{c}{\cellcolor{black!30}\textbf{Year}} & 
  \multicolumn{1}{c}{\cellcolor{black!30}\textbf{Action}}  & 
  \multicolumn{1}{c}{\cellcolor{black!30}\textbf{Cash-Flow}} & 
  \multicolumn{1}{c}{\cellcolor{black!30}\textbf{A.F.}}  & 
  \multicolumn{1}{c}{\cellcolor{black!30}\textbf{A.C.F.}} \\ \hline
				$0$ & Investment & $-10,000$ & $1$ & $-10,000$\\ \hline
				$1$ & Input & $3,250$ & $0.909$ & $2,955$\\ \hline
				$2$ & Input & $3,750$ & $0.826$ & $3,099$\\ \hline
				$3$ & Input & $4,250$ & $0.751$ & $3,193$\\ \hline
				$4$ & Input & $4,750$ & $0.683$ & $3,244$\\ \hhline{|=|=|=|=|=|}
				\textbf{Sums} & \textbf{O.N.C.} & $\mathbf{6,000}$ & \textbf{O.N.A.C.} & $\mathbf{2,491}$\\ \hline
		\end{tabular}
	\end{center}
	\caption{Detailed actualization table under accounting form}
	\end{table}
	where in the above table A.F. means "\NewTerm{actualization factor}", A.C.F. "\NewTerm{actualized cash-flow}", O.N.C "\NewTerm{overall net cash flow}" and finally O.N.A.C. "\NewTerm{overall net actualized cash-flow}".
	
	So in this table, the investment has a positive return of $2,491$.- more than a $10\%$ investment operation after $4$ years in certain future (deterministic Universe) without taxes.

	We will notice on the way that starting from the third year, the initial investment is repaid. We speak then of "\NewTerm{payback period}" (see further below).
	
	\pagebreak	
	\paragraph{Internal Rate of Return}\mbox{}\\\\
	\textbf{Definition (technical \#\mydef):} The "\NewTerm{internal rate of return (IRR)}" , also sometimes named "\NewTerm{limit of profitability}" or "\NewTerm{marginal efficiency of capital}" or "\NewTerm{discounted cash flow rate of return}" is the (actualization) rate $t\%$ for which the actualized value of net capital cash-flows (present values) resulting from an investment project is equal to the present value of disbursements required to make this investment.

	To quote the definition of J.M .Keynes: \textit{The marginal efficiency of capital is the discount rate that, applied to the series of annual payments made by the expected return of this capital during its entire existence, makes the present value of the annuities equal to the offering price of capital}.

	In other words, this is equivalent to asking what is the geometric average rate of the market (discount rate) for which the NPV of an investment is zero (ie: above the first IRR root the NPV is negative!). Either that satisfy the relation:
	
	Since the IRR calculation is a manifestation of the general problem of finding the roots of the equation $\text{NPV}(t\%)=0$, there are many numerical methods that can be used to estimate $t\%$. For example, using the secant method that we have study in the Numerical Methods section we have recursively:
	
	where $t\%_{n}$ is considered the $n$th approximation of the IRR and which can only be calculated quickly with IT tools (the Target tool in Microsoft Excel 11.8346 for example) or simply using the \texttt{IRR( )} function integrated in almost all the spreadsheet softwares like Microsoft Excel.
	\begin{tcolorbox}[colframe=black,colback=white,sharp corners]
	\textbf{{\Large \ding{45}}Example:}\\
	A project manager proposes us to buy a machine for an immediate investment of $2,000$.- in a project with a cash flow that doubles each period on a basis of $400$.- guarantee for the $3$ periods while the geometric average market interest is rate (discount rate) is of $5\%$. The IRR (internal rate of return) from which the NPV is zero is given with Microsoft Excel 14.0.7106:
	\begin{center}
	\texttt{=IRR(-2000,400,800,1600)=15.117\%}
	\end{center}
	Or analytically:
	
	When:
	\begin{center}
		\texttt{=-2000+NPV(5\%,400,800,1600)=488.71}
	\end{center}
	Therefore this project is obviously good as the free-risk market rate should be at leat at $15.117\%$ to be more interesting than investing to our project!
	\end{tcolorbox}
	So between two investments, we choose in companies that whose IRR is the highest and meeting the internal political and economical constraints.

	This type of calculation is therefore applicable on return on investment projects against market investments and not against the return on development projects. Thus, this is only computational tool to help the purely financial decision  and not industrial or commercial one.
	
	However, the NPV and IRR methods will obviously return conflicting results when mutually exclusive projects obviously differ in the timing of cash flows. In most cases, utilizing either the NPV or IRR method will lead to the same accept-or-reject decision.  An exception exists when evaluating mutually exclusive projects with crossing NPV profiles and the cost of capital is less than the crossover rate.  When these conditions are present, the NPV and IRR results will conflict in which project to accept or reject.

	 The presence of non-normal cash flows will lead obviously to multiple IRRs. But this is not an issue as financial analyst will keep the smallest solution.
	 
	 So what do managers really use for capital budgeting? In a survey in 2001, Graham and Harvey (from Duke University) surveyed 392 managers, primarily chief financial officers (CFOs), asking them what techniques they use when deciding on projects or acquisitions. The results are listed in the figure below. The two most prominent measures
are also the correct ones: They are the "internal rate of return" and the "net present value" methods. Alas, the troublesome "payback period" method and its cousin, the "discounted payback period" still remain surprisingly common:
	\begin{figure}[H]
		\centering
		\includegraphics[scale=0.48]{img/economy/cfo_valuation_techniques.jpg}
		\caption[CF Valuation Techniques]{CF Valuation Techniques (source: \textit{Corporate Finance An Introduction}, Ivo Welch)}
	\end{figure}
	\textbf{Definition (\#\mydef):} The "\NewTerm{profitability index}" is an index that attempts to identify the relationship between the costs and benefits of a proposed project through the use of a ratio calculated as:
	
	A ratio of $1.0$ is logically the lowest acceptable measure on the index, as any value lower than $1.0$ would indicate that the project's present value is less than the initial investment. As values on the profitability index increase, so does the financial attractiveness of the proposed project.

	\paragraph{Internal Rate of Return}\mbox{}\\\\
	\textbf{Definition (\#\mydef):} The "\NewTerm{payback}" or "\NewTerm{payback period $p$}" is another indicator for decision support as part of the selection of investment projects and simpler of use (and understanding) than the NPV but that does not provide obviously the same type of information.

    This indicator has for simple and only purpose to show when, in time, the investment will be repaid. In other words, it indicates the number of periods needed for the cash flow amounts $C_i$ to cover the initial investment $V_0$. This is a very simple information to determine which is equivalent of found the smallest $p$ such that:
      
    or even:
      
    In other words, it is the time for a project where the cash flows balance the initial investment.
  
   \textbf{Definition (\#\mydef):} The "\NewTerm{amortization time}" is the time for a project where the actualized cash flows balance the initial investment. That is the smallest $p$ such that:
     
    or even:
     
    
    \subsubsection{Company Valuation Methods} 
    For anyone involved in the field of corporate finance, understanding the mechanisms of company valuation is an indispensable requisite\footnote{Following Warren Buffet an investor shoud put its money only in corporations that have a dominant position, a sustainable business, whose know-how is quite hard to copy and that have shown in the past a sustainable capacity to make profits.}. This is not only because of the importance of valuation in acquisitions and merges but also because the process of valuing the company and its business units helps identify sources of economic creations and destruction within the company and also its IPO value.
    
    \textbf{Definition (\#\mydef):} The "\NewTerm{Initial public offering (IPO)}" or "\NewTerm{stock market launch}" is a type of public offering in which shares of a company usually are sold to institutional investors[ that in turn, sell to the general public, on a securities exchange, for the first time. Through this process, a privately held company transforms into a public company. Initial public offerings are mostly used by companies to raise the expansion of capital, possibly to monetize the investments of early private investors, and to become publicly traded enterprises. A company selling shares is never required to repay the capital to its public investors. After the IPO, when shares trade freely in the open market, money passes between public investors. Although IPO offers many advantages, there are also significant disadvantages, chief among these are the costs associated with the process and the requirement to disclose certain information that could prove helpful to competitors. The IPO process is colloquially known as going public.

   The methods for valuing companies can be classified commonly in sis groups:
   \begin{enumerate}
        \item Balance sheet (book value/adjusted book value)

        \item Income statement (multiples per sales or other multiples)

       \item Mixed Goodwill (Classic Union of European Accounting Experts standard method)

        \item Cash flow discounting (equity cash flow dividende or other cash flows)

        \item Value creation (Economic profit)

        \item Options (Black \& Scholes types evaluation)
    \end{enumerate}
    In this book as always we will focus only on the mathematical aspect of theses methods and we will provide also a list that contains most common errors!
    
     The methods that are popular (and are conceptually "correct") are those based on cash stochastic flow discounting. These methods view the company as a cash flow generator and, therefore, assessable as a financial asset. We will briefly comment on other methods since - even tough they are conceptually "incorrect" - they continue to be used frequently.
  
  A valuation may be used for a wide range of purpose:
  \begin{enumerate}
      \item In company buying an selling operations:
           \begin{itemize}
               \item For the buyer, the valuation will tell him the highest price he should pay
               \item For the seller, the valuation will tell him the lowest price at which he should be preparer to sell
           \end{itemize}
      \item Valuations of listed companies:
           \begin{itemize}
               \item The valuation is used to compare the value obtained with the share's price on the stock market and to decide whether to sell, buy or hold the share
               \item The valuation of several companies is used to decide the securities that the portfolio should concentrate on and therefore to compare companies
           \end{itemize}
     \item To valuate the price at which shares are offered to the public and therefore to compare it with other assets
     \item  To identify and stratifying the main values drivers of a company
     \item As decision tool (key development point) to continue business:
     	\begin{itemize}
               \item For deciding what products/business lines/countries/customers... to maintain, grow or abandon.
               \item To measure the impact of the company's possible policies and strategies on value creation and destruction
        \end{itemize}
	\end{enumerate} 
	
	\paragraph{Balance sheets-based method}\mbox{}\\\\
	\textbf{Definition (\#\mydef):} The "\NewTerm{Balance sheets-based method}" or "\NewTerm{Shareholders' Equity}" seek to determine the company's value by estimating the value of its assets. These are traditionally used methode that consider a company's value lies basically in it's balance sheet. Therefore this method is a static deterministic viewpoint, which therefore, does not take into account the company's possible future evolution or money's temporary value and also implicitly estimation errors (tolerance interval). Neither do they take into account other factors that also affect the value such as: the industry's current situation, human resources or organizational problems, contracts, etc. that do not appear in the accounting statements.

  Some of the methods are the following: adjusted book value, liquidation value and substantial value.
  
  	This method and all accounting base method are quite poor as they only give a valuation for today times $t$ when the investor is interested to $t+\Delta t$. However this method seems to be the most popular because the required knowledge to apply it is quite low and the results can therefore be understand by the board committees that have most of time no knowledge in engineering finance (their knowledge is limited to spreadsheet softwares sum and averages most time).
  
  	\subparagraph{Book Value}\mbox{}\\\\
  	\textbf{Definition (\#\mydef):} A company's "\NewTerm{book value}", or "\NewTerm{net worth}" or "\NewTerm{common equity}", is the value of the shareholder's equity stated in the balance sheet (capital and reserves). This quantity is also the difference between total assets and liabilities, that is, the surplus of the company's total goods and rights over its total debts with third parties.
 	\begin{tcolorbox}[colframe=black,colback=white,sharp corners]
	\textbf{{\Large \ding{45}}Example:}\\\\
	In the table below the shares' book value (capital plus reserves) is $80$ M\$. It can also be calculated as the difference between total assets ($160$) and liabilities ($40+10+30$), that is $80$ millions dollars.
	\begin{table}[H]
	\begin{center}
		\definecolor{gris}{gray}{0.85}
			\begin{tabular}{|lcclc|}
				\hline
				\multicolumn{1}{c}{\cellcolor{black!30}\textbf{Assets}} &  & 
   & 
  \multicolumn{1}{c}{\cellcolor{black!30}\textbf{Liabilities}}  & 
  \\ \hline
				Cash & $5$ &  & Accounts payable & $40$\\ 
				Account receivables & $10$ &  & Bank Dept & $10$\\ 
				Inventories & $45$ &  & Long term dept & $30$\\ 
				Fixed Assets & $100$ &  & Capital and reserves & $80$\\ \hhline{|=====|}
				\textbf{Total assets} & $\mathbf{160}$ &  & \textbf{Total liabilities} & $\mathbf{160}$\\ \hline
		\end{tabular}
	\end{center}
	\caption[]{Alfa Inc. Official Balance Sheet (Millions Dollars)}
	\end{table}
	\end{tcolorbox}
	The implicit and most of time unknown implicit error of balance sheet and the fact that it is only a static today estimation makes that the book value of a company almost never matches the market value.
	
	\pagebreak
	A company's standard book value may be significantly affected by intangible assets and/or goodwill such as intellectual property (patents and trademarks), brand recognition, and trade secrets. On the balance sheet, intangible assets add to the total asset value and, as such, are a component of the total book value.

	\textbf{Definition (\#\mydef):} Therefore, a "\NewTerm{tangible book value}" (alternately "\NewTerm{net tangible assets}" or "\NewTerm{net tangible equity}") is the total net asset value of a company (book value) minus intangible assets and goodwill.
	\begin{tcolorbox}[title=Remark,colframe=black,arc=10pt]
	In most cases, GAAP (Generally Accepted Accounting Principles) does not allow internally generated intangibles to show up on the balance sheet. 
	\end{tcolorbox}
	
	\subparagraph{Adjusted Book Value}\mbox{}\\\\
	\textbf{Definition (\#\mydef):} The "\NewTerm{adjusted book value}" is a method that seeks to overcome the shortcomings that appear when purely accounting criteria are applied in the valuation.

	When the values and liabilities match their market value, the adjusted net worth is obtained. Continuing with the example used above, we will analyze a number of balance sheet items individually in order to adjust them to their approximate market value. For example if we consider that:
	\begin{itemize}
		\item \textit{Accounts receivable} includes $2$ M\$ of bad debt, this item should have a value of $8$ M\$.
		
		\item \textit{Inventories}, after discounting obsolete, worthless items and revaluing the remaining items at their market value, has a value of $52$ M\$
		
		\item \textit{Fixed assets} (land, buildings and machinery) have a value of $150$ M\$ according to an expert
		
		\item \textit{Accounts payable}, \textit{Bank dept} and \textit{Long-term dept} is equal to their market value
	\end{itemize}
	The adjusted balance sheet would be that show below:
	\begin{table}[H]
	\begin{center}
		\definecolor{gris}{gray}{0.85}
			\begin{tabular}{|lcclc|}
				\hline
				\multicolumn{1}{c}{\cellcolor{black!30}\textbf{Assets}} &  & 
   & 
  \multicolumn{1}{c}{\cellcolor{black!30}\textbf{Liabilities}}  & 
  \\ \hline
				Cash & $5$ &  & Accounts payable & $40$\\ 
				Account receivables & $8$ &  & Bank Dept & $10$\\ 
				Inventories & $52$ &  & Long term dept & $30$\\ 
				Fixed Assets & $150$ &  & Capital and reserves & $135$\\ \hhline{|=====|}
				\textbf{Total assets} & $\mathbf{215}$ &  & \textbf{Total liabilities} & $\mathbf{215}$\\ \hline
		\end{tabular}
	\end{center}
	\caption[]{Alfa Inc. Official Adjusted Balance Sheet (Millions Dollars)}
	\end{table}
	The adjusted book value is the $135$ M\$ (in other words: different from the book value!).
	
	\pagebreak
	\subparagraph{Liquidation Value}\mbox{}\\\\
	\textbf{Definition (\#\mydef):} The "\NewTerm{liquidation value}" is the company's value if it is liquidated, that is, its assets are sold and its debts are paid off. This value is calculated by deducting the business's liquidation expenses (redundancy payments to employees, tax expenses and other typical liquidation expenses) for the adjusted net worth.

	Taking the example given in the table above, if the redundancy payments and other expenses associated with the liquidation of the company Alfa Inc. were to amount $60$ M\$, the shares' liquidation value would be $75$ M\$ ($135-60$).

	Obviously, this method's usefulness is limited to a highly specific situation, namely, when the company is bought with the purpose of liquidating it at a later date. However, it always represents the company's minimum value as a company's value, assuming it continues to operate, is greater than its liquidation value.
	
	\subparagraph{Substantial Value}\mbox{}\\\\
	\textbf{Definition (\#\mydef):} The "\NewTerm{substantial value}" represents the investment that must be made to form a company having identical conditions as those of the company being valued. It can also be defined as the "\NewTerm{asset's replacement value}", assuming the company continues to operate, as opposed to their liquidation value. Normally, the substantial value does not include those assets that are not used for the company's operations (unused land, holdings in other companies, etc.).

	Three types of substantial value are usually defined for the book value and to avoid the reader going up and down in the page let us copy here the above table:
	\begin{table}[H]
	\begin{center}
		\definecolor{gris}{gray}{0.85}
			\begin{tabular}{|lcclc|}
				\hline
				\multicolumn{1}{c}{\cellcolor{black!30}\textbf{Assets}} &  & 
   & 
  \multicolumn{1}{c}{\cellcolor{black!30}\textbf{Liabilities}}  & 
  \\ \hline
				Cash & $5$ &  & Accounts payable & $40$\\ 
				Account receivables & $8$ &  & Bank Dept & $10$\\ 
				Inventories & $52$ &  & Long term dept & $30$\\ 
				Fixed Assets & $150$ &  & Capital and reserves & $135$\\ \hhline{|=====|}
				\textbf{Total assets} & $\mathbf{215}$ &  & \textbf{Total liabilities} & $\mathbf{215}$\\ \hline
		\end{tabular}
	\end{center}
	\end{table}
	\begin{enumerate}
		\item "\NewTerm{Gross substantial value}": this is the assets' value at market price (in the above example the value of: $215$ M\$).
		
		\item "\NewTerm{Net substantial value}" or "\NewTerm{corrected net assets}": this is the gross substantial value less liabilities. It is also know as "\NewTerm{adjusted net worth}", which we have already seen earlier (in the above example the value of: $135$ M\$).
		
		\item "\NewTerm{Reduced gross substantial value}": this is the gross substantial value reduce only by the value of the cost-free debt (in our example: $175=215-40$). The remaining $40$ millions dollars correspond to accounts payable.
	\end{enumerate}
	
	\subparagraph{Book Value and Market Value}\mbox{}\\\\
	 \textbf{Definition (\#\mydef):} The "\NewTerm{price-book value ratio $P$/BV}" (or simply $P/B$), also sometimes known as "\NewTerm{Market-to-Book ratio}", is the ratio of the market value of equity to the book value of equity. Price stands for the current market price of a stock. Book value is the total assets minus liabilities, or net worth, which is the accounting measure of shareholders' equity in the balance sheet:
	 
	where $\mathcal{M}_{\mathcal{E}}$ is the market value of equity, and $\mathcal{B}_{\mathcal{E}}$ the book value of equity.
	 
	 In general, the equity or also the tangible book value has little bearing to its market value. This can be seen in the figure below:
	 \begin{figure}[H]
		\centering
		\includegraphics[scale=0.8]{img/economy/jpm_wfc_price_to_book_ratio.jpg}
		\caption[$P/B$ for JP Morgan Chase and Wells Fargo Company]{$P/B$ for JP Morgan Chase and Wells Fargo Company (source: Source: YCharts)}
	\end{figure}
	
	\paragraph{Income Statemented-Based Methods}\mbox{}\\\\	
	Unlike the balance sheet-based methods, these methods are based on the company's income statement. They seek to determine the company's value through the size of its earnings, sales or other indicators. Thus, for example, it is a common practice to perform quick valuations of companies by multiplying their annual production capacity (or sales) in metric tons by a ratio (multiple). It is also common to value car parking lots by multiplying the number of parking spaces by a multiple and to value insurance companies by multiplying annual premiums by a multiple. This category includes the methods based on the PER: accord to this method, the share's price is a multiple of the earnings.
	
	 The income statement of an imaginary company is show in the table below
	\begin{table}[H]
	\begin{center}
		\definecolor{gris}{gray}{0.85}
			\begin{tabular}{|lr|}
				\multicolumn{2}{c}{\cellcolor{black!30}\textbf{Imagininary LLC Income Statement (M\$)}}  \\ \hline
				Sales & $300$ \\
				Cost of sales & $136$\\ 
				General expenses & $120$\\ 
				Interest expense & $4$\\ 
				\hhline{|==|}
				\textbf{Earnings before tax} & $\mathbf{40}$\\ 
				Tax ($35\%$) & $14$\\ 
				\hhline{|==|}
				\textbf{Total assets} & $\mathbf{26}$\\ \hline
		\end{tabular}
	\end{center}
	\end{table}
	
	\subparagraph{Value of Earnings with PER}\mbox{}\\\\
	According to this method, the equity value is obtained by multiplying the annual net income by the PER ratio (\SeeChapter{see section Economy page \pageref{price earning ratio}}), that is:
	
	The figure below shows the evolution of the PER for the S\&P 500 for little bit more than a hundred years:
	 \begin{figure}[H]
		\centering
		\includegraphics[scale=1]{img/economy/pe_sp500_evolution.jpg}
		\caption[$10y$ P/E Ratio for S\&P 500]{$10y$ P/E Ratio for S\&P 500 (source: ValueExplorer.com)}
	\end{figure}
	
	Sometimes, the relative PER is also used, which is simply the company's PER divided by the country's PER.

	\subparagraph{Value of Dividends}\mbox{}\\\\
	Dividends are the part of the earnings effectively paid out to the shareholders and, in most cases, are the only regular flow recevied by shareholders. According to this method, a shar'es value is the net presetn value of the dividends that we expect to obtain from it. In the perpetuity case, that is, a company from which we expect constant dividends every year.
	
	The Bates formula establishes a relationship between the current price earning ratio (\SeeChapter{see section Economy page \pageref{price earning ratio}}) at time $0$ ($\text{PER}_0$) of a stock and its future P/E after $n$ years ($\text{PER}_n$), given the dividend per share $D_i$ at period $i$, the risk free rate of return $t\%$ during the periods (supposed as constant). Obviously is formula is based on the method used to determine the present value of cash flows on a given period ($n$ years) and the terminal value $V_n$ of the company.
	
	We start for the proof from the company valuation formula or "Irving Fisher fundamental value\index{Irving Fisher fundamental value}" given under simplistic assumptions by a simple present value calculation (\SeeChapter{see section Economy page \pageref{net present value}}):
	
	according to the notations in usage in the field of corporation evaluation.
	
	Knowing that in general:
	
	where BPS represent the benefice per share a the year $i$ and that:
	
	where $d$ is the fraction of the benefice per share distributed to shares holders, we can rewrite the above relation as following:
	
	Following the same assumption as the Gordon-Shapiro model (\SeeChapter{see section Economy page \pageref{gordon shapiro model}}) and using the corresponding usual notation we know that we can write:
	\begin{equation}
		\text{BPS}_n=\text{BPS}_0(1+g)^n \Rightarrow \text{BPS}_i=\text{BPS}_0(1+g)^i
	\end{equation}
	with $\text{BPS}_0$ being the actual benefice per share.
	
	We get then get:
	\begin{equation}
		\dfrac{\text{BPS}_i}{\text{BPS}_0}=(1+g)^i
	\end{equation}
	So rearranging the prior previous relation we get:
	
	The last equality is the "\NewTerm{Bates Formula}":
	
	However for more elegance it is convenient to notice that the first sum is a geometric series of common ratio:
	
	Which leads us to write (\SeeChapter{see section Economy page \pageref{power sum in finance}}):
	
	So we can rewrite the Bates formula as following:
	
	Empirical evidence shows that the companies that pay more dividends (as a percentage of their earning) do not obtain a growth in their share price as a result. This is because when a company distributes more dividends, normally it reduces its growth because it distributes the money to its shareholders instead of plowing it back into new investments.
	
	Empirical observations also show that shares value increase many times just before dividend pay time and decease just after.
	
	\subparagraph{Sales Multiples}\mbox{}\\\\
	This valuation method, which is used in some industries with a certain frequency, consists of calculating a company's value by multiplying its sales by a number. For example, a pharmacy is often valued by multiplying its annual sales (in dollars) by $2$ or another number, depending on the market situation. It is also a common practice to value a soft drinkg bottling play by mujltiplying its annual sales in lters by $500$ or another number, depending on the market situation.
	
	\paragraph{Goodwill-Based Methods}\mbox{}\\\\
	Generally speaking, goodwill is the value that a company has above its book value or above the adjusted book value, Goodwill seeks to represent the value of the company's intangible assets, which often do not appear on the balance sheet but which, however, contribute an advantage with respect to other companies operating in the industry (quality of the customer portfolio, industry leadership, brands, strategic alliances, etc.). The problem arises when one tries to determine its value, as there is no consensus regarding the methodology used to calculate it. Some of the methods used to value the goodwill rise to the various valuation procedures described below.

	These methods apply a mixed approach: on the one hand, they perform a static valuation of the company's assets and, on the other hand, they try to quantify the value that the company will generate in the future. Basically, these methods seek to determine the company's value by estimating the compbine value of its assets plus a capital gain resulting from the value of its future earnings: they start by valuing the company's assets and the add a quatity relative with future earnings.	
		
	\pagebreak
	\subsubsection{Capital Goods}
	A capital good is a durable good (one that does not quickly wear out) that is used in the production of goods or services. Capital goods are one of the three types of producer inputs, the other two being land and labor, which are also known collectively as primary factors of production. In the study of economic systems and in Marxian economics, the term means of production is often used with the same meaning as capital goods. This classification originated during the classical economic period and has remained the dominant method for classification.

	Capital goods are acquired by a society by saving wealth which can be invested in the means of production. In terms of economics one can consider capital goods to be tangible. They are used to produce other goods or services during a certain period of time. Machinery, tools, buildings, computers, or other kind of equipment that is involved in production of other things for sale represent the term of a Capital good. The owners of the Capital good can be individuals, households, corporations or governments. Any material that is used in production of other goods also is considered to be capital good.
	
	Capital goods undergo a gradual depreciation due to wear or obsolescence. This decline in value recorded as an expense in accounting in many countries around the world, is named "\NewTerm{book depreciation}". The reader must not confuse financial amortization as seen in the section of Economy and corresponding to debt repayment and the accounting depreciation that is a decrease in value of the means of production.
	
	Certain assets have lost value fairly uniform in time unlike others that depreciate quickly the first few years, some are just empirical at defined by governments laws. Here, we discuss some of the accounting methods used in practice that describe either of these phenomena.
	
	\paragraph{Linear Amortization}\mbox{}\\\\
	\textbf{Definition (\#\mydef):} We speak of a "\NewTerm{linear amortization}" of a capital good when its asset value is decreased (depreciated) of a periodic constant  amount (annually in accounting) during its lifetime.
	
	Thus, if we denote by $A_n$ the amount of the $k$-th amortization and the initial value of the capital good $V_0$ and $V_n$ its desired final value (which in most cases will be zero), we have obviously:
	
	It is possible to obtain the depreciation value using the function \texttt{SLN( )} of any spreadsheet software like  of Microsoft Excel 11.8346.

	The equivalent constant depreciation rates based on the purchase and residual value is then given of course by:
	
	Notice that this constant rate and the final value are sometimes imposed by legislation in some countries (as is the case in Switzerland for example) to a residual value of zero or non-zero as defined by the law!
	
	\paragraph{Arithmetic Declining Amortization}\mbox{}\\\\
	\textbf{Definition (\#\mydef):} When the asset value (depreciation) of good decreases inversely to the order of periods (accounting years ) then we talk of "\NewTerm{arithmetic declining depreciation}" (there is therefore so no amortization rate that can be required by the government!).
	
	For example, a good of $4$ years life, will be depreciated the first year $4/10$, $3/10$ the second years, $2/10$ the third year, $1/10$ the last year. The common base "$10$" (in this example) is the arithmetic sum of $1 + 2 + 3 + 4$ so that the sum of all are equal to unity. This is a purely American tax rule named "\NewTerm{Sum-of-years'-digits}" (SYD).
	
	\begin{tcolorbox}[title=Remark,colframe=black,arc=10pt]
	Do not confuse the "\NewTerm{arithmetic declining depreciation}" with the "\NewTerm{declining balance}" that consist to apply a multiplier (enacted by the tax authorities) to the linear amortization and that we will not discuss here (except on explicit request of readers). It can only be used for new goods and does not apply to all types of asset. This declining balance rate will apply each year on the residual accounting value of the good.
	\end{tcolorbox}
	Given the $k$-th amortization and $V_0$ the initial value of the good and $V_n$ its desired final value (which in most cases will be zero) then we have:
	
	That can be written in the most general case:
	
	and as we proved it in the section of Sequences And Series:
	
	Which takes us to write:
	
	You can get the amortization at a period $k$ with a spreadsheet software using for example the function  \texttt{SYD( )} of Microsoft Excel 11.8346.
	
	\paragraph{Geometric Declining Amortization (declining balance)}\mbox{}\\\\
	\textbf{Definition (\#\mydef):} When the capital value of good decreases at a constant rate of depreciation based on the residual value of the previous period (that is to say like compound interest with negative rate), then we talk "\NewTerm{Simple geometric declining amortization}".
	Thus, the value of the good after $n$ years is given by:
	
	with then (elementary algebra):
	
	\begin{tcolorbox}[title=Remark,colframe=black,arc=10pt]
	We see that the value $t\%$ is in the range between $[0,1[$. The limit $(1-t\%)^n$ as $n$ tends to big values is never zero. Thus, the residual value never will either!
	\end{tcolorbox}
	Knowing by definition of this depreciation that:
	
	we get:
	
	By injecting the expression of the in the above relation, we get:
	
	We notice that the depreciation values do not need to know the depreciation rate explicitly. Just know the final and initial value is enough. This is precisely why the \texttt{DB( )} function of the spreadsheet software Microsoft Excel 11.8346 does not require the depreciation rate.
	
	We notice also that this constant rate and the final value are sometimes imposed by legislation in some countries (as is the case in Switzerland for example) to a residual value of zero or non-zero also defined by the law!
	\begin{tcolorbox}[title=Remark,colframe=black,arc=10pt]
	The declining geometric depreciation is particularly suitable for goods with a very sharp depreciation in the early years.
	\end{tcolorbox}
	
	\pagebreak
	\subsubsection{Wages model}
	Obviously, the assumption that all people are elementally selfish organic logic robots is as offensive as it is incorrect (at the beginning of medical trials a lot of physicians were also thinking the same...). For such a model to be applied it is the environment (context) that must be strongly normalized and this, not in the purpose to be able to put a sticker to every individual but (furthermore when we know that almost all jobs will disappear in a near horizon), to decrease the inner entropy and reduce decisions costs significantly and thereafter increase efficiency.
	
	Build a mathematical model for lower bound wages has many purposes:
	\begin{enumerate}
		\item It gives a positive professional image to employees and to the market as it as proof of transparency and also a proof of rigor on how the company works.
		
		\item It helps companies to plan annual budget and forecast expenses in long time projects and investments.
		 
		\item It give an opportunity to spare time in decision making when hiring people or consultants or for each annual employee review. 
		
		\item It avoids to forget some factors that need after an adjustment of the wage and thus costs in administrative processes.			
		
		\item Avoid some subjective and biased  inequalities (typically between men and woman or ethnics) but justify some others.
		
		\item It helps employees to know what they have to do to have a better wage and therefore encourages self-motivation and a better analysis of well described job opportunities.
	\end{enumerate} 
		
	We can enumerate some constraints to the model:
	\begin{enumerate}
		\item The wage $W$ is a real positive value (therefore there is no upper bound).			
		\item In the worst situation the employee must be able to live with a minimum of comfort.
		\item The wage has not to take in consideration the family situation of the employee.
		\item Each parameter must be no subjective and has to be something that can be quantified.
		\item The model must be based mainly on relative factors to some given reference index and values. Indeed, this model cannot be applied identically in all countries as the taxes, prices and money value are not the same (it must therefore be a "background independent" model as in General Relativity).
		\item The model is not error-free and therefore has a standard deviation (as the variables will be supposed as independent, then we will have a lower bound of the standard deviation as most variables are positively correlated).
	\end{enumerate}
	
	We must focus to build such a model only on parameters that can easily be retrieved by companies and also easily implemented. Here is a list of what could be taken at least into consideration:
	\begin{enumerate}
		\item The gender of the employee denoted by $S$ (sport shows that physical endurance are not the same between man and woman).
		\item The age of the employee denoted by $A$.
		\item The education level in years of studies after the period of compulsory education denoted by $E$.
		\item The score to cognitive/physics tests (depending on the job)
		\item The number of people the employee will have to manage denoted by $N_e$.
		\item The amount of cash (budget) the employee will have to manage denoted by $B$.
		\item The number of years of experience in the same field denoted by  $Y_e$.
		\item The number of useful foreign languages $N_l$ that the employee know and the corresponding level $L_l$.
		\item The reputation (or "connections") in the real life (can be quantified with social networks and followers by making a clear distinction between ingoing and outgoing connections obviously...)
		\item The number of people that the employee has to manage\footnote{With a maximum threshold of $150$ as above this number, named \NewTerm{Dunbar's number}\index{Dunbar's number}, it has been shown that it is the limit to the number of people with whom one can maintain stable social relationships.}
	\end{enumerate}
	\StickyNote[2.5cm]{\LARGE To finish and write before year 2020}[6.5cm]
	
	\pagebreak
	\subsection{Project Management}
	Project management is the discipline of initiating, planning, executing, controlling, and closing the work of a team to achieve specific goals and meet specific success criteria. A project is a temporary endeavor designed to produce a unique product, service or result with a defined beginning and end (usually time-constrained, and often constrained by funding or deliverables) undertaken to meet unique goals and objectives, typically to bring about beneficial change or added value. The temporary nature of projects stands in contrast with business as usual (or operations), which are repetitive, permanent, or semi-permanent functional activities to produce products or services. In practice, the management of these two systems is often quite different, and as such requires the development of distinct technical skills and management strategies.
	
	As many technical Project Management subjects are related to Corporate Finance (empirical index, sensitivity analysis) and hence to Economy (Actuarial, Interests calculations, Loan Amortization/Repayment, Portfolio Theory and Forecasting), Optimization techniques (\SeeChapter{see section Numerical Methods page \pageref{operational research}}) and also Industrial Engineering (see section of the same name page \pageref{industrial engineering}) we will not come back on this subjects that we have already studied previously in details or that we will study in details.
	
	We will focus here only on subjects that are very specific to quantitative project management and are commonly designated under the name of "\NewTerm{Project Management Engineering}".
	
	\subsubsection{Probabilistic PERT (Beta-PERT Distribution)}\label{probabilitic pert}
	Always remaining within the framework of statistics and probabilities with respect to mathematical management techniques, there is a project management empirical law (domain we assume known by the reader as trivial) widely used in the context of "\NewTerm{probabilistic PERT}" (where PERT stands for Program Evaluation and Review Technic) and included in a lot of specialized softwares to make cost and time predictions for project plannings (\SeeChapter{see section Graph Theory page \pageref{gantt chart}}). We speak then about "\NewTerm{Probabilistic Network Evaluation Technique (PNET)}".

	This empirical distribution, named "\NewTerm{beta distribution}\index{beta distribution}" or sometimes "\NewTerm{PERT law}\index{PERT law}", or also "\NewTerm{Beta-PERT Distribution}\index{Beta-PERT Distribution}", is often presented as follows in the books and without proof (...) and applies as well to costs or time estimation:
	
and gives the expected mean duration (or cost) $t_{Pr}$ of an elementary task (not downsizable into sub-tasks) where we have $t_O,t_V,t_P$ which are respectively the optimistic, most likely and pessimistic duration (or cost) of the task. We will show below the origin of this relation and why it is partially false when written in a such simple way!

	\begin{tcolorbox}[title=Remark,colframe=black,arc=10pt]
This classical approach dates from 1962 and is due to C.E. Clark. It is said to be based on a project task database analysis. In facts you should always check what kind of law (probabilistic distribution) follow the tasks in your company and not apply just stupidly and simply the beta distribution.
	\end{tcolorbox}

The principles of this approach is based on the idea that the duration (or cost) of each project task is considered random and Beta distribution is systematically used as a more accurate approach than the triangular distribution (\SeeChapter{see section Statistics page \pageref{triangular distribution}}). The parameters of this distribution as we will prove it are determined through calculation of a fairly strong assumption, based on  extreme values $a$ and $b$ as the execution time (cost) can take, and is modal value.

So we just have to ask ourself following two questions:
	\begin{enumerate}
		\item What is the minimum physically possible time or cost execution?
		\item What is the maximum physically possible time or cost execution?
	\end{enumerate}
to obtain the parameters $a,b,M_0$, which are then used to calculate the average (expected mean), the variance and the shape of the distribution of this random time/cost.

Then we determine the critical path of the project (using the Potential Metra Method supposed to be known by the reader after its study of the section on Graph Theory), placing himself in some universe and using a given probabilistic duration for each task obtained with the beta distribution, allowing to find afterwards the probabilistic critical paths and thus the total probabilistic duration/cost of the whole project.

This done some practitioners place themselves then in random universe where the project duration is considered as the sum of the durations of critical path tasks previously identified. They use then the central limit theorem that we have proved in section Statistics (recall that this theorem states, under generally respected conditions that the random variable consisting of a sum of $n$ independent random variables identically distributed approximately follows a Normal distribution, whatever are the original as soon as $n$ is large enough) to approximate the density distribution law of the whole project itself. Even if this practice is simple and useful it is in fact dangerous because in reality task are not equally distributed and can takes the practitioners to huge bias error. 

	\begin{tcolorbox}[title=Remark,colframe=black,arc=10pt]
In a rigorous way the practionner should use Monte Carlo simulations with special softwares like @Risk of Palisade to estimate the probabilistic time/cost of the whole project.
	\end{tcolorbox}
	
If $X_i$ is the corresponding cost or duration of task $i$ the using the property of linearity of the mean (\SeeChapter{see section Statistics page \pageref{properties of the mean}}) we have for the mean cost or duration of the whole project:
	
and in the strong hypothesis that all tasks of a project are independent we can still write that the global variance (\SeeChapter{see section Statistics page \pageref{properties of the variance}})  is:
	
Now let us go a little bit more deeply in technical tools for project duration and cost estimation.

Recall that we saw in the sections of Statistics and of Differential and Integral Calculus that:
	
and:
	
In management we notice that often (applying an Anderson-Darling Goodness of fit test as seen in the section of Statistics) that tasks durations follow a distribution that we name "\NewTerm{beta distribution of the first kind}\label{beta distribution application}" (\SeeChapter{see section Statistics page \pageref{beta distribution}}) given by:
	
For an interval $[a, b]$ which includes $x$, we can obtain the more general form:
	
\begin{theorem}
Let us check that we have well:
	
\end{theorem}
\begin{dem}
	By the change of variable:
	
we get:
	
\end{dem}
Let us now determine the expected mean:
	
Always with the same change of variable, we get:
	
But we know from the study of the section on Differential and Integral Calculus that:
	
Thus:
	
Let us calculate now the variance using the Huygens relation that we have proved in the section Statistics:
	
First we calculate $\text{E}(X^2)$:
	
Always with the same change of variable, we get:
	
But we have:
	
Thus:
	
To finish:
	
Let us now calculate the modal $M_0$ value of this distribution (for recall of the section Statistics the modal value is by definition the overall maximum of the distribution function):
	
To get the expression of the modal value we just need to solve the equation:
	
After derivation, we get:
	
dividing by $(x-a)^{\alpha-1}(b-x)^{\gamma-1}$ we get:
	
that is to say:
	
Now the reader will have noticed that the value $a$ is the smallest value and the greatest is $b$. In between there is therefore the modal value $M$.

In project management $a$ corresponds to the optimistic duration (or cost) $t_O$ and $b$ the duration $t_P$ duration (or cost) of a task.
	
or:
	
This means that we have:
	
as well as:
	
And finally:
	
This gives us to resume the following relations:
\begin{equation}
  \addtolength{\fboxsep}{5pt}
   \boxed{
   \begin{gathered}
		\begin{aligned}
			M_0&=\dfrac{2(a+b)+\sqrt{2}(b-a)}{4}\\
			\text{V}(X)&=\dfrac{(b-a)^2}{6^2}\\
			\text{E}(X)&=\dfrac{a+4M_0+b}{6}
		\end{aligned}
   \end{gathered}
   }
\end{equation}
that is assimilated in some textbooks to the Beta-PERT distribution with a scale parameter $\lambda$ equal to $4$ as a more general derivation (that we will not present here) gives :
	
	\begin{tcolorbox}[title=Remark,colframe=black,arc=10pt]
The last two expressions of the variance and expected mean are the ones you can find in any project management book (of course without proof...) such as the famous PMBOK (Project Management Body of Knowledge). Unfortunately, there are two errors in this project management reference book as far as y know (the first error was corrected at the beginning of the years 2010). Indeed, the first error is that the modal value is said to be imposed by the project manager but as we know, in fact it should be calculated with the first of the three above relations using the pessimistic and optimistic task duration (or cost)!!! The second mistake is that the PMBOK say that the modal value is always smaller than the expected mean. Which is obviously wrong! To check this we can put for example $a = 0$:
   
and simplifying by $b$ we immediately see that:
	
	\end{tcolorbox}

It is important to know that exactly the same developments made so far are valid with the following formulation of the 4-parameter beta distribution (which will also be very useful in the section of Industrial Engineering for reliability calculations):
	
The latter formulation is that available in spreadsheets softwares such as Microsoft Excel version of 11.8346 for example using the \texttt{BETADIST()} function or its inverse using the \texttt{BETAINV()} function.

We then get for the modal value, the variance and expected mean (we can detail again all calculations if some readers want to...):
	
But to get the P.M.I. result you must use in spreadsheet softwares (if you put this two values in the latter relations you can quickly check that you find back the classic modal value, variance and expected mean of the P.M.I.):
	
To calculate the probability or quantile in the English version of Microsoft Excel 14.0.7153:
	\begin{itemize}
		\item \texttt{BETA.DIST(quantile},$3+\sqrt{2}$,$3-\sqrt{2}$,\texttt{TRUE},\texttt{optimistic value},\texttt{pessimistic value)}
		\item \texttt{BETA.INV(probability},$3+\sqrt{2}$,$3-\sqrt{2}$,\texttt{optimistic value},\texttt{pessimistic value)}
	\end{itemize}
	\begin{tcolorbox}[colframe=black,colback=white,sharp corners]
	\textbf{{\Large \ding{45}}Example:}\\\\
	Plot of the distribution and cumulative function for the beta law of parameters $(t_o,t_v,t_p)=(0,1,3)$:
	\begin{figure}[H]
	\centering
	\includegraphics[scale=0.75]{img/economy/beta_distribution.eps}
	\caption{Distribution and cumulative function of the beta law}
	\end{figure}
	\end{tcolorbox}
We also define the "\NewTerm{action risk}" by the ratio (whose interpretation is left to the project managers and customers...):
	
	By the NASA Probabilistic analysis of cost and/or schedule estimates is required (in 2014) for tightly coupled programs and single-project programs (regardless of life-cycle cost), and projects with an LCCE (Life-Cycle Cost Estimate) greater than \$ $250$ million. When the probabilistic analysis is developed for only one parameter (i.e., cost or schedule) or when generally referring to a probabilistic assessment of the level of confidence of achieving a specific goal, the analysis is referred to merely as a "confidence level". When the probabilistic analysis is developed to measure the likelihood of meeting both cost and schedule, the analysis is referred to as a "\NewTerm{joint cost and "schedule confidence level (JCL)}. A JCL is defined as the probability that actual cost and schedule will be equal to or less than the targeted cost and schedule. For example, a $70\%$ JCL is the point on the joint cost and schedule probability distribution curve where there is a $70\%$  percent probability that the project or program will be completed at or lower than the estimated cost and on or before the estimated schedule.
	
	\pagebreak
	\subsubsection{Project planning variance reduction}
	As we know a lot of projects budgeting and time planning lack in accuracy because there is always an implicit variance to each task on the critical path and most project manager don't have the time to do detailed planning as for most board comitee (and even most project managers!!!) they is no reason that the total variance of a project decrease if we increase the detailes of the planification (that is to say: we detail a maximum of tasks).
	
	We suppose here that we will work in the framework of softwares used by project management engineers and able to display for each task and early finish ($t_o$), expected finish ($t_V$) and late finish ($t_p$). Even if there advanced risk modelization softwares doing this work in cutting edge technology companies (@Risk, Crystal Ball, Risky Project, etc), most project managers calculate these three point estimates based on a rule of thumb that is most of time of the type:
	
	where $b$ is a security factor that we will here name the "bias" of estimation.

	\begin{theorem}
	Under some assumptions relevant in real projects variance will decrease if the planning is detailed and that this variance tends to zero when the number of detailed tasks tends to infinity (I always try to explain without maths that this is obvious since if we do a task for every gesture and action in the project we forget less stuff than doing just a macroscopic planning).
	\end{theorem}

	For the proof let we have only $3$ known assumptions (perhaps they are other hidden one but not identified so far...):
	\begin{enumerate}
		\item[H1.] When a project manager estimates a task duration their brain use a non-robust estimator that is the mean $\mu$.
		
		\item[H2.] Tasks duration are independent and therefore their covariance is zero.

		\item[H3.]  The project managers always have the same biased $b$ (percentage) for all tasks \footnote{In practice I see that most project managers have a significant bias much greater than $20\%$ in task duration and budget estimation}.
	\end{enumerate}
	\begin{dem}
	Let us consider the case (that can be easily generalized with high-school maths) of a task of duration $T$ of mean $\mu$ and standard-deviation:
	
	We take the special case where we detail this task of duration $T$ into two critical (ie.e. consecutive) sub tasks (still with same bias) of non-zero duration  such that:
	
	where obviously $w_1+w_2=T$. This is quite a natural process for most project managers until now so nothing new...
	
	But now let us calculate the standard deviation of the sum under the assumption of independence and stability:
	
	So we see that as if $0<w_i<1$ then we have:
	
	as:
	
	Because:
	
	So as we see, the deeper the granularity of a project planning is, the lower is the variation the latter. That is:
	
	\begin{flushright}
		$\square$  Q.E.D.
	\end{flushright}
	\end{dem}
	It is quite easy to notice that in the best case, for every $i>1$ we minimize $\sigma_{\sum T_i}$ if all $w_i$ are equal to:
	
	For example with $i=2$ in the best case we can reduce the initial variance of almost $30\%$, if $i=3$ of almost $42\%$, if $i=4$ of exactly $50\%$, if $i=5$ of almost $55\%$ and so on. The variance reduction relation is simple given in the general best case by: 
	 
	This result corrobarte the fact the division of labor increases production. In other words, when the tasks involved with producing a good or service are divided and subdivided, workers and businesses can produce a greater quantity of output. In his observations of pin factories, Adam Smith observed that one worker alone might make $20$ pins in a day, but that a small business of $10$ workers (some of whom would need to do two or three of the $18$ tasks involved with pin-making), could make $48,000$ pins in a day. How can a group of workers, each specializing in certain tasks, produce so much more than the same number of workers who try to produce the entire good or service by themselves? Smith offered three reasons.

	First, specialization in a particular small job allows workers to focus on the parts of the production process where they have an advantage. (In later chapters, we will develop this idea by discussing comparative advantage.) People have different skills, talents, and interests, so they will be better at some jobs than at others. The particular advantages may be based on educational choices, which are in turn shaped by interests and talents. Only those with medical degrees qualify to become doctors, for instance. For some goods, specialization will be affected by geography—it is easier to be a wheat farmer in North Dakota than in Florida, but easier to run a tourist hotel in Florida than in North Dakota. If you live in or near a big city, it is easier to attract enough customers to operate a successful dry cleaning business or movie theater than if you live in a sparsely populated rural area. Whatever the reason, if people specialize in the production of what they do best, they will be more productive than if they produce a combination of things, some of which they are good at and some of which they are not. 

	Second, workers who specialize in certain tasks often learn to produce more quickly and with higher quality. This pattern holds true for many workers, including assembly line laborers who build cars, stylists who cut hair, and doctors who perform heart surgery. In fact, specialized workers often know their jobs well enough to suggest innovative ways to do their work faster and better.

	A similar pattern often operates within businesses. In many cases, a business that focuses on one or a few products (sometimes named its "core competency") is more successful than firms that try to make a wide range of products. 

	Third, specialization allows businesses to take advantage of economies of scale, which means that for many goods, as the level of production increases, the average cost of producing each individual unit declines. For example, if a factory produces only 100 cars per year, each car will be quite expensive to make on average. However, if a factory produces $50,000$ cars each year, then it can set up an assembly line with huge machines and workers performing specialized tasks, and the average cost of production per car will be lower. The ultimate result of workers who can focus on their preferences and talents, learn to do their specialized jobs better, and work in larger organizations is that society as a whole can produce and consume far more than if each person tried to produce all of their own goods and services. The division and specialization of labor has been a force against the problem of scarcity.
	
	\pagebreak
	\subsubsection{Process Reliability}
	Almost every people and even child know what why process are important and how to draw them using various normalized symbols (ISO 5807:85 or BPMN) as show the figure below:
	\begin{figure}[H]
		\centering
		\includegraphics[scale=0.68]{img/economy/bpmn.jpg}
		\caption{BPMN process representative diagram with Microsoft Visio 2010}
	\end{figure}
	But in some high level and cutting edge industries just drawing process is not enough. We consider mainly three situations about processes:
	
	\begin{itemize}
		\item An organization has processes, people respect them or not and there is not statistical analysis about their usage.
		
		There is nothing to say about this as it is what 99.9999\% of undergraduate project managers and business analysts do.

		\item  An organization has processes, people respect them or not and there is a Six Sigma statistical analysis about the capability of the processes or their subparts (\SeeChapter{see section Industrial Engineering page \pageref{capability indices}}).
		
		For the capability analysis we will study and prove in details in the section of Industrial Engineering that we can (or have to) calculate the following indicators:
		
		and:
		
		associates with various run charts (\SeeChapter{see section Industrial Engineering page \pageref{run chart}}) to do real scientific robust process management analysis.

		\item An organization has processes, people respect them or not and there is a reliability analysis of the processes or their subparts (\SeeChapter{see section Industrial Engineering page \pageref{capability indices}}).
		
		This is typically used in Nuclear Engineering or high level Pharmaceutical products quality tracking.
		
		The mathematical relations used in this case are those proven in the section of Industrial Engineering during our study of Topological Systems in Preventive Maintenance (\SeeChapter{see section Industrial Engineering page \pageref{preventive maintenance}}) also associated with Monte Carlo simulations (\SeeChapter{see section Numerical Methods page \pageref{monte carlo simulations}}). That is to say:
	\begin{itemize}
		\item The reliability (probability of operation) of a pure series topology:
		
		\item 	The reliability (probability of operation) of a pure parallel topology:
		
		\item 	The reliability of a $k/n$ topology:
		
		\item 	The reliability of a series topologies in parallels:
		
		\item 	The reliability of parallel topologies in series:
		
		\item The reliability of bridged topology in series:
		
	\end{itemize}
	\end{itemize}
	Put these calculations is quite easy even with a spreadsheet software. But obviously it is strongly recommended to use specialized softwares the add-in Process Simulator of ProModel\textsuperscript{\textregistered} for Microsoft Office Visio.
	
	\subsubsection{Divide and rule?}
	Divide and rule (or divide and conquer) in management is gaining and maintaining a better control by breaking up groups into pieces that individually have less incompatibilities. The concept refers to a strategy that breaks up existing structures, and especially prevents individuals in group to disagree, fight together or and generate issues and poor efficiency.
	\begin{center}
		\includegraphics[scale=1]{img/economy/divide_and_rule.jpg}
	\end{center}
	We have proved in the section of Statistical Mechanics and of Statistics above that for a group of items (individuals) having a unique common property\footnote{For example a property: Technical skill with a binary value of the type "Yes"/"No"} (characteristic) its entropy was given by:
	
	Now let us consider the case of a population of size $P$ with only one property (so that we can split the population in two distinct groups) in proportion:
	
	and therefore $w_2=1-w_1$ and $w_1P+w_2P=P$. So the total entropy if we consider the two possible groups the first having a proportion $p_1^1$ of $w_1$ and $p_1^2$ of $w_2$ (and therefore the second group has $p_2^1=1-p_1^1$ and $p_2^2=1-p_1^2$):
	
	A first simplification by $P$ gives:
	
	We can work in percentage of the population $P$. So if we put $P=100\%=1$, we get:
	
	So we have a function of three variables $S_\text{tot}(w_1,p_1^1,p_1^2)$ as for recall:
	
	If we fix one of them we have a function of two variable that we can easily plot and study. So let us fix $w_1$.
	
	First case we fix $w_1=0.5$ and we use the following Maple 4.00b command lines:\\
	
	\texttt{>w\_1:=0.5;w\_2:=1-w\_1;\\
	>p\_21:=1-p\_11;p\_22:=1-p\_12;\\
	>S\_tot:=((p\_11*w\_1+p\_12*w\_2)*log10(p\_11*w\_1+p\_12*w\_2)-(p\_11*w\_1*log10(p\_11*w\_1)\\
	+p\_12*w\_2*log10(p\_12*w\_2)))/(p\_11*w\_1+p\_12*w\_2)+((p\_21*w\_1+p\_22*w\_2)*\\
	log10(p\_21*w\_1+p\_22*w\_2)-(p\_21*w\_1*log10(p\_21*w\_1)+p\_22*w\_2*log10(p\_22*w\_2)))/\\
	(p\_21*w\_1+p\_22*w\_2);\\
	>plot3d(S\_tot,p\_11=0.1..0.9,p\_12=0.1..0.9,filled=true,axes=boxed);}
	
	That gives:
	\begin{figure}[H]
		\centering
		\includegraphics[scale=1]{img/economy/entropy_maple_plot_01.jpg}
		\caption[]{$S_\text{tot}$ Maple 4.00b with $w_1=0.5$}
	\end{figure}
	\begin{figure}[H]
		\centering
		\includegraphics[scale=1]{img/economy/entropy_maple_plot_02.jpg}
		\caption[]{$S_\text{tot}$ Maple 4.00b with $w_1=0.25$}
	\end{figure}
	\begin{figure}[H]
		\centering
		\includegraphics[scale=1]{img/economy/entropy_maple_plot_03.jpg}
		\caption[]{$S_\text{tot}$ Maple 4.00b with $w_1=0.95$}
	\end{figure}
	\begin{figure}[H]
		\centering
		\includegraphics[scale=1]{img/economy/entropy_maple_plot_04.jpg}
		\caption[]{$S_\text{tot}$ Maple 4.00b with $w_1=0.99$}
	\end{figure}
	With the above chart we can conclude that:
	\begin{itemize}
		\item If the both groups or individuals are almost in the same proportions, it is a bette strategy to divide them in two groups as it minimize entropy. Divide and rule is better!

		\item If one of the group is in huge majority, it is better to put everybody together to minimize the entropy. Merge and rule is better!
	\end{itemize}
	
	\pagebreak
	\subsection{Lean Management (Six Sigma Process)}

As we saw in the section Statistics, the Normal distribution is a function whose shapes looks like a "bell" curve given by:
	
whose standard deviations are used to give the cumulative probability interval to lie within these limits centered on the average as shown below:
	\begin{figure}[H]
		\centering
		\includegraphics[scale=0.75]{img/economy/six_sigma_gauss.eps}
		\caption{Reminder of sigma intervals for the Normal distribution}
	\end{figure}
This being recalled, we also presented in the section of Industrial Engineering the joint probabilities as part of Six Sigma analysis tools for a serial process/workflow chain $W$.

In fact the mentioned processes/workflow are not necessarily industrial processes but can be treated under identical assumptions to arbitrary processes (administrative, procedures, check-lists, etc.).

We had seen that the joint probability of such a process/workflow is named in Six Sigma "\NewTerm{Rolled throughput Yield R.T.Y.}" and was given by (\SeeChapter{see section Probability page \pageref{joint probability}}):
	
For example, applying the above relation gives for a 4-step serial process ($i=4$) whose reliability is $P_i=90\%$ at each step (for more complex structures, remember that we sometimes talk about "\NewTerm{weighted probability trees}" or "\NewTerm{Topological Systems}"):
\begin{figure}[H]
\centering
\includegraphics{img/economy/process_rty.eps}
\caption{Serial/Process Workflow}
\end{figure}
We find ourselves in the final with a $65.6\%$ reliability that is to say a cumulative probability of default of $34.4\%$ for the entire process (you can also imagine that these are sequential tasks in a project!).

	\begin{tcolorbox}[title=Remark,colframe=black,arc=10pt]
The reader that is attentive will have noticed that the serial system is always less reliable than its least reliable component!!
	\end{tcolorbox}

Let us communicate again the Six Sigma table according to the worst situation of a not centered process with an average deviation of (thus to the right but we might take left and the results will be the same) relative to the target and with a unit standard-deviation and also with symmetrical USL and LSL (which restricts the scope):

	\begin{table}[H]
	\begin{center}
		\definecolor{gris}{gray}{0.85}
			\begin{tabular}{|c|c|c|c|c|}
				\hline
				\multicolumn{1}{c}{\cellcolor{black!30}\textbf{$C_p$}} & 
\multicolumn{1}{c}{\cellcolor{black!30}\textbf{$C_{pk}$}} & \multicolumn{1}{c}{\cellcolor{black!30}\textbf{Defectives (PPM)}} & \multicolumn{1}{c}{\cellcolor{black!30}\textbf{Sigma Quality Level}}  & \multicolumn{1}{c}{\cellcolor{black!30}\textbf{Criteria}}\\ \hline
		0.5 & 0 & 501,350 & 1.5 & Bad\\ \hline
		0.6 & 0.1 & 382,572 & 1.8 & {}\\ \hline
		0.7 & 0.2 & 274,412 & 2.1 & {}\\ \hline
		0.8 & 0.3 & 184,108 & 2.4 & {}\\ \hline
		0.9 & 0.4 & 115,083 & 2.7 & {}\\ \hline
		1 & 0.5 & 66,810 & 3 & {}\\ \hline
		1.1 & 0.6 & 35,931 & 3.3 & {}\\ \hline
		1.2 & 0.7 & 17,865 & 3.6 & {}\\ \hline
		1.3 & 0.8 & 8,198 & 3.9 & Limit\\ \hline
		1.4 & 0.9 & 3,467 & 4.2 & {}\\ \hline
		1.5 & 1 & 1,350 & 4.5 & {}\\ \hline
		1.6 & 1.1 & 483 & 4.8 & {}\\ \hline
		1.7 & 1.2 & 159 & 5.1 & {}\\ \hline
		1.8 & 1.3 & 48 & 5.4 & {}\\ \hline
		1.9 & 1.4 & 13 & 5.7 & {}\\ \hline
		2 & 1.5 & 13.4 & 6 & Excellent \\ \hline
	\end{tabular}
	\end{center}
	\caption{Defectives and Sigma quality level of a non-center process}
	\end{table}	

where we proved in the section of Industrial Engineering (see page \pageref{quality levels}) that the PPM values were given by:	
	
Which gives for a level of quality of 3 sigma with Maple 4.00b:\\

\texttt{>evalf((1-1/sqrt(2*Pi)*int(exp(-(x-1.5)\string^ 2/2),x=-infinity..(1*3))))}
\texttt{*1E6+evalf((1/sqrt(2*Pi)*int(exp(-(x-1.5)\string^ 2/2),x=-infinity..-(3*(1+1)))))*1E6;}

This gives a value of $\sim 66,810$ that is to say the value of the 6th line in the above table. This corresponds to $\sim 6.68\%$ in terms of cumulative probability of non-quality ($66,810$ divided by 1 million and in percent) and thus respectively to a cumulative probability of $93.32\%$ of quality.

We have the following table that can summarize some important values of the above table by using the Maple 4.00b command given above:

	\begin{table}[H]
	\begin{center}
		\definecolor{gris}{gray}{0.85}
			\begin{tabular}{|p{2cm}|p{1cm}|p{1cm}|p{1cm}|p{1.5cm}|}
				\hline
				\multicolumn{1}{c}{} & 
  \multicolumn{1}{c}{\cellcolor{black!30}$\pm 3\sigma $} & 
  \multicolumn{1}{c}{\cellcolor{black!30}$\pm 4\sigma$} & 
  \multicolumn{1}{c}{\cellcolor{black!30}$\pm 5\sigma$} & 
  \multicolumn{1}{c}{\cellcolor{black!30}$\pm 6\sigma$}\\ \hline
				 {\cellcolor{black!30}Quality \%} & 93.32 & 99.38 & 99.98 & 99.996 \\ \hline
				 {\cellcolor{black!30}$C_p$} & 1 & 1.33 & 1.68 & 2 \\ \hline
				 {\cellcolor{black!30}Judgement} & Bad & Limit & Good & Excellent \\ \hline
		\end{tabular}
	\end{center}
	\caption{Important values of the process control}
	\end{table}
Under the assumption that each step in a process series follows the same law with the same moment and the same deviations from the target then we have for the R.T.Y.:

	\begin{table}[H]
	\begin{center}
		\definecolor{gris}{gray}{0.85}
			\begin{tabular}{|p{2cm}|p{1cm}|p{1cm}|p{1cm}|p{1.5cm}|}
				\hline
				\multicolumn{1}{c}{\cellcolor{black!30}\textbf{Step/Quality \%}} & 
\multicolumn{1}{c}{\cellcolor{black!30}\textbf{$\pm 3\sigma$}} & \multicolumn{1}{c}{\cellcolor{black!30}\textbf{$\pm 4\sigma$}} & \multicolumn{1}{c}{\cellcolor{black!30}\textbf{$\pm 5\sigma$}}  & \multicolumn{1}{c}{\cellcolor{black!30}\textbf{$\pm 5\sigma$}}\\ \hline
		1 & 93.32 & 99.38 & 99.98 & 99.9996 \\ \hline
		7 & 61.63 & 95.73 & 99.84 & 99.9976 \\ \hline
		10 & 50.08 & 93.96 & 99.77 & 99.9966 \\ \hline
		20 & 25.08 & 88.29 & 99.54 & 99.9932 \\ \hline
		40 & 6.29 & 77.94 & 99.07 & 99.9864 \\ \hline
		60 & 1.58 & 68.81 & 98.61 & 99.9796\\ \hline

		80 & 0.40 & 60.75 & 98.16 & 99.9728 \\ \hline
		100 & 0.10 & 53.64 & 97.70 & 99.966\\ \hline
		... & ... & ... & ... & ...\\ \hline
	\end{tabular}
	\end{center}
	\caption{Defectives and Sigma quality level of a non-center process}
	\end{table}	

where each line represents the troughput Rolled Yield (RTY) thus calculated:
	
where $i$ is the number of process steps $(1, 7, 10, 20, 40)$ and where the probabilities are given by the first line of the before last table. Thus, with a quality level of $\pm 3\sigma$ and a deviation to the target of $\pm 1.5\sigma$ for a process of 20 steps identically distributed:
	
This is lamentable! An industrial process in 20 steps of a complete product should have a minimum level of $4\sigma$ quality to be acceptable in terms of yield and quality.

Thus, the objective of Lean Six Sigma in a company is to increase the level of quality with maximum R.T.Y. for a given number of steps in a process.

	\begin{tcolorbox}[title=Remark,colframe=black,arc=10pt]
In the case of parallel or mixed serial/parallel process, the reader is invited to read the section of Industrial Engineering.
	\end{tcolorbox}
	
	\subsubsection{Pareto Analysis}\label{pareto analysis}
	The Pareto chart, as we shall see, is an easy technique (among others) to classify phenomena in order of importance. But the Pareto distribution is often used in business as a basis for stochastic simulation of random variables representing projects  investment (about as often used as the triangular distribution, Beta distribution or log-normal distribution in modern industry).

	Remember that a random variable is said by definition follow a Pareto distribution if its cumulative probability  function is given by (\SeeChapter{see section Statistics page \pageref{pareto distribution}}):

	
	with $k \in \mathbb{R}_{+}$ and $x \geq x_m \geq 0$ (thus x>0) and for the mean (average):
	
	and for variance:
	
	To illustrate this type of representation (Pareto Diagram), we assume that a study to reorganize the LAMBDA company's sales network will lead the sales manager to be interested in the distribution of purchase of  $55,074$ orders (bottom right value in the table below) received during a given year depending on the city where customers are domiciled (we used here the first 200 cities in the imaginary country this is why we have 200 rows in the table below).
	
	To calculate the parameters of the Pareto distribution, the reader should refer to section of Industrial Engineering because in the present section we will have a qualitative approach as do managers, quantitative analysis being reserved mainly for engineers.

	\begin{table}[H]
	\begin{center}
		\definecolor{gris}{gray}{0.85}
			\begin{tabular}{|c|c|c|c|c|c|c|c|c|c|}
				\hline
				\multicolumn{1}{c}{\cellcolor{black!30}\textbf{$n_i$}} & 
\multicolumn{1}{c}{\cellcolor{black!30}\textbf{$N_i$}} & \multicolumn{1}{c}{\cellcolor{black!30}\textbf{$n_i$}} & \multicolumn{1}{c}{\cellcolor{black!30}\textbf{$N_i$}} & \multicolumn{1}{c}{\cellcolor{black!30}\textbf{$n_i$}} & \multicolumn{1}{c}{\cellcolor{black!30}\textbf{$N_i$}} & \multicolumn{1}{c}{\cellcolor{black!30}\textbf{$n_i$}} &
\multicolumn{1}{c}{\cellcolor{black!30}\textbf{$N_i$}} &
\multicolumn{1}{c}{\cellcolor{black!30}\textbf{$n_i$}} &
\multicolumn{1}{c}{\cellcolor{black!30}\textbf{$N_i$}}\\ \hline
		$8965 $ & $8965 $ & $240 $ & $40027 $ & $125 $ & $46378 $ & $88 $ & $50916 $ & $55 $ & $53752$ \\\hline
		$4555 $ & $13520 $ & $236 $ & $40263 $ & $124 $ & $46862 $ & $87 $ & $51003 $ & $54 $ & $53806$ \\\hline
		$3069 $ & $16589 $ & $224 $ & $40487 $ & $123 $ & $46985 $ & $87 $ & $51090 $ & $54 $ & $53860$ \\\hline
		$2336 $ & $18925 $ & $223 $ & $40710 $ & $123 $ & $47108 $ & $86 $ & $51176 $ & $54 $ & $53914$ \\\hline
		$1085 $ & $25987 $ & $202 $ & $41772 $ & $118 $ & $47704 $ & $83 $ & $51597 $ & $48 $ & $54164$ \\\hline
		$934 $ & $26921 $ & $202 $ & $41972 $ & $117 $ & $47821 $ & $81 $ & $51678 $ & $47 $ & $54211$ \\\hline
		$843 $ & $27764 $ & $200 $ & $42172 $ & $115 $ & $47936 $ & $79 $ & $51757 $ & $44 $ & $54255$ \\\hline
		$746 $ & $28510 $ & $189 $ & $42363 $ & $114 $ & $48050 $ & $78 $ & $51835 $ & $43 $ & $54298$ \\\hline
		$722 $ & $29232 $ & $190 $ & $42553 $ & $112 $ & $48162 $ & $78 $ & $51913 $ & $42 $ & $54340$ \\\hline
		$607 $ & $31828 $ & $173 $ & $43278 $ & $107 $ & $48597 $ & $74 $ & $52218 $ & $40 $ & $54503$ \\\hline
		$539 $ & $32367 $ & $170 $ & $43448 $ & $106 $ & $48703 $ & $72 $ & $52290 $ & $39 $ & $54542$ \\\hline
		$502 $ & $32868 $ & $161 $ & $43609 $ & $105 $ & $48808 $ & $72 $ & $52362 $ & $38 $ & $54580$ \\\hline
		$462 $ & $33330 $ & $160 $ & $43769 $ & $102 $ & $48910 $ & $72 $ & $52434 $ & $37 $ & $54617$ \\\hline
		$459 $ & $33790 $ & $158 $ & $43927 $ & $102 $ & $49012 $ & $71 $ & $52505 $ & $37 $ & $54654$ \\\hline
		$422 $ & $34212 $ & $153 $ & $44080 $ & $102 $ & $49114 $ & $71 $ & $52576 $ & $36 $ & $54690$ \\\hline
		$372 $ & $34584 $ & $153 $ & $44233 $ & $101 $ & $49215 $ & $71 $ & $52647 $ & $36 $ & $54726$ \\\hline
		$362 $ & $34945 $ & $149 $ & $44382 $ & $100 $ & $49315 $ & $69 $ & $52716 $ & $34 $ & $54760$ \\\hline
		$355 $ & $35300 $ & $149 $ & $44531 $ & $100 $ & $49415 $ & $68 $ & $52784 $ & $33 $ & $54793$ \\\hline
		$347 $ & $35647 $ & $148 $ & $44679 $ & $99 $ & $49514 $ & $67 $ & $52581 $ & $32 $ & $54825$ \\\hline
		$342 $ & $35989 $ & $147 $ & $44826 $ & $98 $ & $49612 $ & $67 $ & $52918 $ & $31 $ & $54586$ \\\hline
		$338 $ & $36327 $ & $147 $ & $44973 $ & $98 $ & $49710 $ & $65 $ & $52983 $ & $28 $ & $54884$ \\\hline
		$329 $ & $36656 $ & $146 $ & $45119 $ & $97 $ & $49807 $ & $64 $ & $53047 $ & $25 $ & $54909$ \\\hline
		$314 $ & $36970 $ & $146 $ & $45265 $ & $97 $ & $49904 $ & $63 $ & $53110 $ & $22 $ & $54931$ \\\hline
		$313 $ & $37283 $ & $146 $ & $45411 $ & $96 $ & $50000 $ & $63 $ & $53173 $ & $20 $ & $54951$ \\\hline
		$310 $ & $37593 $ & $142 $ & $45553 $ & $95 $ & $50095 $ & $63 $ & $53236 $ & $20 $ & $54971$ \\\hline
		$310 $ & $37903 $ & $140 $ & $45693 $ & $95 $ & $50190 $ & $60 $ & $53296 $ & $20 $ & $54991$ \\\hline
		$300 $ & $38203 $ & $137 $ & $45830 $ & $94 $ & $50284 $ & $60 $ & $53356 $ & $19 $ & $55010$ \\\hline
		$284 $ & $38755 $ & $136 $ & $45966 $ & $92 $ & $50376 $ & $58 $ & $53414 $ & $15 $ & $55025$ \\\hline
		$268 $ & $38755 $ & $135 $ & $46101 $ & $92 $ & $50468 $ & $57 $ & $53471 $ & $13 $ & $55038$ \\\hline
		$267 $ & $39022 $ & $131 $ & $46232 $ & $91 $ & $50559 $ & $57 $ & $53539 $ & $11 $ & $55049$ \\\hline
		$265 $ & $39287 $ & $128 $ & $46360 $ & $91 $ & $50660 $ & $57 $ & $53528 $ & $11 $ & $55060$ \\\hline
		$251 $ & $39538 $ & $127 $ & $46487 $ & $90 $ & $50740 $ & $56 $ & $53641 $ & $8 $ & $55068$ \\\hline
		$249 $ & $39787 $ & $126 $ & $46613 $ & $88 $ & $50828 $ & $56 $ & $53967 $ & $6 $ & $55074$ \\\hline
	\end{tabular}
	\end{center}
	\caption[]{Dataset for Pareto Analysis}
	\end{table}	

where the $200$ $n_i$ values (number of orders from a city $i$) have been classified by decreasing values and cumulated in a column $N_i$.

The first city is characterized by $8,965$ purchase orders (which corresponds to $16.28\%$ of the total orders), the second by $4,555$, so that the first two cities made $13,520$ purchase orders (that corresponds to $24.55\%$ of the total orders), the first three cities, spent $16,589$ purchase orders (which corresponds to $30.12\%$ of the total orders), etc. Another way to describe this phenomenon is to say: $0.5\%$ of cities (ranked by decreasing value of the chosen criterion) made $16.28\%$ of the total orders, 1\% of cities have passed $24.55\%$ of the total orders, etc.

These calculations are partly shown in the table below:

	\begin{table}[H]
	\begin{center}
		\definecolor{gris}{gray}{0.85}
			\begin{tabular}{|c|c|c|c|c|c|c|c|c|c|}
				\hline
				\multicolumn{1}{c}{\cellcolor{black!30}\textbf{$i\%$}} & 
\multicolumn{1}{c}{\cellcolor{black!30}\textbf{$N_i\%$}} & \multicolumn{1}{c}{\cellcolor{black!30}\textbf{$i\%$}} & \multicolumn{1}{c}{\cellcolor{black!30}\textbf{$N_i\%$}} & \multicolumn{1}{c}{\cellcolor{black!30}\textbf{$i\%$}} & \multicolumn{1}{c}{\cellcolor{black!30}\textbf{$N_i\%$}} & \multicolumn{1}{c}{\cellcolor{black!30}\textbf{$i\%$}} &
\multicolumn{1}{c}{\cellcolor{black!30}\textbf{$N_i\%$}} &
\multicolumn{1}{c}{\cellcolor{black!30}\textbf{$i\%$}} &
\multicolumn{1}{c}{\cellcolor{black!30}\textbf{$N_i\%$}}\\ \hline
		$0.5$ & $16.28$ & $20.5$ & $72.68$ & $40.5$ & $84.86$ & $60.6$ & $92.45$ & $80.5$ & $97.6$ \\ \hline
		$1$ & $24.55$ & $21$ & $73.11$ & $41$ & $85.09$ & $61$ & $92.61$ & $81$ & $97.7$ \\ \hline
		$1.5$ & $30.12$ & $21.5$ & $73.51$ & $41.5$ & $85.32$ & $61.5$ & $92.77$ & $81.5$ & $97.8$ \\ \hline
		$2$ & $34.36$ & $22$ & $73.92$ & $42$ & $85.54$ & $62$ & $92.92$ & $82$ & $97.89$ \\ \hline
		$\cdots$ & $\cdots$ & $\cdots$ & $\cdots$ & $\cdots$ & $\cdots$ & $\cdots$ & $\cdots$ & $\cdots$ & $\cdots$ \\ \hline
		$19$ & $71.33$ & $39$ & $84.18$ & $59$ & $91.97$ & $79$ & $97.3$ & $99$ & $99.7$  \\ \hline
		$19.5$ & $71.79$ & $39.5$ & $84.41$ & $59.5$ & $92.13$ & $79.5$ & $97.4$ & $99.5$ & $99.99$ \\ \hline
		$20$ & $72.24$ & $40$ & $84.64$ & $60$ & $92.29$ & $80$ & $97.5$ & $100$ & $100$  \\ \hline
	\end{tabular}
	\end{center}
	\caption[]{Aggregated and normalized data for Pareto analysis}
	\end{table}	
	
	translated graphically (according to the traditions of use) below as a Pareto chart with any modern spreadsheet software:
	\begin{figure}[H]
		\centering
		\fbox{\includegraphics[scale=0.75]{img/economy/pareto_chart.eps}}
		\caption{Pareto chart}
	\end{figure}
	or in a more relevant way using a "\NewTerm{Lorenz diagram}" also named sometimes "gain chart" (the horizontal axes with the numbers of cities is replaced by the cumulative\% of towns as in the data table above):
	\begin{figure}[H]
		\centering
		\fbox{\includegraphics[scale=0.75]{img/economy/lorenz_chart.eps}}
		\caption{Lorenz chart}
	\end{figure}

	This analysis shows that a small number of elements is the root of most of the studied phenomenon (e.g. here $31\%$ of cities generate $80\%$ of purchase orders). This explains why it is one of the major techniques used in the field of "\NewTerm{Total Quality Management}" and in the analysis of the importance of the roots of a quality problem. It is also used by managers to structure the organization, especially to differentiate processes based on characteristic of demand (differentiated customer tracking according to its importance, for example) and when it is used to define $3$ classes, this technique often takes the name "\NewTerm{ABC method}".

	The dotted line in the Lorenz graph represents what we would have observed if the case of equal distribution of the studied phenomenon, that is to say if each city was characterized by the same number of purchase orders.

	In general, the more a Lorenz curve is close to the perfect equality of the straight line and more the distribution of mass seen in the population is equal. Indeed, in this case, the majority (wages, wealth, income, etc.) is not concentrated only on a  few items.

	\begin{tcolorbox}[title=Remark,colframe=black,arc=10pt]
	The presentation of this analysis was conducted by classifying the observations by decreasing values, but we could just as well from a ranking by increasing values and, in the latter case, the resulting curve would have been symmetric, the center of symmetry being the coordinate point $(0.5,0.5)$.
	\end{tcolorbox}

	In an industrial environment, potential improvement points are obviously almost innumerable. It could even improve indefinitely, everything and anything. It should however not be forgotten that the improvement also costs and in return should provide added value, or at least offset losses!


	\paragraph{Gini Index}\label{gini index}\mbox{}\\\\
The studied phenomenon is even less equally distributed as the curve moves away from the straight line of equal distribution. Economists, managers or corporate departments responsible sometimes use (for their performance dashboard) a synthetic indicator to measure this phenomenon and its evolution: the "\NewTerm{Gini index}" (also called "\NewTerm{Gini coefficient}").

This coefficient is defined by the ratio:

where the surfaces $A$ and $B$ relate to the following figures:
\begin{figure}[H]
\centering
\includegraphics[scale=0.75]{img/economy/gini_index.eps}
\caption{The two most common situations of the Gini index}
\end{figure}
The Gini coefficient is thus a real number between $0$ and $1$. In case of perfect equality, it is equal to zero (since $A=0$). In case of total inequality is equal to $1$, since $B=0$. Therefore, as $G$ increases from $0$ to $1$, inequality of the distribution increases.

Knowing that the Lorenz curve $1 \times 1$ is seen that the area $A + B$ is equal to half of this surface. So we have:
	
We can thereby write:
	
Also as:
	
Finally, we can write that:
	
Using the trapezoidal numerical integration (\SeeChapter{see section Numerical Methods page \pageref{trapezoidal numerical integration}}) we get that the area $B$ which is given by:
	
	where as reminder $h$ is the length of the intervals all assumed equals (which is always the case in the context of Lorenz analysis). This relation is correct if and only if the data table configuration is such that we get the following type of chart (which is rare...):
	\begin{figure}[H]
	\centering
	\includegraphics[scale=0.75]{img/economy/gini_index_calculation_first_case.eps}
	\caption{Gini Index first case calculation}
	\end{figure}
	Traditionally we find the last relation in the form:
		
	where $n$ is the number of statistical units (the size of the population).
	In the situation where the chart looks as (this case is significantly more common in the business and this also corresponds to our data table above!!):
	\begin{figure}[H]
	\centering
	\includegraphics[scale=0.75]{img/economy/gini_index_calculation_second_case.eps}
	\caption{Gini Index second case calculation}
	\end{figure}
	The trapezoidal numerical integration method brings us then to write:
	
Traditionally we find this last relation in the form:
	
In the case that serves us as an example from the beginning with the table:

	\begin{table}[H]
	\begin{center}
		\definecolor{gris}{gray}{0.85}
			\begin{tabular}{|p{1cm}|p{1cm}|p{1cm}|p{1cm}|p{1cm}|p{1cm}|p{1cm}|p{1cm}|p{1cm}|p{1cm}|}
				\hline
				\multicolumn{1}{c}{\cellcolor{black!30}\textbf{$i\%$}} & 
\multicolumn{1}{c}{\cellcolor{black!30}\textbf{$N_i\%$}} & \multicolumn{1}{c}{\cellcolor{black!30}\textbf{$i\%$}} & \multicolumn{1}{c}{\cellcolor{black!30}\textbf{$N_i\%$}} & \multicolumn{1}{c}{\cellcolor{black!30}\textbf{$i\%$}} & \multicolumn{1}{c}{\cellcolor{black!30}\textbf{$N_i\%$}} & \multicolumn{1}{c}{\cellcolor{black!30}\textbf{$i\%$}} &
\multicolumn{1}{c}{\cellcolor{black!30}\textbf{$N_i\%$}} &
\multicolumn{1}{c}{\cellcolor{black!30}\textbf{$i\%$}} &
\multicolumn{1}{c}{\cellcolor{black!30}\textbf{$N_i\%$}}\\ \hline
		0.5 & 16.28 & 20.5 & 72.68 & 40.5 & 84.86 & 60.6 & 92.45 & 80.5 & 97.6 \\ \hline
		1 & 24.55 & 21 & 73.11 & 41 & 85.09 & 61 & 92.61 & 81 & 97.7 \\ \hline
		1.5 & 30.12 & 21.5 & 73.51 & 41.5 & 85.32 & 61.5 & 92.77 & 81.5 & 97.8 \\ \hline
		2 & 34.36 & 22 & 73.92 & 42 & 85.54 & 62 & 92.92 & 82 & 97.89 \\ \hline
		$\cdots$ & $\cdots$ & $\cdots$ & $\cdots$ & $\cdots$ & $\cdots$ & $\cdots$ & $\cdots$ & $\cdots$ & $\cdots$ \\ \hline
		19 & 71.33 & 39 & 84.18 & 59 & 91.97 & 79 & 97.3 & 99 & 99.7  \\ \hline
		19.5 & 71.79 & 39.5 & 84.41 & 59.5 & 92.13 & 79.5 & 97.4 & 99.5 & 99.99 \\ \hline
		20 & 72.24 & 40 & 84.64 & 60 & 92.29 & 80 & 97.5 & 100 & 100  \\ \hline
	\end{tabular}
	\end{center}
	\end{table}	
and adopting the notations accordingly, this gives:
	
	
	\subsubsection{Weighted Ishikawa Diagram}
	The traditional "\NewTerm{Ishikawa diagram}\index{Ishikawa diagram}\label{Ishikawa diagram}", or "\NewTerm{fishbone diagram}\index{fishbone diagram}", is a qualitative tool of management. Using this tool one can show the relations between causes and the analyzed effect. The most often used is the Ishikawa diagram in a form named the "6M+E model". The symbol 6M+E describes next general causes: man, machine, material, method, management, measurement and environment.

	This diagram is presented in the figure below:
	\begin{figure}[H]
		\centering
		\includegraphics[scale=0.8]{img/economy/ishikawa_6me.jpg}
		\caption{6M+E Ishikawa diagram}
	\end{figure}
	But the model of the classical Ishikawa diagram is not complete.
There is no quantitative information to obtain from this diagram and as we know in management: can be improved only what can be quantified! So own in 2005 to Aleksander Gwiazad and first approach to quantify this diagram! This approach is in my point of view an "a priori" analysis as in don't make usage of the frequence of events. But obviously it is quite easy to improve this model by including a frequency data or also with Bayesian point of view or even with Monte Carlo simulations.

	\pagebreak
	Below is presented the method of preparing the weighted Ishikawa diagram:
	\begin{enumerate}
		\item Determination of a set of main causes
		\item Determination of subcauses
		\item Determination of weights of main causes
		\item Preparing the weighted Ishikawa diagram
		\item Conducting the stratification analysis
		\item Determination the set of important causes and subcauses
	\end{enumerate}

	To determine the weights of connections (causes) it is proposed to use a form of the Saaty matrix (from the name of the Author of the AHP decision tool that we have study in the section of Game and Decision Theory). As we know this matrix is an non-symmetric influence Matrix which influential elements are in the rows, this is why the summations are done for each row and not for each column: 
	\begin{table}[H]
  		\centering\settowidth\rotheadsize{Next concept/}
		\renewcommand\cellalign{cl}
	    \renewcommand\arraystretch{1.25}
  		\begin{tabular}{|l|c|c|c|c|c|c|c|c|c|}
		\hline
	    \diagbox[height=1.25\rotheadsize]{\raisebox{3ex}{Input}}{\raisebox{-4ex}{Input}} & \rotcell{Man} & \rotcell{Machine} & \rotcell{Material} & \rotcell{Method} & \rotcell{Management} & \rotcell{Measurement} & \rotcell{Environment} & Sum & Normalised Sum \\
	    \hline
	    Man & X & $0.5$ & $1$ & $0.5$ & $1$ & $1$ & $0.5$ & $4.5$ & $0.214$ \\ \hline
	    Machine & $0.5$ & X & $1$ & $0.5$ & $0.5$ & $1$ & $0$ & $3.5$ & $0.167$ \\ \hline
	    Material & $0$ & $0$ & X & $0$ & $0.5$ & $0.5$ & $0$ & $1$ & $0.048$ \\ \hline
	    Method & $0.5$ & $0.5$ & $1$ & $X$ & $0.5$ & $1$ & $0.5$ & $4$ & $0.19$ \\ \hline
	   	Management & $0$ & $0.5$ & $0.5$ & $0.5$ & X & $1$ & $0$ & $2.5$ & $0.119$ \\ \hline
	   	Measurement & $0$ & $0$ & $0.5$ & $0$ & $0$ & X & $0$ & $0.5$ & $0.024$ \\ \hline
	   	Environment & $0.5$ & $1$ & $1$ & $0.5$ & $1$ & $1$ & X & $5$ & $0.238$ \\ \hline
	  	\end{tabular}
		\caption{Ishikawa Satty Matrix}
	\end{table}
	Using this matrix one can compare in pairs every cause on the same level of analysis. The scale of notes is as presented: $0$, $0.25$,
$0.5$, $0.75$, $1$. Where $0$ means that analyzed cause is of no importance. The note 0.25 is for causes of small importance. If the importance of causes is equal one gives them the note 0.5. The notes 0.75 and 1 are for cause that are more important and very important. Next, in the column $\sum$ there is calculated the sum of all component notes. This sum is the weight of each cause. In the next column the normalized weight is computed. The normalized weights are inscribed to the Ishikawa diagram.

	In the same manner the weights of subcauses, for each main cause separately, are determined. Next are computed the absolute weights by comparising the weights of subcauses with the weight of the main cause.
	
	For example consider that for Man branch we have three causes {Subcause1, Subcause2, Subcause3} with respective weights $\{0.17,0.33,0.5\}$ that obviously sum up to $1$.
	
	After each influence is given by:
	
	A complete example is given in the figure below:
	\begin{figure}[H]
		\centering
		\includegraphics[scale=0.8]{img/economy/ishikawa_weighted.jpg}
		\caption[Weighted Ishikawa diagram]{Weighted Ishikawa diagram (source: Aleksander Gwiazad/Industrial Management and Organisation)}
	\end{figure}
	The next step is the stratification analysis. This analysis bases
on the Pareto rule. The first step in this analysis is to prepare a
table of weights of all subcauses:
	\begin{table}[H]
	\begin{center}
		\definecolor{gris}{gray}{0.85}
			\begin{tabular}{|c|c|c|c|}
				\hline
				\multicolumn{1}{c}{\cellcolor{black!30}\textbf{N$^\circ$}} & 
  \multicolumn{1}{c}{\cellcolor{black!30}\textbf{Subcause}} & 
  \multicolumn{1}{c}{\cellcolor{black!30}\textbf{Weight}} & 
  \multicolumn{1}{c}{\cellcolor{black!30}\textbf{Cumulate Weight}}\\ \hline
				\multicolumn{1}{|c|}{\cellcolor{green!30}$1$} & \multicolumn{1}{|c|}{\cellcolor{green!30}environment 1} & \multicolumn{1}{|c|}{\cellcolor{green!30}$0.159$} & \multicolumn{1}{|c|}{\cellcolor{green!30}$0.159$} \\ \hline
				\multicolumn{1}{|c|}{\cellcolor{green!30}$2$} & \multicolumn{1}{|c|}{\cellcolor{green!30}machine 1}& \multicolumn{1}{|c|}{\cellcolor{green!30}$0.112$} & \multicolumn{1}{|c|}{\cellcolor{green!30}$0.271$} \\ \hline
				\multicolumn{1}{|c|}{\cellcolor{green!30}$3$} & \multicolumn{1}{|c|}{\cellcolor{green!30}man 3} & \multicolumn{1}{|c|}{\cellcolor{green!30}$0.107$} & \multicolumn{1}{|c|}{\cellcolor{green!30}$0.378$} \\ \hline
				\multicolumn{1}{|c|}{\cellcolor{green!30}$4$} & \multicolumn{1}{|c|}{\cellcolor{green!30}method 1} & \multicolumn{1}{|c|}{\cellcolor{green!30}$0.095$} & \multicolumn{1}{|c|}{\cellcolor{green!30}$0.473$} \\ \hline
				\multicolumn{1}{|c|}{\cellcolor{green!30}$5$} & \multicolumn{1}{|c|}{\cellcolor{green!30}method 2} & \multicolumn{1}{|c|}{\cellcolor{green!30}$0.095$} & \multicolumn{1}{|c|}{\cellcolor{green!30}$0.0.568$} \\ \hline
				\multicolumn{1}{|c|}{\cellcolor{green!30}$6$} & \multicolumn{1}{|c|}{\cellcolor{green!30}environment 2} & \multicolumn{1}{|c|}{\cellcolor{green!30}$0.079$} & \multicolumn{1}{|c|}{\cellcolor{green!30}$0.647$} \\ \hline
				\multicolumn{1}{|c|}{\cellcolor{green!30}$7$} & \multicolumn{1}{|c|}{\cellcolor{green!30}man 2} & \multicolumn{1}{|c|}{\cellcolor{green!30}$0.071$} & \multicolumn{1}{|c|}{\cellcolor{green!30}$0.718$} \\ \hline
				\multicolumn{1}{|c|}{\cellcolor{green!30}$8$} & \multicolumn{1}{|c|}{\cellcolor{green!30}machine 2} & \multicolumn{1}{|c|}{\cellcolor{green!30}$0.055$} & \multicolumn{1}{|c|}{\cellcolor{green!30}$0.773$} \\ \hline
				$9$ & management 1 & $0.050$ & $0.823$ \\ \hline
				$10$ & man 1 & $0.036$ & $0.859$ \\ \hline
				$11$ & management 2 & $0.030$ & $0.889$ \\ \hline
				$12$ & management 3 & $0.030$ & $0.919$ \\ \hline
				$13$ & material 1 & $0.018$ & $0.937$ \\ \hline
				$14$ & material 3 & $0.018$ & $0.955$ \\ \hline
				$15$ & material 2 & $0.012$ & $0.967$ \\ \hline
				$16$ & measurement 1 & $0.012$ & $0.979$ \\ \hline
				$17$ & management 4 & $0.009$ & $0.988$ \\ \hline
				$18$ & measurement 2 & $0.008$ & $0.996$ \\ \hline
				$19$ & measurement 3 & $0.002$ & $0.998$ \\ \hline
		\end{tabular}
	\end{center}
	\end{table}

	In the presented case the first eight subcauses constitute the group of important causes.

	\pagebreak
	\subsection{Supply Chain Management}

The challenge of inventory management and supply chain is something critical for all big companies worldwide. The goal is to implement processes that optimize the economic function under the constrained of in theory a flawless availability. These are the objectives of the stocks manager. This entails having a visibility on the inventory and appropriate methodologies to different situations.

A production without stock is almost inconceivable (even if it is the final goal!) given the many functions performed by stocks. Indeed, stockpiling is required if:
	\begin{enumerate}
		\item There is non-coincidence in time or space of production and consumption: the stock is essential in this case because it is hard or costly to produce where and when the demand arises (this is typically the purpose of the Lean Six Sigma). Classic examples are the manufacture of toys or confectionery for non-coincidence in time, and supermarkets for the non-coincidence in space.
		\item There is uncertainty about the level of demand or on the price: if there is uncertainty about the quantity requested, we will provide a safety stock allowing to face to a peak demand (taking care to avoid the "bullwhip effect" that is to say a to big difference between demand and supply). If there is uncertainty about the price, we will build a speculative stock. For example, oil companies buy much needed crude oil when its price is relatively low on the market.
		\item There are risk of issues in chain: this is to avoid that a failure on a machine affects the whole supply chain. A delay in performance on the previous post (manufacturing step) or a transport strike will not immediately stop the entire production process if there are stocks buffers.
		
		\item There is presence of launch costs: in this case, working in batches allows an economy of scale on the cost of production launch but, in contrast, causes an increase in cost of ownership of the stock.
	\end{enumerate}
	The control of stocks and supply chain of a company is as fundamental in the life of it. To reduce (or optimize it depends...) the various costs that surround everything that has to do with storage, we must use our knowledge in mathematical statistics and differential and integral calculus as we shall see later.

	\begin{tcolorbox}[title=Remarks,colframe=black,arc=10pt]
	\textbf{R1.} The application of the tools we will see now does not really apply to SMEs with fewer than 50 employees producing small pieces irregularly but rather to multinationals producing in enormous quantities of small objects or small quantities of significant size objects and this regularly. In the facts almost nobody use the models we will see below because the assumptions are very difficulty to satisfy but it is still a very good study to be able to build more complicated heuristic models.\\
	
	\textbf{R2.} In facts we use a combination of Monte Carlo simulations and Data Mining (Machine Learning) techniques to improve supply chain management in a very efficient ways (application of Data Mining to spare millions of dollars per year are numerous in the specialized literature).
	\end{tcolorbox}

First, we will establish how to determine the required initial stock for a company based on statistical data using very simple naive models afterwhat we will do the same with replenishment models whose the approach is somewhat different and allows like for the first one to achieve satisfactory results for large scale stable markets.

The models we will build will therefore give the opportunity under the constraints of stable market (stable in time or "quasi-static" as will say thermodynamics specialists...) to:

	\begin{enumerate}
		\item To regulate the flow of supplies
		\item Allow batch production (reduced production costs)
		\item To meet seasonal demands
	\end{enumerate}

As we can guess, additional stocks can generate "\NewTerm{interest costs}" (fixed capital), "\NewTerm{obsolescence costs}" (between the products become obsolete time), "\NewTerm{storage costs}", "\NewTerm{insurance costs}" (protection against accidents that can provide on the products) and many others...

We distinguish in the field of supply chain at least 3 types of stocks:
	
	\begin{enumerate}
		\item The "\NewTerm{safety stock}" that responds to the most common issues associated with the consumption and delivery.
		\item The "\NewTerm{alert stock}" or "\NewTerm{critical stock}" is the stock level for which an order is triggered at risk to be out of stock. 
			\item The "\NewTerm{minimum stock"}, also named "\NewTerm{buffer stock}" or "\NewTerm{order point}" or "\NewTerm{reorder point}" is by definition the sum of the safety stock and minimum stock.
	\end{enumerate}
	
	We often consider at least three stocks management strategies:
		\begin{enumerate}
			\item The "\NewTerm{supply chain management by control point}": the supply of the stock is raised when it is observed that the stock falls below the reorder point.
			\item The "\NewTerm{supply chain calendar management}": the supply of the stock is triggered at regular intervals $T$, for example, daily or weekly.
			\item The "\NewTerm{supply chain conditional calendar management}": the supply of the stock is triggered at regular intervals $T$, but only when it is observed that the stock falls below the control point.
		\end{enumerate}
		A stock is formed to meet future demand. In case of random demand, there may be no coincidence between the application and the stock. Two cases are of course possible:
		\begin{enumerate}
			\item A demand higher than the available stock: we speak then of "\NewTerm{out of stock}" or "\NewTerm{inventory shortages}".
			\item A demand lower than the stock: we speak then of "\NewTerm{residual stock}"
		\end{enumerate}
		The criteria commonly adopted in inventory management is that of cost minimization (...). We denote this function by the letter $C$ followed in brackets (or as index) of the system control variables.
		
		\subsubsection{Supply Chain Management in uncertain future}
		
		Let's begin our study with the simplest case which requires that consumption is statistically steady in time and under statistical control. It follows (see sections Statistics page \pageref{central limit theorem} and Industrial Engineering page \pageref{short and long term process}) that the instantaneous consumption then follows a Normal law $\mathcal{N}(\mu,\sigma)$ or more technically: a Gaussian stochastic process where for reminder each day is independent of the previous one (\SeeChapter{see section Economy page \pageref{stochastic process}}).
		
		Let us take a concrete example, since the theory has already been studied extensively in the section of Statistics and Economy.
		
	\begin{tcolorbox}[title=Remark,colframe=black,arc=10pt]
	The case where we we want to forecast sales using different mathematical assumptions is presented in the section of Economy when we study Time Series Analysis (Forecasting Techniques).
	\end{tcolorbox}
	
	\begin{tcolorbox}[colframe=black,colback=white,sharp corners]
\textbf{{\Large \ding{45}}Example:}\\\\
	Consider an article whose instantaneous demand $X$ follows approximately a Normal law of parameters:
	
	Suppose that stock available at the time of restock order is $500$ units and that the restock delay is $5$ days. We would like to know the cumulative probability of being above or equal to the out of stock $5$ days after the last restock and the cumulative probability of being below the assumed consumption?\\
	
	By using the property of stability of the Normal distribution (\SeeChapter{see section Statistics page \pageref{stability of the sum in statistics}}) the statistical cumulative consumption after $5$ days  give us:
		
		Therefore we have after $5$ days the consumption that follows a Normal law of characteristics:
		
		Using a tabsheet software like Microsoft Microsoft Excel 11.8346 we get the cumulative probability $P(X \geq 500)$ to be out of stock at the fifth day:
		\begin{center}
		\texttt{=1-NORM.DIST(500,450,44.72,1)=13.17\%}
		\end{center}
		And we can also easily calculate the cumulative probability of being below the assumed consumption each day:
		\begin{center}
		\texttt{=1-NORM.DIST(500/5,90,20,1)=30.85\%}
		\end{center}
		Obviously, in practice the reality is far less simple and more..., the Normal law has in fact a probability of negative sales, which is unrealistic... But it is a first model that can be confronted with the reality to see its predictive power and afterwards adapted with more complicated distributions.
	\end{tcolorbox}

	\pagebreak
	\subsubsection{Optimal initial stock management with zero rotation}
	
	Imagine a scenario to develop another statistical model but a little more elaborate than the previous one (strongly by the book \textit{Gestion de la Production} \cite{giard1981gestion} of Vincent Giard). We will limit ourselves however here to a discrete statistical law and we will do late again an anoter example with a continuous law.
	
	Consider for this purpose that the company MAC is the specialist of a given product whose direct manufacturing cost per unit is 25.- and the sales price is $60.-$ selling per unit (assumed constant prices and thus independent of the quantity and without discount).
	
	The daily sales of this product is, weekdays from $2.5$ units on average and the statement of claims during three months suggests that it follows a discrete distribution like the Poisson law (this time the domain of definition is at least positive in contrast to the previous example...), that is to say that we have a probability distribution of the number $X$ of these products sold during a day. We will truncate to $x=10$ the realization of this random variable as the likelihood of sales above $10$ units will be assumed to be zero.
	
	\begin{table}[H]
	\begin{center}
		\definecolor{gris}{gray}{0.85}
			\begin{tabular}{|c|c|c|}
				\hline
				\multicolumn{1}{c}{\cellcolor{black!30}\textbf{$X$}} & 
\multicolumn{1}{c}{\cellcolor{black!30}\textbf{$P(X)$}} & \multicolumn{1}{c}{\cellcolor{black!30}\textbf{Cumulated Probability}}\\ \hline
		0 & 0.0821 & 0.0821 \\ \hline
		1 & 0.2052 & 0.2873 \\ \hline
		2 & 0.2565 & 0.5438 \\ \hline
		3 & 0.2138 & 0.7576 \\ \hline
		4 & 0.1336 & 0.8912 \\ \hline
		5 & 0.0668 & 0.958 \\ \hline
		6 & 0.0278 & 0.9858 \\ \hline
		7 & 0.0099 & 0.9957 \\ \hline
		8 & 0.0031 & 0.9988 \\ \hline
		9 & 0.0009 & 0.9997 \\ \hline
		10 & 0.0003 & 1 \\ \hline
	\end{tabular}
	\end{center}
	\caption{Cumulated probability of sales}
	\end{table}	
	We then have the above table that shows us that the most commonly sold quantity to an economic agent is approximately $2$ (modal value) and calculation of the expected mean give us for this table:
	
	result which can also be interpreted as the average daily sales than the average level of stock shortages.
	
	We will assume that the stock is lean. In other words, from one day to the next, no unit is reported for sales of the day after because it is not supposed to have anymore available (what we name "\NewTerm{inventory management by zero rotation}" ). The question therefore is to know, given the table above, how many products we have to manufacture (or to order) every day to maximize profit and minimize losses\footnote{In many corporates there exist a report commonly used by analysts and produced by the "product control" department that attributes or explains the daily fluctuation of a tracked variable to the root causes of the changes and named "\NewTerm{PnL report}\index{PnL report}" used a lot in manufacturing, supply chain and portfolio management.}.
	
	Therefore, from the perspective we want a minimization of cost of ownership $C_p$ (often named "\NewTerm{unit cost of ownership}" or "\NewTerm{unit cost of possession}") and we know that associated cost for unsold is $25.-$ per unit, while the cost of shortage (often named "\NewTerm{unit cost of shortage}" ) is equal to the loss of profit following the failed sales, that is to say the selling price $60.-$ substracted by the $25.-$ of manufacturing cost, gives a shortage cost of $35.-$ per unit.
	
	\begin{tcolorbox}[title=Remark,colframe=black,arc=10pt]
	The cost of shortage is not treated only as the loss caused by a failed sales in practice but also include the cost of a temporary replacement solution proposed by the seller to the customer on hold to provide the right product. Internally in companies, the cost of shortage often leads to layoffs of downstream positions when a piece necessary for the manufacturing is no longer available to the production line.
	\end{tcolorbox}	
	
	A rational management must consist to calculate  the initial stock $S$ (i.e. the number of products to order or to manufactured for the next day) that minimizes the indicator of "\NewTerm{management cost}" $C (S)$ defined as the sum of the cost of possession associated with average unsold stock $I_P(S)$ (often named "\NewTerm{average stock owned}" or "\NewTerm{average stock possessed}") and the shortage cost associated with average stock sales missed $I_R(S)$ (often name the "\NewTerm{average stock of unsold}"):
	
	This last relation is, however, a simplification of the relation:
	
	where the last term is the order cost.
	
	Minimizing the cost function $C (S)$ consist in the mathematical point of view to look for an extreme (minimum) of the management cost function such that for the optimal value of the initial procurement $S^{*}$ the cost $C(S^{*})$ is immediately below $C(S^{*}\mp 1)$ that is to say less by the right and by the left. In other words:
	
	That is to say the function in convex. Or after rearrangement:
	
	inequalities that the diagram below may help to better understand:
	\begin{figure}[H]
		\begin{center}
			\includegraphics{img/economy/cost_management_minimisation.jpg}
		\end{center}	
		\caption{Arbitrary example of the extremum of the function $C(S)$}
	\end{figure}
	From now the question is how to determine $S^{*}$. In fact the idea is subtle but  simple as it is clearly and thoughtfully exposed.
	
	Let us take the probability distribution of the daily demand law and suppose we want to calculate the average shortage (i.e. expected mean) for a given level of initial stock $S$ for example:
	
	Or:
	
	associated with the respective initial stocks with respect to the given distribution (obviously $x$ should be taken to be greater than $S$). Either in full generality:
	
	Obviously, if the probability distribution is continuous (or at least approximated as such), the latter relation will be written:
	
	The idea is then to write the probability density distribution over the missing quantity of stock and not the sold one:
	\begin{table}[H]
	\begin{center}
		\definecolor{gris}{gray}{0.85}
			\begin{tabular}{|c|c|c|c|c|c|}
				\hline
				\multicolumn{1}{c}{\cellcolor{black!30}\textbf{$x$}} & 
\multicolumn{1}{c}{\cellcolor{black!30}\textbf{$P(x)$}} & \multicolumn{1}{c}{\cellcolor{black!30}\textbf{$x-4$}} & \multicolumn{1}{c}{\cellcolor{black!30}\textbf{$(x-4)P(x)$}}  & \multicolumn{1}{c}{\cellcolor{black!30}\textbf{$x-5$}} & \multicolumn{1}{c}{\cellcolor{black!30}\textbf{$(x-5)P(x)$}}\\ \hline
		0 & 0.0821 & - & - & - & -\\ \hline
		1 & 0.2052 & - & - & - & -\\ \hline
		2 & 0.2565 & - & - & - & -\\ \hline
		3 & 0.2138 & - & - & - & -\\ \hline
		4 & 0.1336 & - & - & - & -\\ \hline
		5 & 0.0668 & 1 & 0.0668 & - & -\\ \hline
		6 & 0.0278 & 2 & 0.0556 & 1 & 0.0278 \\ \hline
		7 & 0.0099 & 3 & 0.0297 & 2 & 0.0198 \\ \hline
		8 & 0.0031 & 4 & 0.0124 & 3 & 0.0198 \\ \hline
		9 & 0.0009 & 5 & 0.0045 & 4 & 0.0036 \\ \hline
		10 & 0.0009 & 6 & 0.0045 & 5 & 0.0015 \\ \hhline{|=|=|=|=|=|=|}
		$\sum$ & 1 & - & $I_R(4)=0.1708$ & - & $I_R(5)=0.062$\\ \hline
	\end{tabular}
	\end{center}
	\caption[]{Probability density distribution with respect to the missing quantity}
	\end{table}	
	It appears from the table above that to get the initial stock $S$ from $4$ to $5$, decrease the average shortage stock by passing him from 0.1708 to 0.062. But from this result, we can't do anything for now because at our present level of development, this would mean that by taking an initial stock of $10$, we would have an average shortage stock of zero (... which does not take us to anything interesting...) and that if we take no initial stock, we would have a total shortage stock...
	
	But still we can draw an interesting intermediate result. Indeed let us look on how the varies  the difference average break (easily generalizable result - we can make the proof on request if necessary):
	
	The result is the same if the probability distribution is continuous!
	
	In other words (be careful to what we say !!!) the probability decrease in average rupture caused by increasing the initial stock of one unit previously dimensioned to $S=4$ is equal to the cumulative probability that the demand is strictly greater than that of the original stock, that is to say $P(X>4)$.
	
	In other words, in case this is not clear, the fact of increasing the initial stock certainly decreases the probability of average rupture, but in exchange creates a risk that there are fewer buyers who may meet the supply and the only ones who can are those corresponding to the cumulative probability $P(X>4)$.
	
	Finally, we can write:
	
	The table below shows the decreasing of expected out of stock we obtain by increasing by one unit the initial stock (and respectively the probability of customers able to consume the whole stock...):
	\begin{table}[H]
	\begin{center}
		\definecolor{gris}{gray}{0.85}
			\begin{tabular}{|c|c|c|}
				\hline
				\multicolumn{1}{c}{\cellcolor{black!30}\textbf{$x$}} & 
\multicolumn{1}{c}{\cellcolor{black!30}\textbf{$P(x)$}} & \multicolumn{1}{c}{\cellcolor{black!30}\textbf{$I_R(x)-I_R(x+1)$}} \\ \hline
		0 & 0.0821 & 0.9179 \\ \hline
		1 & 0.2052 & 0.7127 \\ \hline
		2 & 0.2565 & 0.4562  \\ \hline
		3 & 0.2138 & 0.2424 \\ \hline
		4 & 0.1336 & 0.1088  \\ \hline
		5 & 0.0668 & 0.0420 \\ \hline
		6 & 0.0278 & 0.0142  \\ \hline
		7 & 0.0099 & 0.0043  \\ \hline
		8 & 0.0031 & 0.0012  \\ \hline
		9 & 0.0009 & 0.0003  \\ \hline
		10 & 0.0009 & 0 \\ \hline
	\end{tabular}
	\end{center}
	\caption[]{Average rupture by increasing the initial stock of one unit}
	\end{table}	
	Now look at us look at the unsold $I_P(S)$ as a random function of the initial stock $S$. Their mean is of course given by (use tables if needed to understand):
	
	Obviously, if the probability distribution is continuous (or at least approximated as such), the latter relation is written:
	
	What we can write:
	
	Therefore:
	
	which is then the residual stock of end of period. It is a remarkable result that will allow us to determine $S^*$  only from $I_R(S)$. We can also again consider as obvious that the result will be the same if the probability distribution is continuous.
	
	This last relation can also be written after rearrangement:
	
	where the term on the left represents the average demand satisfied and the right term average offer used. So this relationship is a special relationship of balance between offer and demand.
	
	We can also check this from the tables below with a specific example:
	
	\begin{table}[H]
	\begin{center}
		\definecolor{gris}{gray}{0.85}
			\begin{tabular}{|c|c|c|}
				\hline
				\multicolumn{1}{c}{\cellcolor{black!30}\textbf{$x$}} & 
\multicolumn{1}{c}{\cellcolor{black!30}\textbf{$4-x$}} & \multicolumn{1}{c}{\cellcolor{black!30}\textbf{$(4-x)P(X)$}} \\ \hline
		0 & 4 & 0.3284 \\ \hline
		1 & 3 & 0.6156 \\ \hline
		2 & 2 & 0.5130  \\ \hline
		3 & 1 & 0.2138\\ \hline
		4 & 0 & -  \\ \hline
		5 & - & - \\ \hline
		6 & - & -  \\ \hline
		7 & - & -  \\ \hline
		8 & - & -  \\ \hline
		9 & - & -  \\ \hline
		10 & - & - \\ \hhline{|=|=|=|}
		 &  & $I_P(4)=1.6708$ \\ \hline
	\end{tabular}
	\end{center}
	\caption[]{Mean of unsold items}
	\end{table}	
	Finally we can write an expression of the management cost $C(S)$, according to the average rupture:
	
	or what is written after rearrangement:
	
	Then it follows under the assumption that:
	
	Which gives using the results obtained previously:
	
	Under these conditions, the relations:
	
	can be written using the prior-previous result:
	
	Therefore:
	
	therefore $S^*$ is optimal if:
	
	The result is know in the literature under "\NewTerm{Newsvendor optimal stochastic stock problem}" (there are many versions of this problem).
	
	In our numerical example, we then have:
	
	and this lies between thanks to our table between:
	
	hence the initial daily optimal inventory also named "\NewTerm{minimum stock}" to put into production (or order) to minimize the management cost $C(S)$:
	
	It's a number that is obviously very close to the modal value of the initial distribution.
	
	Note that in practice, however, employees who are controlling supply chain in multinational companies have no idea of prices, margins and costs that occur in the shop of their customers. Therefore they get these information (sales, prices, margins, etc.) by paying their customers to get the data (most of time into Microsoft Excel data file). 
	
	Often practitioners simply take the modal value or the value corresponding to a $95\%$ coverage of the periodic request (this becomes somewhat arbitrary). In the case of our little exercise, for example, the modal value of the initial stock rounded to the nearest integer value would be $2$ and at a threshold of $95\%$  would be equal to $5$. As a precaution, however, there is a tendency to take rather the modal value (it is better to avoid to throw and it is better to say to the boss we missed and unknown amount of sales quantity rather than to give an exact amount of products returned or throw\footnote{This is a typical behavior of supply chain manager that don't know statistics or how to communicate them}...).
	
	\subsubsection{Wilson's Models}
	There are several stock management optimization models (statistical, Wilson, ABC, 20/80, VSM, ...). Among them, we wanted to make stop on the "\NewTerm{Wilson's Models}" with its three variants that are best known in the academic world (but not necessarily the most realistic...).
	
	We therefore consider three situations:
	\begin{enumerate}
		\item The company resupply by ordering batches thanks to external suppliers and then there is no stocks inertia and it redo an order when the safety stock is reached. We speak then of "\NewTerm{Wilson's model with resupply}".
		
		\item The company resupply by an internal production, which then implies a given inertia to return to the desired level of stock. We speak then of "\NewTerm{Wilson's model without resupply}" (for which all results can be very quickly deduced from the first model without redoing all the mathematical developments).
		
		\item The company resupply by ordering batches thanks to external suppliers and then there is no stocks inertia and it redo an order when the safety stock is reached but with inertia and please customers to wait. We speak then of "\NewTerm{Wilson's model with waiting time}".
	\end{enumerate}
	\begin{tcolorbox}[title=Remarks,colframe=black,arc=10pt]
	\textbf{R1.} This model also named the "\NewTerm{Economic Lot Size Model}" or "\NewTerm{Economic Batch Quantity EBQ}" has for purpose to determine the frequency and optimal resupply quantity for a store, a factory, etc. It is commonly used by supply chain services in large structures. It was actually introduced in 1913...\\
	
	\textbf{R2.} When the Wilson model is applied to cash management but rather than with inventory costs we work with with trading costs, fixed cost of making securities, etc. We then speak of "\NewTerm{Baumol-Allais-Tobin model}".
	\end{tcolorbox}
	The purpose is to determine the strategy that must be adopted so that the total periodic (annual, monthly, weekly, daily, ...) of the orders or manufacturing of goods minimizes the total cost of acquisition and ownership for stocks of the company. We talk then sometimes about "\NewTerm{just in time supply chain management}" or more simple "\NewTerm{JIT supply chain management}".
	
	The existence of stocks within a company leads the supply chain manager to ask the question of the optimal level of these latter, avoiding at least two major pitfalls:
	\begin{enumerate}
		\item "\NewTerm{Over-stocking}", source of costs for the company (physical storage costs, handling costs, premises and land used costs, associated costs, security assurances costs\footnote{The FSA company (Financial Security Assurance) provides insurance to corporate and municipal borrowers in the bond market. FSA determines what added collateral and surplus cashflow each deal needs to bring the bond to investment grade and arranges for the issuer to place necessary collateral into a special corporation controlled by FSA. FSA then rents the issuers its triple-A rating, saving the issuers money through lower interest rates and exacting a hefty premium for itself.}, costs immobilized capitals, human unemployment, etc.).
		
		\item "\NewTerm{Sub-stocking}" which may result in stock ruptures that are detrimental to the production activity or the business activity of the company (factory stoppage, sales loss,  customers loss, quality loss, increase in stress for employees, etc.).
	\end{enumerate}
	Thus, the different models of supply chain management are intended to minimize the management cost in this system of constraints by determining the resupply frequency and the associated quantity.
	
	Consider first a purely qualitative approach. For each reference, the quantities in stock change over time, for example in a form:
	\begin{figure}[H]
		\begin{center}
			\includegraphics{img/economy/resupply_first_strategy.jpg}
		\end{center}	
		\caption{First resupply strategy}
	\end{figure}
	Simplifying we get a graph named "\NewTerm{sawtooth supply chart}":
	\begin{figure}[H]
		\begin{center}
			\includegraphics{img/economy/resupply_first_strategy_simplified.jpg}
		\end{center}	
		\caption[]{Simplified sawtooth corresponding diagram}
	\end{figure}
	where we are in a typical situation of "\NewTerm{supply chain management at non-zero rotation}" (that is to say, when unsold can be sold at a later time).
	
	To avoid stock rupture, we must of course ensure that the entry of an order takes place, at the latest, when the inventory quantity becomes zero:
	\begin{figure}[H]
		\begin{center}
			\includegraphics{img/economy/visual_idea_of_the_strategy.jpg}
		\end{center}	
		\caption[]{Visual principle of the strategy}
	\end{figure}
	If we consider a constant consumption of a quantity $N$ per unit of time (day, month, year, etc.) and that we know in advance the lead time (in days, months, years, etc.) then if everything is put in equivalent units (e.g. daily) we have the critical level which is given by:
	
	which is also assimilated to the justified terminology of "\NewTerm{control point}" since this is the amount we have in stock from which we must launch a new procurement order:
	\begin{figure}[H]
		\begin{center}
			\includegraphics{img/economy/control_point.jpg}
		\end{center}	
		\caption[]{Control point representation (sawtooth diagram)}
	\end{figure}
	Thus, if the consumption is of $10$ units per day, and the lead time is $15$ days, then the critical level is of $150$ units.
	
	To prevent hazards (strike, transport, consumption variation, replacements, ...) we plan a safety stock $S_s$:
	\begin{figure}[H]
		\begin{center}
			\includegraphics{img/economy/visual_idea_of_the_strategy_with_safety_stock.jpg}
		\end{center}	
		\caption[]{Representation of safety stock on the sawtooth pattern}
	\end{figure}
	
	We then have for the critical level:
	
	Let us now see the influence of the number of deliveries on the cost of storage (since the higher the rate of retention is bigger, more the storage costs are high). Suppose  for this that the market consumes $100$ units per month and this regularly. In the case of a single annual supply, the consumption is represented by the following diagram (no safety stock in order to simplify the example):
	\begin{figure}[H]
		\begin{center}
			\includegraphics{img/economy/single_supply_representation.jpg}
		\end{center}	
		\caption[]{Representation of the effect of a single supply}
	\end{figure}
	where we immediately see that the average stock is $600$. The average stock is obtained by simple arithmetic average or simply using the definition of the integral average (\SeeChapter{see section Statistics page \pageref{integral average}}) of the consumption function:
	
	thus trivially half maximum stock (or otherwise seen: the area of the rectangle  triangle divided by $T$...).
	
	And if we divide into two resupply, then we have:
	\begin{figure}[H]
		\begin{center}
			\includegraphics{img/economy/double_supply_representation.jpg}
		\end{center}	
		\caption[]{Representation of the effect of a double supply}
	\end{figure}
	so an average stock twice lower and therefore an average storage cost twice less (therefore we understand better what the JIT supply chain management is the best way to management stocks in the point of view of costs). But of course we must associate the cost of supply to it (when there is one). It is at this level of complexity that comes precisely the mathematical formalization of Wilson.
	
	Each purchase order or production order so costs the company. The "\NewTerm{launch cost}" or "\NewTerm{cost of purchase}" or "\NewTerm{restocking cost}" to produce or order new batches represent by definition all related costs (administrative, machine settings, preparation, communication, ages, etc.) relatively to the fact of placing an order (or production) and is assumed to be proportional to the quantity. These costs are determined using analytical cost accounting.
	
	Thus, the cost of an order is obtained by dividing the total operating cost of the purchasing department by a significant and relevant magnitude. For example, the number of annually purchased orders (or orders of manufacturing) by example. The cost of a manufacturing launch will be calculated by dividing the total cost of running the manufacturing services, in which, we  must add the cost of machines maintenance and wages by of manufacturing launches.
	
	These values depend mainly on the company, of its choices in cost accounting methods. It is difficult to define a range of standard values. Many companies do not know accurately how much cost them an order or a production launch (and many simply do not know how to make an analysis...) and as most manager don't know how to make use of probability distributions the decisions are even more difficult for the executive board.
	
	The cost of ownership of the stock consists of expenses related to the physical storage but also the non-remuneration of capital tied up in the stock (or even the cost of debt to finance the stock). For this last reason, this cost is considered to be proportional to the value of the average stock and the holding period of the stock (at lest on short periods because in reality it is an exponential as a factor of $(1+t\%)^n$).
	
	The annual possession/ownership rate $t_\text{O}\%$ is the cost of ownership down to a monetary unit of stored material. It is calculated by dividing the total annually cost of ownership  by the average annual stock. This rate typically and obviously fluctuate depending on the product value stored in the stock, depending on the type of articles and the quality of their inventory management.
	
	Wislon has establish a famous formula to determine the optimum stock quantity based on some assumptions that we will study right now:
	
	\paragraph{Wilson's model with resupply}\mbox{}\\\\
	The highly simplified assumptions of this model are:
	
	\begin{enumerate}
		\item[H1.] The periodic demand (ask) is known and certain (deterministic and stable in other words).
		
		\item[H2.] The ordered quantities are constant at each period.
		
		\item[H3.] Shortages (stock-outs) occurs at end of period (so there is never a lack of stock and therefore no unknown risks).
		
		\item[H4.] The production time is constant and the resupply is instantaneous.
		
		\item[H5.] All costs (stocks, articles, procurement, production, manufacturing ...) are time invariant.
		
		\item[H6.] Cost of ownership (also named "carrying cost" sometimes) is proportional to the value (short time hypothesis).
		
		\item[H7.] The planning horizon is infinite.
	\end{enumerate}
	and according to the authors this list of assumptions vary more or less (certain assumptions being implicit or being relatively trivial or forgotten/hidden).
	\begin{tcolorbox}[title=Remarks,colframe=black,arc=10pt]
	\textbf{R1.} We assume that the stock management is done in a given time period.\\
	
	\textbf{R2.} Based on these assumptions/hypothesis, we conclude that there will be the same level of resupply at every period, and that total cost of shortage is zero.
	\end{tcolorbox}
	\begin{itemize}
		\item $t_\text{O}\%$ the periodic possession/ownership rate.
		
		\item $N$ the quantity corresponding to a demand or respectively the number of items consumed per period.
		
		\item $Q$ is the quantities of resupplies or parts launched for manufacturing in once during this same time (batch sizes).
		
		\item $P_U$ the unit purchase price of the pieces (price assumed as constant!).
		
		\item $S_S$ the considered safety stock for this pieces (also assumed as constant over time) to answer to fluctuations over time (fluctuations also assume to be constant over time).
		
		\item $C_L$ the global cost by resupply/acquisition by order or of manufacturing launch.
	\end{itemize}
	From this primitive variables we can also create the following common variables:
	
	\begin{itemize}
		\item The "\NewTerm{unit cost of storage}" often (but not always!) calculated on the basis of the unit purchase price of a pieces:
		
		
		\item The "\NewTerm{stocks inertia}" which can be seen more explicitly as the "number of periodic launches" (or the replenishment rate) to meet the demand. The inertia of the stock thus has the dimension of a reciprocal of the time:
		
		
		\item The "\NewTerm{inertia cost}" or respectively the "\NewTerm{cost of acquisition}" or "\NewTerm{launch cost}" or "\NewTerm{restocking cost}" is therefore per period unit:
		
		The latter is assumed to be proportional to the consumption! What is important to  is that the launch cost is inversely proportional to the amount $Q$ and therefore it tends to zero as $Q$ tends to infinity. Having said that, normally we will have in most theoretical cases $Q\leq N$.
		
		\item The average stock in the company in the assumption of a constant consumption (linear decrease of the stock) and a constant level of security within the time period is trivially by period by:
		
	\end{itemize}
	The "\NewTerm{periodic cost of ownership}", also named "\NewTerm{cost of ownership}" or "\NewTerm{management cost}" or  "\NewTerm{storage cost}" is then given by:
	
	So this is the function of a straight line (whose intercept is nonzero if the safety stock is not zero) if we consider that only $Q$ is the variable. It is important to note that this cost does not take into account the volume discount concepts made by commercial...	
	
	These proposals therefore lead us to the equation of the "\NewTerm{total cost of resupply}", also named "\NewTerm{total cost of storage}" that we will seek to minimize:
	 
	and giving an hyperbolic type curve of cumulative costs:
	\begin{figure}[H]
		\begin{center}
			\includegraphics{img/economy/supply_chain_total_cost.jpg}
		\end{center}	
		\caption{Cumulative costs $C_{\text{tot}}(Q)$ with Maple 4.00b}
	\end{figure}
	Find the economic quantity $Q_e$ is finding the value of $Q$ for which the total cost is minimal (in other words this is to seek the bathc size that allows to make the best return on investment of all the costs of storage, manufacturing, etc.), that is to say, the mathematically, the value $Q_e$ for which the derivative of the total cost $C_{tot}$ with respect to the quantity $Q$ is zero:
	 
	Hence the "\NewTerm{Wilson relation}" (after an elementary algebraic computation), also simply named "\NewTerm{Wilson formula}" for the "\NewTerm{optimal economic quantity/batch}":
	
	which is therefore for reminder, the batch size that minimizes the total cost (independent of the safety stock which vanishes when we do the derivative). We notice that the economic quantity is proportional to the square root of the consumption. So an item sold $100$ more times as expected by a company require a $10$ times greater average stock.
	
	Obviously once known the economic quantity, it becomes easy to calculate the "\NewTerm{optimal management cost for period}" $C_{\text{tot},e}$ by injecting $Q_e$ in the relation obtained previously:
	 
	and also the "\NewTerm{optimal resupply rate or frequency}" as given by the ratio:
	
	We then have after a little bit elementary algebraic manipulations the optimal cost of management by period that is given by:
	
As often, managers like to make sensitivity analysis (we have already mentioned this in our study of "breakeven analysis"). In the area of supply chain management it traditional to analyze the sensitivity of the relative change in total storage cost compared to the optimal cost of  management by period:	
	
	But as $S_SC_{US}$ (cost of safety stock) is assumed to be independent of the quantity, it is customary to consider only the relative variation regardless of this term such that (...):
	
	Finally, it is customary to simplify the analytical form of this last relation in the following form after some elementary algebraic manipulations:
	
	so this is a fairly aesthetic result... that the reader will notice is always positive!
	
	Obviously we can also deduce from the relation of the optimum resupply frequency the "\NewTerm{optimal resupply period}" then given by (the inverse of the frequency definition):
	
	If we report on a graph the functions:
	\begin{itemize}
		\item The launching cost (supply) in function of the quantities:
		
		
		\item The cost of ownership based in function of the quantities:
		
		
		\item The total costs in function of the quantities:
		
	\end{itemize}
	the economic quantity is at the intersection of the two curves, launch and possession (when the cost of ownership is equal to the cost of acquisition), or at the inflection point of the total cost curve. In practice, however, it is possible to accurately control the economic quantity, we will choose a batch size that meets the various constraints and included in the "\NewTerm{economic area}":
	\begin{figure}[H]
		\begin{center}
			\includegraphics{img/economy/wilson_economic_area.jpg}
		\end{center}	
		\caption{Representation of the economic area}
	\end{figure}
	Obviously in some companies the goal is rather to try to reduce the unit cost of storage $C_{US}$ to reach the equivalent of the demand as economic quantity $Q_e$. We then have with some elementary algebra:
	
	that is to say the optimal unit storage costs if the supply amount $Q$ is imposed:
	
	So to summarize, the reader must be informed that we often consider that the entire Wilson model can be summarized to the $5$ below relations:
	\begin{equation}
	  	\addtolength{\fboxsep}{5pt}
	   	\boxed{
	   	\begin{gathered}
	   		\begin{aligned}
			&C_{\text{tot},e}(Q)=\sqrt{2NC_LC_S}+S_SC_{US}\\
			&Q_e=\sqrt{\dfrac{2NC_L}{t_O\%P_U}}=\sqrt{\dfrac{2NC_L}{C_{US}}}\\
			&T_e=\sqrt{\dfrac{2C_L}{NP_Ut_O\%}}\\
			&C_{US,e}=\dfrac{2C_L}{N}\\
			&\dfrac{C_{\text{tot}}}{C_{\text{tot},e}}\cong\dfrac{1}{2}\left(\dfrac{Q_e}{Q}+\dfrac{Q}{Q_e}\right)
	   		\end{aligned}
	   	\end{gathered}
	   	}
	\end{equation}
	There is another type of scenario that must be studied. If order (resupply) in larger quantities, thus benefiting from a discount, it certainly increases the costs of ownership but theoretically reduces the number of annual orders.
	
	The objective for the manager is of course to check mathematically that the discount granted by the supplier does not lead to higher induced costs to the discount (that would be an incompetent manager!).
	
	To do this, we must reduce all costs to a unique item such that the unit total cost will be written:
	
	This relation is important as it gives the possibility to determine the discount value for the latter to be interesting.

	To know the discount threshold $R$ for a given quantity, we replace in the above relation, $Q$ by the target quantity $Q'$ and $P_U$ by $P_U(1-R)$, where $R$ is the discount percentage.

	Then we have to solve:
	
	We will determine therefore the value of $R$ under which the discount does not compensate the internal costs.

	In the practice we cannot order most of time the exact optimal economic quantity $Q_e$, especially because of the minimum order quantity (MOQ) imposed by suppliers. It is therefore more interesting to focus on the "economic area", defined by the lower area of the total cost curve.
	
	Because of its simplified assumption and its deterministic aspect, the Wilson model can only provide at best an order of amplitude of the solution if the consumption and/or prices are subject to fluctuations (as the economic quantity is dependent of the parameters: storage costs and resupply cost).

	Also it should be notice that the use of resupply is "anti-flexible" par definition. This type of politic takes frequently important consequences, risk to inflate the stock of finite products, reporting costs and losses downstream of the process.
	
	However, the Wilson's model is interesting for the fact that it can also be apply quite well to human resources in some special cases.
	\begin{tcolorbox}[colframe=black,colback=white,sharp corners]
	\textbf{{\Large \ding{45}}Example:}\\\\
	The company MAC use an article X330 for which the projected consumption of the year $N$ should be $4,000$ items. The data are the following:   
	\begin{itemize}
		\item The unit cost of Article X330 is $P_U=8$ (regardless of currency)
		
		\item The procurement/launch cost of an order is $C_L=100$

		\item The rate of possession of the stock is $t\%=10\%$ by year.

		\item The safety stock is of $250$ units
	\end{itemize} 
	The supplier of this article, to encourage its customers to increase the size of their orders, offer to the company the following conditions:
	\begin{enumerate}
		\item[C1.] Ordered quantity less than $2,000$ units: unit price $P_U=8$

		\item[C2.] Ordered quantity less between $2,000$ and $3,500$ units: discount of $2\%$

		\item[C3.] Ordered quantity more than $3,500$ units: discount fo $3\%$
	\end{enumerate}
	The question is to know the is the best choice for the company (buyer)?\\

	The price varies then depending on the quantity such that given a selected quantity, the discount applies on same manner equivalent to all items in the batch (this is named a "\NewTerm{uniform discount}").\\
	
	According to the statement and according to the quantity $Q$ of supply, we know that:
	\begin{enumerate}
		\item If $Q<2,000 \Rightarrow P_U:=P_1=8$

		\item If $2,000< Q <3,500 \Rightarrow P_U:=P_2=8-2\%\cdot 8=7.84$

		\item If $Q\geq 3,500 \Rightarrow P_U:=P_3=8-3\%\cdot 8=7.76$
	\end{enumerate}
	Using the Wilson relation of the optimal economic batch, we will calculate the economic quantity $Q_e$ for the most advantageous price, that is to say $P_3$:
	
	\end{tcolorbox}
	
	\pagebreak
	\begin{tcolorbox}[colframe=black,colback=white,sharp corners]
	But to get the price $P_3$ must be order at least $3,500$ items so there is contradiction and this solution is therefore out of the area. Identical calculations (that we leave to the reader...) show that only the economic quantity $Q_e$ of $P_1$ satisfies the constraints $Q<2,000$.\\
	
	Therefore, the optimal resupply period will be:
	
	and the optimal management cost per period is given by:
	
	Caution!!! For the latter calculation, some authors include the cost of safety stock $S_SC_S$ the quantity $N$. This is a choice rather questionable and that would need to be clarified.\\
	
	Finally, the impact on the total cost of management if we increase or decrease the amount of $\pm30\%$ compared to the economic quantity is given by (sensitivity analysis):
	
	So obviously, this only increases costs and as we have seen on the graph earlier, the curve being not symmetrical, the two results are not equal.
	\end{tcolorbox}
	
	\pagebreak
	\paragraph{Wilson's model without resupply}\mbox{}\\\\
	In the previous model, we have assume for recall a sawtooth stock management of the following form:
	\begin{figure}[H]
		\begin{center}
			\includegraphics{img/economy/wilson_model_with_resupply.jpg}
		\end{center}	
		\caption{Wilson model with resupply}
	\end{figure}
	In the model that will interests us now, we will consider a supply chain management with a resupply inertia often due to an internal house production such as follows:
	\begin{figure}[H]
		\begin{center}
			\includegraphics{img/economy/wilson_model_without_resupply.jpg}
		\end{center}	
		\caption{Wilson model without resupply}
	\end{figure}
	As for the Wilson model with resupply, the demand in production is done at the command point. But there is big difference! We can observe that that during the production at a rate of $P$ parts per unit of time $t_1$ the consumption $N$ per period $T$ (reported to the production unit time), is already using the newly created stocks!!!
	
	Do we have to redo all the calculations of the previous model? The answer is: not at all!!! Because if we think a little bit by observing the hyperbolic function of the total cost of storage that we have proved earlier above:
	
	It is obvious that the first term due to resupply has no reason to change. The third term, due to safety stock, has no reason to change either! So there remains only the second term represented by the average cost of storage that is necessarily different!
	
	Therefore the average storage is no longer:
	
	We can easily calculate the average area of the triangle using the Heron's formula available for any triangle and alread proved in the section of Geometric Shapes. But the problem in this case is that we want to make the variable $Q$ appear in the average stock! We can then not use Heron's formula. The idea will be the following (refer the previous figure):
	
	When the company don't have any stock more, it launch for a given time $t_1$ the production at a rate of $P$ elements per unit time until it reach the cumulative targeted quantity $Q$. Then we have:
	
	But at the same time production is running (and also after), the company uses (sell) $N$ elements per unit time. The real stock will therefore be at time $t_1$"(under the obvious condition that $P$ is greater than $N$) the maximum inventory:
	
	This maximum inventory corresponds therefore to the height of our general triangle (always refer to previous figure) but we wish it to make appear our target stock $Q$. So, for this, there is now nothing more simple using prior-previous relation:
	
	Therefore the average stocks become:
	
	So we see that the only difference between the two models is the factor:
	
	Therefore, the relations obtained for the Wilson model with resupply (at least the most important one) and that were for reminder:
	\begin{equation}
	  	\addtolength{\fboxsep}{5pt}
	   	\begin{gathered}
	   		\begin{aligned}
			&C_{\text{tot},e}(Q)=\sqrt{2NC_LC_S}+S_SC_{US}\\
			&Q_e=\sqrt{\dfrac{2NC_L}{t_O\%P_U}}=\sqrt{\dfrac{2NC_L}{C_{US}}}\\
			&T_e=\sqrt{\dfrac{2C_L}{NP_Ut_O\%}}\\
			&C_{US,e}=\dfrac{2C_L}{N}\\
			&\dfrac{C_{\text{tot}}}{C_{\text{tot},e}}\cong\dfrac{1}{2}\left(\dfrac{Q_e}{Q}+\dfrac{Q}{Q_e}\right)
	   		\end{aligned}
	   	\end{gathered}
	\end{equation}
	become (the last relations does not change):
	\begin{equation}
	  	\addtolength{\fboxsep}{5pt}
	  	\boxed{
	   	\begin{gathered}
	   		\begin{aligned}
			&C_{\text{tot},e}(Q)=\sqrt{2NC_LC_S\left(1-\dfrac{N}{P}\right)}+S_SC_{US}\\
			&Q_e=\sqrt{\dfrac{2NC_L}{\left(1-\dfrac{N}{P}\right)t_O\%P_U}}=\sqrt{\dfrac{2NC_L}{\left(1-\dfrac{N}{P}\right)C_{US}}}\\
			&T_e=\sqrt{\dfrac{2C_L}{\left(1-\dfrac{N}{P}\right)NP_Ut_O\%}}=\sqrt{\dfrac{2C_L}{\left(1-\dfrac{N}{P}\right)C_{US}}}\\
			&C_{US,e}=\dfrac{2C_L}{N\left(1-\dfrac{N}{P}\right)}\\
	   		\end{aligned}
	   	\end{gathered}
	   	}
	\end{equation}
	The adaptation of the relations of periodic cost of ownership and total cost of supply are also immediate.

	So unlike the model with resupply, we see that only a new variable $P$ appears and that is the production speed (elements produced per unit time) to allow a production of $N$ elements per unit time. What the reader must not forget when applying in practic this model is to bring $P$ and $N$ at the same time unit!!!
	
	\paragraph{Wilson's model with resupply and break-up}\mbox{}\\\\
	Still in the classics of academic studies (the rest being only a stochastic mix of different scenarios based on the same reasoning), let us see now the Wilson's model with (instantaneous) resupply but where the break-up and short selling (with penalty counterpart) is allowed (so there is no safety stock anymore!!!). We will represent this scenario by the following saw-tooth figure:
	\begin{figure}[H]
		\begin{center}
			\includegraphics{img/economy/wilson_model_with_resupply_and_break_up.jpg}
		\end{center}	
		\caption{Wilson model with resupply and break-up}
	\end{figure}
	We will consider $C_R$ as the financial loss (cost) per piece unit during the short-selling period (not having immediate input cash-flow makes that we lost money anyway as we can not place it on funds with return yield).
	
	In this case, we have not found a simple way to avoid the redoing the calculations of the relations of the Wilson's model with resupply (but if someone has a tip he can contact us!). However it is still elementary algebra and therefore readers should not suffer to much with this kind of exercise style.
	
	We first have trivially by referring to the figure above:
	
	First, the average cost of storage is trivially equal to:
	
	The average cost of selling in break-out is also trivially given by:
	
	The average total cost for a period is then:
	
	Let us report this to the time of one period:
	
	The average total cost per period is then:
	
	The problem here is that $Q$ depends on $M$ and vice versa. We'll have to use partial derivatives to fixsometimes one and sometimes the other. Therefore to find the economic quantity we will first arbitrarily determine which is the amount $M$ that will minimize the total average cost per period (in order to get rid of the presence of the variable $M$ in the expression of the quantity economic). We then have taking the partial derivative (thus implying that we maintain $Q$ constant):
	
	Therefore:
	
	Now for the economic quantity, we respectively have by taking the partial derivative (so implicitly we fix $M$):
	
	By injecting in it the prior previous result it comes:
	
	Thus after some elementary algebraic manipulations:
	
	Which gives finally:
	
	We then have explicitly:
	
	The optimal resupply period is then:
	
	The economic quantity $m$ of rupture gives resupply launch trigger that is then:
	
	A practical difficulty of this model is to determine $C_R$. The idea is then to estimate it on the maximum supposed waiting time of the customer when there is break-up. Starting from:
	
	And by squaring in the idea of extracting $C_R$:
	
	So we finally have an equation of the second degree:
	
	Whose two real roots are (\SeeChapter{see section Calculus page \pageref{double root}}):
	
	The root to keep is obviously the one that gives a positive cost of rupture. That is to say:
	
	Before moving on to the practical example, the summary is reduced to:
	\begin{equation}
	  	\addtolength{\fboxsep}{5pt}
	  	\boxed{
	   	\begin{gathered}
	   		\begin{aligned}
			Q_e&=\sqrt{\dfrac{2NC_L}{C_S}}\sqrt{\dfrac{C_S+C_R}{C_R}}\\
			M_e&=\sqrt{\dfrac{2NC_L}{C_S}}\sqrt{\dfrac{C_R}{C_S+C_R}}\\
			T_e&=\sqrt{\dfrac{2NC_L}{C_S}}\left(\sqrt{\dfrac{C_S+C_R}{C_S+C_R}}+\sqrt{\dfrac{C_R}{C_S+C_R}}\right)\\
			t_\text{limit}&=\sqrt{\frac{2C_LC_S}{NC_R(C_S+C_R)}}\\
			C_R&=-\dfrac{C_R}{2}+ \sqrt{\left(\frac{C_R}{2}\right)^2+\dfrac{2C_LC_S}{Nt_\text{limit}^2}}
	   		\end{aligned}
	   	\end{gathered}
	   	}
	\end{equation}
	\begin{tcolorbox}[colframe=black,colback=white,sharp corners]
	\textbf{{\Large \ding{45}}Example:}\\\\
	The company MAC supply an article X330. In the objective to not to lose customers, the managers decide of an  optimal break-up time (meaning: maximum) of $2$ days. We have the following data:
	
	Which give:
	
	\end{tcolorbox}
	
	\pagebreak	
	\subsection{Queueing Theory}\label{queueing theory}
	The queueing theories are an extremely powerful and extensive tool (a complete presentation requires at a minimum $300$ A4 pages) that gives the possibility to consider and model the bottlenecks in business processes or in logistics, phone (telecom) exchanges, requests on servers, at the  check-out line of shops, or in the toilet of major sports stadiums (...), and also for ticket distribution for huge events, etc. depending on the starting assumptions and constraints. There are numerous example in newspaper of companies or events having very bad reputation because their manager did not use such modelization tools (or consulta
	nts knowing these tools....) during a given period of their existence and that have generate a huge frustration by the customers (reputation very hard to correct afterwards!).

   Typically, customers clearly during waiting non-value added activity, and if they wait too long, they combine this waste of time to a poor quality of service and share their experience on social networks and this can have afterwards a huge impacts (through other medias such at the television). Similarly, within companies, unoccupied  employees or unused equipment represent non-value added activities. To avoid these situations, some the best of companies and governmental institutions have implemented continuous improvement process whose ultimate goal is the elimination of all forms of waste, including waiting. All these examples show the importance of the analysis of queues. To notice that if a company or administration does not have the financial or intellectual capacity to make use of the queuing theories, the minimum respect for customers is at least to indicate with an inexpensive display, the median waiting time and to provide during this waiting times some entertainment (television, newspapers, magazines, food, drinks, etc.)!
   
   Here is typical small list of common systems relatives to queuing theory:
   \begin{table}[H]
	\begin{center}
		\definecolor{gris}{gray}{0.85}
		\begin{tabular}{|c|c|c|}
		\hline
		\multicolumn{1}{c}{\cellcolor{black!30}\textbf{System}} & 
\multicolumn{1}{c}{\cellcolor{black!30}\textbf{Customers}} & \multicolumn{1}{c}{\cellcolor{black!30}\textbf{Services}} \\ \hline
		Counter-reception & People & Receptionists\\ \hline
		Repair workshop & Machines & Technicians \\ \hline
		Customs p & People & Customs officers \\ \hline
		Ticketing & People & Salers  \\ \hline
		Garage & Trucks/Cars & Mechanician \\ \hline
		Hospital & Patients & Nursing  \\ \hline
		Computer & Tasks & Processors, HDD, BUS\\ \hline
		Airport & Airplane & Traffic controler   \\ \hline
		Roads & Cars & Traffic lights  \\ \hline
		Manufacturing & Tasks & Machines/Workers \\ \hline
		Phone & Calls & Exchangers\\ \hline
		Public transport & Travelers & Bus/Metro \\ \hline
		Washroom & Lines & Washers / Dryers \\ \hline
		Safety Check & People & Security guards \\ \hline
		... & ... & ... \\ \hline
		\end{tabular}
	\end{center}
	\caption[]{Examples of typical queues}
	\end{table}
	These theories proves to be particularly useful to justify investments, hiring or equipment purchases. More generally, it is an integral part of mathematics management techniques when necessary to find an economic optimum between waiting costs and a system service costs.

    The typical problem in companies and governmental administration can be expressed as:
   \begin{itemize}
       \item The human brain is subject to cognitive bias so not accurate enough to take a good decision hence the need of mathematical tool in this field

       \item What is the optimal number of stations / terminals to be commissioned to process all applications, while avoiding an excessive queue and the departure of some "customers"?

       \item What is the average/median waiting time of a "customer" in front the station / terminal?

       \item What is the average/median number of "customers" waiting in the queue?
   \end{itemize}
   These questions express the goals of quality and level of service that are:
    \begin{itemize}
        \item An average or median waiting time in the queue or service that should not to be exceed

        \item A maximum probability of waiting time
 
        \item An average/median number of waiting customers 
    \end{itemize}
    In practice, when the customer is external to the company/administration, the waiting cost is difficult to evaluate because it is an impact rather than a cost that can be accounted (but sure the cost is not null!!!). However, the waiting time can be regarded as a measurement standard service level. The manager decides the acceptable waiting time and he sets up the capacity that can provide this level of service. 
    \begin{figure}[H]
		\centering
		\includegraphics[scale=0.9]{img/economy/queue.jpg}
	\end{figure}
    
    When the customer is internal to the company/administration, we can evaluate some costs directly related customer waiting time (machines/information). Moreover, we must not conclude too quickly that for now, the cost of an employee that is waiting is equal to his wage during the waiting time (even if this can be true for a small proportion of people...). This would imply that the net decrease in earnings of the company/administration, due to the inactivity of an employee is equal to the wage of the latter, which, a priori, is not easy to prove (excepted for IT or auditing companies that have for only purpose to put consultant into contracts). The employee, whether he works or wait will receive most of time the same salary (absurdity but than can be corrected with modern communication technologies). By cons, its contribution to the earnings of the company is really lost as its productivity is lower. When a machine operator is inactive because he is waiting its productive force (which may include, in addition to his wage, a proportion of the company's fixed costs) is lost. In other words, we must consider not only the physical resource that is on hold, but the value (cost) of all inactive economic resources, and then estimate the loss of profit from this lost productivity. Then the  objective of the analysis of queues is to find a compromise between the cost associated with the service capacity and cost of waiting customers.
    
    These theories therefore rely on statistics and algebraic methods that we have study in the sections of Statistics and Graphs Theory. They are just as most exciting.

    To introduce the topic, rather than doing an abstract generalization, we chose to develop the theory around a concrete and classic example that is the Teletraffic. A generalization to other case studies then being relatively easily by analogy. 
    
    Let us consider, therefore, a central telephone system regrouping the lines of a group of buildings in a city and not having as many lines going to the network as lines going to the different individuals it serves.

	So we can legitimately ask how many lines we need to serve all those subscribers.

	To dimension its network, an operator will have to calculate the number of resources to implement so that with a probability extremely close to $1$, a user who picks up his telephone can have a circuit (connexion). To do this, it will be necessary to develop some relations of probability of blocking. These relations will require statistical modeling of the start and end times of calls as well as the durations of these calls. The following paragraphs will introduce the probability laws used to dimension such networks.

	Finally, before we begin, we would like to make available the following summary table of the most important notations that the reader will discover as he reads the developments below and to which he can refer in case of confusion:
  	\begin{table}[H]
		\centering
		\begin{tabular}{|c|p{7cm}|c|}
		\hline
		\rowcolor[HTML]{9B9B9B} 
		\textbf{Variables} & \textbf{Information} & \textbf{Unit} \\ \hline
		$\lambda$ & Incoming flow of customers in the queue (communication), also named "average arrival rate of calls", or "average arrival frequency". The inverse $\lambda^{-1}$ gives the average time between arrivals (calls) in the queue. & $[h^{-1}]$ \\ \hline
		$\mu$ & Customer exit flow (communication) corresponding to the processing rate. The inverse $\mu^{-1}$ gives the average waiting time during the service (thus once arrived at the end of the queue) also named "average service time". & $[h^{-1}]$ \\ \hline
		$A$ or $\rho$ & Service utilization rate (per server unit). Assimilated to the concept of "traffic" (somewhat abusively) or "load". Is equal to the ratio $\lambda/\mu$ and must be strictly less than $1$ to avoid congestion. & - \\ \hline
		$C$ & Total number of customers in the system & - \\ \hline
		$C_Q$ & Number of customers waiting in the queue & - \\ \hline
		$C_S$ & Number of customers in service (processing) & - \\ \hline
		$T$ & Waiting time in the system & {[}s{]} \\ \hline
		$T_Q$ & Waiting in the queue & {[}s{]} \\ \hline
		$p_k$ & Probability of having $k$ customers in the system & - \\ \hline
		\end{tabular}
		\caption{Conventional notations in Queuing Theory}
	\end{table}
	And let us specify that the parameters to be taken into account are often:
	\begin{itemize}
		\item The type of customer arrival process (data)

		\item The Statistical distribution of service time (processing)

		\item The number of servers (checkouts)

		\item The capacity of the system (often assumed to be infinite in practice ...)

		\item The size of the population

		\item The service discipline
	\end{itemize}
	
	\pagebreak
	\subsubsection{$M/M/\ldots$ arrival times modelisation}
	In this "$M/M/\ldots$ model whose origins of notation will be explained further below, we consider calls that start randomly. Let us then take a time interval $t$ and divide this interval into $n$ subintervals of duration $t/n$.

	We choose $n$ large enough for the following assumptions (hypothesis) to be respected:
	\begin{enumerate}
		\item[H1.] Only one incoming call can occur in an interval $t / n$

		\item[H2.] Call arrival times are independent of each other (the arrival rate is not influenced by the number of calls from the population). This presupposes an infinite population.

		\item[H3.] The probability that a call arrives in a given subinterval is proportional to a constant factor close to the duration of the subinterval.
	\end{enumerate}
	We then write the probability $p_1$ of one call in a subinterval $(1)$ as follows:
	
	Where the $1$ in index of the $p$ represents the analysis on $1$ call, the $1$ between brackets the fact that the analysis is done on $1$ subinterval and finally the term $\lambda$ represents the coefficient of proportionality between the probability and the duration $t/n$ of the subinterval.

	The starting hypothesis of considering as zero the probability of having several calls in a subinterval is then written:
	
	The probability of having no call during a sub-interval of time $t/n$ is thus written:
	
	By developing, we get:
	
	and using the property defined just above:
	
	The probability of having $k$ calls arriving during $n$ time intervals is then obtained by considering the number of ways to choose $k$ intervals among $n$... (since there can not be more than one call per interval).

	For each of these solutions, we will then necessarily have $k$ intervals with a single incoming call and $n-k$ intervals with no incoming call. We have seen in the section Statistics that the law which made it possible to get the probability of choosing a certain arrangement of binary issues from a total number of issues was Bernoulli's law given by:
	
	It therefore comes in our case that the probability of one of the solutions will be:
	
	The global probability is obtained by summing the probabilities of all the cases, which gives us the binomial law (\SeeChapter{see section Statistics page \pageref{binomial distribution}}):
	
	Or, by replacing the probabilities by their values as a function of $\lambda$, $t$, $n$:		
	
	The limit of the probability $p_k(n)$ when $n$ tends to infinity will be equal to the probability of having $k$ call arrivals during a time interval $t$. We denote by $p_k$ this probability:
	
	Then, taking up the different terms of the expression of $p_k(n)$, it comes:
	
	Using the Taylor developments (\SeeChapter{see section Sequences and Series page \pageref{usual maclaurin developments}}):
	
	Either by taking only the first term, that is, considering $x$ very small:
	
	Therefore:
	
	And for the last part:
	
	hence after grouping:
	
	This distribution function is therefore extremely important, since it represents the probability of observing $k$ calls arriving in an interval of duration $t$ (or the probability of observing $k$ customers that are in front of a checkout or server in an interval of duration $t$) and it s therefore a Poisson distribution (\SeeChapter{see section Statistics page \pageref{poisson distribution}}). In practice, it is therefore necessary to ensure before using the relation that will follow in the further developments that this distribution is well respected (with a Chi-square test typically).

	Let us consider now the distribution function, named the "\NewTerm{Erlang distribution function}\index{Erlang distribution function}\label{erlang distribution}", which for a given time interval $t$ gives the cumulative probability of having a number $k$ of arrivals greater than or equal to $n$:
	
	If we derive this function with respect to $t$ this gives the "\NewTerm{Erlang distribution function}\index{Erlang distribution function}" (the reader may have noticed perhaps that in the penultimate line we thus find an example for the field of Managnement of the concept of "telescopic series" which had been introduced in the section of Sequences and Series):
	
	and thus we fall back on a particular form of the distribution function of the Gamma law introduced in the section of Statistics. Let us recall the equivalence already presented in the section of Statistics with Microsoft Excel 14.0.6123 between the Poisson law, the Chi-2 law and the Gamma law:
	\begin{center}
		\texttt{=POISSON.DIST(}$x \in \mathbb{N}$\texttt{,}$\mu$\texttt{,TRUE)}\\
		\texttt{=1-CHISQ.DIST(}$2*\mu$\texttt{,}$2*(x+1)$\texttt{,TRUE)}\\
		\texttt{=1-GAMMA.DIST(}$\mu$\texttt{,}$x+1$\texttt{,}$1$\texttt{TRUE})
	\end{center}
	It follows by analogy with the general form of the Poisson law that the parameter $\lambda$ is the average arrival rate (mean rate of calls) per unit of time (or "\NewTerm{Poisson Arrivals See Time Average}\index{Poisson Arrivals See Time Average}": PASTA ...). Typically this will be an average number of calls per second (see the Poisson law estimators in the section Statistics).

	Thus, we have for the mean and variance (\SeeChapter{see section Statistics page \pageref{poisson distribution}}) the number of calls:
	
	When the model consider a constant event occurrence rate $\lambda$ we then speak of "\NewTerm{homogeneous Poisson process}\index{homogeneous Poisson process}" (HPP).
	\begin{tcolorbox}[colframe=black,colback=white,sharp corners]
	\textbf{{\Large \ding{45}}Example:}\\\\
	A very small company wishing to set up a hotline estimates that at the beginning it will receive $8$ phone calls per day (ie a probability of $1$ chance out of $2$ to have one call per hour and therefore an average rate $\lambda$ of $0.5$ calls per hour). Then the probability that it receives exactly $4$ calls ($k$) per day and at least $4$ calls ($k$) per day according to the theoretical model of the queuing theory is:
	
	Where we used the function \texttt{POISSON.DIST()} embedded in Microsoft Excel 14.0.6123.
	\end{tcolorbox}
	Now let us introduce the random variable $\tau$ representing the time between two call arrivals. We define for this the probability $A(t)$ which is the probability that the time $\tau$ is less than or equal to a value $t$:
	
	We therefore have:
	
	But, $P(\tau>t)$ represents the cumulative probability that there is no arrival of calls during a time $t$. This cumulative probability was precisely established above:
	
	From this we deduce:
	
	It is therefore the function of distribution of an exponential law! We can also introduce the probability density of the random variable $\tau$. We thus get:
	
	\begin{tcolorbox}[title=Remark,colframe=black,arc=10pt]
	Let us recall that in the section Statistics, we often did the opposite reasonning. That is to say, given a probability density $a(t)$, we sought the distribution function $A(t)$ through an integration on the domain of definition of the random variable.
	\end{tcolorbox}
	The probability density thus makes it possible to calculate the expected mean duration between two call arrivals:
	
	By integrating by parts, it comes:
	
	We thus obtain that, for an arrival rate of $\lambda$ calls, that the expected mean time between calls is equal to $1/\lambda$ (a relatively logical result but still had to be rigorously proved). Indeed, if we have $\lambda$ that is worth $2$ calls per hour, the average arrival time is indeed $0.5$ hours ($1/2$) between calls.

	Let us now suppose that no call has arrived until an time $t_0$ and we want to calculate the probability that a call arrives during a time $t$ after the time $t_0$

	We must therefore calculate the probability of having a duration between two calls less than $t+t_0$ while being greater than $t_0$

	This probability is written $P(\tau\leq t+t_0|\tau>t_0)$. Using the Bayes formula (\SeeChapter{see section Probabilities page \pageref{bayes formula}}):
	
	But with the appropriate notations, it comes:
	
	This probability can also be written:
	
	By taking the expressions of the different probabilities:
	
	We thus see that the probability of occurrence of a call during a time $t$ after a duration $t_0$ during which none has occurred is the same as the probability of occurrence of a call for a duration $t$, independent of what happened before. We therefore consider that the phenomenon (the exponential law) is "without memory"\label{without memeory process}.
	
	\pagebreak
	\subsubsection{$M/M/\ldots$ service times modelisation}
	In this model $M/M/\ldots$ for which we will explain again the origin of the notation later, we will study the probability laws that model the durations of the calls (implicitly: in service once the end of the queue reached). To do this, we proceed as before.

	We therefore consider a time interval of duration $t$ which we decompose into $n$ subintervals of duration $t/n$. We choose $n$ so that the following assumptions remain justified:
	\begin{enumerate}
		\item[H1.] The probability that a call terminates during a subinterval is proportional to the duration of that subinterval.

		We will denote:
		
		this probability, in which $\mu$ represents the coefficient of proportionality.

		\item[H2.]  The probability of a call terminating during a subinterval is independent of the subinterval considered.
	\end{enumerate}
	We then introduce the random variable $\theta$ representing the duration of a call and the probability $H(t) $that the duration of a call is less than or equal to $t$:
	
	The probability that a call that started at $t = 0$ does not end before $t$ is then written:
	
	This probability is equal to the probability that the call does not end in any of the $n$ subintervals of duration $t/n$:
	
	By taking $n$ towards infinity, we get:
	
	We thus obtain the expression of the probability that a call has a duration less than or equal to $t$:
	
	We can deduce the associated probability density, denoted $h(t)$:
	
	which corresponds to an exponential law (the time that a customer use to be served at a check-out or server - period of service - or so that his call is processed once at the end of the queue, follows then an exponential law!). In practice, it is therefore necessary to ensure before using the relations that follow that this distribution is well respected (with a Chi-square test typically).

	In the same way as in the preceding paragraphs, the average call duration (average time between two ends of calls in service) is obtained by calculating (always the same integration by parts as above):
	
	In conclusion, we have $\mu$ calls that cease per second and we have an expected mean call duration in service equal to:
	
	Indeed, if we have, for example, $2$ calls in-service  that stop per hour, this gives us an average time of $0.5$ hour ($1/2$) of service per call.

	The ratio:
	
	represents therefore the number of calls that appear in the queue on the number of calls in service that end during the same time interval (mean service time), that is, it simply represents the traffic (or in other words, the "traffic intensity") or from another point of view the average use of the service whose unit is the "\NewTerm{Erlang}" (we will recall this definition several times thereafter...).
	\begin{tcolorbox}[colframe=black,colback=white,sharp corners]
	\textbf{{\Large \ding{45}}Example:}\\\\
	In a store, there are $240$ customers per hour and it takes an average of $28$ seconds to process a customer (average service time). Knowing that the length of service follows an exponential law and the distribution of arrivals a Poisson law, what is the intensity of the traffic and the rate of occupation of the check-out stations if the store has only two of them?

	The traffic is therefore given by the number of customers per hour divided by the number of customers processed per hour. As it takes $28$ seconds to process one and there are $3,600$ seconds in an hour, we have the following traffic intensity:
	
	The average load (or occupancy rate) per check-out station (server), knowing that there are two of them, is therefore:
	
	\end{tcolorbox}
	It is this latter value which will ultimately be taken as the value of traffic $A$ per station (check-out/server) for subsequent calculations.

	The probabilities of appearance of calls and end of calls that have been developed in the preceding paragraphs make it possible to model the complete process of appearance and end of calls.
	
	\subsubsection{Kendall queues notation}
	A notation was developed by D. G. Kendall in 1953 to represent the queues following the intensive (and numerous) developments of the mathematical models concerning them. The reduced form of this notation named "\NewTerm{Kendall notation}\index{Kendall notation}" or "\NewTerm{Kendall-Lee notation}" is:
	\begin{gather*}
		\boxed{A/B/C}
	\end{gather*}
	where:
	\begin{itemize}
		\item $A$ is the customer arrival process in the system (distribution interarrival law)
		\item $B$ is the distribution of the service time of the customers in the system (distribution law)
		\item $C$ is the number of servers in the system.
	\end{itemize}
	Common notations for $A$ are:
	\begin{itemize}
		\item $M$ for Markovian or memoryless arrival Poisson time process
		\item $G$ general distribution (any arbitrary distribution allowed)
		\item $E_k$ Erlange distribution
	\end{itemize}
	Common notations for $B$ are:
	\begin{itemize}
		\item $M$ for Markovian or memoryless arrival exponential time process
		\item $G$ general distribution (any arbitrary distribution allowed)
		\item $E_k$ Erlange distribution
	\end{itemize}
	For example, the $M/M/1$ notation means that the clients arrive in the system according to a Poisson distribution modeled by a Markov chain (\SeeChapter{see section Probabilities page \pageref{markov chains}}), that the processing time follows an exponential distribution (modeled by a Markov chain as well) and the system consisting of a single server based on a first-in first-out principle (FIFO)  in an infinite-population and steady-state queue. This corresponds respectively to the three following relations which we have proved above for the probability of arrival of $k$ calls in a given time:
	
	The probability that the processing time (service time) is equal to a certain value:
	
	and the probability of having $k$ customer (communications):
	
	In general, we use the notation:
	\begin{gather*}
		\boxed{A/B/C:d/e/f}
	\end{gather*}
	where:
	\begin{itemize}
		\item $d$ is the capacity of the system, or the maximum number of customers allowed in the system including those in service. When the number is at this maximum, further arrivals are turned away. If this number is omitted, the capacity is assumed to be unlimited, or infinite.

		\item $e$ is the service discipline or priority order that jobs in the queue, or waiting line, are served (see examples further below)

		\item $f$ is the size of calling source. The size of the population from which the customers come.  If this number is omitted, the population is assumed to be unlimited, or infinite.
	\end{itemize}
	The parameter of queue's discipline $e$ has typically the following properties:
	\begin{itemize}
		\item FIFO/FCFS (First In First Out/First Come First Served): The customers are served in the order they arrived in.

		\item LIFO/LCFS	(Last in First Out/Last Come First Served): The customers are served in the reverse order to the order they arrived in.

		\item SIRO (Service In Random Order): The customers are served in a random order with no regard to arrival order.

		\item SJF (Shortest Job First): The customer with shorthest ranked treatments are served before

		\item $\ldots$
	\end{itemize}
	 If this parameter is omitted, the discipline is assumed to be a FIFO one.
	 
	Thus, $M/M/1$ would be written without omitting anything in our study case above:
	\begin{gather*}
		M/M/1:\infty/\text{FIFO}/\infty
	\end{gather*}
	We also find variants in the literature of the definition above which does not always help in the reading of various textbooks about the subject (absence of ISO standards doesn't help)...
	
	\subsubsection{Modeling of arrivals and departures $M/M/1$}
	At every instant a certain number of calls will appear and others will end. We can therefore model the state where we are at a given moment as a chain of states.

	Each state represents the number of calls in progress. We therefore conceive that if at a given instant there are $k$ calls, we can pass only in two adjacent states according to our hypotheses: $k-1$ and $k + 1$.

	We then recognize a Markov chain (\SeeChapter{see section Probabilities page \pageref{markov chains}}). The probability of going from a state $i$ to a state $j$ during a time $\mathrm{d}t$ will therefore be denoted:
	
	We then introduce the following state transition probabilities:
	\begin{itemize}
		\item Being in state $k$, the probability $p_{k,k+1}(\mathrm{d}t)$ to go to the state $k+1$ during a time interval $\mathrm{d}t$ will be denoted $\lambda_k\mathrm{d}t$

		\item Being in state $k$, the probability $p_{k,k-1}(\mathrm{d}t)$ to go to the state $k-1$ during a time interval $\mathrm{d}t$  will be denoted $\mu_k\mathrm{d}t$

		\item Being in the state $k+1$, the probability $p_{k+1,k}(\mathrm{d}t)$ to pass to the state $k$ during a time interval $\mathrm{d}t$ will be denoted $\mu_{k+1}\mathrm{d}t$

		\item Being in the state $k-1$, the probability $p_{k-1,k}(\mathrm{d}t)$ to go to state $k$ during a time interval $\mathrm{d}t$ will be denoted $\lambda_{k-1}\mathrm{d}t$
	\end{itemize}
	\begin{figure}[H]
		\centering
		\includegraphics{img/economy/mm1_markov_chain_state.jpg}	
	\end{figure}
	The quantities $\lambda_k$ and $\mu_k$ are rates of arrival (appearance) and departure (end) of calls of the same type as those used in the preceding paragraphs. The only difference is that these rates have in index the state in which the system is.

	We can then introduce the state probability, that is, the probability of being in a state $k$ at time $t$. Let us denoted $p_k(1)$ thisthis probability (to be compared with the notation $p_j(n)$ used for the discrete-time Markov chains in the section of Probabilities).

	The variation of this probability during a time interval $\mathrm{d}t$ is then equal to the probability of joining this state from a state $k-1$ or a state $k+1$ minus the probability of leaving this state to go to a state $k-1$ or to a state $k+1$.

	What is written:
	
	Assuming the system stable, that is, assuming that it stabilizes on fixed state probabilities when time tends to infinity, we can write that:
	
	We can then denote $p_k=p_k(t)$ hence finally:
	
	We could have introduced this last relation in another way: It simply expresses the fact that the probability of starting from a state is equal to that to arrive at it (it is perhaps more simple like this).

	This relation is satisfied for any $k\geq 0$ with the following mathematical conditions (because otherwise these terms have no mathematical meaning):
	
	and the following real logical condition (calls not yet existing can not end...):
	
	\begin{tcolorbox}[title=Remark,colframe=black,arc=10pt]
	Let us emphasize that the stability of probabilities means that there is an equal probability $p_k$ of leaving the state that to join it.
	\end{tcolorbox}
	By using (see above:
	
	we get therefore:
	
	We then find quite easily the general form:
	
	Since the system is necessarily in one of the states, we have the following relation which must be satisfied:
	
	By replacing with the antecedent relationship:
	
	This also gives:
	
	and therefore:
	
	If we now consider a system with a single line and infinite capacity (steady state), the quantities $\lambda_k$ (call arrival rate) and $\mu_k$ (call departure rates) will have identical values for all $k$. That is, we consider that the arrival rate and the starting rate are constant regardless of the position in which the queue is located. We then have the last relation which simplifies:
	
	Using the result proved in the section Sequences and Series (Gauss series) we have under precise necessary conditions of convergence ($A$ must be strictly smaller than $1$):
	
	Since $k$ represents the number of calls, this last relation gives the probability of having $0$ calls (clients) for a given permanent traffic $A$ on the line.

	Using:
	
	We then have in generality for this type of system the probability of having $k$ calls in steady state which is given by:
	
	It follows that the expected mean of number of communications (clients) in the system (in the queue $+$ in service) is then by definition of the expected mean:
	
	since $p_k$ represents the probability that there are $k$ calls in the system at any time (queue $+$ in service). However, we have seen in the section of Sequences and Series that:
	
	and if $q$ is strictly less than $1$ and $n$ tends to infinity, we immediately have:
	
	If we derive this last relation:
	
	and by multiplying by $q$ then it comes:
	
	It then comes to the end for the expected mean of the number of customers in the system:
	
	If we want to know the expected mean of the number of customers waiting in the queue only, we must understand that at every moment we have a probability $p_k$ that there are $k$ customers in the system but as $1$ among these is always in service (ie out of the queue), we still have $k-1$ really waiting in the queue. Therefore:
	
	Knowing the number of communications (or customers) we have on the single line throughout the system, we then have the expected mean of the waiting time (average waiting time), denoted $\text{E}(T)$ which will be given by the ratio of the expected mean of the number of communications (customers) $\text{E}(C)$ in steady state on line  by the arrival rate of the calls:
	
	and often written in textbooks as:
	
	that is to say long-term average number of customers in a stable system $L$ is equal to the long-term average effective arrival rate, $\lambda$, multiplied by the average time a customer spends in the system, $W$.
	
	This result, named "\NewTerm{Little's law}\index{Little's law}" (John Little having strictly demonstrated that the relation is valid for any type of queue), is intuitive in this case. Actually, let's take a random call. When it arrives in the system, it will be statistically facing $\text{E}(C)$ customers waiting. When it leaves the system, there will have been in it an average time $\text{E}(T)$. So during this mean time, $\lambda\text{E}(T)$ calls will have arrived behind him in the system. In steady state, the number of calls left behind on departure must equal the number of arrivals. Hence the equality $\lambda\text{E}(T)=\text{E}(C)$ from which we then deduce immediately the Little's law.
	
	We then have for the expected mean of waiting time in the system:
	
	
	To determine the waiting time in the queue only, it is sufficient to subtract the processing / service time from the linearity property of the mean (average call duration in service):
	
	that we often found in the literature in the following form:
	
	If there are $n$ servers (or cashiers) we don't change the value of $\mu$ as from the point of view of one customer that has to choose ONE server (or cashier) the fact that there are more than $1$ server (or cashier) does not influence the fact the he has to pass anyways by one of the server (or cashier), but we divide $\lambda$ by $n$ as if the servers are equiprobably choosed by the customers he can divide the number of customers arriving customers by the number of cashiers such that:
	
	with the condition that (otherwise the expected mean will be negative):
	
	
	\begin{tcolorbox}[colframe=black,colback=white,sharp corners]
	\textbf{{\Large \ding{45}}Example:}\\\\
	If we have $n=3$ (cashiers), $\lambda=1/2$ per minute ($30$ customers arriving per hour), and $\mu=1/5$ per minute (customers rate served by each cashier equal to $12$ customers served per hour) then:
	
	\end{tcolorbox}
	To summarize, since there are many parameters and results, we have for a type $M/M/1$ queue (according to the Kendall notation):
	\begin{table}[H]
		\centering
		\begin{tabular}{|l|c|}
		\hline
		\rowcolor[HTML]{9B9B9B} 
		\multicolumn{1}{|c|}{\cellcolor[HTML]{9B9B9B}\textbf{Information}} & \textbf{$\pmb{M/M/1}$} \\ \hline
		Probability of empty system & $1-A$ \\ \hline
		Waiting probability & $A$ \\ \hline
		Number of customers in the system & $\text{E}(C)=\dfrac{A}{1-A}$ \\ \hline
		Average number of customers in the system & $\text{E}(C_Q)=\dfrac{A^2}{1-A}$ \\ \hline
		Number of customers in service (treatment) & $A=\dfrac{\lambda}{\mu}$ \\ \hline
		Average time in the whole system & $\text{E}(T)=\dfrac{\text{E}(C)}{\lambda}=\dfrac{1}{\mu-\lambda}$ \\ \hline
		Average time in the queue & $\text{E}(T_Q)=\dfrac{1}{\mu}\left(\dfrac{A}{1-A}\right)$ \\ \hline
		Condition of equilibrium (for one server!) & $\dfrac{\lambda}{\mu}<1$ \\ \hline
		Probability of having $k$ customers & $p_k=A^k(1-A)$ \\ \hline
		\end{tabular}
		\caption{$M/M/1$ queue summary of important relations}
	\end{table}
	We see that some relations diverge to infinity when the permanent traffic $A$ tends to unity. This is why we had imposed above that this parameter is strictly less than unity!

	We should for each possible type of queue, detail the proof of the corresponding relation above, which is long and laborious (it is a job/specialization on its own right and there are really good textbooks of more than $400$ pages on the subject like \cite{ng2008queueing}).
	\begin{tcolorbox}[colframe=black,colback=white,sharp corners]
	\textbf{{\Large \ding{45}}Example:}\\\\
	Let us consider that we have a numerical control machine processing parts one at a time. Assume that $\lambda=8$ parts per hour (number of pieces arriving on average per hour) and that $\mu=10$ parts per hou (number of pieces coming out on average per hour). We have then:
	
	Which corresponds to the traffic or occupancy rate of the machine.\\
	
	The expected number of parts in the system (machine + queue) is given by:
	
	The expected number of parts waiting in the queue only is given by:
	
	The expected waiting time in the system is given by:
	
	that is to say $30$ minutes.\\
	
	The expected waiting time in the queue only is given by:
	
	that is to say $24$ minutes.\\
	
	And the probability that there are 5 pieces in the system (execution + expectation):
	
	\end{tcolorbox}
	
	\pagebreak
	\subsubsection{Probability of standby in a $M/M/k/k$ queue (Erlang-B formula)}
	We are interested here in a system with $N$ communication channels (each channel is supposed to support a call rate with an immediate response). If the $N$ channels are busy, incoming calls are considered lost (eg, no dial tone). We are talking about "blocking the system" or "ruining the system". It is therefore a limited queue of type $M/M/k/k$ according to the Kendall notation, also named "system with loss".

	We will try to estimate this probability of blocking according to the number of available channels and the traffic.

	Given what was stated on the memoryless character of the call arrival process, we can consider that the probability:
	
	to have $k$ calls in the state $k$ is independent of the state of the system such that:
	
	Thus, at each state $k$ of the system, the Poisson-type probability law is valid. The difference in treatment is that rather than considering states, we will consider that a communication channel can be considered as a state of its own.

	But for the probability of end of call, we have:
	
	Indeed, this probability just reflects the fact that if $k$ calls are in progress each has a probability $\mu\mathrm{d}t$ to terminate, hence the sum that gives $k\mu\mathrm{d}t$. We have then:
	
	Thus, by injecting these relations in:
	
	it comes:
	
	Introducing then (which must be strictly less than $1$ if we want the following developments to converge to a finite value: ergodic Markov chain):
	
	which is for recall the number of calls that appear on the number of calls that end in a given time interval (average service time), which is actually simply the "traffic" (or the "traffic intensity" or "traffic rate"), then it comes:
	
	or by introducing the $1$ in the summation:
	
	Since $k$ represents the number of calls, this last relation gives the probability of having $0$ calls (clients) for a given permanent traffic $A$ in the system.

	By reporting $p_0$ in the following expression of $p_k$ (probability of being in the state $k$ therefore...) obtained above:
	
	it comes:
	
	And considering the character without memory, we have the relation:
	
	it comes:
	
	where the $k$ can unfortunately be confusing. It is necessary to do some cleaning! In the numerator, the $k$ refers to the number of channels (servers, lines, operators or terminals) and at the denominator the $N$ also. It is therefore necessary to rewrite this more appropriately:
	
	which therefore gives the probability of placing on hold (and thus the probability of saturation / blocking) of a system having $N$ channels of finite capacity according to the principle of first in / first out and for a traffic $A$ (expressed in "Erlang") and in which communications are lost if placed on hold.

	This relation is sometimes denoted in the literature in the form of:
	
	This relation is very important in queuing theory and is named the "\NewTerm{Erlang-B formula}\index{Erlang-B formula}".
	\begin{figure}[H]
		\centering
		\includegraphics{img/economy/erlang_b_plot.jpg}	
		\caption{Probability of saturation (smoothed plot of the Erlang-B function)}
	\end{figure}
	This relation is at the basis for the design of circuit-switched networks. Indeed, the problem of sizing a circuit switch is as follows: Given the traffic in number of calls per unit of time $A$ in Erlang, find the number of service units $k$ such that the probability that a call arrives in a system that has become blocking is lower than a certain value.

	The probability obtained then represents the quality of service offered by the network from the point of view of the usage. And about the traffic $A$, that latter is estimated on the basis of the number of existing and / or future telephone installation based on an average activity per station and per application (telephone, fax, terminal, computer server, etc.).
	\begin{tcolorbox}[colframe=black,colback=white,sharp corners]
	\textbf{{\Large \ding{45}}Example:}\\\\
	E1. What is the probability of saturation of a hotline (whose duration of service follows an exponential law and the distribution of the arrivals follows a Poisson law) knowing that the traffic $A$ of the line is estimated to be $2$ [Erlang] ($1$ call per hour for $1$ call processed by $1/2$ hour - hence a ratio of $2$ to $1$) for a single telephone line ($N = 1$) using the Erlang-B model:
	\\
	
	E2. In a company, we have counted $200$ calls during the peak hours of an average of $6$ minutes per hour (average service time). What is the probability of saturation with $20$ operators (knowing that the length of service follows an exponential law and the distribution of arrivals a Poisson law)?

	The biggest difficulty here is to calculate the traffic! There are thus $200$ calls per hour with $10$ calls processed only per hour (since $6$ minutes per call in an hour of $60$ minutes gives: $10$ calls). The traffic $A$ is therefore $200/10$ or $20$ Erlang. Applying then the previous relation, we have:
	
	In the industry, a saturation rate of $0.01\%$ is assumed. By playing with a spreadsheet like Microsoft Excel and the Target Value tool, we quickly find that $N$ must be equal to $30$.
	\end{tcolorbox}
	
	\subsubsection{Probability $M/M/K/+\infty$ of standby (Erlang-C formula)}
	Let us consider now a system for which calls can be queued before being served when the $k$ servers are blocked. It is therefore an unlimited capacity queue of type $M/M/k/+\infty$ according to the Kendall notation.

	With this system, we always have:
	
	But for the probability of end of call the analysis becomes more subtle. First there is the probability that the calls on the available $N$-channels will cease and that is given by:
	
	But as soon as the number of calls is greater than the number of communication channels available, the probability that the calls cease is:
	
	Which means that regardless of the number of calls, $N$ have the probability of being put on hold (standby) as soon as $k$ (the number of calls) is greater than or equal to $N$.

	Thus, to summarize:
	
	Using:
	
	We get by decomposition of the product term:
	
	Hence finally:
	
	Using the expression of $p_0$:
	
	We can decompose the summation:
	
	and decompose the second product:
	
	Under the assumption that:
	
	The sum:
	
	can be simplified. Indeed, by putting:
	
	it comes:
	
	and as we have prove it in our study of the Zeta function (\SeeChapter{see section Sequences and Series page \pageref{zeta function}}) this sum can be written in the form:
	
	Therefore:
	
	So finally:
	
	Since $k$ represents the number of calls, this last relation gives the probability of having $0$ calls (customers) for a given permanent traffic $A$ in the system.

	We thus have for $\forall k\ge N$:
	
	The cumulative probability of queuing is denoted $C(N, A)$. It is equal to:
	
	By proceeding exactly as in the preceding paragraphs, we get finally:
	
	which therefore gives the probability of putting on hold (and thus of saturation / blocking - that is to say all servers are blocked) upon arrival in a system having $N$ channels with infinite capacity according to the principle of the first in/first out (FIFO) and for a constant traffic $A$ (expressed in "Erlang") and in which communications can be placed on hold (at the opposite to the Erlang-B model). This last relation is named the "\NewTerm{Erlang-C formula}\index{Erlang-C formula}". Traditionally we write this last relation as following:
	
	the "$W$" coming from "Wait".

	The model proposed above is obviously open to criticism because in reality the capacity of the queue is finite and some clients give up when the wait is too long.
	\begin{tcolorbox}[colframe=black,colback=white,sharp corners]
	\textbf{{\Large \ding{45}}Example:}\\\\
	If we take an arrival rate of $1$ call per minute and an average service time of $5$ minutes, then we have:
	
	If we take a number $N$ of $7$ servers, we have:
	
	Therefore $32.41\%$ cumulative probability of being put on hold (standby). What a is little too much ... (an empirical rule is to look for the number $N$ of servers so that the latter value falls below $20\%$).
	\end{tcolorbox}
	
	
	\pagebreak
	\subsection{Insurance}\label{insurance}
	Insurance is an operation in which a person (the "insurer") groups other persons (the "insured") into mutual insurance schemes in order to put them in a position to compensate each other for possible losses (losses) to which expose them certain risks\footnote{For a list of most common riks in project management the reader can refer to the French free PDF: \textit{Gestion de projets pour ingénieurs et scientifiques} (Project management for engineers and scientists)}, thanks to amounts of cash (premiums or contributions) paid by each insured individual to a common mass (cohort) managed by the insurer.

	The common denominator of the majority of laws defines the insurance contract as a contract whereby, subject to the payment of a fixed or variable premium, one party, the insurer, undertakes to another party, the insured (also named "policyholder") to provide a benefit stipulated in the contract in the event of an uncertain event that the insured or the beneficiary, as the case may be, has an interest in not being realized.

	Obviously, the insurance transaction results in the transfer (total or partial) of the financial consequences of the risk incurred by the insured to an insurance company. Therefore, upon entering into the contract, the insurer and the insured agree to:
	\begin{itemize}
		\item Of an event or list of events, included in the insurance policy, and guaranteed by the insurer.

		\item A premium paid by the insured to the insurer.
	\end{itemize}
	The expenses assumed by the insurer may correspond to:
	\begin{itemize}
		\item Either to indemnities to be paid to third parties, in respect of the liability (civil, professional or other) of the insured in a given laps of time

		\item Either to repair partially or totally to the damage sustained by the latter.
	\end{itemize}
	\begin{tcolorbox}[title=Remark,colframe=black,arc=10pt]
	We decided to put the study of Insurance in the section of Quantitative Management rather in that of Economy as we have never seen until now a Manager (even the best one) deal with options, shares and so on... (this is the job of Finance graduate people). But we have seen quite often Managers, and this is normal as the degree of difficulty is quite low, seek for insurances contracts to mitigate the risks of some projets that they have to manage.
	\end{tcolorbox}
	
	\pagebreak
	The main assumptions used in the insurance industry are as follows:
	\begin{enumerate}
		\item[H1.] From a legal point of view, an insurance contract is a "random contract" valid only to cover risks with a random component.
	
		\item[H2.] The "rule of the game" of the risk must be stable within a period of time considered as long (at least a few years).
	
		\item[H3.] The maximum possible loss must not be too great in relation to the insurer's solvency margin.
	
		\item[H4.] The average risk premium must be identifiable and quantifiable according to well-chosen explanatory statistical variables in order to allow a segmentation of risk management (small anecdote: when actuaries are criticized for their segregation practices, we often hear the answer: \textit{We do not discriminate, but differentiate}...).
	
		\item[H5.] The risks must be independent (and if they are identical distributed in terms of probabilities and weighting this is better ...) and demonstrated to be such significantly using statistical tools.
	
		\item[H6.] There must be a market in the sense that the supply and demand for insurance must arrive at an equilibrium price (in a sense, the equivalent of the absence of arbitrage opportunity in finance).
	
		\item[H7.] Mathematical expected mean is considered to be the price of the right pure premium to be paid by the insured.
	\end{enumerate}
	 \begin{figure}[H]
		\centering
		\includegraphics{img/economy/risk_insurance_cartoon.jpg}	
	\end{figure}
	\textbf{Definitions (\#\mydef):}
	\begin{enumerate}
		\item[D1.] The "\NewTerm{pure premium}\index{pure premium}" is in the field of insurance chosen as the mathematical expeted mean value of the burden. It corresponds to the minimum premium that an insurer may ask for not be ruin with certainty.

		\item[D2.] The "\NewTerm{safety charge}\index{safety charge}" is the amount that comes in addition of the pure premium and allows the insurer to withstand the volatility of refunds.

		\item[D3.] The "\NewTerm{management fee charge}" is the amount that is added to the previous two and is related to the operation of their company, contract management, premium recovery, investment of assets (basic technical premium), taxes...

		\item[D4.] The "\NewTerm{commercial expenses charge}" is added to the previous three and is linked to the acquisition of contracts (commission of intermediaries, costs of commercial networks, advertising).

		\item[D5.] In the end, all the costs are reflected in the "\NewTerm{commercial premium}\index{commercial premium}" that is the one communicated to the customer.
	\end{enumerate}
	We can already conclude that in the case of a compulsory or voluntary state Care insurance system, management and commercial costs will always be lower, under the assumption of an equal management method, and a private system of wage attribution...
	\begin{tcolorbox}[title=Remark,colframe=black,arc=10pt]
	For obvious reasons, it shall be mandatory (as for every business), and required by an ad hoc legislation, that for every price (an especially in the insurance industry), that the percentage of the price allocated to marketing, commercial, management, etc. to be communicated.
	\end{tcolorbox}
	
	\subsubsection{Premium pricing}
	So the insurer does not know exactly what the amount of claims is going to be. By taxing contracts at the pure premium level (and assuming a symmetrical loss distribution), the insurer loses money one year on two. In the absence of capital, this situation would quite quick lead to bankruptcy.

	To protect himself, the insurer adds to its premium a security charge. Numerous methods of determining it are possible, none having so far largely supplanted the others:
	\begin{itemize}
		\item Charge proportional to the pure premium: The coefficient of proportionality reflects the insurer's view of the volatility of risk.

		\item Charge dependent on the standard deviation of the losses: This method is a slight formalization of the previous one. It poses a problem because it will introduce a security load that will depend on winnings (actual loss less than the pure premium), excepted if we use the semi-variance.

		\item Charge dependent on a certain quantity of losses (eg the third quartile). Such a charge ensures that the premium will be sufficient in a predetermined number of cases but does not give any information on technical losses.
	\end{itemize}
	Let us consider the case where the population would be heterogeneous, in extenso two classes $S_A$, $S_B$ of risk coexist in the population of respective weight $n_A$, $n_B$ with of course:
	
	Let us consider a health insurance with two categories of insured (healthy young $A$ / seniors at risk $B$).
	
	To simplify the example to the extreme, let us imagine that the insurer knows, by means of its internal statistics, that an individual in group $A$ will cost the insurance a sum defined by a statistical distribution law that we will denote here:
	
	Then to any cost $x$ is associated a given some cumulative probability given by the function $F_X^A$. Similarly for group $B$:
	
	If we randomly select an individual in the portfolio, if $\theta$ refers to the group (traditional insurance notation for groups), the insurer should therefore claim in pure premiums for group $A$:
	
	hence the expectated mean of the function $F_X^A$.
	
	Same for group $B$:
	
	And for a random individual in both groups:
	
	The latter case is named the "\NewTerm{solidarity mechanism}\index{solidarity mechanism}". So the good risks pay for the bad risks ... In fact it's a good way to guarantee "equality" between insured people but not "equity" hence the huge problems that encounters some governments for their system of insurance. The mathematical models of all insurance I know so fare don't take into account the equity that is in physics a fundamental principle of proportion (but don't ask economists to do physics...).

The variance of the pure premium will be given by the relation proved in the Statistics section when studying the properties of the variance (variance of two statistical series):
	
	Where we see that the term:
	
	at the numerator corresponds to the homogeneity of the two groups. Hence, heterogeneity increases the variability of the pure premium in the solidarity mechanism especially when equity is not taking into account.

	Now let us imagine the case where:
	
	and a private insurance (1) which practices the solidarity mechanism and another insurance (2) which does not practice it. In the present case, we must distinguish two situations from an economic point of view:
	If all the insured are rational (therefore somewhat... egoistic) then:
	\begin{itemize}
		\item Healthy young people representing group $A$ will go to private insurance (2) which has segmented the risks and therefore allows the young people to pay less.

		\item Seniors in poorer health, representing group $B$, go to private insurance (1), which has not segmented risks, but which, by ideology, has applied the principle of solidarity.
	\end{itemize}
	The conclusion is that private insurance (1) will quickly go bankrupt because:
	\begin{itemize}
		\item The good risks no longer compensate the solidarity mechanism because of selfishness

		\item  We are in a competitive market where insurance are not ideological ...
	\end{itemize}
	This fact is named the "\NewTerm{anti-selection problem}\index{anti-selection problem}" or "\NewTerm{adverse selection}\index{adverse selection}" based on the approach of G. Akerlof, Nobel Prize for Economics in 2001.

	The conclusion is that the privatization of insurance based on a solidarity principle can not work without incurring additional costs for pure premiums because of a periodic anti-selection that generates a constant flow of insured persons moving from one insurance to another and thus generates phenomenal administrative and computer costs! So in every point the suppression of competition (concurrence) remains better in terms of solidarity but not in terms of jobs for insurance employees (who would then be for the majority unemployed...) or for the advantages of some CEO and politicians....!!!!!

	Thus, the elimination of competition (concurrence) for insurance services which all have equivalent quality services (as in the case of health insurance in Switzerland, for example) eliminates the principle of anti-selection, applies in a concrete way the solidarity mechanism wanted by the state and finally reduces the administrative costs due to the comings and goings (flow) of the insured and the development of IT tools of business-management costing millions to each insurance and which incur costs which in the end are passed on to the price of the commercial premium!

	In addition, in order to reduce overall volatility (global standard deviation), insurance should theoretically segment the risks indefinitely, making it an unsustainable system for certain specific areas of insurance.

	But it must be remembered that this conclusion is valid only if the insurance benefits are identical (or quasi-similar) in a given market!

	There is also a recurring issue in the field of insurance, named "\NewTerm{moral hazard}\index{moral hazard}", which is based on the observation that those who insure tend to be less cautious than those who do not insure themselves. In other words, insurance generates risk and this is very important factor that explain cost increase of solidarity insurance systems!
	
	\pagebreak
	\subsubsection{Taking account of the return on experience}
	So far, to determine an insurance premium, we have noticed that it was possible to integrate exogenous variables (gender, age, children, vehicle power, nationality, environment, etc.). But an important point not to be neglected is the return on experience of an insured's claims. Let's see a concrete example!

	Suppose that the number of claims over one year for a given insured follows a Poisson law (law of rare events) given for recall by (\SeeChapter{see section Statistics page \pageref{poisson distribution}}):
	
	which is written in the field of insurance:
	
	Let us suppose that the insured population is separated into three classes of risk according to a Poisson distribution such as:
	
	In other words, $70\%$ of the total population follows a Poisson's law of expected mean equal to $1$ (good risk class), $20\%$ follows Poisson's law of expected mean equal to $2$ (average risk class) , and $10\%$ follows a Poisson's lawofexpected mean equal to $3$ (class of bad risks).

	Obviously, the overall expected mean of an individual in the portfolio is the expected mean of the "\NewTerm{mixed distribution}\index{mixed distribution}":
	
	Let us suppose that the costs of an incident are fixed and of the indemnity type of $1,000$.-. We can then ask what the premium should be for an insured, knowing that the first year he had $2$ claims. What is equivalent to ask ourselves: what is the number of claims that he could have in the second year.

	If we denote by $X_1$ the number of claims of the first year, and $X_2$ the number of claims a posteriori knowing $X_1$ we thus find ourselves with a problem of conditional probability (a Bayesian approach in other words ...) according to what we have studied in the section of Probabilities:
	
	where we adopted in accordance with the tradition in the field of insurance the notation "$|$" to indicate we are dealing with conditional probabilities.

	But we can not calculate the numerator because we do not know in advance what the will be the value of $X_2$. We will therefore calculate the expected conditional mean in order to get around this problem:
	
	First we have:
	
	This last calculation (expected mean of the mixed distribution) being written in the field of insurance in the following way:
	
	If we consider the two random variables to be independent:
	
	It then comes immediately that:
	
	We therefore fall back on a known value corresponding to the pure premium:
	
	Identical to the calculation of:
	
	Therefore the random variables $X_1$, $X_2$ are confused with that of the class of risk!

	This system obviously is not conform to the bonus-malus concept. We must then consider that the two random variables are not independent (in fact it is the very definition of the bonus-malus!).

	Thus, if the two random variables are not independent, we have:
		
	and therefore:
	
	To compare with the pure premium without bonus-malus of $1.4$!
	
	What is tricky with this method is that when we accumulate years... this Bayesian approach becomes painful. Indeed, imagine that we would like to determine the pure premium of the third year knowing that of the second year, the insured had $1$ accident the second year. We then have the conditional expected mean:
		
	that we leave the development to the reader ...

	We notice from what has been seen above that the expected mean of a mixed distribution defined by:
	
	is therefore trivially given by:
	
	
	
	\subsubsection{Discounting factor of a retirement insurance}
	Let us now look for something that has absolutely nothing to do with what we have just calculated and which is the discount factor (\SeeChapter{see section Economy page \pageref{discount factor}}) by which we must divide a savings balance at the beginning of retirement, to allocate the resulting amount every year during retirement, depending onf life expectancy, as well as interest and inflation rates. The approach here will be to question ourselves what level of savings it will be necessary to have accumulated at the beginning of retirement to obtain a pension of $1$.- (real) each year during retirement. It is from this level of saving necessary to receive $1$.- that it will be possible to use any savings balance to derive what annuity will be available to the saver.

	First, we will define the parameters used:
	\begin{itemize}
		\item Number of retirement periods (years or months): $m$
		\item Savings balance at period $k$: $S_k$
		\item Periodic benefit (which we assume here annual): $p$
		\item Periodic rate of inflation (assumed here constant and annual): $i$
		\item Average geometric market rate of return (assumed constant and annual): $R$
		\item Discount factor at age $m$: $a_m$
	\end{itemize}
	Thus, as mentioned above, we are looking for the initial level of savings, denoted $S_0$, which will make possible to get a periodic benefit (therefore annual in this case), $p$, equal to $1$.- indexed to inflation $i$.

	Starting from the initial balance $S_0=S$, assuming that benefits are paid at the beginning of the year (praenumerando), we can derive the remaining balance at the end of the first retirement period as well:
	
	More precisely, the pensioner will have received an amount $p$ for the first period, leaving the residual to be exposed to the geometric mean return on market $R$.

	Then, in a similar way and considering that the benefits are indexed to inflation $i$, we can calculate this balance at the end of the second period and any subsequent period $k$:
	
	or:
	
	or:
	
	And so on ... It comes then:
	
	or otherwise written:
	
	Thus we have, to one given term, a geometric series of reason $q$ known in the last term. We have indeed proved in the section of Sequences and Series that for $j$ ranging from $0$ to $n$ this one is equal to:
	
	and therefore:
	
	It is therefore possible for us to rewrite, for the last retirement period $m$, the balance as follows:
	
	Let us notice that in the unlikely and particular case where the average geometric market return is equal to inflation, ie if $i=R$, we have:
	
	At this stage, the reason why the term $S_m$ is an important parameter is that it is the retiree's balance at the end of his retirement period, that is, at his death. It will be assumed here that the retiree does not plan a legacy and therefore:
	
	Consequently, for $R\neq i$, this is equivalent to find $S$ for which:
	
	We can then establish in this case that for $R\neq i$:
	
	And in the unlikely and particular case where the average geometric return of the market is equal to inflation, that is, if $R=i$, we have:
	
	and we deduce in this special case a logical result:
	
	For what follows, we come back to the realistic case where $r\neq i$ and thus with:
	
	and let us put:
	
	The prior-previous relation is then written in a somewhat more condensed form:
	
	where:
	
	is therefore the discounting factor. In addition, we have the following relation giving the expected mean of being alive at age $a$ (age corresponding to the assumed age of death in the present case) introduced and proved in the section of Populations Dynamics:
	
	with for recall the function $e(a, s)$ representing the conditional life expectancy at age $a$ and for sex $s$.
	
	\subsubsection{Pension insurance (life annuities)}
	We will now study a rent that is not paid in a certain way but in a "life annuity" way, that is to say, which depends on the survival of the insured. It will therefore be necessary to multiply each term of the rent by the probability that it will be paid to the insured.

	Obviously, in the majority of cases, what will interest us will be the "present value" of the life annuities in order to know what the insurer will have to pay (and thus what it must have capitalized).
	
	We consider mainly the following type of annuity:
	\begin{enumerate}
		\item Life annuity: You will receive regular income as long as you live. You can opt for monthly, quarterly, half-yearly or annual payouts depending on the product.

		\item Joint life annuity: You will receive regular income as long as you live. After you, your life partner will continue getting regular income

		\item Life annuity with return of purchase price: You will receive regular income as long as you live. After you, your children will get back the initial lump sum amount paid by you

		\item Life annuity guaranteed for $5$/$10$/$15$ years: You will receive regular income for a guaranteed period ($5$/$10$/$15$ years). In case of your demise within this period, your nominee will receive regular income till the end of the guaranteed period.

		\item Joint life annuity with return of purchase price: You will receive regular income as long as you live. After you, your life partner will continue getting regular income. After both of you, your children will get the initial lump sum amount paid by you.

		\item Immediate annuity with regular payout: Your income flow will start immediately upon investing, followed by payouts at regular intervals selected by you.

		\item Deffered annuity: You will receive regular periodic payments starting at a future date selected by you.
	\end{enumerate}
	 We will see what some of them! But before that, let us recall that we have seen in the section of Populations Dynamics that the probability of being alive at a year $x + 1$ knowing that one is alive at an age $x$ is given by:
	
	and let us recall (following a reader's request) that the present value of a capital $C_0$ at constant rate payable in $1$ year is given by (\SeeChapter{see section Economy page \pageref{compound interest}}):
	
	that gives the sum to be placed at a rate $t\%$ to have after one year the capital $C_1$.

	By combining probability of survival with current value, we can define in the case of an annuity the unique premium (UP) of the old-age pension as:
	
	If the insured is alive at maturity, the insured wins the game (in reality they often win both ...), otherwise it is the insurer (where in that latter case he is really winning a lot of money!).
	
	Obviously in the above case, extremely simplified, there is only one payment from the insurer. But in reality there may be several (this is what we are going to study), the payment of the insured may not be in the form of a single premium, the average geometric market rate may not be constant (the latter being treated normally with Monte Carlo simulations).

	So even if what we are going to see is very simplified compared to what is done in insurance by the actuaries, it already gives a search path for more elaborate models.
	
	\paragraph{Temporary life annuity}\mbox{}\\\\
	The "\NewTerm{temporary life annuity}\index{temporary life annuity}" consist for the insurer to pay annuity terms, as long as the insured is alive but not more than $n$ years after he has reached a given age $x$ but immediately after the insured has paid his single premium (which corresponds precisely to the age $x$ in question). The objective is therefore to determine the capital (current value) that the insurer must have accumulated at the time of the payment of these pensions (this capital may also be claimed in the form of multiple payments rather than in the form of a unique premium).

	Obviously this case is quite unrealistic because few people will pay a single premium (big sum of money) at a time $x$ of their lives to start right away to receive annuities. But it is pedagogically interesting before discussing the type of deferred  life annuity and which is a little more realistic.

	The temporary praenumerando  life annuity (ie the present value of the unit life annuity payable praenumerando as long as the insured is alive but at most for $n$ years) is given by the following relation which follows from the developments made in the section Economy (see page \pageref{praenumerando annuities}) with the only small difference that one finds there coefficients of probability of survival:
	
	where for recall (\SeeChapter{see section Economy page \pageref{discount factor}}) we have the discount factor (in this case we should instead use the term "capitalization factor") which is:
	
	In general, the term (\SeeChapter{see section Population Dynamics page \pageref{life probability}}):
	
	is named "\NewTerm{probable value of deferred capital}".
	
	The postnumerando  temporary life annuity (present value of a unit life annuity payable postnumerando as long as the insured is alive but for maximum during $n$ years) is given by:
	
	\begin{tcolorbox}[colframe=black,colback=white,sharp corners]
	\textbf{{\Large \ding{45}}Example:}\\\\
	We wish to pay an annuity of $1,000$.- during $3$ years to a $40$ year old man. What is the single premium that he will have to pay if neither interest nor mortality are involved in the calculations is given by:
	
	if only the interest is considered in the calculations:
	
	and finally taking into account interest and mortality
	
	\end{tcolorbox}
	
		\pagebreak
	\paragraph{Deferred life annuity }\mbox{}\\\\
	The "\NewTerm{deferred life annuities}\index{deferred life annuities}" account for the majority of annuity insurance contracts. The insured person always pays a single premium in order to receive an annuity if he is alive at a certain age (at the age of retirement for example), so he begin receiving the annuity $k$ years after paying the single premium, but this time until his death.

	The praenumerando deferred life annuity (present value of a unit life annuity deferred of $k$ years and payable praenumerando until death of the insured) is given by:
	
	where $\omega$ is the last value (line number) of the life table. The reader may notice that if we set $k = 1$ (without delay) and $\omega=x+n$, we fall back on a temporary postnumerando life annuity $a_{\lcroof{x:n}}$.

	Thus, normally for a same age $x$ and a same supposed annuity duration, the present value of the deferred life annuity (single premium) is lower than that of the temporary life annuity and this the smaller is the delay.

	The postnumerando  deferred life annuity (current value of a deferred life annuity of $k$ years and payable postnumerando until death of the insured) is given by:
	
	Obviously, the two annuities above have the same end because given that it is until death, it is difficult to pay in postnumerando of the death...

	Let us also notice that it is possible mathematically to create a temporary AND deferred annuity but it is rare and therefore we will not study it here.
	\begin{tcolorbox}[title=Remark,colframe=black,arc=10pt]
	Unlike certain annuities, as far as we know there is no simplified formula for a quick calculation of these pensions based on mortality tables. The use of a spreadsheet software or the programming of these functions makes it easy to calculate the current values of the lifetime benefits. At the time when computing were not existing, we built tables named "\NewTerm{commutations numbers}\index{commutations numbers}" that made it possible to obtain the required information.
	\end{tcolorbox}
	
	
	\pagebreak	
	\subsection{Sensitivity Analysis}\label{sensitivity analysis}
	The mathematical models we have seen previously (and anyone in general) are defined by a series of equations, input variables and parameters aimed at characterizing some process under investigation. Some examples might be a climate model, an economic model, a project planning model, etc.. Increasingly, such models are highly complex, and as a result their input/output relationships may be poorly understood. In such cases, the model can be viewed as a black box, i.e. the output is an opaque function of its inputs.

	Quite often, some or all of the model inputs are subject to sources of uncertainty, including errors of measurement, absence of information and poor or partial understanding of the driving forces and mechanisms. This uncertainty imposes a limit on our confidence in the response or output of the model. Further, models may have to cope with the natural intrinsic variability of the system (aleatory), such as the occurrence of stochastic events.

	Good modeling practice requires that the modeler provides an evaluation of the confidence in the model. This requires, first, a quantification of the uncertainty in any model results ("\NewTerm{uncertainty analysis}) and second, an evaluation of how much each input is contributing to the output uncertainty. Sensitivity analysis addresses the second of these issues (although uncertainty analysis is usually a necessary precursor), performing the role of ordering by importance the strength and relevance of the inputs in determining the variation in the output.
	
	\begin{tcolorbox}[title=Remarks,colframe=black,arc=10pt]
	\textbf{R1.} Uncertainty analysis investigates the uncertainty of variables that are used in decision-making problems in which observations and models represent the knowledge base. In other words, uncertainty analysis aims to make a technical contribution to decision-making through the quantification of uncertainties in the relevant variables. Many high level like to say that a management model is never complete until it has not a quantification tool associated with it!\\
	
	\textbf{R2.} Sensitivity and Uncertainty analysis belongs both to the field of knowledge named "\NewTerm{uncertainty quantification} that is the science of quantitative characterization and reduction of uncertainties in both computational and real world applications.
	\end{tcolorbox}
	
	The process of recalculating outcomes under alternative assumptions to determine the impact of variable under sensitivity analysis can be useful for a range of purposes, including:
	\begin{itemize}
	\item Testing the robustness of the results of a model or system in the presence of uncertainty.
	
	\item Increased understanding of the relationships between input and output variables in a system or model.
	
	\item Identifying model inputs that cause significant uncertainty in the output and should therefore be the focus of attention if the robustness is to be increased (perhaps by further research).
	
	\item Searching for errors in the model (by encountering unexpected relationships between inputs and outputs).
	
	\item Model simplification – fixing model inputs that have no effect on the output, or identifying and removing redundant parts of the model structure.
	
	\item Enhancing communication from modelers to decision makers (e.g. by making recommendations more credible, understandable, compelling or persuasive).
	
	\item Finding regions in the space of input factors for which the model output is either maximum or minimum or meets some optimum criterion (see optimization and Monte Carlo filtering).
	
	\item In case of calibrating models with large number of parameters, a primary sensitivity test can ease the calibration stage by focusing on the sensitive parameters. Not knowing the sensitivity of parameters can result in time being uselessly spent on non-sensitive ones.
\end{itemize}

	Taking an example from economics, in any budgeting process there are always variables that are uncertain. Future tax rates, interest rates, inflation rates, headcount, operating expenses and other variables may not be known with great precision. Sensitivity analysis answers the question, "if these deviate from expectations, what will the effect be (on the business, model, system, or whatever is being analyzed), and which variables are causing the largest deviations?"

	Sensitivity analysis is a variation on scenario analysis (deterministic naive modelization) that is useful in pinpointing the areas where forecasting risk is especially severe in any field of corporate finance, project management, lean management, supply chain management. 
	
	There are a large number of approaches to performing a sensitivity analysis, many of which have been developed to address one or more of the constraints discussed above. They are also distinguished by the type of sensitivity measure, be it based on (for example) variance decompositions, partial derivatives or elementary effects. In general, however, most procedures adhere to the following outline:
	\begin{enumerate}
		\item Quantify the uncertainty in each input (e.g. ranges, probability distributions). Note that this can be difficult and many methods exist to elicit uncertainty distributions from subjective data.
		
		\item Identify the model output to be analysed (the target of interest should ideally have a direct relation to the problem tackled by the model).
		
		\item Run the model a number of times using some design of experiments, dictated by the method of choice and the input uncertainty.
		
		\item Using the resulting model outputs, calculate the sensitivity measures of interest.
	\end{enumerate}
	In some cases this procedure will be repeated, for example in high-dimensional problems where the user has to screen out unimportant variables before performing a full sensitivity analysis.
	
	However, in the field of management (projects, finance, statistics) and multivariate case, the sensitivity analysis is at a graduate level simple technique based on the use of the linear Pearson correlation coefficient that we have study in the section Statistics and for which a possible writing form in the bivariate case was:
	
	Some common difficulties in sensitivity analysis include:
	\begin{enumerate}
		\item Too many model inputs to analyze. Screening can be used to reduce dimensionality.
		
		\item The model takes too long to run
		
		\item There a unsuspected nonlinearities and correlations
		
		\item There is not enough information to build probability distributions for the inputs. Probability distributions can be constructed from expert elicitation, although even then it may be hard to build distributions with great confidence. The subjectivity of the probability distributions or ranges will strongly affect the sensitivity analysis.
		
		\item Piecewise sensitivity. This is when one performs sensitivity analysis on one sub-model at a time. This approach is non conservative as it might overlook interactions among factors in different sub-models
	\end{enumerate}
	
	Let us begin with the theory of the first method that is strictly speaking not a sensitivity analysis but that many managers (non-scientific) consider as it...
	
	\subsubsection{Direct Bias Method}
	The direct bias method can also be named the "\NewTerm{one factor at a time method}" (OFAT).  In fact it is the most easiest method and in can be applied in a deterministic, probabilistic or stochastic way. It is quite useful when the board committee want to know the influence of a given factor only for a given amplitude!
	
	OFAT customarily involves:
	\begin{enumerate}
		\item Chaning one input variable, keeping others at their baseline (nominal) values, monitoring the output then,
		
		\item Returning the variable to its nominal value, the repeating for each of the other inputs in the same way
	\end{enumerate}
	
	Consider that we have an output model given by:
	
	where $\vec{x}$ is the vector of all the variables $x_i$ components  that can be time dependent or not.
	
	\begin{enumerate}
		\item The "\NewTerm{deterministic univariate direct bias method} consists simply by doing the following relative calculation:
		
		where the variation $\delta_{i,t}$ is chosen arbitrarily by the practitioner that must keep in mind that the output is not necessarily proportional to the input variation!!!!

		The ratio above is a "relative-output/nominal output" but some practitioners prefers a "relative/relative" analysis. Thus if $\delta_i$ is express in percentage change of $x_i(t)$ we will write obviously:
		
		 Furthermore depending on the amplitude of $\delta_{i,t}$ some non-null correlation can appear with other factor could in reality occur! 
		
		This technique is quite limited as you must do as many calculations as you have scenarios (variable inputs ranges).
		
		When the practitioner use this punctual naive estimation method we often speak of "\NewTerm{univariate Scenario Analysis}" or "\NewTerm{univariate What-If Analysis}" and in the special case where only three punctual estimates are used such as the optimistic, expected, pessimistic values:
		
		 we then speak of "\NewTerm{univariate three point estimate analysis}".
		  
		 So in this case sensitivity analysis is done using a large number of probabilistic What-If analysis or Three point estimate analysis. Sadly spreadsheet software like Microsoft Excel have made this term and method popular and most people thinks that this tool is the Graal... In fact it is just a high-school level analysis tool that should never be use in a firm if the CEO or CFO are graduate from a good university otherwise you will have quite a lot of problems to keep your job since the results are very poorly predictive and poorly predictive!
		 
		 Such an analysis is very easy to implement in any spreadsheet software like Microsoft Excel with multiple input values. As this is more a high school subject we will not give any example about this.
		 \begin{figure}[H]
			\centering
			\includegraphics{img/economy/what_if_scenario_manager.jpg}	
			\caption[]{What-If Scenario Manager tool in Microsoft Excel 14.0.7166}
		\end{figure}
		Microsoft never implemented until now an equivalent tool for multivariate case (with more than two vector) even if their spreadsheet has the capacity to do it with a little bit more formatting (typically a structure near that one of the PivotTables).
		
		 If we apply the same procedure as above but with to variables:
		 
		 The total number of combinations we can do is obviously:
		 
		 Why speak about this special case??? Because as it can be visualized as a double entry table, many spreadsheet software offer a "Data Table" tool to make "deterministic bivariate direct bias method". Indeed, a spreadsheet software like Microsoft Excel provide us a tool in this purpose (for more details, see a Microsoft Excel book as this is only basic stuff):
		\begin{figure}[H]
			\begin{center}
				\includegraphics{img/economy/exce_data_table.jpg}
			\end{center}	
			\caption[]{Data Table tool in Microsoft Excel 14.0.7166}
		\end{figure}
		And example based on the minimal order quantity of Wilson's model gives with a little bit formatting:
		\begin{figure}[H]
			\begin{center}
				\includegraphics[scale=0.8]{img/economy/data_table_excel_example.jpg}
			\end{center}	
			\caption[]{Data Table tool illustration with Microsoft Excel 14.0.7166}
		\end{figure}
		 
		 \item The "\NewTerm{univariate probabilistic direct bias method} consist simple to admit that $x_i(t)$ is not punctual but fluctuate and follow a density probabilistic function $f_i(t)$:
		 
		 and that:
		 
		 Using the property we saw in the section of Numerical Methods, we have:
		  
		 Therefore the idea is not the measure $\Delta y(t)/y(t)$ anymore but based on a simulation of $n$ values:
		 
		If once again the practitioner prefer a "relative input/relative output" then we take:
		
		and then the analysis becomes:
		
		The information we get therefore is not the same as previously. Before we had a relative variation for the output for a given variation of the input. Now we have a probabilistic distribution of all possible outcomes. These are two distinct things that are complementary!!!
		
		\item The "\NewTerm{probabilistic or deterministic univariate forecast bias method} consists simply by doing the same both calculations as before but at any given time $t$. Therefore the practitioner must make usage of the forecasting techniques we saw in the chapter of Economy. 
		
		For example the most common usage is a "nominal input/nominal output" with probabilistic bias and therefore we get a stochastic model such that:
		
		
	\end{enumerate}
	
	\pagebreak
	\subsubsection{Correlation Method}
	The correlation method, involves fitting a regression to the model response  and using the regression coefficient as direct measures of sensitivity. This method is consider by many consultants as the real "sensitivity analysis".
	
	Let us first recall that the correlation coefficient is given as (\SeeChapter{see section Statistics page \pageref{linear correlation coefficient}}):
	
	and a perfectly equivalent notation already proved in the section of Numerical Methods:
	
	with for recall:
	
	The idea is that if we have a main variable that depends on multiple dependent variables then we calculate the correlation coefficient (or the coefficient of determination which is simply the square of the correlation coefficient) of the main variable with each of the probabilistic simulations o the underlying variables.

	Let us take a simple example that applies to the world of project management (as it is the expertise domain of the author of these lines). Imagine a project schedule that is made only of a critical path (\SeeChapter{see section Graph Theory page \pageref{critical path}}) based solely only on two tasks $A$ and $B$. Imagine that first task $A$ time execution distribution is the best fitted with a non-truncated Normal law (yes it's stupid but its only for the example!!!) such as:
	
	that is to say an expected duration of $8$ days and a standard deviation of $1$ day and the task $B$ is supposed to be spread over a period following a beta distribution (according to the recommendation of PMI and the theory seen previously...) of parameters (5.12). Thus an optimistic duration $5$ days and a pessimistic duration of $12$ days:
	
	Let us do for an application example with a Monte Carlo simulation (\SeeChapter{see section Numerical Methods page \pageref{monte carlo simulations}}) of $1,000$ iterations with the English version of Microsoft Excel 11.8346 (even if it is most more efficient to use a software like Palisade @Risk):
	\begin{figure}[H]
		\begin{center}
		\includegraphics[]{img/economy/tasks_monte_carlo_simulation_excel.jpg}
		\end{center}	
		\caption{Simulation of two small tasks respectively with a Normal and Beta distribution}
	\end{figure}
	Then we calculate always with the English version of Microsoft Excel 11.8346 the correlation coefficient:
	\begin{figure}[H]
		\begin{center}
		\includegraphics[]{img/economy/correlation_tasks_excel.jpg}
		\end{center}	
		\caption{Correlation coefficient between the sums and each task}
	\end{figure}
	Which gives:
	\begin{figure}[H]
		\begin{center}
		\includegraphics[]{img/economy/correlation_tasks_excel_values.jpg}
		\end{center}	
		\caption{corresponding correlation values}
	\end{figure}
	Either as it is customary in the field of sensitivity analysis i the form of a "\NewTerm{Tornado chart}":
	\begin{figure}[H]
		\begin{center}
		\includegraphics[]{img/economy/sensitivity_correlation_excel_tasks.jpg}
		\end{center}	
		\caption[]{Tornado chart build with Microsoft Excel 11.8346}
	\end{figure}
	It is dangerous to use the  (Pearson) correlation coefficient when the relationship between the endogenous variable and the exogenous variable is not linear! However, modern software can identify almost any other type of relation (exponential, logarithmic, power, polynomial) and make the necessaries transformation to bring the equation to that of the straight line (\SeeChapter{see section Numerical Methods page \pageref{simple linear regression}}). However, this manipulation can also lead to major errors in some cases. The majority of specialized softwares then use the empirical definition of the regression we saw in the section of Numerical Methods:
	
and then use optimization techniques (\SeeChapter{see section Numerical Methods page \pageref{operational research}}) to determine the parameters of the that minimize the $R^2$ according to previous relation.

	A specialized software that the author of these lines like to use and to teach to quickly get this kind of analysis on complex situations is @RISK from Palisade that gives the following graph:
	\begin{figure}[H]
		\begin{center}
		\includegraphics{img/economy/sensitivity_correlation_palisade_risk_tasks.jpg}
		\end{center}	
		\caption[]{Tornado chart of the simulation obtained with @RISK}
	\end{figure}
	Finally, when we have a large number of underlying variables, it may be interesting to group the interval correlation coefficients with the following type of graph (example taken from the stock exchange) where the ordinate represents the $\%$ of the number of total input variables concerned (as far as I know @Risk sadly don't do such chart automatically):
	\begin{figure}[H]
		\begin{center}
		\includegraphics{img/economy/sensitivity_correlation_grouping_variables.jpg}
		\end{center}	
		\caption[]{Chart for correlation group analysis  with MS Excel 11.8346}
	\end{figure}
	Thus for this special above example, $32\%$ of the endogenous variables have a zero correlation coefficient with the exogenous variable (which is the performance index of the relevant portfolio) and $14\%$ of the endogenous variables have a correlation coefficient between $+0.75$ and $+1$ .
	
	It is also common to represent the average influence in $\%$ of each endogenous variable on the exogenous variable by averaging (or taking the median as more robust indicator) the influence on each other. This gives the following chart if we have a large number of endogenous variable:
	\begin{figure}[H]
		\begin{center}
		\includegraphics{img/economy/sensitivity_correlation_group_excel.jpg}
		\end{center}	
		\caption[]{Chart for correlation group analysis  with MS Excel 11.8346}
	\end{figure}
	For example, we can see in the graph above a that an endogenous variable has an average impact of $+4\%$ on the portfolio index with correlation of $+0.5$.

	Anyway ... the practitioners are free to do as many graph types they want depending on their business needs!
	
	
	\begin{flushright}
	\begin{tabular}{l c}
	\circled{85} & \pbox{20cm}{\score{3}{5} \\ {\tiny 28 votes,  65.71\%}} 
	\end{tabular} 
	\end{flushright}

	%to force start on odd page
	\newpage
	\thispagestyle{empty}
	\mbox{}
	\section{Music Mathematics (physics of hearing)}\index{acoustic}\index{music maths}\label{acoustic}
	\lettrine[lines=4]{\color{BrickRed}M}usic theorists sometimes use mathematics to understand music, and although music has no axiomatic foundation in modern mathematics, mathematics is "the basis of sound" and sound itself "in its musical aspects... exhibits a remarkable array of number properties", simply because nature itself "is amazingly mathematical".
	
	The theory of music encompass all theoretical aspects of a particular music system. There are, not one, but an infinity of musical theory, each type of music having its own. All musical system is indeed based on a number of usage, more or less constraint, that could be encompass into a theory, orally or written.
	
	A theory of music has often a religious, philosophical or magic basis and sometimes and arithmetic or scientific starting point (acoustic). It is on this latter that we focus in this section of course...!
	
	We are going to begin, consider in this section the study of elastic waves in a gas, resulting from changes in pressure in the gas. The sound is the most important example of this type of waves that is part of our daily environment.
	
	\subsection{Longitudinal Sound Waves}
	In elastic medium: gases and liquids, longitudinal sound waves propagate in the following mechanism (for solids this are transverse waves that we have already study in the section of Continuum Mechanics): a middle layer moves in the direction of propagation of the wave (hence the name of "longitudinal") and compresses the next layer, which advance under the effect of pressure and compresses the next layer and so on. This also works for a layer going backwards: the pressure on the next layer decreases, which has the effect of reducing the next layer which in turn decreases the pressure on the next, etc.
	
	The speed at which the longitudinal waves will move depends as often on the characteristics of the environment. In general, it is lower in gases than in liquids and lower in the liquid than in solids. For example in the air $\sim 340\;[\text{ms}^{-1}]$, in water $\sim 1,500\;[\text{ms}^{-1}]$ and $\sim 5,000\;[\text{ms}^{-1}]$ for the transverse waves in the steel.
	
	As regards to the frequencies, there is almost no limit (in positive real values). We can generate sound waves at fractions of Hz (Hertz: oscillations per second) up to hundreds of MHz. By reference to the frequency range audible to humans, we name "\NewTerm{infrasound}" frequencies lower than $20\; [\text{Hz}]$ and "ultrasound" frequencies greater than $20\; [\text{kHz}]$.
	
	\begin{figure}[H]
		\begin{center}
			\includegraphics[scale=0.73]{img/economy/sound_frequency_instruments.jpg}
		\end{center}	
		\caption{Some instruments frequencies range}
	\end{figure}
	
	In general, the waves are generated by a source of limited dimensions and, from this source, are propagated in all directions. In isotropic mediums the wavefront of a disturbance is spherical. We will avoid introducing spherical coordinates limiting us to the study of parts of wavefronts enough away from the source and sufficiently small compared with the distance to the source so we can assimilate the piece of sphere to a flat surface. That is to say: in this section we will only deal with the longitudinal plane waves.
	
	There is also another important difference between the longitudinal elastic waves in a gas or liquid and the transverse elastic waves in a solid bar. The gas is highly compressible, and if pressure fluctuations are established in a gas, its density will undergo the same type of the pressure fluctuation.
	
	Let us consider now the waves propagating inside a cylindrical tube or pipe (horizontal along the $x$-axis) of section $S$. Let us denote by $P_0$ and $\rho_0$ the pressure and density of the gas in equilibrium. In these conditions of equilibrium, $P_0$ and $\rho_0$ are the same throughout the whole gas volume of the cylinder, that is to say independent of any coordinate.
	
	If the gas pressure is disturbed for example by one of the two opening of the cylinder, a volume element thereof $S\mathrm{d}x$ will intuitively move because the pressures $P$ and $P'$ on the two sides $S, S'$ of this little volume will be different and therefore produce a resultant force (and therefore a power surface density)\label{basic frequency spectrum human ear}.
	\begin{figure}[H]
		\centering
		\includegraphics{img/economy/frequency_spectrum.jpg}
		\caption{Basic frequency spectrum with transported surface power  density}
	\end{figure}
	
	\begin{tcolorbox}[title=Remark,colframe=black,arc=10pt]
	Even if they have a very high speed, in a gas, molecules undergo very frequent collisions with each other. They travel in fact less than one micron in average (mean free path), under normal conditions, before hitting another one (see section Continuum Mechanics page \pageref{mean free path} for the proof).
	\end{tcolorbox}
	This results in a displacement of the section $S$ of a quantity $\xi$ and of the section $S'$ in a an amount $\xi'$ necessarily different because pressure balance has not had time to be done.
	
	Thus, the volume element at the start has a width $\mathrm{d}x$ but after the pressure has changed, it will have a width (if the pressure changes are small) as a first approximation equal to:
	
	However, due to the change in volume, there is also now a density variation due to the compressibility of gas. The mass contained in the volume is initially undisturbed:
	
	If $\rho$ is the density of the disturbed gas, the mass of the perturbed volume is in the end:
	
	The conservation of matter claim that these two masses are equal, that is to say:
	
	where:
	
	By solving relatively to $\rho$ we get:
	
	As we consider only the small changes in pressure relatively to the ambient pressure, $\mathrm{d}\xi/\mathrm{d}x$ is small, the we can replace:
	
	by its limited Taylor series development (\SeeChapter{see section Sequences and Series page \pageref{usual maclaurin developments}}):
	
	Therefore:
	
	By now admitting that the pressure is only connected to the density (for simplicity) we can write:
	
	Using the general univariate form of the Taylor expansion (\SeeChapter{see section Sequences and Series page \pageref{usual maclaurin developments}}):
	Therefore:
	
	Then we have:
	
	The quantity:
	
	usually denoted:
	
	is named "\NewTerm{coefficient of compressibility}" or more technically "\NewTerm{coefficient of isothermal compressibility}".

	Let us recall now that we have proved in the section of Weather and Marine Engineering during our study of the adiabatic atmospheric model that starting from Laplace's law we get:
	
	where for recall $\gamma$ is the "\NewTerm{Laplace coefficient}\index{Laplace coefficient}", also named "\NewTerm{adiabatic coefficient}\index{adiabatic index}\index{adiabatic coefficient}" defined by (\SeeChapter{see section Thermodynamics page \pageref{adiabatic coefficient}}):
	
	Then we get (relation that we will use later):
	
	The isothermal compressibility coefficient will obviously be written in the general case:
	
	By the way, let us notice that by changing to the density to the volume, we get:
	
	often denoted $\chi$ in the field of Thermodynamics.

	Let us return to the two relations:
	
	and let inject them into the Taylor series above. Then we have:
	
	This expression links the pressure at any point of the gas to the deformation at the same point.

	We then need the equation of motion of the volume element. The mass of the element is $\rho_0 S \mathrm{d}x$ and its acceleration $\mathrm{d}\xi^2/\mathrm{d}t^2$.
	
	We have naturally in terms of force (the minus sign indicates that the force change the pressure is opposed to the initial pressure in the cylinder):
	
	Therefore:
	
	In this problem, there are two fields, the displacement field $\xi$ and the pressure field $P$. We can combine them as follows by taking:
	
	and differentiating with respect to $x$ and remembering that $P_0$ is independent of the position in the gas. Then we have:
	
	What we can combine with:
	
	We thus find a the final expression of the wave equation of (for recall) a Poisson's equation (more particularly it is a D'Alembert's equation):
	
	So we get a similar relation to that we get in the section of Wave Mechanics for Waves Mechanics or in the section of Electrodynamics in the wave propagation equation. We conclude that displacement due to pressure disturbance in a gas spreads to speed:
	
	where for recall the gas constant worth $8.314\;[\text{J}\cdot\text{K}^{-1}\cdot\text{mol}^{-1}]$.

	The relation:
	
	is sometimes Newton "\NewTerm{Newton-Laplace speed of sound relation}".
	\begin{tcolorbox}[colframe=black,colback=white,sharp corners]
	\textbf{{\Large \ding{45}}Example:}\\\\
	If we consider the air as an ideal diatomic gas then (\SeeChapter{see section Thermodynamics page \pageref{adiabatic index diatomic gas}}) we have $\gamma\cong 7/5=1.4...$ of molar mass $29\cdot 10^3\;[\text{kg}\cdot\text{mol}^{-1}]$ (weighted average molecular weights of $\mathrm{N}_2$ and $\mathrm{O}_2$). It comes then at a temperature of $300\;[\text{K}]$:
	
	This is in perfect agreement with the experience! By cons, when we say that a plane flies at Mach $2$ (Mach is the ratio between the speed of an airplane and that of sound as we will see further below during our study of wave shocks), we do not really know in reality the speed of sound (neither that of the plane) nor the temperature.
	\end{tcolorbox}
	But, we also have the following result (\SeeChapter{see section Continuum Mechanics page \pageref{bulk modulus}}):
	
	But, we have proved in the section of Mechanical Engineering that the Poisson coefficient $\eta$ respected:
	
	Taking the approximation that for a gas $\eta\cong 1/3$... then we have:
	
	and therefore:
	
	and therefore:
	
	The speed of sound is given by the same type of expression for fluids or solids! Thus, the propagation of a longitudinal deformation \label{propagation of a longitudinal deformation} in a solid is given by:
	
	and for the transverse strain (\SeeChapter{see section of Continuum Mechanics page \pageref{transverse wave in solids}}):
	
	We also have:
	
	Dividing these equalities member by member, we get:
	
	Thus after simplification:
	
	For laboratories usage, the sound wave model is more convenient than the previous one as we measure more easily pressure variations in a liquid or a gas, than the movements of molecules. It is interesting to know that all the analyzes we have made with the string wave equation (\SeeChapter{see section Wave Mechanics page \pageref{wave equation}}) or electromagnetic waves (\SeeChapter{see section Electrodynamics page \pageref{electromagnetic wave equation}}) can also be applied to the equation above (energetic aspects, particular solutions, decrease in amplitude in function on the distance from the source, etc.).

	The movement of the waves in gases is an adiabatic process (\SeeChapter{see section see Thermodynamics page \pageref{adiabatic system}}) so there is no exchange of energy in heat form by the elementary volume of gas.

	\subsubsection{Power carried by a sound wave}
	We have seen just earlier above that:
	
	Let us write the density variations caused by the sound wave by:
	
	We will do the calculation of the power carried by a wave only in the case of sine waves here. In this case $\xi$ will be of the form:
	
	As $\xi$ is known, we can calculate the corresponding pressure by using:
	
	By introducing the pressure variations due to the sound wave:
	
	and by using:
	
	It comes:
	
	Then we have:
	
	To calculate the power carried by a sound wave, we will calculate the work done, during a period, on a surface $S$ on a plane perpendicular to the $x$-axis and of coordinates $x$.

	The force apply on this surface will be:
	
	The work done when the surface moves of $\mathrm{d}\xi$ will be:
	
	As we are doing a calculation for the displacements $\xi$ of a layer whose equilibrium position $x$ does not vary, only the variable $t$ varies:
	
	Substituting we get:
	
	The word applied during one period $T=2\pi/\omega$ will be:
	
	We have already proved the expression of such an integral in the section of Differential and Integral Calculus:
	
	Hence:
	
	Replacing it comes:
	
	To calculate the power, we have to divide the work by the time during which it was made:
	
	and to calculate the power transmitted per unit area, we have to divide by the surface:
	
	If we replace $k$ by:
	
	We get:
	
	Before we continue, notice that there are several ways to measure the amplitude of a sound, and by extension, any signal of wave nature:
	\begin{itemize}
		\item The average amplitude (the arithmetic mean value of the positive signal)

		\item The effective amplitude (equivalent continuous amplitude in power)

		\item The peak amplitude (maximum positive)

		\item The peak to peak amplitude (the maximum difference of positive and negative amplitude)
	\end{itemize}
	\begin{figure}[H]
		\begin{center}
			\includegraphics{img/economy/various_type_of_amplitudes.jpg}
		\end{center}	
		\caption{Various type of amplitudes}
	\end{figure}
	In practice, the average amplitude of little interest and is not used to much. However, the effective value or "\NewTerm{Root Mean Square}", that is the square root of the mean square value of the signal is universally adopted to measure the value of AC voltages (it is like the variance in statistics in fact!), within the general framework than in acoustic. An amplifier which is given with the indication "$10$ watts RMS" will have approximately $14$ watts peak and approximately $28$ watts peak-to-peak (also denoted by "pp"). Measurements of power peak to peak are quite often named "\NewTerm{music watts}" by the audio-video retailers because the values are quite nice in terms of communication/marketing..

	The unit of measurement of the amplitude depends on the physical measured quantity:
	\begin{itemize}
		\item For  a vibrating string, it's a distance.

		\item For a sound wave is the air pressure, or movement of the diaphragm

		\item For the electromagnetic radiation, the amplitude corresponds to the electric field.

		\item For an electrical signal that corresponds to the maximum value.
	\end{itemize}
	For sinusoidal regimes (or more generally any periodic signal!), it is easy to prove that regardless of the field in physics the effective value (square root of the mean square of the signal) is the peak value divided by the square root of two. Indeed:
	
	Therefore:
	
	Therefore in RMS (this is only an arbitrary choice and nothing else!!!... so that you understand what will follow!!), the prior previous relations is:
	
	As we have the amplitude of the pressure which is equal to:
	
	and its RMS by:
	
	Finally, we have for the power (that we will represent later by a stylized capital $\mathcal{P}$ letter in order not to be confused with the pressure $P$):
	
	
	\subsubsection{Measuring the intensity of a sound}
	To characterize the sound, a unit of measurement was invented: the "\NewTerm{Bel}" and its submultiple the "\NewTerm{decibel}" which is $1/10$ of Bel. This acoustic intensity, assimilated to the "loudness" was defined from the effective sound pressure (the $P_\text{eff}$ given above) and of the sound pressure of a wave (at a specific frequency!) at the limit of the hearing threshold of a small elite of humanity (about $10\%$).

	This effective pressure reference is:
	
	The loudness of a wave whose RMS sound pressure is $P_\text{eff}$ is by definition and by convention equal to:
	
	and is often simply denoted $S$ in the literature to indicate that it is the "\NewTerm{physiological sensation}" of the human ear. How did we get to this relation? Well by observing on a panel of individuals that the physiological sensation of the ear was not linear but that the absolute variations of the sensation are proportional to relatives pressure variationss such that:
	
	Therefore after integration:
	
	Thus after integration:
	
	where by convention the minimum sensitivity is put as being equal to zero and where for historical reasons we take the decimal logarithms instead of Napierian. It comes then:
	
	In some books, we find the definition of the Bel from a reference intensity of (at $1$ [kHz]):
	
	It comes therefore (the definition is equivalent because it is a ratio  and that power is directly proportional to the square of pressure as we have proved earlier above):
	
	The decibel is then a tenth of Bel ... hence its name as already mentioned!
	
	The reason of the logarithmic choice is that the it has been observed that the auditory sensation of human ear is also logarithmic (this is quite a beautiful engineering of Nature): we have the same sensation of sound increase when it goes from $1$ to $10$ as when it moves from $10$ to $100$.
	
	To give an idea of the value of a decibel, here is a table of typical values (for a given frequency and a given reference pressure):
	\begin{table}[H]
	\begin{center}
		\definecolor{gris}{gray}{0.85}
		\begin{tabular}{|c|l|}
		\hline
		\multicolumn{1}{c}{\cellcolor{black!30}\textbf{Decibels}} & 
\multicolumn{1}{c}{\cellcolor{black!30}\textbf{Source/Effect}} \\ \hline
		$180$ & Rocket launch\\ \hline
		$160$ & (damage to the eardrum)\\ \hline
		$140$ & Shotgun blast (pain threshold) \\ \hline
		$130$ & Jet engine 30 meters away \\ \hline
		$120$ & Rock concert (threshold of discomfort) \\ \hline
		$110$ & Car horn, snowblower \\ \hline
		$100$ & Blow dryer, subway, helicopter, chainsaw ,heavy machinery \\ \hline
		$90$ & Motorcycle, lawn mower, convertible ride on highway\\ \hline
		$85$ & Trucks\\ \hline
		$80$ & Factory, noisy restaurant, vacuum, screaming child \\ \hline
		$75$ & Average factory  \\ \hline
		$70$ & car, alarm clock, city traffic \\ \hline
		$60$ & Conversation, dishwasher,\\ \hline
		$50$ & Moderate rainfall \\ \hline
		$40$ & Cinema audience noise/Refrigerator  \\ \hline
		$20$ & Broadcasting Studio/Watch ticking  \\ \hline
		$10$ & Anechoic room \\ \hline
		$0$ & Threshold of hearing\\ \hline
		\end{tabular}
	\end{center}
	\caption[]{Examples of typical dB values}
	\end{table}
	As we have already say it the minimum effective sound pressure which causes auditory sensation is:
	
	that is to say approximately $10^{-10}$ times smaller than atmospheric pressure. It must nevertheless be noticed that the frequency of this wave is around $3$ [kHz] where the sensitivity is maximum for the human ear in the 20th century.
	
	It is very interesting to calculate at which amplitude of effective displacement that corresponds. We then use the relation proved above:
	
	So for the limit case at $1$ [kHz], at the standard conditions for temperature and pressure, we will have a displacement of (values taken from the tables of the swiss-french mathematical commission):
	
	and the peak amplitude value is therefore:
	
	This value is less than the radius of atoms according to the Dalton model. Nature has given us a ear with an exquisite sensitivity!
	
	It is easy to calculate at which power value per unit area corresponds a sound wave at the limit of hearing. Using the relation proved above, we get:
	
	Since the surface of the section of the ear canal is less than one square centimeter, the power that comes to the eardrum is lower than $10^{-6}\;[\text{W}]$.
	
	\subsection{Spherical  Sound Waves}
	In reality the sound waves are generated by sources of finite extent. At lower or comparable distances to the extent of the source, the shape of the wavefront that moves away from the source can be very complicated (imagine the wavefront of a thunder caused by choppy Flash) . But far from the source, it is seen as a point object and the wave-front becomes more spherical as one moves away from the source.
	
	The difference is that the energy carried by the sound wave is distributed in an area which expands as $R^2$ (where $R$ is the distance to the source) as for the light (electrodynamics waves) emitted by stars. So the power per unit area should decrease as $1/R^2$ (exactly according to the same mathematical development than for electromagnetic waves seen in the section of Electrodynamics) and as the power per unit area is proportional to $\xi_\text{eff}^2$ or to $P_\text{eff}^2$ this implies that a spherical wave, both displacement $\xi_\text{eff}$, as the sound pressure $P_\text{eff}$ must decrease as $1/R$.
	\begin{figure}[H]
		\centering
		\includegraphics[scale=0.8]{img/electromagnetism/spherical_wave_front.jpg}
		\caption[Spherical wave front]{Spherical wave front (source: ?)}
	\end{figure}
	In the case of sine waves, we will have to modify the wave equation:
	
	by adding a coefficient $1/R$. If we denote by $r$ the distance between the source and the observation position, and that, instead of measuring the position with $x$ we do it with $r$ we get:
	
	and also for the sound pressure:
	
	
	For those who want more details, let us recall the in the section of Calculus with a prove the expression of the laplacian in spherical coordinates:
	
	Remember that we are looking for a description of spherical waves, waves that are spherically symmetrical (ie ones that do not depend on $\theta$ and $\phi$) so that:
	
	The laplacian then reduces then simply to:
	
	The differential wave equation that is for recall (\SeeChapter{see section Wave Mechanics page \pageref{three dimensions wave equation}}):
	
	will be written:
	
	But:

	Therefore we get the differential wave equation in spherical coordinates:
	
	Multiplying bot sides by $r$ (independent of $t$):
	
	It is immediate that a special obvious solution is of the form:
	
	and it is common to choose the "\NewTerm{harmonic spherical wave}\index{harmonic spherical wave}" form (don't forget that $kv=\omega$):
	
	or:
	
	Obviously most wave sources are spherical but far away from the source, the radius is such that the front wage con be considered as a plane.
	
	\subsection{Doppler effect}\label{acoustic doppler effect}
	In what has been studied so far, the source $S$ and the observer O of the wave have always been considered to be immobile relative to each other. The Doppler effect explains the wave frequency (sound, electromagnetic, fluid) changes observed when $S$ and O are in relative motion.
	
	The Doppler effect (or the Doppler shift) is the change in frequency or wavelength of a wave (or other periodic event) for an observer moving relative to its source. It is named after the Austrian physicist Christian Doppler, who proposed it in 1842 in Prague. It is commonly heard when a vehicle sounding a siren or horn approaches, passes, and recedes from an observer. Compared to the emitted frequency, the received frequency is higher during the approach, identical at the instant of passing by, and lower during the recession.
	
	When the source of the waves is moving towards the observer, each successive wave crest is emitted from a position closer to the observer than the previous wave. Therefore, each wave takes slightly less time to reach the observer than the previous wave. Hence, the time between the arrival of successive wave crests at the observer is reduced, causing an increase in the frequency. While they are traveling, the distance between successive wavefronts is reduced, so the waves "bunch together". Conversely, if the source of waves is moving away from the observer, each wave is emitted from a position farther from the observer than the previous wave, so the arrival time between successive waves is increased, reducing the frequency. The distance between successive wavefronts is then increased, so the waves "spread out".
For waves that propagate in a medium, such as sound waves, the velocity of the observer and of the source are relative to the medium in which the waves are transmitted. The total Doppler effect may therefore result from motion of the source, motion of the observer, or motion of the medium. Each of these effects is analyzed separately. For waves which do not require a medium, such as light or gravity in general relativity, only the relative difference in velocity between the observer and the source needs to be considered.

	\subsubsection{Fixed source-Moving observer}
	So consider the first case where the source is fixed and the observer in uniform rectilinear motion:
	\begin{figure}[H]
		\begin{center}
			\includegraphics{img/economy/doppler_fixed_source_moving_observer.jpg}
		\end{center}	
		\caption{Schematic diagram of a fixed source and an approaching observer (Doppler Effect)}
	\end{figure}
	Given:
	
	the frequency of the source $S$ and the $v_\text{phase}$ the propagation speed of the wave in the medium, assumed immobile.

	If O does not move, he sees the wavefronts arrive at speed $v_\text{phase}$ and there is not much to say. If O moves in the direction of $S$ at the speed $v_\text{obs}$, it receives the wavefronts at the speed:
	
	and crosses the distances $\lambda$ in equal times:
	
	During each interval $\Delta t$ perceives a full cycle of the wave. He concludes that $\Delta t$ is the period $T'$ of a wave of frequency $f'$:
	
	So finally:
		
	The sound becomes then more acute when approaching a source.
	
	If O go away from $S$ ar speed $v_\text{obs}$, it receives the wavefronts at the speed:
	
	and similar developments lead us then to:
	
	The limit of validity of this relation is for $v_\text{obs}\leq v_\text{phase}$. 
	
	The sound becomes more low-pitched sound when we move far from the source.
	\begin{tcolorbox}[title=Remark,colframe=black,arc=10pt]
	If the speed of the observer is oriented in any manner, relatively to the line $\overline{\text{O}S}$, only the radial component is to be taken into account.
	\end{tcolorbox}
	
	\subsubsection{Moving source-Fixed observer}
	It is not necessarily intuitive to see that the system is not symmetrical. But taking a particular case one quickly realizes that it is actually not. Indeed, if the receiver is moving from the source at a speed higher than $v_\text{phase}$, he will never receive wawefront, whereas if the source is moving aways from the observer, that latter will still receive a wavefront. Can not reverse the role of the transmitter (source) and receiver (observer).

	In the classical case, there is therefore an asymmetry in the frequency offset following that the source or observer is moving.

	Let us consider the following situation:
	\begin{figure}[H]
		\begin{center}
			\includegraphics{img/economy/doppler_fixed_observer_moving_source.jpg}
		\end{center}	
		\caption{Schematic diagram of a fixed observer and an approaching source (Doppler Effect)}
	\end{figure}
	Or in a more detailed form highlighting also the wave aspect of the phenomenon:
	\begin{figure}[H]
		\begin{center}
			\includegraphics[scale=0.4]{img/economy/doppler_fixed_observer_moving_source_detailed.jpg}
		\end{center}
	\end{figure}
	Given:
	
	the frequency of the source $S$ in uniform rectilinear motion and the $v_\text{phase}$ the speed of the wave propagation in the medium and O the assumed fixed observer.

	If the source moves at speed $v_\text{source}$, it travels a distance:
	
	between each cycle. The wavelength seen by the observer thus varies of
	
	For the observer, it decreases such that:
	
	Then we have:
	
	So after simplification:
	
	The limit of validity of this relations being for $v_\text{source}\leq v_\text{phase}$.

	Therefore it comes:
	
	The sound becomes more acute when the source approaches (that same feeling as when it is the observer who approaches).

	Similarly, when the source is moving away from the observer, we get immediately:
	
	The sound becomes more low-pitched when the source is moving away (so same sensation as when it is the observer who moves away).
	\begin{tcolorbox}[title=Remark,colframe=black,arc=10pt]
	In the case of electromagnetic waves, the wave velocity is the speed of light $c$ which does not depend on the repository. Then we must address the problem within the framework of Special Relativity and then we expect to find a perfectly symmetrical effect since it can not possible to distinguish between speed the velocity of the transmitter or that of the receiver, only must be taking into account the relative speed between the both (see the developments done in  the section of Special Relativity).
	\end{tcolorbox}
	
	\subsubsection{Moving source and observer}
	If both (source and observer) are in rectilinear motion on a common straight line, the two effects are additive. We then have when they approach each other:
	
	And when they move away:
	
	
	\subsection{Shock waves}
	Unlike ordinary sound waves, the speed of a shock wave varies with its amplitude. The speed of a shock wave is always greater than the speed of sound in the fluid and decreases as the amplitude of the wave decreases. When the shock wave speed equals the normal speed, the shock wave dies and is reduced to an ordinary sound wave.
	\begin{figure}[H]
		\begin{center}
			\includegraphics[scale=0.6]{img/economy/fight_jet_supersonic.jpg}
		\end{center}
		\caption{Jet fighter planes with conical shock waves made visible by condensation}
	\end{figure}
	The ratio of the speed $v$ of a moving object to the speed of sound $v_\text{sound}$ in a fluid is known as the "\NewTerm{Mach number}" in honor of Ernst Mach (1838–1916):
	
	The Mach number is therefore a dimensionless measure of speed common in aerodynamics. Mach $0.5$ is half the speed of sound, Mach $2$ is twice the speed of sound, and so on. Speeds less than the speed of sound have a Mach number between zero and one and are described as "\NewTerm{subsonic}". Those greater than the speed of sound have Mach numbers greater than one are a described as "\NewTerm{supersonic}". Speeds approximately equal to the speed of sound have Mach numbers approximately equal to one and are described as "\NewTerm{transonic}".
	
	The shock wave from a supersonic moving conical object is a cone composed of overlapping spherical wavefronts (if the moving object is cube at a far observation distance enough the shock wave will be also seen as a cone). As any one of these wavefronts forms, it propagates radially outward at speed $v_\text{sound}$ and acquires a radius $v_\text{sound}t$ relatively to the contact peak point. At the same time the source, traveling at speed $v$ moves forward $vt$. These two displacements form the leg and hypotenuse, respectively, of a right triangle and can be used to determine the Mach angle at the vertex of the shock cone (prolongation of the cone sides if the moving object is a cube...):
	
	\begin{figure}[H]
		\begin{center}
			\includegraphics[scale=0.75]{img/economy/fight_jet_supersonic_cone.jpg}
		\end{center}
		\caption[Jet fighter with conical shock waves in the case of a conical "nose"]{Jet fighter with conical shock waves in the case of a conical "nose" (source: Wikipedia, author: Ignacio Icke)}
	\end{figure}
	When an object travels slower than sound, the ratio in this equation is greater than one, and the equation does not have a real solution. This makes absolute sense as there is no shock wave to speak of at subsonic speeds. Traveling at the speed of sound makes the ratio equal one and results in a Mach angle of ninety degrees. At transonic speeds the shock wave is a wall of high pressure moving with the object, perpendicular to its velocity. Above the speed of sound, the ratio is less than one and the Mach angle is less than ninety degrees. The faster the object moves, the narrower the cone of high pressure behind it becomes. Measuring the vertex angle is thus a simple way to determine the speed of a supersonic object.
	
	\pagebreak
	\subsection{Music Scales}
	When we do vibrate a guitar string, violin string or other stringed instrument whose ends are fixed (stationary wave), we have proved in the section of Wave Mechanics that we are in conditions name "Dirichlet boundaries".

	We obtained after many calculations, the following result:
	
	Therefore:
	
	thus the frequency (which is one of the factors of the feel of the "\NewTerm{height}" of a sound by humans) is inversely proportional to the length $L$ of a string attached to the ends (as it would be deducted by Pythagoras long time before that theory of string was established).

	The previous relationship highlights that: more the vibrant part is long (more $n$ is small) the more the sound is low (low frequency) and the vibrant part is short (large $n$), the more the sound will be acute (high frequency).

	Let us recall that for $n=1$, we speak of "fundamental frequency" and for $n\geq 2$ of "harmonic frequency." Thus, $n=2$ is the first harmonic, $n=3$ the second harmonic, etc.
	
	\textbf{Definitions (\#\mydef):} 
	\begin{enumerate}
		\item[D1.] We say that two frequencies (or two periodic waves) are "\NewTerm{in harmony}" with each other if the ratio of their frequencies is equal to unity:
		

		\item[D2.] We say that two frequencies (or two periodic waves) are "\NewTerm{at the octave}" to each other if the ratio of their frequencies is equal to two:
		
		Let us notice that the first harmonic ($n=1$) corresponds to the octave of the fundamental $f_0$:
		
	For example, some singers are capable of singing on $8$ octaves (that is to say a frequency ratio of $16$). Thus, in the case of a stringed instrument with Dirichlet conditions this is equivalent obviously to pluck the string in the middle (for guitars that match to press the rope on the fret of the 12th box with respect to the frequency of an open string to get one octave).
	
		The octave ratio sounds pretty good for the majority of humans.		

		\item[D3.] We say that two frequencies (or two periodic waves) are "\NewTerm{at the quint}" to each other if the ratio of their frequencies is equal to $3/2$:
		
		Except the octave, this interval is the simplest of all, and since always, in occidental music, it was considered as the consonant interval with excellence, so the more remarkable sound for ear. For this reason, the quint has played a key role in establishing music scales, the Pythagorean scale being even built exclusively on this particular interval.

		We notice that we can put at most an integer of two quints in one octave (beyond we exceed the value of an octave).
		
		When a stringed instrument has its empty vibrations that are in a ratio of frequencies corresponding to the quint, we say that the instrument is "\NewTerm{tuned in quint to quint}".

		\item[D4.] We say that two frequencies (or two periodic waves) are "\NewTerm{at the quart}" to each other if the ratio of their frequencies is equal to $4/3$:
		

		\item[D5.] The ratio of the frequencies of two sounds (or two periodic waves) distant from each other by a "\NewTerm{tone}" may be $9/8$. This is named the "\NewTerm{major tone}":
		
		
		\item[D6.] The "\NewTerm{minor ton}" has a ratio of frequencies of $10/9$:
		
	\end{enumerate}
	\begin{tcolorbox}[title=Remark,colframe=black,arc=10pt]
	 The octave, the quint, the quart and the tones are obviously named "interval ratio" in music. Ratios, rather than direct frequency measurements, allow musicians to work with relative pitch measurements applicable to many instruments in an intuitive manner, whereas one rarely has the frequencies of fixed pitched instruments memorized and rarely has the capabilities to measure the changes of adjustable pitch instruments.
	\end{tcolorbox}
	Obviously, if we want to make music, we have to choose a mark (reference). As the frequency range is infinite, there was chosen a frequency at which the current human ear is the most sensitive: $440$ [Hz] that is named the "\NewTerm{La}" and is denoted by (A) by the Anglo Saxon.
	
	In theory, the occidental students learn the empirical sequences of $7$ notes and whose writing origin is purely mnemonic (we find this result on electronic tuners with the Anglo-Saxon system using certain letters of the alphabet):
	\begin{figure}[H]
		\begin{center}
			\includegraphics[scale=1]{img/economy/music_scale_example.jpg}
		\end{center}
	\end{figure}
	where the horizontal lines of the symbolic solfegic above represent a quinte in the Pythagorean scale. 

	Between the first $Do$ and the second $Do$, we have then a range of one octave. Moreover, the number of notes within the octave interval is often insufficient to make good music and artists have added other notes by altering the frequencies of these $7$ notes, by "diesing" them (multiplication of the frequency by $25/24$) or by "bemolising" (multiplication by $1-25/24$).

	Note that there are a dozen ways to cut the solfegic octave and are almost all are still used in the early 21st century (they are all completely empirical so we will not go into details):
	\begin{itemize}
		\item The "natural scale" (also named "harmonic range")

		\item The "tempered scale"

		\item The "Pythagorean scale"

		\item The "chromatic scale"

		\item The "Zarlino scale" (also named "physicists scale")

		\item etc.
	\end{itemize}
	The method and the aforementioned learning sequence is a mathematical aberration (but as ti is music... it's an empirical art so deeply linked with non-sense).
	
	\pagebreak
	\subsection{Harmonic Oscillator}
	In classical mechanics, a harmonic oscillator is a system that, when displaced from its equilibrium position, experiences a restoring force proportional to the displacement.
	Let us consider the mechanical system of the figure below consisting of a spring of constant $k$ of a mass $m$ and a viscous damping constant $c$. This system is often equated with the mechanics of a speaker (or a microphone) where the air acts as a viscous damper and the magnetic field of the exciter magnet as source of the force or also to a car damper:
	\begin{figure}[H]
		\begin{center}
			\includegraphics[scale=1]{img/economy/harmonic_oscillator.jpg}
			\caption{Schematic diagram of the harmonic oscillator}
		\end{center}
	\end{figure}
	This model is also used widely in molecular chemistry and solid physics to study crystalline structure on to model the bonds between atoms.
	
	We have proved in the section of Mechanical Engineering that for an helical spring we had:
	
	and this is the expression of the force to  which the spring will resist to an external force of compression or tension. At equilibrium, we necessarily the sum of the forces that is zero:
	
	Therefore:
	
	We have proved in the section of Continuum Mechanics that Stokes' law in the case of a cylinder was given by:
	
	We will assume that our viscous damper is based on a similar law where the resistance force is proportional to the speed. At equilibrium, we necessarily have the sum of the forces is still zero:
	
	Therefore:
	
	We will simplify this expression:
	
	We are left with exactly the same differential equation to that of a series RLC circuit in  steady state (\SeeChapter{see section Electrical Engineering page \pageref{rlc circuit}}):
	
	Let us recall that if rearranged in such a way that we write it as:
	
	So we have in the same way, the critical viscosity:
	
	instead of the critical resistance of the RLC circuit:
	
	which will determine if the system is in critical regime:
	
	in aperiodic regime (supercritical)
	
	or in pseudo-periodic regim (damped oscillations) with:
	
	In the case of cars or sound speakers, it is often useful to arrange the system to be in pseudo-periodic regim. Then we find exactly the same results as the RLC circuit with damped oscillations since the differential equation is the same.

	Specialists from acoustic or mechanics then arrange the system to get the concept of "damping factor" of the RLC circuit:
	
	which will be written in the present case by identification of the terms:
		
	With the natural frequency:	
		
	which will be written in the present case:
	
	The damped oscillator loses energy over time. Let us calculate that loss. We have (\SeeChapter{see section of  Classical Mechanics page \pageref{principle of conservation of energy} and Mechanical Engineering page \label{potential energy harmonic oscillator}}):
	
	Therefore:
	
	Now, according to the equation of motion:
	
	It comes then:
	
	The loss of energy (or power loss) is proportional to the square of the instantaneous velocity!

	We have shown in the section of Electrical Engineering for the RLC circuit in regime of damped oscillations, we had:
	
	What becomes here:
	
	Therefore:
	
	
	\subsubsection{Damped oscillator}
	Let us still consider the same oscillator but this time excited  by a harmonic force (thus: forced regime) as:
	
	Again, this is exactly the same type of differential equation as for the RLC oscillator in forced regime studied in the section of Electrical Engineering for which we had:
	
	It follows the same results and plots and conclusions that the reader may find in section of Electrical Engineering.
	
	\begin{flushright}
	\begin{tabular}{l c}
	\circled{20} & \pbox{20cm}{\score{3}{5} \\ {\tiny 13 votes,  61.54\%}} 
	\end{tabular} 
	\end{flushright}
	
		
\chapter{Engineering}

	\textit{\textbf{Engineering is the set of practices consisting to apply the results of the exact sciences and basic research to practical industrial or daily problems.}} (Larousse)
	\minitoc
	\pagebreak
	\section{Marine \& Weather Engineering}\label{meteorology}
	\lettrine[lines=4]{\color{BrickRed}M}eteorology is the study of atmospheric phenomena such as clouds, weather depression and precipitation to understand how they form and evolve. It is a discipline that deals mainly with fluid mechanics applied to air but which makes use of various other branches of physics and chemistry. It therefore allows for weather forecasts based on mathematical models on (actually) relatively short and long term. It is also applied for the prediction of air quality, climate change and the study in several areas of human activity (construction, air traffic, etc.).
	
	Meteorology is linked to a large amount of variables that would be very difficult to make a list even not exhaustive... However on our planet Earth, an important factor not to be neglected is the one formed by ocean surfaces and their intrinsic dynamics which we will try to present through a summary mathematical study some of their properties. What is nice is that some results can also be applied to other Planets and sometimes very exotic one like Jupiter.
	
	Before beginning to the content of this section we also strongly recommend the reader to have a look of the section of Astronomy as the distance of the Earth (in general any planet...) to its main star (placed on the foci of the orbit) has an influence on the power of the radiation (\SeeChapter{see section Optical Geometry page \pageref{radiant flow}}) and is therefore directly related to the eccentricity $e$ of the orbit. Especially the inclination of the Earth rotation relatively to the ecliptic plan has a strong influence on the weather conditions as it related to the existence of the seasons!
	\begin{figure}[H]
		\centering
		\includegraphics[scale=0.6]{img/engineering/earth_inclination.jpg}
		\caption[Earth's inclination]{Earth's inclination (source: Wikipedia)}
	\end{figure}
	And the reader must be careful about the statement that at the periapsis people say it's winter because in fact... it depends where you live!!! Indeed, if you live in the northern hemisphere ti will be winter, but at the southern hemisphere it will be summer (people that have the money to travel during holidays know this fact very well) as illustrated in the figure below:
	\begin{figure}[H]
		\centering
		\includegraphics[scale=0.6]{img/engineering/earth_seasons.jpg}
		\caption[Seasons space point of view]{Seasons space point of view (source: NASA)}
	\end{figure}
	The reader must also consider that the main factors of surface climate are:
	\begin{itemize}
		\item Latitude
		\item Topography
		\item Temperature
		\item Cloud cover
		\item Water cover
		\item Snow/Ice cover
		\item Wind circulation
	\end{itemize}
	The combination of all these factors is well represented in the figure below:
	\begin{figure}[H]
		\centering
		\includegraphics{img/engineering/weather_factors.jpg}
		\caption{Annual mean temperatures in $2$ [m] height in the atmosphere from the NCEP data set over the period 1950-2000. Values in $^\circ$C.}
	\end{figure}
	This section is a general introduction to the basic to the field of technical applications of thermodynamics and fluid mechanics. It will allow the reader to become familiar with the language and some fundamental methods of calculations used by engineers in this branch. Of course, this study must be completed by practical laboratory and simulation work on computer.
	
	
	\subsection{Visual horizon}
	Let us consider first a little funny subject making often friendly debate during holidays or more seriously ... In some meteorological softwares, it is required during temperature and pressure measurement to manually enter the visual horizon... but this is difficult to determine when the weather is very good and we are in a height position (on a hill or small mountain).

	For this, consider the Earth as a perfect sphere with radius $R$ and a point of view of height $h$ relative to sea level which we will denote by $A$ (and without atmosphere to avoid optical effects...). The question is to know how far can be by maximum the point $C$ if it is by definition given by the tangent $\overline{AC}$ which is simply the horizon line:
	\begin{figure}[H]
		\begin{center}
			\includegraphics{img/engineering/visual_horizon_01.jpg}
			\caption{Visual Horizon configuration experience}
		\end{center}	
	\end{figure}	
	The reader can probably already observed that the study will mainly appeal to trigonometry concepts (see corresponding section of this book) and elementary geometry (also see corresponding sections).
	
	The angle $\widehat{\text{O}CA}$ is a right angle. Indeed, a straight tangent at a point on a circle is perpendicular to the radius at that point. The triangle $\widehat{\text{O}CA}$  is therefore rectangle on $C$.
	
	Therefore we have:
	
	But, we have $\overline{\text{O}B}=\overline{\text{O}C}=R$. Hence we deduce:
	
	The $\overline{AC}$ distance is the distance in a straight line between our viewpoint and the (floor of the) boat that we see on the horizon. The distance, however, that interest us here is $\overline{BC}$: it is the distance that we should travel at the altitude $0$ to reach the ground of the boat we see on the horizon.
	
	For what will follows we will put for the curvilinear distance: $\overline{BC}=d$.
	
	When the angle $\alpha$ varies from $0^{\circ}$ to $360^{\circ}$ (full circle), we describe the entire circumference of the Earth, that is to say $2\pi R$, since the Earth is assumed to be round.
	
	Using the rule of three:
	If an angle of $360^{\circ}$ corresponds to a distance of $2\pi R$ then an angle $\alpha$ in degrees correspond to a distance:
	
	But we have seen previously that:
	
	Hence finally:
	
	With $R\cong 6,378\;[\text{km}]$, we find ($h$ must be expressed in kilometers):
	
	In Microsoft Excel this is given by:
	\begin{center}
	\texttt{=111.32*DEGREES(ACOS(6378/(6378+h)))}
	\end{center}
	We have then in vacuum, in a landscape without obstacles... the following table:
	\begin{table}[H]
	\begin{center}
		\definecolor{gris}{gray}{0.85}
			\begin{tabular}{|c|c|}
				\hline
				\multicolumn{1}{c}{\cellcolor{black!30}\textbf{Altitude $h$ [m]}} & 
\multicolumn{1}{c}{\cellcolor{black!30}\textbf{Horizon distance $d$ [km]}} \\ \hline
		$1.5$ & $4.3$\\ \hline
		$5$ & $8$\\ \hline
		$10$ & $11.3$\\ \hline
		$50$ & $25.3$\\ \hline
		$100$ & $35.7$\\ \hline
		$200$ & $50.5$\\ \hline
		$400$ & $71.4$\\ \hline
		$600$ & $87.5$\\ \hline
		$800$ & $101$\\ \hline
		$1,000$ & $113$\\ \hline
		$2,000$ & $159.7$\\ \hline
		$3,000$ & $195.6$\\ \hline
		$4,000$ & $225.8$\\ \hline
		$5,000$ & $252.5$\\ \hline
		$10,000$ & $357$\\ \hline
	\end{tabular}
	\end{center}
	\caption{Visual horizon curvilinear distance in function of the altitude}
	\end{table}	
	\begin{tcolorbox}[title=Remark,colframe=black,arc=10pt]
	If we do not take into account atmospheric refraction, we find in the table above that would need to go to an altitude of the order of several kilometers to see beyond $200$ [km] away. Yet, without going very far, at the heights of Nice (Alpes-Maritimes in France), it is possible to observe the Cap Corse (island) which is about $220$ [km] from the continent !!! Atmospheric refraction plays then a significant role in this phenomenon.
	\end{tcolorbox}
	The question that arise is: given that we are at a height $h_0$ and we look far at the horizon:
	\begin{enumerate}
		\item What is the distance of the horizon (same as above but with a different approach use by people that argue that the Earth is flat and that the NASA lie...)?

		\item At what level under the horizon is an object at a distance $d$ of the observer (eye)?
	\end{enumerate}
	\begin{figure}[H]
		\begin{center}
			\includegraphics{img/engineering/visual_horizon_02.jpg}
			\caption{Visual Horizon configuration experience}
		\end{center}	
	\end{figure}
	The application of Pythagoras gives us immediately:
	
	and:
	
	So let us compare the value of $d_1$ just obtained above (straight distance) for $d$ obtained previously that is the curvilinear distance:
	\begin{table}[H]
	\begin{center}
		\definecolor{gris}{gray}{0.85}
			\begin{tabular}{|c|c|c|}
				\hline
				\multicolumn{1}{c}{\cellcolor{black!30}\textbf{Altitude $h$ [m]}} & 
\multicolumn{1}{c}{\cellcolor{black!30}\textbf{Horizon distance $d$ [km]}}  & 
\multicolumn{1}{c}{\cellcolor{black!30}\textbf{Horizon distance $d_1$ [km]}} \\ \hline
		$1.5$ & $4.374$ & $4.374$\\ \hline
		$5$ & $7.98$ & $7.98$ \\ \hline
		$10$ & $11.29$ & $11.29$ \\ \hline
		$50$ & $25.25$ & $25.25$ \\ \hline
		$100$ & $35.71$ & $35.71$ \\ \hline
		$200$ & $50.51$ & $50.50$ \\ \hline
		$400$ & $71.43$ & $71.43$ \\ \hline
		$600$ & $87.48$ & $87.48$\\ \hline
		$800$ & $101.01$ & $101.02$\\ \hline
		$1,000$ & $112.93$ & $112.94$ \\ \hline
		$2,000$ & $159.70$ & $159.73$\\ \hline
		$3,000$ & $195.58$ & $195.64$\\ \hline
		$4,000$ & $225.83$ & $225.92$ \\ \hline
		$5,000$ & $252.47$ & $252.59$\\ \hline
		$10,000$ & $356.93$ & $357.29$ \\ \hline
	\end{tabular}
	\end{center}
	\caption{Visual horizon curvilinear and straight distance in function of the altitude}
	\end{table}
	So the bias of the people that believe that the Earth (and all other planets) is flat ("flat Earthers") is that when they do the experiment to prove their arguments they forget the "perfectly spheric" assumption of that model. Indeed as we have seen in the section of Astronomy the Earth's shape is more like a irregular potatoes and is locally (at human scale, i.e. a few hundreds of kilometers sometimes) absolutely not spherical due to relief irregularities.
	\begin{figure}[H]
		\begin{center}
			\includegraphics[scale=0.9]{img/engineering/visual_horizon_03.jpg}
			\caption{Earth curvature visibility}
		\end{center}	
	\end{figure}
	Anyways an another argument is that we can prove with Isoperimetric Inequality that natural final shape for a cloud of dust is a sphere and otherwise that the acceleration of objects that fall all respect Newton's law in it's... spherical form! And also... if the Earth was flat... nobody so far as we know has seen it's border...
	
	Therefore the two relations to remember are:
	
	
	\pagebreak
	\subsection{Wind direction}
	We will now prove mathematically something quite intuitive: the winds are moving from high to low pressure (it's silly like that but we must still prove it!).

	We know (\SeeChapter{see section Continuum Mechanics page \pageref{pressure}}) that the pressure force exerted on a surface $S$ is normal to this surface and is equal in scalar form to $P\cdot S$.
	
	For an air parcel volume $\mathrm{d}V=\delta x\delta y\delta z$ the total pressure force in the $x$ direction is then:
	
	\begin{figure}[H]
		\begin{center}
		\includegraphics{img/engineering/wind_direction.jpg}
		\end{center}	
	\end{figure}
	In addition, we have (\SeeChapter{see section Differential and  Integral Calculus page \pageref{differential calculus}}):
	
	Therefore:
	
	The mass pressure force is then:
	We can do the same calculation following $y$. Finally, the massic horizontal pressure force is given by:
	
	Thus, the pressure force (massic or not) is opposite to the horizontal gradient. The transmission of information (of the force) is donce at the speed of sound for this equation (which explains the speed of air calls in your home or apartment and then force that can make slam doors or window).

It is therefore:

	\begin{itemize}
		\item  Directed from high to low pressure, perpendicular to the isobars

		\item Inversely proportional to the spacing of the isobars.
	\end{itemize}
	If we measure the values of the atmospheric pressure at different points of the globe and we connect between them the points of same pressure on a drawing, we get a series of curves, named "\NewTerm{isobars}\index{isobars}". The wind is directly determined by the atmospheric relief since it is a movement of air between the high to low pressure.

	The wind speed is then determined by the pressure gradient: that is, if the atmospheric pressure changes rapidly with distance, the wind blows strong, while it will be low in abarometric  "swamp" where this pressure remains almost unchanged on great distances. In summary, more the isobars are close together, more the wind will blow strong.

	The isobars are traditionally marked with a pitch of $5$ millibars on weather maps such as shown in the example below:
	\begin{figure}[H]
		\begin{center}
			\includegraphics{img/engineering/depression.jpg}
		\end{center}	
		\caption{Typical representation of isobars (on a depression)}
	\end{figure}
	Charts showing isobars are useful because they identify features such as anticyclones (areas of high pressure) and depressions (areas of low pressure).

	Areas of high and low pressure are caused by ascending and descending air. As air warms, it ascends leading to low pressure at the surface. As air cools, it descends leading to high pressure at the surface. This is illustrated in the diagram below.
	\begin{figure}[H]
		\begin{center}
			\includegraphics{img/engineering/pressure_sytem.jpg}
		\end{center}	
		\caption{High and low pressure systems}
	\end{figure}
	In general, low pressure leads to unsettled weather conditions and high pressure leads to settled weather conditions.

	In an anticyclone (high pressure) the winds tend to be light and blow in a clockwise direction (in the northern hemisphere as we will prove later below). Also the air is descending, which reduces the formation of cloud and leads to light winds and settled weather conditions.

	In a depression (low pressure), air is rising and blows in an anticlockwise direction around the low (in the northern hemisphere). As it rises and cools, water vapor condenses to form clouds and perhaps precipitation. This is why the weather in a depression is often unsettled - there are usually frontal systems associated with depressions.
	
	To close this topic about winds, let us notice that meteorologists empirically defined (that is fun for the general culture) a unit of wind measurement that is just a match between the wind force and the distance between two isobars in steps of $5$ by $5$ [mb]:
	\begin{table}[H]
		\begin{center}
		\definecolor{gris}{gray}{0.85}
		\begin{tabular}{|l||c|c|}
			\hline
			{\cellcolor{black!30}Distance between isobares [km]} & {\cellcolor{black!30}Unit [Beaufort]} & {\cellcolor{black!30}Speed $[\text{m}\cdot \text{s}^{-1}]$}  \\ \hline
			$600$ (light breeze) & $2$ & $1.6$-$3.3$\\ \hline
			$500$ (average breeze) & $4$ & $3.4$-$5.4$\\ \hline
			$400$ (cold breeze) & $5$ & $8$-$10.7$\\ \hline
			$300$ (strong wind) & $6$ & $10.8$-$13.8$\\ \hline
			$200$ (big wind) & $7$ & $13.9$-$17.1$\\ \hline
			$100$ (storm) & $9$ & $20.8$-$24.4$\\ \hline
		\end{tabular}
		\end{center}
		\caption{Distance between isobars and wind speed (Beaufort units)}
	\end{table}
	
	\subsection{Atmospheric Profile Models}
	An atmospheric profile model is a mathematical model constructed to predict the profile of temperature, pressure or other in the atmosphere, locally depending on the measured height supposing a steady state of the atmosphere (same as if we take out a cylinder piece of the atmosphere and freeze time to analyze some of the properties of the atmosphere inside depending on the height in the cylinder). 

	There a lot of models but let us see just two easy one:
	
	\subsubsection{Atmospheric Exponential Profile Model}
	Consider that the atmosphere is a perfect fluid in a uniform gravity field. Then from the following Bernoulli's theorem relation, proved in the section of Continuum Mechanics (static fluid):
	
	It comes then:
	
	Thus, to know the variation of pressure depending on height in the atmosphere or depth in the ocean, we have taken as hypothesis the "\NewTerm{hydrostatic equilibrium}\index{hydrostatic equilibrium}", that is the variation of pressure with height / depth is proportional to the gravity and density of the fluid.
	
	This is obviously not valid in the case of rapid movements of convection, like thunderstorms, but can easily be satisfied for slow movements (quasi-static) and at large scale: the "\NewTerm{synoptic scale}\index{synoptic scale}" ($2,000$-$20,000$ [km]).
	\begin{figure}[H]
		\begin{center}
			\includegraphics[scale=0.7]{img/engineering/meteoric_scale.jpg}
		\end{center}	
		\caption{Various meteorological scales}
	\end{figure}
	We will then combine this last relation with a state equation, such as that of the ideal gas at temperature $T$ and of density $\rho$ whose constituent particles have a mass $m$. So we have the ideal gas equation (\SeeChapter{see section Continuum Mechanics page \pageref{ideal gas law}}):
	
	with for recall $P$ that is the pressure expressed in Pascals, $V$ the volume in cubic meters, $R$ is the gas constant, $T$ is the temperature in Kelvin, $n$ the number of moles, $k$ is the Boltzmann constant, $\rho$ the particle density, $m$ the total mass of the particles.
	
	In the isothermal case (e.g. in the terrestrial stratosphere, above $10$ [km] altitude where the temperature is almost constant around $-55$ degrees Celsius), we found out easily by substitution that:
	
	Therefore, at a given pressure, the vertical pressure gradient is inversely proportional to temperature.

	Let us now consider the following relation:
	
	Using the exponential:
	
	The pressure therefore decreases exponentially with the altitude. $P_0$ being the pressure at ground level.

	Let us return to the relation:
	
	It can of course also be written as:
	
	which tells us that the distance $z$ between isobaric surfaces is directly proportional to temperature.

	We have also another common approach. Let us start again from the relation proved just earlier above, but for a mass $m$ of $1$ [kg]:
	
	and let us denote this relation as follow:
	
	Let us recall that (\SeeChapter{see section Differential and Integral Calculus page \pageref{usual derivatives}}):
	
	Therefore:	
	
	and now let assume that the temperature variation is linear in the atmosphere (which is not far from the truth for the first $10$ to $20$ km of the atmosphere such that:
	
	with $\lambda<0$ which is the temperature gradient in $[\text{K}\cdot\text{m}^{-1}]$.
	\begin{figure}[H]
		\begin{center}
			\includegraphics[scale=0.4]{img/engineering/weather_temperature_pression_altitude_profile.jpg}
		\end{center}	
		\caption[Typical profile of temperature and pressure on Earth]{Typical profile of temperature and pressure on Earth... late 20th century (source: ?)}
	\end{figure}
	Then we have:
	
	Therefore:
	
	Which give:
	
	After simplification:
	
	Hence:
	
	Which can written more aesthetically:
	
	A good practical example of application of this relation is the gliders and hang gliders. They expect the weather forecast that it communicates to them the height of the isotherm of zero degrees during its bulletins. They then deduce the temperature gradient per meter. For these athletes, good condition is to have a gradient of $1\;[^\circ \text{C}]$ by $100 $meters. It is therefore easy with the above relation to calculate the pressure at an altitude of $2,000$ meters and to derive the pressure gradient which allows them to use some the updrafts for their aerobatic practices.
	
	\subsubsection{Adiabatic Atmosphere Model}
	The adiabatic temperature gradient is, in the atmosphere, the air temperature change with altitude (ie the air temperature gradient), which only depends on the atmospheric pressure, that is to say:
	\begin{itemize}
		\item Without considerating heat exchange with the environment (other air masses, relief, ground, etc.)

		\item  Without considerating condensation (cloud formation) or precipitation.
	
		\item Without taking into account the Sun position and variations
	\end{itemize}
	This concept is of great importance in meteorology, as well as aviation and maritime navigation.

	We have prove in the section Thermodynamics the Laplace's thermodynamics law (satisfied under certain conditions!):
	
	with the Laplace coefficient:
	
	Therefore in massic form the Laplace's law becomes:
	
	We can take the logarithm:
	
	But by taking the logarithmic differential:
	
	We then have (relationship that we will reuse in the section of Music Mathematics)
	
	We could also have found this result directly from the relation proved in the section Thermodynamics:
	
	By taking also the logarithmic differential  of the ideal gas law where $n$ is, for recall, the number of moles (\SeeChapter{see section Continuum Mechanics page \pageref{ideal gas law}}):
	
	But in the massic form for one mole:
	
	where $M_m$ is the molar mass (\SeeChapter{see section Analytical Chemistry page \pageref{molar mass}}), we have:
		
	Therefore:
	
	We then get:
	
	Therefore:
	
	Let us use the relation proved earlier shown above:
	
	It comes then:
	
	So we have an atmosphere with constant and negative thermal gradient (temperature decreases linearly with altitude):
	
	The last form using the molar mass is more convenient because it allows to characterize the studied medium. Note that this is a model that looks to work well for an altitude of $0$ to $90$ [km] on Venus but much less well for the planet Earth.
	\begin{tcolorbox}[colframe=black,colback=white,sharp corners]
	\textbf{{\Large \ding{45}}Example:}\\\\
	We then have for example for Earth (thus knowing that the model is not well adapted):
	\begin{gather*}
		g\cong 9.81\;[\text{m}\cdot{s}^{-2}]\qquad R\cong 8.314\;[\text{J}\cdot \text{K}^{-1}\cdot \text{mol}^{-1}]
	\end{gather*}
	and the adiabatic coefficient for air is equal to $\gamma=1.4$, and its molecular weight:
	\begin{gather*}
		M_{m,\text{air}}\cong 28.96\cdot 10^{-3}\;[\text{kg}\cdot\text{mol}^{-1}]
	\end{gather*}
	Therefore:
	
	which corresponds to the current idea: $1$ degree per $100$ meters.
	\end{tcolorbox}
	
	\pagebreak
	\paragraph{Hypsometric equation}\mbox{}\\\\
	The hypsometric equation, also known as the thickness equation, relates an atmospheric pressure ratio to the equivalent thickness of an atmospheric layer under the assumptions of constant temperature and gravity. It is derived from the hydrostatic equation and the ideal gas law.
	
	So we have just prove for to hydrostatic equilibrium:
	
	
	We can integrate this relationship if we know $T$ as a function of $P$ or $z$. The direct measurement of $P$ in practice is easier (single altimeters are actually barometers if they don't use GPS).

	We can then separate the variables:
	
	By integrating between two levels $a$ and $b$:
	
	Since:
	
	Then to continue we use a trick. We will define the average temperature by the relation:
	
	Which then allows us to write:
	
	Therefore:
	
	or written differently:
	
	or after rearrangement and change the notation for the average temperature:
	
	Both boxed relations are each respectively named "\NewTerm{hypsometric equation}\index{hypsometric equation}" (from the Greek "hypso" for "height").
	
	\pagebreak
	\subsection{Planetary equilibrium temperature}
	The "\NewTerm{planetary equilibrium temperature}\index{planetary equilibrium temperature}" is a theoretical temperature that a planet would be at when considered simply as if it were a black body being heated only by its parent star. In this model, the presence or absence of an atmosphere (and therefore any greenhouse effect) is not considered, and one treats the theoretical black body temperature as if it came from an idealized surface of the planet.

	Other authors use different names for this concept, such as "\NewTerm{equivalent black-body temperature of a planet}\index{equivalent black-body temperature of a planet}", or the "\NewTerm{effective radiation emission temperature of the planet}\index{effective radiation emission temperature of the planet}".

	If the incident solar radiation ("\NewTerm{insolation}\index{insolation}") - or incident power radiatin -on the planet at its orbital distance from the Sun is $M_\otimes$, the amount of energy absorbed by the planet ($\mathrm{d}\Phi/\mathrm{d}S$) will depend on its reflection coefficient (albedo $\rho$) and and its cross-sectional area $S$ or average radius $R$ as:
	
	Note that the albedo would be zero for a blackbody as we have seen it in the section of Optical Geometry. However, in planetology, more useful results are obtained by accounting for a measured or assumed planetary albedo $\rho>0$.

	The infrared power radiated by the planet as thermal radiation will depend on its emissivity and its surface area, according to the Stefan–Boltzmann equation proved in the section of Thermodynamics:
	
	and as it is as power by unit surface we will multiply it by the above surface area of the planet such that:
	
	 For a spherical planet, the surface of a perfect spherical area is well know (\SeeChapter{see section Geometric Shapes page \pageref{surface of a sphere}})! Therefore:
	
	But a planet that does not absorb all incident radiation (sometimes known as a grey body) emits less total energy than a black body and is characterized by an emissivity, $\varepsilon< 1$, therefore (by definition emissivity + albedo = 1):
	
	The equilibrium temperature is then calculated by setting $P_\text{in}=P_\text{out}$. Thus:
	
	Therefore:
	
	It is interesting to note that the equilibrium temperature does not depend on the size of the planet, because both the incoming radiation and outgoing radiation depend on the area of the planet!
	
	Using a more astrophysical point of view if we don't know $M_\odot$, based on the luminosity $L_\odot$ (power) of the star and its distance $d$ to the planet we get as proved in the using the relation proved in the section of astrophysics:
	
	Therefore:
	
	
	Earth's overall, average albedo is about $\rho\cong 0.31$ and the emissivity of earth is considered as being about $\varepsilon\cong 0.96$ and $M_\odot\cong 1376$ [W$\cdot$m$^{-2}$].

	A numerical application for earth gives $T_\text{eq}=253.7$ [K] (that is $-19.5$ [$^\circ$C]).

	Based on this calculation, Earth's expected average global temperature is well below the freezing point of water!

	Earth's actual average global temperature is around $14$ [$^\circ$C]. Our planet is warmer than predicted with  a pretty big difference!

	Why is Earth's temperature so much warmer than our calculations predicted? Certain gases in the atmosphere trap some extra heat, warming our planet like a blanket. This extra warming is called the greenhouse effect. Without it, our planet would be a frozen ball of ice. Thanks to the natural greenhouse effect, Earth is comfortable place for life as we know it. However, too much of a good thing can cause problems. In recent decades, a rise in the amount of greenhouse gases has begun to warm Earth a bit too much.
	
	\begin{tcolorbox}[title=Remark,colframe=black,arc=10pt]
	For extrasolar planets the temperature of the star can be calculated from the color of the star using Planck's law (\SeeChapter{see section Thermodynamics page \pageref{planck law}}). The calculated temperature of the star can be used with the Hertzsprung–Russell diagram (\SeeChapter{see section Astrophysics page \pageref{hertzsprung russell diagram}}) to determine the absolute magnitude of the star, which can then be used with observational data to determine the distance to the star and finally the size of the star. Orbital simulations are used to determine what orbital parameters (including orbital distance) produce the observations seen by astronomers.[10] Astronomers use a hypothesized albedo and can then estimate the equilibrium temperature.
	\end{tcolorbox}
	
	\pagebreak
	\subsubsection{Greenhouse effect}
	The greenhouse effect is the process by which radiation from a planet's atmosphere warms the planet's surface to a temperature above what it would be without its atmosphere.

	If a planet's atmosphere contains radiatively active gases (i.e., greenhouse gases) the atmosphere will radiate energy in all directions. Part of this radiation is directed towards the surface, warming it. The downward component of this radiation – that is, the strength of the greenhouse effect – will depend on the atmosphere's temperature and on the amount of greenhouse gases that the atmosphere contains.

	Earth's natural greenhouse effect is critical to supporting life. Human activities, primarily the burning of fossil fuels and clearing of forests, have intensified the natural greenhouse effect, causing global warming.
	\begin{figure}[H]
		\begin{center}
			\includegraphics[scale=0.9]{img/engineering/greenhouse_effect.jpg}
		\end{center}	
		\caption{Greenhouse effect concept}
	\end{figure}
	
	For a mathematical analysis of the greenhouse effect we must improve our above model. Consider for this that we denoted $T_S$ the temperature of surface of the Earth (or any other planet) and $T_A$ the temperature of its atmosphere.	The thermodynamic equilibrium of atmosphere gives us in the point of view of the radiation using again the Boltzmann law and considering that Earth's atmosphere radiate $50\%$ back in space and reflect back $50\%$ back to earth surface:
	
	Therefore it is immediate that:
	
	Remember now the equilibrium proved earlier:
	
	That simplify to:
	
	Using our new notation:
	
	Therefore after rearranging:
	
	Now we add to this the back-radiation due to atmosphere:
	
	That is:
	
	Therefore after rearrangement:
	
	A numerical application gives $T_S\cong 297$ [K] that is $14$ [$^\circ$ C]. This is quite more accurate than the previous model!

	\subsubsection{Milankovitch cycles}\label{milankovic cycles}
	The Milankovitch cycles are periodic or quasiperiodic changes in the parameters of the Earth's orbit and tilt, which in turn affect the climate. The three major types of Milankovitch cycle are:
	\begin{itemize}
		\item changes in the eccentricity of the Earth's orbit
		\item changes in the obliquity, or tilt of the Earth's axis
		\item precession, meaning changes in the direction of the Earth's axis relative
to the fixed stars
	\end{itemize}
These changes do not affect the overall annual amount of solar radiation hitting the Earth, but they affect the strength of the seasons in different ways at different latitudes. It is widely believed that they are partially responsible for the glacial cycles. However, the details of how this occurs are complex and poorly understood.
	\begin{figure}[H]
		\begin{center}
			\includegraphics[scale=0.8]{img/engineering/milankovitch_cylces.jpg}
		\end{center}	
		\caption[Past and future Milankovitch cycles]{Past and future Milankovitch cycles (source: Wikipedia)}
	\end{figure}
	\begin{tcolorbox}[title=Remark,colframe=black,arc=10pt]
	By the mid-seventies the Earth’s orbital parameters were known over the last million years to a good accuracy thanks geological records. A decisive step was made by  André Berger (1978), who expressed in an analytical form the Fourier decomposition of the Earth's orbital parameters relevant for the astronomical theory of palaeoclimates. This work constitutes the first demonstration that the spectrum of climatic precession is dominated by periods of $19,000$, $22,000$ and $24,000$ years, that of obliquity, by a period $41,000$ years, and eccentricity has periods of $400,000$ years, $125,000$ years and $96,000$ years. 
	\end{tcolorbox}
	
	\subsection{Weather (sounding) balloon}
	A pretty interesting example of applied mathematics to weather engineering is the study of the famous weather balloons and especially the characteristic of their volume depending on the altitude which is often subject to debate in discussion groups when nobody formalizes the problem once and for all. You will therefore understand that this is what we will discuss here and especially we will try to determine the theoretical diameter thereof at a given altitude.
	\begin{figure}[H]
		\begin{center}
			\includegraphics[scale=0.8]{img/engineering/weather_balloons.jpg}
		\end{center}	
		\caption[]{Small weather balloon}
	\end{figure}
	The statement of the often discussed problem is the following:

	Given a PVC (Polyvinyl Chloride) weather balloon of mass $m$ used to take at high-altitude an apparatus for performing measurements. The envelope of the balloon contains $n$ moles of the hydrogen ideal gas thus having a molar mass:
	
	The atmosphere will be treated as a perfect gas, of average molecular weight:
	
	at N.T.P. (Normal conditions of Temperature and Pressure).

	We first we want to find what is the ascension force experienced by the balloon?

	Then we want to determine the minimum amount of material $n_0$ providing the take off of the balloon for a given total mass (including the balloon istself!) of $2.6$ [kg]. then the volume $V_0$ corresponding to the starting altitude.

	Let us recall two things first to solve this first point:
	\begin{enumerate}
		\item Any body immersed in a liquid (or gas) undergoes an upward force equal to the weight of the volume it displaces (Archimedes force) according to the relation proved in the section Continuum Mechanics:
		

		\item An ideal gas with a mass in grams equal to the molar mass occupies according to the ideal gas law a volume of $22.4$ [L] at $273.15$ [K] and at a pressure of $1$ [atm] as we have proved it in the section of Thermochemisty. Which gives at N.T.P:
		
	\end{enumerate}
	So for the balloon to fleet at constant height (without mounting but without falling too ...) with just sufficient quantity $n_0$ of hydrogen, it is necessary according to the Archimedes principle that the volume of air that it moves has a weight equal to the total mass of the ballloon and the probe, thus $2.6$ [kg] in our case!

	So since $22$ [L] air weigh about $29$ grams, it is necessary that the volume that it moves to be equal to $2.6$ [kg] of air. Either by a simple rule of three:
	
	So if the balloon is spherical, it gives a radius of:
	
	Thus a diameter of about $1.56$ [m] at the ground. This is consistent with reality!

	We still need to determine the number of moles of hydrogen. It comes immediately:
	
	Now that we know the number of moles in the balloon , if we know the temperature and pressure at a height of $22,000$ [m] (typical altitude of a small weather balloon) it only remains to apply ideal gas law to determine the volume at this altitude then given by the relation provedin the section Continuum Mechanics:
	
	and at $22,000$ [m] above sea level, we have following the numerical tables available on the Internet:
	
	But because solar radiation is about $30\%$ higher in this altitude and the balloon is considered as adiabatic system (without heat exchange) and does not restore the power stored in the external environment. We consider that the temperature is at least $30\%$ higher which gives us such for values:
	
	
	We could also use (because of the adiabatic assumption) the Boyle's relation (\SeeChapter{see section Continuum Mechanics page \pageref{boyle mariotte law}}) to achieve the same result:
	
	This gives a radius of about $2.33$ [m] (thus diameter  of about $4.6$ [m]) instead of $0.78$ [m] to the ground! An increase of the diameter of about $300\%$. However, it is more important to focus on the increasing surface to determine the elastic stress forces on the PVC. So we have before:
	
	and after:
	
	Thus an increase of the surface of about $1,000\%$ while a standard elastomer (including PVC as a part) is not resistant to a relative change more than $500\%$ !!! It is then much easier to understand from the point of view of the surface, why the balloon does not withstand an increase in the diameter of about $300\%$.

	Moreover, if we apply a little abusively Hook's law to the balloon (\SeeChapter{see section Continuum Mechanics page \pageref{hooke law}}), with PVC Young's modulus which is between (source Wikipedia):
	
	We have:
	
	Which complies with the numeral data tables which give an elastic lower limit value of $50$ [MPa] and an upper limit of $80$ [MPa] for PVC (source Wikipedia). We can then calculate the minimum and maximum theoretical height that the balloon can reach.

	So for the minimum height, we will write first:
	
	Which then corresponds to a final radius of:
	
	Which corresponds to a volume of:
	
	Applying Boyle-Martiotte:
	
	which is a pressure corresponding to a height of about $18,000$ [m] according to the experimental measurements (\url{www.engineeringtoolbox.com}) and this is indeed a rare height at which PVC balloons blow up.

	Now let us do the same with the maximum height:
	

	Which then corresponds to a final radius of:
	
	Which corresponds to a volume of:
	
	Applying Boyle-Martiotte:
	
	which is a pressure corresponding to a height of about $24,000$ [m] according to the experimental measurements and this is indeed a height at which the highest PVC balloons burst.
	
	\subsection{Cyclogenesis and Anticyclogenesis}
	Cyclogenesis is the development or strengthening of cyclonic circulation in the atmosphere (a low-pressure area). 
	
	Most of the atmospheric mass is contained within the first $20$ kilometers in altitude, so that the large-scale meteorology runs on a thin spherical shell (and can be assimilated to the mechanics of a two-dimensional fluid).

	The circulation engine of the atmosphere in the tropics is solar heating. Because of the inclination of $23.5$ degrees to the axis of rotation of the Earth, the Sun is never more than a few tenths of a degree from the zenith at noon throughout the year in the tropics, which gives maximum warming around the geographic equator.

	We must therefore distinguish air circulation in the vicinity of the tropics, characterized by strong vertical movement due to thermal convection, and circulation at mid-latitudes, made almost of horizontal movements:
	\begin{figure}[H]
		\begin{center}
			\includegraphics[scale=1]{img/engineering/wind_circulation_earth.jpg}
		\end{center}	
		\caption[Simplified schematic and idealized caricature of wind circulation on Earth]{Simplified schematic and idealized caricature of wind circulation on Earth (source: ?)}
	\end{figure}
	Let us suppose that for a moment that we completely stopped the movement of air in the atmosphere relative to the surface of the planet, and that we should leave then start turning from West to East (left to right on the images) from the rest position. The pressure gradient force causes the air to move from high pressure areas to low pressure areas ("vacuum call"). These convective movements are named "\NewTerm{Hadley cells}\index{Hadley cells}".
	\begin{figure}[H]
		\begin{center}
			\includegraphics[scale=0.7]{img/engineering/hadley_and_ferrel_cells.jpg}
		\end{center}	
		\caption[Hadley and Ferrel cells]{Hadley and Ferrel cells (source: \url{www.climatica.org.uk})}
	\end{figure}
	However, when the rotation movement began the Coriolis force (due to the rotation of the Earth) deflects the North-South winds towards the West and South-North winds eastward for an observer falling in the North Pole (\SeeChapter{see section Classical Mechanics page \pageref{coriolis force}}). We see therefore the formation of cyclones turning in in the opposite clockwise direction  in the Northern Hemisphere and vice versa in the Southern hemisphere (due to the direction of the vector $\vec{\omega}$ in this part of the hemisphere).
	
	The higher the air velocity increases, the more the Coriolis force increases accentuating the deviation. Eventually the Coriolis force reaches a value equal and opposite to that of the force of the pressure gradient, thus producing a flow of constant velocity (no acceleration), parallel to the isobars thus defining the geometric limit of the Hadley cell. This is what we name "\NewTerm{geostrophic balance}\index{geostrophic balance}". In practice, the flow outside the tropics is almost always in quasi-geostrophic equilibrium.

	In the absence of direct wind measurement, meteorologists can estimate the wind speed at a given point by measuring on a weather map the pressure gradient and the latitude. The geostrophic approximation is purely diagnostic. It has no predictive value because its equation contains no term of change depending on time.
	
	In the tropics, where the Coriolis force is becoming weaker until to be zero at the equator, there are other forces, such as centrifugal force, that balance the pressure gradient force.

	This is what we will prove here mathematically using Continuum Mechanics (fluids) and Classical Mechanics (see corresponding sections for the prerequisites).
	
	We know that in our system intervenes pressure forces (gradient), centrifugal forces, gravity forces. Forces to which we must not forget to add the Corlios force (implicitly its corresponding acceleration) of the system (meaning: the cyclone) the pulsation $\vec{\omega}$ (\SeeChapter{see section Classical Mechanics page \pageref{pulsation frequency period wave number}}):
	
	and the Coriolis force (implicitly: acceleration) of the fluid by unit mass (the reason for this unit choice will appear obvious few paragraphs further below) with respect to the pulsation $\Omega$ of the Earth:
	
	Thus, as we know (\SeeChapter{see section Classical Mechanics page \pageref{coriolis force}}), the Coriolis force will tend to deflect any descending movement to the right (East) in the Northern Hemisphere and any ascending movement to the left (West) in the Southern hemisphere (following we place ourselves and look in the direction of fluid movement in the previous figures).

	This is why the air at the base of the Hadley cell, traveling at low altitude from the Tropic to the Equator will be deflected to the West to give the east trade winds.

	We have also proved in the section of Continuum Mechanics a particular form of the Euler equation of the second form that was:
	
	Redesigned, this relation can also be written:
	
	But we had also proved that:
	
	It comes in the Earth reference frame:
	
	It is therefore of the equation defining the pressure within the fluid considered as isolated. In this relation, we must still subtract the Coriolis pressure  forces due to the geocentric reference frame to get the system the dynamic of the "cyclone"
	
	Which finally gives:
	
	\begin{figure}[H]
		\begin{center}
			\includegraphics[scale=1]{img/engineering/north_south_earth_slice_for_cyclogenesis.jpg}
		\end{center}	
		\caption[]{North-South slice of Earth for cyclogenesis study}
	\end{figure}
	If we zoom on the reference frame related to the cyclone and translate to it the vector pulse of the Earth, we have:
	\begin{figure}[H]
		\begin{center}
			\includegraphics[scale=1]{img/engineering/north_south_earth_slice_for_cyclogenesis_zoom.jpg}
		\end{center}	
		\caption[]{Reference frame linked to the cyclone with the planar pulse}
	\end{figure}
	Therefore:
	
	Therefore we have:
	
	As we study the movements (almost) horizontal in the atmosphere at this latitude, we can consider that the fluid particles are liable to remain in the horizontal plane $(\vec{e}_x,\vec{e}_y)$. The components of the Coriolis force a plane motion are therefore ($v_z=0$):
	
	where $f$ is named the "\NewTerm{Coriolis parameter}\index{Coriolis parameter}". So the Coriolis force in oceanography and meteorology is traditionally denoted:
	
	The number $f$, positive in the northern hemisphere, negative in the southern hemisphere, ranging from $0$ to $1.458$ at the poles while the force is of the order of thousandths of Newton for fluid masses (ocean currents) and of the same magnitude (because the speed compensates for the low density) for gas (air currents).
	
	We now apply the approximation of the geostrophic equilibrium, that is to say that we consider that the air is animated by a uniform rectilinear motion (geostrophic wind), in other words, we neglect the action of centrifugal force due to the rotation of the vortex relatively that of the Coriolis force due to the rotation of the Earth, which means we assume that:
	
	with $R$ being the radius of the vortex and $\omega$ its pulsation. Since (\SeeChapter{see section Classical Mechanics page \pageref{kinematics of circural motion}}):
	
	the latter inequality becomes:
	
	where:
	
	is named the "\NewTerm{Rossby number}\index{Rossby number}" and has no dimensions. A small Rossby number signifies a system which is strongly affected by Coriolis forces, and a large Rossby number signifies a system in which inertial and centrifugal forces dominate. For example, in tornadoes, the Rossby number is large ($\cong 10^3$), in low-pressure systems it is low ($\cong 0.1-1$) and in oceanic systems it is of the order of unity, but depending on the phenomena can range over several orders of magnitude ($\cong 10^2-10^2$). As a result, in tornadoes the Coriolis force is negligible, and balance is between pressure and centrifugal forces (named "\NewTerm{cyclostrophic balance}\index{cyclostrophic balance}). 
	\begin{tcolorbox}[title=Remark,colframe=black,arc=10pt]
	For the middle latitudes ($\lambda=45^\circ$), the experiences and measures give $f=10^{-4}\;[\text{m}\cdot\text{s}^{-1}]$ and $v=10\;[\text{m}\cdot \text{s}^{-1}]$. The limit value for which $R_0=1$ is $R=100\;[\text{km}]$. For a larger scale, that is the case with hurricanes where $R\cong 100\;[\text{km}]$ we are close to the geostrophic balance. For an  a smaller scale, Coriolis is negligible and the wind is accelerated from high to low pressure.
	\end{tcolorbox}
	In other words the Rossby number therefore represents the ratio between the forces of inertia and the forces due to rotation that characterize the motion of a fluid in a rotating frame.

	So we can make the difference between a geophysical flow wight a large Rossby number or small Rossby number. If the Rossby number is greater than one, then the Coriolis forces due for example to the Earth's rotation are negligible relatively to the flow inertia. Otherwise for a Rossby very smaller than unity, the Coriolis forces dominate the movement of the fluid.

	Thus, if one approaches the equator, $f$ tends to $0$ the Rossby number becomes very large and at the poles it becomes very weak respectively.

	Under this approximation, our Euler equation can be written as:
	
	and since we are interested only to the horizontal plane of the atmosphere this simplifies even more to the form:
	
	Thus in fully developed and vector form by taking back the capital $P$ for pressure as it is customary in meteorology:
	
	It comes then:
	
	Therefore:
	Therefore in conventional form:
	
	The norm being given by:
	
	Therefore:
	
	which is named the "\NewTerm{equation of geostrophic winds}\index{equation of geostrophic winds}".
	
	Four scenarios are to be considered (with some reminders about some concepts already introduced at the beginnin of the section):
	\begin{enumerate}
		\item We are in the northern hemisphere and therefore $f$ is positive. Let us suppose that $\mathrm{d}P / \mathrm{d}R$ is positive, then the pressure increases away from the vortex center (which latter is therefore a minimum of low pressure). Therefore $v$ is positive and we have a vortex named "depression" in the northern hemisphere. Thus, the fluid (the wind) blows around the depression counterclockwise (West direction) in the Northern Hemisphere.

		\textbf{Definitions (\#\mydef):} 
		\begin{enumerate}
			\item[D1.] A "\NewTerm{depression}\index{depression}" (or "low pressure") is an area where the atmospheric pressure decreases horizontally toward a center of low pressure, that is to say a local minimum pressure.
	
			\item[D2.] The intense weather systems circulating around a closed center of low pressure (like a vacuum that attracts clouds from which the fact such systems are visible on satellite photos) consistently receive in  general the term "\NewTerm{cyclone}\index{cyclone}" or "\NewTerm{tropical cyclone}\index{tropical cyclone}".
		\end{enumerate}
		\begin{figure}[H]
			\begin{center}
				\includegraphics[scale=0.8]{img/engineering/cyclones.jpg}
			\end{center}	
			\caption[]{Some northern hemisphere cyclones}
		\end{figure}
		\begin{tcolorbox}[title=Remark,colframe=black,arc=10pt]
		On Earth we frequently associate depressions to bad weather e because the dynamics surrounding depression presupposes the existence of updrafts (can hardly get into the ground so the only way to escape is up!) That cause clouds and precipitation. In addition, the pressure gradient over a depression can cause strong winds.
		\end{tcolorbox}
	
		\item We are still in the Northern Hemisphere and therefore $f$ is positive. Let us suppose that $\mathrm{d}P / \mathrm{d}R$ is negative this time, the pressure then decreases away from the vortex center (which is thus reaching a maximum pressure). Therefore $v$ is negative and we have a vortex naed a "\NewTerm{high pressure}\index{high pressure}" in the northern hemisphere. Thus, the fluid (wind) blows around the high pressure clockwise (to the East) in the Northern Hemisphere.

		\textbf{Definitions (\#\mydef):} 
		\begin{enumerate}
			\item[D1.] A "\NewTerm{high pressure}\index{high pressure}" is an area where the atmospheric pressure increases horizontally toward a center of high pressure, that is to say a local maximum pressure.
	
			\item[D2.] The intense weather systems to flow around a closed center of high pressure (as a fan it rejects and disperses the clouds from which the fact that the high pressure are not simply visible on satellite photos) consistently receive the name of "\NewTerm{anticyclone}\index{anticyclone}".
		\end{enumerate}
		\begin{tcolorbox}[title=Remark,colframe=black,arc=10pt]
		Anticyclones usually bring good weather and clear skies. Atmospheric dynamics causes air at medium altitudes to be relatively warm and dry, and therefore without clouds.
		\end{tcolorbox}

		\item We are still in the southern hemisphere and therefore $f$ is negative. Let us suppose $\mathrm{d}P /\mathrm{d}R$ is positive, then the pressure increases away from the vortex center. Therefore $v$ is negative and we have a vortex still named a "high pressure" (or "anticyclone") in the Southern Hemisphere. Thus, the fluid (wind) blows around the high pressure but counterclockwise (to the West) in the Southern Hemisphere.

		\item We are still in the southern hemisphere and therefore $f$ is negative. Let us suppose that $\mathrm{d}P/\mathrm{d}R$ is negative, the pressure decreases away from the vortex center. Therefore $v$ is positive and we have a vortex still named "low pressure" (or "Hurricane") in the Southern Hemisphere. Thus, the fluid (the wind) blows around the low pressure clockwise (to the East) in the Southern Hemisphere.
	\end{enumerate}
	\begin{tcolorbox}[title=Remark,colframe=black,arc=10pt]
	It is therefore possible to say generally on larger dimensions that the wind comes from high pressure (anticyclone) to move towards low pressure (depression).
	\end{tcolorbox}
	Here is an illustration example of a depression (anticyclone) and high pressure (cyclone) in the Northern Hemisphere such as represented by professional practitioners of meteorology:
	\begin{figure}[H]
		\centering
		\includegraphics[scale=1]{img/engineering/low_high_pressure.jpg}	
		\caption{Low pressure and high pressure illustration}
	\end{figure}
	We can indeed observe that depression (D) rotates counterclockwise and the high pressure (A) clockwise (and vice versa in the southern hemisphere).
	\begin{tcolorbox}[title=Remark,colframe=black,arc=10pt]
	It is therefore possible to say generally on larger dimensions that the wind comes from high pressure (anticyclone) to move towards low pressure (depression).
	\end{tcolorbox}
	
	\pagebreak
	\subsection{Tides}
	Tides are the rise and fall of sea levels caused by the combined effects of the gravitational forces exerted by the Moon and the Sun and the rotation of the Earth. The times and amplitude of tides at a locale are influenced by the alignment of the Sun and Moon, by the pattern of tides in the deep ocean, by the amphidromic systems of the oceans, and the shape of the coastline and near-shore bathymetry. Tides vary on timescales ranging from hours to years due to a number of factors.

	Tidal phenomena are not limited to the oceans, but can occur in other systems whenever a gravitational field that varies in time and space is present. For example, the solid part of the Earth is affected by tides, though this is not as easily seen as the water tidal movements.
	
	Among the phenomena of nature, the tide is one of the most majestic in its scope and power, one of the more surprising by its regularity and the discretion of its causes. We can then easily understand not only that it is imposed to the attention of browsers but it has, since the remotest antiquity, aroused the research of the most distinguished scholars.

	To approach the subject of the tides in a simply way, we can from a logical conclusion: If the lunar attraction was identical at each point of the Earth, there would be no tides. We must therefore approach the study of tides on the differences of forces. The influence of the moon on the tides is named the "\NewTerm{diurnal component}\index{diurnal component}".
	\begin{figure}[H]
		\centering
		\includegraphics[scale=0.74]{img/engineering/mont_saint_michel_tides.jpg}
		\caption{High and low tide at Mont Saint-Michel in France}
	\end{figure}
	Another nice phenomenon to study mathematically (and to see in real life or on YouTube) are the "tidal bore" which is the leading edge of the incoming tide forms a wave (or waves) of water that travels up a river or narrow bay against the direction of the river or bay's current:
	\begin{figure}[H]
		\centering
		\includegraphics[scale=0.55]{img/engineering/tidal_bore.jpg}
		\caption{Tidal bore}
	\end{figure}
	
	\subsubsection{First approach}
	Let us consider for the naive study of tilde a mass of water $m$ at the equator and at the poles. We will calculate the attractive force on the mass relatively to the center of the Earth and taking into account the influence of the Moon that has a mass $M_L$ (Lunar influence).
	\begin{figure}[H]
		\centering
		\includegraphics[scale=1]{img/engineering/tide_first_approach.jpg}
		\caption[]{Earth-Moon configuration for a first study of tides}
	\end{figure}
	Let us first compute the force $F_a$ at the equator to the nearest point to the Moon relatively to the figure above.

	Then we have:
	
	considering that $R \gg r$ and denoting the "\NewTerm{static tidal force}\index{static tidal force}":
	
	A numerical application for a mass $m= 1$ [kg] gives $f=1\cdot 10^{-6}\;[\text{m}\cdot \text{s}^{-1}]$.

	In vector form the previous relations is obviously written:
	
	As the Earth-Moon distance is approximately $60$ Earth radii, the intensity of acceleration varies approximately linearly (...) along the land portion of a line passing through the center of the Moon. This is particularly the case for the segment that connects the two antipodal points $A$ and $C$ in the figure above. We can write, O denoting the center of the Earth, that:
	
	We must now have to separate the two contributions of the Moon:
	\begin{enumerate}
		\item The force $\vec{F}_0$ applied on the center of mass $G$ is then uniform to the planet by construction. It is this force that is responsible for the revolution of our planet around the common center of mass of the two celestial objects.

		\item The esidual term $\pm \Delta \vec{F}_L$ is superimposed and takes opposite values at the antipodes. It is responsible for the tides (in the first approximation in this simplistic model).
	\end{enumerate}
	Thus, the force due to the Moon is of opposite sign to the horizontal. We then have two (lunar) tides per day at antipodal locations:
	\begin{itemize}
		\item That of the Moon who attracts (from this side of the Earth)

		\item That of the Moon that pushes (at the opposite side of the Earth)
	\end{itemize}
	Or schematically (without any respect for real proportions):
	\begin{figure}[H]
		\centering
		\includegraphics[scale=1]{img/engineering/tide_antipodal.jpg}
		\caption[]{Antipodal tide with antipodal bulge}
	\end{figure}
	If we considered the surface of the Earth as perfectly spherical and covered with water, it would then take the form of an ellipsoid whose axis would be directed to the celestial object generating the tide. We should then observe high tide and low tide that would take place twice a day and always at the same time. We name this the "\NewTerm{static tide}\index{static tide}" and the corresponding model "\NewTerm{static model of the tides}\index{static model of the tides}".

	It should be noted that the subtle play between the rotation of the Earth and the Moon produces enormous friction in water volume that have the effect of slowing the speed of rotation of the Earth about $1$ second every thousand years.
	
	\pagebreak
	\subsubsection{Second approach}
	For the second approach, which is worth seeing for general culture and also because it presents another interesting aspect of the explanation of the attraction between two celestial objects, let us consider the following scheme:
	\begin{figure}[H]
		\centering
		\includegraphics[scale=1]{img/engineering/tide_second_approach.jpg}
		\caption[]{Antipodal tide with antipodal bulge}
	\end{figure}
	We have proved in the section Trigonometry the cosine theorem (Al-Kashi formula) which gives us for recall:
	
	Hence the gravitational potential (\SeeChapter{see section Astronomy page \pageref{gravitation potential}}):
	
	But:
	
	and the gravitational potential is of the form:
	
	Which we can develop in series of Maclaurin (\SeeChapter{see section Sequences and Series page \pageref{usual maclaurin developments}}) until order $2$:
	
	The gravitational potential then becomes:
	
	If we keep only the terms of power $1$ and $2$ on $r / a$, it remains:
	
	The first term of the potential is:
	The first term of the potential is:
	
	It is the potential for $r=0$, that is to say the potential created by the Moon at the center of the Earth.
This term contains no variable, it is constant and therefore its gradient is zero, it gives no force since:
	
	The second term:
	
	contains the two variables $r$ and $\theta$. Its gradient will not be zero. It will generate a gravitational force that we will calculate, but using a trick.

	Since:
	
	It comes:
	
	Consequently, the gradient in Cartesian coordinates is reduced for the mass $\mathrm{d}m$ situated at point $A$ by:
	
	Thus all the elements of the Earth undergo from the Moon parallel forces (on the $x$-axis only) directed towards the Moon. The total mass of the Earth is the sum of all these masses and the total force that the Earth undergoes on the part of the Moon is the sum of the elementar forces. Therefore:
	
	And therefore the total force is only along the $x$-axis and is given by:
	
	This force is the same as if all the mass of the Earth were concentrated at the center $T$ and if the whole mass of the Moon were concentrated at the center $L$. The Moon undergoes on the part of the Earth the same force of contrary sense, it is the principle of mutual actions. It is the latter which forces the Moon to revolve around the Earth.

	The third term is the one responsible for the tides:
	
	The force deriving from the gradient in polar coordinates is given by (\SeeChapter{see section Vector Calculus page \pageref{gradient in polar coordinates}}):
	
	Therefore, the radial component of the force is:
	
	The orthoradial component:
	
	Therefore:
	
	For a mass $m$ of water, the tidal force is:
	
	For $\theta=\{0,\pi/2,\pi\}$, the orthoradial component $F_\theta$ vanishes. We then have for these three angles:
	
	For $\theta=0$ the force is only radial:
	
	The sign of this force is negative and it is directed towards the Moon. For $\theta=0$ the force is also only radial and we fall back on the same expression:
	
	Therefore, the amplitude is the same for the angle $\theta=0=\pi$.
	
	On the other hand, if the reader remember that to return to Cartesian components we have (\SeeChapter{see section Vector Calculus page \pageref{polar coordinates}}):
	
	We see then that for $\theta=0$, the force is oriented in the positive direction of the $x$-axis since $x$ will be positive (but zero following $y$). There is therefore a tide in the direction of the Moon (almost intuitive).

	We also see that for $\theta=\pi$, the force is oriented in the negative direction of the $x$-axis since $x$ will be negative (but zero following $y$). There is therefore a tide in the opposite direction to the moon (counter-intuitive for most people).

	For $\theta=\{\pi/2,3\pi/2\}$, the component $x$ is zero and the radial component will be:
	
	And it is directed towards the center of the Earth since:
	
	We see then that for $\theta=\pi/2$ the force is oriented in the positive direction of the axis $y$ since $y$ will be positive (but zero following $x$). We also see that for $\theta=3\pi/2$, the force is oriented in the negative direction of the $y$ axis since $y$ will be negative (but zero following $x$).

	In reality, the tides are much more complex than the above model (eccentricity of the lunar orbit, superimposition of the diurnal tide, lunar orbit, Moon-Sun alignment, inclination of the plan of the Moon's orbit, equinoxes, etc.). Here is a superb series of images of the elevation of the surface of the oceans in meters, on one tide cycle , calculated from a more elaborate model:
	\begin{figure}[H]
		\centering
		\begin{subfigure}{.4\textwidth}
		  \centering
		  \includegraphics[width=1\linewidth]{img/engineering/tide_01.jpg}
		\end{subfigure}
		\begin{subfigure}{.4\textwidth}
		  \centering
		  \includegraphics[width=1\linewidth]{img/engineering/tide_02.jpg}
		\end{subfigure}
		\begin{subfigure}{.4\textwidth}
		  \centering
		  \includegraphics[width=1\linewidth]{img/engineering/tide_03.jpg}
		\end{subfigure}
		\begin{subfigure}{.4\textwidth}
		  \centering
		  \includegraphics[width=1\linewidth]{img/engineering/tide_04.jpg}
		\end{subfigure}
		\begin{subfigure}{.4\textwidth}
		  \centering
		  \includegraphics[width=1\linewidth]{img/engineering/tide_05.jpg}
		\end{subfigure}
		\begin{subfigure}{.4\textwidth}
		  \centering
		  \includegraphics[width=1\linewidth]{img/engineering/tide_06.jpg}
		\end{subfigure}
		\begin{subfigure}{.4\textwidth}
		  \centering
		  \includegraphics[width=1\linewidth]{img/engineering/tide_07.jpg}
		\end{subfigure}
		\begin{subfigure}{.4\textwidth}
		  \centering
		  \includegraphics[width=1\linewidth]{img/engineering/tide_08.jpg}
		\end{subfigure}
		\begin{subfigure}{.4\textwidth}
		  \centering
		  \includegraphics[width=1\linewidth]{img/engineering/tide_09.jpg}
		\end{subfigure}
		\begin{subfigure}{.4\textwidth}
		  \centering
		  \includegraphics[width=1\linewidth]{img/engineering/tide_10.jpg}
		\end{subfigure}
		\begin{subfigure}{.4\textwidth}
		  \centering
		  \includegraphics[width=1\linewidth]{img/engineering/tide_11.jpg}
		\end{subfigure}
		\begin{subfigure}{.4\textwidth}
		  \centering
		  \includegraphics[width=1\linewidth]{img/engineering/tide_12.jpg}
		\end{subfigure}
	\end{figure}
	
	\begin{figure}[H]
		\centering
		\begin{subfigure}{.4\textwidth}
		  \centering
		  \includegraphics[width=1\linewidth]{img/engineering/tide_13.jpg}
		\end{subfigure}
		\begin{subfigure}{.4\textwidth}
		  \centering
		  \includegraphics[width=1\linewidth]{img/engineering/tide_14.jpg}
		\end{subfigure}
		\begin{subfigure}{.4\textwidth}
		  \centering
		  \includegraphics[width=1\linewidth]{img/engineering/tide_15.jpg}
		\end{subfigure}
		\begin{subfigure}{.4\textwidth}
		  \centering
		  \includegraphics[width=1\linewidth]{img/engineering/tide_16.jpg}
		\end{subfigure}
		\begin{subfigure}{.4\textwidth}
		  \centering
		  \includegraphics[width=1\linewidth]{img/engineering/tide_17.jpg}
		\end{subfigure}
		\begin{subfigure}{.4\textwidth}
		  \centering
		  \includegraphics[width=1\linewidth]{img/engineering/tide_18.jpg}
		\end{subfigure}
		\begin{subfigure}{.4\textwidth}
		  \centering
		  \includegraphics[width=1\linewidth]{img/engineering/tide_19.jpg}
		\end{subfigure}
		\begin{subfigure}{.4\textwidth}
		  \centering
		  \includegraphics[width=1\linewidth]{img/engineering/tide_20.jpg}
		\end{subfigure}
		\begin{subfigure}{.4\textwidth}
		  \centering
		  \includegraphics[width=1\linewidth]{img/engineering/tide_21.jpg}
		\end{subfigure}
		\begin{subfigure}{.4\textwidth}
		  \centering
		  \includegraphics[width=1\linewidth]{img/engineering/tide_22.jpg}
		\end{subfigure}
		\begin{subfigure}{.4\textwidth}
		  \centering
		  \includegraphics[width=1\linewidth]{img/engineering/tide_23.jpg}
		\end{subfigure}
		\begin{subfigure}{.4\textwidth}
		  \centering
		  \includegraphics[width=1\linewidth]{img/engineering/tide_24.jpg}
		\end{subfigure}
		\caption[Real complexity of tides on Earth]{Real complexity of tides on Earth (source: Wikipedia)}
	\end{figure}
	\begin{tcolorbox}[title=Remark,colframe=black,arc=10pt]
	The phenomenon is therefore due to the deformation of the surface of the oceans as a result of the combined attractions of the celestial bodies. This movement can even destroy the celestial boject which undergoes it: if the tidal force prevails over the gravitational force of its constituents, the celestial objects disintegrates. This limit where the tidal forces outweigh the gravitational force is named the "Roche limit" (\SeeChapter{see section Astronomy page \pageref{roche limit}}).
	\end{tcolorbox}
	Besides the diurnal tide due to the attraction of the Moon, it is necessary to count on a tide due to the centrifugal force of the movement of the Earth and the Moon around their center of mass (but it depends on the latitudes, the relief and a lot of other parameters objectively because in some places of the planet there is only one tide per day). Indeed, the Earth and the Moon revolve around the center of mass which defines the orbit of Earth-Moon couple (the scales are not respected):
	\begin{figure}[H]
		\centering
		\includegraphics[scale=1]{img/engineering/tides_superposition.jpg}
		\caption{Principle of tides superimposed on diurnal tides}
	\end{figure}
	And we will ignore the tides of equinoxes and others... until today...
	
	\pagebreak
	\subsection{Lorenz equation}
	The "\NewTerm{free convection}\index{free convection}" or "\NewTerm{natural convection}\index{natural convection}" is the flow regime obtained when we heat a fluid without imposing any external flow. This is the case for atmospheric convection movements (hot gases in cold gases), convection movements of the molten rock responsible for plate tectonics, movements of hot water under pressure in geysers or cakes... and many other phenomena...

	These flows are inexplicable if we do not couple the equations of dynamics and thermodynamics!

	We will in this context establish the famous system of Lorenz equations at the price, however, of numerous approximations and assumptions in order to simplify as much as possible the calculations and the mathematical tools used (because at the time of the development of the model the computers were not what they are today).

	We will therefore prove in the context of convection (one of the important dynamics of our atmosphere) that the equations which determine certain parameters of the motion are very sensitive to the initial conditions, the purpose of which is to show the difficulty of predicting to a more or less long term with deterministic theoretical models (reason why in meteorology we use nowadays the method of the finite elements).

	A priori, the density $\rho$ is a function of temperature and pressure by the state law of ideal gases (\SeeChapter{see section Continuum Mechanics page \pageref{ideal gas law}}). It is therefore natural to think that if we heat a wall, the temperature of the surrounding fluid increases by diffusion. The pressure stratification is changed, the pressure gradient creates the movement.
	\begin{figure}[H]
		\centering
		\includegraphics[scale=0.5]{img/engineering/free_convection.jpg}
		\caption{One of the everyday life most known convection...}
	\end{figure}
	In all sections of this book, we have so far neglected any variation of $\rho$. But decoupling is no longer valid here since it is heating that causes movement. We will therefore allow a variation of the density with the heating assuming however that this perturbation is small. It is therefore necessary to reintroduce a variation of $\rho$ around an equilibrium position: rest. On the other hand, the viscosity will be assumed constant...

	Thus, given a far fluid at rest and at the temperature $T_{\infty}$, it is in the presence of a wall heated to the temperature $T_P$. To obtain the dependence of $\rho$, let us recall some classical thermo-elastic coefficients (\SeeChapter{see section Thermodynamics page \pageref{thermo-elastic}}):
	\begin{itemize}
		\item Coefficient of isobaric compressibility (or dilatation according to the writing in terms of density) :
		

		\item Isothermal compressibility coefficient:
		
	\end{itemize}
	By assuming now that density is mainly related to temperature (for simplicity) we can write (this hypothesis works well for fluids but not too much ... for gas !!):
	
	Using the general form of Taylor's development (\SeeChapter{see section Sequences and Series page \pageref{taylor series}}):
	
	We then have an approach with the engineer way ...:
	
	Therefore:
	
	where $\bar{T}$ is therefore a coefficient without dimensions (as $\varepsilon$ ...) more easily measurable experimentally.

	The continuity equation (\SeeChapter{see section Thermodynamics page \pageref{continuity equation}}) or of mass balance that is for recall:
	
	Then becomes:
	
	to the first order in $\varepsilon$. Moreover, we have proved in the section of Continuum Mechanics  that if the fluid is incompressible:
	
	Let us remember that in a first approximation the fluid is incompressible. It then only remains:
	
	As we wish to study a flow in the presence of gravity, it would be judicious to put:
	
	and therefore to focus only on variations around the hydrostatic equilibrium position ($\delta p$ is dimensionless!). We have proved still in the same section of Continuum Mechanics that in the case of the incompressible fluid with viscosity, the first form of Euler's equation (equation of motion):
	
	Let us first consider the both terms:
	
	which are written along the $Z$ axis:
	
	When there is movement, the projection along $Z$ thus make appears:
	
	which we then rewrite:
	
	Therefore:
	
	since:
	
	It comes:
	
	It remains therefore a buoyancy force directed upward.

	The variation of the density as a function of temperature in the product $\rho\mathrm{d}_t\vec{v}$ of the relation:
	
	will be neglected ($\rho\mathrm{d}_t\vec{v}\cong \rho_\infty \mathrm{d}_t\vec{v}$) because we will restrict ourselves in case where the speed is small. We have then by reintroducing the viscosity...:
	
	and we have then material derivative (\SeeChapter{see section Continuum Mechanics page \pageref{material derivative}}):
	
	there also a useful relation:
	
	We then have as expression of the force density:
	
	To continue, we will seek to determine the energy law of the equation of behavior proved in the section Continuum Mechanics that was for recall
	
	so that it also accounts for the relation between the stresses and thermodynamic characteristics of the fluid, such as heat flow and temperature. We will do this by characterizing the diffusion of energy in the medium due to the effects  (supposedly decoupled) of the viscosity of the fluid and the thermal conduction of the fluid.

	We rewrite this relation with new constants and another notation for the divergence:
	
	where $\mu_d$ and $\lambda$ are in this context named the "\NewTerm{Lamé coefficients}\index{Lamé coefficients}" (also named the "\NewTerm{Lamé parameters}\index{Lamé parameters}" or "\NewTerm{Lamé constants}\index{Lamé constants}"). The terms $\lambda$ and $\mu$ are individually referred to as "\NewTerm{Lamé's first parameter}" and "\NewTerm{Lamé's second parameter}".
	
	We have also proved in the sectoin of Continuum Mechanics the relation:
	
	Therefore:
	
	Which gives:
	
	Let us denote the total energy as:
	
	where $e$ is the internal mass energy of the fluid (so related to a unit of fluid mass). Now the instantaneous variation of the internal energy of the fluid is equal to the contribution of a mechanical power and of the supply of heat (according to what has been seen in the section of Thermodynamics):
	
	where $P$ gives the power of the external forces  to the system given necessarily by the force of the ambient potential field and the mechanical forces given by the stress tensor only (we are always in the situation of a perfect fluid). That is:
	
	and using the Ostrogradsky theorem (\SeeChapter{see section Vector Calculus page \pageref{gauss ostrogradsky theorem}}):
	
	which has indeed the units of a power we have indeed:	
	
	For the heat power $\dot{Q}$ it is very easy also thanks to the developments we had made in the section of Thermodynamics where we obtained the equation of heat:
	
	Therefore:
	
	We have finally:
	
	So all this gives us the equation of the energy of a fluid:
	
	Therefore:
	
	and as (\SeeChapter{see section Thermodynamics page \pageref{fourier law}}) the heat flow follows the Fourier law:
	
	We have then:
	
	Either by using the Laplacian definition of a scalar field (\SeeChapter{see section Vector Calculus page \pageref{scalar laplacian}}):
	
	Or explicitly:
	
	Now, by making the scalar product of:
	
	With the velocity $\vec{v}$ we get the balance of the kinetic energy:
	
	That is to say explicitly:
	
	By subtracting (just proved earlier): 
	
	and (just proved also earlir above):
	
	we get a local relation of the specific internal energy $e$:
	
	That is to say explicitly:
	
	But we also have (derivation of a product):
	
	Therefore:
	
	Indeed:
	
	We have therefore:
	
	And as the stress tensor $\sigma$ is symmetric:
	
	we have therefore:
	
	which is sometimes written (yes I also don't believe what I see sometimes...):
	
	where $\overline{\overline{D}}$ is named the "\NewTerm{tensor of deformation rate}\index{tensor of deformation rate}" and $\overline{\overline{\sigma}} : \overline{\overline{D}}$ represents the double contracted product of the stress tensor and deformation rate tensor.

	We have proved, for recall, in the section of Continuum Mechanics that:
	
	where:
	
	Therefore, it is simple to make the difference between normal forces and tangential forces. Whatever, to come back on the energy equation:
	
	and replacing explicitly the stress tensor $\sigma$, we get:
	
	But in our case:
	
	therefore we can write:
	
	But we also have (using the trace as seen in the section of Linear Algebra):
	
	We have therefore:
	
	Thus into technical condensed form (just for information...):
	
	It is clear that from the point of view of the entropy (\SeeChapter{see section Thermodynamics page \pageref{entropy}}) we have:
	
	We also have:
	
	Thus reduced to the massic values:
	
	The time variation giving:
	
	But with have the continuity equation (\SeeChapter{see section Thermodynamics page \pageref{continuity equation}}):
	
	Which gives finally:
	
	or written slightly differently:
	
	Injected in:
	
	This gives:
	
	If we consider the velocity gradient to be very small (quasi-static) then we can write the approximation:
	
	Let us now give the expression of the entropy (exact total differential) as a function of the temperature and pressure parameters only:
	
	Either in massic form:
	
	But have also proved in the section Thermodynamics the following relation:
	
	Either in massic form:
	
	Which gives us:
	
	But, we have also proved in the section Thermodynamics, one of the following Maxwell's relations:
	
	Either in massic form:
	
	hence:
	
	Therefore:
	
	Then our relation:
	
	can be written:
	
	If we assume that the variation of the density with the temperature is small, we then have at the atmospheric scale:
	
	and remembering that (material derivative):
	
	It finally comes:
	
	We now have two important equations:
	
	Therefore:
	
	
	\subsubsection{Rayleigh-Bénard convection cells (Benard-Marangoni instability)}\label{convection cells}
	Let us now briefly examine the Rayleigh-Bardard\index{Rayleigh-Bénard convection cells} problem, which consists of two plates limiting one fluid being more heated than the other.
	\begin{tcolorbox}[title=Remark,colframe=black,arc=10pt]
	The "\NewTerm{Benard-Marangoni instability}\index{Benard-Marangoni instability}" is a free surface configuration of Rayleigh-Benard instability. In fact, a thin layer of fluid is placed on a horizontal plane but its upper face is not in contact with an other plane but it is a free surface in contact with the air. The temperature of the lower plane must be superior to the temperature of the air to develop instabilities.
	\end{tcolorbox}
	We can then observe parallel longitudinal rolls in a film of viscous fluid (silicone oil) held between two plates at a hot temperature at the bottom and cold at the top. Here is a picture of these rolls seen from sides (but normally we don't notice it, because they only get visible when you add mark-particles):
	\begin{figure}[H]
		\centering
		\includegraphics[scale=1]{img/engineering/rayleigh_benard_convection_cell_oil_slice.jpg}	
		\caption{Rayleigh-Bénard convection cells (instability)}
	\end{figure}
	view from the top:
	\begin{figure}[H]
		\centering
		\includegraphics[scale=1]{img/engineering/rayleigh_benard_convection_cell_oil_from_top.jpg}	
		\caption{Rayleigh-Bénard convection cells viewed from top (instability)}
	\end{figure}
	It is a problem of natural convection: the heated fluid at the bottom expands and ascends entrained by the Archimedes principle, arrived at the top it cools and falls down. It is this movement that must be explained that is similar to that of the Earth's atmosphere, sun convection cells and also Earth's mantel dynamics as illustrated below:
	\begin{figure}[H]
		\centering
		\includegraphics[scale=0.65]{img/engineering/rayleigh_benard_convection_cell_sun.jpg}	
		\caption[Rayleigh-Bénard convection cells on our Sun]{Rayleigh-Bénard convection cells on our sun\index{solar granulation} (source: Swedish Vaccum Solar Telescope 1997-07-10)}
	\end{figure}
	\begin{figure}[H]
		\centering
		\includegraphics[scale=0.55]{img/engineering/rayleigh_benard_convection_cell_earth.jpg}	
		\caption[Rayleigh-Bénard convection cells on Earth]{Rayleigh-Bénard convection cells on Earth (source: ?)}
	\end{figure}
	\begin{figure}[H]
		\centering
		\includegraphics[scale=1]{img/engineering/rayleigh_benard_convection_cell_earth_mantle.jpg}	
		\caption[Rayleigh-Bénard convection cells as supposed in Earth mantle]{Rayleigh-Bénard convection cells as supposed in Earth mantle (source: ?)}
	\end{figure}
	We also notice experimentally that the convection movements are made approximately according to a torus (see the photo seen from the side just above). We can take advantage of this symmetry to simplify the analysis.
	
	Let us consider then a vertical loop of fluid circulating at constant speed (thus without too much turbulence ...):
	\begin{figure}[H]
		\centering
		\includegraphics[scale=1]{img/engineering/rayleigh_benard_convection_cell_simplified_illustration.jpg}	
		\caption{Vertical fluid loop with gradient in homogeneous gravific field}
	\end{figure}
	The configuration will be set as follows:
	\begin{figure}[H]
		\centering
		\includegraphics[scale=1]{img/engineering/rayleigh_benard_convection_cell_configuration_study.jpg}	
		\caption{Required configuration for the theoretical model of convection cell}
	\end{figure}
	Where $T_0$ is the average temperature of the fluid (caution!: do not forget that it is not an extensive quantity!) and where we indicated respectively the temperatures inside the torus and outside it (ie the environment ) which may all vary with time.

	We see that the difference of temperature is of $2T_2$ between the top and the bottom and of $2T_3$ between the right and the left.

	We put that the temperature varies linearly with the height (which of course is false in an atmospheric model ...):
	
	We notice that it is possible to parametrise the temperature along the inside of the torus with the following relation (parametric equation of the circle):
	
	We have then according to the figure above:
	
	Having put this, let us return to:
	
	We will pass this system in polar coordinates best matching the geometry of our problem. Let us recall first that in term:
	
	the differential operator $\vec{\nabla}$ is the divergence. Now, we have proved in the section of Vector Calculus that this one was then written in polar coordinates:
	
	Now, if we rely on the hypothesis that in the volume of the torus the velocity varies neither as a function of the angle nor inside the torus (and therefore does not vary according to the radius $r$), then in polar coordinates :
	
	We have then:
	
	We will reduce the analysis to a single dimension which will be that the phenomenon depends only on the angle. We then have in polar coordinates and explicating all the terms:
	
	where we have at the same time made the projection along the $z$ axis such that:
	
	The differential coefficient of the last term will bother us. We replace it by a coefficient $\Gamma$ which we assume to be constant and which opposes the motion such as we have:
	
	or more explicitly:
	or more explicitly:
	
	We now integrate this on the whole loop as a function of $\phi$. We have then:
	
	We then have the pressure term which disappears, because there is no pressure gradient along the loop. So:
	
	We then for the first term (\SeeChapter{see section Differential and Integral Calculus page \pageref{usual primitives}}):
	
	and for the second term:
	
	Then it remains:
	
	Therefore:
	
	We see in this equation that the motion is driven by the horizontal temperature difference $T_3$.

	Now, let us come back to:
	
	If we neglect the tangential forces inside fluid, then we have:
	
	where $D$ is the thermal diffusion coefficient (\SeeChapter{see section Thermodynamics page \pageref{diffusion coefficient}}).

	In polar coordinates this is reduced to:
	
	and we will also make another approximation:
	
	And we have the two relations:
	
	By subtracting:
	
	Therefore:
	
	and also:
	
	Therefore:
	
	After derivation:
	
	We group together the terms:
	
	We then have the following three differential equations that govern the dynamics of the system:
	
	We finish the multiple simplifications by putting...:
	
	Which gives us:
	
	Putting it this in a more clean way, we get:
	
	Now, let introduce the following dimensionless variables:
	
	where we can assimilate:
	\begin{itemize}
		\item $X$ at the non-dimensional speed

		\item $Y$ to the non-dimensional temperature difference between rising and falling currents

		\item $Z$ to the dimensionless deviation of the convection equilibrium
	\end{itemize}
	We then have indeed:
	
	Therefore:
	
	In an even more condensed and traditional form:
	
	where we have:
	
	where we have:
	
	that correspond to the "\NewTerm{Prandtl number}\index{Prandlt number}" that a dimensionless number, named after the German physicist Ludwig Prandtl, defined as the ratio of "\NewTerm{momentum diffusivity}\index{momentum diffusivity}" to "\NewTerm{thermal diffusivity}\index{thermal diffusivity}" and where for recall:
	\begin{itemize}
		\item $\Gamma$ is the momentum diffusivity (kinematic viscoscity) in $[\text{m}^2\cdot \text{s}^{-1}]$
		\item $K$ is the thermal diffusivity in $[\text{m}^2\cdot \text{s}^{-1}]$
		\item $\mu_d$ is the dynamic viscoscity in $[\text{Ps}^2\cdot \text{s}]$
		\item $\kappa$ is the thermal conductivity in $[\text{W}\cdot \text{m}^{-1}\cdot\text{K}^{-1}]$
		\item $c_P$ is the specific heat in $[\text{J}\cdot\text{kg}^{-1}\cdot\text{K}^{-1}]$
	\end{itemize}
	with some typical values:
	\begin{table}[H]
		\begin{center}
		\definecolor{gris}{gray}{0.85}
		\begin{tabular}{|l|c|}
			\hline
			{\cellcolor{black!30}Material} & {\cellcolor{black!30}Prandtl Number Value}  \\ \hline
			Molten potassium at $975$ [K] & $0.003$\\ \hline
			Mercury & $0.015$\\ \hline
			Noble gases & $0.16$-$0.17$ \\ \hline
			Oxygen & $0.63$ \\ \hline
			Molten lythium at $975$ [K] & $0.065$\\ \hline
			Air & $0.7$-$0.8$\\ \hline
			Gaseous ammonica & $1.38$\\ \hline
			Water & $7$\\ \hline
			Seawater at $293$ [K] & $7.2$\\ \hline
			Seawater at $273$ [K] & $13.4$\\ \hline
			$n$-butanol & $50$\\ \hline
			Engine oil & $100$-$40,000$\\ \hline
			Glycerol & $1,000$\\ \hline
			Polymer melts & $10,000$\\ \hline
			Earth's mantle & $1\cdot 10^{25}$\\ \hline
		\end{tabular}
		\end{center}
		\caption[Typical values for Pr]{Typical values for Pr (source: Wikipedia)}
	\end{table}
	And:
	
	which is assimilated to the "\NewTerm{Rayleigh number}\index{Rayleigh number}"  that is also a dimensionless number associated with "\NewTerm{buoyancy-driven flow}\index{buoyancy-driven flow}", also known as "\NewTerm{free convection}\index{free convection}" or "\NewTerm{natural convection}\index{natural convection}". When the Rayleigh number is below a critical value for that fluid, heat transfer is primarily in the form of conduction; when it exceeds the critical value, heat transfer is primarily in the form of convection.
	
	\subsubsection{Lorenz attractor and chaos}
	The previous system of three equations is essentially the same as that of the famous "\NewTerm{Lorenz system}\index{Lorenz system}". With one difference (!), the (real) Lorenz system contains a factor $b$ in the last equation (which does not change the result anyway since we still get a strange attractor in the end as we go on see):
	
	Pr, Re and $b$ are strictly positive, and we often put $\text{Pr}=10$, $\text{Re}=28$, $b=8/3$ where the Prandtl number corresponds to the value of that of water.

	The Lorenz equations describe the convection phenomena of an ideal two-dimensional fluid in a reservoir heated from below.

	We see by this by this proof that contrary to the statements not demonstrated on many Internet pages and forums that:
	\begin{enumerate}
		\item The system is by no far not simple mathematically and is very approximate

		\item That there are simpler systems to study that are also chaotic (\SeeChapter{see section Population Dynamics page \pageref{chaotic logistic model}})
	\end{enumerate}
	The interest of the Lorenz equations lies, however, in the sensitivity to the initial conditions and the convergence of the dimensionless variables.

	Let's see an example with Maple 4.00b:

	\texttt{>with(DEtools):\\
		>lorenz:=diff(x(t),t) = 10*(y(t)-x(t)),diff(y(t),t) = 28*x(t)-y(t)-x(t)*z(t),\\
		diff(z(t),t) = x(t)*y(t)-8/3*z(t);\\
		>DEplot3d({lorenz}, [x(t),y(t),z(t)], t=0..100, stepsize=0.01, 
		[[x(0)=10, y(0)=10, z(0)=10]], orientation=[-35,75], linecolor = t, thickness = 1);\\
	}
	
	This gives for the first $100$ units of dimensionless time:
	\begin{figure}[H]
		\centering
		\includegraphics[scale=1]{img/engineering/lorenz_phase_system_maple.jpg}	
		\caption{Phase Space of the Lorenz Equations System with Maple 4.00b}
	\end{figure}
	or for the first $10$ units of dimensionless time:
	\begin{figure}[H]
		\centering
		\includegraphics[scale=1]{img/engineering/lorenz_phase_system_first_ten_time_units_maple.jpg}	
		\caption[]{Phase Space of the Lorenz Equations System with Maple 4.00b}
	\end{figure}
	Well so far we realize that the dimensionless parameters revolve around two points named the "\NewTerm{strange attractors}\index{strange attractors}".

	\textbf{Definition (\#\mydef):} In the study of dynamic systems, an "\NewTerm{attractor}\index{attractor}" (or "\NewTerm{set-limit}\index{set-limit}") is a set, a curve or a space towards which a system evolves irreversibly in the absence of perturbations.

	Now always for the same values of adimensional time, we take $x(0)=10.1$, that is to say a relatively small change of the initial conditions. We have then:
	\begin{figure}[H]
		\centering
		\includegraphics[scale=1]{img/engineering/lorenz_phase_system_small_perturbation_maple.jpg}	
		\caption[]{Small variation in initial conditions of the Lorenz system with Maple 4.00b}
	\end{figure}
	We thus notice that the phenomenon is no longer really similar.

	Let us consider for example the variable $x$ taking as initial conditions:
	\begin{gather*}
		x(0)=10 \qquad y(0)=10 \qquad z(0)=10
	\end{gather*}
	then:
	\begin{gather*}
		x(0)=10 \qquad y(0)=10.1 \qquad z(0)=10
	\end{gather*}
	thus a small variation of $0.01$ on the value of $y(0)$.

	Either in Maple 4.00b:
	
	\texttt{>DEplot({lorenz}, [x(t), y(t), z(t)], t=0..15, stepsize = 0.01, [[x(0)=10, y(0)=10, z(0)=10],[x(0)=10, y(0)=10.01, z(0)=10]], scene = [t,x], linecolor = [blue,green], thickness = 1);}
	\begin{figure}[H]
		\centering
		\includegraphics[scale=1]{img/engineering/lorenz_phase_system_under_small_variation_of_initial_conditions_maple.jpg}	
		\caption[]{Analysis of the dimensionless variable $x$ for a small variation of initial conditions with Maple 4.00b}
	\end{figure}
	We see that the system shifts rather quickly from the initial model whereas at the beginning it remains identical but the overall shape remains.

	Another thing ... depending on the parameters the system can converge. Indeed, by changing the factor $28$ by the value $22$ we have for example (convergence on the left):
	\begin{figure}[H]
		\centering
		\includegraphics[scale=1]{img/engineering/lorenz_phase_system_convergence_maple.jpg}	
		\caption[]{Convergence of the Lorenz system with Maple 4.00b}
	\end{figure}
	or with the value $19$ the result is even more trivial:
	\begin{figure}[H]
		\centering
		\includegraphics[scale=1]{img/engineering/lorenz_phase_system_direct_convergence_maple.jpg}	
		\caption[]{Convergence of the Lorenz system with Maple 4.00b}
	\end{figure}
	Or with a value close to $1$:
	\begin{figure}[H]
		\centering
		\includegraphics[scale=1]{img/engineering/lorenz_phase_system_direct_fast_convergence_maple.jpg}	
		\caption[]{Fast convergence of the Lorenz system with Maple 4.00b}
	\end{figure}
	We notice a last interesting case is that if the Prandtl number is $1$ then the system is stable:
	\begin{figure}[H]
		\centering
		\includegraphics[scale=1]{img/engineering/lorenz_phase_system_stability_maple.jpg}	
		\caption[]{System Stability with Maple 4.00b}
	\end{figure}
	This sensitivity to the initial conditions, as well as the shape of the strange attractor of Lorenz, led the meteorologists to make a metaphor with the following sentence: can the flutter of a butterfly in Brazil cause a tornado in Texas? (By stopping the dissipation of the error due to the scales considered ...).

	Hence the name afterwards of "\NewTerm{butterfly effect}\index{butterfly effect}" for the study of the attractor of Lorenz also named in the field the "\NewTerm{Lorenz's butterfly}\index{Lorenz's butterfly}".
	
	\pagebreak
	\subsection{Waves}
	When one finds oneself on a boat, the movement of the water makes many people sick. It is also the case when one is in the galley to calculate this type of movement... Indeed, the waves in the water are neither transverse waves nor longitudinal waves. They are a bit of both. The water particles describe circles or ellipses, in a vertical plane parallel to the displacement of the wave. The big problem is that one of the boundary conditions is on the surface of the water and we do not know where that surface is. It is precisely one of the unknowns.

	The problem is arduous enough for this calculation not to be found in books of general or specialized physics. We then tried to include it here by simplifying enormously and keeping only the simplest cases. The following calculation can be taken as an example of how physicists simplify life by making approximations and also as a masterful example of a common but complex phenomenon to be modeled.

	We will not compute the wave equation. We must calculate the motion of the particles from the surface to the bottom of the water, since the movement extends to the bottom. But since particles describe ellipses, we need at least two variables per point: we need a vector field! We will therefore calculate the vector field of the velocities of the particles, but we will not calculate it directly. What we will find is a scalar potential of velocities whose gradient will be the vector field of velocities.

	A final word on the waves that we will calculate. These waves are named "\NewTerm{gravity waves}\index{gravity names}" (not to be confused with "gravitational waves" in General Relativity), because it is the weight of the water that acts as a restoring force to bring down the summits and ascend the hollows.

	I would like to thank Mr. Louis Peralta for his mathematical mastery and his pedagogy without which these exciting developments could never have been presented in this book. I am indebted to him, with his agreement, to copy / paste $99\%$ of his course material.

	We will make calculations in the simplest case: waves in a rectilinear channel of rectangular section and of constant width and depth... This reduces the dimensions of the problem to two (since the third one has an identical wave profile independently of where we are positioned in the width of the channel): the depth and direction of advance of the waves. Of course, we will neglect the friction on the bottom or sides of the canal and ... many other things.

	The coordinate system chosen is then:
	\begin{itemize}
		\item $x$ horizontal, positive in the direction of wave advancement (direction of channel length)

		\item $y$ horizontal in the direction perpendicular to the advancement of the waves (direction of the width of the channel). Again, no variable will depend on $y$ for the reasons already mentioned earlier

		\item $z$ vertical and positive upwards. The zero is chosen on the surface of undisturbed water. Under water, $z$ will therefore be negative. The depth of the channel being $h$, the bottom of the water will then be $z=-h$
	\end{itemize}
	Now, let us recall that we proved after very long developments in the section of Continuum Mechanics, that if the fluid is incompressible (divergence of the vector field of the velocities is therefore zero such that $\vec{\nabla}\circ\vec{v}=0$) and the dynamic viscosity is also ($\mu_d=0$), we then we have the Euler equation of the first form:
	
	because $U$ is for recall of the scalar potential of the gravitational field.

	What we will explicitly write in the following form in the case of our study of gravity waves:
	
	Before continuing, we will change the form of the term $\mathrm{d}_t\vec{v}$ which is a total derivative and not partial. Let us consider one of the two components ($x$ or $z$) of this term. Let us take the component $x$ of the velocity vector and use what we saw in the section of Differential and Integral Calculus (we omitted the $y$ component since the system does not depend on it):
	
	We have in our situation:
	
	Then we get:
	
	The first term $\partial_t v_x$ is a linear term, whereas the other two are quadratic terms (as they include the product of a velocity by the derivative of a velocity) which are an infection (...) in all formal calculations. We shall therefore limit our calculations to cases where these quadratic terms are negligible relatively to the linear terms (implicitly implying that the speed must vary very little, otherwise this hypothesis is no longer valid). With this restriction, we then have:
	
	We can then write:
	
	We will add an additional restriction to our calculations. This restriction consists in imposing that the rotational velocity is zero:
	
	That is to say, as we have seen in the section of Contiuum Mechanics, that there are no vortices in the fluid. We had already imposed that the velocity was zero in the $y$-direction, which prevents horizontal vortices horizontal axis of rotation). Now we eliminate the possibility of vortices with vertical axis of rotation also. It is not, in fact, too restrictive, we know that an object abandoned on the waves does not turn on itself unless it is caught in rolls. In the cases that we are going to calculate, the amplitudes are limited, we are in the linear case and, above all, not in the case of the rollers!!

	We have proved in the section of Vector Calculus that when the curl of a vector variable is zero, the variable can be expressed as the gradient of a scalar potential:
	
	It comes then that we can write:
	
	Then we get:
	
	What can be written:
	
	This equation acts as a wave equation, even if it does not directly describe the position or velocity of the particles. Once the equation of the function $\phi(x,z,t)$ are solved, we can deduce the velocities:
	
	It should not be forgotten that since the divergence of the vector field of the velocities is zero (incompressible field assumption):
	
	Therefore the scalar Laplacian of $\phi$ is zero:
	
	Therefore:
	
	with $-h<z<0$.
	
	The solution must satisfy boundary conditions. The most obvious is that the vertical velocity must be zero at the bottom of the water:
	
	To find the boundary conditions at the surface, we will define the variable $\xi(x,t)$ which is equal to the vertical displacement of a particle of water located on the surface of the water whose equilibrium position (in the absence of waves) is $x$ (not to be confused with the other $x$...!). The height of the surface of the water, measured by reference to the level of the water without wave, will therefore be given by $\xi(x,t)$. Thus, the vertical velocity $v_z$ at the surface of the water will be:
	
	This is a very subtle tip and not obvious at all to anticipate!

	At the surface, we can consider that the pressure $P$ is always equal to the atmospheric pressure for all the positions in $x$ and in $\xi$. Hence, a priori nothing will change if the atmospheric pressure changes (...), and it is more convenient to consider it null. With this, the equation:
	
	Becomes:
	
	From which we deduce:
	
	If we place ourselves on the surface, which corresponds to $z=0$, the preceding relation becomes:
	
	From which we deduce:
	
	If we place ourselves on the surface, which corresponds to $z=0$, the preceding relation becomes:
		
	and deriving this relation relatively to the time again, we get:
	
	Then using the relation proved just earlier above:
	
	It comes:
	
	Let us sum up the conditions that the scalar potential $\phi$ must satisfy:
	
	We will only look for solutions of the separable type (\SeeChapter{see section Differential and Integral Calculus page \pageref{separation vaiables method}}). That is to say that $\phi$, which is a function of $x$, $z$ and $t$, will then be able to be written as the product of three functions, each of which depends on a single variable. Moreover, we will work only in sinusoidal regime (an intuition ... of the observation in real life) and using the formalism of the phasors (\SeeChapter{see section Wave Mechanics page \pageref{phasors}}). Therefore, we have:
	
	As usual, we will only keep the real part of the phasors of the solutions at the end. 

	Let us now calculate the second partial derivatives with respect to $x$ and $z$:
		
	The following equation:
	
	tells us that under the surface, these two expressions are equal to a given sign:
	
	from which we deduce:
	
	$\lambda$ can only be a constant, since both sides of the expression depend on independent variables (which is very tricky for an observation!). This constant can take, in principle, any real or complex value.

	This gives us the following system of two independent differential equations of the second order:
	
	It is therefore a system of differential equations of order 2 independent, almost identical to that solved in the section of Thermodynamics when we studied the equation of heat.

	That being said, we do not know however the sign of the constant $\lambda$. What we know from what we have studied in the section of Differential and Integral Calculus is that if $b$ is positive then the solution of the differential equation in $z$:
	
	will have harmonic solutions. This would be very surprising ... and counter-intuitive with what we observe in reality. So we will impose $\lambda$ as being negative and then we have seen in the section of Differential and Integral Calculus that we will have by adopting pretty much the same notations:
	
	where we have the discriminant of the second differential equation (\SeeChapter{see section Differential and Integral Calculus page \pageref{second order differential equations}}) which holds:
	
	So in other words:
	
	He then comes:
	
	and the discriminant of the first differential equation will be purely complex:
	
	and therefore can be written:
	
	Our system is simplified even more:
	
	If we consider the advance as zero (null phase shift) we have:
	
	Since:	
	
	we have then:
	
	So in the second parenthesis, we just have two waves that go in opposite directions. Let's keep one of the two to simplify the analysis (no matter what!):
	
	To determine the constants, we will use the limit condition given above:
	
	Which in the present case gives us:
	
	So of course we are not going to simplify totally so that $k$ is zero otherwise it would not correspond with reality. On the other hand, we will simplify in the following way:
	
	Which imposes in the non-trivial case that:
	
	So we can choose what we want, but if we want to simplify life by anticipating what is coming ... it would be better to take the following choice:
	
	Therefore:
	
	Therefore we get:
	
	Using the hyperbolic functions (\SeeChapter{see section Trigonometry page \pageref{hyperbolic trigonometry}}) we have:
	
	We will continue to try to determine the constants using the second limit condition:
	
	Then it comes (for the derivative of the hyperbolic cosine see the section of Differential and Integral Calculus):
	
	and therefore by putting $z=0$:
	
	and simplifying:
	
	or written differently:
	
	Let us recall that (\SeeChapter{see section Wave Mechanics page \pageref{pulsation frequency period wave number}}):
	
	The prior-previous relation does not, to my knowledge, have any solution for expressing the phase velocity directly from an analytic expression (it is therefore transcendent) as a function of the depth $h$ and of the frequency $f$ (implicitly the pulsation $\omega$) of the wave without passing through a Maclaurin development (\SeeChapter{see section Sequences and Series page \pageref{usual maclaurin developments}}) of the hyperbolic tangent for small values of $kh$ such that:
	
	from which we derive when the depth $h$ and $k$ are also small by making an elementary simplification of the preceding relation:
	
	Relation which is named "\NewTerm{Lagrange celerity formula}\index{Lagrange celerity formula}". For example, for a tsunami, the wavelengths $\lambda$ are immense (on the order of a hundred kilometers) so the product $hk$ is small (of the order of $0.25$ in the oceans).

	If, on the other hand, the depth $h$ is very large as well as $k$ then we have in Taylor development  series (\SeeChapter{see section Sequences and Series page \pageref{usual maclaurin developments}}):
	
	from which we derive also after an elementary simplification:
	
	Relation which is named "\NewTerm{Newton celerity formula}\index{Newton celerity formula}".
	However we can numerically solve the relation:
	
	and then we have:
	\begin{figure}[H]
		\centering
		\includegraphics[scale=1]{img/engineering/gravity_wave_speed_period_plot.jpg}	
		\caption[Speed-Period gravity wave relation plot]{Speed-Period gravity wave relation plot (source: LPFR)}
	\end{figure}
	We can also by observing this plot understand why waves break (supercritical flux). Indeed, since the phase velocity for a given period increases as a function of the depth $h$, the upper part of the wave goes faster than the lower part and thus does not collapse due to lack of support. The supercritical part of the wave is destructive in the case of tsunamis because it is accelerated in its fall by gravity.

	It also follows that the larger waves catch up with the small ones and that their amplitudes are superimposed over a certain distance. We will determine later the function linking the amplitude of the wave to the distance.
	\begin{figure}[H]
		\centering
		\includegraphics[scale=1]{img/engineering/gravity_waves_type.jpg}	
		\caption[Most common type of well-known gravity wave]{Most common type of well-known gravity wave (source: ?)}
	\end{figure}
	Let us now return to:
	
	by conserving only the real part:
	
	Let us recall that we saw at the beginning:
	
	To get $Z$ we must therefore derive the prior-previous relation with respect to $z$ and then integrate the result with respect to time.

	The derivative thus gives:
	
	Therefore:
	
	The primitive gives:
	
	Therefore:
	
	And placing ourselves on $z=0$ (on the surface) we get:
	
	and by denoting by $R$ the amplitude of the wave:
	
	With this the vertical displacement $Z$ becomes:
	
	and remembering that the potential of speeds is given by:
	
	we can then rewrite it with this new constant $R$ as following:
	
	The horizontal velocity of a particle of water will then be:
	
	and the horizontal displacement will be the integral of the previous relation with respect to time:
	
	Therefore:
	
	and at the surface $z=0$:
	
	
	\subsubsection{Depth of a wave}
	It is now interesting to calculate the depth of penetration of the wave. Indeed, we know from experience that the waves are less and less visible as the water height is big (excepted for Rogue Wave that are special non-linear cases).

	Remember that we have just proved:
	
	Which corresponds to the vertical displacement of the wave. If:
	
	where we will consider that the wavelength of the waves is small and that the depth $h$ of the medium in which they propagate is large.

	We then have:
	
	and:
	
	and we see that for this approximation to be satisfied, it is necessary that:
	
	So that the remaining partofcalculations to be valid, we will impose small values of $z$ (ie close to the surface).

	Since then:
	
	Without forgetting that $-h<z<0$ (so $z$ is negative for recall!). The amplitude of the movements thus decreases exponentially under these conditions with the depth. The reader can verify that for some values of small $kh$ we have:
	
	which shows the limitations of the model with this approach...
	
	\subsubsection{Wave's amplitude}
	For the amplitude of the horizontal movement the situation is the same. That is to say:
	
	where we will consider that the wavelength of the waves is small and that the depth $h$ of the medium in which they propagate is large.

	As we have proved earlier above that:
	
	It comes then:
	
	where for recall, we always have $-h<z<0$ and that this model is valid only near the surface.

	So for these two approximations, we have then when $z$ is small in absolute value with respect to $h$ (therefore in shallow water):
	
	Thus we can observe that for the case of equation and in shallow water (for $z$ small), the water particles describe a circular motion (since the two components $Z$ and $X$ have the same amplitude).

	On the other hand, in the general case, we have:
	
	The amplitudes are therefore no longer the same and the movements are then elliptic.
	\begin{figure}[H]
		\centering
		\includegraphics[scale=1]{img/engineering/gravity_wave_summary.jpg}	
		\caption[Water gravity wave study summary]{Water gravity wave study summary (source: ?)}
	\end{figure}

	\begin{flushright}
	\begin{tabular}{l c}
	\circled{70} & \pbox{20cm}{\score{2}{5} \\ {\tiny 30 votes,  48.00\%}} 
	\end{tabular} 
	\end{flushright}

	%to make section start on odd page
	\newpage
	\thispagestyle{empty}
	\mbox{}
	\section{Mechanical Engineering}\label{mechanical engineering}
	\lettrine[lines=4]{\color{BrickRed}M}echanical Engineering\index{mechanical engineering} is the set of knowledge related to mechanics, at the physical sense (sciences of movements) and at the technical sense (the study of mechanisms). This field of knowledge goes from design of mechanical product to the recycling of this latter through the way, of course, of production, maintenance, etc. Its applications are very important in many areas of everyday life either for the manufacture of machinery, toys, home appliances, medical devices or buildings or a variety of means of transport... and the list goes on...
	
	Again, we will focus here only on the mathematical formalization of practical cases of current applications in the industry! So this section is only a general introduction to technical applications of mechanics and must be complemented by laboratory practices (or computer simulation with for the SimuLink™ MATLAB™ toolbox and the SimDriveline™ MATLAB™ toolbox) and by reading also the section of Thermodynamics where the solid state equation is treated and the section of Continuum Mechanics where the Navier-Stokes equation is proved. The mix of mechanics and magnetism systems (AC motors for example) is treated in the section Electrodynmics.
	
	\begin{tcolorbox}[title=Remark,colframe=black,arc=10pt]
	It would be pretentious to claim in this section that we want to do as good as the two free French PDF \textit{Éléments de machine} and \textit{Résistance des Matériaux} of Nicolet Gaston Raymond that are at our point of view unrivaled in content and quality at this date (compared to the same non-free books on the same subjects!). It is therefore strongly recommended to refer to them if you want to drive full information about mechanical engineering (see the download section of the site).
	\end{tcolorbox}
	
	\subsection{Gears}
	A "\NewTerm{gear}\index{gear}" is a mechanical system consisting of two intermeshing gears for the transmission of the rotational movement therebetween or for the propulsion of a fluid (when the speak about: gear pump).
	\begin{figure}[H]
		\centering
		\includegraphics[scale=0.9]{img/engineering/gear_train.jpg}	
		\caption{Small Gear train}
	\end{figure}
	
	Two or more gears working in a sequence (train) are named a "\NewTerm{gear train}\index{gear train}" or, in many cases, a "\NewTerm{transmission}"\index{transmission}; such gear arrangements can produce a mechanical advantage through a gear ratio and thus may be considered a simple machine. Geared devices can change the speed, torque, and direction of a power source. The most common situation is for a gear to mesh with another gear; however, a gear can also mesh with a non-rotating toothed part, called a rack, thereby producing translation instead of rotation.
	
	The inventor of the cogwheel is considered by many practitioners as being no less than the famous mathematician, engineer and physicist Archimedes.
	
	We find gears absolutely everywhere in our daily lives: cars, bicycles, watches, adjustable chairs, etc. We strongly recommend also the reader to think about the gears (trays and sprockets connected by a chain of transmission) of his bike to interpret the results that follow below.
	
	The common vocabulary about a simple standard gear is given in the figure below:
	\begin{figure}[H]
		\begin{center}
			\includegraphics{img/engineering/gear_definitions.jpg}
		\end{center}	
		\caption[Single gear simple definitions]{Single Gear simple definitions (source: Wikipedia)}
	\end{figure}
	It seemed important to us to first introduce briefly how to calculate the tooth pitch of a cylindrical wheel depending on another one for the most famous type of gears is the "\NewTerm{spur gear}\index{spur gear}" like the one shown below:
	\begin{figure}[H]
		\begin{center}
			\includegraphics{img/engineering/spur_gear_definitions.jpg}
		\end{center}	
		\caption[]{Spur Gear definitions for our analysis}
	\end{figure}
	For the driving to operate so we must ensure that the teeth of a one of the cylindrical wheel of diameter $\varnothing=D_1$ well interspersed between the teeth of another wheel of diameter $\varnothing=D_2$. For this, we must introduce the concept of "\NewTerm{circular pitch}\index{circular pitch}" from the teeth of each wheel which we will denote respectively by $P_1$ and $P_2$.
	
	For this we will use the assumption that the teeth contact point can be assimilate to the diameters (represented above by black circles) that the cylindrical wheels would have if there were no slip (and therefore we would not need teeth ...) in the same way that the cog-wheel. These circles are named "\NewTerm{primitive circles}\index{primitive circles}", or "\NewTerm{primitive cylinders}\index{primitive cylinders}" or "\NewTerm{primitive diameters}\index{primitive diameters}".
	
	The tooth pitch will thus be expressed as a function of the circumference and the number of teeth on each wheel. If we denote by $Z_1$ the number of teeth a cylindrical gear $1$ and the $Z_2$ the number of teeth of a cylindrical gear $2$, then we have the tooth pitch values that are equal respectively to:
	
	and as for the gear train to work we must have both tooth pitch to be equal such that:
	
	Therefore we have:
	
	where $m$ is named the "\NewTerm{tooth module}\index{tooth module}". So we will retain that:
	
	We also observe by the above relation that the tooth pitch is not proportional to the tooth module (the choice of a large module provides a number of small teeth and the choice of a small module a large number of teeth).
	
	For people interested in more stuff:
	\begin{figure}[H]
		\begin{center}
			\includegraphics{img/engineering/spur_gear_detailed_definitions.jpg}
		\end{center}	
		\caption{Spur Gear definitions}
	\end{figure}
	
	\subsubsection{Transmission ratios}
	The "\NewTerm{transmission ratio}\index{transmission ratio}" also named "\NewTerm{reduction ratio}\index{reduction ratio}" of a gear or a pulley system is very important in the field of mechanics and is defined by:
	
	and therefore if the speed (pulsation) is constant (see the section of Classical Mechanics for the detailed calculations of the kinematics of the circular motion) and no longer varies between starting and nominal operation, it comes:
	
	This is a widely used technology in everyday life but for people that like famous examples, we have such transmission rations in cars (and generally almost all engines):
	\begin{figure}[H]
		\begin{center}
			\includegraphics[scale=0.8]{img/engineering/transmission_ratio_example_car.jpg}
		\end{center}	
		\caption{Motor with few pulleys and drive belts}
	\end{figure}
	as well as in almost all mechanical watches:
	\begin{figure}[H]
		\begin{center}
			\includegraphics{img/engineering/transmission_ratio_example_watch.jpg}
		\end{center}	
		\caption{A gear system with particular transmission ratio of a watch}
	\end{figure}
	and bicycles:
	\begin{figure}[H]
		\begin{center}
			\includegraphics[scale=0.6]{img/engineering/bicycle_gear_ratio.jpg}
		\end{center}	
		\caption{Bicycle gears ratios}
	\end{figure}
	We also of have of course:
	
	Therefore, the ratio of the rotational speed (angular frequency) between the output gear and the input gear is equal to the ratio of the angles traveled between the output gear and the input gear or the inverse ratio of their respective time period.
	
	We even by our study of kinematics of the circular motion:
	
	
	Thus, to use the example of the bicycle ... if we want to apply the greatest moment of force possible to the rear wheel by minimizing the moment of force we provide on the pedal so the best strategy according to the relationship given above is to take the smallest possible plate associated to the largest sprocket (the ideal would be to have a plate smaller than the largest sprockets for a transmission ratio greater than one).
	\begin{figure}[H]
		\begin{center}
			\includegraphics[scale=0.8]{img/engineering/bycicle_gears_ratio.jpg}
		\end{center}	
		\caption[Gear example with bicycle]{Gear example with bicycle (source: Wikipedia)}
	\end{figure}
	We also better understand why it is recommended for countries having periods of snow to maximize the transmission factor when driving car (on snows). As the moment of force will be minimized avoiding too much slipping.
	
	Caution! The principle of pulley/gear with transmission ratio is a great force multiplicator but in no case it multiplies (increase or decrease) the work or the power as a simple calculation with the above figure can be done since for the work:
	
	is the same for the bot situations.
	
	\begin{tcolorbox}[title=Remark,colframe=black,arc=10pt]
	The norm ISO 1122-1  provides the inverse definition and denote the transmission ratio $i$ instead of $r$:
	
	\end{tcolorbox}

	\subsubsection{Gears association}
	For reasons of geometrical or mechanical constraints, it is sometimes necessary to build gear stages as shown for example with the gear train of 4 wheels below:
	\begin{figure}[H]
		\begin{center}
			\includegraphics[scale=0.8]{img/engineering/gear_train_multiple_wheels_3d.jpg}
		\end{center}	
		\caption{Gear train example in 3D with multiple wheels}
	\end{figure}
	Either into technical schematic form (not according to Swiss VSM Standards):
	\begin{figure}[H]
		\begin{center}
			\includegraphics{img/engineering/gear_train_multiple_wheels_2d.jpg}
		\end{center}	
		\caption{Equivalent gear train example in 2D with multiple wheels}
	\end{figure}
	The overall transmission ratio will then be given for the speed of rotation by:
	
	where the fourth equality is simplified because in the case above:
	
	We can also express the total transmission in terms of diameters. Since we have proven that:
	
	he then comes immediately:
	
	expression that we can not simplify!
	\begin{tcolorbox}[title=Remark,colframe=black,arc=10pt]
	Thus, in the context of a transmission of a watch with astronomical complication, the transmission ratio was obtained by determining the rational fraction $1802/217$. This being very difficult to implement with only two gears, we'll just build with three axes and four gears (with $7$, $31$, $34$ and $53$ teeth) the same ratio - which fortunately is not irreducible - as follows:
	
	\end{tcolorbox}
	\begin{tcolorbox}[colframe=black,colback=white,sharp corners]
	\textbf{{\Large \ding{45}}Example:}\\\\
	We would like with a $4$ wheel gear train on $2$ axes, driven by the time axis (which spins through $12$ hours), make a very precise transmission ratio $r$, to make a watch complication that shows a moon phase with the classic disc having $2$ moons and turning behind a mask:
	\begin{figure}[H]
		\centering
		\begin{subfigure}{0.4\textwidth}
			\includegraphics[width=\textwidth]{img/engineering/moon_phase_disc.jpg}
		\end{subfigure}
		\begin{subfigure}{0.4\textwidth}
			\includegraphics[width=\textwidth]{img/engineering/moon_phase_mask.jpg}
		\end{subfigure}				
	\end{figure}
	\end{tcolorbox}
	
	\pagebreak
	\begin{tcolorbox}[colframe=black,colback=white,sharp corners]
	As constraints the maximum number of teeth per wheel shall not exceed $300$ teeth and do not fall below $7$. The accuracy will be the best permitted within these numbers of teeth.\\
	
	We will take $29$ days $12$ hours and $44$ minutes for the lunar month.\\
		
	The calculation of the desired ratio is relatively simple. The two moons disc must make a full turn in $5,102,885.8$ seconds. The wheel of the timepiece makes one revolution in $12$ hours, or $43,200$ seconds.\\
		
	For a turn of the two moons disc (leading axis) the axis of the $12$h wheel (driven axis):
	
	and rounding as below, we have an error of about $3$ ten-thousandth of a second per $12$ hours. Approximately $0.2$ seconds per year of delay. What is acceptable for a mechanical watch.

	By rounding:
	
	we have an error of about $30$ seconds per year, which is still acceptable for a mechanical watch. If we take again away one decimal place, then we have a delay of $6$ minutes per year, even removing an additional decimal place, we would have a $49$ minutes delay (which is still acceptable for many mechanical watches). By cons, beyond, this is no longer acceptable!\\
	
	Now, to find the nearest decomposable rational fraction to this number there exist numerous empirical methods (by trial and error or by using a Brocot tree) and tables but the least worst ... for us ... is that using the continuous fractions (\SeeChapter{see section Number Theory page \pageref{continued fraction}}) when it is applicable...\\
	
	Remember that we have proved that:
	
	\end{tcolorbox}
	
	\pagebreak
	\begin{tcolorbox}[colframe=black,colback=white,sharp corners]
	considering:
	
	If we denote by $x$ the ratio $a / b$ then the relations above give us that $q_1$ is the integer part of $x$, $q_2$ the integer part of $b/r_1$ thus of $b/((x-q_1)b)=1/(x-q_1)$ either and so on...\\
	
	We then have in our case:
	
	Thus after we put everything to a common denominator:
	
	\end{tcolorbox}
	
	\pagebreak
	\begin{tcolorbox}[colframe=black,colback=white,sharp corners]
	If the difference (error) between this fraction and the exact value is acceptable (just have to calculate the error of time this causes after a year of operation of the watch for example), we stop ourselves here. But there is a second criterion ... the numerator and denominator must be decomposable in a satisfactory manner in relation to requirements of the number of teeth of the gears. Or in this case, the denominator ($5788$) can not be decomposed satisfactorily in relation to our constraints (do the decomposition with the command \texttt{ifactor} of Maple 4.00b and you will see!).\\
	
	If we continue to develop our continued fraction, we will not find an acceptable solution before the numerator or the denominator exceeds the maximum permissible value of $300^2$.

	Thus, we return for example to our fraction (well this is not an acceptable approximation because in reality the maladjustment of the watch will be too fast):
	
	and we cheat a bit by trial and error to find a good ratio:
	
	Then, as we have a four gears train, we must have:
	
	and then we take (always helping with to function \texttt{ifactor} of Maple 4.00b):
	
	\end{tcolorbox}
	
	\paragraph{Odd/Even Gear "problem"}\mbox{}\\\\
	The gear association below has a \underline{closed loop} of four different gears. No two of them have the same diameter, and no two have the same number of teeth. They turn smoothly. Any even number of gears, with parallel axles, or even smooth wheels, can be put into such a loop, and they will turn freely, no matter what their tooth count or size. There's a simple way to prove this, using elementary geometry.
	\begin{figure}[H]
		\begin{center}
			\includegraphics{img/engineering/even_loop_gear.jpg}
		\end{center}	
	\end{figure}
	The underlying principle isn't physics, but geometry, and it applies not only to gears, but to smooth friction wheels as well. The figure shows two friction wheels of different diameter. When they turn without slipping, the two circles must turn through the same arc, but in opposite sense of rotation. That is, if the left wheel turns clockwise through arc $A$, the right wheel turns counter-clockwise through arc $B$, and $B = - A$:
	\begin{figure}[H]
		\begin{center}
			\includegraphics{img/engineering/friction_gear.jpg}
		\end{center}	
	\end{figure}
	If these were gears, then if one gear turns through $N$ teeth, the other gear turns through $N$ teeth in the other direction. It follows that if you had a string of an odd number of gears and tried to make a loop by meshing the gears at the end of the loop, their points of contact would be moving in opposite directions. If you closed a loop of an even number of gears, they would all turn quite smoothly through the same arc.
	
	There are even exotic gears that have non-circular perimeters, such as oval or elliptical. You could make a model with an even number of these in a loop, and the gears in the model would turn smoothly.
	
	A typical mistake is to think that the following will work:
	\begin{figure}[H]
		\begin{center}
			\includegraphics[scale=0.6]{img/engineering/serial_gears_loop.jpg}
		\end{center}	
	\end{figure}
	
	\subsubsection{Type of Gears}
	Here are some basic types of gears and how they are different from each other.
	
	\begin{itemize}
		\item The most common gears are "\NewTerm{spur gears}\index{spur gears}" and are used as we have seen previously in series for large gear reductions along a same plane! The teeth on spur gears are straight and are mounted in parallel on different shafts. These are particularly loud, due to the gear tooth engaging and colliding. Each impact makes loud noises and causes vibration, which is why spur gears are not used in machinery like cars. 

		\item "\NewTerm{Helical gears}\index{helical gears}" operate more smoothly and quietly compared to spur gears due to the way the teeth interact. The teeth on a helical gear cut at an angle to the face of the gear. When two of the teeth start to engage, the contact is gradual--starting at one end of the tooth and maintaining contact as the gear rotates into full engagement. Helical is the most commonly used gear in transmissions.

		\item "\NewTerm{Bevel gears}\index{bevel gear}" are used to change the direction of a shaft’s rotation (bevel gears are most often mounted on shafts that are 90 degrees apart). Bevel gears have teeth that are available in straight, spiral, or hypoid shape. Straight teeth have similar characteristics to spur gears and also have a large impact when engaged.

		\item "\NewTerm{Worm gears}\index{worm gears}" are used in large gear reductions. The setup is designed so that the worm can turn the gear, but the gear cannot turn the worm. The angle of the worm is shallow and as a result the gear is held in place due to the friction between the two. The gear is found in applications such as conveyor systems in which the locking feature can act as a brake or an emergency stop.
		
		\item "\NewTerm{Geneva gears}\index{Geneva gears}" or "\NewTerm{Maltese cross gears}\index{Maltese cross gears}" are gears mechanism that translates a continuous rotation into an intermittent rotary motion. The rotating drive wheel has a pin that reaches into a slot of the driven wheel advancing it by one step. The drive wheel also has a raised circular blocking disc that locks the driven wheel in position between steps. In the most common arrangement, the driven wheel has four slots and thus advances by one step of $90^\circ$ for each rotation of the drive wheel. If the driven wheel has n slots, it advances by $360^\circ/n$ per full rotation of the drive wheel.
	\end{itemize}
	While manual transmissions (manually shifting a gear selector mechanism that disengages one gear and selects another) have remained relatively unchanged over the years, electronically controlled automatic, semi-automatic, and continuously variable transmissions (CVTs) have become increasingly complex, but also easier to use than ever before. However, modern transmissions of all types have become more prone to failure, primarily because of this higher level of complexity.
	
	The Continuously Variable Transmissions (CVT) is a transmission mechanism that doesn’t use gears as its means of producing various vehicle speeds at different engine speeds. Instead of gears, the system relies on a rubber or metal belt running over pulleys that can vary their effective diameters:
	\begin{figure}[H]
		\centering
		\includegraphics{img/engineering/cvt.jpg}
		\caption{Continuously Variable Transmissions (CVT)}
	\end{figure}
	To keep the belt at its optimum tension, one pulley will increase its effective diameter, while the other decreases its effective diameter by exactly the same amount. This action is exactly analogous to the effect produced when gears of different diameters are engaged.
	
	Now just a word about epicyclic gears\label{epicyclic gears}!
	
	\textbf{Definition (\#\mydef):} An "\NewTerm{epicyclic gear train}\index{epicyclic gear train}" or "\NewTerm{planetery gear}\index{planetery gear}" consists of two gears mounted so that the center of one gear revolves around the center of the other. A carrier connects the centers of the two gears and rotates to carry one gear, called the planet gear, around the other, called the sun gear. The planet and sun gears mesh so that their pitch circles roll without slip. A point on the pitch circle of the planet gear traces an epicycloid curve. In this simplified case, the sun gear is fixed and the planetary gear(s) roll around the sun gear.
	
	The epicyclic gear train below consists of a sun gear (yellow), planet gears (blue) supported by the carrier (green) and an annular gear (pink). The red marks show the relative displacement of the sun gear and carrier, when the carrier is rotated $45^\circ$ clockwise and the annular gear is held fixed:
	\begin{figure}[H]
		\centering
		\includegraphics[scale=0.5]{img/engineering/epicyclic_gear_train.jpg}
		\caption[Epicyclic gear]{Epicyclic gear (source: Wikipedia)}
	\end{figure}
	Yes ok... but why use a epicyclic train gear excepted in watches knowing that the latter or not one of the most important thing in life!
	
	There is one well know example: modern cars using CVT (and not only) have epicyclic gears and also almost all bicycles:
	\begin{figure}[H]
		\centering
		\includegraphics{img/mechanics/bycicle_gear.jpg}
		\caption{Bicycle hub gear (special example)}
	\end{figure}

	Why do CVT and bicycles use this type of gear? 
	
	First because this type of gear is very compact in surface and in volume and as forces are distributed across $4$ planets gears (sometimes more!) the effort on the sun gear is divided. This is for the technical point of view...
	
	Now for the utility point of view: Consider the figure above, where for example the sun gear is driven by the motor of the car and the annular gear  is connected in one way or another to the transmission of the car. Therefore if we make the sun gear and the carrier solidary (connected) they turn together in the same sens. Then we speak of "\NewTerm{forward gear}\index{forward gear}" as the whole mechanism move as a single unique gear. If we remove what makes the sun gear and annular solidary than as it is a double epicyclic gear, the sun gear will force the annular gear to turn in the reverse direction and the we get a "\NewTerm{reverse gear}\index{reverse gear}".
	\begin{tcolorbox}[title=Remark,colframe=black,arc=10pt]
	If the reader want to see such a gear in action he can take a look to YouTube where there is a ton of video on this subject.
	\end{tcolorbox}
	The gear ratio of an epicyclic gearing system is somewhat non-intuitive, particularly because there are several ways in which an input rotation can be converted into an output rotation. Indeed, in some hybrid vehicle transmissions, two of the components are used as inputs with the third providing output relative to the two inputs!!
	
	OK let us now introduce some mathematical (physical properties of this gear. For this let us consider the following figure where for simplification purposes our epicyclic has its planetary gears that are horizontal:
	\begin{figure}[H]
		\centering
		\includegraphics{img/engineering/planetary_gear_configuration_study.jpg}
		\caption{Epicyclic gear simplified schema}
	\end{figure}
	From the figure above we want to prove a canonical relation for the tooth counts. For this purpose let us denote the number of teeth of the ring gear by $N_r$, of the sun gear by $N_s$, and of the planet gears by $N_p$.
	
	We will assume that our planetary gear to work out is that all teeth have the same pitch, or tooth spacing. This ensures that the teeth mesh.
	
	The determination of a canonical between all teeth can  be made more clear by imagining "gears" that just roll (no teeth), and imagine an even number of planet gears. From the illustration above we can see that the diameters of the sun gear, plus two planet gears be must equal to the ring gear size. That is:
	
	And if we assume that all teeth have the same pitch that we will denote here by $P$, that is:
	
	Therefore injecting and simplifying in the prior-previous relation we get:
	
	We see that the special case $N_p=0$ brings us to the obvious results $N_r=N_s$.
	
	Let us denote first by $\omega_r,\omega_s,\omega_p,\omega_c$  the angular velocity of respectively the Ring (annulus) gear (pink color in the first figure), Sun gear (yellow color in the first figure), Planet gears (blue color in the first figure) and Carrier (green color in the first figure) and $N_r,N_s,N_p$ the corresponding number of teethes of each gear (the Carrier no have teeth of sure!).
	
	Now, usually in a planetary gear, one of the gears is held fixed. For example, if we hold the ring gear in a fixed position, $\omega_r$ will always be zero. So we can ignore. And consider also that in this example we drive the sun gear and we want to know $\omega_c$ in function of $\omega_s$.
	
	For this we use the previous teeth canonical relation. Indeed, if the carrier do one turn, this means that each of the planetary gears have traveled $N_r$ teeth, and therefore made $N_r/N_p$ turns. 
	
	
	\pagebreak
	\subsection{Strength of materials}
	Strength of materials (or SoM for the intimes...) is, like all the other sections of this book, an extremely vast area which level of detail and complexity of the calculations can explode rather quickly. We will in the following paragraphs focus on the essentials that the engineer (in business) needs to know. The developments are oversimplified for trivial special cases (straight bars and beams). In reality, we use the tensor calculus, design of experiments or computer modeling experience with FEM (finite element methods).
	
	Before we begin to study some simple concrete cases let us de some reminders of the proofs we have proceed during our study of this subject in the section of Continuum Mechanics:
	\begin{itemize}
		\item A solid considered as rigid does not exist, it is only a convenient approximation. Experience shows that a solid is in fact still slightly deformable under the effect of external forces.

		\item Relationships between deformations and tensions are usually complicated due to the anisotropy of the crystal lattices. However, the solids are generally not single crystals but polycrystalline substances consisting of assemblies of microcrystals associated randomly, they can be thus considered as isotropic.
	\end{itemize}
	Then, we should consider globally the following assumptions relatively to the developments that will follow:
	\begin{itemize}
		\item[H1.] Matter is homogeneous, that is to say for recall that it has the same physical constitution and the same structure throughout all its volume.

		\item[H2.] Matter is isotropic, that is to say for recall that its mechanical properties are the same in any point of the volume.

		\item[H3.] The material is perfectly elastic, that is to say, for recall that after removal of the external forces, the volume immediately takes back its original dimensions (unlike the plastic limit!).

		\item[H4.] The deformations (displacements of the points of the characteristics line ) are small compared with the dimensions of the objects studied.

		\item[H5.] Any straight section (cross-sections) before deformation remain straight after deformation (\NewTerm{Navier-Bernoulli hypothesis}\index{Navier-Bernoulli hypothesis}).

		\item[H6.] The results obtained in material strength theory can only be applies to a sufficient distance from the concentrated efforts applied on the volume (\NewTerm{Barré Saint Venant hypothesis}\index{Barré Saint Venant hypothesis}).

		\item[H7.] In the elastic range, matter obeys the law of proportionality and then the deformations are given by Hooke's law alread proved in the section of Continuum Mechanics. This linear law allows us to apply the principle of superposition to forces and to resistive deformations.
	\end{itemize}
	We proved in the section of Continuum Mechanics that Hooke's law that states, when the deformations are reversible, that there is proportionality between tension and deformation (it's a variation formulation of Hooke's law):
	
	or:
	
	where $E$ is the Young's modulus, $\varepsilon$ the normal deformation and $\sigma$ the normal stress. 

	Let us indicate that the ratio:
	
	is often named "\NewTerm{stiffness (rigidity) of the bar}\index{stiffness (rigidity) of the bar}" in the literature and is often denoted by $k$.

	We have also proved in the section of Continuum Mechanics that the shear stress was given by:
	
	where $G$ is the "shear modulus", $\gamma$ the angle of deformation, and $\eta$ the Poisson's ratio that is a dimensionless number. thus we have a relation between the modulus of elasticity and rigidity in the case of small deformations.

	We have proved also in the same chapter that for a solid or a liquid subjected to a uniform isotropic pressure we had:
	
	The compressibility coefficient $\kappa$ is therefore  a positive number, so using the above relation, we have:
	
	and then comes a known result:
	
	So the Poisson coefficient can not be bigger than $1/2$ and can be negative (in the latter case we speak then of "\NewTerm{auxetic materials}\index{auxetic materials}").

	Finally, let us remember that we saw in the section of Continuum Mechanics that the unit contraction along the $z$-axis was given during a traction along the $x$-axis by:
	
	That is to say when written differently (focusing on the $XZ$ plane)
	
	Therefore:
	
	and that is what shows the figure below:
	\begin{figure}[H]
		\centering
		\includegraphics{img/engineering/test_piece_traction.jpg}
		\caption{Traction on a test piece}
	\end{figure}
	We also proved in the section of Continuum Mechanics the following relation during in our study of flexural modulus:
	
	which expresses the bending moment for a beam under a moment of force $M$ (torque), then the span describes an arc of with a curvature of radius $R$ and where $I$ characterizes the "\NewTerm{shape stiffness (rigidity)}\index{shape stiffness (rigidity)}" of the material having a given cross sectional area $S$. This is a very important relation in many areas of mechanical and civil engineering (shipbuilding, automobile, architecture, etc.).
	\begin{tcolorbox}[title=Remark,colframe=black,arc=10pt]
	$I$ is named the "\NewTerm{static moment of inertia}\index{static moment of inertia}" or "\NewTerm{quadratic moment}\index{quadratic moment}" as we have already specified it in the section of Continuum Mechanics.
	\end{tcolorbox}
	
	\pagebreak
	\subsubsection{Quadratic moments}
	Let us see the three conventional static moments of inertia $I_x$ in the field of Material Strengths because they are often encountered in practice (construction).
	
	\begin{tcolorbox}[title=Remark,colframe=black,arc=10pt]
	The theory of moments of inertia is presented for recall in the section of Classical Mechanics. In the section on Geometric Shapes we proved in detail the moments of inertia of the most common volumes.
	\end{tcolorbox}
	Let us see first the quadratic moment of inertia $I_x$ of a  rectangular plate of side $b$ and height $h$:
	\begin{figure}[H]
		\centering
		\includegraphics{img/engineering/quadratic_moment_rectangular_plate.jpg}
	\end{figure}
	The domain occupied by the plate is given by:
	
	Then we have:
	
	As we will see further below , it is this result that explains why the right one below configuration will have more strength:
	\begin{figure}[H]
		\centering
		\includegraphics{img/engineering/wood_strength.jpg}
	\end{figure}
	\pagebreak
	And now let us study the quadratic static moment of a disc of diameter:
	\begin{figure}[H]
		\centering
		\includegraphics{img/engineering/quadratic_moment_circular_plate.jpg}
	\end{figure}
	Here the domain of integration is:
	
	where $d$ is the diameter of the disc.
	
	To calculate this integral, we use polar coordinates that are for recall given by (\SeeChapter{see section Vector Calculus page \pageref{polar coordinates}}):
	
	
	We always have:
	
	
	The Jacobian of the transformation is equal to $r$ (\SeeChapter{see section Differential and Integral Calculus page \pageref{jacobian in polar coordinates}}). We get:
	
	
	Now let us calculate the quadratic static inertia of a ring of external diameter $D$ and internal diameter $d$:
	\begin{figure}[H]
		\centering
		\includegraphics{img/engineering/quadratic_moment_ring_plate.jpg}
	\end{figure}
	Here is the domain of integration is:
	
	where $D$ and $d$ are the diameters of the large and small discs.

	If we denote by $S_1$ the domain of the large disc and $S_2$ that of the small disk then:
	
	using the quadratic static moment of inertia of the disk.
	
	For summary, we have therefore:
	
	and finally there exist also the static polar quadratic moment of $S$ with respect to a point O:
	
	It is therefore easy in simple cases to know the static polar moment of inertia and it is very useful for the study of torsion.

	It follows from these tools as more elements of the section are located away from the axis, the more the static quadratic moment will be important and the more (we will prove it just further below) the "arrows" will be weak.
	
	\pagebreak
	\subsubsection{Equation of the elastic line}
	For this case study example but widely used in practice we will first have to obtain mathematically the geometric form that takes the neutral axis of a beam subjected to bending forces.
	
	\begin{tcolorbox}[title=Remark,colframe=black,arc=10pt]
	 If all the reactions of a solicited system can be found from the static equilibrium equations, the problem is say to be "\NewTerm{statically determined}\index{statically determined}" or "\NewTerm{isostatic}\index{isostatic}". When the equations of static fail to find the equilibrium of a system, the system is say to be "\NewTerm{hyperstatic}\index{hyperstatic}" or "\NewTerm{statically indeterminate}\index{statically indeterminate}". We also say that while there is "freedom of movement" despite of $n$ connection points, we are in isostatic situation.
	\end{tcolorbox}
	\begin{figure}[H]
		\centering
		\includegraphics{img/engineering/neutral_fiber.jpg}
		\caption{Beam (simplified ...) subjected to a force}
	\end{figure}
	By definition of the derivative and under the assumption of small deformations (however that still works well up to $45^\circ$...):
	
	Or by derivating once again:
	
	Moreover, the figure shows that (\SeeChapter{see section Trigonometry page \pageref{spherical trigonometry}}):
	
	But the fact that the neutral fiber curve deviates little from the $y$ axis (small deformations), we can write:
	
	Therefore:
	
	Therefore we can write using the relationships obtained above:
	
	which is the differential equation giving $z=f(y)$, named "\NewTerm{equation of the elastic line}\index{equation of the elastic line}".
	
	Another common but less intuitive approach to get the same relation is to start from:
	
	and to recall that the radius of curvature $R$ is given by (\SeeChapter{see section Differential Geometry page \pageref{radius of curvature}}):
	
	Either by adapting the notation to our context:
	
	and neglecting the first derivative for the small deformations we find indeed:
	
	
	\pagebreak
	\begin{tcolorbox}[colframe=black,colback=white,sharp corners]
	\textbf{{\Large \ding{45}}Examples:}\\\\
	\textbf{E1.} Cantilevered beam (only at one side) that is classic case in construction and housing) with concentrated load (punctual) at the end:
	\begin{figure}[H]
		\centering
		\includegraphics[scale=0.9]{img/engineering/cantilevered_beam_punctual_force_extremity.jpg}
		\caption{Cantilevered beam with punctual force on extremity}
	\end{figure}
	In the section $S$ at any point, the moment of force (bending) is therefore:
	
	On the other hand:
	
	By eliminating $R$ between these two relations, it remains:
	
	The figure shows that the boundary conditions are:
	
	We get after integration:
	
	Therefore:
	
	\end{tcolorbox}
	
	\begin{tcolorbox}[colframe=black,colback=white,sharp corners]
	If $y=L$ the deformation is maximum and $z$ therefore takes the maximum value $f$ named the "\NewTerm{arrow}\index{arrow (mechanical engineering)}". It follows:
	
	This give us finally the maximum vertical deflection of the end to a beam fixed on one side:
	
	All data from this relation are known to us (foce, length, Young's modulus, static momentum) and it is then possible to determine whether the bar will break or not because we simply need to apply the relation proved above:
	
	So thanks to the relation:
	
	and knowing experimentally from what experimental value of $\sigma$ the material breaks down we will know when the beam will also break down (at least approximately!) knowing the arrow $f$, the bending moment $M$ and the static quadratic moment $I$.

	So we have a result that will be useful to us later:
	
	and integrating from $0$ to $L$ we find the arrow of our previous beam!\\

	\textbf{E2.} The Sustained beam (also named "\NewTerm{beam two times supported}" or "\NewTerm{isostatic beam}\index{isostatic beam}") is the most classic example in construction and therefore architecture (and for those who played during childhood to put wooden slats to move over a small river). This is a homogeneous beam, of constant section, resting on two free bearings at its ends and subjected to a load $F$ at the center:
	\begin{figure}[H]
		\centering
		\includegraphics{img/engineering/sustained_beam.jpg}
		\caption{Sustained beam with punctual centered force}
	\end{figure}
	\end{tcolorbox}
	
	\begin{tcolorbox}[colframe=black,colback=white,sharp corners]
	We can therefore consider that it is as if we had $F / 2$ at both ends of two beams of length $L / 2$ (thus the sum is indeed equal to $F$, that is to say, the bending moment). Notice that we neglect the weight of the beam relatively to that of $F$, but $F$ may be simply the weight of the beam! Using the last relation of the previous example, we have:
	
	Therefore:
	
	Thus the maximum vertical deflection of a beam connected on both sides is finally (changing a bit the notation):
	
	Thus, for a same length of beam, at identical $F$ the arrow  is $16$ times smaller than for a fixed beam! It was intuitive that it was lower for the same force but hard to guess that it would be of a factor $16$...!\\

	It is this relationship that is also used for IPN beams (famous beams in building construction!).\\

	\textbf{E3}. Beam built only one side still named "cantilever beam" (also classic case in the construction and housing) but now with a constant line load denoted $w$:
	\begin{figure}[H]
		\centering
		\includegraphics{img/engineering/cantilevered_beam_line_constant_load.jpg}
		\caption{Beam fixed on one side with constant line load}
	\end{figure}
	\end{tcolorbox}
	
	\begin{tcolorbox}[colframe=black,colback=white,sharp corners]
	The development is simple but some simplifications are smart to get at the end an elegant result. So we always start from the equation of the elastic line by adopting the notations relatively to the selected configuration (see figure above):
	
	Thus explicitly:
	
		As the moment of force $M$ is zero at $x = 0$, we have the constant that is zero. Since then:
	
	Integrating again, we get:
	
	As for $x=L$ by hypothesis the deformation is zero, we have then the constant that is given by:
	
	Therefore:
	
	By integrating a last time:
	
	And as on $x=L$ we have $y$ that is also zero, it comes for the constant:
	
	Therefore:
	
	And as the arrow is anyway on $x=0$, we have then:
	
	
	\end{tcolorbox}
	
	\paragraph{Euler-Bernoulli Beam equation}\mbox{}\\\\
	Let us now consider the case of a beam fixed on both sides (case even more common that the three previous examples!). The analysis will be a bit more difficult and we will have to introduce several new concepts.

	A beam in practice must withstand at least the following efforts:
	\begin{itemize}
		\item Tension or Compression:
		\begin{figure}[H]
			\centering
			\includegraphics{img/engineering/beam_tension_or_compression.jpg}
		\end{figure}
		
		\item Shearing (shear):
		\begin{figure}[H]
			\centering
			\includegraphics{img/engineering/beam_shearing.jpg}
		\end{figure}

		\item Flexion (bending stress):
		\begin{figure}[H]
			\centering
			\includegraphics{img/engineering/beam_flexion.jpg}
		\end{figure}
	\end{itemize}
	If a beam is in equilibrium, then the internal forces must satisfy at all points the following relations:
	
	Let us consider now a beam embedded at its two ends (dual-embedded beam) and let take in a slice of infinitesimal length $\mathrm{d}y$ such that locally its curvature is zero. The beam will be assumed to be subjected to a uniform force over its entire length (a force which can also be assimilated to its own weight, as already mentioned above). It is then customary to denote by $\vec{w}$ the force per unit length (total weight divided by the length) which is obviously a linear load:
	\begin{figure}[H]
		\centering
		\includegraphics{img/engineering/beam_euler_bernoulli_equation_01.jpg}
	\end{figure}
	If the beam is at equilibrium once deformed (weakly or stronly deformed, this does not matters !) then the sums of the forces of tension, compression, shear and bending must be zero at each point as we have already said! This does not however mean that at each point of the elastic line the values of each of the forces is equal! On the contrary! There are of course differences (if not, there would be no deformation...!).

	Let us first sum the local forces of tension and compression (horizontal) of the element of length $\mathrm{d}y$. We then have schematically:
	\begin{figure}[H]
		\centering
		\includegraphics{img/engineering/beam_euler_bernoulli_equation_02.jpg}
	\end{figure}
	That is to say algebraically (the variational can be negative or positive whatever!):
	
	If now we take care of the vertical forces at the source of the shear then we have schematically:
	\begin{figure}[H]
		\centering
		\includegraphics{img/engineering/beam_euler_bernoulli_equation_03.jpg}
	\end{figure}
	That is to say algebraically (the variational can be negative or positive whatever!):
	
	And finally for the bending moments the sum is also necessarily zero at equilibrium. However, in contrast to the two preceding algebraic sums where we could only use the differential, we have to choose a reference point $R$ for the bending moments since, for recall... the moment of force is by definition the product of a force by a distance. We will therefore naturally choose the center of gravity:
	\begin{figure}[H]
		\centering
		\includegraphics{img/engineering/beam_euler_bernoulli_equation_04.jpg}
	\end{figure}
	The uniform linear load $\vec{w}$ on the length $\mathrm{d}y$ generates a force at mid distance of:
	
	but as it is confounded with the choices of our reference point, then its moment of strength is zero!

	We then have algebraically:
	
	And if we neglect the differentials of order two it remains only:
	
	Finally, we have:
	
	To determine the bending moment ofrom the linear load (which is of primary interest to the practitioner), we derive the third relation twice and make a substitution:
	
	Hence:
	
	The problem with this last relation is the knowledge of the moments. We should get rid of this term absolutely. What we know easily is the deformation function and we have proved earlier above that:
	
	It then comes immediately by substituting the preceding relation in the prior previous one:
	
	where the product $EI$ is known as the "\NewTerm{flexural rigidity}".
	
	This is the most important elementary relation of the theory of beams, that covers the case for small deflections of a beam that is subjected to lateral loads only, because it allows by knowing the linear load to determine the function of deformation or conversely! It is so important to know that it is named the "\NewTerm{static beam equation}\index{static beam equation}" or in honor to those who have determined it: "\NewTerm{Euler-Bernoulli equation}\index{Euler-Bernoulli equation}".  It seemed it was first enunciated circa 1750,but was not applied on a large scale until the development of the Eiffel Tower and the Ferris wheel in the late 19th century

	Since it is a differential equation of order $4$ that will generate four constants at each integration, we will then need $4$ initial conditions to solve it completely.
	\begin{tcolorbox}[title=Remark,colframe=black,arc=10pt]
	The Euler-Bernoulli equation by its assumptions is only a special case of a more general theory named the "\NewTerm{Timoshenko beam theory}\index{Timoshenko beam theory}" published in 1921.
	\end{tcolorbox}
	
	\begin{tcolorbox}[colframe=black,colback=white,sharp corners]
	\textbf{{\Large \ding{45}}Example:}\\\\
	We would like to calculate the deformation of a beam fixed on both sides and loaded uniformly knowing its length $L$, its modulus of elasticity $E$, its moment of inertia $I$. \\

	We start from (we change the notations only to show that according to some text books the axes can be denoted differently):
	
	With the initial conditions:
	
	And integrating by repetition, we have:
	
	From the two initial conditions:
	From the two initial conditions:
	
	It follows that:
	
	With the other two remaining conditions, we obtain the following system which must allow us to determine the two remaining constants:
	
	Then it is simply a matter of solving a simple linear system (\SeeChapter{see section Linear Algebra page \pageref{linear systems}}):
	
	Subtracting both relation in the simple adequate way we get immediately:
	\end{tcolorbox}
	
	\begin{tcolorbox}[colframe=black,colback=white,sharp corners]
	
	Therefore for an ideal beam embedded on both sides subjected to a uniform load and described by:
	
	To determine the arrow, we must therefore look for the point $x$ where this relation has an optimum. We then have:
	
	It follows that the arrow has a maximum at $x = L / 2$. Injecting this into $y (x)$ we get the famous relation often given in the literature but rarely proved:
	
	It follows that the arrow of a beam is proportional to the fourth power of the length of the beam! Such high dependence imposes significant limitations on Civil Engineering structures based on beams.
	\end{tcolorbox}
	\begin{tcolorbox}[title=Remark,colframe=black,arc=10pt]
	The constant linear load, either over the total length of the beam or in successive sections, is a frequent observed type of stress in horizontal axis of pieces. It may come from the weight of the pieces with a constant section or from a load caused by an external force (eg gravity).
	\end{tcolorbox}
	
	\pagebreak
	\paragraph{Potential elastic energy}\mbox{}\\\\
	After a quick overview of the various elastic deformations of the parts stressed by the fundamental forces, we will establish here the general expression of the elastic energy accumulated in a beam of any shape stressed by external forces.

	Let us recall for this study that we can write Hook's law (\SeeChapter{see section Continuum Mechanics page \pageref{hooke law}}) in the following form:
	
	and the elastic potential energy of a spring proved in the section of Classical Mechanics is given by:
	
	Or in relative displacement:
	
	In the traction domain (or compression) of the beams, it is customary to consider the beam as a spring (...) and then to use the stiffness constant of Hook's law ... hoping that this is conforming to experimental observations...:
	
	Where, according to usage, we denote the longitudinal displacement $L$ instead of $x$. By injecting Hook's law (yes ... it turns a little bit a loop... but that's engineering...):
	
	It is then sufficient to divide by the length of the bar to have the elastic linear energy:
	
	The energy density, noted conventionally denoted as in thermodynamics by a minuscule letter, is then:
	
	
	\pagebreak
	\subsubsection{Torsion}\label{torsion}
	Let us recall first to the reader a study we made in the section on Classical Mechanics on the torsional pendulum where some elements had deliberately been muted.... Let's study this in more detail as it is very useful for transmission shafts or springs in everyday life.

	Let us consider for this a cylindrical wire fixed at its base and subjected to a torsion moment $\vec{M}$. Under the effect of this torsion moment, the upper face of the wire is offset by an angle $\theta$ with respect to the lower face, the material undergoing a torsion stress (or shearing $\tau$):
	\begin{figure}[H]
		\centering
		\includegraphics[scale=1]{img/engineering/wire_into_torsion.jpg}
		\caption{Wire under torsion}
	\end{figure}
	Let us imagine an extraction of the inside the wire an elementary tube of radius $r$, of thickness $\mathrm{d}r$, and let us observe the effect of the torsion on this tube unrolled (this will allow us an approximate approach of the concerned phenomenon):
	\begin{figure}[H]
		\centering
		\includegraphics[scale=1]{img/engineering/torsion_piece_wire_extraction.jpg}
		\caption[]{Extraction an unroll of an element of the wire}
	\end{figure}
	Let us look for a relation between torsion moment $\vec{M}$ and torsion angle $\theta$ that would therefore also apply to a beam as as below:
	\begin{figure}[H]
		\centering
		\includegraphics[scale=0.5]{img/engineering/rectangular_torsion_beam.jpg}
	\end{figure}
	For the unrolled tube, let us apply shear relations as proved and introduced in the section of Continuum Mechanics:
	
	but the figure shows that (the deformations being weak) to the first order in Taylor series (\SeeChapter{see section Sequences and Series page \pageref{usual maclaurin developments}}):
	
	hence:
	
	The elementary moment due to this force is by definition of the moment of force (torque):
	
	Since $\vec{r}$ and $\vec{F}$ are perpendicular:
	
	The total moment (torque) is then:
	The total moment (torque) is then equal to:
	
	therefore:
	
	We thus fall back the relation of the torsional pendulum that we pout during our study of the torsional pendulum in the section of Classical Mechanics:
	
	 with the difference that this time the constant $k$, the "\NewTerm{torsion constant}\index{torsion constant}" is explicit!!!!

	The numerator of the constant $k$ is named in the domain of the strength of materials the "\NewTerm{torsional rigidity}\index{torsional rigidity}" or "\NewTerm{torsional stiffness}\index{torsional stiffness}" and the constant $k$ itself is often referred to as "\NewTerm{shaft rigidity}\index{shaft rigidity}" instead of "torsional constant". In practice, the main aim is to find the numerical value of the following expression:
	
	since this will give the angular amplitude of the torsion.

	Let us see therefore a very important application to the compression spring of helical type (the approach is approximate again...) working in torsion.

	First, it must be realized that when a force is applied to the spring, the ends will rotate by a small angle $\theta$ (torsion) corresponding to the movement by a distance $x$ which itself corresponds to the narrowing of the spring (yes indeed! this length must be taken somewhere....).

	Given then a spring of external radius $R$ (or of diameter $D$), of shear modulus $G$, with a body diameter $d$ (diameter of the folded cylinder of which the spring is composed):
	\begin{figure}[H]
		\centering
		\includegraphics[scale=1]{img/engineering/helicoidal_spring.jpg}
		\caption{Spiral spring under load}
	\end{figure}
	For the analysis we will simply need to mix several of the relations demonstrated so far. In the first place, the torsion angle of a beam of length $L$ (length of the spring in this case!):
	
	With:
	
	and:
	
	Moreover, the moment of torsion is written:
	
	We thus arrive at:
	
	\begin{tcolorbox}[title=Remark,colframe=black,arc=10pt]
	It would be pretentious to claim to do with this section as well and also complete as the \textit{Statique} free French PDF of Nicolet Gaston Raymond that is a priori unrivaled in content and quality to this date (even compared to non-free books on the subject!). It is therefore strongly recommended to refer to it if the reader wants to drive full information about civil engineering (see the download section of the companion website).
	\end{tcolorbox}
	The ratio $M/\theta$, as well as for the drive shaft, is named the "\NewTerm{spring stiffness}\index{spring stiffness}" or  "\NewTerm{spring rigidity}\index{spring rigidity}".

	The displacement (deformation) $x$ is equal to (\SeeChapter{see section Trigonometry page \pageref{spherical trigonometry}}):
	
	We finally arrive at:
	
	Which brings us to the worldwide known relation in the world in the strength of material regading to springs:
	
	where $k$ is the "\NewTerm{spring stiffness}\index{spring stiffness}" constant also named "\NewTerm{spring rigidity}\index{spring rigidity}"! If now we use the expression of the elastic potential energy of a spring demonstrated in the section of Classical Mechanics:
	
	We can then determine the energy a spiral spring can absorb.
	
	\subsubsection{Buckling}
	We conclude this study of the material strengths in this book for now, with "\NewTerm{buckling}\index{buckling}" (classical study case in construction and mechanics) which consists in determining (in a particular simple case) the minimum force $F_0$ from which a bar of length $L$, of Young's modulus $E$ fixed at its two ends can fold (with a curvature radius $R$) until it breaks without the need to increase the force $F_0$ (this is again an indication value!).

	Buckling is characterized by a sudden sideways failure of a structural member subjected to high compressive stress, where the compressive stress at the point of failure is less than the ultimate compressive stress that the material is capable of withstanding. Mathematical analysis of buckling often makes use of an "artificial" axial load eccentricity that introduces a secondary bending moment that is not a part of the primary applied forces being studied. As an applied load is increased on a member, such as a column, it will ultimately become large enough to cause the member to become unstable and is said to have buckled. Further load will cause significant and somewhat unpredictable deformations, possibly leading to complete loss of the member's load-carrying capacity. If the deformations that follow buckling are not catastrophic the member will continue to carry the load that caused it to buckle. If the buckled member is part of a larger assemblage of components such as a building, any load applied to the structure beyond that which caused the member to buckle will be redistributed within the structure.
Theoretically, buckling is caused by a bifurcation in the solution to the equations of static equilibrium. At a certain stage under an increasing load, further load is able to be sustained in one of two states of equilibrium: a purely compressed state (with no lateral deviation) or a laterally-deformed state.

	The ratio of the effective length of a column to the least radius of gyration of its cross section is named the "\NewTerm{slenderness ratio $\lambda$}\index{slenderness ratio }". This ratio affords a means of classifying columns. Slenderness ratio is important for design considerations. All the following are approximate values used for convenience.
	\begin{itemize}
		\item A "\NewTerm{columns!short steel column}\index{short steel column}" is one whose slenderness ratio does not exceed $50$; an intermediate length steel column has a slenderness ratio ranging from about $50$ to $200$, and its behavior is dominated by the strength limit of the material, while a long steel column may be assumed to have a slenderness ratio greater than $200$ and its behavior is dominated by the modulus of elasticity of the material.
	
		\item A "\NewTerm{short concrete column}\index{columns!short concrete column}" is one having a ratio of unsupported length to least dimension of the cross section equal to or less than $10$. If the ratio is greater than $10$, it is considered as a "\NewTerm{long column}\index{columns!long column}" (sometimes referred to as a "\NewTerm{slender column}\index{slender column}).
	
		\item "\NewTerm{Timber}\index{columns!timber}" columns may be classified as short columns if the ratio of the length to least dimension of the cross section is equal to or less than $10$. The dividing line between intermediate and long timber columns cannot be readily evaluated. One way of defining the lower limit of long timber columns would be to set it as the smallest value of the ratio of length to least cross sectional area that would just exceed a certain constant of the material.
	\end{itemize}
	If the load on a column is applied through the center of gravity (centroid) of its cross section, it is named an "\NewTerm{axial load}\index{axial load}". A load at any other point in the cross section is known as an "\NewTerm{eccentric load}\index{eccentric load}":
	\begin{figure}[H]
		\centering
		\includegraphics[scale=1]{img/engineering/buckling.jpg}
		\caption[Column under a concentric axial load exhibiting the characteristic deformation of buckling]{Column under a concentric axial load exhibiting the characteristic deformation of buckling (source: Wikipedia)}
	\end{figure}
	A short column under the action of an axial load will fail by direct compression before it buckles, but a long column loaded in the same manner will fail by buckling (bending), the buckling effect being so large that the effect of the axial load may be neglected.
	\begin{figure}[H]
		\centering
		\includegraphics[scale=0.5]{img/engineering/buckling_in_real_life.jpg}
		\caption{Real life buckling (not to be confus with "crashing"!)}
	\end{figure}
	 The intermediate-length column will fail by a combination of direct compressive stress and bending.

	In 1757, mathematician Leonhard Euler derived a relation that gives the maximum axial load that a long, slender, ideal column can carry without buckling. An ideal column is one that is perfectly straight, homogeneous, and free from initial stress (see proof further below). The maximum load, sometimes named the "\NewTerm{critical load}\index{critical load}", causes the column to be in a state of unstable equilibrium; that is, the introduction of the slightest lateral force will cause the column to fail by buckling. However, if lateral forces are taken into consideration the value of critical load remains approximately the same.

	For the study of this phenomenon, we consider that as soon as the bar begins to bend we have then $F_0$ (and we are then very far from the force allowing to break it):
	\begin{figure}[H]
		\centering
		\includegraphics[scale=1]{img/engineering/buckling_study.jpg}
		\caption{Example of buckling}
	\end{figure}
	When the bar begins to bend we then have a force $\vec{F}_0$ that applies to each volume element of volume of the bar but since these are not distributed in the same way along the $z$ axis, they do not create the same moment of force (torque)!

	At the equilibrium of the buckling force, the bar is subjected to a moment of recall. We then have:
	
	Expressing the moment of bending $M$ by means of the relation (see beginning of this section)
	
	It comes:
	
	Using the equation of the elastic line and substituting, we get:	
	
	thus:
	
	Which is the "\NewTerm{buckling differential equation}\index{buckling differential equation}" for calculating the buckling force with the initial conditions that are:
	
	Let us indicate that the relation:
	
	If often written in the following form in text books:
	
	The resolution of the second order differential equation:
	
	is relatively easy (\SeeChapter{see section Differential and Integral Calculus page \pageref{second order differential equations}}) since the characteristic equation is:
	
	We then have the homogeneous solution:
	
	The condition  $y=0\Rightarrow z=0$ imposes:
	
	Then it comes:
	
	So it comes immediately that:
	
	Hence the famous relation given in many text books without proof:
	
	This relation is sometimes referred to as the "\NewTerm{Euler's formula}\index{Euler's formula}" (not to be confused with the formula of the same name proved in the section of Graph Theory) and the load limits the "\NewTerm{Euler's critical load}\index{Euler's critical load}" for a beam perfectly embedded at the ends. The whole study being the "\NewTerm{Euler's buckling}".
	
	Examination of this formula reveals the following interesting facts with regard to the load-bearing ability of slender columns:
	 \begin{enumerate}
		\item Elasticity $E$ and not the compressive strength $\sigma$ of the materials of the column determines the critical load.

		\item The critical load is directly proportional to the second moment of area of the cross section.

		\item The boundary conditions have a considerable effect on the critical load of slender columns. The boundary conditions determine the mode of bending and the distance between inflection points on the deflected column. The inflection points in the deflection shape of the column are the points at which the curvature of the column change sign and are also the points at which the internal bending moments are zero. The closer together the inflection points are, the higher the resulting capacity of the column.
	\end{enumerate}
	
	In the relation above it is physically logic that $k$, named the "\NewTerm{column length factor}\index{column length factor}", to consider the following cases:
	\begin{itemize}
		\item $k=1$: Both ends pinned (hinged, free to rotate)
		\item $k=0.5$: Both ends fixed
		\item $k\cong 0.7071$: One end fixed and the other end pineed
		\item $k=2.0$: One end fixed and the other end free to move laterally
	\end{itemize}
	Now the reader should know that in structural dynamics sometimes we rewrite:
	
	as:
	
	where $r_g$ is the classical radius of gyration (\SeeChapter{see section Classical Mechanics page \pageref{radius of gyration}}). That is to say we consider that for all beam that we are able to concentrate all the mass at the location of radius of gyration so that we get the same effect and satisfy the same equation of dynamics as that of a complex body.

	For example if we were trying to study the equation of motion of a cylinder about it's major axis then instead of developing complex equations you can rather concentrate all the mass at a point located at the distance calculated from above formula and still get same result.

	So now if we rewrite Euler's formular using this radius:
	
	we see the slenderness ratio that we have introduced earlier above that appears:
	
	
	\pagebreak
	\paragraph{Self-buckling}\mbox{}\\\\
	A column can buckle due to its own weight with no other direct forces acting on it, in a failure mode named "\NewTerm{self-buckling}\index{self-buckling}\label{self-buckling}". In conventional column buckling problems, the self-weight is often neglected since it is assumed to be small when compared to the applied axial loads. However, when this assumption is not valid, it is important to take the self-buckling into account.
	
	One interesting example for the use of the equation was suggested by A. G. Greenhill in his paper (1881) after a request of the son of Horace Darwin (son of the famous Charles Darwin). The request aw estimated the maximal height of a pine tree.
	
	For this study, let us suppose a uniform column fixed in a vertical direction at its lowest point, and carried to a height $l$, in which the vertical position becomes unstable and flexure begins:
	\begin{figure}[H]
		\centering
		\includegraphics[scale=0.6]{img/engineering/self_buckling.jpg}
		\caption[Column exhibiting a compressive buckling load due to its own weight]{Column exhibiting a compressive buckling load due to its own weight (source: Wikipedia)}
	\end{figure}
	The reader can notice that we take the origin O at the top of the pole in its vertical position and the axis $\text{O}x$ directed vertically downward!
	
	There is a body force $f$ per unit length ($f=F/d$):
	
	where $S$ is the cross-sectional area of the column, $g$ is obivously the acceleration due to gravity and $\rho$  is its mass density.

	The column is slightly curved under its own weight, so the curve $w(x)$ describes the deflection of the beam in the $y$ direction at some position $x$.
	
	Looking at any point on the column, we can write the moment equilibrium:
	
	where the right-hand side of the equation is sum of all moment of the weight of $\overline{BP}$ about the point $P$ or coordinate $y$.

	According to Euler–Bernoulli beam theory (see earlier above):
	
	Where $E$ is the Young's modulus of elasticity of the substance, $I$ is the moment of inertia.

	Therefore, the differential equation of the central line of $\overline{BP}$ is:
	
	Differentiating with respect to $x$, we get:
	
	Or:
	
	That is:
	
	We get that the governing equation is the third order linear differential equation with a variable coefficient. 
	
	The way to solve the problem is to use new variables in a very clever way. First we put:
	
	Therefore:
	
	And we multiple left and right by $x^2$ and we divide by $EI$ to get:
	
	We put now $p=\sqrt{x}z$ then:
	
	That is:
	
	and:
	
	we divide by $\sqrt{x}$:
	
	We simplify a bit:
	
	Therefore:
	
	Finally:
	
	Now, we put $x=r^{2/3}$ and then:
	
	Therefore:
	
	By derivating relatively to $x$ the expression above we get:
	
	But we remember that $r=x^{3/2}$ and therefore that:
	
	We replace this $\mathrm{d}r/\mathrm{d}x$ we have just get into the prior previous relation and we get:
	
	Finally the two terms:
	
	becomes after having replace $\mathrm{d}^2z/\mathrm{d}x^^2$ and $\mathrm{d}z/\mathrm{d}x$ by the expression we just get:
	
	Therefore:
	
	Can be written:
	
	After rearranging:
	
	This is of the form of Bessel's differential equation
	
	Let us now solve that the Bessel equation
	
	with $x(0)=0$.
	
	Now the first step is to consider a solution $y(x)$ expanded in the generalized power series in the vicinity of $x(0)=0$ :
	
	with $a_0\neq 0$ and $\left|x\right|\le \infty$.
	
	The second step is to use:
	
	and also:
	
	Third step is to inject this both relations in the original:
	
	gives:
	
	To continue we use:
	
	and therefore we can simplify:
	
	Now we rearrange in a very clever way (I'm always surprise that some people get such ideas...):
	
	with $a_0\neq 0$.
	
	Now to get this equal to zero a trivial solution is that each facto is equal to zero such that:	
	
	That last relation can be rewritten as:
	
	with obviously $n=2,3,\ldots$.
	
	The characteristic equation is of the first factor:
	
	of the set of three equations is obviously:
	
	If we consider this solution, it means that we must accept that $a_1=0$ for the second relation to be also equal to zero. The same reasoning apply for the third relation and therefore we must have $a_{2k+1}=0$  $k = 0,1,2,\ldots$!
	
	If we put the positive solution into $a_n$, we get:
	
	still with the conditions that $n=2,3,\ldots$ and $a_0\neq 0$.
	
	The denominator of this last can be rewritten. First let us develop:
	
	and now we put $n=2k$, therefore the previous expression can be written:
	
	Therefore:
	
	still with the condition that $a_0\neq 0$ and now with $k=1,2,3,\ldots$.
	
	Now let us write a few recursive developments of the previous relation:
	
	still with $k=1,2,\dots$ and $a_0\neq 0$.
	
	So finally for summary we have:
	
	with $k=1,2,\dots$ and $a_0\neq 0$ and:
	
	with $k=0,1,2,\ldots$.
	
	So for the first partial solution $(\rho=1/3)$ we have:
	
	
	Using the same reasoning but with $\rho=-1/3$ we get:
	
	
	The general solution is then the sum of the both solution, that:
	
	We can extract of the both constants a factor $2^{-\frac{1}{3}}$ to be able to write:
	
	That is:
	
	The solution of the transformed equation is:
	
	Where $J_n$ is the Bessel function of the first kind. Then, the solution of the original equation is if we remember that:
	
	Then:
	
	Now, we will use the boundary conditions. 
	
	\begin{itemize}
		\item As we consider no moment at $x=0$ (since there is no material above this point), then:
		
		So to see what must be the values of the constants $A$ and $B$ let us rewrite the stuff explicitly:
		
		Therefore:
		
		Therefore we see quickly that when $x=$ everything vanish expect a term $1/2AC_1$. Therefore this first condition impose $A=0$.
	
		\item Since it is fixed at $x=l$, we have  $w(x=l)=0$, and therefore also $\mathrm{d}w/\mathrm{d}x=0$, thus:
		
		And after a small simplification this reduce to:
		
		If we denote by $c$ the first root of $J_{-\frac {1}{3}}$, that means:
		
		with:
		
	\end{itemize}
	Therefore from the second boundary condition, we can get the critical length in which a vertical column will buckle!
	
	Indeed, from the first that we have chosen to denote $c$, we get:
	
	as for recall:
	
	Therefore:
	
	Now with Maple 17.00 we can determiner the smallest root of $J_{-1/3}$ to get $c$:
	
	\texttt{>evalf(BesselJ(-1/3, 1));}
	
	that gives $0.6068875051$. 
	
	Therefore:
	
	Hence:
	
	This done let us do a numerical application!
	
	\begin{tcolorbox}[colframe=black,colback=white,sharp corners]
	\textbf{{\Large \ding{45}}Example:}\\\\
	For a column of steel (Structural ASTM-A36) of radius $r=0.1$ [m] we have $E=2.1\cdot 10^{11}\;[\text{N}\cdot\text{m}^{-2}]$ (some steel go to $E=4.8\cdot 10^{11}\;[\text{N}\cdot\text{m}^{-2}]$) and:
	
	and $f=2481\;[\text{N}\cdot \text{m}^{-1}]$. Therefore:
	
	With a square column of $55\times 132$ [cm] side (as biggest World Trade Center building beams in New York) we would get as $I=bh^3/12$ the value $x_{\max}\cong 30.13$[m].
	\end{tcolorbox}
	So what can we conclude so far:
	\begin{enumerate}
		\item The development above has for now not been peer-reviewed so he may contain errors. We have compared with different sources that have a slightly different result but they all have missing details that avoid us to make a strict comparison.
		
		\item We don't know any laboratory that can confirm the value that we get and some people on Internet with different approach get $80$ meters (seems a priori more accurate than our result) or a few kilometers...
	\end{enumerate}
	
	\subsubsection{Traction}
	Let us now consider the case of a bar hanging only to its own weight. The surface of its circular cross-section is $S$ and $h$ the total height of this bar. The Young's modulus of its material is denoted $E$ (\SeeChapter{see section of Continuum Mechanics page \pageref{young modulus}}) and its density $\rho$.

	It is easy to see that a section situated at an altitude $z$ supports the weight of the piece of bar under it:
	
	The constraint is therefore not constant in the bar:
	
	And the deformation either:
	
	where $z$ is the abscissa on the bar, the inhomogeneous deformation is related to the displacement by the relation:
	
	After integration, we get the general form of the displacement:
	
	where the constant is to be determined using any bonding conditions at the ends of the bar. If the upper end is embedded, the displacement there is therefore zero:
	
	The displacement at any point of the bar is thus expressed by:
	
	The elongation of the bar is the difference in displacement between the two ends of the bar:
	
	We then have trivially:
	

	
	\begin{flushright}
	\begin{tabular}{l c}
	\circled{20} & \pbox{20cm}{\score{2}{5} \\ {\tiny 32 votes,  78.75\%}} 
	\end{tabular} 
	\end{flushright}

	%to make section start on odd page
	\newpage
	\thispagestyle{empty}
	\mbox{}
	\section{Electrical Engineering}	
	\lettrine[lines=4]{\color{BrickRed}W}e will see in this section the study - under its mathematical aspect of sure - of general electronic circuits, chips, and electronic machinery, transmitter / receivers that the engineer must know formalize, analyze, understand, build and simulate after his studies . For this reason we have chosen in this book to focus - as always - on mathematical case of practical (useful!)  applications In the every days life, mentioning the potential pitfalls and dangers of the electronic assembly when necessary.
	
	We will deal here first with analogic electronics, then power electronics (electrotechnic), digital electronics and of the physics of semiconductor  to understand the foundations of some components.
	
	Electrical engineering is therefore hierarchy of models. This is the only way to approach the design of complex systems. In principle, the operation of many devices can certainly always be reduced to the application of Maxwell's equations (\SeeChapter{see section Electrodynamics page \pageref{maxwell equations}}) or Schrödinger Equation for more complicated case, however it is humanly impossible to understand the design of some systems by staying at this theoretical level.
	
	It is therefore customary in the industry to perform the analysis in 5 stages:
	\begin{enumerate}
		\item[L0.] Solid State Physics: This model is essential for the analysis of electric and magnetic properties of matter. It is based on the laws of quantum physics and essentially leads to the description of energy bands and to the calculation of their degree of occupation. This model explains for example the fundamental properties of semiconductors.
		
		\item[L1.] Electromagnetism: This model is essential for the analysis of devices working at microwave frequencies and the electromagnetic devices. It is based on Maxwell's relations and uses the mathematical theory of partial differential equations. This model don't allows anymore to analyze the influence of an atom as the studied objects are at a macroscopic level, described by their dimensions, their permittivity, their conductivity, etc.
		
		\item[L2.] Circuit Theory: This model is essential for the analysis of electronic devices in the very common case where the dimensions of the device are well below the wavelength of the phenomenon studied. This model is based on Kirchhoff's lemma and the definition of a half-dozen of discrete elements, resistance, capacitance, inductance, etc. There is nore more geometries  in such a model but only a topologic (structural) approach. One can calculate the current and voltage, scalar quantities, and the fields have most of time meaning anymore at this scale. Mathematical techniques are those of ordinary differential equations, Laplace transformations, complex computation and matrices, etc.
		
		\item[L3.] Block Diagrams: At this level, we do not take into account currents or voltages, and we don't care about the geometry (topology) of the system. This is constituted by the block connections doing functions/works characterized by the relations between output and input variables.
		
		\item[L4.] Systems: At this level, we schematize as a functional block a set of blocks of level 3. A computer is such an interconnection of various logical systems each performing a particular function.
		
		\item[L5.] Software: From this level, the engineer don't add any more additional devices, don't combines them anymore into larger systems, but program the machine. The theoretical methods then are usually closer to the linguistic than to mathematics.
	\end{enumerate}
	
	\subsection{Elementary Primitive Electrical Symbols}	
	This section will help you become familiar with the language and some methods of calculation used by engineers of the industry for the first 4 levels. However and strictly speaking, any training in this area should be completed by practical laboratory work!
	Sadly still in 2016 there are several national and international standards for graphical symbols in circuit diagrams, in particular:
	\begin{itemize}
		\item IEC 60617 (also known as British Standard BS 3939).
		\item ANSI standard Y32.2 (also known as IEEE Std 315).
		\item IEEE Std 91/91a: graphic symbols for logic functions (used in digital electronics). It is referenced in ANSI Y32.2/IEEE Std 315.
Australian Standard AS 1102.
	\end{itemize}
	
	We have tried to respect at best in diagrams the norm NF EN 60617 for electrical components symbols. As they are very numerous, we propose below a list only of the components used or mentioned so far in this book. This table will therefore evolve over time!
	
	First some generators:
	\begin{figure}[H]
		\vspace{1cm}
		% First row	
		\begin{center}
		\begin{picture}(50,50)(0,0)
		\vst{25,10}{E}
		\cn{25,10}
		\tx{25,0}{vst}
		\end{picture}
		%
		\begin{picture}(50,50)(0,0)
		\vdst{25,10}{E}
		\cn{25,10}
		\tx{25,0}{vdst}
		\end{picture}
		%
		\begin{picture}(50,50)(0,0)
		\vba{25,10}{E}
		\cn{25,10}
		\tx{25,0}{vba}
		\end{picture}
		%
		\begin{picture}(50,50)(0,0)
		\vdba{25,10}{E}
		\cn{25,10}
		\tx{25,0}{vdba}
		\end{picture}
		%
		\begin{picture}(50,50)(0,0)
		\vj{25,10}{J}
		\cn{25,10}
		\tx{25,0}{vj}
		\end{picture}
		%
		\begin{picture}(50,50)(0,0)
		\vdj{25,10}{J}
		\cn{25,10}
		\tx{25,0}{vdj}
		\end{picture}
		%
		\begin{picture}(50,50)(0,0)
		\vln{25,10}{50}
		\cn{25,10}
		\tx{25,0}{vln}
		\end{picture}
		\end{center}
	\end{figure}
	
	\begin{figure}[H]
		\begin{center}
		\begin{picture}(50,50)(0,0)
		\hst{0,5}{E}
		\cn{0,5}
		\tx{25,-25}{hst}
		\end{picture}
		%
		\begin{picture}(50,50)(0,0)
		\hlst{0,5}{E}
		\cn{0,5}
		\tx{25,-25}{hlst}
		\end{picture}
		%
		\begin{picture}(50,50)(0,0)
		\hba{0,5}{E}
		\cn{0,5}
		\tx{25,-25}{vba}
		\end{picture}
		%
		\begin{picture}(50,50)(0,0)
		\hlba{0,5}{E}
		\cn{0,5}
		\tx{25,-25}{vdba}
		\end{picture}
		%
		\begin{picture}(50,50)(0,0)
		\hj{0,5}{E}
		\cn{0,5}
		\tx{25,-25}{vj}
		\end{picture}
		%
		\begin{picture}(50,50)(0,0)
		\hlj{0,5}{E}
		\cn{0,5}
		\tx{25,-25}{vdj}
		\end{picture}
		%
		\begin{picture}(50,50)(0,0)
		\hln{0,5}{50}
		\cn{0,5}
		\tx{25,-25}{vln}
		\end{picture}
		\\[2cm]
		\end{center}
	\end{figure}
	
	\begin{figure}[H]
		\vspace{1cm}
		\begin{center}
		\begin{picture}(50,50)(0,0)
		\ve{25,10}{E}
		\cn{25,10}
		\tx{25,0}{ve}
		\end{picture}
		%
		\begin{picture}(50,50)(0,0)
		\vde{25,10}{E}
		\cn{25,10}
		\tx{25,0}{vde}
		\end{picture}
		%
		\begin{picture}(50,50)(0,0)
		\he{0,35}{E}
		\cn{0,35}
		\tx{25,0}{he}
		\end{picture}
		%
		\begin{picture}(50,50)(0,0)
		\hle{0,35}{E}
		\cn{0,35}
		\tx{25,0}{hle}
		\end{picture}
		%
		\begin{picture}(50,50)(0,0)
		\vcvs{25,10}{Au}
		\cn{25,10}
		\tx{25,0}{vcvs}
		\end{picture}
		%
		\begin{picture}(50,50)(0,0)
		\vccs{25,10}{gu}
		\cn{25,10}
		\tx{25,0}{vccs}
		\end{picture}
		%
		\begin{picture}(50,50)(0,0)
		\vnst{25,10}{e}
		\cn{25,10}
		\tx{25,0}{vnst}
		\end{picture}
		\\[2cm]
		\end{center}
	\end{figure}
	
	And now some passive elements (impedance):
	\begin{figure}[H]
		\vspace{1cm}
		\begin{center}
		\begin{picture}(50,50)(0,0)
		\vz{25,10}{Z}
		\cn{25,10}
		\tx{25,0}{vz}
		\end{picture}
		%
		\begin{picture}(50,50)(0,0)
		\vc{25,10}{C}
		\cn{25,10}
		\tx{25,0}{vc}
		\end{picture}
		%
		\begin{picture}(50,50)(0,0)
		\vl{25,10}{L}
		\cn{25,10}
		\tx{25,0}{vl}
		\end{picture}
		%
		\begin{picture}(50,50)(0,0)
		\vlr{25,10}{L}
		\cn{25,10}
		\tx{25,0}{vlr}
		\end{picture}
		%
		\begin{picture}(50,50)(0,0)
		\vx{25,10}{X}
		\cn{25,10}
		\tx{25,0}{vx}
		\end{picture}
		%
		\begin{picture}(50,50)(0,0)
		\vr{25,10}{R}
		\cn{25,10}
		\tx{25,0}{vr}
		\end{picture}
		%
		\begin{picture}(50,50)(0,0)
		\veng{25,10}{M}
		\cn{25,10}
		\tx{25,0}{veng}
		\end{picture}
		\end{center}
	\end{figure}
	
	\begin{figure}[H]
		\begin{center}
		\begin{picture}(50,50)(0,0)
		\hz{0,5}{Z}
		\cn{0,5}
		\tx{25,-25}{hz}
		\end{picture}
		%
		\begin{picture}(50,50)(0,0)
		\hc{0,5}{C}
		\cn{0,5}
		\tx{25,-25}{hc}
		\end{picture}
		%
		\begin{picture}(50,50)(0,0)
		\hl{0,5}{L}
		\cn{0,5}
		\tx{25,-25}{hl}
		\end{picture}
		%
		\begin{picture}(50,50)(0,0)
		\hld{0,5}{L}
		\cn{0,5}
		\tx{25,-25}{hld}
		\end{picture}
		%
		\begin{picture}(50,50)(0,0)
		\hx{0,5}{X}
		\cn{0,5}
		\tx{25,-25}{hx}
		\end{picture}
		%
		\begin{picture}(50,50)(0,0)
		\hr{0,5}{R}
		\cn{0,5}
		\tx{25,-25}{hr}
		\end{picture}
		%
		\begin{picture}(50,50)(0,0)
		\hrold{0,5}{R}
		\cn{0,5}
		\tx{25,-25}{hrold}
		\end{picture}
		\\[2cm]
		\end{center}
	\end{figure}
	
	And some transformators:	
	\begin{figure}[H]
		\vspace{1cm}
		\begin{center}
		\begin{picture}(50,50)(0,0)
		\vmlc{0,10}
		\cn{0,10}
		\tx{20,0}{vmlc}
		\end{picture}
		%
		\begin{picture}(50,50)(0,0)
		\vml{0,10}{L_1}{L_2}{M}
		\cn{0,10}
		\tx{25,0}{vml}
		\end{picture}
		%
		\begin{picture}(70,50)(-10,0)
		\vt{0,10}{L_1}{L_2}{M}
		\cn{0,10}
		\tx{25,0}{vt}
		\end{picture}
		%
		\begin{picture}(50,50)(0,0)
		\vm{0,10}{L_1}{L_2}{M}
		\cn{0,10}
		\tx{25,0}{vm}
		\end{picture}
		%
		\begin{picture}(70,50)(-10,0)
		\vlm{0,10}{M}{L_1}{L_2}
		\cn{0,10}
		\tx{25,0}{vlm}
		\end{picture}
		%
		\begin{picture}(50,50)(0,0)
		\vlmi{0,10}{M}{L_1}{L_2}
		\cn{0,10}
		\tx{25,0}{vlmi}
		\end{picture}\hfill\\[1cm]
		%
		\end{center}
	\end{figure}
	
	Some domestic elements:
	\begin{figure}[H]
		\vspace{1cm}
		\begin{center}
		\begin{picture}(50,50)(0,0)
		\vlamp{25,10}{L}
		\cn{25,10}
		\tx{25,0}{vlamp}
		\end{picture}
		%
		\begin{picture}(50,50)(0,0)
		\hso{0,35}{t=0}
		\cn{0,35}
		\tx{25,0}{hso}
		\end{picture}
		%
		\begin{picture}(50,50)(0,0)
		\hsc{0,35}{S}
		\cn{0,35}
		\tx{25,0}{hsc}
		\end{picture}
		%
		\begin{picture}(50,50)(0,0)
		\vso{25,10}{S}
		\cn{25,10}
		\tx{25,0}{vso}
		\end{picture}
		\end{center}
	\end{figure}
	
	Some measurement elements:
	\begin{figure}[H]
		\vspace{1cm}
		\begin{center}
		\begin{picture}(50,50)(0,0)
		\vavo{25,10}{A}{I}
		\cn{25,10}
		\tx{25,0}{vavo}
		\end{picture}
		%
		\begin{picture}(50,50)(0,0)
		\havo{0,35}{V}{U}
		\cn{0,35}
		\tx{25,0}{havo}
		\end{picture}
		%
		\begin{picture}(50,50)(0,0)
		\vpm{25,35}{W}{P}
		\cn{25,35}
		\tx{25,0}{vpm}
		\end{picture}\\[1cm]
		\end{center}
	\end{figure}
	
	Some equipotentials:
	\begin{figure}[H]
		\vspace{1cm}
		\begin{center}
		\begin{picture}(50,50)(0,0)
		\uu{25,10}{U}
		\cn{25,10}
		\tx{25,0}{uu}
		\end{picture}
		%
		\begin{picture}(50,50)(0,0)
		\du{25,10}{U}
		\cn{25,10}
		\tx{25,0}{du}
		\end{picture}
		%
		\begin{picture}(50,50)(0,0)
		\ru{0,35}{U}
		\cn{0,35}
		\tx{25,0}{ru}
		\end{picture}
		%
		\begin{picture}(50,50)(0,0)
		\lu{0,35}{U}
		\cn{0,35}
		\tx{25,0}{lu}
		\end{picture}
		%
		\begin{picture}(50,50)(0,0)
		\dcru{25,10}{U}
		\cn{25,10}
		\tx{25,0}{dcru}
		\end{picture}
		%
		\begin{picture}(50,50)(0,0)
		\ucru{25,10}{U}
		\cn{25,10}
		\tx{25,0}{ucru}
		\end{picture}
		%
		\begin{picture}(50,50)(0,0)
		\lcuu{0,35}{U}
		\cn{0,35}
		\rcuu{0,10}{U}
		\cn{0,10}
		\tx{25,0}{lcuu}
		\tx{25,-10}{rcuu}
		\end{picture}
		\end{center}
		
		%10th row
		Some current nodes:
		\begin{center}
		\begin{picture}(50,50)(0,0)
		\ri{25,25}{I}
		\cn{25,25}
		\tx{25,0}{ri}
		\end{picture}
		%
		\begin{picture}(50,50)(0,0)
		\li{25,25}{I}
		\cn{25,25}
		\tx{25,0}{li}
		\end{picture}
		%
		\begin{picture}(50,50)(0,0)
		\ui{25,25}{I}
		\cn{25,25}
		\tx{25,0}{ui}
		\end{picture}
		%
		\begin{picture}(50,50)(0,0)
		\di{25,25}{I}
		\cn{25,25}
		\tx{25,0}{di}
		\end{picture}
		%
		\begin{picture}(50,50)(0,0)
		\rui{25,25}{I}
		\cn{25,25}
		\tx{25,0}{rui}
		\end{picture}
		%
		\begin{picture}(50,50)(0,0)
		\hcj{25,25}
		\cn{25,25}
		\tx{25,0}{hcj}
		\end{picture}
		%
		\begin{picture}(50,50)(0,0)
		\vcj{25,25}
		\cn{25,25}
		\tx{25,0}{vcj}
		\end{picture}\\[2cm]
		\end{center}
	\end{figure}
	
	Some ground elements and bipoints measurements:
	\begin{figure}[H]
		\vspace{1cm}
		\begin{center}
		\begin{picture}(50,50)(0,0)
		\hg{25,25}
		\cn{25,25}
		\tx{25,0}{hg}
		\end{picture}
		%
		\begin{picture}(50,50)(0,0)
		\vg{25,25}
		\cn{25,25}
		\tx{25,0}{vg}
		\end{picture}
		%
		\begin{picture}(50,50)(0,0)
		\hgp{25,25}
		\cn{25,25}
		\tx{25,0}{hgp}
		\end{picture}
		%
		\begin{picture}(50,50)(0,0)
		\hgpu{25,25}
		\cn{25,25}
		\tx{25,0}{hgpu}
		\end{picture}
		%
		\begin{picture}(50,50)(0,0)
		\hp{25,25}{a}{b}
		\cn{25,25}
		\tx{25,0}{hp}
		\end{picture}
		%
		\begin{picture}(50,50)(0,0)
		\vp{25,50}{a}{b}
		\cn{25,50}
		\tx{50,0}{vp}
		\end{picture}\\[2cm]
		\end{center}
	\end{figure}
	
	Various diodes (simple and capacitive Diodes, Zener diodes, Schottky diodes):
	\begin{figure}[H]
		\vspace{0.5cm}
		\begin{center}
		\begin{picture}(50,50)(0,0)
		\ud{25,25}{D}
		\cn{25,25}
		\tx{25,0}{ud}
		\end{picture}
		%
		\begin{picture}(50,50)(0,0)
		\dd{25,25}{D}
		\cn{25,25}
		\tx{25,0}{dd}
		\end{picture}
		%
		\begin{picture}(50,50)(0,0)
		\rd{0,50}{D}
		\cn{0,50}
		\tx{25,0}{rd}
		\end{picture}
		%
		\begin{picture}(50,50)(0,0)
		\ld{0,50}{D}
		\cn{0,50}
		\tx{25,0}{ld}
		\end{picture}
		%
		\begin{picture}(50,50)(0,0)
		\rdb{0,50}{D}
		\cn{0,50}
		\tx{25,0}{rdb}
		\end{picture}
		%
		\begin{picture}(50,50)(0,0)
		\rde{0,50}{D}
		\cn{0,50}
		\tx{25,0}{rde}
		\end{picture}
		%
		\begin{picture}(50,50)(0,0)
		\vrd{0,50}{D}
		\cn{0,50}
		\tx{25,0}{vrd}
		\end{picture}
		\\[2cm]
		\end{center}
	\end{figure}
	
	\begin{figure}[H]
		\vspace{0.5cm}
		\begin{center}
		\begin{picture}(50,50)(0,0)
		\zud{25,25}{Z}
		\cn{25,25}
		\tx{25,0}{zud}
		\end{picture}
		%
		\begin{picture}(50,50)(0,0)
		\zdd{25,25}{Z}
		\cn{25,25}
		\tx{25,0}{zdd}
		\end{picture}
		%
		\begin{picture}(50,50)(0,0)
		\srd{0,50}{S}
		\cn{0,50}
		\tx{25,0}{srd}
		\end{picture}
		%
		\begin{picture}(50,50)(0,0)
		\sld{0,50}{S}
		\cn{0,50}
		\tx{25,0}{sld}
		\end{picture}
		\end{center}
	\end{figure}
	
	Some diode transistors:
	\begin{figure}[H]
		\begin{center}
		\begin{picture}(50,50)(0,0)
		\snpn{0,50}{T}
		\cn{0,50}
		\tx{25,0}{snpn}
		\end{picture}
		%
		\begin{picture}(50,50)(0,0)
		\npnb{25,25}{T}
		\cn{25,25}
		\tx{25,0}{npn}
		\end{picture}
		%
		\begin{picture}(50,50)(0,0)
		\pnpb{25,25}{T}
		\cn{25,25}
		\tx{25,0}{pnp}
		\end{picture}
		%
		\end{center}
	\end{figure}
	
	\begin{figure}[H]
		\vspace{2cm}
		\begin{center}
		\begin{picture}(50,50)(0,0)
		\npnc{0,50}{T}
		\cn{0,50}
		\tx{25,0}{npnc}
		\end{picture}
		%
		\begin{picture}(50,50)(0,0)
		\pnpc{0,50}{T}
		\cn{0,50}
		\tx{25,0}{pnpc}
		\end{picture}
		%
		\begin{picture}(50,50)(0,0)
		\pnpcr{50,50}{T}
		\cn{50,50}
		\tx{25,0}{pnpcr}
		\end{picture}
		%
		\begin{picture}(50,50)(0,0)
		\npr{50,50}{T}
		\cn{50,50}
		\tx{25,0}{npr}
		\end{picture}
		%
		\begin{picture}(50,50)(0,0)
		\pnr{50,50}{T}
		\cn{50,50}
		\tx{25,0}{pnr}
		\end{picture}
		\begin{picture}(50,50)(0,0) % + 5 for a little space at the top of the page
		\npnnc{0,50}{T}
		\cn{0,50}
		\tx{25,0}{npnnc}
		\end{picture}
		\end{center}
	\end{figure}
	where he circle on the gate indicates that the transistor is in "depletion mode".  You can think of a "normally-on" switch where applying voltage to the gate turns OFF the channel conductance.  
	
	Conversely the symbol without the circle is "enhancement mode".
Equivalent to a "normally-off" switch where applying voltage to the gate turns ON the channel conductance.
	\begin{figure}[H]
		\vspace{1cm}
		\begin{center}
		\begin{picture}(50,50)(0,0)
		\npnwoc{0,50}{T}
		\cn{0,50}
		\tx{25,0}{npnwoc}
		\end{picture}
		%
		\begin{picture}(50,50)(0,0)
		\pnpwoc{0,50}{T}
		\cn{0,50}
		\tx{25,0}{pnpwoc}
		\end{picture}
		%
		\begin{picture}(50,50)(0,0)
		\npnwocu{0,50}{T}
		\cn{0,50}
		\tx{25,0}{npnwocu}
		\end{picture}
		%
		\begin{picture}(50,50)(0,0)
		\pnpc{0,50}{T}
		\cn{0,50}
		\tx{25,0}{pnpc}
		\end{picture}
		%
		\begin{picture}(50,50)(0,0)
		\nprwoc{50,50}{T}
		\cn{50,50}
		\tx{25,0}{nprvoc}
		\end{picture}
		%
		\begin{picture}(50,50)(0,0)
		\nprwocu{50,50}{T}
		\cn{50,50}
		\tx{25,0}{nprwocu}
		\end{picture}\\[2cm]
		\end{center}
	\end{figure}

	\begin{figure}[H]
		\begin{center}
		\begin{picture}(50,50)(0,0)
		\enmosna{0,50}{F}
		\cn{0,50}
		\tx{25,0}{enmosna}
		\end{picture}
		%
		\begin{picture}(50,50)(0,0)
		\enmos{0,50}{F}
		\cn{0,50}
		\tx{25,0}{enmos}
		\end{picture}
		%
		\begin{picture}(50,50)(0,0)
		\epmos{0,50}{F}
		\cn{0,50}
		\tx{25,0}{epmos}
		\end{picture}
		%
		\begin{picture}(50,50)(0,0)
		\dnmos{0,50}{F}
		\cn{0,50}
		\tx{25,0}{dnmos}
		\end{picture}
		%
		\begin{picture}(50,50)(0,0)
		\dpmos{0,50}{F}
		\cn{0,50}
		\tx{25,0}{dpmos}
		\end{picture}
		%
		\begin{picture}(50,50)(0,0)
		\njfet{0,50}{F}
		\cn{0,50}
		\tx{25,0}{njfet}
		\end{picture}
		%
		\begin{picture}(50,50)(0,0)
		\pjfet{0,50}{F}
		\cn{0,50}
		\tx{25,0}{pjfet}
		\end{picture}\\[2cm]
		\end{center}
	\end{figure}
	
	\begin{figure}[H]
		\vspace{1cm}
		\begin{center}
		\begin{picture}(50,50)(0,0)
		\enmosrna{50,50}{F}
		\cn{50,50}
		\tx{25,0}{enmosna}
		\end{picture}
		%
		\begin{picture}(50,50)(0,0)
		\enmosr{50,50}{F}
		\cn{50,50}
		\tx{25,0}{enmos}
		\end{picture}
		%
		\begin{picture}(50,50)(0,0)
		\epmosr{50,50}{F}
		\cn{50,50}
		\tx{25,0}{epmos}
		\end{picture}
		%
		\begin{picture}(50,50)(0,0)
		\epmosdd{0,50}{F}
		\cn{0,50}
		\tx{25,0}{epmosdd}
		\end{picture}
		%
		\begin{picture}(50,50)(0,0)
		%%%
		\end{picture}
		%
		\begin{picture}(50,50)(0,0)
		\fet{0,50}{F}
		\cn{0,50}
		\tx{25,0}{fet}
		\end{picture}\\[2cm]
		\end{center}
	\end{figure}
	
	\begin{figure}[H]
		\vspace{1cm}
		\begin{center}
		\begin{picture}(50,50)(0,0)
		\benmos{0,50}{F}
		\cn{0,50}
		\tx{25,0}{benmos}
		\end{picture}
		%
		\begin{picture}(50,50)(0,0)
		\bdnmos{0,50}{F}
		\cn{0,50}
		\tx{25,0}{bdnmos}
		\end{picture}
		%
		\begin{picture}(50,50)(0,0)
		\bepmos{0,50}{F}
		\cn{0,50}
		\tx{25,0}{bepmos}
		\end{picture}
		%
		\begin{picture}(50,50)(0,0)
		\bdpmos{0,50}{F}
		\cn{0,50}
		\tx{25,0}{bdpmos}
		\end{picture}
		%
		\begin{picture}(50,50)(0,0)
		\njfetm{0,50}{F}
		\cn{0,50}
		\tx{25,0}{njfetm}
		\end{picture}
		%
		\begin{picture}(50,50)(0,0)
		\njfetr{0,50}{F}
		\cn{0,50}
		\tx{25,0}{njfetr}
		\end{picture}\\[2cm]
		\end{center}
	\end{figure}
	
	The most common type of transistor is named "bipolar transistor" and these are divided into NPN and PNP types.
Their construction-material is most commonly silicon (their marking has the letter B) or germanium (their marking has the letter A). Original transistor were made from germanium, but they were very temperature-sensitive. Silicon transistors are much more temperature-tolerant and much cheaper to manufacture.

	Some operational amplifier: 
	\begin{figure}[H]
		\vspace{1cm}
		\begin{center}
		\begin{picture}(50,50)(0,0)
		\hoa{0,25}{O}
		\cn{0,25}
		\tx{25,0}{hoa}
		\end{picture}
		%
		\begin{picture}(50,50)(0,0)
		\ho{0,25}{O}
		\cn{0,25}
		\tx{25,0}{ho}
		\end{picture}
		%
		\begin{picture}(50,50)(0,0)
		\hop{0,00}{O}
		\cn{0,00}
		\tx{25,0}{hop}
		\end{picture}
		%
		\begin{picture}(50,50)(0,0)
		\hopi{0,00}{O}
		\cn{0,00}
		\tx{25,0}{hopi}
		\end{picture}
		%
		\begin{picture}(50,50)(0,0)
		\hoar{50,25}{O}
		\cn{50,25}
		\tx{25,0}{hoar}
		\end{picture}
		%
		\begin{picture}(50,50)(0,0)
		\hod{0,75}{O}
		\cn{0,75}
		\tx{25,0}{hod}
		\end{picture}
		%
		\begin{picture}(50,50)(0,0)
		\hou{0,25}{O}
		\cn{0,25}
		\tx{25,0}{hou}
		\end{picture}\\[2cm]
		\end{center}
	\end{figure}
	
	Some logic (boolean) gates:
	\begin{figure}[H]
		\vspace{1cm}
		\begin{center}
		\begin{picture}(50,50)(0,0)
		\hand{0,25}
		\cn{0,25}
		\tx{10,0}{hand}
		\end{picture}
		%
		\begin{picture}(50,50)(0,0)
		\hnand{0,25}
		\cn{0,25}
		\tx{10,0}{hnand}
		\end{picture}
		%
		\begin{picture}(50,50)(0,0)
		\hor{0,25}
		\cn{0,25}
		\tx{10,0}{hor}
		\end{picture}
		%
		\begin{picture}(50,50)(0,0)
		\hnor{0,25}
		\cn{0,25}
		\tx{10,0}{hnor}
		\end{picture}
		%
		\begin{picture}(50,50)(0,0)
		\hxor{0,25}
		\cn{0,25}
		\tx{10,0}{hxor}
		\end{picture}
		%
		\begin{picture}(50,50)(0,0)
		\hxnor{0,25}
		\cn{0,25}
		\tx{10,0}{hxnor}
		\end{picture}
		%
		\begin{picture}(50,50)(0,0)
		\hnot{0,25}
		\cn{0,25}
		\tx{10,0}{hnot}
		\end{picture}\\[2cm]
		\end{center}
	\end{figure}
	
	\begin{figure}[H]
		\vspace{1cm}
		\begin{center}
		\begin{picture}(50,50)(0,0)
		\sr{0,25}
		\cn{0,25}
		\tx{10,0}{sr}
		\end{picture}
		%
		\begin{picture}(50,50)(0,0)
		\jk{0,25}
		\cn{0,25}
		\tx{10,0}{jk}
		\end{picture}
		%
		\begin{picture}(50,50)(0,0)
		\jkff{0,25}
		\cn{0,25}
		\tx{10,0}{jkff}
		\end{picture}
		%
		\begin{picture}(50,50)(0,0)
		\dff{0,25}
		\cn{0,25}
		\tx{10,0}{dff}
		\end{picture}
		%
		\begin{picture}(50,50)(0,0)
		\hnott{0,25}
		\cn{0,25}
		\tx{10,0}{hnott}
		\end{picture}
		%
		\begin{picture}(50,50)(0,0)
		\hnots{0,25}
		\cn{0,25}
		\tx{10,0}{hnots}
		\end{picture}\\[2cm]
		\end{center}
	\end{figure}
	

	\begin{figure}[H]
		\vspace{1cm}
		\begin{center}
		\begin{picture}(125,50)(0,0)
		\htp{0,25}{N}
		\cn{0,25}
		\tx{50,0}{htp}
		\end{picture}
		%
		\begin{picture}(50,50)(0,0)
		\hstp{0,25}{N}
		\cn{0,25}
		\tx{25,0}{hstp}
		\end{picture}
		%
		\\[2cm]
		\end{center}
	\end{figure}
	
	\begin{figure}[H]
		\vspace{1cm}
		\begin{center}
		\begin{picture}(50,50)(0,0)
		\hs{50,25}{Z}
		\cn{50,25}
		\tx{25,0}{hs}
		\end{picture}
		%
		\begin{picture}(50,50)(0,0)
		\hsr{0,25}{N}
		\cn{0,25}
		\tx{25,0}{hsr}
		\end{picture}
		%
		\begin{picture}(50,50)(-10,0)
		\vtl{0,25}{Z}
		\cn{0,25}
		\tx{25,0}{vtl}
		\end{picture}
		%
		\begin{picture}(125,50)(-25,0)
		\htl{0,25}{Z}
		\cn{0,25}
		\tx{50,0}{htl}
		\end{picture}\\[2 cm]
		\end{center}
	\end{figure}
	
	\begin{figure}[H]
		\vspace{1cm}
		\begin{center}
		\begin{picture}(50,50)(0,0)
		\amp{0,25}{A}
		\cn{0,25}
		\tx{25,0}{amp}
		\end{picture}
		%
		\begin{picture}(50,50)(0,0)
		\vo{0,25}{A}{15}
		\cn{0,25}
		\tx{25,0}{vo}
		\end{picture}
		%
		\begin{picture}(50,50)(0,0)
		\voi{0,25}{Z}{15}
		\cn{0,25}
		\tx{25,0}{voi}
		\end{picture}
		%
		\begin{picture}(50,50)(0,0)
		\vor{50,25}{Z}{15}
		\cn{50,25}
		\tx{25,0}{vor}
		\end{picture}
		%
		\begin{picture}(50,50)(0,0)
		\node{25,35}{A}
		%\cn{25,25}
		\tx{25,0}{node}
		\end{picture}
		%
		\begin{picture}(50,50)(0,0)
		\bridge{0,50}
		\cn{0,50}
		\tx{35,0}{bridge}
		\end{picture}
		\\[2 cm]
		\end{center}
	\end{figure}

	
	\begin{figure}[H]
		\begin{center}
		\begin{picture}(50,50)(0,0)
		\htf{0,35}{T}
		\cn{0,35}
		\tx{25,0}{htf}
		\end{picture}
		%
		\begin{picture}(50,50)(0,0)
		\vtf{25,10}{T}
		\cn{25,10}
		\tx{25,0}{vtf}
		\end{picture}
		\end{center}
	\end{figure}
	Or in a much more modern and nice way:	
	\begin{figure}[H]
		\centering
		\includegraphics[width=1.0\textwidth]{img/engineering/electronics_components.jpg}
		\caption[Electronic cheat sheet poster]{Electronic cheat sheet poster (author: Joseph Ricafort Jr.)}
	\end{figure}
	
	\pagebreak
	\subsection{Alternative current VS Direct current}
	The reader will notice that throughout this section, we will mainly work with alternating current. It seems important to explain the origin of this trend of the contemporary industrial world for the alternating current before going further.

	In fact, the origin of this trend is relatively simple:

	When power plants came into being, especially in remote areas of urban centers, it was necessary to transport electric power over long distances. But the cables that carry the electricity have some resistance and this posed a major problem.

	Indeed, an average city can largely need a power of about $10$ [MW]. If this quantity had to be transported under a modest voltage of about $100$ [V], as we have $P=UI$ (\SeeChapter{see section Electrokinetics page \pageref{electric power}}), the current had to be enormous: $100,000$ [A]!

	But the Joule effect in the copper of $1$ [cm] diameter has a linear resistance of $R\cong 0.4\;[\Omega\cdot \text{km}^{-1}]$. With a current of $100,000$ [A], the energy loss per Joule effect would be about (neglecting the potential drop):
	
	we see quite quick the problem if we not have any supra-conductors...

	At the price of $0.1 \;[\$\cdot\text{kW}\cdot \text{h}]$, this represents a cost (loss) of about:
	
	humm ...!
	
	There were no other economic choices than to lower the current. Clearly, if the voltage reached $10^5$ [V], the same power could be carried efficiently by $100$ [A]. Thus, by raising the voltage by a factor of $1,000$, we can reduce the current by a factor of $1,000$ as well, and thus the Joule loss by a factor $10^6$.

	As there was already a simple device for raising and lowering the AC voltage (transformers) without any comparable device for DC voltage (at least at the time), the race was won by the adepts of alternating current.

	It should also be added as a second interest that some linear electrical components (see below) do not have much interest in DC ... we will come back to it!

	Let's see a simple assembly to generate single-phase alternating current:
	\begin{figure}[H]
		\centering
		\includegraphics{img/engineering/ac_generator_principle.jpg}
		\caption{Schematic diagram of an alternating current generator}
	\end{figure}
	The voltage (respectively the current) is determined by the Faraday law demonstrated in the section Electrokinetics:
	
	Which therefore gives the induced electromotive force (or voltage in the case of a generator without resistance...) .

	We obviously have in the situation above if the magnet is permanent and the length of the square turn is $L$:
	
	We already see that in order to obtain a given electromotive force it will be preferable to play with the frequency of rotation rather than with the surface or the intensity of the magnetic field... Or to increase the number of turns by a mounting allowing to arrive at the following relation:
	
	It should be pointed out for the skeptics that there is conservation of energy in this system! Indeed, the energy needed to turn the coil will be that used in part by the system (and that is why dams run turbines with water and nuclear power plants with steam and wind turbines with wind...).
	
	Obviously the opposite case as an alternating current injected into the coil will cause it to rotate in the magnetic field. So in a situation, we have an electric generator and in the opposite case an electric motor.

	It is possible with similar equipment to produce a nearly reliable DC voltage in the following manner named "\NewTerm{dynamo}\index{dynamo}":
	\begin{figure}[H]
		\centering
		\includegraphics{img/engineering/dynamo.jpg}
		\caption{Example of direct current generator (dynamo)}
	\end{figure}
	The simple generator given first with some horseshoe magnets producing the magnetic field was widely used at the beginning of the era of electrical technology but at high voltages (several [kV]) and high internal currents too (more that $50$ [A]). Metal brushes and slip rings produced sparks and deteriorated relatively quickly. Currently, this rotating armature machine is no longer widely used.

	To avoid the difficulties associated with voltages exceeding about $600$ [V], we now turn electromagnets around a stationary armature. The current supplying the rotating electromagnets (which may also be permanent magnets) is relatively low and does not pose any problem to the operation of the rings and brushes. This configuration is then referred to as an "\NewTerm{alternator}\index{alternator}".

	With linear electrical components it is also possible to straighten the tension (we will see this much further). Then we fall back on a dynamo again!
	
	\subsubsection{Average power}
	We have defined in the section Electrokinetics that Joule power was given in constant supply for a conductor by:
	
	In alternating regime system and at low frequency (in order to consider the resistance as constant) we have in the purely resistive case:
	
	It then comes (\SeeChapter{see section Statistics page \pageref{integral average}}) for the mean value of a periodic signal of period $T$:
	
	The term in brackets can therefore be compared to the value that would have a direct current producing the same Joule power. Therefore:
	
	We name this equivalent current the "\NewTerm{effective current value}\index{effective current value}" or "\NewTerm{current root mean square}\index{current root mean square}" (abbreviated "\NewTerm{RMS current}\index{RMS current}") and we denote it by:
	
	The RMS values being what the multimeters measure. For the sinusoidal regimes, we then have:
	
	Therefore we have in this special case:
	
	And using the same method, wishing to calculate the average voltage, we get in sinusoidal regime:
	
	Therefore for a voltage and current having the same phase, we can write:
	
	Strictly speaking, we must generalize this last relation for situations where the current and the voltage are out of phase with an angle (or time) $\varphi$.
	\begin{tcolorbox}[title=Remark,colframe=black,arc=10pt]
	Unfortunately, many electronic books or even research articles have defined a power that has no physical meaning and that is named "\NewTerm{RMS power}\index{RMS power}" and measures "\NewTerm{RMS watts}\index{RMS watter}" defined by analogy by:
	
	Even though this term is used by advertisers and some editors, it has no place in good technical publications. It often appears to give a semblance of technical expertise...
	\end{tcolorbox}	
	Let us consider now another approach of the Power this time from the instantaneous point of view in a more general than purely resistive case. In a more general case, we have the right to write the voltage and the current in the sinusoidal steady state with cosines in the following form (which implicitly also takes into account the phase shift of current and voltage):
	
	We then have using the remarkable trigonometric relations (\SeeChapter{see section Trigonometry page \pageref{remarkable trigonometric identities}}):
	
	where we are free to to put:
	
	such that this term $\varphi$ is positive or null (we can always choose to subtract one of the two terms to the other in order to have a strictly positive or zero $\varphi$ value without changing the developments and conclusions that follow below) and where we used the results obtained previously:
	
	Therefore, in the expression:
	
	the instantaneous power thus comprises a constant component (first term) and a pulsed component with a frequency double that of the current and the voltage. 

	By using:
	
	as well as the following remarkable trigonometric relations (\SeeChapter{see section Trigonometry page \pageref{remarkable trigonometric identities}}):
	
	We then have:
	
	Denoted:
	
	This last relation highlights that the instantaneous power can always be reduced in a permanent sinusoidal regime as the sum of two terms where:
	\begin{itemize}
		\item The first term is a pulsed component, always positive (by construction), which oscillates around the mean value $U_\text{eff}I_\text{eff}\cos(\varphi)$ and which translates a unidirectional energy exchange between a source and a load.

		\item The second term is an alternating component that varies sinusoidally with an amplitude $U_\text{eff}I_\text{eff}\sin(\varphi)$  and a zero average value. It is therefore alternately positive and negative and translates an oscillatory exchange of energy between a source and a charge.
	\end{itemize}

	When $\varphi=0$ (perfect purely resistive load), we then have:
	
	The mean value $U_\text{eff}I_\text{eff}\cos(\varphi)$ is then maximum and equal to $U_\text{eff}I_\text{eff}$, while the second term (the alternative component) is zero.

	On the other hand, when $\varphi=\pm\dfrac{\pi}{2}$ (purely reactive load as an ideal inductance or capacity), then:
	
	and in this case the instantaneous power is reduced to the only alternative component. From this arises the following definitions:
	
	\textbf{Definitions (\#\mydef):}
	\begin{enumerate}
		\item[D1.] We name "\NewTerm{active power}\index{active power}" the term:
		
		and this quantity is measurable by a wattmeter and which corresponds to a real supply of energy convertible to work or heat and which is therefore maximum in the case of a purely resistive (ideal) load and zero in the case of a purely reactive (ideal) charge.

		\item[D2.] We name "\NewTerm{reactive power}\index{reactive power}" the term:
		
		which is not measurable by a wattmeter since equal to zero as it is alternative.

		\item[D3.] We name "\NewTerm{apparent power}\index{apparent power}" the product $U_\text{eff}I_\text{eff}$ that is in faced denoted most of time as:
		
		hence the famous "\NewTerm{power triangle}\index{power triangle}":
		\begin{figure}[H]
		\centering
		\includegraphics{img/engineering/electric_power_triangle.jpg}
		\caption{Electric Power triangle}
	\end{figure}
		Which is in apparence a power but does not necessarily provide work, hence its name ... Traditionally it is the apparent power that is indicated on large industrial installations. It can be recognized because the units are often indicated in volts-amps [VA] rather than in watts [W] (because this is not totally a power)!
	\end{enumerate} 	
	
	\subsection{Transformers}
	A static electric transformer is an electrical machine which makes it possible to modify the voltage and current intensity values delivered by an alternative electrical power source into a voltage and current system of different values but with the same frequency and the same form.
	
	Since the invention of the first constant potential transformer in 1885, transformers have become essential for the transmission, distribution, and utilization of alternating current electrical energy. A wide range of transformer designs is encountered in electronic and electric power applications. Transformers range in size from RF transformers less than a cubic centimeter in volume to units interconnecting the power grid weighing hundreds of tons for Nuclear Plant.
	Let us recall that we have proved in the section of Electrokinetics that:
	
	If we can in one way or another to pass the flow of a first solenoid, then named "\NewTerm{primary winding}\index{primary winding}", having:
	
	to a second solenoid, then named "\NewTerm{secondary winding}\index{secondary winding}", without any loss (or at least a negligible loss) such as:
	
	Then it comes by term-by-term identification:
	
	And if the internal resistances are negligible the induced electromotive force is then equal to the voltage at the terminals, then it comes:
	
	And if all the energy of the magnetic field is transmitted in the secondary winding (without loss) by the conservation of enery, we have:
	
	from where:
	
	Thus, the ratio of the number of primary turns to the number of secondary turns totally determines the transformation ratio of the transformer which can then be used as a voltage transformation station by raising or lowering it as a function of the number of turns of l Secondary winding. It should also be noted that a transformer which increases the voltage simultaneously decreases the current and vice versa.

	A typical historical and pedagogical instrument to do this is the single-phase transformer with a ferromagnetic core:
	\begin{figure}[H]
		\centering
		\includegraphics[scale=1]{img/engineering/schematic_monophasic_transformer.jpg}
		\caption{Typical example of a single-phase transformer with a ferromagnetic core}
	\end{figure}
	In practice, monophasic transformers have windings that are concentric to minimize flux leakage. An insulation is inserted between the primary circuit and the secondary:
	\begin{figure}[H]
		\centering
		\includegraphics[scale=1]{img/engineering/transformer_3d_monophasic_concentric.jpg}
		\caption[Concentric monophasic transformer type]{Concentric monophasic transformer type (source Wikipedia)}
	\end{figure}
	Closed-core transformers are constructed in "core form" or "shell form". When windings surround the core, the transformer is core form; when windings are surrounded by the core, the transformer is shell form. Shell form design may be more prevalent than core form design for distribution transformer applications due to the relative ease in stacking the core around winding coils. Core form design tends to, as a general rule, be more economical, and therefore more prevalent, than shell form design for high voltage power transformer applications at the lower end of their voltage and power rating ranges (less than or equal to, nominally, $230$ [kV] or $75$ [MVA]). At higher voltage and power ratings, shell form transformers tend to be more prevalent. Shell form design tends to be preferred for extra-high voltage and higher [MVA] applications because, though more labor-intensive to manufacture, shell form transformers are characterized as having inherently better kVA-to-weight ratio, better short-circuit strength characteristics and higher immunity to transit damage:
	\begin{figure}[H]
		\centering
		\includegraphics[scale=0.7]{img/engineering/schematic_transformer_type.jpg}
		\caption[Schematic representation of various Transformers]{Schematic representation of various Transformers (source: Wikipedia, author: Spinningspark)}
	\end{figure}
	Transformers by their ability to increase voltage and reduce current (saving by Joule loss as we have already mentioned) play an important role in the transmission of electricity from domestic infrastructures (low voltage: LV, medium voltage: MV, high voltage HV). Thus we find the step-up transformers (SUT) and step-down transformers (STD) in everyday life:
	\begin{figure}[H]
		\centering
		\includegraphics[scale=1]{img/engineering/transformer_cascade.jpg}
		\caption{Typical civilian voltage transformation cascade installation in Switzerland}
	\end{figure}
	\pagebreak
	Here is an example of the largest triphasic step-up transformer (SUT) in the world in 2014:
	\begin{figure}[H]
		\centering
		\includegraphics[scale=0.7]{img/engineering/transformer_largest_2014.jpg}
		\caption[Largest plant-oil-based Transformer in the World in 2014]{Largest plant-oil-based Transformer in the World in 2014 (source: TransnetBW GmbH)}
	\end{figure}
	And a primary triphasic step-down transformer with a slice cut:
	\begin{figure}[H]
		\centering
		\includegraphics[scale=0.3]{img/engineering/transformer_triphasic_stepdown_cut_slice.jpg}
		\caption[Cutaway view of liquid-immersed construction transformer]{Cutaway view of liquid-immersed construction transformer. The conservator (reservoir) at top provides liquid-to-atmosphere isolation as coolant level and temperature changes. The walls and fins provide required heat dissipation balance (source: Wikipedia)}
	\end{figure}
	And a typical USA pole-mounted distribution step-down transformer  with center-tapped secondary winding used to provide split-phase power for residential and light commercial service:
	\begin{figure}[H]
		\centering
		\includegraphics[scale=0.25]{img/engineering/transformer_monophasic_step_down.jpg}
		\caption[Typical pole-mounted distribution transformer in USA with center-tapped secondary winding]{Typical pole-mounted distribution transformer in USA with center-tapped secondary winding (source: Wikipedia)}
	\end{figure}
	And finally typical small size domestic transformers for computer, TV, Hi-Fi, etc. :
	\begin{figure}[H]
		\centering
		\includegraphics[scale=0.7]{img/engineering/transformer_family.jpg}
		\caption[Small domestic size Transformers]{Small domestic size Transformers (source: Mohawk Electro Techniques Inc.)}
	\end{figure}
	
	\subsubsection{Transformer universal EMF equation}
	If the flux in the core is purely sinusoidal, the relationship for either winding between its RMS voltage $e_\text{eff}$ of the winding, and the supply frequency $f$, number of turns $N$, core cross-sectional area $S$ and peak magnetic flux density $B_\text{peak}$ is given by the universal EMF equation that we will prove now!
	
	The derivation of EMF Equation of the transformer is shown below. First remember that we have:
	
	Explicitly:
	
	So we see again that the induced EMF lags flux by $90^\circ$.

	The maximum value of EMF is obviously equal to:
	
	The root mean square RMS value is therefore:
	
	As $\Phi=B_\text{peak}\cdot S$ we get finally the "\NewTerm{transformer universal EMF equation}\index{transformer universal EMF equation}":
	
	
	\pagebreak
	\subsection{Steady State linear circuits}
	We will see here circuits composed of simple elements such as resistance, capacitance and impedance. These circuits, whose representative equation is a linear differential equation, are named "\NewTerm{linear circuits}\index{linear circuits}". Moreover, they are an excellent example to see the cumbersome developments using classical mathematical tools as opposed to other more flexible and powerful techniques (representation by phasors and Laplace transforms).
	
	\subsubsection{RC series circuit}\label{rc series circuit}
	Any circuit having a capacitor also have a resistance, if only that of the connection wires. Such RC (Resistance-Capacitor) series circuits are very common and sometimes of great importance (pacemaker for example). Indeed, when we close a circuit which contains only resistors (purely resistive circuit), the current rises to its nominal value in an extremely short time, so that we can consider that the current and the voltage are constant with an excellent approximation a short time after the circuit was closed (at least in common civil applications...). Thus, the permanent regime is established after a very brief transitional regime. On the contrary, in a series RC circuit, voltage and current take a long time to reach their nominal values. This dependence over time has multiple applications and allows to produce a whole range of signals that can be modulated over time as required.
	\begin{figure}[H]
		\centering
		\includegraphics{img/engineering/rc_serie_circuit.jpg}
		\caption{RC serie circuit}
	\end{figure}
	We assume that initially the capacitor is charged and that no current flows (open switch):
	
	When we close the switch the electrons go away from the capacitor $C$. We then have at the terminals of the resistor:
	
	At the terminals of the capacitor:
	
	The equation of the circuit is then:
	
	Therefore:
	
	Trivial differential equation (but we can detail on readers request) whose solution is with the initial conditions:
	
	Therefore:
	
	The current then has the following form:
	
	It is therefore a system in which the current decreases exponentially and this even faster than the RC factor named the "\NewTerm{time constant}\index{time constant}" is small. We see this kind of system when the light inside a car turns off slowly after closing the doors.

	When this regime is placed under a constant voltage equation, then we have an equation of the form:
	
	An obvious particular solution is then:
	
	We then have the general solution:
	
	either for the current:
	
	And for the voltage at the terminals of the capacitor:
	
	Which therefore represents the voltage at the terminals of the capacitor during charging. So in the end we have the two relations:
	
	\begin{figure}[H]
		\centering
		\includegraphics{img/engineering/rc_serial_voltage_profile.jpg}
		\caption[]{Charging and discharging of the capacitor when opening / closing the switch}
	\end{figure}
	Let us now study the energetic aspect of this circuit that is important in engineering, since power consumption or loss of power is a major selling factor in some applications!

	For this purpose let us take up the relation:
	
	And let us multiply by $i$:
	
	What we write:
	
	As:
	
	Where we see that as soon as the transient charge or discharge is completed, the voltage at the terminals of the capacitor being zero then the current is also zero.

	We have then:
	
	where the first term is the power supplied by the generator to the circuit, the second term is the Joule effect term and the third is the power stored in the capacitor.

	The energy supplied by the generator is stored in the capacitor and dissipated by the resistance by Joule effect.

	What is the most important is to make an energetic balance study over the whole duration of charge of the capacitor to indicate the power dissipated in the characteristics of sale (it better for must buyer than to put equations in it...). To do this, it suffices to integrate the preceding relation from $0$ to infinity to obtain the energy dissipated.

	The first term to the left of the equality gives:
	
	The second term is integrated using $i(t)$:
	
	The third term integrates immediately since we already have the primitive:
	
	Finally, we get:
	
	Thus, for long durations, half of the energy supplied by the generator is dissipated by the Joule effect and the other stored in the capacitor.
	
	
	\subsubsection{RL series circuit}
	Let us consider now the following circuit:
	\begin{figure}[H]
		\centering
		\includegraphics{img/engineering/rl_serie_circuit.jpg}
		\caption{RL serie circuit}
	\end{figure}
	When we close the switch, we then have at the terminals of the resistor:
	
	and at the terminals of the inductance:
	
	and $U_0$ at the terminals of the DC voltage generator.
	
	The equation of the circuit is then:
	
	Therefore:
	
	By reversing:
	
	Let us do a change of variable:
	
	Then:
	
	It then comes after integration:
	
	Let us multiply the two members by $-R$, then let us take the exponential of the two members:
	
	and:
	
	Then:
	
	Where we have the time constant defined by:
	
	Thus, when the switch is closed, the current increases exponentially with an asymptote at the value $U_0/R$. Thus, unlike the RC circuit, the current tends to a non-zero fixed value when $t$ tends to infinity!
	
	We therefore have:
	
	So in the end we have the two relations:
	
	Let us now study the energy aspect that is important in engineering, since power consumption or loss of power is a major selling factor in some applications! As for the RC circuit, we will directly make a energetic balance study of the entire duration of the transient regime to signal the power dissipated in the sales characteristics (it passes better than to put equations in it...):
	
	Let us multiply the terms of the differential equation by $i(t)$:
	
	What we will write:
	
	To calculate the dissipated energy, we proceed in the same way as for the series RC circuit. We have after integration:
	
	Therefore:
	
	Unlike the case of the RC circuit, we can not integrate above with the given terminals because of the "$1-$" which is in front of the exponential because it causes the consumed power to reach the infinite, which is logical, unlike the RC circuit which ends up blocking itself after the capacitor has been charged (the current $i$ tending towards zero very quickly).

	Thus, we only integrate up to a sufficiently large limit time with respect to the values of the passive elements (two or three $\tau$), or we are interested purely in terms of the instantaneous value of the power. We have then:
	
	And therefore at the end of the transitional regime when $t\rightarrow +\infty$:
	
	Therefore, in a steady state, the resistance is the only energy dissipative element in the circuit and it is sufficient to multiply the power dissipated by the desired time interval in order to have an estimate of the energy dissipated.
	
	\subsubsection{RLC circuit}\label{rlc circuit}
	An electrical wire (an antenna for example) is not a perfect conductor. In reality it can be assimilated to a resistance, a capacitance and an internal inductance in series. If we take the case, for example, of the generators, we often consider only the internal resistance as non-negligible, and this obviously reduces the nominal voltage of the generator by a factor to a first approximation proportional to the current flowing through it.

	To study the behavior of such an element often named "\NewTerm{RLC circuit}\index{RLC circuit}", we first represent it in the following form:
	\begin{figure}[H]
		\centering
		\includegraphics{img/engineering/rlc_circuit_serie.jpg}
		\caption{RLC serie circuit}
	\end{figure}
	We assume that initially the capacitor is charged and that no current flows (open switch):
	
	When we close the switch the electrons go away from the capacitor $C$. We then have at the terminals of the resistor:
	
	and at the terminals of the condensator:
	
	and at the terminal of the inductance:
	
	The sum of the potential differences of the circuit is equal to the initial difference of potential, hence:
	
	Or written differently:
	
	It is a second-order linear differential equation very well known in physics (we find it identically in other domains with just different constants). To solve it, we must look for the roots of the associated characteristic equation (\SeeChapter{see section Differential and Integral Calculus page \pageref{method of characteristic polynomial}}):
	
	The latter has for discriminant:
	
	The resistance value for which this discriminant is zero is named the "\NewTerm{critical resistance}\index{critical resistance}":
	
	We can also write the discriminant in the following form:
	
	The solutions of the differential equation are different according to the number and type of the roots of the characteristic equation.
	
	\pagebreak
	\paragraph{Critically damped response}\mbox{}\\\\
	This is the case where $R=R_C$. The characteristic equation then admits a real double root since:
	
	We then have:
	
	with:
	
	The differential equation then admits a solution of the following type when the discriminant is zero (\SeeChapter{see section Differential and Integral Calculus page \pageref{discriminant differential equation}}):
	
	by omitting the delay.
	
	Which gives for the intensity:
	
	The constants are defined by the initial conditions:
	
	We thus obtain for the global solution:
	
	So in the end, we have the two relations:
	
	The following figures illustrate the trend of the time evolution of the charge of the capacitor and of the current through the inductance. The intensity is maximum for $t=\tau$ in the inductance:
	\begin{figure}[H]
		\centering
		\includegraphics{img/engineering/rlc_discharge_behavior_plot.jpg}
		\caption[]{Behavior of the current in the circuit during the discharge of the capacitor}
	\end{figure}
	
	\paragraph{Overdamped response (hypercritic)}\mbox{}\\\\
	This is the case where $R>R_C$. The characteristic equation then admits two distinct real roots:
	
	Therefore:
	
	The two roots are of the same sign, because by using the the Vieta relations (\SeeChapter{see section Calculus page \pageref{vieta relations}}) we have:
	
	The two roots are therefore necessarily negative. We denote their absolute values:
	
	that therefore satisfies:
	
	We have seen in the section of Differential and Integral Calculus that at this moment the solution (without phase shift) is of the form:
	
	Which gives for intensity:
	
	The constants $A$ and $B$ are defined by the initial conditions:
	
	Which gives us:
	
	Either in conventional form:
	
	Let us refer to the expressions of the charge and of the intensity:
	
	The following figures illustrate the temporal evolution of these functions (remember that the roots are negative!):
	 \begin{figure}[H]
		\centering
		\includegraphics{img/engineering/rlc_discharge_behavior_plot.jpg}
		\caption[]{Behavior of the current in the circuit during the discharge of the capacitor}
	\end{figure}
	
	\paragraph{Underdamped response (decaying oscillation) }\mbox{}\\\\
	This is the case where $R<R_C$. The characteristic equation then admits two conjugate complex roots:
	
	Which are assimilated to the resistance of the circuit. We name it "\NewTerm{complex impedance}\index{complex impedance}".

	We will see that contrary to the intuition of that time they were invented (as often in pure mathematics) complex roots have a real physical meaning.

	Let us denote for this $\alpha$ and $\omega$ the absolute values of the real and imaginary parts of these roots:
	
	with:
	
	and:
	
	We have seen in the chapter of Differential and Integral Calculus  that the solution of the differential equation is then written:
	
	What gives us for the intensity:
	
	The constants $C'$ and $\phi$ are determined by the initial conditions:
	
	Which give us:
	
	Therefore:
	
	Thus by reporting in the expressions of the charge $q$ and of the current $i$:
	
	and:
	
	Let us try to simplify this last equality a little. First we ahve proved just before:
	
	Therefore:
	
	Which gives:
	
	But, we also have:
	
	and:
	
	We then have:
	
	We have in the end the following two relations:
	
	So a plot of the current $i$ in the inductance and $q$ of the capacitance will give:
	 \begin{figure}[H]
		\centering
		\includegraphics{img/engineering/rlc_discharge_behavior_underdamped_response_plot.jpg}
		\caption[]{Behavior of the current in the circuit during the discharge of the capacitor in underdamped response}
	\end{figure}
	where:
	
	is the "\NewTerm{amortization factor}\index{amortization factor}". If we wish to have beautiful oscillations damped a but, it is advantageous to have this term as small as possible and therefore have a small value of $R$.

	When $R$ is zero we then have:
	
	with therefore:
	
	which we name the "\NewTerm{resonance pulsation}\index{resonance pulsation}\index{resonance frequency}". Either a period of:
	
	It is then necessary to play with $C$ or $L$ to obtain the desired period in the case where the resistance is zero. Note also that this particular situation is named a "\NewTerm{harmonic oscillator}\index{harmonic oscillator}".

	Finally, from the results obtained, we thus have the generalization of the RC, RL or LC series circuits.

	Now, suppose that in the circuit we were placing a continuous supply in series in the circuit. We have then:
	
	The linear differential equation with constant coefficients now has a second member (constant in this case). We then immediately find a particular solution which it is then sufficient to add to all the solutions which we have obtained previously.

	A particular solution is therefore:
	
	Hence:
	
	Therefore:
	
	This particular solution, which is to be added to the preceding solutions, has no influence on the equations of the current (its derivative being zero). On the other hand, it shifts the plot of $q(t)$ upwards. This is the effect of adding a constant voltage source (such as a simple battery).
	
	So for summary to compare the three configurations:
	 \begin{figure}[H]
		\centering
		\includegraphics{img/engineering/rlc_modes.jpg}
		\caption[]{RLC modes}
	\end{figure}
	\begin{tcolorbox}[title=Remark,colframe=black,arc=10pt]
	It seems that the first practical use for RLC circuits was in the 1890s in spark-gap radio transmitters to allow the receiver to be tuned to the transmitter.
	\end{tcolorbox}
	
	\pagebreak
	\subsection{Linear circuit in forced regime}
	he objective here will be to study the behavior of a series linear RLC circuit excited by a sinusoidal voltage generator since it is a generalization of the $RL$ or $RC$ circuits  (it is enough to cancel $L$ or $C$ respectively to back on the solutions of an $RC$ or $RL$ circuit).

	We then have:
	
	which is a differential equation of known form and therefore important in the field of acoustics because we find it identically for the study of loudspeakers.

	We could very well add a phase shift to the sine on the right  of the equality (phase arbitrary). This would not change the developments that follow, and let us recall that the cosine is only a sinus with a very precise phase shift!

	Finally, the most important thing is that if we find a particular solution to the differential equation above, then since the amplitude and the pulse can take any value at an arbitrary phase shift, then we have an infinity of particular solutions. And as we have shown in our study of differential equations that the sum of particular solutions is also a solution then it means that an excitation obtained with a Fourier series will also have a solution and by passing to the limit we have a Fourier transform (\SeeChapter{see section Sequences and Series page \pageref{fourier transform}})!!!

	So let's move on to our study. To do this, let us derive this relation with respect to $t$:
	
	Let us then seek for a particular solution of the form:
	
	We notice that this proposition of solution is in every point identical to the fundamental of a Fourier series whose term $a_0$ is zero (which is the mean of the signal or the continuous component if it exists) as the reader can check by going back to the section of Sequences and Series where we have introduced Fourier series.!

	Then let us inject these relationships into:
	
	By grouping the trigonometric terms of the same nature:
	
	What is equivalent to:
	
	Hence by identifying the terms:
	
	We can factorize:
	
	And simplifying by $\omega$:
	
	and by changing the sign of the second line:
	
	It is therefore a system of two equations with two unknowns $a$, $b$ which we solve by writing:
	
	where we thus fall back on the "inductive reactance" and the "capacitive reactance" introduced in the section of Electrokinetics.

	Which immediately gives us:
	
	hence:
	
	and therefore:
	
	We also put traditionally that (we shall see later that this expression can be assimilated to the concept of impedance by analogy with the norm of a vector):
	
	This gives the following particular solution:
	
	to a given abritrary phase value.
	
	It is possible to find $\theta$ such as:
	
	Or otherwise written (thus one sees better that we browse all the possible values outside singularities):
	
	We then have using the remarkable trigonometric relations (\SeeChapter{see section Trigonometry page \pageref{remarkable trigonometric identities}}):
	
	$\theta$ is therefore the phase of the current, that is to say the advance or the delay of the current on the voltage. If $\theta=0$ then we have:
	
	and then:
	
	We then say that there is "resonance of the circuit" with therefore:
	
	Either when the inductive reactance is equal to the capacitive reactance.
	
	\subsubsection{Low-pass filter}
	Let us consider the case where $L$ is zero. We have then:
	
	Therefore:
	
	Hence:
	
	Finally:
	
	We then have at the terminals of the capacitor:
	
	We see then that the voltage at the terminals of the capacitor acts as what we name a "\NewTerm{low-pass filter}". That is to say that the amplitude of the voltage across the capacitor with respect to the excitation voltage of the circuit will be reduced, and this is all the more as the frequency will be high.

	This type of tool is very useful for example to eliminate the high-frequency harmonics of a periodic signal obtained by Fourier series or for cleaning a high-frequency noise. Cascaded low-pass filters can also be used to perform spectrum analyzers.

	Here is the plot of the factor:
	
	\begin{figure}[H]
		\centering
		\includegraphics{img/engineering/low_pass_filter_plot.jpg}
	\end{figure}
	We can see that at low frequencies (on the left) the amplitude is preserved (the low-pass filter therefore let pass the low frequencies). Beyond that, the signal is turned off.
	
	The ratio:
	
	Is often expressed in decibels Db either by definition using the measure:
	
	And is then referred to as the "\NewTerm{transfer function}\index{transfer function}" of the filter.
	
	\subsubsection{High-pass filter}
	With regard to the voltage at the terminals of the resistor, we have:
	
	Which is traditionally modified in the following form:
	
	So we see that the voltage at the terminals of the resistor acts as what we name a "\NewTerm{high-pass filter}". That is to say that the amplitude of the voltage across the resistor with respect to the excitation voltage of the circuit will be reduced, and the more so as the frequency will be small.

	Here is the plot of the factor:
	
	\begin{figure}[H]
		\centering
		\includegraphics{img/engineering/high_pass_filter_plot.jpg}
	\end{figure}
	We can see that at small frequencies (on the right) the amplitude is preserved (the high-pass filter allows to pass the high frequencies). Beyond this the signal is switched off.

	The ratio:
	
	Is often expressed in decibels either by definition using the measure:
	
	and is then also referred to as the "\NewTerm{transfer function}" of the filter.

	With different types of assembled filters we can thus remove (but never completely) frequency ranges. We're talking about a "\NewTerm{band-pass filter}\index{band-pass filter}". This is the technique used, for example, for the reception of a certain radio or television channel in a specific frequency range, or also in electronic music to attenuate low pitched or high pitched sounds. Or to separate the ADSL signal from the voice of a telephone line.

	A "\NewTerm{passive filter}\index{passive filter}" is characterized by the exclusive use of linear passive components (resistors, capacitors, coupled coils or not). Therefore, their gain (power ratio between output and input) can not exceed one. They can therefore only partly attenuate signals, but not amplify them, because this would require an input of energy (which is the role of an "active filters").
	
	\subsubsection{Integrator and differentiator}
	We therefore have at the terminals of the capacitor:
	
	Now, if $\omega \gg (RC)^{-1}$, we have:
	
	If we make things so that $\theta=0$ we must have:
	
	Therefore:
	
	Since then:
	
	The circuit is then what we name logically enough ... an "\NewTerm{integrator}\index{integrator}".

	Let us now look at the resistance side:
	
	But we have:
	
	Therefore:
	
	As:
	
	We have then:
	
	If $\omega \ll (RC)^{-1}$ then:
	
	If we make things so that $\theta=\pi/2$ we must have:
	
	The circuit is then what we name logically enough ... a "\NewTerm{derivativator}".

	The utility of an integrating circuit is, for example, to transform a periodic signal into a constant (since the time average of a periodic signal having an offset will never be zero).
	
	\pagebreak
	\subsection{Amplifiers}
	An "\NewTerm{amplifier}\index{amplifier}" or "\NewTerm{electronic amplifier}\index{electronic amplifier}" is an electronic device that can increase the power of a signal (a time-varying voltage or current). An amplifier functions by using electric power from a power supply to increase the amplitude of the voltage or current signal.
	
	 An amplifier can either be a separate piece of equipment or an electrical circuit contained within another device. Amplification is fundamental to modern electronics, and amplifiers are widely used in almost all electronic equipment. Amplifiers can be divided into voltage amplifiers, current amplifiers, transconductance amplifiers, and transresistance amplifiers. 
	
	We can make a very simple amplifier using the circuit shown:
	\begin{figure}[H]
		\centering
		\includegraphics[scale=1]{img/engineering/basic_amplifier_follower.jpg}
		\caption{Simple amplifier (follower)}
	\end{figure}
	This circuit is called a follower because the output follows closely the input voltage.  The voltage gain is slightly less than $1$, but there is a high current gain.
	\begin{figure}[H]
		\centering
		\includegraphics[scale=0.45]{img/engineering/basic_amplifier_config.jpg}
		\caption{Basic operational amplifier configurations}
	\end{figure}

	\begin{flushright}
	\begin{tabular}{l c}
	\circled{20} & \pbox{20cm}{\score{3}{5} \\ {\tiny 23 votes,  64.35\%}} 
	\end{tabular} 
	\end{flushright}

	%to make section start on odd page
	\newpage
	\thispagestyle{empty}
	\mbox{}
	\section{Civil Engineering}\label{civil engineering}
	\lettrine[lines=4]{\color{BrickRed}C}ivil Engineering\index{civil engineering} represents all the techniques for civil constructions and tools associated with it. Civil engineers are involved in the design, implementation, operation and rehabilitation of construction works and urban infrastructure that they manage to meet the needs of society, while ensuring public safety and environmental protection at least in theory. Very interesting and varied, their achievements are distributed mainly in five major areas: structures, geotechnics, hydraulics, transport and environment. As usual in this book, we will focus only on the mathematical formalization of very common cases that have a practical application in everyday life.
	
	Civil Engineering is traditionally broken into several sub-disciplines including architectural engineering, environmental engineering, geotechnical engineering, control engineering, structural engineering, earthquake engineering, transportation engineering, forensic engineering, municipal or urban engineering, water resources engineering, materials engineering, wastewater engineering, offshore engineering, facade engineering, quantity surveying, coastal engineering, construction surveying, and construction engineering. Civil engineering takes place in the public sector from municipal through to national governments, and in the private sector from individual homeowners through to international companies.
	
	Letus notice that in civil engineering it is sometimes made use of the calculation of minimal surfaces. This has been already covered by a concrete examples in the section of Analytical Mechanics. Regarding the efforts of the heat on beams, this is also already been discussed in the section of Thermodynamics.
	
	We will limit ourselves in this section to provable accurate calculations and not on experimental formulas an this in order to have a general introduction to civil engineering techniques and become familiar with the language and some methods of calculation used by engineers in this field. In practice, civil engineers make use of a standard packages with formulas including coefficients or based on tables or maps. For example, in Switzerland, the civil engineers refer to SIA standards (Swiss Society of Engineers and Architects), which contain among others a lot of empirical formulas that engineer use without knowing where they come from. The theoretical approach is obviously insufficient and any training in this field should be obviously completed by practical laboratory work.
	\begin{tcolorbox}[title=Remark,colframe=black,arc=10pt]
	It would be pretentious to claim to do with this section as well and also complete as the \textit{Statique} free French PDF of Nicolet Gaston Raymond that is a priori unrivaled in content and quality to this date (even compared to non-free books on the subject!). It is therefore strongly recommended to refer to it if the reader wants to drive full information about civil engineering (see the download section of the companion website).
	\end{tcolorbox}
	
	\pagebreak
	\subsection{Static}
	Civil engineering uses a lot the tools of the Static for constructions. We're not going here obviously to rewrite the whole section here of Classical Mechanics with the fundamental principle of statistics and everything that goes with it, nor static analysis of beams that has already be done in the section of Mechanical Engineering, but only present some applicative aspects of the principle.
	
	To begin this section, let's look at least the smaller non-trivial cases encountered in practice. We would like to put agains a wall a solid object and we would like to know the strength that will endure this wall. We can represent this basically by the following scheme:
	\begin{figure}[H]
		\centering
		\includegraphics{img/engineering/object_against_wall.jpg}
		\caption{Massive object against a wall}
	\end{figure}
	For the wall holds it is necessary (but not  a sufficient condition, it is just necessary!!!) that the force momentum of gravity equalizes the moment of force of the wall. Then we have:
	
	Therefore:
	
	Verbatim, the foundations of the wall must be able to opposite to the force momentum.
	
	\pagebreak
	\subsection{Pulleys}
	In the context of the study of static forces, there is an industrial example that is famous and that we meet almost every week by walking or driving in front of working sites (construction cranes), train stations (stretchers), ports (ships) or by going to fitness hall or garages: the pulley! Its origin is once again an Archimedes idea (it seems...) who used it for the movement of large masses needed in various projects of his time. The relation with the civil engineering is then completely justified! Let us see this more closely (exceptionally there are very few equations).
	
	Let u consider the following situation named "\NewTerm{simple fixed pulley}\index{simple fixed pulley}" with a mass of $10$ [kg] (ie a force of $100$ [N] with Earth gravity rounded gravity to the nearest ten) hung to rope slipped into the gutter of a pulley:
	\begin{figure}[H]
		\centering
		\includegraphics{img/engineering/simple_fixed_pulley.jpg}
		\caption[Simple fixed pulley]{Simple fixed pulley (source: Wikipedia)}
	\end{figure}
	A simple fixed pulley has the only mechanical advantage to be able to exert the force in a different direction to that of the movement, the force that has to be applied is the same as that required to move the object without the pulley!
	
	The anchor point of the pulley must support the force required to move the object plus the traction force, thus twice this force in the worst case. Otherwise the total load to must support the anchor point is a function of the "\NewTerm{pull angle}\index{pull angle}" of the rope (between $90^{\circ}$ and $180^{\circ}$ ) of course:
	\begin{figure}[H]
		\centering
		\includegraphics{img/engineering/pull_angle_180.jpg}
		\caption{Pull angle at $180^{\circ}$}
	\end{figure}
	For an angle of $180^{\circ}$ the load factor is $200\%$. A load of $10$ [kg] on the rope will represent a load of $20$ [kg] on the pulley.
	\begin{figure}[H]
		\centering
		\includegraphics{img/engineering/pull_angle_90.jpg}
		\caption{Pull angle at $90^{\circ}$}
	\end{figure}
	For an angle of $90^{\circ}$, the load factor is $140\%$. A load of $10$ [kg] on the rope is a load of $14$ [kg] on the pulley.
	
	Now let consider a situation where we set one end of the rope to the support and we draw with the other end to move both the pulley AND the load of $10$ [kg]. This configuration is named "\NewTerm{simple moving pulley}\index{simple moving pulley}" or "\NewTerm{reverse pulley}\index{reverse pulley}" (the legend says that it is this system that Archimedes used it to pull an entier boat):
	\begin{figure}[H]
		\centering
		\includegraphics{img/engineering/simple_mobile_pulley.jpg}
		\caption[Single mobile pulley]{Single mobile pulley (source: Wikipedia)}
	\end{figure}
	In fact in this system (set up vertically or horizontally whatever!) it is as if there were two people who shared the effort of displacement: the wall and the free end of the rope (which is pulled).
	
	The single mobile pulley therefore reduces the force required to move the load to half (the anchor supporting the other half) and thus adding other mobile pulleys, we continue to divide the action to apply! It's stupid but he had to be think!
	
	But this system requires a pull movement of the rope end equal to twice the distance of movement of the load regardless of the radius of the pulley.
	
	Let us also indicate that more the pulley has a large radius, more the force momentum will be big too! So in the case of very heavy loads we will favor large radii for the pulleys if the system that pulls can provide only weak force.
	
	A more realistic configuration (as we will rarely put us above the anchor point to pull the rope and furthermor the previous system is not very stable mechanically speaking ...) of the single mobile pulley presented above is as follows:
	\begin{figure}[H]
		\centering
		\includegraphics{img/engineering/simple_mobile_pulley.jpg}
		\caption[Single fixed and mobile mixed pulley]{Single fixed and mobile mixed pulley (source: Wikipedia)}
	\end{figure}
	Obviously, when we represent systems like these in schools, we neglect for simplification purposes the mass of the pulleys themselves that we should strictly speaking take into consideration!
	
	When we use several pulleys systems working together, we say that we have a configuration of "\NewTerm{composed pulleys}\index{composed pulleys}". The most common type of such a configuration is named a "\NewTerm{hoist}\index{hoist}": the pulleys are distributed in two groups (or block), one fixed and one mobile:
	\begin{figure}[H]
		\centering
		\includegraphics{img/engineering/pulley_system_fixed_and_mobile.jpg}
		\caption[System of combined fixed and moving pulleys]{System of combined fixed and moving pulleys (source: Wikipedia)}
	\end{figure}
	In each group we set an arbitrary number of pulleys that demultiply by the same factor the initial load. The load is obviously united with the mobile group.
	
	So we have $25$ [N] at the end of the rope. The reader can try to have fun finding the $4$ anchor points in the previous illustration and the two pulleys which each divide by $2$ the necessary pull force ... If necessary here is the same configuration but represented in and "unfolded" way:
	\begin{figure}[H]
		\centering
		\includegraphics{img/engineering/pulley_system_fixed_and_mobile_unfolded.jpg}
		\caption[Unfolded system of combined fixed and moving pulleys]{Unfolded system of combined fixed and moving pulleys (source: Wikipedia)}
	\end{figure}
	We already see that the big top pulley is useless except to change the direction of the pulling force. In fact, the two pulleys that are used to divide the force by $2$ are the two lower one, the remainder being provided only for convenience for the movement of the rope.
	
	Let us recall to the reader that in reality we would have to take into account the friction of the rope on the pulley (science of tribology) and we have proven in the section of Classical Mechanics that the real force (stress) of a rope end relatively to the other (end) was given by:
	
	and therefore that the real useful force (stress) to lift a load was given by:
	
	So the force (stress) supplied to lift the object being equal to $T_2$, the force (real stress) lifting the weight being $T_1$, the difference gives the force (stress) making the pulley turn by the intermediate of the friction. Then we have:
	
	The moment of force of the pulley of radius $R$ is:
	
	Let us see an another well know application of a system of fixed and mobile pulleys in train stations (an not only!!!):
	\begin{figure}[H]
		\centering
		\includegraphics{img/engineering/pulley_system_fixed_and_mobile_real.jpg}
		\caption{System of mixed fixed and mobile pulley}
	\end{figure}
	This is a hoist to tension electrical cables with a non-visible counterweight on the photograph (bottom right) that ensures a certain strength so some tension. The advantage of this system is that it allows you to add loads gradually as the cable relaxes and these loads are $4$times higher at the electrical cable level thanks to the wo two movable pulleys (on the left). The pulleys to the right are only here for convenience for the movement of the rope and... the pulling direction for the pulley at the right end.
	
	In the case of a horizontal or vertical load lift, it is easy to determine the gear ratio $D$. Indeed, if we consider $F$ as  the force required to lift vertically the object from a height $h$ by pulling the rope over a length $d$ and $F_g$ the gravitational force on the mass drawn, we then have neglecting the friction and weight of all mobile pulleys:
	
	Finally, let us notice that it is possible to play with the radius of the deflection (deviation) pulley to reduce the force to provide while keeping constant the moment of force (we then speak of "\NewTerm{differential hoist}\index{differential hoist}") but in the end the energy used remain always the same to lift an object at the same height (and we will to pull the rope even more to lift the load at the same height).
	
	\pagebreak
	To conclude about pulley the illustration below and the corresponding gives an excellent summary of what wee saw until now:
	\begin{figure}[H]
		\centering
		\includegraphics[scale=0.96]{img/engineering/pulley_summary.jpg}
		\caption[Summary of typical pulleys]{Summary of typical pulleys (source: Wikipedia)}
	\end{figure}
	\begin{enumerate}
		\item[(1)] Fixed Pulley:
		
	
		\item[(2)] Mobile Pulley:
		
		
		\item[(3)] Simple hoist:
		

		\item[(4)] Double hoist:
		
		That can be generalized to the $n$ case!
	\end{enumerate}
	
	\pagebreak
	We can also considered the differential hoist:
	\begin{figure}[H]
		\centering
		\includegraphics{img/engineering/pulley_differential_hoist.jpg}
		\caption[Summary typical differential pulleys]{Summary typical differential pulleys (source: Wikipedia)}
	\end{figure}
	with:
	
	
	\subsubsection{Windlass}
	The windlass is an apparatus for moving heavy weights. Typically, a windlass consists of a horizontal cylinder (barrel), which is rotated by the turn of a crank or belt. A winch is affixed to one or both ends, and a cable or rope is wound around the winch, pulling a weight attached to the opposite end.
	
	Mathematically all equations about windlass can be derivative by analogy from pulleys.
	
	Consider for example the simple following windlass:
	\begin{figure}[H]
		\centering
		\includegraphics[scale=0.7]{img/engineering/windlass_simple.jpg}
		\caption[Simple Windlass]{Simple Windlass (source: Fortec, Charles Pache)}
	\end{figure}
	For this case we have obviously (gears rules applies as seen in the section of Mechanical Engineering):
	
	\begin{figure}[H]
		\centering
		\includegraphics[scale=0.7]{img/engineering/windlass_differential.jpg}
		\caption[Differential Windlass]{Differential Windlass (source: Fortec, Charles Pache)}
	\end{figure}
	For this case we have obviously the same as the differential hoist:
	
		
	\pagebreak
	\subsection{Cornu spiral}
	The "\NewTerm{clothoid}\index{clothoid}" is a plane transcendental curve\footnote{A "\NewTerm{transcendental curve}\index{transcendental curve}" is a curve which is not defined by an algebraic equation (typically polynomial or trigonometric), but by a transcendental equation, that is to say, the unknowns are not finite quantities, but differential ($\mathrm{d}x$, $\mathrm{d}y$, representing an infinitesimal variation of the variables $x$ and $y$). Such a curve can not be constructed geometrically precisely because its equation can not be reduced in a simple form $y = f (x)$.}  which the curvature is proportional to the curvilinear abscissa. It is also named "\NewTerm{Cornu spiral}\index{Cornu spiral}", referring to Alfred Cornu, the French physicist who invented it. More rarely, it may appear under the name of "\NewTerm{radioïde arc}\index{radioïde arc}" or "\NewTerm{Euler spiral}\index{Euler spiral}" or "\NewTerm{Fresnel spiral}\index{Fresnel spiral}".
	
	This geometry is also suitable for curves for railways because a train following the path of this geometry at a constant speed will be under constant angular acceleration whatever the point of the curve, which reduces both the stresses on the rail or on the wheels and the discomfort of passengers of the train:
	\begin{figure}[H]
		\centering
		\includegraphics[scale=0.8]{img/engineering/clothoide_bernina_express.jpg}
		\caption{Train loop with a clothoid shape}
	\end{figure}
	Same for car:
	\begin{figure}[H]
		\centering
		\includegraphics{img/engineering/clothoide_highways.jpg}
		\caption{Highway bifurcation with a clothoid shape}
	\end{figure}
	Finally, the railway cable of pylons of cable car and supporting the suspension cable, take this form. Like this, it is possible to circulate the cab at maximum speed on the pylon, without inconveniencing the passengers.
	\begin{figure}[H]
		\centering
		\includegraphics[scale=0.8]{img/engineering/clothoide_sabot.jpg}
		\caption{Cable car pylon with a clothoid shape}
	\end{figure}
	Also this curve is used for vertical loops or loops in the roller coaster for passenger comfort, so that the vertical acceleration is continuous.
	\begin{figure}[H]
		\centering
		\includegraphics{img/engineering/clothoide_roller_coaster.jpg}
		\caption{Roaller cooaster vertical loop with a clothoid shape}
	\end{figure}
	When a vehicle moves in a circular motion, he will undergo a force $\vec{F}=m\vec{a}$ perpendicular to its direction (centrifugal force) having for norm (\SeeChapter{see section Classical Mechanics page \pageref{centrifugal force}}):
	
	from the start of it entry into the circular curve. This effect is problematic because for some small cars on a highway, the centrifugal force can equal the force of gravity ot the car itself (when speed is within the legal values!).

	Thus, the acceleration rise brutally from $0$ to $v^2/R$, then the engineers build sometimes the curves with an declination to improve the adhesion, but it is also possible to try to find curves for which the acceleration is more gradual. For example if the curvature $C$ given by (\SeeChapter{see section Differential Geometry page \pageref{curvature radius}}):
	
	is proportional to the path $s$ (curvilinear abscissa) traveled in the curve, we will have at the beginning of the curve $C=0$ so the acceleration will be zero.

	So what we seek are then curves such that:
	
	For this, let us recall that we can also write naturally for a circle, the curvature into the following form:
	
	Indeed, if we turn of an angle $\mathrm{d}\theta$ the we move of a length $\mathrm{d}s=R\mathrm{d}\theta$ (\SeeChapter{see section Trigonometry page \pageref{spherical trigonometry}}).

	Then we have the relation:
	
	Thus:
	
	hence:
	
	But let us recall that the parametric equation of the circle is (\SeeChapter{see section Analytical Geometry page \pageref{parametric equation of an ellipse}}):
	
	We then have:
	
	Therefore:
	
	We can now write:
	
	Therefore:
	
	with a small change of variables:
	
	it comes:
	
	by taking $x_0=y_0=0$ (we can always do a translation later).
	
	The two integrals are named "\NewTerm{Fresnel integrals}\index{Fresnel integrals}" and cannot calculated in a closed form as far as we know. However, we can express them in a Taylor series expansion form (\SeeChapter{see section Sequence and Series page \pageref{taylor series}}) as:
	
	The plot of the Fresnel integral in Maple 4.00b gives:
	
	\texttt{>plot([FresnelC(t),FresnelS(t),t=-5..5]);}
	\begin{figure}[H]
		\centering
		\includegraphics[]{img/engineering/fresnel_integral_plot_maple.jpg}
		\caption{Plot of the Fresnel integral in Maple 4.00b}
	\end{figure}
	By zooming on the part relevant for us:
	\begin{figure}[H]
		\centering
		\includegraphics[]{img/engineering/fresnel_integral_plot_zoom_origin_maple.jpg}
		\caption[]{Focus on the origin of the Fresnel integral}
	\end{figure}
	The same thing to a given constant using the Taylor series previously presented:
	\begin{figure}[H]
		\centering
		\includegraphics[]{img/engineering/fresnel_integral_taylor_development_plot_maple.jpg}
		\caption[]{Equivalent Taylor series expansion}
	\end{figure}
	The engineering office use special software incorporating clothoids spirals in 2D or 3D environments based on topographic surveys done by GIS specialists.
	
	\pagebreak
	\subsection{Overhead cable}
	An overhead cable is a cable for the transmission of information, laid on utility poles. Overhead telephone and cable TV lines are common in North America. Elsewhere, overhead cables are laid mainly for telephone connections of remote buildings and temporary mechanisms, as for example building sites. The same poles sometimes carry overhead power lines for the supply of electric power. Power supply companies may also use them for an in-house communication network. Sometimes these cables are integrated in the ground or power conductor. Otherwise an additional line is strung on the masts.
	
	Galilee was probably the first to be interested in the chain ovearhead cable shape that he interpreted for a parabolic arc. Jean Bernoulli, Huygens and Leibniz found (independently) in response to the challenge of Jakob Bernoulli, his true nature in 1691: it generated by a hyperbolic cosine.
	\begin{figure}[H]
		\centering
		\includegraphics{img/engineering/cable_overhead_high_voltage.jpg}
		\caption[High-voltage overhead cable]{High-voltage overhead cable (source: chronomaths)}
	\end{figure}
	\begin{figure}[H]
		\centering
		\includegraphics{img/engineering/cable_suspended_pipeline.jpg}
		\caption[]{Suspended Pipeline over a river}
	\end{figure}
	
	
	\pagebreak
	\subsubsection{Free overhead cable (catenary)}
	Let us consider (source: ChronoMath) for the study a homogeneous free, flexible cable attached at two points $A$ and $B$. In its equilibrium position, the cable hangs in a vertical plane and seems to take a parabolic shape. In fact, not really ...
	\begin{figure}[H]
		\centering
		\includegraphics[]{img/engineering/free_overhead_cable_configuration_study.jpg}
		\caption{Configuration for studying suspended cables}
	\end{figure}
	Let us create in this plane an orthonormal reference frame $(\text{O},\vec{i},\vec{j})$, where O denotes the lowest point of the cable and let us denote as always the field of gravity $\vec{g}$ that we consider as uniform.

	Let us denote by $\vec{T}_0$ the tension (force) at the point O which defeats the tension at the point $M$ so that the cable portion $\overline{\text{O}M}$ of length $L$, subjected to its linear weight $\mu\vec{g}$ at the point $G$, is in equilibrium in the static sense:
	
	\begin{tcolorbox}[title=Remark,colframe=black,arc=10pt]
	The product $\mu\vec{g}$ of the linear mass $\mu$ with the gravitational acceleration $\vec{g}$ is also very often denoted $\vec{w}$ asw as we have already see in the section of Mechanical Engineering to represent the linear load.
	\end{tcolorbox}
	Let us project on the coordinate axes by denoting by $\alpha$ the angle $\widehat{\vec{i}\vec{T}}$. We then have the following decomposition:
	
	We can then write the following system:
	
	Either after simplification:
	
	Therefore:
	
	By calculating the ratio:
	
	To get the differential equation... (here this is subtle...):
	
	where $\mathrm{d}L$ is the curvilinear abscissa of the cable (often denoted $\mathrm{d}s$ in the literature according to what is done in Differential Geometry).

	Then:
	
	But the tangent is also the derivative of the function describing the chain. So:
	
	Therefore it comes:
	
	Following the intervention of one of our reader, we will propose two ways to solve this differential equation. The first is the original one and it is a bit complicated and the second one (available much further below) is that proposed by a reader and that  is perhaps more elegant and simple.
	\begin{itemize}
		\item First approach:

		Let us put $u=y'$ and seek the primitive of the left member at first (that of the right member being obvious). The calculations made in the section of Differential and Integral Calculus in the determination of the usual primitives give us:
		
		So we have:
		
		By passing to the exponential:
		
		observing that in our problem at $x=0$ we have indeed $y'=0$.
		To find $y'$ we use a trick: We know that the function is symmetric. So if we replace $x$ by $-x$ the tangent also changes sign and passes from $y'$ to $-y'$:
		
		By subtracting:
		
		it comes:
		
		So after integration:
		
		where the expression after the first equality is named the "\NewTerm{catenary curve}\index{catenary curve}". It comes to the end:
		
		Where the constant will be determined by the initial conditions.
	
		We see with Maple 4.00b the difference between a parabola and the chain easily:
	
		\texttt{>plot([x\string^2,cosh(x)],x=-4...4);}
		\begin{figure}[H]
			\centering
			\includegraphics[scale=0.8]{img/engineering/catenary_vs_parabola.jpg}
			\caption{Plot between a chain and parabola with Maple 4.00b}
		\end{figure}
		In real life as an artwork:
		\begin{figure}[H]
			\centering
			\includegraphics[scale=0.4]{img/engineering/st_louis_catenary.jpg}
			\caption[The Gateway Arch St. Louis Arch]{The Gateway Arch St. Louis Arch (source: Wikipedia, authot: Daniel Schwen}
		\end{figure}
		Let us consider now two points $(x_1,y_1)$ and $(x_2,y_2)$ in the plane and let us determine the equation of the chain of length $L$ having these two points as ends.

		We have the two equations:
		
		We obtain a third equation using the length $L$ which is known. Indeed (\SeeChapter{see section of Analytical Mechanics page \pageref{parametric curve length}}):
		
		where we always have:
		
		Thus, we get a nonlinear system of three equations with three unknowns $(k,c_1^{te},c_2^{te})$:
		We have the two equations:
		
		\begin{tcolorbox}[colframe=black,colback=white,sharp corners]
		\textbf{{\Large \ding{45}}Example:}\\\\
		Let us determine for example the chain of length $38$ [cm] passing through the points $(-9,0)$ and $(9,10)$.\\
	
		The following system must then be solved:
		
		Here are the Maple 4.00b commands that allow us to get the result.\\
		
		\texttt{> E1: = 0 = k * cosh (-9 / k + c1) + c2;\\
		> E2: = 10 = k * cosh (9 / k + c1) + c2;\\
		> E3: = 38 = k * (sinh (9 / k + c1) -sinh (-9 / k + c1));
		\\> Fsolve ({e1, e2, e3}, {k, c1, c2}, {k = 0..infinity});
		}
		Maple gives:
		\begin{center}
			\texttt{K = 4.073758798, c1 = .2694982504, c2 = -14.46356329}
		\end{center}
		Graphically we have then:
		\begin{figure}[H]
			\centering
			\includegraphics[scale=0.75]{img/engineering/overhead_cable_small_plot_maple.jpg}
			\caption{Plot of a small overhead cable in Maple 4.00b}
		\end{figure}
		\end{tcolorbox}	
		
	
		\item Second approach:
		For this second approach of solving the differential equation, we will keep the notation proposed by the reader who contacted us. We start from the differential equation:
		
		We make the following change of variable:
		
		Therefore it comes:
		
		and then:
		
		The integration gives according to the usual primitive proved in the section of Differential and Integral Calculus:
		
		And the condition:
		
		on $x=0$ (lowest point of the overhead string for recall) imposes that the integration constant is zero and therefore that the points of attachment are by symmetry located at the same height. We have then:
		
		Therefore:
		
		that has to be compared with the previous method where we had obtained:
		
		The integration constant is determined by the points of attachment of the overhead cable distant from a distance $D$ where we have $y=0$ on $x=-D/2$ and $x=D/2$ (so the ends are on the same horizontal). We have then:
		
		It is under this last form that the overhead cable was obtained independently by Leibniz, Huyghens and Bernoulli at the end of the 17th century...!
	
		We can now easily calculate the maximum deviation (the "arrow" as the engineers in civil engineering say, that we will denoted by the letter $f$ as in the section of Mechanical Engineering) with respect to the horizontal passing through the points of attachment:
		
		Finally it can be interesting to calculate the length of the cable at equilibrium. For this, let us recall that we have demonstrated above that:
		
		In the case where the fasteners are at the same height, the chain is symmetrical and we then have:
		
		It remains to determine the constant! If $D$ is zero then $L$ must be zero. The constant is then zero and it remains:
		
		In the case of electrified railway lines, we solve the problem of the arrow by a main cable that carries of the catenary: the upper cable (below on the photo) undergoes an accepted deflection, which reduces the tension between the pylons. The catenary thus remains very linear thanks to the multiple auxiliary hooks to an auxiliary cable.
		\begin{figure}[H]
			\centering
			\includegraphics[scale=1]{img/engineering/overhead_cable_train.jpg}
			\caption[Overhead cable of railways ]{Overhead cable of railways (source: Chronomaths)}
		\end{figure}
		Otherwise let us also notice that we also find the overhead cable in all the places of the life of every day where a cable is suspended between two points on the same horizontal.
		\begin{tcolorbox}[colframe=black,colback=white,sharp corners]
		\textbf{{\Large \ding{45}}Example:}\\\\
		Consider a suspended cable with the following data:
		
		We have since the linear load of the cable is constant (taking the notation of the section of Mechanical Engineering on the way...):
		
		and therefore:
		
		Thus the arrow of the cable is $6$ meters below the horizontal of the two hangers.
		\end{tcolorbox}	
	\end{itemize}
	
	\subsubsection{Charges overhead cable (suspended bridge)}
	Let us consider the following suspended bridge carrying a constant linear load, hence the arrow $h$ is imposed by the architect, as well as the distance $D$ between the two pillars:
	\begin{figure}[H]
		\centering
		\includegraphics[scale=1]{img/engineering/suspended_cable_bridge.jpg}
		\caption[Suspension bridge]{Suspension bridge (source: ISBN 0-13-814929-1)}
	\end{figure}
	The subtlety of this case study lies in the fact that the linear load is no longer in the cable itself but in the span of the bridge which has a much higher linear mass. We can no longer use the analyzes given above. Development done earlier above for normal suspended cable remains valid only until the relation:
	
	Now, we must remember that we also have:
	
	by rearranging and deriving again by $\mathrm{d}x$:
	
	On the bridge however, each portion $\mathrm{d}L$ of the cable is negligible in front of the portion $\mathrm{d}x$ of the bridge. This marks the difference between the suspended bridge and the simple overhead cable as we need to replace $\mathrm{d}L$ by $\mathrm{d}x$ (it is quite subtle with this approach but there are several possible approaches to this development). We then have the preceding relation which becomes:
	
	Either after rearrangement:
	
	By integrating a first time it comes then:
	
	And a second and last time:
	
	The constants are determined by the initial conditions. The place where we have placed the reference frame imposes that $y = 0$ in $x = 0$ and that $\mathrm{d}y / \mathrm{d}x = 0$ in $x = 0$, we have the two constant which are zero and it then remains:
	
	It is therefore the equation of a parabola and not of a ovearhead cable anymore! A small concern is that we do not know the tension (stress) in the cable. We should get rid of it. Now, because the architect requires that the arrow be $h$ at $x = D / 2$, then it comes:
	
	therefore:
	
	Again this remains a parable regardless of the weight! This is due to the fact that it is uniform in this case.

	We then have to determine the length of the cable. For this we take up the relation already recalled above (proved in the chapter section Analytical Mechanics and of Geometric Shapes):
	
	By the vertical symmetry of the function of the suspension bridge we can write:
	
	In the section of Differential and Integral Calculus in our proofs of the usual primitives, we have demonstrated for recall that:
	
	Therefore it comes:
	
	\begin{tcolorbox}[colframe=black,colback=white,sharp corners]
	\textbf{{\Large \ding{45}}Example:}\\\\
	Let us consider the characteristics of the Golden Bridge of San Francisco. Its arrow $h$ is about $230$ meters and its main bearing is $1280$ [m]. We have then:
	
	\end{tcolorbox}
	
	\pagebreak
	\subsubsection{Very tense cable}
	Let us recall that in the general case we get the following relation:
	
	Let us take the differential:
	
	Thus for the component $y$:
	
	But we also have:
	
	So injected into the previous relation this gives:
	
	Therefore:
	
	And under the assumption that the cable is very tight (under hight stress):
	
	Therefore it comes:
	
	After a first integration it comes:
	
	Before going further, let us notice that the constant is easy to determine since in $x = L / 2$, the derivative must be zero. We have then:
	
	After a second integration (where the integration constant is zero):
	
	We then for the deformation of the string:
	
	Thus, in the middle of the cable (string), the arrow is then:
	
	Or respectively if the arrow is known and the tension is sought to be determined:
	
	And we could also calculate the length of the cable (string) using the same technique as for the suspension bridge.
	
	\pagebreak
	\subsection{Falling chimney (naive approach)}
	The blasting of a falling chimney problem is a well know study in high-school and an interesting application of circular motion kinematics and moment of inertia. It is especially interesting as we can found many approach to this problem on Internet and textbooks but many known one are vicious as they hide important assumptions or are even wrong but still gives the correct final result.
	
	Our purpose here is to give the simple possible way to determine the result without forgetting to enumerate any assumption! Our purpose is also to take this opportunity to introduce first a physical phenomenon that also appears withing the falling chimney study framework that some people consider as "magic": the Hinged Stick and Ball trick!
	
	Let us consider a plank that is able to pivot about one end, with the other end free to move. As in figure below:
	\begin{figure}[H]
		\centering
		\includegraphics{img/engineering/falling_plank.jpg}
	\end{figure}
	 the free end of the plank is raised a certain angle $\theta$ above a table, with a ball set on the raised end. The plank is then released. All points on the plank (except those at the pivot) will follow circular trajectories, each with a different radius. The linear velocity of each point, directed tangentially to the circular path of the point, is then dependent upon its distance from the fixed end of the plank. The ball, however, falls independently of the plank.
	 
	Then we assume:
	\begin{itemize}
		\item There is no friction at the pivot point
		\item We are in vacuum
		\item The plank is infinitely rigid
		\item The gravitational force is constant at any point
		\item The plank is assimilated to a cylinder
	\end{itemize}

	It is of interest to find the acceleration of various points on the plank. First, consider the rotational dynamics of the system while the plank is in motion. Once the plank is released from its rest position, the two forces acting on it are gravity and a normal force. The latter is exerted by the table at the pivot, so it does not exert a torque about the pivot. The gravitational force does exert a torque on the plank about the pivot, giving the net torque $\tau$ (we change the notation of the torque that is normally $M$ in this book because later we will have $M$ for the notation of the whole mass so it would be a mess if we don't change now!) of on the plank as:
		
	where $m$ is the mass of the plank, $g$ is the acceleration due to gravity $g = 9.81\;[\text{m}\cdot \text{s}^{-2}]$, $l$ is the plank's length and $\theta$ is the angle between the raised plank and the table.
	
	The rotational equivalent of Newton's second law applies to the plank’s motion, namely:
	
	And we have proven in the section of Geometrical Shapes that the moment of inertia of a cylinder rotating around a fixed extremity point was given by:
	
	Therefore:
	
	Next, the linear acceleration $a(x,\theta)$ of a point on the plank that is at a distance $x$ from the pivot is given by (dimensional analysis):
	
	Therefore we can write:
	
	And we can equate the both $M$ to get:
	
	from which it is possible to solve for $a$, with the result being:
	
	Let $x = L$, so that the point of consideration is the moving end of the stick. Then it remains (notice that the mass $m$ does not appear!):
	
	Therefore we have that $a>g$ for the of the stick when:
	
	As we know that
	
	and in our case the plank has $x_0=0$ and $v_0=0$ it comes:
	
	The distance $x$ we are looking for depends on the initial angle $\theta$ but is trivially given (definition of the radian!) by for the extremity of the plank:
	
	Therefore (notice that the mass $m$ again... does not appear!):
	
	\begin{tcolorbox}[colframe=black,colback=white,sharp corners]
	\textbf{{\Large \ding{45}}Example:}\\\\
	If we consider a plastic ruler of $l=30$ [cm] and of mass $m= 68$ [g] with starting angle $\theta=\pi/4$ [rad] we get:
	
	\end{tcolorbox}
	OK this basic stuff being done. Let us now focus on the chimney rupture point.
	
	In the frame of the falling chimney we have for a perfect rigid chimney:
	
	But first we have:
	
	Therefore:
	
	or what remains the same:
	
	But in reality what we can observe is that a falling object consisting on a of smaller parts (bricks) breaks apart on some points:
	\begin{figure}[H]
		\centering
		\includegraphics[scale=0.6]{img/engineering/falling_chimney.jpg}
		\caption{A special case of falling chimney}
	\end{figure}
	Obviously for chimneys the smaller bricks are not independent and the point of rupture depends on the material properties. So we will imagine a chimney made of a stack of bricks without any strengthening. So in this situation we observe that a falling stacks of bricks always break into parts. So the previous relations is not complete. We don't know the origin of the force that breaks apart but we know that we can insert a term that contains all possible sources of internal forces just by writing:
	
	Or we also could have written:
	
	where $\mathrm{d}F$ is the force due to the mass that is above $\mathrm{d}m$.
	\begin{tcolorbox}[title=Remark,colframe=black,arc=10pt]
	Another way to explain this is to remember that we have proved earlier above that:
	
	Therefore the angular acceleration depends on the length of the chimney. Now if the force of gravity were the only force acting on $\mathrm{d}m$, the angular acceleration on $\mathrm{d}m$ would be independent of the length of the chimney.
	\end{tcolorbox}
	
	Now we know that what breaks such a hypothetical chimney is a torque!
	
	An element at distance $x_0$ from $0$ experiences a torque due to the rest of the chimney above it. This torque is given by:
	
	The extremal torques (maxima and minima) are found by:
	
	Therefore after derivation by $x_0$ it remains only:
	
	Or simplified a bit more:
	
	So there is a first obvious visible solution $x_0=L$ that gives:
	
	So this is the minima!
	
	Now remember that:
	
	Therefore:
	
	After simplification (assuming $\cos(\theta)\neq 0)$:
	
	Or:
	
	Hence:
	
	Or:
	
	This is a second degree polynomial on $x_0$. Therefore (\SeeChapter{see section Calculus page \pageref{double root}}):
	
	
	The maximum torque is therefore found at distance $x_0 = L/3$
from $0$. This is where the chimney might POSSIBLY break.
	
	\pagebreak
	\subsection{Dams}
	A dam is a barrier that impounds water or underground streams. Reservoirs created by dams not only suppress floods but also provide water for activities such as irrigation, human consumption, industrial use, aquaculture, and navigability. Hydropower is often used in conjunction with dams to generate electricity. A dam can also be used to collect water or for storage of water which can be evenly distributed between locations. Dams generally serve the primary purpose of retaining water, while other structures such as floodgates or levees (also known as dikes) are used to manage or prevent water flow into specific land regions.
	\begin{figure}[H]
		\centering
		\includegraphics[scale=1]{img/engineering/dam_dixence.jpg}
		\caption{Dixence (Switzerland) Dam}
	\end{figure}
	There exist various type of dams as illustrated in the figure below:
	\begin{figure}[H]
		\centering
		\includegraphics[scale=0.4]{img/engineering/type_of_dams.jpg}
		\caption[Types of dams]{Types of dams (source: Tata \& Howard)}
	\end{figure}
	We will consider here, in a first time, only a "\NewTerm{gravity dam}\index{gravity dam}" that is by definition a massive sized dam fabricated from concrete or stone masonry. Such dams are designed to hold back large volumes of water. By using concrete, the weight of the dam is actually able to resist the horizontal thrust of water pushing against it. This is why it is named a "gravity dam". Gravity essentially holds the dam down to the ground, stopping water from toppling it over.

	Gravity dams are well suited for blocking rivers in wide valleys or narrow gorge ways. Since gravity dams must rely on their own weight to hold back water, it is necessary that they are built on a solid foundation of bedrock.
	
	Therefore let us consider the following gravity dam (consisting of a solid with infinite rigidity and perfectly waterproof...) of height $z$, of length $L$ and storing water of density $\rho$:	
	\begin{figure}[H]
		\centering
		\includegraphics[scale=1]{img/engineering/gravity_dam.jpg}
		\caption[]{Simplified approach for the study of pressure on a gravity dam}
	\end{figure}
	We have proved in the section of Continuum Mechanics that the hydrostatic pressure was given by:
	
	But in this situation we obviously have:
	
	So when we place ourselves on the surface of the water at $z=h$:
	
	thus the pressure of the air at the surface of the dam sea...

	On an surface element $\mathrm{d}S$ it applies an elementary force:
	
	But:
	
	Therefore:
	
	hence after integration
	
	It is therefore the force exerted on the immersed face. The force on the emerged face (left in the illustration) is simply given by putting $\rho=0$. We have therefore for the part due to atmospheric pressure alone:
	
	and for the part due to water only:
	
	\begin{tcolorbox}[title=Remark,colframe=black,arc=10pt]
	On average between the empty state and full state (following what we can read on the Internet) the top of a gravity dam of certain height would sometimes move with an amplitude of the order of $80$ [cm].
	\end{tcolorbox}

	\begin{flushright}
	\begin{tabular}{l c}
	\circled{70} & \pbox{20cm}{\score{3}{5} \\ {\tiny 25 votes,  54.40\%}} 
	\end{tabular} 
	\end{flushright}

	%to make section start on odd page
	\newpage
	\thispagestyle{empty}
	\mbox{}
	\section{Aerospace Engineering}
	\lettrine[lines=4]{\color{BrickRed}I}n this section we will see some useful and simple mathematical practical cases that we studied in the section of Analytic Geometry, Classical Mechanics and Continuum Mechanics and even Astronomy.
	
	Aerospace engineering is the primary field of engineering concerned with the development of aircraft and spacecraft. It is divided into two major and overlapping branches: "aeronautical engineering" and "astronautical engineering". the Aerospace Engineering (astrodynamics) which is then scientific discipline that brings together aerospace engineering techniques  (travel in the atmosphere, using airplanes or helicopter, for example) and those of astronautics (space travel, that is to say outside the atmosphere and interplanetary journeys, using shuttles space and rockets).
	
	The reader will certainly notice that the examples below are only the examples that we often find in the text textbooks as exercises.
	
	We will not here come back on the theory of conical (very important for the orbiting satellites) seen in the section Analytic Geometry, neither the gyroscope theory very useful in guiding / force the axis of rotation (spin) of satellites and already seen in the section of Classical Mechanics (by cons we can not use the theory of ballistic that can be found in the same section as we had assumed there the constant initial speed...) or theory of Lagrangian points sometimes useful to put satellites into orbit on distant stars (which does not mean that we will not use the theoretical results of these subjects we studied). Also hypothetically we will consider the body in a non-relativistic motion which is so far the most frequent case... at least actually...
	
	In this section we will neglect many things like friction, vibration, the cases with more than two bodies (stars, planets ...) and many other factors. We will also do not see the tricks specific to space engineering (like the fact that some satellites have weights attached by adjustable cables to increase or decrease their gyro momentum).
	\begin{tcolorbox}[title=Remark,colframe=black,arc=10pt]
	Some details may seem insignificant to the reader but he must be aware that the price per kilo of rocket launch is between $20,000$ and $30,000$ on this beginning of the 21st century and this excluding insurance... So anything that can be optimized without increasing the risk hast to be optimized!
	\end{tcolorbox}
	
	\pagebreak
	\subsection{Airfoil Lift}
	A fluid flowing past the surface of a body exerts a force on it (and not magic...). The "\NewTerm{lift}\index{lift}" is the component of this force that is perpendicular to the oncoming flow direction. It contrasts with the "\NewTerm{drag force}\index{drag force}", which is the component of the surface force parallel to the flow direction. If the fluid is air, the force is named an "\NewTerm{aerodynamic force}\index{aerodynamic force}". In water, it is named a "\NewTerm{hydrodynamic force}\index{hydrodynamic force}".
	
	Because lift is a force, it is a vector quantity, having both a magnitude and a direction associated with it. Lift acts through the center of pressure of the object and is directed perpendicular to the flow direction. There are several factors which affect the magnitude of lift.
	
	There are many explanations for the generation of lift found in encyclopedias, in basic physics textbooks and on Web sites. Unfortunately, many of the explanations are misleading or partially incorrect (because oversimplified) and does not quantify the value of the corresponding lift. Therefore it is naturally because this lack of scientific approach that the generation of lift have become a source of great controversy and a topic for heated arguments (this is why NASA has a special sub-site dedicated to this subject that has significantly inspired us for the below pages).

	Lift occurs when a moving flow of gas is turned by a solid object. The flow is turned in one direction, and the lift is generated in the opposite direction, according to Newton's Third Law of action and reaction. Because air is a gas and the molecules are free to move around, any solid surface can deflect a flow. For an aircraft wing, both the upper and lower surfaces contribute to the flow turning. Neglecting the upper surface's part in turning the flow leads to an incorrect theory of lift.
	
	The proponents of the arguments usually fall into two main camps: 
	\begin{enumerate}
		\item Those who support the "Bernoulli" position that lift is generated by a pressure difference across the wing

		\item Those who support the "Newton" position that lift is the reaction force on a body caused by deflecting a flow of gas
	\end{enumerate}
	Notice that we place the names in quotation marks because neither Newton nor Bernoulli ever attempted to explain the aerodynamic lift of an object. 
	
	Lift is generated by the difference in velocity between the solid object and the fluid. There must be motion between the object and the fluid: no motion, no lift.... It makes no difference whether the object moves through a static fluid, or the fluid moves past a static solid object. 
	
	Which camp is correct? How is lift generated? In fact there is no right or wrong camp. All different effects are superposed and have more or left amplitude depending on the configuration of the wing (Angle Of Attack (AOA)\index{angle of attack}).

	When a gas flows over an object, or when an object moves through a gas, the molecules of the gas are free to move about the object; they are not closely bound to one another as in a solid. Because the molecules move, there is a velocity associated with the gas. Within the gas, the velocity can have very different values at different places near the object. Bernoulli's equation, which was named for Daniel Bernoulli, relates the pressure in a gas to the local velocity; so as the velocity changes around the object, the pressure changes as well. Adding up (integrating) the pressure variation times the area around the entire body determines the aerodynamic force on the body. The lift is the component of the aerodynamic force which is perpendicular to the original flow direction of the gas. The drag is the component of the aerodynamic force which is parallel to the original flow direction of the gas. Now adding up the velocity variation around the object instead of the pressure variation also determines the aerodynamic force. The integrated velocity variation around the object produces a net turning of the gas flow. From Newton's third law of motion, a turning action of the flow will result in a re-action (aerodynamic force) on the object. So both "Bernoulli" and "Newton" are correct. Integrating the effects of either the pressure or the velocity determines the aerodynamic force on an object. We can use equations developed by each of them to determine the magnitude and direction of the aerodynamic force.
	
	What is the argument?

	Arguments arise because people mis-apply Bernoulli and Newton's equations and because they over-simplify the description of the problem of aerodynamic lift. The most popular incorrect theory of lift arises from a mis-application of Bernoulli's equation. The theory is known as the "equal transit time" or "longer path" theory which states that wings are designed with the upper surface longer than the lower surface, to generate higher velocities on the upper surface because the molecules of gas on the upper surface have to reach the trailing edge at the same time as the molecules on the lower surface. The theory then invokes Bernoulli's equation to explain lower pressure on the upper surface and higher pressure on the lower surface resulting in a lift force. The error in this theory involves the specification of the velocity on the upper surface. In reality, the velocity on the upper surface of a lifting wing is much higher than the velocity which produces an equal transit time. If we know the correct velocity distribution, we can use Bernoulli's equation to get the pressure, then use the pressure to determine the force. But the equal transit velocity is not the correct velocity. Another incorrect theory uses a Venturi flow to try to determine the velocity. But this also gives the wrong answer since a wing section isn't really half a Venturi nozzle. There is also an incorrect theory which uses Newton's third law applied to the bottom surface of a wing. This theory equates aerodynamic lift to a stone skipping across the water. It neglects the physical reality that both the lower and upper surface of a wing contribute to the turning of a flow of gas.

	The real details of how an object generates lift are very complex and do not lend themselves to simplification. For a gas, we have to simultaneously conserve the mass, momentum, and energy in the flow. Newton's laws of motion are statements concerning the conservation of momentum. Bernoulli's equation is derived by considering conservation of energy. So both of these equations are satisfied in the generation of lift; both are correct. The conservation of mass introduces a lot of complexity into the analysis and understanding of aerodynamic problems. For example, from the conservation of mass, a change in the velocity of a gas in one direction results in a change in the velocity of the gas in a direction perpendicular to the original change. This is very different from the motion of solids, on which we base most of our experiences in physics. The simultaneous conservation of mass, momentum, and energy of a fluid (while neglecting the effects of air viscosity) are called the Euler Equations after Leonard Euler. Euler was a student of Johann Bernoulli, Daniel's father, and for a time had worked with Daniel Bernoulli in St. Petersburg. If we include the effects of viscosity, we have the Navier-Stokes equations. To truly understand the details of the generation of lift, one has to have a good working knowledge of the Euler Equations.
	
	Keep in mind that we may explain the same phenomena in many ways, one doesn't exclude the other. So now let us focus on all common possible calculations of lift in a special configuration case to be able to compare their effect numerically!
	
	\subsubsection{Newton's lift argument (skipping stone argument)}
	About paper planes... even a flat surface can produce a lift with an angle of attack. But the "equal time argument" (Bernoulli's equation) cannot explain such phenomenon.
	
	The Newton's lift is given by the relation (\SeeChapter{see section Continuum Mechanics page \pageref{stoke law}}) that we have determined during out study of Stoke's law:
	
	
	Let's use the information we've just learned to evaluate the "skipping stone argument":
	\begin{enumerate}
		\item This theory is concerned with only the interaction of the lower surface of the moving object and the air. It assumes that all of the flow turning (and therefore all the lift) is produced by the lower surface. Experiments shows the upper surface also turns the flow. In fact, when one considers the downwash produced by a lifting airfoil, the upper surface contributes more flow turning than the lower surface. This theory does not predict or explain this effect.

		\item Because this theory neglects the action/reaction of molecules striking the upper surface, it does not predict the negative lift present in our experiment when the angle of attack is negative. On the top of the airfoil, no vacuum exists. Molecules are still in constant random motion on the upper surface (as well as the lower surface), and these molecules strike the surface and impart momentum to the airfoil as well.
		
		\item The upper airfoil surface doesn't enter into the theory at all. So using this theory, we would expect two airfoils with the same lower surface but very different upper surfaces to give the same lift. We know this doesn't occur in reality. In fact, there are devices on many airliners named "spoilers" which are small plates on the upper surface, between the leading and trailing edges. They are used to change the lift of the wing to maneuver the aircraft by disrupting the flow over the upper surface. This theory does not predict or explain this effect.
		
		\item If we make lift predictions based on this theory, using a knowledge of air density and the number of molecules in a given volume of air, the predictions are totally inaccurate when compared to actual measurements. The chief problem with the theory is that it neglects the physical properties of the fluid. Lift is created by turning a moving fluid, and all parts of the solid object can deflect the fluid.
	\end{enumerate}
	But.... this theory is not totally inaccurate. In certain flight regimes, where the velocity is very high and the density is very low, few molecules can strike the upper airfoil surface and the Newtonian theory gives very accurate predictions. These are the conditions which occur on the Space Shuttle during the early phases of its re-entry into the Earth's atmosphere at altitudes above about $80$ [km] and at velocities above $13$ Mach. For these flight conditions, the theory gives a good prediction. However, for most normal flight conditions, like those on an airliner ($10$ [km], $800\;[\text{km}\cdot \text{h}^{-1}]$), this argument does not give the right answer.
	
	\subsubsection{Bernoulli's lift argument (equal time argument)}
	The "equal time argument" states that airfoils are shaped with the upper surface longer than the bottom. The air molecules have farther to travel over the top of the airfoil than along the bottom. In order to meet up at the trailing edge, the molecules going over the top of the wing must travel faster than the molecules moving under the wing. Because the upper flow is faster, then, from Bernoulli's generalized equation (\SeeChapter{see section Continuum Mechanics page \pageref{generalized bernoulli equation}}):
	
	the pressure is lower. The difference in pressure across the airfoil produces the lift.
	\begin{figure}[H]
		\centering
		\includegraphics[]{img/engineering/airfoil_bernoulli_argument.jpg}
		\caption[Bernoulli force pressure lift effect illustration]{Bernoulli force pressure lift effect illustration (source: HyperPhysics, author: Rod Nave)}
	\end{figure}
	This approach is a beautiful and simple way to explain lift but... in fact, as we have seen in it in the section of Continnum Mechanics, the Bernoulli's equation must be applied on a unique stream line and therefore it is wrong to take for the starting point and endpoint by mixing the both path as they are distinct stream lines!
	
	However would be a shame that a reader thinks the second Bernoulli is never applicable to flow around an airfoil. As already mentioned we must be careful with the assumptions of this theorem, but there are still relatively easy to satisfy:
	\begin{itemize}
		\item The flow has to be incompressible. With air, it is not strictly accurate to zero speed. However, as long as one remains in low subsonic, air can reasonably be considered an incompressible fluid (in practice, as long as one remains at a Mach number of less than $0.2$ or $0.3$). And even the first Bernoulli remains available for all the barotropic compressible fluids along a streamline.
		
		\item The fluid must be perfect. The air is a viscous fluid, though the phenomena related to the viscosity remain confined to the boundary layer outside the boundary layer, the fluid can be considered perfect.
		
		\item The flow must be irrotational. In fact, it is sufficient that the flow is irrotational globally (this is the general case). The fact that there is locally turbulence does not stop to apply more generally the theorem.
	\end{itemize}
	Let's use the information we've just learned to evaluate the various parts of this theory:
	\begin{itemize}
		\item Lifting airfoils are designed to have the upper surface longer than the bottom: This is not always correct. The symmetric airfoil also generates plenty of lift and its upper surface is the same length as the lower surface. Think of a paper airplane. Its airfoil is a flat plate, top and bottom exactly the same length and shape and yet they fly just fine. This part of the theory probably got started because early airfoils were curved and shaped with a longer distance along the top. Such airfoils do produce a lot of lift and flow turning, but it is the turning that's important, not the distance. There are modern, low-drag airfoils which produce lift on which the bottom surface is actually longer than the top. This theory also does not explain how airplanes can fly upside-down which happens often at air shows and in air-to-air combat. The longer surface is then on the bottom!

		\item Air molecules travel faster over the top to meet molecules moving underneath at the trailing edge: Experiments shows us that the flow over the top of a lifting airfoil does travel faster than the flow beneath the airfoil. But the flow is much faster than the speed required to have the molecules meet up at the trailing edge. Two molecules near each other at the leading edge will not end up next to each other at the trailing edge as shown experiments. This part of the theory attempts to provide us with a value for the velocity over the top of the airfoil based on the non-physical assumption that the molecules meet at the aft end. We can calculate a velocity based on this assumption, and use Bernoulli's equation to compute the pressure, and perform the pressure-area calculation and the answer we get does not agree with the lift that we measure for a given airfoil. The lift predicted by the "equal time argument" theory is much less than the observed lift, because the velocity is too low. The actual velocity over the top of an airfoil is much faster than that predicted by the "equal time argument" theory and particles moving over the top arrive at the trailing edge before particles moving under the airfoil.

		\item The upper flow is faster and from Bernoulli's equation the pressure is lower. The difference in pressure across the airfoil produces the lift: Experiments show that this part of the theory is correct. In fact, this theory is very appealing because many parts of the theory are correct. In our discussions on pressure-area integration to determine the force on a body immersed in a fluid, we mentioned that if we know the velocity, we can obtain the pressure and determine the force. The problem with the "Equal Transit" theory is that it attempts to provide us with the velocity based on a non-physical assumption as discussed above.
	\end{itemize}
	 \begin{tcolorbox}[title=Remark,colframe=black,arc=10pt]
	The misconception that wings must be curved on top and flat on the bottom is commonly associated with the previously-discussed misconception that the air is required to pass above and below the wing in equal amounts of time. In fact, an upside-down wing produces lift by exactly the same principle as a rightside-up wing.
	\end{tcolorbox}
	

	\subsubsection{Euler's lift argument}
	The simultaneous conservation of mass, momentum, and energy of a fluid (while neglecting the effects of air viscosity) use the Euler Equations and we have proved in the section of Continuum Mechanics:
	
	that was the Bernoulli equation was a special case of Euler's equation (as it is itself a special case of Navier-Stokes equations).
	
	\subsubsection{Coandă lift argument}
	In its original sense, the Coandă effect refers to the tendency of a fluid jet to stay attached to an adjacent surface that curves away from the flow, and the resultant entrainment of ambient air into the flow.
	
	More broadly, some consider the effect to include the tendency of any fluid boundary layer to adhere to a curved surface, not just the boundary layer accompanying a fluid jet. It is in this broader sense that the Coandă effect is used by some to explain why the air flow remains attached to the top side of an airfoil.
	
	\subsubsection{Kutta-Joukowski lift argument}
	Many discussion of airfoil lift invoke a vortex around the moving airfoil and it well know and experimentally proven that that there is a lift around a spinning cylinder. This is named the "\NewTerm{Kutta-Joukowski theorem}\index{Kutta-Joukowski theorem}" (Kutta–Joukowski theorem relates lift to circulation much like the Magnus effect relates side force (named Magnus force) to rotation).
	
	The two early aerodynamicists, Kutta in Germany and Joukowski in Russia, worked to quantify the lift achieved by an airflow over a spinning cylinder. The lift relation is given by:
	
	\begin{figure}[H]
		\centering
		\includegraphics[]{img/engineering/kutta_joukowski_lift_effect_illustration.jpg}
		\caption[Kutta-Joukowski lift effect illustration]{Kutta-Joukowski lift effect illustration (source: HyperPhysics, author: Rod Nave)}
	\end{figure}
	Like all aerodynamic lift, this seems a bit mysterious, but it can be looked at in terms of a redirection of the air motion. If the cylinder traps some air in a boundary layer at the cylinder surface and carries it around with it, shedding it downward, then it has given some of the air a downward momentum. That can act to give the cylinder an upward momentum in accordance with the principle of conservation of momentum. Another approach is to say that you have exerted a downward component of force on the air and by Newton's 3rd law there must be an upward force on the cylinder. Yet another approach is to say that the top of the cylinder is assisting the airstream, speeding up the flow on the top of the cylinder. Then by the Bernoulli equation, the pressure on the top of the cylinder is diminished, giving an effective lift.
	
	We have seen that several physical principles are involved in producing lift. Each of the following statements is correct as far as it goes:
	\begin{itemize}
		\item The wing produces lift "because" it is flying at an angle of attack.
		\item The wing produces lift "because" of circulation.
		\item The wing produces lift "because" of Bernoulli's principle.
		\item The wing produces lift "because" of Newton's law of action and reaction.
	\end{itemize}

	\pagebreak
	\subsection{Cosmic speeds}
	Placing takeoff of a rocket near the equator will be of significant assistance since the tangential velocity (horizontal) of the rotation of the Earth is given by (applying the trivial circular kinematics studied in the section of Classical Mechanics):
	
	That is to says slightly more than the speed of sound on the ground in standard conditions of temperature and pressure, which represents about $5\%$ of the first cosmic speed we will deal with just a little further below (that's already something to optimize fuel usage!). And obviously, as the Earth rotates on itself from west to east it is better to leave the rocket to the East rather than straight or even worse ... to the West (note in passing that for security reasons of people on the ground it would be even better if the rocket can be sent to the East over the ocean...).
	
	Also note that we are seeking also to acquire of course vertical speed to quickly leave the dense layers of the atmosphere, but we also want to communicate to the launcher a horizontal velocity component: that is to say the velocity that allows to put satellites into orbit of the Earth or of something else less or far away (hence another reason to launch to the East).
	
	As about the difference of the gravitational acceleration it is also a factor to be taken into account. Indeed some specialized book provide:
	
	Thus, a difference of about $0.527\%$ of the weight of a space shuttle and this is not negligible to once again save fuel.
	
	c
	The "\NewTerm{first cosmic speed}\index{first cosmic speed}" or "\NewTerm{low-level orbit speed}\index{low-level orbit speed}" is the minimum velocity at which a body should be brought to lie in low orbit around the Earth. It is determined by the relation balance between centrifugal and centripetal force (\SeeChapter{see section Classical Mechanics page \pageref{centrifugal force}}):
	
	from which we deduce trivially:
	
	In fact in our present time (...) an orbit escapes to the Earth's atmosphere usury only if it is at an altitude greater Than $200$ [km]. We then have (speed that approximately corresponds to that of the International Space Station after-several overlapping):
	
	That is to say an orbital period of:
	
	
	The "\NewTerm{second cosmic speed}\index{second cosmic speed}" corresponds to the release velocity of a body leaving Earth. This is the minimum speed beyond which a body can move away permanently from Earth, at least... as we neglect the presence of the Sun, the Moon and of our Galaxy... We have already proven the corresponding relation in the section of Classical Mechanics so its no use to do it again. Let us just that the following relation was obtained:
	
	What gave for Earth (taking the average radius at the sea level...)
	
	
	Because of the Earth's atmosphere, it is difficult (and not very useful) to bring an object close to its surface at this speed, this speed being too far in the hypersonic regime to be achievable by most existing propulsion systems. In addition, it will cause a destruction of most objects by atmospheric friction or compression. In practice, an object is first placed in circular orbit of the Earth orbit and then accelerated from that altitude as the friction is there almost zero, the thrust has then a very good yield (see further below the study of Hohmann transfer orbit). In addition, the following table provided by NASA shows this is their strategy they use (you may notice that starting from $105$ kilometers of altitude the speed gain is very effective):
		\begin{table}[H]\centering
	\begin{center}
		\definecolor{gris}{gray}{0.85}
			\begin{tabular}{|c|c|c|c|}
				\hline
  \multicolumn{1}{c}{\cellcolor{black!30}\textbf{Time [s]}} & 
  \multicolumn{1}{c}{\cellcolor{black!30}\textbf{Altitude [m]}} & 
  \multicolumn{1}{c}{\cellcolor{black!30}\textbf{Speed [ms$^{-1}$]}} & 
  \multicolumn{1}{c}{\cellcolor{black!30}\textbf{Acceleration [ms$^{-2}$]}}\\ \hline
				$0$ & $-8$ & $0$ & $2.45$\\ \hline
				$20$ & $1,244$ & $139$ & $18.62$\\ \hline
				$40$ & $5,377$ & $298$ & $16.37$\\ \hline
				$60$ & $11,617$ & $433$ & $19.40$\\ \hline
				$80$ & $19,872$ & $685$ & $24.50$\\ \hline
				$100$ & $31,412$ & $1,026$ & $24.01$\\ \hline
				$120$ & $44,726$ & $1,279$ & $8.72$\\ \hline
				$140$ & $57,396$ & $1,373$ & $9.70$\\ \hline
				$160$ & $67,893$ & $1,490$ & $10.19$\\ \hline
				$180$ & $77,485$ & $1,634$ & $10.68$\\ \hline
				$200$ & $85,662$ & $1,800$ & $11.17$\\ \hline
				$220$ & $92.481$ & $1,986$ & $11.86$\\ \hline
				$240$ & $98.004$ & $2,191$ & $12.45$\\ \hline
				$260$ & $100,301$ & $2,417$ & $13.23$\\ \hline
				$280$ & $105,321$ & $2,651$ & $13.92$\\ \hline
				$300$ & $107,449$ & $2,915$ & $14.90$\\ \hline
				$320$ & $108,619$ & $3,203$ & $15.97$\\ \hline
				$340$ & $108,942$ & $3,516$ & $17.15$\\ \hline
				$360$ & $108,543$ & $3,860$ & $18.62$\\ \hline
				$380$ & $107,690$ & $4,216$ & $20.29$\\ \hline
				$400$ & $106,539$ & $4,630$ & $22.34$\\ \hline
				$420$ & $105,142$ & $5,092$ & $24.89$\\ \hline
				$440$ & $103,775$ & $5,612$ & $28.03$\\ \hline
				$460$ & $102,807$ & $6,184$ & $29.01$\\ \hline
				$480$ & $102,552$ & $6,760$ & $29.30$\\ \hline
				$500$ & $103,297$ & $7,327$ & $29.01$\\ \hline
				$520$ & $105,069$ & $7,581$ & $0.10$\\  \hline
 		\end{tabular}
		\end{center}
		\caption[Ascensional data of Discovery Shuttle]{Ascensional data of Discovery Shuttle (source: NASA)}
	\end{table}
	
	\subsection{Fundamental Equation of Propulsion (Tsiolkovsky rocket equation)}
	A space launcher has for mission most of time to place a load into orbit, for that it must operate in the atmosphere and vacuum. The principles used are those of the action and reaction of Newton, and the conservation of momentum (\SeeChapter{see section Classical Mechanics page \pageref{newton third law} and page \pageref{conservation of linear momentum}}): roughly we can say that a rocket accelerates high speed by ejecting gas (the balloon that deflates gives a good idea of the phenomenon). After a simple mechanical study, we can obtain the fundamental equation of propulsion.
	
	The "\NewTerm{Tsiolkovsky rocket equation}\index{Tsiolkovsky rocket equation}", or "\NewTerm{ideal rocket equation}\index{ideal rocket equation}", describes the motion of vehicles that follow the basic principle of a rocket: a device that can apply acceleration to itself (a thrust) by expelling part of its mass with high speed and thereby move due to the conservation of momentum. The equation relates the $\Delta v$ (the maximum change of velocity of the rocket if no other external forces act) with the effective exhaust velocity and the initial and final mass of a rocket (or other reaction engine).
	
	Considering the different mobile and independent parts of the rocket (in fact the main structure that ejected gas separately), we can state thanks to the principle of action and reaction (\SeeChapter{see section  Classical Mechanics page \pageref{newton third law}}) for a given system, the sum of the external forces are:
	
	Thus, taking the rocket system and gas together, we have:
	
	That is to say:
	
	The principle of propulsion is then stated through the conservation of linear momentum that we have just established.
	
	Let us consider a mass $m$ and of speed $v_f$ rocket, the ejection velocity of the gases being $v_g$. At time $t$, we have:
	
	at the time $t+\mathrm{d}t$ we have:
	
	but the $\mathrm{d}m$ is a mass loss so we must change its sign otherwise the above relation does not correspond to the interpretation of reality. So:
	
	based the principle of conservation of linear momentum:
	
	Therefore after simplification:
	
	hence:
	
	which by integration (\SeeChapter{see section Differential and Integral Calculus page \pageref{first order differential equations}}) gives the "\NewTerm{fundamental equation of propulsion}\index{fundamental equation of propulsion}" or the "\NewTerm{delta-v}\index{delta-v}" as often said in American literature and for a (non-relativistic ...) rocket outside from gravity field and in vacuum with a constant speed  of ejection of gas ...:
	
	The difference between the initial mass $m_i$ and final mass $m_f$ is often named "\NewTerm{dead mass}\index{dead mass}".
	
	Then we understand easily why rockets are composed of several staged propulsion elements. This allows them to increase their final speed in getting rid of the mass of tanks that they initially take with them.

	We conclude therefore that a launcher accelerates especially since the gas velocity is large, the thrust depends on the amount of gas supplied and speed, and the ratio of initial and final masses must be at maximum to ensure a good propulsion, that is to say that the final system structure must be as negligible as possible (final mass is then minimal).
	
	Let us notice that we also by extension the following relation between the acceleration of the rocket and the mass flow rate ejected:
	
	However, in a gravitational field (of course it is necessary that the acceleration of the rocket is greater than that of gravity at any time ...) we must add the term which slows down the rocket (the "loss by gravity") and this will give us the expression:
	
	if we assume the gravity $g$ as constant during the main acceleration phase. The latter relationship is the "\NewTerm{Tsiolkovsky formula}\index{Tsiolkovsky formula}". Denoting by $D_e$ the mass flow of propellant, we sometimes find this last relation in the following form where time no longer explicitly intervenes:
	
	as:
	
	However, it should also take into account the variation of gravity depending on the distance as we have proved it in the section Astronomy. Then we have:
	
	So if we assume the ejection speed of the gas as constant and the trajectory path in straight line from the main body of attraction, then we have that more time passes more will the rocket speed increases due at its mass decrease (of course it will at the end be constant) but at the same time less the influence of gravity is high.
	
	The theoretical value of the altitude reached, even if its expression is very simple to determine, is so false in our point of view that it is not useful to present it.
	
	The rockets do not go at the release velocity with only one propulsion stage and even ... they often have for only objective of going only to the low-orbit speed... (first cosmic speed ). The rochet is therefore in practice not freed of gravity, far from it!
	\begin{tcolorbox}[colframe=black,colback=white,sharp corners]
	\textbf{{\Large \ding{45}}Example:}\\\\
	Let us calculate the final velocity of the first phase of Ariane 5 launch assuming constant gravity, knowing the ejection gas velocity (assumed as constant speed), the initial total mass of the rocket, the ejected mass and the mass flow (...also assumed constant). Then we have:
	
	with:
	
	While the actual communicated value is between $2,000$ and $2,800\; [\text{m}\cdot\text{s}^{-1}]$ after the cross-checking of the information given by several websites and videos of the launch control center of Ariane (so we are very far from the first cosmic speed! ). If we add the variation of gravity with altitude the calculated result would be even closer to $2,800\; [\text{m}\cdot\text{s}^{-1}]$ (for information separation of the first stage boosters seems to be about $80$ [km] of altitude and after the cross-checking of several sources about $132$ [s] after ignition).\\
	
	However, this remains an acceptable order of (and similar to the US shuttle takeoffs!!) given that we did ot take into account of the friction of the air (that for recall was proven in the section of Continuum Mechanics as being proportional to the square of the velocity in the subsonic case... !!!).
	\end{tcolorbox}
	Let us now determine the distance reached after a given time $t$ in the case of constant gravitational field approximation. We then have obviously in a first time (we change a little bit the notation to condense the developments):
	
	Therefore:
	
	We will do the following change of variable:
	
	Therefore:
	
	Which give us (\SeeChapter{see section Differential and Integral Calculus page \pageref{usual primitives}}):
	
	Let us determine the integration constant the fact that at time $t = 0$, we must have $z (t)$ that is zero. It then comes immediately that:
	
	That is to say finally:
	
	where we can again get rid of the explicit time variable by reusing that:
	
	Which gives:
	
	\begin{tcolorbox}[colframe=black,colback=white,sharp corners]
	\textbf{{\Large \ding{45}}Example:}\\\\
	Let us calculate the height of the first launch phase of Ariane 5 assuming constant gravity with the same vaéies data as the previous example. This then gives:
	
	Which according to the cross-check of many website and videos of the Ariane  launch control center would not be too wrong because the main propellers (solid stages accelerators) are dropped at an altitude about $70$ to $125$ [km] (between $132$ and $205$ [s] after ignition) and even the cap that protects the satellites and the propulsion stage (main cryogenic stage) are sometimes dropped just a few seconds after (it always makes a few tons less! ).
	\end{tcolorbox}
	So as we have seen with the calculations of the previous example, there are still about $100$ [km] to reach to go to the low orbit and its corresponding cosmic speed using the storable propellant (EPS).
	
	Let us notice that if we take off the rocket with an acceleration equal to that of gravity (assuming the gravity constant), we would obtain a height equal to:
	
	So we can approximatively conclude that the propulsion acceleration of Ariane 5 is significantly higher than that of gravity.
	
	We can also calculate the change of the gravity at the altitude of $100$ [km] to see if the variation is significant or not in $\%$ of that at the ground level:
	
	So we see that this is $3\%$, which is quite significant.
	
	Furthermore, here is a diagram supposedly taken from a book of Arianespace showing the horizontal acceleration of the Ariane 5 in multiple of the local gravity on the launch center:
	\begin{figure}[H]
		\centering
		\includegraphics[]{img/cosmology/ariane_5_acceleration_profile.jpg}
		\caption[Horizontal acceleration profile of Ariane 5]{Horizontal acceleration profile of Ariane 5 (source: ?)}
	\end{figure}
	Now suppose that based on the relation:
	
	that we know the limited total mass ejected. We will then also have the combustion period that will be limited. Let us denote this by:
	
	Using the relations obtained earlier in the case of a constant gravitational field, we then have of obviously:
	
	After reaching the final velocity, the speed of the rocket will be supposed given by the classical relation of the rectilinear kinematics (\SeeChapter{see section Classical Mechanics page \pageref{kinematics of rectilinear motion}}):
	
	and also the height:
	
	By injecting the relations obtained above, we first have for the speed:
	
	and for the height:
	
	But when the maximum height is reached, the speed is zero, then we have:
	
	Then injected into the prior previous relation we get:
	
	The first two terms are positive. The third and last term is negative. Therefore, we see that to maximize the height reached, it is best to make the mass flow $D_e$ tends to infinity (ie: give all the linear momentum from the start!). Therefore, we have:
	
	\begin{tcolorbox}[colframe=black,colback=white,sharp corners]
	\textbf{{\Large \ding{45}}Example:}\\\\
	Let us alculate the maximum height of the first launch phase of Ariane 5 assuming constant gravity with the same values data as the previous example. This then gives:
	
	So this is the altitude at which the rocket would have a zero vertical speed and begin to fall. As we can see, it is well above the low orbit but well below the geostationary orbit as we will calculate thereafter.
	\end{tcolorbox}
	
	\subsection{Geostationary orbit}
	The geostationary orbit is an orbit at $35'786$ [km] altitude above the equator of the Earth in the equatorial plane and zero orbital eccentricity. This is a particular case of the geosynchronous orbit (otherwise the orbital period is always the period of revolution of the Earth orbit but the orbit then deviates north and south of the equator describing a analemma in the sky when viewed from a fixed point on the surface of the Earth).
	
	An object in such an orbit has an orbital period equal to the Earth's rotational period (one sidereal day) and thus appears motionless, at a fixed position in the sky, to ground observers. Communications satellites and weather satellites are often placed in geostationary orbits, so that the satellite antennas (located on Earth) that communicate with them do not have to rotate to track them, but can be pointed permanently at the position in the sky where the satellites are located. Using this characteristic, ocean color satellites with visible and near-infrared light sensors (e.g. the Geostationary Ocean Color Imager (GOCI)) can also be operated in geostationary orbit in order to monitor sensitive changes of ocean environments.
	
	Geostationary satellites are then necessarily located vertically or at the zentih of a point on the equator or, in other words, located in the equatorial plane of the Earth.
	
	To calculate the position of the geostationary orbit, we will first use the Newton's second law (\SeeChapter{see section Classical Mechanics page \pageref{newton second law}}):
	
	and we have shown in the chapter of classical mechanics when the movement is circular, we have for the centrifugal force:
	
	And we will use the law of gravity presented in the section of Classical Mechanics (and proved in the section General Relativity):
	
	We will use these relations with the mass of the Earth $M_\text{Earth}=5.9736\cdot 10^{24}$ [kg], $m_s$ the mass of the satellite, the average radius of the Earth $R_\text{Earth}=6,378.14$ [km] at the equator, $h$ the height of the satellite relatively to the ground and $v$ the speed of the satellite.
	
	On geostationary orbit, there is therefore by definition a balance between the attractive gravitational forces of the planet and the centrifugal force of the satellite:
	
	By adopting the notations given above so this gives:
	
	We see then that the mass of the satellite $m_s$ can be simplified. So the geostationary orbit is independent of it! It is also important to notice that since the speed $v$ is after simplification independent of the mass of the satellite in a circular orbit, while an astronaut inside it will have the same speed and will be weightless inside thereto (it is the same for all nearby objects inside or outside of the satellite).
	
	The speed for a circular path is the ratio of the circumference of the circle on the period of time to travel the whole circumference. So we have:
	
	Therefore $T$ being equal by definition in the context geostationary orbit at the time of the day on Earth, we have taking the available tables:
	
	Let us come back to prior-previous relation after simplification:
	
	and by injecting in it the explicit relation of speed we get:
	
	Therefore:
	
	Therefore it comes finally:
	
	and the speed of the satellite in geostationary orbit is then:
	
	Note also that using the previous relations, if the radius of the orbit of a non-geostationary satellite is given and also the mass of the Earth is known to us, we can then also determine the period of revolution without having to know its speed:
	
	\begin{figure}[H]
		\centering
		\includegraphics[scale=0.4]{img/engineering/beoing_2006_geostationnary_satellites.jpg}
		\caption[Geostationary satellites as given by public data in 2006]{Geostationary satellites as given by public data in 2006 (source: Boeing)}
	\end{figure}
	But let us now calculate the valueo of $g$ at the alititue of the International Space Station (ISS):
	
	We often see video of astronauts in space stations, apparently weightless. But clearly, the force of gravity is acting
on them. Comparing the value of $g$ we just calculated to that on Earth ($9.80\;[\text{m}\cdot \text{s}^{-1}]$) , we see that the astronauts in the International Space Station still have $88\%$ of their weight. In fact they only appear to be weightless because they are in free fall!!
	\begin{figure}[H]
		\centering
		\includegraphics[scale=0.5]{img/engineering/space_station_over_earth.jpg}
		\caption[International Space Station]{International Space Station (source: NASA)}
	\end{figure}
	Centrifugal force is obviously often used in space engineering calculations. For example, to calculate the speed that should have a circular space laboratory of radius $r$ to simulate terrestrial gravity for attached item on the ground as illustrated below:
	\begin{figure}[H]
		\centering
		\includegraphics{img/cosmology/spatial_station_simulate_gravity.jpg}
	\end{figure}
	we simply apply the analysis of centrifugal force again:
	
	where $g$ is the desired gravity to simulate.
	
	and we see immediately that the mass vanish to get only:
	
	Now if we considered that $2$ revolutions per minute is safe psychologically for an astronaut (it does not disturbs our brain too much). Then we get using the above relation $r\cong 200$ [m] (for comparison, in the movie \textit{2001: A Space Odyssey}, the rotating ring had a radius of $280$ [m]). So we can imagine the cost of such a structure in space and why it has not been realized until now...
	
	Once again, this circular cylinder only works as artificial gravity if you are touching it. So if you placed an basketball anywhere where it is not touching the cylinder, it would stay exactly still. If a person jumped to the center of the cylinder, he would indeed keep going!

	\subsection{Vis-Viva Equation}
	 The vis-viva equation, also referred to as "\NewTerm{orbital-energy-invariance law}\index{orbital-energy-invariance law}", is one of the equations that model the motion of orbiting bodies. It is the direct result of the principle of conservation of mechanical energy which applies when the only force acting on an object is its own weight.
	 \begin{tcolorbox}[title=Remark,colframe=black,arc=10pt]
	Vis viva (Latin for "live force") is a term from the history of mechanics, and it survives in this sole context. It represents the principle that the difference between the aggregate work of the accelerating forces of a system and that of the retarding forces is equal to one half the vis viva accumulated or lost in the system while the work is being done.
	\end{tcolorbox}
	In the vis-viva equation the mass $m$ of the orbiting body (e.g., a spacecraft) is taken to be negligible in comparison to the mass $M$ of the central body (e.g., the Earth). In the specific cases of an elliptical or circular orbit, the vis-viva equation may be readily derived from conservation of energy and momentum.

	Specific total energy is constant throughout the orbit. Thus, using the subscripts $a$ and $p$ to denote apoapsis (apogee) and periapsis (perigee), respectively (see the study of conics in the section of Analytical Geometry) we know that on any point of an elliptic orbit we have by conservation of energy:
	
	But using the first cosmic speed:
	
	We get (we will focus only on the apogee as the developments are the same for the perigee):
	
	Therefore:
	
	Notice that $E_\text{tot} < 0$. This is named a "\NewTerm{bound orbit}\index{bound orbit}".
	
	Let's equate the last special case expression with for any point of the orbit:
	
	Doing a little algebra and putting $a:=r_a$ we get:
	
	which gives the object's speed at any point along the orbit. This is the "\NewTerm{vis-viva equation}\index{vis-viva equation}". We often gives only the apogee version of this equation as it is the position used in Hohmann transfer orbit (see further below). It is interesting to notice that if $a=r$ (circular orbit) we fall back on the first cosmic speed (low-level orbit speed):
	
	This powerful equation does not depend on orbital eccentricity! For instance, if we observe a new object in the solar system and know its current velocity and distance, we can determine its orbital semi-major axis and thus have some idea where it came from.	
		
	Notice that the second cosmic speed (release velocity) can be obtained from the Vis-viva equation by taking the limit as $a$ approaches $+\infty$:
	
	
	\pagebreak
	\subsection{Hohmann Transfer orbit}
	 A Hohmann transfer is a transfer orbit which uses an ellipse where one end of the ellipse is at the starting planetary orbit and the other end is at the destination planet orbit in the same plane as shown in the figure below:
	\begin{figure}[H]
		\centering
		\includegraphics[]{img/cosmology/hohmann_transfer_between_earth_and_mars.jpg}
		\caption{Hohmann Transfer Between Earth and Mars}
	\end{figure}
	The orbital maneuver to perform the Hohmann transfer uses two engine impulses, one to move a spacecraft onto the transfer orbit and a second to move off it.
	
	 In the above the transfer orbit meets Earth's orbit at periapse and Mars' orbit at apoapse. The velocities needed at these two points can easily be found using the Vis Viva proved previously.
	 
	The figure above show that the initial circular orbit around the Earth and departure (transfer) orbit share a common point. Therefore the satellite must apply a maneuver at this point in order to change its velocity. The initial velocity of the spacecraft is that of an object in circular orbit a given as we know by the first cosmic speed (low-level orbit speed):
	
	
	a numerical application for Earth gives $7.73\;[\text{km}\cdot\text{s}^{-1}]$ when we want to put something at $300$ [km] height. The corresponding kinetic energy is immediate:
	
	
	But at the same time we know thanks to the Vis-Viva equation what must be the speed at a given point $r$ of an ellipse so that the satellite we later be on point $a$ (transfer orbit) :
	
	So as we know $r$ this is the common point between the Earth's orbit and transfer orbit and that we know $a$ that is the common point of the transfer orbit and Mars' orbit the difference gives us the speed to gain:
	
	And as we have (kinematics):
	
	where $a$ is the acceleration and not the apoapse, we can know what is the acceleration (and therefore the force) that has to be applied to the satellite as quick as possible to put it on its transfer orbit.
	
	If we want for example to take a satellite to elliptic orbit (transfer orbit) to pout it in geostationnary orbit ($42,300$ [km]) for it's actual orbit of ($300$ [km], ie $6,680$ [km] for Earth's center), we then have first by assimilating Earth's center naturally by the focus of this elliptic transfer orbit of semi-grand axis $a$:
	
	we have (taking $r=6,680$ [km]):
	
	So to our satellite will have to change from approximately $7.73\;[\text{km}\cdot \text{s}^{-1}]$ to $10.16\;[\text{km}\cdot \text{s}^{-1}]$ to be put in a transfer orbit that will take him at $42,300$ [km] of Earth.
	
	The corresponding energy is immediate:
	
	Therefore a energy variation of $\Delta E_c\cong 1.63\cdot 10^9$ [J].
	
	The last step is obviously to change the speed of the satellite when it is at the perigee of the transfer orbit. So first we calculate its speed when it is at the perigee of the orbit transfer:
	
	when the speed for a circular orbit at this same distance is:
	
	We have therefore to have to speed up the satellite from $1.6\;[\text{km}\cdot \text{s}^{-1}]$ to $3.07\;[\text{km}\cdot\text{s}^{-1}]$, to transform the elliptic transfer orbit into the final circular geostationary orbit.
	
	And again as the necessary kinetic energy of the geostationnary orbit is:
	
	Therefore the energy that we must give for the last time to satellite is:
	
	
	Finally, one can calculate the transit time for the Hohmann transfer to complete. As seen in the figure above, the satellite on the transfer orbit travels through $\pi$ ($180^\circ$) of true anomaly. Thus the transfer time can be found by taking half of the elliptic orbit period given by the third's Kepler law (\SeeChapter{see section Astronomy page \pageref{third kepler law}}):
	
	 
	The Hohmann transfer for which the above analysis applies is often used because it is a highly efficient transfer. Because the Hohmann travels through $\pi$ of true anomaly, it allows the satellite to apply tangential burns which achieve maximum $\Delta v$ for minimum fuel consumption. However, some missions may wish to consider other transfer variables, such as launch window or transfer time, over $\Delta v$. These missions may consider type I or II trajectories. A type I trajectory travels through less than $\pi$ of true anomaly while a type II trajectory travels through greater than $\pi$.
	
	\begin{flushright}
	\begin{tabular}{l c}
	\circled{25} & \pbox{20cm}{\score{3}{5} \\ {\tiny 22 votes,  59.09\%}} 
	\end{tabular} 
	\end{flushright}

	%to make section start on odd page
	\newpage
	\thispagestyle{empty}
	\mbox{}
	\section{Software Engineering}
	\lettrine[lines=4]{\color{BrickRed}S}oftware Engineering will be defined in this section as the set of techniques or approaches to solving problems using university-level mathematics and we find frequently implemented in many IT solutions.
	
	The reader will find in the texts that will follow mathematical tools whose details have all been proved in other chapters and sections, but for whose the computer applications had no place in the theoretical parts of this book. We have also choosen to focus only algorithms discuessed and used in our trainings that seems to make problems to our students and trainees. The reader must indeed keep in mind that there are so many algorithms that their are jobs around the world and at least as many as their are humans...
	
	\begin{tcolorbox}[title=Remark,colframe=black,arc=10pt]
	If the reader is looking for mathematical methods to simulate 3D objects on a screen, he will have to refer to the section of Projective Geometry and if he is looking for machine learning/data mining/deep learning techniques he will have to refer to the section of Theoretical Computing.
	\end{tcolorbox}
	
	\subsection{Algorithm}
	In mathematics and computer science, an algorithm  is a self-contained step-by-step set of operations to be performed. Algorithms perform calculation, data processing, and/or automated reasoning tasks.

	An algorithm is an effective method that can be expressed within a finite amount of space and time and in a well-defined formal language for calculating a function. Starting from an initial state and initial input (perhaps empty), the instructions describe a computation that, when executed, proceeds through a finite number of well-defined successive states, eventually producing "output" and terminating at a final ending state. The transition from one state to the next is not necessarily deterministic; some algorithms, known as randomized algorithms, incorporate random input.

	Algorithms can be expressed in many kinds of notation, including natural languages, pseudocode, flowcharts, programming languages or control tables (processed by interpreters). Natural language expressions of algorithms tend to be verbose and ambiguous, and are rarely used for complex or technical algorithms. Pseudocode, flowcharts, drakon-charts and control tables are structured ways to express algorithms that avoid many of the ambiguities common in natural language statements. Programming languages are primarily intended for expressing algorithms in a form that can be executed by a computer, but are often used as a way to define or document algorithms.

	In computer systems, an algorithm is basically an instance of logic written in software by software developers to be effective for the intended "target" computer(s) to produce output from given input (perhaps null). An optimal algorithm, even running in old hardware, would produce faster results than a non optimal (higher time complexity) algorithm for the same purpose, running in more efficient hardware; that is why the algorithms, like computer hardware, are considered technology.
	
	\subsection{Dichotomic Search algorithm}
	In computer science, a dichotomic search is a search algorithm that operates by selecting between two distinct alternatives (dichotomies) at each step. It is a specific type of divide and conquer algorithm. A well-known example is "\NewTerm{"binary search}\index{binary search}".

	Abstractly, a dichotomic search can be viewed as following edges of an implicit binary tree structure until it reaches a leaf (a goal or final state). This creates a theoretical tradeoff between the number of possible states and the running time. We have proved in the section of Theoretical Computing that the complexits of such an algorithm was $\mathcal{O}(\log_2(n))$

	In the case of the search of a root of a function the dichotomic search is equivalent to the bisection method (\SeeChapter{see section Theoretical Computing page \pageref{bisection method}}) as the $x$-axes is ordered by construction.
	
	But let us do here a small recall first on the bisection method:
	
	\subsubsection{Bisection algorithm}
	
	The bisection method in mathematics is also root-finding method that repeatedly bisects an interval and then selects a subinterval in which a root must lie for further processing. It is a very simple and robust method, but it is also relatively slow. Because of this, it is often used to obtain a rough approximation to a solution which is then used as a starting point for more rapidly converging methods. The method is also named the "\NewTerm{interval halving method}\index{interval halving method}", the "\NewTerm{binary search method}\index{binary search method}", or the "\NewTerm{dichotomy method}\index{dichotomy method}".
	
	The method is applicable for numerically solving the equation $f(x) = 0$ for the real variable $x$, where $f$ is a continuous function defined on an interval $[a, b]$ and where $f(a)$ and $f(b)$ have opposite signs such that $f(a)f(b)<0$. In this case $a$ and $b$ are also said to bracket a root since, by the intermediate value theorem, the continuous function f must have at least one root in the interval $[a, b]$.

	At each step the method divides the interval in two by computing the midpoint $c = (a+b) / 2$ of the interval and the value of the function $f(c)$ at that point. Unless $c$ is itself a root (which is very unlikely, but possible) there are now only two possibilities: either $f(a)$ and $f(c)$ have opposite signs and bracket a root, or $f(c)$ and $f(b)$ have opposite signs and bracket a root. The method selects the subinterval that is guaranteed to be a bracket as the new interval to be used in the next step. In this way an interval that contains a zero of $f$ is reduced in width by $50\%$ at each step. The process is continued until the interval is sufficiently small.

	\begin{figure}[H]
		\centering
		\includegraphics{img/computing/root_bissection_method.jpg}
		\caption{Bisection method scheme}
	\end{figure}

	Explicitly, if $f(a)$ and $f(c)$ have opposite signs, then the method sets $c$ as the new value for $b$, and if $f(b)$ and $f(c)$ have opposite signs then the method sets $c$ as the new $a$. (If $f(c)=0$ then $c$ may be taken as the solution and the process stops.) In both cases, the new $f(a)$ and $f(b)$ have opposite signs, so the method is applicable to this smaller interval.
	
	The implementation, on a computer, of this method is particularly simple. The conditions to be satisfied being only that in the interval $[a,b]$:
	\begin{itemize}
		\item $f$ must be continuous

		\item $f$ must be monotone near the root $\bar{x}$
		
		\item $f(a)f(b)<0$ to sure that there is a root
	\end{itemize}
	
	The algorithm consists therefore in performing the following steps:
	\begin{enumerate}
		\item We fix $\varepsilon>0$ as upper bound of the admissible error tolerance.
		
		\item We calculate $x=(a+b)/2$

		\item We evaluation $f(x)$
		
		\item If $|f(x)|<\varepsilon$ then the job is done, we have to display $x$ and $f(x)$
	
		\item Otherwise we proceed as following
			\begin{enumerate}
				\item we replace $a$ by $x$ if $f(x)f(a)>0$.
	
				\item we replace $b$ by $x$ if $f(x)f(b)>0$ or $f(x)f(a)<0$.
				
				\item we go back in (2)
			\end{enumerate}
	\end{enumerate}
	The previous step (4) imposes the condition for stopping the calculations. Sometimes it is better to choose another criterion calculation ending. It requires the calculated solution to be contained in an interval of length equation containing the root $x^{*}$. This test is enunciate as follows:
	\begin{enumerate}
		\item[4'.] If $|b-a|<\varepsilon$, the job is finished and $x=(a+b)/2$ is displayed. It is for sure obvious that $|x-x^{*}|<\varepsilon/2$
	\end{enumerate}
	In pseudo-code (non-unique and not optimized):\\\\
	\begin{algorithm}[H]
	 \KwData{$a$,$ b$, $\varepsilon$ expression of $f$ }
	 \KwResult{$x^{*}$}
	 initialization\;
	$x=(a+b)/2$\;
	\While{$|f(x)|>\varepsilon$}{
	    \uIf{$f(x)f(a)>0$}{
     		$a:=x$\;
	 	}
		\uElseIf{$f(x)f(b)>0\; \vee \; f(x)f(a)<0$}{
			$b:=x$\;
		}
		$x=(a+b)/2$\;
	 }
	 Display $x,f(x)$\;
	 \caption{Proportional Parts bisection pseudo-code algorithm}
	\end{algorithm}
	The equivalent Maple 4.00b code is given by:
	
	\texttt{>zero:=proc(f,a,b,pre)
	local M;\\
	M:=f((a+b)/2);\\
	if abs(M)<pre then \\
	     RETURN((a+b)/2)\\
	elif M>0 then\\
	     zero(f,a,(a+b)/2,pre)\\
	else zero(f,(a+b)/2,b,pre)\\
	     fi\\
	end:}
	
	\pagebreak
	\subsubsection{Binary search algorithm}
	"\NewTerm{Binary search}\index{binary search}", also known as "\NewTerm{half-interval search}\index{half-interval search}" or "\NewTerm{logarithmic search\footnote{because of its complexity that we proved in the section of Theoretical Computing as being of order $\mathcal{O}(\log_2(n))$}}\index{logarithmic search}", is a search algorithm that finds the position of a target value within a sorted array. It compares the target value to the middle element of the array; if they are unequal, the half in which the target cannot lie is eliminated and the search continues on the remaining half until it is successful.

	Although the idea is simple, implementing binary search correctly requires attention to some subtleties about its exit conditions and midpoint calculation.
	\begin{figure}[H]
		\centering
		\includegraphics{img/computing/binary_search.jpg}
		\caption[Visualization of the binary search algorithm]{Visualization of the binary search algorithm where $4$ is the target value (source: Wikipedia)}
	\end{figure}
	
	So as we said a binary search works on sorted arrays. A binary search begins by comparing the middle element of the array $A$ with the target value $T$. If the target value matches the middle element, its position in the array is returned. If the target value is less or more than the middle element, the search continues the lower or upper half of the array respectively with a new middle element, eliminating the other half from consideration.

	Here is the pseudo-code:
	
	\begin{algorithm}[H]
		\KwData{Array $A=[A_0,\ldots,A_i,\ldots,A_{n-1}]$, $T$ }
		\KwResult{$i$}
		initialization\;
		$L:=0$\;
		$R:=n-1$\;
		\uIf{$L>R$}{
	     	Exit \; '\texttt{nothing to found so exit program}  
		}
		$i:=\text{E}[(L+R)/2]$\;
		\While{$A_i<>T$}{
			\uIf{$L>R$}{
	     		Exit \; '\texttt{found nothing so exit program}  
			}
			\uIf{$A_i<T$}{
		    	$L:=i+1$\;
			}
			\uElseIf{$A_i>T$}{
				$R:=i-1$\;
			}
			$i:=\text{E}[(L+R)/2]$\;
		}
			
		Display $i$\;
		\caption{Binary search pseudo-code algorithm}
	\end{algorithm}
	When the computer scientist and professor Jon Bentley assigned binary search as a problem in a course for professional programmers, he found that $90\%$ failed to provide a correct solution after several hours of working on it, and another study published in 1988 shows that accurate code for it is only found in five out of twenty textbooks.

	Furthermore, Bentley's own implementation of binary search, published in his 1986 book \textit{Programming Pearls}, contained an overflow error that remained undetected for over twenty years. The Java programming language library implementation of binary search had the same overflow bug for more than nine years.

	In a practical implementation, the variables used to represent the indices will often be of fixed size, and this can result in an arithmetic overflow for very large arrays. If the midpoint of the span is calculated as $(L+R)/2$, then the value of $L+R$ may exceed the range of integers of the data type used to store the midpoint, even if $L$ and $R$ are within the range. This can be avoided by calculating the midpoint as $L+(R-L)/2$.

	\pagebreak
	\subsection{Tower of Hanoi algorithm}
	The Tower of Hanoi is a mathematical childish game or puzzle. It consists of three rods, and a number of disks of different sizes which can slide onto any rod. The puzzle starts with the disks in a neat stack in ascending order of size on one rod, the smallest at the top, thus making a conical shape.

	The objective of the puzzle is to move the entire stack to another rod, obeying the following simple rules:
	\begin{itemize}
		\item Only one disk can be moved at a time.
		\item Each move consists of taking the upper disk from one of the stacks and placing it on top of another stack i.e. a disk can only be moved if it is the uppermost disk on a stack.
		\item No disk may be placed on top of a smaller disk.
	\end{itemize}
	With three disks, the puzzle can be solved in $7$ moves. The minimum number of moves required to solve a Tower of Hanoi puzzle is $2^n - 1$, where $n$ is the number of disks.

	Many people working in corporations and that never learned programming and wish to do so (typically to learn VBA) don't know before attending a programming course that most of time they don't have the capacity the conceptualize any algorithm. An excellent exercise used by teacher or trainers to exclude people from corporate training class is to ask them before the training as a test to write a general algorithm for the Hanoi tower as it is a very childish algorithm that a normal constituted brain should be able to write after $30$ minutes of reflection. The purpose is not tow write the most optimal algorithm must just to be able to write one...
	
	Approximately following my experience only $20-30\%$ of people are first able to write the solution in a human language as below:
	
	\begin{algorithm}[H]
	 \KwData{$n$ : number of disks}
	 initialization\;
	 \If{$n$ $\mathrm{is\;even}$}{
		 \nl\Repeat{until complete}{
		make the legal move between pegs A and B\;
		make the legal move between pegs A and C\;
		make the legal move between pegs B and C\;
	 	}
	   }
	   \If{$n$ $\mathrm{is\; odd}$}{
		\nl\Repeat{until complete}{
	 	make the legal move between pegs A and C\;
		make the legal move between pegs A and B\;
		make the legal move between pegs C and B\;
		}
	   }
	 \caption{Human intuitive pseudo-code algorithm of Hanoi Tower}
	\end{algorithm}
	
	Or a little bit better:
	
	\begin{algorithm}[H]
	\KwData{$n$ : number of disks}
	initialization\;
	\eIf{$n$ $\mathrm{is\; even}$}{
		\nl\Repeat{until complete}{
		make the legal move between pegs A and B\;
		make the legal move between pegs A and C\;
		make the legal move between pegs B and C\;
	 	}
	}{
		\nl\Repeat{until complete}{
	 	make the legal move between pegs A and C\;
		make the legal move between pegs A and B\;
		make the legal move between pegs C and B\;
		}
	}
	\end{algorithm}
	That is to say:
	\begin{figure}[H]
		\centering
		\includegraphics{img/computing/hanoi_algorithm_1.jpg}
		\includegraphics{img/computing/hanoi_algorithm_2.jpg}
	\end{figure}
	\begin{figure}[H]
		\centering
		\includegraphics{img/computing/hanoi_algorithm_3.jpg}
	\end{figure}
	\begin{figure}[H]
		\centering
		\includegraphics{img/computing/hanoi_algorithm_4.jpg}
		\caption[Hanoi Tower algorithm illustration]{Hanoi Tower $n=4$ algorithm illustration (source: Wikipedia)}
	\end{figure}
	
	But this is not a computer oriented pseudo-code!! If we request to the students or trainees to write a pseudo-code that is "computer oriented" that is to say to use arrays, instead of "pegs"... and use numbers instead of "disks" and write in a mathematics form the "is even" as computers don't know what is a even or odd number. When add such as constraint the number of people that successfully pass this test decrease to almost $10\%$ and this show an inability to have a structured reasoning.
	
	For example to move $n$ disks we will use numbers to identify uniquely each disk (as when first humans invented numbers to identify uniquely items in a collection):\\
	
	\begin{algorithm}[H]
	\small
	\KwData{$n$ : number of disks}
	$A$ = dim([\space],$n$)\;
	$B$ = dim([\space],$n-1$)\;
	$C$ = dim([\space],$n$)\;
	initialization\;
	\For{$i=1$ \KwTo $n$}{ 
	 $A[i]=(n+1)-i$
	}
	\If{$n>0$}{
	 	\eIf{$n\equiv 0 \mod(2)$ }{
		\nl\Repeat{$C$=$A$}{
			\eIf{$A[\mathrm{count}[A]]>B[\mathrm{count}[B]]$}{
				$B[\mathrm{count}[B]]=A[\mathrm{count}[A]]$
			}{
				$A[\mathrm{count}[A]]=B[\mathrm{count}[B]]$
			}
			\eIf{$A[\mathrm{count}[A]]>C[\mathrm{count}[C]]$}{
				$C[\mathrm{count}[C]]=A[\mathrm{count}[A]]$
			}{
				$A[\mathrm{count}[A]]=C[\mathrm{count}[C]]$
			}
			\eIf{$B[\mathrm{count}[B]]>C[\mathrm{count}[C]]$}{
				$C[\mathrm{count}[C]]=B[\mathrm{count}[B]]$
			}{
				$B[\mathrm{count}[B]]=C[\mathrm{count}[C]]$
			}
	 	}
		}{
		\nl\Repeat{$C$=$A$}{
			\eIf{$A[\mathrm{count}[A]]>C[\mathrm{count}[B]]$}{
				$C[\mathrm{count}[C]]=A[\mathrm{count}[A]]$\;
			}{
				$A[\mathrm{count}[A]]=C[\mathrm{count}[C]]$\;
			}
			\eIf{$A[\mathrm{count}[A]]>B[\mathrm{count}[B]]$}{
				$B[\mathrm{count}[B]]=A[\mathrm{count}[A]]$\;
			}{
				$A[\mathrm{count}[A]]=B[\mathrm{count}[B]]$\;
			}
			\eIf{$C[\mathrm{count}[C]]>B[\mathrm{count}[B]]$}{
				$B[\mathrm{count}[B]]=C[\mathrm{count}[C]]$\;
			}{
				$C[\mathrm{count}[C]]=B[\mathrm{count}[B]]$\;
			}
		}
	}
	}
	\caption{Hanoi Tower computer pseudo-code algorithm }
	\end{algorithm}
	This is obviously not optimal but a minimum we can expect from a human being. 
	
	\subsection{Sorting Algorithms}
	A sorting algorithm is an algorithm that puts elements of a list in a certain order. The most-used orders are numerical order and lexicographical order. Efficient sorting is important for optimizing the use of other algorithms (such as search and merge algorithms) which require input data to be in sorted lists; it is also often useful for canonicalizing data and for producing human-readable output. 

	There is a consequent list of sorting algorithms  (see Wikipedia for an exhaustive list of almost $20$ of them). As already mention at the beginning of this section we will focus only on the algorithms used by redactor of this book when he teacher computer science in schools or firms.

	\subsubsection{Bubble sort}
	Bubble sort, sometimes referred to as sinking sort, is a simple sorting algorithm that repeatedly steps through the list to be sorted, compares each pair of adjacent items and swaps them if they are in the wrong order. The pass through the list is repeated until no swaps are needed, which indicates that the list is sorted. The algorithm, which is a comparison sort, is named for the way smaller elements "bubble" to the top of the list. Although the algorithm is simple, it is too slow and impractical for most problems.
	
	Bubble sort can operate in-place on an array, requiring small additional amounts of memory to perform the sorting.
	
	So let us see the idea behind the Bubble sort:
	\begin{figure}[H]
		\centering
		\includegraphics{img/computing/bubble_sort.jpg}
		\caption{Schematic principle the Bubble Sort algorithm}
	\end{figure}
	
	Here is the pseudo-code of this algorithm:
	
	\begin{algorithm}[H]
	 \KwData{$A$ : list of sortable items }
	 \KwResult{$x_1$}
	 initialization\;
	 $n :=$ length$(A)$\;
	 \nl\Repeat{not swapped}{
	 swapped: = false\;
	 \For{$i=1$ \KwTo $n-1$}{ 
	 \If{$A[i-1]>A[i]$}{
	 	swap($A[i-1]$,$A[i]$)\;
		swapped := true\;
	   }
	}
	}
	display $A$ : list of sorted items\;
	 \caption{Bubble sort pseudo-code algorithm}
	\end{algorithm}
	The bubble sort algorithm can be easily optimized by observing that the $n$-th pass finds the $n$-th largest element and puts it into its final place. So, the inner loop can avoid looking at the last $n-1$ items when running for the $n$-th time:
	
	\begin{algorithm}[H]
	 \KwData{$A$ : list of sortable items }
	 \KwResult{$x_1$}
	 initialization\;
	 $n =$ length$(A)$\;
	 \nl\Repeat{not swapped}{
	 swapped := false\;
	 \For{$i=1$ \KwTo $n-1$}{ 
	 \If{$A[i-1]>A[i]$}{
	 	swap($A[i-1]$,$A[i]$)\;
		swapped := true\;
	   }
	}
	$n:=n-1$\;
	}
	Display $A$ : list of sorted items\;
	 \caption{Alternative optimized Bubble sort pseudo-code algorithm}
	\end{algorithm}

	As the bubble sort has two nested loops, the worst-case and average complexity both $\mathcal{O}(n^2)$, where $n$ is the number of items being sorted (obviously when the list is already sorted (best-case), the complexity of bubble sort is only $\mathcal{O}(n^2)$). There exist many sorting algorithms with substantially better worst-case or average complexity of  $\mathcal{O}(n \log(n))$. Even other $\mathcal{O}(n^2))$ sorting algorithms, such as insertion sort, tend to have better performance than bubble sort. Therefore, bubble sort is not a practical sorting algorithm when $n$ is large.
	
	\subsubsection{QuickSort algorithm}
	Quicksort (sometimes named "partition-exchange sort") is an efficient sorting algorithm, serving as a systematic method for placing the elements of an array in order. Developed by Tony Hoare in 1959,[1] with his work published in 1961,[2] it is still a commonly used algorithm for sorting.

	Quicksort can as bubble sort operate "in-place" on an array, requiring small additional amounts of memory to perform the sorting.

	Quicksort is a divide and conquer algorithm. Quicksort first divides a large array into two smaller sub-arrays: the low elements and the high elements. Quicksort can then recursively sort the sub-arrays.

	The steps are:
	\begin{enumerate}
		\item Pick an element, named a pivot, from the array.

		\item Partitioning: reorder the array so that all elements with values less than the pivot come before the pivot, while all elements with values greater than the pivot come after it (equal values can go either way). After this partitioning, the pivot is in its final position. This is called the partition operation.

		\item Recursively apply the above steps to the sub-array of elements with smaller values and separately to the sub-array of elements with greater values.
	\end{enumerate}

	Here is an illustration of the concept:
	\begin{figure}[H]
		\centering
		\includegraphics{img/computing/quick_sort_algorithm.jpg}
		\caption{Schematic principle for QuickSort algorithm}
	\end{figure}
	 In pseudocode, a quicksort that sorts elements \texttt{lo} through \texttt{hi} (inclusive) of an array $A$ can be expressed as (Lomuto partition scheme\footnote{This scheme is attributed to Nico Lomuto and popularized by Bentley in his book \textit{Programming Pearls} and Cormen et al. in their book \textit{Introduction to Algorithms}.}
):
	 
	 \begin{algorithm}[H]
		\SetAlgoLined\DontPrintSemicolon
		\SetKwFunction{algo}{quicksort($A$, lo, hi)}
		\SetKwFunction{proc}{proc}
		\KwData{Global $A$ : list of sortable items }
		\SetKwProg{myalg}{Algorithm}{}{}
		\myalg{\algo{}}{
		\If{lo$<$hi}{
	 	p:=partition($A$, lo, hi)\;
		quicksort($A$, lo, $p-1$)\;
		quicksort($A$, $p+1$, hi)\;
	    }
		\nl \KwRet\;}{}
		\setcounter{AlgoLine}{0}
		\SetKwProg{myproc}{Procedure}{}{}
		\myproc{\proc{}}{
		pivot:=A[$hi$]\;
		\For{$j=$lo \KwTo hi$-1$}{ 
			\If{$A[i-1]\leq$ pivot}{
				swap($A[i]$,$A[j]$)\;
				$i:=i+1$\;
			}
		}
		swap($A[i]$,$A[hi]$)\;
		\nl  \KwRet $i$ \;}
  		\caption{QuickSort pseudo-code algorithm with function}
  	\end{algorithm}
  	
  	\subsection{Dijkstra's algorithm }
	The "\NewTerm{Dijkstra's algorithm}\index{Dijkstra's algorithm}" is an algorithm for finding the shortest paths between nodes in a graph, which may represent, for example, road networks. It was conceived by computer scientist Edsger W. Dijkstra in 1956 and published three years later.
	
	The algorithm exists in many variants; Dijkstra's original variant found the shortest path between two nodes,[2] but a more common variant fixes a single node as the "source" node and finds shortest paths from the source to all other nodes in the graph, producing a shortest-path tree.
	
	For a given source node in the graph, the algorithm finds the shortest path between that node and every other. It can also be used for finding the shortest paths from a single node to a single destination node by stopping the algorithm once the shortest path to the destination node has been determined. For example, if the nodes of the graph represent cities and edge path costs represent driving distances between pairs of cities connected by a direct road, Dijkstra's algorithm can be used to find the shortest route between one city and all other cities. As a result, the shortest path algorithm is widely used in network routing protocols, most notably IS-IS and Open Shortest Path First (OSPF). It is also employed as a subroutine in other algorithms.
	
	So how does a Dijkstra Algorithm work? I will explain this using four simple steps:
	\begin{enumerate}
		\item Assign to every node a tentative distance value: set it to zero for our initial node and to infinity for all other nodes.
	
		\item Set the initial node as current. Mark all other nodes unvisited. Create a set of all the unvisited nodes called the unvisited set.
	
		\item For the current node, consider all of its neighbors and calculate their tentative distances. Compare the newly calculated tentative distance to the current assigned value and assign the smaller one. 
	
		\item When we are done considering all of the neighbors of the current node, mark the current node as visited and remove it from the unvisited set. A visited node will never be checked again.
	
		\item If the destination node has been marked visited (when planning a route between two specific nodes) or if the smallest tentative distance among the nodes in the unvisited set is infinity (when planning a complete traversal; occurs when there is no connection between the initial node and remaining unvisited nodes), then stop. The algorithm has finished.
	
		\item Otherwise, select the unvisited node that is marked with the smallest tentative distance, set it as the new "current node", and go back to step 3.
	\end{enumerate}
	Suppose you would like to find the shortest path between two intersections on a city map: a starting point and a destination. Dijkstra's algorithm initially marks the distance (from the starting point) to every other intersection on the map with infinity. This is done not to imply there is an infinite distance, but to note that those intersections have not yet been visited; some variants of this method simply leave the intersections' distances unlabeled. Now, at each iteration, select the current intersection. For the first iteration, the current intersection will be the starting point, and the distance to it (the intersection's label) will be zero. For subsequent iterations (after the first), the current intersection will be a closest unvisited intersection to the starting point (this will be easy to find).
	
	From the current intersection, update the distance to every unvisited intersection that is directly connected to it. This is done by determining the sum of the distance between an unvisited intersection and the value of the current intersection, and relabeling the unvisited intersection with this value (the sum), if it is less than its current value. In effect, the intersection is relabeled if the path to it through the current intersection is shorter than the previously known paths. To facilitate shortest path identification, in pencil, mark the road with an arrow pointing to the relabeled intersection if you label/relabel it, and erase all others pointing to it. After you have updated the distances to each neighboring intersection, mark the current intersection as visited, and select an unvisited intersection with minimal distance (from the starting point) – or the lowest label—as the current intersection. Nodes marked as visited are labeled with the shortest path from the starting point to it and will not be revisited or returned to.

	Continue this process of updating the neighboring intersections with the shortest distances, then marking the current intersection as visited and moving onto a closest unvisited intersection until you have marked the destination as visited. Once you have marked the destination as visited (as is the case with any visited intersection) you have determined the shortest path to it, from the starting point, and can trace your way back, following the arrows in reverse; in the algorithm's implementations, this is usually done (after the algorithm has reached the destination node) by following the nodes' parents from the destination node up to the starting node; that's why we also keep track of each node's parent.

	This algorithm makes no attempt to direct "exploration" towards the destination as one might expect. Rather, the sole consideration in determining the next "current" intersection is its distance from the starting point. This algorithm therefore expands outward from the starting point, interactively considering every node that is closer in terms of shortest path distance until it reaches the destination. When understood in this way, it is clear how the algorithm necessarily finds the shortest path. However, it may also reveal one of the algorithm's weaknesses: its relative slowness in some topologies.
	%%%%%%%%%%%%%%%%%%%%% Styles for Dijktra algorithm %%%%%%%%%%%%%%%%%%%%%
	\pgfdeclarelayer{background}
	\pgfsetlayers{background,main}
	\tikzstyle{preno}=[semithick] % arêtes non orientées.
	\tikzstyle{pre}=[->,>=stealth,semithick] % arêtes orientées.
	\tikzstyle{select}=[-,cap=round,style=nearly transparent,line width=6pt] %
	\tikzstyle{fl_actif}=[-,cap=round,style=nearly transparent,line width=6pt,color=black!50!red]
	\tikzstyle{infini} =[circle,thick,inner sep=0pt,minimum size=6mm,node distance=20mm,draw=black!50,fill=blue!5]
	\tikzstyle{encours}=[circle,thick,inner sep=0pt,minimum size=6mm,node distance=20mm,draw=blue!50,fill=blue!20]
	\tikzstyle{fini}   =[circle,thick,inner sep=0pt,minimum size=6mm,node distance=20mm,draw=black!80,fill=black!30]
	\tikzstyle{actif}  =[circle,thick,inner sep=0pt,minimum size=6mm,node distance=20mm,draw=black!80!red,fill=red!20!black!30]
	
	\StickyNote[2.5cm]{\LARGE To finish translate and write before year 2020}[6.5cm]
	
	Now aim of this is to make the Dijkstra algorithm work on a small concrete example.

	Let us look for the shortest paths of origin $ A $ in this graph to node $C$:
	%%%%%%%%%%%%%%%%%%%%% Présentation du graphe %%%%%%%%%%%%%%%%%%%%%
	\begin{center}
	\begin{tikzpicture}[>=stealth,scale=1]
	  \node[encours,label={[red]left:A}]      (s)                            {};
	  \node[infini,label={[red]above:B}]      (t)        [above right of=s]  {};
	  \node[infini,label={[red]below:E}]      (y)        [below right of=s]  {};
	  \node[infini,label={[red]above right:C}]      (x)        [right of=t]  {};
	  \node[infini,label={[red]below right:D}]      (z)        [right of=y]  {};
	% flèches soulignées.
	
	% flèches et numéros
	\draw[pre] (s) to node[auto] {$10$} (t) ;
	\draw[pre] (s) to node[auto,swap] {$5$} (y) ;
	\draw[pre] (t) to node[auto] {$1$} (x) ;
	\draw[pre] (y) to node[auto,near end] {$9$} (x) ;
	\draw[pre] (y) to node[auto,swap] {$2$} (z) ;
	\draw[pre,bend right=20] (y) to node[auto,swap] {$3$} (t) ;
	\draw[pre,bend right=20] (t) to node[auto,swap] {$2$} (y) ;
	\draw[pre,bend right=20] (x) to node[auto,swap] {$4$} (z) ;
	\draw[pre,bend right=20] (z) to node[auto,swap] {$6$} (x) ;
	\draw[pre,bend left=80] (z) to node[auto] {$7$} (s) ;
	\end{tikzpicture}
	\end{center}
	
	We place ourselves on $A$ and we put the infinites values:
	%%%%%%%%%%%%%%%%%%%%% Étape 1 %%%%%%%%%%%%%%%%%%%%%
	\begin{center}
	\begin{tikzpicture}[>=stealth,scale=1]
	  \node[encours,label={[red]left:A}]      (s)                            {$0$};
	  \node[infini,label={[red]above:B}]      (t)        [above right of=s]  {$\infty$};
	  \node[infini,label={[red]below:E}]      (y)        [below right of=s]  {$\infty$};
	  \node[infini,label={[red]above right:C}]      (x)        [right of=t]        {$\infty$};
	  \node[infini,label={[red]below right:D}]      (z)        [right of=y]        {$\infty$};
	% flèches soulignées.
	\begin{pgfonlayer}{background}
	
	\end{pgfonlayer}
	% flèches et numéros
	\draw[pre] (s) to node[auto] {$10$} (t) ;
	\draw[pre] (s) to node[auto,swap] {$5$} (y) ;
	\draw[pre] (t) to node[auto] {$1$} (x) ;
	\draw[pre] (y) to node[auto,near end] {$9$} (x) ;
	\draw[pre] (y) to node[auto,swap] {$2$} (z) ;
	\draw[pre,bend right=20] (y) to node[auto,swap] {$3$} (t) ;
	\draw[pre,bend right=20] (t) to node[auto,swap] {$2$} (y) ;
	\draw[pre,bend right=20] (x) to node[auto,swap] {$4$} (z) ;
	\draw[pre,bend right=20] (z) to node[auto,swap] {$6$} (x) ;
	\draw[pre,bend left=80] (z) to node[auto] {$7$} (s) ;
	\end{tikzpicture}
	\end{center}
	We initialize the following table according to the algorithm:
	\vspace{7mm}
	\begin{center}
	\begin{tabular}{|c|c|c|c|c|} \hline
	A & B & C & D & E \\ \hline
	\fbox{$0$} & $\infty$ & $\infty$ & $\infty$ & $\infty$ \\ 
	$\bullet$ & & & & \\ 
	$\bullet$ & & & & \\ 
	$\bullet$ & & & & \\ 
	$\bullet$ & & & & \\ 
	$\bullet$ & & & & \\
	\end{tabular}
	\end{center}
	
	Each of the edges starting from the selected noded are studied:
	%%%%%%%%%%%%%%%%%%%%% Étape 2.1 %%%%%%%%%%%%%%%%%%%%%
	\begin{center}
	\begin{tikzpicture}[>=stealth,scale=1]
	  \node[actif,label={[red]left:A}]      (s)                            {$0$};
	  \node[infini,label={[red]above:B}]      (t)        [above right of=s]  {$10$};
	  \node[infini,label={[red]below:E}]      (y)        [below right of=s]  {$5$};
	  \node[infini,label={[red]above right:C}]      (x)        [right of=t]        {$\infty$};
	  \node[infini,label={[red]below right:D}]      (z)        [right of=y]        {$\infty$};
	% flèches soulignées.
	\begin{pgfonlayer}{background}
	\draw[fl_actif] (s) to (t) ;
	\draw[fl_actif] (s) to (y) ;
	\end{pgfonlayer}
	% flèches et numéros
	\draw[pre] (s) to node[auto] {$10$} (t) ;
	\draw[pre] (s) to node[auto,swap] {$5$} (y) ;
	\draw[pre] (t) to node[auto] {$1$} (x) ;
	\draw[pre] (y) to node[auto,near end] {$9$} (x) ;
	\draw[pre] (y) to node[auto,swap] {$2$} (z) ;
	\draw[pre,bend right=20] (y) to node[auto,swap] {$3$} (t) ;
	\draw[pre,bend right=20] (t) to node[auto,swap] {$2$} (y) ;
	\draw[pre,bend right=20] (x) to node[auto,swap] {$4$} (z) ;
	\draw[pre,bend right=20] (z) to node[auto,swap] {$6$} (x) ;
	\draw[pre,bend left=80] (z) to node[auto] {$7$} (s) ;
	\end{tikzpicture}
	\end{center}
	We report the distance to the adjacent node (keeping also the source as we must never use it anymore later!):
	\vspace{7mm}
	\begin{center}
	\begin{tabular}{|c|c|c|c|c|} \hline
	A & B & C & D & E \\ \hline
	\fbox{$0$} & $\infty$ & $\infty$ & $\infty$ & $\infty$ \\ 
	$\bullet $ & $10_A$ & $\infty$ & $\infty$ & $5_A$ \\ 
	$\bullet$ & & & & \\ 
	$\bullet$ & & & & \\
	$\bullet$ & & & & \\
	$\bullet$ & & & & \\
	\end{tabular}
	\end{center}
	In the columns, we put the distance to $A$, and the node from which we come from.

	We put ourselves on the node with the smallest weight, here $ E $ (the corresponding node color has been darken in the graph below to put it in evidence):	
	%%%%%%%%%%%%%%%%%%%%% Étape 2.2 %%%%%%%%%%%%%%%%%%%%%
	\begin{center}
	\begin{tikzpicture}[>=stealth,scale=1]
	  \node[actif,label={[red]left:A}]      (s)                            {$0$};
	  \node[infini,label={[red]above:B}]      (t)        [above right of=s]  {$10$};
	  \node[encours,label={[red]below:E}]      (y)        [below right of=s]  {$5$};
	  \node[infini,label={[red]above right:C}]      (x)        [right of=t]        {$\infty$};
	  \node[infini,label={[red]below right:D}]      (z)        [right of=y]        {$\infty$};
	% flèches soulignées.
	\begin{pgfonlayer}{background}
	\draw[fl_actif] (s) to (t) ;
	\draw[fl_actif] (s) to (y) ;
	\end{pgfonlayer}
	% flèches et numéros
	\draw[pre] (s) to node[auto] {$10$} (t) ;
	\draw[pre] (s) to node[auto,swap] {$5$} (y) ;
	\draw[pre] (t) to node[auto] {$1$} (x) ;
	\draw[pre] (y) to node[auto,near end] {$9$} (x) ;
	\draw[pre] (y) to node[auto,swap] {$2$} (z) ;
	\draw[pre,bend right=20] (y) to node[auto,swap] {$3$} (t) ;
	\draw[pre,bend right=20] (t) to node[auto,swap] {$2$} (y) ;
	\draw[pre,bend right=20] (x) to node[auto,swap] {$4$} (z) ;
	\draw[pre,bend right=20] (z) to node[auto,swap] {$6$} (x) ;
	\draw[pre,bend left=80] (z) to node[auto] {$7$} (s) ;
	\end{tikzpicture}
	\end{center}
	And we highlight in the table our new starting point having the minimal weight:
	\begin{center}
	\vspace{7mm}
	\begin{tabular}{|c|c|c|c|c|} \hline
	A & B & C & D & E \\ \hline
	$0$ & $\infty$ & $\infty$ & $\infty$ & $\infty$ \\ 
	$\bullet $ & $10_A$ & $\infty$ & $\infty$ & \fbox{$5_A$} \\ 
	$\bullet$ & & & & $\bullet$\\ 
	$\bullet$ & & & & $\bullet$\\ 
	$\bullet$ & & & & $\bullet$\\ 
	$\bullet$ & & & & $\bullet$\\ 
	\end{tabular}
	\end{center}

	And we iterate again the procedure:

	%%%%%%%%%%%%%%%%%%%%% Étape 3.1 %%%%%%%%%%%%%%%%%%%%%
	\begin{center}
	\begin{tikzpicture}[>=stealth,scale=1]
	  \node[fini,label={[red]left:A}]      (s)                            {$0$};
	  \node[infini,label={[red]above:B}]      (t)        [above right of=s]  {$8$};
	  \node[actif,label={[red]below:E}]      (y)        [below right of=s]  {$5$};
	  \node[infini,label={[red]above right:C}]      (x)        [right of=t]   {$14$};
	  \node[infini,label={[red]below right:D}]      (z)        [right of=y]   {$7$};
	% flèches soulignées.
	\begin{pgfonlayer}{background}
	\draw[select] (s) to (y) ;
	\draw[fl_actif,bend right=20] (y) to (t) ;
	\draw[fl_actif] (y) to (x) ;
	\draw[fl_actif] (y) to (z) ;
	\end{pgfonlayer}
	% flèches et numéros
	\draw[pre] (s) to node[auto] {$10$} (t) ;
	\draw[pre] (s) to node[auto,swap] {$5$} (y) ;
	\draw[pre] (t) to node[auto] {$1$} (x) ;
	\draw[pre] (y) to node[auto,near end] {$9$} (x) ;
	\draw[pre] (y) to node[auto,swap] {$2$} (z) ;
	\draw[pre,bend right=20] (y) to node[auto,swap] {$3$} (t) ;
	\draw[pre,bend right=20] (t) to node[auto,swap] {$2$} (y) ;
	\draw[pre,bend right=20] (x) to node[auto,swap] {$4$} (z) ;
	\draw[pre,bend right=20] (z) to node[auto,swap] {$6$} (x) ;
	\draw[pre,bend left=80] (z) to node[auto] {$7$} (s) ;
	\end{tikzpicture}
	\end{center}
	And we get:
	\begin{center}
	\vspace{7mm}
	\begin{tabular}{|c|c|c|c|c|} \hline
	A & B & C & D & E \\ \hline
	$0$ & $\infty$ & $\infty$ & $\infty$ & $\infty$ \\ 
	$\bullet $ & $10_A$ & $\infty$ & $\infty$ & \fbox{$5_A$} \\ 
	$\bullet $ & $8_E$ & $14_E$ & $7_E$ & $\bullet $\\ 
	$\bullet $ & & &  & $\bullet $\\ 
	$\bullet $ & & &  & $\bullet $\\ 
	\end{tabular}
	\end{center}

	And we put ourselves on node $D$ (highlighted in the figure below):
	%%%%%%%%%%%%%%%%%%%%% Étape 3.2 %%%%%%%%%%%%%%%%%%%%%
	\begin{center}
	\begin{tikzpicture}[>=stealth,scale=1]
	  \node[fini,label={[red]left:A}]      (s)                            {$0$};
	  \node[infini,label={[red]above:B}]      (t)        [above right of=s]  {$8$};
	  \node[actif,label={[red]below:E}]      (y)        [below right of=s]  {$5$};
	  \node[infini,label={[red]above right:C}]      (x)        [right of=t]   {$14$};
	  \node[encours,label={[red]below right:D}]      (z)        [right of=y]   {$7$};
	% flèches soulignées.
	\begin{pgfonlayer}{background}
	\draw[select] (s) to (y) ;
	\draw[fl_actif,bend right=20] (y) to (t) ;
	\draw[fl_actif] (y) to (x) ;
	\draw[fl_actif] (y) to (z) ;
	\end{pgfonlayer}
	% flèches et numéros
	\draw[pre] (s) to node[auto] {$10$} (t) ;
	\draw[pre] (s) to node[auto,swap] {$5$} (y) ;
	\draw[pre] (t) to node[auto] {$1$} (x) ;
	\draw[pre] (y) to node[auto,near end] {$9$} (x) ;
	\draw[pre] (y) to node[auto,swap] {$2$} (z) ;
	\draw[pre,bend right=20] (y) to node[auto,swap] {$3$} (t) ;
	\draw[pre,bend right=20] (t) to node[auto,swap] {$2$} (y) ;
	\draw[pre,bend right=20] (x) to node[auto,swap] {$4$} (z) ;
	\draw[pre,bend right=20] (z) to node[auto,swap] {$6$} (x) ;
	\draw[pre,bend left=80] (z) to node[auto] {$7$} (s) ;
	\end{tikzpicture}
	\end{center}
	So we put the new starting point and corresponding node in evidence in the table:
	\vspace{7mm}
	\begin{center}
	\begin{tabular}{|c|c|c|c|c|} \hline
	A & B & C & D & E \\ \hline
	$0$ & $\infty$ & $\infty$ & $\infty$ & $\infty$ \\ 
	$\bullet $ & $10_A$ & $\infty$ & $\infty$ & $5_A$ \\ 
	$\bullet $ & $8_E$ & $14_E$ & \fbox{$7_E$} & $\bullet $\\ 
	$\bullet $ & & & $\bullet $ & $\bullet $\\ 
	$\bullet $ & & & $\bullet $ & $\bullet $\\ 
	\end{tabular}
	\end{center}
	Now starting from node $D$ it remains only one possible path:
	%%%%%%%%%%%%%%%%%%%%% Étape 4.1 %%%%%%%%%%%%%%%%%%%%%
	\begin{center}
	\begin{tikzpicture}[>=stealth,scale=1]
	  \node[fini,label={[red]left:A}]      (s)                            {$0$};
	  \node[infini,label={[red]above:B}]      (t)        [above right of=s]  {$8$};
	  \node[fini,label={[red]below:E}]      (y)        [below right of=s]  {$5$};
	  \node[infini,label={[red]above right:C}]      (x)        [right of=t]        {$13$};
	  \node[actif,label={[red]below right:D}]      (z)        [right of=y]        {$7$};
	% flèches soulignées.
	\begin{pgfonlayer}{background}
	\draw[select] (s) to (y) ;
	\draw[select,bend right=20] (y) to (t) ;
	\draw[fl_actif,bend right=20] (z) to (x) ;
	\draw[select] (y) to (z) ;
	\end{pgfonlayer}
	% flèches et numéros
	\draw[pre] (s) to node[auto] {$10$} (t) ;
	\draw[pre] (s) to node[auto,swap] {$5$} (y) ;
	\draw[pre] (t) to node[auto] {$1$} (x) ;
	\draw[pre] (y) to node[auto,near end] {$9$} (x) ;
	\draw[pre] (y) to node[auto,swap] {$2$} (z) ;
	\draw[pre,bend right=20] (y) to node[auto,swap] {$3$} (t) ;
	\draw[pre,bend right=20] (t) to node[auto,swap] {$2$} (y) ;
	\draw[pre,bend right=20] (x) to node[auto,swap] {$4$} (z) ;
	\draw[pre,bend right=20] (z) to node[auto,swap] {$6$} (x) ;
	\draw[pre,bend left=80] (z) to node[auto] {$7$} (s) ;
	\end{tikzpicture}
	\end{center}
	
	\vspace{7mm}
	\begin{center}
	\begin{tabular}{|c|c|c|c|c|} \hline
	A & B & C & D & E \\ \hline
	$0$ & $\infty$ & $\infty$ & $\infty$ & $\infty$ \\ 
	$\bullet $ & $10_A$ & $\infty$ & $\infty$ & $5_A$ \\ 
	$\bullet $ & $8_E$ & $14_E$ & \fbox{$7_E$} & $\bullet $\\ 
	$\bullet $ & $8_E$ & $13_D$ & $\bullet$ & $\bullet $\\ 
	$\bullet$  &  &  & $\bullet$ & $\bullet$ \\ 
	$\bullet$  &  &  & $\bullet$ & $\bullet$ \\ 
	\end{tabular}
	\end{center}

	%%%%%%%%%%%%%%%%%%%%% Étape 4.2 %%%%%%%%%%%%%%%%%%%%%
	\begin{center}
	\begin{tikzpicture}[>=stealth,scale=1]
	  \node[fini,label={[red]left:A}]      (s)                            {$0$};
	  \node[encours,label={[red]above:B}]      (t)        [above right of=s]  {$8$};
	  \node[fini,label={[red]below:E}]      (y)        [below right of=s]  {$5$};
	  \node[infini,label={[red]above right:C}]      (x)        [right of=t]        {$13$};
	  \node[actif,label={[red]below right:D}]      (z)        [right of=y]        {$7$};
	% flèches soulignées.
	\begin{pgfonlayer}{background}
	\draw[select] (s) to (y) ;
	\draw[select,bend right=20] (y) to (t) ;
	\draw[fl_actif,bend right=20] (z) to (x) ;
	\draw[select] (y) to (z) ;
	\end{pgfonlayer}
	% flèches et numéros
	\draw[pre] (s) to node[auto] {$10$} (t) ;
	\draw[pre] (s) to node[auto,swap] {$5$} (y) ;
	\draw[pre] (t) to node[auto] {$1$} (x) ;
	\draw[pre] (y) to node[auto,near end] {$9$} (x) ;
	\draw[pre] (y) to node[auto,swap] {$2$} (z) ;
	\draw[pre,bend right=20] (y) to node[auto,swap] {$3$} (t) ;
	\draw[pre,bend right=20] (t) to node[auto,swap] {$2$} (y) ;
	\draw[pre,bend right=20] (x) to node[auto,swap] {$4$} (z) ;
	\draw[pre,bend right=20] (z) to node[auto,swap] {$6$} (x) ;
	\draw[pre,bend left=80] (z) to node[auto] {$7$} (s) ;
	\end{tikzpicture}
	\end{center}


	\vspace{7mm}
	\begin{center}
	\begin{tabular}{|c|c|c|c|c|} \hline
	A & B & C & D & E \\ \hline
	$0$ & $\infty$ & $\infty$ & $\infty$ & $\infty$ \\ 
	$\bullet $ & $10_A$ & $\infty$ & $\infty$ & $5_A$ \\ 
	$\bullet $ & $8_E$ & $14_E$ & $7_E$ & $\bullet $\\ 
	$\bullet $ & \fbox{$8_E$} & $13_D$ & $\bullet$ & $\bullet $\\ 
	$\bullet$  & $\bullet$ &  & $\bullet$ & $\bullet$ \\ 
	$\bullet$  & $\bullet$ &  & $\bullet$ & $\bullet$ \\ 
	\end{tabular}
	\end{center}
	

	%%%%%%%%%%%%%%%%%%%%% Étape 5 %%%%%%%%%%%%%%%%%%%%%
	\begin{center}
	\begin{tikzpicture}[>=stealth,scale=1]
	  \node[fini,label={[red]left:A}]      (s)                            {$0$};
	  \node[actif,label={[red]above:B}]      (t)        [above right of=s]  {$8$};
	  \node[fini,label={[red]below:E}]      (y)        [below right of=s]  {$5$};
	  \node[encours,label={[red]above right:C}]      (x)        [right of=t] {$9$};
	  \node[fini,label={[red]below right:D}]      (z)        [right of=y] {$7$};
	% flèches soulignées.
	\begin{pgfonlayer}{background}
	\draw[select] (s) to (y) ;
	\draw[select,bend right=20] (y) to (t) ;
	\draw[select] (y) to (z) ;
	\draw[fl_actif] (t) to (x) ;
	\end{pgfonlayer}
	% flèches et numéros
	\draw[pre] (s) to node[auto] {$10$} (t) ;
	\draw[pre] (s) to node[auto,swap] {$5$} (y) ;
	\draw[pre] (t) to node[auto] {$1$} (x) ;
	\draw[pre] (y) to node[auto,near end] {$9$} (x) ;
	\draw[pre] (y) to node[auto,swap] {$2$} (z) ;
	\draw[pre,bend right=20] (y) to node[auto,swap] {$3$} (t) ;
	\draw[pre,bend right=20] (t) to node[auto,swap] {$2$} (y) ;
	\draw[pre,bend right=20] (x) to node[auto,swap] {$4$} (z) ;
	\draw[pre,bend right=20] (z) to node[auto,swap] {$6$} (x) ;
	\draw[pre,bend left=80] (z) to node[auto] {$7$} (s) ;
	\end{tikzpicture}
	\end{center}
	
	\vspace{7mm}
	\begin{center}
	\begin{tabular}{|c|c|c|c|c|} \hline
	A & B & C & D & E \\ \hline
	$0$ & $\infty$ & $\infty$ & $\infty$ & $\infty$ \\ 
	$\bullet $ & $10_A$ & $\infty$ & $\infty$ & $5_A$ \\ 
	$\bullet $ & $8_E$ & $14_E$ & $7_E$ & $\bullet $\\ 
	$\bullet $ & $8_E$ & $13_D$ & $\bullet$ & $\bullet $\\ 
	$\bullet$  & $\bullet$ & \fbox{$9_B$} & $\bullet$ & $\bullet$ \\ 
	$\bullet$  & $\bullet$ & $\bullet$ & $\bullet$ & $\bullet$ \\ 
	\end{tabular}
	\end{center}

	%%%%%%%%%%%%%%%%%%%%% Étape 6 %%%%%%%%%%%%%%%%%%%%%
	\begin{center}
	\begin{tikzpicture}[>=stealth,scale=1]
	  \node[fini,label={[red]left:A}]      (s)                            {$0$};
	  \node[fini,label={[red]above:B}]      (t)        [above right of=s]  {$8$};
	  \node[fini,label={[red]below:E}]      (y)        [below right of=s]  {$5$};
	  \node[fini,label={[red]above right:C}]      (x)        [right of=t]        {$9$};
	  \node[fini,label={[red]below right:D}]      (z)        [right of=y]        {$7$};
	% flèches soulignées.
	\begin{pgfonlayer}{background}
	\draw[select] (s) to (y) ;
	\draw[select,bend right=20] (y) to (t) ;
	\draw[select] (y) to (z) ;
	\draw[select] (t) to (x) ;
	\end{pgfonlayer}
	% flèches et numéros
	\draw[pre] (s) to node[auto] {$10$} (t) ;
	\draw[pre] (s) to node[auto,swap] {$5$} (y) ;
	\draw[pre] (t) to node[auto] {$1$} (x) ;
	\draw[pre] (y) to node[auto,near end] {$9$} (x) ;
	\draw[pre] (y) to node[auto,swap] {$2$} (z) ;
	\draw[pre,bend right=20] (y) to node[auto,swap] {$3$} (t) ;
	\draw[pre,bend right=20] (t) to node[auto,swap] {$2$} (y) ;
	\draw[pre,bend right=20] (x) to node[auto,swap] {$4$} (z) ;
	\draw[pre,bend right=20] (z) to node[auto,swap] {$6$} (x) ;
	\draw[pre,bend left=80] (z) to node[auto] {$7$} (s) ;
	\end{tikzpicture}
	\end{center}

	\vspace{7mm}
	\begin{center}
	\begin{tabular}{|c|c|c|c|c|} \hline
	A & B & C & D & E \\ \hline
	$0$ & $\infty$ & $\infty$ & $\infty$ & $\infty$ \\ 
	$\bullet $ & $10_A$ & $\infty$ & $\infty$ & $5_A$ \\ 
	$\bullet $ & $8_E$ & $14_E$ & $7_E$ & $\bullet $\\ 
	$\bullet $ & $8_E$ & $13_D$ & $\bullet$ & $\bullet $\\ 
	$\bullet$  & $\bullet$ & $9_B$ & $\bullet$ & $\bullet$ \\ 
	$\bullet$  & $\bullet$ & $\bullet$ & $\bullet$ & $\bullet$ \\ 
	\end{tabular}
	\end{center}

	Here is the corresponding pseudocode:
	
	\begin{verbatim}
	 1  function Dijkstra(Graph, source):
	 2      for each vertex v in Graph: // Initializations
	 3          dist[v] := infinity     
	        // Unknown distance function from source to v
	 4          previous[v] := undefined
	        // Previous node in optimal path from source
	 5      dist[source] := 0           
	        // Distance from source to source
	 6      Q := the set of all nodes in Graph
	        // All nodes in the graph are unoptimized - thus are in Q
	 7      while Q is not empty:       // The main loop
	 8          u := vertex in Q with smallest dist[]
	 9          if dist[u] = infinity:
	10              break               
	        // all remaining vertices are inaccessible
	11          remove u from Q
	12          for each neighbor v of u:
	        // where v has not yet been removed from Q.
	13              alt := dist[u] + dist_between(u, v) 
	14              if alt < dist[v]:    // Relax (u,v,a)
	15                  dist[v] := alt
	16                  previous[v] := u
	17      return previous[]
	\end{verbatim}
	As we can see above, this Dijkstra simple algorithm has a complexity of order $\mathcal{O}(v^2)$ as two loops depending on $v$ (the number of vertices) are inside each other. Some algorithms are faster using various empirical techniques (adjacency lists or quantum computing).	
	
	\subsection{Google PageRank algorithm}
	At its creation in 1998, Google (search engine currently well know by many Internet users) has dominated in 2011 and still dominates the market for Internet search engines. His initial strength is based on the simple idea to sort the results by relevance, what did not make the other major search engines before him.
	
	The main idea behind the initial Google algorithm is interesting to present in this book because it is a practical example of purely mathematical concepts we have presented so far (graph theory and Markov chains in the respective sections of Graph Theory and Probability) and created one of the largest international companies in new technologies in the years 2000's (Google earns billions thanks to linear algebra ...) and even since the beginning of worldwide economy! As it is often the case in a market economy, it is primarily theoretical research (mathematical or physics) that helped to develop new tools!
	
	\begin{tcolorbox}[title=Remark,colframe=black,arc=10pt]
	\textbf{R1.} Obviously the basic algorithm has changed since 1998 because you can add many mathematical parameters (weighting by a distance for example) or empirical (visits, clicks, advertising, design, etc.).\\
	
	\textbf{R2.} This algorithm is not only used in the field of Internet. Indeed, it can be used to highlight any object in a multiple relation with other objects (e.g. variables in a project that influence each other by having direct or indirect relations: a structural analysis) or also to determiner influencers in a social network.
	\end{tcolorbox}
	
	To find an information in the amorphous and not really structured content that is the Internet (given the few developers who properly comply with the standards set by the W3C...), the user will search by keywords. This obviously requires for sure some preparation to be effective: the search engine copy first in the local memory all web pages and sorts extracted words (or combination of words) in alphabetical order using traditional empirical algorithms index of the IT domain (b trees, b+ trees or even others...). The result is a directory of keywords with their associated web pages.
	
	To analyze how Google identifies the most relevant web pages, we will ignore the content of pages and only count the links between them. What we get then is the structure of a graph (not necessarily connected!).
	
	The following image shows a miniature example (there would $25\cdot 10^9$ websites from Google in the early 2007):
	\begin{figure}[H]
		\centering
		\includegraphics[]{img/engineering/google_pagerank_schematic_idea.jpg}
		\caption{Schematic principle of the graph to study}
	\end{figure}
	For what follow, we will denote the web pages by:
	
	and we will write:
	
	if the page $P_j$ directly quotes the page $P_i$ (directly related citation or link). Thus, in the previous image, we have a link $1\longrightarrow 5$, for example, but no link $5\longrightarrow 1$.
	
	Obviously, a link $jj\longrightarrow i$ is a recommendation of the page $P_j$ to go read the page $P_i$. Thus it is a vote of $P_j$ in favor of the of the page $P_i$ (we then understand easily then easily why exchanging links between two websites with no contextual relation was somehow banned by Google).
	
	Let us analyze the previous image in a more consistent aspect of graph theory (see corresponding section in this book), which hierarchy will have to be justified to be as following:
	\begin{figure}[H]
		\centering
		\includegraphics{img/engineering/google_pagerank_graph_idea.jpg}
		\caption[]{Preceding figure into more mathematical aspect}
	\end{figure}
	For purely educational reasons, we will introduce three theoretical models of ranking web pages. The latest model is say to be the closest to that implemented by Google in 1998 and the first two models were just here to prepare the basis.
	
	\subsubsection{Weighted Count}
	It is highly possible that if a page is important, it gets a lot of links. With a bit of naivety, we can assume that the reverse is true (that is to say: if a page gets a lot of links, it means it is important). Thus, we could define the importance of $\mu_i$ of a page $P_i$ as the number of directly related links $j\rightarrow i$. In mathematical form, we write it in this way:
	In other words, $\mu_i$ is equal to the number of votes for the page $P_i$, where each vote contributes for the same value $1$. This is certainly easy to calculate, but often do not match with the importance felt by the user.

	Therefore in our above example we have:
	
	that is in front of (in terms of ranking):
	
	what is worse, is that this naive count can be too easily manipulated by adding pages (another Internet domain name for example) without interest recommending any other page.
	\begin{tcolorbox}[title=Remark,colframe=black,arc=10pt]
	So the maximum weight is verbatim equal to $n-1$.
	\end{tcolorbox}
	Some pages emit a lot of (hyper)links. They therefore seem less specific (specialized) and their weight will (must) therefore be lower. We therefore share the vote of the page $P_j$ in $l_j$ equal parts, where $l_j$ denotes the number of emitted links. So we can define a finer measurement:
	
	In other words, $\mu_i$ counts now the number of weighted votes for the page $P_i$. It is easy to calculate, but we always still remain with the above problem.
	
	In our example, we have therefore:
	
	and:
	
	and:
	
	\begin{tcolorbox}[title=Remark,colframe=black,arc=10pt]
	If all the pages were directly related (connex), then we would have equation for all $i$.
	\end{tcolorbox}
	
	\subsubsection{Recursive counting}
	Heuristically, a page $P_1$ seems important if many important pages cite it. This leads us to define the $\mu_i$ recursively as follows:
	
	It is therefore a system of homogeneous linear equations with $n$ unknowns, where in the chosen example $n=12$. More traditionally, the system may be denoted:
	
	or more explicitly (\SeeChapter{see section Linear Algebra page \pageref{linear algebra}}):
	\setcounter{MaxMatrixCols}{20}
	
	Which gives with our example:
	\setcounter{MaxMatrixCols}{20}
	
	In addition to the trivial solution (null vector), this system has an infinity of solutions because its determinant is zero (\SeeChapter{see section Linear Algebra page \pageref{determinant matrix inverse}}), which can easily be verified with the \texttt{MDETERM( )} function of the of Microsoft Excel 11.8346.

	Let us now write the previous system in the following equivalent form:
	\setcounter{MaxMatrixCols}{20}
	
	We notice two things before continuing:
	\begin{enumerate}
		\item In the above matrix, each column is such that the sum of its values equals unity. It is therefore the transpose of a stochastic matrix of a Markov chain (\SeeChapter{see section Probabilities page \pageref{markov chains}}).

		\item The importance $\mu_i$ is positive or zero. So the vector $\vec{\mu}$ is strictly positive and it is the stochastic vector of the Markov chain (\SeeChapter{see section Probabilities page \pageref{markov chains}}) whose sum of the components must be equal to unity.
	\end{enumerate}
	We will write this system in condensed form:
	
	Thus, $1$ is the eigenvalue and $\mu$ is the eigenvector of the application $P$. We know from our study of Markov chains that this relation is verified only for the invariant measure denoted traditionally $\pi$ in the domain of study of Markov chains:
	
	To determine the measure, it is necessary to first rewrite the network in the format of a matrix application of a Markov chain (\SeeChapter{see section Probabilities page \pageref{markov chains}}):
	
	With Microsoft Excel 11.8346, the modeling is quite simple enough to reproduce to determine the invariant measure with any starting point (any distribution) in the graph:
	\begin{figure}[H]
		\centering
		\includegraphics{img/computing/google_pagerank_graph_matrix_importance_vector_excel.jpg}
		\caption[]{Determining the Invariant Measure with Microsoft Excel 14.0.7173}
	\end{figure}
	with the following explicit formulas:
	\begin{figure}[H]
		\centering
		\includegraphics[scale=0.8]{img/computing/google_pagerank_graph_matrix_importance_vector_excel_explicti_formulas.jpg}
		\caption[]{Explicit formulas for determining the Invariant measure with Microsoft Excel 14.0.7173}
	\end{figure}
	The system converges fairly quickly, after the $30$th step, we already have a convergence to the second decimal:
	\begin{figure}[H]
		\centering
		\includegraphics[scale=1]{img/computing/google_pagerank_rank_vector.jpg}
		\caption[]{Convergence at the $35$th iteration}
	\end{figure}
	So finally the invariant measure is:
	
	This gives us the convergent distribution per page of a cohort of $17$ Internet users (sums of values of the stochastic vector).

	The pages where $\mu_i$ is large are the most popular in terms of equilibrium probabilities. In the quest to classify web pages, it is still an argument to use the measure $\mu$ as an empirical indicator.
	
	\begin{tcolorbox}[title=Remark,colframe=black,arc=10pt]
	The reader will be able to found in the R companion book how we plot the above network and calculate the invariant measure in a more efficient way that in Microsoft Excel...
	\end{tcolorbox}
	
	\subsubsection{Absorbing states}
	However, if our graph contains a page (or group of pages) without issue then this one absorbs all the probability, because our cohort will fall sooner or later on this page and will not be able to leave it ("absorbing state" as defined In the Probabilities section). We represent this in the following graphical form if, for example, we add a page $P_{13}$ that is an absorbing page:
	\begin{figure}[H]
		\centering
		\includegraphics[scale=1]{img/computing/graph_with_absorbing_state.jpg}
		\caption{Graph with an absorbent state}
	\end{figure}
	With the associated transition matrix:
	\setcounter{MaxMatrixCols}{20}
	
	Our model is not yet satisfactory. We could then construct a matrix that allows to escape the absorbent states by putting that any column that contains only one $1$ with only zeros teleport the visitor on all pages with an equal weight. Which would give us since we have $13$ pages:
	\setcounter{MaxMatrixCols}{20}
	
	But this is not very optimal, because it loads the matrix so it is very greedy in terms of memories and calculations.

	To escape the absorbing states, Google uses a more refined empirical model.

	We take the initial application which is a linear mapping:
	
	Where for reminder the sum of the components of $\mu$ is always $1$. We rewrite this in the form:
	
	where the $1$ represents the fact that there is a $100\%$ probability of having the state $P^T\mu$. Now the trick is to write:
	
	with:
	
	which means that the Internet user has at all times a probability $(1-c)$ of being in the state $P^T\mu$ and a probability $c$ of being in the state $\varepsilon$, that is to say to found ourselves with an equiprobable probability in any point of the graph. Finally, we sum up the two to obtain a stochastic vector (whose sum of the components is equal to $1$).

	Indeed, if we take for example $10\%$ for the value of $c$, we have the sum of the components of the vector:
	
	which will be equal to $0.9$ and respectively the sum of the components of the vector $\varepsilon c$ that will be equal to $0.1$ (since the sum of the components of $\varepsilon$ equals $1$ by construction).
	
	So to summarize, with a fixed probability $c$ the indexing engine (or the Internet user) abandons the current page $P_i$ and starts again on one of the $n$ pages of the web, this one being chosen, in order not to privilege anyone! Otherwise, with a probability $1-c$ the crawl engine follows one of the links of the page $P_j$, chosen equally. The only delicate point that remains is to calibrate the value of $c$ which is therefore between $0$ and $1$. If it is $0$ we fall back on the initial model with the problem of the absorbing states, if it is $1$ we have a model that gives an equal note to each page. It is therefore necessary to choose $c$ with low values close to $0$.

	This teleportation tip avoids being trapped by an absorbing page and guarantees to arrive anywhere in the graph, regardless of connectivity issues.

	We must put the whole back in the form of components, so we have the writing (whose second term converges):
	
	which has the enormous advantage of being able to calculate the value of any component $\mu_i$ of the stochastic vector of any step of a random walk on the graph and this without necessitating at any moment the manipulation of an immense matrix that would be too greedy in terms of memory capacity !!

	\begin{flushright}
	\begin{tabular}{l c}
	\circled{20} & \pbox{20cm}{\score{2}{5} \\ {\tiny 7 votes,  42.86\%}} 
	\end{tabular} 
	\end{flushright}

	%to make section start on odd page
	\newpage
	\thispagestyle{empty}
	\mbox{}
	\section{Industrial Engineering}\label{industrial engineering}
	\lettrine[lines=4]{\color{BrickRed}I}ndustrial Engineering involves the design, improvement and installation of systems. It uses the knowledge from the mathematical, physical and social sciences as well as the principles and methods specific to the art of engineering in order to specify, predict and evaluate the results arising from these systems.

We can summarize all areas affecting the industrial engineering (and not only... industrial this can apply with ad hoc adaptation to the administration) with the objective to optimize and monitor the overall performance of the business (costs , deadlines, quality) because:
\begin{center}
\textit{Only what is measured can be improved!}	
\end{center}
Note that some industrial engineering techniques have been discussed in other sections and subsections as: quantitative management techniques, optimization (operations research), financial analysis, queues analysis, etc. and that this section also includes "\NewTerm{Quality Engineering}\index{quality engineering}".

In this chapter we will only deal with the minimum theoretical aspects of SQC (Statistical Quality Control) relating to statistical quality control (that's the job of the "quality controller") in the manufacturing environment and into production of goods or services and which is the minimum-minimorum knowledge of any quality engineer in any organization (industrial or administrative) under penalty of having no credibility! Also beware of companies - especially multinationals - who are looking for quality specialists mastering Microsoft Excel or Microsoft Access. Because it means that they use non-professional tools to do a job which should be done with the appropriate tools (and Microsoft Excel or Microsoft Access are not) !!! So in terms of internal organization, you can ensure that these companies organize and analyze anything, anyhow, with an unsuitable tools and therefore that internally the organization is a general mess.

Depending on the use we distinguish three main areas of application of Industrial Engineering that are in the conventional order:
\begin{enumerate}
	\item "\NewTerm{Statistical process control}\index{statistical process control}", production monitoring and quality settings (Statistical Process Control, SPC). This is the monitoring of a manufacturing process for the mass production of products to discover the differences in quality and to be able to step in and drive directly corrective actions.
	
	The engineers must absolutely consult the norm ISO/TR 13425:2006 \textit{Guidelines for the selection of statistical methods in standardization and specification} as well as the norm ISO/TR 8258 \textit{Shewhart control cards} and finally the ISO/TR 18532 \textit{Guidelines for the application of statistical methods to quality and to industrial standardization} before implementing SPC tools within their company.
	
	\item "\NewTerm{Delivery check or receipt sample test (Acceptance Sampling)}\index{delivery check or receipt sample test (acceptance sampling)}". This is an input control, a control during production and a final product inspection in a company (or factory) without direct influence on the production. Thus, the amount of product is measured. The initial inspection is also used to reject the incoming goods. It therefore influences the production only indirectly.
	
	The engineers must absolutely consult the family of norms ISO 3591 \textit{Sampling procedures for inspection by variables and attributes} before setting up reception control tools within their company.
	
	\item Preventive maintenance and control of aging and failure and critical impacts best known as "\NewTerm{Failure Mode Analysis, Effects and Criticality Analysis FMEA"\index{Failure Mode Analysis, Effects and Criticality Analysis}}. This is mainly to calculate the lifespan of components or machines to provide replacements in advance and actions related thereto to be taken to avoid human or financial emergencies.
	
	The engineer must absolutely consult the norm IEC 61649 \textit{Weibull Analysis} and the norm NF EN 13306 \textit{Terminology for maintenance} before putting in place preventive maintenance tools within their company.
\end{enumerate}

These three areas using statistics in general, the engineer should always refer to the family of norms ISO 3534 \textit{Vocabulary and symbols}, ISO 3534-1 \textit{Probability and general statistical term}s, ISO 3534-2 \textit{Statistical Quality Control}, ISO 3534- 3 \textit{Design of experiments}.

For information, since the late 20th century, it is fashionable to combine the first two points in a working methodology named "Six Sigma" that we will study immediately. Finally, note that in practice, to get an interest of chiefs executives the engineer must always find a quantitative relations between non-quality and costs in order to make things change...

\subsection{Six Sigma}\label{six sigma}

Two objects are never perfectly identical. Whatever the techniques used to manufacture these objects or precise are the tools, there is variability in any production process. The aim of any industrial is that this natural variability remains within acceptable limits. This is a major concern in the improvement of industrial quality.

One of the tools used to work towards this quality is the "\NewTerm{Statistical Process Control S.P.C.}\index{statistical process control}" .If you produce a certain type of object, and if you want to keep your customers to sustain your business, you must ensure that the lots that you deliver them are consistent with what has been agreed between you, usually by contract. Any serious  industrial performs checks on produced lots to check the quality, that he is the producer of the goods or that he receives them. Various statistical techniques related to sampling are then used to avoid, in most cases, to check one by one all the objects in a batch. The samples control  taken in batches is essential if the controls at destroying the artefact, as in an analysis of the active ingredient dose contained in a tablet. However, there are cases where it is preferred to check all objects (for example it is desirable that the brakes of a vehicle works and a control of the braking function on a sample of produced cars does not guarantee that all vehicles have good braking...).

When a batch is controlled, it is conform or it is not. If it is conform, we deliver it (we are the supplier) or we accept it (we are the customer). If it is not conform, we can destroy it, check one by one all the elements and destroy only those who do not comply, etc. All solutions to handle non-conforming lots are expensive. If the batch does not comply, the damage is done. The S.P.C. methods aim to avoid producing non-conforming goods by monitoring production and intervening whenever anomalies are found. A good SPC implementation eliminates a significant number of controls of conformity by setting up statistical tools for monitoring manufacturing processes.

To resume, the S.P.C. is therefore to control the process during manufacturing and to act on the process rather than on the product if problems are detected. This approach tries to go up at the highest possible level of the production line to prevent the occurrence of defective product and goods. We discuss in this particular case of "process control".

This is a useful working methodology to satisfy customers and the idea of this methodology is to deliver products/services of quality, knowing that quality is inversely proportional to the variability. Moreover, the introduction of quality should be optimized so as not to overly increase costs. The subtle interplay between these two parameters (quality/costs) and their joint optimization is often associated with the term "\NewTerm{lean management}\index{lean management}". If we integrate Six Sigma, then we speak of "\NewTerm{Lean Six Sigma}\index{lean Six Sigma}".

Six Sigma integrates all aspects of the control of variability in business whether at the level of manufacturing, services, organization, marketing or management. Hence why it is such interesting! Moreover, in Six Sigma a defect must be paradoxically welcome, because it is a source of progress of an initially hidden problem. Then you have to ask several times the question "Why?" (traditionally 5 times) to better trace the source of it.

We distinguish two types of variability in practice:
\begin{enumerate}
	\item The "\NewTerm{inherent variability}\index{inherent variability}" in the process (and difficult to change) that induces the concept of measures distribution (usually accepted by businesses as a Normal distribution).
	\item The "\NewTerm{external variability}\index{external variability}" which induces most often a bias (deviation) in the distributions over time.
\end{enumerate}
Manufacturing processes in the high tech industry having a strong tendency to become terribly complex, it should be noted that the basic components used for each product are not always of equal quality or performance. And if in addition, production procedures are difficult to establish, the drifts will inevitably exists.

Whether for one reason or another, ultimately many products will be outside of the normality and will deviate of the range corresponding to the acceptable quality for the customer. This drift is very costly for the company. The management of waste, rework or customer returns for nonconformity generate substantial costs seriously amputating the expected benefits.

As we shall see in the texts, a pretty fair possible definition of Six Sigma is: solving problems based on the exploitation of data. Thus it is a scientific method of management!

\pagebreak
\subsubsection{Quality Control}

In the situation of quality studies in a business, we often give up a 100\% control because of the price it would generate. We then do a sampling. These must obviously be representative, that is to say, representatives and with equal probabilities (i.e. the mixture is good).

The purpose of sampling is obviously the real probability of failure of the whole lot on the basis of observed failures in the sampling.

Let us recall before going further that we have seen in the section of Statistics the hypergeometric distribution (and its interpretation) given by (\SeeChapter{see section Statistics page \pageref{hypergeometric distribution}}):
	
where the notation of the binomial coefficient is consistent with that defined and chosen in the section of Probabilities (therefore non-compliant with ISO 31-11).

During a sampling, we normally have a pack of n elements which we draw p of them.  Instead of taking m (remember it is an integer!) as the number of defective parts, we will implicitly defined it as being equal to:
	

where $p_d$ is the probability (assumed known or imposed...) that a part is defective. Thus, we have for probability to find $k$ defective parts in a sample of $p$ among $n$:
	
The cumulative probability of finding $k$ defective parts (between $0$ and $k$ defectives in other words) is then simply calculated with the cumulative hypergeometric distribution\label{quality control hypergeometric}:
	

	\begin{tcolorbox}[colframe=black,colback=white,sharp corners]
\textbf{{\Large \ding{45}}Example:}\\\\
In a batch of 100 machines, we assume that a maximum of 3 are defective (i.e. that is to say $p_d=3\%$). We conduct a control sampling $p$ at each output order of 20 machines.\\

We want to know at first what is the probability that in this sample of size $p$, 3 machines are defective and secondly what is the maximum number of defective machines allowed in this sample of size $p$ that says us with 90\% confidence that the batch of $n$ machines contain only 3 defective.

	\begin{table}[H]
	\begin{center}
		\definecolor{gris}{gray}{0.85}
			\begin{tabular}{|c|c|c|}
				\hline
				\multicolumn{1}{c}{\cellcolor{black!30}$x$} & 
  \multicolumn{1}{c}{\cellcolor{black!30}$H(x)$} & 
  \multicolumn{1}{c}{\cellcolor{black!30}$\displaystyle\sum_{x=0}^k H(x)$} \\ \hline
				 0 & 0.508 & 0.508 \\ \hline
				 1 & 0.391 & 0.899 \\ \hline
				 2 & 0.094 & 0.993 \\ \hline
				 3 & 0.007 & 1.000 \\ \hline
		\end{tabular}
	\end{center}
	\caption[]{Application of the hypergeometric distribution}
	\end{table}
	Thus, the probability of sampling in one series three defective machines in the sample of $20$ is $0.7\% (0.007)$ and the maximum number of defective machines allowed in this sample of $20$ that allows us with at least $90\%$ of confidence to have $3$ defectives is $1$ defective machine found in the sample (cumulative probability: $0.899$)!

	$H(x)$ values can be calculated easily with the English version of Microsoft Excel 11.8346. For example, the first value is obtained by the function:
	\begin{center}
	\texttt{=HYPGEOM.DIST(0,20,3,100,FALSE)\\=COMBIN(30,0)*COMBIN(97,20-0)/COMBIN(100,20)=0.508}
	\end{center}
	\end{tcolorbox}
	
\subsubsection{Defaults/Errors}

Let us now exhibit for the general culture and a practical and particular example of what is only a simple application of the theory of statistics and probabilities. To understand the importance of quality and the concept of "zero defects", let us consider the following general example of the inventor of the method:

For example, consider that the installation of a tourism car includes $2,500$ operations (pieces) and each operation is perfect $99$ times out of $100$. A priori, a having a successful operation in $99\%$ of cases is a sign of an almost perfect mastery of the quality. But in fact, the perfection of the whole assumes that $2,500$ times, the operations are fully perfectly realized! If the daily production is of $2,000$ units, one the $5$ million operations that are done daily in our factory, there is $100\%-99\%=1\%$ of defects assemblies that is to say $50,000$, and an average of $25$ defects per car, which is hardly acceptable. Also, let us imagine  this imaginary factory is equipped with a control service sampling systematically at the end of the assemblies line. This represents a considerable cost in working hours for control. If the defects can be corrected, this will require rework, replace perhaps some parts and work in unexpected conditions to correct the defects. If the defects are too important, these defects makes the final product unusable, and scrap are extremely expensive depending on the product. Even worse: if the control service sees $99\%$ of defects it remains about $500$  of them daily and this requires costly repairs and returns and a significant deterioration of the image depending on the performance of competitors.

This example serving as a school case study, imagine now a company manufacturing $3$ copies of a same product coming out of the same production line, each copy being composed of $8$ elements.

	\begin{tcolorbox}[title=Remark,colframe=black,arc=10pt]
We can just as easily imagine a services company developing (manufacturing) 3 copies of a software (product) out of the same development team (production line), each composed of an equal number of modules (elements).
	\end{tcolorbox}

Suppose that the product $P1$ has $1$ defects in average, the product $P2$ has $0$ defects $P3$ has $2$ defects and that each product is composed of 8 items.

Here Six Sigma implicitly assumes that defects are independent variables, which is relatively rare in machinery production lines but more common in chains in which humans are involved. However, we can consider in the application of the S.P.C. on machines that a sampling in time in the measurement process is equivalent to having a random independent variable!!

	\begin{tcolorbox}[title=Remarks,colframe=black,arc=10pt]
\textbf{R1.} As part of the example above with the software, independence is unlikely if we do not take an example in which the programming modules are customized according to customer needs.\\\\
\textbf{R2.} The inconstancy of production results of certain machines whose settings move during operation... (which is common!), and also where raw material quality changes during production (which is also common!) generates major issues in S.P.C. methods and is this inconstancy problem is the heart of Six Sigma.
	\end{tcolorbox}
The arithmetic average of the defects is named in the Six Sigma methodology "\NewTerm{Defects Per Unit (D.P.U.)}\index{defects per unit}\index{defects per unit}" and is defined by:
	
And this will give with our small example above:
	
which means that on average each product has a design or manufacturing defect. 
	\begin{tcolorbox}[title=Remark,colframe=black,arc=10pt]
Warning! This value is not a probability for simple reasons it may initially be higher than 1 and then it has the dimension of [defects]/[products].
	\end{tcolorbox}

Similarly, the analysis can be done with the total number of possible defective elements that make up the product so that we are led naturally to define following the Six Sigma methodology the "\NewTerm{Defects per Unit Opportunity D.P.O.}\index{defects per unit opportunity}":
	
And this will give with our small example:
	
and this can be seen as the probability of a default product element since it is a dimensionless value and the value is always less than or equal $1$:
	
By extension, we can argue that $87.5\%$ of the components of a unit does not have defects and because Six Sigma enjoys working with examples of the order of a million (it's more awesome...) we then define the "\NewTerm{Defects per Million Opportunities D.P.M.O.}\index{defects per million opportunities}" as:
	
And this will give wit our small example:
	
	As $D_i$ is the probability that an element $i$ is not defective is $87.5\%$ (that is to says $12.5\%$ of scrap) then, by the axiom of joint probabilities (\SeeChapter{see section Probabilities page \pageref{joint probability}}), the probability that product as a whole is not defective if all elements are serial is:
	
	This is not good...

	In Six Sigma, the joint probabilities are also naturally used to calculate the joint probability of non-defective products in a production process connected in series or to calculate the defect probability of an administrative serial workflow. This joint probability is named in Six Sigma the "\NewTerm{Rolled throughput Yield R.T.Y.}\index{Rolled throughput Yield }" and is denoted by:
	
	\pagebreak

	\begin{tcolorbox}[colframe=black,colback=white,sharp corners]
	\textbf{{\Large \ding{45}}Example:}\\\\
	Consider the following serial process/workflow or production line with its given yields (non-default rates):

	\begin{figure}[H]
		\centering
		\includegraphics{img/engineering/process_rty.eps}
		\caption{Serial Process/Workflow}
	\end{figure}
	
	We find ourselves in final with a $65.6\%$ reliability that is to say a cumulative probability of default for the entire process/workflow of $34.4\%$ (you can also imagine that these are four tasks in sequence of a project!).
	\end{tcolorbox}
	\begin{tcolorbox}[title=Remark,colframe=black,arc=10pt]
The reader that is attentive will have noticed that the serial system is always less reliable than its least reliable component!!
	\end{tcolorbox}
	
This type of calculation is widely used by supply chain manager and the result is named "\NewTerm{availability rate}\index{availability rate}" as well as by project managers for the duration of a phase of a project when they consider durations as independent tasks (for more complex structures, remember that we sometimes talk about "\NewTerm{weighted probability trees}\index{weighted probability trees}" or "\NewTerm{Topological Systems}\index{Topological Systems}").

Thus, in an industrial production line based on the previous example to have a well-defined amount $Q$ of products (supposed to use only one component of each stage) at the end of the production line, it will be necessary at the step $A$ to bring:

	
That is to say $52.42\%$ of components of type $A$ more than expected. At stage $B$:
	

Recall now that the probability density of having $k$ times the event $p$ and $N-k$ times the event $q$ in any arrangement (or order) is given by (\SeeChapter{see section Statistics page \pageref{binomial distribution}}):
	
and is nothing less than the binomial law whose expected mean and standard deviation are (\SeeChapter{see section Statistics page \pageref{binomial distribution}}):
	
Thus, in the Six Sigma methodology, we would apply the binomial distribution (to make things simple because in facts we should use the Hypergeometric law for small samples) to determine what is the probability of having zero defective items and $8$ others  working well on a product of the production line of our example (if all elements have the same probability to fail for sure...):
	
and we obviously fall back on the value obtained with the joint probabilities:
	
Or the probability to have one defective item and seven working others on a product of the production line:
	
we see that the binomial distribution gives us $39.26\%$ probability to have one defective item on $8$ in a product.	

Moreover, in the section of Statistics, we have shown that when the probability $p$ is very small and tends to zero but however the average value $n\cdot p$ tends to a fixed value when $n$ approaches infinity, the binomial law of average $\mu=np$ with $k$ trials was then given by a Poisson law:
	
with (\SeeChapter{see section Statistics page \pageref{poisson distribution}}):
	
	\begin{tcolorbox}[title=Remark,colframe=black,arc=10pt]
In a practical context, it is made use of the maximum likelihood estimators of the exponential law to determine the mean and standard deviation above (\SeeChapter{see section Statistics page \pageref{poisson distribution mle}}).
	\end{tcolorbox}
What in Six Sigma methodology is naturally written:
	
with:
	
Thus, in our example, it is interesting to see the value obtained (which will necessarily be different since we are far from having an infinity of individuals and $p$ is far from being small):
	
with:
	
that is a result even worst than with the binomial law for our products.

However, if $p$ is fixed and small, the average $\mu=np$ also tends to infinity in theory with the Binomial Law, moreover the standard deviation $\sigma=\sqrt{npq}$ also tends to infinity.

If we want to calculate the limit of the binomial distribution, we will therefore need to make a change of origin which stabilizes the average at $0$ for example and a change of unit that stabilizes the standard deviation for example to $1$. This calculation has already been done in the Statistics section, we know that the result is the Normal distribution:
	
For our example we know we have to take $k=0,\sigma=0,\mu=0$:
	
Thus, applying the Normal distribution, we have $24.19\%$ chance to sample the first time a defective product. This difference compared to other methods is simply explained by the assumptions (finite number of individuals, significant probability of defect, etc.).

	\subsubsection{Capability Indices}\label{capability indices}

The Six Sigma methodology (and also the series of norms ISO 22514) defines several indexes that permits to quantify during the manufacturing process the capability of control in the case of a large amount of defects measures distributed as Gaussian-Laplace law (Normal law).

In the case where the measures do not follow a Normal law we have therefore to transform the data using various empirical techniques (Johnson transformations typically).

Basically, if we imagine ourselves working in a company, in charge of the manufacturing quality of a new machine of a new series of pieces, we will be faced with two major situations:
	\begin{enumerate}
		\item At the beginning of production, there may be big quality differences due to defects in the machine or poorly initialization of important settings of the machine. These are defects that will often be rapidly corrected (on the short-term) thanks to a systematic control (destructive or not) of small samples. 
		
		\item Once the big anomalies corrected, we will have in theory only minimal issues that will be very difficult to control and even on the long term. Therefore systematic control is not necessary anymore and once this period of large corrections passed, we make checks per batch (between each correction) and each will be considered an independent and identically distributed random variable (according to a Normal law), but obviously with a different average and standard deviation.
	\end{enumerate}

These two scenarios show that we do not then logically perform the same  tests at the beginning of the production and then on the long term. This is why we define in S.P.C. several indices (whose symbols are unique to this book because they change according the used standards) whose two most important are:
	\begin{enumerate}
		\item[D1.] We name "\NewTerm{short-term process potential capability}\index{short-term process potential capability}" or "\NewTerm{short-term dispersion index}\index{short-term dispersion index}" the ratio of the range of control $E$  of the distribution of values tolerated by the customer and the Six Sigma quality ($6 \sigma$) when the process is almost under control as:
			
		Where USL is the upper control limit or officially the "\NewTerm{Upper Specification Level}\index{upper specification level}" of the distribution and LSL the lower control limit or officially the "\NewTerm{Lower Specification Level}\index{lower specification level}" we often impose (but not always!) in the industry as at equal distances to the desired theoretical mean $\mu$. 
		
		Normally within manufactures, the range of control is fixed (the numerator) and so when the value of the standard deviation is large (more variations, not enough controls) the value of the index is low and when the standard deviation is low (less variation, lot of controls) the value of the index is high as shown by the two examples below:
		
		\begin{figure}[H]
		\centering
		\includegraphics[scale=0.75]{img/engineering/capability.eps}
		\caption{Illustration of $C_p$ index}
		\end{figure}
		
		The above capability is therefore an indication that we will seek to maximize (since the standard deviation in the denominator should be minimized!).
		
		This ratio is useful in the industry in the sense that the range $E$ (which is important because it represents the tolerated dispersion/process variation\label{range six sigma}) is considered as the "voice of the customer" (his request) and the $6$ sigma denominator the actual behavior of the process/process meant to include virtually all possible outcomes. It is better to hope that this report is at worst equal to unity!
		
		Here is a typical example in project management where, when the customer does not pay for a fine tuning risk modeling (the customer therefore accepts by contract a change in time and costs that can exceed 50\%), we come across with this type of chart:
		
		\begin{figure}[H]
		\centering
		\includegraphics[scale=0.75]{img/analysis/six_sigma.eps}
		\caption{Typical plot of a control histogram with tolerances and limits}
		\end{figure}
		
		\begin{tcolorbox}[title=Remark,colframe=black,arc=10pt]
In S.P.C., the range $E$ is sometimes denoted TI, meaning "tolerance interval".
		\end{tcolorbox}
		
		The standard deviation in the denominator is simply defined as the average of the variances in the case of $k$ independent random variables:
			
		where it is useful perhaps to remember that ST is the abbreviation for "Short Term" (abbreviation often unspecified in the practice as assumed known in the context). The standard deviations $\sigma_{\text{ST}}$ is obviously the easiest to apply for the first scenario which we have mentioned above. Because between each big correction, batches are considered independent and can not be analyzed as a single lot (it would be an aberration!).
		
		\begin{tcolorbox}[title=Remark,colframe=black,arc=10pt]
		According to the type of industry some use the standard deviation based on the range for $\sigma_{\text{ST}}$ and that is given as we proved it in the section Statistics during our study of extreme values by the ratio of the average of ranges of all measurements made by different operators and divided by the Hartley's constant such that:
		
		\end{tcolorbox}
		
		
		Be careful! As is often the situation in the short term process (during the correction of the big sources of error) tested batches are small, even very small, and thus to reduce costs. Then the standard deviation below the root is for sure not really a correct value...This is why it is important to calculate a confidence interval! So let us see that it is possible to easily construct a hypothesis test for the $C_p$ indicator under the assumption that each sample is identically distributed following a Normal law. Indeed, remembering that we proved in the section on Statistics that:
		
		It is immediate that:
		
		Thus:
		
		Therefore:	
		
		or:	
		
		and we can apply the same reasoning for all types of indicators of the same kind that we will see later!
		\begin{tcolorbox}[title=Remark,colframe=black,arc=10pt]
As part of our study later of control charts, we will see that it is possible to use special expressions for the standard deviation when working with samples of measures. These expressions will be based for one on the chi-square law and the other on order statistics.
		\end{tcolorbox}
				\item[D2.] We name "\NewTerm{Overall performance of long-term process}\index{overall performance of long-term process}" the ratio of the range of control $E$ of the distribution of values tolerated by the customer and the Six Sigma quality ($6 \sigma$) when the process is centered that is to say under statistical control (product manufacturing parameters vary only a little bit) as:
			
			But where the standard deviation in the denominator is given this time by the fact we consider all the big issues corrected and the process as stable such that we can consider all manufactured pieces in the long term as a single controlled batch:
			
			where it is useful perhaps to remember that LT is the abbreviation for "Long Term" (abbreviation often unspecified in the practice as assumed known in the context). This standard deviation is obviously the simplest for the second scenario which we have mentioned above. Because the variations are now, by hypothesis, very small, the entire production can be assumed as a single lot of control over the long term (this does not avoid that sometimes it is necessary to clean the extreme values that may occur).
	\end{enumerate}
Tolerancing the characteristics is therefore important for obtaining the desired quality and reliability of assembled products. Traditionally, a tolerance is expressed in the form of a bi-point $[\text{Min}, \text{Max}]$. A characteristics is then accepted as conform if it is within tolerance.

The problem of the art of tolerancing is to try to conciliate the fixation of the acceptable variability limits to reduce production costs and ensure the highest level of quality in the finished assembled product.

Two approaches attempt to solve this problem:
	\begin{enumerate}
		\item "\NewTerm{Worst Tolerancing}\index{Worst Tolerancing}" ensures the assembly in all situations from the time the basic characteristics are within tolerance. This is the far the best method when the production is very small (and in the extreme case when there is only one unit...).
		\item "\NewTerm{Statistical Tolerancing}\index{statistical tolerancing}" takes into account the low probability of extreme assemblies and enables to significantly expand the tolerances to reduce costs and so it is to this that we will look further below. For sure this can be apply only for large production sets.
	\end{enumerate}

A process is said to "\NewTerm{limit capable}\index{limit capable process}" if the ratios $C_p,P_p$ given above (choosing six times the standard deviation) is almost equal to $1$. But in the industry, we actually prefer to take  the value $1.33$  in the case of a Normal distribution that is not perfectly centered (we will see later where this last value comes from).

Obviously, the value of the standard deviation  $\sigma$  can be calculated by using the maximum likelihood estimators with and without bias seen in the Statistics and in practice you must never forget that this is just an estimator and not the real theoretical value! Furthermore, we will see later that depending on the standard deviation used, the notations change!

	\begin{tcolorbox}[title=Remark,colframe=black,arc=10pt]
In manufacture, one must be careful because the measuring instrument adds its own standard deviation (error) on these of the production process.
	\end{tcolorbox}

As we have proved it in the section on Statistics, the standard error (standard deviation of the mean) is:
	
	In the Six Sigma methodology, we then often for long term process and under statistical control:
	\begin{equation}
  \addtolength{\fboxsep}{5pt}
   \boxed{
   \begin{gathered}
		\begin{aligned}
			\text{LCL}&=\mu-3\dfrac{\sigma}{\sqrt{n}}\\
			\text{UCL}&=\mu+3\dfrac{\sigma}{\sqrt{n}}
		\end{aligned}
   \end{gathered}
   }
	\end{equation}
when we analyze control charts (see further below) whose random variables is a sample of $n$ independent and identically Normally distributed random variables and that the limits were not imposed by a customer or an internal policy or technical constraints! Obviously, we must be aware that UCL and LCL are not the same expression in more complex cases and therefore for distributions other than the Normal distribution!

Furthermore, the above expression differs for short-term process because the example given above is for a case of measures over the long term for reminder!

As we know $C_p$ requires that the average (target) is centered between USL and LSL. Therefore, the average is confounded with what we name the "\NewTerm{target $T$}\index{process target}" of the process.

But the average $\mu$ in reality can be offset from the original target T which must always (in common use) be equidistant between USL and LSL as shown in the figure below in the particular case a Normal distribution:

		\begin{figure}[H]
		\centering
		\includegraphics[scale=0.75]{img/engineering/target.eps}
		\caption{Measures under statistical control offset from the target (typical with index $C_p$)}
		\end{figure}

But this is not necessarily the case in reality where engineers (whatever their application field) can choose asymmetrical a LSL and USL with respect to the average and this only because the distribution is not always following a Normal law (typically the case in project management ...)! Hence the following definition:

\textbf{Definition (\#\mydef):} We name "\NewTerm{short term non-centered potential Capability  short for the process}\index{short term non-centered potential Capability  short for the process}" (in the biased case) or more frequently "\NewTerm{Process Capability Index (Within)}\index{process capability index (within)}" the relation:
	
	with:
	
	where $k\geq 0$ is named the "\NewTerm{degree of bias}\index{degree of bias}" or "\NewTerm{index position}\index{index position}" and $T$ is as we know the "target" naturally given by:
	
	which gives the middle of the distribution relative to the bi-point $[\text{LSL}, \text{USL}]$ requested by the customer (remember that the standard deviation in the denominator of the prior-previous relation is the standard deviation short term!).
	
	In fact this control capability indicator $C_{pk}$ may seem very artificial, but it is not completely... Indeed, there are some outstanding values (those that interest the engineer) that can help to get a good idea of what happens with it:
	\begin{enumerate}
		\item If the average and the target are confused, then we have:
			
			We find ourselves with $k=0$ and therefore $C_{pk}=C_p$ and judgment criterion of the value of the index will then be based on the short-term capability index.
			\item If because of a poor control of the process we have:
			
			Then the mean $\mu$ is either above USL or below LSL which has the consequence of having $k>1$ and therefore $C_{pk}<0$.	
			\item If we have:
			
			Then the average $\mu$ is between the values USL and LSL which has the consequence of having $0<k<1$ and therefore $0<C_{pk}<C_p$.	
			 \item If we have:
			 
			Then it simply means that the average is aligned with USL or LSL and we therefore have $k=1$ and $C_{pk}=0$.
			As the interpretation remains sometimes delicate and difficult, we build unilateral capability indices "Upper Capability Index CPU" and "Lower Capability Index CPL" given by:
			
			That we will obviously also seek to maximize. Let's see where these two definitions comes from and how to use them:
			\begin{dem}
			First, we need two specific formulations of the degree of bias $k$.
			\begin{enumerate}
				\item If:
				
				then we can get rid of the absolute value:
				
				\item If:
				
				then we can get rid of the absolute value:
				
			\end{enumerate}
	We have therefore when $T>\mu$:
		
		and when when $T<\mu$:
		
		\begin{flushright}
			$\square$  Q.E.D.
		\end{flushright}
		\end{dem}
	\end{enumerate}
		In the long term, in some companies (firms), it is interesting to know what are the worst values taken by the CPU indices and CPL (this is the case in the field of production but not necessarily in project management).
		
		The worst values being by construction the smallest one, we often write (with a few different notations that can be found in the literature ...):
		

Below an example of an analysis diagram of the capability generated by the software Minitab 15.1.1 Software with the various aforementioned factors on a sample of 68 measures that can not be rejected as not following a Normal distribution (a normality test was done before):

\begin{figure}[H]
\centering
\includegraphics[scale=0.75]{img/engineering/minitab_cp_cpk_cpu_cpl.jpg}
\caption{Serial/Process Workflow}
\end{figure}

Two typical readings of this chart are possible (we will explain the lower left part of the chart later):
	\begin{enumerate}
		\item In manufacturing: The process is capabable (value $>1.33$) but with a (too) strong deviation to the left in relation to the defined target, which is not good (CPL having the smallest value) and must be corrected.
		\item In Project Management: Redundant tasks are under control (value$>1.33$) but with a strong deviation to the left, which can be good if our goal is to get ahead compared to planning (nothing to correct).
	\end{enumerate}
	You really have to take care of the fact that in reality it is not always possible to take the Normal distribution (we recall this because all the examples given above and below are based on this simplifying assumption!).
	
	Always in the context of quality management in manufacturing, the figure below is represents well the reality in the context of a short or long-term process\label{short and long term process}:
	\begin{figure}[H]
		\centering
		\includegraphics[scale=0.75]{img/engineering/variation_process_control.jpg}
		\caption[Short/Long term process]{Short/Long term process (source: Maurice Pillet ISBN13: 978-2708133495)}
	\end{figure}
	Every little Gaussian in light gray, represents a batch analysis assimilated to the concept of "\NewTerm{instantaneous dispersion}\index{instantaneous dispersion}". We see that their average don't keep moving during the measurement period (that this variation is large or very small) and that's what we name the "\NewTerm{global dispersion}\index{global dispersion}". The goal in organizations (industries or administrations) is to ensure that the instantaneous or global variability is limited as much as possible.
	
	But the relation defining $C_p$ assumed, as we mentioned, that the process is under control and also centered (so all Gaussian are aligned) and on a short term.
	
	Similarly, the relation defining $C_{pk}$ assumed, as we mentioned it before, the that the process is under control, on a short-term perspective and non-centered by choice (or because of the fact that the law is not Normal).
	
	If the process is not centered because it is not under control when it should be, the random variable measured could be the sum of the small random behavior of the machine $X$ and uncontrollable small random variations of constraint measurement of the pieces $Y$.
	
	The total standard deviation is then, if the two random variables follow a Normal distribution (and that they are independent!), the square root of the sum of deviations (\SeeChapter{see section Statistics page \pageref{sum of two random normal variables}}):
	
	Now if we only have one measure, we get taking the unbiased estimator (it's a little stupid to use it in this case but...):
	
	But in the case of study that interests $Y$ represents the experimental mean (measured) of the process we are trying to bring under control (this is also why we can put $n=1$). This average is traditionally denoted $m$ in the field of industrial engineering.
	
	Then $m_Y$ being not known we take what it should be: that is the target $T$ in the process. Thus, we introduce a new index named "\NewTerm{short term non-centered biased process potential capability}\index{short term non-centered biased process potential capability}":
	
	where once again it must be remembered that the standard deviation in the denominator inside the root of the standard deviation short term of the machine!
	We immediately see that most $C_{pm}$ is close to $C_{p}$ better it is (at least in manufacturing field).
	
	We finally have three common centered and non-centered short term capabilities indices (we deliberately chose to standardize the notations and put as much information inside the relations below):
	
	Similarly we also have three common long-term capabilities indicators centered or non-centered (we deliberately chose to standardize the notations and put as much information inside the relations below):
	
	Finally, it is good to know that although if this is not very relevant, sometimes some engineers do both analyzes (short term + long-term) at the same time on the same measurements.
	
	\begin{tcolorbox}[title=Remark,colframe=black,arc=10pt]
	Let us indicate the capabilities of an industrial process methods when applied to machines are denoted respectively by $C_m,C_{mk},C_{mm}$. Anyway ... for more information (but without proofs) refer to the norm ISO 22514-2:2013 or the old version ISO 21747:2006.
	\end{tcolorbox}
	
	However, to make an objective analysis on the indexes of capability seen so far, it would first have to check measuring instruments are themselves capable... what we often name the "\NewTerm{R\&R methods}\index{R\&R methods}" (Repeatability, Reproducibility).
	
	The basic principle (a little bit more advanced principle is to make use of the two-factor ANOVA with repetition as studied in details in the section of Statistics) is then to evaluate the short term or respectively  long term dispersion of the measuring instrument to calculate a "\NewTerm{process control capability}\index{process control capability}" defined by:
	
	or according to the type of industry some use the standard deviation based on the range for $\sigma_{\text{instrument}}$ and as we proved it in the section of Statistics during our study of extreme values and therefore named "\NewTerm{Equipment Variation EV}\index{equipment variation}" consisting in the calculation of the average of ranges of all measurements made by different operators and divide by the Hartley's constant:
	
	In the classical cases, we declare the control method as capable for an SPC tracking when its capability is greater than $4$ and we will immediately see why. 
	
	Recall for this first that:
	
	But the variance observed is in fact the sum of the "true" variance and this of the instrument such that:
	
	But we have:
	
	Putting it all squared, we deduce:
	
	And now under the strong assumptions that all $E$ are equals we have:
	
	Therefore:
	
	That give us with some rearrangements:
	
	Finally:
	
	This can be translated in graph in the figure below which shows the interest of a $C_{pc}$ of at least equal to 4!
	
\begin{figure}[H]
\centering
\includegraphics[scale=0.8]{img/engineering/observed_cp.jpg}
\caption{Relations between $C_{p,\text{observed}}$ and $C_{p,\text{true}}$ for fixed $C_{pc}$}
\end{figure}

	In practice, note that to determine $C_{pc}$ we use a standard prototype measured by laser interferometry and then we ensure that all repeated measurement take place on the same two measuring points.
	
	Once this done, we performs several measures of the standard prototype and we take the standard deviation of these measures. This will give the $\sigma_{\text{instrument}}$.
	
	The range $E$ is imposed by the customer or by internal engineers to the company. It will often be taken as maximum to the tenth of the tolerance unit of the standard prototype.
	
	For example, if we have a piece with an internal diameter of $36\pm 1\;[\mu \text{m}]$ (tolerance range of $2\; [\mu\text{m}]$ which is already a good level precision because in our time the standard is closer to $3\;[\mu\text{m}]$!), our device will then need to have according the rule cited above range of $0.2\;[\mu\text{m}]$...
	
	\pagebreak
	\subsubsection{Quality Levels}
	
	Some engineers as we already mention it like to know how many elements by million units produced (parts per million: PPM) will be considered as defective.
	
	The calculation is then easy since the engineer has at his disposal at least the following information:
	
	and under the strong assumption that the data follow a Normal distribution and that the target is centered on the mean it is immediate (\SeeChapter{see section Statistics page \pageref{gauss distribution}}) in this simple case the for the defective below the LSL we get (for long term or short term):
	
	and:
	
	values very easy to get with any spreadsheet like Microsoft Excel for example.
	We for sure:
	
	If $\mu$ and $T$ are still aligned and that USL and LSL are symetric to this two values we have the special case:
	
Note now an important point relatively to Six Sigma! In fact, objectively, the idea of this method is, of course, to make SPC (among others, but that's nothing really new) but especially to: guarantee to the customer according to tradition with a commonly accepted standard deviation having an upper limit of $\sigma=1$ with an absolute deviation to the mean (in absolute value) of $1.5\sigma$ relative to the target, which guarantees at maximum the famous 3.4 PPM (that is to say 3.4 defectives per million).	

	\begin{tcolorbox}[title=Remark,colframe=black,arc=10pt]
	This choice seems to come from the empirical practice of Six Sigma by one of its main creator (Bill Smith). It is said that he observed in his business (Motorola) that under statistical control, it was almost always a deviation ranging between $1.2$ and $1.8\sigma$ of the average for almost all its industrial processes.
	\end{tcolorbox}
	Let's see where this value comes from with the following two tables:
	\begin{enumerate}
		\item First we construct an ideal type of table presenting data from a short-term process (but the calculations are identical for a long-term process) centered on the target (target will be zero here, which is a typical case), with a zero mean (i.e. on target and therefore $C_p=C_{pk}$) and unit standard deviation with USL and LSL symmetrical (which against restricted by the scope of application):
	\begin{table}[H]
	\begin{center}
		\definecolor{gris}{gray}{0.85}
			\begin{tabular}{|c|c|c|c|c|}
				\hline
				\multicolumn{1}{c}{\cellcolor{black!30}\textbf{$C_p$}} & 
\multicolumn{1}{c}{\cellcolor{black!30}\textbf{$C_{pk}$}} & \multicolumn{1}{c}{\cellcolor{black!30}\textbf{Defectives (PPM)}} & \multicolumn{1}{c}{\cellcolor{black!30}\textbf{Sigma Quality Level}}  & \multicolumn{1}{c}{\cellcolor{black!30}\textbf{Criteria}}\\ \hline
		0.5 & 0.5 & 133,614 & 1.5 & Bad\\ \hline
		0.6 & 0.6 & 71,861 & 1.8 & {}\\ \hline
		0.7 & 0.7 & 35,729 & 2.1 & {}\\ \hline
		0.8 & 0.8 & 16,395 & 2.4 & {}\\ \hline
		0.9 & 0.9 & 6,934 & 2.7 & {}\\ \hline
		1 & 1 & 2,700 & 3 &{}\\ \hline
		1.1 & 1.1 & 967 & 3.3 & {}\\ \hline
		1.2 & 1.2 & 318 & 3.6 & {}\\ \hline
		1.3 & 1.3 & 96 & 3.9 & Limit\\ \hline
		1.4 & 1.4 & 27 & 4.2 & {}\\ \hline
		1.5 & 1.5 & 6.8 & 4.5 & {}\\ \hline
		1.6 & 1.6 & 1.6 & 4.8 & {}\\ \hline
		1.7 & 1.7 & 0.34 & 5.1 & {}\\ \hline
		1.8 & 1.8 & 0.067 & 5.4 & {}\\ \hline
		1.9 & 1.9 & 0.012 & 5.7 & {}\\ \hline
		2 & 2 & 0.002 & 6 & Excellent \\ \hline
	\end{tabular}
	\end{center}
	\caption{Defectives and Sigma quality level of a centered process}
	\end{table}
		where all the values are obtained using the following relations from the potential capability index only. First (remember the assumption that $T=\mu$ and $\sigma=1$:
	
	if the standard deviation is reduced (that can always be done and does not change the accuracy of the results!). And since in the above table LSL and USL are symmetrical with respect to the target:
	
	and PPM are conforming with what we saw just before given by:
	
	and therefore as in the example above LSL and USL are symmetric with respect to the target it simplifies to:
	
where, for example, the value of PPM given at the line "Limit" is obtained with Maple 4.00b using the following command:

	\texttt{>evalf((1-1/sqrt(2*Pi)*int(exp(-x\string^ 2/2),x=-infinity..3.9))*2)*1E6;}
	
	or with Microsoft Excel 11.8346:
\begin{center}
	\texttt{=(1-NORMDIST(3.9,0,1,1))*1E6}
\end{center}
		Let us recall that the "sigma quality level" denoted by $\sigma_q$ is given in fact with the following table we had built in the section Statistics under the assumption of a Normal law (thanks to John Cannin for the \LaTeX{} figure):
		\begin{center}
		\pgfplotsset{compat=1.7}
		\pgfmathdeclarefunction{gauss}{2}{\pgfmathparse{1/(#2*sqrt(2*pi))*exp(-((x-#1)^2)/(2*#2^2))}%
		}
		\begin{tikzpicture}
		\begin{axis}[no markers, domain=0:10, samples=100,
		axis lines*=left, xlabel=Standard deviations, ylabel=Frequency,,
		height=6cm, width=10cm,
		xtick={-3, -2, -1, 0, 1, 2, 3}, ytick=\empty,
		enlargelimits=false, clip=false, axis on top,
		grid = major]
		\addplot [fill=cyan!20, draw=none, domain=-3:3] {gauss(0,1)} \closedcycle;
		\addplot [fill=orange!20, draw=none, domain=-3:-2] {gauss(0,1)} \closedcycle;
		\addplot [fill=orange!20, draw=none, domain=2:3] {gauss(0,1)} \closedcycle;
		\addplot [fill=blue!20, draw=none, domain=-2:-1] {gauss(0,1)} \closedcycle;
		\addplot [fill=blue!20, draw=none, domain=1:2] {gauss(0,1)} \closedcycle;
		\addplot[] coordinates {(-1,0.4) (1,0.4)};
		\addplot[] coordinates {(-2,0.3) (2,0.3)};
		\addplot[] coordinates {(-3,0.2) (3,0.2)};
		\node[coordinate, pin={68.2\%}] at (axis cs: 0, 0.4){};
		\node[coordinate, pin={95\%}] at (axis cs: 0, 0.3){};
		\node[coordinate, pin={99.7\%}] at (axis cs: 0, 0.2){};
		\node[coordinate, pin={34.1\%}] at (axis cs: -0.5, 0){};
		\node[coordinate, pin={34.1\%}] at (axis cs: 0.5, 0){};
		\node[coordinate, pin={13.6\%}] at (axis cs: 1.5, 0){};
		\node[coordinate, pin={13.6\%}] at (axis cs: -1.5, 0){};
		\node[coordinate, pin={2.1\%}] at (axis cs: 2.5, 0){};
		\node[coordinate, pin={2.1\%}] at (axis cs: -2.5, 0){};
		\end{axis}
		\end{tikzpicture}
		\end{center}
	\begin{table}[H]
	\begin{center}
		\definecolor{gris}{gray}{0.85}
			\begin{tabular}{|c|c|c|}
				\hline
				\multicolumn{1}{c}{\cellcolor{black!30}\textbf{Sigma Quality Level}} & 
\multicolumn{1}{c}{\cellcolor{black!30}\textbf{Non-defective assured rate in \%}} & \multicolumn{1}{c}{\cellcolor{black!30}\textbf{Defective  in PPM}}  \\ \hline
		$1\sigma_q$ & 68.26894 & 317,311\\ \hline
		$2\sigma_q$ & 95.4499 & 45,500 \\ \hline
		$3\sigma_q$ & 99.73002 & 2,700 \\ \hline
		$4\sigma_q$ & 99.99366 & 63.4 \\ \hline
		$5\sigma_q$ & 99.999943 & 0.57 \\ \hline
		$6\sigma_q$ & 99.9999998 & 0.002 \\ \hline
	\end{tabular}
	\end{center}
	\caption{Capability, non-defective rate in\% and PPM}
	\end{table}
	and for which we had given the Maple 4.00b command to obtain the values that are valid for all standard deviations and all means!

	\item Now let us build the table at worst as considered by Six Sigma, that is to say a table for non-centered process (that is to say where $C_p=C_{pk}$ is not satisfied) with a deviation of the average of $+1.5\sigma$ (so on the right but could be taken as shifted on left and the results would be exactly the same!) against the target and unit standard deviation with symmetrical USL and LSL (which still restricts the scope of application):
	\begin{table}[H]
	\begin{center}
		\definecolor{gris}{gray}{0.85}
			\begin{tabular}{|c|c|c|c|c|}
				\hline
				\multicolumn{1}{c}{\cellcolor{black!30}\textbf{$C_p$}} & 
\multicolumn{1}{c}{\cellcolor{black!30}\textbf{$C_{pk}$}} & \multicolumn{1}{c}{\cellcolor{black!30}\textbf{Defectives (PPM)}} & \multicolumn{1}{c}{\cellcolor{black!30}\textbf{Sigma Quality Level}}  & \multicolumn{1}{c}{\cellcolor{black!30}\textbf{Criteria}}\\ \hline
		0.5 & 0 & 501,350 & 1.5 & Bad\\ \hline
		0.6 & 0.1 & 382,572 & 1.8 & {}\\ \hline
		0.7 & 0.2 & 274,122 & 2.1 & {}\\ \hline
		0.8 & 0.3 & 184,108 & 2.4 & {}\\ \hline
		0.9 & 0.4 & 115,083 & 2.7 & {}\\ \hline
		1 & 0.5 &  66,810 & 3 & {}\\ \hline
		1.1 & 0.6 & 35,931 & 3.3 & {}\\ \hline
		1.2 & 0.7 & 17,865 & 3.6 & {}\\ \hline
		1.3 & 0.8 & 8,198 & 3.9 & Limit\\ \hline
		1.4 & 0.9 & 3,467 & 4.2 & {}\\ \hline
		1.5 & 1 & 1,350 & 4.5 & {}\\ \hline
		1.6 & 1.1 & 483 & 4.8 & {}\\ \hline
		1.7 & 1.2 & 159 & 5.1 & {}\\ \hline
		1.8 & 1.3 & 48 & 5.4 & {}\\ \hline
		1.9 & 1.4 & 13 & 5.7 & {}\\ \hline
		2 & 1.5 & 3.4 & 6 & Excellent \\ \hline
	\end{tabular}
	\end{center}
	\caption{Defectives and Sigma quality level of a non-centered process}
	\end{table}
	where all the values are obtained using the following relations from the potential capability index only. First (remember the assumption that $T=\mu$ and $\sigma=1$:
	
	if the standard deviation is reduced (that can always be done and does not change the accuracy of the results!). And since in the above table LSL and USL are symmetrical with respect to the target:
	
	Therefore:
	
	and PPM are conforming with what we saw just before given by:
	
where, for example, the value of PPM given at the line "Limit" is obtained with Maple 4.00b using the following command:

	\texttt{>evalf((1-1/sqrt(2*Pi)*int(exp(-(x-1.5)\string^ 2/2),x=-infinity..(1.3*3))))*1E6}\\
	\texttt{+evalf((1/sqrt(2*Pi)*int(exp(-(x-1.5)\string^ 2/2),x=-infinity..-(3*(1.3+1)))))*1E6;}\\
	
	or with Microsoft Excel 11.8346:
\begin{center}
	\texttt{=(((1-NORMDIST(3*1.3,1.5,1,1))+NORMDIST(-3*(1.3+1),1.5,1,1)))*1E6}
\end{center}
	\end{enumerate}	
	
	We finally understand seeing this famous "limit" row, why a process under control is said to be "\NewTerm{limit capable}\index{limit capable}" with a potential capability index of 1.33 potential given the number of PPM!
	
	So the goal in practice is obviously to be in the situation of the first table with the corresponding values and with in the first table at a quality level of $\sim4.7\sigma$ for the equivalent of 3.4 PPM of the second table (because it is easier to center a process thant to control its variations).
	
	\begin{tcolorbox}[colframe=black,colback=white,sharp corners]
	\textbf{{\Large \ding{45}}Example:}\\\\
	Let us now make a summary of all this considering a new small production of $n=50$ pieces in packs of $k=n_i=10$ (in order to adjust the machines during production). Measuring the specifications of 5 pieces per hour during 10 hours with a tolerance of $10\pm0.07$ that is to say in terms of a range of hundredths:
	
	and a target of $T=0$. We have the data:
	\begin{table}[H]
	\begin{center}
	\begin{tabular}{|c|c|c|c|c|c|c|c|c|c|c|}
	\hline 
	{} & {\cellcolor{black!30}1} & {\cellcolor{black!30}2} & {\cellcolor{black!30}3} & {\cellcolor{black!30}4} & {\cellcolor{black!30}5} & {\cellcolor{black!30}6} & {\cellcolor{black!30}7} & {\cellcolor{black!30}8} & {\cellcolor{black!30}9} & {\cellcolor{black!30}10} \\ 
	\hline 
	{\cellcolor{black!30}1} & -2 & -4 & -1 & 0 & 4 & 0 & 3 & 0 & 1 & -1  \\ 
	\hline 
	{\cellcolor{black!30}2} & 0 & -3 & 0 & -2 & 1 & -2 & 0 & 1 & -1 & 2  \\ 
	\hline 
	{\cellcolor{black!30}3} & -1 & 0 & -3 & -1 & 0 & 0 & -1 & -1 & 3 & 1  \\ 
	\hline 
	{\cellcolor{black!30}4} & 1 & 1 & -2 & 2 & 2 & 0 & 1 & 0 & 4 & 0  \\ 
	\hline 
	{\cellcolor{black!30}5} & -1 & -1 & -3 & 0 & 0 & 3 & 3 & 2 & 1 & 0  \\ 
	\hline 
	{\cellcolor{black!30}$\mu$} & -0.6 & -1.4 & -1.8 & -0.2 & 1.4 & 0.2 & 1.2 & 0.4 & 1.6 & 0.5  \\ 
	\hline 
	{\cellcolor{black!30}$\sigma$} & 1.14 & 2.07 & 1.30 & 1.48 & 1.67 & 1.79 & 1.79 & 1.14 & 1.95 & 1.14  \\ 
	\hline 
	\end{tabular} 
	\end{center}
	\caption{Application of statistical process control analysis}
	\end{table}
	We immediately see that the manufacturing process was not stationary during this first production, we will therefore need to make corrections in the future:
	\begin{figure}[H]
		\centering
		\includegraphics{img/engineering/application_non_stationnary_process.jpg}
		\caption[]{Visual evidence by a plot that the process seems not stationary}
	\end{figure}
	or as a very simple control average-standard deviation chart with $3\sigma$ (as I like do it for beginners):
	\end{tcolorbox}
	
	\pagebreak
	\begin{tcolorbox}[colframe=black,colback=white,sharp corners]
	\begin{figure}[H]
		\centering
		\includegraphics{img/engineering/application_non_stationnary_process_control_chart.jpg}
		\caption[]{Visual evidence by a plot that the process seems not stationary}
	\end{figure}
	So we easily guess that the process is limit capable...
	\begin{tcolorbox}[title=Remark,colframe=black,arc=10pt]
	An interesting thing to know is that this type of chart can be analyzed using the mathematical tools of the time series analysis as we will see later.
	\end{tcolorbox}
	First, if we want to make a relevant statistical study of the different data above we can calculate the average of deviations that under the assumption of a Normal distribution is the arithmetic mean (\SeeChapter{see section Statistics page \pageref{normal distribution mle}}):
	
	Then the standard deviation of all pieces taken as a single group is using the maximum likelihood estimator of the variance of the Normal distribution (long term standard deviation) and taking the some notation for this variance as the one used in the section of Statistics:
	
	So the standard error  (the estimator of the standard error of the mean)  is as proved in the section Statistics (page \pageref{standard error}):
	
	So the confidence interval to $95\%$ of the average is (\SeeChapter{see section Statistics page \pageref{one sample z test}}):
	
	\end{tcolorbox}
	
	\pagebreak
	\begin{tcolorbox}[colframe=black,colback=white,sharp corners]
	In our case this gives:
	
	And the statistical inference with our long-term standard deviation using the chi-square $\chi^2$ bilateral hypothesis test gives (\SeeChapter{see section Statistics page \pageref{ci on the variance with empirical mean}}):
	
	What gives us in our case:
	
	Therefore:
	
	We notice therefore that on a long-term analysis we have the intervals:
	
	The variations can therefore be huge with a cumulative probability of 95\% and we must then take care in practical cases to make adjustments as quickly as possible to reduce as possible the values of those moments!\\
	
	Let us now calculate the potential long-term process capability (if assumed centered!). We have:
	
	But with an instrument having $P_{pc}$ of $4$, it really corresponds to:
	
	Furthermore, let us indicate that if we can do a calculation of the confidence interval for $\sigma_{\text{LT}}$ (see calculations done previously), it is then easy to have one too for $P_p$!\\
	If the analysis of the overall performance of the long-term process is not centered (which is the case here)  we use:
		
	and we know once again that because of the instrument, this value is somewhat undervalued!
	\end{tcolorbox}
	
	\pagebreak
	\begin{tcolorbox}[colframe=black,colback=white,sharp corners]
	We have of course:
	
	So the process is not centered (as we suspected...). Then, we must calculate the potential long-term average non-centered capability $P_{pm}$ of the process using the relations defined above:
	
	To resume, either the value of $P_{pk}$, $P_p$ or $P_{pm}$, we see that all the values are limit capable (that is to say that the value is greater than $1$ - see definition above for a reminder of what means "limit capable").\\
	
	If we do then our calculations of PPM according to the relations obtained above with the value of $S_{\text{LT}}$ and $\bar{X}$ previously obtained then we have:
	
	Then say this number  is good or bad is difficult because we lack the information to know what is the cost of production, cost of repair, the cost of a product and the whole is itself dependent of the total amount produced! But we can also use the model of Taguchi (see further below) to know the value of the parameters (moments) calculated that it would be preferable not to exceed!\\
	
	Let us now calculate the short term capability indices! For this, we need the estimator of the average of the whole considering each individual as a random variable. We know (\SeeChapter{see section Statistics page \pageref{normal distribution mle}}) that this average is the arithmetic mean in the case of a Normal distribution and is strictly equal to that which is calculated by considering the set of individuals as a single random variable. So it follows that:
	
	Regarding the standard deviation this is not the same! But we know (\SeeChapter{see section Statistics page \pageref{stability of the sum in statistics}}) that the Normal law that is stable by the sum. For example, we have proved that given two independent random variables distributed according to a Normal distribution (by imagining that each variable represents two of our ten samples), we hat for their sum:
	
	But we have also proved in the section of Statistics using the property of linearity of expected mean, that we have:
		
	\end{tcolorbox}

	\pagebreak
	\begin{tcolorbox}[colframe=black,colback=white,sharp corners]
	and for the variance:
		
	Therefore:
	
	and in our particular case:
	
	\begin{tcolorbox}[title=Remark,colframe=black,arc=10pt]
	A software like Minitab used the pooled standard deviation (\SeeChapter{see section Statistics page \pageref{pooled standard deviation}}) that for recall is given by:
	
	But it's still not accurate enough. As we will prove it further below during our study of control charts, by default (but there is an option to deactivate this) Minitab use unbiasing constants such that finally:
	
	\end{tcolorbox}
	So the standard error (the estimator of the standard error of the mean) is:
	
	So the confidence interval to $95\%$ of the average is (\SeeChapter{see section Statistics page \pageref{ci on the mean with know variance}}):
	
	In our case this gives:
	
	We note that in the short term, the range is much wider in the long term, which is normal given the low value of $k$ (which is only $5$ in our example).
	\end{tcolorbox}
	
	\pagebreak
	\begin{tcolorbox}[colframe=black,colback=white,sharp corners]
	And the statistical inference with our short-term standard deviation using the chi-square $\chi^2$ bilateral hypothesis test gives (\SeeChapter{see section Statistics page \pageref{ci on the variance with empirical mean}}):
	
	What gives us in our case:
	
	Therefore:
	
	We notice therefore that on a short-term analysis we have the intervals:
	 
	The variations can therefore be huge with a cumulative probability of 95\% and we must then take care in practical cases to make adjustments as quickly as possible to reduce as possible the values of those moments!\\
		
	Let us now calculate the potential long-term process capability (if assumed centered!).
	We have:
	
	So we have logically (the opposite would be problematic):
	
	But with an instrument having $C_{pc}$ of $4$, it really corresponds to:
	
	Furthermore, let us indicate that if we can do a calculation of the confidence interval for $\sigma_{\text{ST}}$ (see calculations done previously), it is then easy to have one for $C_p$ too!\\
	
	If the analysis of the overall performance of the long-term process is not centered (which is the case here)  we use:
		
	and we know once again that because of the instrument, this value is somewhat undervalued! We have of course:
	\end{tcolorbox}
	
	\pagebreak
	\begin{tcolorbox}[colframe=black,colback=white,sharp corners]
	
	So the process is not centered (as we suspected...). Then, we must calculate the potential long-term average non-centered capability $C_{pm}$ of the process using the relations defined above:
	
	To resume, either the value of $C_{pk}$, $C_p$ or $C_{pm}$, we see that all the values are limit capable (that is to say that the value is greater than $1$ - see definition above for a reminder of what means "limit capable").\\
	
	If we do then our calculations of PPM according to the relations obtained above with the value of $S_{\text{ST}}$ and $\bar{X}$ previously obtained then we have:
	
	With same remarks as for $\text{PPM}_{\text{tot}}^{\text{LT}}$.\\
	
	To close this long detailed example, here is the output of a program like Minitab 17.3.1 in which we find all the calculations above plus the control charts which we will further detailed the proofs (see the section of Statistics for the details about AD Anderson-Darling test):
	\begin{figure}[H]
		\centering
		\includegraphics[scale=0.7]{img/engineering/six_pack_analysis.jpg}
		\caption[]{Minitab 17.3.1 Six Pack Capability Analysis}
	\end{figure}
	\end{tcolorbox}
	
	\pagebreak
	\begin{tcolorbox}[colframe=black,colback=white,sharp corners]
	and with a screenshot highlighting the fact that they are multiple ways to calculate $\sigma_{\text{ST}}$ with or without unbiasing constants:
	\begin{figure}[H]
		\centering
		\includegraphics{img/engineering/capability_short_term_standard_deviation_options.jpg}
	\end{figure}
	\end{tcolorbox}
	
	\pagebreak
	\subsection{Taguchi Model}
	As part of the SPC (Statistical Process Control), it is interesting for a manufacturer to estimate the financial losses generated by the differences in the target (caution! can also apply this approach in other areas that the industry!).
	
	We can have a relatively simple and satisfactory estimate of its (financial) losses under the following assumptions/hypothesis:
	\begin{enumerate}
		\item[H1.] The process is under control (constant standard deviation) and follows a symmetric decreasing left and right mass function relative to the target (which can be a measurement, a number of errors per periods, etc.).
		\item[H2.] The cost of error is zero when production (or work) is centered on the target (minimum).
		
		 \item[H3.] The cost increases in the same way as production off center on the left and on the right (which is not the case in the field of administration for example). The cost function therefore pass, following assumptions/hypothesis H2 and H3, through a minimum on the target.
	\end{enumerate}
	Therefore, if we note  $Y$ the off-centering with respect to the target $T$ and $L$ financial loss. We have:
	
	Although we do not know the form of this function, we can write it as a Taylor expansion (\SeeChapter{see section Sequence and Series page \pageref{taylor series}}) around $T$ as:
	
	If we develop to the third order:
		
	But by the assumption/hypothesis H2, we have $L (T)$ is zero. It then remains:
	
	and as by H3, the derivative of the function $L (Y)$ is zero in $T$ as it is a minimum then:
	
	That is traditionally noted in SPC:
	
	and is named "\NewTerm{Taguchi (centered) loss function}\index{Taguchi (centered) loss function}" or just "\NewTerm{quality loss function (centered)}\index{quality loss function (centered)}".
	
	Well it's nice to have this relation, but how should we use it?
	
	In fact, it's relatively simple. Under the assumptions mentioned above, if we have production defective measurement (ribs, delays, breakdowns, bug, etc.) then just calculate their arithmetical mean $\mu$ (estimator of the expected mean of a Normal distribution) and then to know the financial or hourly cost $L$ that the defect generates for the company, institution or manufacture (sometimes this average is calculated on the basis of a single sample...).
	
	Therefore, the above relation becomes:
	
	with known $L$ and $\mu$.
	
	And since $T$ is given by the customer or depending on the context it is therefore easy to get the $k$ factor:
	
	which is the fact mathematically the inflection point of the mathematical function $L$ (\SeeChapter{see section Differential and Integral Calculus page \pageref{inflection point}}).
	
	The latter relationship is sometimes written:
	
	
	Once we have $k$ with a good estimation, it is possible to know $L$ for any value of $Y$ and so we can calculate in production the cost of any deviation from the target.
	
	\begin{tcolorbox}[colframe=black,colback=white,sharp corners]
	\textbf{{\Large \ding{45}}Example:}\\\\
	Consider a power supply for a stereo for which voltage $T$ is $110 [V]$. If the voltage is out of the range $110\pm 20 [V]$ then the stereo fails and must be repaired. Suppose the repair cost (including all direct and indirect costs!) is of $100.-$. Then, the associated cost for a given value of the voltage is:
	
	\end{tcolorbox}
	
	Let us now see an elegant way to calculate the average Taguchi cost (average unit loss). We obviously have for a production line on several pieces of the same family:
	
	where the $X_i$ are supposed to be normal random variables (Gaussian by  assumption because the manufacturing process is under statistical control). But we have proved in the section of Statistics in our study of the confidence interval for the variance with known empirical mean that:
	
	Therefore:
	
	And finally:
	
	This last expression has the advantage of showing very clearly that to minimize loss, we must act on the dispersion and the adjustment of the average on the nominal value.
	
	Now remember that we have proved in the section Statistics during our study of likelihood estimator that (it is important in these developments that we use the notation which distinguish different estimators!):
	
	Therefore
	
	If $n$ is large, then we have for a lot of products:
	
	where the first term in brackets represents the variance of $Y$ around its own mean and the second term of the deviation of $Y$ with respect to the target $T$.
	
	We also find this last expression in the literature and in softwares often denoted as follows:
	
	where the index $N$ means "Nominal". Obviously, when the target $T$ (or nominal value) is taken as zero, the relations simplify even more.
	
	Now, remember the following properties of the expected mean and variance (\SeeChapter{see section Statistics page \pageref{properties of the mean}}):
	
	Therefore, if the process is non-centered and that we must correct the random variable to recenter it, it comes intuitively that the correction factor will then be (if linear):
	
	Therefore, it comes immediately:
	
	So to minimize $L$, knowing that the other terms are imposed, it is necessary to minimize the ratio:
	
	or, equivalently, maximize the inverse ratio (which remains without dimensions):
	
	This number is often very large, it is customary in the literature and in statistical software to take the base-10 logarithm and multiply the result by 10 (it is an idea inspired by the acoustic physics). We then have what we name the "\NewTerm{signal/noise ratio}\index{signal/noise ratio}", given in decibels:
	
	for what engineers call "\NewTerm{nominal is the best}\index{nominal is the best}".
	
	Relation that will meet in a software like Minitab during the analysis of Taguchi design of experiment (see below what are experimental designs). Some very rare software also offer the following special case:
	
	For what follow let us return to the following relation:
	
	Engineers talk about minimizing the quality loss function when $T$ (the target) must be zero and therefore it is necessary to minimize all $y_i$. So the above relation is finally reduced to:
	
	What engineers sometimes like to summarize as follows (whereas I personally think that taking the logarithm in base $10$ as well as putting a "-" in this case is really not justified...):
	
	and even sometimes they say it is still a signal/noise ratio... while obviously there is no noise in this specific case that is named "\NewTerm{smaller is better SIB}\index{smaller is better }". This is exactly in this form in which you can find this target in the software Minitab.
	
	In the context of a parameter to maximize it is customary to say that we seeks to minimize:
	
	because given that $T$ tends to infinity it would be difficult to do anything mathematically speaking.
	
	So because $T$ tends to infinity, the preceding relation is reduced to:
	
	What engineers also likes sometimes summarized as follows (here the base-10 logarithm is justified in my point of view!):
	
	and even sometimes they say also it is still a signal/noise ratio... while obviously there is no noise in this specific case that is named "\NewTerm{larger is better}\index{larger is better}". This is exactly  under this form in which you can find this target written in a software like Minitab.
	
	It is, however, being clever, possible to introduce noise in the relation:
	
	For this let us write:
	
	We put:
	
	So in this case, using the Taylor series as proved in the section of Sequences And Series:
	
	Then we have:
	
	and under statistical control we have:
	
	and assuming that we work on all parts of the population, we have:
	
	It comes then:
	
	So in the end by replacing a little bit abusively by the estimators, we get:
	
	
	\pagebreak
	\subsection{Preventive Maintenance}\label{preventive maintenance}
	The evolution of production techniques towards greater automation of complex technical systems has increased the importance of the reliability of production machines or servers. Also, an unexpected shutdown is costly or even deadly to a business. Similarly, in the aerospace or nuclear industries, problems of reliability, maintainability, availability are crucial. Preventive maintenance has the purpose to quantify the reliability for all the components (mechanical, electromechanical, IT, etc.).
	
	The existence of a maintenance service has for purpose to maintain the equipment (systems) and also the reduction in breakdowns. Indeed, these are expensive, they generate:
	\begin{enumerate}
		\item Intervention, repair costs		
		\item Cost of product non-quality		
		\item Indirect costs such as fixed costs, production losses, lost profit...
		\item Life-loss in some cases...
	\end{enumerate}
	Therefore, every effort should be made to avoid the blackout/failure, act quickly when it occurs to increase equipment availability. To do this, one must model the life of equipment and train people to react. The set of methods and techniques for its problems are usually classified under the name of "\NewTerm{Failure Mode and Effects Analysis FMEA}\index{Failure Mode and Effects Analysis}", but we can also found it under the name of "\NewTerm{Probabilistic risk assessment (PRA)}\index{probabilistic risk assessment}". Especially, that latter is a very official systematic and comprehensive methodology to evaluate risks associated with a complex engineered technological entity (such as an airliner or a nuclear power plant) or the effects of stressors on the environment (Probabilitic Environmental Risk Assessment- PERA) for example\footnote{In addition to the mentioned methods, PRA studies require special but often very important analysis tools like human reliability analysis\index{human reliability analysis} (HRA) and common-cause-failure analysis\index{common-cause-failure analysis} (CCF). HRA deals with methods for modeling human error while CCF deals with methods for evaluating the effect of inter-system and intra-system dependencies which tend to cause simultaneous failures and thus significant increase in overall risk.}.
	
	"\NewTerm{Quantitative Reliability Engineering}\index{quantitative reliability engineering}" analysis involves more than just reliability predictions and reliability demonstration that are performed against a given program or project requirements. Quantitative Reliability Engineering analysis can play a key role in supporting a broad range of applications. It is critical in addressing design and manufacturing deficiencies.
	
	We distinguish two main classes of systems:
	\begin{enumerate}		
		\item Non-repairable systems (satellites, although low cost goods, etc.)
		\item Repairable systems (production machinery, means of transport, etc.)
	\end{enumerate}
	where the theoretical approaches are different. For the second category, you can also use Markov chains (where the states represent the number of functional components or failure of a system in accordance with the standard ISO 31010), Petri nets or Monte-Carlo simulation.
		
	The idea is in the texts that follow to make the point on one of these methods, to seek their efficiency and enable engineers practitioners and technicians to better understand these problems. An emphasis will be placed on the Weibull model, which has a significant importance in this field.
		
	\subsubsection{Planned Obsolescence}
	The planned obsolescence goes against preventive maintenance philosophy whose primary purpose is to provide a quality product and respectful of its environment with precise knowledge of the product's lifetime and quantifiable risk of obsolescence.
		
	At the opposite, the purpose of planned obsolescence is to reduce the lifetime (or "lifespan") of a product in order to increase the rate of replacement by the buyer (sometimes motivated by political strategy or short term  money return...). Therefore the induced demand would have for aim (not proved in the very long term ... as this is quite bad for the reputation of the brand) to benefit the producer, or its competitors. The sector then has a greater production, stimulating productivity growth (economies of scale) and technical progress (which accelerates the obsolescence of previous products) and maintains the use of an ever growing population which is therefore problematic by definition.
	\begin{figure}[H]
		\centering
		\includegraphics[scale=0.7]{img/engineering/capacitor_life_expectancy.jpg}
		\caption[Example of a reliability plot of capacitor]{Example of a reliability plot of capacitor (source: Cash Investigation about Samsung TV issue) }
	\end{figure}
	The reader will therefore have to understand that I am strongly against these methods and that every engineer and every scientist working for a company must refuse by deontology this type of request coming from the hierarchy even denounce these methods anonymously on the Internet or to consumer associations.
	\begin{tcolorbox}[title=Remark,colframe=black,arc=10pt]
	The consumer must be very careful with TV shows or newspapers about the subject of planned obsolescence! Indeed as they don't have any statisticians consultants, these shows or newspapers communicate numbers in a very wrong and bad way (a journalist being not a scientist...).
	\end{tcolorbox}
		
	It must be also notice that lifetime (lifespan) will be a major subject of concern for future generations as Earth's resources are not unlimited and that in this 21st century people still don't recycle everything. So it will be a challenge for these future generations (at least if we will not be able to bring material from other planets, satellites or asteroids) to found solutions to get the longest lifespan. Having the longest lifespan is however also in the 21st century sometimes a requirement as it is an important factor in aeronautics, trains, boats or all other areas of cutting edge technologies and where human life are in concerns (or very rarely in luxury goods also).

	But let us also show a small chart of the status of the known reserves - in years - of row material resources (so not including unknows reserves!) if humans continue to consume as they do in the beginning of this 21st century (sorry did not find the french version and also the original paper of this study...):
	\begin{figure}[H]
		\centering
		\includegraphics[scale=0.38]{img/engineering/raw_materials_reserves.jpg}
	\end{figure}
		
	\pagebreak
	\subsubsection{Reliability Empirical Estimators}
	As part of the non-accelerated reliability study (accelerated aging will be treated in a near future), we are led to define some variables which are listed below:
	\begin{itemize}
		\item $N_0$ will the number of good elements at time $t_0$ (initial time)
		\item $N_i$ the number of good elements at time $t_i$
		\item $n_i$ the number of defective elements between $t_i$ and $t_{i-1}$ also denoted  by $\Delta N_i$
		\item $\Delta t_i$ the time interval between $t_i$ and $t_{i-1}$
	\end{itemize}
	\textbf{Definitions (\#\mydef):}
	\begin{enumerate}
		\item[D1.] We define the "\NewTerm{failure rate per time-slot}\index{failure rate per time-slot}" $\Delta t$ by the relation:
		
		which therefore is interpreted as being the number of defective elements relative to the number of surviving elements on a given timeslot (this is therefore a percentage of non-conformities related to a time slot).
		
		The latter relation is also sometimes named "\NewTerm{hazard function}\index{hazard function}" (or "\NewTerm{hazard ratio HR}\index{hazard ratio}") or "\NewTerm{relative survival}\index{relative survival}" and often denoted by $\bar{h}(t_i)$.
		
		\item[D2.] We define the "\NewTerm{failure function}\index{failure function}" by the relation (probability density of failure at time $t_i$):
		
		noting that the denominator is not the same as that which defines the default rate per timeslot!
		
		This function is therefore interpreted as the \% of defective elements in the studied  time frame relative to the total number of elements initially tested. This is the indicator that interests most often the engineer because can be assimilate to a probability and having the properties of a probability!
		
		\item We naturally define the "\NewTerm{cumulative failure function}\index{cumulative failure function}" by:
		
		which tends to $1$ when time tends to infinity.
		
		This function is therefore interpreted as the \% of cumulative defective components based on the total number of elements initially tested. It is therefore the function of empirical distribution of probabilities of failures.
		
		\item[D4.] We define verbatim the "\NewTerm{reliability function}\index{reliability function}" by (this is the second term of the previous relation):
		
		Its name comes from the interpretation of the ratio in the context of the definition of the cumulative failure function.
		
		It must be remembered that the letter $R$ is derived from the word "\NewTerm{reliability}\index{reliability}" and $F$ is derived from the word "\NewTerm{failure}\index{failure}" which means "breakdown".
	\end{enumerate}
	It follows from these definitions that:
	
	The latter relation is used for the calculation of laws reliability as we will see later! We deduce a notation which sometimes helps to better understand the failure rate per time slot. Effectively:
	
	Because:
	
	the failure function can be seen as a probability as we have already mentioned it, which naturally led us to define its expected mean:
	
	Very useful relation in practice which in theory gives the average percentage of failed elements at the moment $t_i$.
	\pagebreak
	So to summarize and with different notations that we can find in the literature, this gives (we will see later the same summary in the continuous case):
	\setlength\extrarowheight{12pt}
	\begin{table}[H]
		\begin{center}
			\definecolor{gris}{gray}{0.85}
				\begin{tabular}{|l|l|}
					\hline
					\cellcolor{black!30}\textbf{Function type} & \cellcolor{black!30}\textbf{Mathematical expression} \\ \hline
					Failure density function & \centering\arraybackslash\ $\hat{f}\Delta t_i=\displaystyle\frac{n_i}{N_0}$ \\ \hline
					Failure repartition function & \centering\arraybackslash\ $\bar{F}(t_i)=\displaystyle\sum_i \hat{f}(t_i)\Delta t_i=\displaystyle\sum_i \dfrac{n_i}{N_0}=P(T\leq t_i)$  \\ \hline
					Survival repartition function & \centering\arraybackslash\ $\hat{R}(t_i)=\hat{S}(t_i)=1-\hat{F}(t_i)=P(T\geq t_i)=1-\displaystyle\sum_i \hat{f}(t_i)\Delta t_i$  \\ \hline
					Risk (hazard) function & \centering\arraybackslash\ $\hat{\lambda}(t_i)=\hat{h}(t_i)=\dfrac{\hat{f}(t_i)}{\hat{R}(t_i)}$  \\ \hline
					Cumulated risk (hazard) & 		\centering\arraybackslash\ $\hat{H}(t_i)=\displaystyle\sum_i\dfrac{n_i}{N_i\Delta t_i}$  \\ \hline
			\end{tabular}
		\end{center}
		\caption{Summary of discrete maintenance analysis functions}
	\end{table}
	\setlength\extrarowheight{0pt}
	Note that we will see later below a statistical mathematical model based on the Kaplan-Meier estimator that gives in fact a non-parametric estimation of the survival function. The following example will perhaps help you better understand the concept!
	\begin{tcolorbox}[colframe=black,colback=white,sharp corners]
	\textbf{{\Large \ding{45}}Example:}\\\\
	We have identified on a batch of 37 engines of a given type the following listed failures in slices (given by customers or internally measured on test benches):
	\begin{table}[H]
	\begin{center}
		\definecolor{gris}{gray}{0.85}
			\begin{tabular}{|c|c|c|c|c|c|}
				\hline
				\multicolumn{1}{c}{\cellcolor{black!30}\vtop{\hbox{\strut \textbf{0 to }}\hbox{\strut \textbf{1,000h.}}}} & 
\multicolumn{1}{c}{\cellcolor{black!30}\vtop{\hbox{\strut \textbf{1,000 to }}\hbox{\strut \textbf{2,000h.}}}} & \multicolumn{1}{c}{\cellcolor{black!30}\vtop{\hbox{\strut \textbf{2,000 to }}\hbox{\strut \textbf{3,000h.}}}} & \multicolumn{1}{c}{\cellcolor{black!30}\vtop{\hbox{\strut \textbf{3,000 to }}\hbox{\strut \textbf{4,000h.}}}} & \multicolumn{1}{c}{\cellcolor{black!30}\vtop{\hbox{\strut \textbf{4,000 to }}\hbox{\strut \textbf{5,000h.}}}} &\multicolumn{1}{c}{\cellcolor{black!30}\vtop{\hbox{\strut \textbf{5,000 to }}\hbox{\strut \textbf{6,000h.}}}} \\ \hline
		$1$ & $4$ & $7$ & $12$ & $11$ & $2$\\ \hline
	\end{tabular}
	\end{center}
	\caption{Engine failures in effort slices}
	\end{table}
	We must consider the value of the reliability function $\hat{R}(t_i)$, of the failure function $\hat{f}(t_i)$ and the failure by time slice $\hat{\lambda}(t_i)$. The calculations are elementary and we get the following table:
	\begin{table}[H]
	\begin{center}
		\definecolor{gris}{gray}{0.85}
			\resizebox{\textwidth}{!}{\begin{tabular}{|p{0.1cm}|p{0.8cm}|p{0.9cm}|p{0.6cm}|p{0.7cm}|p{0.9cm}|p{0.7cm}|p{1.5cm}|}
				\hline
				\multicolumn{1}{c}{\cellcolor{black!30} \textbf{$i$}} & 
\multicolumn{1}{c}{\cellcolor{black!30}\vtop{\hbox{\strut \textbf{Observation}}\hbox{\strut \textbf{interval}}}} & \multicolumn{1}{c}{\cellcolor{black!30}\vtop{\hbox{\strut \textbf{Failures in}}\hbox{\strut \textbf{the interval}}}} & \multicolumn{1}{c}{\cellcolor{black!30}\textbf{Survivors}} & \multicolumn{1}{c}{\cellcolor{black!30}\vtop{\hbox{\strut \textbf{Cumulated}}\hbox{\strut \textbf{failures}}}} & \multicolumn{1}{c}{\cellcolor{black!30} \textbf{$\hat{R}(t_i)$}} & \multicolumn{1}{c}{\cellcolor{black!30} \textbf{$\hat{f}(t_i)\Delta t_i$}} & \multicolumn{1}{c}{\cellcolor{black!30}\vtop{\hbox{\strut \textbf{$\hat{\lambda}(t_i)$}}\hbox{\strut $10^{-6}[h^{-1}]$}}}\\ \hline
		\cellcolor{black!30}$0$ & $0$ & $-$ & $37$ & $0$ & $100\%$ & $0\%$ & $0$\\ \hline
		\cellcolor{black!30}$1$ & \vtop{\hbox{\strut 0 to}\hbox{\strut 1,000}} & $1$ & $36$ & $1$ & $97\%$ & $2.7\%$ & \tiny{$(1/36)/1000\cong27.7$}\\ \hline
		\cellcolor{black!30}$2$ & \vtop{\hbox{\strut 1,000 to}\hbox{\strut 2,000}} & $4$ & $32$ & $5$ & $86\%$ & $10.8\%$ & \tiny$(4/32)/1000=125$\\ \hline
		\cellcolor{black!30}$3$ & \vtop{\hbox{\strut 2,000 to}\hbox{\strut 3,000}} & $7$ & $25$ & $12$ & $67\%$ & $18.9\%$ & \tiny$(7/25)/1000=280$\\ \hline
		\cellcolor{black!30}$4$ & \vtop{\hbox{\strut 3,000 to}\hbox{\strut 4,000}} & $12$ & $13$ & $24$ & $35.1\%$ & $32.4\%$ & \tiny$(12/13)/1000\cong 923$\\ \hline
		\cellcolor{black!30}$5$ & \vtop{\hbox{\strut 4,000 to}\hbox{\strut 5,000}} & $11$ & $2$ & $35$ & $5.4\%$ & $5.4\%$ & \tiny$(11/2)/1000\cong 5,500$\\ \hline
		\cellcolor{black!30}$5$ & \vtop{\hbox{\strut 6,000 to}\hbox{\strut 6,000}} & $2$ & $0$ & $37$ & $0\%$ & $-$ & $-$\\ \hline
	\end{tabular}}
	\end{center}
	\caption{Analysis of engine failures per effort slices}
	\end{table}
	We see above  that the failure rate $\hat{\lambda}(t_i)$ is of course not constant! The failure rate $\hat{f}(t_i)$ would therefore be about $2.5\%$ to $3\%$ the first year, about $10\%$ to $11\%$ the second year,etc.
	\end{tcolorbox}
	Regarding default rates, engineers often recognize three analysis slices following that the studied objects are: 1) new, 2) in normal operation or 3) considered aging.
	
	In practice we considered quite intuitively (and sometimes roughly) that for very common goods the failure rate follows a bathtub curve as shown below (in the technical tables it is often the normal operation failure rate that is given):
	
	\begin{figure}[H]
		\centering
		\includegraphics{img/engineering/reliability_bathtube.jpg}
		\caption[]{Reliability bathtube}
	\end{figure}
	whereas if you look at the table of the example above, the failure rate does not follow at all a bathtub curve (so it is a cons-example).
	
	Reliability engineers often cut the bath in the three visible parts above, but under the following technical names:
	\begin{enumerate}
		\item D.F.R. for "\NewTerm{Decreasing Failure Rate}\index{decreasing failure rate}": young unidentified components with manufacturing problems during the process are eliminated from the batch which has the effect of reducing the failure rate. The Weibull distribution is relatively well suited to model this phase (see later below the details of this model).
		
		\item C.F.R. for "\NewTerm{Constant Failure Rate}\index{constant failure rate}": the components are in stable conditions.
		
		\item I.F.R. for "\NewTerm{Increasing Failure Rate}\index{increasing failure rate}" components are in end of life and their failure rate increases. The Weibull distribution is again relatively well adapted to model this phase.
	\end{enumerate}
	\pageref{queueing theory}
	The Poisson distribution (\SeeChapter{see section Statistics page \pageref{poisson distribution}}) is relatively well adapted to model the failure arrival time, and the exponential distribution (\SeeChapter{see section Statistics page \pageref{exponential distribution}}) to model the time between successive failures. This fact follows mathematical proofs available in the section of Quantitative Management (queueing theory). So, this means that failures are often considered independent. Therefore, the law on the number of failures in a given period is then a law "without memory" and can be modeled by a Poisson distribution (proof also made in the section of Quantitative Management).
	
	\begin{tcolorbox}[title=Remark,colframe=black,arc=10pt]
	Contrary to what a lot of theorists think... many general public softwares have also sometimes a failure rate following a bathtub curve. Indeed, at the beginning there undetected bugs that cause the failure will decrease following the identification and correction. Then, because of frequent updates of the environment that tend to add other issues (service pack), the failure rate remains fairly constant. Finally, over time, the evolution of surrounding technologies (frameworks) make the application obsolete and some functions no longer properly work which is again an increase the failure rate.
	\end{tcolorbox}
	
	Vis-à-vis the effectiveness of renovation, let us indicate (by simplifying) that they can be frequently divided into three categories:
	
	\begin{enumerate}
		\item "\NewTerm{As good as new}\index{as good as new}": This is preventive maintenance in the sense that we change a piece when his life times leads it to a failure rate that we consider too high and that the unanticipated failure will cost more that its non-anticipation.
		
		\item "\NewTerm{As bad as old}\index{as bad as old}": It is the poor maintenance in the sense that we change a piece only when it comes ot break, which mainly generates higher costs of maintenance that the preventative maintenance that involves in anticipating issues.
		
		\item "\NewTerm{Partial Restore}\index{partial restore}": This is the minimum preventive maintenance in the sense that we repair the defective part rather than replace it with a new one. Again the cost of the problem must be calculated by an audit of requirements and of the deadlines of the business.
	\end{enumerate}
	
	Let us return to other definitions in the passage to the limit of the continuum!
	
	So we know that the "\NewTerm{instantaneous failure rate}\index{instantaneous failure rate}" will have for unit the inverse of time as $[\lambda]=s^{-1}$. This rate is, within the framework of our study, not necessarily constant over time as we have seen above!
	
	Given $R(t)$ the cumulative percentage of objects analyzed (tracked) always good working of a sample tested at time $t$. The number of objects falling down during the infinitesimal time $\mathrm{d}t$ is therefore equal to:
	
	this corresponds to the decrease of the initial stock in functioning at time $t$.
	
	We can then write the relation:
	
	therefore:
	
	Which can often be found in the literature under the following equivalent forms:
	

	\textbf{Definition (\#\mydef):} The "\NewTerm{conditional probability of failure}\index{conditional probability of failure}" between $t$ and $t + \mathrm{d}t$ is defined as the temporal conditional probability of experiencing the event at a given time $t$, knowing that we have not experienced before and so we survived up to a time $t$ (i.e. it only increase with time...):
	
	where $F(t)$ and $R(t)$ are, for reminder, respectively the cumulative probability fonction of failure (cumulative probability to fail at time $t$) and the cumulative probability function of reliability also named "\NewTerm{survival function}\index{survival function}". $R (t)$ is $1$ at time $0$ and ... $0$ after an infinite time as we have already seen before!
	
	We will come back on this conditional probability further below with a more pedagogical approach during our study of Cox's proportional hazard model.
	
	\begin{tcolorbox}[title=Remark,colframe=black,arc=10pt]
	By the same intellectual approach, rather than defining a function of failure $F(t)$ and survival $R(t)$ with its associated hazard function, we can define a repairability function with its function $M(t)$ which would then be a "\NewTerm{maintainability function}\index{maintainability function}".
	\end{tcolorbox}
	If we integrate (caution! $u$ now represent the time!):
	
	As $F(t_0=0)=0$ we have:
	
	Therefore:
	\begin{equation}
  \addtolength{\fboxsep}{5pt}
   \boxed{
   \begin{gathered}
   		\begin{aligned}
		R(t)&=e^{-\int\limits_0^t\lambda(u)\mathrm{d}u}\\
		F(t)&= 1 - e^{-\int\limits_0^t\lambda(u)\mathrm{d}u}
   		\end{aligned}
   \end{gathered}
   }
	\end{equation}
		Moreover, since we have seen that $\hat{f}(t_i)=\hat{\lambda}(t_i)\hat{R}(t_i)$, then we have the "\NewTerm{density/repartition function of instantaneous failure}\index{density/repartition function of instantaneous failure}":
	
	We can get this relation and interpretation in another way:
	
	So where we find back therefore $F (t)$ the function of cumulative probability of failure. Obviously to determine the law $f (t)$, we use the usual statistical tools for goodness of fit (\SeeChapter{see section Statistics page \pageref{anderson darling gof test}}).
	
	We then get the very important equation (see the most important one!) in practice, which connects the density function of the instantaneous failure and the reliability's law:
	
	We have above the three most general expressions binding laws of reliability and the instantaneous failure rates. Let us add that we have also the following instantaneous hazard function that follows:
	
	It then follows immediately that:
	
	Since $f (t)$ is the failure density function, the expected value of the failure is given by:
	
	Thus, if the distribution of failures is equally likely (uniform density function), which is rare..., half of the equipment will be off at $E(t)$.
	
	\begin{tcolorbox}[title=Remark,colframe=black,arc=10pt]
	Observing $100,000$ hard drives, Google engineers have observed an average of $8\%$ loss per year! So a higher loss rate than would have announced that manufacturers that was about $300,000$ hours! The loss rate is higher the first $3$ years! But perhaps the disks that survive live longer!?
	\end{tcolorbox}
	We also have use the integration by parts and the above relation:
	
	Hence another way to express it:
	
	So to summarize and with different notations that we can find in the literature, this gives:
	\setlength\extrarowheight{12pt}
	\begin{table}[H]
		\begin{center}
			\definecolor{gris}{gray}{0.85}
				\begin{tabular}{|l|l|}
					\hline
					\cellcolor{black!30}\textbf{Function type} & \cellcolor{black!30}\textbf{Mathematical expression} \\ \hline
					Failure density function & \centering\arraybackslash\ $f(t)=\lim_{\mathrm{d}t\rightarrow 0} \dfrac{P(T\in [t+\mathrm{d}t])}{\mathrm{d}t}$ \\ \hline
					Failure repartition function & \centering\arraybackslash\ $F(t)=\displaystyle\int\limits_0^T f(t)\mathrm{d}t=P(T\leq t)$  \\ \hline
					Survival repartition function & \centering\arraybackslash\ $R(t)=S(t)=1-F(t)=P(T\geq t)=1-\displaystyle\int\limits_0^Tf(t)\mathrm{d}t$  \\ \hline
					Risk (hazard) function & \centering\arraybackslash\ $h(t)=\lim_{\mathrm{d}t\rightarrow 0} \dfrac{P(T\in [t+\mathrm{d}t]|T\geq t)}{\mathrm{d}t}=\dfrac{f(t)}{R(t)}$  \\ \hline
					Cumulated risk (hazard) & 		\centering\arraybackslash\ $H(t)=\displaystyle\int\limits_0^th(s)\mathrm{d}s=\displaystyle\int\limits_0^t\dfrac{f(s)}{R(s)}\mathrm{d}s=-\text{ln}(S(t))$  \\ \hline
			\end{tabular}
		\end{center}
		\caption{Summary of continuous maintenance analysis functions}
	\end{table}
	\setlength\extrarowheight{0pt}
	Let us notice again (!) that we will see further statistical mathematical model based on the Kaplan-Meier estimator giving a non-parametric estimation of the survival function!
	
	Before continuing let us give some other definitions of maintenance indicators, the most important are the European Maintenance Standard EN 13306:2010 (I precise it because on the web we can find the best and the worst on the definitions of these reliability indicators...):
	
	\begin{itemize}
		\item The "\NewTerm{Time Between Maintenance T.B.M.}\index{time between maintenance}" is the contractual time between two visits or maintenance control (not defined in the EN 13306:2010 standard).
		
		\item The "\NewTerm{Mean Operating Time Between Failures M.O.T.B.F.}\index{mean operating time between failures }" applicable only for repairable items is the expected mean of the operational time of a system between the end of the first failure and the beginning of the next (defined in the EN 13306:2010 standard). In the industry M.O.T.B.F. is also named "\NewTerm{Mean Up Time M.U.T.}\index{mean up time}".
		
		\item The "\NewTerm{Mean Time Between Failures M.T.B.F.}\index{mean time between failures}" applicable only for repairable items is the expected mean of operational time of a system between the start of two failures (defined in the standard EN 13306:2010). This indicator includes the time of non-operation. In the area of services, the M.T.B.F. is sometimes named "\NewTerm{Mean Time Between System Accidents M.T.B.A.}\index{mean time between system accidents}". In the field of the production lines MTBF is named "\NewTerm{Mean Time Between Defects M.T.B.D.}\index{mean time between defects}" (we do not stop the production because only one defect was detected as we will see during our study of control charts further below). In the field of computing science the M.T.B.F. is named "\NewTerm{Mean Time Between Errors M.T.B.E.}\index{mean time between errors}" (we also rarely stops a software or a website because one bug or error was detected!).
		
		\item The "\NewTerm{Mean Time To Failure M.T.T.F.}\index{mean time to failure}" applicable for repairable or not repairable items is the expected mean of operational time of a system to its first failure (not defined in the standard EN 13306:2010). In fact it is the latter which is calculated most often in tests laboratories.
		
		\item The "\NewTerm{Mean Repair Time M.R.T.}\index{mean repair time}" applicable only for repairable items is the expected mean of repair time (defined in the standard EN 13306:2010).
		
		\item The "\NewTerm{Mean Time To Restore M.T.T.R.}\index{mean time to restore}" applicable only for repairable items is the expected mean restart time (defined in the standard EN 13306:2010). In the area of services, M.T.T.R. is sometimes named "\NewTerm{Mean Time To Restore Service M.T.T.R.S.}\index{mean time to restore service}".
		
		\item The "\NewTerm{Mean Down Time M.D.T.}\index{mean down time}" is equal to the mathematical expected mean of the time that the system is not operational. This indicator (not defined in the standard EN 13306:2010) therefore includes the M.R.T. It follows that the M.T.B.F. is equal to the sum of the M.O.T.B.F. (M.U.T.) and of the M.D.T.
		
		\item The "\NewTerm{Mean Time To Detection M.T.T.D.}\index{mean time to detection}" is equal to the mathematical expected mean of the time that the system is not detected as being non operational. This indicator (not defined in the standard EN 13306:2010), as well as the M.R.T. is included in the M.D.T.
		
		\item The "\NewTerm{Mean Time to Answer M.T.T.A.}\index{mean time to answer}" is equal to the mathematical expected mean of the time during which the user expected an answer or response of manufacturer after reporting a failure. This indicator (not defined in the standard EN 13306:2010), as well as the M.R.T. and M.T.T.D. are included in the M.T.T.A.
		
		\item The "\NewTerm{Overall Equipment Effectiveness O.E.E.}\index{overall equipment effectiveness}" in the field of maintenance is traditionally defined as being simply the ratio of the operational running time and time expected operational performance (not defined in the standard EN 13306:2010).
		
		\item The "\NewTerm{Service Ability S.A.}\index{service ability}" which in the field of maintenance is often given by the following two relations and whose goal is to make it as close as possible to $1$ (that is to say to make the M.D.T. approaching zero):
		
	\end{itemize}
	and we can continue to define many others indicators outside the standard...
	
	The below diagram summarizes maybe a little better all these definitions:
	\begin{figure}[H]
		\centering
		\includegraphics{img/engineering/reliability_indicators.jpg}
		\caption{Common Reliability Indicators}
	\end{figure}
	\begin{tcolorbox}[title=Remarks,colframe=black,arc=10pt]
	\textbf{R1.} For each of these indicators it will be indicated if external causes have been taken into account or not.\\
	
	\textbf{R2.} If the tested sample size is very small or just unitary (only one machine), it is customary to use the arithmetic average as an estimator of the expected mean.... Obviously it can lead to huge errors...\\
	
	\textbf{R3.} It is of course possible from the defined indicators to calculate whether a machine (element) is able to ensure the desired service (such as a number of pieces per year to produce for example!). So not only do these indicators can be measured, but they can therefore also be used to verify the achievement of a desired goal!\\
	
	\textbf{R4.} In the beginning of the 21st century countries should strictly speaking legislate to require all manufacturers to communicate the MTTF of their products so that consumers can make the best choice to buy and compare the values in relation to the guarantee provided! Unfortunately, this is not the case and it would highlight a current bad tradition in the consumer products industry that is to manufacture components whose life revolves around 200,000 hours to ensure industrial renewal of their market.
	\end{tcolorbox}
	
	\begin{tcolorbox}[colframe=black,colback=white,sharp corners]
	\textbf{{\Large \ding{45}}Example:}\\\\
	A machine is supposed to work 24/24 hours 7/7 days. It turned up today for $5,020$ hours with $2$ stops of a total of $20$ hours (thus included in the $5,020 [\text{h}]$). Give the conventional indicators relative to the limited information provided:
	

	\end{tcolorbox}
	The classification of systems in terms of availability commonly leads to 7 classes from "not available" (system available $90$\% of the time, and therefore unavailable over a month per year) to "ultra available" (available $99.99999$\% of the time and therefore unavailable only $3$ seconds per year): these classes corresponds arbitrarily to the numbers of "$9$" in the percentage of time that the class systems are available (one year includes $525,600$ minutes for refresh):
	\begin{table}[H]\centering
	\begin{center}
		\definecolor{gris}{gray}{0.85}
			\begin{tabular}{|p{4cm}|p{3.5cm}|p{2cm}|p{1cm}|}
				\hline
				\multicolumn{1}{c}{\cellcolor{black!30}Type} & 
  \multicolumn{1}{c}{\cellcolor{black!30}Non-availability [min/year]} & 
  \multicolumn{1}{c}{\cellcolor{black!30}Percentage availability} & 
  \multicolumn{1}{c}{\cellcolor{black!30}Class} \\ \hline
				 non-managed & $50,000\; (\sim3\;5 \text{days})$ & $90\%$ & $1$ \\ \hline
				 managed & $5,000\;  (\sim3.5\; \text{days})$ & $99\%$ & $2$ \\ \hline
				 well-managed & $500\; (\sim8\; \text{hours})$ & $99.9\%$ & $3$\\ \hline
				 faultily tolerance & $50$ & $99.99\%$ & $4$ \\ \hline
				 high availability  & $5$ & $99.999\%$ & $5$  \\ \hline
				 very high availability & $0.5$ & $99.9999\%$ & $6$  \\ \hline
				 very high availability & $0.05$ & $99.99999\%$ & $7$  \\ \hline
		\end{tabular}
	\end{center}
	\caption[]{Classes failures in FMEA}
	\end{table}
		The use of these parameters in the context reliability framework make me us told that we have an "approach through average values".
		
		Note however one thing! This is not because an event has a smaller or infinitely less probability than another one that it is less important!! Indeed we must obviously take into account (certainly a little in a empirical way) the gravity of the failure. Thus, a nuclear power plant may well be $1,000$ times more secure than a thermal central in terms of probabilities ... nothings is comparable in terms of severity when a major failure occurs in a nuclear plant! Indeed in the case of a thermal central there may be in the worst case $500$ persons affected while with a nuclear power plant that is another story (see Tchernobyl or Fukushima...)!
		
		Let us also note a simple case: Some components (typically electronic) have in their period of a maturity constant failure rate. The cumulative probability distribution of the failure that results is then immediately deducted as the instantaneous hazard function is given by $\lambda(t)=\lambda$ and therefore:
	
	Let us indicate that we find very often the latter relation in the following form in the literature:
	
	and then we have well:
	
	To summarize a bit for the exponential law:
	
	And as we have:
	
	And we proved in the section of Differential and Integral Calculus that:
	
	Therefore:
	
	The previous hazard and survival are then graphically of the following form:
	\begin{figure}[H]
		\centering
		\includegraphics{img/engineering/exponential_law_reliability.jpg}
		\caption{Plot of the hazard function and Reliability of the exponential law}
	\end{figure}	
	The density function of faulty components at time $t$ is obviously:
	
	It therefore follows an exponential law! This law and its moments are known to us (\SeeChapter{see section Statistics page \pageref{exponential distribution}}). It then becomes easier to determine the M.T.T.F. and its standard deviation (inverse of failure rate) and also a confidence interval for it.
	
	Furthermore, if we calculate the reliability $R(t)$ at the time corresponding to the M.T.T.F. (inverse of the failure rate in the case of the exponential law) we will always obtain a cumulative probability of $F(t)=1-e^{-1}=36.8\%$ (so basically almost one chance on three chance to run at this time and $2$ chances on $3$ to be down) and not $50\%$ as sometimes we can intuitively think (this is the case only if the probability distribution is symmetric).
	
	Since the technical tables of reliability in the industry almost always assume the reliability rate to be constant then we better understand the importance of it (we will have an example in our presentation further below of topological systems).
	
	Also remember that we saw in the section of Probability that if we have a set of independent events (independent mechanisms) with a given probability, then the probabilities calculated on the whole as a globality involves the multiplication of probabilities!
	
	Therefore in a mechanism having independent but essential parts (system said to be "\NewTerm{serial system}\index{serial system}") the overall reliability density function will be equal to the multiplication of cumulative probability distributions $R (t)$ and which is therefore equivalent in the case of a exponential having a single exponential law whose overall failure rate is equal to the sum of the various parts failure rate!
	
	Another example... in mechanics, where the phenomenon of wear is the cause of the failure, the failure rate is often linear (so it increases steadily over time and is not zero at the initial starting of the device):
	
	Then:
	
	Therefore:
	
	As (for an equipment whose repair time is negligible):
	
	this integral can be calculated by numerical integration methods (\SeeChapter{see section Statistics page \pageref{numerical integration}}). Therefore, in practice, we prefer to take closed form type of law thanks to goodness of fit type statistical tests (\SeeChapter{see section Statistics page \pageref{goodness of fit tests}}), such as the Weibull distribution (see just below for detailed developments), which is a little bit the trash to put everything in the field of reliability engineering...
	
	You should know that in common and free public reliability data banks reliable on the market are normally given: the name of the material or component, the M.T.B.F., the average default rate, the statistical heritage (...), a multiplicative factor of failure rates depending on the environment or usage constraints. 
	
	\pagebreak
	\paragraph{Average Failure Rate}\mbox{}\\\\
	As we have seen, the hazard function usually varies with time. What is sometimes of interest to the practitioner is the average rate of failures within a given time interval, which is clearly specified by the functional specification. We then define the average failure rate or "\NewTerm{average failure rate}\index{average failure rate}":
	
	Therefore:
	
	This rate, therefore denoted by $\text{AFR}(t_1,t_2)$, is a single number that can be used as a specification or target for the population failure rate over that interval. If $t_1$ is $0$, it is dropped from the expression. Thus, for example, $\text{AFR}(40,000)$ would be the average failure rate for the population over the first $40,000$ hours of operation.  

	An interval often taken in practice is $0$ to $\tau$. Then in comes:
	
 	Sometimes in the literature considering the following variant:
	
	and considering that $F(\tau)$ is small, by by a Taylor development to the second order, we have:
	
 	We can transform the antecedent relations into a failure distribution function:
	
 	and therefore:
	
 	And finally:	
	
	
	\pagebreak
	\subsubsection{Weibull Distribution}\label{weibull distribution}
	Once again maintenance techniques use probability and statistics so we refer the reader to the section of the same name in this book. However, there exists in the field of maintenance (and not only!) a probability density function widely used named "\NewTerm{Weibull law}\index{Weibull law}".
	
	It seemed to be completely empirical and defined by:
	
	where $f(x)\geq 0,x\geq 0$ with $\beta >0,\eta>0,-\infty<\gamma<+\infty$ that are named respectively "\NewTerm{scale parameter}\index{scale parameter}" $\eta$, "\NewTerm{shape parameter}\index{shape parameter}" $\beta$ and "\NewTerm{location parameter}\index{location parameter}"  $\gamma$.
	
	\begin{tcolorbox}[title=Remark,colframe=black,arc=10pt]
	The engineers in corporations have to to refer to \textit{IEC 61649 Weibull analysis} to make a standardized usage of this law!
	\end{tcolorbox}	
	The Weibull distribution is often written as following by putting $x:=x-\gamma$, $\eta:=\beta$, $\beta:=\alpha$ (in this case the location parameter is implicitly posed as equal to zero):
	
	It can be calculated in the English version of Microsoft Excel 11.8346 in this latter form using \texttt{WEIBULL( )} function.
	\begin{tcolorbox}[title=Remark,colframe=black,arc=10pt]
	The engineers in corporations have to to refer to I\textit{EC 61649 Weibull analysis} to make a standardized usage of this law!
	\end{tcolorbox}
	The Weibull law can be seen as a generalization of the exponential distribution function (we'll see why later!) with the advantage that it is possible to play with the three parameters to obtain almost anything.
	\begin{figure}[H]
		\begin{center}
			\includegraphics{img/engineering/weibull_law.jpg}
		\end{center}	
		\caption[Weibull Density and Distribution functions for different $\alpha$ values]{Weibull Density and Distribution functions for different $\alpha$ values (source: Wikipedia)}
	\end{figure}
	By canceling the location parameter $\gamma$, we get the "\NewTerm{two-parameters Weibull distribution}\index{two-parameters Weibull distribution}":
	
	which has an important practical application and for which we calculated the estimators of the parameters in the Statistics section.
	
	By assuming again $\gamma=0$ and assuming also that we have $\beta=C=c^{te}$ we get "\NewTerm{one-parameter Weibull distribution}\index{one-parameter Weibull distribution}" (it's a little stupid as a definition but never-mind...):
	
	where the only unknown parameter is the scale parameter $\eta$. We assume that the parameter $\beta$ is known a priori from past experiences on similar samples. Notice that the two-parameter Weibull distribution function a is the derivative of the following repartition function (cumulative probability function or "unreliability" as say practitioner in the field of maintenance) where there are three successive interiors derivatives to fall back on the Weibull distribution function ... which is a good application of this we saw in the section of Differential and Integral Calculus:
	
	It follows therefore in reliability engineering that in this particular case, we have then:
	
	We also find sometimes this relation in the literature in the form:
	
	Note that if $\beta=1$, then we fall back on an exponential of mean $\eta$:
	
	The MTBF is then given by the mean of the Weibull distribution with nonzero $\gamma$:
	
	Let us put:
	
	with:
	
	What gives:
	
	The first integral above is already known to us, we have already study it in the section of Integral and Differential Calculus. This is the gamma Euler function:
	
	Therefore we have finally:
	
	By canceling the location parameter $\gamma$ we get the following current case in reliability:
	
	\begin{tcolorbox}[title=Remark,colframe=black,arc=10pt]
	If then by chance $\beta^{-1} \in \mathbb{N}$, then as we have proved during our study of the gamma Euler function:
	
	In case where $\beta^{-1} \in \mathbb{R}$ we must make use of the numerical tables obtained by the algorithms seen in the section of Theoretical Computing.
	\end{tcolorbox}	
	Similarly:
	
	Finally when the location parameter is equal to zero, we get:
	
	\begin{tcolorbox}[title=Remark,colframe=black,arc=10pt]
	Some specialists, when it is made usage of two moments of order two in the probabilistic analysis of reliability, sometimes talk about "\NewTerm{mean square approach method}\index{mean square approach method}"...
	\end{tcolorbox}	
	To conclude on the Weibull Law, note that it is customary to consider five important and relevant situations of the Weibull distribution based on the values of its parameters:
	
	\begin{enumerate}
		\item Survival density function looks like a rapid exponential decay and therefore the hazard function (failure rate) is high initially and decreases over time (first portion of the "bathtube" hazard function/failure rate). This is typically the situation where the failures are early because  failures occur in the initial period of the life of the product.
		
		\item Survival density looks like an exponential decay and therefore the hazard function (failure rate) is constant over time (second portion of the "bathtube" hazard function/failure rate). This is typically a situation where failures are random and sources of multiple causes. In this situation, the Weibull models the phase of "useful life" of the product.
		
		\item The survival density function rises to a peak and then decreases with a fairly pronounced asymmetry (high density on the left) and therefore the hasard function (failure rate) increases rapidly at first and eventually stabilize. This is typically a situation where the initial wear is very fast.
		
		\item The survival density function looks like a half large Gaussian Law and the hasard function (failure rate) is growing rapidly (exponential type of growth). This is typically the situation where we reach the end of life of a set of parts (third portion of the "bathtube" hazard function/failure rate).
	\end{enumerate}
	\begin{figure}[H]
		\begin{center}
			\includegraphics[scale=0.85]{img/engineering/weibull_reliability_examples.jpg}
		\end{center}	
		\caption{Plot of four 4 Weibull survival functions for different parameter values}
	\end{figure}
	And with the corresponding hasard functions:
	\begin{figure}[H]
		\begin{center}
			\includegraphics[scale=0.85]{img/engineering/weibull_hasard_examples.jpg}
		\end{center}	
		\caption{Plot of four 4 Weibull hasard functions for different parameter values}
	\end{figure}
	To summarize  we have therefore for the Weibull distribution with two parameters:
	
	So we see that if well that if we pub $\beta=1$ we fall back on all the reliability functions based on the exponential law.
	\begin{tcolorbox}[colframe=black,colback=white,sharp corners]
	\textbf{{\Large \ding{45}}Example:}\\\\
	A car manufacturer knows the distribution of time until the first failure of one of its vehicles under average driving conditions. The Weibull distribution function obtained by the tests is a two-parameter Weibull function given by:
	
	where $x$ is the distance in kilometers of the tested vehicles.	
	If we want to find the warranty in kilometers encompassing $5\%$ of the vehicles, then we must solve the equation:
	
	That is to say with some elementary algebra:
	
	\end{tcolorbox}
	
	\paragraph{Two-parameter Weibull distribution linearization}\mbox{}\\\\
	In the case of the 2-parameter Weibull, the CDF (also the "unreliability" for recall) is given as we have just seen by (we use the temporal variables $t$ instead of $x$ now):
	
 	This function can then be linearized (i.e., put in the common form of $y=ax+b$ format) as follows for practial regression (adequation) analysis.
 	
 	We start from the previous relation and rearrange the whole:
	
 	and we take the natural logarithm:
	
 	Therefore:
	
	Rearranging and taking the natural logarithm againa we get:
	
	Using the properties of the logarithm (\SeeChapter{see section Functional Analysis page \pageref{logarithms}}):
	
	Then by setting:
	
	and:
	
	the equation can then be rewritten as:
 	
	which is now a linear equation with a slope of:
	
	and an intercept of:
	
	
	\begin{tcolorbox}[colframe=black,colback=white,sharp corners]
	\textbf{{\Large \ding{45}}Example:}\\\\
	Let us now calculate as application example the Average Failure Rate of this Weibull law:
	
	In the case of a time interval of the type $[0,\tau]$ we then have:
	
	\end{tcolorbox}
	
	\pagebreak
	\subsubsection{Topology of Systems}
	When working with non-repairable real systems (mechanical, electronic or other), we face different constraints depending on the type of installation that we have. The study of such systems is also named "\NewTerm{Reliability Block Diagram}\index{reliability block diagram}".
	
	The two main assumptions of the study of these systems are:
	\begin{enumerate}
		\item The failure of a component is independent of the  others.
		\item No breakdowns arrive at the same time.	
	\end{enumerate}
	We recognize five main topologies in which each component is represented by a block:
	\begin{enumerate}
		\item Serial topology:
		
		If only one component fails the whole system does not work anymore (the examples are so numerous and easy to find that we did not mention any of them).
		
		So in the hypothesis that the failure of a component is independent of the other, the cumulative probability of failure is the product of cumulative probabilities (\SeeChapter{see Section Probabilities page \pageref{joint probability}}).
		\begin{figure}[H]
			\begin{center}
				\includegraphics{img/engineering/topology_serial.jpg}
			\end{center}	
			\caption{Serial Topology}
		\end{figure}
		In the study of probabilistic fault tree (see further below), this topology corresponds to an OR gate with $n$ inputs, as it is sufficient that onE blocks may be down so that the output no longer works:
		\begin{figure}[H]
			\begin{center}
				\includegraphics{img/engineering/gate_or.jpg}
			\end{center}	
			\caption{OR gate}
		\end{figure}
		As often made use of the exponential distribution. The multiplication of several terms of the accumulated probabilities are relatively long to write, we prefer the use of the cumulative probability of reliability $R$.
		
		Thus, the reliability (probability of operation) of a series system is given by:
		
		Which brings us well to zero reliability if at least one block has zero reliability.
		
		What is traditionally noted in the field:
		
		\begin{tcolorbox}[title=Remark,colframe=black,arc=10pt]
		The attentive reader will have noticed that the serial system is always less reliable than its least reliable component!
		\end{tcolorbox}	
		Caution!! In the case of electronic components, the failure rate is often regarded as constant for simplicity and density function is that of an exponential law:
		
		However, we have proved that in the case of a non-repairable system:
		
		And as we have proved it in the section Statistics the expected mean of the exponential law:
		
		We have also proved earlier that:
		
		Therefore:
		
		So for a series constant failure rate system:
		
		Thus, in this particular case:
		
		or written differently:
		
		and... the problem is that the a lot of Internet pages that talk about series topology systems (or parallel) implicitly use an exponential law, this is why sometimes beginner practitioners use this last relation without having studied mathematics underlying details and this brings to huge errors in the reliability of their product.
		\item Parallel topology:
		
		Unlike the previous system, parallel system continues to operate if at least one component still works (typically redundancy systems in airplanes, rockets, nuclear power plants or critical computer systems). Therefore is is considered as the most important system to study.
		\begin{figure}[H]
			\begin{center}
				\includegraphics{img/engineering/topology_parallel.jpg}
			\end{center}	
			\caption{Parallel Topology}
		\end{figure}
		In the study of probabilistic fault tree (see further below), this topology corresponds to an AND gate with $n$ inputs, as it is necessary that all blocks are down so that the output no longer works:
		\begin{figure}[H]
			\begin{center}
				\includegraphics{img/engineering/gate_and.jpg}
			\end{center}	
			\caption{AND gate}
		\end{figure}
		In other words, it stops working if and only if the cumulated probability failure is equal to:		
		
		So it comes immediately for a purely parallel system:
		
		In the very common case where the blocks are redundant with equal reliability $R_i=R$ and that $R_P$ is fixed by the customer we must found a component such that is reliability is therefore equal to:
		
		\begin{tcolorbox}[title=Remark,colframe=black,arc=10pt]
		In Project management this type of parallel structure is typically used when one wishes to know the total reliability of several supplier of identical parts (in this case if any of them has delays or other problems, the others must afford to mitigate the negative impact) knowing or believing obviously the reliability of each...
		\end{tcolorbox}	
		We therefore have the parallel system whose components have a constant failure rate and all identical (to simplify the study):
		
		Therefore we have:
		
		If we put:
		
		and replacing in the prior-previous expression we get:
		
		As far as we know is not possible to integrate this latter relation, but if we compared for different values of $n$ fixed then we see quickly that:	
		
		or written in another way:
		
		Let us still do the detailed development for a parallel system with two components whose failure rate is constant and not identical. For this, we start from the relation proved earlier above:
		
		Either in the case where $n$ is equal to $2$, we have:
		
		and therefore:
		
		So if $\lambda_1=\lambda_2$, we have:
		
		Therefore:
		
		Before continuing with other topologies, here is an example of a small school assignment (with solution!) mixing parallel/serial topology that shows that this does not apply only to machines.
		
		\begin{tcolorbox}[colframe=black,colback=white,sharp corners]
		\textbf{{\Large \ding{45}}Example:}\\\\
		A laborant must perform $3$ different tests on a sample set: Tests $A$, $B$ and $C$ in the given order! We know that the laborant miss the test $A$ ne out of four times but may start it again $2$ times in case of failure (so they can do it $3$ times in total), they succeed the terst $B$ in $96\%$ of times and may also start it again only once on failure (so they can do it $2$ times in total), they fail the test $C$ half of time and can not start it again. What is the probability that a laborant having successfully made the $3$ lab tests in only one time (without failed)?\\
		
		To solve this exercise you have to see each of reproducible tests as a parallel process and all the tests as a serial process. We then the have total probability that is given by:
		
		\end{tcolorbox}
				
		\item Topology $k/n$:
		This system works if $k$ out of $n$ components of same reliability law $R$ still work. This is typically the case of RAID hard disks computer where it is necessary to more than one to always works in the end and this number is determined by the volume of data. It is also operating principle of redundancy voting system on airplanes where we have a  $2/3$ system named "triplex".
		
		We then have the following schematic representation:
		\begin{figure}[H]
			\begin{center}
				\includegraphics{img/engineering/topology_kn.jpg}
			\end{center}	
			\caption{$k/n$ Topology}
		\end{figure}
		In the study of probabilistic fault tree (see further below), this topology corresponds to a $k$-OR (Voting OR), as it is necessary that $k$ blocks are down so that the output no longer works:
		\begin{figure}[H]
			\begin{center}
				\includegraphics{img/engineering/gate_or_kn.jpg}
			\end{center}	
			\caption{Gate $k$-OR (Voting OR)}
		\end{figure}
		Therefore seek the cumulative probability law of reliability returns to the question of the cumulative probability of having $k$ elements that work among $n$ that are indistinguishable.
		
		This therefore is equivalent to use the binomial distribution (\SeeChapter{see section Statistics page \pageref{binomial distribution}}) for which we have proved that the cumulative probability was given by:
		
		This relationship is extremely important in the practice of reliability engineering !!! We notice as well as for $k=n$ we find the reliability of a series system whose elements all have the same reliability:
		
		And for $k=1$, we fall back on the reliability of a parallel system whose elements all have the same reliability (remember the properties of the binomial coefficient in the section Calculus if necessary):
		
		\begin{tcolorbox}[colframe=black,colback=white,sharp corners]
		\textbf{{\Large \ding{45}}Example:}\\\\
		In the case of the triplex, we have therefore:
		
		So, if the reliability is given by an exponential law where we have already proved that:
		
		Then we have:
		
		So the overall MTBF of the $2/3$ system is larger than a simple system which obviously the goal.
		\end{tcolorbox}
		
		\pagebreak
		\item[T4.] Serial/Parallel and Parallel/Serial topology with symmetrical configuration:
		
		These are just simple compositions of the first two previously studied systems.
		
		We first have the series/parallel system:
		\begin{figure}[H]
			\begin{center}
				\includegraphics{img/engineering/topology_serial_parallel.jpg}
			\end{center}	
			\caption{Serial/Parallel symmetric Topology}
		\end{figure}
		However, as the series systems are given by:
		
		and the parallel by:
		
		the composition of the two gives trivially in the case above:
		
		And we also have in the family of parallel/series systems
		\begin{figure}[H]
			\begin{center}
				\includegraphics{img/engineering/topology_parallel_serial.jpg}
			\end{center}	
			\caption{Parallel/Serial symmetric Topology}
		\end{figure}
		when by using exactly the same approach as previously we have:
		
		
		\item[T5.] Complex Topology:
		
		In fact, it is not really about complex systems but they just require a little knowledge of probability axioms. The particular example which will interest us is the following (typical cascading RLC filter):
		\begin{figure}[H]
			\begin{center}
				\includegraphics{img/engineering/topology_bridge.jpg}
			\end{center}	
			\caption{Complex topology (bridge type)}
		\end{figure}
		named "\NewTerm{bridged network}\index{bridged network}".
		
		And we guess that what makes the complex system robust is the component number $5$. We can then consider a first approach that is to break down the problem.
		
		The system relatively to component $5$ will be in one of the state:
		\begin{figure}[H]
			\begin{center}
				\includegraphics{img/engineering/topology_bridge_first_decomposition.jpg}
			\end{center}	
			\caption[]{Bridged Network system decomposition (first stage)}
		\end{figure}
		if defective with a probability density function:
		
		and having itself a reliability:
		
		according to our previous results of complex series/parallel system.
		
		Or in the following state if $5$ is functional with a probability density law $R_5$:
		\begin{figure}[H]
			\begin{center}
				\includegraphics{img/engineering/topology_bridge_second_decomposition.jpg}
			\end{center}	
			\caption[]{Bridged Network system decomposition (second stage)}
		\end{figure}
		and having itself a reliability of:
		
		according to our previous results on the parallel/serial complex system.
		
		Since the system can not be in both states at the same time (at this time it is not a quantum system...), we are dealing with a disjoint probability (\SeeChapter{see section Probabilities page \pageref{disjoint events}}) that is to say the sum of the densities to which we must join the other components. Therefore we have:
		
	\end{enumerate}
		We can then using all 5 previous topologies evaluate the reliability of any system!
	
	Another strategy for complex systems is to decompose them into simple series or parallel systems. If this is not possible, we can calculate the reliability of each system configuration that is considered to operate and then sum the probabilities of operation of each configuration.
	\begin{tcolorbox}[colframe=black,colback=white,sharp corners]
		\textbf{{\Large \ding{45}}Example:}\\\\
		Let's make an example by considering the following case:
		\begin{figure}[H]
			\begin{center}
				\includegraphics{img/engineering/topology_complex_example.jpg}
			\end{center}	
			\caption[]{Special case assumed as a complex system}
		\end{figure}
		And consider the following truth table with the $2^n$ possible configurations:
	\end{tcolorbox}
	\begin{table}[H]\centering
	\begin{center}
		\definecolor{gris}{gray}{0.85}
			\begin{tabular}{|c|c|c|c|c|}
				\hline
				\multicolumn{1}{c}{\cellcolor{black!30}\textbf{State Number}} & 
  \multicolumn{1}{c}{\cellcolor{black!30}\textbf{Block 1}} & 
  \multicolumn{1}{c}{\cellcolor{black!30}\textbf{Block 2}} & 
  \multicolumn{1}{c}{\cellcolor{black!30}\textbf{Block 3}} & 
  \multicolumn{1}{c}{\cellcolor{black!30}\textbf{Output}}\\ \hline
				 1 & $0$ & $0$ & $0$ & $0$\\ \hline
				 2 & $1$ & $0$ & $0$ & $0$\\ \hline
				 3 & $0$ & $1$ & $0$ & $0$\\ \hline
				 4 & $1$ & $0$ & $1$ & $1$\\ \hline
				 5 & $0$ & $1$ & $1$ & $1$\\ \hline
				 6 & $1$ & $1$ & $1$ & $1$\\ \hline
				 7 & $1$ & $1$ & $0$ & $0$\\ \hline
				 8 & $0$ & $0$ & $1$ & $0$ \\ \hline
		\end{tabular}
	\end{center}
	\caption[]{Truth table of a preventive maintenance system}
	\end{table}
	The principle (not always easy to guess...) is to say that an UP state (thus worth equal $1$) is assigned to the value $R_1$ofthe concerned block $i$ and the DOWN state (thus worth equal $0$) is assigned to the value $1-R_i$ of the concerned block $i$. Each value will be multiplied with others to get the total reliability of the system in a given state.
	
	Thus, in the previous example, we have $3$ states that allow the system to operate. Let us calculate their respective reliability. The states No $4$ and No $5$ thus provide by definition:
	
	The state No $6$ gives him by definition:
	
	and we sum all as previously indicated:
	
	And we can check that this approach is actually correct by taking the general relation of such a system proved above:
	
	this shows that we have the same result and that the decomposition approach also works.
	
	Finally, let us notice in conclusion that when we include in the systems elements that can tolerate or accept some errors, we then speak of "\NewTerm{fault tolerance}\index{fault tolerance}" and we distinguish mainly three types:
	\begin{enumerate}
		\item \NewTerm{Active redundancy}\index{active redundancy} In this case all redundant components operate simultaneously.
		
		\item \NewTerm{Passive redundancy}\index{passive redundancy} One redundant element is working, the others are waiting, which has for advantage to reduce or remove the aging of redundant elements, but in return requires the insertion of a component of failure detection and switching.
		
		\item \NewTerm{Majority Redundancy}\index{majority redundancy} This redundancy mainly concerns signal processing. The output signal will be that of the majority of redundant components.
	\end{enumerate}
		
	\pagebreak
	\paragraph{Fault Tree Analysis}\mbox{}\\\\
	A "\NewTerm{fault tree}\index{fault tree}" is a widely used technique in engineering safety studies and reliability of systems consisting of represent the possible combinations of events that allow the realization of a predefined undesirable event. Such a representation therefore highlights the relationships of cause and effect:
	\begin{figure}[H]
		\centering
		\begin{tikzpicture}[
		% Gates and symbols style
		    and/.style={and gate US,thick,draw,fill=red!60,rotate=90,
				anchor=east,xshift=-1mm},
		    or/.style={or gate US,thick,draw,fill=blue!60,rotate=90,
				anchor=east,xshift=-1mm},
		    be/.style={circle,thick,draw,fill=green!60,anchor=north,
				minimum width=0.7cm},
		    tr/.style={buffer gate US,thick,draw,fill=purple!60,rotate=90,
				anchor=east,minimum width=0.8cm},
		% Label style
		    label distance=3mm,
		    every label/.style={blue},
		% Event style
		    event/.style={rectangle,thick,draw,fill=yellow!20,text width=2cm,
				text centered,font=\sffamily,anchor=north},
		% Children and edges style
		    edge from parent/.style={very thick,draw=black!70},
		    edge from parent path={(\tikzparentnode.south) -- ++(0,-1.05cm)
					-| (\tikzchildnode.north)},
		    level 1/.style={sibling distance=7cm,level distance=1.4cm,
					growth parent anchor=south,nodes=event},
		    level 2/.style={sibling distance=7cm},
		    level 3/.style={sibling distance=6cm},
		    level 4/.style={sibling distance=3cm}
		%%  For compatability with PGF CVS add the absolute option:
		%   absolute
		    ]
		%% Draw events and edges
		    \node (g1) [event] {No flow to receiver}
			     child{node (g2) {No flow from Component B}   
			     	child {node (g3) {No flow into Component B}
			     	   child {node (g4) {No flow from Component A1}
			     	      child {node (t1) {No flow from source1}}
			     	      child {node (b2) {Component A1 blocks flow}}
					}
			     	   child {node (g5) {No flow from Component A2}
			     	      child {node (t2) {No flow from source2}}
			     	      child {node (b3) {Component A2 blocks flow}}
					}
				   }
			     	child {node (b1) {Component B blocks flow}}
				};
		%% Place gates and other symbols
		%% In the CVS version of PGF labels are placed differently than in PGF 2.0
		%% To render them correctly replace '-20' with 'right' and add the 'absolute'
		%% option to the tikzpicture environment. The absolute option makes the 
		%% node labels ignore the rotation of the parent node. 
		   \node [or]	at (g2.south)	[label=-20:G02]	{};
		   \node [and]	at (g3.south)	[label=-20:G03]	{};
		   \node [or]	at (g4.south)	[label=-20:G04]	{};
		   \node [or]	at (g5.south)	[label=-20:G05]	{};
		   \node [be]	at (b1.south)	[label=below:B01]	{};
		   \node [be]	at (b2.south)	[label=below:B02]	{};
		   \node [be]	at (b3.south)	[label=below:B03]	{};
		   \node [tr]	at (t1.south)	[label=below:T01]	{};
		   \node [tr]	at (t2.south)	[label=below:T02]	{};
		%% Draw system flow diagram
		   \begin{scope}[xshift=-7.5cm,yshift=-5cm,very thick,
				node distance=1.6cm,on grid,>=stealth',
				block/.style={rectangle,draw,fill=cyan!20},
				comp/.style={circle,draw,fill=orange!40}]
		   \node [block] (re)					{Receiver};
		   \node [comp]	 (cb)	[above=of re]			{B}  edge [->] (re);
		   \node [comp]	 (ca1)	[above=of cb,xshift=-0.8cm]	{A1} edge [->] (cb);
		   \node [comp]	 (ca2)	[right=of ca1]			{A2} edge [->] (cb);
		   \node [block] (s1)	[above=of ca1]		{Source1} edge [->] (ca1);
		   \node [block] (s2)	[right=of s1]		{Source2} edge [->] (ca2);
		   \end{scope}
		\end{tikzpicture}	
		\caption[Example of Simple Fault Tree]{Example of Simple Fault Tree (author: Zhang Long)}
	\end{figure}
	This technique is complemented by mathematical processing that allows the combination of single failures and their probability of occurrence. It therefore allows to quantify the probability of occurrence of an adverse event.
	
	The idea is very simple as long as we use only binary probabilities, by cons when it is necessary to use continuous distributions functions (the most common case in practice), we must go through Monte Carlo simulations (\SeeChapter{section Theoretical Computing}) with typical software like Weibull++.
	
	Consider the following fault tree that has $5$ primitive events and $3$ combined events and where the probabilities of the primitive events are assumed to be known:
	\begin{figure}[H]
		\begin{center}
		\includegraphics{img/engineering/fault_tree.jpg}
		\end{center}	
		\caption{Example of Simple Fault Tree with binary probabilities and AND / OR gates only}
	\end{figure}
	We then for the next event $E6$ the following probability since it is an AND gate (equivalent to a parallel system):
	
	and for the event $E7$ since it is an OR gate (similar to a series system), we have:
	
	And so in the end:
	
	Thus, the probability of having the room without light is (a posteriori) of $14.24\%$.
	
	\paragraph{Markov Chain Reliability Model}\mbox{}\\\\
	The notable strengths of Markov chain (\SeeChapter{see section Statistics page \pageref{markov chains}}) models for reliability analysis is that they can account for repairs as well as failures. This makes the technique particularly useful for assessing the long-term average reliability of one or more devices with established maintenance and repair strategies.  With the advent of high-integrity "fault-tolerant" systems, the ability to account for repairs of partially failed (but still operational) systems has become increasingly important. Markov modeling is well-suited to the task of determining inspection and repair intervals needed to achieve a desired level of safety.

	For any given system, a "\NewTerm{Markov chain reliability model}\index{Markov chain reliability model}" consists of a list of the possible states of that system, the possible transition paths between those states, and the rate parameters of those transitions. 
	\begin{tcolorbox}[title=Remark,colframe=black,arc=10pt]
	We should speak rigorously about "\NewTerm{semi-Markov chain reliability model}\index{Markov chain reliability model}" as mathematicians do the difference between a Markov chain where the jumps from one state to another are at constant probability an a semi-Markov chain where the jumps are random variables with any distribution, whose distribution function may depend on the two states between which the jump is made. For example semi-Markov process where all the holding times are exponentially distributed is named a "continuous time Markov chain/process".
	\end{tcolorbox}	
	In reliability analysis the transitions usually consist of failures and repairs. When representing a Markov model graphically, each state is usually depicted as a "bubble", with arrows denoting the transition paths between states, as depicted in the figure below for a single component that has just two states: healthy and failed.
	\begin{figure}[H]
		\centering
		\includegraphics{img/engineering/markov_reliability_chain.jpg}	
		\caption{Markov Chain Reliability Model example}
	\end{figure}
	This system is then formalized in the following form (not obvious to guess the very first time):
	
	We can rewrite system in matrix form as well:
	
	With:
	
	A possible way to solve this system consists in a first time in diagonalizing (if possible) the matrix, that is to say to find an invertible matrix and a diagonal matrix such that:
	
	We thus fall back an application of the spectral theorem proved in the section of Linear Algebra and applied to the resolution of differential equations (it is sometimes terrifying and admirable the extent of the striking power of pure mathematics...).
	
	From here we want to be able to write:
	
	or also (it remains the same):
	
	By doing thechange of variable:
	
	the system to be solved becomes obviously:
	
 	Let us denote by $\gamma_2$,$\gamma_1$, $\gamma_0$ the diagonal elements of $D$ and $\vec{X}=(X_2,X_1,X_0)^T$. We then want to rewrite the previous system as:
	
	But the solutions of this system are easy to determine, they are given by:
	
	for $i=0,1,2$ where $c_i^{te}\in \mathbb{R}$.

	Finally, we will determine the solutions of the system of departure with the help of the equality:
	
	Well... now let us put all this into practice!

	To diagonalize $M$ (if it is possible) we must determine its eigenvalues and its eigenspaces. Calculations made (see chapter of Linear Algebra for the general method) we find that the values of $M$ are:
	
	The corresponding eigenspaces are given by the vectors (see chapter of Linear Algebra for the general method):
	
	Hence, if we write:
	
	We can then indeed write:
	
	The diagonal elements of $D$ are:
	
	The components of the vector $\vec{X}$ being given earlier above by:
	
	we therefore have:
 	
	To finish the solutions of the original system are for recall given by:
	
	So finally:
	
 	If we impose the initial condition:
	
	we then have the system of three equations with three unknowns:
	
 	Its resolution should be quite trivial for the reader (if it is not the case do not hesitate to contact us we will add the details) and we get:
	
	So at the end:
	
	We can also verify that for any $t$ we have:
	
 	This is the reason why we often find the above system in the following form:
	
 	From the above formulas it follows that the availability on the long-term system is given by:
	
 	Because $\lambda+\mu>0$ and because we also have:
	
	
	\pagebreak
	\subsubsection{Maximum Likelihood for failure rate determination of samples}
	We studied in the section Statistics and Theoretical Computing that the technics using the maximum likelihood allowed us to estimate the maximum or minimum value of the parameters of a statistical law under the assumption that events are independent.
	
	This technique, even empirical and sometimes quite questionable, is also used in the field of the reliability. As the theory has already been developed in the section Statistics let us see directly some practical cases.
	\begin{tcolorbox}[colframe=black,colback=white,sharp corners]
	\textbf{{\Large \ding{45}}Examples:}\\\\
	E1. Consider a system whose time intervals between failures is distributed according to an exponential law:
	
	of unknown parameter $\lambda$ and where we consider (as always ...) that every failure is independent.\\
	
	Observations gave us the following number of months between each failure: $10, 5, 11, 12, 15$.\\
	
	Then using maximum likelihood which idea is for refresh is to maximize (minimize) the total probability of independent events (therefore we must multiply the probabilities), we have:
	
	Therefore determine the failure rate that maximizes the likelihood is equivalent as we know to calculate when the derivative (\SeeChapter{see section Differential and Integral Calculus page \pageref{differential calculus}}) is zero:
	
	We then have two roots that are trivial:
	
	and of course we will keep the second as root failure rate of our system.
	\end{tcolorbox}
	
	\pagebreak
	\begin{tcolorbox}[colframe=black,colback=white,sharp corners]
	E2. We consider now see the same data as before but this time with a Weibull distribution with the following simplified notation:
	
	and therefore:
	
	To maximize this kind of function, as far as we know, we can use only numerical methods. By cons, a simple spreadsheet software like Microsoft Excel with its solver allows to find the two parameters of the Weibull distribution in three clicks!\\
	
	E3. We imaginer that we put under observation $10$ similar manufacturedelements. On the $10$, we observed a stop of $4$ of them after respectively $700, 800, 900$ and $950$ hours. The remaining $6$ items will be considered as being over the $1,000$ work hours and we stop observing from this value. Assuming that the failure distribution is exponentially again, then we have (the idea is subtle but simple):
	
	It comes then:
	
	We then have two roots that are trivial:
	
	and of course we will keep the second root as failure rate of our system.
	\end{tcolorbox}

	\pagebreak
	\subsubsection{Kaplan-Meier Survival Rate}
	In areas such as high-level industry, high-level medicine or high-level biology, we often interested in the:
	\begin{enumerate}
		\item Survival duration after a serious event
		\item Duration of remission after treatment or surgery
		\item Duration of symptoms after an abnormality
		\item Duration of a symptomless infection
	\end{enumerate}
	We seek very often to distinguish at least "the event of interest":
	\begin{enumerate}
		\item System shutdown after serious event
		\item End of remission 
		\item End of a symptom after anomaly
		\item Beginning of a symptom during an infection
	\end{enumerate}
	of the variable to explain "duration before the event of interest":
	\begin{enumerate}
		\item Elapsed duration before shutdown
		\item Elapsed duration before the end of remission
		\item Elapsed duration before the end of the symptom
		\item Elapsed duration without symptoms
	\end{enumerate}
	\textbf{Definitions (\#\mydef):}
	\begin{enumerate}
		\item[D1.] We name "\NewTerm{remission}\index{remission}", the reduction of a disease or a temporarily malfunction.
		
		\item[D2.] The "\NewTerm{survival time}\index{survival time}" or "\NewTerm{lifetime}\index{lifetime}" $T$ means the time which elapses from an initial time (start of treatment, diagnosis or failure, etc.) until the occurrence of a final event of interest (patient death, relapse, remission, cure, repair, etc.). We say that the object of the study "survives at time $t$" if at this moment the final interest event has not yet occurred.
	\end{enumerate}
	\begin{tcolorbox}[title=Remark,colframe=black,arc=10pt]
	Although this type of study can be associated with preventive maintenance (see page \pageref{preventive maintenance}), statisticians associate this type of study rather to the domain of "\NewTerm{survival analysis}\index{survival analysis}".
	\end{tcolorbox}
	
	We will focus in this book on a particular context but that can be easily generalized:
	\begin{itemize}
		\item Cohort/Clinical trial where we study the survival time of each patient (machine).
		
		\item All patients (machines) do not have the same observation time (different instants of entry into the study).
		
		\item We have information on the survival time of each patient (machine) but we do not know exactly when it happens.
	\end{itemize}
	From the last two points, we conclude that the survival time can be censored. So the usual statistical techniques does not apply directly as censored data require special treatment (of course, if we remove the censored data we lose information). It goes without saying that this situation is extremely common and therefore the study of the Kaplan-Meier estimate is very important (for other survival models the reader should refer to the Statistics section in part concerning Survival Models).
	
	\textbf{Definition (\#\mydef):} The duration $T$ is said to be censored if the duration is not observed completely. The different types of censoring are:
	\begin{itemize}
		\item Type I censoring: fixed right. In this situation, the time is not observable beyond a maximum, fixed, named "\NewTerm{fixed-censoring}\index{fixed-censoring}" and imposed. So either we have the opportunity to observe the real duration of the event of interest for the item if it occurs before the fixed-censorsing, or we limit ourselves to the fixed-censoring time if the vent of interest has not occurred before.
		
		\item Type II censoring: wait. In this situation, we observe the lives of $n$ individuals until $m\leq n$ individuals have seen the event to occur (deceased). The duration considered is therefore that of the beginning of the experiment until the event of interest for the $m$-th.
		
		\item Type III censoring: random left. In this situation, we do not know when the event of interest has occurred (because we started to observe the subject of study too late). We can not then deal with "durations" in the measurable sense and we have to limit ourselves to a simple count.
		
		\item Type III censoring: random right. In this situation, we do not know when the event will take place (because we stopped to observe the subject before it occurs  for any reason). We can not then also not deal with "times" in the measurable sense and we must limit ourselves to a simple count.
		
		\item Type IV censoring: random intervals. In this situation, we have a mixture of random left and right censoring. That is to say that for some study subjects, we do not know when the event of interest has begun, and for others we do not know when it will occur (if any. ...).
	\end{itemize}
	In the machinery industry, we often deal with the type I censoring: fixed right. In the medical field, in clinical trials, we often deal with a type III censoring: random right. In the case of pandemics, we are dealing with type III censoring: random left.
	
	To introduce this subject, rather than doing obscure theory, as always in this book we prefer at pragmatic approach. Suppose that the study is a clinical trial involving two groups of patients receiving two types of treatments. Two important questions raised to the physicians in a Phase II clinical trial (phase I is for non-toxicity approval to human and phase 0 for animals):
	\begin{enumerate}
		\item Is one of the two treatments more effective than the other in terms of improving patient survival?
		
		\item Can we highlight prognostic factors, that is to say that improve / deteriorate survival?
	\end{enumerate}
	
	To answer the first question we can develop statistical methods that will allow us to compare the two groups of patients who receive both types of treatment.
	
	To answer the second question we propose a model that links patient survival time for explanatory variables and highlight the prognostic factors.
	
	Let us as always assist the theory with an example. For this consider the following table where two cohorts of patients (we move from mechanical engineering to human engineering...) of same initial size with acute leukemia drug test (6-MP ) against a placebo (of course blindly).
	
	We have the following remission times for $21$ patients (the table of $21$ lines therefore indicates the number of weeks for where patient is considered cured after treatment before falling again ill):
	\begin{table}[H]\centering
	\begin{center}
		\definecolor{gris}{gray}{0.85}
			\begin{tabular}{|c|c|}
				\hline
				\multicolumn{1}{c}{\cellcolor{black!30}\textbf{6-mercaptopurine (6-MP)}} & 
  \multicolumn{1}{c}{\cellcolor{black!30}\textbf{Placebo}} \\ \hline
				 $6$ & $1$ \\ \hline
				 $6$ & $1$ \\ \hline
				 $6$ & $2$ \\ \hline
				 $6+$ & $2$ \\ \hline
				 $7$ & $3$ \\ \hline
				 $9+$ & $4$ \\ \hline
				 $10$ & $4$ \\ \hline
				 $10+$ & $5$ \\ \hline
				 $11+$ & $5$ \\ \hline
				 $13$ & $8$ \\ \hline
				 $16$ & $8$ \\ \hline
				 $17+$ & $8$ \\ \hline
				 $19+$ & $8$ \\ \hline
				 $20+$ & $11$ \\ \hline
				 $22$ & $11$ \\ \hline
				 $23$ & $12$ \\ \hline
				 $25$ & $12$ \\ \hline
				 $32+$ & $15$ \\ \hline
				 $32+$ & $17$ \\ \hline
				 $34+$ & $22$ \\ \hline
				 $35+$ & $23$ \\ \hline
		\end{tabular}
	\end{center}
	\caption[]{Two cohort survival analysis with right censoring}
	\end{table}
	The $+$ sign corresponds to patients who left the study for that week. They are therefore censored. For example, the fourth patient was lost of view for any reason after $6$ weeks of treatment with 6-MP: it has therefore has a duration of remission greater than $6$ weeks. So in the study 6-MP, there are $21$ patients and $12$ with censored data.
	
	\begin{tcolorbox}[title=Remark,colframe=black,arc=10pt]
	The theoretical model assumes that censorsing is independent of the survival time (not informative censoring). But if censoring is due to the discontinuation of treatment, the independence assumption is not valid anymore!
	\end{tcolorbox}
	For the placebo group it is simple to make a survival curve. It is sufficient to produce the following table (for the omitted weeks, obviously we impose the number of remission as constants):
	\begin{table}[H]\centering
	\begin{center}
		\definecolor{gris}{gray}{0.85}
			\begin{tabular}{|c|c|c|}
				\hline
				\multicolumn{1}{c}{\cellcolor{black!30}\textbf{Week $i$}} & 
  \multicolumn{1}{c}{\cellcolor{black!30}\textbf{Total \# remission at week $i$}} & 
  \multicolumn{1}{c}{\cellcolor{black!30}\textbf{Proportion (probability) of remission at week $i$}} \\ \hline
				 $0$ & $21$ & $100\%$ \\ \hline
				 $1$ & $19$ & $19/21=90\%$ \\ \hline
				 $2$ & $17$ & $17/21=81\%$ \\ \hline
				 $3$ & $16$ & $16/21=76\%$ \\ \hline
				 $4$ & $14$ & $14/21=67\%$ \\ \hline
				 $5$ & $12$ & $12/21=57\%$ \\ \hline
				 $8$ & $8$ & $8/21=38\%$ \\ \hline
				 $11$ & $6$ & $6/21=29\%$ \\ \hline
				 $12$ & $4$ & $4/21=19\%$ \\ \hline
				 $15$ & $3$ & $3/21=14\%$ \\ \hline
				 $17$ & $2$ & $2/21=10\%$ \\ \hline
				 $22$ & $1$ & $1/20=0.05\%$ \\ \hline
				 $23$ & $0$ & $0\%$ \\ \hline
		\end{tabular}
	\end{center}
	\caption[]{Two cohort survival analysis with right censoring}
	\end{table}
	So if the data are not censored, the survival $S(t)$ can be estimated by the proportion of individuals surviving at time $t$, that is customary to write under the following mathematical form:
	
	The idea is therefore to estimate:
	
	by the proportion of patients who survived until time $t$.
	
	If the data are censored, the estimated survival function requires specific tools. Kaplan and Meier have proposed in this particular case the following calculation:
	
	Let's see it in a slightly more mathematical form:
	
	With of course:
	
	If we denote by $X(1)\leq X(2)\leq ...\leq X(n)$ the moments (sorted) where an event occurred (death or censored), then we have:
	We estimate:
	
	where $d_k$ is the number of deaths (failures) observed in the time corresponding to the event $X(k)$ and $R_k$ is the number of individuals at risk (at risk of death/failure) just before $X(k)$.
	
	We define the Kaplan-Meier estimator for any $X(0)\leq t <X(k)$ by:
	
	Therefore we get doing now for 6-MP group (not the placebo group!!!!!) the following :
	\begin{table}[H]\centering
	\begin{center}
		\definecolor{gris}{gray}{0.85}
			\resizebox{\textwidth}{!}{\begin{tabular}{|c|c|c|c|}
				\hline
				\multicolumn{1}{c}{\cellcolor{black!30}\textbf{Remission duration}} & 
  \multicolumn{1}{c}{\cellcolor{black!30}\textbf{Items in remission}} & 
  \multicolumn{1}{c}{\cellcolor{black!30}\textbf{Probability of not relapse}} & 
  \multicolumn{1}{c}{\cellcolor{black!30}\textbf{Survival probability of}} \\ 
  				\multicolumn{1}{c}{\cellcolor{black!30}\textbf{(uncensored) observed}} & 
  \multicolumn{1}{c}{\cellcolor{black!30}\textbf{at the beginning of $k$}} & 
  \multicolumn{1}{c}{\cellcolor{black!30}\textbf{at $k$ knowing that we are at}} & 
  \multicolumn{1}{c}{\cellcolor{black!30}\textbf{Kaplan-Meier}} \\
   \multicolumn{1}{c}{\cellcolor{black!30}} & 
   \multicolumn{1}{c}{\cellcolor{black!30}} & 
  \multicolumn{1}{c}{\cellcolor{black!30}\textbf{at $k-1$ ($\hat{P}_k$)}} &  \multicolumn{1}{c}{\cellcolor{black!30}} \\ \hline
				 $0$ & $21$ & $21/21=100\%$ & $100\%$\\ \hline
				 $6$ & $21$ & $18/21=85.7\%$ & $100\%85.7\%=85.7\%$ \\ \hline
				 $7$ & $17$ & $16/17=94.1\%$ & $85.7\%94.1\%=80.7\%$\\ \hline
				 $10$ & $15$ & $14/15=93.3\%$ & $80.7\%93.3\%=75.3\%$\\ \hline
				 $13$ & $12$ & $11/12=91.7\%$ & $75.3\%91.7\%=69\%$\\ \hline
				 $16$ & $11$ & $10/11=90.9\%$ & $69\%90.9\%=62.7\%$\\ \hline
				 $22$ & $7$ & $6/7=85.7\%$ & $62.7\%85.7\%=53.8\%$\\ \hline
				 $23$ & $6$ & $5/6=83.3\%$ & $53.8\%83.3\%=44.8\%$ \\ \hline
		\end{tabular}}
	\end{center}
	\caption[]{6-MP survival analysis}
	\end{table}
	We thus find the same values as those given for example by Minitab Statistical Software 15.1.1 (for the details see the Minitab companion book).
	
	\pagebreak
	\subsubsection{ABC Method}
	In a company, the tasks are various and the maintenance teams are systematically reduced to a minimum. In addition, the most advanced technologies in maintenance can be very expensive, and should not be applied indiscriminately.
	
	It is therefore appropriate to be organized effectively and efficiently. The ABC analysis\footnote{Not to be confused with Activity-based costing (ABC) that is a costing methodology that identifies activities in an organization and assigns the cost of each activity with resources to all products and services according to the actual consumption by each.} (one should say in all rigor "ABC objective"), using implicitly use the cumulated Pareto law (\SeeChapter{see section Statistics page \pageref{pareto distribution}}), allows remedy this relatively well with an empirical approach. Thus, a classification of costs in relation to the types of failure gives priority interventions to carry on (this empirical method is also used in many other areas including one that is very well known: supply chain management).
	
	The idea is at first like Pareto analysis (\SeeChapter{see section Quantitative Management page \pageref{pareto analysis}} to classify faults in order of increasing maintenance costs (or cost impact in case of failure) every failure relating a simple or complex system and to establish a chart matching the percentage of cumulative cost  to the cumulative percentages type of failure.
	
	Then we distinguish traditionally in the industry three zones:
	\begin{enumerate}
		\item Zone A: In most cases, about $20\%$ of total failures represent $80\%$ of costs and so this is the priority area.
		
		\item Zone B: In this zone, representing $30\%$ of total failures we have the following $15\%$ additional costs.
		
		\item Zone C: In this zone $50\%$ of the remaining failures generates only $5\%$ of costs.
	\end{enumerate}
	The field of participatory web (web 2.0) considers areas (targets) of respectively $90-9-1$ percents so we found very everything and anything in this field as it is empirical...
	
	Let us see an example assuming that the following data were collected and we would like to do an analysis of the percentage of machines that should be that we focus on to reduce the cost off failures hours of approximately $80\%$ (in reality we will focus more on the percentage of the financial cost!).
	\begin{table}[H]\centering
	\begin{center}
		\definecolor{gris}{gray}{0.85}
			\begin{tabular}{|c|c|c|}
				\hline
				\multicolumn{1}{c}{\cellcolor{black!30}\textbf{Machine Number}} & 
  \multicolumn{1}{c}{\cellcolor{black!30}\textbf{Nonworking hours}} & 
  \multicolumn{1}{c}{\cellcolor{black!30}\textbf{Number of failures}} \\ \hline
				 $1$ & $100$ & $4$ \\ \hline
				 $2$ & $32$ & $15$ \\ \hline
				 $3$ & $50$ & $4$ \\ \hline
				 $4$ & $19$ & $14$ \\ \hline
				 $5$ & $4$ & $3$ \\ \hline
				 $6$ & $30$ & $8$ \\ \hline
				 $7$ & $40$ & $12$ \\ \hline
				 $8$ & $80$ & $2$ \\ \hline
				 $9$ & $55$ & $3$ \\ \hline
				 $10$ & $150$ & $5$ \\ \hline
				 $11$ & $160$ & $4$ \\ \hline
				 $12$ & $5$ & $3$ \\ \hline
				 $13$ & $10$ & $8$ \\ \hline
				 $14$ & $20$ & $8$ \\ \hline
		\end{tabular}
	\end{center}
	\caption[]{ABC analysis on machines failures}
	\end{table}
	Then, in a spreadsheet like Microsoft Excel or other ... we can easily establish the following table:
	\begin{table}[H]\centering
	\begin{center}
		\definecolor{gris}{gray}{0.85}
			\begin{tabular}{|c|c|c|c||c|c|c|}
				\hline
				\multicolumn{1}{c}{\cellcolor{black!30}\textbf{Machine}} & 
  \multicolumn{1}{c}{\cellcolor{black!30}\textbf{Stop [h.]}} & 
  \multicolumn{1}{c}{\cellcolor{black!30}\textbf{Sum [h.]}} & 
  \multicolumn{1}{c}{\cellcolor{black!30}\textbf{\% Cumulated}} & 
  \multicolumn{1}{c}{\cellcolor{black!30}\textbf{Failures}} & 
  \multicolumn{1}{c}{\cellcolor{black!30}\textbf{Sum}}& 
  \multicolumn{1}{c}{\cellcolor{black!30}\textbf{\% Cumulated}} \\ \hline
				 $11$ & $160$ & $160$  & $21.19\%$ & $4$ & $4$ & $4.30\%$\\ \hline
				$10$ & $150$ & $310$ & $41.06\%$ & $5$ & $9$ & $9.68\%$\\ \hline
				$1$ & $100$ & $410$ & $54.30\%$ & $4$ & $13$ & $13.98\%$\\ \hline
				$8$ & $80$ & $490$ & $64.90\%$ & $2$ & $15$ & $16.13\%$\\ \hline
				$9$ & $55$ & $545$ & $72.19\%$ & $3$ & $18$ & $19.35\%$\\ \hline
				$3$ & $50$ & $595$ & $78.81\%$ & $4$ & $22$ & $23.66\%$\\ \hline
				$7$ & $40$ & $635$ & $84.11\%$ & $12$ & $34$ & $36.56\%$\\ \hline
				$2$ & $32$ & $667$ & $88.34\%$ & $15$ & $49$ & $52.69\%$\\ \hline
				$6$ & $30$ & $697$ & $92.32\%$ & $8$ & $57$ & $61.29\%$\\ \hline
				$14$ & $20$ & $717$ & $94.97\%$ & $8$ & $65$ & $69.89\%$\\ \hline
				$4$ & $19$ & $736$ & $97.48\%$ & $14$ & $79$ & $84.95\%$\\ \hline
				$13$ & $10$ & $746$ & $98.81\%$ & $8$ & $87$ & $93.55\%$\\ \hline
				$12$ & $5$ & $751$ & $99.47\%$ & $3$ & $90$ & $96.77\%$\\ \hline
				$5$ & $4$ & $755$ & $100.00\%$ & $3$ & $93$ & $10.00\%$\\ \hline
		\end{tabular}
	\end{center}
	\caption[]{Data normalization for ABC analysis on machines failures}
	\end{table}
	We then have graphically:
	\begin{figure}[H]
		\begin{center}
		\includegraphics{img/engineering/abc.jpg}
		\end{center}	
		\caption{Graph of the ABC method}
	\end{figure}
	where the areas $A$, $B$ and $C$ are rounded to existing points. Thus, the critical area $A$ contains the machines $11$, $10$, $1$, $8$, $9$ and $3$. improving the reliability of these machines can therefore get up to $78.8\%$ time saving on downtime.
	
	Now let us determine the parameters of the Pareto distribution law (\SeeChapter{see section Statistics page \pageref{pareto distribution}}):
	
We have to determine $k$ and $x_m$ and the other parameters are given us by our measurements (the table).

	We can play in the following way:
	
	Therefore:
	
	So thanks to:
	
	we should be able to determine the two seeked parameters by considering the expression as the equation of a linear function whose $k$ is the slope and $k\log(x_m)$ the intercept:
	
	A simple linear regression (\SeeChapter{see section Theoretical Computing page \pageref{simple linear regression}}) gives us:
	
	Therefore:
	
	So we have finally:
	
	What then gives the following table (the $x_i$ of the Pareto distribution are the Cumulative\% failure):
	\begin{table}[H]\centering
	\begin{center}
		\definecolor{gris}{gray}{0.85}
			\begin{tabular}{|c|c|c|c|}
				\hline
				\multicolumn{1}{c}{\cellcolor{black!30}\textbf{Machine}} & 
  \multicolumn{1}{c}{\cellcolor{black!30}\textbf{\% Cumulated cost [h.]}} & 
  \multicolumn{1}{c}{\cellcolor{black!30}\textbf{\% Cumulated failures}} & 
  \multicolumn{1}{c}{\cellcolor{black!30}\textbf{\% Cumulated Pareto Law}} \\ \hline
				 $11$ & $21.19\%$ & $4.30\%$ & $61.76\%$\\ \hline
				$10$ & $41.06\%$ & $9.68\%$ & $78.01\%$\\ \hline
				$1$ & $54.30\%$ & $13.98\%$ & $82.89\%$\\ \hline
				$8$ & $64.90\%$ & $16.13\%$ & $84.48\%$\\ \hline
				$9$ & $72.19\%$ & $19.35\%$ & $86.29\%$\\ \hline
				$3$ & $78.81\%$ & $23.66\%$ & $88.05\%$\\ \hline
				$7$ & $84.11\%$ & $36.56\%$ & $91.12\%$\\ \hline
				$2$ & $88.34\%$ & $52.69\%$ & $93.08\%$\\ \hline
				$6$ & $92.32\%$ & $61.29\%$ & $93.75\%$\\ \hline
				$14$ & $94.97\%$ & $69.89\%$ & $94.29\%$\\ \hline
				$4$ & $97.48\%$ & $84.95\%$ & $95\%$\\ \hline
				$13$ & $98.81\%$ & $93.55\%$ & $95.32\%$\\ \hline
				$12$ & $99.47\%$ & $96.77\%$ & $95.43\%$\\ \hline
				$5$ & $100.00\%$ & $10.00\%$ & $95.43\%$\\ \hline
		\end{tabular}
	\end{center}
	\caption[]{Comparisons experimental/ theoretical data }
	\end{table}
	Then we can get the real corresponding Pareto curve easily in Microsoft Excel 11.8346:
	\begin{figure}[H]
		\begin{center}
		\includegraphics{img/engineering/abc_measurement_vs_theory.jpg}
		\end{center}	
		\caption{ABC method associated with a Pareto curve}
	\end{figure}
	The difference between the experimental curve and the theoretical is significant... As $k$ is less than $1$, then as we have seen proved it the section of Statistics, the Pareto law has neither expected mean nor variance.
	
	The practitioner must be careful to the fact that in the field of management, the Pareto law is used wrongly for a little bit everything when an another probability distribution could be much more appropriate.
	
	Furthermore, any GoF test (\SeeChapter{see section Statistics page \pageref{goodness of fit tests}}) shows us that we must reject the approximation by a Pareto Distribution. Moreover, specialized software such @RISK reject the Pareto approximation beyond the $20$ best adjustments, the best fit with this same software is the Log-Normal law.
	
	\pagebreak
	\subsection{Design of Experiments (DoE)}\label{doe}
	The behavior or the subjective assessment of industrial or manufactured products is generally a function of many phenomena, often dependent on each other. To predict the behavior / assessment, the products and phenomena are modeled and simulations are performed. The relevance of simulation results obviously depends on the quality of the models.
	
	In particular, as part of the design or redesign of a product, the models generally involve a number of physical quantities (parameters) which we are authorized or not to modify. The behavior of industrial products generally depends on many phenomena, often dependent on each other. To predict this behavior, product and phenomena are modeled and simulations are performed.
	
	However, these tests (simulations) are many times expensive, and this even more if the number of parameters to vary is important. Indeed, changing a parameter may for example require disassembly and reassembly of the product, a destructives analysis, or the manufacture of several different prototypes (case of a piece produced in series), or the interruption of production to change tool (for a manufacturing process) ... The cost of an experimental study depends on the number and order of the tests carried out.

	The idea then is to select and order tests in the purpose to identify, at lower costs, the effects of parameters on the response of the product. These are statistical methods using most often simple mathematical concepts. The implementation of these methods has three stages (but the practitioner should better refer to the ISO 3534-3: 1999 norm on the subject to be more rigorous!):
	\begin{enumerate}
		\item Postulate a behavior model of the system (with coefficients being unknown)
		
		\item Do data transformation to adjust the model to the assumptions of DoE tools and techniques (typically log-transformation for rates, etc.)
		
		\item Check that the assumptions of the DoE models are satisfied for your experiment after data transformation
		
		\item Define an experiment protocol, i.e. a series of tests ("\NewTerm{runs}\index{runs}") for identifying the model coefficients
		
		\item Do the tests, identify the coefficients ("\NewTerm{runs}") and influential variables
		
		\item Determine the values of the variables that allow you to get as closed as desired to the target result or with minimal variability and conclude.
	\end{enumerate}
	The experimental designs, or "\NewTerm{Design of Experiment (DoE)}\index{design of experiment}" (related to the field of "\NewTerm{diagnostic analytics}\index{diagnostic analytics}") allow to better organize the tests that accompany scientific research or industrial research. They are applicable to many disciplines and to all industries from the moment the practitioner is seeking for the relation between a variable of interest, $y$ (amount of scrap, defects detection, amplitude, vacations, satisfaction, etc.) and control or environmental variables $x_i$ in the purpose of:
	\begin{itemize}
		\item Determining which variables $x_i$ are most influential on the response $y$
		
		\item Determining where to set the influential variables $x_i$ so that $y$ is almost always near the desired nominal value
		
		\item Determining where to set the influential $x_i$ so that variability in $y$ is small
		
		\item Dtermining where to set the influential variables $x_i$ so that the effects of uncontrollable variables (denote $z_i$ most of times) are minimized
	\end{itemize}
	This is why they are softwares to treat them such as R, Minitab, QI Macros or JMP mainly to name the most known one.
	
	Let us also indicate that the experimental designs are a pillar of chemometrics (mathematical tools, especially statistics, to get the maximum information from chemical data) but are also used in supply chain optimization, marketing product design and and computer network dimensioning. But designed experiments were first used by agronomists during the 20th. This method seemed highly theoretical at first, and was initially restricted to agronomy. Genichi Taguchi made the designed experiment approach more accessible to practitioners in the manufacturing industry.

	In the last years of the 20th century, the application of design of experiments has developed, particularly given the acknowledged fact that they are essential to improving the quality of goods and services. Even if statistical quality control, managerial solutions, inspections, and other quality tools also perform this function, the design of experiment is the methodology of choice in the case of an environment with complex parameters, variables and interactions and when we need to quantify the facts and benefits. From a historical point of view, the design of experiment have been developed and evolved in the agriculture sector. Medicine/Pharma has also a long history of design of experiments developed with caution. Currently, industrial environments prove the significant benefits of this methodology, because of the ease of initiation of efforts (of user-friendly application software), better training, influential trainers, and many successes achieved by DoE. Furthermore, today the process of analysis and manipulations is automated by robots in most high level industries.
	
	There are three main types of recognized DoE actually:
	\begin{enumerate}
		\item "\NewTerm{Screenings designs}\index{Screenings designs}"  or "\NewTerm{factor screening experiments}\index{factor screening experiments}" whose purpose is to discover the most influential factors on a given response with a minimal number of experiences. This is the simplest family as considered close to the experimental intuition (it is sometimes considered a subfamily of the second family below because it disregards the interactions of factors and is thus reduced to a purely additive model).
		
		\item "\NewTerm{Modelization plans}\index{modelization plans}" whose purpose is to find the mathematical relations between the measured responses to the variables associated with factors through an analytical approach or purely matrix one. Complete factorial and fractional designs (2 levels by factors with linear models) and response surfaces designs (at least 3 levels in models with factors of the second degree) are part of this family.
		
		\item "\NewTerm{Mixture designs}\index{mixture designs}" whose purpose is the same as the second family, but where the factors are not independent and are forced by a dependence relation (e.g. the sum of their ratio must be typically equal to a given constant).
	\end{enumerate}
	The technique of DoE has for purpose to be scientifically more advantageous than techniques consisting in only change one factor at a time (thus still let fixed the other factors) because the latter does not allow a deep statistical study of errors and (especially!) hides completely the interactions.
	
	The general principle of the strategy of DoE is based on the fact that the study of a phenomenon can, usually, be summarized as follows: we are interested in a quantity $y$ that depends on many variables, $x_1,x_2,…,x_n$ (and their order has no influence ... which is a problem in some chemistry experimentation...).
	
	The mathematical modeling consists to find a function $f$ such that:
	
	that takes into account the influence of each single factor or combination of factors (interactions). 
	
	A conventional study method consists of measuring the response $y$ for several values of the variables $x_1$ o while leaving the other $n-1$ variables fixed. We then iterate this method for each variable.
	
	Therefore, for example, if we have $8$ variables and if we decides to give $2$ experimental values to each of them, we are led to make $2^8=256$ experiences.
	
	The use of an experimental design then gives a strategy in choosing the test methods. The success of experiments in research and industry is related to the need for business competitiveness: they allow quality improvement and cost reduction!
	
	\begin{tcolorbox}[title=Remark,colframe=black,arc=10pt]
	The methodofDoE was developed early in the century, in the 1920s, by Ronald A. Fisher (the same for man as for the Fisher Test!). It has grown considerably with the development of information technology and computing power that accompanies it.
	\end{tcolorbox}
	The treatment of results is made most of time using univariate or multivariate linear regression techniques (\SeeChapter{see section Numerical Method page \pageref{regression techniques}}) and that of analysis of variance (\SeeChapter{see section Statistics page \pageref{anova}}).
	
	With the DoE, the goal is therefore to obtain the maximum information (but not all information!) with minimum of experiences (and therefore a minimum of cost) in order to model and optimize the studied phenomena.
	
	An experimenter who starts a study is interested in a quantity that he measure at each test. This quantity is named obviously the "\NewTerm{response}" or "\NewTerm{exogenous variable}\index{exogenous variable}" and is the quantity of interest. The value of this quantity depends on several variables. Instead of the term "variable" or "endogenous variable" we will use the word "\NewTerm{factor}\index{factor}". The value given to a factor to achieve a test is named a "\NewTerm{level}\index{factor level}". When studying the influence of a factor, in general we limit is variations between two bounds (yes we must have to stop one day and be reasonable...) respectively named "\NewTerm{lower level}\index{factor lower level}" and "\NewTerm{upper level}\index{factor upper level}".
	
	Obviously, when we have several factors $x_1,x_2,…,x_n$ they represent a point in $\mathbb{R}^n$ named "\NewTerm{experimental space}\index{experimental space}".
	
	The set of all possible values that a factor can take between the lower level and the upper level is named the "\NewTerm{domain of variation of the factor}\index{domain of variation of the factor}" or simply the "\NewTerm{factor domain}\index{factor domain}". The usage is to denote the lower level by $-1$ and the upper level by $+1$ when all the factors have only two levels, otherwise is customary to note the levels from $1$ to $n$ (number of levels). Of course, depending on the chosen notation, the expressions of the mathematical models must be adapted accordingly.
	\begin{tcolorbox}[title=Remark,colframe=black,arc=10pt]
	If the factors don't have all the same number of level we say that we have an "irregular experimental region" and the most simple corresponding design is then a complete (general) factorial design.
	\end{tcolorbox}
	So for example for a factor having an domain of variation between an upper level of $20\; [^{\circ}\text{C}]$ corresponds to $+1$ and a lower level of $5\; [^{\circ}\text{C}]$ corresponding to $-1$ we will have at the end of our study to transform all the experimental values in "\NewTerm{centered reduced units}\index{centered reduced units}" in which the $x_i$ must be used.

	Thus, we have two entry points $(20,5)$ and two outputs $(+1, -1)$. Any intermediate value is simply given by the equation of the line:
	\begin{figure}[H]
		\begin{center}
		\includegraphics{img/engineering/doe_units_normalization.jpg}
		\end{center}	
		\caption{Principle of construction of centered reduced units}
	\end{figure}
	The slope is obvious to obtain, to get $b$, we just have a simple linear equation with one unknown to solve:
	
	or (what remains the same):
	
	Therefore the transition form the non-normalized variables, denoted by $x$, to the normalized one, denoted by $X$, is then written:
	
	Therefore in the case of an $(+1,-1)$ output:
	
	and vice-versa:
	
	that is to say in the case of an $(+1,-1)$ output:
	
	The set of domains of all factors defines the "\NewTerm{study domain}\index{study domain}". This study domain is the experimental space chosen by the  to make his tests. A study, that is to say multiple well defined experiments, is represented by points distributed in the study domain.

	For example, for two factors, the study domain is an experimental plane surface in $\mathbb{R}^2$:
	\begin{figure}[H]
		\begin{center}
		\includegraphics{img/engineering/doe_plane_study_domain.jpg}
		\end{center}	
		\caption{Two normalized factors study domain example}
	\end{figure}
	The levels of the reduced centered $x_i$  represents the coordinates of an experimental point and $y$ is the answer of this point. The graphical geometric representation of a DoE and the answer need always one dimension more than the dimension of the experimental space. A DoE with two factors need therefore a three dimension space to be represented: one dimension for the answer, two dimensions for the factors.

	When we have determined the mathematical model, to each point of the study domain, corresponds then a set of answers that are located on a surface (or hypersurface) named "\NewTerm{response surface}\index{response surface}" (this is why typically DoE are named in the general case: "DoE for response surfaces") that look like for example with two factors:
	\begin{figure}[H]
		\begin{center}
		\includegraphics{img/engineering/doe_response_surface.jpg}
		\end{center}	
		\caption{Two centered reduce factors quadratic surface response}
	\end{figure}
	The number and the position of the experimental points is one of the fundamental problems of DoE. Indeed, we are looking for the best precision on the response surface by minimizing at the same time the number of experimentations. The practitioner will therefore look for a mathematical function that links the response variable to the factors.
	
	For this, we simplify the problem first by remembering (\SeeChapter{see section Sequences and Series page \pageref{taylor series}}) that any continuous function, whatever is its number of variables, can be approximated in a sum of power series on a given point.

	We then take a limited Taylor development of a bivariate function (for example\label{bivariat taylor expansion doe}):
	
	Therefore at the neighborhood of $x_0=0,y_0=0$ (what we can do as we take the centered reduced variables... and this is also one of the reason we centered and reduced them!), we have the following second order Maclaurin series by change notation accordingly to the traditions in DoE for $i=1,2$:
	
	where $y$ is the response and the $x_i$ the factors and the $a_0,a_i,a_{ij},a_{ii}$ the coefficients of the a priori adopted mathematical model. They are not known and must be calculated from the results of experiments.
	
	The benefit (pros) of modeling the response by a polynomial is to calculate afterwards all the answers of the study area without having to do the experiments thanks to a model named "\NewTerm{postulated model}\index{postulated model}" or "\NewTerm{prior model}\index{prior model}".
	
	The disadvantage (cons) of a polynomial model it is that it is not bounded (diverge or: non-lipschitz function) and have therefore not asymptotic answer. In some situation NURBS (or splines in the case of a univariate function) could be more accurate but sadly we do not know actually how to make statistical inference with NURBS  to determine which factors are significant or not...
	
	Two complements need to be made to the model described above:
	\begin{enumerate}
		\item The first complement is the "\NewTerm{lack of fit}\index{lack of fit}". This expression reflects the fact that prior model is likely almost surely different (only by the approximation of the approach) of the real model that governs the studied phenomenon. There is a difference between these two models. This gap is the lack of fit!

		\item The second complement is to take into account the random nature of the response (without the latter to be stochastic otherwise we have to use other tools!). Indeed, if we measure a response several times in the same experimental point, we do not get exactly the same result. The results are scattered. The dispersions of this kind are named "\NewTerm{experimental error}\index{experimental error}" or "\NewTerm{residuals}\index{residuals}" and are supposed to follow a supposed to follow a Normal law.
	\end{enumerate}
	These two differences, lack of fit and experimental error, are most of time merged into only one error denoted $e$ (or sometimes as in linear regression by $\varepsilon$).
	
	The model used by the experimenter will then be written at the second order term and first degree:
	
	and is sometimes named "\NewTerm{synergic model}\index{synergic model}" and in the particular case of two variables we speak of a " \NewTerm{hyperbolic paraboloid model}\index{hyperbolic paraboloid model}" typically represented by:
	\begin{figure}[H]
		\centering
		\includegraphics[scale=1]{img/engineering/doe_hyperbolic_paraboloid.jpg}	
		\caption{Generic example of interaction for two factors (hyperbolic paraboloid)}
	\end{figure}
	and as resqueste by a reader, an example of univariate functions (to differentiate the linear a non-linear case when variables transformations could be necessary):
	\begin{figure}[H]
		\centering
		\includegraphics[scale=0.8]{img/engineering/generice_univariate_functions.jpg}	
	\end{figure}
	\begin{tcolorbox}[title=Remark,colframe=black,arc=10pt]
	If we stop the Taylor developement to the first order term and first degree (without interactions), we then speak of "\NewTerm{affine model}\index{affine model}".
	\end{tcolorbox}
	
	In practice we write the latter relation as following (we remove the error term for readiness purposes):
	
	where we have the abusive notations for what it is common to name in this field the "\NewTerm{rectangle term}\index{rectangle term}":
	
	Therefore it comes that for $2$ factors, the expression contains $4$ terms, for $3$ factors it contains $8$ terms, for $4$ factors it contains $16$ terms, and so on... There are always power of $2$!
	
	This model without error is many times named "\NewTerm{controlled (linear) model (or order $2$) with interactions}\index{controlled (linear) model with interactions}".
	
	\begin{tcolorbox}[colframe=black,colback=white,sharp corners]
	\textbf{{\Large \ding{45}}Example:}\\\\
	For example, a response surface associated with a relation like the previous one looks like with Maple 4.00b:\\
	
	\texttt{>plot3d(5+3*x1+2*x2+4*x1*x2,x1=-10..10,x2=-10..10,\\
view=[-10..10,-10..10,-10..10],contours=10,style=PATCHCONTOUR,\\
		axes=frame,numpoints=10000);
		}
	\begin{figure}[H]
		\begin{center}
		\includegraphics[scale=0.6]{img/engineering/doe_hyperbolic_paraboloid.jpg}
		\end{center}	
		\caption{Generic example of interaction for two factors (hyperbolic paraboloid)}
	\end{figure}
	where we have represented the isolines (\SeeChapter{see section Functional Analysis page \pageref{isoline}}). This permits the practitioner to search the combination of explanatory variables that for a same result is the less expensive!\\

	Obviously if we remove the interaction terms we simply get a plane:\\
	
	\texttt{>plot3d(5+3*x1+2*x2,x1=-10..10,x2=-10..10,\\
	view=[-10..10,-10..10,-10..10],contours=10,\\
	style=PATCHCONTOUR,axes=frame,numpoints=10000);
	}
	\begin{figure}[H]
		\begin{center}
		\includegraphics[scale=0.65]{img/engineering/doe_plane.jpg}
		\end{center}	
		\caption{Generic example of interaction for two factors (hyperbolic paraboloid)}
	\end{figure}
	\end{tcolorbox}
	Obviously, the higher the degree of the polynomial is, the higher the theoretic model will be close to reality. But (!!!) the high degree polynomials require a lot of data points and their validity can quickly diverge outside the experimental area (\SeeChapter{see section Calculus page \pageref{polynomial}}). If the study requires it, we prefer to use specific mathematical functions to better fit the model to the experimental results.
	
	However, in practice, the higher order interactions often have very little influence on the response (in fact this assumption depends strongly on the industrial/economical field ...!). It is therefore possible to not to include them in the model, which leads to less trials. This principle is used in the construction of many experimental designs, as we shall see in the next subsection. In many applications, we obtain results quite satisfactory results by limiting ourselves to double interactions.
	
	Why do we satisfy us of this approximate relation with four terms? For the simple reasons that:
	
	\begin{enumerate}
		\item The answer may be non-zero when all factors are zero (thanks to the first coefficient $a_0$).

		\item The answer depends trivially (intuitively) of the sum of the effects of the first and second factors $x_1,x_2$ in an independent way (coefficients $a_1,a_2$).

		\item The answer also depends on the interaction between the two factors $x_1,x_2$ (coefficient $a_{12}$).
	\end{enumerate}
	Each experimental point for which the $x_i$ are given then provides a value of the response $y$. This response is modeled by a polynomial whose coefficients are the unknowns an must be determined and also categorized as significant or not.
	
	\subsubsection{Two levels factorial Designs}
	So in an experimental design of $2$ factors with $2$ levels, we need at least (and at most for cost reasons!) $4$ measurements (trials) to have a system of $4$ equations with $4$ unknowns which are the coefficients $a_0,a_1,a_2,a_{12}$ and such that the system has a unique solution (\SeeChapter{see section Linear Algebra page \pageref{unique solution linear system}})
	
	\begin{tcolorbox}[title=Remark,colframe=black,arc=10pt]
	For a study of $2$ factors with $3$ levels, we can no longer take a linear model. Then we must take the quadratic terms of the Taylor expansion!
	\end{tcolorbox}	
	Since for each of the factors we need to set a low and a high level in order to work reasonably... so if we have two factors, we have an experimental space defined by $4$ points {(up, up), (down, down ) (up, down), (down, up)}, corresponding to the $2$ times $2$ levels ($2^2$), which is enough for us then to get our system of $4$ equations with $4$ unknowns and then determine the $4$ coefficients (uniquely).
	
	Thus, the points to be taken in our experience naturally correspond to the vertices (geometrically speaking) of the experimental space!
	
	We then have the following system of equations (recall that this approach works only for $2$ level factors, the case with $3$ levels will be study much further below with an example):
	
	or explicitly:
	
	As we are interested about the coefficients (and that the values of the variables are known!), the problem simply consist in solving a linear system (\SeeChapter{see section Linear Algebra page \pageref{linear systems}}).
	
	If the variables were not coded, we would have for example for a (two-level) complete factorial design with one variable (velocity of car) having for lower value $40 \;[\text{km}\cdot\text{h}^{-1}]$ and for upper value  $50 \;[\text{km}\cdot\text{h}^{-1}]$, and for a second variable (wheel pressure) having for lower value $1.5\;[\text{Pa}]$ and for upper value $3\;[\text{Pa}]$ the following system to solve knowing that we have drive respectively for each combination the distances $32700, 32680, 31710, 33222\;[\text{km}]$:
	
	Hence in matrix form:
	
	By solving this system by hand (\SeeChapter{see section Linear Algebra page \pageref{linear systems}}), or with a calculation software (spreadsheet or statistical software), we get:
	\begin{figure}[H]
		\begin{center}
		\includegraphics{img/engineering/doe_full_factorial_excel_values.jpg}
		\end{center}	
		\caption[]{Putting in equation and implicit resolution of the FF DoE in Microsoft Excel 14.0.6123}
	\end{figure}
	thus explicitly:
	\begin{figure}[H]
		\begin{center}
		\includegraphics[scale=0.7]{img/engineering/doe_full_factorial_excel_formulas.jpg}
		\end{center}	
		\caption[]{Putting in equation and explicit resolution of the FF DoE in Microsoft Excel 14.0.6123}
	\end{figure}
	Therefore the solution is finally:
	
	thus exactly the same coefficients as those given by a specialized software like Minitab 15.1.1 (see the companion Minitab book).
	
	Going back to our system with the coded variables, it then comes when solving the system algebraically (relations valid if and only if the variables are coded!):
	
	which can be written in the following matrix form:
	
	What is written in a more general way for linear models of the second order under the following form (\SeeChapter{see section Linear Algebra page \pageref{linear systems}}):
	
	The matrix $X$ containing $2^n$ rows is named "\NewTerm{full factorial design (FFDoE) with $2^n$ interactions}\index{full factorial design with $2^n$ interactions}" (the term "\NewTerm{factorial}" coming from the fact that all factors vary simultaneously).
	
	The matrix $X$ in practice is named the "\NewTerm{experimental matrix}\index{experimental matrix}" or "\NewTerm{effect matrix}\index{effect matrix}" and is often represented as follows in the previous case:
	\begin{table}[H]\centering
	\begin{center}
		\definecolor{gris}{gray}{0.85}
			\begin{tabular}{|c|c|c|c|c|c|}
				\hline
				\multicolumn{1}{c}{\cellcolor{black!30}\textbf{Trial N${}^\circ$}} & 
  \multicolumn{1}{c}{\cellcolor{black!30}\textbf{Rest}} & 
  \multicolumn{1}{c}{\cellcolor{black!30}\textbf{Factor 1}} & 
  \multicolumn{1}{c}{\cellcolor{black!30}\textbf{Factor 2}} & 
  \multicolumn{1}{c}{\cellcolor{black!30}\textbf{Factor 12}} & 
  \multicolumn{1}{c}{\cellcolor{black!30}\textbf{Answer}} \\ \hline
				 $1$ & $+1$ & $-1$ & $-1$ & $+1$ & $y_1$\\ \hline
				 $2$ & $+1$ & $+1$ & $-1$ & $-1$ & $y_2$\\ \hline
				 $3$ & $+1$ & $-1$ & $+1$ & $-1$ & $y_3$\\ \hline
				 $4$ & $+1$ & $+1$ & $+1$ & $+1$ & $y_4$\\ \hline
		\end{tabular}
	\end{center}
	\caption{Experimental Full Factorial ($2$ factor) matrix}
	\end{table}
	But we see immediately that in practice, the second column ("Rest") is unnecessary because it is always $+1$ and it is also only implicitly implemented in statistical softwares.
	
	It is the same for the fifth column (Factor 12), because it is automatically deductible from the third and fourth columns (it is equal to the multiplication of row elements... what some practitioners name the "\NewTerm{Box multiplication}\index{Box multiplication}").
	
	The reader will also notice that by passing from one column to another or from one row to another, there are always two factors that have the level that change! By cons, if we focus only on the columns Factor 1 and Factor 2, we see that from one row to another, there is one factor that changes at a time.
	\begin{tcolorbox}[title=Remark,colframe=black,arc=10pt]
	Once again notice that the first column has only $+1$ and the there is also always at least on row with only $+1$ values!
	\end{tcolorbox}
	Thus, in practice (softwares) and in many books, we represent rightly only the following reduce form table (which hides the fact that we are dealing in reality with a square matrix):
	\begin{table}[H]\centering
	\begin{center}
		\definecolor{gris}{gray}{0.85}
			\begin{tabular}{|c|c|c|c|}
				\hline
				\multicolumn{1}{c}{\cellcolor{black!30}\textbf{Trial N${}^\circ$}} & 
  \multicolumn{1}{c}{\cellcolor{black!30}\textbf{Factor 1}} & 
  \multicolumn{1}{c}{\cellcolor{black!30}\textbf{Factor 2}} & 
  \multicolumn{1}{c}{\cellcolor{black!30}\textbf{Answer}} \\ \hline
				 $1$ &  $-1$ & $-1$ &  $y_1$\\ \hline
				 $2$ &  $+1$ & $-1$ &  $y_2$\\ \hline
				 $3$ &  $-1$ & $+1$ &  $y_3$\\ \hline
				 $4$ &  $+1$ & $+1$ &  $y_4$\\ \hline
		\end{tabular}
	\end{center}
	\caption{Experimental Full Factorial ($2$ factor) reduced matrix}
	\end{table}
	for a $2$ factors experimental design with $2$ levels with interactions in linear model (no errors) we have even more extreme ("\NewTerm{Yates notation}\index{Yates notation}") ... in terms of writing:
	\begin{table}[H]\centering
	\begin{center}
		\definecolor{gris}{gray}{0.85}
			\begin{tabular}{|c|c|c|c|}
				\hline
				\multicolumn{1}{c}{\cellcolor{black!30}\textbf{Trial N${}^\circ$}} & 
  \multicolumn{1}{c}{\cellcolor{black!30}\textbf{Factor 1}} & 
  \multicolumn{1}{c}{\cellcolor{black!30}\textbf{Factor 2}} & 
  \multicolumn{1}{c}{\cellcolor{black!30}\textbf{Answer}} \\ \hline
				 $1$ &  $-$ & $-$ &  $y_1$\\ \hline
				 $2$ &  $+$ & $-$ &  $y_2$\\ \hline
				 $3$ &  $-$ & $+$ &  $y_3$\\ \hline
				 $4$ &  $+$ & $+$ &  $y_4$\\ \hline
		\end{tabular}
	\end{center}
	\caption{Experimental Full Factorial ($2$ factor) reduced matrix Yates notation}
	\end{table}
	We see better in this form of writing that in addition to the fact that the two columns Factor $1$ and Factor $2$ are orthogonal (the norm ISO 3534-3: 1999 speaks of "\NewTerm{orthogonal contrast}\index{orthogonal contrast}"), they are also "\NewTerm{balanced}\index{balanced design}" in the sense that there are so many $+$ and $-$ in each column.
	\begin{tcolorbox}[title=Remark,colframe=black,arc=10pt]
	By default, most software randomize the order of testing of the design (whether it is complete, fractional, factorial or not). Generally, it is recommended to randomize the order of testing to mitigate the effects of factors that are not included in the study and that parasitize it ("\NewTerm{nuisance factor}\index{nuisance factor}"), including effects related to the time. This is, we don't know that the factor exists and it may even be changing levels while we are conducting the experiment. However, in some cases, randomization does not produce a sequence of interesting essays and perhaps even dangerous because it can mask certain sources of systematic errors which were not identified before the experiment. For example, in industrial applications, modifying factor levels may be difficult or expensive. It is also possible that after the change done, the return to a steady state of the system takes a lot of time. In such cases, it may be desirable not to randomize the design to minimize the level changes. Also they are some cases where the nuisance factor is know but uncontrollable, but then we can compensate it by using the analysis of covariance (\SeeChapter{see section Statistics}).
	\end{tcolorbox}
	Practitioners appreciate to calculate the average of the answers and the average effect of a given factor since the system is linear. Thus, in at the level $+1$ for the factor $x_1$, by remaining on the linear system:
	
	then we can build and define the "average answers" given by:
	
	and we have the same with the level $-1$ for the same factor:
	
	We then the "\NewTerm{overall effect}\index{overall effect}" of the factor $x_1$ that will be given by:
	
	and the "\NewTerm{mean effect}\index{mean effect}" of the factor $x_1$ that is therefore define by the semi-difference between the average of the answers at the upper level of the factor $x_1$ and the average answers at the lower level of the same factor:
	
	but after some simplification of elementary algebra, it comes quickly that:
	
	It is obvious that if the overall effect (and verbatim the mean effect) is not zero, we can doubt that there is an interaction between the factors as the variation in amplitude of the response is not the same depending on the value of the level of the other factor (see the study of the analysis of variance of two factors in the Statistics section for more details). Obviously, in practice, the study of interactions will be made, as often for ANOVA, also with graphics (interaction diagrams).
	
	The average of all responses thus gives the value of the response at the center of the experimental domain:
	
	By doing some elementary algebra, that the plane is written in normalized and centered form or not, this expression reduces to:
	
	Obviously, we can summarize this simple case graphically if this can help the reader:
	\begin{figure}[H]
		\begin{center}
		\includegraphics{img/engineering/doe_mean_effect.jpg}
		\end{center}	
		\caption[]{Two factors with the response surface}
	\end{figure}
	and with the figure above, the changes being not very parallel when a factor is set, we can assume that there is interaction between the factors and thus this will require the use of an ANOVA to have a more in-deep numerical analysis.
	
	When the number of factors is large, it is not always easy for everyone to put the $+,-$ without the help of a software. So there is a small procedure named "\NewTerm{Yates algorithm}\index{Yates algorithm}" or "\NewTerm{Yates and Hunter algorithm}\index{Yates and Hunter algorithm}" which enables to reach very fast the desired result for factorial designs (whose factors have two levels) whose number of factors is a power of $2$: first, we start all columns by $-1$ and after we alternate the $-1$ and $+1$ all the $2^{j-1}$ rows for the $j$th row.
	\begin{tcolorbox}[title=Remarks,colframe=black,arc=10pt]
	\textbf{R1.} If the type of table above contains encoded values, we speak of "\NewTerm{experimental design}\index{experimental design}" if not with the usual physical units we speak of "\NewTerm{experimental table}\index{experimental table}".\\
	
	\textbf{R2.} In the case of coded tables, it is customary to indicate under the table a second table with the correspondences between the coded units and physical units.
	\end{tcolorbox}
	Let us insist on one important thing: If we had $3$ factors with $2$ levels each, then we have $2^3$ possible experiences opportunities (therefore $8$). But the number $8$ matches exactly with the number of coefficients that we also have in the linear model with interactions of a response with three variables:
	
	which also corresponds also only to the linear terms and condensed form of the Maclaurin series expansion of a function $f$ of three variables for recall.

	And so on ... for $n$ factors with two levels each. This is why the linear full factorial designs $2^n$ are traditionally the most used because they are mathematically intuitive and simple to prove and develop as many results simplifies at the opposite of the general factorial designs!

	\pagebreak
	\paragraph{Replicated full factorial designs}\mbox{}\\\\
	Replicated full factorial designs are important as the give (among others that we will see during a companion example for the study of general factorial designs\footnote{especially the calculation of an ANOVA}) is the ability to calculate the confidence intervals of the effects.
	
	As a companion example let us consider the $2^3$ factorial design below replicated twice:
	\begin{table}[H]\centering
	\begin{center}
		\definecolor{gris}{gray}{0.85}
			\begin{tabular}{|c|c|c|c|c|c|c|}
				\hline
				\multicolumn{1}{c}{\cellcolor{black!30}\textbf{Trial N${}^\circ$}} & 
  \multicolumn{1}{c}{\cellcolor{black!30}$x_1$} & 
  \multicolumn{1}{c}{\cellcolor{black!30}$x_2$} & 
  \multicolumn{1}{c}{\cellcolor{black!30}$x_3$} & 
  \multicolumn{1}{c}{\cellcolor{black!30}$y_{i1}$} & 
  \multicolumn{1}{c}{\cellcolor{black!30}$y_{i2}$} & 
  \multicolumn{1}{c}{\cellcolor{black!30}$\bar{y}_i$}\\ \hline
				 $1$ & $-1$ & $-1$ & $-1$ & $59$ & $61$ & $60$\\ \hline
				 $2$ & $+1$ & $-1$ & $-1$ & $74$ & $70$ & $72$\\ \hline
				 $3$ & $-1$ & $+1$ & $-1$ & $50$ & $58$ & $54$\\ \hline
				 $4$ & $+1$ & $+1$ & $-1$ & $69$ & $67$ & $68$\\ \hline
				 $5$ & $-1$ & $-1$ & $+1$ & $50$ & $54$ & $52$\\ \hline
				 $6$ & $+1$ & $-1$ & $+1$ & $81$ & $85$ & $83$\\ \hline
				 $7$ & $-1$ & $+1$ & $+1$ & $46$ & $44$ & $45$\\ \hline
				 $8$ & $+1$ & $+1$ & $+1$ & $79$ & $81$ & $80$\\ \hline
		\end{tabular}
	\end{center}
	\end{table}
	Or in more complete form:
	\begin{table}[H]\centering
	\begin{center}
		\definecolor{gris}{gray}{0.85}
			\begin{tabular}{|c|c|c|c|c|c|c|c|c|c|}
				\hline
				\multicolumn{1}{c}{\cellcolor{black!30}\textbf{Trial N${}^\circ$}} & 
  \multicolumn{1}{c}{\cellcolor{black!30}\texttt{I}} &
  \multicolumn{1}{c}{\cellcolor{black!30}$x_1$} & 
  \multicolumn{1}{c}{\cellcolor{black!30}$x_2$} & 
  \multicolumn{1}{c}{\cellcolor{black!30}$x_3$} &
  \multicolumn{1}{c}{\cellcolor{black!30}$x_{12}$} & 
  \multicolumn{1}{c}{\cellcolor{black!30}$x_{13}$} & 
  \multicolumn{1}{c}{\cellcolor{black!30}$x_{23}$} & 
  \multicolumn{1}{c}{\cellcolor{black!30}$x_{123}$} &  
  \multicolumn{1}{c}{\cellcolor{black!30}$\bar{y}_i$}\\ \hline
				 $1$ & $+$ & $-$ & $-$ & $-$ & $+$ & $+$ & $+$ & $-$ & $60$\\ \hline
				 $2$ & $+$ & $+$ & $-$ & $-$ & $-$ & $-$ & $+$ & $+$ & $72$\\ \hline
				 $3$ & $+$ & $-$ & $+$ & $-$ & $-$ & $+$ & $-$ & $+$ & $54$\\ \hline
				 $4$ & $+$ & $+$ & $+$ & $-$ & $+$ & $-$ & $-$ & $-$ & $68$\\ \hline
				 $5$ & $+$ & $-$ & $-$ & $+$ & $+$ & $-$ & $-$ & $+$ & $52$\\ \hline
				 $6$ & $+$ & $+$ & $-$ & $+$ & $-$ & $+$ & $-$ & $-$ & $83$\\ \hline
				 $7$ & $+$ & $-$ & $+$ & $+$ & $-$ & $-$ & $+$ & $-$ & $45$\\ \hline
				 $8$ & $+$ & $+$ & $+$ & $+$ & $+$ & $+$ & $+$ & $+$ & $80$\\ \hline
		\end{tabular}
	\end{center}
	\end{table}	
	We will assume to follower underlying model:
	
	The global average is:
	
	The average of all responses at high level of $x_1$ is equal to:
	
	the average of all responses at bottom level of $x_1$ is equal to:
	
	So the overall effect of the $x_1$ is equal to:
	
	Or equivalently using in one shot only all the $y_{i1}$ and $y_{i2}$:
	
	or using only the averages $\bar{y}_i$:
	
	And so on:
	
	Finally we get:
	
	Now the question that interest us here are:
	\begin{itemize}
		\item Which of these effects are important?
		\item Which of these effects are distinguishable from the noise in the experimental environment?
	\end{itemize}
	Let's say we run experiment numerous times. Sample effects, $E_{Gx_i}$ will be supposed to be Normally distributed. We can obviously do a Normal probability plot (Henry plot) of the measurements but here we will focus rather on the calculation of a confidence interval as it is more easy to communicate in corporations.

	So our experiment is replicated, we calculate a sample variance for each run:
	
	Then we get the following table:
	\begin{table}[H]\centering
	\begin{center}
		\definecolor{gris}{gray}{0.85}
			\begin{tabular}{|c|c|c|c|c|}
				\hline
				\multicolumn{1}{c}{\cellcolor{black!30}\textbf{Trial N${}^\circ$}} & 
  \multicolumn{1}{c}{\cellcolor{black!30}$y_{i1}$} & 
  \multicolumn{1}{c}{\cellcolor{black!30}$y_{i2}$} & 
  \multicolumn{1}{c}{\cellcolor{black!30}$\bar{y}_i$} & 
  \multicolumn{1}{c}{\cellcolor{black!30}$\hat{\sigma}_{y_i}^2$} \\ \hline
				 $1$ & $59$ & $61$ & $60$ & $2$\\ \hline
				 $2$ & $74$ & $70$ & $72$ & $8$\\ \hline
				 $3$ & $50$ & $58$ & $54$ & $32$\\ \hline
				 $4$ & $69$ & $67$ & $68$ & $2$\\ \hline
				 $5$ & $50$ & $54$ & $52$ & $8$\\ \hline
				 $6$ & $81$ & $85$ & $83$ & $8$\\ \hline
				 $7$ & $46$ & $44$ & $45$ & $2$\\ \hline
				 $9$ & $79$ & $81$ & $80$ & $2$\\ \hline
		\end{tabular}
	\end{center}
	\end{table}	
	We now assume that the standard deviation is constant for all runs (very strong assumption!) and given by the pooled sample variance:
	
	So obviously we can generalize:
	
	The reader must keep in mind that the null hypothesis we would like to reject is:
	
	and under that null hypothesis they would follow a Normal distribution (so if we reject that one we will be quite happy as the effects are significantly non-zero).
	
	We may, based on each calculated effect, develop a confidence interval for the true mean effect:
	
	and we know (\SeeChapter{see section Statistics page \pageref{standard error}}) that this is written explicitly:
	
	Therefore in our example we have for the average:
	
	Numerically:
	
	That is to say:
	
	shortly:
	
	Since $0$ is not in the confidence interval, we are happy to reject $H_0$ that $\mu_\text{Average}=0$.
	
	Now let us calculate the confidence interval for the other effects. We know that we calculate effects generally by:
	
	Each bases on $N/2$ observations ($N/2=8$ for our example). We have obviously:
	
	with obviously:
	
	where $r$ is the number of replications. Therefore:
	
	Bus as we assume for all runs the standard deviation:
	
	Therefore:
	
	Hence:
	
	So finally:
	
	sometimes named the "\NewTerm{}\index{standard error of an effect}".
	
	For our example:
	
	Therefore:
	
	Now we get easily for each effect the following confidence interval:
	
	So in our example:
	
	Numerically:
	
	Hence:
	
	So finally
	
	So $0$ does not line on the confidence interval for $E_{Gx_1}$, $E_{Gx_2}$ and $E_{Gx_{13}}$. We now know that only $a_0$, $a_1$, $a_2$ and $a_{13}$ are not noise and therefore the fitted model becomes:
	
	That is to say:
	
	For comparison we can see the special way on how Minitab 17.1.3 show these same results:
	\begin{figure}[H]
		\begin{center}
		\includegraphics[scale=1]{img/engineering/replicated_design_minitab_analysis.jpg}
		\end{center}	
		\caption[]{Replicated factorial design effects C.I. with Minitab 17.1.3}
	\end{figure}
	\begin{tcolorbox}[title=Remark,colframe=black,arc=10pt]
	What Minitab denote by the $T$-value is the ratio:
	
	That is to say the value that should take the $T$ distribution for not rejecting the null hypothesis.\\
	
	So for example for the factor $A$ we get:
	
	and the column $p$-value is as always (\SeeChapter{see section Statistics page \pageref{p value}}) the corresponding percentile of the $T$-value.
	\end{tcolorbox}
	And finally for those who are interested to see the Henry plot of the effect:
	\begin{figure}[H]
		\begin{center}
		\includegraphics[scale=1]{img/engineering/replicated_design_henry_plot_minitab.jpg}
		\end{center}	
		\caption{Henry plot of the replicated factorial design with Minitab 17.1.3}
	\end{figure}
	
	\pagebreak
	\paragraph{Plackett-Burman Designs}\mbox{}\\\\
	It is important to notice that all these full linear designs approximated to the second order are in matrix form orthogonal square matrices $M_{n}$ and therefore of course can be inverted (\SeeChapter{see section Linear Algebra page \pageref{orthogonal matrix}})!
	
	Also the previous matrices do not satisfy the following relation as seen in the section of Linear Algebra (scalar product of basis-columns with themselves):
	
	where for recall $\mathds{1}$ is a matrix with only $1$ in the diagonal and $0$ everywhere else, but have the particularity for all full experimental design to satisfy the relation:
	
	So unlike orthogonal matrices, which by definition have all the columns (or rows) that form an orthonormal base (unit norm), the experimental designs matrices have the particularity of not having the norm orthogonal basis vectors equal to the unit.
	
	\textbf{Definitions (\#\mydef):}
	Thus, we define the matrix whose components are all $+1$ or $-1$ and satisfying the previous relation:
		
	as a "\NewTerm{Hadamard matrix}\index{Hadamard matrix}" denoted sometimes $H$. 
	
	Put in another way, a $(+1,-1)$-matrix is Hadamard if the inner
product of two distinct rows is 0 and the inner product of a
row with itself is $n$.

	A few examples of Hadamard matrices are:
	
	It is quite apparent that if the rows and columns of an Hadamard matrix are permuted, the matrix remains Hadamard. It is also true that if any row or column is multiplied by $-1$, the Hadamard property is retained.

	\begin{theorem} 
	Hadamard matrices have the property to exist only for orders $1$, $2$, $4$, $8$, $12$, $16$, $20$, $24$, ... That is to say only for the multiple for the orders that are multiple of $4$ or written differently:
	
	 (if we omit the case $1$ and $2$ that are obvious).
	\end{theorem}
	\begin{dem}
	Knowing that the case of $1$ and $2$ are trivial and that the odd case must immediately intuitively be removed (try and you will see almost very quickly...), we will do the proof for $n\geq 4$.
	
	Since all the columns are mandatory orthogonal (so that the matrix is invertible and therefore the system uniquely solvable) and from the fact that the rows and columns of an Hadamard matrix are permuted or that if any row or column is multiplied by $-1$ the Hadamard property is retained, it is therefore always possible to arrange to have the first row and first column of an Hadamard matrix contain only $+1$ entries. An Hadamard matrix in this form is said to be "normalized":
	
	The reader must obviously consider that in the above system we can consider two case in facts:
	\begin{enumerate}
		\item It is a system that represents indeed an order $4$ matrix, so $x$, $y$, $z$ and $w$ have to be replaced in its mind by the value $1$

		\item It is a system that can represents only the $4$ first rows of a matrix of order $n$ greater $4$ (as all the other rows are equal to zero base on the same principle we have the habit to not represent them), so $x$ must therefore be in the reader mind replace by a given number of $1$ ($x=1\ldots 1$), same for $y$, $z$ and $w$. More explicitly (for example for $n=8$):
		
		as we can always normalize a Hadamard matrix to get such as system and that the Hadamard properties still remain valids.
	\end{enumerate}
	Solving the system gives:
	
	therefore $n$ must be divisible by $4$ for $n \geq 4$ if we want the number of terms in $x$, $y$, $z$, $w$ to be an integer!
	
	The result can seems not obvious but if the reader has the time... he can try for example to construct a Hadamard matrix of order $n=6$ and he will quickly see why it is not possible.
	\begin{flushright}
		$\square$  Q.E.D.
	\end{flushright}
	\end{dem}
	It then follows trivially the following relation (with a very abusive notation because it omits the notation of the unit matrix):
	
	However, we will see a little further below in our study of fractional factorial designs that Hadamard matrices of order $1$, $4$, $8$, $16$, $32$, ..., $2^n$ are almost never used under the expression "Hadamard matrices" as they are confused with the fractional factorial designs.
	
	By cons, it is interesting to observe that we have the potential existing design of experiments with $12$, $20$, $24$, $28$, ... trials (in other words:  the number of runs is a multiple of $4$ but not a power of $2$) that have some interest when the number of factors is greater than or equal to $4$ (some software like JMP - in 2012 - however don't propose these designs if the number of factors is less than $5$). These designs are named "\NewTerm{Plackett-Burman Design of Experiments}\index{Plackett-Burman design}" and sometimes "\NewTerm{irregular Plackett-Burman Design of Experiments}\index{irregular Plackett-Burman design}" or more rarely "\NewTerm{non-geometric Plackett-Burman Design of Experiments}\index{non-geometric Plackett-Burman Design}" (a software like Minitab propose however arbitrarily - and indicate it explicitly in the shoftware Help - a Plackett-Burman design of order $32$ and Plackett-Burman designs for $2$ or $3$ factors).

	 Robin L. Plackett and J. P. Burman have try with algorithms and by trial and error the expression of matrices corresponding  experimental designs that carry their name and which therefore contain $12$, $20$, $24$, $28$, ... trials. They have proposed a very useful tip so that practitioners can create these experimental designs without software. Let us do to start an example  with a Hadamard matrix of order $12$ (without focus for the moment on the number of factors that we use). The Plackett-Burman tables for the special $12$ order give only the first row (in fact... it is a column in the matrix):
	 
		Then we build the following table of experiment by following the algorithm proposed by Plackett and Burman (software use the same as far as we know):
	\begin{enumerate}
		\item First step, we put the first line as a column in a $12\times 12$ table:
		 
		\begin{table}[H]
		\begin{center}
			\begin{tabular}{|c|c|c|c|c|c|c|c|c|c|c|c|}
			\hline 
			+ & {} & {} & {} & {} & {} & {} & {} & {} & {} & {} & {} \\ 
			\hline 
			+ & {} & {} & {} & {} & {} & {} & {} & {} & {} & {} & {} \\ 
			\hline 
			- & {} & {} & {} & {} & {} & {} & {} & {} & {} & {} & {} \\ 
			\hline 
			+ & {} & {} & {} & {} & {} & {} & {} & {} & {} & {} & {} \\ 
			\hline 
			+ & {} & {} & {} & {} & {} & {} & {} & {} & {} & {} & {} \\ 
			\hline 
			+ & {} & {} & {} & {} & {} & {} & {} & {} & {} & {} & {} \\ 
			\hline 
			- & {} & {} & {} & {} & {} & {} & {} & {} & {} & {} & {} \\ 
			\hline 
			- & {} & {} & {} & {} & {} & {} & {} & {} & {} & {} & {} \\ 
			\hline 
			- & {} & {} & {} & {} & {} & {} & {} & {} & {} & {} & {} \\ 
			\hline 
			+ & {} & {} & {} & {} & {} & {} & {} & {} & {} & {} & {} \\ 
			\hline 
			- & {} & {} & {} & {} & {} & {} & {} & {} & {} & {} & {} \\ 
			\hline 
			{} & {} & {} & {} & {} & {} & {} & {} & {} & {} & {} & {} \\ 
			\hline 
			\end{tabular}
		\end{center}
		\end{table}	
		
		\item Second step: we deduce the second column from the first column by shifting the signs a step to the bottom, the last "$-$" sign being ascended to the top of the second column (it is therefore equivalent to a circular permutation):
		\begin{table}[H]
		\begin{center}
			\begin{tabular}{|c|c|c|c|c|c|c|c|c|c|c|c|}
			\hline 
			+ & - & {} & {} & {} & {} & {} & {} & {} & {} & {} & {} \\ 
			\hline 
			+ & + & {} & {} & {} & {} & {} & {} & {} & {} & {} & {} \\ 
			\hline 
			- & + & {} & {} & {} & {} & {} & {} & {} & {} & {} & {} \\ 
			\hline 
			+ & - & {} & {} & {} & {} & {} & {} & {} & {} & {} & {} \\ 
			\hline 
			+ & + & {} & {} & {} & {} & {} & {} & {} & {} & {} & {} \\ 
			\hline 
			+ & + & {} & {} & {} & {} & {} & {} & {} & {} & {} & {} \\ 
			\hline 
			- & + & {} & {} & {} & {} & {} & {} & {} & {} & {} & {} \\ 
			\hline 
			- & - & {} & {} & {} & {} & {} & {} & {} & {} & {} & {} \\ 
			\hline 
			- & - & {} & {} & {} & {} & {} & {} & {} & {} & {} & {} \\ 
			\hline 
			+ & - & {} & {} & {} & {} & {} & {} & {} & {} & {} & {} \\ 
			\hline 
			- & + & {} & {} & {} & {} & {} & {} & {} & {} & {} & {} \\ 
			\hline 
			{} & {} & {} & {} & {} & {} & {} & {} & {} & {} & {} & {} \\ 
			\hline 
			\end{tabular}
		\end{center}
		\end{table}	
		and so on until the $11$th column:
		\begin{table}[H]
		\begin{center}
			\begin{tabular}{|c|c|c|c|c|c|c|c|c|c|c|c|}
			\hline 
			+ & - & + & - & - & - & + & + & + & - & + & {} \\ 
			\hline 
			+ & + & - & + & + & - & - & - & + & + & - & {} \\ 
			\hline 
			- & + & + & - & + & +-& - & - & + & + & + & {} \\ 
			\hline 
			+ & - & + & + & - & + & - & - & - & + & + & {} \\ 
			\hline 
			+ & + & - & + & + & - & + & - & - & - & + & {} \\ 
			\hline 
			+ & + & + & - & + & + & - & + & - & - & - & {} \\ 
			\hline 
			- & + & + & + & - & + & + & - & + & - & - & {} \\ 
			\hline 
			- & - & + & + & + & - & + & + & - & + & - & {} \\ 
			\hline 
			- & - & - & + & + & + & - & + & + & - & + & {} \\ 
			\hline 
			+ & - & - & - & + & + & + & - & + & + & - & {} \\ 
			\hline 
			- & + & - & - & - & + & + & + & - & + & + & {} \\ 
			\hline 
			{} & {} & {} & {} & {} & {} & {} & {} & {} & {} & {} & {} \\ 
			\hline 
			\end{tabular}
		\end{center}
		\end{table}
		
		\item Last step, we add a row and a column of signs "$-$":
		\begin{table}[H]
		\begin{center}
			\begin{tabular}{|c|c|c|c|c|c|c|c|c|c|c|c|}
			\hline 
			+ & - & + & - & - & - & + & + & + & - & + & - \\ 
			\hline 
			+ & + & - & + & + & - & - & - & + & + & - & - \\ 
			\hline 
			- & + & + & - & + & +-& - & - & + & + & + & - \\ 
			\hline 
			+ & - & + & + & - & + & - & - & - & + & + & - \\ 
			\hline 
			+ & + & - & + & + & - & + & - & - & - & + & - \\ 
			\hline 
			+ & + & + & - & + & + & - & + & - & - & - & - \\ 
			\hline 
			- & + & + & + & - & + & + & - & + & - & - & - \\ 
			\hline 
			- & - & + & + & + & - & + & + & - & + & - & - \\ 
			\hline 
			- & - & - & + & + & + & - & + & + & - & + & - \\ 
			\hline 
			+ & - & - & - & + & + & + & - & + & + & - & - \\ 
			\hline 
			- & + & - & - & - & + & + & + & - & + & + & - \\ 
			\hline 
			- & - & - & - & - & - & - & - & - & - & - & - \\ 
			\hline 
			\end{tabular}
		\end{center}
		\end{table}
	\end{enumerate}
	The reader can easily verify that each row or column taken in pairs are orthogonal.
	
	Now arises the question of whether we choose to associate for example this design of experience to a $4$ factors design (meaning reduced centered design of course!):
	
	to reduce the number of tests from $16$ to $12$, which column should we associate to what? Or going further: for how many $2$-level factors can we associated this type of design having $12$ trials???
	\begin{enumerate}
		\item The first one is to say that the Plackett-Burman designs of order $n$ should only be used for the study of main effects (so no interaction is taken into account) of $n-1$ factors. Thus, a Plackett-Burmann design of order $12$ will be reserved for a study of $11$ factors and only of their main effects (purely additive linear model).

		\item The second one is to say that the Plackett- Burman design contain a concept that we will see further and who's named "\NewTerm{aliases}\index{aliases}" (Plackett-Burman designs are all DoE of resolution III of the second order). Therefore, we have to use this type of design when we are ready to consider initially that the interactions with two factors are very significant.
	\end{enumerate}
	The complexity of the alias with the Plackett- Burmann design makes that in practice they are rather used in the first religion release... by beginners and in the second by consultants.
	\begin{tcolorbox}[title=Remark,colframe=black,arc=10pt]
	A software like Minitab 15.1.1 does not display all the aliases used in the implementation of Plackett-Burmann designs.
	\end{tcolorbox}
	To conclude this section, let us summarize with simple observation:
	\begin{table}[H]\centering
	\begin{center}
		\definecolor{gris}{gray}{0.85}
			\begin{tabular}{|c|c|c|c|}
				\hline
				\multicolumn{1}{c}{\cellcolor{black!30}\textbf{Plane}} & 
  \multicolumn{1}{c}{\cellcolor{black!30}\textbf{Factors}} & 
  \multicolumn{1}{c}{\cellcolor{black!30}\textbf{Interactions}} & 
  \multicolumn{1}{c}{\cellcolor{black!30}\textbf{Sum}} \\ \hline
				 $1$ &  $-$ & $-$ &  $y_1$\\ \hline
				 $2$ &  $+$ & $-$ &  $y_2$\\ \hline
				 $3$ &  $-$ & $+$ &  $y_3$\\ \hline
				 $4$ &  $+$ & $+$ &  $y_4$\\ \hline
		\end{tabular}
	\end{center}
	\caption{Types of plans and factors \& interactions}
	\end{table}
	So using full factorial designs, the user is sure to have the optimal experimental procedure since these plans are based on Hadamard matrices and we have proved that we could not do better.
	
	\pagebreak
	\paragraph{Fractional Factorial Designs}\mbox{}\\\\
	In practice, full factorial designs can only be used on systems with very few factors, or when each test takes very little time. When $n$ is greater than or equal $3$ then experiences costs can quickly become expensive.

	The smallest case where it is interesting to optimize the number of tests is the one consisting of $3$ factors with $2$ levels each. We then have the following equation and experience table:
	
	\begin{table}[H]\centering
	\begin{center}
		\definecolor{gris}{gray}{0.85}
			\begin{tabular}{|c|c|c|c|c|}
				\hline
				\multicolumn{1}{c}{\cellcolor{black!30}\textbf{Trial N${}^\circ$}} & 
  \multicolumn{1}{c}{\cellcolor{black!30}\textbf{Factor 1}} & 
  \multicolumn{1}{c}{\cellcolor{black!30}\textbf{Factor 2}} & 
  \multicolumn{1}{c}{\cellcolor{black!30}\textbf{Factor 3}} & 
  \multicolumn{1}{c}{\cellcolor{black!30}\textbf{Answer}} \\ \hline
				 $1$ & $-$ & $-$ & $-$ & $y_1$\\ \hline
				 $2$ & $+$ & $-$ & $-$ & $y_2$\\ \hline
				 $3$ & $-$ & $+$ & $-$ & $y_3$\\ \hline
				 $4$ & $+$ & $+$ & $-$ & $y_5$\\ \hline
	 			 $5$ & $-$ & $-$ & $+$ & $y_5$\\ \hline
  				 $6$ & $+$ & $-$ & $+$ & $y_6$\\ \hline
  				 $7$ & $-$ & $+$ & $+$ & $y_7$\\ \hline
  				 $8$ & $+$ & $+$ & $+$ & $y_8$\\ \hline
 		\end{tabular}
	\end{center}
	\caption{Experimental Full Factorial (2 factor) matrix}
	\end{table}
	Either as full experiment table:
	\begin{table}[H]\centering
	\begin{center}
		\definecolor{gris}{gray}{0.85}
			\begin{tabular}{|c|c|c|c|c|c|c|c|c|c|}
				\hline
				\multicolumn{1}{c}{\cellcolor{black!30}\textbf{Trial N${}^\circ$}} & 
  \multicolumn{1}{c}{\cellcolor{black!30}\textbf{Rest}} & 
  \multicolumn{1}{c}{\cellcolor{black!30}\textbf{F1}} & 
  \multicolumn{1}{c}{\cellcolor{black!30}\textbf{F2}} & 
  \multicolumn{1}{c}{\cellcolor{black!30}\textbf{F3}} & 
  \multicolumn{1}{c}{\cellcolor{black!30}\textbf{F12}} & 
  \multicolumn{1}{c}{\cellcolor{black!30}\textbf{F13}} & 
  \multicolumn{1}{c}{\cellcolor{black!30}\textbf{F23}} & 
  \multicolumn{1}{c}{\cellcolor{black!30}\textbf{F123}} & 
  \multicolumn{1}{c}{\cellcolor{black!30}\textbf{Answers}}\\ \hline
				$1$ & $+$ & $-$ & $-$ & $-$ & $+$ & $+$ & $+$ & $-$ & $y_1$\\ \hline
				$2$ & $+$ & $+$ & $-$ & $-$ & $-$ & $-$ & $+$ & $+$ & $y_2$\\ \hline
				$3$ & $+$ & $-$ & $+$ & $-$ & $-$ & $+$ & $-$ & $+$ & $y_3$\\ \hline
				$4$ & $+$ & $+$ & $+$ & $-$ & $+$ & $-$ & $-$ & $-$ & $y_4$\\ \hline
				$5$ & $+$ & $-$ & $-$ & $+$ & $+$ & $-$ & $-$ & $+$ & $y_5$\\ \hline
				$6$ & $+$ & $+$ & $-$ & $+$ & $-$ & $+$ & $-$ & $-$ & $y_6$\\ \hline
				$7$ & $+$ & $-$ & $+$ & $+$ & $-$ & $-$ & $+$ & $-$ & $y_7$\\ \hline
				$8$ & $+$ & $+$ & $+$ & $+$ & $+$ & $+$ & $+$ & $+$ & $y_8$\\ \hline
 		\end{tabular}
	\end{center}
	\caption{Full Design of experiment with $3$ factors and interactions under Yates form}
	\end{table}
	or in matrix form:
	
	where once again it is easy to control that all columns are orthogonales and balanced (same number of $+$ or $-$ in each column, or in other words the sum of each column - excepted the first one - is equal to zero) and that the matrix is a Hadamard matrix.	 Therefore the full factorial design for $3$ factors needs $8$ trials.

	We can also represent his in the following form:
	\begin{figure}[H]
		\begin{center}
		\includegraphics[scale=0.9]{img/engineering/doe_full_factorial_design_scheme_three_factor.jpg}
		\end{center}	
		\caption{Traditional representation of a full factorial design with $3$ factors}
	\end{figure}
	and the corresponding $2$-way interactions in the following form (typical of ANOVA studies):
	\newcommand\drawplane[2]
	{%
	    \draw
	    [
	        thick,
	        opacity=.6,
	        draw=#2,
	        fill=#2!60,
	    ] #1 -- cycle;%
	}
	
	\newcommand\drawonecase[4]
	{
	    \begin{tikzpicture}[scale=2]
	
	        \tikzset
	        {
	            edgevis/.style={black},
	            edgehid/.style={dashed,black},
	        }
	
	        \def\vertexradius{.7pt}
	
	        \coordinate (OOO) at (0,0);
	        \coordinate (OOI) at (xyz cs:z=1);
	        \coordinate (OIO) at (xyz cs:y=1);
	        \coordinate (OII) at (xyz cs:y=1,z=1);
	        \coordinate (IOO) at (xyz cs:x=1);
	        \coordinate (IOI) at (xyz cs:x=1,z=1);
	        \coordinate (IIO) at (xyz cs:x=1,y=1);
	        \coordinate (III) at (xyz cs:x=1,y=1,z=1);
	
	        \drawplane{#1}{#2}
	        \drawplane{#3}{#4}
	
	        \draw[edgevis] (OOI) -- (OII) -- (OIO) -- (IIO) -- (IOO) -- (IOI) -- cycle;
	        \draw[edgevis] (III) -- (IIO);
	        \draw[edgevis] (III) -- (IOI);
	        \draw[edgevis] (III) -- (OII);
	        \draw[edgehid] (OOO) -- (OOI);
	        \draw[edgehid] (OOO) -- (OIO);
	        \draw[edgehid] (OOO) -- (IOO);
	
	        \draw (OOO) circle (\vertexradius);
	        \draw (OOI) circle (\vertexradius);
	        \draw (OIO) circle (\vertexradius);
	        \draw (OII) circle (\vertexradius);
	        \draw (IOO) circle (\vertexradius);
	        \draw (IOI) circle (\vertexradius);
	        \draw (IIO) circle (\vertexradius);
	        \draw (III) circle (\vertexradius);
	    \end{tikzpicture}
	}
	\begin{figure}[H]
	\centering
    \begin{subfigure}[b]{\textwidth}
    	 \centering
        \begin{tabular}{ccc}
            \drawonecase
                {(OOO) -- (OOI) -- (OII) -- (OIO)}{red}
                {(IOO) -- (IOI) -- (III) -- (IIO)}{blue}
            &
            \drawonecase
                {(OOO) -- (IOO) -- (IIO) -- (OIO)}{blue}
                {(OOI) -- (IOI) -- (III) -- (OII)}{red}
            &
            \drawonecase
                {(OOO) -- (IOO) -- (IOI) -- (OOI)}{red}
                {(OIO) -- (IIO) -- (III) -- (OII)}{blue}
        \end{tabular}
        \caption[]{Main effects $A$, $B$ and $C$}
    \end{subfigure}
    \par
    \vspace{1em}
    \begin{subfigure}[b]{\textwidth}
    	 \centering
        \begin{tabular}{ccc}
            \drawonecase
                {(OOI) -- (OII) -- (IIO) -- (IOO)}{blue}
                {(OOO) -- (OIO) -- (III) -- (IOI)}{red}
            &
            \drawonecase
                {(OII) -- (OIO) -- (IOO) -- (IOI)}{red}
                {(OOI) -- (OOO) -- (IIO) -- (III)}{blue}
            &
            \drawonecase
                {(OOI) -- (IOI) -- (IIO) -- (OIO)}{blue}
                {(OII) -- (III) -- (IOO) -- (OOO)}{red}
        \end{tabular}
        \caption[]{Two-factor interactions $AB$, $AC$, $BC$}
    \end{subfigure}
    \caption[]{Geometric presentation of contrast (in each case, high levels are highlighted in blue, low levels in red)}
	\end{figure}
	A "\NewTerm{reduced design}\index{reduced design}", more commonly named "\NewTerm{fractional factorial designs  (FFD)}\index{fractional factorial designs}" or  "\NewTerm{screening plans}\index{screening plans}" (following the norm ISO 3534-3:1999), consist in selecting some combinations have therefore been proposed. They naturally gives the opportunity to reduce the experimental costs, but also decrease the information available on the system! We must therefore be sure of the adequate choice relatively to the model to identify!
	
	A first elementary method is to do the hypothesis that there is no interaction. Therefore our function is reduce to a purely additive mode:
	
	and to resolve thy system, $4$ trials are enough. We therefore go from a $8$ trial design to a $4$ trial design just by supposing that there are no interactions and therefore we fall back on a simple target design.
	
	To reduce the experimental costs by keeping the interactions implicitly, we can play with mathematics stuff. First let us take the actual problem with a full factorial design of $3$ factors into explicit form:
	
	Can we reduce the writing of this system to reduce the trials to do? The answer is: Yes! But in counterpart we will lost the measurement of pure effects (we therefore sometimes speak of "\NewTerm{confusion}\index{confusion}" or of "\NewTerm{confounding}\index{confounding}").
		
	The inferior nearest writting is the Hadamard matrix of order $4$. This means obviously that we need to conserve $4$ rows on the $8$ and that these $4$ rows must remain orthogonal, balanced and must satisfy the relation:
	
	The idea, named the "\NewTerm{Box and Hunter method}\index{Box and Hunter method}" and that works only for design with two level factors, is in a first time to merge together the influential factors (in indices) such that (the developments are similar for any $n$):
	
	The grouping choice is also made so that the interaction terms that are supposed to be negligible by the centered reduce normalisation are merged ("confused") with a coefficient of a main factor supposed as not negligible!! Therefore it is also logic that in each grouping we never found indices with a number of a main factor (as merging two main factors that are not negligible will be very bad!).
	
	We then say that we have an "\NewTerm{alias structure}\index{alias structure}" (the norm ISO 3534-3:1999 name this a "\NewTerm{concomitance}\index{concomitance}" when it is the practitioner that force the choice of the regroupement and "\NewTerm{alias}\index{alias}" if the regroupement choice is due to the nature of the experience) of the type:
	 \begin{center}
	 \texttt{0+123;1+23;2+13;3+12}
	 \end{center}
	 and if we write this in the same way as many statistical software, this give (this is exactly the alias given by a software like Minitab 15.1.1):
	\begin{center}
	\texttt{I+ABC;A+BC;B+AC;C+AB}
	\end{center}
	Also some practitionner writhe this (as \texttt{I} must always have a positive sign):
	\begin{center}
	\texttt{I=ABC;A+BC;B+AC;C+AB}
	\end{center}
	Let us write the previous system as following:	
	
	Let us change our notations:
	
	Naturally, if we consider this new notation as new variables, this unique system can now be splitted into two sub-systems (the grouping being named "\NewTerm{contrasts}\index{contrasts}" in this field but this is not corresponding to the vocabulary of the norm ISO 3534-3) to be solvable:
	
	which gives the possibility to divide the number of trials by $2$ relatively to the initial full factorial design!!!! 

	By solving of these two systems, we we say that the interactions are "\NewTerm{aliased}" (also said in "\NewTerm{confusion}\index{confusion}") with the pure effects in negative or in positive (in the present case, some main effects are aliased with two factors interactions).
	
	It is afterwards the tradition to keep only the positive aliased system:
	
	because if the interactions are zero, we fall back on the same experience of the matrix of a full $2^2$ factorial design! The approach therefore leads to select only the runs $2$, $3$, $5$ and $8$, which gives the possibility to divide the number of tests by two compared to a full factorial design. Thus, the fractional factorial design of an experiment with three factors may be reduced to $4$ runs with this method. The fractional factorial design above will naturally be represented by the following experimental matrix:
	
	\begin{table}[H]\centering
	\begin{center}
		\definecolor{gris}{gray}{0.85}
			\begin{tabular}{|c|c|c|c|c|}
				\hline
				\multicolumn{1}{c}{\cellcolor{black!30}\textbf{Trial N${}^\circ$}} & 
  \multicolumn{1}{c}{\cellcolor{black!30}\textbf{Factor 1}} & 
  \multicolumn{1}{c}{\cellcolor{black!30}\textbf{Factor 2}} & 
  \multicolumn{1}{c}{\cellcolor{black!30}\textbf{Factor 3}} & 
  \multicolumn{1}{c}{\cellcolor{black!30}\textbf{Answer}} \\ \hline
				 $2$ & $+$ & $-$ & $-$ & $y_2$\\ \hline
				 $3$ & $-$ & $+$ & $-$ & $y_3$\\ \hline
				 $5$ & $-$ & $-$ & $+$ & $y_5$\\ \hline
				 $8$ & $+$ & $+$ & $+$ & $y_8$\\ \hline
 		\end{tabular}
	\end{center}
	\caption{Fractional design with 3 factors into Yates form}
	\end{table}
	Or more visually:
	\begin{figure}[H]
		\begin{center}
		\includegraphics{img/engineering/full_factorial_design_vs_fractional_factorial_design.jpg}
		\end{center}	
		\caption{Visual $2^2$ factorial design and fraction factorial design}
	\end{figure}
	\begin{tcolorbox}[title=Remark,colframe=black,arc=10pt]
	We have seen that the $8$ runs $2^3$ experiment above is splitted into $2\times 4$ fractional experiments. In a software like Minitab (see companion book) the practitioner can choose if he wants to use the first fraction or the second one in the options of the design builder.\\ 
	
	For a fraction DoE splitted for example into $4$ parts the practitioner can therefore choose in Minitab $4$ possible fractions (but they are obviously equivalent!!!).
	\end{tcolorbox}
	There is however a problem: Even if the triple interaction is really equal to zero, it can remain up to $7$ other coefficients in the model, while we have only $4$ trial results to identify them. In other words, unless we know a priori that at least $3$ of these coefficients are equal to zero (to reduce the problem to $4$ equations with $4$ unknowns), we will get at best only relations between the coefficients and the rigorous identification will therefore be impossible. Thus, it is not possible to indefinitely reduce the cost of an experimental design without degrading its robustess (information it provides).
	
	It is important to notice that in the fractional factorial design above, the third factor is confused (is aliased) with the interaction $12$ of factors $1$ and $2$. We name this the "\NewTerm{initial alias}\index{initial alias}" or "\NewTerm{alias generator}\index{alias generator}" and we can notice by reiterating the calculations for factorial designs with $4$, $5$, $6$... factors that the generators helps us to immediately identify the trials (runs) to preserve (you can control when using softwares like JMP or Minitab that they don't let you choose the alias generator, these softwares take by default the theoretical one). For example, the generator ("alias" or "confusion") of the above experimental table will be written by the tradition:
	
	\begin{center}
	\texttt{C=AB}
	\end{center}
	
	This does not mean that the coefficient of the model are equal but simply that the design is unable to separate the analysis of these two entities. We must therefore pay a particular attention to the interpretation of the respective coefficients!
	
	We have just seen that a $2^{(3-1)}$ design (resolution \texttt{III}) has the serious drawback of confusing a main factor with an interaction of order $2$. A $2^{(4-1)}$ (resolution \texttt{IV}) design has the only disadvantage of confusing (aliazing) a main factor with an interaction of order $3$ and two interactions of order $2$. A $2^{(5-1)}$ design has even lower drawbacks. It is for this reason that the factorial theory uses the concept of "resolution". The higher the resolution, the greater the design is accurate.
	
	\textbf{Definitions (\#\mydef):}\index{interactions}
	\begin{enumerate}
		\item[D1.] When no main effect is aliased with another main effect, but the main effects have aliases with interactions of $2$ factors, we speak then of ""\NewTerm{resolution III designs}\index{resolution III designs}". From a practical point of view, resolution III designs are mainly intended to allow exploratory research because they allow to explore a number of factors with a certain economy. Even if the fail to obtain a sufficiently accurate model they possibly can eliminate a lot of factors in first time and then reduce the number of trials (runs).

		\item[D2.] When no main effect has alias with another main effect or another $2$ factors interaction, but some of the $2$ factors interactions have aliases with other $2$ factor interactions  and some main effects have aliases wither $3$ factor interactions, we speak of "\NewTerm{resolution IV designs}\index{resolution IV designs}".

		\item[D3.] When no main effect or interaction of $2$ factors have alias with another different main effect or another $2$ factors interaction, but $2$ factors interactions have alias with $3$ factors interactions and main effects have alias interactions with $4$ factors interactions we speak of "\NewTerm{resolution V designs}\index{resolution V designs}".
	\end{enumerate}
	and so on...
	
	This concept of resolution is important. We find it in programs like Minitab 15.1.1 in the selection of fractional factorial designs with the most common known alias known under the name "\NewTerm{minimum aberration designs}\index{minimum aberration designs}":
	\begin{figure}[H]
		\begin{center}
		\includegraphics{img/engineering/doe_fractional_design_selection_minitab.jpg}
		\end{center}	
		\caption{Display of fractional factorial designs resolutions selector in Minitab 15.1.1}
	\end{figure}
	or with the Design Expert software (part of the picture but is more explicit than Minitab 15.1.1):
	\begin{figure}[H]
		\begin{center}
		\includegraphics{img/engineering/doe_fractional_design_selection_designexpert.jpg}
		\end{center}	
		\caption{Display of fractional factorial designs resolutions selector in Design Expert}
	\end{figure}
	Thus, the reader will observe that a full factorial design of $5$ factors with 32 trials (runs) can be reduced to $16$ trials by bringing together the influential factors in pairs or $2$-tuples (hence the division by $2$ of the number of tests) , or $8$ tests by bringing together the influential factors by $4$-tuples.
	
	Then it's the job of the experimenter to know well its analysis and whether:
	\begin{enumerate}
		\item Among the aliased factors if there are interactions or not!

		\item In the aliased factor, the strong influence on the answer comes from the interaction or from the pure effect alone!
	\end{enumerate}
	Once determined coefficients, assuming that each of the factors or interactions is independent (acceptable limit hypothesis ...) some engineers are analyzing the variance of the regression line obtained final or determine the correlation coefficient to whether the linear approximation of the model is acceptable in the field of study and application.
	
	Regarding the generators of fractional factorial designs here is a non-exhaustive summary table:
	\begin{table}[H]\centering
	\begin{center}
		\definecolor{gris}{gray}{0.85}
			\begin{tabular}{|c|c|c|c|c|c|}
				\hline
				\multicolumn{1}{c}{\cellcolor{black!30}\textbf{Factors}} & 
  \multicolumn{1}{c}{\cellcolor{black!30}\textbf{Trials}} & 
  \multicolumn{1}{c}{\cellcolor{black!30}\textbf{Resolution}} & 
  \multicolumn{1}{c}{\cellcolor{black!30}\textbf{Reduction}} & 
  \multicolumn{1}{c}{\cellcolor{black!30}\textbf{Generators}} & 
  \multicolumn{1}{c}{\cellcolor{black!30}\textbf{Aliases}}\\ \hline
				 $3$ & $4$ & III & $2^{(3-1)}$ & \texttt{C=AB} & \parbox{4.5cm}{\texttt{I+ABC,A+BC,B+AC,C+AB}} \\ \hline
				 $4$ & $8$ & IV & $2^{(4-1)}$ & \texttt{D=ABC} & \parbox{4.5cm}{\texttt{I+ABCD,A+BCD,B+ACD}\\\texttt{C+ABD,D+ABC,AB+CD}\\\texttt{AC+BD,AD+BC}} \\ \hline
				 $5$ & $8$ & III & $2^{(5-2)}$ & \parbox{2cm}{\texttt{D=AB}\\\texttt{E=AC}} & \parbox{4.5cm}{\texttt{I+ABD+ACE+BCDE}\\\texttt{A+BD+CE+ABCDE}\\\texttt{B+AD+CDE+ABCE}\\\texttt{C+AE+BDE+ABCD}\\\texttt{D+AB+BCE+ACDE}\\\texttt{E+AC+BCD+ABDE}\\\texttt{BC+DE+ABE+ACD}\\\texttt{BE+CD+ABC+ADE}} \\ \hline
				 $6$ & $16$ & V & $2^{(5-1)}$ & \parbox{2cm}{\texttt{E=ABCD}} & \parbox{4.5cm}{\texttt{A+BCDE}\\\texttt{B+ACDE}\\\texttt{C+ABDE}\\\texttt{D+ABCE}\\\texttt{E+ABCD}\\\texttt{AB+CDE}\\\texttt{AC+BDE}\\\texttt{AD+BCE}\\\texttt{AE+BCD}\\\texttt{BC+ADE}\\\texttt{BD+ACE}\\\texttt{BE+ACD}\\\texttt{CD+ABE}\\\texttt{CE+ABD}\\\texttt{DE+ABC}} \\ \hline
				... & ... & ... & ... & ... & ... \\ \hline
 		\end{tabular}
	\end{center}
	\caption{Some fractional factorial designs with generators and alias}
	\end{table}
	Of course, use of fractional design is an economic (and temporal) bet. If the conclusions are clear, then we have saved time and reduced our effort. But sometimes we lose the bet. We can therefore use a "\NewTerm{complementary design}\index{complementary design}" the consists of adding in a pertinent way enough lines to the initial design to unaliased some desired coefficients based on the choice of the alias generator.
	
	Before moving on to another type of design, let us come back on to the fractional factorial design with $3$ factors and therefore with $4$ trials for educational reasons. Consider that we have done these $4$ trials, and for each, we got a measurement as given in the figure below:
	\begin{figure}[H]
		\begin{center}
		\includegraphics{img/engineering/doe_real_three_factors_factorial_design.jpg}
		\end{center}	
		\caption{Representation of a real fractional factorial design with $3$ factors}
	\end{figure}
	We want to show to the reader how not determine the coefficients by calculations (its all about solving a simple linear system and then there will be an example about its it a little further below) but how to calculate the effects in the context of this particular case?

	Indeed, the effect of $x_1$ is:
	
	and this of $x_2$ is:
	
	and this of $x_3$ is:
	
	The effect of interaction is somewhat subtle in the case of $3$ factors using the figure above (we will see in the example further below with a picture that with a table it is much more intuitive). Thus, we have for the interaction $x_1x_2$:
	
	and for $x_1x_3$:
	
	and for $x_2x_3$:
	
	
	For the triple interaction $x_1x_2x_3$ with the figure above it is also not obvious (it is a little more intuitive with a table as we will see in the example further below). We have:
	
	
	\pagebreak
	\subsubsection{General factorial Designs}
	In this book we name "\NewTerm{general factorial design}\index{general factorial designs}" designs that have a given quantity of factors but with more than $2$ levels and that are not reduced. 
	
	Given $n$ the number of factor, $l_i$ the number of levels of each factor. The number of runs $N$ of a general factorial design is given obviously:
	
	As we will see below, many of the techniques relatives to complete factorial design (designs with only $2$ levels for recall) applies identically to general factorial design. The main differences that surprise people starting to work in the field of DoE are:
	\begin{itemize}
		\item The final regression model don't have a unique coefficient for each factor therefore we cannot do a statistical regression analysis.
		
		\item The calculation methods of the standard error of the coefficients we have seen for the $2$ factorial design cannot be applied (we have to use another methods of calculations)
	\end{itemize}

	Rather than doing a long theoretical introduction to general factorial designs, let us use a companion example that will show us that in fact most concepts that we have learned so far also applies!
	
	As part of the study of tires, the criterion retain by a company is the longevity (number of kilometers traveled before the tire blow up) and the chosen factors are the type of use (city or highway), average speed ($40$ [km/h] or $50$ [km/h]) and the inflation pressure ($1.5$ or $2$ or $2.5$ [kg]).\\
	
	Therefore we have a general factorial design with the following number of runs:
	
 	Furthermore, it is more healthy to repeat the experiments (to analyze the dispersion of the values in order to run an ANOVA), so to make $24$, $36$, etc. experiments or to make some central points measurements.\\

	We will denote as usual $y$ the variable of interest and in this case $x_1,x_2,x_3$ the variables factors. We are seeking a simple model in the form:
	
	As each factor can have several levels, for will have for example the value of $y$ at the level $2$ of factor $1$ and at level $1$ of factor $2$ and at level $3$ of factor $3$ that will be denoted:
	
	
	The full factorial design is the presentation of the various experiences (measurement) depending on the levels of the variables. For example, to study the tires we will have:
	\begin{itemize}
		\item For the factor 1, the place, denoted $x_1$: city$=1$, town $=2$

		\item For the factor 2, the speed, denoted $x_2$: $40$ [km/h] $=1$, $50$ [km/h] $=2$

		\item For the factor 3, the pressure, denoted $x_3$: $1.5$ [kg] $=1$, $2$ [kg] $=2$, $3$ [kg] $=3$
	\end{itemize}
	The experimental matrix will then be (this is exactly the same as that generally obtained with the software Minitab: a full complete general design for $2$ factors with $2$ levels and $1$ factor with $3$ levels without randomization):
	\begin{table}[H]\centering
		\centering
			\definecolor{gris}{gray}{0.85}
				\begin{tabular}{|c|c|c|c|}
					\hline
					\multicolumn{1}{c}{\cellcolor{black!30}\textbf{Trial N${}^\circ$}} & 
	  \multicolumn{1}{c}{\cellcolor{black!30}$x_1$} & 
	  \multicolumn{1}{c}{\cellcolor{black!30}$x_2$} & 
	  \multicolumn{1}{c}{\cellcolor{black!30}$x_3$} \\ \hline
					$1$ & $1$ & $1$ & $1$\\ \hline
					$2$ & $1$ & $1$ & $2$\\ \hline
					$3$ & $1$ & $1$ & $1$\\ \hline
					$4$ & $1$ & $2$ & $1$\\ \hline
					$5$ & $1$ & $2$ & $2$\\ \hline
					$6$ & $1$ & $2$ & $3$\\ \hline
					$7$ & $2$ & $1$ & $1$\\ \hline
					$8$ & $2$ & $1$ & $2$\\ \hline
					$9$ & $2$ & $1$ & $3$\\ \hline
					$10$ & $2$ & $2$ & $1$\\ \hline
					$11$ & $2$ & $2$ & $2$\\ \hline
					$12$ & $1$ & $2$ & $3$\\ \hline
	 		\end{tabular}
	\end{table}
	At this table, we can associate the results of the experiments. For example, for the tires, we did three tests (three "\NewTerm{replication}\index{replication}") for each of the levels and we got:
	\begin{table}[H]\centering
		\begin{center}
			\definecolor{gris}{gray}{0.85}
				\begin{tabular}{|c|c|c|c|c|c|c|}
					\hline
					\multicolumn{1}{c}{\cellcolor{black!30}\textbf{Trial N${}^\circ$}} & 
	  \multicolumn{1}{c}{\cellcolor{black!30}$x_1$} & 
	  \multicolumn{1}{c}{\cellcolor{black!30}$x_2$} & 
	  \multicolumn{1}{c}{\cellcolor{black!30}$x_3$} & 
	  \multicolumn{3}{c|}{\cellcolor{black!30}\textbf{Measured values of $y$}}                                             \\ \hline
		$1$ & $1$ & $1$ & $1$ & $32700$ & $32750$ & $32960$\\ \hline
		$2$ & $1$ & $1$ & $2$ & $33430$ & $33360$ & $32910$\\ \hline
		$3$ & $1$ & $1$ & $3$ & $31710$ & $32100$ & $32220$\\ \hline
		$4$ & $1$ & $2$ & $1$ & $32680$ & $32270$ & $33130$\\ \hline
		$5$ & $1$ & $2$ & $2$ & $34070$ & $33100$ & $33610$\\ \hline
		$6$ & $1$ & $2$ & $3$ & $33220$ & $33700$ & $33285$\\ \hline
		$7$ & $2$ & $1$ & $1$ & $33180$ & $32160$ & $32640$\\ \hline
		$8$ & $2$ & $1$ & $2$ & $34430$ & $34380$ & $34460$\\ \hline
		$9$ & $2$ & $1$ & $3$ & $33570$ & $33300$ & $32570$\\ \hline
		$10$ & $2$ & $2$ & $1$ & $33270$ & $33080$ & $32415$\\ \hline
		$11$ & $2$ & $2$ & $2$ & $33440$ & $33570$ & $34204$\\ \hline
		$12$ & $1$ & $2$ & $3$ & $32840$ & $33210$ & $32470$\\ \hline
	 		\end{tabular}
		\end{center}
	\end{table}

	Thanks to these values we can calculate the effects of the various factors. For this we start with principle that their average (effect) is zero for a given factors. Therefore the global arithmetic average (that can be calculated as the arithmetic average if and only the variables are code):
	
	is the constant coefficient of the model (intercept):
	
	The effect of $x_1$ at the level $1$ is the variation of this average when we consider only the cases where $x_1$ is at the level $1$. The arithmetic average becomes $32955.8$ and therefore the effect is:
	
	We will write this obviously as we already know:
	
	At levelof$x_1$, the average becomes $33282.7$, therefore the effect of $x_1$ at level $2$ is equal to:
	
	The difference:
	
	is named the "\NewTerm{effect}\index{effect}" of the factor.\\
	
	We notice that as assumed, the sum of the effects $x_1$ is zero, as it is the sum of the deviations from the arithmetic average of the sub-populations of the same size (and thus the overall average is the arithmetic average of the levels).\\
	
	Exactly in the same way, we get:
	
	and we also find that the sum of the effects of a variable (factor) is zero.\\

	We modelize these results into the form using the average response (notice that the sum of each vector is 
zero):
	
	giving the model found. Function that we can rewrite under the form of absolute average effects (the most interesting form mathematically speaking):
	
	intermediate result to be compared with a simple linear regression that (perfectly identical result between Microsoft Excel 14.0.6117 and Minitab 15.1.1):
	
	but that does not make sense since the linear regression is not built for factors over two levels (\SeeChapter{see section Theoretical Computing page \pageref{dummy variable regression}}).\\

	From these results, we already derive the best tire usage policy: road, speed $50$ [km/h] and pressure $2$ [kg], we can expect to do:
	
	Let us notice immediately that in some cases, the experiment did better (see the values in the above table), even in town or at $40$ [km / h]. This come for the obvious dispersion of the results. But then... did the effects found come perhaps only from this dispersion!? Furthermore we studied the effects independently of each other. But it may have some that reinforce some others. We will explore these issues in the further below!\\
	
	Let us take back our $36$ experiments, comparing the differences $\Delta$ between the values measured (denoted here $\hat{y}$) and the model predictions (denoted here $y$):
	\begin{table}[H]\centering
		\begin{center}
			\definecolor{gris}{gray}{0.85}
			\begin{tabular}{|c|c|c|c|c|c|c|c|c|c|c|c|c|}
			\hline				
			  \multicolumn{1}{c}{\cellcolor{black!30}$x_1$} & 
			  \multicolumn{1}{c}{\cellcolor{black!30}$x_2$} & 
			  \multicolumn{1}{c}{\cellcolor{black!30}$x_3$} & 
			  \multicolumn{1}{c}{\cellcolor{black!30}$\hat{y}$} & 
			  \multicolumn{1}{c}{\cellcolor{black!30}$y$} & 
			  \multicolumn{1}{c}{\cellcolor{black!30}$\Delta$} &
			  \multicolumn{1}{c}{\cellcolor{black!30}$x_1$} & 
			  \multicolumn{1}{c}{\cellcolor{black!30}$x_2$} & 
			  \multicolumn{1}{c}{\cellcolor{black!30}$x_3$} & 
			  \multicolumn{1}{c}{\cellcolor{black!30}$\hat{y}$} & 
			  \multicolumn{1}{c}{\cellcolor{black!30}$y$} & 
			  \multicolumn{1}{c}{\cellcolor{black!30}$\Delta$} \\ \hline
				$1$ & $1$ & $1$ & $32700$ & $32527.5$ & $172.5$ & $2$ & $1$ & $1$ & $32160$ & $32854.3$ & $-694.3$\\ \hline
				$1$ & $1$ & $2$ & $33430$ & $33496.6$ & $-66$ & $2$ & $1$ & $2$ & $34280$ & $33823.4$ & $456.6$\\ \hline
				$1$ & $1$ & $3$ & $31710$ & $32607.5$ & $-897.5$ & $2$ & $1$ & $3$ & $33300$ & $32943.3$ & $365.7$\\ \hline
				$1$ & $2$ & $1$ & $32680$ & $32684.9$ & $-4.9$ & $2$ & $2$ & $1$ & $33080$ & $33011.7$ & $68.3$\\ \hline
				$1$ & $2$ & $2$ & $34070$ & $33654$ & $416$ & $2$ & $2$ & $2$ & $33570$ & $33980.8$ & $-410.8$\\ \hline
				$1$ & $2$ & $3$ & $33220$ & $32764.9$ & $455.1$ & $2$ & $2$ & $3$ & $33210$ & $33091.7$ & $118.3$\\ \hline
				$2$ & $1$ & $1$ & $33180$ & $32854.3$ & $325.7$ & $1$ & $1$ & $1$ & $32960$ & $32527.5$ & $432.5$\\ \hline
				$2$ & $1$ & $2$ & $34430$ & $33823.4$ & $606.6$ & $1$ & $1$ & $2$ & $32910$ & $33496.6$ & $-586.6$\\ \hline
				$2$ & $1$ & $3$ & $33570$ & $32934.3$ & $635.7$ & $1$ & $1$ & $3$ & $32220$ & $32607.5$ & $-387.5$\\ \hline
				$2$ & $2$ & $1$ & $33270$ & $33011.7$ & $258.3$ & $1$ & $2$ & $1$ & $33130$ & $32684.9$ & $445.1$\\ \hline
				$2$ & $2$ & $2$ & $33440$ & $33980.8$ & $-540.8$ & $1$ & $2$ & $2$ & $33610$ & $33654$ & $-44$\\ \hline
				$2$ & $2$ & $3$ & $33840$ & $33091.7$ & $-251.7$ & $1$ & $2$ & $3$ & $33285$ & $32764.9$ & $520.1$\\ \hline
				$1$ & $1$ & $1$ & $32750$ & $32527.5$ & $222.5$ & $2$ & $1$ & $1$ & $32640$ & $32854.3$ & $-214.3$\\ \hline
				$1$ & $1$ & $2$ & $33360$ & $33496.6$ & $-136.6$ & $2$ & $1$ & $2$ & $34460$ & $33823.4$ & $636.6$\\ \hline
				$1$ & $1$ & $3$ & $32100$ & $32607.5$ & $-507.5$ & $2$ & $1$ & $3$ & $32570$ & $32934.3$ & $-364.3$\\ \hline
				$1$ & $2$ & $1$ & $32270$ & $32684.9$ & $-414.9$ & $2$ & $2$ & $1$ & $32415$ & $33011.7$ & $-596.7$\\ \hline
				$1$ & $2$ & $2$ & $33100$ & $33654$ & $-554$ & $2$ & $2$ & $2$ & $34204$ & $33980.8$ & $223.2$\\ \hline
				$1$ & $2$ & $3$ & $33700$ & $32764.9$ & $935.1$ & $2$ & $2$ & $3$ & $32470$ & $33091.7.8$ & $-621.7$\\ \hline
	 		\end{tabular}
		\end{center}
	\end{table}
	We see that the differences between the model and the predicted values are sometimes important. This may be a natural random dispersion of values, or an incorrect model or it could be due to the fact that combination of factors have more effect than separate. The positive or negative effects of two factors may do more than just add themselves (purely additive model may therefore be reject).\\
		
	It is therefore probably interactions (chemists speak of "\NewTerm{potentiation}\index{potentiation}"). To see if there are second orders interactions, a good way is to study all the factors in pairs (and ignore the existence of others). So if we start with the factors (location) $x_1$ and (speed) $x_2$ by ignoring the tire pressure $x_3$, we get (the explanations of how the values in this table are calculated are given below the table):
	\begin{table}[H]\centering
		\centering
		\definecolor{gris}{gray}{0.85}
		\begin{tabular}{|c|c|c|c|c|}
		\hline
		\multicolumn{1}{c}{\cellcolor{black!30}\textbf{Levels}} & 
	  	\multicolumn{1}{c}{\cellcolor{black!30}Experimental average} & 
	  	\multicolumn{1}{c}{\cellcolor{black!30}Total effect} & 
	  	\multicolumn{1}{c}{\cellcolor{black!30}Sum of effects} & 
	  	\multicolumn{1}{c}{\cellcolor{black!30}Interaction effect} \\ \hline
		$x_{1,1}x_{2,1}$ & $32682.2$ & $-437.1$ & $-242.2$ & $-194.9$ \\ \hline
		$x_{1,1}x_{2,2}$ & $33229.4$ & $110.2$ & $-84.7$ & $-194.9$ \\ \hline
		$x_{1,2}x_{2,1}$ & $33398.9$ & $279.6$ & $84.7$ & $194.9$ \\ \hline
		$x_{1,2}x_{2,2}$ & $33166.6$ & $47.3$ & $242.2$ & $-194.9$ \\ \hline
	 	\end{tabular}
		\caption[]{Table of interaction of factors $1$ and $2$}
	\end{table}
	So in the table above $32,682.2$ is the arithmetic average of the experiences done at level $x_{1,1}$ (City) and level $x_{2,1}$ ($40$ [km/h]). By subtracting the main average $x_0$ (always $33,119.3$), we get an effect of $-437.1$, while the effect of $x_1$ ($-163.4$) plus the effect of $x_2$ (-78.7) gives only $-242.2$.\\

	So there a surplus off $-194.9$, largely due to the interaction $x_{1,1,}x_{2,1}$ (notice that this is the same for $x_{1,2}x_{2,2}$, and the opposite for $x_{1,1}x_{2,2}$ and $x_{1,2}x_{2,1}$). We notice immediately that the interaction  $x_{1,1}x_{2,1}$ has a much greater effect than the effects of $x_1$ and $x_2$!\\

	Let us see the two other possible interactions of respectively factors $1$ and $3$ and $2$ and $3$:
		\begin{table}[H]\centering
		\centering
		\definecolor{gris}{gray}{0.85}
		\begin{tabular}{|c|c|c|c|c|}
		\hline
		\multicolumn{1}{c}{\cellcolor{black!30}\textbf{Levels}} & 
	  	\multicolumn{1}{c}{\cellcolor{black!30}Experimental average} & 
	  	\multicolumn{1}{c}{\cellcolor{black!30}Total effect} & 
	  	\multicolumn{1}{c}{\cellcolor{black!30}Sum of effects} & 
	  	\multicolumn{1}{c}{\cellcolor{black!30}Interaction effect} \\ \hline
		$x_{1,1}x_{3,1}$ & $32748.3$ & $-370.9$ & $-513.1$ & $142.2$ \\ \hline
		$x_{1,1}x_{3,2}$ & $33143.3$ & $294.1$ & $455.9$ & $-161.9$ \\ \hline
		$x_{1,1}x_{3,3}$ & $32705.8$ & $-413.4$ & $-433.1$ & $19.7$ \\ \hline
		$x_{1,2}x_{3,1}$ & $32790.8$ & $-328.4$ & $-186.3$ & $-142.2$ \\ \hline
		$x_{1,2}x_{3,3}$ & $34064.0$ & $944.7$ & $782.8$ & $161.9$ \\ \hline
		$x_{1,2}x_{3,3}$ & $32993.3$ & $-125.9$ & $-106.3$ & $-19.7$ \\ \hline
	 	\end{tabular}
	\end{table}
	\begin{table}[H]\centering
		\centering
		\definecolor{gris}{gray}{0.85}
		\begin{tabular}{|c|c|c|c|c|}
		\hline
		\multicolumn{1}{c}{\cellcolor{black!30}\textbf{Levels}} & 
	  	\multicolumn{1}{c}{\cellcolor{black!30}Experimental average} & 
	  	\multicolumn{1}{c}{\cellcolor{black!30}Total effect} & 
	  	\multicolumn{1}{c}{\cellcolor{black!30}Sum of effects} & 
	  	\multicolumn{1}{c}{\cellcolor{black!30}Interaction effect} \\ \hline
		$x_{2,1}x_{3,1}$ & $32731.7$ & $-387.6$ & $-428.4$ & $40.8$ \\ \hline
		$x_{2,1}x_{3,2}$ & $33811.7$ & $629.4$ & $540.7$ & $151.7$ \\ \hline
		$x_{2,1}x_{3,2}$ & $33811.7$ & $692.4$ & $540.7$ & $151.7$ \\ \hline
		$x_{2,1}x_{3,3}$ & $32578.3$ & $-540.9$ & $-348.4$ & $-192.5$ \\ \hline
		$x_{2,2}x_{3,1}$ & $32807.5$ & $-311.8$ & $-271.0$ & $-40.8$ \\ \hline
		$x_{2,2}x_{3,2}$ & $33665.7$ & $546.5$ & $698.1$ & $-151.7$ \\ \hline
		$x_{2,2}x_{3,3}$ & $33120.8$ & $1.6$ & $-191.0$ & $192.5$ \\ \hline
	 	\end{tabular}
	\end{table}
	We can represent the interactions differently by crossing the factors:
	\begin{table}[H]\centering
		\centering
		\definecolor{gris}{gray}{0.85}
		\begin{tabular}{|c|c|c|}
		\hline
		\multicolumn{1}{c}{\cellcolor{black!30}} & 
	  	\multicolumn{1}{c}{\cellcolor{black!30}$x_{2,1}$} & 
	  	\multicolumn{1}{c}{\cellcolor{black!30}$x_{2,2}$} \\ \hline
		\multicolumn{1}{c}{\cellcolor{black!30}$x_{1,1}$}  & $-194.9$ & $194.9$ \\ \hline
		\multicolumn{1}{c}{\cellcolor{black!30}$x_{1,2}$} & $194.9$ & $-194.9$ \\ \hline
	 	\end{tabular}
	\end{table}
	\begin{table}[H]\centering
		\centering
		\definecolor{gris}{gray}{0.85}
		\begin{tabular}{|c|c|c|c|}
		\hline
		\multicolumn{1}{c}{\cellcolor{black!30}} & 
	  	\multicolumn{1}{c}{\cellcolor{black!30}$x_{3,1}$} & 
	  	\multicolumn{1}{c}{\cellcolor{black!30}$x_{3,2}$}  & 
	  	\multicolumn{1}{c}{\cellcolor{black!30}$x_{3,3}$}  \\ \hline
		\multicolumn{1}{c}{\cellcolor{black!30}$x_{1,1}$}  & $142.2$ & $-161.9$ & $19.7$ \\ \hline
		\multicolumn{1}{c}{\cellcolor{black!30}$x_{1,2}$} & $-142.2$ & $161.9$ & $-19.7$ \\ \hline
	 \end{tabular}
	\end{table}
		 \begin{table}[H]\centering
		\centering
		\definecolor{gris}{gray}{0.85}
		\begin{tabular}{|c|c|c|c|}
		\hline
		\multicolumn{1}{c}{\cellcolor{black!30}} & 
	  	\multicolumn{1}{c}{\cellcolor{black!30}$x_{3,1}$} & 
	  	\multicolumn{1}{c}{\cellcolor{black!30}$x_{3,2}$}  & 
	  	\multicolumn{1}{c}{\cellcolor{black!30}$x_{3,3}$}  \\ \hline
		\multicolumn{1}{c}{\cellcolor{black!30}$x_{2,1}$}  & $40.8$ & $151.7$ & $-192.5$ \\ \hline
		\multicolumn{1}{c}{\cellcolor{black!30}$x_{2,2}$} & $-40.8$ & $-151.7$ & $192.5$ \\ \hline
	 	\end{tabular}
	\end{table}      
	We notice then that each line or each column has a total of zero. We also see that in this case, it is difficult to neglect the interactions, and that $x_2$ (for example) influence the results more by its interactions that its own effect!\\

	We can still try to check if the interaction of the three factors has a significant effect. As the experiment has been repeated $3$ times at each level (we then speak of "three-blocks"), we can calculate the mean of $y$, and subtract to it the model with the interactions calculated above (notice that the sum of each vector is zero!):
	
	that it is customary to write for simplification purposes (notic that the sum of each vector is still zero):
	
	\begin{tcolorbox}[title=Remark,colframe=black,arc=10pt]
	It suffices to know that for the coefficients of interaction terms the missing effects (not written) are of opposed sign.
	\end{tcolorbox}
	Some statistical software go further in the simplification of notation by writing (still in the same idea that the other coefficients are of opposed sign):
	
	and other software (such as Minitab 17.1.3 for example) gives:
	\begin{figure}[H]
		\centering
		\includegraphics[scale=0.6]{img/engineering/doe_regression_equation_minitab.jpg}	
	\end{figure}

	We can then summarize all these results in the form:
	\begin{table}[H]\centering
		\begin{center}
		\definecolor{gris}{gray}{0.85}
		\begin{tabular}{|c|c|c|c|c|c|c|c|c|c|c|}
		\hline
		\multicolumn{1}{c}{\cellcolor{black!30}\textbf{Trial N${}^\circ$}} & 
	  \multicolumn{1}{c}{\cellcolor{black!30}$x_1$} & 
	  \multicolumn{1}{c}{\cellcolor{black!30}$x_2$} & 
	  \multicolumn{1}{c}{\cellcolor{black!30}$x_3$} & 
	  \multicolumn{3}{c|}{\cellcolor{black!30}\textbf{Measured values of $y$}} & 
	  \multicolumn{1}{c}{\cellcolor{black!30}\textbf{Average}} & 
	  \multicolumn{1}{c}{\cellcolor{black!30}\textbf{Model}} & 
	  \multicolumn{1}{c}{\cellcolor{black!30}\textbf{$\Delta$}}                                                                                                                                                                         \\ \hline
		$1$ & $1$ & $1$ & $1$ & $32700$ & $32750$ & $32960$ & $32803.3$ & $32515.6$ & $287.8$\\ \hline
		$2$ & $1$ & $1$ & $2$ & $33430$ & $33360$ & $32910$ & $33233.3$ & $33291.5$ & $-58.1$\\ \hline
		$3$ & $1$ & $1$ & $1$ & $31710$ & $32100$ & $32220$ & $32010.0$ & $32239.8$ & $-229.8$\\ \hline
		$4$ & $1$ & $2$ & $1$ & $32680$ & $32270$ & $33130$ & $32693.3$ & $32981.2$ & $-287.8$\\ \hline
		$5$ & $1$ & $2$ & $2$ & $34070$ & $33100$ & $33610$ & $33593.3$ & $33535.3$ & $58.1$\\ \hline
		$6$ & $1$ & $2$ & $3$ & $33220$ & $33700$ & $33285$ & $33401.7$ & $33172.0$ & $229.7$\\ \hline
		$7$ & $2$ & $1$ & $1$ & $33180$ & $32160$ & $32640$ & $32660$ & $32947.8$ & $-287.8$\\ \hline
		$8$ & $2$ & $1$ & $2$ & $34430$ & $34380$ & $34460$ & $34390.0$ & $34331.9$ & $58.1$\\ \hline
		$9$ & $2$ & $1$ & $3$ & $33570$ & $33300$ & $32570$ & $33146.7$ & $32917.0$ & $229.7$\\ \hline
		$10$ & $2$ & $2$ & $1$ & $33270$ & $33080$ & $32415$ & $32921.7$ & $32633.8$ & $287.9$\\ \hline
		$11$ & $2$ & $2$ & $2$ & $33440$ & $33570$ & $34204$ & $33738.0$ & $33796.1$ & $-58.1$
\\ \hline
		$12$ & $1$ & $2$ & $3$ & $32840$ & $33210$ & $32470$ & $32840.3$ & $33069.6$ & $-229.6$\\ \hline
	 	\end{tabular}
		\end{center}
	\end{table}
	We see then that the values of the model with interactions continues to diverge quite significantly for the average values, which can be interpreted either by the fact that the effects are not linear, either by an interaction between the three factors.\\
	
	It would be necessary to redo the calculations above with the interaction of the three factors. But as it is always the same principle, we will do that if some readers explicitly request us to add it.\\

	But there is still a source of error in the model is to take into account effects that do not occur in reality. To ensure that the effect calculated on a variable or interaction is real, we will use to start the one-way fixed factor analysis of variance (ANOVA) that we have studied in detail in the section Statistics. For this, it is important to have repeated the experiments in order to highlight the dispersion due to external and almost unknown uncontrolled factors.\\

	For example, the factor $x_2$ has a rather small effect. Does he have a real influence? For this, we separate the $36$ experiments in two samples (level $x_{2,1}$ and level $x_{2,2}$, and we calculate the sums of squares and the degrees of freedom (we can detail the following calculations on request) as we have see it in the section of Statistics:
	\begin{table}[H]\small
	\renewcommand{\arraystretch}{1.2}
	\begin{tabular}{cccccc}\hline
	Source & Sum of squares & $\chi^2$ df & Average of squares \\ \hline
	Inter-Class & $Q_A=\displaystyle\sum_{i}n_i\left(\bar{x}_{i}-\bar{\bar{x}}\right)^2=170019$ & $k-1=1$ & $\text{MSk}=\displaystyle\frac{Q_A}{k-1}=170019$ \\
	Intra-Class & $Q_R=\displaystyle\sum_{ij}\left(x_{ij}-\bar{x}_i\right)^2=16025448$ & $N-k=34$ & $ \text{MSE}=\displaystyle\frac{Q_R}{k-1}=471337$ \\
	Total & $Q_T=\displaystyle\sum_{ij}\left(x_{ij}-\bar{\bar{x}}\right)^2=16195467$ & $N-1=35$ & \\ \hline
	\end{tabular}
	\end{table}
	And we get from this table:
	
	What gives with Minitab 17.3.1 (we can see also that the model is probably not linear) as detailed in the Minitab companion book:
	\begin{figure}[H]
		\begin{center}
		\includegraphics[scale=0.6]{img/engineering/minitab_speed_factor_anova.jpg}
		\end{center}	
	\end{figure}
	and visually:
	\begin{figure}[H]
		\begin{center}
		\includegraphics[scale=0.6]{img/engineering/minitab_speed_factor_box_plot.jpg}
		\end{center}	
	\end{figure}	
	
	As we can see we cannot reject the null hypothesis $H_0$ of this ANOVA. Therefore this factor has no significant impact on our experience.
	
	We can do the same calculations for $x_1$ (the place: city or town):
	\begin{table}[H]\small
	\renewcommand{\arraystretch}{1.2}
	\begin{tabular}{cccccc}\hline
	Source & Sum of squares & $\chi^2$ df & Average of squares \\ \hline
	Inter-Class & $Q_A=\displaystyle\sum_{i}n_i\left(\bar{x}_{i}-\bar{\bar{x}}\right)^2=961707$ & $k-1=1$ & $\text{MSk}=\displaystyle\frac{Q_A}{k-1}=961707$ \\
	Intra-Class & $Q_R=\displaystyle\sum_{ij}\left(x_{ij}-\bar{x}_i\right)^2=14906680$ & $N-k=34$ & $ \text{MSE}=\displaystyle\frac{Q_R}{k-1}=438432$ \\
	Total & $Q_T=\displaystyle\sum_{ij}\left(x_{ij}-\bar{\bar{x}}\right)^2=15868387$ & $N-1=35$ & \\ \hline
	\end{tabular}
	\end{table}
	And we get from this table:
	
	What gives with Minitab 17.3.1 (we can see also that the model is probably not linear) as detailed in the Minitab companion book:
	\begin{figure}[H]
		\begin{center}
		\includegraphics[scale=0.7]{img/engineering/minitab_placefactor_anova.jpg}
		\end{center}	
	\end{figure}
	and visually:
	\begin{figure}[H]
		\begin{center}
		\includegraphics[scale=0.7]{img/engineering/minitab_place_factor_box_plot.jpg}
		\end{center}	
	\end{figure}	
	As we can see we also cannot reject the null hypothesis $H_0$ of this ANOVA. Therefore this factor has also no significant impact on our experience.\\
	
	We can do the same calculations for $x_3$ (the pressure):
	\begin{table}[H]\small
	\renewcommand{\arraystretch}{1.2}
	\begin{tabular}{cccccc}\hline
	Source & Sum of squares & $\chi^2$ df & Average of squares \\ \hline
	Inter-Class & $Q_A=\displaystyle\sum_{i}n_i\left(\bar{x}_{i}-\bar{\bar{x}}\right)^2=6943967$ & $k-1=2$ & $\text{MSk}=\displaystyle\frac{Q_A}{k-1}=3471983$ \\
	Intra-Class & $Q_R=\displaystyle\sum_{ij}\left(x_{ij}-\bar{x}_i\right)^2=8924421$ & $N-k=33$ & $ \text{MSE}=\displaystyle\frac{Q_R}{k-1}=270437$ \\
	Total & $Q_T=\displaystyle\sum_{ij}\left(x_{ij}-\bar{\bar{x}}\right)^2=15868387$ & $N-1=35$ & \\ \hline
	\end{tabular}
	\end{table}
	And we get from this table:
	
	What gives with Minitab 17.3.1 (we can see also that the model is probably not linear) as detailed in the Minitab companion book:
	\begin{figure}[H]
		\begin{center}
		\includegraphics[scale=0.6]{img/engineering/minitab_pressure_factor_anova.jpg}
		\end{center}	
	\end{figure}
	and visually:
	\begin{figure}[H]
		\begin{center}
		\includegraphics[scale=0.6]{img/engineering/minitab_pressure_factor_box_plot.jpg}
		\end{center}	
	\end{figure}

	As we can see here we reject the null hypothesis $H_0$ of this ANOVA at a threshold of $5\%$. Therefore this factor has a significant impact on our experience.\\
	
	Let us recall that the fact we reject the null hypothesis $H_0$ highlight an influential factor, the one who is controlled. So the basic model to consider (leaving aside the interactions) is not:
	
	but (the sum of the vectors still being equal to zero):
	
	Caution!!! That fact that $x_1$ (speed) and $x_2$ (place) have no significant influence does not result that the interactions between $x_1$ and $x_2$, or $x_2$ and $x_3$, or even $x_1x_2$ and $x_3$ are not. For example, here, $x_2x_3$ is has no significant influence (the Fischer-Snedecor test succeeds), but at the level $1$ of $x_3$, the interaction $x_1x_2$ has a significant influence (in the sens $x_1=x_2$ against $x_1\neq x_2$).\\

	Let us see this again with Minitab 17.3.1 (the calculations below can be detailed on request):
	\begin{figure}[H]
		\begin{center}
		\includegraphics[scale=0.9]{img/engineering/minitab_interactions_anova.jpg}
		\end{center}	
	\end{figure}
	We see that by doing a multifactorial ANOVA, that finally the factor \textit{Place} is statistically significant ($p$-value less than $5\%$), the factor \textit{Speed} is not ($p$-value greater than $5\%$), the pressure has a very significant influence (almost zero $p$-value). The interaction \textit{Place * Speed} is significant ($p$-value less than $5\%$). For cons, the interactions \textit{Speed * Pressure} and \textit{Place * Pressure} are not significant ($p$-value greater than $5\%$). The triple interaction \textit{Place * Speed * Pressure} is significant!\\
	
	So as we can see it, its not because a single given factor is not significant in the model without interaction that it will remain not significant when we take into account the interactions!!!\\
	\begin{tcolorbox}[title=Remark,colframe=black,arc=10pt]
	All the results that we have obtained above and also the charts can be obtained with a simple spreadhseet software like Microsoft Excel. So if the reader want we add the corresponding screenshots he must not hesitate, as always, to contact us.
	\end{tcolorbox}
	To close this subject, remember that we talked earlier about the interest to randomize the order of the measurement to eliminate unknown confounders. In reality, we must consider three important cases of harmful factors in practice and which will only change a little bit the used ANOVA and that most modern statistical software implement nowadays:
	\begin{enumerate}
		\item We have "\NewTerm{unknown harmful and uncontrollable factors}\index{unknown harmful and uncontrollable factors}". Therefore we make a usual ANOVA but simply where the order of measurement were randomized.

		\item We have "\NewTerm{known harmful factors but still uncontrollable}\index{known harmful factors but still uncontrollable}". We speak then of "\NewTerm{covariates}\index{covariates}" or "\NewTerm{cofactors}\index{cofactors}" and we use an ANCOVA (\SeeChapter{see section Statistics}) instead of a simple ANOVA

		\item We have "\NewTerm{known harmful factors and that under control}\index{known harmful factors and that under control}" that we want to eliminate from the ANOVA. The idea is to use a technique named "\NewTerm{blocking}\index{blocking}" it is simply a hierarchical ANOVA (\SeeChapter{see section Statistics}).
	\end{enumerate}
	
	\pagebreak
	\subsubsection{Taguchi Designs and Nomenclature (robust designs)}
	"\NewTerm{Taguchi designs}" also named "\NewTerm{robust designs}" or "\NewTerm{Taguchi Orthogonal Array Designs}" (abbreviated "Taguchi OA designs" in some softwares) are only a particular technique to find factorial or multifactorial fractional or full designs (from a matrix, a triangular table or from graph). In other words: design is a type of general fractional factorial design!
	
	\textbf{Definitions (\#\mydef):} An "\NewTerm{orthogonal array}\index{orthogonal array}"  of $s$ elements, denoted $\text{OA}_N(s^m)$, is a $N\times m$ matrix whose columns have the property that in every pair of columns each of the possible ordered pairs of elements appears the same number of time. Taguchi refers to $\text{OA}_N(s^m)$ by the notation $L_n(s^m)$ where $L$ stands for "Level".
	
	Below for example are given $\text{OA}_4(2^3)$ and $\text{OA}_8(2^7)$ respectively:
	\begin{table}[H]
		\newcolumntype{C}[1]{>{\centering\let\newline\\\arraybackslash\hspace{0pt}}m{#1}}
		\centering
		\label{my-label}
		\begin{tabular}{|
		>{\columncolor[HTML]{EFEFEF}}c |C{1cm}|C{1cm}|C{1cm}|}
		\hline
		\multicolumn{1}{|l|}{\cellcolor[HTML]{9B9B9B}} & \multicolumn{3}{c|}{\cellcolor[HTML]{9B9B9B}\textbf{Column number}} \\ \hline
		\multicolumn{1}{|l|}{\cellcolor[HTML]{9B9B9B}\textbf{Row number}} & \cellcolor[HTML]{EFEFEF}$\pmb{1}$ & \cellcolor[HTML]{EFEFEF}$\pmb{2}$ & \cellcolor[HTML]{EFEFEF}$\pmb{3}$ \\ \hline
		{\color[HTML]{333333} $\pmb{1}$} & $0$ & $0$ & $0$ \\ \hline
		{\color[HTML]{333333} $\pmb{2}$} & $0$ & $1$ & $1$ \\ \hline
		{\color[HTML]{333333} $\pmb{3}$} & $1$ & $0$ & $1$ \\ \hline
		{\color[HTML]{333333} $\pmb{4}$} & $1$ & $1$ & $0$ \\ \hline
		\end{tabular}
		\caption{$\text{OA}_4(2^3)$ Orthogonal array}
	\end{table}
	\begin{table}[H]
		\newcolumntype{C}[1]{>{\centering\let\newline\\\arraybackslash\hspace{0pt}}m{#1}}
		\centering
		\begin{tabular}{|
		>{\columncolor[HTML]{EFEFEF}}c |C{1cm}|C{1cm}|C{1cm}|C{1cm}|C{1cm}|C{1cm}|C{1cm}|}
		\hline
		\multicolumn{1}{|l|}{\cellcolor[HTML]{9B9B9B}} & \multicolumn{7}{c|}{\cellcolor[HTML]{9B9B9B}\textbf{Column number}} \\ \hline
		\multicolumn{1}{|l|}{\cellcolor[HTML]{9B9B9B}\textbf{Row number}} & \cellcolor[HTML]{EFEFEF}$\pmb{1}$ & \cellcolor[HTML]{EFEFEF}$\pmb{2}$ & \cellcolor[HTML]{EFEFEF}$\pmb{3}$ & \cellcolor[HTML]{EFEFEF}$\pmb{4}$ & \cellcolor[HTML]{EFEFEF}$\pmb{5}$ & \cellcolor[HTML]{EFEFEF}$\pmb{6}$ & \cellcolor[HTML]{EFEFEF}$\pmb{7}$\\ \hline
		{\color[HTML]{333333} $\pmb{1}$} & $0$ & $0$ & $0$ & $0$ & $0$ & $0$ & $0$\\ \hline
		{\color[HTML]{333333} $\pmb{2}$} & $0$ & $0$ & $0$ & $1$ & $1$ & $1$ & $1$ \\ \hline
		{\color[HTML]{333333} $\pmb{3}$} & $0$ & $1$ & $1$ & $0$ & $0$ & $1$ & $1$ \\ \hline
		{\color[HTML]{333333} $\pmb{4}$} & $0$ & $1$ & $1$ & $1$ & $1$ & $0$ & $0$ \\ \hline
		{\color[HTML]{333333} $\pmb{5}$} & $1$ & $0$ & $1$ & $0$ & $1$ & $0$ & $1$ \\ \hline
		{\color[HTML]{333333} $\pmb{6}$} & $1$ & $0$ & $1$ & $1$ & $0$ & $1$ & $0$ \\ \hline
		{\color[HTML]{333333} $\pmb{7}$} & $1$ & $1$ & $0$ & $0$ & $1$ & $1$ & $0$ \\ \hline
		{\color[HTML]{333333} $\pmb{8}$} & $1$ & $1$ & $0$ & $1$ & $0$ & $0$ & $1$ \\ \hline
		\end{tabular}
		\caption{$\text{OA}_8(2^7)$ Orthogonal array}
	\end{table}
	With softwares that automatically generate the corresponding designs, this technique has become a bit old fashioned but it had the advantage at the time of its use to provide a list of over a hundred of tables with factors at $2$ or more levels with or without interactions. Let us see some classics examples  for the general culture (as it is well to know and plus it's nice and still use in practice and is softwares!!!!).
	
	Dr. Genichi Taguchi has proposed to organized experiences following tables and graphs to identify interactions (the Taguchi tables include some Plackett-Burman designs) and afterwards to be able to found the settings for the controllable variables that minimize the variability transmitted to the response from the uncontrollable variable.
	
	In the Taguchi OA design, only the main effects and two-factor interactions are considered, and higher-order interactions are assumed to be nonexistent!!!! In addition, designers are asked to identify (based on their knowledge of the subject matter) which interactions might be significant before conducting the design.

	Let us start with an example of table that we know well, the $L_8$ table. By Taguchi this table can be read as the table of a fractional factorial design of $7$ factors (thus without interaction) or  as the table of a full factorial design for $3$ factors. This is why it is denoted $L_8(2^7)$ or $L_8(2^3)$ and represented by:
	\begin{table}[H]\centering
	\begin{center}
		\definecolor{gris}{gray}{0.85}
			\begin{tabular}{|c|c|c|c|c|c|c|c|}
				\hline
				\multicolumn{1}{c}{\cellcolor{black!30}\textbf{Trial N${}^\circ$}} & 
  \multicolumn{1}{c}{\cellcolor{black!30}$1$} & 
  \multicolumn{1}{c}{\cellcolor{black!30}$2$} & 
  \multicolumn{1}{c}{\cellcolor{black!30}$3$} & 
  \multicolumn{1}{c}{\cellcolor{black!30}$4$} & 
  \multicolumn{1}{c}{\cellcolor{black!30}$5$} & 
  \multicolumn{1}{c}{\cellcolor{black!30}$6$} & 
  \multicolumn{1}{c}{\cellcolor{black!30}$7$}\\ \hline
				 $1$ & $1$ & $1$ & $1$ & $1$ & $1$ & $1$ & $1$\\ \hline
				 $2$ & $1$ & $1$ & $1$ & $2$ & $2$ & $2$ & $2$\\ \hline
				$3$ & $1$ & $2$ & $2$ & $1$ & $1$ & $2$ & $2$\\ \hline
				$4$ & $1$ & $2$ & $2$ & $2$ & $2$ & $1$ & $1$\\ \hline
				$5$ & $2$ & $1$ & $2$ & $1$ & $2$ & $1$ & $2$\\ \hline
				$6$ & $2$ & $1$ & $2$ & $2$ & $1$ & $2$ & $1$\\ \hline
				$7$ & $2$ & $2$ & $1$ & $1$ & $2$ & $2$ & $1$\\ \hline
				$8$ & $2$ & $2$ & $1$ & $2$ & $1$ & $1$ & $2$\\ \hline
 		\end{tabular}
	\end{center}
	\caption{$L_8$ table with common factorial notation by Taguchi and associated graph}
	\end{table}
	\begin{figure}[H]
		\begin{center}
		\includegraphics{img/engineering/taguchi_L8.jpg}
		\end{center}	
	\end{figure}
	We'll see if this table really differs from what we already know. First, each factor also takes two levels in this table, we can replace the Taguchi notation with our usage notation for the factorial replacing $1$ by "$+$" and "$2$" by "$-$":
	\begin{table}[H]\centering
	\begin{center}
		\definecolor{gris}{gray}{0.85}
			\begin{tabular}{|c|c|c|c|c|c|c|c|}
				\hline
				\multicolumn{1}{c}{\cellcolor{black!30}\textbf{Trial N${}^\circ$}} & 
  \multicolumn{1}{c}{\cellcolor{black!30}$1$} & 
  \multicolumn{1}{c}{\cellcolor{black!30}$2$} & 
  \multicolumn{1}{c}{\cellcolor{black!30}$3$} & 
  \multicolumn{1}{c}{\cellcolor{black!30}$4$} & 
  \multicolumn{1}{c}{\cellcolor{black!30}$5$} & 
  \multicolumn{1}{c}{\cellcolor{black!30}$6$} & 
  \multicolumn{1}{c}{\cellcolor{black!30}$7$}\\ \hline
				$1$ & $+$ & $+$ & $+$ & $+$ & $+$ & $+$ & $+$\\ \hline
				$2$ & $+$ & $+$ & $+$ & $-$ & $-$ & $-$ & $-$\\ \hline
				$3$ & $+$ & $-$ & $-$ & $+$ & $+$ & $-$ & $-$\\ \hline
				$4$ & $+$ & $-$ & $-$ & $-$ & $-$ & $+$ & $+$\\ \hline
				$5$ & $-$ & $+$ & $-$ & $+$ & $-$ & $+$ & $-$\\ \hline
				$6$ & $-$ & $+$ & $-$ & $-$ & $+$ & $-$ & $+$\\ \hline
				$7$ & $-$ & $-$ & $+$ & $+$ & $-$ & $-$ & $+$\\ \hline
				$8$ & $-$ & $-$ & $+$ & $-$ & $+$ & $+$ & $-$\\ \hline
 		\end{tabular}
	\end{center}
	\caption{$L_8$ table with common notation}
	\end{table}
	We see already much better than all the rows and columns are orthogonal taken in pairs (Hadamard matrix). We add a column of "$+$":
	\begin{table}[H]\centering
	\begin{center}
		\definecolor{gris}{gray}{0.85}
			\begin{tabular}{|c|c|c|c|c|c|c|c|c|}
				\hline
				\multicolumn{1}{c}{\cellcolor{black!30}\textbf{Trial N${}^\circ$}} & 
  \multicolumn{1}{c}{\cellcolor{black!30}$1$} & 
  \multicolumn{1}{c}{\cellcolor{black!30}$2$} & 
  \multicolumn{1}{c}{\cellcolor{black!30}$3$} & 
  \multicolumn{1}{c}{\cellcolor{black!30}$4$} & 
  \multicolumn{1}{c}{\cellcolor{black!30}$5$} & 
  \multicolumn{1}{c}{\cellcolor{black!30}$6$} & 
  \multicolumn{1}{c}{\cellcolor{black!30}$7$} & \multicolumn{1}{c}{\cellcolor{black!30}\textbf{Rest}} \\ \hline
				$1$ & $+$ & $+$ & $+$ & $+$ & $+$ & $+$ & $+$ & $+$\\ \hline
				$2$ & $+$ & $+$ & $+$ & $-$ & $-$ & $-$ & $-$ & $+$\\ \hline
				$3$ & $+$ & $-$ & $-$ & $+$ & $+$ & $-$ & $-$ & $+$\\ \hline
				$4$ & $+$ & $-$ & $-$ & $-$ & $-$ & $+$ & $+$ & $+$\\ \hline
				$5$ & $-$ & $+$ & $-$ & $+$ & $-$ & $+$ & $-$ & $+$\\ \hline
				$6$ & $-$ & $+$ & $-$ & $-$ & $+$ & $-$ & $+$ & $+$\\ \hline
				$7$ & $-$ & $-$ & $+$ & $+$ & $-$ & $-$ & $+$ & $+$\\ \hline
				$8$ & $-$ & $-$ & $+$ & $-$ & $+$ & $+$ & $-$ & $+$\\ \hline
 		\end{tabular}
	\end{center}
	\caption{$L_8$ table with common notation}
	\end{table}
	and we notice that we fall back on the full factorial design with $3$ factors at $2$ levels if we rewrite the title row of the columns as follows:
	\begin{table}[H]\centering
	\begin{center}
		\definecolor{gris}{gray}{0.85}
			\begin{tabular}{|c|c|c|c|c|c|c|c|c|}
				\hline
				\multicolumn{1}{c}{\cellcolor{black!30}\textbf{Trial N${}^\circ$}} & 
  \multicolumn{1}{c}{\cellcolor{black!30}$F3$} & 
  \multicolumn{1}{c}{\cellcolor{black!30}$F2$} & 
  \multicolumn{1}{c}{\cellcolor{black!30}$F23$} & 
  \multicolumn{1}{c}{\cellcolor{black!30}$F1$} & 
  \multicolumn{1}{c}{\cellcolor{black!30}$F13$} & 
  \multicolumn{1}{c}{\cellcolor{black!30}$F12$} & 
  \multicolumn{1}{c}{\cellcolor{black!30}$F123$} & \multicolumn{1}{c}{\cellcolor{black!30}\textbf{Rest}} \\ \hline
				$1$ & $+$ & $+$ & $+$ & $+$ & $+$ & $+$ & $+$ & $+$\\ \hline
				$2$ & $+$ & $+$ & $+$ & $-$ & $-$ & $-$ & $-$ & $+$\\ \hline
				$3$ & $+$ & $-$ & $-$ & $+$ & $+$ & $-$ & $-$ & $+$\\ \hline
				$4$ & $+$ & $-$ & $-$ & $-$ & $-$ & $+$ & $+$ & $+$\\ \hline
				$5$ & $-$ & $+$ & $-$ & $+$ & $-$ & $+$ & $-$ & $+$\\ \hline
				$6$ & $-$ & $+$ & $-$ & $-$ & $+$ & $-$ & $+$ & $+$\\ \hline
				$7$ & $-$ & $-$ & $+$ & $+$ & $-$ & $-$ & $+$ & $+$\\ \hline
				$8$ & $-$ & $-$ & $+$ & $-$ & $+$ & $+$ & $-$ & $+$\\ \hline
 		\end{tabular}
	\end{center}
	\caption[]{Previous table with traditional notation for the title row}
	\end{table}
	and that this also corresponds to the fractional factorial design of a $7$-factor experimental design at $2$ levels (without interactions).
	
	The question that will obviously ask the reader to himself is how we have would have identify what factor belonged to which column in the context of a full factorial design with $3$ factors if we did not know the table prepared with the techniques view previously? Well  in fact just by using the following linear graph:
	\begin{figure}[H]
		\begin{center}
		\includegraphics{img/engineering/taguchi_L8_linear_graph.jpg}
		\caption{Taguchi L8 linear graph}
		\end{center}	
	\end{figure}
	that indicates the main factors in the vertices of the graph (therefore the columns $1$, $2$ and $4$ are the main factors). The column $3$ is the interaction between the columns $1$ and $2$ (hence the fact that it is on the edge between the two vertices), the column $5$ is the interaction between the columns $1$ and $4$ (hence the fact that it is on the edge between the two vertices), and column $6$ is the interaction between the columns $2$ and $4$ (hence the fact that it is on the edge between the two vertices). The fact that $7$ is outside the linear graph outside is because a triple interaction can not be represented with the technique of planar graph. We then used the symbols of vertices to signify that this is the superposition of the vertices $1$, $2$ and $4$ (circle + ring + dics).

	The second linear graph associated with this table permits to highlight another potential use:
	\begin{figure}[H]
		\begin{center}
		\includegraphics{img/engineering/taguchi_L8_linear_graph_complement.jpg}
		\end{center}	
	\end{figure}
	That is to say, to make use of it for an analysis of $4$ factors (always with two modality in this case) represented by columns $1$, $2$, $4$, $7$ (the reader can indeed check that these four columns correspond to a fractional factorial design with for generator \textbf{12} for $4$ factors either by hand or with software) with $3$ interactions ($1$ and $2$; $1$ and $4$, $1$ and $7$).
	
	\begin{tcolorbox}[title=Remark,colframe=black,arc=10pt]
	The practitioner must be careful because there are therefore two Taguchi tables therefore denoted $L_8$: one that is a full table for $3$ factors to $2$ levels and one for $7$ factors at $2$ levels without interactions as we have just seen (so with $7$ columns), but there also tables $L_8$ for $5$ factors splitten into $4$ factors at $3$ levels and $1$ factor to $4$ levels (without interactions) but only with $5$ columns. This is why books listing the tables Taguchi normally explicitly specify the scope of application of the given tables.
	\end{tcolorbox}
	Let's see some other tables (a small part of the complete list):
	\begin{itemize}
		\item Table $L_4 (2^3)$:
		\begin{table}[H]\centering
			\begin{center}
				\definecolor{gris}{gray}{0.85}
					\begin{tabular}{|c|c|c|c|}
						\hline
						\multicolumn{1}{c}{\cellcolor{black!30}\textbf{Trial N${}^\circ$}} & 
		  \multicolumn{1}{c}{\cellcolor{black!30}$F1$} & 
		  \multicolumn{1}{c}{\cellcolor{black!30}$F2$} & 
		  \multicolumn{1}{c}{\cellcolor{black!30}$F3$} \\ \hline
						$1$ & $1$ & $1$ & $1$\\ \hline
						$2$ & $1$ & $2$ & $2$\\ \hline
						$3$ & $2$ & $1$ & $2$\\ \hline
						$4$ & $2$ & $2$ & $1$\\ \hline
		 		\end{tabular}
			\end{center}
			\caption{Taguchi $L_4$ table}
		\end{table}
		\begin{figure}[H]
			\begin{center}
			\includegraphics{img/engineering/taguchi_L4.jpg}
			\end{center}	
		\end{figure}
		
		\item Table $L_9 (3^4)$:
		\begin{table}[H]\centering
			\begin{center}
			\definecolor{gris}{gray}{0.85}
			\begin{tabular}{|c|c|c|c|c|}
			\hline
			\multicolumn{1}{c}{\cellcolor{black!30}\textbf{Trial N${}^\circ$}} & 
			\multicolumn{1}{c}{\cellcolor{black!30}$F1$} & 
			\multicolumn{1}{c}{\cellcolor{black!30}$F2$} & 
			\multicolumn{1}{c}{\cellcolor{black!30}$F3$} & 
			\multicolumn{1}{c}{\cellcolor{black!30}$F4$} \\ \hline
		  	$1$ & $1$ & $1$ & $1$ & $1$\\ \hline
		  	$2$ & $1$ & $2$ & $2$ & $2$\\ \hline
		  	$3$ & $1$ & $3$ & $3$ & $3$\\ \hline
		  	$4$ & $2$ & $1$ & $2$ & $3$\\ \hline
		  	$5$ & $2$ & $2$ & $3$ & $1$\\ \hline
		  	$6$ & $2$ & $3$ & $1$ & $2$\\ \hline
		  	$7$ & $3$ & $1$ & $3$ & $2$\\ \hline
		  	$8$ & $3$ & $2$ & $1$ & $3$\\ \hline
		  	$9$ & $3$ & $3$ & $2$ & $1$\\ \hline
		 		\end{tabular}
			\end{center}
			\caption{Taguchi $L_9$ table}
		\end{table}
		\begin{figure}[H]
			\begin{center}
			\includegraphics{img/engineering/taguchi_L9.jpg}
			\end{center}	
		\end{figure}
		
		\item $L_{16} (2^{15})$:
		\begin{table}[H]\centering
			\begin{center}
			\definecolor{gris}{gray}{0.85}
			\begin{tabular}{|c|c|c|c|c|c|c|c|c|c|c|c|c|c|c|c|}
			\hline
			\multicolumn{1}{c}{\cellcolor{black!30}\textbf{Trial N${}^\circ$}} & 
			\multicolumn{1}{c}{\cellcolor{black!30}$F1$} & 
			\multicolumn{1}{c}{\cellcolor{black!30}$F2$} & 
			\multicolumn{1}{c}{\cellcolor{black!30}$F3$} & 
			\multicolumn{1}{c}{\cellcolor{black!30}$F4$} & 
			\multicolumn{1}{c}{\cellcolor{black!30}$F5$} & 
			\multicolumn{1}{c}{\cellcolor{black!30}$F6$} & 
			\multicolumn{1}{c}{\cellcolor{black!30}$F7$} & 
			\multicolumn{1}{c}{\cellcolor{black!30}$F8$} & 
			\multicolumn{1}{c}{\cellcolor{black!30}$F9$} & 
			\multicolumn{1}{c}{\cellcolor{black!30}$F10$} & 
			\multicolumn{1}{c}{\cellcolor{black!30}$F11$} & 
			\multicolumn{1}{c}{\cellcolor{black!30}$F12$} & 
			\multicolumn{1}{c}{\cellcolor{black!30}$F13$} & 
			\multicolumn{1}{c}{\cellcolor{black!30}$F14$} & 
			\multicolumn{1}{c}{\cellcolor{black!30}$F15$}\\ \hline
		  	$1$ & $1$ & $1$ & $1$ & $1$ & $1$ & $1$ & $1$ & $1$ & $1$ & $1$ & $1$ & $1$ & $1$ & $1$ & $1$\\ \hline
			$2$ & $1$ & $1$ & $1$ & $1$ & $1$ & $1$ & $1$ & $2$ & $2$ & $2$ & $2$ & $2$ & $2$ & $2$ & $2$\\ \hline
			$3$ & $1$ & $1$ & $1$ & $2$ & $2$ & $2$ & $2$ & $1$ & $1$ & $1$ & $1$ & $2$ & $2$ & $2$ & $2$\\ \hline	
			$4$ & $1$ & $1$ & $1$ & $2$ & $2$ & $2$ & $2$ & $2$ & $2$ & $2$ & $2$ & $1$ & $1$ & $1$ & $1$\\ \hline	
			$5$ & $1$ & $2$ & $2$ & $1$ & $1$ & $2$ & $2$ & $1$ & $1$ & $2$ & $2$ & $1$ & $1$ & $2$ & $2$\\ \hline	
			$6$ & $1$ & $2$ & $2$ & $1$ & $1$ & $2$ & $2$ & $2$ & $2$ & $1$ & $1$ & $2$ & $2$ & $2$ & $2$\\ \hline	
			$7$ & $1$ & $2$ & $2$ & $2$ & $2$ & $1$ & $1$ & $1$ & $1$ & $2$ & $2$ & $2$ & $2$ & $1$ & $1$\\ \hline
			$8$ & $1$ & $2$ & $2$ & $2$ & $2$ & $1$ & $1$ & $2$ & $2$ & $1$ & $1$ & $1$ & $1$ & $2$ & $2$\\ \hline
			$9$ & $2$ & $1$ & $2$ & $1$ & $2$ & $1$ & $2$ & $1$ & $2$ & $1$ & $2$ & $1$ & $2$ & $1$ & $2$\\ \hline
			$10$ & $2$ & $1$ & $2$ & $1$ & $2$ & $2$ & $1$ & $2$ & $1$ & $2$ & $1$ & $2$ & $1$ & $2$ & $1$\\ \hline
			$11$ & $2$ & $1$ & $2$ & $2$ & $1$ & $2$ & $1$ & $1$ & $2$ & $1$ & $2$ & $2$ & $1$ & $2$ & $1$\\ \hline
			$12$ & $2$ & $1$ & $2$ & $2$ & $1$ & $2$ & $1$ & $2$ & $1$ & $2$ & $1$ & $1$ & $2$ & $1$ & $2$\\ \hline
			$13$ & $2$ & $2$ & $1$ & $1$ & $2$ & $2$ & $1$ & $1$ & $2$ & $2$ & $1$ & $1$ & $2$ & $2$ & $1$\\ \hline
			$14$ & $2$ & $2$ & $1$ & $1$ & $2$ & $2$ & $1$ & $2$ & $1$ & $1$ & $2$ & $2$ & $1$ & $1$ & $2$\\ \hline
			$15$ & $2$ & $2$ & $1$ & $2$ & $1$ & $1$ & $2$ & $1$ & $2$ & $2$ & $1$ & $2$ & $1$ & $1$ & $2$\\ \hline
			$16$ & $2$ & $2$ & $1$ & $2$ & $1$ & $1$ & $2$ & $2$ & $1$ & $1$ & $2$ & $1$ & $2$ & $2$ & $1$\\ \hline
		 	\end{tabular}
			\end{center}
			\caption{Taguchi $L_{16}$ table}
		\end{table}
		\begin{figure}[H]
			\begin{center}
			\includegraphics{img/engineering/taguchi_L16.jpg}
			\end{center}	
		\end{figure}
		
		\item Table $L_{16} (4^5)$:
		\begin{table}[H]\centering
			\begin{center}
			\definecolor{gris}{gray}{0.85}
			\begin{tabular}{|c|c|c|c|c|c|}
			\hline
			\multicolumn{1}{c}{\cellcolor{black!30}\textbf{Trial N${}^\circ$}} & 
			\multicolumn{1}{c}{\cellcolor{black!30}$F1$} & 
			\multicolumn{1}{c}{\cellcolor{black!30}$F2$} & 
			\multicolumn{1}{c}{\cellcolor{black!30}$F3$} & 
			\multicolumn{1}{c}{\cellcolor{black!30}$F4$} & 
			\multicolumn{1}{c}{\cellcolor{black!30}$F5$} \\ \hline
		  	$1$ & $1$ & $1$ & $1$ & $1$ & $1$\\ \hline
		  	$2$ & $1$ & $2$ & $2$ & $2$ & $2$\\ \hline
		  	$3$ & $1$ & $3$ & $3$ & $3$ & $3$\\ \hline
		  	$4$ & $1$ & $4$ & $4$ & $4$ & $4$\\ \hline
		  	$5$ & $2$ & $1$ & $2$ & $3$ & $4$\\ \hline
		  	$6$ & $2$ & $2$ & $1$ & $4$ & $3$\\ \hline
		  	$7$ & $2$ & $3$ & $4$ & $1$ & $2$\\ \hline
		  	$8$ & $2$ & $4$ & $3$ & $2$ & $1$\\ \hline
		  	$9$ & $3$ & $1$ & $3$ & $4$ & $2$\\ \hline
		  	$10$ & $3$ & $2$ & $4$ & $3$ & $1$\\ \hline
		  	$11$ & $3$ & $3$ & $1$ & $2$ & $4$\\ \hline
		  	$12$ & $3$ & $4$ & $2$ & $1$ & $3$\\ \hline
		  	$13$ & $4$ & $1$ & $4$ & $2$ & $3$\\ \hline
		  	$14$ & $4$ & $2$ & $3$ & $1$ & $4$\\ \hline
		  	$15$ & $4$ & $3$ & $2$ & $4$ & $1$\\ \hline
		  	$16$ & $4$ & $4$ & $1$ & $3$ & $1$\\ \hline
		 		\end{tabular}
			\end{center}
			\caption{Taguchi $L_{16}$ table}
		\end{table}		
		\begin{figure}[H]
			\begin{center}
			\includegraphics{img/engineering/taguchi_L16_4.jpg}
			\end{center}	
		\end{figure}
		
		\item etc.
	\end{itemize}
	The Taguchi tables together with their graphs are therefore full or fractional factorial designs with all their advantages. Taguchi's merit is to have tried to simplify the use of factorial designs to make them accessible to a large number of experimenters. 
	
	Taguchi's designs are usually highly fractionated, which makes them very attractive to practitioners. Doing a half-fraction, quarter-fraction or eighth-fraction of a full factorial design greatly reduces costs and time needed for a designed experiment. The drawback of a fractionated design is that some interactions may be confounded with other effects. It is important to consider carefully the role of potential confounders and aliases. Failure to take account of such confounded effects can result in erroneous conclusions and misunderstandings.

	When using a Taguchi design, one needs to guess which interactions are most likely to be significant - even before any experiment is performed.
	
	The Taguchi methodology is say to have generated considerable debate and controversy. Part of the controversy arose because Taguchi's methodology was advocated in the West initially (and primarily) by entrepreneurs, and the underlying statistical science had not been adequately peer-reviewed. By the late 1980s the results of a very comprehensive peer review indicated that although Taguchi's engineering concepts and the overall objective Taguchi's robust parameter design were well-founded\cite{co2008confirmation}, there were substantial problems with his experimental strategy (maximizing signal to noise does not necessarily minimize variability...) and methods of data analysis\cite{kacker1991taguchi}.
	\begin{tcolorbox}[title=Remark,colframe=black,arc=10pt]
	For example to see how this is vicious... Taguchi's catalog contains $20$ arrays. However, only $18$ of these arrays are in reality orthogonal arrays...
	\end{tcolorbox}

	\pagebreak
	Finally let summarize what we have seen so far with the respective names (this list is to take with precaution as there is a priori no clear international standard to our knowledge regarding these definitions) in the order of mathematical generalization:
	\begin{itemize}
		\item Designs that contain only factors at two levels, are referred to as "\NewTerm{factorial design}\index{factorial design}" that they are based on a linear or nonlinear model, additive or not.

		\item The factorial designs that allow only to determine the coefficients of the main factors (ie additive model: without interactions) are referred to as "\NewTerm{Koshal designs}\index{Koshal designs}" or "\NewTerm{screening designs}\index{screening designs}" (also named "\NewTerm{one shot at a time designs}\index{one shot at a time designs}" because as the technique of "one at a time" it can't analyze any interaction but the comparison stops there!).

		\item The factorial designs where all the interactions of three and higher order are neglected are referred to as "\NewTerm{Rechtschaffner designs}\index{Rechtschaffner designs}".

		\item The factorial designs where the coefficients are aliased while keeping interactions are referred to as "\NewTerm{fractional factorial designs}\index{fractional factorial designs}" or of "\NewTerm{Box and Hunter designs}\index{Box and Hunter designs}" or even under the name of "\NewTerm{matrix Hadamard based designs}\index{matrix Hadamard based designs}". With fractional factorial designs, it should also be specify the method of resolution.

		\item The factorial designs whose order is a multiple of $4$ but not a power of two (ie $12$, $16$, $20$, $24$, etc.) are designated under the name of "\NewTerm{Plackett-Burman designs}\index{Plackett-Burman design}"

		\item The designs with any number of factors, of interactions of any order and any number of levels are referred to as "\NewTerm{Fisher designs}\index{Fisher designs}" (in honor of to the original creator of the concept of experimental design ).

		\item The designs which contain as many tests  as the number of coefficients to be determined are designated under the name of "\NewTerm{saturated designs}\index{saturated designs}".

		\item The designs which contain fewer tests than  the number coefficients to be determined are referred to as "\NewTerm{oversaturated designs}\index{oversaturated designs}".
	\end{itemize}
	One aspect still needs to be clarified: it is the verification of the validity of the mathematical model of the first degree. None of these designs provides, as far as we know, such a validity test using elaborated statistics. This is why it is recommended to always add at least one test point at the center of the experimental range. The value of the response at this point will be compared to the deduced value of the other data points through mathematical model. If both values are similar, the mathematical model will be adopted if they are not we will have to reject this model and complete the results already obtained by experiments by going through a model of the second degree.

	
	Finally if we have one response variable for the problem to take up for DOE then we have the following situations (options):
	\begin{enumerate}
		\item The multiple responses are independent. Then we do our job as usually (as optimizing one under the independence assumption must not change the other responses)

		\item The multiple responses are not independent. The we add each model as a weighted sum and we optimize that latter (Overall Evaluation Criterion method as suggested in Taguchi designs) with any solver.

		\item We use $2$ stage least squares regression techniques (\SeeChapter{see section Theoretical Computing}) and after we optimize the whole with a solver.				
	\end{enumerate}
	
	
	
	\pagebreak
	\subsubsection{Response Surface Methodology (Box Domains)}
	In statistics, "\NewTerm{response surface methodology (RSM)}\index{response surface methodology}" explores the relationships between several explanatory variables and one or more response variables. The method was introduced by G. E. P. Box and K. B. Wilson in 1951. The main idea of RSM is to use a sequence of designed experiments to obtain an optimal response. Box and Wilson suggest using a second-degree polynomial model to do this. They acknowledge that this model is only an approximation, but use it because such a model is easy to estimate and apply, even when little is known about the process.
	
	An easy way to estimate a first-degree polynomial model is as we have seen to use a factorial experiment or a fractional factorial design based in the two variables case for recall by:
	
	This is sufficient to determine which explanatory variables affect the response variable(s) of interest. Once it is suspected that only significant explanatory variables are left, then a more complicated design, such as a central composite design can be implemented to estimate a second-degree polynomial model, which is still only an approximation at best given by:
			
	However the second-degree model can be used to optimize the process (maximize, minimize, or attain a specific target for).
	
	Obviously in the case of two variables, we speak of "response surface" (beyond we speak of a "volume" or "hypervolume") and then the previous relation will be written:
	
	The designs including the following three terms:
	
	are named "\NewTerm{Doehlert's composite designs}\index{Doehlert's composite designs}". We will have then to determine $4$ terms:
	\begin{itemize}
		\item $a_0$: the constant term
		\item $a_i$: term of first degree
		\item $a_{ij}$: rectangle term
		\item $a_{ii}$: quadratic term
	\end{itemize}
	We see that in the case of a $2$-factor design, we need then to estimate $6$ coefficients:
	
 	and for $3$ factors we will have to estimate $10$ coefficients:
	
	Therefore after trial and error we found that a Doehlert design has a cost (number of runs) that can be simply written as:
	
	where $k$ is the number of factors and $N_0$ the number of additional points.
	
	The idea often adopted to pass to a model of the second degree after a factorial study while retaining the tests already carried out is simply to complete the existing design by the tests for the estimation of the model of higher degree when possible.
	\begin{tcolorbox}[colframe=black,colback=white,sharp corners]
	\textbf{{\Large \ding{45}}Example:}\\\\
	For example, a response surface associated with a relation like the previous one looks like with Maple 4.00b:\\
	
	\texttt{>5+3*x1+2*x2+4*x1*x2+2.5*x1\string^2+3*x2\string^2,x1=-10..10,x2=-10..10,\\
view=[-10..10,-10..10,0..50],contours=10,style=PATCHCONTOUR,\\
		axes=frame,numpoints=10000);
		}
	\begin{figure}[H]
		\begin{center}
		\includegraphics[scale=0.65]{img/engineering/quadratic_response_surface_design.jpg}
		\end{center}	
		\caption{Generic example of interaction for two factors quadratic polynomial}
	\end{figure}
	\end{tcolorbox}
	
	\pagebreak
	\paragraph{Pure quadratic curvature test}\mbox{}\\\\
	In running a two-level factorial experiment, we usually anticipate fitting the first order model, but we should be alert to the possibility that the second order model (response surface methodology) is really more appropriate. Therefore before running response surface will should be aware of any curvature assumption!
	
	 There is a method for replicating certain points in a $2^k$ factorial that will provide protection against curvature from second-order effects as well as allow an independent estimate of error to be obtained. The method consist of adding center points to the $2^k$ design. These consist of $n$ replicates run at the points $x_i=0$. One important reason for adding the replicate runs at the design center is that center points do not affect the usual effect estimates in a $2^k$ design.
	
	To illustrate the approach, consider $2^2$ design, with one observation at each of the factorial points $(-,-)$, $(+,-)$, $(-,+)$, and $(+,+)$ and $n_c$ observations at the center point $(0,0)$. Let $\bar{y}_F$ be the average of the four runs at the four factorial points and let $\bar{y}_C$ be the average of the $n_c$ runs at the center point. If the difference:
	
	is small then the center points lie on or near the plane passing through the factorial points, and there is no quadratic curvature. On the other hand, if this same difference is large, then quadratic curvature is present. A single-degree-of-freed sum of squares for pure quadratic curvature is given by (see proof just below):
	
	where in general, $n_F$ is the number of factorial design points. This quantity may be compared to the error mean square to test for pure quadratic curvature. More specifically, when point are added to the center of the $2^k$ design, then the test for curvature actually test the hypotheses the coefficients of the quadratic terms are all non zero):
	
	Or more explicitly
	
	First let us prove that:
	
	Indeed, let us recall the full model:
	
	And let us rewrite if for the for the $(x_1,x_2)$ combinations $(-1,+1)$, $(-1,-1)$, $(+1,+1)$, $(+1,-1)$:
	
	We see immediately that the sum gives:
	
	Therefore the average is (the sum divided by the number of corresponding measurement points):
	
	and in general:
	
	It is immediate that for the center point $(0,0)$ we have:
	
	Therefore:
	
	Now the variance of $\bar{y}_F-\bar{y}_C$ is under the assumption of equality of variances and independence:
	
	Consequently a hypotheses can be conducted using the homoscedastic Student T-test (\SeeChapter{see section Statistics}):
	
	with $n=n_F+n_C$.
	
	As we have proved in the section Statistics, we have $T_{n-2}^2=F_{1,n-2}$ (it is convenient in statistical packages to put the curvature test in the ANOVA table where all other tests are based on Fisher distribution). Therefore:
	
	and therefore:
	
	This finish our proof of the "\NewTerm{pure quadratic curvature test}\index{pure quadratic curvature test}" and notice that this do not give us the factors that are responsible of the curvature!
	
	The reader can found an application example of this test in our Minitab companion book.
	
	\pagebreak
	\paragraph{Box-Wilson Central Composite Designs}\mbox{}\\\\
	Let us recall that so far we have seen that according to the principle of analysis of factorial experimental designs that in the 1D case (a single explanatory variable), the interpolated point must have a left measurement point ($-1$) and a right measurement point ($+1$). In the 2D case (two explanatory variables) it is natural to think then to have at least $3$ points around the interpolated point  such that we have a triangle whose barycenter is the point to be interpolated. Identically in the 3D case (three explanatory variables) it is natural to think of having at least $4$ points surrounding the point to be interpolated and forming a cube or tetrahedron as needed. Thus, by generalizing the reasoning, in an $n$-dimensional space we will need at least $n + 1$ points to construct a simplex around the point to be interpolated.

	Now let us recall that we saw in the section of Theoretical Computing different techniques of linear regression and we also studied the fact that a polynomial could be analyzed with a simple linear regression. We obtained in the same section (during the study of the factor of inflation of the variance) the following relation with its abusive notation ... dangerously confusing matrix of covariance and simple variance:
	
	with for recall:
	
	We have therefore to obtain a scalar from the explanatory values:
	
 	And therefore:
	
	Experts in the field routinely write the latter relation in the following form and name it "\NewTerm{variance function}\index{variance function}":
	
	The idea now is to come back to our experimental designs is therefore to choose our measurement points which minimizes the standard deviation of the predictions introduced just previously!

	Let us consider as an example the problem of finding a linear regression to the polynomial:
	
	in the square $-1\leq x_1 \leq 1$,$-1\leq x_2\leq 1$ and let us compare the maximum value of the predictive variance for the case of a complete factorial design (ie a square in the plane) and then for a fractional factorial design (ie a triangle in the plane) that omits the point $(1,1)$.
	
	In the first case of the complete factorial design, we have therefore (\SeeChapter{see section Linear Algebra page \pageref{transposed matrix} and page \pageref{determinant of three by three matrix}}):
	
	and therefore:
	
	What is written more frequently:
	
	where $R$ is the distance (radius) to the central point. When a predictive variance depends only on the distance $R$, we say that we have a "\NewTerm{rotation invariant experimental design}\index{rotation invariant experimental design}" or simply a "\NewTerm{rotational experimental design}\index{rotational experimental design}" (the variance function is then say to be "spherical"). This type of plan is attractive to practitioners who do not know a priori where in the space of experimentation the most precise forecasts must be found!

	Therefore:
	
	The variance has then for value at the point of origin $(0,0)$:
	
	And:
	
 	at the ends. This case thus shows a small variation between the origin and the extremities (a variation exactly of $\sqrt{3}$).

	In the second case of the fractional factorial design we have:
	
	and therefore:
	
	Therefore:
	
	So first this design is by definition not invariant by rotation since the variance does not depend only on $R$!

	The variance corresponding to the point of origin $(0,0)$ is then:
	
 	and at the end $(1,1)$ we have:
	
 	If we now compute by the partial derivatives the coordinates which minimizes the error such that:
	
 	We get:
	
 	Hence:
	
	and therefore:
	
	Therefore it comes:
	
	is the point of minimum variance and at this point (which coincides with the barycenter of the triangle, that is to say the center of the inscribed circle), we have:
	
 	From this example, we can already easily conclude that for a complete factorial design, the variance of the predictive error is written:
	
 	and as well as the maximum error is immediate and is equal in the case of a complete factorial design:
	
 	Thus, the quality of the fit (regression) becomes very good when $n$ grows!

	Let us now consider now the fractional factorial design based on three points in the plane forming an equilateral triangle:
	
	Similar calculations to before give us (as always on readers request we can detail!):
	
	Thus, we notice that, by judiciously choosing the points of the design, we may have a lower prediction error than with a complete factorial design both at the origin and at the points $(1, 1)$, $(-1,1)$, $(1, -1)$ and $(-1, -1)$. However the price to pay is that we have to make measurements outside the unit square which complicates the task of the practitioner.
	
	
	
	\subparagraph{Circumscribed Center Designs}\mbox{}\\\\
	The "\NewTerm{Composite Center Design (CCD)}\index{composite center design}" is a two-level experimental design for surface planes. In the 2D case (two explanatory variables) this is equivalent to taking the usual points of the squares to which we add the points of intersection of the axes with the circumscribed circle (+ in practice one or more central control points):	
	\begin{figure}[H]
		\centering
		\includegraphics[scale=1]{img/engineering/composite_center_design_2d.jpg}	
		\caption{2D Composite Center Design}
	\end{figure}
	We thus have the following points, which are all at the same distance from the origin of the axes:
	
	where of course the root of $2$ comes from the fact that the radius of the circle is given by the vertices of the square and thus by stupidly applying Pythagoras:
	
 	Due to the fact that in the above case the circle (in 2D) or the sphere (in 3D) is circumscribed to the square (or the cube in the 3D case) we also talk about CCC design for "\NewTerm{Circumscribed Center Design (CCD)}\index{circumscribed center design}".

	Let's take the example of the CCD case where the columns in the matrix below, the first is filled only with $1$, the other columns corresponding respectively to the terms: $x_1$,$x_2$,$x_1x_2$, $x_1^2$, $x_2^2$ (notice that not all columns are orthogonal!):
	
	where the last row correspond to the fact of adding a control point a the center of the domain. This empirical matrix is sometimes named "\NewTerm{composite equiradial matrix}\index{equiradial}", it request as we see a strong symmetry and therefore implies what we named an "\NewTerm{isovariance}\index{isovariance}" (see plots with Maple further below of the variance).
	
	We also see that the factors have $5$ levels $(-1,1,-\sqrt{2},+\sqrt{2},0)$. This is a characteristic of CCD designs!
	
	Therefore:
	
	Using a CAD software we get:
	
	Then we have:
	
	We therefore have here an experimental design whose variance depends indeed only on the distance $R$ from the origin as many textbooks state but without proof.

	Also at the opposite of $99\%$ of textbooks we will give the Maple 4.00b code to plot the famous variance corresponding figure:

	\texttt{>contourplot3d(sqrt(1-7/8*(x\string^2+y\string^2)+11/32*(x\string^2+y\string^2)\string^2),\\x=-sqrt(2)..sqrt(2),y=-sqrt(2)..sqrt(2),contours=10,filled=true);}
	
	We have:
	\begin{figure}[H]
		\centering
		\includegraphics[scale=1]{img/engineering/composite_center_design_2d_variance_plot_perspective.jpg}	
		\caption{2D Composite Center Design variance perspective plot in Maple 4.00b}
	\end{figure}
	Either seen from above we get the corresponding famous chart available in many textbooks regarding to the rotational invariance property of surface design with two factors:
	\begin{figure}[H]
		\centering
		\includegraphics[scale=1]{img/engineering/composite_center_design_2d_variance_plot_top.jpg}	
		\caption{2D Composite Center Design variance top plot in Maple 4.00b}
	\end{figure}
	We can see that CCD have a number cost (run) of:
	
	where $k$ is the number of factors, $r$ the reduction integer and $N_0$ the number of center points. Therefore the three terms can be read as following:
	\begin{itemize}
		\item $2^{k-r}$: Common full factorial of fraction factorial design of the  composite equiradial matrix.

		\item $2k$: Number of star points

		\item $N_0$: Number of center points
	\end{itemize}
	Therefore for a typical two factors design with a full factorial subset we have:
	
	and as $5$ is recommended for $N_0$, it comes that $N_{\text{CCD}}=13$ for a two factor response surface design with $5$ center points.
	
	\pagebreak
	\subparagraph{Face Centered Designs}\mbox{}\\\\
	Repeating the same maneuver but with a experience design said to be "\NewTerm{face-centered design}\index{face-centered design}" such as:
	
	Then we have:
	
	And:
	
	We then have after:
	
	Notice that the Circumscribed Center Design (CCC) was a rotatable design but that the Face Centered Design (CCF) is not as it does not depends only on $R$.
	
	Which give us with Maple 4.00b by proceeding exactly with the same above given code:
	We have:
	\begin{figure}[H]
		\centering
		\includegraphics[scale=1]{img/engineering/face_centered_design_2d_variance_plot_perspective.jpg}	
		\caption{2D Faced-centered Design variance perspective plot in Maple 4.00b}
	\end{figure}
	Either seen from above we get the corresponding famous chart available in many textbooks:
	\begin{figure}[H]
		\centering
		\includegraphics[scale=1]{img/engineering/face_centered_design_2d_variance_plot_top.jpg}	
		\caption{2D Faced-centered Design variance top plot in Maple 4.00b}
	\end{figure}
	The two previous designs are frequently summarized with the following figure:
	\begin{figure}[H]
		\centering
		\includegraphics[scale=1]{img/engineering/ccd_fcd_summary.jpg}	
	\end{figure}
	In fact there are three Composite Central Design:
	\begin{enumerate}
		\item Circumscribed (CCD)
		\item Inscribed (ICD)
		\item Face centered (FCD)
	\end{enumerate}
	Response surface methodology is obviously often applied to pilot plant operations by research and development personnel but when it is applied to a full-scale production process, it is usually only done once (or very infrequently) because the experimental procedure is relatively elaborate. However, conditions that were optimum for the pilot plant may not be optimum for the full-scale process. Indeed, the operation "scale-up" of the pilot plant to the full-scale production process usually results in distortion of the optimum conditions. Even if the full-scale plant begins operation at the optimum, it will eventually' 'drift" away from that point because of variations in raw materials, environmental changes, and operating personnel.

	A method is needed for the continuous monitoring and improvement of a full-scale process with the goal of moving the operating conditions toward the optimum or following a "drift." The method should not require large or sudden changes in operating conditions that might disrupt production. "\NewTerm{Evolutionary operation (EVOP)}"\index{ Evolutionary operation} was proposed by Box (1957) as such an operating procedure. It is designed as a method of routine plant operation that is carried out by manufacturing personnel with minimum assistance from the research and development staff.
	
	EVOP consists of systematically introducing small changes in the levels of the operating variables under consideration. Usually, a $2^k$ design is employed to do this. The changes in the variables are assumed to be small enough that serious disturbances in yield, quality, or quantity will not occur, yet large enough that potential improvements in process performance will eventually be discovered. Data are collected on the response variables of interest at each point of the $2^k$ design. When one observation has been taken at each design point, a cycle is said to have been completed. The effects and interactions of the process variables may then be computed. Eventually, after several cycles, the effect of one or more process variables or their interactions may appear to have a significant effect on the response. At this point, a decision may be made to change the basic operating conditions to improve the response. When improved conditions have been detected, a phase is said to have been completed. 
	
	In testing the significance of process variables and interactions, an estimate of experimental error is required. This is calculated from the cycle data. Also, the 2k design is usually centered about the current best operating conditions. By comparing the response at this point with the 2k points in the factorial portion, we may check on curvature or change in mean (CIM); that is, if the process is really centered at the maximum, say, then the response at the center should be significantly greater than the responses at the 2k-peripheral points.
	
	In theory, EVOP can be applied to k process variables. In practice, only two or three variables are usually considered. We will give an example of the procedure for two variables. Box and Draper (1969) give a detailed discussion of the three-variable case, including necessary forms and worksheets. Myers and Montgomery (1995) discuss the computer implementation of EVOP.
	
	\pagebreak
	\subsubsection{Optimal Designs}
	In the design of experiments, "\NewTerm{optimal designs}\index{optimal designs}" are a class of experimental designs that are optimal with respect to some statistical criterion. The creation of this field of statistics has been credited to Danish statistician Kirstine Smith.

	In the design of experiments for estimating statistical models, optimal designs allow parameters to be estimated without bias and with minimum variance. A non-optimal design requires a greater number of experimental runs to estimate the parameters with the same precision as an optimal design. In practical terms, optimal experiments can reduce the costs of experimentation.
	
	The optimality of a design depends on the statistical model and is assessed with respect to a statistical criterion, which is related to the variance-matrix of the estimator. Specifying an appropriate model and specifying a suitable criterion function both require understanding of statistical theory and practical knowledge with designing experiments.

	We will begin here with the study of "\NewTerm{D-optimal design}" which is a form of experimental design provided by a computer algorithm for which we will give the mathematical background a little further below. These types of computer-aided designs are particularly useful when conventional plans do not apply.

	Unlike standard classical models such as factorial design and fractional factorial designs or Taguchi designs, D-optimal designs are generally not orthogonal but are very flexible, hence their great practical importance.

	D-optimal plans are always an option whatever the degree of the model the experimenter wishes to use (eg first order, second order with or without interactions, complete quadratic, cubic, etc.).

	The D-optimal designs are associated with a regression model based on the fact that since in the case of the multiple linear regression we have (\SeeChapter{see section Theoretical Computing page \pageref{multiple linear regression gaussian model}}):
	
 	then we must be able to find the component values of the matrix $X$ which minimize the variances (the elements of the diagonal of this matrix as we have proved in the section of Theoretical Computing!). As this optimization operation is tricky, it is necessary to optimize a scalar value relative to the dispersion matrix $(X^TX)^{-1}$

	If we write:
	
	Then let us recall that we have proved in the section of Linear Algebra that the components of the inverse of the matrix $A$ are given by:
	
	Therefore minimize all the components is equivalent as to maximize the determinant of $A$ and hence to maximize the determinant of $(X^TX)^{-1}$. It is for this reason that we speak of D-optimal designs for "\NewTerm{Determinant-optimal designs}\index{determinant-optimal designs}" because trying to minimize the diagonal components (and in general all components) of the variance-covariance matrix is equivalent to seeking to maximize the determinant of $(X^TX)$:
	
	Another approach of minimizing the diagonal components of the variance-covariance matrix can be in choosing the scalar value to minimize, and therefore to minimize the trace of the information matrix:
	
 	and we then speak of an "\NewTerm{A-optimal design}\index{A-optimal design}" which is numerically more expensive in algorithmic terms than the search for the D-optimal plan because it is necessary to reverse the information matrix but it is a criterion that is often considered more natural.

	However, the majority of software specialized in experimental designs propose to determine the D-optimal or A-optimal designs but there are many other possible criteria that we will not present here such as E-optimality, G-optimality, etc.

	Obviously before that the D or A-optimal design or other can be generated, the experimenter must know the theoretical model he wants! Thus, knowing the model, the number of factors and levels of each factor we known then the complete factorial design and from the number of real tests (runs) that can be made by the experimenter (economically speaking!), we must find the most effective sub-set that will be a D-optimal, A-optimal design or other...

	It seems that to date the software uses evolutionary algorithms (\SeeChapter{see section Theoretical Computing page \pageref{genetic algorithms}}) to determine the optimal design so in extenso there is no guarantee that the given optimal design is really the optimal one (except if one restarts the calculation a great number of times always falling on the same result).

	The advantage of D-optimal or A-optimal designs or other designs of the same family is that they are very flexible and often more suited to the real needs of the experimenters.
	
	OK... it may seems quite abstract for some readers so let us see a companion example.
	
	Suppose that an experimenter has $3$ controlled variables and that seeks to determine a model of the type:
	
 	and we will consider that it has the following parameters:
	
	and that he can only do $12$ trials (for economic reasons or times reasons or else...!). In this case, given that one of the factors have $5$ levels, it is out of question for this to use fractional factorial designs or the Plackett-Burman designs. Even if Taguchi's designs manage certain factors at more than $2$ levels there is no Taguchi plan for this particular scenario (at least as far as we know...) so we must also forget them. The same applies to composite surface designs or Box-Behnken designs.

	The only thing that can be done is either to use a complete factorial design (but again it would require $20$ tests which is not conceivable!) or a D or A-optimal design.

	Obviously the complete factorial plan would require $5\cdot 2\cdot 2=20$ tests (runs) which would be:
	\begin{table}[H]
	\begin{center}
		\definecolor{gris}{gray}{0.85}
			\begin{tabular}{|c|c|c|c|}
				\hline
				\cellcolor{black!30}\textbf{Run} & \cellcolor{black!30}\textbf{$\pmb{x_1}$} & \cellcolor{black!30}\textbf{$\pmb{x_2}$} & \cellcolor{black!30}\textbf{$\pmb{x_3}$}\\ \hline
				$1$ & $-1$ & $-1$ & $-1$ \\ \hline
				$2$ & $-1$ & $-1$ & $+1$ \\ \hline
				$3$ & $-1$ & $+1$ & $-1$ \\ \hline
				$4$ & $-1$ & $+1$ & $+1$ \\ \hline
				$5$ & $-0.5$ & $-1$ & $-1$ \\ \hline
				$6$ & $-0.5$ & $-1$ & $+1$ \\ \hline
				$7$ & $-0.5$ & $+1$ & $-1$ \\ \hline
				$8$ & $-0.5$ & $+1$ & $+1$ \\ \hline
				$9$ & $0$ & $-1$ & $-1$ \\ \hline
				$10$ & $0$ & $-1$ & $+1$ \\ \hline
				$11$ & $0$ & $+1$ & $-1$ \\ \hline
				$12$ & $0$ & $+1$ & $+1$ \\ \hline
				$13$ & $+0.5$ & $-1$ & $-1$ \\ \hline
				$14$ & $+0.5$ & $-1$ & $+1$ \\ \hline
				$15$ & $+0.5$ & $+1$ & $-1$ \\ \hline
				$16$ & $+0.5$ & $+1$ & $+1$ \\ \hline
				$17$ & $+1$ & $-1$ & $-1$ \\ \hline
				$18$ & $+1$ & $-1$ & $+1$ \\ \hline
				$19$ & $+1$ & $+1$ & $-1$ \\ \hline
				$20$ & $+1$ & $+1$ & $+1$ \\ \hline
		\end{tabular}
	\end{center}
	\caption{Starting full design for D-optimal design seeking}
	\end{table}
	So that everyone can use an accessible tool, we will show how to extract a D or A-optimal design of $12$ tests using a simple spreadsheet software like Microsoft Excel 14.0.7015 by first constructing the following table corresponding to the matrix in its totality:
	\begin{figure}[H]
		\centering
		\includegraphics[scale=1]{img/engineering/full_factorial_design_for_d_optimum_design_microsoft_excel.jpg}
		\caption[]{Starting full design for D-optimal design seeking in Microsoft Excel 14.0.7015}	
	\end{figure}
	Then we construct the following structure a little further on the same sheet:
	\begin{figure}[H]
		\centering
		\includegraphics[scale=0.65]{img/engineering/d_optimum_design_structure_preparation_in_microsoft_excel.jpg}
		\caption[]{Preparation of D-optimal design with formulas in Microsoft Excel 14.0.7015}	
	\end{figure}
	Now let's us recall that finding a D-optimal design is equivalent to find the components of $X$ such as:
	
	so in one of the cells we should write (be careful this is a matrix function so we must validate by Ctrl + Shift + Enter):
	\begin{figure}[H]
		\centering
		\includegraphics[scale=1]{img/engineering/d_optimum_design_formal_determinant_in_microsoft_excel.jpg}
	\end{figure}
	But if we follow the method advocated by the NIST on their page here:
	\begin{center}
	\url{http://www.itl.nist.gov/div898/handbook/pri/section5/pri521.htm}
	\end{center}
	We would have to maximize the determinant of the submatrix (which in all rigor is false but anyway let us do as they say...) and then we write rather:
	\begin{figure}[H]
		\centering
		\includegraphics[scale=1]{img/engineering/d_optimum_design_nist_determinant_in_microsoft_excel.jpg}
	\end{figure}
	Then we launch the solver of this version of Microsoft Excel in evolutionary mode:
	\begin{figure}[H]
		\centering
		\includegraphics[scale=0.7]{img/engineering/d_optimum_design_nist_determinant_solver_settings_microsoft_excel.jpg}
		\caption{Evolutionary solver settings for seeking D-optimal design in Microsoft Excel 14.0.7015}
	\end{figure}
	You may perhaps be wondering why more variable cells should be selected than necessary. This is because of the intrinsic operation of the Microsoft Excel solver. If we take that cells \texttt{I2} to \texttt{I13} it will not find any solution that you put additional constraints or not to the solver.

	If we run the solver, we get:
	\begin{figure}[H]
		\centering
		\includegraphics[scale=0.8]{img/engineering/d_optimum_design_fina_according_nist_in_microsoft_excel.jpg}
		\caption{Final D-optimum design following NIST recommendation in in Microsoft Excel 14.0.7015}
	\end{figure}
	that give us a determinant of $1296$. The result provided by  the NIST has in comparison a determinant of $1156$. The D-optimal experimental design given by the NIST is indeed:
	\begin{table}[H]
	\begin{center}
		\definecolor{gris}{gray}{0.85}
			\begin{tabular}{|c|c|c|c|}
				\hline
				\cellcolor{black!30}\textbf{Run} & \cellcolor{black!30}\textbf{$\pmb{x_1}$} & \cellcolor{black!30}\textbf{$\pmb{x_2}$} & \cellcolor{black!30}\textbf{$\pmb{x_3}$}\\ \hline
				$1$ & $+1$ & $-1$ & $-1$ \\ \hline
				$2$ & $-1$ & $-1$ & $+1$ \\ \hline
				$3$ & $-1$ & $+1$ & $-1$ \\ \hline
				$4$ & $-1$ & $+1$ & $+1$ \\ \hline
				$5$ & $0$ & $-1$ & $-1$ \\ \hline
				$6$ & $0$ & $-1$ & $+1$ \\ \hline
				$7$ & $0$ & $+1$ & $-1$ \\ \hline
				$8$ & $0$ & $+1$ & $+1$ \\ \hline
				$9$ & $+1$ & $-1$ & $-1$ \\ \hline
				$10$ & $+1$ & $-1$ & $+1$ \\ \hline
				$11$ & $+1$ & $+1$ & $-1$ \\ \hline
				$12$ & $+1$ & $+1$ & $+1$ \\ \hline
		\end{tabular}
	\end{center}
	\caption{Final D-optimal design}
	\end{table}
	Where does this difference comes from???!!!!! This is quite easy to understand: It is impossible to found all coefficients with the system we get with Microsoft Excel as we still have $4$ levels, $2$ levels and $2$ levels so this system cannot be solved as we would need $16$ runs to solve it without alias!!!! So the real algorithm to reach $12$ runs has to eliminate one of the level of the first factor and the main purpose of $D$-optimum design is to found which level should be eliminate to keep the maximum information!
	\begin{tcolorbox}[title=Remark,colframe=black,arc=10pt]
	Caution! Depending on the software packages of used for DoE, you might come across a different result because they basically use almost all different algorithms (Federov's Method, Modified Federov's Method or K-Exchange Method for most known one):
	\begin{figure}[H]
		\centering
		\includegraphics[scale=0.8]{img/engineering/reliasoft_optimal_design.jpg}
		\caption[]{Reliasoft DOE++ Screenshot of algorithms options dialog box}
	\end{figure}
	\end{tcolorbox}
	Then, it is customary in practice and if we conform ourselves to the NIST results... to make the calculations of the following efficiency indices relative to the matrix obtained (the choice is completely empirical but well... it seems that most statistical software has its own indicators...):
	
	where in the G-efficiency, the term $\sigma_G$ is the smallest variance in the diagonal of the projection matrix $X(X^TX)^{-1}X^T$.
 
	What gives in Microsoft Excel:
	\begin{figure}[H]
		\centering
		\includegraphics[scale=0.8]{img/engineering/index_efficiency_optimal_designs_formulas_excel.jpg}
		\caption{Explicit formulas for index efficiency conforming to NIST in Microsoft Excel 14.0.7015}
	\end{figure}
	where in the relations of the cells \texttt{S1} to \texttt{S3}, the $5$ is the number of columns of the experiment matrix and the $12$ is the number of rows. To do the calculations of the A-efficiency and G-efficiency we need in Microsoft Excel both matrices explicitly (since this spreadsheet does not have functions for the trace of a matrix nor for the maximum of the trace) as that we have in \texttt{R6} (and the whole range of the relative matrix) the following matrix relation (to be validated with Ctrl+Shit+Enter):
	\begin{center}
		\texttt{=MINVERSE(MMULT(TRANSPOSE(J2:N13),J2:N13))}
	\end{center}
	and in \texttt{R13} (and in all the corresponding range) the following matrix relaction (to be validated with Ctrl+Shit+Enter):
	\begin{center}	
\texttt{=SQRT(MMULT(J2:N13,MMULT(MINVERSE(MMULT(TRANSPOSE(J2:N13),J2:N13))\\
		,TRANSPOSE(J2:N13))))}
	\end{center}
	What gives:
	\begin{figure}[H]
		\centering
		\includegraphics[scale=0.8]{img/engineering/index_efficiency_optimal_designs_excel.jpg}
		\caption{Index efficiency conforming to NIST in Microsoft Excel 14.0.7015}
	\end{figure}
	and perfectly matches (obviously...) the values given by the NIST! We also see that JMP (SAS) also gives (with the difference that it multiplies D-efficiency and G-efficiency by a factor $100$) as shown in the printscreen of the french version below:
	\begin{figure}[H]
		\centering
		\includegraphics[scale=1]{img/engineering/index_efficiency_optimal_designs_jmp.jpg}
		\caption{JMP (SAS) efficiency index of optimal designs}
	\end{figure}
	while the Minitab software (version 16.1.1 and earlier) gives the D and G optimality but calculates it in a bizarre and obscure way in my point of view.
	
	In order to fit an optimality criterion the companion example above show us that we have to define the number of experiments we want to have in the design. The selection of this factor is very important because changing the number of the design runs alters the model matrix and another optimal design is chosen. There are no rules to define this number, but the minimum is model-dependent. A model with $p$ coefficients can only be investigated with a D-optimal design which has at least $p$ runs. In most cases, it is useful to create different designs that differ in the number of runs and compare the efficiency of the designs. A design with a few more or less design runs than the desired one can have a higher determinant and hence is the best design to perform. A software like Minitab for reasons that are unknown to us don't give however the possibility  to have a number of runs equal to the number of factors.
	
	\pagebreak
	\subsubsection{Mixture Design}
	When a product is formed by mixing together two or more ingredients, the product is named a "mixture", and the ingredients are named "mixture components". In a general mixture problem, the measured response is assumed to depend only on the proportions of the ingredients in the mixture, not the amount of the mixture. For example, the taste of a fruit punch recipe (i.e., the response) might depend on the proportions of watermelon, pineapple and orange juice in the mixture. The taste of a small cup of fruit punch will obviously be the same as a big cup.
	
	Sometimes the responses of a mixture experiment depend not only on the proportions of ingredients, but also on the settings of variables in the process of making the mixture. For example, the tensile strength of stainless steel is not only affected by the proportions of iron, copper, nickel and chromium in the alloy; it is also affected by process variables such as temperature, pressure and curing time used in the experiment.

	One of the purposes of conducting a mixture experiment is to find the best proportion of each component and the best value of each process variable, in order to optimize a single response or multiple responses simultaneously. In this chapter, we will discuss how to design effective mixture designs and how to analyze data from mixture experiments with and without process variables.
	
	There are several different types of mixture designs. The most common ones are "simplex lattice", "simplex centroid", "simplex axial" and "extreme vertex designs", each of which is used for a different purpose.
	\begin{itemize}
		\item If there are many components in a mixture, the first choice is to screen out the most important ones. Simplex axial and Simplex centroid designs are used for this purpose.

		\item If the number of components is not large, but a high order polynomial equation is needed in order to accurately describe the response surface, then a simplex lattice design can be used.

		\item Extreme vertex designs are used for the cases when there are constraints on one or more components (e.g., if the proportion of watermelon juice in a fruit punch recipe is required to be less than $30\%$, and the combined proportion of watermelon and orange juice should always be between $40\%$ and $70\%$).
	\end{itemize}
	So often in practice we are interested in studying mixtures. In other words, we want to know the proportion of ingredients needed to achieve a certain result. The cases of application are so numerous that it would be futile to enumerate them all here.

	The fact that the proportions must be added is the key element of the simple mixture design since if we denote by $x_i$ the proportion of the ingredient $i$ then under unconstrained mixture we have:
	
 	or written slightly differently:
	
	where the last relation is named logically "\NewTerm{fundamental constraint of unconstrainted mixtures}\index{fundamental constraint of unconstrainted mixtures}".

	So we already notice that it is not possible to vary one parameter (ingredient) without the other being modified. So the control variables are not independent! We can not then use the techniques seen so far!

	As we will see just below, the response surface of a mixture design is a simplex (connected surface as introduced in the section of Geometric Shapes) of dimension $k$ with $k-1$ edges (and generally plunged into a space of dimension $k$).

	For example, with two factors (binary mixture) the simplex  is simply a line segment ranging from $(0,1)$ to $(1,0)$ in terms of coordinates. With three factors, the simplex is a triangle with three edges with three vertices of coordinates $(1,0,0)$, $(0,1,0)$ and $(0,0,1)$:
	\begin{figure}[H]
		\centering
		\includegraphics[scale=1]{img/engineering/mixture_three_factors_simplex.jpg}	
		\caption{Three factor mixtures design simplex}
	\end{figure}
	where the segments of the point $M$ at the concentration points  are logically the translations of the edges of the triangle with for condition that the sum of the segments being equal to a unit length.

	There are several ways of arranging the experimental points whose in our field of study are (from top to bottom in the figure below):
	\begin{itemize}
		\item Simplex lattice designs
		\item Simplex centroid designs
		\item Augmented simplex centroid designs
		\item and some other exotics one...
	\end{itemize}
	\begin{figure}[H]
		\centering
		\includegraphics[scale=1]{img/engineering/mixture_simplex_arrangements.jpg}	
		\caption{Various arrangements of three factors mixture simplex}
	\end{figure}
	Giving some classes I have notice that also an example of a $4$ factors simplex centroid example design is also useful for some students:
	\begin{figure}[H]
		\centering
		\includegraphics[scale=0.9]{img/engineering/mixture_centroide_simplex_four_factors.jpg}	
		\caption{$4$ factors simplex centroid example}
	\end{figure}
	
	\paragraph{Network mixture designs (simplex lattice designs)}\mbox{}\\\\
	The "\NewTerm{network mixture designs}\index{network mixture designs}" more commonly named "\NewTerm{Simplex lattice designs}\index{simplex lattice designs}" and sometimes just "\NewTerm{L-simplex design}\index{L-simplex design}" or "\NewTerm{Scheffé's designs}\index{Scheffé's designs}" are those most often studied in schools because they are the least abstract and algebraically the most accessible. Thus, in a network mixture design of $p$ variables and where the intervals of values are cut in $m$ pieces, the reader will notice by trying manually on a sheet of paper that we get the following table of number of runs (that is to say: the cost of the corresponding design):
	\begin{table}[H]
		\centering
		\begin{tabular}{|c|c|c|c|}
				\hline
		          & \multicolumn{3}{c|}{{\cellcolor{black!30}$\pmb{m}$}} \\ \hline
		         \cellcolor{black!30}$\pmb{p}$ &   {\cellcolor{black!30}$1$ (pure simplex)}    & {\cellcolor{black!30}$2$}      &  {\cellcolor{black!30}$3$}    \\ \hline
		         {\cellcolor{black!30}$3$}  &   $3$    &   $6$    &   $10$   \\ \hline
		         {\cellcolor{black!30}$4$} &    $4$   &    $10$   &   $20$   \\ \hline
		         {\cellcolor{black!30}$5$} &    $5$  &     $15$  &   $35$   \\ \hline
		         {\cellcolor{black!30}$6$} &    $6$   &    $21$   &  $56$    \\ \hline
		         {\cellcolor{black!30}$7$} &    $7$  &    $28$   &  $84$   \\ \hline
		\end{tabular}
		\caption{Number of points in a mixture lattice design (non-centered)}
	\end{table}
	OK this table table be a quite abstract for some people. So let us consider an illustration of the case $p=3$ and $m=1$ (we see clearly the $N=3$ runs):
	\begin{figure}[H]
		\centering
		\includegraphics[scale=1]{img/engineering/mixture_design_non_centered_p3_m1.jpg}
	\end{figure}
	and for the case $p=3$ and $m=2$ (we see clearly the $N=6$ runs):
	\begin{figure}[H]
		\centering
		\includegraphics[scale=1]{img/engineering/mixture_design_non_centered_p3_m2.jpg}
	\end{figure}
	and finally for the case $p=3$ and $m=3$ (we see clearly the $N=10$ runs):
	\begin{figure}[H]
		\centering
		\includegraphics[scale=1]{img/engineering/mixture_design_non_centered_p3_m3.jpg}
	\end{figure}
	A very common mixture design is the $\{p,m\}={3,2}$ with $6$ experimental points as seen above but with an addition point. This is why in many textbooks we can found: $\{p,m\}={3,2}=7$ as there is an addition center point!
	
	In the general case of non-centered lattice mixture designs we find by a little bit trial and error that the cost (runs) of such a simplex is given by:
	
	and for a centered lattice design of $q$ factors by:
	
	Let us also indicate that logically nothing prevents the support (domain of definition) of the mixtures from being constrained, we then speak of "\NewTerm{extreme vertices designs}\index{extreme vertices designs}" that cover only a subportion or smaller space within the simplex. Thus, in the case of a ternary mixture design, nothing prevents us from having a case where each of the components has only upper limit values but the lower value can be zero! Here is for example a special case of constraint mixture design:
	\begin{figure}[H]
		\centering
		\includegraphics[scale=1]{img/engineering/mixture_constraint_domains.jpg}
		\caption{Three factor mixture constraint domains}
	\end{figure}
	or lower ($L$) and at the same time upper ($U$) limit values:	
	\begin{figure}[H]
		\centering
		\includegraphics[scale=1]{img/engineering/mixture_highly_constraint_domains.jpg}
	\end{figure}
	or that the ratio of two ingredients is constant, or that the sum of two ingredients gives a constant proportion. In short there may be many scenarios, but in the case of limited boundary, it is customary to write:
	
	and this is what we name "\NewTerm{linear constraints}\index{linear constraints}" (as it is obvious if each ingredient is bounded that the sum is also bounded!).
	\begin{tcolorbox}[title=Remark,colframe=black,arc=10pt]
	It is quite possible to treat both the mixing variables (the proportions of the constituents) and the design factors of the experiments. To illustrate this situation, we can take the example of a chocolate manufacturer. The study of the composition of the chocolate gives rise to a mixtures design and the conditions of preparation give rise to a factorial design (or other ...). At each point of experiments of the factorial design, it is necessary to realize mixture design!!!! There is thus rapidly a large number of tests to be carried out since it is necessary to carry out $np$ experiments if the mixture design has $n$ runs and the factorial design has $p$ runs.
	\end{tcolorbox}
	We then consider the particular simple linear model (thus still an approximation ... and there are many other theoretical models that are not linear):
	
	or seen with the statistician point of view:
	
 	But since we have the constraint that $\sum_{i=1}^k x_i=1$ the $\hat{\beta}_i$ can not be determined uniquely if we go through the usual linear system. One of the $x_i$ must be used to determine the coefficients uniquely and must therefore be eliminated (concept of degree of freedom)! It is obviously very annoying in practice.

	So mathematically, nothing prevents us from writing:
	
 	and if we change the notations:
	
 	the previous relation can then be written:
	
	The mathematical model may of course also contain interactions. For example in the case of two factors, we could write with interactions and at the second degree:
	
 	and by applying the identity:
	
 	then:
	
	Therefore:
 	
	with therefore:
	
 	Either with three ingredients this gives for example:
	
 	and so on for $n$ components and the idea is the same for the third degree and beyond. It is Henry Scheffé that suggested to describe mixture properties by reduced polynomials subject to the normalization condition.
	
	The number of terms will then obviously be for this model of the second degree:
	
 	We can then use the following table of control (for those who like to do the calculations by hand) by stopping at the power of three (yes we have to stop one day or another...):
	\begin{table}[H]
		\centering
		\begin{tabular}{|c|l|l|l|}
			\hline
		         \cellcolor{black!30}{\textbf{\# of factors}} &   \cellcolor{black!30}{\parbox{2.5cm}{\textbf{\# terms with linear model}}}  &  \cellcolor{black!30}\parbox{3cm}{\textbf{\# terms with quadratic model}}  &  \cellcolor{black!30}\parbox{2.5cm}{\textbf{\# terms with cubic model}}     \\ \hline
		       	$2$  &   $2$    &   $3$   &   $-$   \\ \hline
			$3$  &   $3$    &   $7$   &   $7$   \\ \hline
		        $4$  &   $4$    &   $10$   &   $14$   \\ \hline
		        $5$  &   $5$    &   $15$   &   $24$   \\ \hline
		        $\ldots$  &   $\ldots$    &   $\ldots$   &   $\ldots$   \\ \hline
			$n$  &   $n$    &   $\dfrac{n(n+1)}{2}$   &   $\dfrac{n(n^2+5)}{6}$   \\ \hline
		\end{tabular}
		\caption{L-mixture designs terms for linear, quadratic and upper models}
	\end{table}
	Now let us come back to our initial example:
	
	\begin{figure}[H]
		\centering
		\includegraphics[scale=1]{img/engineering/simple_lattice_design_three_factor_design.jpg}
	\end{figure}
	for the following both reasons:
	\begin{enumerate}
		\item There are $6$ measurement points and this is good because we have $6$ coefficients to determine and we will need $6$ equations at least.

		\item The choice of the arrangement of the $6$ points is quite intuitive. Indeed, if you imagine first taking $3$ points, which ones are furthest from each other? They will be those of the summits of course! Once these three points are taken, what are the three other potential points that are the furthest away from it, but at the same time the furthest away from each other? They will be those in the middle of the edges!
	\end{enumerate}
	We will then write algebraic the following that resume the above figure:
	\begin{itemize}
		\item For $3$ of the $6$ points on the vertices: $y_i$ for $x_i=1$, $x_j=0$ with $i,j=1,2,3$ under the constraint that $j\neq 1$

		\item For $3$ of the $6$ points on the middle of the edges: $y_{ij}$ for $x_i=\dfrac{1}{2}$, $x_j=\dfrac{1}{2}$, $x_k=0$ under the constraint that $i<j,k\neq i,j$
	\end{itemize}
	What injected into:
	
	gives $6$ equations with $6$ unknowns:
	
	The search for the coefficients for these $6$ equations is immediate:
	
	Either in a generic way:
	
 	If we make repeated measurements in quantities $r_{ij}$, $r_i$, $r_j$, then we have:
	
 	Either by extension:
	
 	and for the variance (which will be useful for making inference on the values of the coefficients):
 	
	Some statistical software and authors even consider that (we must then verify that this hypothesis is well satisfied with typically a Fisher test):
	
	Then it remains:
	
	Some statistical softwares assumes that (so take care the this hypothesis is not rejected in the practice).
	
	It is alos interesting to measure the experimental variance of the model. Therefore in the general second degree case that is for recall written:
	
	Then we have:
	
	or if the measurement are not in equal quantities:
	
	which can sometimes found in the following way in the literature:
	
	and as:
	
	Therefore it comes:
	
	with:
	
	
	\begin{tcolorbox}[colframe=black,colback=white,sharp corners]
	\textbf{{\Large \ding{45}}Example:}\\\\
	Let us consider that three constituents that are polyethylene $x_1$, polystyrene $x_2$ and polypropylene $x_3$, are mixed in order to obtain a fiber whose measure of interest is the elongation at a constant force with the following values of the mean elongations obtained for each of the points of the mixture design network:
	\begin{figure}[H]
		\centering
		\includegraphics{img/engineering/mixture_design_without_process_variable_example.jpg}	
	\end{figure}
	with the measurements detailed in the corresponding table below:
	\begin{table}[H]\small
		\renewcommand{\arraystretch}{1.1}
		\centering
		\begin{tabular}{|c|c|c|c|c|c|}
			\hline
		         \cellcolor{black!30}{\parbox{2cm}{\textbf{\begin{center}Experiment point \end{center}}}} &   \cellcolor{black!30}{\parbox{2.1cm}{\textbf{\begin{center}Polyethylene $\pmb{x_1}$\end{center}}}}  &  \cellcolor{black!30}{\parbox{2cm}{\textbf{\begin{center}Polystyrene $\pmb{x_2}$\end{center}}}}  & \cellcolor{black!30}{\parbox{2.3cm}{\textbf{\begin{center}Polypropylene $\pmb{x_3}$\end{center}}}} & \cellcolor{black!30}{\parbox{2cm}{\textbf{\begin{center}Observed elongation $\pmb{y_i}$\end{center}}}} & \cellcolor{black!30}{\parbox{2cm}{\textbf{\begin{center}Average elongation $\pmb{\bar{y}_i}$\end{center}}}} \\ \hline
		       	$1$  &   $1$    &   $0$   & $0$ & $11.0$, $12.4$ & $11.7$  \\ \hline
				$2$  &   $0.5$  & $0.5$   & $0$ & $15.0$,$14.8$,$16.1$ & $15.3$ \\ \hline
		        $3$  &   $0$    &   $1$   & $0$ & $8.8$,$10.00$ & $9.4$ \\ \hline
		        $4$  &   $0$    &   $0.5$ & $0.5$ & $10.0$,$9.7$,$11.8$ & $10.5$ \\ \hline
		        $5$  &   $0$    &   $0$   & $1$ & $16.8$,$16.0$ & $16.4$  \\ \hline
		        $6$  &   $0.5$  & $0$   &   $0.5$ & $17.7$,$16.4$,$16.6$ & $16.9$  \\ \hline
		        \hhline{|=|=|=|=|=|=|}
		        &  & & \parbox{2.2cm}{\begin{flushleft}
		          { {\footnotesize Number of measurements:}}\end{flushleft}} & $15$ &  \\ \hline
				&  &  &  {\footnotesize  Average:} & $13.54$ &   \\ \hline
		\end{tabular}
	\end{table}
	and we will use the model:
	
 	and so we will use:
	
 	Which gives us:
	\end{tcolorbox}
	
	\begin{tcolorbox}[colframe=black,colback=white,sharp corners]
	
	So at the level of the influences, we already see that (without even needing to make a statistical test of significant difference):
	
	with the obvious that follows. We also observe already that:
	
	which means that the components $1$ and $2$, $1$ and $3$ contributes positively to the effect were are looking for when the interaction $2$ and $3$ decrease that later.\\

	Now, let us calculate and check if we have well:
	
	Therefore:
	
	With for recall (requested by reader) for example:
	
	and then:
	
	and also:
	\end{tcolorbox}
	
	\begin{tcolorbox}[colframe=black,colback=white,sharp corners]
	
	We see then obviously that we quite far from the assumption of equality of variance...\\

	Therefore what can we do???\\
	
	In fact apart to do the calculations of $\hat{\sigma}$ using the global average $\bar{y}$ as detailed just above, we notice that was is not realistic in the choice of the calculation of $\hat{\sigma}$ as done is that we consider that the average $\bar{y}$ should be the same for all the levels of the components which is obviously not true! Therefore as during our study of the ANOVA, we will use the residual variance $\text{V}_R$ (that is we know is also named "grouped variance" or "pooled variance") and that is given therefore by:
	
	We see then that the compromise is much better ... (without however of course to be exact!). We then have using the results prove earlier above:
	
	Our model is then traditionally written:
	
	which corresponds exactly to the second decimal point to what a software like Minitab gives us. Then, to find the triplet that maximizes the function it is enough to use a solver like the one  integrated into Microsoft Excel.\\
	
	Let us recall now that:
	\end{tcolorbox}
	
	\begin{tcolorbox}[colframe=black,colback=white,sharp corners]
	Let us now recall that:
	
	And as in our example the model has no pure quadratic term, it reduces to:
	
	We have then:
	
	We see then that to determine $\hat{\beta}_0$ we have to make a choice on one of the $\hat{\beta}_i$ (as we are not able to refer to the experimental value in the present case since the absence of any ingredient in the mixture will only give a empty entity and then it is quite difficult to test an elongation on nothing... but this, however, raises the principle of the approach). We will choose empirically $\hat{\beta}=0$, then it comes:
	
	The model with the ordinate at the origin (at least if it had a physical sense) would then be written:
	
	with in brackets the variances that do not change for the relative coefficients. There are two coefficients whose variances will change!\\
	
	We have indeed:
	
	Therefore:
	
	\end{tcolorbox}
	
	\begin{tcolorbox}[colframe=black,colback=white,sharp corners]
	Hence:
	
	We can even make an ANOVA of the regression (\SeeChapter{see section Theoretical Computing page \pageref{anova for linear regression}}) as for the more" traditional" experimental design and as we have studied (proved!) in the section of Theoretical Computing, we will need the following values:
	\begin{table}[H]
		\centering
		\begin{tabular}{ccccc}
		$Q_T$ & $=$ & $Q_A$ & $+$ & $Q_R$  \\
		$\displaystyle\sum_i (y_i-\bar{\bar{y}})^2$ & $=$ & $\displaystyle\sum_i (\hat{y}_i-\bar{y})^2$ & $+$  & $\displaystyle\sum_i(y_i-\hat{y}_i)^2$  \\
		 $N-1$ & $=$ & $k-1$ & $+$ & $N-k$  \\
		\end{tabular}
	\end{table}
	Then let us use the following table:
	\begin{table}[H]
		\centering
		\begin{tabular}{|c|c|c|c|c|c|}
			\hline
	         \cellcolor{black!30}{\parbox{1.9cm}{\textbf{\begin{center}Observed value $\pmb{y_i}$\end{center}}}} &   \cellcolor{black!30}{\parbox{2cm}{\textbf{\begin{center}Predicted  value $\pmb{\hat{y}_i}$\end{center}}}}  &  \cellcolor{black!30}{\parbox{1.9cm}{\textbf{\begin{center}Residuals $\pmb{(y_i-\hat{y}_i)}$\end{center}}}}  & \cellcolor{black!30}{\parbox{2cm}{\textbf{\begin{center}Deviation $\pmb{(y_i-\bar{y}_i)}$\end{center}}}} & \cellcolor{black!30}{\parbox{2.3cm}{\textbf{\begin{center}Deviation of regression $\pmb{(\hat{y}_i-\bar{y})}$\end{center}}}}  \\ \hline
	       	$11$  &   $11.7$    &   $-0.7$   & $-2.5$ & $-1.8$   \\ \hline
			$12.4$	& $11.7$ &	$0.7$ & $-1.1$ & $-1.8$ \\ \hline
	        $15.0$	& $15.3$ &	$-0.3$ & $1.5$ & $1.8$  \\ \hline
	        $14.8$	& $15.3$ &	$-0.5$ & $1.3$ & $1.8$  \\ \hline
	        $16.1$ & $15.3$ & $0.8$ & $2.6$ & $1.8$  \\ \hline
	        $10.0$ & $9.4$ & $0.6$ & $-3.5$ & $-4.1$  \\ \hline
	        $8.8$ &	$9.4$ &	$-0.6$ & $-4.7$ & $-4.1$ \\ \hline
	        $10.0$ & $10.5$ & $-0.5$ & $-3.5$ &	$-3.0$ \\ \hline
	        $9.7$ &	$10.5$ & $-0.8$ & $-3.8$ & $-3.0$ \\ \hline
	        $11.8$ & $10.5$ & $1.3$ & $-1.7$ & $-3.0$ \\ \hline
	        $16.8$ & $16.4$ & $0.4$ & $3.3$ & $2.9$ \\ \hline
	        $16.0$ & $16.4$ & $-0.4$ & $2.5$ & $2.9$ \\ \hline
	        $17.7$ & $16.9$ & $0.8$ & $4.2$ & $3.4$ \\ \hline
	        $16.4$ & $16.9$ & $-0.5$ & $2.9$ & $3.4$ \\ \hline
	        $16.6$ & $16.9$ & $-0.3$ & $3.1$ & $3.4$ \\ \hline
		\end{tabular}
	\end{table}
	Then we have:
	\begin{table}[H]
		\centering
		\begin{tabular}{ccccc}
		$Q_T$ & $=$ & $Q_A$ & $+$ & $Q_R$  \\
		$\displaystyle\sum_i (y_i-\bar{\bar{y}})^2$ & $=$ & $\displaystyle\sum_i (\hat{y}_i-\bar{y})^2$ & $+$  & $\displaystyle\sum_i(y_i-\hat{y}_i)^2$  \\
		$138.88$ & $=$ & $128.32$ & $+$  & $6.56$  \\
		 $N-1$ & $=$ & $k-1$ & $+$ & $N-k$  \\
		 $14$ & $=$ & $5$ & $+$ & $9$  \\
		\end{tabular}
	\end{table}
	\end{tcolorbox}
	
	\begin{tcolorbox}[colframe=black,colback=white,sharp corners]
	We then have the following one-way fixed factor ANOVA table (\SeeChapter{see section Statistics page \pageref{anova one way fixed factor}}):
	\begin{table}[H]\small
		\renewcommand{\arraystretch}{1.4}
		\begin{tabular}{llclcc}\hline
		Source & Sum of squares &  $\chi^2$ df & Average of squares & $F$ & Critical $F$\\ \hline
		Inter-Class & $Q_A=128.32$ & $5$ & $\text{MSk}=\displaystyle\frac{128.32}{5}$ &
		$\frac{\displaystyle\text{MSk}}{\displaystyle\text{MSE}}=31.5$ & $F_{5,9}^{\text{crit.}}=3.481$ \\
		Intra-Class & $Q_R=6.56$ & $9$ & $\text{MSE}=\displaystyle\frac{6.56}{5}$  & & \\
		Total & $Q_T=134.88$ & $14$ & & &\\ \hline
		\end{tabular}
		\caption[]{One way fixed factor ANOVA table for mixture design}
	\end{table}
	where the limit has been calculated with the following function of the spreadsheet software Microsoft Excel 14.0.7106:
	\begin{center}
		\texttt{=F.INV(95\%,5,9)=3.481}
	\end{center}
	Hence as $31.5$ (corresponding $p$-value of $0.002\%$) is far above $3.481$ we reject the null hypothesis $H_0$ that the model coefficients are non-significant é(ie they are therefore significantly different from zero). In extenso this also leads us to reject the fact that the response surface is plane (since the coefficients of the interactions are also significant).\\
	
	The reader interested can see the study of this mixture in the R or Minitab companion books.
	\end{tcolorbox}
	
	\pagebreak
	\paragraph{Full Factorial Combined with Mixture Design-Crossed Design}\mbox{}\\\\
	In 1981, Cornell\cite{cornell2011experiments} published in its book the results of experimental studies where the quality of a fish patty was defined by both its composition and production process. The experiment included seven different mixtures prepared by mixing different species of fish and then subjecting the resulting patty to various cooking conditions and defined in accord with simplex centroid design (see figure below).
	\begin{figure}[H]
		\centering
		\includegraphics[scale=1]{img/engineering/simplex_centroid_design.jpg}
		\caption{Simplex Centroid Design}
	\end{figure}
	The preparation procedure of the defined mixtures or a formulation included three control process factors with associated
variation levels: baking temperature from $190\;[^\circ\text{C}]$  to $218\;[^\circ\text{C}]$; time in the oven from $25$ [min] to $40$ [min]; deep fat frying time from $25$ [s] to $40$ [s]. Process factor levels have been varied in accordance with the $2^3$ full factorial experiment. Design of experiment $7\times 2^3$ with $56$ trials of a simplex centroid design $\times 2^3$ full factorial design
has been sufficient for mathematical modeling of the observed phenomenon (see figure below).
	\begin{figure}[H]
		\centering
		\includegraphics[scale=1]{img/engineering/cross_design_mixture_on_vertices.jpg}
		\caption{Simplex centroid design in each point of a $2^3$ full factorial experiment}
	\end{figure}
	\begin{figure}[H]
		\centering
		\includegraphics[scale=1]{img/engineering/cross_design_factorial_on_vertices.jpg}
		\caption{$2^3$ factorial design in each point of a simplex centroid design}
	\end{figure}
	A regression model with $56$ regression coefficients or reduced regression model with $18$ coefficients has been sufficient for an adequate description of the
problem.	
	
	This was the first example of application of a mixture design $\times$ process factor design in experimental studies. Since such designs contain a relatively large number of factors, it is of interest to replace full factorial designs of process factors with fractional factor designs. Today all modern Statistical softwares include the analysis of "\NewTerm{design-crossed design}\index{design-crossed design}".
	\begin{tcolorbox}[colframe=black,colback=white,sharp corners]
	\textbf{{\Large \ding{45}}Example:}\\\\
	Assume a design has $3$ mixture components and $2$ process variables, as illustrated in the above figure.\\
	
	For the $3$ mixture components, the following special cubic Scheffé model is used:
	
	For the $2$ process variables the following model is used:
	
	The combined model with both mixture components and process variables is assumed, for being able to make an ANOVA, as following (quite strong assumption in the point of view of the physicist...):
	
	The above combined model has total of $7\times 4=28$ terms. By expanding it, we get the following model:
	
	The combined model basically crosses every term in the mixture components model with every term in the process variables model. From a mathematical point of view, this model is just a regular regression model. Therefore, the traditional regression analysis method can still be used for obtaining the model coefficients and calculating the ANOVA table.
	\end{tcolorbox}
	The reader can see a real application example of crossed designs in the Minitab companion book if necessary.
	
	\pagebreak
	\pagebreak
	\subsubsection{General DoE diagnostic tools}
	We have seen so far the almost all DOE use statistical analysis based on ANOVA, ANCOVA and obviously linear regression null hypothesis tests. But most DoE softwares propose a bunch of other more or less empirical diagnostics for the analysis of the designs and this is what we will focus below in details.
	
	\paragraph{Lenth's PSE Pareto Margin Error for unreplicated factorial designs}\mbox{}\\\\
	G.E.P. Box and R.D. Meyer introduced in 1986 a method for assessing the sizes of contrasts in unreplicated factorial and fractional factorial designs. This is was useful technique, and an associated graphical display popularly known as a "Bayes plot" makes it even more effective but very hard to interpret. Therefore Russell V. Lenth proposed a new method 1989 that is still currently used in many statistical softwares.
	
	The "\NewTerm{Lenth's Pseudo Standard Error}\index{Lenth's PSE}" (abreviated Lenth's PSE) is a commonly indicator in DoE softwares Pareto outputs (even if it is not real Pareto charts...) that provides an analytical basis for testing the effects in an unreplicated (single replicate) two-level factorial design\footnote{It should not then be necessary to precise that it therefore not applies to General factorial Designs}. Lenth's method estimates the variance of an effect from the smallest (presumably insignificant) effect estimates.
	
	So... Lenth proposed a method for estimating the standard deviation of the effects on the basis that if the coded factors $x_i$ (for simplification reasons but it also applies to non-coded and non-normalized effects) have their effects that follow a Normal distribution $E_i=\mathcal{N}(0,\sigma)$ then the median of the absolute value of the coded effects $|E_i|$ is in theory equal to the half-Normal distribution (\SeeChapter{see section Statistics page \pageref{half normal distribution}}):
	
	and again (!) this last relation can be easily generalized to non-coded effects (just return back to that proof and add a non-zero mean in the development of the proof)!
	
	What Lenth has proposed is first to eliminate all effects that satisfies:
	
	that is to say:
	
	denoted $s_0$.
	
	The R software gives with the \textbf{fdrtool} package:
	\begin{center}
	\texttt{>1-phalfnorm(2.5,theta=sqrt(pi/2),lower.tail=T,log.p=F)=0.012}
	\end{center}
	
	The reason why Lenth's made this choice are explained in a quite obscure way in its original article... in our point of view. His argument is that simulations shows that the $1.5M_e$ is overestimate in practice therefore eliminate factors with high amplitude effect seems to give a more accurate result.
	
	After the elimination the proposition is to calculate a pseudo standard error with the remaining effects (that therefore should converge to the $\sigma$ of all effects):
	
	where Lenth's assume that $\sigma_{\text{effect}}$ is the standard error of the mean and that therefore we can do inference using the Student T distribution such at an $\alpha$ bilateral threshold we have the following margin error
	
	where $m$ is the total number of effects and the division by $3$ comes from a Monte Carlo study that was conducted by Lenth.
	\begin{tcolorbox}[colframe=black,colback=white,sharp corners]
	\textbf{{\Large \ding{45}}Example:}\\\\
	To make an example let us consider the following experiment that we already know (but we have removed the replicaton):
	\begin{table}[H]\centering
	\begin{center}
		\definecolor{gris}{gray}{0.85}
			\begin{tabular}{|c|c|c|c|c|}
				\hline
				\multicolumn{1}{c}{\cellcolor{black!30}\textbf{Trial N${}^\circ$}} & 
  \multicolumn{1}{c}{\cellcolor{black!30}$x_1$} & 
  \multicolumn{1}{c}{\cellcolor{black!30}$x_2$} & 
  \multicolumn{1}{c}{\cellcolor{black!30}$x_3$} & 
  \multicolumn{1}{c}{\cellcolor{black!30}$y_{i}$}\\ \hline
				 $1$ & $-1$ & $-1$ & $-1$ & $59$\\ \hline
				 $2$ & $+1$ & $-1$ & $-1$ & $74$\\ \hline
				 $3$ & $-1$ & $+1$ & $-1$ & $50$\\ \hline
				 $4$ & $+1$ & $+1$ & $-1$ & $69$\\ \hline
				 $5$ & $-1$ & $-1$ & $+1$ & $50$\\ \hline
				 $6$ & $+1$ & $-1$ & $+1$ & $81$\\ \hline
				 $7$ & $-1$ & $+1$ & $+1$ & $46$\\ \hline
				 $8$ & $+1$ & $+1$ & $+1$ & $79$\\ \hline
		\end{tabular}
	\end{center}
	\end{table}
	We now already how to perform the calculations manually of the effect so we will not come back on this topic. We will just communicate the output given by Minitab 17.1.3:
	\begin{figure}[H]
		\centering
		\includegraphics[scale=1]{img/engineering/lenth_pse_minitab_effects.jpg}	
	\end{figure}
	\end{tcolorbox}
	\begin{tcolorbox}[colframe=black,colback=white,sharp corners]
	Therefore in abolute value, we have:
	\begin{gather*}
		24.5, 5, 1, 1.5, 7.5, 2, 0.5
	\end{gather*}
	Ordered:
	\begin{gather*}
		0.5, 1, 1.5, 2, 5, 7.5, 24.5
	\end{gather*}
	So it is immediate that the median is equal to:
	
	So all the effects that are smaller or equal to $2.5(1.5M_e)=7.5$ are (Lenth takes strictly smaller but Minitab don't follow this rule):
	\begin{gather*}
		0.5, 1, 1.5, 2, 5, 7.5
	\end{gather*}
	Therefore:
	
	So finally:
	
	Where the $T$ value was obtained with R since Microsoft Excel don't accept fractional degrees of freedom:
	\begin{center}
	\texttt{>qt(.975,df=7/3)=3.764123}
	\end{center}
	Minitab 17.1.3 makes a summary of all calculations done above as following:
	\begin{figure}[H]
		\centering
		\includegraphics[scale=0.8]{img/engineering/lenth_pse_minitab_pareto.jpg}	
	\end{figure}
	\end{tcolorbox}
	
	\paragraph{Pareto Margin Error for replicated factorial designs}\mbox{}\\\\
	Let us recall that we have studied the following replicated experiment earlier above:
	\begin{table}[H]\centering
	\begin{center}
		\definecolor{gris}{gray}{0.85}
			\begin{tabular}{|c|c|c|c|c|c|c|}
				\hline
				\multicolumn{1}{c}{\cellcolor{black!30}\textbf{Trial N${}^\circ$}} & 
  \multicolumn{1}{c}{\cellcolor{black!30}$x_1$} & 
  \multicolumn{1}{c}{\cellcolor{black!30}$x_2$} & 
  \multicolumn{1}{c}{\cellcolor{black!30}$x_3$} & 
  \multicolumn{1}{c}{\cellcolor{black!30}$y_{i1}$} & 
  \multicolumn{1}{c}{\cellcolor{black!30}$y_{i2}$} & 
  \multicolumn{1}{c}{\cellcolor{black!30}$\bar{y}_i$}\\ \hline
				 $1$ & $-1$ & $-1$ & $-1$ & $59$ & $61$ & $60$\\ \hline
				 $2$ & $+1$ & $-1$ & $-1$ & $74$ & $70$ & $72$\\ \hline
				 $3$ & $-1$ & $+1$ & $-1$ & $50$ & $58$ & $54$\\ \hline
				 $4$ & $+1$ & $+1$ & $-1$ & $69$ & $67$ & $68$\\ \hline
				 $5$ & $-1$ & $-1$ & $+1$ & $50$ & $54$ & $52$\\ \hline
				 $6$ & $+1$ & $-1$ & $+1$ & $81$ & $85$ & $83$\\ \hline
				 $7$ & $-1$ & $+1$ & $+1$ & $46$ & $44$ & $45$\\ \hline
				 $8$ & $+1$ & $+1$ & $+1$ & $79$ & $81$ & $80$\\ \hline
		\end{tabular}
	\end{center}
	\end{table}
	and that we get by hand exactly the same results as given by Minitab 17.1.3:
	\begin{figure}[H]
		\begin{center}
		\includegraphics[scale=1]{img/engineering/replicated_design_minitab_analysis.jpg}
		\end{center}	
		\caption[]{Replicated factorial design effects C.I. with Minitab 17.1.3}
	\end{figure}
	It must be remembered to for the constant and all coefficients the $T$-value was respectively of the form:
	
	Therefore we see that all Student's $T$ distribution have always the same degrees of freedom!!! That will be written:
	
	So we could use the limit $T$-value (we have seen earlier above how to calculate it) in absolute value to compare the amplitude of every factor. And this is what Minitab give as output when the design is replicated:
	\begin{figure}[H]
		\centering
		\includegraphics[scale=1]{img/engineering/replicated_design_minitab_pareto_plot.jpg}
		\caption[]{Replicated factorial design $T$-value with Minitab 15.1.1}
	\end{figure}
	Now the question that arise is: What is this red line and how it is calculated?
	
	The choice (not rigorously justify as far as I know) used by most softwares is to compare all these values by the with the quantile $T_{\alpha/2,2^k}$ that gives in our case with Microsoft Excel 14.0.7177:
	\begin{center}
		\texttt{=TINV(97.5\%,8)=2.30600414}
	\end{center}
	and this is the value of the red line we can see on the plot above.
	
	\paragraph{Desirability}\mbox{}\\\\
	The "\NewTerm{desirability function}\index{desirability function}" approach to simultaneously optimizing multiple equations was originally proposed by Edwin C. Jr Harrington (1980). Essentially, the approach is to translate the functions to a common scale ($[0, 1]$), combine them using the geometric mean and optimize the overall metric. The equations may represent model predictions or other equations.

	For example, desirability functions are popular in response surface methodology as a method to simultaneously optimize a series of quadratic models. A response surface experiment may use measurements on a set of outcomes. Instead of optimizing each outcome separately, settings for the predictor variables sought to satisfy all of the outcomes at once.

	Also, in drug discovery, predictive models can be constructed to relate the molecular structures of compounds to characteristics of interest (such as absorption properties, potency and selectivity for the intended target). Given a set of predictive models built using existing compounds, predictions can be made on a large set of virtual compounds that have been designed but not necessarily synthesized. Using the model predictions, a virtual compound can be scored on how well the model results agree with required properties. In this case, ranking compounds on multiple endpoints may be sufficient to meet the scientist's needs.
	
	Originally, Harrington used exponential functions to quantify desirability. In this text we will introduce simple discontinuous functions of G. Derringer and R. Suich.

	Suppose that there are $N$ equations or function to simultaneously optimize, denoted $f_i(\vec{x})$ with $i= 1 \ldots N$. For each of the $i$ functions, an individual "desirability" function is constructed that is high when $f_i(\vec{x})$ is at the desirable level (such as a maximum, minimum, or target) and low when $f_n(\vec{x})$ is at an undesirable value. Derringer and Suich proposed three forms of these functions, corresponding to the type of optimization goal. For maximization of $f_i(\vec{x})$, the function:
	
	can be used, where $L$ is the lower limit that we allow to our variable of interest, $H$ its high (maximal) value that we allow, and $w$ is the weight that are all three chosen by the investigator. When the equation is to be minimized, they proposed the function:
	
	and for target $T$ situations:
	
	These functions are on the same scale (output in range $[0,1]$) and are discontinuous at the points $L$, $H$, and $T$. The values
of $w$, $w_1$ or $w_2$ can be chosen so that the desirability criterion is easier or more difficult to satisfy. The scaling factors $w$ are useful when one equation is of great importance than the other. Examples of the functions are given in the figure below:
	\begin{figure}[H]
		\centering
		\includegraphics[scale=0.5]{img/engineering/desirability.jpg}
		\caption[Derringer and Suich Desirability functions plot]{Derringer and Suich Desirability functions plot (source: Max Kuhn R Software)}	
	\end{figure}
	It should be noted that any function can be used to mirror the desirability of a model. For example, Del Castillo, Montgomery, and McCarville (1996) develop alternative desirability functions that can be used in conjunction with gradient based optimization routines.

	Given that the $N$ desirability functions $d_1,\ldots, d_r$ are on the $[0,1]$ scale, they can be combined to achieve an overall desirability function, $D$. One method of doing this is by the geometric mean\index{geometric mean} and the we speak of "\NewTerm{composite desirability}\index{composite desirability}":
	
	The geometric mean has the property that if any one model is undesirable ($d_i=0$), the overall desirability is also unacceptable ($D=0$). Once $D$ has been defined and the prediction equations for each of the $N$ equations have been computed, it can be use to optimize or rank the predictors.

	In Minitab (see our companion book on that software) the composite desirability is the geometric mean of the individual desirabilities if and only if the "importance" parameter  (whose value must be between $0.1$ and $10$) are all equals to $1$.
	\begin{figure}[H]
		\centering
		\includegraphics[scale=1]{img/engineering/optimizer_setup_minitab.jpg}
		\caption[]{Minitab 15.1.3 Design of experiment optimizer setup}	
	\end{figure}
	
	\pagebreak
	\subsection{Quality Control on Reception (Lot Acceptance Sampling Plans)}\label{sampling plans}
	The "\NewTerm{quality control on reception}\index{quality control on reception}" or simply "\NewTerm{acceptance test}\index{acceptance test}" or "\NewTerm{statistical batch control}\index{statistical batch control}" or following NIST\footnote{National Institute of Standards and Technology} "\NewTerm{Lot Acceptance Sampling Plans (LASPs)}\index{Lot Acceptance Sampling Plans}" is an extremely important field, wide and mathematically technical of statistical quality control in the industry (food, pharma, consumer goods) and services companies (repetitive action, process controls, surveys, assessments). Its purpose, according to the norm NF X06-021, is to allow the application on a controlled batch (products or services) of one or the other following decisions: Acceptance or Rejection at different stages of a production / development to see if a batch is admissible for the following steps or to check the deliverable  before sending it to the customer.
	\begin{figure}[H]
		\centering
		\includegraphics[scale=0.6]{img/engineering/quality_control.jpg}	
	\end{figure}
	\begin{tcolorbox}[title=Remark,colframe=black,arc=10pt]
	We will introduce here only the basics of this subject because the particular cases require considerable mathematical background! It is also for this reason why many companies (SMEs and multinationals) that use interns or students that don't have a mathematical background to define quality controls in order to save money are found later with considerable problems in compliance with laws and quality standards! However, what we introduce here is should be the minimum-minimorum knowledge of any active  quality responsible or quality engineer or responsible for reception control of suppliers in any organization (industrial or administrative) otherwise they should have absolutely no credibility from people hiring such "specialists" or from external consultants! Also beware of companies - particularly multinationals - who are looking for quality specialists mastering Microsoft Excel or Microsoft Access. Because it will mean that they use non-professional tools to do a job which ought to be with him the appropriate tools (Microsoft Excel and Microsoft Access or are not) !!! So in terms of internal organization, you can ensure that these companies organize and analyze anything, anyhow, with an unsuitable tool and therefore that there is a general mess internally.
	\end{tcolorbox}
	The purpose of quality control on reception (acceptance sampling) is not to estimate the quality of a lot, but only to accept or reject it. The acceptance is not a substitute for process control methods such as control charts or capability analysis. A dynamic use of these tools during manufacturing will result in reducing and, in some cases, eliminate the need to perform incoming or outcoming inspection/controls.

	There are three main approaches to evaluate a batch:
	\begin{itemize}
		\item Either accept the batch without inspection (useful in situations where absolute trust relations exists between the customer and supplier or we now that defects are impossible).

		\item Either take a sample (sampling) and draw conclusions about the whole batch from the sample (useful in most cases because inexpensive and highly motivates the supplier to improve quality but there is a risk of rejecting good batches or accept bad one). For example Mulhouse PSA control $5\%$ of vehicles coming out of their factories every day.

		\item Either inspect all units in the batch (useful when the supplier can not prove that its process control and is capable of producing in the specifications or when the monitored characteristic is very critical).
	\end{itemize}
	For the result of a batch control by sampling to be scientifically reliable, it is trivially important that, when collecting a sample of individuals, no preference is given to individuals from the whole population. Ideally, each individual batch must have the same probability of being draw from the sample.

	The characteristics controlled can be:
	\begin{itemize}
		\item Qualitative (attributes): the appearance of a product, the presence or absence of a non-compliant element, failure to comply with a caliber of control, an achieved result  or not. Individuals are then directly classified as compliant or non-compliant identified by the statistical number of defects they have. 
		
		\begin{tcolorbox}[title=Remark,colframe=black,arc=10pt]
		In my job of consultant I saw once a customer that had for process to reject definitively any supplier that do one and only one first failure. This is a zero tolerance approach and is quite not scientific as I warn this customer... But when people don't have any statistical culture it's a waste of time trying to eplain them why it's probable not the best strategy...
		\end{tcolorbox}

		\item Quantitative (variables or measurement): characteristic measured on a continuous scale: volume, length, weight, ...
	\end{itemize}
	Let us notice finally that there are two ways to classify acceptance plans. The first family is to define the following two categories:
	\begin{enumerate}
		\item The "\NewTerm{control of the proportion of non-compliant items by counting}\index{control of the proportion of non-compliant items by counting}" or also simply named "\NewTerm{attribute control}\index{attribute control}": One or more characteristics of qualitative or quantitative type are controlled for each item in order to classify it as compliant or non-compliant according certain criteria. The decision is based on the number of non-compliant individuals found in the samples.

		\item The "\NewTerm{control of the proportion of non-compliant individuals by measuring}\index{control of the proportion of non-compliant individuals by measuring}" or or also simply named "\NewTerm{variable control}\index{variable control}". A characteristic is measured for each item and a decision is made based on the average and dispersion of the characteristic calculated on all the items collected.
	\end{enumerate}
	We can also classify the acceptance plans depending on the number of samples taken.
	\begin{itemize}
		\item A "\NewTerm{simple sampling plan}\index{simple sampling plan}" involves taking $n$ items in a batch of size $N$ and make a decision based on them.

		\item A "\NewTerm{double sampling plan}\index{double sampling plan}" involves taking a sample from the first batch and draw one of these three conclusions: (1) accept the batch, (2) reject the batch or (3) resample. If a second sample is taken, the information of the two samples are gathered to decide whether to accept or reject the batch (if we repeat this strategy, then we speak of "\NewTerm{multiple sampling plan}\index{multiple sampling plan}"). The idea is illustrated in the following figure:
		\begin{figure}[H]
			\begin{center}
			\includegraphics[scale=0.75]{img/engineering/double_sampling_plan.jpg}
			\end{center}	
		\end{figure}
		\item A "\NewTerm{progressive or sequential sampling plan}\index{progressive or sequential sampling plan}" is the extreme case of multiple sampling plan. The items are removed one by one and after each sampling a decision is made: (1) accept, (2) reject or finally (3) take a new item.
	\end{itemize}
	\begin{tcolorbox}[colframe=black,colback=white,sharp corners]
	\textbf{{\Large \ding{45}}Example:}\\\\
	A soft drink producer wants to control a property of the bottles are supplied to him. For this, it uses a simple reception plan by monitoring the proportion of nonconforming items by a measuring when a batch of bottles are delivered to him:

	\begin{enumerate}
		\item Collect a number $n$ of units (bottles) of the batch (single sampling)

		\item Measure pressure at which each of the $n$ bottles broke (destructive testing)

		\item Calculate the statistical indicators of the observed values.

		\item Follow the acceptance decision rule or rejection that will be determined.
	\end{enumerate}
	\end{tcolorbox}
	\begin{tcolorbox}[title=Remark,colframe=black,arc=10pt]
	As for the control charts we will see further below, applying a sampling plan is a hypothesis test. It can lead to right or wrong decisions. The risks are those of accepting a non-compliant batch (Type II Error) or to reject a batch that is compliant (Type I Error).
	\end{tcolorbox}
	Obviously sampling has advantages and disadvantages (pros \& cons if you prefer). Here is the list of the most common advantages:
	\begin{enumerate}
		\item It's cheaper because you do not control everything ...

		\item There is less manipulation of the products thus less waste

		\item There needs to mobilize less people or control machines

		\item This dramatically reduces inspection errors

		\item The return of a whole batch just because of an non-compliant sample tends to motivate more the supplier to the quality ...
	\end{enumerate}
	But it also has disadvantages such as:
	\begin{enumerate}
		\item The risk of accepting non-compliant batches (type II error) and reject good one (Type I error)...

		\item Requires more intellectual skills of employees working in quality department (mathematical capacities...) and well-written contracts ...
	\end{enumerate}
	
	\subsubsection{Simple acceptance sampling plan by measurement for a unique tolerance with known standard deviation}
	For this type of sampling plan, we will assume that the sample $n$ is much smaller than the total batch of size $N$ (a factor of $10$ at least) such that $n\ll N$ and the standard deviation $\sigma$ is known!
	
	The characteristic $X$ measured for each item of the sample is assumed to follow a Normal distribution with  identical mean $\mu$ and standard deviation $\sigma$ (or variance) and the calculation of these last two parameters is based on the use of the sample mean (estimator of maximum likelihood of the mean as seen in the section Statistics) and of the unbiased estimator of the standard deviation (maximum likelihood estimator of the standard deviation as seen in the section Statistics). This hypothesis implies that the product of supplier is manufactured under statistical control!
	
	\begin{tcolorbox}[colframe=black,colback=white,sharp corners]
	\textbf{{\Large \ding{45}}Example:}\\\\
	In our example, the producer of soft drinks, he gets lots of $10,000$ bottles (so we will consider that the sample satisfy the above condition of $n \ll N$).
	\end{tcolorbox}
	We write (\SeeChapter{see section Statistics page \pageref{gauss distribution}}):
	
	\begin{tcolorbox}[title=Remark,colframe=black,arc=10pt]
	The proper way to compute probability of acceptance should be to use the hypergeometric distribution as we have seen in the section Statistics. But if the lot size is large relative to the sample size, the binomial distribution may be used. If the probabilities are quite small the binomial distribution can be approximated by a Normal distribution. This approximation is quite satisfactory if the lot size is more than ten times the sample size and non-compliant items are rare. In other words, if the lot size $N$ is large enough to be declared as infinite, the distribution of the number of non-compliants in a random sample of $n$ units will be binomial with parameters $\mu=np$ and $\sigma=np(1-p)$ as proved in the section Statistics.
	\end{tcolorbox}
	We will agree that the customer and the supplier have agreed on specification limits (tolerances). We denote will denote as usual these limits USL and LSL.

	Given $D$ the proportion (unknown!) of non-compliant products (defective), it is naturally given in the field statistical processes control by:
	
	In practice, in order to save money in the context of quality tests, we try to reduce (reformulate) any control problem to a single caliber. In fact the problem can always with or without reformulation be reduced to have a proportion of non-compliant below the LSL or above USL to reject the batch if the supplier's production is in statistical control. Hence the fact that in practice the mathematical developments are made only in relation to a single bound which we will denote by BSL for "\NewTerm{batch specification limit}\index{batch specification limit}".

	Thus, we reduce the problems of single sampling (or we manage the problems such that it can be reduced to such a formulation):
	
	\begin{tcolorbox}[colframe=black,colback=white,sharp corners]
	\textbf{{\Large \ding{45}}Example:}\\\\
	In the context of our companion example, the producer of soft drinks, it is the pressure resistance of a batch of $10,000$ bottles that we would like to control. The producer has agreed with its supplier to a lower specification limit for this property, choosen at $22.5$ limits a majority of the bottles must exceed without blow up. So:
	\begin{gather*}
		D=P(X<LSL)=P(22.5<LSL)
	\end{gather*}
	We will assume that past data were used to estimate precisely the standard deviation of the blow up pressure limit and which can therefore be considered as known and equal to $1.5$ and is under statistical control.
	\end{tcolorbox}
	Because obviously if the proportion of non-compliant items is more (or less) that defines the quality contract in a boud, it will be also the same the opposite bound in the same proportions as the supplier is supposed to have its manufacturing process under statistical control!!!

	Before going further into the mathematical developments, there is an important thing to remember:

	If we imagine ourselves to be a supplier, it is obvious that we would like to have a level of quality such that the cumulative probability that the customer mistakenly reject a batch is small (because as a supplier we are contractually obliged to take the back the batch even if a full second sampling control later shows that he seems to be wrong!). For this purpose, the supplier defines a value $D_s$ of the percentage of non-compliant products, named "\NewTerm{Acceptable Quality Level (AQL)}\index{acceptable quality level}" below which he assumes that the batch may be rejected only very rarely. In addition, it also defines $\alpha$ as the maximum cumulative probability of seeing a batch being reject that has a percentage of non-conformity less than or equal to $D_f$ (so it is the cumulative probability that a customer would reject by error a batch). This is written as:
	
	more of often written:
	
	where is named course ... "\NewTerm{suppliers's risk}\index{suppliers's risk}" and is generally in the range of $0.1$ to $10\%$.
	
	Finally, if we imagine that we are the customer, it is obvious that we would like the cumulative probability of wrongly accepting a batch that does not meet the contract quality to be quite small for obvious cost reasons. For this, the customer must on his side defines a value $D_c$ of the proportion of non-compliant items, know under the name "\NewTerm{Quality Level Limit QLL}\index{quality level limit}" or following NIST "\NewTerm{Lot Tolerance Percent Defective LTPD}\index{Lot Tolerance Percent Defective}", beyond which it considers that the batch can be accepted wrongly only rarely. He also defines $\beta$ as the maximum cumulative probability of having to accept a batch that has a proportion of non-compliant item greater than or equal to $D_c$ (so it is the cumulative probability that wrongly accept a batch of the supplier) . This is written as:
	
	more often written:
	
	where $\beta$ is named obviously ... "\NewTerm{customer risk}\index{customer risk}" and is also generally in the range of $5$ to $10\%$.
	
	To summarize the situation may be illustrated by the following table (which is similar in all point to the table of Type I and II errors in already seen in the section Statistics):
	\begin{center}
	  \renewcommand{\arraystretch}{2.6}
	  \begin{tabular}{|l|c|c|c|}
	  \hline
	    \cellcolor{black!30}   & \multicolumn{2}{|c|}{\cellcolor{black!30}\textbf{State of Nature}} \\ \hline
	\cellcolor{black!30}\textbf{\parbox{3.5cm}{Decision resulting\\ from sampling}} & \parbox{5cm}{The batch is not good (QLL)\\\centering ($H_0$ \textbf{false})} & \parbox{5cm}{The batch is good (AQL)\\\centering ($H_0$ \textbf{true})} \\ \hline
	\textbf{Reject (batch) $H_0$} & \cellcolor{green!30}\parbox{5.5cm}{Correctly reject null decision\\ \centering($1-\beta$: Power of the test)} & \cellcolor{red!30}\parbox{3cm}{Type I Error\\ \centering(Risk $\alpha$)} \\[3ex] \hline
	\textbf{Fail to reject (batch) }$H_0$ & \cellcolor{red!30}\parbox{3cm}{Type II Error\\ \centering(Risk $\beta$)} & \cellcolor{green!30}\parbox{4.5cm}{\centering Correct decision}  \\[3ex] \hline
	  \end{tabular}
	\end{center}

	\begin{tcolorbox}[colframe=black,colback=white,sharp corners]
	\textbf{{\Large \ding{45}}Example:}\\\\
	In the context of our companion example - the producer of soft drinks - which receives batches of $10,000$ bottles whose pressure resistance (BSL) must be greater than $22.5$ with a standard deviation $\sigma$ under statistical control equal to $1.5$, the supplier and the customer each have fixed the risks they are ok to take:
	\begin{itemize}
		\item The supplier would like that a batch with less than $1\%$ of non-compliant to be rejected by error by the customer to be at maximum of $5\%$ of cases (cumulative probability of type I error). Thus:
		\begin{gather*}
			P(\text{Reject a batch}|D\leq \text{AQL})\leq \alpha\Rightarrow  P(\text{Reject a batch}|D\leq 1\%)\leq 5\%
		\end{gather*}

		\item The customer would like that a batch withmore than $5\%$ of non-compliant bottles to be wrongly accepted in less than $10\%$ of cases (cumulative probability of type II error). Thus:
		\begin{gather*}
			P(\text{Accept a batch}|D\leq \text{QLL})\geq \alpha\Rightarrow  P(\text{Accept a batch}|D\geq 5\%)\leq 10\%
		\end{gather*}
	\end{itemize}
	\end{tcolorbox}
	\begin{tcolorbox}[title=Remark,colframe=black,arc=10pt]
	The decision rule to accept or reject a batch for a control by measurement is generally based on the estimated arithmetic mean of the characteristic of the sample rather than based on a estimate of non-compliant proportion.
	\end{tcolorbox}
	For what follows, when we write:
	
	the specificity is that in practice $D$ is requsted and $\sigma$ is given. The remainder is either to calculate or to eliminate (indeed the quality practitioner should normally only talk in terms of cumulative probability and proportions or factors  to define its quality policy and LSL is not a cumulative probability neither a percentage or a factor independent of the thing that we analyze!). Thus we write:
	
	So with the usual score (centered reduce variable as seen in the section Statistics):
	
	Obviously we draw from this that limit average value of the sample is:
	
	where $Z_D$ is the level $D$ of percentile of the standard centered Normal distribution (as usual). Unfortunately, we still have LSL that is there. Can we eliminate it? For this we have in the context that if the measurement (or manufacturing) is under statistical control:
	
	The cumulative probability to accept (we could have done it for the rejection but we had to choose one of the both for the example...) the batch when the non-compliant rate is $D$ is deducted in the context of using the average of the measurements becomes:
	
	in the special case where the measurement should not be less than some given value BSL (as is the case of the above example with the soft drink bottles!). Obviously $k$ could be a real positive or negative number as we know but not equal to $Z_D$ and we will how to found a way to determine it from the supplier and customer risk level!!!

	Then we have:
	
	Then we have a very interesting result! The cumulative probability to accept the batch when the non-compliant rate is $D$ is then given in the particular case there where the measurement should not be below a specific limit by:
	
	and the probability to reject it in the case of a measure beyond which we should not go:
	
	and what is good in this result is that the problem is then reduced to a percentile  $Z_D$ that is easy to choose (norm or policy choices), of the sample size $n$ and a factor $k$ so only independent elements of the type of measurement itself and that can then be used in any job for any type of object and facilitates the writing of contract (plus we have eliminated the explicit use of the standard deviation) !!

	Obviously, as we know, the supplier and the customer must agree to a sampling plan corresponding to build quality tests meeting expectations of the one and the other. What we know so far is that they respectively require an $\alpha$ (supplier risk) and a $\beta$ (customer risk) that allows them to immediately calculate the $Z_{D,\alpha},Z_{D,\beta}$ respectively. However it is necessary to calculate the number of individuals $n$ sampled and the factor $k$ that satisfy the respective requirements.

	To do this, remember that if $\alpha$ is the supplier risk (cumulative probability that the customer mistakenly rejects the batch) and the $\beta$ the customer risk (cumulative that he wrongly accepts supplier batch) and that:
	
	the cumulative probability of accepting a batch with a lower limit, then we must (!) have:
	
	or after rearrangement:
	
	To understand why we write these both relations the reader can take the two extreme situation where $\alpha=\beta=0$ and $\alpha=\beta=1$ (with their respective $Z_{D,\alpha},Z_{D,\beta}$) to better understand (we can detail on request with an explicit example).

	If we solve this system we can get the $n$ and $k$ we are looking for so that the customer and supplier get the sampling plan their both expecting!

	So we can rewrite the previous system as:
	
	Therefore:
	
	from which we can take out $n$:
	
	Finally:
	
	and for $k$:
	
	So that finally:
	
	\begin{tcolorbox}[colframe=black,colback=white,sharp corners]
	\textbf{{\Large \ding{45}}Examples:}\\\\
	E1. In the context of our companion example - the producer of soft drinks - which receives batches of $10,000$ bottles whose pressure resistance (LSL) must be at least of 22.5 with a standard deviation under statistical control equal to $1.5$, remember that the supplier and the customer each have fixed the risks they are ok to assume such that:
	\begin{gather*}
			P(\text{Reject a batch}|D\leq \text{AQL})\leq \alpha\Rightarrow  P(\text{Reject a batch}|D\leq 1\%)\leq 5\%\\
			P(\text{Accept a batch}|D\leq \text{QLL})\geq \alpha\Rightarrow  P(\text{Accept a batch}|D\geq 5\%)\leq 10\%
	\end{gather*}
	We then have using Microsoft Excel 11.8346:
	
	and (calculated with the same Microsoft Excel functions so we omit to explicit them again):
	
	The sampling plan is then to collect and analyze a sample of size $n=18$, calculate the arithmetic average $\bar{X}$ and apply the following decision rule:
	
	Let us suppose now that a sample size of $18$ is taken and gives the following results:
	\begin{gather*}
		25.6, 26.0, 23.0, 27.0, 27.5, 29.0, 28.5, 26.0, 25.6\\
25.8, 26.5, 28.8, 27.3, 25.2, 27.1, 29.8, 26.5, 27.8\\
	\end{gather*}
	Then we have:
	\begin{gather*}
		\bar{X}=26.8\geq 25.4
	\end{gather*}
	and therefore the batch is accepted. You should know that many practitioners prefer the following calculation using a "\NewTerm{quality index}\index{quality index}", denoted $Q$, simply using the following equation:
	\begin{gather*}
		Q=\dfrac{\bar{X}-LSL}{\sigma}=\dfrac{26.8-22.5}{1.5}\cong 2.86>k
	\end{gather*}
	therefore the batch is accepted.
	\end{tcolorbox}
	
	\begin{tcolorbox}[colframe=black,colback=white,sharp corners]
	\begin{tcolorbox}[title=Remark,colframe=black,arc=10pt]
	In the latter case it is still recommended in practice to calculate the unbiased estimator of the standard deviation of the sample to ensure it is not too far from the standard deviation of the contract (even if it may not be the same for small samples)!
	\end{tcolorbox}	
	E2. A second typical calculation made by practitioners (often just for curiosity or sometimes to make a quality report to the board committee) resulting from the first example is: What is the cumulative probability that we reject with $n$ and $k$ given, a batch of which the proportion of non-compliant $D$ would be $5\%$? Then, we use the following proved relation:
	\begin{gather*}
		P_D=P(Z\leq \sqrt{n}(k+Z_D))
	\end{gather*}
	Thus with the previous values:
	\begin{gather*}
		P_D=P(Z\leq \sqrt{18}(1.943+\texttt{NORMSINV(5\%)}))\cong 89.70\%
	\end{gather*}
	and the cumulative probability that we accept the batch is then of $1-89.70\%$ or $10.29\%$.
	\end{tcolorbox}
	
	\paragraph{Calculation of the parameters using the norms AF-X06-023}\mbox{}\\\\
	The norm AFNOR X06-023 offers a slightly different method to determine $k$ and $n$, not directly based on the customer's risk and the supplier's risk, but only base on the "Acceptable Quality Level" (AQL.). In addition, this norm also takes into account the size $N$ of the tested batch (considered previously as infinite) and of the desired level of control.
	\begin{tcolorbox}[title=Remark,colframe=black,arc=10pt]
	The equivalent international norm to the  AFNOR X06-023 seems to be the ISO 3951:1981 that to my knowledge gives very different values from that of the AFNOR for reasons that I don't understand quite well (I hope it's just because I did not understand something...).
	\end{tcolorbox}	
	The method proposed by AFNOR X06-023 isofthe type "recipe" but is requested by certain European standards so the industry have no choice but to apply it. But at the end, the numerical values are relatively close to those obtained with the previously demonstrated relations. In fact their real usefulness is that they are simple to implement in much more complex situations than those we have just studied.

	Depending on the context, the norm proposes to apply different levels of control:
	\begin{itemize}
		\item The "normal control" that has to be chosen when we are in close trusted relation with the supplier and we have a look in real time on the quality of its production using computer remote monitoring tools .

		\item The "reinforced control", which is stricter than the normal control ($n$ is greater) and is intended to better protect customers against the risk of having to accept bad batches (Type I Error). This type of control should be performed temporarily when there are serious reasons for considering that the production quality is not (or is no more) what is should be until we return back to the normal (deliveries are suspended between this time).

		\item Finally, the "reduced control" which is the most economical one and can be applied when based on previously tested and accepted batches, we can believe that the supplier master well the quality its processes
	\end{itemize}
	\begin{tcolorbox}[colframe=black,colback=white,sharp corners]
	\textbf{{\Large \ding{45}}Examples:}\\\\
	E1. In the context of our companion example - the producer of soft drinks - which receives batches of $10,000$ bottles whose pressure resistance (LSL) must be at least of 22.5 with a standard deviation under statistical control equal to $1.5$, remember that the supplier and the customer each have fixed the risks they are ok to assume such that:
	\begin{gather*}
			P(\text{Reject a batch}|D\leq \text{AQL})\leq \alpha\Rightarrow  P(\text{Reject a batch}|D\leq 1\%)\leq 5\%\\
			P(\text{Accept a batch}|D\leq \text{QLL})\geq \alpha\Rightarrow  P(\text{Accept a batch}|D\geq 5\%)\leq 10\%
	\end{gather*}
	We must then use the norm AFNOR X06-023 with a NQA of $1\%$.\\

	The norm requires the choice of a control level . We will take that of type II. A table in this norm tells us that we must use the code \textbf{L} for the next step. The final table thus gives us for a NQA of $1\%$ and a $L$ coll the following cell:
	\begin{center}
	  \begin{tabular}{|c|c|c|}
	  \hline
	    \multicolumn{2}{|c|}{$1.06\%$} \\ \hline
		$n=25$ & $k=1.97$ \\ \hline
		\multicolumn{2}{|c|}{$4.28\%$} \\ \hline
	  \end{tabular}
	\end{center}
	corresponding to:
	\begin{center}
	  \begin{tabular}{|c|c|c|}
	  \hline
	    \multicolumn{2}{|c|}{$D_s$} \\ \hline
		$n$ & $k$ \\ \hline
		\multicolumn{2}{|c|}{$D_c$} \\ \hline
	  \end{tabular}
	\end{center}
	or more traditionally:
	\begin{center}
	  \begin{tabular}{|c|c|c|}
	  \hline
	    \multicolumn{2}{|c|}{\textbf{AQL}} \\ \hline
		$n$ & $k$ \\ \hline
		\multicolumn{2}{|c|}{\textbf{LTPD (QLL)}} \\ \hline
	  \end{tabular}
	\end{center}
	Therefore, the norm indicates that a sample size $n$ of $25$ should be used and a $k$ of $1.97$! The plan insures risks slightly smaller than the plan calculated above but is more expensive because requires bigger samples.
	\end{tcolorbox}
	\pagebreak
	\begin{tcolorbox}[colframe=black,colback=white,sharp corners]
	E2. Practitioners when they use norms often want to calculate the customer and supplier risk associated to the selected AQL and the corresponding parameters given by the norm. Then using:
	
	It comes:
	
	Either as part of our example and always using Microsoft Excel 11.8346:
	
	Therefore:
	
	to be compared to the $5\%$. As well as:
	
	Either as part of our example and always using Microsoft Excel 11.8346:
	
	it comes:
	
	and finally:
	
	to be compared to the $10\%$.
	\end{tcolorbox}
	
	\pagebreak
	\subsubsection{Simple acceptance sampling plan by attribute}
	The attributes acceptance plans are simpler than the sampling plans by measurement, but they are often more expensive (in number of samples) than acceptance plans by measurement and are therefore advise only when we can not summarize information on the quality of a unit by measurement or that the cost of analyzes my measurement are too high compared to a simple classification of compliant or non-compliant units.

	The "\NewTerm{simple acceptance plan by attributes}\index{simple acceptance plan by attributes}", as already mentioned, is an acceptance test procedure the consist to draw a random sample of a given size $n$ in a batch. The sampled units (in a homogeneous way!) rre then inspected and classified as compliant or non-compliant (avoid using the terme "defective" as it is not same!). 

	Given $X$ be the number of non-compliant components found in the sample. If $X\leq A$, the quality of the batch is considered good and the batch is accepted. If $X\geq R$ with in the case of simple acceptance plan by attribute: $R = A+1$ the batch quality is considered as poor and the batch is not accepted.
	\begin{tcolorbox}[colframe=black,colback=white,sharp corners]
	\textbf{{\Large \ding{45}}Example:}\\\\
	Bottles by batch size of $N = 10,000$ are received by a customer. The control procedure used empirically a simple acceptance plan by attribute: it collects a sample of size $n = 200$. The acceptance criterion $A$ is $X\leq 1$ and the rejection criterion $R$ is $X\geq 2$.
	\end{tcolorbox}
	Let us recall before going futher into the mathematical details of the calculations of the strategy of a single sampling plan by attributes the principles of double and multiple sampling plans by attributes for the general culture:
	\begin{itemize}
		\item The "\NewTerm{double acceptance plan by attributes}\index{double acceptance plan by attributes}" is an acceptance test procedure having two stages of decision. It involves first taking a first sample size of a size smaller  than the one we would collect for a single acceptance plan. The first stage of decision is based on information provided by the first sample. If the quality of the first sample is considered good enough, the batch is accepted. If judged bad enough, the batch is not accepted. However if the quality of the first sample is judged as intermediate, a second sample is taken and inspected in order to accept or reject the batch. the second stage of the decision criteria are based on information collected on the two samples.
		\begin{tcolorbox}[colframe=black,colback=white,sharp corners]
		\textbf{{\Large \ding{45}}Example:}\\\\
		Bottles by batch size of $N = 10,000$ are received by a customer. The control procedure used empirically the following acceptance plan by attributes: it takes a first sample size $n_1=125$. The acceptance criterion $A$ in the first stage is the $X_1=0$ and the rejection criterion $R$ is $X_1\geq 2$. If the quality is not satisfactory, we take a second sample of size $n_2=125$ with an acceptance criterion $A$ in the second stage of $X_2\leq 1$ and the rejection criterion $R$ is $X_2\geq 2$.
		\end{tcolorbox}
		We will come back on a much more detailed and formal example of double acceptance plan by attribute further below!!
		
		\item The "\NewTerm{multiple acceptance sampling plan by attribute}\index{multiple acceptance sampling plan by attribute}" is simply a generalization of the double acceptance sampling plan by attributes having $D$ decision steps instead of $1$ for the simple acceptance plan or $2$ for the double acceptance plan...
	\end{itemize}
	Let us come back to our mathematical developments of a single sampling acceptance plan by attributes. If we denote now $n$ the batch size, $p$ the sample size and $m$ the total number of defective in the batch, then the probability of having $k$ non-compliant items among $p$ is (always using as from the beginning of this book a notation for the binomial coefficient that is not-compliant with the norm ISO 31-11):
	
	To be consistent with the tradition of the field of acceptance plan, let's first change these notations. Let us denote by $n$ the sample size and $N$ the batch size. It comes then:
	
	Let's do a second change of notation. Let us denote by $p$ the proportion of non-compliant items in the batch. It then comes the traditional customary notation:
	
	Thus, the cumulative probability of accepting a batch of quality $p$ is given by:
	
	\begin{tcolorbox}[colframe=black,colback=white,sharp corners]
	\textbf{{\Large \ding{45}}Example:}\\\\
	What is the cumulative probability of drawing $5$ non-compliant items (acceptance criterion) of a batch of $10,000$ bottles ($N$) from which we took a sample of $200$ individuals ($n$), the proportion of non-compliant is known to be of $5\%$ ($p$). So with Microsoft Excel 14.0.6123 we get as cumulative probability of acceptance:
	
	\end{tcolorbox}
	\begin{tcolorbox}[title=Remark,colframe=black,arc=10pt]
	A classic common mistake business that I see also Quality Manager of the best luxury international brand based in Switzerland is to think that the acceptance criteria is proportional to the batch size. In other words that if for $10,000$ bottles, the fact to a test sample of $200$ itmes ($2\%$) gives a cumulative probability of acceptance of $\cong 6\%$, then on $1,000$ bottles, a sample of $20$ items (still $2\%$) gives the same cumulative probability of acceptance. In fact, with Microsoft Excel 14.0.6123, we then not have $6.05\%$ anymore but $99.7\%$ (!!!)... the difference is then considerable !!
	\end{tcolorbox}	
		If we recall (see above the simple acceptance plans by measurement) that if $\alpha$ is the supplier risk (cumulative probability that the customer rejects by error the batch whose proportion of non-compliant items is equal to $p_\alpha$: type I Error) and $\beta$ the customer risk (cumulative probability of accepting by error a batch of a supplier whose proportion of non-compliant items is $p_\beta$: type II Error), then so that the supplier and the customer have the sampling plan corresponding to each of their quality requirements we have to solve following system:
	
	As far as we know it is not possible to solve this system analytically. The techniques of operational research techniques  (\SeeChapter{see section Theoretical Computing page \pageref{operational research}}) using the simplex method, the gradient method or Newton method fail miserably in finding a solution to this system (which is quite normal in fact...). By cons, with evolutionary algorithms (tool available in Microsoft Excel 14.0.6123 or also in MATLAB) we can find acceptable solutions, but the settings are not easy.

	Therefore, assuming that the size of batches submitted for inspection is large relative to the sample size $N\gg n$, and using the binomial approximation of the hypergeometric distribution (approximation proved in the section Statistics), we have:
	
	The advantage of this approximation is huge! The total size $N$ of the batch is no longer involved in the problem!
	\begin{tcolorbox}[colframe=black,colback=white,sharp corners]
	\textbf{{\Large \ding{45}}Examples:}\\\\
	E1. What is the cumulative probability of drawing $5$ non-compliant items (acceptance criterion) of a batch of $10,000$ bottles ($N$) from which we took a sample of $200$ individuals ($n$), whose proportion of non-compliant is known to be as equal to $5\%$ ($p$). While  Microsoft Excel 14.0.6123 we get:
	
	E2. The latter relation is also used in reliability engineering (see earlier above)!!! Indeed, suppose we wish to know the number of elements $n$ we need to take to make a reliability demonstration plan showing with $90\%$ of cumulative probability that the number $A$ of failures  (non-compliance) is equal to zero knowing that their reliability is $80\%$. Then this is equivalent to search $n$ such that we are the closest to:
	
	Therefore:
	
	So the nearest value $n$ for this is to take $11$ items (this is the value that also returns the worldwide reference reliability software what Weibull++). We are then let make use of the same previous calculation:
	\begin{center}
		\texttt{BINOMDIST(0,11,20\%,1)}$=8.59\%$
	\end{center}
	Within the area of reliability, it is customary to name this approach "\NewTerm{non-parametric binomial demonstration test plan of the  reliability}\index{non-parametric binomial demonstration test plan of the  reliability}"... (we will come back on this subject further below).\\
	
	E3. Let us see now a last example relatively to reliability. Remember that we have proved earlier above that for the Weibull distribution with one parameter, we had:
	
	And let suppose that we previous tests have shown that at a time $t$ equal to $2,000$ days, we had a reliability of $80\%$ and we know that the shape parameter $\beta$is equal to $2$. We would like to know what sample size we need to take for a reliability test of $1,500$ days shows with maximum $90\%$ cumulative probability $1$ non-compliance (that is: $1$ failure). For this, we first determine the scale parameter:
	\end{tcolorbox}
	
	\begin{tcolorbox}[colframe=black,colback=white,sharp corners]
	
	Then it comes for this same test at $1,500$ day a reliability of:
	
	By injecting into the binomial law, we then have:
	
	Solving for $n$, we get the integer value $n$ of $32$ items (the same value as that returned the by the worldwide reference reliability software Weibull++). We then left with:
	\begin{center}
		\texttt{= BINOMDIST (1, 32, 1-88.2\%, 1)} = 10.50\%
	\end{center}
	\begin{tcolorbox}[title=Remark,colframe=black,arc=10pt]
	Again... a common (beginner) mistake that is classic in corporation is to think that the acceptance criterion is proportional to the batch size. In other words that if if for $10,000$ bottles, the fact to test a sample of $200$ items ($2\%$) gives a cumulative probability of acceptance of $\cong 6\%$, the a on a batch of $1,000$ bottles, a sample of $20$ itmes ($2\%$) gives the same cumulative probability of acceptance... In fact, with Microsoft Excel 14.0.6123, we have don't have $6.23\%$ anymore but $99.7\%$... the difference is huge!!
	\end{tcolorbox}	
	\end{tcolorbox}
	Returning to the theory .... the system we had then with the Hypergeometric law can be written when $N\gg n$ as:
	
	But once again, it is not possible (as far as we know) to solve the previous system analytically. In 1967, HR Larson then created a chart of the cumulative binomial frequently referred to as the "\NewTerm{Larson's binomial nomograph}\index{Larson's binomial nomograph}" (I never found or seen an equivalent for a hypergeometric law unfortunately...) which allows approximately to solve the problem of determining the value of $A$ and $n$ for the acceptance plan:
	\begin{figure}[H]
		\centering
		\includegraphics{img/engineering/larson_binomial_monograph.jpg}
		\caption[Larson's Binomial Monograph]{Larson's Binomial Monograph (source: Montgomery)}	
	\end{figure}
	\begin{tcolorbox}[colframe=black,colback=white,sharp corners]
	\textbf{{\Large \ding{45}}Example:}\\\\
	A batch of bottles is delivered under the form of $10,000$ units corresponding to $N$. We seek to establish an approval control acceptance plan by attributes with the cumulative probability (supplier risk) $\alpha$ of $1\%$ that the customer rejects the batch by error with less than $2.5\%$ ($p_\alpha$) of non-compliant. At his side the customer wishes a cumulative probability (customer risk) $\beta$ of $10\%$ of accepting by error a batch with more than $5\%$ ($p_\beta$) of non-compliant (in fact the requirements are independent of the size $N$ of the batch).\\

	The goal is therefore to determine the value of $A$ and $n$ using the nomograph. For this, as shown in the nomograph, we must draw two lines:

	\begin{itemize}
		\item Starting points of the two lines: $0.99$ ($100\%$ -$1\%$) and $0.1$ ($10\%$) of the right axis.

		\item Finishing point of the two lines: $0.025$ ($2.5\%$) and $0.05$ ($5\%$) of the left axis.
	\end{itemize}
	\end{tcolorbox}

	\begin{tcolorbox}[colframe=black,colback=white,sharp corners]
	and the intersection of the two lines will give the desired parameters. Thus in this case:
	
	We also see that if we put these two values inside:
	
	with Microsoft Excel, we're actually far from the expected result! So the error of the nomograph can lead to the impression of rough approximations! But change in Microsoft Excel  the value $30$ by $27$ and the value $700$ by $695$ and you'll see what you can conclude...
	\end{tcolorbox}
	We also see through this example that an attribute acceptance plan is actually also much more expensive than a acceptance plan by measurement (since we took the same values of $\alpha$ and $\beta$ as for the example with the same parameters as during our study of the acceptance plans with measurement but we have $n=700$ instead of $n\cong 18$).
	
	The reader will also have perhaps notic that $N$ does not affect the value of $A$ and $n$ when we use the Larson's nomograph. So if we use computer tools to find the solution, we can therefore put that $n$ is between $0$ and $1,000$ and $A$ between $0$ and $150$ as we can see it on the nomograph.
	
	\begin{tcolorbox}[title=Remark,colframe=black,arc=10pt]
	Finally, let indicate that some use the Poissonian approximation of the binomial distribution (when the probability (proportion) of non-compliants is very small and that the batch size is very big and sometimes even when these criteria are far from being respected ...) following:
	
	\end{tcolorbox}
	
	\paragraph{Calculation of the parameters using the norm ISO 2859-1}\mbox{}\\\\
	The norm ISO 2859 (which derives from the norm MIL 105E) is dedicated to sampling procedures for inspection by attributes. Along with the norm AFNOR X06-023 it is based on the concept of AQL (percentage of non-compliant items who should not be exceeded for a production, controlled on a series of batches, so that it can be considered ass satisfactory).

	Following this norm, the NQA must be one of the $26$ values recommended by the norm. We will choose for example here the inspection level number II of this norm (they are described in the norm documentation). With respect to the chosen effective, as the AFNOR norm, ISO 2589 provides a symbol (letter) which will then be used in a table. Finally, depending on the type of acceptance plan (simple, double or triple) we find the parameters $A$ and $n$.
	
	\begin{tcolorbox}[colframe=black,colback=white,sharp corners]
	\textbf{{\Large \ding{45}}Examples:}\\\\
	E1. In the context of our companion example - the producer of soft drinks - which receives batches of $10,000$ bottles, the AQL quality level is $1\%$. As for example with the AFNOR norm, we will take a type II control based on a simple acceptance plan.

	A table in the norm (see below) tells us that we must use the code $L$ for the next step. Then for simple control, the standard says to use to use the table IIA.
	\begin{figure}[H]
		\centering
		\includegraphics[scale=1.5]{img/engineering/mil_acceptance_plan.jpg}
		\caption{MIL 105E acceptance sampling table}	
	\end{figure}
	We can then read that $n$ must be equal to $200$ and $A$ is equal to $5$. We therefore find that the value obtained is very different from that the one calculated theoretically. Honestly the reason for this difference is unknown to me...\\

	E2. Practitioners in the use of this norm often want to calculate the supplier and customer risk factors associated to the values of $n$ and $A$ provided by the norm in relation to the NQA. selected for a batch of a given quality. So as in this case, $200/10,000$ is smaller than $10\%$, we can use the binomial approximation of the hypergeometric law. Suppose we therefore wish to calculate the risk for the supplier of the previous example (the one with the use of the nomograph):
	
	\end{tcolorbox}
	
	\begin{tcolorbox}[colframe=black,colback=white,sharp corners]
	hence a supplier risk of $100\% -61.59\%$ that is $38.40\%$ which is considerable! Thus, this sampling plan obtained via the norm is suitable for proportions of non-compliants much lower that $2.5\%$ (in fact less than $\cong 1.25\%$).\\
	
	For the customer risk, we proceed in the same way:
	
	which is here a more acceptable result. In fact we find that the norm is more appropriated to the criteria of the customer rather than that of the supplier.
	\end{tcolorbox}
	\begin{tcolorbox}[title=Remark,colframe=black,arc=10pt]
	The software Quick Control Pro (used sometimes by the Swiss Watch industry) of the company Logystem SA manages the acceptance sampling according to the norms ISO 2859 and ISO 3951.
	\end{tcolorbox}
	
	\subsubsection{Double acceptance sampling plan by attribute}
	Now let us come back on the double acceptance sampling using the binomial approximation that if for recall:
	
	and let us consider a double sampling plan with:
	
	and where the proportion of non-compliant is known as being equal to $5\%$. So, in this case the cumulative probability of having, for example, at least $1$ non-compliant item at the first sampling and corresponding to the probability of accepting the batch is equal to:
	
	So far nothing special! Now to get the probability of acceptance of the batch on the second sampling, we list all the ways we can get the second sampling. 

	For this, suppose for example that the batch is accepted if and only if the cumulative number of non-compliant items overall the both sampling is between $2$ and $3$ units (arbitrary simplified choice for the example).
	
	We then have the following possible ordered combinations of drawing between the two samplings (notice that the sum is always between $0$ and $3$):
	\begin{gather*}
		 \{0,2\}, \{0,3\}, \{1,1\}, \{1.2\}, \{3.0\}, \{2.1\}, \{ 2.0\}
	\end{gather*}
	 But as already mentioned we accept the batch if and only the first sampling gives a number of non-compliant equal or less than $1$, it then remains only the following combinations:
	\begin{gather*}
		 \{3,0\}, \{2,1\}, \{2,0\}
	\end{gather*}
	to consider in this double acceptance sampling. So we need to calculate the following probabilities:
	
	
	\subsubsection{Operating characteristic curve (OC)}
	The "\NewTerm{operating characteristic curve (OC)}\index{operating characteristic curve}"  depicts the discriminatory power of an acceptance sampling plan (by measurement or by attribute!). The OC curve plots the probabilities of accepting a lot versus the fraction defective.

	If we assume that the lot size $N$ is very large, as compared to the sample size $n$, so that removing the sample doesn't significantly change the remainder of the lot, no matter how many defects are in the sample. Then the distribution of the number of defectives, $d$, in a random sample of $n$ items is approximately binomial as we know.

	When the OC curve is plotted, the sampling risks are obvious. You should always examine the OC curve before using a sampling plan.

	The idea in practice is to compare OC curves to help choose the appropriate sampling plan for various $n$ (sampling size) and $A$ (acceptance number often denoted $c$ in the US literature of "criterion").
	
	This curve that that is obtained by plotting on a graph the cumulative probability of acceptance of a batch in function of the proportion of it non-compliant item (named sometimes "effective quality" axes).

	We can build this curve for the Normal approximation, Binomial approximation or also we the exact case of the Hypergeometric law!
	
	\begin{tcolorbox}[colframe=black,colback=white,sharp corners]
	\textbf{{\Large \ding{45}}Example:}\\\\
	For a quality control by attribute acceptance plan, we apply the control plan on a sample of $n=700$ items, the batch will be accepted if the number of non-compliant items is less than or equal to $30$ ($A\leq 30$) or in percentages ($A\leq 4.28\%$) as determined in our previous example!\\

	We want to plot the cumulative probability to accept the batch for a range of non-compliant proportion between $0-7\%$ by step of $0.1\%$.\\
	
	Thus the purpose is explicitly to plot:
	
	That is often written in the US literature as:
	
	Therefore in our example:
	
	In a spreadsheet software like Microsoft Excel 14.0.7166 this gives (first rows only):
	\begin{figure}[H]
		\centering
		\includegraphics[scale=0.9]{img/engineering/oc_curve_binomial_excel.jpg}
	\end{figure}
	\end{tcolorbox}
	
	\begin{tcolorbox}[colframe=black,colback=white,sharp corners]
	And if we do a plot adding at the same time the OC curve using the hypergeometric law and the ideal OC curve (sampling plan that discriminated perfectly between good and bad batches) we get:
	\begin{figure}[H]
		\centering
		\includegraphics[scale=0.7]{img/engineering/oc_curve_binomial_hypergeometric_ideal.jpg}
		\caption{Binomial, Hypergeometric and Ideal OC for AQL$=2.5\%$ and $n=200$}	
	\end{figure}
	Where remember that the values of $A$ and $n$ came from the fact that we seek to establish an approval control acceptance plan by attributes with the cumulative probability (supplier risk) $\alpha$ of $1\%$ that the customer rejects the batch by error with less than $2.5\%$ ($p_\alpha$) of non-compliant. At his side the customer wishes a cumulative probability (customer risk) $\beta$ of $10\%$ of accepting by error a batch with more than $5\%$ ($p_\beta$) of non-compliant (in fact the requirements are independent of the size $N$ of the batch).\\
	
	So as we can see, for $A=30$, $n=700$ we have almost $100\%$ probability to accept a batch that has between $0$ and almost $2.5\%$ proportion of non-compliant items. After the probability to accept a batch by error decrease as the real proportion of non-compliant items increase (this is quite obvious!).\\
	
	The reader can see in the Minitab companion book that we get the same binomial OC curve!
	\end{tcolorbox}
	An elegant way to represent the concepts of an operating curve without forgetting that in the binomial case:
	
	is as follows:
	\begin{figure}[H]
		\centering
		\includegraphics{img/engineering/oc_curve_summary.jpg}
		\caption{Operating Curve Summary}	
	\end{figure}
	The operating curve shows that the probability of accepting a defective batch decreases - for a constant batch size - when the percentage of defective items in the batch increases: when the quality is good, the probability of accepting the batch by error is high; when the quality is bad, the probability of accepting the batch by error is small.
	
	This type of operating curve based on large sampled  batches (dixit using the binomial distribution) is named in the case of using the binomial distribution "\NewTerm{operating  curve of type B (OC-B)}\index{operating  curve of type B}". If obviously the sample size is significant relatively to the size of the batch we will use the hypergeometric and we'll talk then of "\NewTerm{operating curve of type A (EC-A)}\index{operating curve of type A}".

	\subsubsection{Average outgoing quality (AOQ)}
	The batches not accepted by a sampling plan will usually be $100\%$ inspected or screened for nonconforming or defective units. After screening, nonconforming units may be rectified or discarded or replaced by good units, usually taken from accepted lots. Such a programmed of inspection is known as a "\NewTerm{rectifying inspection}\index{rectifying inspection}" or "\NewTerm{screening inspection}\index{screening inspection}". For those batches accepted by the sampling plan, no screening will be done and the outgoing quality will be
the same as that of the incoming quality $p$.

	Since the cumulative probability of accepting by error a bad batch is $ P_a$ , the outgoing batches will contain obviously proportion of $P_a\cdot p$ defectives. If only the nonconforming units found in
the sample of size $n$ are replaced by good ones, the "\NewTerm{average outgoing quality (AOQ)}\index{average outgoing quality}" (its an average proportion!!!) in in batches size of $N$ will be given obviously by:
	
	and for large $N$ we get the approximation:
	
	Using always our companion example with the bottle batch and using always the same spreadsheet software we have (first rows only):
	\begin{figure}[H]
		\centering
		\includegraphics{img/engineering/average_outgoing_quality_binomial_excel.jpg}
	\end{figure}
	and explicitly:
	\begin{figure}[H]
		\centering
		\includegraphics{img/engineering/average_outgoing_quality_binomial_explicit_excel.jpg}
	\end{figure}
	and the correspoding chart made made by selecting the both last columns:
	\begin{figure}[H]
		\centering
		\includegraphics[scale=0.7]{img/engineering/average_outgoing_quality_binomial_chart_excel.jpg}
		\caption{AOQ (Average Outgoing Quality) binomial chart $A=30$, $n=700$}
	\end{figure}
	The maximum ordinate of the AOQ curve is known as the "\NewTerm{average outgoing quality limit (AOQL)}\index{average outgoing quality limit}".
	
	So to close this subject we can that Minitab 17.3.1 gives us a resultat that is not far from our estimation of the Larson's nomograph (yes remember that we get earlier above $A\cong 30$ and $n\cong 700$) and that we also get the same operating curve and average outgoing quality plot:
	\begin{figure}[H]
		\centering
		\includegraphics[scale=0.65]{img/engineering/oc_by_attributes_minitab.jpg}
	\end{figure}
	
	\pagebreak
	\subsection{Quality Control Charts (CC)}\label{quality control charts}
	A "\NewTerm{control chart}\index{control chart}" (sometimes also refereed to as "\NewTerm{sequence chart}\index{sequence chart}" or "\NewTerm{Shewhart charts}\index{Shewhart charts}" or "\NewTerm{process-behavior charts}\index{process-behavior charts}") is a planar empirical representation of the variation of an accurate measurement of a process or a method where the vertical axis represents the quantitative indicator selected and the axis horizontal the time or corresponding number of outgoing  units (so it is a subfamily of "run charts").

	This nondestructive method of quality control that is say to be "\NewTerm{on-line}\index{on-line quality control}" (as the acquisition can be done in real time) permits to highlight special causes of variation and then to optimize the frequency of maintenance operations or recalibration) in the purpose to determine if a manufacturing or business process is in a state of statistical control (or move it in such a state) and to separate signal from noise! 
	
	This technique seems to have been created by physicist engineer Walter A. Shewhart while working for Bell Labs in the 1920s and is new since some decades (1991 to be exact) an ISO norm under the name \textit{ISO 8258:1991 - Shewhart control charts}. Moreover, it is also Shewhart who inspired the well known W. Edwards Deming (statistician) in the field of project management and quality (all the masters in the field of advanced project management have a scientific background...).

	In the early use of control charts, it is advisable to put control charts on all important measurable product characteristics. The resulting chart generally show quickly what charts are necessary or unnecessary (in reality the quality manager has also to conduct a CoQ- Cost of Quality - analysis). Unnecessary or inappropriate control charts will be removed and other control charts might be added.
	
	The most basic and intuitive planar control chart could looks like this:
	\begin{figure}[H]
		\centering
		\includegraphics[scale=0.6]{img/engineering/control_chart_naive.jpg}
		\caption{Naive control chart}
	\end{figure}
	The upper and lower limits can be requested / specified by a customer and therefore we note them, as we already know, USL (Upper Specification Limit) and LSL (Lower Specification Limit), otherwise they are calculated internally and therefore we denote them by "\NewTerm{UCL (Upper Control Limit)}\index{upper control limit}" and "\NewTerm{LCL (Lower Control Limit)}\index{lower control limit}". To improve the effectiveness of a control chart, we sometimes add on chart narrower limits named "\NewTerm{monitoring limits}\index{monitoring limits}" (in reality we almost always represent the limits  calculated AND requested by the quality control policy on the control chart). 
	
	The "\NewTerm{Center Line (CL)}\index{Center Line}" is most of time chosen as the arithmetic average of all measurement but it should be better to choose the Median as we already know. The "\NewTerm{Target ($T$)}\index{target}" is the value request as an ideal by the customer specifications.
	
	The equivalent run chart\label{run chart} of the above chart made with Minitab 17.3.1 is:
	\begin{figure}[H]
		\centering
		\includegraphics[scale=1]{img/engineering/control_chart_minitab_run_chart.jpg}
		\caption{Run Chart with Minitab 17.1.3}
	\end{figure}
	
	In practice I highly recommend to represent  the center line (CL) that  and also UCL, LCL and as in reality there is normally always limits and goals that are imposed contractually by the customer, to also represent the target $T$ and  USL, LSL on the same control chart when possible!!!
	
	The control chart is therefore a measurement tool for viewing and sometimes anticipating the changes and thus determine when an assignable cause occurs causing a drift of a business process or manufacturing processor a variation of a financial value on stock markets requiring rapid corrective action to reduce the costs of non-quality or losses. Thus, ideally the process will be stopped or corrected at the right time, that is to say before it produces too many non-compliant deliverables / services (outside the tolerance) and therefore to avoid overcontrol (useless re-calibrating because of common variability or application of acceptance plan).

	The type of control chart corresponds to the type of characteristic that is the subject of a control. However the economic aspect may be an important factor in the choice of the control chart to implement (time for measurement, destructive or non-destructive procedure and son...). If we look at whether a characteristic complies or not with certain norms, an "\NewTerm{attribute control chart}\index{attribute control chart}" (see further below for the details) will be used, but requires a big sample size. Control on quantitative and almost continuous characteristics requires the use of "\NewTerm{measurement control charts}\index{measurement control charts}" and if the measurement are not independent of "\NewTerm{time series control charts}\index{time series control charts}". This type of measurement control is more efficient and more precise but also more expensive since it requires the use of measuring instruments that should be checked or re-calibrated regularly. We have also for two dimensional correlated controls the "\NewTerm{multivariate control charts}\index{multivariate control charts}".
	\begin{tcolorbox}[title=Remark,colframe=black,arc=10pt]
	The subject of control charts is immensely vast and there is an abundant English literature on the subject. We wish to give here only the fundamental (and therefore only basic control charts!) to show a practical application of the theory or Rank Statistics and of Statistics of extreme values outliers developed that we have developed in detail in the section Statistics. To place an order of magnitude, most specialized software offers between 10 and 30 different control charts (Minitab, R, JMP, QI Macros MATLAB)!
	\end{tcolorbox}
	Before going into the mathematical details notice that the control charts among the most used one are the control charts by measurement  of the average $\bar{X}$ in combination with the control charts by range $R$ with Shewhart's limits (when we speak of "\NewTerm{Shewart's limits}\index{Shewart's limits}" we are implicitly referring to the fact that the upper control limit UCL and the lower limit control LCL are empirically calculated at $\pm 3\sigma$ of the arithmetic average of the distribution that the latter is symmetrical or not!). These both control maps are used and interpreted most of time together because both parameters are independent and complementary and simple to understand (even if the majority of employees in companies does not know what is a standard deviation). Indeed, the arithmetic average value can vary without that the range varies and vice versa\footnote{Remember the trap of the Anscombe's quartet that we saw in the section Statistics}.
	\begin{tcolorbox}[title=Remarks,colframe=black,arc=10pt]
	\textbf{R1.} In the case of a control charts with Shewhart's limits, monitoring limits are traditionally placed at $\pm 3\sigma$ (at the opposite of the control limits that are placed at $\pm 3\sigma$).\\
	
	\textbf{R2.} In the case of a Normal distribution $\mathcal{N}(\mu,\sigma)$, $\pm \sigma$ corresponds to a $99.73\%$ confidence interval as we have seen earlier in this section and also in the section Statistics. This therefore corresponds to an $\alpha$ of about:
	
	Most of the industry therefore takes $\pm 3\sigma$ as limits, but strictly speaking distributions of certain control chorts are not symmetrical. We should then take terminals (limits) corresponding to a cumulative probability of $\alpha/2$ which can be calculated quite easily with many distributions (but not all!) just by having available an inexpensive spreadsheet software. The choice of the value $\alpha$ depends on the company's quality policy and has the advantage of being a much more accurate indicator that the Shewhart's limits (the latter being wrong when the distributions are skewed).\\

	\textbf{R3.} Shewhart seems to proposed that there must b at least $25$ samples of $4$ individuals for the validity of certain control cards (that we will see further below) begins to be acceptable.\\
	
	\textbf{R4.} As we will see it, almost all measurement control charts to measures assume Normal distribution and that the observations are independent (between sub-groups and within subgroups!!). For example (example that will be detailed mathematically further below), for already stable processes as since started a long time ago, we perform several individual observations on several subgroups numbered at a given time frequency (hourly, three times per day ...). On each ordered subgroup $k$ (chronological incrementation), we perform $n$ observations. We report on the control chart the arithmetic average $\bar{X}$ of subgroup according to its chronological number $k$ to be reported on the horizontal axis of the control chart. Because of the central limit theorem (CLT) proved in the section Statistics, the average values $\bar{X}_k$ of the control chart follows a Normal distribution $\mathcal{N}(\mu,\sigma/\sqrt{n})$ (so symmetrical distribution!) that the observations of the subgroups are Normally distributed or not but at the condition there are independent and identically distributed over time!!! This assumption is valid even for small sample size and a manufacturing process under control, which is common in quality control. Production is said to be "stable" as we know if the tendency and dispersion are statistically constant over time (and considered identically distributed independent variables).
	\end{tcolorbox}
	
	\pagebreak
	\subsubsection{WECO's empirical rules}
	The interpretation of a control chart as mathematically  powerful and elegant it is is not easy and requires experience and expertise to know if corrective action is required or not.  The "\NewTerm{Western Electric Rules}\index{Western Electric Rules}" were codified by a specially-appointed committee of the manufacturing division of the Western Electric Company and appeared in the first edition of its \textit{Statistical Quality Control Handbook} in 1956. Their purpose was to ensure that line workers and engineers interpret control charts in a uniform way (the French health ministry did the same 50 years later ... but with rules that differ a little).

	The WECO based at this time on the assumption that the statistical distributions are always symmetrical (same assumption as Shewhart) and has then adopted in the statement of control  rules limits that are integer multiples $k$ of the standard deviation. Of course, nothing prevents the quality specialist to adapt these rules with probabilistic limits corresponding to a given confidence interval (the majority of specialized software gives the possibility to select the WECO rules to be applied automatically for the detection of non-conformities).

	In the measurement control charts type, among these rules here are those that are applicable in order to identify whether a process has failed and that have been completed over the years by other specialists (we have represent below only some  rules into a single control chart for obvious pedagogical reasons):
	\begin{figure}[H]
		\centering
		\includegraphics[scale=0.65]{img/engineering/control_chart_weco_rules_example.jpg}
		\caption[]{Some WECO rules on a Run Chart}
	\end{figure}
	On the next page you can found some of the WECO rules:
	
	\pagebreak
	\begin{enumerate}
		\item[W1.] A measured point is beyond the USL, LSL specified by the customer or beyond the empirical or probabilistic UCL, LCL corresponding to $\pm 3\sigma$.

		\item[W2.] Two consecutive measured points are beyond the empirical or probabilistic UCL, LCL corresponding to $\pm 2\sigma$.

		\item[W3.] Four consecutive measured points are beyond the empirical or probabilistic UCL, LCL corresponding to $\pm 2\sigma$ (in this case named "\NewTerm{alert limits}\index{alert limits}").

		\item[W4.] Eight consecutive points fall on the same side of the mean (even if once again the use of the median is more accurate and robust).

		\item[W5.] Six consecutive points are on an uptrend or downward trend respectively.

		\item[W6.] Fourteen consecutive points on a systematic up/down alternation trend.
	\end{enumerate}
	The rules presented above apply to control charts with symmetric control limits. The WECO handbook provides additional guidelines for control charts where the control limits are not symmetrical, as for $R$ charts and $p$-charts...
	
	As the reader can see below, Minitab includes at least $8$ empirical rules and that can be furthermore customized:
	\begin{figure}[H]
		\centering
		\includegraphics{img/engineering/control_chart_minitab_weco_measurement_rules.jpg}
		\caption[]{Example of WECO rules implemented in Minitab 17.3.1}
	\end{figure}
	\begin{tcolorbox}[title=Remark,colframe=black,arc=10pt]
	The reader can also refer to "\NewTerm{Westgard Rules}\index{Westgard Rules}" that are a set of rules used for laboratory quality control. They are a copyrighted set of modified Western Electric rules, developed by James Westgard and provided in his books and seminars on quality control.
	\end{tcolorbox}
	If the process/method is out of control, corrective actions must be implemented (process review, training, recalibration, code update, etc.). The causes of variation may be obviously random or deterministic. If the causes are only due to chance, then they are named "\NewTerm{random causes}\index{random causes}" (Shewhart) or "\NewTerm{common causes}\index{common causes}" (Deming) and are therefore treated as instantaneous dispersion. Not all changes are due to chance, of course, some are specific and identifiable in a deterministic a certain way. In the latter case, the variations are named "\NewTerm{assignable variations}\index{assignable variations}" (Shewhart) or "\NewTerm{special causes}\index{special causes}" (Deming) and are assimilated to the overall dispersion (see in the section Statistics the subsection about the propagation of errors). 

	Correct the special causes is obviously most of time much easier than correcting random causes. The primary purpose of control charts is obviously identifying special causes...
	
	OK that's all regarding the customs and traditions of interpretation of measurement points. Now this non-mathematical prelude done, let us see the different types of control charts with a concrete example for each.
	
	\subsubsection{Sample size and Sampling frequency for Control Charts}
	The use of control charts pose early two major difficulties:
	\begin{enumerate}
		\item Choosing the family and type of control chart as we have ($6$ main families, with $20$ control charts and including the typical variants to calculate a same control chart, we have a total almost $60$ control charts...!):
		 \begin{itemize}
			\item Variable charts for subgroups
			\begin{itemize}
				\item Xbar-R/S
				\item I-MR-R/S
				\item R
				\item S
			\end{itemize}
			\item Variable charts for individuals
			\begin{itemize}
				\item Levey-Jennings
				\item I-MR
				\item Z-MR
				\item Individuals
				\item Moving Range
			\end{itemize}
	
			\item Attribute charts
			\begin{itemize}
				\item P
				\item Laney P'/U'
				\item NP
				\item U
				\item C
			\end{itemize}
	
			\item Time-weighted charts
			\begin{itemize}
				\item Moving Average
				\item EWMA
				\item CUSUM
			\end{itemize}
	
			\item Multivariate charts
			\begin{itemize}
				\item T-squared
				\item Multivariate EWMA
			\end{itemize}
	
			\item Rare events charts
			\begin{itemize}
				\item G
				\item T
			\end{itemize}
		\end{itemize}

		\item The choice of the sample size $n$ and of the sampling frequency $f_s$.
	\end{enumerate}
	We would here discuss the second point! We must first observe that it is obviously the pace of execution and the quality of operation and also the time it takes to measure the feature (characteristic) that will give us the answer. This quality is appreciated by the average number of corrective interventions observed during a specific period in the past (MTBF typically). The more corrective action there was, the less the quality of the operation of the method / process is good and higher will be the sampling frequency.
	
	The choice of the number of the sample size is not a problem because as it will result from the statistical calculation of the confidence interval of the statistical indicator that we impose ourselves (see the study of Acceptance Plan earlier above) and therefore of the used distribution law and its properties and also of the assumptions of the chosen the control chart (see below in the text that follows the mathematical description of main control charts).
	
	The challenge is there to determine the sampling frequency. There are for this four main methods and the two first are quite empirical (rule of thumbs):
	\begin{itemize}
		\item Choose a frequency such that the corrective actions $T_c$ are at least four times lower than the sampling frequency (this criterion is independent of the sample size and therefore sometimes not very applicable...):
		
		
		\item We consider $T_p$ as the life of the method / process between each change / modification. We denote as usual $n$ the sample size respecting acceptance plan rules (see previously). We will denote $N$ the number of executions of the method / process (number of produced items) during the period $T$. So we have a maximum $N/n$ samples that are possible.:
		
		
		\item Another method consist to use the MTBF (Mean Time Between Failure) and that as we have proved during our study of reliability techniques is given for recall in the case of the Weibull distribution with three parameters by:
		
		or in the case of an exponential law:
		
		
		\item The last method consist to use a concept we will prove further below and that give a lower bound for the sampling frequency and that is the "average run length" given by:
		
	\end{itemize}
	In all control charts that we will present below, we have presented examples with constant sample size (hence the limits of control charts are always constant with some exceptions that we will also see). Indeed, if the size of the samples change it is that the production rate (or batch size of delivery) has changed or also the overall quality and therefore the manufacturing parameters too. Then we should start a new control chart! It is the same after a re-engineering or changes of the process as shown in the Minitab 15.1.3 screenshot below:
	\begin{figure}[H]
		\centering
		\includegraphics[scale=0.6]{img/engineering/control_chart_before_after_minitab.jpg}
		\caption[]{Example of before/after control chart in Minitab 15.1.4}
	\end{figure}
	
	\subsubsection{Attributes Control Charts (qualitative CC)}
	To measure qualitative variables (\% of defective, \% failures, ...), we use attributes $p$ control charts (binomial with proportion), $np$ control charts (binomial with counts), $c$ control charts (Poisson with counts), $U$ control charts (Poisson with percentages) or Laney control charts (two by two range normalized proportions) to monitor the attributes quality control over time.

	The attributes by control charts are very easy to study (undergraduate maths only!) use and interpret this is why we teach them always first, but has several disadvantages such as the high asymmetry of distributions, no lower limit for small sample sizes, low efficiency of deterioration detection.
	
	\pagebreak
	\paragraph{$P$ Control Charts (binomial proportion CC)}\mbox{}\\\\
	The $p$ control chart (binomial by proportion) is used when it comes to working with ratios, proportions or percentages of compliance or non-compliance of samples.

	Good examples of $p$-type control charts are: product inspection of a production line or receiving batches of a supplier or control malfunction of a number of devices, respect of deadlines, product specifications or also coding line errors (bugs).

	\textbf{Pros}: This type of control chart is very simple to construct because does not require any advanced statistical knowledge to be calculated.

	\textbf{Cons}: 
	\begin{itemize}
		\item In reality its a control chart which control limits may be asymmetric if the practical application are at the limit of acceptable when the theoretical assumptions of construction of this chart are not met. 
		
		\item Sometimes a difficulty encountered by practitioners of this chart is to determine the batch size and the level of acceptance ($A$) or rejection ($R$) equivalent to the USL by taking the decision based on the developments (and norms) of acceptance plans samples see earlier above. 
		
		 \item This control chart is not suitable for exhaustive sampling since it would then required the use of the hypergeometric law. Unfortunately, as far as we know, there is no analytic relation for the confidence interval of the proportion of a hypergeometric distribution, and therefore is not feasible to build a control chart  based on a hypergeometric law.
		
		\item A final difficulty is that some employees and board committee are not proficient with the concept of percentages.
	\end{itemize}

	The probability law used in this context is the binomial distribution (\SeeChapter{see section Statistics page \pageref{binomial distribution}}) where $p$ will represent the proportion in $\%$ of non-compliants and $q$ (which therefore is $1-p$) that proportion in $\%$ of compliants.

	We saw in the section Statistics that we were able to write formally, under certain assumptions, the following confidence interval for the real proportion:
	
	So we can calculate the probabilities of non-compliant batches by excess or by defect using the fact that $p$ follows under statistical control a Normal distribution of parameters:
	
	In short, we have no choice but to base this control chart on this assumption. Therefore, the first step in creating a $p$-type control chart is to calculate the proportion of non-compliance, at least its estimator for each batch $i$:
	
	where $k_i$ is the number of non-compliant items in the batch of size $N_i$.

	The average proportion of non-conformities, which corresponds to the center line (CL) of the control chart is given by (don't forget that using the median would be better in reality!):
	
	and if the $n$ batches of size $N_i$ are all of equal size $N$ this simplifies obviously to:
	
	Therefore we have:
	
	and if the batches size are equal  we have obviously:
	
	According to the study that we made during our study of acceptance plan they can not be any lower limit (LSL) imposed by the customer/board committee or target ($T$) in this field of application.

	So, with the following list where the batches size (special case) are identical (made with Microsoft Excel 14.0.7166):
	\begin{figure}[H]
		\centering
		\includegraphics{img/engineering/control_chart_p_excel_list_data.jpg}
	\end{figure}
	or explicitly:
	\begin{figure}[H]
		\centering
		\includegraphics[scale=0.55]{img/engineering/control_chart_p_excel_list_data_with_formulas.jpg}
	\end{figure}
	and the corresponding chart still with the same spreadsheet software:
	\begin{figure}[H]
		\centering
		\includegraphics[scale=0.7]{img/engineering/control_chart_p_plot_excel.jpg}
	\end{figure}
	with Minitab 17.1.3 this gives for people that don't trust our calculation made with Microsoft Excel (we can also do such a chart with R but the plot quality is not acceptable in a corporation):	
	\begin{figure}[H]
		\centering
		\includegraphics[scale=0.95]{img/engineering/control_chart_p_plot_minitab.jpg}
	\end{figure}
	where we took as example for limits imposed by the customer/board committee:
	
	obviously not visible on the chart as too far away on the $y$-axes.

	This control chart can also be interpreted as any another Shewhart chart using the empirical WECO rules.
	
	\paragraph{$NP$ Control Charts (binomial counting CC)}\mbox{}\\\\
	The $np$ control chart is obviously similar to the previous $p$ control but is the number (count) of non-compliant items units than the proportion that is plotted. This chart is easier to interpret as the $p$ control chart since not all employees are able to represent themselves the concept of percentage, but against it is very difficult to compare objectively two points coming from batches that do not have the same size.
	
	\textbf{Pros}: This type of control chart is very simple to construct because does not require any advanced statistical knowledge to be calculated. For some employees whose concept of percentage is poorly mastered, it is easier to read a $np$ chart.

	\textbf{Cons}: 
	\begin{itemize}
		\item In reality its a control chart which control limits may be asymmetric if the practical application are at the limit of acceptable when the theoretical assumptions of construction of this chart are not met. 
		
		\item Sometimes a difficulty encountered by practitioners of this chart is to determine the batch size and the level of acceptance ($A$) or rejection ($R$) equivalent to the USL by taking the decision based on the developments (and norms) of acceptance plans samples see earlier above. 
		
		 \item This control chart is not suitable for exhaustive sampling since it would then required the use of the hypergeometric law. Unfortunately, as far as we know, there is no analytic relation for the confidence interval of the proportion of a hypergeometric distribution, and therefore is not feasible to build a control chart  based on a hypergeometric law.
		
		\item It is very difficult to compare objectively two points coming from batches that do not have the same size
	\end{itemize}
	
	For the values of the control chart, we have still using:
	
	And therefore using the results of the previous control chart is comes immediately in the general case:
	
	But since it is very difficult (unlike the $p$ control chart) to compare objectively with this control two points when the size of the sample is not the same, some specialists required - or recommended - that the batch size is the same such that finally:
	 
	According to the study of acceptance plans we know that they can not be any lower limit (LSL) required by the customer / board committee or target ($T$) in this field of application.

	So, with the following list where the batches size (special case) are identical (made with Microsoft Excel 14.0.7166):
	\begin{figure}[H]
		\centering
		\includegraphics{img/engineering/control_chart_np_excel_list_data.jpg}
	\end{figure}
	or explicitly:
	\begin{figure}[H]
		\centering
		\includegraphics[scale=0.5]{img/engineering/control_chart_np_excel_list_data_with_formulas.jpg}
	\end{figure}
	and the corresponding chart still with the same spreadsheet software:
	\begin{figure}[H]
		\centering
		\includegraphics[scale=0.6]{img/engineering/control_chart_np_plot_excel.jpg}
	\end{figure}
	with Minitab 17.1.3 this gives for people that don't trust our calculation made with Microsoft Excel (we can also do such a chart with R but the plot quality is not acceptable in a corporation):	
	\begin{figure}[H]
		\centering
		\includegraphics[scale=0.95]{img/engineering/control_chart_np_plot_minitab.jpg}
	\end{figure}
	where we took as example for limits imposed by the customer/board committee:
	

	This control chart can also be interpreted as any another Shewhart chart using the empirical WECO rules.
	
	\paragraph{$C$ Control Charts (Poisson counting CC)}\mbox{}\\\\
	The $C$ control charts are used to look at variation in counting type attributes data. They are used to determine the variation in the number of defects in a constant subgroup size. To use the $C$ chart, the opportunities for defects to occur in the subgroup must be very large, but the number that actually occurs must be small. 
	
	Indeed, during our study of statistical laws (or statistical distributions), we have proved that when the probability $p$ is very small and tends to zero, but that instead the average value $\mu=np$ tends to a fixed value when $n$ approaches infinity, the binomial distribution of mean $\mu=np$ with $k$ trials was therefore given by a Poisson law:
	
	with:
	
	In practice, some replace the binomial distribution by a Poisson law since $n>30$ and $np<5$ or when $n>50$ and $p<0.1$... In practice I personally recommend the use of the approximation when $n>1000$ and $p<0.01$.
	
	As in practice such situations does not happen very often. Some books and authors say that this control chart is reserved to the count of defects per unit opportunity (see our study of Six Sigma at the beginning of this chapter). Indeed, the fact that each unit can be composed of multiple items this increase "artificially" the $n$ (as $1$ unit opportunity can have dozens, hundreds of millions of components) and decrease most of time the probability $p$ such that the assumption of using the Poisson Law are respected.
	
	Following the previous rules it comes obviously:
	 
	what is traditionally noted in this field:
	 
	where $\bar{c}$ is simply the average of the count (that's form this word that comes from the "c" in the name of the chart) of nonconformities:
	
	According to the study that we made during our study of acceptance plan they can not be any lower limit (LSL) imposed by the customer/board committee or target ($T$) in this field of application.

	\textbf{Pros}: This type of control chart is very simple to construct because does not require any advanced statistical knowledge to be calculated.

	\textbf{Cons}:
	\begin{itemize}
		\item In reality its a control chart which control limits may be asymmetric if the practical application are at the limit of acceptable when the theoretical assumptions of construction of this chart are not met. 
		
		\item Sometimes a difficulty encountered by practitioners of this chart is to determine the batch size and the level of acceptance ($A$) or rejection ($R$) equivalent to the USL by taking the decision based on the developments (and norms) of acceptance plans samples see earlier above. 
		
		 \item This control chart is not suitable for exhaustive sampling since it would then required the use of the binomial or hypergeometric law. 		\end{itemize}
		 
		Once again using our spreadsheet software here is an example where the batch sizes are necessarily identical for the reasons mentioned just above (and once again... non-compliant items can not be expressed in $\%$ of the batch size as the Poisson distribution is a discrete law!):
	\begin{figure}[H]
		\centering
		\includegraphics{img/engineering/control_chart_c_excel_list_data.jpg}
	\end{figure}
	or explicitly:
	\begin{figure}[H]
		\centering
		\includegraphics[scale=0.6]{img/engineering/control_chart_c_excel_list_data_with_formulas.jpg}
	\end{figure}
	and the corresponding chart still with the same spreadsheet software:
	\begin{figure}[H]
		\centering
		\includegraphics[scale=0.6]{img/engineering/control_chart_c_plot_excel.jpg}
	\end{figure}
	with Minitab 17.1.3 this gives for people that don't trust our calculation made with Microsoft Excel (we can also do such a chart with R but the plot quality is not acceptable in a corporation):	
	\begin{figure}[H]
		\centering
		\includegraphics[scale=0.95]{img/engineering/control_chart_c_plot_minitab.jpg}
	\end{figure}
	where we took as example for limits imposed by the customer/board committee:
	

	This control chart can also be interpreted as any another Shewhart chart using the empirical WECO rules.
	
	\paragraph{$U$ Control Charts (normalized Poisson)}\mbox{}\\\\
	The $u$ control chart is similar to the $c$ control chart except that it has a for role to normalize the data to the unit to have a $c$ control chart based on a Poisson distribution but with percentages.

	According to the study that we made during our study of acceptance plan they can not be any lower limit (LSL) imposed by the customer/board committee or target ($T$) in this field of application.

	\textbf{Pros}: This type of control chart is very simple to construct because does not require any advanced statistical knowledge to be calculated and it is now more easy to compare objectively two points coming from batches that do not have the same size

	\textbf{Cons}:
	\begin{itemize}
		\item In reality its a control chart which control limits may be asymmetric if the practical application are at the limit of acceptable when the theoretical assumptions of construction of this chart are not met. 
		
		\item Sometimes a difficulty encountered by practitioners of this chart is to determine the batch size and the level of acceptance ($A$) or rejection ($R$) equivalent to the USL by taking the decision based on the developments (and norms) of acceptance plans samples see earlier above. 
		
		 \item This control chart is not suitable for exhaustive sampling since it would then required the use of the binomial or hypergeometric law.
		
		\item A final difficulty is that some employees and board committee are not proficient with the concept of percentages.
 	\end{itemize}
 	For this, we use the same working assumptions as those demonstrated in the during our study of the Poisson distribution:
	
	This makes that our previous $c$ control chart is then written:
	
	If we divide the random variable $k=n_i\bar{\hat{p}}$ by the batch size (or number of components) we have then by the property of the mean and variance:
	
	what is traditionally written:
	
	with:
	
	
	Once again using our spreadsheet software here is an example where the batch sizes are non-necessarily identical this time using the "trick" above (and once again... non-compliant items can now be expressed in $\%$ of the batch size!):
	\begin{figure}[H]
		\centering
		\includegraphics{img/engineering/control_chart_u_excel_list_data.jpg}
	\end{figure}
	or explicitly:
	\begin{figure}[H]
		\centering
		\includegraphics[scale=0.5]{img/engineering/control_chart_u_excel_list_data_with_formulas.jpg}
	\end{figure}
	and the corresponding chart still with the same spreadsheet software:
	\begin{figure}[H]
		\centering
		\includegraphics[scale=0.6]{img/engineering/control_chart_u_plot_excel.jpg}
	\end{figure}
	with Minitab 17.1.3 this gives for people that don't trust our calculation made with Microsoft Excel (we can also do such a chart with R but the plot quality is not acceptable in a corporation):	
	\begin{figure}[H]
		\centering
		\includegraphics[scale=0.95]{img/engineering/control_chart_u_plot_minitab.jpg}
	\end{figure}
	where we took as example for limits imposed by the customer/board committee:
	

	This control chart can also be interpreted as any another Shewhart chart using the empirical WECO rules.
	
	\paragraph{Laney's $p'$ and $u'$ control charts}\mbox{}\\\\
	The $p$ and $u$ control charts as we hav seen it have a big problem that resides in the fact that the limits use a standard deviation that is proportional to:
	
	which therefore tends to $0$ when the $N_i$ are very by (which typically happens in international call centers where its value is often greater than $5,000$ per day) and then the standard deviation becomes so small that we get almost systematically false alarms (therefore the control chart is a little bit much ... too sensitive). Moreover, their standard deviation assumes that the proportion (probability) is constant over time and this is also unrealistic.

	It should be noticed that the problem of too much sensitivity almost never appears in the industry since it is often less than a hundred of items!

	An alternative proposed by David Laney in 2002 is then to consider in a first time the proportions as continuous quantitative values and then of using a control chart taking first the range by paired values of the consecutive proportions:
	
	And as (\SeeChapter{see section Statistics page \pageref{range statistics}}):
	
	It comes:
	
 	because comparing only the proportion two-by-two, the Hartley constant $d_2(n)$ must be taken with $n=2$.

	We then have the possibility to take:
	
	Practice shows that this control chart is already much less sensitive, which is a progress but... however it suffers from a big issue!: the upper and lower limits are constant whereas we would like to take into account that the size of the sample varies (or may vary) for each $i$. Then we have to find another solution!
	
	A track (a little drawn by the hair) then consists first of all in considering that if the $N_i$ are very large and the proportion (probability) is neither too close to $0$ nor too close to $1$, then each can be approximated By a Normal law and to be centered reduced:
	
	and therefore:
	
 	Then if follows that:
 	
 	and therefore the true variance of $\hat{p}_i$ can be thank to:
	
	Obviously, in the reality the standard deviation of $Z$ will not really be equal to $0$. So the idea (questionable ...) is to consider that as now we have a continuous variable we will be able to estimate the true variance of $Z$ thanks to:
	
	and
	
	Therefore:
	
	We then have:
	
	where if $\text{LCL}_{Z(p)}$ is smaller than zero then we will assume that this limit is zero (as for the previous control charts).
	
	We thus have a control map which is an attribute-range mixture which according to the return on experience (REX) is less sensitive (less systematic false alerts) and which has variable limits depending on the size of the subgroups.

	The same reasoning is applicable with the $U$ control chart (as the Poisson's law can also be approximated by a Normal law, as we have demonstrated in the section of Statistics):
	
	We notice that if by hazard we have we have for the both set of relations above:
	
 	Then we fall back on the standard parameters of a control chart of type $p$ or respectively of type $u$.	

	There are however corrections/variants to these two models ... so it is necessary to know well what your software does! It is for this reason that we will not take screenshots of the procedure with Microsoft Excel this time and compare it with Minitab (excepted if requested by our readers).
	
	\subsubsection{Measurement Control Charts (quantitative CC)}
	Quantitative variables are continuous measurements of the type: weights, lengths, thicknesses, temperature, diameter ...

	Unlike autocorrelated control charts and quality (attributes) control charts , quantitative control charts (with measurement) are often - if not always - presented in pairs (or should be!):
	\begin{itemize}
		\item A control chart showing the analysis / evolution of the central statistical trend (often the mean, because it is relatively easy to construct confidence intervals on it).

		\item A control chart showing the analysis / evolution of the dispersion via the range or the standard deviation.
	\end{itemize}

	The controls limits are often taken (for lack of thinking time in business) as the Shewhart or WECO limits and not probabilistic limits.
	
	We also make the distinction between:
	\begin{itemize}
		\item "\NewTerm{Individual control charts}\index{Individual control charts}" (each point on the control chart invole one and only one measurement)

		\item "\NewTerm{Subgroups control charts}\index{subgroups control charts}" (each point on the control chart invole implicitly more than one item)
	\end{itemize}
	
	\paragraph{Individual measurement control chart with required limits}\mbox{}\\\\
	This is the easiest control chart and most used one by non specialist because it does not require any special knowledge and statistical hypothesis. It consist only to report the measurements and to take corrective actions when it seems necessary by the rule of thumb.
	
	\textbf{Pros}: Simple to build because does not require any special knowledge or statistical hypothesis and as all previous control charts, permits the user to check in a timely manner whether a production meets or not the constraints specified by customers.

	\textbf{Cons}:
	\begin{itemize}
		\item It may represent temporary random deviations that may make the reader think mistakenly that there is problem in the method / process and will lead to unnecessary corrective actions (false alarms). 
		
		\item  It does not allow to identify if the process / method is under statistical control or not that is to say that as for attribute control charts, it is not possible to make a Capability SixPack analysis.
 	\end{itemize}
 	For example, let us consider the following list of data still in the same spreadsheet software:
 	\begin{figure}[H]
		\centering
		\includegraphics{img/engineering/control_chart_individual_measurement_excel_list_data.jpg}
	\end{figure}
	where we took empirically (rule of thumb) the limits:
	
	Which gives the following control chart:
	\begin{figure}[H]
		\centering
		\includegraphics[scale=0.65]{img/engineering/control_chart_individual_measurement_plot_excel.jpg}
	\end{figure}
	While this card is the most used in practice (since most people don't have training in SPC), the problem remains that it does not really tell us whether the method or process appears to be stable in time and if the statistical control are in the customer limitations (it shows us only the punctual statistics which is as we know not very robust...!). We will partially remedy with this with the next control chart.
	
	This type of control chart is available in most statistical software but as there is nothing complicated we will not provide here any Minitab (or other...) screenshot.
	
	\paragraph{Individual measurement control chart with moving limits}\mbox{}\\\\
	When we begin the first observations (measurements), the method or process is rarely stable and there is no way to have fun do manufacture a huge quantity of trial pieces to do a control chart of the average in such conditions and especially if a piece is quite expensive and time consuming!
	
	\textbf{Pros}: Highlights the evolution over time of the statistical indicators of positions (average or median) and dispersion and permits to observe the process/methods stabilization   during the prototyping phase (often called "\NewTerm{Phase I}\index{phase I}" in the literature).

	\textbf{Cons}:
	\begin{itemize}
		\item It may represent temporary random deviations that may make the reader think mistakenly that there is problem in the method / process and will lead to unnecessary corrective actions (false alarms). 
		
		\item  It does not allow to identify statistically if the process / method is under statistical control or not that is to say that as for attribute control charts, it is not possible to make a Capability SixPack analysis.
 	\end{itemize}
 	We stay with a measured control chat to which we add the arithmetic average and the unbiased standard deviation recalculated at each new item. That is, as we have not enough samples to know the probability distribution and is out of the question make waste pieces in this purpose and to train people for weeks to statistics, we add to the previous Shewhart control chart control the following limits:
	
	where CL means "\NewTerm{centerline}\index{centerline}".
	
	Obviously for the estimator of the mean and the standard deviation, we take:
	
	Once again using our spreadsheet software here is an example where $N=150$ and therefore where $k$ moves from $1$ to $150$ with the $63$ first rows visible:
	\begin{figure}[H]
		\centering
		\includegraphics{img/engineering/control_chart_running_statistics_measurement_excel_list_data.jpg}
	\end{figure}
	or explicitly:
	\begin{figure}[H]
		\centering
		\includegraphics[scale=1]{img/engineering/control_chart_running_statistics_measurement_excel_list_data_with_formulas.jpg}
	\end{figure}
	and the corresponding chart still with the same spreadsheet software:
	\begin{figure}[H]
		\centering
		\includegraphics[scale=0.7]{img/engineering/control_chart_running_statistics_plot_excel.jpg}
	\end{figure}
	where we took as example for limits imposed by the customer/board committee:
	
	We see for example the in the above control chart that the CL indicator (so the estimator of the mean) stabilizes close to the target $T$ from the beginning to the end. It's the same for UCL and LCL (because the estimator of the standard deviation stabilizes) by cons following the usage in engineering the results are not good, since the UCL and LCL are beyond the required specifications by the customer! Note that this control chart would according to the WECO rules have only one single point beyond the $\pm 3\sigma$ (we indicate that for comparison reasons  with the next control charts).
	
	\begin{tcolorbox}[title=Remark,colframe=black,arc=10pt]
	On all control charts with no exceptions, the customer specifications (limits) must be indicated in addition to the calculated limits!
	\end{tcolorbox}
	
	This type of control chart is as far as we know not available in  statistical softwares. But there is an easy one with the following very common and obvious fixed limits:
	
	named a "\NewTerm{Levey-Jennings control chart}\index{control chart!Levey-Jennings control chart}".
	
	With Minitab 15.1.1.0 this gives for people that don't trust simple Microsoft Excel calculations (we can also do such a chart with R but the plot quality is not acceptable in a corporation):
	\begin{figure}[H]
		\centering
		\includegraphics[scale=0.95]{img/engineering/control_chart_levey_jennings_minitab.jpg}
	\end{figure}
	At this level, we supposed that we are unable to determine the probability distribution of the data (sample problem or statistical skills problems) and therefore unable to calculate the confidence intervals for the mean or the standard deviation!

	The next control charts is supposed to remedy to this situation under some assumption!
	
	\paragraph{Subgroups measurement control chart with standard error}\mbox{}\\\\
	Once the process or method stabilized, we consider that the mean and standard deviation are stabilized and even if the probability distribution that follows a group of $k$ samples is not the same from one day to another, by the central limit theorem, the set of averages follows a Normal distribution as we know of the type:
	
	and written in SPC as (for the reason we will see further below):
	
	Therefore it may be more interesting to represent the change in the average over time and not just the single measurement. This also gives us the opportunity to do a capability analysis of our process or method.
	
	\textbf{Pros}:
	\begin{itemize}
		\item Hides (smooth) non-systematic and random variations and often allows to avoid unnecessary corrective actions (false alarms).
		
		\item Gives the possibility to associate the control chart with a capability analysis.
	\end{itemize}

	\textbf{Cons}:
	\begin{itemize}
		\item This control chart assumes that the samples are identically distributed by the mathematical assumptions it is build on. Moreover, in the framework of fixed Shewhart limits, the distribution law is assumed symmetrical. 
		
		\item  The major issue of this control chart is that the number of individuals per sample must be large enough for the unbiased estimator of the standard deviation is not too far from reality. This is why we do not found this control chart in statistical software (this is why this control chart is replaced with another chart that we will see a little further below).
 	\end{itemize}
 	As companion example let us consider the following example of $25$ samples (that we will identify by the variable $k$) of $6$ individuals each (that we will identify by the variable $n_i=n$) taken from a continuous manufacturing process (without machine recalibration or without machine change!):
 	\begin{figure}[H]
		\centering
		\includegraphics{img/engineering/control_chart_subgroup_measurement_standard_error_excel_list_data.jpg}
	\end{figure}
	where every day we took $6$ individuals of all manufactured items produced the same day. On the control chart, then we will represent each point as the average of these $6$ individuals.

	We have proved in the section Statistics that the standard deviation of the mean (standard error) of a series of $n$ independent and identically distributed random variables were if the samples are big enough:
	
	and using the finite population correction factor if the sampled population is not infinite (remember that $N$ is the population size):
	 
	We then have the following Shewhart control limit indicators (obviously as we use the average estimators and standard deviation, the greater the number of samples and individuals, the greater the limits are asymptotically accurate but also... known problem... if the samples $n_i$ are too big we will have systematic alarms as we know!):
	
	With:
	
	and if all the $n_i$ are equals:
	
	Let us notice that it is the first control chart where we can calculate the probabilistic limits of non-compliance by excess or defect of subgroups by using the fact $\overline{X}$ that follows a Normal distribution of parameters $\mathcal{N}\left(\overline{X},\sigma_X/\sqrt{n}\right)$.
	
	Once again using our spreadsheet software here is the calculations we get in the simple case where all sample still have the same size (calculations based on the previous list of data given just earlier above):
	\begin{figure}[H]
		\centering
		\includegraphics[scale=0.8]{img/engineering/control_chart_subgroup_measurement_standard_error_excel_calculations.jpg}
	\end{figure}
	or explicitly:
	\begin{figure}[H]
		\centering
		\includegraphics[scale=0.55]{img/engineering/control_chart_subgroup_measurement_standard_error_excel_explicit_formulas.jpg}
	\end{figure}
	and the corresponding chart still with the same spreadsheet software:
	\begin{figure}[H]
		\centering
		\includegraphics[scale=0.65]{img/engineering/control_chart_standard_error_plot_excel.jpg}
	\end{figure}
	where we took as example for limits imposed by the customer/board committee:
	
	Let us notice that this control chart would according to the WECO rules have $6$ points beyond the $\pm 3\sigma$ (we indicate that for reasons of comparison with the previous control chart and the next one). It is therefore more sensitive to changes card than the control chart. Also notice that the UCL and LCL limits are wider than the imposed USL and LSL.
	
	With the $\pm 3\sigma_{\text{X}}$ limits and according to the hypothesis of Normally distributed date, we have using the symmetry of the Normal distribution and writing as usual $\Phi$ the cumulative probability function of the Normal centered reduced distribution (that is to where we suppose $\text{LCL}_\text{SE}=\text{UCL}_\text{SE}=\sigma_{\overline{X}}=1$):
	
	So we will have statistically $1$ point (piece/action) on the control chart on $370$ outside the $\pm \sigma_{\overline{X}}$ by unit time/sample measured (so if each point on the map is made by frequency of $1$ hour then we have an ARL of $0.0027$ [h$^{-1}$] ) only for statistical purposes and that must be therefore considered as false alarms (Type I Error\footnote{Obviously there are also type II errors (no alarm when there should have). But as I've never seen anyone use it in practice for control charts.}!!!) that the data are centered and reduced or not!!!!
	
	We name the latter number "\NewTerm{Average Run Length (ARL)}\index{average run length}" (some use the French appellation "Average Operational Period (AOP)"). When it reported to a specific unit of time, then we speakof"\NewTerm{Average Time to Signal (ATS)}" and write it:
	
	or more commonly:
	
	Obviously, by extension, the addition of the WECO empirical rules add false alarms and it has for immediate effect of reducing the ARL.
	
	As we have already mention it at our beginning of the study of control charts, the ARL is important in practice, because it gives the interval sampling size beyond which we must not go! We will ensure in practice (when possible) to take a lower value close to the ARL for large productions but possibly not beyond!
	
	\begin{tcolorbox}[title=Remark,colframe=black,arc=10pt]
	As we have already mentioned, sometimes several levels of controls are added to the control charts, so we can observe the $\pm 3\sigma_{\overline{X}}$ and simultaneously $\pm 2\sigma_{\overline{X}}$ . In the U.S.A factor $2$ or $3$ seems to be mainly used that the distribution is symmetrical or not ... as in European countries like France and Germany (industrial countries), we use probabilistic limits that have a much better reliability when data are not distributed following a symmetric (as is the case for well known cards using the range $R$ that we will see later).
	\end{tcolorbox}
	Recall that the problem of this card is that equation is asymptotically correct if equation is great. While in reality it is by far not the case since often this value is less than 10. Therefore we will see below a map of the average, the fixed limits are calculated based on the extent and not R no equation. However, in the chapter of Statistics, we have shown that when the standard deviation is estimated, then we have:
	
	So we can use the Shewhart control chart with the following probabilistic limits that is more accurate than the previous one:
	
	Also as part of a process or method under statistical control, it can be interesting to communicate the average control chart (with fixed or probabilistic limits) previously presented with a control chart showing also the evolution of the standard deviation considering a Normal distribution of the random variables. Indeed, remember that the Normal distribution is completely defined by its parameters $\mu,\sigma$. Hence the reason to have a look on both of them and this is why many SPC softwares plots automatically two control charts at the same time!
	
	
	\paragraph{$\overline{S}-S$ Subgroups measurement control chart for standard deviation}\mbox{}\\\\
	Also as part of a process or method under statistical control, it can be interesting to communicate the average control chart (with fixed or probabilistic limits) previously presented with a control chart showing also the evolution of the standard deviation considering a Normal distribution of the random variables. Indeed, remember that the Normal distribution is completely defined by its parameters $\mu,\sigma$. Hence the reason to have a look on both of them and this is why many SPC softwares plots automatically two control charts at the same time!
	
	\textbf{Pros}:
	\begin{itemize}
		\item Complete well the previous control chart and permits to visualizes if the volatility (standard deviation) of the process stabilizes. 

		\item Represents well the variation within the data of each sampling (instantenous dispersion).
	\end{itemize}

	\textbf{Cons}:
	\begin{itemize}
		\item This control chart must be used only when the prototyping phase is completed (often named "Phase II" in the literature) because it is based on an inescapable strong assumption that the data are Normally distributed. 

		\item The limits of Shewhart controls chart are not adapted in this case because, as we will prove in detail the variance distribution law is not a symmetrical law while Shewhart limits suppose this is the case.
 	\end{itemize}
 	The idea is to represent:
	
	always with the estimator of mean ($k$ being the number of samples for recall...):
	
	and always with the unbiased maximum likelihood estimator of the standard deviation denoted $S_i$ in the field of quality (at the opposite $\hat{\sigma}$ in the field of pure statistics):
	
	where $n$ corresponds obviously to the number of individuals for the sample $i$.

	To calculate the mean and the variance of $\sigma_S$, we must consider $S$ as also as a Normally distributed random variable (it is the strong and essential assumption in this control chart that we had mentioned in the "cons" just before!). We then have proved in sections Statistics that we had:
	
	which we will denote $t$ thereafter as the corresponding random variable. 

	For the remaining development, we will consider $(n-1)$ and $\sigma$ as simple constant coefficients of $S$. So the random variable $t$ is actually implicit function of $S$ as:
	
	We will focus now to the calculation of mean of $S$. It is then more convenient to take the square root of $t$ such that:
	
	Therefore it comes:
	
	Let us recall that we proved in the section Statistics that the chi-square law was defined by the following probability density function:
	
	Therefore it comes:
	
	Let us put for what follows:
	
	Then we have:
	
	Therefore:
	
	Finally for finish on this point, be aware that it is traditional to denote the "\NewTerm{standard deviation unbiasing constants $c_4$}\index{standard deviation unbiasing constants}" or "\NewTerm{Burr constants}\index{Burr constants}" as follows:
	
	and for which we find the values in the tables of various book or softwares. But still almost everybody has a spreadsheet software, these tables are useless as we can obtain their values for examples with spreadsheet software like Microsoft Excel 14.0.6123 by writing the following formula:
	\begin{center}
		\texttt{=SQRT(2/($n_i$-1))*EXP(GAMMALN($n_i$/2))/EXP(GAMMALN(($n_i$-1)/2))}
	\end{center}
	or also with:
	\begin{center}
		\texttt{=SQRT(2/($n_i$-1))*GAMMA($n_i$/2)/GAMMA(($n_i$-1)/2)}
	\end{center}
	or in OpenOffice Calc (the formula is there little bit simpler):
	\begin{center}
		\texttt{=SQRT(2/($n_i$-1))*GAMMA($n_i$/2)/GAMMA(($n_i$-1)/2)}
	\end{center}
	Let us now calculate the variance of $S$ using the Huygens relation proved in the section Statistics that is for recall:
	
	Therefore we can write:
	
	
	Therefore:
	
	hence:
	
	Therefore we have can write now:
	
	that is to say finally as sometimes written in the litterature:
	
	where by definition:
	
	That can be also sometimes be found in some books written as:
		
	with by definition the following unbiasing constants:
	
	The unbiasing constants $c_4,B_3,B_4$ (all dependent of $n_i$) can obviously be found in many tables in various books or softwares, but once again the can be easily computed using a simple spreadsheet software like Microsoft Excel as proved earlier above.

	As companion example let us once again consider the following example of $25$ samples (that we will identify by the variable $k$) of $6$ individuals each (that we will identify by the variable $n_i=n$) taken from a continuous manufacturing process (without machine recalibration or without machine change!):
 	\begin{figure}[H]
		\centering
		\includegraphics{img/engineering/control_chart_sbar_s_measurement_excel_list_data.jpg}
	\end{figure}
	where every day we took $6$ individuals of all manufactured items produced the same day.
	
	Once again using our spreadsheet software here is the calculations we get in the simple case where all sample still have the same size (calculations based on the previous list of data given just earlier above):
	\begin{figure}[H]
		\centering
		\includegraphics[scale=0.9]{img/engineering/control_chart_subgroup_measurement_sbar_s_excel_calculations.jpg}
	\end{figure}
	or explicitly:
	\begin{figure}[H]
		\centering
		\includegraphics[scale=0.8]{img/engineering/control_chart_subgroup_measurement_sbar_s_excel_explicit_formulas.jpg}
	\end{figure}
	and the corresponding chart still with the same spreadsheet software:
	\begin{figure}[H]
		\centering
		\includegraphics[scale=0.65]{img/engineering/control_chart_sbar_s_plot_excel.jpg}
	\end{figure}
	where we took as example for limits imposed by the customer/board committee:
	
	With Minitab 17.1.3 this gives for people that don't trust our calculations made with Microsoft Excel (we can also do such a chart with R but the plot quality is not acceptable in a corporation):
	\begin{figure}[H]
		\centering
		\includegraphics{img/engineering/control_chart_sbar_s_plot_minitab.jpg}
	\end{figure}
	Let us notice that this control chart would according to the WECO rules have $0$ points beyond the $\pm 3\sigma$ (we indicate that for reasons of comparison with the previous control chart and the next one). It is therefore more sensitive to changes card than the control chart. Also notice that the UCL and LCL limits are wider than the imposed USL and LSL.
	
	For recall, the major problem with this control chart is that it is correct if and only if the data are Normally distributed and that Shewart limits suppose that $S$ is symetric and here in fact it is note case (if we took probabilistic limits). Therefore, the quality manager then combine with the average control chart another optional control which we will introduce now.
	
	\pagebreak
	\paragraph{$\overline{X}-S$ Subgroups measurement control chart}\mbox{}\\\\
	During our study of the control chart with averages with fixed limits based on the standard deviation of the mean (standard error), we have clearly shown that it suffered from the handicap that $\sigma_X$ is too approximative in practice (because $n_i$ is often too small).

	Therefore, a workaround to this is to use the results obtained from the study of the previous control chart.
	
	\textbf{Pros}:
	\begin{itemize}
		\item Corrects the problem of the control chart for averages with standard error when the number of individuals per sample is too small. 

		\item Represents well the change of level in the average of data (central tendency).
	\end{itemize}

	\textbf{Cons}:
	\begin{itemize}
		\item The mathematical proof of the origin of the control chart limits is quite difficult for undergraduates (see the section Statistics for the detailed proof).  

		\item This control chart only works if and only if the data are identically distributed and are Normally distributed! 
		
		\item Moreover, since the limits are based on the estimator of the standard deviation it would require however $n_i$ that in the idea should not be below the value of $10$ ...
 	\end{itemize}
 	First let us recall that for the average control chart with standard error limits we get:
	
	and we have prove earlier above:
	
	We have therefore:
	
	and that we sometimes find in the literature written as:
	
	with the unbiasing constant:
	
	Notice then that we can calculate the probabilities of non-compliance by excess or defect of subgroups using the fact that $\overline{X}$ follows a Normal distribution of parameters:
	
	As companion example let us once again consider the following example of $25$ samples (that we will identify by the variable $k$) of $6$ individuals each (that we will identify by the variable $n_i=n$) taken from a continuous manufacturing process (without machine recalibration or without machine change!):
 	\begin{figure}[H]
		\centering
		\includegraphics{img/engineering/control_chart_xbar_s_measurement_excel_list_data.jpg}
	\end{figure}
	where every day we took $6$ individuals of all manufactured items produced the same day.
	
	We can calculate, or find in the tables that:
	
	We then have the following table:
	\begin{figure}[H]
		\centering
		\includegraphics[scale=0.9]{img/engineering/control_chart_subgroup_measurement_xbar_s_excel_calculations.jpg}
	\end{figure}
	or explicitly:
	\begin{figure}[H]
		\centering
		\includegraphics[scale=0.6]{img/engineering/control_chart_subgroup_measurement_xbar_s_excel_explicit_formulas.jpg}
	\end{figure}
	and the corresponding chart still with the same spreadsheet software:
	\begin{figure}[H]
		\centering
		\includegraphics[scale=0.65]{img/engineering/control_chart_xbar_s_plot_excel.jpg}
	\end{figure}
	where we took as example for limits imposed by the customer/board committee:
	
	With Minitab 17.1.3 this gives for people that don't trust our calculations made with Microsoft Excel (we can also do such a chart with R but the plot quality is not acceptable in a corporation):
	\begin{figure}[H]
		\centering
		\includegraphics{img/engineering/control_chart_xbar_s_plot_minitab.jpg}
	\end{figure}
	Let us notice that this control chart would according to the WECO rules have $1$ point beyond the $\pm 3\sigma$ (we indicate that for reasons of comparison with the previous control chart and the next one). It is therefore more sensitive to changes card than the control chart. Also notice that the UCL and LCL limits are wider than the imposed USL and LSL.
	
	For recall, the major problem with this control chart is that it is correct if and only if the data are Normally distributed and that Shewart limits suppose that $S$ is symmetric and here in fact it is note case (if we took probabilistic limits). Therefore, the quality manager then combine with the average control chart another optional control chart which we will introduce now.
	
	\paragraph{$\overline{X}-S_P$ Subgroups measurement control chart with pooled variance}\mbox{}\\\\
	As we have seen it the standard error based control chart suppose all measurements as whole (that means we suppose that each subgroup has the same mean). This is in fact not really accurate as every group is supposed as being independent and has to be compared with the others. So  identically at the $t$-test or ANOVA for independent groups we use the "pooled variance\index{pooled variance}" as overall indicator and as a generalization of the standard error (the pooled estimate the variance of several different populations when the mean may be different\footnote{Therefore it increase the power of a corresponding test}, but one may assume that the variance of each population is for recall the same!!!).
	
	Therefore the pooled variance control chart is based on the following relation proved in the section Statistics during our study of ANOVA:
	
	or in a more common obvious form for control charts:
	
	But as above... we can use if we want the unbiasing constant $c_4$ of the variance as it applies similarly at the difference that:
	
	where the $1+$ is here because in the case where $n_i=2$ (minimum case for the calculation of a variance) we would have without it $c_4(1)$ and that latter is not defined, that's why!
	
	But in fact it's quite useless in the situation of the pooled variance. Normally unbiasing constant $c_4$ is for small samples... but when the parameter of the constant is $\sum_{i}(n_i-1)$ then:
	
	and it follows that:
	
	To convinced yourself you can see the calculated some $c_4$ values using the relation constant given earlier above. As requested by some of my students here are some tabulated values:
	\begin{table}[H]
	\begin{center}
		\definecolor{gris}{gray}{0.85}
			\begin{tabular}{|c|c|}
				\hline
				\cellcolor{black!30}$\pmb{n}$ & 
\cellcolor{black!30}$\pmb{c_4}$ \\ \hline
		$1$ & $*$\\ \hline
		$2$ & $0.797885$\\ \hline
		$3$ & $0.886227$\\ \hline
		$4$ & $0.921318$\\ \hline
		$5$ & $0.939986$\\ \hline
		$10$ & $0.972659$\\ \hline
		$15$ & $0.982316$\\ \hline
		$20$ & $0.986934$\\ \hline
		$30$ & $0.991418$\\ \hline
		$40$ & $0.993611$\\ \hline
		$50$ & $0.994911$\\ \hline
		$70$ & $0.996383$\\ \hline
		$90$ & $0.997195$\\ \hline
	\end{tabular}
	\end{center}
	\caption{Unbiasing constant $c_4$}
	\end{table}	
	But as most of time the correction is smaller than the measurement precision... I let you therefore guess if it is very useful...
	
	But we still can write (we like to write it by highlighting the fact that we are using the pooled variance so we avoid to use too much condensed notations!):
	
	where we often write in practice:
	
	As companion example let us consider the following example of $25$ samples of $6$ individuals each (that we will identify by the variable $n_i=n$) taken from a continuous manufacturing process (without machine recalibration or without machine change!):
 	\begin{figure}[H]
		\centering
		\includegraphics{img/engineering/control_chart_subgroup_measurement_standard_error_excel_list_data.jpg}
	\end{figure}
	and let us focus on an example with the use of the unbiasing constant and always providing a detailed Microsoft Excel example and showing afterwards the corresponding Minitab plot!
	
	So every day we took $n_i=6$ individuals of all manufactured items produced the same day.
	
	We can calculate, or find in the tables that:
	
	We then have the following table:
	\begin{figure}[H]
		\centering
		\includegraphics[scale=0.9]{img/engineering/control_chart_subgroup_pooled_variance_measurement_xbar_sp_excel_calculations.jpg}
	\end{figure}
	or explicitly (the reader can zoom on it with its PDF reader):
	\begin{figure}[H]
		\centering
		\includegraphics[scale=0.4]{img/engineering/control_chart_subgroup_pooled_variance_measurement_xbar_sp_excel_explicit_formulas.jpg}
	\end{figure}
	and the corresponding chart still with the same spreadsheet software:
	\begin{figure}[H]
		\centering
		\includegraphics[scale=0.65]{img/engineering/control_chart_xbar_sp_plot_excel.jpg}
	\end{figure}
	With Minitab 17.1.3 this gives for people that don't trust our calculations made with Microsoft Excel (we can also do such a chart with R but the plot quality is not acceptable in a corporation):
	\begin{figure}[H]
		\centering
		\includegraphics[scale=0.93]{img/engineering/control_chart_xbar_sp_plot_minitab.jpg}
	\end{figure}
	Let us notice that this control chart would according to the WECO rules have $0$ point beyond the $\pm 3\sigma$ (we indicate that for reasons of comparison with the previous control chart and the next one). 
	
	For recall, the major problem with this control chart is that it is correct if and only if the data are Normally distributed!
	
%	\pagebreak
	\paragraph{$\overline{R}-R$ Subgroups measurement control chart}\mbox{}\\\
	The range control chart is often associated with the average control chart. It aims to advantageously replace the standard deviation control chart $\overline {S}-S$ when data are not Normally distributed, because it's mathematical construction allows if desired to choose the underlying probability distribution! So we use this new control chart when $n_i\leq 10$ (the number of individuals per sample), while for the control chart for the standard deviation we must instead $n_i\geq 10$.
	
	\textbf{Pros}:
	\begin{itemize}
		\item Completes well the average control chart and replaces well the control chart of the standard deviation when the number of samples is small. 

		\item Represents well the variation within each sample of data (instantaneous dispersion) that we could detect.
	\end{itemize}

	\textbf{Cons}:
	\begin{itemize}
		\item The mathematical proof of the origin of the limits of controls is quite hard (see the section of Statistics for the detailed proof) and the determination of the corresponding unbiasing coefficients requires the use of Monte Carlo simulations (\SeeChapter{see section Theoretical Computing page \pageref{monte carlo simulations}}). 

		\item Another problem is that even if the probability distribution of sample size can be chosen, it is often given in specialized books only for the Normal distribution... (but softwares such as Minitab manage them).
 	\end{itemize}
 	The idea is to represent (due to the possible assymetry of $R$ it is impossible - nonsense - that the customer sets coherent USL and LSL or a target $T$ as regards to the range, hence the fact that these parameters are this will time not be listed or represented):
 	
	and this time we will not use estimators of the standard deviation for the range as it do not converge fast enough. Therefore, we will use the results proved in the section Statistics during our study of rank statistics. We obtained for the estimator of the standard deviation with the unbiasing Hartley's constants:
	
	with:
	
	or depending of the softwares (the latter one is more logic if we think a little bit):
	
	Therefore:
	
	which is written traditionally:
	
	The unbiasing coefficients $D_3,D_4$ defined by:
	
	are available in the literature (or on Internet or also in the Help of Minitab) as a function of the $n_i$ of the sample size and often (unfortunately) only for a Normal distribution (see the example in the section Statistics).
	
	As companion example let us consider the following example of $25$ samples (that we will identify by the variable $k$) of $6$ individuals each (that we will identify by the variable $n_i=n$) taken from a continuous manufacturing process (without machine recalibration or without machine change!):
	\begin{figure}[H]
		\centering
		\includegraphics{img/engineering/control_chart_rbar_r_measurement_excel_list_data.jpg}
	\end{figure}
	where every day we took $6$ individuals of all manufactured items produced the same day.
	
	We can find in the tables that:
	
	We then have the following table:
	\begin{figure}[H]
		\centering
		\includegraphics{img/engineering/control_chart_subgroup_measurement_rbar_r_excel_calculations.jpg}
	\end{figure}
	or explicitly:
	\begin{figure}[H]
		\centering
		\includegraphics[scale=0.8]{img/engineering/control_chart_subgroup_measurement_rbar_r_excel_explicit_formulas.jpg}
	\end{figure}
	and the corresponding chart still with the same spreadsheet software:
	\begin{figure}[H]
		\centering
		\includegraphics[scale=0.65]{img/engineering/control_chart_rbar_r_plot_excel.jpg}
	\end{figure}
	With Minitab 17.1.3 this gives for people that don't trust our calculations made with Microsoft Excel (we can also do such a chart with R but the plot quality is not acceptable in a corporation):
	\begin{figure}[H]
		\centering
		\includegraphics{img/engineering/control_chart_xbar_s_plot_minitab.jpg}
	\end{figure}
	which (is it necessary to recall it?) underestimates the number of error detections since the probability distribution of the range is in reality not symmetrical (hence the application of fixed Shewhart's control limits is not really suited ...) and because almost all of the tables assume the data still Normally distributed as well.
	
	Let us notice that this control chart would according to the WECO rules have $0$ point beyond the $\pm 3\sigma$.
	
%	\paragraph{$\overline{X}-R$ Subgroups measurement control chart}\mbox{}\\\
	During our study of the averages control chart averages with fixed limits based on the standard deviation of the mean and also of the $\overline{S}-S$ control chart, we have clearly discusses that it suffered from the handicap that $\sigma_X$ is too approximate in practice (because $n_i$ is often too small).

	Therefore, a workaround is to use the results obtained from the study of the previous $\overline{R}-R$ control chart.
	
	\textbf{Pros}:
	\begin{itemize}
		\item Corrects the problem of the control charge with for averages using the traditional standard deviation or also the unbiasing standard deviation approach when the number of individuals per sample is quite small. 

		\item Represents well the change in the average level of data (central tendency).
	\end{itemize}

	\textbf{Cons}:
	\begin{itemize}
		\item The mathematical proof of the origin of the limits of controls is quite hard (see the section of Statistics for the detailed proof) and the determination of the corresponding unbiasing coefficients requires the use of Monte Carlo simulations (\SeeChapter{see section Theoretical Computing page \pageref{monte carlo simulations}}). 

		\item Another problem is that even if the probability distribution of sample size can be chosen, it is often given in specialized books only for the Normal distribution... (but softwares such as Minitab manage them).
 	\end{itemize}
 	\begin{tcolorbox}[title=Remark,colframe=black,arc=10pt]
	This control chard is used a lot as part of the study of reproducibility and repeatability (R\&R Studies). This is why it is found it often in statistical process control software as automatically processes control chart when launching a R\&R analysis (typically what do Minitab).
	\end{tcolorbox}
	Let us recall first that we had:
	
	and let us recall that we have proved in the section Statistics in the context of our study of order statistics that:
	
	It comes (it is impossible that the customer sets a limit for the USL, LSL or target $T$ as regards the range as in general it is asymmetric, hence the fact that these parameters are this time not listed):
	
	still with:
	
	and the whole is traditionally written in the field:
	
	Notice that we can then calculate the probabilities of non-compliance by excess or defect of the subgroups by using the fact that $\overline{X}$ follows a Normal distribution of parameters:
	
	
	As companion example let us consider the following example of $25$ samples (that we will identify by the variable $k$) of $6$ individuals each (that we will identify by the variable $n_i=n$) taken from a continuous manufacturing process (without machine recalibration or without machine change!):
	\begin{figure}[H]
		\centering
		\includegraphics{img/engineering/control_chart_xbar_r_measurement_excel_list_data.jpg}
	\end{figure}
	where every day we took $6$ individuals of all manufactured items produced the same day.
	
	We can find in the tables that:
	
	We then have the following table:
	\begin{figure}[H]
		\centering
		\includegraphics{img/engineering/control_chart_subgroup_measurement_xbar_r_excel_calculations.jpg}
	\end{figure}
	or explicitly:
	\begin{figure}[H]
		\centering
		\includegraphics[scale=0.8]{img/engineering/control_chart_subgroup_measurement_rbar_r_excel_explicit_formulas.jpg}
	\end{figure}
	and the corresponding chart still with the same spreadsheet software:
	\begin{figure}[H]
		\centering
		\includegraphics[scale=0.65]{img/engineering/control_chart_xbar_r_plot_excel.jpg}
	\end{figure}
	With Minitab 17.1.3 this gives for people that don't trust our calculations made with Microsoft Excel (we can also do such a chart with R but the plot quality is not acceptable in a corporation):
	\begin{figure}[H]
		\centering
		\includegraphics{img/engineering/control_chart_xbar_r_plot_minitab.jpg}
	\end{figure}
	which (is it necessary to recall it?) underestimates the number of error detections since the probability distribution of the range is in reality not symmetrical (hence the application of fixed Shewhart's control limits is not really suited ...) and because almost all of the tables assume the data still Normally distributed as well.
	
	Let us notice that this control chart would according to the WECO rules have $0$ point beyond the $\pm 3\sigma$.
	
	Finally for the people interested here are some data that may perhaps be useful to some people:
	\begin{center}
	  \resizebox{\textwidth}{!}{\begin{tabular}{|r|c|c|c|c|c|c|c|c|c|c|c|c|c|c|c|}
	    \multicolumn{16}{c}
	    {\textbf{Factors useful in the Construction of Control Charts}} \\
	    \multicolumn{16}{c}{\ } \\
	    \hline
	    & \multicolumn{3}{c|}{\textit{Chart for}}
	    & \multicolumn{5}{c|}{\textit{Chart for standard deviations}} 
	    & \multicolumn{7}{c|}{\textit{Chart for Ranges}} \\
	    & \multicolumn{3}{c|}{\textit{Averages}} 
	    & \multicolumn{5}{c|}{\ } & \multicolumn{7}{c|}{\ } \\ \hline
	    & \multicolumn{3}{c|}{\ } & \multicolumn{5}{c|}{\textit{Factors for:}} 
	    & \multicolumn{2}{c|}{\ } & \multicolumn{5}{c|}{\ } \\
	      \cline{5-9}
	    & \multicolumn{3}{c|}{\textit{Factors for}} & \textit{Central}
	    & \multicolumn{4}{c|}{\textit{Control Limits}} 
	    & \multicolumn{2}{c|}{\textit{Factors for}}
	    & \multicolumn{5}{c|}{\textit{Factors for Control Limits}} \\
	    & \multicolumn{3}{c|}{\textit{Control Limits}} & \textit{Line} 
	    & \multicolumn{4}{c|}{\ } 
	    & \multicolumn{2}{c|}{\textit{Central Line}} 
	    & \multicolumn{5}{c|}{\ } \\ \hline
	       &       &        &       &       &       &       &       &
	       &       &        &       &       &       &       &       \\
	    n & A & A$_2$ & A$_3$ & c$_4$ & B$_3$ & B$_4$ & B$_5$ & B$_6$ 
	      & d$_2$ & 1/d$_2$ & d$_3$ & D$_1$ & D$_2$ & D$_3$ & D$_4$ \\
	       &       &        &       &       &       &       &       &
	       &       &        &       &       &       &       &       \\
	     2 & 2.121 & 1.880 & 2.659 & 0.7979 &   0   & 3.267 &   0   & 2.606
	       & 1.128 & 0.8862 & 0.852 &   0   & 3.686 &   0   & 3.266 \\
	     3 & 1.732 & 1.023 & 1.954 & 0.8862 &   0   & 2.568 &   0   & 2.276
	       & 1.693 & 0.5908 & 0.888 &   0   & 4.357 &   0   & 2.574 \\
	     4 & 1.500 & 0.729 & 1.628 & 0.9213 &   0   & 2.266 &   0   & 2.088
	       & 2.059 & 0.4857 & 0.879 &   0   & 4.697 &   0   & 2.281 \\
	     5 & 1.342 & 0.577 & 1.427 & 0.9400 &   0   & 2.089 &   0   & 1.964
	       & 2.326 & 0.4299 & 0.864 &   0   & 4.918 &   0   & 2.114 \\
	       &       &        &       &       &       &       &       &
	       &       &        &       &       &       &       &       \\
	     6 & 1.225 & 0.483 & 1.287 & 0.9515 & 0.030 & 1.970 & 0.029 & 1.874
	       & 2.534 & 0.3946 & 0.848 &   0   & 5.078 &   0   & 2.003 \\
	     7 & 1.134 & 0.419 & 1.182 & 0.9594 & 0.118 & 1.882 & 0.113 & 1.806
	       & 2.704 & 0.3698 & 0.833 & 0.206 & 5.203 & 0.076 & 1.924 \\
	     8 & 1.061 & 0.373 & 1.099 & 0.9650 & 0.185 & 1.815 & 0.179 & 1.751
	       & 2.847 & 0.3512 & 0.819 & 0.389 & 5.306 & 0.137 & 1.863 \\
	     9 & 1.000 & 0.337 & 1.032 & 0.9693 & 0.239 & 1.761 & 0.232 & 1.707
	       & 2.970 & 0.3367 & 0.807 & 0.548 & 5.392 & 0.184 & 1.816 \\
	    10 & 0.949 & 0.308 & 0.975 & 0.9727 & 0.284 & 1.716 & 0.276 & 1.669
	       & 3.078 & 0.3249 & 0.797 & 0.688 & 5.467 & 0.223 & 1.777 \\
	       &       &        &       &       &       &       &       &
	       &       &        &       &       &       &       &       \\
	    11 & 0.905 & 0.285 & 0.927 & 0.9754 & 0.321 & 1.679 & 0.313 & 1.637
	       & 3.173 & 0.3152 & 0.787 & 0.813 & 5.533 & 0.256 & 1.744 \\
	    12 & 0.866 & 0.266 & 0.886 & 0.9776 & 0.354 & 1.646 & 0.346 & 1.610
	       & 3.258 & 0.3069 & 0.778 & 0.924 & 5.593 & 0.284 & 1.716 \\
	    13 & 0.832 & 0.249 & 0.850 & 0.9794 & 0.382 & 1.618 & 0.374 & 1.585
	       & 3.336 & 0.2998 & 0.770 & 1.026 & 5.646 & 0.307 & 1.693 \\
	    14 & 0.802 & 0.235 & 0.817 & 0.9810 & 0.406 & 1.594 & 0.399 & 1.563
	       & 3.407 & 0.2935 & 0.763 & 1.119 & 5.695 & 0.328 & 1.672 \\
	    15 & 0.775 & 0.223 & 0.789 & 0.9823 & 0.428 & 1.572 & 0.421 & 1.544
	       & 3.472 & 0.2880 & 0.756 & 1.204 & 5.739 & 0.347 & 1.653 \\
	       &       &        &       &       &       &       &       &
	       &       &        &       &       &       &       &       \\
	    16 & 0.750 & 0.212 & 0.763 & 0.9835 & 0.448 & 1.552 & 0.440 & 1.526
	       & 3.532 & 0.2831 & 0.750 & 1.283 & 5.781 & 0.363 & 1.637 \\
	    17 & 0.728 & 0.203 & 0.739 & 0.9845 & 0.466 & 1.534 & 0.458 & 1.511
	       & 3.588 & 0.2787 & 0.744 & 1.357 & 5.819 & 0.378 & 1.622 \\
	    18 & 0.707 & 0.194 & 0.718 & 0.9854 & 0.482 & 1.518 & 0.475 & 1.496
	       & 3.640 & 0.2747 & 0.738 & 1.425 & 5.855 & 0.392 & 1.608 \\
	    19 & 0.688 & 0.187 & 0.698 & 0.9862 & 0.497 & 1.503 & 0.490 & 1.483
	       & 3.689 & 0.2711 & 0.733 & 1.490 & 5.888 & 0.404 & 1.596 \\
	    20 & 0.671 & 0.180 & 0.680 & 0.9869 & 0.510 & 1.490 & 0.504 & 1.470
	       & 3.735 & 0.2677 & 0.728 & 1.550 & 5.920 & 0.415 & 1.585 \\
	       &       &        &       &       &       &       &       &
	       &       &        &       &       &       &       &       \\
	    21 & 0.655 & 0.173 & 0.663 & 0.9876 & 0.523 & 1.477 & 0.516 & 1.459
	       & 3.778 & 0.2647 & 0.724 & 1.607 & 5.950 & 0.425 & 1.575 \\
	    22 & 0.640 & 0.167 & 0.647 & 0.9882 & 0.534 & 1.466 & 0.528 & 1.448
	       & 3.819 & 0.2618 & 0.719 & 1.661 & 5.978 & 0.435 & 1.565 \\
	    23 & 0.626 & 0.162 & 0.633 & 0.9887 & 0.545 & 1.455 & 0.539 & 1.438
	       & 3.858 & 0.2592 & 0.715 & 1.712 & 6.004 & 0.444 & 1.556 \\
	    24 & 0.612 & 0.157 & 0.619 & 0.9892 & 0.555 & 1.445 & 0.549 & 1.429
	       & 3.895 & 0.2567 & 0.712 & 1.761 & 6.030 & 0.452 & 1.548 \\
	    25 & 0.600 & 0.153 & 0.606 & 0.9896 & 0.565 & 1.435 & 0.559 & 1.420
	       & 3.931 & 0.2544 & 0.708 & 1.807 & 6.055 & 0.460 & 1.540 \\
	       &       &        &       &       &       &       &       &
	       &       &        &       &       &       &       &       \\
	    Over 25 & $3/\sqrt{n}$ & $3/d_2\sqrt{n}$ & $\cdots$ & $\cdots$ & * & ** 
	       & $\cdots$ & $\cdots$ & $\cdots$ & $\cdots$ & $\cdots$ & $\cdots$ 
	       & $\cdots$ & $\cdots$ & $\cdots$ \\ \hline
	  \end{tabular}}
	  \[ *\ 1-\frac{3}{\sqrt{2n}} \qquad\qquad\qquad\qquad
	     \text{**}\ 1+\frac{3}{\sqrt{2n}} \]
	\end{center}
	
	\pagebreak
	\subsubsection{Autocorrelated Measurement Control Charts (time weighted control charts)}\label{autocorrelated measurement control charts}
	The family ofautocorrelated control charts is defined by the fact that either the points on the chart or the control limits are calculated based on a number of the foregoing points.

	These conthrol chart have the advantage of requiring less data than others and to exacerbate the problem of deviation and therefore detect anomalies faster.

	The WECO rules do not apply to autocorrelated control chart. If a point is outside the calculated limits the process requires an immediate corrective action.
	
	\paragraph{$\text{I}-\text{MR}/\overline{X}$ Individual moving range measurement control chart}\mbox{}\\\
	In prototyping phase, we can not put in place the concept of sampling. In addition, the measurement short term control chart on the mean with standard deviation  does not gives the possibility to make statistical inference, because the distribution law is supposed unknown simple because the process is unstable.

	An intermediate solution is to consider then that the process is stable (assumption of Normality of the data) and calculate the limits from a minimum number of individuals that it is possible to have to make a statistic. That is to say: two!

	This situation is then well suited for prototyping phases or in the case of processes/methods with slow frequencies for which the measurements (pieces)/controls are very expensive.
	
	\textbf{Pros}:
	\begin{itemize}
		\item Complete the first control charts (measurement on subgroups) that we saw at the beginning for situations where it is difficult to have a large number of individuals by sample for reasons of cost or time.

		\item This control chart is sometimes regarded as the Swiss Army knife of quality control ... and is increasingly used because it is the policy of the "Just in Time" (JIT) production for  small series (for the decrease of costs).
	\end{itemize}

	\textbf{Cons}:
	\begin{itemize}
		\item The mathematical proof of the origin of the control limits is quite difficult (see the section Statistics for detailed profed) and the determination of the coefficients requires the use of  Monte Carlo simulation (\SeeChapter{see section Theoretical Computing page \pageref{monte carlo simulations}})). 

		\item Another problem is that although the probability distribution of the range can be chosen, it is often given in specialized books only for the Normal distribution (but most advanced softwares manage various distribution laws).
 	\end{itemize}
 	As workaround of the problem of an acceptable estimate of the standard deviation, the idea is to consider two successive measurements $X_{i-1}$,$x_i$ as the random variables of a rank statistics of which we calculate the range in absolute terms:
	
	where MR stands for "Moving Range" (the control chart is sometimes named "\NewTerm{MR($2$) control chart}\index{control chart!MR($2$) control chart}"). And then we have:
	
	since we have necessarily $n-1$ moving ranges.

	Then we take the back the definitions of the control chart my measurement for the mean with short term standard deviation:
	
	and we can get rid of the of the estimation problem of $\sigma_X$ using the relation proved in the section Statistics during our study of rank statistics:
	
	Therefore rearraning and using the prior previous relation:
	
	It comes (it is impossible that the customer sets a limit for the USL, LSL or target $T$ as regards the range as in general it is asymmetric, hence the fact that these parameters are this time not listed):
	
	\begin{tcolorbox}[title=Remark,colframe=black,arc=10pt]
	The same unbiasing constant can be used for subgroup control charts! Indeed, rather than using $R_i=x_{i-1}-x_i$ we use $R_i=\overline{X}_{i-1}-\overline{X}_i$ where $i$ is then the number of the sample!!! This is why in fact Minitab propose in the menu of measurement subgroup control charts the a MR chart!!!
	\end{tcolorbox}
	As companion example let us consider the following data:
	\begin{figure}[H]
		\centering
		\includegraphics{img/engineering/control_chart_i_mr_excel_list_data.jpg}
	\end{figure}
	Then we have for the first $17$ visible rows only:
	\begin{figure}[H]
		\centering
		\includegraphics{img/engineering/control_chart_i_mr_excel_list_data_with_calculations.jpg}
	\end{figure}
	Explicitly:
	\begin{figure}[H]
		\centering
		\includegraphics[scale=0.7]{img/engineering/control_chart_i_mr_excel_list_data_with_formulas.jpg}
	\end{figure}
	Then we have the following chart:
	\begin{figure}[H]
		\centering
		\includegraphics[scale=0.65]{img/engineering/control_chart_individual_measurement_i_mr_xbar_plot_excel.jpg}
	\end{figure}
	With Minitab 17.1.3 this gives for people that don't trust our calculations made with Microsoft Excel:
	\begin{figure}[H]
		\centering
		\includegraphics{img/engineering/control_chart_i_mr_plot_minitab.jpg}
	\end{figure}
	Notice that this control chart would be according to the WECO rules have $0$ points beyond the $\pm 3\sigma$ (we indicate that for comparison reasons with previous control charts and the next one).
	
	As the reader have probably notice it in the figure above, Minitab also plot an I-MR/$\overline{\text{MR}}$ control chart. So let us study this one just now:
	
	\paragraph{$\text{I}-\text{MR}/\overline{\text{MR}}$ Individual moving range measurement control chart}\mbox{}\\\
	So as we just say and notice it, the previous central tendency control chart must be completed (same normally as for almost all previous control charts with central tendency) with a control chart showing the tendency of dispersion (the benefits and problems of this control chart are the same as the previous one) .

	Then we take the definitions of a measurement control chart to get:
	
	Again, the standard deviation of the moving range (MR) is constructed in the same way as the standard deviation of the range. Then we have:
	
	It comes (it is impossible that the customer sets a limit for the USL, LSL or target $T$ as regards the range as in general it is asymmetric, hence the fact that these parameters are this time not listed):
	
	which is denoted as well as the range to control chart:
	
	with for recall:
	
	As companion example let us consider the following data:
	\begin{figure}[H]
		\centering
		\includegraphics{img/engineering/control_chart_i_mr_mr_excel_list_data.jpg}
	\end{figure}
	Then we have for the first $25$ visible rows only:
	\begin{figure}[H]
		\centering
		\includegraphics{img/engineering/control_chart_i_mr_mr_excel_list_data_with_calculations.jpg}
	\end{figure}
	Explicitly:
	\begin{figure}[H]
		\centering
		\includegraphics[scale=0.7]{img/engineering/control_chart_i_mr_mr_excel_list_data_with_formulas.jpg}
	\end{figure}
	Then we have the following chart:
	\begin{figure}[H]
		\centering
		\includegraphics[scale=0.65]{img/engineering/control_chart_individual_measurement_i_mr_mr_plot_excel.jpg}
	\end{figure}
	We see that this control chart with moving range seems much more sensitive than any previous control chart. But the problem remains as to know whether our observation of this control chart is correct in the sense that the moving range does not follow a symmetrical distribution law at the contrary to what presupposes fixed Shewhart limits.

	Also be aware that some of the WECO rules (the rules for sequences for example) do not apply to this type of control chart, since the data are autocorrelated there.
	
	\paragraph{Individual Moving Average control chart}\mbox{}\\\
	We always assume here that the characteristic follows a Normal distribution (...) and that the samples were taken of constant size $n$ (yes... Moving Average control charts can also be used for subgroups using moving range standard deviation or or $S$ estimation method, not only for individual measurement!!!). The moving average of duration $h$ at time $t$, written $M_t$ is defined by moving average seen in the section Statistics:
	
	The parameter $h$ is also named "horizon". Greater is the value of $h$ better is the efficiency of this control chart for the detection of small deviations.

	When $t\geq h$, assuming independence between samples, we get using the property of the variance:
	
	And if all the samples have the same size (if samples there are...)
	
	Therefore:
	 
	And:
	 	
	So Finally:
	
	Ok this done, let us now as always make a summary of the pros and cons of this control chart.
	
	\textbf{Pros}:
	\begin{itemize}
		\item The fact that we smooth the fluctuations in the purpose to reduce strong variations makes this control chart very well suited for small productions for which the total number of items is around $10$ (however as already mentioned this control chart can also be used for subgroup measurement). Also in the example that will follow, we will only focus on this particular scenario of individual measurement! 

		\item We also see that the boundaries of this control will be even smaller than $h$ is large. So this control chart can be very sensitive to small variations on the long term.
	\end{itemize}

	\textbf{Cons}:  The standard deviation is assumed to be known or at least calculated precisely so that the estimator converges to the theoretical value. All samples should be of identical size $n$ if we use the method developed above (otherwise most SPC softwares implement the necessary mathematical tools of this condition is not satisfied).
	
	As companion example let us consider the following data:
	\begin{figure}[H]
		\centering
		\includegraphics{img/engineering/control_chart_i_ma_excel_list_data.jpg}
	\end{figure}
	Since the control chart is with individual values, we have $n=1$. And the system using the moving range for the estimate of the standard deviation is reduces to:
	
	therefore apart the denominators with the root of $h$, the parameters CL, UCL and LCL are the same as the I-MR $\overline{X}$ control chart. So we fall back on the same calculation method as the one provided par the NCSS Statistical Analysis \& Graphics Software.
	
	We must take in the case of our special example the value of $d_2$ for $n_i=2$. The tables or direct calculations give us:
	
	Which gives us for the calculations:
	\begin{figure}[H]
		\centering
		\includegraphics{img/engineering/control_chart_i_ma_excel_list_data_with_calculations.jpg}
	\end{figure}
	Explicitly:
	\begin{figure}[H]
		\centering
		\includegraphics[scale=0.65]{img/engineering/control_chart_i_ma_excel_list_data_with_formulas.jpg}
	\end{figure}
	Then we have the following chart:
	\begin{figure}[H]
		\centering
		\includegraphics[scale=0.65]{img/engineering/control_chart_individual_measurement_i_ma_xbar_plot_excel.jpg}
	\end{figure}
	Notice that this control chart would be according to the WECO rules have $1$ points beyond the $\pm 3\sigma$ based (we indicate that for comparison reasons with previous control charts and the next one).
	
	We don't get the same results with Minitab 17.1.3 on any known prior version as the use a strange method to calculate the standard deviation for which we can not found any proof:
	\begin{figure}[H]
		\centering
		\includegraphics{img/engineering/control_chart_i_ma_plot_minitab.jpg}
	\end{figure}
	
	\paragraph{CUSUM (cumulated sum) control chart with empirical V-mask}\mbox{}\\\
	A major disadvantage of the control chart of the Shewhart type is to base that analysis on the latest information collected (assumption of independance of measurements). They ignore the information of trends of the process contained in the latest estimates. You should know that many quality standards institutes worldwide (AFNOR or ASQ) consider Shewhart charts as completely obsolete (wrongly or rightly?) and advice the usage of CUSUM control charts because using autocorrelation and the amplification variations.

	In the years 1950 control charts named "\NewTerm{CUSUM (cumulated sum)}\index{control chart!CUSUM control chart}" or "\NewTerm{Page-Hinkley control chart}\index{control chart!Page-Hinkley control chart}" (in honor of their inventors) were introduced by statisticians to detect small variations. Although they are mathematically more optimal, they are difficult to properly configure and very sensitive to the initial settings.

	The principle of CUSUM control charts (because there are several for one given distribution law...) is to sum the differences between the estimates of the position of the method / process and its target. When the accumulated positive variances or accumulated negative variances exceed a certain value, we conclude with a off-centering of the process.
	
	We put:
	
	and:
	
	that is obviously a sum of centered reduced variables!
	
	Where often the first equality is written:
	
	Therefore there comes a famous writing in the literature:
	
	In general, the CUSUM control chart behaves as follows:
	\begin{itemize}
		\item If the process is under control, the cumulative sum fluctuates randomly around zero.

		\item If the average undergoes a positive drift of the mean, the statistic $C_n$ or $S_n$ grows quick by accumulation of a systematic bias.
	\end{itemize}
	\begin{tcolorbox}[title=Remarks,colframe=black,arc=10pt]
	\textbf{R1.} In many books and numerous application cases, we do not use $S_n$ because the variance is unknown and we restrict ourselves to the use of $C_n$ only. The mathematical developments which then follows can be adapted very easily.\\
	
	\textbf{R2.} The same approach can be made with average of samples following:
	
	
	\textbf{R3.} There are many different versions of the CUSUM control charts and this even for the CUSUM with V-mask there are multiple sub-families. To be honest there is in my opinion quite a mess...
	\end{tcolorbox}
	For the example, let us take again the following data table:
	\begin{figure}[H]
		\centering
		\includegraphics{img/engineering/control_chart_cusum_excel_list_data.jpg}
	\end{figure}
	and we calculate the cumulative quantity still considerating a target $\mu=T=10$:
	\begin{figure}[H]
		\centering
		\includegraphics{img/engineering/control_chart_cusum_excel_calculations_cumulated_sum.jpg}
	\end{figure}
	Thus explicitly:
	\begin{figure}[H]
		\centering
		\includegraphics{img/engineering/control_chart_cusum_excel_calculations_cumulated_sum_formulas.jpg}
	\end{figure}
	Then we have the following chart:
	\begin{figure}[H]
		\centering
		\includegraphics[scale=0.65]{img/engineering/control_chart_cusum_plot_excel.jpg}
	\end{figure}
	We then very quickly see with this control chart that the cumulate sum seems to decrease. This indicates that the true mean quality is lower than the target (you can try with $9.994$ as the target to see the sensitivity).
	
	We must now consider two cases:
	\begin{enumerate}
		\item[H1.] lt comes under the assumption of normality (process under control) and by the property of stability of the Normal distribution for a centered reduced random variable:
		
		or otherwise:
		
		Therefore, under this hypothesis, we have the cumulative sums $(S_i)_{i=1\ldots n}$ which are distributed around a regression line of slope $0$ (since the average of the sums of a random variables with zero mean is zero...) and the distributions have their standard deviation, which is proportional to the root of $i$ (respectively the variance which is proportional to $i$).
		
		\item If the process is out of control (global dispersion), there is a moment $t<n$ such as $X_i=\mathcal{N}(\mu,\sigma)$ for $i\leq r-1$ amd $X_i=\mathcal{N}(\mu+\delta,\sigma)$ for $i\geq r$ with $\delta>0$ (the latter value being often imposed by the constraints of quality and named "\NewTerm{large middle translation}\index{large middle translation}"). Therefore, we have:
		
		or:
		
		for $i=1,\ldots,r-1$ and:
		
		for $k\geq r$. That said, it is obvious that the cumulative sums:
		
		are arranged around regression line of slope $\delta/\sigma$ right from the moment $r$.
	
		Respectively we have:
		
	\end{enumerate}
	To continue, the idea is to discriminate hypothesis H1 of a regression line of zero slope and the hypothesis H2 of a regression line of $\delta/\sigma$ for $S_k$ by fixing for imaginary border the half of the slope (the right middle in other words...):
	
	respectively:
	
	and a shift (offset) $h\geq 0$ also empirical with $h\in \mathbb{N}$. At the step $n$ we reject H1 if for an integer $m\leq n$, we have:
	
	It is obviously also possible to do it by reject with a geometrical representation (this has the advantage of going much faster and to the control in one time only for any $m$ less than or equal to $n$). The idea then is to draw a line of slope $K$ starting from the point of abscissa $n+1h$. If there exist a point $(m,S_m)$ with $m<n$ below the regression line, then we pull an alarm.
	
	Let us do an example with the previous graph. Consider the following empirical values of the parameters:
	
	Thus:
	
	Therefore:
	\begin{figure}[H]
		\centering
		\includegraphics{img/engineering/control_chart_cusum_vmask_calculations_excel.jpg}
	\end{figure}
	Explicitly:
	\begin{figure}[H]
		\centering
		\includegraphics{img/engineering/control_chart_cusum_vmask_calculations_explicit_excel.jpg}
	\end{figure}
	We then for example at the point $26$ corresponding to $25 + h$:
	\begin{figure}[H]
		\centering
		\includegraphics[scale=0.65]{img/engineering/control_chart_cusum_vmask_plot_excel.jpg}
	\end{figure}
	With Minitab 17.1.3 this gives for people that don't trust our calculations made with Microsoft Excel:
	\begin{figure}[H]
		\centering
		\includegraphics[scale=0.85]{img/engineering/control_chart_cusum_vmask_plot_minitab.jpg}
	\end{figure}
	The reader can easily observe that with the choice we made, we would have no alarm with the selected empirical parameters. It would be necessary in practice to reduce the value $\delta$.

	To resume, the benefits of CUSUM control charts compared to Shewhart control charts are:
	\begin{itemize}
		\item They are more effective for detecting small drifts

		\item The drift of the process visually appears on the control chart

		\item It is usually easy to detect at what point the process began to drift
	\end{itemize}
	The disadvantages of these control charts are the following:
	\begin{itemize}
		\item They can be slow to detect large drifts

		\item They are not very interesting to analyze past data and look for cycles or gaits characteristics in the studied distribution.

		\item They are less easily accepted by operators as less intuitive and do not directly represent the characteristic of interest.
	\end{itemize}
	For the reasons discussed above, it is advisable to represent in parallel of CUSUM control charts to communicate a classicl Shewhart control chart.
	\begin{tcolorbox}[title=Remark,colframe=black,arc=10pt]
	There are still other CUSUM V-mask type. So be careful with the software you used to be sure what you are doing!!!
	\end{tcolorbox}
	
	\paragraph{EWMA control charts (exponential weighted moving average) with fixed limits}\mbox{}\\\
	An interesting alternative to the CUSUM control charts in the context of autocorrelation and the detection of small deviations are the EWMA (Exponentially Weighted Moving Average) control charts that have their origin in the field of Time Series Analysis (\SeeChapter{see section Economy page \pageref{time series analysis}}).

	\begin{tcolorbox}[title=Remark,colframe=black,arc=10pt]
	Some also use control charts using empirical techniques like the simple or double moving average as discussed in section Economy when data are serially correlated.
	\end{tcolorbox}
	An EWMA control chart is easy to implement and is not too sensitive to parameter changes and non-Normal data, it remains just as powerful as the CUSUM control charts to detect small variations.

	By definition, the statistical reported on the EWMA control chart is calculated by the following equation (notice that the indices of the different terms are not the same as exponential smoothing of a time series as seen in section Economy since the goal is not to make predictions !!):
	
	
	with the particularity that for some softwares this software control chart (as Minitab) take:
	
	The constant $\lambda$ (smoothing constant) determines the weight that we want to assign to the latest measures. The smallest is this value, more the control chart is sensitive to sudden deviations.

	Let us prove that the last relation can be written as:
	
	Indeed:
	
	Let us suppose for what will follow that:
	
	Then we have:
	
	and remembering one of the properties of the variance (\SeeChapter{see section Statistics page \pageref{properties of the variance}}):
	
	It comes:
	
	Let us do a change of variable:
	
	The it comes using the result proved in the context of our study of arithmetic sequences in the section of Sequences and Series:
	
	when $t$ is sufficiently large, then the variance is reduced to (approximation that the majority of softwares - like Minitab - don't do and therefore the limits are not constant depending on $t$!):
	
	the EWMA control charts limits then are also build with the $\pm\sigma$:
	
	We therefore have the a target is defined and also a standard deviation (it is impossible that the customer sets  a LSL, USL or target $T$ regarding to the exponential weighting hence the fact that these parameters are this time not listed):
	that we find in the following form when it comes to working with samples:
	
	Some softwares uses to calculate the standard deviation the following relation that we have proved earlier above:
	
	and others (such as Minitab):
	
	the latter option has the advantage of working if the sample size is unitary (we speak then of course "EWMA for individual values"). Let us recall that in fact we have already mentioned that the EWMA control chart is poorly suited to mass production.
	
	It should be noticed that regardless of the option chosen, as we have:
	
	the controls limits are always significantly below those of a range or moving range control chart. This is also a reason why we smooth the data!

	Let us do an example for each variant of calculation of the standard deviation since each is of equal importance in practice. To do this, let us start with the EWMA control chart with:
	
	and taking the following table of data:
	\begin{figure}[H]
		\centering
		\includegraphics{img/engineering/control_chart_ewma_data_list_excel.jpg}
	\end{figure}
	We get:
	\begin{figure}[H]
		\centering
		\includegraphics{img/engineering/control_chart_ewma_calculations_excel.jpg}
	\end{figure}
	Thus explicitly (we have cut the screenshot into two parts because of horizontal space on the page):
	\begin{figure}[H]
		\centering
		\includegraphics{img/engineering/control_chart_ewma_calculations_formula_excel_part1.jpg}
	\end{figure}
	and for the second part:
	\begin{figure}[H]
		\centering
		\includegraphics[scale=0.7]{img/engineering/control_chart_ewma_calculations_formula_excel_part2.jpg}
	\end{figure}
	We then have the following control chart:
	\begin{figure}[H]
		\centering
		\includegraphics[scale=0.6]{img/engineering/control_chart_ewma_plot_excel.jpg}
	\end{figure}
	With Minitab 17.3.1 we get a small difference in comparison to our calculations by hand and this is normal as we have made an approximation for a large $t$ and also we made on of the two choice for $F_0$:
	\begin{figure}[H]
		\centering
		\includegraphics{img/engineering/control_chart_ewma_plot_minitab.jpg}
	\end{figure}
	For the last example, we take the famous case of an EWMA control chart with individual values and:
	
	with the following data:
	\begin{figure}[H]
		\centering
		\includegraphics{img/engineering/control_chart_ewma_moving_range_excel_list_data.jpg}
	\end{figure}
	by taking therefore (caution! the $n$ in the following relation is not the same as that in the root containing the smoothing constant!):
	
	and:
	
	as well as:
	
	We then with a smoothing constant of $0.6$:
	\begin{figure}[H]
		\centering
		\includegraphics{img/engineering/control_chart_ewma_moving_range_excel_calculations.jpg}
	\end{figure}
	Thus explicitly (we have cut the screenshot into two parts because of horizontal space on the page):
	\begin{figure}[H]
		\centering
		\includegraphics{img/engineering/control_chart_ewma_moving_average_calculations_formula_excel_part1.jpg}
	\end{figure}
	and for the second part:
	\begin{figure}[H]
		\centering
		\includegraphics[scale=0.85]{img/engineering/control_chart_ewma_moving_average_calculations_formula_excel_part2.jpg}
	\end{figure}
	We then have the following control chart:
	\begin{figure}[H]
		\centering
		\includegraphics[scale=0.6]{img/engineering/control_chart_ewma_moving_range_plot_excel.jpg}
	\end{figure}
	With Minitab 17.3.1 we get a small difference in comparison to our calculations by hand and this is normal as we have made an approximation for a large $t$ and also we made on of the two choice for $F_0$:
	\begin{figure}[H]
		\centering
		\includegraphics{img/engineering/control_chart_ewma_moving_range_plot_minitab.jpg}
	\end{figure}
	As with other control charts, the EWMA control chart must be accompanied by a control chart for monitoring the variability, so either a control chart based on ranges with fixed limits or a control chart with standard deviations with fixed limits.
	
	\subsubsection{Rare events control charts}
	Rare events inherently occur in all kinds of processes. In hospitals, there are medication errors, infections, patient falls, ventilator-associated pneumonias, and other rare, adverse events that cause prolonged hospital stays and increase healthcare costs. 

	But rare events happen in many other contexts, too. Software developers may need to track errors in lines of programming code, or a quality practitioner may need to monitor a low-defect process in a high-yield manufacturing environment. Accidents that occur on the shop floor and aircraft engine failures are also rare events, ideally.

	Whether you're in healthcare, software development, manufacturing or some other industry, statistical process control is an important component of quality improvement. Using control charts, we can graph these rare events and monitor a process to determine if it's stable or if it's out of control and therefore unpredictable and in need of attention.
	
	\paragraph{Frequency $T$ control chart with probabilistic limits}\mbox{}\\\
	The monitoring of frequency of occurrence of defects / accidents / anomalies is a very simple technique to implement and very easy to read for people with little or no knowledge of statistics. Moreover it is a technique adapted to cases where the monitored events are very rare.

	The idea then is not to track the number of special events, but the time $T$ between two appearances of the monitored events. The time $T$ is then measured in either number of pieces or number of accidents, defects, days, etc.

	For companion example let us consider the following list of data:
	\begin{figure}[H]
		\centering
		\includegraphics{img/engineering/control_chart_frequency_excel_list_data.jpg}
	\end{figure}
	Which gives graphically:
	\begin{figure}[H]
		\centering
		\includegraphics[scale=0.65]{img/engineering/control_chart_frequencies_plot_excel.jpg}
	\end{figure}
	Well it's nice, but we we need an UCL and LCL for this control chart to be reaaly useful! To achieve this we will use the queing theory with the same assumptions and the same checkpoints (\SeeChapter{see section Quantitative Management page \pageref{queueing theory}}). Let us first recall that we had showed there  that the probability of observing $k$ events in a time interval $t$ (or an interval of pieces number $t$) under some very specific assumptions (!!!), followed a Poisson distribution:
	
	where the parameter $\lambda$ is the average arrival rate (or "average rate of occurrence") of events per unit time (or "\NewTerm{Poisson Arrivals See Time Average (PASTA)}\index{Poisson Arrivals See Time Average}") assumed constant over the entire period:
	
	and where we have for mean and variance (\SeeChapter{see section Statistics page \pageref{poisson distribution}}) the number of events:
	
	This is a probability law with discreet support as we know, which fits well with our case studies so the Poisson distribution can be a good candidate (but there are others, like for example the Geometric law!!!).

	But we also proved that the random variable $\tau$ representing the time (or the number of pieces) separating two arrivals of unwanted events (or desired...) was given by the distribution function based on the exponential law:
	
	Still with:
	
	where:
	
	So we could take the following Shewhart limits :
	
	But it is more rigorous (as for all other control charts) to use probabilistic limits. So if we want to set limits with a usual bilateral risk threshold $\alpha/2$ of $0.5\%$ then we calculate for the lower limit:
	
	and for the superior limit:
	
	Then we have:
	\begin{figure}[H]
		\centering
		\includegraphics[scale=0.75]{img/engineering/control_chart_frequency_calculations_excel.jpg}
	\end{figure}
	Thus explicitly:
	\begin{figure}[H]
		\centering
		\includegraphics[scale=0.75]{img/engineering/control_chart_frequency_calculations_formulas_excel.jpg}
	\end{figure}
	We then have the following control chart (we have exceptionally not represented the USL on it):
	\begin{figure}[H]
		\centering
		\includegraphics[scale=0.65]{img/engineering/control_chart_frequencies_plot_with_limits_excel.jpg}
	\end{figure}
	With Minitab 17.3.1 we get something quite different all control limits. As they don't give the proofs but only the formulas we are not able to say who has the most correct way between we and them to make the calculations:
	\begin{figure}[H]
		\centering
		\includegraphics[scale=0.9]{img/engineering/control_chart_frequencies_plot_minitab.jpg}
	\end{figure}
	
	\paragraph{Frequency $G$ control chart of rare events}\mbox{}\\\\
	Developed by James Benneyan in 1991, the g chart  is a control chart that is based on the geometric distribution. Benneyan has since published several papers about the G chart and a companion chart, the "$h$ chart". The majority of applications cited in these papers are for monitoring infection rates in healthcare, such as nosocomial infections. Nosocomial infections are infections that occur as a direct result of a patient’s treatment in a medical facility.

	$P$ charts and $U$ charts are often used to monitor adverse events such as nosocomial infections. But $P$ charts and $U$ charts require very large quantities of data and specific definitions of the data. For example, if you use a $U$ chart to monitor nosocomial infections, each patient day is considered an area of opportunity in which one or more infections could occur. Thus, the data are the number of infections per patient day. If you use a $P$ chart, the data are the number of patient days in which one or more infections occur. As for the data requirements, if you follow the standard practice of requiring a minimum of $25$ to $35$ subgroups to establish control limits, and the infection rate is low (for example, $< 1\%$), the required amount of data is at least $12,500$ patients ($500$ patients per subgroup multiplied by $25$ subgroups). This means that it could take weeks, months, or perhaps even years to accumulate enough data to detect and respond to changes in the infection rate.
	
	The geometric distribution provides an alternative probability model. In the geometric distribution, you count the number of opportunities before or until the defect occurs. Thus, in a healthcare setting where you monitor the infection rate, the ideal would be to count the number of patients or procedures until an infection is observed. While this is the ideal, it is also rarely done, because of complications with counting the actual number of patients through the system, or the number of procedures. What is most often done is to count the number of days between observed infections. The key assumption used when counting the number of days is that the number of
patients or procedures per day is fairly constant.

	We have proved in the section Statistics that the mean to have $R$ successes before the $E$-th failure knowing that the probability of a failure is $p$ was given by the mean of the negative binomial distribution:
	
	and we get for to the variance:
	
	Obviously, if we set $E$ as the first failure, the mean is reduced to that of the geometric law:
	
	We have proved in the section Statistics that the estimator of the parameter of the geometric law was given by:
	
	Therefore it comes:
	
	and:
	
	and therefore:
	
	Therefore we have:
	
	where if LCL is negative, we put it as always as being equal to zero.
 
 	Let us see an example by considering the following list in Microsoft Excel 14.0.6123:
 	\begin{figure}[H]
		\centering
		\includegraphics{img/engineering/control_chart_frequencies_g_list_data_excel.jpg}
	\end{figure}
	with the following formulas:
	\begin{figure}[H]
		\centering
		\includegraphics[scale=0.7]{img/engineering/control_chart_frequencies_g_formulas_excel.jpg}
	\end{figure}
	Which gives graphically:
	\begin{figure}[H]
		\centering
		\includegraphics[scale=0.75]{img/engineering/control_chart_frequency_g_plot_minitab.jpg}
	\end{figure}
	With Minitab 17.3.1 we get something quite different all control limits. As they don't give the proofs but only the formulas we are not able to say who has the most correct way between we and them to make the calculations:
	\begin{figure}[H]
		\centering
		\includegraphics{img/engineering/control_chart_frequencies_g_plot_minitab.jpg}
	\end{figure}
	For a software like Minitab, the lower control limit of the rare events chart is set to the percentile $0.0013499$ of the geometric distribution (taking the nearest decimal value by linear interpolation). The upper control limit is set to the percentile $0.99865$ (also taking the nearest decimal value by linear interpolation). The central line is set at $0.5$ percentile (taking the nearest decimal value by linear interpolation).
	
	It should be also remembered that the Geometric law  (or the Negative Binomial law) is not necessarily symmetrical. This is why, in practice, you should rather use the median for the CL (centerline which can be determined by simple simulations of the law and taking the nearest value of the $50\%$ percentile) and for the LCL and UCL control limits use also the probabilistic simulations with the same simulations.
	
	\pagebreak
	\subsubsection{Control Charts Operating Characteristic (OC) Curves}
	In the usage of control charts we face two types of errors (putting apart human error and incompetence error...) that are well known to us in the context of our study of statistics:
	\begin{itemize}
		\item Type I error: We detect an alarm when there is actually no alarm

		\item Type II error: We are out of control but we do not detect alarm
	\end{itemize}
	The aim here will be to determine the conditions for controlling these two errors.
	
	As we will see the operating characteristic function for a Shewhart chart, is the probability, the statistics of interest falls between the lower and upper control limits.

	\paragraph{OC for $\bar{X}$ measurement control charts}\mbox{}\\\\
	We have already demonstrated earlier above that the capability that a process centered at $3\sigma$ had a cumulative probability of $0.27\%$ ($1/370$) to have a point outside the $3\sigma$ and this only for reasons of statistical fluctuations (error of Type I). So in extenso to $3\sigma$ the error type II is $1-0.27\%$ which is quite huge but acceptable in practice (as practitioner consider a power of $80\%$ as being acceptable)! The reason for this high amplitude of type II error is implicitly due to the fact that we calculate here:
	
 	that is to say that the data have all been centered reduced as well as in extenso that the USL and the LSL and where for recall is the symbol $\Phi$ stand for the cumulative probability function of the Normal law $\mathcal{N}(0,1)$.

	But if the data is not centered, then we must calculate:
	
 	But if we are interested in a certain lag by loss of control of the process, then we have by expressing this shifts in multiples $m$ of $\sigma_X$ as it is done by tradition (this choice will be justified a little later):
	
 	Now, let us recall that we have in the framework of an $\bar{X}$ control chart:
	
	Therefore it comes:
	
 	We see that the choices at the beginning in the way of noting the offset advantage us by the fact that the error of type II is a function of more than three parameters of which $k$ is almost always fixed at $3$ and $n$ is chosen by the experimenter.

	So finally we have:
	
 	\begin{tcolorbox}[colframe=black,colback=white,sharp corners]
	\textbf{{\Large \ding{45}}Example:}\\\\
	Suppose we use an $\bar{X}$ control chart with a $k$ of $3\sigma_{\bar{X}}$, a sample size of $n=5$ and we want to know the Type II error to detect an offset of $2\sigma_X$ ($m=2$). From then on, it comes:
	
	So the risk of type II is this time much lower but the error of type I then becomes $1-0.0705 = 0.9295$ which is huge. This is why it is often necessary to find the right compromise between the quality policy and the adjustment adjustment costs.
	\end{tcolorbox}
	Let us indicate that the general case, for $k$ equal to $3$, we have the following curve which is named "\NewTerm{$\bar{X}$ control chart operating characteristic curve}\index{operating characteristic curve}":
	 \begin{figure}[H]
		\centering
		\includegraphics[scale=1]{img/engineering/operating_characteristic_curve_xbar_control_chart.jpg}
		\caption{$\bar{X}$ control chart operating characteristic curve}
	\end{figure}
	So on the plot above we have $k=3$, $n=5$, $m=2$:
	
 	either with the English version of Microsoft Excel 14.0.7177:
	\begin{center}
		\texttt{=NORM.S.DIST(3-2*SQRT(5),TRUE)-NORM.S.DIST(-3-2*SQRT (5),TRUE)}
	\end{center}
	
	\paragraph{OC for $P$-type attribute control charts}\mbox{}\\\\
	The same type of reasoning applies for a attributes $P$-type control chart! We have then:
	
	However, we saw in the Statistics section that when $N$ is large the interval could be estimated using a Normal law such that:
	
 	Hence, since $\bar{\hat{p}}$ follow approximately a Normal law, we arrive at the same type of calculations as for the previous control chart:
	
	Thus finally:
	
	
	 \pagebreak
	 \subsection{Design of reliability tests}\label{design of reliability tests}
	 "\NewTerm{Reliability test plans}\index{reliability test plan}" have two main functions: One is to determine the sample size and testing time needed to estimate model parameters. The other is to demonstrate that we have met specified reliability requirements.

	A test plan includes:
	\begin{itemize}
		\item The number of units you need to test.
		\item A stopping rule: the amount of time you must test each unit or the number of failures that must occur.
		\item Success criterion: the number of failures allowed while the test still passes (for example, all units are tested for the specified amount of time and there are no failures).
	\end{itemize}
	Three types of test plans are available: demonstration, estimation, and accelerated life.
	\begin{itemize}
		\item "\NewTerm{demonstration plan}\index{demonstration plan}" to determine the sample size or testing time needed to demonstrate, with some level of confidence, that the reliability is higher than a particular standard.

		There are two types of demonstration tests:
		\begin{enumerate}
			\item "\NewTerm{Substantiation tests}\index{substantation tests}" that provide statistical evidence that a redesigned system suppressed or significantly reduced a known cause of failure. For example, is the redesigned system better than the previous system?\\
			
			 We can rewrite these hypotheses in terms of the scale (Weibull or exponential distribution) or location (other distributions), a percentile, the reliability at a particular time, or the mean time to failure (MTTF). For example, we can test whether the MTTF for a redesigned system is greater than the MTTF for the previous system.\\

			\item "\NewTerm{Reliability test}\index{reliability test}" that provide statistical evidence that a reliability specification is achieved. For example, is the system reliability greater than a goal value?
		\end{enumerate}

		\item "\NewTerm{Estimation test plans}\index{estimation test plans}" to determine the number of test units that we need to estimate percentiles or reliabilities with a specified degree of precision. Estimation test plans are similar to classic sample-size problems, but computations are more intensive because the data are usually censored.

		\item "\NewTerm{Accelerated life test plans}\index{accelerated life test plans}" to determine - under some very strong assumptions - the number of units to test and the allocation of those test units across stress levels for an accelerated life test. We can also use these test plans to determine the standard error for the parameter we want to estimate for a fixed number of test unit
	\end{itemize}
	 \begin{figure}[H]
		\centering
		\includegraphics[scale=0.52]{img/engineering/pnnl_demonstation_plan.jpg}
		\caption{Pacific Northwest National Laboratory automatic long-term performance measurement equipment for  Philips LED 60 Watt, which can test up to $200$ lamps at a time}
	\end{figure}
	Several methods have been designed to help engineers for all this purposes: Cumulative Binomial, Non-Parametric Binomial, Exponential Chi-Squared and Non-Parametric Bayesian. They are discussed in the following paragraphs.
	
	\subsubsection{Chi-squared time of test}
	A method for designing tests for products that have an assumed constant failure rate, or exponential life distribution, draws on the chi-squared distribution. These represent the true exponential distribution confidence bounds referred to in The Exponential Distribution. This method only returns the necessary accumulated test time for a demonstrated reliability or $\overline{\text{MTTF}}$.
	
	We have seen earlier above that for a survival function of the form:
	
	The corresponding density function was the exponential law:
	
 	and that we had:
	
 	Our objective here will be to find the value that permits a reliability engineer to know how long a test must be performed to demonstrate with a $90\%$ confidence level (CL) that its product has zero failure.

	What looks like in the worldwide reference Weibull++ software for reliability engineering to the following:
	\begin{figure}[H]
		\centering
		\includegraphics[scale=1]{img/engineering/design_of_reliability_exponential_chi_squared_time_of_test.jpg}
		\caption{Exponential Chi-squares time test in Weibull++ 7 software}
	\end{figure}
	We will prove in the following that this result follows from a close relationship between the exponential distribution, the Poisson distribution through the queue theory (\SeeChapter{see section Quantitative Management page \pageref{queueing theory}}) and the chi-square distribution.

	To prove this result, let us recall that we have prove in the section of Quantitative Management that under certain hypotheses of a queue, the probability of seeing an event in an interval of duration $t$ is given by the function:
	
 	and that the corresponding density function of the corresponding time between two events is an exponential law. In other words, when the time of failure follows an exponential law, the probability of a certain number of failures in a certain time interval follows the relation given just before!

	The previous relation can be used in the case of reliability to obtain an upper bound of $\lambda$ for a given time interval $t$ since we can then write with the Erlang distribution function (\SeeChapter{see section Quantitative Management page \pageref{erlang distribution}}):
	
	A small and simple change of variable gives:
	
	We therefore recognize the function of the distribution of the Poisson distribution! However, this is impossible to reverse by hand (as far as we know). Reason why it is wiser to find a relation with another known distribution that is continuous in the purpose to reverse it by hand.

	For this, let us recall that we have proved during our study of the chi-square law in the section Statistics that:
	
	Therefore, by re-reading carefully the changes of variables made in the section Statistics we have in other words:
	
 	It is therefore immediate that:
	
	It follows that:
	
	Conversely:
	
	Therefore:
	
	
	\subsubsection{Binomial sampling size}
	In demonstration plans, an important calculation is to determine the size of a sample in order to demonstrate a certain degree of reliability for a given level of confidence. In the case where the test time is equal to the demonstration time, the binomial distribution is widely used in practice because nonparametric and relative only to the estimated parameter $R$ (see earlier above):
	
	with $f$ the number of failures, $n$ the size of the sample, and $R$ the reliability that is sought to be demonstrated. We then speak of a "\NewTerm{nonparametric binomial demonstration plan}\index{nonparametric binomial demonstration plan}".
	
	As the reader can notice, there is no time value associated with this methodology, so one must assume that the value of $R$  is associated with the amount of time for which the units were tested.
	\begin{tcolorbox}[colframe=black,colback=white,sharp corners]
	\textbf{{\Large \ding{45}}Example:}\\\\
	Consider that an engineer wants to set up a demonstration plan in order to demonstrate the zero failure of a product with an $80\%$ reliability with a confidence level of $90\%$. We have then:
	
	This gives by taking the logarithm:
	
	\begin{figure}[H]
		\centering
		\includegraphics[scale=0.7]{img/engineering/design_of_reliability_binomial_sample_size_test.jpg}
		\caption{Binomial sample size demonstration plan in Weibull++ 7  software}
	\end{figure}
	\end{tcolorbox}
	However, in reliability tests, in order to better estimate reliability, Bayesian methods are used, especially when sample size is financially limited. If there is an a priori information about the reliability of the system, this information can then be used to build better demonstration plans.
	
	We will see later that the application of Bayesian techniques gives a quasi-similar result to the above example for certain values a priori well chosen...
	
	\subsubsection{Beta-binomial sampling size}
	The regular non-parametric analyses performed based on either the binomial or the chi-squared equation were performed with only the direct system test data. However, if prior information regarding system performance is available, it can be incorporated into a Bayesian non-parametric analysis. This subsection will demonstrate how to incorporate prior information about system reliability and also how to incorporate prior information from subsystem tests into system test design.

	Thus, since $R$ is bounded in the interval $[0, 1]$ a well-known a priori function is the beta function (\SeeChapter{see section Statistics page \pageref{beta distribution}}). We proved in the section Probabilities that the a posteriori law was then a beta-binomial for of which we had specified that reliability engineers noted the cumulative probability under the somewhat unfortunate notation:
	
 	Therefore, we have in our case a posteriori:
	
 	We know $f$ which is imposed to us by the demonstration plan, we are always trying to determine $N$. We know that for the beta law:
	
 	that we have:
	
	With a little elementary algebra we get:
	
 	The question is then what to do now ...? Then here the statistician engineer uses a trick ... It takes the generalized beta law with $4$ parameters including thus the famous optimistic, pessimistic interval $[O, P]$ that we studied in the section of Quantitative Management and uses the corresponding relation of mean and variance that we have proved:
	
	Where normally we should use for the modal value as we have proved in the section of Quantitative Management:
	
	but the overwhelming majority of engineering project management and reliability software make the fundamental error of asking the modal value to the user (we will see a concrete example a little later with the most famous reliability software used in the USA).

	Once this is done. The idea is to return to a beta law with two parameters by injecting the values of expectation and variance obtained by the beta law with $4$ parameters in:
	
 	which obviously imposes indirectly that beta laws with $4$ and $2$ parameters have then same variance and even expectation!
	\begin{tcolorbox}[colframe=black,colback=white,sharp corners]
	\textbf{{\Large \ding{45}}Example:}\\\\
	Consider that an engineer wants to set up a demonstration plan in order to demonstrate that the reliability of its product is $90\%$ with an $80\%$ confidence level assuming that a single failure is allowed and that the engineer believes that the optimistic reliability is of $O=87\%$, the most likely (modal) valueof$M_0=90\%$ and the greatest reliability is equal to $P=99\%$.\\

	We have then:
	
	and thus the parameters of the beta law with two parameters are then:
	
	We must then find $N$ such that:
	
 	It is easy to get $N$ with a spreadsheet software like Microsoft Excel using the BETA.DIST with Microsoft Excel 2007 and higher:
	\begin{center}
	\texttt{= BETA.DIST(90\%;N+184.4125-1;19.337;1)}
	\end{center}
	And with the target value tool, we get the result $N=24$ by rounding to the nearest integer. This corresponds well to the Weibull++ software output (see next page):\\
	\end{tcolorbox}
	
	\begin{tcolorbox}[colframe=black,colback=white,sharp corners]
	\begin{figure}[H]
		\centering
		\includegraphics[scale=1]{img/engineering/design_of_reliability_beta_binomial_sample_size_test.jpg}
		\caption{Beta-Binomial sample size demonstration plan in Weibull++ 7 software}
	\end{figure}
	It is interesting to know that if the modal value $M_0$ had been calculated rather than requested, we would have therefore $N=-30$ since in the latter case $M_0=97.246\%$... So we see what any model hat obviously its limits...\\

	If we use the same parameters with $O=87\%$, $M_0=90\%$, $P=99\%$ with our case of the Binomial law that we used as the first example above, we then rounded to the nearest integer $N=13$.
	\end{tcolorbox}
	
	\begin{flushright}
	\begin{tabular}{l c}
	\circled{60} & \pbox{20cm}{\score{3}{5} \\ {\tiny 19 votes,  53.68\%}} 
	\end{tabular} 
	\end{flushright}

	
	
 	\chapter{Epilogue}
	You have traveled a long distance with us through this book. We have now reached the epilogue, where by tradition, the main redactor is allowed to give voice its own personal opinions. 

	Indeed, I want to leave you with some of my thoughts on theory versus practice, business and engineering education, applied mathematics research, and what I hope you will take with you after having read this book.
	
	By nature, academic engineering is very closely related to its practice. Theory and practice are ruled by the same ideas. As an academic myself teaching since 2001 in Fortune 500 and SME companies, I am proud to claim that the majority of engineering ideas were either invented or developed in academia first before they crossed over into practice.
	
	But engineering research is not just for aspiring academics: As i know it very well management consultants and economics consultants are basically researchers even if most of them have a very low analytic level. Firms like McKinsey, Ernst and Young, KPMG, Accenture may have different audiences, production speeds, team systems, and publication and evaluation processes, but they research the same issues that academics do and with the same methods at the exception that the global internal analytic level is much lower.

	There is also much cross-fertilization: Many professors work regularly with major consulting in USA or asset-investment firms and some have even quit academia altogether to quadruple their pay.

	Because engineering is by nature such an applied discipline, after reading this book,  you should not need anything else to understand engineering research today. In an ideal world, you should be able to read the current state-of-the-art research right now.
	
	So, do we really understand engineering? Certainly not fully. We haven seen through this that finance is as much an art as it is a science. Given our deficiencies, given that all our methods have their errors, what should we do? My best advice to you is to use common sense, to employ a number of different techniques to come up with a range of possible answers, and to then make a judgment at the end of the day as to what estimate appears most reasonable in light of different models. As I have noted many times, engineering is art based on science.
	
	Our book has covered the principles of engineering in some depth and breadth. You should be very well prepared now for the next steps in your engineering education.
	
	I have enjoyed writing this book in the same way that I enjoy writing my training books, and pretty much for the same reason: It has been like solving an intriguing puzzle that no one else has figured out in quite the same way a particular way to see and explain finance. Of course, writing it has taken me quite a long times (20 years, without translation: 15 years!).
	
	But it will all have been worth it if you have learned from it. If you have studied the book, you should now know about $99\%$ of what I know about engineering. Interestingly, there were a number of topics that I thought I had understood, but had not - and it was only my having to explain them to you that clarified them for me, too. And this brings me to a key point that I want to leave you with - never be afraid to ask questions, even about first principles. To do so is not a sign of stupidity on the
contrary, it is often a sign of deepening awareness and understanding.

	I have no illusions: You will not remember all the fine details in this book as time passes - I know I won't. But more than the details, I hope that I will have left you with an appreciation for the big ideas, an arsenal of tools, a method for approaching novel problems, and a new perspective. You can now think like an Engineer. 
	\begin{figure}[H]
		\centering
		\includegraphics[scale=0.1]{img/knowledge_is_power.jpg}	
	\end{figure}
	
	\chapter{Biographies}
	To be informed of Nobel laureates (physics, chemistry, economics), Fields Medal... click this link {\href{http://www.nobelprize.org/}{{\color{blue}Nobel Prizes}}} or on this one {\href{http://www.fields.utoronto.ca/aboutus/jcfields/fields_medal.html}{{\color{blue}Fields Medal}}}

This section contains a list of some humans who have a strange reputation. Under the rules of history that is taught in elementary school, they do not exist, they haven't ordered any army, they sent nobody to death, they did not have any empire and they had only a minor part in major historical decisions. Some have acquired some celebrity, but none was ever a national hero. Yet their work has more influenced on the course of history than many acts by statesmen crowned with a far greater glory. They produced more turmoil than the comings and goings of armies in battle over borders, they have done more for the happiness or unhappiness that the edicts of kings and assemblies, because their work, is to have shaped the mind of man!

Whoever diffuse his ideas, has a power much greater than that of the sword or scepter: this is shy why they have also shaped and directed the world. For the most part they have not lifted any finger to act physically, they worked mainly as intellectuals, in silence and oblivion, without worrying of the surrounding world. But in their wake, empires have crumbled, political regimes have either strengthened or eroded, the classes were pitted against each other, and also did nations. Not under the influence of a dark conspiracy, but by the extraordinary power of their ideas. Who are these humans?: Scientists, economists, chemists, biologists, mathematicians, physicists, computer scientists, engineers, ...

The biographies below of the most famous scientists around the world and cited in the various chapters of this book are sorted alphabetically and almost all texts are simplified copy/paste of the French {\href{http://www.wikipedia.fr}{{\color{blue}Wikipedia}}}. If you want us to add an entry, simply  {\href{mailto:isoz@sciences.ch}{{\color{blue}email}}} us the full name of the concerned person and why you would like to see us include him in the list below. We then study the proposal and take the appropriate decision.

\textbf{We also pay tribute to the hundreds of thousands of scientists, engineers, philosophers, craftsmen, artists, known and anonymous amateurs whose collaboration enabled through millennia the evolution of science and of the human condition!}

The sizes of the biographies are not proportional to the number of articles published or to the discoveries made, but on the amount of information found on the Internet or in the literature. The list is also not exhaustive, but its purpose is to honour and remember the great humans who made of pure sciences what they are today and who have spent part or whole of their life to science: most constrained art.

Caution! In physics (as well in mathematics) a theory, an equation or even a constant rarely wears the name of its true inventor. This is widely known among scientists and is often a source of joke from the community.

	\begin{figure}[H]
		\centering
		\includegraphics[scale=0.5]{img/shoulders_of_giants.jpg}	
	\end{figure}

\begin{center}
\hyperref[sec:A]{A} \hyperref[sec:B]{B} C D E F G H I J K L M N O P Q R S T U V W X Y Z
\end{center}

\phantomsection
\addcontentsline{toc}{section}{A}
\label{sec:A}
		
\pichskip{15pt}% Horizontal gap between picture and text
\parpic[l][t]{
  \begin{minipage}{40mm}
    \fbox{\includegraphics[width=110px,height=140px]{img/medaillons/al.eps}}
  \end{minipage}
}		
\textbf{Al-Biruni, Muhammad Ibn Ahmad Abul-Rayhan}(973-1048) is a mathematician, astronomer, physicist, encyclopedist, philosopher, astrologer, traveller, historian, pharmacologist and a tutor, native of Persia, who contributed greatly to the fields of mathematics, philosophy, medicine and science. He is known for his theory on the Earth's rotation around its axis and around the Sun, and this long before Copernicus. He focused particularly on the calculation of the Sun running (apogee) and also corrected some data of Ptolemy. Excellent mathematician, Al-Biruni developed new equations unknown to his predecessors. He calculated also the local meridian and the coordinates of some localities. But the picture would not be complete if we forgot to mention that six centuries before Galileo, Al Biruni already put forward an Earth that revolved around its axis. With the help of an astrolabe, the sea and a nearby mountain, he calculated the circumference of the Earth by solving a complex equation for its time. The main contribution of Al-Biruni to mathematics lies in its work in trigonometry (calculations of some trigonometric functions values that were not well defined at this time).

\pichskip{15pt}% Horizontal gap between picture and text
\parpic[l][t]{%
  \begin{minipage}{40mm}
    \fbox{\includegraphics[width=110px,height=140px]{img/medaillons/alembert.eps}}
  \end{minipage}
}
\textbf{Alembert, Jean le Rond} (1717-1783), child of a commissioner of artillery, abandoned on the steps of the chapel of Paris Saint-Jean-Le-Rond, the future great philosopher, mathematician and physicist is adopted by a glazier who secretly receive a pension to support the education of the young boy who brilliantly study law, medicine and mathematics. Following the publication of several memoirs (on integral calculus and refraction of solids), d'Alembert entered the Academy of Sciences (1741). He is at the origin of the famous momentum principle, named "D'Alembert's principle" in his \textit{ Traité de dynamique} (1743). In astronomy, he is the author of a treaty on the precession of the equinoxes (1749) explained by using the Newton theory of universal gravitation and with a partial solution to the three bodies problem. D'Alembert also establishes a mathematical theory of vibrating strings by studying the nature of the sound (harmonics).

\pichskip{15pt}% Horizontal gap between picture and text
\parpic[l][t]{%
  \begin{minipage}{40mm}
    \fbox{\includegraphics[width=110px,height=140px]{img/medaillons/ampere.eps}}
  \end{minipage}
}
\textbf{Ampère, André Marie} (1775-1836) at 18 years, he already knows most of the mathematical works of his time. First-class mathematician, he shows how we must use this science, that he was considerating as a branch of philosophy, to the study of physical facts to give a definitive relation. Within a few weeks, Ampere gives the foundation to a science to which he gives the name of "Electromagnetism". He tries to understand the magnetism of magnets and draws a hypothesis of "particulate flows" (today: electronic orbits spin orientation). It also equal the number of molecules in equal volumes of gases of different nature, but measured under identical conditions of temperature and pressure (experimental observation of Gay-Lussac).

\pichskip{15pt}% Horizontal gap between picture and text
\parpic[l][t]{%
  \begin{minipage}{40mm}
    \fbox{\includegraphics[width=110px,height=140px]{img/medaillons/archimede.eps}}
  \end{minipage}
}
\textbf{Archimedes of Syracuse }(287-212 BC.), is a famous Greek mathematician and engineer as both a theorist and as a mechanic machine manufacturer. Archimedes had an exceptional mathematical production, part of which was received in treaties such as \textit{On the Sphere and Cylinder}, \textit{Measurement of the circle; Quadrature of the parabola, spiral and the conoid and spheroids}; \textit{The Method of Bodies floating}... This is from his mechanical work that the main legends starts, like the lever or the bath, will be. The famous maxim: « Give me a place to stand and I will move the earth» is an echo of the popular Archimedean contribution to the static in the treaty of Equilibria. Archimedes proves the law of the lever, introduces the basic concept of center of gravity, and determines the centroids for the main plane geometric figures. It is the same for the story of Archimedes springing naked from his bath, crying  «Eureka», because he came, following the legend, to solve the problem posed to him by King Hiero. In fact, the story is a spectacular staging of the discovery of the fundamental principle of hydrostatics (commonly named "Archimedes principle"). In geometry, the work of Archimedes develops that of Eudoxus of Cnidus as we know it by the book XII of Euclid's Elements: this is to compare measurements of plane figures and solids, in particular from curvilinear figures. Thus, Archimedes proved that the volume of the circumscribing cylinder of a sphere is equal to one and half times it's volume and that the side surface of the cylinder is equal to that of the sphere or four times the surface of a great circle. So if we can calculate the area of the circle, we know that of the sphere, cylinder, its volume and that of the sphere, etc. His most famous result and easiest is for the circle. Archimedes brings it's quadrature to another problem: the correction of its circumference, that is to say "find a equal straight line to it is equal", it solves the problem using a geometric curve that is now named "Archimedean spiral". In addition, it calculates the approximate values of the circumference/diameter ratio (what we name the number "Pi" noted $\pi$).

\pichskip{15pt}% Horizontal gap between picture and text
\parpic[l][t]{%
  \begin{minipage}{40mm}
    \fbox{\includegraphics[width=110px,height=140px]{img/medaillons/avogadro.eps}}
  \end{minipage}
}
\textbf{Avogadro, Amedeo} (1776-1856), son of a magistrate of Turin, Amedeo begins his life by following his father path. He pass a law degree in 1795 and practice in his hometown. But his love for physics and mathematics, which he studied alone, drives him to start on the late scientific studies. In 1809 he presents a paper to the Royal Academy of Turin; its success allows him to get a professorship at the Royal College of Vercelli. In 1820, the University of Turin created for him a chair of physics that he will keep until the end of his life. By studying the laws governing the compression and expansion of gases Avogadro states, in 1811 the hypothesis famously known as "Avogadro's law". Based on the atomic theory of Dalton's and Gay-Lussac law on the volume ratios, Avogadro's theory indicates that two equal volumes of different gases, under the same conditions of temperature and pressure, contain the same number of molecules. Under it's seeming simplicity, this law has important implications, because of it, it becomes possible to determine the molar mass of a gas from one another. But the chemists at this time, more interested in experiences, are not attentive to the theoretical studies of Avogadro who will also be recognized only 50 years later. The name Avogadro also remains linked to that of "Avogadro's number" indicating the number of molecules in one mole.

\phantomsection
\addcontentsline{toc}{section}{B}	
\label{sec:B}

\pichskip{15pt}% Horizontal gap between picture and text
\parpic[l][t]{%
  \begin{minipage}{40mm}
    \fbox{\includegraphics[width=110px,height=140px]{img/medaillons/bachelier.eps}}
  \end{minipage}
}
\textbf{Bachelier, Louis} (1870-1946) was born in Le Havre in a family of merchants. He appears at his majority on the electoral lists of Le Havre in 1892 as a sales representative at the same business address as his father. After completing his military service at age 22, he resumed his studies at the Faculty of Sciences in Paris. He obtained a Bachelor of Science in 1895 (passing grade) and a Ph.D. in 1900 with his famous and unknown subject in mathematics... Although his theory is now considered as a pioneering work in probability and financial theory. From 1913 to 1914 Bachelor teaches probability theory applied to mechanics, ballistics, and biometrics. He was also responsible for additional conferences on general mathematics from 1913 to 1914. It is only after the 1914-1918 war that he obtained a first post of lecturer at the Faculty of Besançon. After several replacements in Dijon and Rennes, he returned to Besançon in 1927 as Chair Professor of calculus, a position he held until his retirement in 1937. Louis Bachelier, among his numerous works, was the first to have introduced the continuity in probabilities problems taking time as a variable. In particular, he developed a mathematical theory of Brownian motion five years before Albert Einstein. He is also, well before Norbert Wiener, the first to have defined the function of Brownian motion and gave many of its properties.

\parpic[l][t]{%
  \begin{minipage}{40mm}
    \fbox{\includegraphics[width=110px,height=140px]{img/medaillons/balmer.eps}}
  \end{minipage}
}
\textbf{Balmer, Johann Jakob} (1825-1898) was a Swiss mathematician and mathematical physicist born in Lausen (Switzerland) and died in Basel. During his schooling he excelled in mathematics, and so decided to focus on that field when he attended university. He studied at the University of Karlsruhe and the University of Berlin, then completed his Ph.D. from the University of Basel in 1849 with a dissertation on the cycloid. Johann then spent his entire life in Basel, where he taught at a school for girls. He also lectured at the University of Basel. Despite being a mathematician, he is not remembered for any work in that field; rather, his major contribution (made at the age of sixty, in 1885) was an empirical formula for the visible spectral lines of the hydrogen atom, the study of which he took up at the suggestion of Eduard Hagenbach also of Basel. A full explanation of his formula worked, however, had to wait until the presentation of the Bohr model of the atom by Niels Bohr in 1913.

\parpic[l][t]{%
  \begin{minipage}{40mm}
    \fbox{\includegraphics[width=110px,height=140px]{img/medaillons/banach.eps}}
  \end{minipage}
}
\textbf{Banach, Stefan} (1892-1945) was a Polish mathematician who define the foundations of functional analysis. Born in Krakow in 1892, Austria-Hungary (now Polish city). Banach went to high school in Krakow, where he revealed to be particularly brilliant in mathematics and natural sciences, but his disinterest in other matters prevented him from obtaining the best evaluation. Banach's life (at least mathematically) will switch in the spring of 1916, when he meets Steinhaus in Krakow. With Otto Nikodym, they decided to found a mathematical society. Banach's mathematical research begins at this moment. His first article was co-authored with Steinhaus. Steinhaus told him about a property he could not be able to prove, and after some days of reflection, Banach exhibited a cons-example. It is difficult to say what would have happened to Banach's mathematical activity without the meeting with Steinhaus, but the fact remains that he began only after it's intense and fruitful researches. Banach returns in Lvov in 1920 as an assistant. He submitted his thesis in 1922, and it is in this thesis that appears for the first time the notion of Banach space and where the fundamentals theorems about these objects are prove and also and also where there is a discussion on weak topology... . In short, this thesis marks the birth of functional analysis. In 1929, he founded with Steinhaus the magazine "Studia Math", dedicated to the development of functional analysis, and in 1939 he was elected president of the Mathematical Society of Poland. In 1945, shortly before the end of World War II, he died of a long cancer. Many theorems are associated with the name Banach, that he has demonstrated himself, or that they refer to it's ideas. These include: the theorem of Hahn-Banach about the extension of continuous linear forms, the theorem of Banach-Steinhaus, Banach-Alaoglu, the Banach fixed point theorem of course the Banach-Tarski paradox.

\parpic[l][t]{%
  \begin{minipage}{40mm}
    \fbox{\includegraphics[width=110px,height=140px]{img/medaillons/bell.eps}}
  \end{minipage}
}
\textbf{Bell, John} (1928-1990) was from earliest childhood attracted to books about science. Because of family financial problems, he could not immediately follow academic studies. He then worked for a year as a technician in the physics department of Queen's University in Belfast before becoming a student in 1945 in the same department. He went out first rank in his class in math-physics. Bell founded in the years 1960 a new inspiration in the foundations of quantum theory, an area supposedly exhausted by the results of the Bohr-Einstein debate thirty years earlier, and ignored by almost everyone who used the quantum theory-between time. Indeed, Bell was intrigued by the Heisenberg quantum uncertainty and wanted to delve deeper by showing that the discussion of concepts such as "realism", "determinism" and "locality" could be affiliated in a rigorous mathematical relation: "the Bell inequalities" experimentally verifiable. Bell pushed very far the doubts he had on the principles of uncertainty to the point that even irritated his teacher (Sloane) who told him that now he was going too far! Bell waited for his thesis to develop his ideas. Unfortunately, because of financial problems again, he had to delay his research and join the Harwell atomic research center. During his career, he married a woman (Mary Bell) who helped him in the development of its work on the fundamentals of quantum theory. It is in 1951, with Rudolf Peierls, that Bell developed his famous CPT theory (Charge, Parity, Time). Unfortunately for Bell, Gerhard Lüders and Wolfgang Pauli came to the same result in the same period and it is to them that were awarded this discovery. The theoretical developments of Bell are at the origin of cryptography and quantum information theory. Another major work of Bell in 1969 was the participation in the development of "the A.B.J. anomaly" (Adler-Bell-Jackiw) in quantum field theory. This three physicists showed that the standard algebraic model merely an error. Indeed, quantification of the fields model broke a symmetry. Bell was nominated for the Nobel Prize, that he certainly would have obtained if he had not died in 1990.

\parpic[l][t]{%
  \begin{minipage}{40mm}
    \fbox{\includegraphics[width=110px,height=140px]{img/medaillons/berners_lee_timothy_john.jpg}}
  \end{minipage}
}
\textbf{Berners-Lee, Timothy John} (1955-) is an English physicist best known as the inventor of the World Wide Web. He is the director of the World Wide Web Consortium (W3C), which oversees the continued development of the Web. He is also the founder of the World Wide Web Foundation, and is a senior researcher and holder of the founders chair at the MIT Computer Science and Artificial Intelligence Laboratory. He is a director of the Web Science Research Initiative (WSRI), and a member of the advisory board of the MIT Center for Collective Intelligence. He worked as an independent contractor at CERN from June to December 1980. While in Geneva, he proposed a project based on the concept of hypertext, to facilitate sharing and updating information among researchers. The address info.cern.ch was the address of the world's first-ever web site and web server, running on a NeXT computer at CERN. The first web page address was http://info.cern.ch/hypertext/WWW/TheProject.htm.

\parpic[l][t]{%
  \begin{minipage}{40mm}
    \fbox{\includegraphics[width=110px,height=140px]{img/medaillons/bernoulli_daniel.eps}}
  \end{minipage}
}
\textbf{Bernoulli, Daniel} (1700-1782) was a Swiss scientist who discovered the basic principles of behavior of a fluid (it is also the son of Johann Bernoulli and the nephew of Jacques Bernoulli). He cultivated both mathematics and natural sciences, taught mathematics, anatomy, botany and physics. Friend of Leonhard Euler, he worked with him in several areas of mathematics and physics (he shared with him ten times the annual prize of the Academy of Sciences of Paris), he made this prize a kind of income. The various problems he tried to solve (elasticity theory, mechanism of tides) led him to focus and develop mathematical tools such as differential equations or series. He also collaborated with Jean le Rond d'Alembert in the study of vibrating strings. He studied the flow of fluids (1738) and formulated the principle (the famous Bernoulli theorem) that the pressure exerted by a fluid is inversely proportional to its velocity. He used atomistic concepts to outline the first kinetic theory of gases, expressing their behavior in terms of probabilities under the particular conditions of pressure and temperature. He can be regarded as a founder of hydrodynamics.

\parpic[l][t]{%
  \begin{minipage}{40mm}
    \fbox{\includegraphics[width=110px,height=140px]{img/medaillons/bernoulli_jacques.eps}}
  \end{minipage}
}
\textbf{Bernoulli, Jacques} (1654-1705), was a Swiss mathematician and physicist, brother of Jean Bernoulli and Daniel Bernoulli and the uncle of Nicolas Bernoulli. Born in Basel in 1654, he met Robert Boyle and Robert Hooke on a trip to England in 1676. After that, he devoted himself to physics and mathematics. He teaches at the University of Basel from 1682, becoming professor of mathematics in 1687. He earned by his work and discoveries to be made a member of the Academy of Sciences in Paris (1699) and that in Berlin (1701). His correspondence with Gottfried Wilhelm Leibniz leads him to study infinitesimal calculus in collaboration with his brother Jean. He was among the first to understand and apply the integral and differential calculus, proposed by Leibniz, discovered the properties of numbers named "Bernoulli numbers" and gave the solution of problems considerated at his time as insoluble. He sets out the principles of probability theory and introduced the Bernoulli numbers in a book published after his death in 1713.

\parpic[l][t]{%
  \begin{minipage}{40mm}
    \fbox{\includegraphics[width=110px,height=140px]{img/medaillons/bernoulli_jean.eps}}
  \end{minipage}
}
\textbf{Bernoulli, Jean} (1667-1748), was a Swiss mathematician and physicist. Jacques Bernoulli's brother and father of Daniel and Nicolas Bernoulli. He taught mathematics at Groningen (1695), then in Basel, after the death of Jacques Bernoulli (1705), and became a member of the Academies of Sciences in Paris, London, Berlin and St. Petersburg. Trained by his brother Jacques Bernoulli, he had long worked with him to develop the implications of the new infinitesimal calculus invented by Gottfried Leibniz, but he then appears between them on the occasion of solving some problems, a rivalry which degenerated into enmity. He also contributed in many areas of mathematics including the problem of a particle moving in a gravitational field. He found the equation of the chain in 1690 and developed the exponential calculation in 1691. He also had the honour to train Leonhard Euler. He came to Paris in 1690 and became intimate with the most distinguished scholars, especially with Hospital. Jean Bernoulli became a member of the Royal Society in 1712.

\parpic[l][t]{%
  \begin{minipage}{40mm}
    \fbox{\includegraphics[width=110px,height=140px]{img/medaillons/bessel.eps}}
  \end{minipage}
}
\textbf{Bessel, Friedrich} (1784-1846) Born in Minden, Westphalia, Bessel began working very young as a clerk. Attracted by shipping, he became interested in nautical observations, constructing his own sextant and studying astronomy in his free time. He calculated the trajectory of Halley's comet, a result which was immediately released and allowed him to obtain, in 1806, a job as assistant at the Lilienthal Observatory. In 1810 he became director of the new observatory in Königsberg, while pursuing mathematical studies. He had to teach mathematics to his students in astronomy until 1825 (when Jacobi came to teach the subject in Königsberg). His whole life was devoted to astronomy (he wrote over 350 articles) and, shortly before his death, he began the study the motion of Uranus, the problem that led to the discovery of Neptune. In mathematics, Bessel is known for introducing the functions that have his name, used for the first time in 1817  for the study of a Kepler problem, and employing more fully seven years later to study planetary perturbations.

\parpic[l][t]{%
  \begin{minipage}{40mm}
    \fbox{\includegraphics[width=110px,height=140px]{img/medaillons/biot.eps}}
  \end{minipage}
}
\textbf{Biot, Jean-Baptiste} (1774-1862) born and died in Paris was a physicist, astronomer and mathematician. Jean-Baptiste followed secondary education (humanities) in Paris at the Collège Louis-le-Grand until 1791. He began studying engineering at the École des Ponts et Chaussées in January 1794, then joined the École Central des Travaux Publiques (later Polytechnic) when it opened in December 1794 at the Palais Bourbon. One year later (1795) he joined the École des Ponts et Chaussées to complete his training as an engineer. It is to teaching that Biot oriented his career after studying engineering. He became professor of mathematics at the École Central de l'Oise in Beauvais in 1797. With the support of Laplace he was appointed in 1800, aged 26, as professor of mathematical physics at the Collège de France. He is between 1816 and 1826 responsible at 50\% of the trainings of physical acoustics, magnetism and optics, Gay-Lussac, having the Chair of Physics, teaches heat, gas, humidity, the electricity and the galvanism. He formulated with Félix Savart, the Biot-Savart law, which gives the value of the magnetic field produced at a point in space by an electric current as a function of distance from this point to the conductor.

\parpic[l][t]{%
  \begin{minipage}{40mm}
    \fbox{\includegraphics[width=110px,height=140px]{img/medaillons/bohr.eps}}
  \end{minipage}
}
\textbf{Bohr, Niels Henrik David} (1885-1962) was a Danish physicist, Nobel Prize of Physics in 1922 for his contributions to nuclear physics and the understanding of atomic structure. The Bohr theory of atomic structure, for which he received the Nobel Prize, was published between 1913 and 1915. His work was inspired by the nuclear model of the Rutherford atom, in which the atom is considered as a compact nucleus surrounded by a cloud of electrons. The model suppose that the atom emits electromagnetic radiation when an electron moves from one quantum level to another. This model contributed enormously to future developments of theoretical atomic physics.

\parpic[l][t]{%
  \begin{minipage}{40mm}
    \fbox{\includegraphics[width=110px,height=140px]{img/medaillons/boltzmann.eps}}
  \end{minipage}
}
\textbf{Boltzmann, Ludwig} (1844-1906) was an Austrian physicist who helped to establish the foundations of statistical mechanics. Educated in Vienna and Oxford, he taught physics at several universities in Germany and Austria for over forty years. Developing the kinetic theory of gases, especially from the work of Maxwell, it establishes that the second law of thermodynamics could be obtained on the basis of statistical analysis. Calculating the number of particles with a given energy, he established the so-named "Maxwell-Boltzmann statistical". He expressed the entropy S of a system according to the probability W of his state (through his famous equation of transport from which he showed that entropy could only increase over time ... result which was previously recognized experimentally but not theoretically proved). It could also establish theoretically the "Stefan's law" concerning the radiation of a black body. But he had to explain how the mechanics principles, where the phenomena are reversible, could lead to thermodynamic laws describing phenomena characterized by irreversibility. He advanced the idea that irreversible changes, although they are only possibilities among others, are so likely that they are almost always occurring.

\parpic[l][t]{%
  \begin{minipage}{40mm}
    \fbox{\includegraphics[width=110px,height=140px]{img/medaillons/boole.eps}}
  \end{minipage}
}
\textbf{Boole, George} (1815-1864) was an English mathematician and logician and the creator of symbolic logic. Born in Lincoln, and son of a shopkeeper, he received his first lessons in mathematics from his father, who also taught him to manufacture optical instruments. Outside advice from his father and several years in local schools, Boole is a self-taught. When his father's business declined, he was obliged to work to help his family and, when sixteen, he taught in village schools, at twenty, he opened his own school in Lincoln. During his hobbies he studied mathematics at the Institute of Mechanics, created around this time, that's where he became acquainted with Newton's Principia, Laplace's celestial mechanics and analytical mechanics of Lagrange and he began to solve problems of higher algebra. Boole submitted to the new Cambridge Mathematical Journal a series of original articles, the first being \textit{Searches on the theory of analytical transformations}, these articles focused on differential equations and the invariant linear transformation. In 1844, he studied the links between algebra and infinitesimal calculus in an important paper published in the Transactions of the Royal Society, which awarded him a medal for his contribution to the analysis (that is, the use of algebra in the study of infinitely small and large entities). Developing new ideas about the method in logic and confident in the symbolism he had created from his mathematical research, he published in 1847, a booklet, \textit{Mathematical Analysis of Logic}, in which he argues that the logic must be attached mathematics, not philosophy. Even he had no university degree, Boole was, on the basis of its publications, in 1849 appointed professor at Queen's College in Cork, Ireland. With Boole, in 1847 and in 1854 began the algebra of logic, that is to say, what we name today the "Boolean Algebra". In his book of 1854, Boole states its completely new symbolic method of logical inference, which allows with proposals containing a number of terms, to obtain, by symbolic processing of the premisses, conclusions which were logically contained in the premises. He also search a general method in probability, which would, from the known probabilities of a given event, determine the probability of any other event logically connected to specific events.

\parpic[l][t]{%
  \begin{minipage}{40mm}
    \fbox{\includegraphics[width=110px,height=140px]{img/medaillons/borel.eps}}
  \end{minipage}
}
\textbf{Borel, Emile} (1871-1956) received major at X and ULM, he chooses the last one and dedicated his time to mathematics. He founded the Institute Henri Poincare and was elected mayor of the Aveyron and Saint-Africa. He studies the measures of sets and in particular, defines the sets of measure zero and all Borel sets on which we can define a measure. He then turns to probability and mathematical physics. Borel is also considered a constructivist mathematician. He is at the origin of strategic game theory and cybernetics that will develop later von Neumann and Morgenstern. His pupil Henri Lebesgue use his results in topology and measure theory for his theory of integration.

\parpic[l][t]{%
  \begin{minipage}{40mm}
    \fbox{\includegraphics[width=110px,height=140px]{img/medaillons/born.eps}}
  \end{minipage}
}
\textbf{Born, Max }(1882-1970) born in Breslau and died in Göttingen was a German and British physicist. Initially he followed his studies at the College of König-Wilhelm and continued at the University of Breslau followed by Heidelberg and Zürich Universities. While studying for his Ph.D. he came in contact with mathematicians such as Klein, Hilbert, Minkowski, Runge, Schwarzschild. In 1921, he was appointed professor of theoretical physics at Göttingen. He emigrated to Scotland in 1933 and became a British citizen in 1939. Outstanding theoretical physicist, he is known for his significant contribution to quantum physics: Development (1925) of quantum matrix mechanics introduced by Werner Heisenberg and, most importantly, he will be the first to give to the square of the modulus of the wave function the meaning of a density of probability of presence. He was also a pioneer in the quantum theory of solids (conditions of Born-von Karmann) and nonlinear electrodynamics of Born-Infeld. He has won half of the Nobel Prize for Physics in 1954 (the other half was given to Walther Bothe) for his fundamental research in quantum mechanics, especially for his statistical interpretation of the wave function. The Royal Society awarded him the Hughes Medal in 1950.

\parpic[l][t]{%
  \begin{minipage}{40mm}
    \fbox{\includegraphics[width=110px,height=140px]{img/medaillons/bose.eps}}
  \end{minipage}
}
\textbf{Bose, Satyendranath} (1894-1974) was an Indian mathematician and physicis, known for his contributions to quantum theory. Born in Calcutta, Bose was educated at Presidency College in Calcutta. In 1924, he offers a statistical description of quantum systems, echoed by Albert Einstein, which places no restrictions on the energy distribution of particles in the system. This description is known as the "Bose-Einstein statistics", as opposed to the "Fermi-Dirac statistics". Applied to the theory of black-body radiation, this new statistic leads to the "Planck distribution" and treats this radiation as a photon gas. In the field of elementary particle physics, the Bose-Einstein statistics requires the wave function of particles (in the Schrödinger equation) to be perfectly symmetrical for all the variables of space and spin. Particles obeying these statistics (photons, mesons $\pi$, etc.) are named "bosons". Professor of physics at the Universities of Calcutta and Dhaka, Satyendranath Bose was appointed in 1958 National Teacher of India.

\parpic[l][t]{%
  \begin{minipage}{40mm}
    \fbox{\includegraphics[width=110px,height=140px]{img/medaillons/broglie.eps}}
  \end{minipage}
}
\textbf{de Broglie, Louis Victor} (1892-1987) was a  French physicist and Nobel laureate who brought an essential contribution to the quantum theory with his studies of electromagnetic radiation. Born in Dieppe, Louis de Broglie was educated in Paris. He tried to understand the dual nature of matter and energy and suggested the association of a wave with any particle. He proposed also directly to explain how it was possible to obtain the quantization rules of Bohr and Sommerfeld atom's model requiring an integer number of waves in a stationary orbit. His discovery of the wave nature of electrons (1924) won him the Nobel Prize for Physics in 1929, however, he did not proposed a wave equation describing quantum phenomena (what Schrödinger will). He was elected to the Academy of Sciences in 1933 and at the Académie Française in 1943. He was appointed professor of theoretical physics at the Université de Paris (1928), Permanent Secretary of the Academy of Sciences (1942), and advisor to the Atomic Energy Commission (1945).

\parpic[l][t]{%
  \begin{minipage}{40mm}
    \fbox{\includegraphics[width=110px,height=140px]{img/medaillons/brouwer.eps}}
  \end{minipage}
}
\textbf{Brouwer, Luitzen Egbertus Jan} (1881-1966) was a great Dutch mathematician of the early 20th century. Born from a father who was teacher, he performed very fast at high school. At the University of Amsterdam, he was trained by Korteweg, who is known for contributions in Applied Mathematics. He presented his Ph.D. in 1904. From 1909 to 1913, Brouwer is interested in topology, and discovered most of the theorems to which his name has remained attached, including his famous fixed point theorem. For many, Brouwer is the father of modern topology. In 1912 he obtained, through Hilbert referrals, a professorship at the University of Amsterdam. He teaches the theory of sets, of functions, and axiomatic. Later, he refused to join Hilbert in Göttingen. During the first World War his health embrittlement and he left some time the fields of scientific research. When he returned, it was to devote himself to his first love (his thesis was already on this subject): the foundations of mathematics. Brouwer is with Poincaré the spearheading of intuitionist mathematics, as opposed to the logicism of Frege and Russell, and Hilbert formalism. In particular, for Brouwer, an existence theorem can be true only if you can show a process, even formal, of construction. This led in particular to reject the law of excluded middle, which says that a property is either true or false! The proofs thus obtained are often longer, but Brouwer was able to rewrite treaties of set theory, theory of measurement and theory of functions in accordance with the rules of intuitionism. Oddly, Brouwer never taught topology. This is probably because the theorems that he had proven itself did not fit anymore in it axiomatic set. According to testimonies of some of his students, he was a really strange character, madly in love with his philosophy, and a teacher with which it was not recommended to ask questions!

\phantomsection
\addcontentsline{toc}{section}{C}	

\parpic[l][t]{%
  \begin{minipage}{40mm}
    \fbox{\includegraphics[width=110px,height=140px]{img/medaillons/cantor.eps}}
  \end{minipage}
}
\textbf{Cantor, Georg }(1845-1918) was a brilliant student, particularly in manual operations. Despite the injunctions of his father, who dreams of making him an engineer, he moved to Berlin in 1862 to study mathematics where his teachers are Weierstrass and Kronecker. He presented his Ph.D. in 1867 (on number theory). The first post-doctoral research of Cantor are devoted to the decomposition of functions into sums of trigonometric series (the famous Fourier series) and especially to the uniqueness of this decomposition. To fully resolve this difficult problem, it was necessary to introduce and study sets named "exceptional sets". This led in 1872 to define precisely what a real number was as limit of a sequence of rational numbers, at the same time his friend Dedekind gives another definition of the straight of the real number using cuts. Cantor and Dedekind note on this occasion there's a lot more real numbers than rational, but there has not robust mathematical definition of this "much more". In 1874, in the prestigious Journal of Crelle, Cantor defines the number of elements of an infinite set which extends naturally that of the cardinal of an infinite set, which extends that of the cardinal of a finite set. It follows, until 1897, a succession of strange discoveries: there are as many even integers than any integers, as many points on a segment than in a square, many more transcendental numbers than rational numbers. This hierarchy of infinite sets gradually led Cantor to establish new numbers, transfinite ordinals, and define an arithmetic on these numbers. Cantor's works had a lot of influence in the 20th century. We have to mention, in 1903, a paradox raised by Russell in the naive set theory: if $A$ is the set of all sets that are not elements of themselves, is $A$ contained in $A$? Logicians will overcome this conceptual difficulty, without changing the conclusions of Cantor. We can refer to the problem of the continuum hypothesis. One of the last lines of research Cantor was to estimate the number of elements of the real line. Specifically, Cantor wanted to prove the absence of any set whose cardinality is strictly between the cardinal integers and the real numbers. This is what we name the "continuum hypothesis". All Cantor's work and this of his successors to prove or disprove the continuum hypothesis were unsuccessful, and for good reason: in 1963, the logician Cohen proved that in a standard theory of sets, the continuum hypothesis is undecidable. We can easily assume it is true or false without obtaining any conflict in the theory.

\parpic[l][t]{%
  \begin{minipage}{40mm}
    \fbox{\includegraphics[width=110px,height=140px]{img/medaillons/carnot.eps}}
  \end{minipage}
}
\textbf{Carnot, Nicolas Léonard Sadi} (1796-1832), physicist and French military engineer, considered as the creator of thermodynamics. Eldest son of Lazare Carnot, nicknamed "the Grand Carnot", Sadi was educated at the École Polytechnique. In 1824, he described his conception of the ideal heat engine, named "Carnot engine", in which all available energy is used. He discovered that heat could pass from a cold body to a hotter body, and the engine performance depended on the amount of heat ha was able to use. This discovery, or Carnot cycle, is the basis of the second law of thermodynamics.\\

\parpic[l][t]{%
  \begin{minipage}{40mm}
    \fbox{\includegraphics[width=110px,height=140px]{img/medaillons/cartan.eps}}
  \end{minipage}
}
\textbf{Cartan, Élie} (1869-1951) received his primary education at the School of Dolomieu, Vienna and then in the Lycée et Collègede Grenoble. He attended Jeanson-de-Sailly high school for the preparation at the École Normale Supérieure, where he entered in 1888. There he followed the teachings of H. Poincare, É. Picard and C. Hermite. The first work of Élie Cartan which would lead to his thesis in 1894, focuses on complex simple Lie groups, where he resumed, corrected and developed the results of structure and classification obtained by W. Killing. Cartan obtained a lectureship at the Université de Montpellier from 1894 to 1896, then at the Faculté de Lyon from 1896 to 1903. The same year he is appointed professor at the Faculté de Nancy, where he remained until 1909. He gives at the same time courses at the École d'Ingénierie Électrique et de Mécanique Appliquées. He wrote two major articles on a generalization in infinite dimensions of simple Lie groups. He develops methods that were to influence the further development of differential geometry. In 1909, he left Nancy to teach at La Sorbonne, where he is appointed professor in 1912. He also provides instruction in the École de Physique et Chimie de Paris. In 1914, he solves the problem of classification of real simple Lie groups, and determines the finite dimensional representations of these groups. During the war, he served as a sergeant in the hospital located in the premises of the École Normale Superieure, while continuing his mathematical work. His subsequent mathematical work is considerable, with nearly 200 publications and several books. Topics covered include the study of Pfaffian systems, the deformation theory, the study of constant negative curvature varieties, the gravitational theory of Einstein's, the theory of affine connections, holonomy groups, the Riemannian symmetric spaces, spinors. He is also the author of several articles on the history of geometry. He retired in 1940.


\parpic[l][t]{%
  \begin{minipage}{40mm}
    \fbox{\includegraphics[width=110px,height=140px]{img/medaillons/cauchy.eps}}
  \end{minipage}
}
\textbf{Cauchy, Augustin-Louis} (1789-1857). It was at Cherbourg that Cauchy started his math researches on polyhedra, and it's first results are promising. But, tired by the cumulative charge of engineer and long evenings of research, Cauchy had a depression that pushes him to return to live with his parents. In Paris, he sought a position in line with its commitment to pure mathematical research. In 1815, he completed a brilliant memory where it shows a famous Fermat's theorem on polygonal numbers. This will do much for his reputation, and in 1816, he became a member of the Academie des Sciences, replacing Carnot and Monge. The course of analysis that Cauchy professes at the École Polytechnique is decried by both his students as colleagues from other matters. However, this course is published in 1821 and 1823, which was to become the reference for the analysis in the 19th century. highlighting the rigour, not just intuition. This is the first time that real definitions of limits, continuity, convergence of sequences, series, are used. This rigour, however, still remains relative, since Cauchy "proves" that the limit of a series of continuous functions is continuous, which is not true. It is true that Cauchy did not yet have a clear definition of real numbers. This is also the time where Cauchy deeply develops the analysis of functions of a complex variable (i.e. establishing the expression of residues), as well as advances in the theory of finite groups. Cauchy was never the leader of a school of mathematicians, and he behaved sometimes awkwardly with young researchers as Abel or Galois, he underestimates, or even lost, memories of the first importance. His relations with his colleagues are generally not very easy.

\parpic[l][t]{%
  \begin{minipage}{40mm}
    \fbox{\includegraphics[width=110px,height=140px]{img/medaillons/cayley.eps}}
  \end{minipage}
}
\textbf{Cayley, Arthur }(1821-1895), born in Richmond (Surrey), showed very early strong predispositions for mathematics. However, despite the great interest of his first publications, he couldn't emerge as a mathematician, he decided to study law and became a lawyer in 1849. During fourteen years he held this job doing at the same time scientific researches. In 1863, Cayley was appointed professor at Cambridge and was finally able to devote himself entirely to mathematics. Throughout the work of Cayley, especially in his early works, there is a sensitive influence of the founders of the English school algebra who formulated the program of modern algebra by giving priority to the formal approach of problems. Educated mathematician and creator, Cayley, in the tradition of the English school, was able to develop new and fruitful theories. The richness of Cayley approach appears from his early work on group theory (1854). Cayley, addressing the work of Galois, Gauss and Cauchy with the methods of English algebraists, provides a definition of abstract groups which led to the notion of isomorphism. The study of linear equations systems led to Cayley to the determinants. In his early work, he established many rules for calculating the determinants, including the relation of multiplication of determinants that was already in the works of Cauchy, Jacobi and Binet. Next to original studies on the determinants, we meet the concept of rectangular array representing the coefficients of a system of linear equations or coefficients of a linear transformation. Cayley studied the rectangular matrices with real coefficients and complex, he also introduced the matrix operations and describes their properties, including the non-commutative multiplication. This is probably the first appearance of linear algebra. A few years later, Cayley also study non-associative systems and publish the results of multilinear algebra. Cayley has spent many of his publications on the problems of geometry and the study of algebraic curves and surfaces. At twenty-two years, he expressed the idea of the geometry of n dimensions, idea that was also made, almost simultaneously, but in a slightly different form, by Grassman. Cayley did not return until much later (in 1870) on the n-dimensional space, but its algebraic method contributed to important discoveries that took place in other areas of geometry. Thus, in the Sixth Memoir on Quantics of 1859 he introduced the projective metric, thereby subordinating the metric geometry to projective geometry; he demonstrated that the basics of the metric geometry (angles and distances) are the invariants and covariants of certain linear transformations of the absolute quadric.

\parpic[l][t]{%
  \begin{minipage}{40mm}
    \fbox{\includegraphics[width=110px,height=140px]{img/medaillons/chandrasekhar.eps}}
  \end{minipage}
}
\textbf{Chandrasekhar, Subrahmanyan} (1910-1995) obtained at the age of 23 his Pd.D. at Trinity College (Cambridge University). Specialist in astrophysics Chandrasekhar made a decisive advance knowledge of the hydrodynamic evolution and hydromagnetic energy transfer by radiation without forgetting the relativistic and quantum effects in the evolution of stars. His major contribution in this area is the transformation of white dwarf stars and beyond with a mass greater than the Chandrasekhar limit (1.44 that of the sun), the collapse in a neutrons star. Objects more massive giving black holes.\\\\

\parpic[l][t]{%
  \begin{minipage}{40mm}
    \fbox{\includegraphics[width=110px,height=140px]{img/medaillons/clairaut.eps}}
  \end{minipage}
}
\textbf{Clairaut, Alexis-Claude} (1713-1765) was a member of the Académie Française des Sciences and was one of the most knew mathematicians and physicists of the 18th century. At age 10, he knew infinitesimal calculus, at 12, he submitted his first study at the Academy of Sciences and at 18, he published a book containing important extensions to the geometry that have permitted him the admission to the academy in 1731. Clairaut was one of the scientists who accompanied Maupertuis in Lapland to acquire the necessary data for determining the shape of the earth. In 1743 he published his Theory of the figure of the Earth, where he calculated more accurately than Newton had done, the shape adopted by a rotating body due to the natural gravitation of its parts. In 1760 he published his \textit{Théorie du mouvements des comètes}, which accurately predicted the date of Halley's comet will arrive at the Sun's nearest point.

\parpic[l][t]{%
  \begin{minipage}{40mm}
    \fbox{\includegraphics[width=110px,height=140px]{img/medaillons/cohen.eps}}
  \end{minipage}
}
\textbf{Cohen, Paul Joseph}(1934-2007) was an mathematician and logician born in the New Jersey and died in Stanford. In 1963, Cohen has discovered a new model building, named "forcing", which now plays a key role in set theory and model theory. He also built models of set theory (assumed consistent) in which the axiom of choice and the continuum hypothesis are not verified, which, given the earlier work of Kurt Gödel, establishes that axiom of choice and the continuum hypothesis are independent of the usual systems of set theory. This work has earned Cohen, in 1966, the Fields Medal of the International Mathematical Union. He is also the author of interesting works in classical analysis.

\parpic[l][t]{%
  \begin{minipage}{40mm}
    \fbox{\includegraphics[width=110px,height=140px]{img/medaillons/connes.eps}}
  \end{minipage}
}
\textbf{Connes, Alain}(1947 -) was born in 1947 in Draguignan (France). Old student of the École Normale Superieure, he received in 1980, the Prix Ampère, one of the most important award of the Académie des Sciences. He was elected member of the academy in 1981 (he was the youngest member). The first work of Alain Connes enrol directly in the tradition of John von Neumann and his immediate followers. The development of quantum physics to the years 1920 had made the agenda of the study areas in three dimensions rather than as one where we believe we live, or four, as in Einsteinian relativity, but to an infinite size (Hilbert spaces). One of the essential tools of quantum physics is the notion of operator in such a space, generalizing the notion of rotation of a Euclidean space. The theory of operator algebras started around 1930 by the work of von Neumann, who showed the importance of a certain type of operator algebras, now named "von Neumann algebras", and that established for these algebras a theorem of prime factorization quite similar to the well known decomposition theorem for ordinary integers. Since the origin, the factors were classified into three types: type I factors, II and III. We had an early understanding of type I factors and a lot of information on those of type II, but the factors of type III remained for a long time much more mysterious. Even the examples were few and von Neumann said about this: «This is the most refractory of all, and tools for its study are lacking, at least for now». The first success of Connes, which immediately gave him international fame, was a major breakthrough towards the elucidation of the structure of type III factors and can be said to be the first to have acquired a practical knowledge of these objects, until now rather enigmatic, as a whole. Very roughly, the results of Connes bring the study of factors of type III to that of type II and their automorphisms. The work of Alain Connes is that of a complete mathematician, capable of solving difficult problems bequeathed by the past, but also to completely change a discipline by introducing new original ideas.

\parpic[l][t]{%
  \begin{minipage}{40mm}
    \fbox{\includegraphics[width=110px,height=140px]{img/medaillons/copernic.eps}}
  \end{minipage}
}
\textbf{Copernic, Nicolas} (1473-1543), studying at Krakow University until 1491, he then went to Italy to take courses in canon law at Universita de Bologna. He also follows the astronomy courses of Domenico Maria Novara, one of the first scientists to question the teachings of Ptolemy. In 1500, he taught mathematics in Rome, before returning for one year in Frauenburg where his uncle take him as canon in 1497. Having obtained permission to continue his studies in Italy, he enrolled in the faculties of law and medicine in Padua and received his Ph.D. in canon law in Ferrara in 1503. Finally, he returned to Frauenburg where he built an observatory and began his famous research in astronomy. He remained there until his death. Cosmology of this time is then based on the geocentric system of Ptolemy. The Earth is motionless at the center of several concentric spheres carrying the Moon, Mercury, Venus, Sun, Mars, Jupiter, Saturn and finally the stars. But this system does not satisfy Copernic, he found it to complicated and flawed. He then consults the authors of antiquity (Cicero, Aristarchus of Samos, etc.) and finds that some of them already consider the rotation of planets, including Earth around the Sun (considered as fixed). Copernic then shows that the combination of movements of the Earth and planets perfectly explains the apparent motion of the planets (in forward and backward). In addition, he establishes that their apparent diameter changes arise as a consequence of their revolution around the Sun. His researches will continue for thirty-six years and show that the Moon is a satellite of the Earth and the Earth's axis is not fixed. His masterpiece \textit{De Revolutionibus orbium coelestium} was published in 1543 in Nuremberg and Copernicus received the first copies only a few hours before his death. In the dedication he made to Pope Paul III, he presents his system as a pure hypothesis, thus avoiding the condemnation of the Church. Adopted a century after his death, after being violently rejected, the Copernican system brought a profound revolution in the conception of the world and more generally in scientific thought.

\parpic[l][t]{%
  \begin{minipage}{40mm}
    \fbox{\includegraphics[width=110px,height=140px]{img/medaillons/coriolis.eps}}
  \end{minipage}
}
\textbf{Coriolis, Gaspard} (1792-1843) was a French mathematician and engineer who brought to light the centrifugal composed forces, named "Coriolis forces". The engineer of Roads and Bridges is the author of important works in mechanics. In 1835, he demonstrated that the acceleration of a moving object in a rotating frame is subjected to an additional (Coriolis force) perpendicular to the direction of movement of the mobile in this referential. Even if the force produced by the rotation of the planet has a low intensity on the surface of the Earth, it influences the direction of ocean and atmosphere currents. It produces a deflection to the east and explains, for example, the circular movement of hurricanes.

\parpic[l][t]{%
  \begin{minipage}{40mm}
    \fbox{\includegraphics[width=110px,height=140px]{img/medaillons/coulomb.eps}}
  \end{minipage}
}
\textbf{Coulomb, Charles Augustin} (1736-1806) was  French physicist pioneer in electrical theory. Born in Angoulême, he served as military engineer for France in the West Indies, but retired to Blois during the French revolution, to continue his research on magnetism, friction, and electricity. In 1777 he invented the torsion balance to measure the strength of the electric and magnetic attraction. With this invention, Coulomb was able to formulate the principle, now known as "Coulomb's law", which governs the interaction between electric charges. In 1779 Coulomb published the treaty Theory of simple machines, an analysis of friction in machines. After the revolution, Coulomb left his retirement and assisted the new government to design a metric system of weights and measures. The unit used to express the amount of electrical charge, the "Coulomb", named after the physicist.

\parpic[l][t]{%
  \begin{minipage}{40mm}
    \fbox{\includegraphics[width=110px,height=140px]{img/medaillons/cournot.eps}}
  \end{minipage}
}
\textbf{Cournot, Antoine Augustin} (1801-1877) studied at the Gray College from 1809 to 1816. He won prizes for it's excellence in mathematics. In 1820 he joined the Collège Royal de Besançon and won the prize of honour in mathematics. With two memories and two translations of various treatises in mathematics, he draws the attention to Poisson, who appointed him in 1834 professor of analysis and mechanics at the Faculté de Lyon. Augustin Cournot is a scientific, that is to say, a man of extensive knowledge in all fields of science, but a scientific philosopher, who by his modesty, has not known celebrity. Cournot was first a teacher and great popularizer. Three mathematics books distinguish Carnot: \textit{Elementary Treatise of the theory of functions and infinitesimal calculus} (1841); \textit{Presentations of the theory of chance and probabilities} (1843), \textit{On the origin and limits of the correspondence between algebra and geometry} (1847). But Cournot's genius lies in the introduction of probability in economics. He is the precursor of modern theories in economics, that inspired Léon Walras who in his autobiography completed in 1904, and in several letters, reminded the role played in the development of his thought, on the one hand, the work of Antoine Augustin Cournot, and on the other hand, that of his father, the economist and philosopher Auguste Walras who was a classmate of Augustin Cournot at the École Normale.

\parpic[l][t]{%
  \begin{minipage}{40mm}
    \fbox{\includegraphics[width=110px,height=140px]{img/medaillons/clausius.eps}}
  \end{minipage}
}
\textbf{Clausius, Rudolf} (1822-1888) is one of the greatest physicists of the 19th century. He is known primarily for his contribution to the study of thermodynamics. The first, this German scientist formulated what is commonly named the "second principle of thermodynamics" and proposed a clear definition of the entropy. He is also one of the main creators of the kinetic theory of gases. Born in Köslin (Pomerania), Clausius attended the universities of Berlin and after Halle which he graduated in 1848. Professor until his death, he was responsible of the Chair of Physics at the Royal School of Artillery and Engineering in Berlin (1850-1855) and, simultaneously, at the University and Polytechnic of Zurich (1855-1867), then at the University of Würzburg (1867-1869), and finally at Bonn from 1869 to his death. His first publication in 1850 in the Annalen der Physik in Poggendorff, attracted a widespread attention. He was searching to reconcile the idea of equivalence between work and heat. Clausius pointed out that the assumption of the conservation of heat in the process of transfer was not an essential part of the theory of Carnot. He establishes that in an ideal machine, the amount of heat taken to the boiler must always be greater than that which is transferred to the condenser, and an amount exactly equal to the work done. This important synthesis performed, Clausius, in the same publication, enunciated what we now name the "second law of thermodynamics". It was the necessary need, already established by Carnot, of the presence, not just of a warm body (the boiler), but also of a cold body (the condenser) for a steam to provide a work. In 1854, Clausius, pushing further the views expressed in 1850, offered the first clear statement of the concept of entropy. He was looking to measure the ability of the heat energy of any system to provide real non-ideal work. In the case of the heat conduction along a solid rod, for example, heat flows from the hot end to the cold end without providing any work, although this transfer is accompanied by a decrease in the ability of the hot end to serve subsequently as a potential source of work. This decrease occurs because at the end of the process the heat energy is held by a body located at a lower temperature than the initial state. It has not been lost, but only deteriorated because, according to the second law of thermodynamics, we can't find the initial temperature without the help of external work. The last major contributions of Clausius are from 1857 and 1858 and related to the kinetic theory of gases. Although he is not the first to have developed it, already proposed and discussed by Joule and Krönig in particular, it ranks with Maxwell among its founders. He introduced the concept of mean free path and establishes the important distinction between the translational energy and internal energy of a gas particle. In addition, we generally recognized him the merit of having, by his theoretical work, make a bridge between atomic theory and thermodynamics.

\parpic[l][t]{%
  \begin{minipage}{40mm}
    \fbox{\includegraphics[width=110px,height=140px]{img/medaillons/curiepierre.eps}}
  \end{minipage}
}
\textbf{Curie, Pierre} (1859-1906) is considered as one of the pioneers of studies on chemistry/physics radioactivity. He is even in its thesis published in 1898 that the term "radioactivity" was used for the first time by his wife Marie and him. The education of Pierre began at a very young age by his father, who was a military General Surgeon. The Curies had the habit of visiting the countryside near Paris on Sunday, Pierre, during his walks, quickly learned all the names of plants and animals. Since the school was not obliged at this time (not before 1881 when Jules Ferry edicted a law for this), Pierre was educated at home with his mother, then with his brother and after with tutors and finally, alone. At the age of 14, the education of Pierre was delegated to Mr. Bazille, who taught him elementary and special mathematics, this greatly developed the mental abilities of Pierre, who had a clear interest in mathematics. At the age of sixteen, he was received Bachelor of Science. In 1877, he obtained a degree in physics from the École des Pharmacies... In subsequent years, he studied crystals and magnetism, which will eventually lead to the discovery of piezoelectricity. In 1877, he took a position as an assistant. Afterwards he became demonstrator of physics experiments for laboratories until 1882 when he became director of all practical work in the schools of physics and industrial chemistry. Pierre married his wife, Marie Sklodowska in 1895 and they had two children together, Irene and Eve. Pierre Curie won in 1903 with his wife, the Nobel Prize in Physics for their work on radioactive substances and their discovery of two new elements: radium and polonium.

\parpic[l][t]{%
  \begin{minipage}{40mm}
    \fbox{\includegraphics[width=110px,height=140px]{img/medaillons/curiemarie.eps}}
  \end{minipage}
}
\textbf{Curie, Marie} (1867-1934) was a chemist and physicist born in Warsaw and died in Haute-Savoie. Daughter of a father professor in mathematics and physics and of a mother who was teacher, she is the youngest of a family of 4 sisters. Between 1876 and 1878 she lost a sister and his mother. She took refuge in studies where she excelled in all subjects, and where the maximum score was granted to her. She thus obtains his graduation from high school with a gold medal in 1883. She wants to pursue higher education and teach, but these studies are forbidden to women. When her older sister, Bronia, left to study medicine in Paris, Marie agrees as a governess in province hoping to save enough money to join her sister while having originally intended to return in Poland to teach. After three years, she returned to Warsaw, where his cousin helped her to enter in a laboratory. In 1891, she moved to Paris, where she was hosted by her sister and brother. The same year, she enrolled to study physics at the Faculté de Paris. Three years later, she graduated in physics, being first in her class. During the summer, a scholarship is granted to Marie, which allows her to continue his studies in Paris. A year later, she graduated in mathematics, being second of her class. Then she hesitates to return to Poland. At a party she met Pierre Curie (her future husband), who is head of physics works at the Écoloe Municipale de Chimie et Physique Industrielle and also studied magnetism, with which she will work. Mary receives (with her husband Pierre Curie) one half of the Nobel Prize for Physics in 1903 (the other half is given to Henri Becquerel) for research on radiation. In 1911, she won the Nobel Prize in Chemistry for his work on polonium and radium.

\phantomsection
\addcontentsline{toc}{section}{D}	

\parpic[l][t]{%
  \begin{minipage}{40mm}
    \fbox{\includegraphics[width=110px,height=140px]{img/medaillons/dalton.eps}}
  \end{minipage}
}
\textbf{Dalton, John} (1766-1844) was a British chemist and physicist who developed the atomic theory upon which was founded modern physical science. Dalton began in 1787 a series of meteorological observations that he continued for fifty-seven years, accumulating some two hundred thousand observations and measurements of time in the Manchester area. Dalton's interest in meteorology led him to study different phenomena and instruments used to measure them. He was the first to prove the validity of the idea that rain is precipitated by a drop in temperature, not by a change in atmospheric pressure. Dalton arrived at his atomic theory by studying the physical properties of atmospheric air and other gases. During his research, he discovered the law of partial pressures of gases mixed, often known as the "Dalton's law" that the total pressure exerted by a mixture of gases is equal to the sum of the individual pressures which would be exerted if each gas alone occupied the whole volume.

\parpic[l][t]{%
  \begin{minipage}{40mm}
    \fbox{\includegraphics[width=110px,height=140px]{img/medaillons/davinci.eps}}
  \end{minipage}
}
\textbf{Da Vinci, Leonardo }(1452-1519) was an Italian painter, sculptor, architect and man of science. Man of universal mind, both artist, scientist, inventor and philosopher Leonardo embodied the universal spirit of the Renaissance and remains one of the great men of that time. At the age of five, his father having noticed his gift for drawing, send it as an apprentice in the workshop of Verrocchio in Florence. He enters at the age of twenty years in the Guild of Painters, and began his career as a painter with famous works such as \textit{La vierge à l'oeillet}, or the \textit{L'Annonciation} (1473). He improves the sfumato technique (printing mist) to a point of refinement never achieved before him. In 1481, the monastery of San Donato order the \textit{Adoration des Mages}, but Leonard annoyed to being selected for the decoration of the Sixtine Chapel in Rome, would never finish this painting and left Florence for Milan. After the completion of \textit{La Vierge aux rochers} for the chapel of San Francesco Grande, and that of the equestrian statue of Francesco Sforza, he finds fame throughout Italy. In 1495, the Dominicans of Santa Maria delle Grazie order him \textit{La Cène}. In 1498, he realized the ceiling of the Sforza palace. During this period he realized the \textit{Mona Lisa} and \textit{La Bataille d'Anghiari}. Leonardo also carries a large amount of studies on zoology, botany, anatomy and geology. He imagines multiple devices and machines, the first flying machine, which will remain at the stage of design. More than itself as a scientist, Leonardo has impressed his contemporaries and subsequent generations by his methodical approach to knowledge, learning skills, observation knowledge, analyse knowledge. The approach he exhibited in all the activities he was interested for, both technical and art (both are also not distinguished in his mind), stemmed from a prior accumulation of detailed observations, knowledge scattered here and there, which tended towards surpassing what was already there, with perfect aim. Many drafts, notes and treated by Leonardo da Vinci are not, strictly speaking, original discoveries, but are the result of research carried out in a encyclopedic purpose. In 1516, he joined the court of Francis I., where he participated in planning urbanistic projects. Form Leonardo da Vinci, remains today 7,000 notes and drawings, forty certified works (eight have disappeared).

\parpic[l][t]{%
  \begin{minipage}{40mm}
    \fbox{\includegraphics[width=110px,height=140px]{img/medaillons/dantzig.eps}}
  \end{minipage}
}
\textbf{Dantzig, George Bernard} (1914-2005) was a mathematician born in Portland and died in Stanford, inventor of the famous "Simplex algorithm" in linear optimization. His father, Tobias, is a Russian mathematician who had studied with Henri Poincaré in Paris. He married a colleague from La Sorbonne, Anja Ourisson, and the couple emigrated to the United States. He is the main actor of a famous story in mathematics. In one of his Ph.D. course at the University of Berkeley, Professor Jerzy Neyman proposed two open problems in statistics. An open problem is a problem that although it was formulated, has not yet been resolved. Such problems are a significant challenge and require research that can extend over several years. Dantzig was late and thought it was homework. Without taking several years but a few days, he solved the problems. He received his Ph.D. from Berkeley in 1946. Six years later, he was hired to do mathematical research at the RAND Corporation, where he implements the simplex algorithm in computers. In 1960, UC Berkeley hired him to teach computer science, and eventually to became the head of the operational research center. Six years later, he held a similar position at Stanford University, a position he held until his retirement in the 1990s. In addition to his work on the simplex algorithm and linear optimization, he also worked on methods for decomposing large problems, sensitivity analysis, methods of resolution matrix with pivot, the nonlinear optimization and linear stochastic optimization.

\parpic[l][t]{%
  \begin{minipage}{40mm}
    \fbox{\includegraphics[width=110px,height=140px]{img/medaillons/debye.eps}}
  \end{minipage}
}
\textbf{Debye, Peter Joseph Wilhelm }(1884-1966) was a  physicist and chemist born in Maastricht and died in New-York. Debye subscribe in 1901 at the Universität von Aix-la-Chapelle in Germany. He studied there mathematics and classical physics and holds in 1905 a degree in electrical engineering. In 1907 he produced his first scientific publication, an elegant mathematical solution of a problem involving Foucault currents. He studied at Aix-la-Chapelle under the direction of Arnold Sommerfeld. In 1906, he accompanied Sommerfeld in Munich as an assistant. He obtained his Ph.D. in 1908 with a dissertation on radiation pressure. In 1910, he proved Planck's law with a method that Max Planck admitted that she was simpler than his. In 1911 Debye was appointed professor at Zurich in Switzerland. He then went to Utrecht in 1912 (Germany), in Göttingen in 1913, he returned to Zurich in 1920, went to Leipzig (Germany) in 1927 and in Berlin in 1934 where he became director of the Kaiser Wilhelm Society that will in 1938 take the name of Max Planck Society. In 1912, he extended Albert Einstein theory of specific heat at low temperatures including contributions from low-frequency phonons know today as the "Debye model". In 1913, he extended the Niels Bohr theory of atomic structure by introducing elliptical orbits, a concept also proposed by Arnold Sommerfeld. Debye benefits in 1938 of a proposal for a conference at Cornell University in Ithaca to go to the United States and then stay at Cornell University, where he became professor and then, for 10 years, director of the Department of Chemistry. He remains to Cornell the rest of his career. He retired in 1952 but continued his research until his death.

\parpic[l][t]{%
  \begin{minipage}{40mm}
    \fbox{\includegraphics[width=110px,height=140px]{img/medaillons/descartes.eps}}
  \end{minipage}
}
\textbf{Descartes, René} (1596-1650) was French philosopher,  scientist and mathematician, founder of modern rationalism. Born in La Hague, of a father consultant at the Parliament of Rennes, Descartes received from 1607 to 1614 a teaching decisive for him from the Jesuits of the Collège Royal de La Flèche. This experience led him to propose an overhaul of Sciences, criticizing the lack of foundation of professed education. He was trained as a lawyer in 1616 and then started a military career in 1618, undertook trips, mixed scientific life and meet high society people, before devoting himself fully to philosophy. He spent his life between France and the Netherlands, fleeing the cities, frequenting libraries and meeting the most illustrious minds of his time, including Bérulle, Fermat, Gassendi, Hobbes, and Pascal. He died of pneumonia in Stockholm, bequeathing to posterity a work surrounded by legends and imbued with a new spirit.

\parpic[l][t]{%
  \begin{minipage}{40mm}
    \fbox{\includegraphics[width=110px,height=140px]{img/medaillons/dirac.eps}}
  \end{minipage}
}
\textbf{Dirac, Paul Adrien Maurice}(1902-1984) was born in Bristol and studied at the Universities of Bristol and Cambridge. In 1926, for his Ph.D. (the first thesis in the world having for subject "quantum mechanics"), he introduced a general formalism for quantum physics shortly and independently after Heisenberg (he finds the non-commutativity of position and momentum operators). In 1928, he developed a relativistic theory to describe the properties of the electron. This theory led to the postulate of a particle identical to the electron in all its aspects but of opposite charge, that is to say positive and that have to annihilate with a negative electron in a collision. Dirac's theory was confirmed in 1932 when the physicist Carl Anderson discovered the positron. Dirac also helps with Fermi, the development of the Fermi-Dirac statistics, describing the collective behavior of particles of half-integer spin. In 1933 Dirac shared the Nobel Prize in Physics with the Austrian physicist Erwin Schrödinger. In 1939, he became a member of the Royal Society. He was professor of mathematics at Cambridge from 1932 to 1968, professor of Physics at the State University of Florida from 1971 until his death, and a member of the Institute of Advanced Studies regularly between 1934 and 1959.

\parpic[l][t]{%
  \begin{minipage}{40mm}
    \fbox{\includegraphics[width=110px,height=140px]{img/medaillons/dirichlet.eps}}
  \end{minipage}
}
\textbf{Dirichlet (-Lejeune), Peter Gustav} (1805-1859) was born in Düren (Germany). Dirichlet was a brilliant student who completed his secondary education at age of 16. Because of the low quality of universities in Germany at this time, Dirichlet decided to go study in Paris, taking with him the \textit{Disquisitiones Arithmeticae} of Gauss as a bible. In the French capital, his personal situation is facilitated by the General Foy, an old important general of the Napoleonic battles who show him kindness and for whom he became the tutor of his children. Dirichlet then met some of the greatest mathematicians, including Legendre, Poisson, Laplace and Fourier. This last especially impress many Dirichlet, and will cause his interest for trigonometric series and mathematical physics. It was in Paris that Dirichlet wrote his first significant contribution to mathematics, in 1825 he is at the initiative of the proof of the case n = 5 of Fermat's last theorem, proof completed by Legendre a little bit later. End of 1825, General Foy died and Dirichlet decides to return to Germany. He first taught at the University of Breslau, at the military school in Berlin and at the University of Berlin in 1829, where he remained for 27 years. Among his students, we note the names of Kronecker and Riemann. Dirichlet is described as a good teacher, but not a perfect one. He gives the appearance of someone dirty, always wearing a cigar and a beer, apparently not really concerned about the image he gives. It was also told that he was often late. In 1848, his master and friend Karl Jacobi is diagnosed as being ill with diabetes. Dirichlet accompany Jacobi on a journey of 18 months in Italy. Back in Germany, Dirichlet begins to be tired of the heavy teaching tasks that he must assume. At Gauss's death, he succeeded him in Göttingen. This is unfortunately not for long, because he also died in 1859 or a heart attack. The scope of work of Dirichlet illustrates the depth of the German Mathematical early at his Golden Age. He is also the first to define a sufficient condition for convergence of a Fourier series (in the case of piecewise continuous functions), the theorem of arithmetic progression, the extension of harmonic functions defined on the boundary of an open and a whole class of partial differential equations is named "Dirichlet problem". We also owe him many contributions in arithmetic that bear his name like the theorem of Dirichlet units, Dirichlet series, etc.

\parpic[l][t]{%
  \begin{minipage}{40mm}
    \fbox{\includegraphics[width=110px,height=140px]{img/medaillons/doppler.eps}}
  \end{minipage}
}
\textbf{Doppler, Christian} (1803-1853) was an Austrian mathematician and physicist famous for his discovery of the Doppler effect. After studying at the Wien Universität, Doppler became assistant professor in this institution in 1829. This job position being not renewed, he has in mind to emigrate to the United-States. He renounces to leave his country after being named in Prague in 1837 and in 1849 at the Wien Polyteknische Schule. In 1850, he founded the institute of physics of the Wien Universität which he is the only professor and first director. Suffering from a lung disease, tuberculosis, he stopped his jobs in 1852. His scientific work is varied: optics, astronomy, electricity ... His most famous publication was presented in 1842 at the Royal Academy of Sciences of Bohème and has for title \textit{On the coloured light of the double stars and other stars of the sky}, using the Doppler effect. His calculations were wrong, the real offset of the light frequency was too low to be detected at this time. In 1846 Doppler published a correction of the initial work that takes into account the relative speeds of the light source and the observer.

\parpic[l][t]{%
  \begin{minipage}{40mm}
    \fbox{\includegraphics[width=110px,height=140px]{img/medaillons/drude.eps}}
  \end{minipage}
}
\textbf{Drude, Paul Karl Ludwig} (1863-1906) was a physicist born in Braunschweig and died in Berlin. Drude began his studies in mathematics at the Göttingen Universität, but then went then to physics. He completed his Ph.D. in 1887 and wrote a thesis on the reflection and diffraction of light in crystals. In 1894 he was appointed professor at the Leipzigs Universität of Leipzig. In 1900 he obtained the post of editor of the scientific journal Annalen der Physik. The same year, he developed a model know today as "Drude model" explaining the thermal, electrical and optical properties of the material that will be taken in 1933 by Arnold Sommerfeld and Hans Bethe and will become the "Drude-Sommerfeld model". He teaches at the University of Giessen from 1901 to 1905 and was promoted Director of the Department of Physics at the University of Berlin. In 1906 he became a member of the Berlin Academy.

\phantomsection
\addcontentsline{toc}{section}{E}	

\parpic[l][t]{%
  \begin{minipage}{40mm}
    \fbox{\includegraphics[width=110px,height=140px]{img/medaillons/einstein.eps}}
  \end{minipage}
}
\textbf{Einstein, Albert} (1879-1955), born in Ulm and die in Princeton, is a theoretical physicist who was successively German and stateless (1896), Swiss (1901) and finally under the Swiss-American dual citizenship (1940). He published his theory of Relativity in 1905, and a theory of gravity named "General Relativity" in 1915. He contributed to the development of quantum mechanics and cosmology, and received the Nobel Prize in Physics in 1921 for his explanation of the photoelectric effect. His work is best known for the equation of equivalence which establishes an equivalence between matter and energy of a system. He is also known for his hypothesis on the corpuscular nature of light. But he also contributed to the development of many other theories (including quantum physics like LASER or EPR paradox). In 1905, Einstein received his Ph.D. from the University of Zurich for a theoretical dissertation on the dimensions of molecules. He also published three theoretical papers of central importance on the development of the physics of the twentieth century. In the first of these articles, on Brownian motion, he made important predictions on the movement of particles randomly distributed in a fluid. During the rest of his life, Einstein devoted a considerable time to generalize even more his theory of General Relativity. He was trying to find a unified field theory, which was not completely successful, and made numerous attempts to describe the electromagnetic interaction and gravitational interaction in a common model.

\parpic[l][t]{%
  \begin{minipage}{40mm}
    \fbox{\includegraphics[width=110px,height=140px]{img/medaillons/erdos.eps}}
  \end{minipage}
}
\textbf{Erdös, Paul} (1913-1996) was the most prolific mathematicians of the 20th century, with about 1,500 articles (we have to go back to Euler for the same volume of publications). More than someone who was building theories, he solved problems, often with elegance and simplicity. Erdös was born in Budapest. Both his parents were teachers of mathematics in secondary. While Erdös was aged just one year, his father was taken prisoner by the Russians and deported to Siberia. These events contributed to the development of a very strong mother/son relation, which greatly influence the course of the life of Paul Erdös. At the age of 19, he began his studies at the university and quickly became known in mathematical associations. He publishes a new proof of Bertrand's postulate, which asserts that there exists a prime number between $n$ and $2n$ for all n. Two years later, he obtained his Ph.D. (21 years old), then goes to a post-doc in Manchester. As Erdös is of Jewish origin, he can not return to Hungary in the late 30s, and he emigrated to the United-States. After several visits in Europe to survivors of his family after the Holocaust, he has problems in the United-States with the McCarthyism, and he sees himself banned from entering the country. Erdös is forced to settle down in Israel. With 1,500 publications, the contributions from Erdös are very important: number theory, combinatorics, discrete mathematics, where he was a master. Erdös had an exceptional ability to surround himself with the most competent mathematicians to solve it's conjecture. As a result, Erdös had many collaborators: 500 mathematicians wrote an article in common with him. Mathematicians had fun to define an Erdös number: any mathematician who published a paper together with Erdös has an Erdös number equal to 1. Anyone who has published a paper with someone who has an Erdös number equal to 1 has an Erdös number equal to 2. And so on ... Albert Einstein Erdös number is 2. However, among all these collaborations, at least one went wrong, and it is all the more regrettable as it concerns the most successful subject of Erdös. At the end of the 19th century, Hadamard and de La Vallée Poussin had proved the prime number theorem, that the number of primes less than or equal to $n$ is equivalent, when $n$ is large, $n/\ln(n)$. Their proof is particularly difficult! In 1949, Atle Selberg found an inequality he thinks that can be an important step towards an elementary proof of the prime number theorem. The inequality is presented to Erdös, who finds the missing key to complete the proof. An co-authored publication would probably have been the most appropriate to measure the contributions of each. But after a misunderstanding related to sending triumph postcards form Erdös, Selberg fears that Erdös take the advantage only for him. He published alone the full proof. He will receive the Fields Medal in 1950, and Erdös will only receive the Wolf Prize in 1984. Erdös life was really strange. He had no home, no wife, material contingencies were painful for him. He traveled alone, with two suitcases that contained all his possessions, going from university to university, living in a hotel or with mathematician friends... He is also the author of numerous "erdosismes" as this famous sentence: «A mathematician is a machine for turning coffee into theorem». Himself was doped with all kinds of amphetamines! Until the end of his life, Erdös didn't slow his mathematical activity. Die meant to him to stop doing math. He died in Warsaw, during a congress.

\parpic[l][t]{%
  \begin{minipage}{40mm}
    \fbox{\includegraphics[width=110px,height=140px]{img/medaillons/erlang.eps}}
  \end{minipage}
}
\textbf{Erlang, Agner Krarup} (1878-1979) was a Danish mathematician who worked on the theory of queues and management of telephone networks. Erlang has worked, on the basis of the work of Poisson that law of rare events has found its application dimension to telecommunications networks, the developmenft of a mathematical model for the design of telecommunications networks on a statistical approach to achieve operating costs likely to enable a mass market.\\\\\\

\parpic[l][t]{%
  \begin{minipage}{40mm}
    \fbox{\includegraphics[width=110px,height=140px]{img/medaillons/euclide.eps}}
  \end{minipage}
}
\textbf{Euclid} (3rd century BC.) We know very little thing about the life of Euclid. He apparently taught mathematics in Alexandria at the request of Ptolemy 1st. He would thus appear as the founder of the famous Alexandrian school which influenced the work of Archimedes. In contrast, theories of Euclid are well known and constitute a reference in the history of mathematics. The masterpiece is undoubtedly of \textit{Euclid's Elements}. This book represents a remarkable synthesis of mathematical results and has left its mark on the discipline as a whole. It consists of thirteen books. The first four deal with plane geometry with the definitions of point, and the straight line and the surface. They also expose the calculation of areas of different polygons. The Book V contains the rudiments of analysis. The VI deals with the similarity of figures and gives the solution of quadratic equations using geometric constructions. Books VII, VIII, and IX deal with arithmetic. The X studies irrational numbers, and finally the last three deal with geometry in space. Euclid, also writted books on geometric analysis, optics and astronomy. Perfect representation of the scientific statement, the \textit{Euclid's Elements} consist of various proposals classified into two groups: the axioms and assumptions. Among the five axioms, we find the famous postulate of «Euclid by any point of the plane passes one and only one line parallel to another line». This axiom is the foundation of Euclidean geometry, as opposed to non-Euclidean geometries that appeared some 2000 years later.

\parpic[l][t]{%
  \begin{minipage}{40mm}
    \fbox{\includegraphics[width=110px,height=140px]{img/medaillons/euler.eps}}
  \end{minipage}
}
\textbf{Euler, Leonhard} (1707-1783), Swiss mathematician born in Basel but also physicist, engineer and philosopher, he was one of the founders of the methods of differential and integral calculus.  His father, Paul Euler, was pastor of the Reformed Church, and Marguerite Brucker, his mother, the daughter of a pastor. Shortly after the birth of Leonhard Euler's family moved from Basel to the town of Riehen, where Euler spent most of his childhood. Paul Euler was a friend of the Bernoulli family. Jean Bernoulli is considerate as the main European mathematician and the one who had the greatest influence on the young Leonhard. The official education of Euler started early in Basel, where he was sent to live with his maternal grandmother. At the age of thirteen, he enrolled at the University of Basel, and in 1723 he received his Master of Philosophy with a dissertation that compared the philosophy of Descartes to this of Newton. At that time, he received every Saturday afternoon lessons from Johann Bernoulli, who quickly discovered that his student has incredible talent for mathematics. Euler began to study theology, Greek and Hebrew at the request of his father to become a pastor, but Jean Bernoulli convinced Paul Euler that Leonhard was destined to become a great mathematician. Euler was the first to treat analytically the algebra, equations, trigonometry and analytic geometry. In this work, he treated the subject of the development of series of functions and formulated the rule that only the convergent infinite series could be properly evaluated. He also discussed the three-dimensional surfaces and proved that the conic sections are represented by the general equation of second degree in two dimensions. Other works deal with simple algebra, the calculus of variations, the theory of numbers, imaginary and transcendental numbers, determinate and indeterminate algebra and graph theory. Euler brought also contributions in the fields of astronomy, analytical mechanics (variational calculus), hydrodynamics, optics and acoustics. Euler is considered as an eminent mathematician of the 18th century and one of the best and most prolific of all time and who introduced much of the notations still used in the early 21st century (symbols for the sum function, logarithm , exponential, etc.).

\phantomsection
\addcontentsline{toc}{section}{F}

\parpic[l][t]{%
  \begin{minipage}{40mm}
    \fbox{\includegraphics[width=110px,height=140px]{img/medaillons/faraday.eps}}
  \end{minipage}
}
\textbf{Faraday, Michael} (1791-1867) was an English scientist who contributed to the fields of electromagnetism and electrochemistry. The young Michael Faraday, who was the third of four children, having only the most basic school education, had to educate himself. At 14 he became apprentice in a local bookbinder. During his seven year apprenticeship he read many books. At this time he also developed an interest in science, especially in electricity. Faraday was particularly inspired by the book \textit{Conversations on Chemistry} by Jane Marcet. It was by his research on the magnetic field around a conductor carrying a direct current that Faraday established the basis for the concept of the electromagnetic field in physics. Faraday also established that magnetism could affect rays of light and that there was an underlying relation between the two phenomena. He similarly discovered the principle of electromagnetic induction at the same time as Joseph Henry, diamagnetism, and the laws of electrolysis. His inventions of electromagnetic rotary devices formed the foundation of electric motor technology, and it was largely due to his efforts that electricity became practical for use in technology.

\parpic[l][t]{%
  \begin{minipage}{40mm}
    \fbox{\includegraphics[width=110px,height=140px]{img/medaillons/feigenbaum.eps}}
  \end{minipage}
}
\textbf{Feigenbaum, Mitchell} (1994-)  was born in New York City, from Polish and Ukrainian immigrants. He attended Samuel J. Tilden High School, in Brooklyn, New York, and the City College of New York. In 1964 he began his graduate studies at the Massachusetts Institute of Technology (MIT). Enrolling for graduate study in electrical engineering, he changed his area to physics. He completed his doctorate in 1970 for a thesis on dispersion relations. After short positions at Cornell University and the Virginia Polytechnic Institute and State University, he was offered a longer-term post at the Los Alamos National Laboratory in New Mexico to study turbulence in fluids. Although that group of researchers was ultimately unable to unravel the currently intractable theory of turbulent fluids, his research led him to study chaotic maps. Some mathematical mappings involving a single linear parameter exhibit the apparently random behavior known as chaos when the parameter lies within certain ranges. As the parameter is increased towards this region, the mapping undergoes bifurcations at precise values of the parameter. At first there is one stable point, then bifurcating to an oscillation between two values, then bifurcating again to oscillate between four values and so on. In 1975, Dr. Feigenbaum, using the small HP-65 calculator he had been issued, discovered that the ratio of the difference between the values at which such successive period-doubling bifurcations occur tends to a constant of around 4.6692... He was able to provide a mathematical proof of that fact, and he then showed that the same behavior, with the same mathematical constant, would occur within a wide class of mathematical functions, prior to the onset of chaos. For the first time, this universal result enabled mathematicians to take their first steps to unraveling the apparently intractable "random" behavior of chaotic systems. This "ratio of convergence" is now known as the first Feigenbaum constant.

\parpic[l][t]{%
  \begin{minipage}{40mm}
    \fbox{\includegraphics[width=110px,height=140px]{img/medaillons/fermat.eps}}
  \end{minipage}
}
\textbf{Fermat, Pierre de} (1601-1665) was a French mathematician, author of a famous theorem without proof in arithmetic and nicknamed "the prince of amateurs". He is at the origin of Femat's principle (optics) and with his friend Blaise Pascal of probabilities. He also created the theory of numbers and made several discoveries in this field. Thus, some consider him the father of the modern theory. He outran the differential calculus for his work on calculus. He left to posterity the task of proving a theorem, the famous "Fermat's last theorem", on which mathematicians are bent for more than three centuries. It was not until 1993 that the British researcher Andrew Wiles proposed a very complexed proof.

\parpic[l][t]{%
  \begin{minipage}{40mm}
    \fbox{\includegraphics[width=110px,height=140px]{img/medaillons/fermi.eps}}
  \end{minipage}
}
\textbf{Fermi, Enrico} (1901-1954) was an Italian physicist, known for making the first controlled nuclear reaction. Very young Enrico Fermi showed an exceptional memory and high intelligence, allowing it to excel in studies. Enrico, deeply marked by the death of one of his very young brother, then throws in the study of physics to overcome his pain. Good student, he developed a passion for physics and mathematics and began studying various books dealing with mechanics, optics, astronomy and acoustics. A friend of his father, Adolfo Amidei engineer, who becomes aware of the unusual qualities of the young Fermi lends him various books on mathematics. Thus, at age seventeen, Enrico Fermi masters analytical geometry, projective geometry, calculus, integral calculus and mechanics. Starting 1918 Fermi studied at the University of Pisa. As usual, he studied alone various problems of mathematical physics and consult the works of Poincaré, Poisson or Appell. From 1919, he is interested in new theories such as relativity and atomic physics, and he acquired a great knowledge of theories such as relativity, the theory of blackbody or the Bohr's hydrogen atomic model. Also Enrico Fermi, who was the only one at university aware of these theories, comes at the insistence of his teachers to give lectures where he exposes teachers and assistants the latest discoveries in atomic physics. In 1922, after four years at the university, Enrico Fermi published his first paper on General Relativity. In an Italian scientific community hostile to the work of Einstein, he is the only one with Levi-Civita to defend the theory of relativity. In 1922, Fermi received his graduate diploma after a submission on the X-ray diffraction. He attended various senior physicists in Italy, before becoming, for two years, a lecturer at the University of Florence. In 1926, he became professor of theoretical physics at the University of Rome La Sapienza. It was during this period that he developed the quantum statistical theory later named the "Fermi-Dirac statistics". From 1932, he focus more specifically on nuclear physics, and it is this same year he wrote an article on the beta radioactivity. In 1934, he developed his theory of the emission of beta radiation by including the neutron postulated in 1930 by Wolfgang Pauli that he renamed neutrino (neutron name was already used for another particle), and he develops the creation of artificial radioactive isotopes by slow neutron bombardment (for which he received the Nobel Prize in 1938).

\parpic[l][t]{%
  \begin{minipage}{40mm}
    \fbox{\includegraphics[width=110px,height=140px]{img/medaillons/feynman.eps}}
  \end{minipage}
}
\textbf{Feynman, Richard Phillips} (1918-1988) was  born in St Far Rockaway, Queens district of New York (United States) of Russian and Polish parents. His father, who encouraged him to ask questions and to challenge the commonly accepted things, has had a major influence. From his mother, he herited a strong sense of humour that never left him. Feynman is one of the most influential physicists of the second half of the 20th century, partly because of his work on relativistic quantum electrodynamics, quarks and superfluid helium. During his last year in the high school of Far Rockaway, Feynman won the championship of Mathematics of the University of New York... He also received a scholarship to study at the Massachusetts Institute of Technology (MIT) where he received his baschelor of advanced studies in 1939 after having initially study electronics and mathematics, and finally he attended all courses offered including physics during its second year course of theoretical physics reserved for graduate students. Feynman gets a remarkable score at the entrance examinations to Princeton University in mathematics and physics, but he had a very low score in the literature exam. During his studies at the Institute for Advanced Study at Princeton (IAS) (recently created and directed by Albert Einstein), Feynman worked under the direction of John Wheeler on the principle of least action applied to quantum mechanics. He established the foundations of Feynman diagrams and the approach of quantum mechanics trough integral paths. He obtained his Ph.D. in 1942. He completely reformulated quantum mechanics using the path integral which generalizes the action principle of classical mechanics and invented diagrams that bear his name and are now widely used in quantum field theory (including electrodynamics quantum part). Musician, teacher remarkable writer of many popular books, he has also been involved in the development of the atomic bomb. After World War II, he taught at Cornell University and then at Caltech where he conducted fundamental research in the theory of superfluidity and quarks. Sin-Itiro Tomonaga, Julian Schwinger and Feynman are co-winners of the Nobel Prize for Physics in 1965 for their work in quantum electrodynamics.

\parpic[l][t]{%
  \begin{minipage}{40mm}
    \fbox{\includegraphics[width=110px,height=140px]{img/medaillons/fisher.eps}}
  \end{minipage}
}
\textbf{Fisher, Ronald Aymler} (1890-1962) born in London was a British biologist and statistician, who contributed greatly to founding modern statistics. Thanks to his works on statistics he earned the Darwin Medal in 1948, the Copley Medal in 1955 and the silver Darwin-Wallace  medal in 1958. In the field of statistics, he introduced many concepts such as Maximum Likelihood, Fisher information and Analysis of Variance (ANOVA). He is considered as a great precursor of Shannon. He is also one of the founders of modern genetics and a great follower of Darwin, in particular through the use of statistical methods, essential in population genetics. He contributed to the mathematical formalization of the principle of natural selection. He was first attracted by physics and obtained in 1912 a degree in astronomy at the University of Cambridge. From 1915 to 1919, he taught mathematics in London in private schools. In 1919, he was hired at Rothamsted Experimental Station to analyse the effect of rainfall on the yield of wheat where remained until 1933. In his publication \textit{On the mathematical foundations of theoretical statistics} of 1922, he defines a couple of basic concepts in statistics such as the notion of convergence, efficiency, likelihood and sufficient statistics. He proposed the maximum likelihood estimator in 1922 after making a first version in 1912. He also introduced in 1924 the analysis of variance. In 1925 he published some innovations in time series analysis and multiple correlations.

\parpic[l][t]{%
  \begin{minipage}{40mm}
    \fbox{\includegraphics[width=110px,height=140px]{img/medaillons/foucault.eps}}
  \end{minipage}
}
\textbf{Foucault, Leon} (1819-1868) was a French physicist famous for his demonstration of the movement of the Earth by the rotation of the plane of oscillation of the pendulum. Born in Paris, he worked with the French physicist Armand Fizeau on the measure of the speed of light. Foucault proved independently, that the speed of light in air was higher than in water. In 1851, he made a spectacular demonstration of the rotation of the Earth by suspending a pendulum with a long cable attached to the dome of the Pantheon in Paris. The pendulum demonstrated the rotation of the Earth on its axis. In 1855 he discovered that the force required to rotate a disk of copper increases when it should rotate with its rim between the poles of a magnet, the disk heating at the same time because of the "foucault's currents" induced in the metal. He also created a method for measuring the curvature of the mirrors of telescopes. He developed other instruments like a prism polariser and a gyroscope which is the basis of modern gyrocompass.

\parpic[l][t]{%
  \begin{minipage}{40mm}
    \fbox{\includegraphics[width=110px,height=140px]{img/medaillons/fourier.eps}}
  \end{minipage}
}
\textbf{Fourier, Joseph} (1768-1830) was a French physicist and mathematician known for the discover of trigonometric series and transformation that bear his name. Fourier lost his father and mother at the age of ten. The organist of Auxerre, Joseph Pallais, take Fourier in a boarding school. Recommended by the Bishop of Auxerre, he studied at the École Militaire d'Auxerre, held by the Benedictines of the Congrégation de Saint-Maur. Destinated at the monastic life, he prefers to devote himself to science for which he won the most first prizes. Brilliant student, he was promoted to professor at the age of sixteen and can therefore start his own research. He joined the École Normale Supérieure at the age of 26, where he has as teachers great scientifics like Joseph-Louis Lagrange and Pierre Gaspard Monge-Simon Laplace, whom he succeeded to the chair at the École Polytechnique in 1797. Fourier has contributed to the numerical resolution of equations and the diffusion where one of the laws have his name. His work has a direct involvement in the convergence of series and infinite sum. He participated with Monge at the Egypt campaign as scientific observer. Ennobled under Napoleon, he was a professor at the École Polytechnique, secretary of the Institute of Egypt and Prefect of Isère. He was also elected to the Académie des Sciences and at the Academie Française. He is considered as one of the founders, with the French Poisson and the Swiss Daniel Bernoulli to what we now name the "Physics-Mathematics".

\parpic[l][t]{%
  \begin{minipage}{40mm}
    \fbox{\includegraphics[width=110px,height=140px]{img/medaillons/fraunhofer.eps}}
  \end{minipage}
}
\textbf{Fraunhofer, Joseph von} (1787-1826) was  German physicist and optician, born in Straubing. Fraunhofer brought many improvements in the manufacture of optical glass, to grinding and polishing of lenses and to the construction of telescopes and other optical instruments. Fraunhofer Joseph was the eleventh child of a glassblower. He was eleven years when his parents died: they also sent him for a apprenticeship in Munich during six years so that he learns the manufacturing of mirrors. In 1801 he nearly being killed in the collapse of the mirror workshop. At the end of his apprenticeship in 1806, he had the opportunity to continue training as an optician in the Mechanics Institute of Reichenbach. The workshops were transferred in 1807 to Benediktbeuern and Fraunhofer was appointed the foreman. There, he developed new polishing machines mirrors and new types of optical glass (flint achromatic glass), which brought a decisive improvement in the quality of the lenses. Fraunhofer also invented many scientific instruments. His name is associated with fixed and black lines in the solar spectrum named "Fraunhofer lines" that he was the first to describe in detail. His research in the field of refraction and dispersion of light led to the invention and development of the spectroscopy.

\parpic[l][t]{%
  \begin{minipage}{40mm}
    \fbox{\includegraphics[width=110px,height=140px]{img/medaillons/fresnel.eps}}
  \end{minipage}
}
\textbf{Fresnel, Augustin Jean} (1788-1827) was French physicist, founder of modern optics, he proposed an explanation of all optical phenomena in the context of the wave theory of light. He began to realize many experiments on light interference, for which he postulated the concept of wavelength and created the Fresnel integrals. He was the first to prove that two beams of light polarized in different planes have no interference effect. He rightly inferred from this experiment that the vibration of the polarized light is transverse and not longitudinal (such as sound) as we thought before him. In addition, he was the first to produce a circularly polarized light. To explain the propagation of light waves, Fresnel used to the notion of ether, unfortunately inconsistent with other experiments. This theory will be left with relativity, but the so-called "Fresnel relations" are always used today. In the field of applied optics, Fresnel designed levelling lens used to increase the illuminating power of the lighthouses. During his lifetime, the scientific work of Fresnel were known only to a small group of scientists and some of his articles were published only after his death.

\phantomsection
\addcontentsline{toc}{section}{G}

\parpic[l][t]{%
  \begin{minipage}{40mm}
    \fbox{\includegraphics[width=110px,height=140px]{img/medaillons/galilee.eps}}
  \end{minipage}
}
\textbf{Galileo Galilei} (1564-1642) was an Italian physicist and astronomer born in Pisa and at the origin of the scientific revolution of the 17th century. His theories and those of the German astronomer Johannes Kepler served as the basis for the work of British physicist Sir Isaac Newton's law of universal gravitation. His main contribution to astronomy was a considerable improvement (when the technique worked...) of the telescope (which allowed him to make observations that revolutionize the discipline) and the discovery of sunspots, lunar mountains and valleys, the four largest satellites of Jupiter and the phases of Venus. In physics, he discovered the law of falling bodies and projectiles parabolic movements. His studies on the oscillations of the pendulum weight led him to invent the pulsometer. This device enabled pulse measurement and provided a standard time, which did not exist at that time. He also started his studies on falling bodies. In the history of culture, Galileo is the symbol of the battle against the religious authorities for the freedom of research (he had however a very good reputation and good relations with religious people that helped...). In mathematics and physics, he helped to advance the knowledge about the kinematics and dynamics, thus laying the foundations of the mechanical sciences. He is therefore considered as the founder of modern physics.

\parpic[l][t]{%
  \begin{minipage}{40mm}
    \fbox{\includegraphics[width=110px,height=140px]{img/medaillons/galois.eps}}
  \end{minipage}
}
\textbf{Galois, Evariste} (1811-1832) was a French mathematician, who gave his name to a branch of mathematics: Galois theory. His life is so legendary that it is sometimes difficult to distinguish between myth and reality. Starting 1827-1828, the fury of mathematics dominates. Galois reads Legendre, Lagrange, Euler, Gauss, Jacobi. Professor, Louis-Paul-Émile Richard admires the mathematical genius of his student and keeps his copies and entrust him to another of his students: Charles Hermite. This is the time when he published his first article in the Annales des Mathématiques of Joseph Gergonne (he proves a theorem on periodic continued fractions). He also wrote a first paper on the theory of equations, sent to the Académie des Sciences, lost by Cauchy... He failed the entrance exam to Polytechnique. Some people say that Galois threw the cloth to erase the chalk at the head of his examiner because of the stupidity of the questions. On the advice of his teacher, Galois entered the Preparatory School (the future École Normale). He wrote the results of his research in a paper - \textit{Requirements for an equation to be solvable by radicals} - to compete for the grand prize of mathematics of the Académie des Sciences. Fourier took the manuscript at home and died shortly after: the manuscript is lost, and the grand prize is awarded to Abel (die the year before), and Jacobi. For political reasons, Galois goes in prison, where he continued his research work. Released in 1832, he fell in love in May 1832 of a woman with whom he broke the same year. It is unclear why, but a duel seems to result a few days later. The night preceding May 29, Galois resume his latest discoveries in a beautiful letter to his friend Auguste Chevalier. From this letter was born the legend that Galois made his major discoveries in one night, caught by the fever of death. On the morning of May 30, Galois, abandoned, severely wounded, is raised by a peasant and leads to the Cochin's Hospital . He died the day after in the arms of his younger brother and was buried in a common grave in the cemetery of Montparnasse. The work of Galois are rediscovered a decade later by Liouville, who announced in 1843 at the Academy of Sciences that he has found in the papers of Galois a solution as accurate as deep to the problem of solvability by radicals. It was only in 1846 that he publishes the texts without adding comments. Starting from 1850, the writings of Galois are finally accessible by the best mathematicians.

\parpic[l][t]{%
  \begin{minipage}{40mm}
    \fbox{\includegraphics[width=110px,height=140px]{img/medaillons/gamow.eps}}
  \end{minipage}
}
\textbf{Gamow, George} (1904-1968) was a Russian-American theoretical physicist, astronomer, cosmologist and scientific author born in Odessa, Ukraine. Gamow came in 1928 in Göttingen, where he uses quantum physics to a develop a quantum theory of alpha radioactivity. Two months later, he joined Niels Bohr in Copenhagen. It makes the idea of an atomic nucleus behaves like a nuclear fluid, model set almost a decade later by Bohr. In 1929, he received a new award and he joined Ernest Rutherford at the University of Cambridge. He develops the idea of tunnelling in order to makes protons interact to obtain nuclei with atomic numbers higher. There he met John Cockcroft, who built shortly after the first particle accelerator, thus achieving validate the Gamow model by a transmutation of lithium. Professor at Washington in 1934, Gamow worked with Edward Teller to formulate the theory of beta decay (1936). Interested by astrophysics, Gamow and Teller give a model of the internal structure of red giant stars (1942). In 1954, interested by biochemistry, he proposed the concept of genetic code determined by the order of the components of DNA. In 1956, he was appointed professor of physics at Boulder (Colorado).

\parpic[l][t]{%
  \begin{minipage}{40mm}
    \fbox{\includegraphics[width=110px,height=140px]{img/medaillons/gauss.eps}}
  \end{minipage}
}
\textbf{Gauss, Carl Friedrich} (1777-1855) was German mathematician who made major contributions to many branches of pure and applied sciences. At the age of seventenn, he tried to find a solution to the classical problem of building a polygon with seven sides using only ruler and compass. He managed to prove the impossibility of this construction and continued his approach by providing methods for constructing polygons with 17, 257, and 65'537 sides. More generally, he proved that the construction, using always only ruler and compass, of a regular polygon with an odd number of sides is possible only if the number of sides is one of the prime numbers 3, 5, 17, 257, and 65'537, or a product of these numbers. For his Ph.D., he showed that any algebraic equation has at least one root. This theorem, whose proof had resisted to the most famous mathematicians, is still named the "Fundamental theorem of algebra" or "Alembert-Gauss theorem". Gauss then turned his attention to the field of astronomy where he developed a new method for calculating the orbits of celestial bodies, developing a theory of errors of observation known as the least squares method (for "overdetermined" linear systems). In the field of probabilities, his name is attached to the Normal distribution (also named "Laplace-Gauss distribution"), whose is described by the famous bell curve or "Gaussian curve". He also worked in geodesy. With the German physicist Wilhelm Eduard Weber, Gauss did, starting from 1831, extensive research in the field of magnetism and electricity. He also conducts research in optics, particularly lenses systems. To return to mathematics, he was the first, studying the hypergeometric series, to give rigorous conditions of convergence of a series. He studied successful generalizations of the law of quadratic reciprocity and discovered their links with the theory of elliptic functions. His memoir of 1828 on the theory of intrinsic surface was the starting point for a general theory of curved spaces (Riemann's work and successors). He also introduced the arithmetic of Gaussian integers (of the form $a+ib$) based on a geometric representation of complex numbers as points in the plane.

\parpic[l][t]{%
  \begin{minipage}{40mm}
    \fbox{\includegraphics[width=110px,height=140px]{img/medaillons/gibbs.eps}}
  \end{minipage}
}
\textbf{Gibbs, Josiah Willard} (1839-1903) was a physicist and mathematician, born and died in New Haven, Connecticut (after spending almost his entire life as single). Coming from a family of scholars, he studied latin and physics, and he began a career as a professor of mathematical physics at Yale College. He lived successively in Paris and Berlin where he took lessons from Heinrich Gustav Magnus and Heidelberg and met Gustav Kirchhoff and Helmholtz Herman Ludwig. Gibbs will be remembered as a scholar of proverbial modesty and with extraordinary power of scientific investigation. His work was first remarkably compact and little known. Today it is considered as a monumental scientific contributions in the 19th century. The two main publications dating from 1877 and 1902. The first is titled \textit{On the Equilibrium of Heterogeneous Substances} and compared in importance to weighted chemistry created by Antoine Laurent Lavoisier. The second, still considered more original, is titled \textit{Elementary Principles in Statistical Mechanics}, and compared, for it's genious, to the analytical mechanics of Joseph Louis Lagrange. Although Gibbs papers are distinguished by exceptional clarity, and how the basic idea is always carefully presented, the first of two papers hardly retained first the attention of chemists of his time, unaccustomed to rigorous language sciences. The wealth of thermodynamic methods on which it relies has however defined the foundations of a unified basis of physico-chemical theory of equilibrium states and their stability. Most of the laws that relate to this discipline, which first bore other names were later rediscovered in it's first memory. This is, for example, "the law of phases" giving the variance of equilibrium systems, long attributed to Bakkuis Roozeboom (laws also named "Van't Hoff law" or "Le Chatelier's law"), on the displacements of equilibrium at a constant temperature and constant pressure. It is still the same with the stability criteria of balance, or the moderation theorem also named "theorem of Le Chatelier and Braun". In short, most of the properties that are present in chemical thermodynamics equilibrium states, such as osmotic pressure, the influence of surface tension, the elastic deformation, the Law on the entropy of the gas mixtures and the associated Gibbs paradox have the same memory for origin. Gibbs developed in two previous communications to the previous one, a complete diagrams and thermodynamic surfaces catalogue which contributed to the spread of their employment by practitioners. Gibbs theory used for the first time the notion of a set as well as the distinction between a canonical set and a microcanonical set and between a large and a small set. Gibbs theory also introduces the concept of phase space, characterized by the coordinates and momenta of each element. This theory also establishes, from the Liouville equation, the law of conservation element extension phase, as well as density and probability of the statistics sate. He finally achieves a remarkable formal agreement with macroscopic laws of thermodynamics governing the behavior of material media in equilibrium. Current developments in statistical mechanics are still on more than one point, extensions of the method of Gibbs. He also defined for chemical reactions two useful quantities, namely the "enthalpy" that is the heat of reaction at constant pressure and the "free energy" that determines whether a reaction can proceed spontaneously at room temperature and constant pressure. This latter quantity is now named "Gibbs energy" in his honour. Gibbs seems to be at origin of the usage to designate the scalar product with a dot, the vector cross product with a St Andrew cross $\times$ and the adoption of nabla and del vector differential operators.

\parpic[l][t]{
  \begin{minipage}{40mm}
    \fbox{\includegraphics[width=110px,height=140px]{img/medaillons/godel.eps}}
  \end{minipage}
}
\textbf{Gödel, Kurt} (1906-1978) born in Brünn and died at Stanford, was a mathematician and logician, that in the entire 20th century, has the most revolutionized the logical foundations of mathematics. He was a man so obsessed with the logic that when he tried to get his American citizenship, he dared to show the judge the contradiction of some articles of the constitution of the United States. His thesis, and especially an article published in 1931 under the title \textit{Über formal unentscheidbare Sätze und der Principia Mathematica verwandter System} (About the undecidability of formal Principia Mathematica and similar systems), will give Gödel an international reputation. Gödel puts an end to hopes of Hilbert about a complete axiomatized mathematics system, and to make of mathematics a field where only mechanical deductions are possible leaving no place for intuition. Thus, Gödel shows that there are true propositions about integers, but that they can not be proved. He shows that even if we add other axioms, there will always be true undecidable propositions (we can not prove). He shows in particular that the continuum hypothesis and the axiom of choice is not in contradiction with the other axioms of set theory. Then he turned to relativity, being directly related to Princeton with his friend Einstein. He is known to physicists as having demonstrated that travel to the past is possible within the framework of the equations of General Relativity.

\parpic[l][t]{%
  \begin{minipage}{40mm}
    \fbox{\includegraphics[width=110px,height=140px]{img/medaillons/goeppertmayer.eps}}
  \end{minipage}
}
\textbf{Göpper-Meyer, Maria}(1906-1972) was a German-born American physicist, Nobel Prize in 1963 for her study of nuclear structure. She was married to the physicist Joseph Mayer, specialized in solid state physics (1904-1983). But in this couple, each worked separately in his speciality. Goeppert-Mayer obtained his Ph.D. at the University of Göttingen, Germany. She taught in many institutions before returning to the University of California at San Diego in 1960. In 1963 she shared with H.D. Jensen and  E.Wigner the Nobel Prize in physics and was cited by the Nobel committee for his independent work in the late 1940s. She proved that the nucleus has a number of neutrons and protons well defined: she introduced a structural model of the atomic nucleus in layers. This model, developed in detail in 1948 assumed that the strong interaction between the intrinsic rotation (quantified by the spin) of nucleons and their orbital motion was responsible for the structure of the energy levels of the nuclei. Many consequences deduced from this hypothesis were verified by experimental measurements. A few years later, James Rainwater, Aage Bohr and Ben R. Mottelson (all three Nobel Prize in Physics 1975) completed the theory taking into account the coupling between the motion of the nucleons in the outer layer and the collective motion of the nuclear core.

\parpic[l][t]{%
  \begin{minipage}{40mm}
    \fbox{\includegraphics[width=110px,height=140px]{img/medaillons/gosset.eps}}
  \end{minipage}
}
\textbf{Gosset, William Sealy} (1876-1937) known under the pseudonym "Student" was an English statistician. Employee of the Guinness brewery to stabilize the flavour of the beer, he invented the $t$-test used actually as a standard in many fields of industry or the economy. He also determined in 1908 the experimental distribution he obtained through his job and after taking a statistics course with Karl Pearson, he obtained his famous result that he published under the pseudonym "Student" with the law that bears and test that still bear his name today.\\

\parpic[l][t]{%
  \begin{minipage}{40mm}
    \fbox{\includegraphics[width=110px,height=140px]{img/medaillons/gottlob.eps}}
  \end{minipage}
}
\textbf{Gottlob, Frege Friedrich Ludwig} (1848-1925) was a German mathematician and philosopher, founder of modern logic. Frege was born in Wismar in 1848, and was educated at the universities of Jena and Göttingen where he received his Ph.D. in philosophy in 1873. From 1879 to 1917 he was professor at the Faculty of Philosophy in Jena. His work focuses on particular mathematical logic and its applications. Faced with the ambiguity of ordinary language and imperfect logic systems available, he invented many symbolic notations, such as quantifiers and variables, then putting the foundations of modern mathematical logic. He is the first to have presented a coherent theory of predicate calculus and the propositional calculus. He was also the first to derive the logical arithmetic. He defined in particular the following integers from the empty set, by applying a few simple rules.

\parpic[l][t]{%
  \begin{minipage}{40mm}
    \fbox{\includegraphics[width=110px,height=140px]{img/medaillons/grothendieck.eps}}
  \end{minipage}
}
\textbf{Grothendieck, Alexander} (1928 -) is born in Berlin, his father was a Russian anarchist who was killed by the Nazis, and his mother a woman of letters refugee in France. He obtained his Bsc. at the Faculté de Montpellier, then spent a year in 1948-1949 at the École Normale Supérieure in Paris, before moving in 1949 to the Université de Nancy. He became there a student of Schwartz and Dieudonné in functional analysis. Dieudonné feel that Grothendiek was a bit pretentious, and also asked him to work on issues that neither Schwartz have solved. Here is what Schwartz says in his autobiography: "Dieudonné, with the aggression (always passing), of which he was capable, reprimanded Grothendiek, arguing that he should not work this way, by generalizing just for the pleasure to generalize. [...] The article ended with 14 questions, open problems that we had not been able to resolve, I and Dieudonné. Dieudonné proposed to Grothendieck to consider some of the problems that he would choose. We never saw him again during a few weeks. When he reappeared he had found the solution of half of them!". Quickly, Grothendieck wrote his thesis on Topological tensor products and nuclear spaces, and became a worldwide specialist in the theory of topological vector spaces. He also became a member of the famous Bourbaki group. In the early 1960s, he gets a function at the recent Institute of Advanced Scientific Studies (IHES), and his focus is directed towards on algebraic geometry. There he made gigantic works, which earned him the Fields Medal in 1966. However, Grothendieck refuses to go to the USSR to receive the prize, to protest against the repression of the Hungarian uprising in 1956. The Fields Institute gives him the Fields medal later, but Grothendieck offers it to Vietnam to use its gold. He also teaches there several weeks under the American bombing. In the late 60s, Grothendieck, who lost the habit of writing (Dieudonné wrote during his seminary years), becomes less and less clear. He will never forgive other mathematicians do not understand him and distorted his ideas. If his relations with the mathematical community had never been easy (he worked a lot alone, his days were 27 or 28 hours, so that sometimes he was shifted - He despised slightly Dieudonné, sequel of the first reprimand - his disputes with Weil caused his departure from Bourbaki ...), they are more strained than ever ... He gradually abandoned mathematics and the IHES after a dispute over military funding in 1970, to retire to his home in the Hérault, where he devoted himself to meditation and ecology. In 1985 he wrote a sort of autobiography that was not published. Those who have read it are unanimous in saying that it contained many attacks against the community of mathematicians.

\phantomsection
\addcontentsline{toc}{section}{H}

\parpic[l][t]{%
  \begin{minipage}{40mm}
    \fbox{\includegraphics[width=110px,height=140px]{img/medaillons/hall.eps}}
  \end{minipage}
}
\textbf{Hall, Edwin Herbert} (1855-1938) was a physicist born in the Main and died in Cambridge (U.S.A.). Hall did his undergraduate work at Bowdoin College, graduating in 1875. He did his graduate schooling and research, and earned his Ph.D. degree (1880) at the Johns Hopkins University where his experiments were performed. The Hall effect was discovered by Hall in 1879, while working on his doctoral thesis in Physics. Hall was appointed as Harvard's professor of physics in 1895. He was notable for lecturing without shoes and wrote numerous books on physics.\\

\parpic[l][t]{%
  \begin{minipage}{40mm}
    \fbox{\includegraphics[width=110px,height=140px]{img/medaillons/hamilton.eps}}
  \end{minipage}
}
\textbf{Hamilton, William Rowan} (1805-1865) was an Irish mathematician, physicist and astronomer (born and died in Dublin) who was the object during his lifetime of the highest honours and was called the "Irish Lagrange" and even "Irish Newton" yet his work was little known and rarely studied. He is known for his discovery of quaternions, but also contributed to the development of optics, dynamics and algebra. His research was important for the development of quantum mechanics. The mathematical work of Hamilton include the study of geometrical optics, the adaptation of dynamic methods for optical systems, applying quaternion and vector problems of mechanical and geometric possibilities of solving polynomial equations, including the general equation of the fifth degree, linear operators, for which he proves a result for these operators in the space of quaternions, which is a special case of Cayley-Hamilton theorem. His scientific career was predestined by its studies at Trinity College, in Dublin, where, at the age of nineteen, he finished a remarkable job on the lens. At the age of 23 years, he became professor of astronomy at Dublin and Royal Astronomer at Dunsink Observatory where he will stay for the rest of his life. Hamilton tries to provide the fundamental principles of mechanics to a simple form to build a deductive theory. To do this, he modifies the principles of previous variations, including the principle of least action, and introduced what is named today the "Hamilton's principle". Finally, we note that he is at the origin of the "canonical" expression of the equations of the dynamics that brings nothing new to physics but provides a more powerful method for solving the equations of motion. In his work of the years 1832 to 1835 Hamilton attaches a great importance to the geometric interpretation of complex numbers, and it is from there that one seeks to interpret algebraic calculation in three-dimensional space. He arrives at this goal in 1843, building the quaternions. In the years following this discovery, he devoted himself to his development and its dissemination, by finding applications in various fields of mathematics and physics. The Hamilton's quaternions are one of the first vector systems and, through their theoretical consequences, contributed significantly to the development of algebra and quantum physics in the 20th century.

\parpic[l][t]{%
  \begin{minipage}{40mm}
    \fbox{\includegraphics[width=110px,height=140px]{img/medaillons/hawking.eps}}
  \end{minipage}
}
\textbf{Hawking, Stephen} (1942-) born in Oxford, is a British theoretical physicist and cosmologist. Just as Albert Einstein, Hawking was not particularly brilliant at the high school, but his taste for the physical sciences leads him to the University of Oxford, a place of relative boredom to where he exits with honours. After receiving his B.A. degree at Oxford in 1962, he stayed to study astronomy. He decided to quit when he found that studying sunspots was not attractive and that he was more interested in theory than by observation. He left Oxford with honours, for Trinity Hall, where he took part in the study of theoretical astronomy and cosmology theory. The University of Cambridge is a different world: on the one hand, there Hawking begins his exciting Ph.D. in General Relativity, on the other, his disease occurs. Despite this difficulty, the study of singularities, allows the researcher to develop theories that will lead him to the Big Bang and Black Holes theory. First, Roger Penrose and Hawking build the mathematical structure answering the question of a singularity as the origin of the Universe. Then, starting from the 1970s, Hawking deepened his research on local infinite densities, and his studies on black holes have advanced many other areas. Finally, the theory of everything, to unify the four physical forces, is at the center of current Hawking researches. The aim is to demonstrate that the universe can be described by a mathematical stable model, determined by the known physical laws, the principle of finite growth but not limited, model for which Hawking gave a lot of credit. His severe handicap can not alone explain the great success of his research, Hawking has tried to popularize his work, and his book A Brief History of Time is one of the most successful scientific literature. In 2001, he released his second book, The Universe in a nutshell that explains the latest state of his thoughts, addressing supergravity and supersymmetry, quantum theory and M-theory, holography and duality theory superstring and p-branes ... He also wondered about the possibility of time travel and the existence of multiple universes.

\parpic[l][t]{%
  \begin{minipage}{40mm}
    \fbox{\includegraphics[width=110px,height=140px]{img/medaillons/hausdorff.eps}}
  \end{minipage}
}
\textbf{Hausdorff, Felix }(1868-1942) The reputation of the German mathematician Felix Hausdorff is mainly based on his book \textit{Grundzüge der Mengenlehre }(1914), who mades him the founder of the topology and the theory of metric spaces. Born in Breslau in a wealthy merchant family, Hausdorff followed a high-school education in Leipzig. After high-school he studied mathematics and astronomy at Leipzig, Freiburg im Breisgau and Berlin. In 1891, he obtained his Ph.D. in Leipzig and taught there from 1896 to 1902. Throughout this time, Hausdorff, while publishing several papers on astronomy, optics and mathematics, was particularly interested in philosophy, literature and art. From 1910 to 1935 he was professor of mathematics at the University of Bonn, with the exception of the years 1913 to 1921, where he taught at Greifswald. Since his forced retirement in 1935, the work of Hausdorff were no longer published in Germany. Jewish, Hausdorff risked the concentration camp, when internment became imminent in 1942, he committed a suicide in Bonn with his wife and sister in law. Hausdorff contributions to the development of mathematics lie in several areas. His study of the series led to the demonstration of theorems on methods of summation and Fourier coefficients (1921). Considering the properties of digital sets, he introduced an important class of measures. He studied in the general theory of sets, partially ordered sets and several theorems on ordered sets (1906-1909). In descriptive set theory, he demonstrated the theorem on the cardinality of Borel sets (1916). Apart isolated but deeps results in topology and set theory, Hausdorff had especially in his \textit{Grundzüge der Mengenlehre} laid the foundations of a discipline. Hausdorff developed a theory of topological spaces and metric encompassing perfectly the previous results. He chose to build his theory of abstract spaces on the notion of neighbourhood. He added many new results in the theory of metric spaces, the most profound is the theorem stating that every metric space can be extended in a unique way to a complete metric space. Hausdorff was a methodic teacher, but his courses, with their rich content and rigorously structured, passed above the level of his listeners.

\parpic[l][t]{%
  \begin{minipage}{40mm}
    \fbox{\includegraphics[width=110px,height=140px]{img/medaillons/heaviside.eps}}
  \end{minipage}
}
\textbf{Heaviside, Oliver} (1850-1925) was born in the city of Camden in London (England) and died in Torquay in Devon (England). This is where he lived the last twenty five years of his life. He comes from a family quite poor. He caught scarlet fever when he was a toddler, which affected his hearing, he remained partially deaf. This has had an impact on his life making difficult his childhood especially in relationships with other children. He compensated by shyness and sarcasm. However, despite all this, his academic performance was rather high. One can even say that a sixteen years old he was a top student, but he failed in Euclidean geometry. He hated having to infer a fact of another. The primacy of rigorous proof in arithmetic, an idea strongly disliked by Heaviside, made him the subject where he was the weakest. Although he stopped his studies at sixteen, he continued to learn by himself. He learned Morse code, studied electricity and other languages in particular Danish and German. He was self-taught. In 1868, after leaving his studies, Heaviside went to Denmark and became a telegraph operator. He progressed rapidly in his profession and he returned to England in 1871. It is his work that led him to study electricity. He then read the new treaty of Maxwell on electricity and magnetism. After reading this treatise, he made changes in his life. He stopped working and he locked himself in a room of the family home to work on Maxwell's theory. Heaviside reduced Maxwell's theory, and it is from this time that the electrical theory took its modern form. Maxwell has written twenty equation with twenty variables. Heaviside reduced the twenty equations into four equations with two variables. Today, we name these equations "The four Maxwell equations", forgetting that they are in fact the "Heaviside equations". However, it was Hertz who got the credit for it, but he admits that his ideas came him from Heaviside.

\parpic[l][t]{%
  \begin{minipage}{40mm}
    \fbox{\includegraphics[width=110px,height=140px]{img/medaillons/heisenberg.eps}}
  \end{minipage}
}
\textbf{Heisenberg, Werner Karl} (1901-1976), born in Würzburg and died in Munich was a German physicist. He was the founder of the rigorous theoretical concepts of quantum mechanics. He was awarded the Nobel Prize for Physics in 1932. He attended the prestigious Maximiliangymnasieum where Max Planck had studied fourty years earlier. At the age of twelve, he began to learn integral calculus and later fascinated by mathematics, he followed as free auditor followed several courses at the University of Munich, including mathematical methods of modern physics. He completed his studies in physics in the record time of three years, and defended his thesis (that he almost missed because of gaps in basic experimental physics) under the direction of Arnold Sommerfeld, with whom he developed a theory explaining the anomalous Zeeman effect at the age of twenty years, which attracted on him the attention of major European physicists (he was regarded as brilliant as Pauli himself who was already considered most brilliant than Einstein). From 1924 he became the assistant of Max Born in Göttingen and he worked with Niels Bohr in Copenhagen. It was during the following years with Max Born and Pascual Jordan, that he threw the theoretical foundations of quantum mechanics. Heisenberg was recruited in 1927 as a professor at the University of Leipzig aged only twenty six. He made of this University one of the highest places of theoretical physics (especially nuclear physics) in Europe. He developed the first formalization of quantum mechanics, in 1925, Erwin Schrödinger at the same time. However, the mathematical formalism was different. Heisenberg adopted a complex matrix formulation (although he did not know what was a matrix as most physicists of his time...) from which emerged naturally the non-commutativity while Schrödinger used an approach based on differential equations (simple wave equation). For this reason we thought at first that the two theories were distinct but the following year, Schrödinger establishes the mathematical equivalence of the two formulations. His uncertainty principle, discovered in 1927, says that the determination of certain pairs of values, such as position and momentum, can not be done with infinite precision. From 1929, he worked with Wolfgang Pauli in the development of quantum field theory. After the discovery of the neutron by James Chadwick in 1932, Heisenberg proposed the proton-neutron model of the atomic nucleus, and used it to explain the nuclear spin isotopes.

\parpic[l][t]{%
  \begin{minipage}{40mm}
    \fbox{\includegraphics[width=110px,height=140px]{img/medaillons/hemlholtz.eps}}
  \end{minipage}
}
\textbf{Helmholtz, Hermann Ludwig Ferdinand von} (1821-1894) was born in Potsdam and died in Berlin. There is no a field of science for which Helmholtz has not made some research. We could also say about him what he says himself about Friedrich von Humboldt in his famous inaugural lecture of scientific the symposium of Innsbruck (on the purpose and progress of the science of Nature, 1869): «He managed to dominate all of the natural sciences of his time and penetrate until each of their specialties». Even if Helmholtz said that in the second half of the 19th century that encyclopedic knowledge is now impossible, and we must resign ourselves to focus in a defined area, just take a look at all of his work to see that it was concerned with matters as diverse as thermodynamics, hydrodynamics, electrodynamics and the theory of electricity, physical meteorology, physiology, and especially the theory of acoustic and physiological optics. Having a remarkable gift for the popularization of the latest scientific findings, he wrote numerous articles and delivered many lectures where scientists subjects were presented side by side with popular aesthetic or philosophical concerns. His name is mainly linked with the formulation of the principle of conservation of energy, even if some of his assertions may seem as uncompromising mechanism and were able to give him the reputation of the last taking of Galilean physics. His name is also linked to some notable inventions like the ophthalmoscope or spherical resonators. At the end of his life, Helmholtz recognize the importance and universality of another physical principle, the "principle of least action", that he will apply, in particular, in electrodynamics.

\parpic[l][t]{%
  \begin{minipage}{40mm}
    \fbox{\includegraphics[width=110px,height=140px]{img/medaillons/henry.eps}}
  \end{minipage}
}
\textbf{Henry, Joseph} (1797-1878) was a scientist born in New-York and died in Washington. His parents were poor, and Henry's father died while he was still young. For the rest of his childhood, Henry lived with his grandmother in New York. He attended a school which would later be named the Joseph Henry Elementary School in his honour. After school, he worked at a general store, and at the age of thirteen became an apprentice watchmaker and silversmith. His interest in science was sparked at the age of sixteen by a book of lectures on scientific topics titled \textit{Popular Lectures on Experimental Philosophy}. In 1819 he entered The Albany Academy, where he was given free tuition. He was so poor, even with free tuition, that he had to support himself with teaching and private tutoring positions. He intended to go into the field of medicine, but in 1824 he was appointed an assistant engineer for the survey of the State road being constructed between the Hudson River and Lake Erie. From then on, he was inspired to a career in either civil or mechanical engineering. Henry excelled at his studies (so much, that he would often be helping his teachers to teach science) that in 1826 he was appointed Professor of Mathematics and Natural Philosophy at The Albany Academy. Some of his most important research was conducted in this new position. His curiosity about terrestrial magnetism led him to experiment with magnetism in general. He was the first to coil insulated wire tightly around an iron core in order to make a more powerful electromagnet. While building electromagnets, Henry discovered the electromagnetic phenomenon of self-inductance. He also discovered mutual inductance independently of Michael Faraday, since Faraday published his results first, he became the officially recognized discoverer of the phenomenon. Using his newly-developed electromagnetic principle, Henry in 1831 created one of the first machines to use electromagnetism for motion. This was the earliest ancestor of modern DC motor. The SI unit of inductance, the henry, is named in his honour.

\parpic[l][t]{%
  \begin{minipage}{40mm}
    \fbox{\includegraphics[width=110px,height=140px]{img/medaillons/hermite.eps}}
  \end{minipage}
}
\textbf{Hermite, Charles} (1822-1901) was born in Dieuze, he published his first work while he was still a student at the École Polytechnique, and at the age thirteen, he was already considered one of the best mathematicians of his time. He was successively professor at the École Polytechnique, the Collège de France and then at the Sorbonne in 1869, where his teaching and his voluminous correspondence had a considerable influence. He lived in Paris until his death. He was elected member of the Academie des Sciences at the age of thirty-four years. In algebra, Hermite took an active part in the early development of the theory of invariants, initiated by Arthur Cayley and James Joseph Sylvester, he completed, among others, the determination of invariants of binary forms of the fifth degree, begun by Sylvester, and discovered the law of reciprocity between covariants of binary forms of various degree. He is also at the origin of improvement of the Lagrange's interpolation method taking into account the values of the first derivatives, and of the discovery of the family of orthogonal polynomials that bear his name. The analytical work of Hermite is marked by his algebraist temperament. His favourite subject throughout his life was the theory of elliptic functions and abelian functions, which he loved to explore the hidden links with algebra and number theory. One of the results that most struck his contemporaries is the solution of equation of the fifth degree using elliptic functions. His virtuosity in the calculation of functions allowed him to directly obtain the remarkable relations on class numbers of quadratic ideals that Kronecker had derived from the complex multiplication. He was a pioneer in the study of Abelian functions, where he developed the theory of transformation and met on this occasion for the first time the symplectic group. Finally, the most famous of Hermite memories is when, in 1872, he proved the transcendence of the number $e$, using results of his research on algebraic continued fractions, and his method has remained almost the only one available today to address issues of transcendence.

\parpic[l][t]{%
  \begin{minipage}{40mm}
    \fbox{\includegraphics[width=110px,height=140px]{img/medaillons/hertz.eps}}
  \end{minipage}
}
\textbf{Hertz, Heinrich Rudolf}(1857-1894) was a German physicist born in Hamburg and died in Bonn. He studied at the Berlin Universität. In 1879, he was the student of Gustav Kirchhoff and Hermann von Helmholtz at the Berlin Physik Institut. He became a lecturer at the University of Kiel in 1883 where he conducts research on electromagnetism. From 1885 to 1889, responsible for wireless telegraphy, he was professor of physics at the Karlsruhe Teknische Schule, and from 1889, professor of physics at the Bonn Universität. Hertz clarified and expanded the electromagnetic theory of light proposed by the English physicist James Maxwell in 1884. He proved that the power could be transmitted by electromagnetic waves which travel at the speed of light and have many other properties of the light. His experiments with these waves led to the development of wireless telegraphy and radio. The unit of frequency, one period per second, was named the "Hertz".

\parpic[l][t]{%
  \begin{minipage}{40mm}
    \fbox{\includegraphics[width=110px,height=140px]{img/medaillons/hilbert.eps}}
  \end{minipage}
}
\textbf{Hilbert, David} (1862-1943) was born in Königsberg, and died in Göttingen. He was a student under the supervision of Lindemann with whom he obtained his Ph.D. in 1885 and he was a comrade of Herman Minkowski, with whom he remained bound by a deep friendship. Although mathematical interests of Hilbert were vast, he preferred to work on one subject at a time. His main areas of interest were: until 1892, the algebraic theory of invariants, from 1892 to 1899 the theory of algebraic numbers, from 1899 to 1905, the calculus of variations, from 1901 to 1912, integral equations, 1912 in 1917, the mathematical foundations of physics. Around 1910, Hilbert supports the efforts of Emmy Noether, mathematician of the first order, who wishes to teach at the Göttingen Universität. To circumvent the system established against women, Hilbert lends its name to Noether who can announce the schedule of his course without damaging the reputation of the university. From 1917 until the end of his life he devoted himself to mathematical logic. He gave a decisive impetus to the development of research on the foundations of mathematics. During the International Congress of Mathematics in 1900, Hilbert presented a list of twenty-three problems many of which remain unresolved today. He adopted and vigorously defended the ideas of Georg Cantor set theory and transfinite numbers. He is also known as one of the founders of proof theory, mathematical logic and clearly distinguished mathematics of meta-mathematics. He is considered by many as on of the greatest mathematician of the 20th century.

\parpic[l][t]{%
  \begin{minipage}{40mm}
    \fbox{\includegraphics[width=110px,height=140px]{img/medaillons/hotelling.eps}}
  \end{minipage}
}
\textbf{Hotelling, Harold} (1895-1973) was a mathematical statistician and an influential economic theorist, known for Hotelling's T-squared distribution in statistics. He was Associate Professor of Mathematics at Stanford University from 1927 until 1931, a member of the faculty of Columbia University from 1931 until 1946, and a Professor of Mathematical Statistics at the University of North Carolina at Chapel Hill from 1946 until his death. A street in Chapel Hill bears his name. In 1972 he received the North Carolina Award for contributions to science. Hotelling is known to statisticians because of Hotelling's T-squared distribution which is a generalization of the Student's t-distribution in multivariate setting, and its use in statistical hypothesis testing and confidence regions. He also introduced canonical correlation analysis. In the United States, Harold Hotelling is known for his leadership of the statistics profession, in particular for his vision of a statistics department at a university, which convinced many universities to start statistics departments.

\parpic[l][t]{%
  \begin{minipage}{40mm}
    \fbox{\includegraphics[width=110px,height=140px]{img/medaillons/hoyle.eps}}
  \end{minipage}
}
\textbf{Hoyle, Fred} (1915-2001) was born in Bingley, Yorkshire and died in Bournemouth. Hoyle studied mathematics and theoretical physics at Cambridge from 1933 to 1939. When hostilities started, he enlisted in the Royal Navy and worked on the development of radar at the Witley's research center. There he met the two physicists Hermann Bondi and Thomas Gold. All three passionate of cosmology, they consider with scepticism the standard model of the universe (in a philosophical point of view it was unacceptable). At the time, the standard model stumbled over a serious difficulty: an estimated Hubble age of the universe was about 2 billion years, yet geological data led to a age of the Earth at least 4 billion years. During the war and in the years following the end of hostilities, Hoyle published several studies on the theory of accretion and the theory of stellar structure, especially for giant stars and white dwarfs. The war ended, the three men returned to Cambridge, where Hoyle gets a chair of mathematics. In 1948, they expose their theory in two articles, one of Bondi and Gold, the other of Hoyle. In 1963, the first quasar is discovered. Its intrinsic brightness is much higher than any other celestial object known: it is a hundred times more luminous than any galaxy! In 1962, Hoyle and William A. Fowler had proposed a theory that could account for the huge luminosity of quasars, it was the theory of supermassive stars. Theoretical considerations can demonstrate that normal stars of masses greater than about 60 solar masses would be the seat of violent instability due to radiation pressure and the generation of nuclear energy. This hypothesis is supported by the fact that we do not observe normal stars beyond the limit of instability. Despite this argument, Hoyle and Fowler proposed the concept of a supermassive star, star that would be maintained almost entirely by radiation pressure. Thus, to achieve the brightness characteristic of a quasar, the supermassive star must have a mass of about 100 million solar masses. When the density becomes sufficiently high, a supermassive star of less than 1 million solar masses explode, while a more massive star undergoes a cataclysmic collapse and form a supermassive black holes. These two possibilities are very important for understanding quasars, and they have been studied by many researchers. Another explanation for the quasar phenomenon, suggested for the first time by Donald Lynden-Bell, suppose the accretion of matter into a supermassive black hole at the center of a galaxy (consensus currently adopted by the scientific community).

\parpic[l][t]{%
  \begin{minipage}{40mm}
    \fbox{\includegraphics[width=110px,height=140px]{img/medaillons/huygens.eps}}
  \end{minipage}
}
\textbf{Huygens, Christian} (1629-1695) was a Dutch astronomer, mathematician and physicist. Many scientific original discoveries earned him widespread recognition and honours among the eminent scientists of the 17th century. With his \textit{Treatise on Light} (1690), he is at the origin of the wave theory of light (which later took its name): each point of wave motion is itself a source of new waves. He studies in 1652 the rules set by Descartes in the \textit{ Principia philosophiae}. Building on the Cartesian conservation of momentum $mv$, he had the idea to solve algebraically the problem of shocks by comparing the quantities $mv^2$ which are introduced only for the harmony of calculations, without particular physical meaning. While discovering these quantities are conserved before and after the shock, he can write the rules in the general case, that Descartes had not done so, including conservation of momentum and kinetic energy. In 1655, he invented a method of grinding and polishing of optical lenses. The finer definition obtained allowed him to discover a satellite of Saturn and to provide the first accurate description of the rings of Saturn. The need for an accurate measurement of time for the observation of the sky led him to apply the laws of the pendulum to adjust the movements of clocks and watches. In 1656, he designed a telescope that bears his name. Between 1658 and 1659, Huygens worked on the theory of pendulum. He has indeed the idea of regulating clocks with a pendulum to make the most accurate measurement of time. He discovered the rigorous isochronism formula in 1659 when the extremity of the pendulum travels an arc of cycloid, the period of oscillation is constant regardless of the amplitude. In \textit{Horologium Oscillatorium} (1673), he determined the true relationship between the length of a pendulum and the period of oscillation, and presented his theories on centrifugal force of circular motions, which helped the English physicist Isaac Newton to formulate the laws gravity. In 1673, Huygens and his young assistant Denis Papin, highlight the principle of internal combustion engines, which will lead to the nineteenth century with the invention of the automobile. In 1678, he found the polarization of the light by the birefringent calcite.

\phantomsection
\addcontentsline{toc}{section}{I}

\parpic[l][t]{%
  \begin{minipage}{40mm}
    \fbox{\includegraphics[width=110px,height=140px]{img/medaillons/ibn.eps}}
  \end{minipage}
}
\textbf{Ibn Al Haytham }(965-1039) was an Arabic mathematician, philosopher and physicist. He is one of the fathers of quantitative physics and modern optics, the pioneer of the modern scientific method and the founder of experimental physics and some, for these reasons, described him as the first scientist. Al Haytham began his scientific career in his hometown of Basra. However, he was summoned by the caliph Hakim who wanted to control the flooding of the Nile that beated Egypt year after year. After leading an expedition in the desert to trace the source of the famous river, Al Haytham realized that this project was impossible. Back to Cairo, he feared that the caliph, who was furious at his failure, avenge and so he decided to feign madness. The caliph only assigned Al Haytham at residence. Al Haytham took advantage of this enforced leisure to write several books on various subjects such as astronomy, medicine, mathematics, scientific method and optics. The exact number of his writings is not known with certainty, but there is an approximation of a number between 80 and 200. Few of these works, in fact, have survived until today. Some of them, those on cosmology and his treatises on optics in particular, have survived only thanks to their translation to latin. Most of his research involved geometrical optics and physiology. Contrary to the popular belief, he was the first to explain why the sun and moon appear bigger (it was long believed that it was Ptolemy), he also establishes that the light of the moon comes from the sun. He also contradicted Ptolemy about that the eye emit light. For Al Haytham, if the eye really emit light we could see at night. He realized that the sunlight was diffused by the object and then entered the eye. In astronomy he attempted to measure the height of the atmosphere and found that the phenomenon of twilight is due to a phenomenon of refraction. He also spoke on the attraction of masses and it is believed that he knew the gravitational acceleration. Al Haytham was ahead a few centuries on several discoveries made by occidental scientists during the Renaissance. He was one of the first to use a scientific method of analysis and greatly influenced scientists like Roger Bacon and Kepler.

\phantomsection
\addcontentsline{toc}{section}{J}

\parpic[l][t]{%
  \begin{minipage}{40mm}
    \fbox{\includegraphics[width=110px,height=140px]{img/medaillons/jacobi.eps}}
  \end{minipage}
}
\textbf{Jacobi, Carl} (1804-1851) was born in Potsdamand and died in Berlin (Germany). He was with N. H. Abel, the founder of the theory of elliptic functions for which he gave many applications in the most various fields of mathematics. We also owed him papers on theoretical mechanics where he goes back on the results of W. R. Hamilton, and applications of the theory of differential equations to dynamics. When Jacobi entered the High school in 1816, he already completed alone the graduate level, quite refractory to traditional teaching, he studied directly works of the great mathematicians, particularly those of Euler and Lagrange. Registered in 1821 at the University of Berlin, he learned philology and mathematics, to which he devoted himself almost completely. In 1825 he obtained a PhD with a thesis in which he generalized some Lagrange's formulas. He taught in Berlin for about a year, then Koenigsberg where he was transferred by ministerial decision. End of 1827, he was appointed extraordinary professor at the Vienna University, where he came into contact with the astronomer Friedrich Wilhelm Bessel. Pensioned by the Prussian government, he was, after a trip to Italy in 1843, named Academician in Berlin, exempt from teaching but authorized to work on any subject that interested him. Presented as a candidate to the elections in 1848, he was persecuted for a time for his liberal views. Jacobi devoted many works to integrals transformations and brought an important contribution to the theory of differential equations and partial differential equations. This is to what are attached his contributions to the calculus of variations, the dynamics of solids and celestial mechanics - three-body problem, perturbations of planetary motion. Algebra owes him important research on quadratic forms and a relation with the theory of determinants that has become a classic, a prelude to his memory on the functional determinants named nowadays "Jacobians". He perfected the theory of elimination and teaches to represent the roots of an algebraic equation by definite integrals or series. He studies the common points between curves and algebraic surfaces, and found directly the number of double tangents of a plane curve, already established by J. Plücker using duality.

\parpic[l][t]{%
  \begin{minipage}{40mm}
    \fbox{\includegraphics[width=110px,height=140px]{img/medaillons/jordan_camille.jpg}}
  \end{minipage}
}
\textbf{Jordan, Camille} (1838-1921) was born in Lyon and died in Paris (France). He was the undisputed specialist in the theory of groups throughout the late 19th century and we owe him numerous famous results, both on finite groups as the so named "classics groups", which he was the first to measure the importance. His analysis courses contributed to the development of the theory of functions of a real variable. In 1855, at the age of seventeen, he is received as best student at the École Polytechnique and finished the École des Mines in 1861. He will be, at least officially, engineer responsible for overseeing the Carrières de Paris until 1885, which will not prevent him from an intense mathematical research. Appointed examiner at the École Polytechnique in 1873 and professor in 1876, he entered the Académie des Sciences in 1881 and succeeded to Joseph Liouville at the Collège de France two years later. From 1885 to 1921, he assumed the management of the Journal de Mathématiques pures et appliquées founded by Liouville. Despite the efforts of Liouville, the work of Evariste Galois remained almost totally unknown to the world of mathematics (Leopold Kronecker had only used some of his results), and this is Jordan with his \textit{Treatise of algebraic substitutions and equations}, published in Paris in 1870, that we owe the first systematic presentation of group theory, enriched with ten years of personal research. Jordan limits his study to the finite groups, specifically the groups of permutations, and introduced many new concepts. In later submissions, Jordan studied in detail, mainly in terms of the factors of composition, the linear and orthogonal group and symplectic groups on a prime body. Jordan studies on the linear group involve considerations of matrix reduction, and in particular, on the shape named "Jordan form". Finally, we note the efforts of Jordan to determine all finite solvable groups in response to the problem posed by Niels Henrik Abel, to find all given degree equations solvable by radicals. In addition to the results given above for the linear group, we owe to Jordan a complete study of the real Euclidean $n$-dimensional geometries using entirely analytical methods. The teaching of Jordan at the École Polytechnique and the Collège de France, leads him to clarify many concepts of the theory of functions of real variable and his \textit{Cours d'analyse de Polytechnique} (first edition in 1880) will help to train generations of mathematicians. We also owe him the concept of a function with bounded variation, which allows him to give a correct definition of the length of a curve and to obtain the general form of the theorem of convergence for Fourier series, but the most famous result is one that says that a simple closed curve (known nowadays, "Jordan curve") divides the plane into two regions.  We also owe him a classical proof of Euler's theorem on polyhedra and the fact that two surfaces of the same kind are applicable to the other one (which, as shown by Poincaré, is not generally true for hypersurfaces).

\parpic[l][t]{%
  \begin{minipage}{40mm}
    \fbox{\includegraphics[width=110px,height=140px]{img/medaillons/jordanp.eps}}
  \end{minipage}
}
\textbf{Jordan, Pascual} (1902-1980) was born in Hanover and died in Hamburg (Germany). He was a theoretical physicist, professor at the University of Göttingen. Jordan passed in 1921 a part of his studies at the Hanover Teknische Universität where he studied a mixture of zoology, mathematics and physics. In 1923 he specialized when he entered at Göttingen Universität, who was then at its zenith from the point of view of mathematics and physics. In Göttingen, Jordan became the assistant to Richard Courant and especially Max Born, who greatly influenced him. He contributed decisively to the foundation of quantum mechanics and quantum field theory. Because of its affiliation with the Nazi party, he was, however, rejected by the physicists community. In 1925, with Max Born, Jordan wrote the canonical commutation relation between momentum and position. In the same article, he also offers the idea that we must also quantify the electromagnetic field, paving the way to quantum field theory. Also in 1925 with Max Born and Werner Heisenberg, Jordan develops the Heisenberg's matrix formulation of quantum mechanics. They introduce the canonical transformations, perturbation theory, the treatment of degenerate systems, and finally the famous canonical commutation relation of the components of angular momentum.

\parpic[l][t]{%
  \begin{minipage}{40mm}
    \fbox{\includegraphics[width=110px,height=140px]{img/medaillons/joule.eps}}
  \end{minipage}
}
\textbf{Joule, James Prescott} (1818-1889) was a  British physicist, born in Salford, Lancashire, and died in Sale. He was one of the greatest physicists of his time. Joule is famous for his research in electricity and thermodynamics. During his research on the heat emitted by an electrical circuit, he formulated the law, known always today as "Joule's law" on heat supply, which indicates that the amount of heat generated per second in a conductor by the passage of electric current is proportional to the electrical resistance of the conductor and the square of the electric current. Joule experimentally verified the law of conservation of energy in his study of the transformation of mechanical energy into thermal energy (relation between joules and calories: it takes 1 calorie or 4.18 joules to raise 1 gram of water of 1 degree).

\phantomsection
\addcontentsline{toc}{section}{K}
\parpic[l][t]{%
  \begin{minipage}{40mm}
    \fbox{\includegraphics[width=110px,height=140px]{img/medaillons/kepler.eps}}
  \end{minipage}
}
\textbf{Kepler, Johannes} (1571-1630) was German astronomer and physicist, famous for his formulation and verification of the three laws of planetary motion. These laws are still known as "Kepler's laws". His main treatise contains the formulations of two laws of planetary motion. The first states that the planets move in elliptical orbits with the Sun as focal point and the second, or "area law" states that the imaginary line that we would trace between the Sun and a planet sweeps out equal areas of an ellipse during equal intervals of time, in other words, the more the planet approaches the Sun, the more quickly it moves. Another treaty contains another discovery of planetary motion: the cube of the distance between a planet and the Sun divided by the squared orbital period of this planet is a constant and is the same for all planets. The English mathematician and physicist Isaac Newton strongly based on the theories and observations of Kepler to formulate his theory of gravity. Kepler also brought his contribution in the field of optics and developed in mathematics an infinitesimal system which was the precursor of infinitesimal calculus.

\parpic[l][t]{%
  \begin{minipage}{40mm}
    \fbox{\includegraphics[width=110px,height=140px]{img/medaillons/keynes.eps}}
  \end{minipage}
}
\textbf{Keynes, John Maynard} (1883-1946) was a British economist. He is the founder of the "Keynesian" economic theory that promotes government intervention in the economy to ensure full employment. Keynes was born into a family of academics. At the age of seven, he entered the Perse School. Two years later, he entered preparatory class at St Faith's. Over the years, he showed great dispositions and in 1894, he finished first in his class and received an award for the first time in mathematics. A year later, he joined the Eton's College where he shines and wins in 1899 and 1900, the price of mathematics. In 1901, he finished first in mathematics, history and English. In 1902, he earned his place for the Cambridge King's College. Keynes is undoubtedly an important figure in the history of economic science that he revolutionized with his main work, The\textit{ General Theory of Employment, Interest and Money}, published in 1936. The book is considered as the most influential treaty of social science's in the 20th century because because it has rapidly and continuously changed the way the world viewed the economy and the role of political power in society. Some believe that no other book has had such importance for Europe, even if the book of Friedrich Hayek, who received a Nobel Prize, \textit{The Road to Serfdom}, make a dramatic demonstration of the limits of Keynesian theory. With the \textit{General Theory}, Keynes developed a theory that could explain the level of production and hence this of employment; the determining factor being the demand. Among the revolutionary concepts introduced by Keynes, we note: those of underemployment equilibrium where unemployment is possible for a given level of effective demand, the absence of a regulatory mechanism for prices to reduce unemployment, a theory of money based on the preference for liquidity, the introduction of uncertainty and forecasts, the concept of marginal efficiency of investment breaking Say's Law (and therefore reversing the causal savings-investment relation). These concepts accredit interventionist policies to eliminate recessions and slow down economic overheating. All of these concepts are now named "Macroeconomics".

\parpic[l][t]{%
  \begin{minipage}{40mm}
    \fbox{\includegraphics[width=110px,height=140px]{img/medaillons/kirchhoff.eps}}
  \end{minipage}
}
\textbf{Kirchhoff, Gustav Robert} (1824-1887) was born in Könisberg (now Kaliningrad, Russia) and died in Berlin (Germany). Kirchhoff studied mathematical physics with Franz Neumann. After a Ph.D. in 1847, he became a lecturer at the Berlin Universität  before obtaining, in 1850, the position of extraordinary professor of physics at the Breslau Universität. This is where he met the chemist Robert Wilhelm Bunsen, with whom he will be working for many years. Their collaboration will continue beyond 1854, when Kirchhoff was appointed professor of physics at the  Heidelberg Universität. Elected vice-president of the same university in 1865, he finally accepted a professorship in theoretical physics at Berlin in 1875. Kirchhoff was still a student when he began to take an interest in issues related to electricity. In 1845, he established the concept of electric potential and sets the laws of networks that bear still bear his name today (Kirchhoff's laws). He also generalizes Ohm's law on the electric current of three dimensional conductors and, later, shows that the flow of current through a conductor occurs at the speed of light. His relation with Bunsen led to the birth of spectroscopy. Together, the two researchers discover the specific nature of the spectrum of light emitted by each chemistry body. With this new analysis tool, they track down two unknown elements: cesium (1860) and rubidium (1861). The development of prism spectroscope to analyse the light burning substances, also allows to establish Kirchhoff's radiation law: the ratio of powers of absorption and emission of a body, independent of the properties of this body is a function of temperature and wavelength. The emittance is thus proportional to that of the "black body" defined by Kirchhoff (1862) as the perfect absorbent body. This law, which explains the presence of such dark lines of absorption (called "Fraunhofer lines") in the spectrum of solar radiation, marks the beginning of a new era in astrophysics and announce the beginning of Planck's quantum theory.

\parpic[l][t]{%
  \begin{minipage}{40mm}
    \fbox{\includegraphics[width=110px,height=140px]{img/medaillons/klein.eps}}
  \end{minipage}
}
\textbf{Klein, Felix} (1849-1925) made his studies at Bonn, Göttingen and Berlin (Germany). In 1872 he became professor of mathematics at the University of Erlangen, where his inaugural lecture was the statement of the outline of his famous Erlangen Program. He then taught at Munich (1875-1880), then at the Leipzig Universität (1880-1886) and finally Göttingen (1886-1913). From 1872, he published the \textit{Mathematische Annalen of Goettingen} and founded in 1895, the great \textit{Mathematical Encyclopedia}, he oversaw the writing until his death in Göttingen. He was the undisputed leader of the German school of mathematics, and his influence was great (he gave numerous lectures in foreign countries, including in the United States), especially on the development of geometry, with his Erlangen's program. With the text, published in his book\textit{ Gesammelte mathematische Abhandlungen} (1921-1923), Klein gives a definition of the geometry including both classical geometry (that is to say, Euclidean) and projective geometry, non-Euclidean geometries, etc., ending the sterile controversies between supporters of those synthetic geometry and analytic geometry. For him, a geometry is the study of invariant properties under a given group of transformations: in this way theorems of classical geometry are the expression of invariants relations of the group of similarities, those of projective geometry between covariants of the projective group. We are indebted to Klein extensive works on the hypergeometric differential equation, on Abelian functions on the group of the regular icosahedron (\textit{Lectures on the Icosahedron}, 1914), on elliptic functions, from which he emerged the notion modular function (\textit{Vorlesungen über die Theorie der Funktionen automorphen}, 1897-1902).

\parpic[l][t]{%
  \begin{minipage}{40mm}
    \fbox{\includegraphics[width=110px,height=140px]{img/medaillons/kolmogorov.eps}}
  \end{minipage}
}
\textbf{Kolmogorov, Andrei} (1903-1987) was a Russian mathematician whose contributions are significant in mathematics. Kolmogorov was born at Tambov. His single mother died at his birth and he was educated by his aunt with the savings of his grandfather. Is is supposed that his father was killed during the Russian Civil War. Kolmogorov was educated at the village school of his aunt, and his first literary efforts and mathematical papers were printed in the school newspaper. Teenager, he designed perpetual motion machines, hiding their intrinsic defects so well that its secondary school teachers could not discover the tricks. In 1910, he was adopted by his aunt and they moved to Moscow, where he entered a Gymnasium and graduated there in 1920. After completing his secondary education, he studied at the University of Moscow and Mendeleev Institute. He studied not only mathematics, but also russian history and metallurgy. In 1922, Kolmogorov published his first results on the theory of sets, and in 1923 he published his work on the theory of integration, Fourier analysis and for the first time on probability theory and is starting to become known abroad. After completing his studies in 1925, he started his Ph.D. with Nikolai Louzine, which he completed in 1929. In 1931, he received a professorship at the University of Moscow. In 1933, published in German, his manual \textit{Fundamentals of probability theory} in which he presents his axiomatization of probability and an appropriate method to treat stochastic processes. The same year, he became director of the Institute of Mathematics of the University of Moscow. In 1934, he published his work on the cohomology and gets, thanks to this thesis, a Ph.D. in mathematics and physics. He gets prizes from Soviet authorities, such as the Order of socialist science (1940), the Stalin Prize (1941) and Lenin Prize several times. In 1941, he developed a famous theory of fluid turbulence. In 1953 and 1954, he describes the KAM theory (Kolmogorov-Arnold-Moser) stability of dynamical systems (a complex mechanical systems exactly solvable is stable only if we disturbs it a little bit). He also introduces the notion of metric entropy for measured dynamic systems. In 1955, he became an honorary doctorate from the Sorbonne (France). In 1962, he awarded the Balzan Prize for mathematics.

\parpic[l][t]{%
  \begin{minipage}{40mm}
    \fbox{\includegraphics[width=110px,height=140px]{img/medaillons/kronecker.eps}}
  \end{minipage}
}
\textbf{Kronecker, Leopold} (1823-1891) was a German mathematician who appears as one of the greatest number theorists of the 19th century and one of the founders of the theory of algebraic numbers. His works on particular class fields prepared those of Hilbert. Born in Liegnitz, in a family of wealthy merchants, Kronecker followed at the gymnasium the courses of Ernst Kummer, who he was to meet later as a professor at the Breslau Universität, then as a colleague in Berlin. Peter Gustav Lejeune-Dirichlet and Ernst Kummer have had a profound influence on the development of his thought. Having argued, in 1845, a highly original theory of cyclotomic units he held for several years, family affairs, and could not deliver entirely new mathematical research until 1853. Elected in 1860 member of the Academy of Sciences in Berlin, he gave, from that time, free courses at the university, where he was appointed professor in 1883 and where he ended his life. Instead wielding virtuosity with all the resources of the analysis (as show his works on elliptic functions, Dirichlet series or even the integral formula giving the number of roots of a system of equations in n-dimensional space ), Kronecker is before all an algebraist and arithmetician. Towards the end of his life, he professed a doctrine to reject the actual infinite in mathematics as valid only keeping what could only be based on integers (his polemics with Cantor remained famous). In algebra, Kronecker was one of the most active leaders of the group of mathematicians who, in the years 1860-1890, succeeded to develop linear and multilinear algebra inaugurated by Arthur Cayley and Hermann Grassmann around 1845. So he went and completed the works of Karl Weierstrass and was one of the first to understand and use the work of Evariste Galois (published in 1846).

\phantomsection
\addcontentsline{toc}{section}{L}

\parpic[l][t]{%
  \begin{minipage}{40mm}
    \fbox{\includegraphics[width=110px,height=140px]{img/medaillons/lagrange.eps}}
  \end{minipage}
}
\textbf{Lagrange, Joseph Louis} (1736-1813) was born in Turin and died in Paris. He was as one of the greatest mathematician and astronomer of the 18th century. Brilliant student from a wealthy family, he studied at the College of Turin. He takes a liking for mathematics by chance at the age of seventeen after reading a paper by Edmund Halley on applications of algebra in optics. The subject interests him at the highest point. Therefore, he is passionate about mathematics that he studied diligently and alone. He quickly became a confirmed mathematician and his first important results arrive quickly. In a letter to Leonhard Euler he laid the foundations of the calculus of variations. This exchange is the beginning of a long correspondence between the two men. Lagrange was then ninteen years old and teaches at the Artillery School in Turin where he was appointed in 1755. He founded in 1758 the Academy of Sciences of Turin which will publish his first results on the application of variational calculus to mechanical problems (sound propagation, vibrating string ...). In 1764, his work on the libration of the Moon (small variations in its orbit) are awarded by the Grand Prix de l'Académie des Sciences in Paris. He introduced new methods for the calculus of variations and the study of differential equations, which enabled him to give a systematic presentation of mechanics in his famous book \textit{Analytical Mechanics} (1788). He worked on additive number theory. We owe him the theorem on the decomposition of an integer into four squares. In the study of algebraic equations, he introduced the concepts that lead to group theory later developed by Abel and Galois. In physics, precising the principle of least action, with the calculus of variations, about 1756, he invented the Lagrange function, which verifies the Lagrange equations, then he develops analytical mechanics, about 1788, where he introduced the Lagrange multipliers. He also undertakes extensive research on the three-body problem in astronomy, one of the results being the highlight of the libration points still today named "Lagrange points" (1772).

\parpic[l][t]{%
  \begin{minipage}{40mm}
    \fbox{\includegraphics[width=110px,height=140px]{img/medaillons/langevin.eps}}
  \end{minipage}
}
\textbf{Langevin, Paul} (1872-1946) was a French physicist born and died in Paris (France). Very young Langevin manifest exceptional gifts. Encouraged by his teachers, he quickly went trough the various levels of obligatory education before entering at the age of sixteen at the École Supérieur de Chimie et Physique Industrielle de Paris. Langevin there follows the laboratory courses and teaching of Pierre Curie, with whom he became friends. On leaving the school, he abandoned a career as an engineer and decided, on the advice of Pierre Curie, to focus on research and teaching. Also, he postulated for a job position at the École Normale Supérieure where he was received first in 1894. In 1897 he received a scholarship to go to work one year at the Cavendish Laboratory of Cambridge, high center of European science where E. Rutherford and J. J. Thomson work. Back in France, he defended his thesis in 1902, was appointed deputy professor, then professor at the Collège de France. In 1904, he succeeded Pierre Curie at the School of Physics and Chemistry, where he became director in 1925. Langevin's work lies in this long transition period from 1900 to 1930 that leads from classical physics to modern physics dominated by relativity theory and quantum theory. His first work (on the ionization of gases) led him to develop his main theoretical model in 1905, which should then form the basis for many other explanations of macroscopic properties of matter, in which electrons within atoms define closed orbits, thereby conferring atoms properties similar to those of small magnets. In 1906 he founded the surprising result that inertia is a property of energy... at least in the case of the electron. It is only a few months later that he will read the Einstein's memory on theory of relativity which he will devote his teaching during his courses at the Collège de France. Langevin is also at the origin of the famous Solvay Conference that, starting from 1911, met periodically all the great names of physics, where the concepts of quantum theory were widely discussed. It is thanks to Langeving that the work of his pupil Louis de Broglie on wave mechanics knew the diffusion that deserved him: first surprised, Langevin was quickly convinced of the correctness of De Broglie's ideas and planned immediately the new wave mechanics program's lectures at the Collège de France. Faithful to the ideal of teaching clarity, Langevin has also conducted, on the concepts still being developed quantum theory, a job and redesign analysis for which we always measure today the epistemological significance.

\parpic[l][t]{%
  \begin{minipage}{40mm}
    \fbox{\includegraphics[width=110px,height=140px]{img/medaillons/langevin.eps}}
  \end{minipage}
}
\textbf{Laplace, Pierre Simon} (1749-1827) was born in Beaumont-en-Auge and died in Paris (France). Son of a farmer, Laplace was initiated in mathematics at the military school of Beaumont-en-Auge and began there his teaching. He was able to follow this education thanks to his affluent neighbours who detected his exceptional intelligence. At the age of eigthenn, he arrived in Paris with a letter of recommendation to meet the mathematician d'Alembert, who refuses. But Laplace insists and sends to d'Alembert an article he wrote on classical mechanics. D'Alembert is so impressed that he is happy to sponsor Laplace and found him teaching math job position. The most important work of Laplace is about probability calculus, differential equations (laplace operator) and celestial mechanics. He also establishes, through his work with Lavoisier between 1782 and 1784 the relation of adiabatic transformations of a gas, as well as two fundamental laws of electromagnetism. In Mechanics, it is with the mathematician Joseph-Louis Lagrange, that Laplace summarizes his work and merged those of Newton, Halley, Clairaut, d'Alembert and Euler, on universal gravitation (especially the problem of stability of the solar system) in the five volumes of his Celestial mechanics (1798-1825). It is reported (but it is most likely a legend) that reading \textit{Celestial mechanics}, Napoleon remarked that there was no mention of God. «I do not need this hypothesis», replied Laplace who was not otherwise modest (considering himself - probably rightly - as the best mathematician of his generation). He is also one of the first scientists to conceive the existence of black holes and the notion of gravitational collapse.

\parpic[l][t]{%
  \begin{minipage}{40mm}
    \fbox{\includegraphics[width=110px,height=140px]{img/medaillons/anonymous.eps}}
  \end{minipage}
}
\textbf{Laurent, Pierre Alphonse} (1813-1854) was a French mathematician born in Paris and who became famous for the discovery of the Laurent series in complex analysis that has a great impact in the calculation of certain integrals in physics. He entered the École Polytechnique de Paris in 1830. Laurent was graduated in 1832 as one of the best students of the year and entered the engineering corps as a lieutenant. During the management of the development projects of the port of Le Havre, Lawrence wrote his first mathematical publication on Laurent series. This research was contained in a memorandum submitted to the Grand Prix de l'Académie des Sciences in 1843, but his application was too late, the article has not been included in the price. However, Cauchy made a reference in his works to Laurent's paper three months later. The same problem occurred again for another major publication of Laurent a few months later. After these events, Laurent, disappointed changed his research field to focus on physics (Applied Mathematics). Cauchy offered him a vacancy job at the Academy of Sciences in 1846, but his application was not accepted. Laurent died in Paris at the age of fourty-one. His writings were published after his death.

\parpic[l][t]{%
  \begin{minipage}{40mm}
    \fbox{\includegraphics[width=110px,height=140px]{img/medaillons/lavoisier.eps}}
  \end{minipage}
}
\textbf{Lavoisier, Antoine Laurent} (1743-1794), was a French chemist know as the founder of modern chemistry. Lavoisier was born in Paris and studied at the Collège Mazarin. He was elected member of the Academie des Sciences in 1768. He held several positions, including Director of Poudreries Nationales in 1776, member of the Commission pour l'établissement du nouveau système de poids et mesures in 1790 and Secretary of the Trésorerie in 1791. He tried to introduce reforms in the French monetary and fiscal policy, as well as in the agricultural system. Lavoisier was one of the first to realize truly quantitative chemical experiments. He showed that despite the change of state of the material in a chemical reaction, the amount of material remained constant between the start and the end of each reaction. These experiments have provided evidence in favour of the law of conservation of matter. Lavoisier also did research on the composition of the water, which he named the components: "oxygen" and "hydrogen". One of the most important experiments of Lavoisier was about the nature of the combustion (or burning). He demonstrated that the combustion process implies the presence of oxygen. He also demonstrated the role of oxygen in the respiration of animals and plants. Lavoisier's explanation of combustion replaced the doctrine of phlogiston. This indeed postulated that a substance emerged, the "phlogiston", when the material is consumed. As one of twenty-eight general farmers, Lavoisier is stupidly branded as a traitor by the revolutionists in 1794 and guillotined during the Terror in Paris in 1794, at the age of 50 years, along with all colleagues.

\parpic[l][t]{%
  \begin{minipage}{40mm}
    \fbox{\includegraphics[width=110px,height=140px]{img/medaillons/lebesgue.eps}}
  \end{minipage}
}
\textbf{Lebesgue, Henri Léon} (1875-1941) was born in Beauvais and died in Paris (France) is a former student of the École Normale Supérieure, he had Émile Borel as teacher (who we own the first major work in measure theory). After a few years in the high school of Nancy, Lebesgue will teach at Rennes. It was during this period that he will be known for his elegant theory of measurement. Professor at the Sorbonne and the College de France, he was elected to the Academie des Sciences in 1922. By his theory of measurable functions (1901) based on the Borel tribe (named after the mathematician Émile Borel), Lebesgue extensively revised and generalized integral calculus. His theory of integration (1902-1904) addresses the needs of physicists to the research and the existence of primitive for "irregular" functions. We owe him also the Fourier transform established in the late 30s. He was appointed professor at the Sorbonne in 1910 and at the Collège de France in 1921. He also teaches at the École de Physique Industrielle et Chimie of Paris from 1927 to 1937 and at the École Normale Supérieure of Sèvres.

\parpic[l][t]{%
  \begin{minipage}{40mm}
    \fbox{\includegraphics[width=110px,height=140px]{img/medaillons/lee.eps}}
  \end{minipage}
}
\textbf{Lee Tsung-Dao} (1926-) is born in Shanghai (China). He is the son of a businessman. The Sino-Japanese War of 1937-1945 made him leave the Kweichow University in the province of Zhejiang to join that of Kunming in Province of Yunnan, where he met Yang Chen-Ning, which will be a long time his friend and collaborator. A Chinese government scholarship enabled him to finish his studies at the University of Chicago (USA), where he defended his thesis on the hydrogen content of white dwarfs in 1950. Member of the Institute for Advanced Study in Princeton (New Jersey) from 1951 to 1953, he soon became, at 29 years old, the youngest professor at Columbia University in New York. In 1956, physicists were subjected to a puzzle emerged from analysis of the data provided by the particle accelerator at Brookhaven National Laboratory, near New York, two particles, called "tau" and "theta", seemed to have the same mass and even nuclear interactions, but differed in their decay products. Lee and Yang proposed that they were a single particle, now denoted "$K_0$", and that the weak interaction responsible for the decay does not respect parity symmetry. They concluded that it was necessary to submit to experimental verification that the weak interaction distinguishes right from left. Six months sufficed for the team from the National Bureau of Standards in Washington, mobilized by the Chinese physicist Chien-Shiung Wu, to show that radioactive Cobalt-60 polarized nuclei emitted more electrons in one direction than in the opposite direction. Quickly confirmed by several other experimental groups, the violation of mirror symmetry earned Tsung-Dao Lee and Chen Ning Yang to share the Nobel Prize in Physics 1957.

\parpic[l][t]{%
  \begin{minipage}{40mm}
    \fbox{\includegraphics[width=110px,height=140px]{img/medaillons/anonymous.eps}}
  \end{minipage}
}
\textbf{Legendre, Adrien Marie} (1752-1833) was French mathematician born in Paris and died in Auteil. He holds the Chair of Mathematics at the École Militaire of Paris from 1775 to 1780. In 1783, he became a member of the Académie des Sciences. In 1787, he was appointed Commissioner for geodetic operations. Legendre interests were varied: analysis, number theory, geometry, statistics (least squares methods) and mechanics (Legendre transform in analytical mechanics and thermodynamics). About a century before we get evidence, he conjectured the prime number theorem and the law of quadratic reciprocity. Throughout his life, he became interested in elliptic integrals, whose work would eventually give rise to elliptic curves, subject studied a lot by contemporary mathematicians. He lets as heritage to the mathematical community of the 19th century a treatise on elementary geometry, which is very precious in the world of education.

\parpic[l][t]{%
  \begin{minipage}{40mm}
    \fbox{\includegraphics[width=110px,height=140px]{img/medaillons/leibniz.eps}}
  \end{minipage}
}
\textbf{Leibniz, Gottfried Wilhelm} (1646-1716) was born in Leipzig and died in Hanover. He was a philosopher, mathematician, lawyer and considered as one of the most brilliant minds of the 17th century. Son of a lawyer he graduated in 1663 in ancient philosophy and later wrote a theory of probability in Law... He then entered the Leipzig Universität and in 1666 obtained his Ph.D. as Lawyer. In 1669 he became an adviser to the Chancellor of the electorate of Mainz. He was sent to Paris in 1672, for a supposed diplomatic mission, to convince Louis XIV shift his conquests to Egypt rather than Germany. He stayed there until 1676 and met the great scientific of this time. It was during this period that Leibniz worked on his main scientific topics. In 1676 he was appointed librarian of Brunswick-Luneburg and also managed mathematics, physics, religion and diplomacy. Leibniz contributed to mathematics by discovering, in 1675, the fundamentals of infinitesimal calculus. This discovery was made independently of the discoveries of Newton, who invented the system of infinitesimal calculation in 1666. Leibniz system was published in 1684, that of Newton in 1687, that's when the notation imagined by Leibniz was adopted and he is also considered as a pioneer in the development of mathematical logic.

\parpic[l][t]{%
  \begin{minipage}{40mm}
    \fbox{\includegraphics[width=110px,height=140px]{img/medaillons/landau.eps}}
  \end{minipage}
}
\textbf{Landau, Lev Davidovich} (1908-1968) was born in Azerbaijan and died in Moscow. He was the son of an engineer and a doctor. After completing his studies at the Physics Department at the University of Leningrad at the age of ninteen, he began his scientific career at the Institute of Technical Physcis at Leningrad. From 1932 to 1937 he was the Head of the Theoretical Technical Physics  Institute in Kharkov (Ukrain) and in 1937 he was appointed head of the Department of Theoretical Institute for Physical Problems at the USSR Academy of Sciences of Moscow. Landau's work covers all branches of theoretical physics at the limits of fluid mechanics to quantum field theory. Much of his papers refers to the theory of condensed state. They started in 1936 by a formulation of a general theory of phase transitions of the second order. After the discovery of Kapitsa, in 1938, the superfluidity of liquid helium, Landau has initiated extensive research that has led to the construction of the complete theory of quantum liquids at very low temperatures. Among his writings, covering a wide range of topics related to physical phenomena, there are more than one hundred articles and several books, including the famous \textit{Course of Theoretical Physics}, published in 1943 with E.M. Lifchitz. Landau has dominated the theoretical physics from 1930 to 1965. He created a series of tests of theoretical physics, named the "theoretical minimum" that students or senior researchers had to pass to get into his research group, which included examination of problems in all branches of mathematics.

\parpic[l][t]{%
  \begin{minipage}{40mm}
    \fbox{\includegraphics[width=110px,height=140px]{img/medaillons/levicivita.eps}}
  \end{minipage}
}
\textbf{Levi-Civita, Tullio} (1873-1941) was born in Padua and died in Rome. He graduated in 1892 from the Faculty of Mathematics of the University of Padua. In 1894, he obtained a teaching degree at the College of Education of the Faculty of Pavia. In 1898, he was appointed head of the chair of celestial and analytical mechanics of Padua where he met Libera Trevisani, one of his students, whom he married in 1914. He remained in Padua until 1918, then was appointed to the chair of analysis at the University of Rome, where he tooks two years later the chair of professor of mechanical engineering. Foremost physicist, his works are mainly related to electromagnetism and to the theories of Lorentz and Maxwell. In 1900, he published with Ricci his Theory of tensors in the methods of differential calculus and their applications that Einstein used to better control the tensor calculus, a key tool for the development of his theory of General Relativity. Levi-Civita also discussed a series of issues about the static gravitational field in his correspondence with Einstein in the years 1915-1917. Their correspondence revolved around the variational formulation of the equations of gravitational fields and their covariant properties, and the definition of the gravitational energy and the existence of gravitational waves. Levi-Civita in 1933 also contributed to the Dirac equations of quantum mechanics.

\parpic[l][t]{%
  \begin{minipage}{40mm}
    \fbox{\includegraphics[width=110px,height=140px]{img/medaillons/lie.eps}}
  \end{minipage}
}
\textbf{Lie, Sophus} (1842-1899) was a Norwegian mathematician educated at the University of Christiana. He gave private lessons to earn money, and spent the winter of 1869-1870 with Klein in Berlin, the summer of 1870 in Paris. In 1872, a mathematics chair was created for him at Christiana, and in 1886 he succeeded Klein in Leipzig. In addition to his work in projective geometry of space,we retain from Lie his studies on new algebraic structures that he applies to geometry, until the creation of the theory of groups and algebras that bear his name. In the concept of Lie group and algebra, involved continuity properties (topological group), announcing the new important branch of mathematics that will be the topology. Lie's work in this area will be mainly pursued by Élie Cartan.

\parpic[l][t]{%
  \begin{minipage}{40mm}
    \fbox{\includegraphics[width=110px,height=140px]{img/medaillons/lindemann.eps}}
  \end{minipage}
}
\textbf{Lindemann, Ferdinand} (1852-1939) was born in Hanover and died in Munich. He was the first mathematician to prove the transcendence of $\pi$. When Ferdinand was two years old he moved to Schwerin where he spent his childhood and primary schooling. As it was the tradition at that time in Germany during the second half of the 19th century, Lindemann moved frequently from one university to another. He began his studies at Göttingen in 1870, where he was greatly influenced by Clebsch. Later, Lindemann who had established very good relations with Clebsch wrote again his geometry lecture notes after his death for their publication in 1876. Then Lindemann studied at Erlangen in Munich where he did his Ph.D. work under the direction of Klein on non-Euclidean geometries and applications to physics. After obtaining his Ph.D., Lindemann made important visits to French and English centers of mathematics. In England, he visited Oxford, Cambridge and London, while in France, he spent most of his time in Paris where he was greatly influenced by Chasles, Bertrand, Jordan and Hermite. When he returned to Germany, Lindemann worked on publications subjects to reintegrate and obtain the recognition of the German scientific community. In 1877 he was finally nominated professor at the University of Würzburg and Professor at the University of Freiburg in 1879. The main work carried by Lindemann was on geometry and analysis and is particularly known for his famous proof of transcendence. In 1873, when Lindemann had just received his Ph.D., Hermite proved the transcendence of the Euler number. Shortly after, Lindemann met Hermite in Paris and discussed the methods used for the demonstration. Thus, using a similar reasoning, Lindemann proved in 1882 the transcendence of $\pi$ (based on the fact that the Euler number is itself transcendent).

\parpic[l][t]{%
  \begin{minipage}{40mm}
    \fbox{\includegraphics[width=110px,height=140px]{img/medaillons/liouville.eps}}
  \end{minipage}
}
\textbf{Liouville, Joseph} (1809-1882) was born in St-Omer and died in Paris. He an active author for the deployment of mathematics and had a considerable activity in the teaching and dissemination of mathematical ideas of his time. He is the founder of the Journal de Mathématiques Pures traditionally named "Journal de Liouville". His main research focuses on the analysis and we owe him an important theorem on the approximation of algebraic irrational. The election of Joseph Liouville in the Constituent Assembly of 1848 is the only event that break the unity of his whole scientific career: He finished the École Polytechnique in 1827, then he returned there in 1833 as a coach and teacher on Analysis. At the age of 31, he was elected to the Academie des Sciences in the section of astronomy. He was one of the best teachers of his time, and his lectures at the École Polytechnique and the Collège de France, took a large part of its time. Liouville founded the Journal de Mathématiques Pures in 1836 and managed it during 39 years. Its academic and editor tasks, for which he complained, stripped him the necessary freedom of mind for thorough research. But he took advantage of the one and the other task to help several young mathematicians with a great future, eg C. Hermite and C. Jordan by glowing reports at the Academy, or the publication of their work in its journal. Meanwhile, he published mostly short notes on a number of issues: analysis, arithmetic, geometry, mechanics, astronomy. He shares with A. Cauchy the merit of having submitted analysis to strict rules often violated in the 18th century, and this merit is even higher as the mathematical language level of his time was not helping...

\parpic[l][t]{%
  \begin{minipage}{40mm}
    \fbox{\includegraphics[width=110px,height=140px]{img/medaillons/lobatchevski.eps}}
  \end{minipage}
}
\textbf{Lobachevsky, Nikolai Ivanovich} (1792-1856) was a Russian mathematician born in Nizhny-Novgorod and died in Kazan. Lobatchevski studied at the Kazan University, where he taught from 1812 and occupied the chair of pure mathematics from 1822 to 1846. Under the influence of Gauss and Laplace, his first works are: \textit{Theory of elliptical motion of celestial bodies} and \textit{On the solution of the simple complex algebraic equation}. But his main research concerns the geometry. His first book, \textit{Geometry} (1823), considered too revolutionary (he used the metric system), won't be published during his lifetime. In 1826 Lobachevsky exposed to his colleagues from the university a memory that shows that he was one of the first mathematicians to be convinced of the possibility of a different geometry than this of Euclid. Despite the scepticism of his colleagues, he continues to study this new geometry (where the Euclidean postulate is replaced by "Lobachevsky postulate": from any point outside a line, it goes an infinite number of parallel to this line) and devotes his life as a mathematician trying to convince the scientific world. He published successively \textit{Elements of Geometry} (1829), \textit{New Elements of geometry with full theory of parallels} (1838) and \textit{Pangeometry} (1855). But the full recognition of the value of his work will come after his death (when Eugenio Beltrami in 1868, built a model of the Lobachevsky geometry: the pseudo-sphere). In addition to his mathematical research, Lobachevsky was the host of the Kazan University: Rector from 1827 to 1846, he was in charge of the university library, set up his observatory, managed the museum and directed the construction of new university building structures.

\parpic[l][t]{%
  \begin{minipage}{40mm}
    \fbox{\includegraphics[width=110px,height=140px]{img/medaillons/lorentz.eps}}
  \end{minipage}
}
\textbf{Lorentz, Hendrik} (1853-1928) was born in Arnhem and died at Haarlem (Netherlands). Lorentz has improved the Maxwell's electromagnetic theory in his Ph.D. thesis on the theory of reflection and refraction of light which he presented in 1875. He was appointed professor of mathematical physics at the University of Leiden in 1878. He remained in this establishment until 1912 where Ehrenfest was appointed in his place. Lorentz was then appointed Director of Research at the Teyler's Institute of Teyler (Haarlem). He held an honorary position in Leiden, where he continued to give some courses. Before the existence of electrons was proved, Lorentz proposed that light waves were due to oscillations of electric charge in the atom. Lorentz developed his mathematical theory of the electron for which he received jointly with Zeeman (one of his students) the Nobel Prize in 1902.  Zeeman has experimentally verified the theoretical work of Lorentz on atomic structure, showing the effect of a strong magnetic field on the oscillations by measuring the change of the wavelength of the light produced. Lorentz is also famous for his work on the Fitzgerald-Lorentz contraction, a contraction in the length of an object at relativistic speeds. The Lorentz transformations, which he presented in 1904, form the basis of the special theory of relativity of Einstein, that was at the beginning named "Einstein-Lorentz special theory of relativity". They describe the increase of the mass, the shortening of the length, and the time dilation of a body moving at speeds near that of light. Lorentz was chairman of the first Solvay Conference that held in Brussels in autumn 1911. This conference was about the two approaches of the atomic theory, namely the classical theory and quantum physics. However, Lorentz never fully accepted quantum theory and has always hoped it would be possible to incorporate it back into the classical approach.

\parpic[l][t]{%
  \begin{minipage}{40mm}
    \fbox{\includegraphics[width=110px,height=140px]{img/medaillons/lucas.eps}}
  \end{minipage}
}
\textbf{Lucas, Edward} (1842-1891) was a French arithmetician born in Amiens and died in Paris. Child from a very modest family, he received a scholarship and passes the entrance examination at the École Normale Supérieure in 1861. On leaving the school, he became assistant astronomer at the Obsérvatoire de Paris, and after the Franco-Prussian war, he obtained a professorship of special mathematics at Moulins from 1872 to 1876. Then he held a professorship at Paris, first at the Lycée Charlemagne in Paris, then to the already prestigious Lycée Saint-Louis. His mathematical works concern non-elementary Euclidean geometry (projective geometry seen through its homographies), and especially the theory of numbers. His main contribution is made to primality tests. Partially forgotten in France (where the algebraic number theory is relegated to the background, waiting Weil), the work of Lucas is recovered and enhanced by the Anglo-Saxons, and especially by Lehmer, which will improve the primality test and prove totally some results of Lucas, to obtain the Lucas-Lehmer test, which is still used in the late 20th century to break records of large prime numbers. These studies are particularly important since the advent of computers that makes the cryptography hungry of very large prime numbers. Lucas is also known for being the inventor of many mathematical recreations. The most common of them is the Hanoi Tower problem, which he published under the name of Claus de Siam, professor at the College of Li-Sou-Tsiam anagram of Lucas d'Amiens, a professor at St. Louis.

\phantomsection
\addcontentsline{toc}{section}{M}

\parpic[l][t]{%
  \begin{minipage}{40mm}
    \fbox{\includegraphics[width=110px,height=140px]{img/medaillons/malthus.eps}}
  \end{minipage}
}
\textbf{Malthus, Thomas Robert} (1766-1834) was born in Guildford and died at Bath. He was an Anglican pastor, who worried about the excessive growth of population in England at the beginning of the industrial revolution (1750 to 1900). His fear revolved around the idea that population growth is faster than the increase of resources, that implies the impoverishment of the population. Because the old regulators of population (wars and epidemics) no longer play their roles, he imagined new obstacles, such as limiting the size of families and the rising age of marriage. These proposals are implemented so far, both, in China, which is indeed obliged to severely limit its demography. The predictions of Malthus are in reality undermined because he could not imagine such a large increase in resources and crop yields (green revolution: chemistry applied to agriculture which is not necessarily beneficial .. .); new means of international exchange of subsistence goods (contributing the way... to the pollution of the oceans the way); the fact that the overflow of people emigrate to the United States or the colonies. However, if the predictions of Malthus are not realized, his theory retains all the attention. It is true that the population is increasing in some countries (Saudi Arabia: 6 children per woman) it is also true (and happy) that advances in health and medicine increase the size of the population, it is true that renewable resources on Earth are limited ultimately by the solar energy it receives, which itself determines biomass, except major scientific discovery... and under these conditions, mathematics is clear: it will not be possible for the terrestrial population to increase indefinitely, and the regulation must occur at one time or another, in one way or another!

\parpic[l][t]{%
  \begin{minipage}{40mm}
    \fbox{\includegraphics[width=110px,height=140px]{img/medaillons/marconi.eps}}
  \end{minipage}
}
\textbf{Marconi, Guglielmo} (1874-1937) was born in Rome and died in Rome. Marconi was a physicist, inventor and Italian businessman. He shared with Karl Ferdinand the Nobel Prize in Physics of 1909 in recognition of their contributions to the development of wireless telegraphy (we can consider that he is the inventor of transmission/reception equipment for electromagnetic waves and so radio and broadcast television). Marconi was born in a wealthy family, the second son of Giuseppe Marconi, an Italian owner, and an Irish mother, Annie Jameson, granddaughter of the founder of the Jameson Whiskey Distillery. He studied at Bologna in the lab of Augusto Righi, in Florence at the Cavallero Institute and, later, in Livorno. He mades in 1985 experiments on waves discovered by Heinrich Rudolf Hertz seven years ago. He reproduces the equipment used by Hertz but improved the Branly coherer to increase the sensitivity and the antenna of Alexander Popov. After his first experiences in Italy, he made in the Swiss Alps at Salvan a link of 1.5 km in the summer of 1895. The following year, being not followed by his compatriots, he went to England to continue and patent his experiments. In 1897 he established the first morse communication over 13 km between Lavernock (Wales) and Brean (England) over the Bristol Channel. The following year, he opened the world first radio factory, at Chelmsford, England. In the early 20th century the name Marconi is (unfortunately) best known as the owner of the Pathé cinema group (who's real complete name is Pathé-Marconi).

\parpic[l][t]{%
  \begin{minipage}{40mm}
    \fbox{\includegraphics[width=110px,height=140px]{img/medaillons/mandelbrot.eps}}
  \end{minipage}
}
\textbf{Mandelbrot, Benoit} (1924-2010) was born in Warsaw and died in Cambridge. His family left Poland for Paris to escape the Nazi threat. It was in Paris that he was introduced to mathematics by two uncles, whom one was a professor at the Collège de France. The German invasion forced the family to flee to Brive-la-Gaillarde. After attending high school at Edmond-Perrier de Tulle, he studied at the Lycée du Parc, in Lyon. After leaving the École Polytechnique (1944), where he studied with a specialist of probabilities (Paul Levy), he became interested in the phenomena of information (the ideas Claude Shannon were at this time in full growth). Mandelbrot made his main studies in France and in the United States and received his Ph.D. in mathematics at the University of Paris in 1952. He taught economics at Harvard University, engineering at Yale, physiology at the Faculty of Medicine and mathematics in Paris and Geneva. From 1958, he worked for IBM at the Thomas B. Watson Research Center in New York on the optimal transmission in noisy environments, while continuing his work on strange objects neglected by mathematicians: objects with recursively defined complexity as the Von Koch curve, which he foresaw a unity: fractal geometry. Fractal geometry is characterized by a more abstract approach to the dimension as it is in the traditional geometry. It finds more and more applications in different fields of science and technology.

\parpic[l][t]{%
  \begin{minipage}{40mm}
    \fbox{\includegraphics[width=110px,height=140px]{img/medaillons/markov.eps}}
  \end{minipage}
}
\textbf{Markov, Andrei Andreyevich} (1856-1922) was a Russian mathematician specializing in number theory, theory of probability and mathematical analysis born in Ryazan and died in Petrograd. Coming from a family of a small government official, he studied at the University of St. Petersburg and received a gold medal for his thesis\textit{ On the integration of differential equations by the method of continued fractions} ( 1878). Professor at the University of St. Petersburg in 1886, he became a member of the Academy of Sciences in 1896. Markov's researches continue the work of his predecessors of the St. Petersburg mathematical school: P. L. Chebyshev, E. I. Zolotarev and A. N. Korkin. His thesis \textit{Bilinear quadratic forms with positive determinant} (1880) inaugurated his works in the field of number theory. In Analysis, his research concerned continued fractions, limits of integrals, series convergence and approximation theory. We owe him a simple solution for determining the upper limit of the derivative of a polynomial (Markov's inequality). After 1910, he turned to the theory of probabilities, and proved rigorously, under fairly general conditions, the central limit theorem on the sum of independent random variables. Trying to generalize this theorem to dependent random variables, he comes to consider the important notion of events chains, known as Markov chains, and establishes a series of laws, the foundation of the theory of Markov processes. He extends several classical results concerning independent events to certain types of chains. His work is at the origin of the modern theory of stochastic processes. Markov was also interested in applications of probability theory, and he justified in a probabilistic way the least squares method.

\parpic[l][t]{%
  \begin{minipage}{40mm}
    \fbox{\includegraphics[width=110px,height=140px]{img/medaillons/markowitz.eps}}
  \end{minipage}
}
\textbf{Markowitz, Harry Maurice} (1927 -) is born in Chicago. He was professor at the University of New York. Markowitz is known for having developed a famous portfolio model in his article \textit{Choice of investment portfolios for a fortune}. Markowitz did not suspect that his article published in 1952 in the Journal of Finance, when he was young, then developed in a book published in 1959, \textit{Portofolio Selection: Efficient diversification}, will lay the foundation of modern portfolio theory and be used by a large number of practitioners. More precisely, Markowitz showed that the investor seeks to maximize his choice, taking into account not only the expected profitability of investments, but also the risk of the portfolio defined mathematically by the variance of profitability. Applying classical theorems of statistical computing and probabilistic techniques, he has demonstrated that a portfolio of several shares is always less risky than a portfolio consisting of one share, even though it would be the least risky. Implementation of Markowitz has quickly raised practical problems. While the volume of data required to calculate increased rapidly with the number of shares held (with 100 shares, the number of necessary statistics was 3,150, but he passed 20,300 for 200 shares and to 125,750 for 300 shares!), information gathering and processing became almost impossible with the available computers in the 1960s, resulting in additional prohibitive treatment costs. This is why William F. Sharpe look for a method for selecting efficient portfolios easier. Markowitz and Sharpe will be recognized as the founding fathers of portfolio management and the doctrinal body on which it is based. The Nobel Prize in Economics will them be awarded as well as to Merton Miller in 1990.

\parpic[l][t]{%
  \begin{minipage}{40mm}
    \fbox{\includegraphics[width=110px,height=140px]{img/medaillons/markx.eps}}
  \end{minipage}
}
\textbf{Marx, Karl} (1818-1883) was born in Trier and died in London. Karl Marx entered the University of Bonn and after at the University of Berlin, after finishing high school in Trier. He studied law in Berlin, but also  history and philosophy. Marx then helped to complete the three main schools of thought of the 19th century: classical German philosophy, classical English political economy and French socialism. Marx's social theory aims to reveal the economic law of capitalist society where the production of goods dominates by seeking the origin of the value of money. Thus, for Marx, money (as the supreme product of the development of exchange and commodity production) fades and hide the character and social ties of individual work. At a certain stage in the development of commodity production, money is also transformed into capital. Thus, the sequence of movements of goods was: $G$ (goods) - $M$ (Money) - $G$ (goods), that is to say, selling a commodity to purchase another. The general sequence of capital is against $M-G-M$, that is to say, the purchase for sale (at a profit). It is this increase in the primitive value of money, so its transformation into capital, which Marx called "capital gain" and that can't come from the movement of goods, because this can only be done by the exchange of the counterparts; it can't either come from an increase in prices, as the reciprocal profits and losses of buyers and sellers equilibrate at large scale. To obtain capital gain it must be according to Marx a commodity whose process of consumption was at the same time a process of value creation. However, this commodity is the human labour. The possessor of money buys labour power at its value, determined as of any other commodity, by the labour time socially necessary for its production. Having bought the labour force, the owner of money is entitled to consume it, that is to say to oblige him to work all day, say, 8 hours. However, in 5 hours (necessary labour time), the worker creates a product that covers the cost of his own subsistence, and for the remaining 3 hours (overtime), it creates an additional product, unpaid by the capitalist, which is the capital gain. Also to express the degree of exploitation of labour by capital, we should compare the capital gain not against the total cost of production, but only to the variable cost of human labour.

\parpic[l][t]{%
  \begin{minipage}{40mm}
    \fbox{\includegraphics[width=110px,height=140px]{img/medaillons/maxwell.eps}}
  \end{minipage}
}
\textbf{Maxwell, James Clerk} (1831-1879) was born in Edinburgh and died in Glenlair. Brilliant student at High school, James Clerk Maxwell continue his studies of mathematics at the University of Cambridge. He obtained a chair of natural philosophy at Aberdeen at the age of twenty five years. Then, from 1860 to 1865, he served as professor at the King's College of London. Following these five years of teaching, he decided to retire to his property of Glenair, Scotland. He will stay there during five more years to study. In 1871, Maxwell was appointed director of the Cavendish Laboratory founded by the Duke of Devonshire. Maxwell will then cease to make it grows so that it becomes the most famous scientific training center. From the beginning of his career, Maxwell focuses on the dynamics of gas. After proving mathematically that the rings of Saturn are composed of discrete particles, he studied the velocity distribution of gas molecules (according to Gauss's law). In 1860, he shows that the kinetic energy of these molecules depends only of their nature. But it was his research in electromagnetism that make Maxwell one of the most known scientific of the 19th century. Based on the work of Faraday, he introduced in 1862 the concept of field. Then, he shows that a magnetic field can be created by varying an electric field (Faraday had discovered induction phenomenon in which the variation of an electric field creates a magnetic field). His purely mathematical teaching will then enable him to prepare the famous differential equations describing the nature of the electromagnetic fields in space and time. He describes them in his treatise On electricity and magnetism published in 1873. Maxwell, by developing the theories of electromagnetism, also defined light as an electromagnetic wave, thus paving the way for further research for other physicist like Heinrich Rudolph Hertz.

\parpic[l][t]{%
  \begin{minipage}{40mm}
    \fbox{\includegraphics[width=110px,height=140px]{img/medaillons/mcfadden.eps}}
  \end{minipage}
}
\textbf{McFadden, Daniel} (1937 -) born in Raleigh is an econometrician who received in 2000, with James Heckman, the Nobel Prize in Economics for his contributions to the theory and methods of discrete choice analysis. He obtained a Bachelor of Science in Physics at the age of nineteen at the University of Minnesota and a Ph.D.. in behavioral sciences (economics) five years later in 1962. In 1964, he joined the University of Berkeley and focuses his research on the behavior of choice, and the links between economic theory and economic measures. In 1975, he was awarded for the John Bates Clark Medal. In 1977, he went to the Massachusetts Institute of Technology, but returned to Berkeley in 1991, as the MIT had no statistics department. After his return, he founded the Laboratory of Econometrics, which is devoted to statistical computing and applied to economics. McFadden has developed micro-econometrics theories and methods for analysing discrete choice behaviors (e.g. data on occupations and places of residence of individuals) and is also famous for his pseudo-R coefficient for the probit logistic regression. From his economic theory on discrete choices, McFadden has developed new statistical methods that have had a decisive influence on the theoretical research, but are also widely used by marketing.

\parpic[l][t]{%
  \begin{minipage}{40mm}
    \fbox{\includegraphics[width=110px,height=140px]{img/medaillons/meitner.eps}}
  \end{minipage}
}
\textbf{Meitner, Lise} (1878-1960) was a physicist born in Vienna and died in Cambridge. In 1899, Lisa began a two-year entry exam accelerated preparation to enter at University. She was received and entered the University of Vienna in 1901, at the age of twenty-two years. After the first year, during which Lise followed many courses in physics, chemistry, mathematics and botany, she focused on physics. From the second year, she chose to take all the courses given by Ludwig Boltzmann ; this reflects the fascination that great theoretical physicist exercised over his students, with whom he developed intellectual but also personal relations. She obtained her Ph.D. in 1905. Lise remained in Vienna during the years that followed his doctorate. As a woman, she could not expect an academic career, but nevertheless continued research. She met Paul Ehrenfest, a former student of Boltzmann, who directed his attention on the articles published by Lord Rayleigh. One described an optical effect that Rayleigh could not explain. Lise founded the theoretical explanation and derived observations. Lise went to Berlin in 1907 to follow the course of Max Planck. Otto Hahn and Lise studied radioactivity and they became famous for their work, including the discovery of protactinium in 1918. Regardless of its work with Hahn, Lise led pioneering research in nuclear physics. She first devoted to the study of spectra of beta and gamma radiation. In 1923, she discovered the non-radiative transition known today as the "Auger effect", named in honour of Pierre Auger, a French scientist who discovered the effect independently two years later. She also discovered the emission of electron-positron pairs in the beta decay more. She made various measurements of the mass of the neutron. In 1939 she played a major role in the discovery of nuclear fission, that she provides with her nephew Otto Frisch, the first theoretical explanation in 1939 using the liquid drop model of Niels Bohr. This is why she is considered as the "mother of the nuclear bomb" by the media of his time.

\parpic[l][t]{%
  \begin{minipage}{40mm}
    \fbox{\includegraphics[width=110px,height=140px]{img/medaillons/mendel.eps}}
  \end{minipage}
}
\textbf{Mendel, Gregor Johann} (1822-1884) was born in Brno and died in Heinzendorf. He was a monk in the monastery of Brno (Moravia). Mendel is widely recognized as a botanist and the father of genetics. He is at the origin of what is now called Mendel's laws, which define the way genes are passed from generation to generation. Mendel was born in a peasant family. Having aptitudes for studies, but having also a depressive tendency which earned him multiple troubles later in his career, the boy was quickly identify by the village priest who decided to send him away from home to continue his studies. Mendel attend in 1851 classes as an auditor of the Institute of Physics of Christian Doppler. He studied in addition to the obliged subjects: botany, plant physiology, entomology and palaeontology. In two years, he acquired the methodological basis which will later give him the possibility to realize his experiences. During his stay in Vienna, Mendel is brought to focus on the theories of Franz Unger, professor of plant physiology. Franz Unger propose the experimental study to understand the emergence of new characteristics in plants over successive generations. He hopes to solve the problem of hybridization in plants. Back to the monastery, Mendel installs an experimental garden in the courtyard and in the greenhouse, in agreement with his abbot, and set up a plan of experiments to understand the laws of the origin and formation of hybrids. He chose for his experiences the pea which has the advantage of being easily cultivated with many known varieties. In 1865, he exhibited at the Society of Natural Sciences Brno and publishes in 1866 the results of its studies after 10 years of painstaking work, Mendel also laid the theoretical foundations of modern genetics and heredity. His work does not generate enthusiasm among his contemporaries who are struggling to understand the mathematical experiences. Very few scientists of his time will speak about his work and Mendel gets only some answers from various correspondents. Of these, only Karl Wilhelm von Nägeli, professor of botany at Munich, wrote him doubting also some of his conclusions. In 1868, Mendel was elected superior of the convent after the death of the abbot.

\parpic[l][t]{%
  \begin{minipage}{40mm}
    \fbox{\includegraphics[width=110px,height=140px]{img/medaillons/mendeleiev.eps}}
  \end{minipage}
}
\textbf{Mendeleev, Dmitri Ivanovich} (1834-1907) was a Russian born in Tobolsk and died in St. Petersburg. He was a famous chemist best known for his periodic table of elements published in 1869. He showed indeed that the chemical properties of elements directly dependent on their atomic weight and that were a periodic functions of that weight. He entered at the age of fourteen years at the Tobolsk high school, after the death of his father. In 1849, the family who became poor moved to St. Petersburg and Mendeleev entered to university in 1850. After graduation, he contracted tuberculosis which forced him to move in the Crimean Peninsula near the Black Sea in 1855, where he became head of the local high school of science. He returns completely healed in St. Petersburg in 1856 where he also studied chemistry and became graduated in 1856. At the age of twenty-five, he works at Heidelberg with scientists like Robert Bunsen and Gustav Kirchhoff. At Heidelberg, he met the Italian chemist Stanislao Cannizzaro, whose ideas on the atomic weight influenced his thinking. Mendeleev returned to St. Petersburg and taught chemistry at the Technical Institute in 1863. He was appointed professor of general chemistry at the University of St. Petersburg in 1866.

\parpic[l][t]{%
  \begin{minipage}{40mm}
    \fbox{\includegraphics[width=110px,height=140px]{img/medaillons/merton.eps}}
  \end{minipage}
}
\textbf{Merton, Robert Cox} (1944-) received the Nobel Prize in Economics in 1997 along with his compatriot Myron Scholes for their development of the evaluation of financial derivatives. This method of evaluation has certainly accelerated the rapid growth of derivatives markets since the 1980s and led to improved management of risks associated with these new financial products. Merton has undoubtedly helped to open a new path in the field of economics and strongly influenced the other two winners. Born in 1944 in New York, he left the California Institute of Technology with a master's degree in Applied Mathematics. He subsequently obtained a Ph.D. in economics at the Massachusetts Institute of Technology (M.I.T.) in Cambridge, under the direction of Paul Samuelson (Nobel Prize for Economics 1970) and specializes in problems of application of probabilistic methods to random evolution of financial asset prices. In 1988 he held the George Fischer Backer chair as professor in Business Administration at Harvard Business School in Cambridge. The pioneering work of Merton start from the early 1970, period during which he develops a new method of calculating the value of derivatives. The failure of his method applied to the management of an investment american fund risk (Long-Term Capital Management) in 1998, has somewhat tarnished its reputation as a specialist in international finance. But Merton himself had told a U.S. television network, following the award of the prize, that is a misunderstanding to think that we can eliminate the risk simply because we understood and can measures them.

\parpic[l][t]{%
  \begin{minipage}{40mm}
    \fbox{\includegraphics[width=110px,height=140px]{img/medaillons/minkowski.eps}}
  \end{minipage}
}
\textbf{Minkowski, Hermann} (1864-1909) was born in Alexotas (Russia) and died in Göttingen (Germany). He was a mathematical physicist who studied at the universities of Berlin and Königsberg. He studied at the high school of Königsberg where he was recognized for his performance in mathematics and he received his Ph.D. in 1885 in the same city. He then taught at several universities in Bonn, Königsberg and Zurich. In Zurich, Einstein was a student in several of his lectures. Minkowski accepted a professorship in 1902 at the University of Göttingen, where he remained for the rest of his life. In Göttingen, he learned the physic-mathematics of Hilbert, he participated to a conference on the theory of the electron in 1905 and learned the latest results in the theory of electrodynamics. In 1907 Minkowski realized that the work of Lorentz and Einstein could be better understood in a non-Euclidean space. He considered space and time, which was previously thought to be independent, to be coupled together in a continuum four-dimensional space-time. Minkowski has established a four-dimensional treatment of electrodynamics. This space-time continuum has provided a framework for all later mathematical works in relativity. These ideas have been used by Albert Einstein in developing the general theory of relativity. Minkowski was mainly interested in pure mathematics and has spent much of his time studying quadratic forms and continued fractions. However, his most original work was his Geometry of numbers. This study has led to work on convex bodies and to questions about packing problems (ways in which the figures of a given form can be placed in another given figure).

\parpic[l][t]{%
  \begin{minipage}{40mm}
    \fbox{\includegraphics[width=110px,height=140px]{img/medaillons/mobius.eps}}
  \end{minipage}
}
\textbf{Möbius, August Ferdinand} (1790-1868) was a German mathematician and astronomer born in Schulpforta and died in Leipzig. Möbius was educated at Leipzig, Göttingen (under the direction of Gauss) and Halle. In 1815 he became professor of astronomy at Leipzig, then director of the observatory of the city, after having directed its construction. He has written several books of theoretical astronomy, including \textit{De computandis occultationibus fixarum per planetas} (1815). His mathematical works concerned mainly geometry and were, for the most part, published in the Journal of Pure and Applied Mathematics of Crelle, from 1828 to 1858, as a complement to his fundamental book \textit{Der barycentrische Calculation} (1827). By introducing a new coordinate system, Möbius studies geometric transformations, mainly projective transformations. His work had a great importance in the development of projective geometry. Studying the static in terms of geometry, Möbius also developed the theory of linear complexes of lines (Lehrbuch der Statik, 1837). Möbius can be considered as one of the pioneers of topology, with the discovery, published in a submission to the French Academy of Sciences, of the famous "Mobius Surface", with only one side.

\parpic[l][t]{%
  \begin{minipage}{40mm}
    \fbox{\includegraphics[width=110px,height=140px]{img/medaillons/monge.eps}}
  \end{minipage}
}
\textbf{Monge, Gaspard} (1746-1818) was born in Beaune and died in Paris. He follows first the college of Beaune and then went to the college of Lyon (France), where he taught from the age of sixteeen physical sciences. An engineering officer, who had seen a map of the town of Beaune made by Monge using new methods of observation and construction graph, recommends Monge to the commander of the military school of Mézières. But he can't be accepted because of its common origin and is accepted only in a technical annex of the school. His scientific talents are recognized when one day he draws the plan of fortifications using a method much faster than previously known methods. He is then admitted to the military school as a mathematics teacher and continued his research, arriving at the general method of geometric representation known since as under the name of "Descriptive Geometry". But his discoveries, considered as valuable military secrets, can't be published. In 1780 he went to Paris to teach hydrodynamics. He immediately entered the Academy of Sciences, where he made a presentation on the lines of curvature drawn on a surface (problem already studied by Euler in 1760). In 1786, he published his famous \textit{Traité élémentaire de la statique} and soon after founded the École Polytechnique, where he had the opportunity to teach descriptive geometry and publish his works hitherto unknown. Chargé de Mission in Italy, Monge meets Bonaparte and is defined as responsible for recruiting scientists for the Egyptian expedition. Back in France, he resumed his education at the École Polytechnique became a senator and was knighted. But the Restoration deprive him of all titles, it will scratch Monge of the list of members of the Institute and will take him away his teaching position. In 1989, his ashes were transferred to the Panthéon. All his research closely intertwined pure geometry, infinitesimal analysis and analytical geometry, allowing, for example, to link each family of surfaces with a partial differential equation, and hence, to find solutions to differential equations using his theory of surfaces. The influence of Monge exerted trough his oral teaching, most of the 19th century French mathematicians that were his students.

\phantomsection
\addcontentsline{toc}{section}{N}

\parpic[l][t]{%
  \begin{minipage}{40mm}
    \fbox{\includegraphics[width=110px,height=140px]{img/medaillons/napier.eps}}
  \end{minipage}
}
\textbf{Napier, John} (1550-1617) was a physicist, astronomer, mathematician and theologian born and died at Merchiston (U.K.). As it was the common practice for members of the nobility at that time, John Napier did not enter schools until he was thirteen. However he did not stay in school very long. It is believed that he dropped out of school in Scotland and perhaps travelled in mainland Europe to better continue his studies. In 1571 Napier, aged twenty-one, returned to Scotland, and bought a castle at Gartness in 1574. On the death of his father in 1608, Napier and his family moved into Merchiston Castle in Edinburgh, where he resided the remainder of his life. Mathematics were not his main activity but he had a lot of ideas to simplify calculations. He establishes some formulas of spherical trigonometry, popularized the use of the point to the English notation of decimal numbers and especially invented logarithms. His objective was to simplify trigonometric calculations needed in astronomy. He defined the logarithm of a sine based on mechanical considerations of moving points and the link between the arithmetic and geometric progressions.

\parpic[l][t]{%
  \begin{minipage}{40mm}
    \fbox{\includegraphics[width=110px,height=140px]{img/medaillons/navier.eps}}
  \end{minipage}
}
\textbf{Navier, Henri} (1785-1836) was born in Dijon and died in Paris. He was an engineer, mathematician and economist best known for his work on hydrodynamics. Henri was orphaned at age of nine, after the death of his father, renowned lawyer and former member during the Revolution. His uncle, engineer at the Corps des Ponts et Chaussées took in charge his education in Paris and consider him as his son before adopting him with his wife, also a close relative of the young Henri. His uncle force him to attend the École Polytechnique. Although one of the last received in 1802, he succeeded his schooling and its classification allows him to integrate the Corps des Ponts et Chaussées. He was appointed resident engineer of Ponts et Chaussées in 1808. Later, he became divisional inspector of this Corpse, and it seems that for a time, General Inspector like his uncle. From 1819 to 1835, he provides the course of Applied Mechanics of the École Nationales des Ponts et Routes (he is nominated professor in 1830 following the retirement of Eisenmann). In the early 1820s, he explores with Augustin-Louis Cauchy aspects of the mathematical theory of elasticity, which allows him to propose the motion equations of Newtonian fluids.

\parpic[l][t]{%
  \begin{minipage}{40mm}
    \fbox{\includegraphics[width=110px,height=140px]{img/medaillons/nash.eps}}
  \end{minipage}
}
\textbf{Nash, John} (1928-2015) was born in West Virginie (U.S.A.) and die in Monroe Township (New Jersey) at the same time as his wife (physicist) of a car accident. Son of John Nash Sr., an engineer, and Virginia Martin, a teacher he spent a lot of time reading and experimenting in his room that he had converted into a small laboratory. From 1945 to  1948, Nash studied at the Carnegie Institute of Technology in Pittsburgh, intending to become an engineer like his father. Instead, he developed an enduring passion for mathematics, and in particular the theory of numbers, Diophantine equations, quantum mechanics and relativity theory. He was admitted at the graduate level at the age of twenty years in all universities he had requested: Harvard, Princeton... He chose to go to Princeton. Having an interest in economics, Nash began to study game theory, an area that had been cleaned by John von Neumann, one of the great names of Princeton, a little over a decade ago. It is on this subject that he decided to make his thesis and he won the Nobel Prize for Economics in 1994. During the summer of 1950, Nash was employed as a consultant at RAND, top-secret institute that employed brainpower to develop various strategies of status quo of victory, in cases of conflict involving nuclear weapons. Nash began to study the compact smooth manifolds, which was the subject of a paper. He then became assistant at M.I.T. in 1951 to 1952, at only twenty-three years old. He really had the temperament of a problem-solver and raised the challenge of solving a question of Waren Ambros: Is it possible to dive any Riemannian manifolds in Euclidean space? Nash found a fundamental original method to achieve this problem. Nash became ill after some personal and professional problems, but he attributed his illness to his attempt to resolve the contradictions of quantum physics. Especially since shortly before he had completed work on non-linear elliptic PDE which earned him much admiration around him, but he finally had to share paternity with a young Italian who had set, independently and a few weeks before him, similar results: This earned them not getting the Fields Medal in 1958...

\parpic[l][t]{%
  \begin{minipage}{40mm}
    \fbox{\includegraphics[width=110px,height=140px]{img/medaillons/newton.eps}}
  \end{minipage}
}
\textbf{Newton, Isaac }(1642-1727) was a mathematician and physicist, considered as one of the greatest scientists in history. Newton was born in Lincolnshire (England), from peasant parents and died in London. At the age of five, he attended primary school at Skillington, then at the age of twelve that of Grantham. He will stay there four years until his mother order him to come back at Woolsthorpe to become a farmer and learn how to administer his domain. However, his mother, seeing that her son was better in mechanics than in livestock, allowed him to return to school to be able perhaps one day to enter at the university. At the age of seventeen, Newton falls in love with a classmate, miss Storey. He is authorized to have her as girlfriend and even got engaged with her, but he must first finish his studies before getting married. Finally, the marriage did not happen and Newton will be single all his life. At age of eighteen, he entered the Trinity College of Cambridge (he will stay there seven years), where he was noticed by his teacher, Isaac Barrow. He also have for professor Henry More who will have a great influence in his conception of absolute space. At Cambridge, he studied arithmetic, geometry in Euclid's Elements and trigonometry, but is particularly interested in astronomy, alchemy and theology. He receive at the age of twenty-five his bachelor of arts, but was forced to suspend his studies for two years following the emergence of the plague that struck the city in 1665. He then returned to his native region. It is during this period that Newton grew strongly in mathematics, physics and especially in optics. He gave important contributions to many areas of science. His discoveries and theories were the basis of much scientific progress after him. Newton was one of the inventors of the branch of mathematics named "infinitesimal calculus" (another inventor was the German mathematician Gottfried Wilhelm Leibniz). He also clarifies the mysteries of light and optics, formulate the three laws of motion and derived the law of universal gravitation based on Kepler's laws. He reached the argument that light is a mixture of different rays of different colors, and because of the phenomena of reflection and refraction, colors appear in separate components. Newton proved his theory of colors by passing light through a prism, which splits the light beam into separate colors. In 1696, he left Cambridge to become the first guardian of the Royal Mint and Master of the Mint in the following year. In 1699, he was promote member of the Royal Society and is elected president in 1703. He held this position until his death.

\parpic[l][t]{%
  \begin{minipage}{40mm}
    \fbox{\includegraphics[width=110px,height=140px]{img/medaillons/neumann.eps}}
  \end{minipage}
}
\textbf{Neumann Von, John} (1903-1957) was a mathematician born in Budapest and died in Washington. Von Neumann was a child prodigy: at the age of six, he converses with his father in ancient Greek and can mentally divide a 8-digit number. An anecdote relates that at the age of only eight years old, he has already read the 44 volumes of the universal history of the family library and he has completely memorized it: with an absolute memory, he is able to quote from memory entire pages of the books readen years ago. He entered the school in Budapest in 1911. At the age of twenty-three years old he received his Ph.D. in mathematics (with minors in experimental physics and chemistry) at the University of Budapest. In parallel, he earned a degree in chemical engineering from the ETH Zurich in Switzerland (at the request of his father, wanting his son to invest in a more remunerative than mathematics). It is interesting to note that von Neumann never followed the courses and went in these two universities only for the exams. Between 1926 and 1930 he was Privatdozent in Berlin and Hamburg (Germany). He also worked with Robert Oppenheimer in Göttingen under the supervision of David Hilbert. During this period, one of the most fruitful of his life, he is also near of Werner Heisenberg and Kurt Gödel. In 1930, von Neumann was invited professor at Princeton University. Then, from 1933 to his death in 1957, he was professor of mathematics at the Faculty of the Institute for Advanced Study that has just been created. He joins there Albert Einstein and Kurt Gödel. Neumann emigrated to the United States in 1933 to join the Institute for Advanced Research in Princeton. He wrote an important book on Applied Mathematics and made a major work in the axiomatization of quantum physics (he founded that a quantum system can be considered as a point in a Hilbert space and introduced linear operators). He participated during the Second World War to the theoretical development of the atomic bomb and the study of shock waves. His mathematical work on ultra-fast simulations of the H-bomb, helped in the development of computers (he is also at the origin of Monte-Carlo method). He also contributed to the theory of games where some of these results had a great influence on the economy.

\parpic[l][t]{%
  \begin{minipage}{40mm}
    \fbox{\includegraphics[width=110px,height=140px]{img/medaillons/abel.eps}}
  \end{minipage}
}
\textbf{Niels, Abel} (1802-1829) was a Norwegian mathematician born in Frindoë and died in Froland. His father was a known Norwegian politician, but at the end of his life, he fell into disgrace, and when he died in 1820, it is Abel who had to bear the entire burden of the family. His father educated Abel himself until 1815, then sent him to the parochial school in Oslo. In this school Latin and Greek religion were taught with the old traditions, with corporal punishment. The situation changed in 1817 after the dismissal of a teacher following the death of a student: the school hired then a young teacher open to new ideas and knowing mathematics. This new teacher discovered that Niels was interested in mathematics, he founded him a scholarship for University. With the financial assistance from his teachers, he manages however his studies and make his first discoveries. But his works were lost by Cauchy and underestimated by Gauss. After his Ph.D., Abel was unable to find a job and his living conditions became increasingly precarious and embrittlement his health: he was thus suffering from tuberculosis. Despite trips to Paris and Berlin, his works are still not perceived at their true value. In his last weeks, he no longer has enough strength to leave it's bed. He died at only twenty-seven years old, while a friend just find him a job in Berlin. It is Jacobi who will understand the genius of the young mathematician. Abel had especially proved at the age of nineteen years, the impossibility of solving algebraic equations of the 5th degree by radicals, result that his contemporary Galois generalized to any degree. Posthumously, in 1830 Abel receive the grand prize of Mathematics of the France Institute.

\parpic[l][t]{%
  \begin{minipage}{40mm}
    \fbox{\includegraphics[width=110px,height=140px]{img/medaillons/noether.eps}}
  \end{minipage}
}
\textbf{Nöther, Emmy} (1882 -1935) was born in Erlangen and died in Princeton. Emmy considered first teaching French and English after passing the required examinations, but finally studied mathematics at the University of Erlangen, where her father was a lecturer. During the winter semester of 1903-1904, she studied at the University of Göttingen and attended the courses of the astronomer Karl Schwarzschild and mathematicians Hermann Minkowski, Felix Klein and David Hilbert. After completing his Ph.D. in 1907 she worked for free at the Mathematics Institute in Erlangen during seven years. In 1915, she was invited by David Hilbert and Felix Klein to join the renowned Department of Mathematics at the University of Göttingen until 1933. In 1935, she was operated because of an ovarian cyst and, despite signs of recovery, died four days later at the age of fifty-three years old. She remains in the history of mathematics as the main founder of abstract algebra or modern algebra, which is one of the essential branches of contemporary mathematics. This abstract algebra takes importances compared to calculations performed in various sets, defined with various operations, and shows what these calculations have in common. In physics, Noether's theorem explains the fundamental connection between symmetry and conservation laws. His ideas have also contributed to the advancement of physics, in particular in the theory of relativity. Despite all her qualities, she had difficulties to lead a normal career as a university professor, because she was a woman in an exclusively male environment. However she enjoyed the esteem and support of David Hilbert, Albert Einstein and Felix Klein.

\phantomsection
\addcontentsline{toc}{section}{O}

\parpic[l][t]{%
  \begin{minipage}{40mm}
    \fbox{\includegraphics[width=110px,height=140px]{img/medaillons/ohm.eps}}
  \end{minipage}
}
\textbf{Ohm, Georg Simon} (1789-1854) was a physicist born in Erlangen and died in Munich (Germany). Although his parents had not made higher education, Ohm's father was a respected man and an autodidact who himself gave his son an excellent education. From his earliest childhood Georg received from his father's very good teachings in physics, mathematics, chemistry and philosophy. Georg attended the school of Erlangen from eleven to fifteen years old and where he received a very limited scientific education, in contrast with the teachings of his father. In 1805, at the age of fifteen, Ohm entered the University of Erlangen. Ohm is dissipated as his father angry at the waste of its potential, sent him to Switzerland where, in 1806, he took up a post as a mathematics teacher in the school of Gottstadt bei Nydau. Ohm left his teaching position at Gottstadt bei Nydau in 1809 to become a private tutor in Neuchâtel (Switzerland) for two years. Then in 1811 he returned to the University of Erlangen. His studies were useful for obtaining his Ph.D. from the University of Erlangen in the same year and immediately join the teaching staff as a lecturer in mathematics. The king Frederick William III of Prussia offered him a position at the Jesuit school of Cologne in 1817. Thanks to the reputation of this school in the teaching of science, Ohm is found to teach both mathematics and physics. The physics laboratory is well equipped, he devoted himself to experimentation. What is now known as Ohm's law appeared in 1827 in the book \textit{Die Kette galvanische, Mathematisch bearbeitet} in which he provides a complete theory of electricity. He entered the Polytechnic School of Nuremberg in 1833 and in 1852 became professor of experimental physics at the University of Munich, where he died later.

\parpic[l][t]{%
  \begin{minipage}{40mm}
    \fbox{\includegraphics[width=110px,height=140px]{img/medaillons/oppenheimer.eps}}
  \end{minipage}
}
\textbf{Oppenheimer, J. Robert} (1904-1967) was a physicist born in New York and died in Princeton. He was the scientific director of the Manhattan Project and also managed the development project of the first atomic bombs. He entered Harvard with a year's delay due to an attack of ulcerative colitis, he took advantage of this period to visit with his former English teacher in New Mexico. He became an amateur of horse riding as well as the mountains and plateaus of this region. Upon his return, he graduated in Chemistry. Percy Bridgman made him discovered experimental physics. It was during his studies at the Rutherford Ernest Cavendish Laboratory of Cambridge that he realizes that he master better the theory than experiments due to his clumsiness. In 1926, he continued his studies under the direction of Max Born at the University of Göttingen (Germandy) and obtained his Ph.D. at the age of twenty-two. At Göttingen, he publishes articles on quantum theory. In 1927, he returned to Harvard and the following year at the Institute of Technology Californie. He is also known for his contribution to the quantum theory and the theory of relativity, and for studies of cosmic rays, positrons and neutron stars. He made important research in astrophysics, nuclear physics, and spectroscopy. He then discovered the Born-Oppenheimer approximation.

\parpic[l][t]{%
  \begin{minipage}{40mm}
    \fbox{\includegraphics[width=110px,height=140px]{img/medaillons/ostrogradsky.eps}}
  \end{minipage}
}
\textbf{Ostrogradsky, Mikhail Vasilyevich} (1801-1862) was an Ukrainian physicist and mathematician. He began his studies in mathematics at the University of Kharkov, and then went to Paris where he was in close contact with the famous French mathematicians Cauchy, Binet, Fourier and Poisson. Back in his homeland, he taught at the School of the Marine Cadet, at the Nicolas Academy of Engineering and at the Artillery School of St. Petersburg. He is famous in particular for establishing the flow divergence theorem, which allows to express the integral over a volume (or triple integral) of the divergence of a vector field as the surface integral (double integral extended to the area surrounding the volume) of the flow defined by this field. He was elected at the American Academy of Arts and Sciences in 1834, the Academy of Sciences of Turin in 1841, and the Academy of Sciences in Rome in 1853. Finally he was elected corresponding member of the Academy of Sciences of Paris in 1856. The scientific work of Ostrogradski are in line with the principles professed at that time at the Polytechnic School in the areas of analysis and Applied Mathematics. In mathematical physics, he imagined a synthesis which would embrace the hydromechanical theory of elasticity, the theory of heat and the theory of electricity under one uniform method.

\phantomsection
\addcontentsline{toc}{section}{P}

\parpic[l][t]{%
  \begin{minipage}{40mm}
    \fbox{\includegraphics[width=110px,height=140px]{img/medaillons/pareto.eps}}
  \end{minipage}
}
\textbf{Pareto, Vilfredo} (1848-1923) was an Italian economist and sociologist, whose most famous contribution to economic theory is the definition of the concept of economic optimum. Born in Paris from an Italian father in exile and a French mother, he returned to Italy at the age of ten. He studied at the University of Turin and became an engineer. In 1893, he was appointed to the chair of political economy at the University of Lausanne (he died in Celigny Switzerland), where he succeeded Léon Walras. Among his works we found the analysis of expectations of economic agents. The fact that they are not independent of each other may give rise to movements of opinion that generate pessimistic crises. Pareto is also the father of the concept of optimum: The economy is optimum when the situation of an agent can not be improved without damaging at least one other agent. This concept is widely used in economics, because it allows to take into account the non-additivity of utilities of different agents. Competition achieves the Pareto optimum. Pareto has also integrated the indifference curves (formalized by Francis Edgeworth) to the Walrasian general equilibrium logic. The sociological work of Pareto was most discussed. In the \textit{Traité de sociologie générale}, published in 1916, he presented his theory of elites, giving that government power in all societies is the subject of a battle between the elites only. This thesis discredited democracies, and implicitly contributed to the development of fascism in Italy.

\parpic[l][t]{%
  \begin{minipage}{40mm}
    \fbox{\includegraphics[width=110px,height=140px]{img/medaillons/pascal.eps}}
  \end{minipage}
}
\textbf{Pascal, Blaise} (1623-1662) was a mathematician, physicist, theologian, mystic, philosopher, moralist and polemicist born in Clermont and died in Paris. Precocious child (at the age of eleven, he composed a short treatise on the sounds of vibrating bodies and proved the 32nd proposition of the first book of Euclid, at the age of sixteen he wrote a treatise on conics), he was educated by his father who was a mathematician. The earliest works of Pascal concern natural and applied sciences. He contributed significantly to the study of fluids. He clarified the concepts of pressure and vacuum by expanding the work of Torricelli. The extent of the areas of interest and genius of Pascal is impressive: inventor of the calculating machine, designer of the first transports in France, architect of the Poitevin marshes drying, he was also one of the finest prose writers of the French language and one of the greatest figures of the 17th century.

\parpic[l][t]{%
  \begin{minipage}{40mm}
    \fbox{\includegraphics[width=110px,height=140px]{img/medaillons/pauli.eps}}
  \end{minipage}
}
\textbf{Pauli, Wolfgang} (1900-1958) was an Austrian physicist, born in Vienna (Austria) and died in Zürich (Switzerland), known for his definition of the exclusion principle in quantum mechanics, for which he received the Nobel Prize in Physics in 1945. Pauli was born from a father who was a university professor and a mother who was journalist and lawyer. At High school in Vienna, Pauli was considered as a prodigy child in mathematics. In 1919, he began his studies in physics at the University of Munich with the Professor Arnold Sommerfeld. Since 1898, Sommerfeld was in charge of writing the 5th volume of the \textit{Enzyklopädie der Wissenschaften Mathematischen} (20,000 pages) devoted to physics. He requires at first the collaboration of Albert Einstein to write the article on relativity, but he refuses. Sommerfeld then asks Pauli, whose speciality was the relativity during registration to Sommerfeld Sommerfeld courses. Thus, at the age of twenty-one, Pauli published his article summarizing the theories of relativity and General Relativity. In 1921, he obtained his Ph.D. on the subject of the hydrogen atom, where he clearly showed the limits of the model of the Bohr's atom, on which he worked as an assistant with Max Born in Göttingen between 1921 and 1922. During the years 1922 and 1923, he worked alongside Niels Bohr in Copenhagen. Between 1923 and 1928, he taught at Hamburg before leaving at the Zürich ETH, where he obtained a professorship in theoretical physics. In 1935, he moved to the United States, where he held invited professor status, including at the Institute for Advanced Study at Princeton during the years 1935-1936, but also at the University of Michigan, in 1931 and 1941, and at Purdue University in 1942. In 1946, he obtained the U.S. citizenship, but returned the same year at the ETH Zurich, where a place as teacher had been kept. In 1949, he became a Swiss citizen. In the 1950s, he regularly returns to Princeton to teach as a visiting professor. In the last years of his life, he participated in the founding of CERN. He died of a peptic ulcer.

\parpic[l][t]{%
  \begin{minipage}{40mm}
    \fbox{\includegraphics[width=110px,height=140px]{img/medaillons/pearson.eps}}
  \end{minipage}
}
\textbf{Pearson, Karl} (1857-1936) was a British mathematician founder of modern statistics born in London and died in Surrey. Statistical analysis has been a great development in the late 19th century in the United Kingdom and Karl Pearson dominates his contemporaries by the extent and variety of his contributions instead having interests in statistics starting only at the age of thirty-three. He develops analytical methods for the study of natural selection and eugenics which he is an ardent promoter. His main contributions are the creation of the test of independence chi-square for judging whether differences in a set of variables with respect to the theoretical values can be assigned or not a random sample and the definition of the correlation coefficient. He received the Darwin medal (biology) in 1898. Pearson was also a business consultant. He also taught statistics at William S. Gosset who introduced the Student law in 1910. He is one of the founders of Biometrika which he was editor for 36 years and has grown to become the best review in mathematical statistics.

\parpic[l][t]{%
  \begin{minipage}{40mm}
    \fbox{\includegraphics[width=110px,height=140px]{img/medaillons/penrose.eps}}
  \end{minipage}
}
\textbf{Penrose, Roger }(1931 -) is a physicist and mathematician born in Colchester (United Kingdom). Penrose get graduated in mathematics from the London's College University and get his Ph.D. from Cambridge University with a thesis on tensor methods in algebraic geometry. Between 1964 and 1973, he taught mathematics at at the Birkbeck London's College and meets the famous physicist Stephen W. Hawking with whom he worked on a theory of the origin of the universe by contributing to the mathematical theory of General Relativity applied to cosmology and the study of black holes. In 1965, at Cambridge, he proves that gravitational singularities can be formed by gravitational collapse of massive stars at the end of their life. In 1971, Penrose discovers spin networks that would later form the geometry of space-time in loop quantum theory. Professor at Oxford, he received, with Hawking, the 1988 Wolf Prize for physics.

\parpic[l][t]{%
  \begin{minipage}{40mm}
    \fbox{\includegraphics[width=110px,height=140px]{img/medaillons/picard.eps}}
  \end{minipage}
}
\textbf{Picard, Charles-Emile }(1856-1941) born and died in Paris made his classical studies at the  École Supérieure de Vanves in 1864, then at the École Supérieure Napoléon(the future Henry IV High School) from 1868 to 1874 where he proved that he was an excellent student, but not really attracted by mathematics. He obtain in 1874 a Bachelor of Arts and the following year a B.Sc. He is received second at the École Polytechnique, and first at the École Normale Supérieure. Finally, passionate for science, he choose this subject to pass the aggregation in 1877. After various assistant positions in Paris and Toulouse, in 1881 he became professor at the École Normale Supérieure. His name is already famous in the circle of mathematicians, because he proved an important theorem on singularities of holomorphic functions which earned him a nomination for membership int the Academy of Sciences. But he is too young, and his election was postponed to 1889. In 1885, Picard was appointed professor at the Sorbonne, where he holds the Chair of differential calculus. Again, his age is a problem (must be at least 30 years for such a position) and it was used a clever procedure to circumvent the legislation. Later, Picard occupy the chair of analysis and algebra, and also exercise at the Central School of Arts and Manufacture (1894-1937): there he trained more than 10,000 mechanical engineers, and is according to Hadamard, a great teacher. Picard's work is difficult, and pave the way for further research. He is the first to use the fixed point theorem in a method of successive approximations, which permit to solve partial differential equations. We also own him works in algebraic geometry, as more applied research on the elasticity or heat. He is also an early defensor of the theories of Einstein. His \textit{Traité d'Analyse} was long considerate as a reference, and Picard was also a philosopher and historian of science. Among the distinctions that Picard has received, he was the president of the International Congress of mathematicians, he was elected to the Académie Française in 1924, and he received the Mittag-Leffler gold medal in 1937.

\parpic[l][t]{%
  \begin{minipage}{40mm}
    \fbox{\includegraphics[width=110px,height=140px]{img/medaillons/planck.eps}}
  \end{minipage}
}
\textbf{Planck, Max} (1858-1947) was a German physicist born in Kiel and died in Göttingen considerate as the founder of quantum physics. After receiving his bachelor's at the age of seventeen in Munich where his father taught, Max Planck went study physics in Berlin. Fascinated by thermodynamics, he supports a thesis on the second law of thermodynamics and the concept of entropy in 1879, which will remain the main concept explaining the majority of his researches. The following year, he became a lecturer at the University of Munich and then became professor of physics at the University of Kiel in 1885. Four years later, he is professor of physics at the University of Berlin, where he worked for nearly 40 years. In 1930 he became director of the Kaiser Wilhelm Institute for Scientific Research, which will soon have his name. Initiated by his doctoral thesis, the research of Planck in thermodynamics are quickly oriented on the black body. Entity purely theoretical, the black body absorbs all radiation it receives (the black carbon, absorbing 97\% of the radiation, is close to this ideal). To explain this phenomenon, Planck developed a new theory. He speculates that the energy of radiation can be emitted or absorbed by matter only in finite quantities, the quanta. He then shows that these "energy packets" are set to hv, where v is the frequency of the radiation and h is a universal constant (the "Planck constant"). Explaining his theory to the German Physical Society in 1900 in Berlin, Planck does not yet know that he has invented a new branch of physics: quantum physics. His discovery will then be at the origin of the creation of the atom model by Niels Bohr, the development of wave mechanics by Louis de Broglie, the explanation of the photoelectric phenomenon by Albert Einstein and the discovery of the uncertainty principle by Werner Heisenberg. Considered as one of the most famous physicists, Planck received the Nobel Prize in 1918.

\parpic[l][t]{%
  \begin{minipage}{40mm}
    \fbox{\includegraphics[width=110px,height=140px]{img/medaillons/poincare.eps}}
  \end{minipage}
}
\textbf{Poincaré, Henri} (1854-1912) was a French mathematician and physicist born in Nancy, died in Paris, who was said that he was the last scientist knowing all the mathematics of his time. Exceptional student at the Lycée Impérial de Nancy, he obtained in 1871 a Bachelor of Arts, with honours, and the same year his B.Sc. He ranks first in the entrance examination at the École Polytechnique in 1873, then at the École des Mines de Paris, as engineer at the Corps des Mines, in 1875. He obtained his Ph.D. in 1876. Named 3rd class engineer in 1879 at Vesoul, he obtained the same year his Ph.D. in mathematics at the Faculty of Sciences in Paris, and became a lecturer in analytical science at the faculty of Caen. The first works of Poincaré are on Fuchsian automorphic functions, the qualitative theory of differential equations and the theory of functions. In a series of 6 articles published from 1894, he is the creator of algebraic topology, expanding science in the 20th century and in which more conjectures due to Poincaré remains open. He was also deeply interested in celestial mechanics: \textit{Les Méthodes nouvelles de la mécanique céleste}, three volumes published between 1892 and 1899, announced modern research on dynamical systems and chaos. In mathematical physics, he founded the properties of the Poincaré-Lorenz group, who were a few months later lead to the fundamental article of Einstein's relativity.

\parpic[l][t]{%
  \begin{minipage}{40mm}
    \fbox{\includegraphics[width=110px,height=140px]{img/medaillons/poisson.eps}}
  \end{minipage}
}
\textbf{Poisson, Siméon Denis} (1781-1840) was a French mathematician whose works were focused on definite integrals, electromagnetic theory and the calculus of probabilities. His family forced him to study medicine that he abandoned in 1798 to study mathematics at the École Polytechnique, where he was a student of Laplace and Lagrange, who became his friends. He taught at the École Polytechnique from 1802 and in 1808, he was appointed astronomer at Bureau des Longitudes, and at its creation, in 1809, professor at the Faculty of Science. The most important work of Poisson focuses on applications of mathematics to physics and mechanics. His \textit{Traité de mécanique} was a mechanical reference for many years. A memoir, published in 1812, contains the most usual laws of electrostatics and the theory that electricity consists of two fluids with similar elements that repel, while different elements attract. In pure mathematics, he published a series of articles on definite integrals, and his research on the Fourier series announced those of Dirichlet and Riemann on this topic. It is in the book \textit{Recherches sur la probabilité des jugements...} (1837), which is an important book on the calculus of probabilities, that for the first time appears the Poisson distribution (or "Poisson law"). Initially obtained as an approximation to the binomial law of Bernoulli it will became fundamental in many problems. The other publications of Poisson include the \textit{Théorie mathématique de la chaleur} (1831) and the\textit{ Théorie mathématique de la chaleur} (1835). The name of Poisson is attached to many mathematical and physical concepts (Poisson integral and equation in potential theory, Poisson brackets in the theory of differential equations, Poisson's ratio in elasticity and Poisson's constant in electricity).

\parpic[l][t]{%
  \begin{minipage}{40mm}
    \fbox{\includegraphics[width=110px,height=140px]{img/medaillons/poynting.eps}}
  \end{minipage}
}
\textbf{Poynting, John Henry} (1852-1912) was a physicist born in Lancashire and died in Birmingham who has worked, among others, on electromagnetic waves. He defined what is named the Poynting vector that represents the power per unit area that carries an electromagnetic wave and the direction of the energy flow. Poynting followed elementary school in a school run by his father. From 1867 to 1872 he attended  Owen College (now Manchester University) where he had as a professor Osborne Reynolds. From 1872 to 1875 he was student at the University of Cambridge where he obtained the honours in mathematics. In the late 1870s he worked at the Cavendish Laboratory under the direction of James Clerk Maxwell. In 1903 he was the first to realize that solar radiation could attract small particles towards the Sun, effect later recognized as the Poynting-Robertson effect. During the year 1884, he analysed the prices of commodity exchanges, including wheat, silk, and cotton, using statistical methods. He was professor of physics at Mason Science College (which later became the University of Birmingham) until his death.

\phantomsection
\addcontentsline{toc}{section}{R}

\parpic[l][t]{%
  \begin{minipage}{40mm}
    \fbox{\includegraphics[width=110px,height=140px]{img/medaillons/ramanujan.eps}}
  \end{minipage}
}
\textbf{Ramanujan, Srivanasa} (1887-1920) was born in Erode, a small village located 400 km south of Madras (India) in a poor family of the Brahmin caste. He spent his childhood in the town of Kumbakonam, where his father worked as an accountant by a draper. From the age of five, he attended different elementary schools before integrating the Town High School in 1898. In 1900, he began to develop his own mathematics based on his first book of mathematics, \textit{The plane Trigonometry}. He defines alone methods to solve the equations of the 3rd and 4th degree, then he also tries to solve those of the 5th degree, unaware that they can not be solved by radicals. We are then in 1902, it was at this time that Ramanujan buys his second (and last!) book that will draw his mathematical working methods, \textit{Synopsis of elementary results in pure mathematics}, compilation of about 6,000 theorems and other formulas by G.S. Carr. This book is essentially a book of results, mostly without proofs, that will influence the future style of Ramanujan, who also left very few details of his own mathematical proofs. At the age of seventeen his approach is already that of a researcher in mathematics. As his results are good, he received a scholarship enabling him to enter the Government College in Kumbakonam in 1904. However, he spends too much time on his research in mathematics and neglects other materials, which earned him the cancellation of the scholarship the following year. Without money, he goes away, without his parents authorization, to Vizagapatnam City where he continues his work on hyper-geometric series and relations between integrals and series. In 1906, he returned to High School again, at Madras this time, with the idea to pass an exam to enter the university. He attends classes a few months and then get sick. During the examination, he succeeded only in mathematics and fails everywhere else, which forbade him the entrance to the University of Madras. In the years that followed, he then goes on to develop his ideas alone, without any outside help and without knowledge of possible research topics, apart from those arising from the concepts presented in the Carr's book. Ramanujan also studied continued fractions and divergent series in 1908. He then falls very ill again and had to undergo, in 1909, an operation which it will be difficult to recover. He began to study and solve mathematical problems in the Journal of the Indian Society of Mathematics (SIM). In 1910, he developed relationships on modular elliptic equations. One year later, the publication of a brilliant article on Bernoulli numbers in the same newspaper earned him the recognition of his work by his peers. Although he has no university degree, he acquired the reputation of a mathematical genius in the area of Madras. The same year, he met the founder of the SIM, which allows him to get a temporary job as accountant in Madras and advises him to contact Ramachandra Rao, a donator member of the SIM. Thanks to this letter, Ramanujan gets the job and starts his work in 1912. He was then fortunate to be surrounded by people with a background in mathematics and interested by his works. The Chief Accountant of the Madras Port is a mathematician who published an article on the work of Ramanujan in 1913, \textit{On the distribution of primes}. On the other hand, a professor of the Madras Engineering College is interested in Ramanujan's abilities. Having himself studied in London, he wrote to one of his mathematics teachers, to whom he sends some results of Ramanujan. The University of Madras allocate later Ramanujan a scholarship in 1913. In 1914 Hardy brought Ramanujan at the Trinity College in Cambridge. This is the beginning of an extraordinary collaboration between the two men. In 1916, he obtained the title of Doctor of the University of Cambridge, even if he does not have the qualifications required to prepare a thesis. In 1918 Ramanujan was elected as a member of the Cambridge Philosophical Society. Three days later, probably the greatest honour of his career, his name appears on the election's list of members of the "Royal Society of London". He was proposed by an impressive list of well-known mathematicians. His election held on 1918 and he was also elected as a member of the Trinity College for six years. Ramanujan go back to India in 1919. However his health continues to deteriorate. He died the following year probably due to severe nutritional deficiencies. Ramanujan left behind a large number of unpublished notebooks (the famous Ramanujan Notebooks), filled with theorems that mathematicians are still studying. Today, his work has still applications in theoretical physics.

\parpic[l][t]{%
  \begin{minipage}{40mm}
    \fbox{\includegraphics[width=110px,height=140px]{img/medaillons/riccicurbastro.eps}}
  \end{minipage}
}
\textbf{Ricci-Curbastro, Gregorio} (1853-1925) born in Lugo and died in Boulogne was a mathematician specialised in differential geometry and one of the fathers of tensor calculus. After studying philosophy and mathematics, Ricci defended his doctoral thesis at the University of Pisa. In 1880, he was appointed professor of mathematical physics at the University of Padua. Levi-Civita was his student and helped to the development of Ricci's absolute differential calculus (1900) to explain mechanics, in abstract spaces (differentiable manifolds), relationships independent of the coordinate system used, inherent to studied the phenomenon (differential invariants). Associated with the differential geometry of Gauss and Riemann, the famous physicist Albert Einstein found in this new mechanics approach named "tensor calculus" (1916), the mathematical tools necessary for his theory of General Relativity.

\parpic[l][t]{%
  \begin{minipage}{40mm}
    \fbox{\includegraphics[width=110px,height=140px]{img/medaillons/riemann.eps}}
  \end{minipage}
}
\textbf{Riemann, Georg Friedrich Bernhard} (1826-1866) was a German mathematician. In high school, Riemann studied the Bible intensively, but he is distracted by mathematics. He even tries to prove mathematically the correctness of the Genesis. His teachers were amazed by his ability to solve complex problems in mathematics. In 1846, with the money from his family, he began studying philosophy and theology to become a priest in order to finance his family. In 1847, his father allows him to study mathematics. He first studied at the University of Göttingen where he met Carl Friedrich Gauss, then at the University of Berlin, where he had as teachers: Jacobi, Dirichlet and Steiner. In his thesis, presented in 1851 under the direction of Gauss, Riemann developed the theory of functions of a complex variable. In 1854 he gave a presentation which lays the foundations of differential geometry. He introduced the right way to extend to $n$-dimensional surfaces the results of Gauss himself. This presentation has changed the conception of geometry, opening the door to non-Euclidean geometry and to the theory of General Relativity. We also own him extensive works on integrals, following those of Cauchy, who gave in particular what we now name "Riemann integrals". Interested in gas dynamics, he lays the foundation for the analysis of partial differential equations of hyperbolic type. He will succeed to Dirichlet for the chair of Gauss in 1859. At 39, he died of tuberculosis.

\phantomsection
\addcontentsline{toc}{section}{S}

\parpic[l][t]{%
  \begin{minipage}{40mm}
    \fbox{\includegraphics[width=110px,height=140px]{img/medaillons/salam.eps}}
  \end{minipage}
}
\textbf{Salam, Abdus} (1926-1996) was a Pakistani physicist who won the Nobel Prize for Physics in 1979 for his works on electroweak interactions and his synthesis of electromagnetism and the weak interactions. Born in Jhang Sadar, he studied at the Government College in Lahore. At the age of fourteen, Salam received the best results ever recorded for the entrance examination at the University of Punjab. Persecuted by the Muslim majority of his country because of his religious affiliation (ahmadiste), he must quit his country. He refugees in Britain, where he obtained in 1952 a Ph.D. in mathematics and physics from the University of Cambridge. His doctoral thesis was a fundamental study on quantum electrodynamics. His work made him famous internationally. He returned to the Lahore Government College as a professor of mathematics, kept this place from 1951 to 1954 and then returned to Cambridge as a lecturer in Mathematics. He teaches in these schools, and in 1957 was appointed professor of theoretical physics at London's Imperial College. He remained there until his retirement. In 1959, he became the youngest member of the Royal Society at the age of thirty-three years. During the 1960s, Salam played an important role in establishing the nuclear research agency of Pakistan and the space research agency of Pakistan, where he was the founding Director. In 1964, he became director of the newly created International Centre for Theoretical Physics in Trieste. That same year, he was awarded the Hughes Medal. In 1967, with the physicist Steven Weinberg, Salam proposed a theory to unify electromagnetism and weak interactions between elementary particles, theory that will be confirmed by experience. Salam will be the first Muslim to win the Nobel Prize for Physics in 1979, together with physicists Sheldon Lee Glashow and Weinberg.

\parpic[l][t]{%
  \begin{minipage}{40mm}
    \fbox{\includegraphics[width=110px,height=140px]{img/medaillons/samuelson.eps}}
  \end{minipage}
}
\textbf{Samuelson, Paul} (1915-2009) was an American economist, Nobel Prize in Economics in 1970 and leader of the school that he named "neoclassical synthesis", which meant endorse both Keynesian macroeconomics and neoclassical microeconomics lessons. Samuelson is considered one of the father of the current mainstream microeconomics and possibly one of the pioneer economists to generalize in an economic framework, the use of mathematical models developed for thermodynamic analysis. Thus, he would have helped to put the economic discipline primarily literary in some field of mathematics highly formalized and axiomatized.

\parpic[l][t]{%
  \begin{minipage}{40mm}
    \fbox{\includegraphics[width=110px,height=140px]{img/medaillons/savart.eps}}
  \end{minipage}
}
\textbf{Savart, Felix} (1791-1841) was a surgeon and physicist, born in Ardennes and died in Paris. Inventor of the sonometer and also of a gear that bears his name and the polariscope. He laid the foundations of molecular physics. With the physicist Jean-Baptiste Biot, he measured the magnetic field created by a current and formulated the Biot-Savart law. He also studied the properties of vibrating strings. He was a member of the Académie des Sciences, elected in 1827, and Chair of General and Experimental Physics of the Collège de France, appointed in 1836, succeeding André-Marie Ampère. He was elected as foreign member of the Royal Society in 1839. His name was given to a unit of measurement of musical intervals.

\parpic[l][t]{%
  \begin{minipage}{40mm}
    \fbox{\includegraphics[width=110px,height=140px]{img/medaillons/say.eps}}
  \end{minipage}
}
\textbf{Say, Jean-Baptiste} (1767-1832) was an economist, journalist and French industrialist born in Lyon and died in Paris. He comes from a family of merchants who emigrated to Amsterdam (Netherlands) and Geneva (Switzerland). It was during a trip to Britain, where the industrial revolution was underway, that he will adopt liberal ideas and especially the theories of Adam Smith, for which he will be a strong advocate when returning to France. In 1789, he published the brochure: \textit{Liberté de la presse}. In 1792, he participated in military campaigns of the French Revolution in Champagne. Initially working in a bank, he managed after a cotton mill at Auchy-lès-Hesdin at the Pas-de-Calais. His many books on political economy made that he was appointed professor at the Conservatoire National des Arts et Métiers in 1821, then at the Collège de France in 1830. The "Say's law" or "law of markets", states that more the producers are numerous and the productions multiple, more the opportunities are easy, varied and vast. In an economy where competition is free and perfect, crises of overproduction are impossible (...). There can't be an imbalance in global market economies and free enterprise (...), there is a spontaneous balancing economic flows (production, consumption, savings, investment). This law is sometimes wrongly reduced to the formula: any supply creates its own demand. The best summary of this approach would be: we spend only the money that we won. The supply-side economics, in the tradition of Say, opposes economic demand, which is that of Malthus and later Keynes.

\parpic[l][t]{%
  \begin{minipage}{40mm}
    \fbox{\includegraphics[width=110px,height=140px]{img/medaillons/schaefer.eps}}
  \end{minipage}
}
\textbf{Schaefer, Milner Baily} (1912-1970) was born in Wyoming and died in San Diego. Schaefer studied at the University of Washington where he received a Bachelor of Science in 1935. After his bachelor, he worked in the fisheries department of the state of Washington in Seattle. From 1937 to 1942 he worked at the Comission of the salmon fisheries of the Pacific Westminster, British-Columbia. He served in the Navy during the war and thereafter, he held various positions as a fisheries biologist. After completing his Ph.D. at the University of Washington in 1950, Schaefer became Director of Investigations of the IATTC (Inter-American Tropical Tuna Commission), an international commission of fisheries. During the 10 years that followed, he worked on the theory of the dynamics of the fishery and developed a population model of marine species known under the name "Schaefer model". During the 1950s, Schaefer became increasingly involved in several committees, groups and organizations concerned with marine resources, particularly fishing and all aspects of oceanography. During this period, he lectured on the dynamics and exploitation of fish populations. In 1962, he resigned from his position as director of investigations at the IATTC for the position of Director of the Institute of Marine Resources of the University of California while serving as a scientific advisor to the IATTC.

\parpic[l][t]{%
  \begin{minipage}{40mm}
    \fbox{\includegraphics[width=110px,height=140px]{img/medaillons/scholes.eps}}
  \end{minipage}
}
\textbf{Scholes, Myron} (1941-) was born in Ontaria, he presented his Ph.D. in 1969 at the University of Chicago. In 1988 he held the Frank E. Buck chair as Professor of Finance at the Graduate School of Business at Stanford University (California) where he also directs research for the Hoover Institution. He received the Nobel Prize in Economics in 1997 with Fischer Black, for the development of an evaluation method of financial derivative instruments (innovative mathematical results to estimate the risks associated with options) that have opened new horizons in the field of economic evaluations. The co-winner of Myron Scholes, Robert Merton, played a very important role in the development of this method of evaluation as well on the applications it has allowed to improve the management of risks related to new financial products. Already in 1900, Louis Bachelier, presented at the Sorbonne a visionary doctoral thesis: \textit{Théorie de la spéculation}. In the 1960s, authors like James Boness and Paul Samuelson (Nobel Prize in Economics in 1970) proposed models to determine the equilibrium price of options. Their assumptions have not proved to be sufficiently realistic for real applications, but improvements to these models in the early 1970s have yielded more satisfactory results. It was in 1973 that Black and Scholes put their skills together and propose the first version of the formula for option pricing which earned them the Nobel Prize. If Myron Scholes and Fischer Black had the fundamental intuition of the demonstration, they took for basis the research base equilibrium model of financial assets (or Capital Asset Pricing Model: CAPM) of their compatriot William Sharpe rewarded for this by the Nobel jury in 1990 (the other two winners were Harry Markowitz and Merton Miller).

\parpic[l][t]{%
  \begin{minipage}{40mm}
    \fbox{\includegraphics[width=110px,height=140px]{img/medaillons/schrodinger.eps}}
  \end{minipage}
}
\textbf{Schrödinger, Erwin} (1887-1961) was born and died in Vienna (Austria). He entered the Gymnasium of that city in 1898. Almost from the first day of class until he left school eight years later, Schrödinger was an excellent student. He was always first in his class thanks to his hard work between the four walls of his personal office. He continued his studies at the University of Jena. In 1920 he was appointed professor at the Stuttgart Technical High School (Germany) and the next year at the University of Breslau. In 1927, he succeeded Max Planck at the University of Berlin. Israelite, he left the country with the advent of National Socialism to go to Oxford where he obtained a professorship in 1933. Seven years later, he became professor of theoretical physics at the Dublin Institute for Advanced Studies of the Irish free State. He will return to Austria only in 1956. Schrödinger, like his contemporary Albert Einstein ,was horrified to learn by heart and be forced to memorize unnecessary facts. Schrödinger's early work focused on the study of color and quantum theory. But he is primarily known for his research in wave mechanics, discipline developed by the French Louis de Broglie. We own him the Schrödinger equation developed in 1926 to calculate the wave function of a particle moving in a field. By establishing this propagation equation, he gives an intuitive tool to quantum mechanics indispensable today (unlike the abstract Heisenberg matrix approach) that Einstein qualified as a Genius Idea. With that of Werner Heisenberg, Schrödinger's theory forms the basis of quantum mechanics. In 1933, Schrödinger shared the Nobel Prize in Physics with Paul Dirac for their contribution to the development of this new discipline. Schrödinger also attempt to apply his theory to biology and genetics in his books \textit{What is life} (1944) and \textit{Science and Humanism} (1951).

\parpic[l][t]{%
  \begin{minipage}{40mm}
    \fbox{\includegraphics[width=110px,height=140px]{img/medaillons/schwartz.eps}}
  \end{minipage}
}
\textbf{Schwartz, Lawrence} (1915-2002) was a French mathematician born and died in Paris. His work is mainly related to Analysis. Old student of the École Normale Supérieure, Laurent Schwartz taught from 1959 to 1960 and from 1963 to 1983 at the École Polytechnique. In 1975 he was elected member of the Academie Des Sciences. His thesis (1943) focuses on the study of approximation and sums of exponentials. Thanks to the theory of distributions, whose original idea came in 1945, he won the Fields Medal in 1950. The language and notation of Schwartz distributions have been naturally adopted by almost all mathematicians and are the natural framework of the theory of partial differential equations. From 1959 to 1962, Schwartz dedicated his time to theoretical physics: the use of distributions allows him to found a correct mathematical formulation for the theory of elementary particles. He has also conducted research on Radon measures on arbitrary topological spaces and has written various publications on cylindrical probabilities and disintegrations of measures.

\parpic[l][t]{%
  \begin{minipage}{40mm}
    \fbox{\includegraphics[width=110px,height=140px]{img/medaillons/schwarzschild.eps}}
  \end{minipage}
}
\textbf{Schwarzschild, Karl} (1873-1916) was a mathematician, astronomer and physicist born in Frankfurt and died in Potsdam who predicted the existence of black holes. His curiosity for the stars appeared from his early school years, when he built a small telescope. Because of to this interest, his father introduced him to a friend mathematician who had a private observatory. Schwarszchild learned to use a telescope and studied alone more advanced mathematics than at school. He became famous with his first two papers on the theory of orbits published at the age of sixteen when he was still in college. He studied at the University of Strasbourg, Munich, and received his Ph.D. at the age of twenty-three for the works on the theories of Henri Poincaré. He was then hired as an assistant at the Kuffner Observatory in Ottakring. He devoted himself mainly to photometry: he performed pioneering works to improve photographic plates and implement their use in astronomy, and in the spectral study of the stars. From 1901 to 1909 he officiated as a professor at the prestigious Institute of Göttingen, where he had the opportunity to work with celebrities such as David Hilbert and Hermann Minkowski. He then held a position at the Astrophysical Observatory of Potsdam in 1909. Schwarzschild is best known for his contributions to theoretical physics, among in the Sun physics as in General Relativity, or stellar kinematics, as well as in various fields of astrophysics. In 1916, he founded a quantity named the "Schwarzschild radius" in the framework of the theory of relativity, stated shortly before by Albert Einstein. When a sufficiently massive star explodes in a supernova, the gravitational contraction produced what is called a "black hole": almost nothing, not even light, can escape this intense gravitational field. When the radius of a gaseous mass falls below the "Schwarzschild radius" for this mass, it collapses into a black hole.

\parpic[l][t]{%
  \begin{minipage}{40mm}
    \fbox{\includegraphics[width=110px,height=140px]{img/medaillons/shannon.eps}}
  \end{minipage}
}
\textbf{Shannon, Claude Elwood} (1916-2001) was born in Massachusetts and died in Migichan. He was a mathematician specializing in Applied Mathematics and electrical engineer, who developed the theory of communication, now known as the "Information Theory". Shannon took courses at the University of Michigan in 1940 and obtained his Ph.D. (he wrote his thesis demonstrating that electrical applications of Boolean algebra could construct any logical, numerical relationship) from the Massachusetts Institute of Technology (MIT), Faculty of which he became a member in 1956, after working in the Bell Telephone laboratories. In 1949, Shannon published the \textit{A Mathematical Theory of Communication}, an article in which he presented his initial concept for a unification theory of the transmission and processing of information. Shannon contributed to the field of cryptanalysis for national defense during World War II, including his basic work on codebreaking and secure telecommunications. Information, according to this theory include all types of messages, including those sent along the nerve channels of living organisms. The information theory is now important in many areas.

\parpic[l][t]{%
  \begin{minipage}{40mm}
    \fbox{\includegraphics[width=110px,height=140px]{img/medaillons/sharpe.eps}}
  \end{minipage}
}
\textbf{Sharpe, William Forsyth} (1934-) is an economist born in Boston. The Sweden Royal Academy of Sciences has awarded in 1990 the Nobel Prize in economics to three American professors: Harry Markowitz, Merton Miller and William Sharpe. Even if the rewarded works were already old and are situated mostly between 1950 and 1970, the Academy decided that the winners were innovators in the field of the theory of financial economics and corporate finance. Indeed, they all contributed to emerge from the shadow of some American universities, a new discipline: quantitative finance. It was the first time that the Royal Swedish Academy rewarded work dealing with stock markets and portfolio management rather than economic equilibrium. William Sharpe, of Stanford University, was rewarded for his equilibrium model of financial assets and for his work on the theory of price formation for financial assets. He was also engaged in his research on the path opened by Harry Markowitz. This last had indeed developed a complicated procedure for selecting stocks to optimize an investment portfolio. But the implementation of his model was quickly raised by practical problems, at the point that the collection of information and treatment became almost impossible with the computers of 1960s. This is why William Sharpe began searching for an easier method of selecting efficient portfolios. He discovered that the variations in the profitability of each title are linked linearly to changes in the overall market, as measured by the concerned index market (e.g. Standard \& Poor 500 index in the United States, or CAC 40 in France). The number of necessary statistics was greatly reduced: 302 statistics instead of 3,150 in the Markowitz model for 100 titles, 602 instead of 20,300 for 200, 10,002 instead of 125,750 for 300 titles, the calculation was immediately easier. It is from this concept, simple in appearance, that Sharpe discovered his famous Beta $\beta$ coefficient linking the profitability of a security to the market index and also being a measure of the risk associated with market volatility. Beyond their practical contribution, the works of Sharpe have contributed decisively to the development of a pricing theory for financial assets more known as a "Capital Asset Pricing Model" (CAPM).

\parpic[l][t]{%
  \begin{minipage}{40mm}
    \fbox{\includegraphics[width=110px,height=140px]{img/medaillons/smith.eps}}
  \end{minipage}
}
\textbf{Smith, Adam} (1723-1790) was a scottish philosopher and economist, born in Kirkcaldy and died in Edinburgh (Scotland). He studied at the universities of Glasgow and Oxford. From 1748 to 1751, he taught rhetoric and literature in Edinburgh. During this time, he meets the philosopher David Hume, whose ideas had a great influence on the conceptions of Smith on ethics and economics. Smith was appointed professor of logic in 1751 and professor of moral philosophy in 1752 at the University of Glasgow. Later, he gathered the ethics courses that he conducted and published in his first masterpiece entitled \textit{Theory of Moral Sentiments}, in 1759. In 1763, he resigned his professorship to accompany the Duke of Buccleuch in a journey of eighteen months in France and Switzerland, as a tutor. From 1766 to 1776 he lived in Kirkcaldy where he worked on his main book: \textit{The Wealth of Nations}. Smith was later appointed commissioner of customs in Edinburgh in 1778, a position he held until his death. In 1787 he was also appointed rector of the University of Glasgow. His famous treatise\textit{ An Inquiry into the Nature and Causes of the Wealth of Nations} (1776), the first study attempting to describe the nature of capital and the historical development of industry and trade between European countries, caused him to be considered as the father of modern economics. \textit{The Wealth of Nations} is the first essay on the history of economic science which considers political economy as an autonomous discipline, distinct from political science, ethics and jurisprudence. Smith proposes a process analysis of production and distribution of wealth, and shows that the main source of any income, that is to say the basic forms in which wealth is distributed, are rents, wages and profits. The Wealth of Nations argues against the physiocrats the principle that labor is the source of all wealth, and presents the development of the industry as a source of increased production. For Smith, the theorist of liberal capitalism, the economics and moral comes from competition, production and trade of goods can only be stimulated, and consequently the general standard of living improved, when governments regulate and control a minimum industrial and commercial activities. To describe this situation, he speaks of a natural order set by the "invisible hand", which may naturally converge the sum of individual interests to the general interest. As a result, too much government intervention in the context of free competition could only be bad.

\parpic[l][t]{%
  \begin{minipage}{40mm}
    \fbox{\includegraphics[width=110px,height=140px]{img/medaillons/sommerfeld.eps}}
  \end{minipage}
}
\textbf{Sommerfeld, Arnold} (1868-1951) was a German physicist born in Königsberg, and died Münch. He studied mathematics and natural sciences at the Königsberg Universität where he received his Ph.D. in 1891. He successively held the chairs of mathematics in Clausthal (1897), Applied Mathematics at Aix-la-Chapelle in France (1900) and theoretical physics in Munich (1906-1931). In 1897, he began with C. F. Klein, a treaty in four volumes of the gyroscope, that he needed thirteen years to complete and at the same time he also did research in other areas of applied physics and engineering, such as friction, lubrication and radio. We own him the improvement of Bohr's model (1916) introducing elliptical orbits and relativistic corrections. This new model, which implies a dependence of energy vis-à-vis the second quantum number, can explain the fine structure of spectral lines emitted by atoms. Sommerfeld also introduced the famous "fine structure constant". He was also interested in Lorenz and after Drude's model of free electrons which explains some properties of metals, particularly conduction, whereas in quantum behavior of electrons. He participated to the development of band theory in solid state physics, presenting in 1928 the idea that electrons occupy quantified states in the material.

\parpic[l][t]{%
  \begin{minipage}{40mm}
    \fbox{\includegraphics[width=110px,height=140px]{img/medaillons/stokes.eps}}
  \end{minipage}
}
\textbf{Stokes, George Gabriel} (1819-1903) was a mathematician and physicist, born in Ireland and died in Cambridge. In 1841, he graduated with honours from the University of Cambridge and began a career as a researcher. Influenced by his former teacher, he devoted himself to the study of viscous fluids. He published in 1845 the results of his works on the movement of fluids in his thesis \textit{On the theories of internal friction of the fluids in motion}. His mathematical approach describing the flow of an incompressible Newtonian fluid in a three-dimensional space, adding a viscous force from the Euler equations (\textit{General principles of fluid motion}, 1755), is the origin of the Navier-Stokes equations. All his researches are synthesized by his treatise \textit{Report on recent research in Hydrodynamics}, published in 1846, the founding text of hydrodynamics. In 1849 he became a professor at the chair of mathematics at the same university. Elected in 1851 at the Royal Society, he will be the president from 1885 to 1890. The last three mentioned positions were occupied by Isaac Newton. He received the Smith Prize in 1841, the Rumford Medal in 1852 and the Copley Medal in 1893.

\parpic[l][t]{%
  \begin{minipage}{40mm}
    \fbox{\includegraphics[width=110px,height=140px]{img/medaillons/stefan.eps}}
  \end{minipage}
}
\textbf{Stefan, Josef} (1835-1893) was an Austrian physicist born in Sankt Peter near Klagenfurt and who died in Vienna. The research works of Stefan include kinetic theory of gases, especially hydrodynamics and radiation theory. After studying at the Wien Universität where he obtained his doctorate in 1858, appointed Privatdozent of mathematical physics, he became professor of physics in 1863, then director of the Institute of Physics (1866). Member of the Academy of Sciences in Vienna, he was the secretary from 1875. Before the work of Stefan, G. R. Kirchhoff had already described the properties of the "perfectly black body", that can absorb all incident radiation and emit a broad spectrum of wavelengths. Stefan proofs empirically in 1879 that the intensity of the black body radiation is proportional to the 4th power of its absolute temperature, relation known since as the "Stefan-Boltzmann law", Boltzmann having also deduced the same results from thermodynamic considerations . This law is one of the important first steps that led to the interpretation of the black body radiation and quantum theory of radiation.

\parpic[l][t]{%
  \begin{minipage}{40mm}
    \fbox{\includegraphics[width=110px,height=140px]{img/medaillons/sturm.eps}}
  \end{minipage}
}
\textbf{Sturm, Charles François} (1803-1855) was born in Geneva (Switzerland) and died in Paris. After studying at the University of Geneva, Sturm went to be tutor in the family of De Broglie in Paris where he attended the greatest scholars of his time and where he settled permanently starting 1825. In 1826 he determines the speed of sound in water, which earned him the following year, the grand prize of mathematics proposed for the best thesis on the compressibility of liquids. In 1829, he stated the famous theorem that bears his name, essential for the study of the properties of the roots of an algebraic equation which specifies the number of real roots of a numerical equation between two limits. He published the proof of this theorem in 1835. In 1830, in conjunction with his friend Liouville, he focused on the problem of the general theory of oscillations and studied differential equations of second order (Sturm-Liouville problems) in several articles, including \textit{Sur les équations différentielles linéaires du second ordre} (1836) and \textit{Sur une classe d'équations à différences partielles} (1836). The methods used will be at the origin of a lot of mathematical works and discoveries. He was elected in 1836 to the Académie des Sciences and work at the École Polytechnique. Succeeding to Poisson, he taught, from 1840, at the Faculté de Paris (mechanical chair). His C\textit{ours d'analyse de l'École Polytechnique} (1857-1863) and his \textit{Cours de mécanique de l'École Polytechnique} (1861) will be published after his death in Paris.

\phantomsection
\addcontentsline{toc}{section}{T}

\parpic[l][t]{%
  \begin{minipage}{40mm}
    \fbox{\includegraphics[width=110px,height=140px]{img/medaillons/taylor.eps}}
  \end{minipage}
}
\textbf{Taylor, Brook} (1685-1731) was an English mathematician born in Edmonton and died in London. He is famous for his contributions to the development of infinitesimal calculus. Taylor was educated at Saint John College, Cambridge. He obtained in 1708 a remarkable solution to the problem of the center of oscillation, which however remained unpublished until 1714 when his priority right was disputed by John Bernoulli. Taylor's book, \textit{Methodus incrementorum directa }(1715), added to higher mathematics a new chapter, named nowadays the "calculus of finite differences". Among other ingenious applications, he used it to determine the movement pattern of a vibrating string with success by reducing the problem to the principles of mechanics. The same book contains the famous formula known as "Taylor's theorem", who's importance appeared only in 1772, when Louis Lagrange realized its power and made it the fundamental principle of differential calculus. In his essay \textit{Linear Perspective}, Taylor sets out the principles of art in an original and more general form than any of his predecessors, but the work suffered from the confusion and lack of clarity that affected most of his writings. Taylor was elected to the Royal Society in 1712. He sat in the same year at the committee to settle priorities disputes between Newton and Leibniz and was secretary of the society from 1714 to 1718. From 1715, his research took a philosophical and religious orientation.

\parpic[l][t]{%
  \begin{minipage}{40mm}
    \fbox{\includegraphics[width=110px,height=140px]{img/medaillons/teller.eps}}
  \end{minipage}
}
\textbf{Teller, Edward} (1908-2003) was a nuclear physicist born in Budapest and died at Stanford. He left Budapest in 1926 to go to Karlsruhe (Germany) to study chemistry, but soon he will develop an affinity with the new theory of quantum physics which led him to study at the Leipzig Universität where he obtain his Ph.D. at the age of twenty-two. Teller won this title under the direction of Werner Heisenberg who participated actively in the later German nationalists camp during World War II. In 1935, Teller expat to the United States and its expertise in advanced physics led him to make a lot of relationships and a good reputation in the scientific community. He was named professor in many American universities and worked on the Manhattan Project in 1942 where he led the very important work that helped to create the first nuclear fission bomb. The work done, Teller argued for the continuation of work looking for a thermonuclear bomb by fear of the Russian advance in this field (Teller was anticommunist and very good friend of Landau who was arrested by the communist police). Teller persuaded the U.S. government to finance research for a hydrogen bomb and led the successful works which make him considered today as the father of the H-bomb.

\parpic[l][t]{%
  \begin{minipage}{40mm}
    \fbox{\includegraphics[width=110px,height=140px]{img/medaillons/tesla.eps}}
  \end{minipage}
}
\textbf{Tesla, Nikola} (1856-1943) was a genius Serbian engineer and inventor in the field of electricity who died in New York. He is often considered as one of the greatest scientists in the history of technology, having for over 300 patents (which are mostly affected to Thomas Edison) dealing with new methods to address the conversion of energy. In 1875, he entered the Polytechnic in Graz (Austria), where he studied mathematics, physics and mechanics. A scholarship given by the administration of the Military Frontier (Vojna Krajina) avoiding him money problems. This did not however prevent him to work hard to assimilate the program for the first two years of study in one year. The following year, the removal of the Military Frontier removes any financial assistance to Tesla, apart from that, very small, that can bring his father, which does not allow him to complete his second year of study. Tesla gained experience in telephony and electrical engineering before immigrating to the United States in 1884 to work for Thomas Edison in New York City. We owe him contributions to the design of the modern alternating current (AC) electricity supply system, the asynchronous electric motor, polyphase alternator, mounting three-phase star, the rotary converter. Tesla discovered the principle of wave reflection on objects in 1900, he studied and published, despite financial problems, the foundations of what would become almost three decades later the radar.

\parpic[l][t]{%
  \begin{minipage}{40mm}
    \fbox{\includegraphics[width=110px,height=140px]{img/medaillons/thom.eps}}
  \end{minipage}
}
\textbf{Thom, René} (1923-2002) was a French mathematician author of important works in differential topology. Born in Montbéliard and died in Bures-sur-Yvette, Thom was a student at the École Normale Supérieure. In 1958, he received the Fields Medal for his theory of cobordism (equivalence relation between compact differential manifolds). In a communication at the conference of Strasbourg (1951), Thom establishes that if the zeros of a polynomial ideal form a variety, it is a border variety, and his thesis, \textit{Espaces fibrés en sphères et carrés de Steenrod }(1951), already contains the germ of the main cobordistes methods. It is in the last chapter of a dissertation of 1954 (\textit{Quelques Propriétés globales des variétés différentiables}) that the theory of cobordism is exposed for the first time. After 1955, Thom has studied especially laminated spaces and stratified sets and morphisms. We owe him results on the approximations of differentiable transformations and their singularities, comparisons of differentiable structures on a triangulated manifold and a Morse theory for laminated varieties. He is also one of the first to use techniques of "surgery" varieties. Since 1969, Thom is devoted to the applications of topology to the phenomena of life. To describe the birth and evolution of forms, he has developed a specific mathematics: his catastrophe theory is a theory of singularities of certain differential equations. Specifically, it allows, from observed phenomena, to trace their unknown causes, at least partially. Thom gave a presentation of his work in the book \textit{Stabilité structurelle et morphogenèse} (1973).

\parpic[l][t]{%
  \begin{minipage}{40mm}
    \fbox{\includegraphics[width=110px,height=140px]{img/medaillons/thales.eps}}
  \end{minipage}
}
\textbf{Thales of Miletus } ($\sim$624 BC. - $\sim$524 BC.) Is one of the first mathematician whose history has retained the name. He was born in Miletus in Minor Asia on the Mediterranean coast of modern Turkey. More than just a mathematician Thales was a universal scholar, curious about everything, astronomer and philosopher, very observant. We did not prove what we say at the time of Thales, we only noticed properties. But how Thales thought, analyzed situations, investigated the causes and effects make him one of the forerunner of science (he based everything by observation and experimentation). One of the big questions for Thales was water, and the causes of the rain. He had noticed that the air turned into rain, and he searched desperately answers. Thales has formulated several geometric properties that he learned perhaps from the Egyptians, instead some elements of this properties were already known long age, he laid the foundations of reasoning with ideal figures through which he obtained several results known today as "Thales' theorem". But the must know fact of Thales is undoubtedly the prediction of a solar eclipse, probably that of 8 May 585 BC. We also owe him the first discovery of electricity through two experiments. First he noticed that amber had the property of attracting light materials . Another experiment realized in Magnesia..., in -600, allows him to highlight the properties of magnetization of iron oxide.

\parpic[l][t]{%
  \begin{minipage}{40mm}
    \fbox{\includegraphics[width=110px,height=140px]{img/medaillons/turing.eps}}
  \end{minipage}
}
\textbf{Turing, Alan} (1912-1954) In his theoretical work in the fields of logic and probabilities, Turing is considered, if not the founder of computers, in any case, as one of the spiritual fathers of artificial intelligence. Born in Paddington (United-Kingdom) Turing does a normal education despite a brilliant mind and net predispositions for sciences. In 1926 he went at Sherborne School. From 1931 to 1934, Turing studied Mathematics at the King's College of the University of Cambridge. During this period, he discovered the work of John von Neumann on quantum mechanics. Stimulated by his researches, he began the study of problems of probability and logic. After his graduation, he learned in the summer of 1936 the developments of Max Newman on a mathematical theory of Gödel incompleteness and the question of Hilbert's decidability. If for many proposals, it is easy to find an algorithm, what about those for which the algorithm, not rigorous enough, is not enough to validate the proposal? Should we infer that they can not be validated? It is now in this direction that the researches of Turing will focus. In 1936 he was awarded the Smith price for his work on probabilities and the concept of "Turing machine". This concept is the basis of all theories of automata and more generally for the theory of computability. The purpose is to formalize the principle of algorithm, represented by a sequence of instructions - acting in sequence on input data - that might provide a result. This formalization requires Turing to develop the notion of computability and identify a class of "decidable" problems. This led him to introduce a new class of functions: "computable functions in the sense of Turing". During his Ph.D. at Princeton University from 1936 to 1938, Turing conceived the idea of building a computer. Returning to Cambridge, he studied mathematics and focuses on the Riemann zeta function. World War II soon offers him the opportunity to put into practice his theories. It is in the British Communications Department of the Ministry of Foreign Affairs that he is confronted to the Enigma secret code, name of the machine used by the German Navy to communicate with submarines. The encryption used by the Nazis always escaped the traditional methods of investigation. With the collaboration of W. G. Welchman, Turing was able to break the code by applying his new method and. Once war finished, Turing joined the National Physical Laboratory, where he began, in competition with U.S. projects to create the first computer. Technological advances suggest him to achieve this goal in the near future. In 1948, thanks to Newman, he obtained a position as a lecturer in mathematics at the University of Manchester that he held until the end of his life. Two years later, he participated with Frederic Williams and Tom Kilburn at the realization of an electronic computer, the Mark I, and wrote on this occasion a programming manual. At the same time, he publishes \textit{Can a machine think?} in which he summarizes the conceptual and mathematical basis of programmable electronic computer and sums up his philosophy of "intelligent machine". He also describes the famous "Turing Test" which is an experiment where a man holds a conversation with a machine and muss guess if a human or processor is behind the screen. Turing was convinced that everything was a only problem of information and that the development of technologies will allow in the next fifty years machines able to defeat the human being at least five minutes. Turing committed suicide by cyanide poisoning because of homophobic persecutions in the United-Kingdom.

\phantomsection
\addcontentsline{toc}{section}{V}

\parpic[l][t]{%
  \begin{minipage}{40mm}
    \fbox{\includegraphics[width=110px,height=140px]{img/medaillons/vanderwaals.eps}}
  \end{minipage}
}
\textbf{Van Der Waals, Johannes Diderik} (1837-1923) was a Dutch physicist born in Leiden (Netherland) and died in Amsterdam. Van Der Waals was first teacher at the age of twenty before becoming, after a lot of solitary efforts, teacher in middle school (1863). He attended classes at the University of Leiden from 1862 to 1865 and teaches as professor of physics at Deventer and The Hague (1866). In 1873, he was received his PhD at the University of Leiden after defending a dissertation entitled: \textit{Over de continuiteit van den gas en vloeistoftoestand} that contains the presentation of the state equation that bears his name and led to much more positive results than the classical equation of ideal gases near the liquefaction zone. This study contributed decisively to support the idea of the existence of intermolecular forces of attraction and to determine the role of molecules bulk volume in the behavior of gas at high pressure, two concepts poorly understood at this time. The rapid success of this new theory is illustrated by the many translations of the original paper that followed its presentation. It is now known that the van der Waals equation is still imperfect and it would be foolhardy to try to preserve the name of "real gas equation" which was once awarded. Indeed, nowadays state equations even more appropriate can achieve an approximation much more accurate which are generally derived from kinetic considerations based on molecular virial theorem forces. From 1877 to 1907, the date of his retirement, Van der Waals was appointed professor of physics at the University of Amsterdam. It was during this period that he made known his law named "the theorem of corresponding states" (1880). This equation of state for all pure bodies greatly contributed, too, to his reputation because it was later used as a guide for prior tests to the liquefaction of hydrogen and helium. From another point of view, the van der Waals contribution is also considered one of the first attempts to express the laws of physics in terms of reduced variables. Among other works of Van der Waals, we can found a major contribution to the theory of binary mixtures and molecular study of capillarity. He received the Nobel Prize in Physics in 1910 for his work on the state equation of gases and liquids aggregation.

\parpic[l][t]{
  \begin{minipage}{40mm}
    \fbox{\includegraphics[width=110px,height=140px]{img/medaillons/viete.eps}}
  \end{minipage}
}
\textbf{Viète, François} (1540-1603) was born in Fontenay-le-Comte (France) and died in Paris. Viète is known today as the inventor of modern algebra. However, at his time, he was best known as a master of requests and Privay Councillor of Henry IV than as a mathematician. His whole life is marked by the duality of a brilliant political career and a strong work practice on the highest problems of mathematics of his century. His scientific work has suffered from numerous political concerns and the limited time they left him. The fact remains that the contribution of Viète to the development of mathematics in the late 16th century is very important. It is characterized by the systematic introduction of the literal representation in algebraic problems for both the unknown and the known quantities, which presents the main advantage of treating the general case and the special cases and not to focus on the structure of problems rather than their expression. Viète in his youth was a student of the Franciscan, at the Collège de Cordeliers. He continued his studies at the Faculté de Droit de Poitiers and entered in the active life as a lawyer. He was appointed council of the Brittany Parliament in 1573, staying there only a few, quite occupied he is by his mathematical work and confidential missions assigned by the king. We found then his trace in Paris in 1579 where he published the \textit{Canon mathematicus}, accompanied by the \textit{Liber singularis}. Appointed master of requests of the king's household in 1580, he resigned from his position in 1585, as a result of people conflicts. In 1589, he is at Tours and prepares the publication of his scientific work. He is also responsible of statistical cryptography for the King. He returned to Paris with the King and was appointed as Privy Councillor. Viète will die after a long period of decline because of disease.

\phantomsection
\addcontentsline{toc}{section}{W}

\parpic[l][t]{
  \begin{minipage}{40mm}
    \fbox{\includegraphics[width=110px,height=140px]{img/medaillons/walras.eps}}
  \end{minipage}
}
\textbf{Walras, Leon} (1834-1910) was a French economist born in Evreux (France) and died in Clarens (Switzerland). He is the son of Auguste Walras, a French economist whose ideas greatly influence her son in the field of social and financial reform in general. He studied at the Collège de Caen in 1844, and the Collège de Douai in 1850. He graduated bachelor-ès-Letters in 1851 and bachelor-ès-Sciences in 1853. The same year, he is not declared eligible for Polytechnic and also after a second trial. In 1854, he is received as external student at the École des Mines de Paris, but he has no interest in engineering and he left the school. Appointed professor at the University of Lausanne (Switzerland), Walras denounced, from the 1870s, the liberal economic theories taught in universities, that he felt unable to explain the economic problems of his time. In his \textit{Éléments d'économie politique pure} (1874), his critics focus especially on the theories of labor value and rent but through it, it is all the classical heritage that he challenges (including that of Adam Smith). Influenced by the mathematician Antoine Cournot, he is one of the first to introduce systematically mathematics in economics. Walras places Companies at the heart of the economy and focuses on its actions in the context of competition between agents, as well as the interdependence of all economic markets: the market of products (goods and services ) and those of production factors (including land, labor and capital). He wonders how to set prices and quantities simultaneously, and defines the problem of general equilibrium, that is to say, the stability of equilibria in all markets. Attention to this issue characterizes the members of the Écolde de Lausanne, in particular the successor of Walras, Vilfredo Pareto. With the Austrian Carl Menger and the Britain Stanley Jevons, who he did not know when he undertook this path, he is considered as one of the founders of neoclassical marginalism.

\parpic[l][t]{
  \begin{minipage}{40mm}
    \fbox{\includegraphics[width=110px,height=140px]{img/medaillons/weber.eps}}
  \end{minipage}
}
\textbf{Weber, Wilhelm} (1804-1891) was a German physicist born in  Wittenberg  and died in Göttingen specialized in electrodynamics. Weber wrote in 1824 a treatise on the wave motion with his older brother, Ernst Heinrich Weber, well known anatomist, and studied with his brother Eduard Friedrich Weber the walking mechanism (1836). In 1831, on the recommendation of Carl Friedrich Gauss, he was hired by the university of Göttingen as professor of physics, at the age of twenty-seven. As a teacher he did what a lot of students would like: a free of charge access to the college laboratory. At Göttingen he collaborated with Carl Friedrich Gauss on the study of geomagnetism, and he connected their laboratories by an electric telegraph: it was one of the first telegraph transmissions that we know. His greatest achievement was that he brought to Leipzig, with F.W.G. Kohlrausch: he determined the ratio of electrostatic and electrodynamic units (Weber's constant) which proved to be the equivalent of a speed, and was later used by James Clerk Maxwell to strengthen his theory of electromagnetism.

\parpic[l][t]{
  \begin{minipage}{40mm}
    \fbox{\includegraphics[width=110px,height=140px]{img/medaillons/weierstrass.eps}}
  \end{minipage}
}
\textbf{Weierstrass, Karl Theodor Wilhelm }(1815-1897) was a mathematician born in Ostenfelde (Prussia) and died in Berlin, who gave to the theory of functions its modern form by specifying in particular the formalism of limits and is thus considered to be the father of modern analysis. His interest in mathematics began while he was a Gymnasium student at Theodorianum in Paderborn. He was sent to the University of Bonn upon graduation to prepare for a government position. Because his studies were to be in the fields of law, economics, and finance, he was immediately in conflict with his hopes to study mathematics. He resolved the conflict by paying little heed to his planned course of study, but continued private study in mathematics. The outcome was to leave the university without a degree. For many years, Weierstrass worked behind the scenes to establish his theory of functions of complex variable, based on entire series developments. After that he studied mathematics at the University of Münster (which was even at this time very famous for mathematics) and his father was able to obtain a place for him in a teacher training school in Münster. Later he was certified as a teacher in that city. In 1854 he published a memoir on \textit{Abelian integrals and hyperelliptic integrals inversion}, which established his reputation as a mathematician and earned him an honorary doctorate from Königsberg Universität. Appointed professor at the Berlin Universität, he taught from 1864 to his death. He published only a little during his lifetime and his reputation came mainly from the influence of his lectures in Berlin. These were followed by many mathematicians who established the theory of functions on the basis of rigour with which his name is attached, the "Weierstrass rigor". He is also known to have published an example of a continuous function differentiable nowhere (Weierstrass function).

\parpic[l][t]{
  \begin{minipage}{40mm}
    \fbox{\includegraphics[width=110px,height=140px]{img/medaillons/weyl.eps}}
  \end{minipage}
}
\textbf{Weyl, Hermann} (1885-1955) is one of the most influential mathematicians of the 20th century, one of the first to combine General Relativity with the laws of electromagnetism. His research mainly concentrated on mathematical topology and geometry. He conducted research in quantum mechanics and number theory. Born in Elmshorn near Hamburg (Germany), Weyl studied from 1904 to 1908 in Göttingen and Munich, mainly interested in mathematics and physics. His doctorate in Göttingen was supported under the direction of Hilbert and Minkowski. In 1910, he obtained a teaching position at Göttingen as a private lecturer. He taught mathematics at the ETH of Zürich in Switzerland in 1913. It is at Princeton that he worked with Einstein. Weyl searched the unification of gravitation and electromagnetism. This research gave an explanation of the violation of the non-conservation of parity, a characteristic of weak interactions. In 1918, he introduced the concept of gauge, the first step in what will become the gauge theory. He laid the foundation, giving rise to spinors, that become familiar around 1930. Weyl continued to work at the Institute for Advanced Studies until his retirement in 1952. In reality, his vision was an unsuccessful attempt to model the electromagnetic and gravitational fields as space-time geometric properties. Those works are fundamental to understand the symmetry of the laws of quantum mechanics. He died in Zürich.

\parpic[l][t]{
  \begin{minipage}{40mm}
    \fbox{\includegraphics[width=110px,height=140px]{img/medaillons/weinberg.eps}}
  \end{minipage}
}
\textbf{Weinberg, Steven} (1933-) is born in New York, he began his studies at New York and then at Cornell University (also in New York) and supported in 1957 at Princeton, his thesis on the effects of strong interaction processes dominated by the weak interaction. Researcher at the University of California at Berkeley from 1959 to 1966, he was interested at many problems in quantum field theory, particle physics and astrophysics. Professor at Harvard in 1973, he contributed decisively to the modern understanding of the fundamental interactions. He joined the University of Texas at Austin in 1982. The unification of the fundamental forces used the efforts of modern physicists since Newton, Maxwell and Einstein who, after having united space and time, tried in vain to unify in a single theory gravitation and electromagnetism. The discovery in the early 20th century, of the two nuclear forces, weak and strong interactions, gave a new impulsion to these efforts. In 1967, Weinberg and the Pakistani physicist Abdus Salam proposed independently that electromagnetism and the weak nuclear interaction are derived from a single electroweak interaction, whose gauge symmetry is spontaneously broken and whose vector is a triplet of bosons massive photon. A few years later, experiments at CERN in Geneva brought the first confirmations of the Weinberg-Salam model. The 1979 Nobel Prize in Physics (shared with American Sheldon Lee Glashow to the importance of his pioneering work) rewarded the two authors of what is now named the "standard model" of electroweak interactions. Excellent teacher, Weinberg is the author of several physics course level, both on the gravitational field theory. Popularizer of talent, his book The \textit{First Three Minutes of the Universe} was a worldwide success.

\parpic[l][t]{
  \begin{minipage}{40mm}
    \fbox{\includegraphics[width=110px,height=140px]{img/medaillons/wilcoxon.eps}}
  \end{minipage}
}
\textbf{Wilcoxon, Frank} (1892-1965) was a chemist and statistician known for the development of famous statistical tests. Frank Wilcoxon was born from american parents in County Cork (Ireland). He grew up in Catskill, New York but received part of his education in England. In 1917, he graduated from Pennsylvania Military College with a B.Sc. After the First World War he entered graduate studies, first at Rutgers University, where he was awarded an M.S. in chemistry in 1921, and then at Cornell University, gaining a Ph.D. in physical chemistry in 1924. Wilcoxon entered a research career, working at the Boyce Thompson Institute for Plant Research from 1925 to 1941. He then moved to the Atlas Powder Company, where he designed and directed the Control Laboratory, before joining the American Cyanamid Company in 1943. During this time he developed an interest in inferential statistics through the study of R.A. Fisher's 1925 text, \textit{Statistical Methods for Research Workers}. He retired in 1957. Over his career Wilcoxon published over 70 papers. His most well-known paper contained the two new statistical tests that still bear his name, the Wilcoxon rank-sum test and the Wilcoxon signed-rank test. These are non-parametric alternatives to the unpaired and paired Student's t-tests respectively. Wilcoxon died after a brief illness in Tallahassee (Florida - USA).

\parpic[l][t]{
  \begin{minipage}{40mm}
    \fbox{\includegraphics[width=110px,height=140px]{img/medaillons/witten.eps}}
  \end{minipage}
}
\textbf{Witten, Edward} (1951-) is a mathematician and physicist, winner of the Fields Medal in 1990 and born in Baltimore (Maryland). Witten completed his graduate studies at Brandeis University in Waltham (Massachusetts), then Princeton University (New Jersey), where he defended his doctoral thesis in physics in 1974. Researcher at Harvard University from 1976 to 1980, he taught at Princeton University, then became a member of the Institute for Advanced Study (IAS) at Princeton in 1987. After work in theoretical physics of elementary particles, Witten focuses his research on mathematical physics and in particular contributes significantly to the development of superstring theories in the hope that they might emerge to an understanding of the gravitational interaction at the quantum level. In mathematics, he has contributed to the study of Morse theory, proving classical Morse inequalities connecting the critical points to homology. In 1987, he proved an infinite sequence of rigidity theorems on the space of solutions of differential equations, such as the Rarita-Schwinger equation, encountered in physics. In knot theory, he showed in 1989 that we can interpret the Vaughan Jones' invariants of knots as Feynman integrals for 3-dimensional gauge theory. He has, furthermore, explored the relationship between quantum field theory and differential topology of 2 or 3-dimensional varieties. Recent advances in the understanding of 2-dimensional models of gravity are largely due to the influence of the innovative ideas of Witten.

\phantomsection
\addcontentsline{toc}{section}{Y}

\parpic[l][t]{
  \begin{minipage}{40mm}
    \fbox{\includegraphics[width=110px,height=140px]{img/medaillons/yang.eps}}
  \end{minipage}
}
\textbf{Yang, Chen-Ning} (1922-) is one of the greatest physicists theorists of the second half of the 20th century. He is professor at the Chinese University of Hong Kong and at the Tsinghua University in Beijing, Professor Emeritus of the University of New York at Stony Brook, Yang. Yang obtained his Master of Science from Tsinghua University in 1944. He enrolled in 1946 at the University of Chicago, that Fermi had just joined. Later, he decided to devote himself to theoretical physics, and in 1949 he defended his thesis work on the phenomenology of nuclear reactions. His career began at the Institute for Advanced Studies (IAS) in Princeton in 1949. In 1965, he refused to succeed to Oppenheimer as director, but he decided in 1966 he finally accepted the Einstein Chair and the position of Director of the Institute of Theoretical Physics of the new University of New York at Stony Brook. From 1971 he actively engages in restoring scientific relations between China and the United States and is involved in the creation of new research institutes, especially in Nanjing. Yang's contributions are characterized by their depth, their amplitude and their variety, from the phenomenology of particle quantum field theory, through the statistical mechanics as well as various forays into physics of condensed matter. His great merit are related to two points: firstly, he showed that the hypothesis of space symmetry had not been tested for weak interactions, and secondly, he devised a whole series of new tests for the space reflection invariance. These advanced in the theory of weak interactions have lead, with the introduction of the Yang-Mills fields, to the electroweak standard model. The idea of Yang was to generalize gauge invariance to groups of rotations in 3-dimensional abstract space intended to describe the internal degrees of freedom of matter fields. The Yang-Mills fields imposed themselves as a fundamental tool for the construction of a predictive theory of all weak, strong and electromagnetic interactions, decisive event that engaged the revolution in physics in the 1970s. All of his work have had a considerable impact in theoretical physics. Nearly 20 years after the publication of his article with Mills, Yang gave a precise reformulation of the theory of Yang-Mills fields under strict fiber spaces. The analogy with the theory of gravitation becomes also apparent and the notions of curvature and parallel transport are introduced naturally. Particular solutions of the Yang-Mills equations, such as this discovered by Gerard't Hooft, are used by mathematicians to explore the properties of differential manifolds in 4 dimensions. Yang has received numerous scientific awards including the Nobel Prize of Physics in 1957 that he shared with Tsung-Dao Lee. This prestigious award was granted for their work on parity laws in the field of elementary particles. These fundamental studies are particularly important because they showed that the left-right symmetry of elementary particles, universally accepted at the time, was simply incorrect, which was later proven experimentally.

\parpic[l][t]{
  \begin{minipage}{40mm}
    \fbox{\includegraphics[width=110px,height=140px]{img/medaillons/yukawa.eps}}
  \end{minipage}
}
\textbf{Yukawa Hideki} (1907-1981) was a physicist Japanese, born and died in Tokyo, he was the 5th of 7 children who became, for the most of them distinguished scholars. He was quickly interested to mathematics and philosophy. He was admitted at the Department of Physics at Kyoto University in 1926. Great reader, Yukawa became fascinated by the new philosophy accompanying relativity and quantum theory, concepts he had discovered especially in the works of Max Planck. In parallel to his studies, he became aware of the contemporary developments in quantum physics that led to its formulation established in the late 1920s. He graduated from the University of Kyoto in 1929 and began therefore personal research in the double direction of relativistic quantum physics and nuclear physics which only started to emerge. He first focused on the problem of the electron-proton nuclear binding then on the quantum field theory. While teaching quantum physics, Yukawa continued his research on the problems of the physics of nuclei. In 1934, he attacked the problem of the nuclear force that the theory of Fermi was unable to solve. He took an idea he had considerate in his early work, that of a force exchange, passed between the neutron and the proton by a new particle associated with a new field, which he proposed to deduce the properties from the nuclear interaction. It is in  1934 that he discovered the solution, obtaining a relation between the mass of the hypothetical exchange particle and scope of action of nuclear forces. The Yukawa particle, the meson, must had a mass 200 times that of the electron. It was assumed that these mesons had integer spin or none, that they were obeying the Bose-Einstein statistics and that they were provided with positive and negative charges. This work did not attract attention until the day when other researchers announced the discovery of a new particle in cosmic rays, with the mass predicted by Yukawa. It appeared, however, that the interaction of the meson with the material was too weak to be the particle of nuclear forces exchange. The theory of the two mesons solved the difficulty. He had discovered in the meantime the mechanism of disintegration of the nucleus by orbital electron capture by applying the Fermi's theory. He was the first Japanese to receive the Nobel Prize for Physics in 1949 for his mesic theory of nuclear forces. Yukawa founded the Research Institute for Fundamental Physics at Kyoto University and directed it until his retirement in 1970. He was not limited to the activity of physicist, he wrote essays on scientific creativity and militated for campaigned for peace, signing the appeal of Albert Einstein and Bertrand Russell against the use of nuclear weapons.

\parpic[l][t]{
  \begin{minipage}{40mm}
    \fbox{\includegraphics[width=110px,height=140px]{img/medaillons/young.eps}}
  \end{minipage}
}
\textbf{Young, Thomas} (1773-1829) was physicist, physician and British egyptologist born in Milverton and died in London, best known for his discoveries in optics (interference phenomena), elasticity of materials and medicine (explanation of color vision). At the age of fourteen he knew already the basics of more than a dozen languages. Young he began studying medicine in 1792 in London, then went to Edinburgh in 1794 and a year later to Göttingen (Germany), where he received his doctorate in physics in 1796. In 1799, he began practicing medicine in London. From 1802 until his death, he served as secretary of the Royal Society. In 1811, Young was appointed to St. George's Hospital in London. He was part of several official scientific committees and, from 1818, he was appointed secretary of the Greenwich Office and editor of the \textit{Nautical Almanac}. In optics, Young discovered the phenomenon of interference, and thus contributed to establish the wave nature of light. He was the first to describe and measure astigmatism and find a physiological explanation for the sensation of color. Young is also known for his work on the theory of capillarity and elasticity. He also contributed to the deciphering of hieroglyphics inscribed on the Rosetta Stone. His writings include extensive work in medicine, physics and Egyptology.

\phantomsection
\addcontentsline{toc}{section}{Z}

\parpic[l][t]{
  \begin{minipage}{40mm}
    \fbox{\includegraphics[width=110px,height=140px]{img/medaillons/zeeman.eps}}
  \end{minipage}
}
\textbf{Zeeman, Pieter} (1865-1943) was a physicist born at Zonnemaire (Netherlands) and died at Amsterdam. He became interested in physics at an early age. In 1883 the aurora borealis happened to be visible in the Netherlands. Zeeman, then a student at the high school in Zierikzee, made a drawing and description of the phenomenon and submitted it to \textit{Nature}, where it was published. After Zeeman passed the qualification exams in 1885, he studied physics at the University of Leiden under Hendrik Lorentz. In 1890, even before finishing his thesis, he became Lorentz's assistant. This allowed him to participate in a research program on the Kerr effect (the reflection of polarized light on a magnetized surface). In 1893 he submitted his doctoral thesis on this effect. After obtaining his doctorate he went for half a year to F. Kohlrausch's Institute in Strasbourg. In 1895, after returning from Strasbourg, Zeeman became Privatdozent in mathematics and physics in Leiden. In 1896, three years after submitting his thesis on the Kerr effect, he disobeyed the direct orders of his supervisor and used laboratory equipment to measure the splitting of spectral lines by a strong magnetic field. He was fired for his efforts, but he was later vindicated: he won the 1902 Nobel Prize in Physics for the discovery of what has now become known as the Zeeman effect. As an extension of his thesis research, he began investigating the effect of magnetic fields on a light source. Because of his discovery, Zeeman was offered a position as lecturer in Amsterdam in 1897. In 1900 this was followed by his promotion to professor of physics at the University of Amsterdam. In 1902, together with his former mentor Lorentz, he received the Nobel Prize for Physics for the discovery of the Zeeman effect. Five years later, in 1908, he succeeded Van der Waals as full professor and Director of the Physics Institute in Amsterdam. He retired as a professor in 1935.

	\chapter{Chronology}
	When arriving at the three thousandth A4 page of writing this book and at the occasion of the 3rd edition, it seemed appropriate to us to try to give a rough timeline of the majority of subjects mentioned in this book\footnote{Even if for some dates it is not possible to check if they are legends or real facts...}. This gives a better perspective on the tools used and also to pay tribute to our illustrious predecessors whom we owe our quality of life, our longevity, our mastery of the environment (not necessarily its respect...) and of its understanding.

If there are important dates missing (but only on subjects near of those presented in the various sections of this book!) or that you identify errors, do not hesitate to let us know, it is a first draft, and therefore the chronology can only be improved.

For more information the reader may refer to \href{http://www.wikipedia.com}{{\color{blue} Wikipedia}}, which has reached the top level in the number of available historical dates and in quality (with verification of sources!).

\begin{center}
\textit{This is the story of how history made science and how science entered in history, and how the ideas which emerged made the modern World.}
\end{center}

\includegraphics[width=\textwidth]{img/raphael_school_of_athens.jpg}

\textbf{+2016}\\
The first observation of gravitational waves was made on 14 September 2015 and was announced by the LIGO and Virgo collaborations on 11 February 2016.

\textbf{+2013}
On 14 March 2013 CERN confirmed that CMS and ATLAS have compared a number of options for the spin-parity of this particle, and these all prefer no spin and even parity. This, coupled with the measured interactions of the new particle with other particles, strongly indicates that it is a Higgs boson. This also makes this particle the first elementary scalar particle to be discovered in nature. In July 2017, CERN confirmed that all measurements still agree with the predictions of the Standard Model.

\textbf{+2006}
The  cognitive psychologist and computer scientist Geoffrey Everest Hinton publish \textit{A fast learning algorithm for deep belief nets}, which rejuvinates interest in Deep Learning. 

\textbf{+2001}\\
First release of Opera Magistris but under the name "Sciences.ch". A compendium on Applied Mathematics that has for purpose to merge all the modern knowledge on STEM (science, technology, engineering, and mathematics) and beyond at the Bachelor/Master level with a maximum of details in mathematical developments.

\textbf{+1995}\\
Michel Mayor and Didier Queloz definitively observe the first extrasolar planet around a main sequence star. The same year the first gaseous Bose-Einstein condensate is produced by the physicists Eric Cornell and physicists Carl Wieman at the University of Colorado at Boulder NIST–JILA lab, in a gas of rubidium atoms cooled to $170$ nanokelvins.

\textbf{+1994}\\
The works of the mathematician Andrew Wiles (more than 10 years of research!) give a solution to the Fermat's last theorem. First algorithm using Quantum computer for Prime factorization by Peter Shor.

\textbf{+1992}\\
Support vector machines (SVMs) are invented.

\textbf{+1986}\\
David E. Rumelhart, Geoffrey E. Hinton and Ronald J. Williams invent backpropagation, a new learning procedure for networks of neurone-like units. The engineers Bill Smith and Mikel J. Harry while working at Motorola introduces Six Sigma ($6\sigma$), a set of techniques and tools for scientific process improvement.

\textbf{+1983}\\
The physicists Carlo Rubbia, Simon van der Meer, and the CERN UA-1 team discovered the W and Z bosons which confirm the unification of the weak nuclear and the electromagnetic forces.

\textbf{+1982}\\
The astrophysicist Werner Becker discover the first millisecond pulsar. The same year, Alain Aspect team observe the violation of Bell inequalities.

\textbf{+1978}\\
The cryptologists mathematicians Ronald Rivest, Adi Shamir and Leonard Adelman propose a public key encryption procedure named "RSA" based on the difficulty of factorization into primes numbers.

\textbf{+1976}\\
Development of a method by a Peter Mansfield and Andrew Maudsley to makes possible nuclear magnetic resonance scanners (NMR).

\textbf{+1975}\\
The computer scientist John Holl invents genetic algorithms.

\textbf{+1973}\\
The mathematician Fischer Black and economist Myron S. Scholes publish a financial asset pricing model.

\textbf{+1969}\\
In his \textit{Control charts for measurements with varying sample sizes} the quality engineer Irving Wingate Burr introduce the famous unbiased constants $c_4$ and $d_2$ that bear his name.

\textbf{+1968}\\
Hidden Markov Model (HMM) invented

\textbf{+1967}\\
Jocelyn Bell Burnell and Antony Hewish discover the first pulsar.

\textbf{+1965}\\
Detection by the astrophysicists Arno Penzias and Robert Wilson of the cosmic microwave due to sky background radiation predicted by the theory of the physicist Robert Dicke. Development of the Fast Fourier Transform algorithm by James W. Cooley and John W. Tukey that has huge application in sciences. The physicist John Stewart Bell discover the Bell inequalities.

\textbf{+1964}\\
The astrophysicist Irwin Shapiro predicts a gravitational time delay of radiation travel as a test of General Relativity. The same year, the physicist John Stewart Bell shows that all local hidden variable theories must satisfy Bell's inequality.

\textbf{+1963}\\
The mathematician and meteorologist Edward Lorenz found what is probably the first strange attractor and opens the way to chaos theory.

\textbf{+1962}\\
The mathematician Benoît Mandelbrot discovered fractals by chance in the analysis of signals located at Bell Laboratories in the United States where he will use computers for repeat graphic patterns endlessly and whose principle is the basis of the theory of fractals. The same year, the economist William Forsyth Sharpe publishes the CAPM (Capital Asset Pricing Model). 

\textbf{+1960}\\
The physicist Abdus Salam postulates the existence of W and Z bosons to explain beta decay and the emergence of a new Z boson, which had never been seen before. The same year, the pysicists Ali Javan and Theodore Maiman invented each particular type of LASER. The mathematicians and engineers Irving Reed and Gustave Solomon present the Reed-Solomon error-correcting code.

\textbf{+1959}\\
The physicists Yakir Aharonov and David Bohm predict the Aharonov-Bohm effect (particle turning around a magnetic field but in a region where the magnetic field is zero is sensible to the field through the vector potential) and the following year the physicist Robert G. Chambers confirmed the effect experimentally. 

\textbf{+1958}\\
The perceptron is developed at Cornell University by Frank Rosenblatt. It is the first program able to learn by trial and errors.

\textbf{+1957}\\
The physicists John Bardeen, Leon Neil Cooper and John Robert Schrieffer propose and theory of supraconductivity.

\textbf{+1956}\\
The physicists Clyde Cowan and Fred Reines observe the neutrino hypothesized 25 years ago by the physicist Wolfgang Pauli. The same year, the  linguist, philosopher, cognitive scientist, historian, social critic, and political activist Avram Noam Chomsky publish Three Models for the Description of Language where he introduces the classification of formal grammars named today the "Chomsky hierarchy", which contains the class of out-of-context grammars, playing an important role in computer science to create programming languages. In the Logic Theorist by the cognitive computer scientist Allen Newell and the political scientist, economist, sociologist, psychologist, and computer scientist Herbert Simon, the first artificial intelligence computer program is proposed, which produced evidence in Russell and Whitehead's Principia Mathematica system. For one of the theorems, the program produces a simpler proof than that presented in the Principia!

\textbf{+1955}\\
The physicists Owen Chamberlain, Emilio Gino Segrè, Clyde Wiegand and Thomas Ypsilantis discover the antiproton.

\textbf{+1954}\\
The economist Harry Markowitz published his thesis on the efficient diversification model of financial assets portfolios. The same year, the physicists John Bell and the duo Wolfgang Pauli and Gerhart Lüders develop the CPT theory analyzing the symmetry of physical laws for transformations involving simultaneously the charge, parity and time. The physicist Charles Hard Townes developed the MASER. The biologist Milner Baily Schaefer publishes his equilibrium populations model.

\textbf{+1953}\\
Monte Carlo Markov Chain (MCMC) invented. Bayesian inference finally becomes tractable on real problems.


\textbf{+1952}\\
The mathematician George Bernard Dantzig developed the simplex algorithm for operational research.

\textbf{+1951}\\
The CPT theorem appears for the first time implicitly in the work of the physicist Julian Schwinger to prove the correlation between spin and statistics.

\textbf{+1950}\\
The physicists Johannes Hans Daniel Jensen and Maria Goeppert-Mayer developed the shell model of the nuclear core. The same year, the economist and mathematician John Forbes Nash developed the concept of non-cooperative games and generalizes the notion of minimax for zero-sum games; the engineer David Huffman found the algorithm used to compress any type of series symbols. In his \textit{Error-detecting and error-correcting codes} the mathematician Richard Wesley Hamming presents the Hamming family of code, linear codes used to o detect and correct transmission errors.

\textbf{+1949}\\
The mathematician and electrical engineer Claude Shannon published an article containing the information theory which will became the foundation of a number of physical theories, statistics and numerical methods. The physicist Richeard Feynman proposed the interpretation of the positron as an electron moving backward in time in his paper \textit{The Theory of Positrons}.

\textbf{+1948}\\
The physicist Maria Göpper-Meyer develops with success a theoretical model for the structure of the atomic nucleus and textile engineer and statistician Genichi Taguchi developed the experimental designs (DOE) that bear his name. The physicist Richard Feynman introduce the diagrams that bear his name and also the path integral formulation. The physicist Polykarp Kusch measure the anomalous magnetic moment of the electron (deviation of theoretical prediction of Dirac theory) thus leading to reconsideration of and innovations in quantum electrodynamics.

\textbf{+1947}\\
The physicists Cecil Powell, Cesare Mansueto Giulio Latte, and Giuseppe Occhialini discover the pion in the study of cosmic rays. The same year the physicists John Bardeen, Walter H. Brattain and William Schockley invent semiconductor transistors in the laboratories of the Bell telephone company in the United States that will cause the computer revolution. Measurement of the Lamb shift by Willis Eugene Lamb (shift of energy spectrum non-predicted by Dirac theory but explain in the framework of Quantum Electrodynamics).

\textbf{+1946}\\
The physicist and chemist Willard Frank Libby develops and discovers the possibility of Carbon 14 dating. The same year, the physicists Walter Houser Brattain, John Bardeen and William Bradford Shockley discover the transistor effect.

\textbf{+1945}\\
The Trinity Test, the first successful detonation of a nuclear weapon by the physicist Robert Oppenheimer and his team in New Mexico.

\textbf{+1944}\\
The mathematician John von Neumann developed the foundations of the mathematical theory of games.

\textbf{+1943}\\
The physicist Tomonaga Sin-Itiro published an article posing the physical basis of quantum electrodynamics.

\textbf{+1942}\\
The physicist Enrico Fermi and his team conducted the first controlled chain reaction in the purpose to build the first atomic bomb.

\textbf{+1941}\\
The physicist Ernst Stueckelberg interprets positrons as electrons with positive energy traveling back in time. The physicist Lev Davidovich Landau publish a theory of superfluidity.

\textbf{+1940}\\
The physicist Edward Teller sees the possibility of using the enormous amount of heat generated by the explosion of a fission bomb to trigger the nuclear fusion process. This is the approach considered as the discovery of nuclear fusion. The same year, the physicist William Donald Kerst developed the first betatron. The physicist John Wheeler in a telephone call to Richard Feynman hypothesizes that all electrons and positrons are actually manifestations of a single entity moving backwards and forwards in time (the "one-electron universe postulate").

\textbf{+1939}\\
The chemists Otto Hahn and Fritz Strassmann bombard uranium with neutrons and discovered that barium is produced by the experience (discovery of nuclear fission). The same year, the physicists Lise Meitner and Otto Robert Frisch determine that nuclear fission occurred during the Hahn-Strassman experiment. The physicists Wolfgang Pauli, Markus Fierz and Frederik Jozef Belinfante prove that the properties of permutation of identical particles, bosons or fermions are controlled by their spin.

\textbf{+1938}\\
The chemist and physicist Isidor Isaac Rabi and his colleagues studied the effects of placing beams of molecules in strong external magnetic fields, leading to the development of nuclear magnetic resonance (NMR). The same year, the physicists Hans Bethe and Carl von Weizsäcker propose a nuclear theory of stars and the mathematician and electrical engineer Claude Shannon publish what is probably the most famous master's thesis of the 20th century (\textit{A symbolic analysis of relay and switching circuits}), and prove that it is possible to simplify the design of logic circuits by using the Boolean algebra. This master's thesis has played an important role in the design of electronic computers. The physicist Piotr Leonidovich Kapitza observed the phenomenon of superfluidity.

\textbf{+1937}\\
The physicists Seth Neddermeyer, Carl Anderson, Jabez Curry Street and E.C. Stevenson discover muons in the traces left by cosmic rays in a bubble chamber. The same year the mathematician John von Neumann developed the Monte Carlo methods for various numerical methods and the physicist Niels Bohr developed the liquid drop model of the nucleus.

\textbf{+1936}\\
The physicists George Gamow and Edward Teller work together to formulate the theory of beta radioactive emissions. The same year, in his \textit{On computable numbers}, the computer scientist, mathematician, logician, cryptanalyst, philosopher and theoretical biologist Alan Mathison Turing analyzes the concept of computability using the Turing machine concept, one of the foundations of theoretical computing. The engineer, statistician, professor, author, lecturer, and management consultant Walter A. Shewhart publish his works on the CUSUM, UWMA and EWMA control charts.

\textbf{+1935}\\
The physicist Hideki Yukawa present the strong interaction theory and predicted the existence of mesons. The same year, the astrophysicist and mathematician Subrahmanyan Chandrasekhar reports the results of his researches on the collapse of stars into white dwarfs and beyond 1.44 solar masses into neutron stars. The article by Albert Einstein, Boris Podolsky and Nathan Rosen on the EPR paradox is published in Physical Review and questioned the nonlocality of the Copenhagen interpretation.

\textbf{+1934}\\
The physicist Pavel Cherenkov Alekseyevich study the emission of light when relativistic particles pass through an amorphous medium. The same year the physicist Enrico Fermi suggested to bombard uranium atoms with neutrons to obtain an element with 93 protons, formulated the theory of beta decay and the physicist Leó Szilárd realizes that a nuclear chain reaction is possible. The physicists Irène Joliot-Curie and Frédéric Joliot bombard aluminum atoms with alpha particles and create artificially radioactive phosphorus-30.

\textbf{+1933}\\
The mathematician Andrei Nikolaevich Kolmogorov published a book containing a solid base of axioms of probability. The physicist Ernst August Friedrich Ruska realize the first electronic microscope by transmission using electrons instead than photons.

\textbf{+1932}\\
The physicist Carl David Anderson discovered the positron. The same year the physicist Werner Heisenberg present the theoretical model of the proton-neutron nuclear core and uses it to explain isotopes. The physicist James Chadwick discovered the neutron and the physicists John Cockcroft and Ernest Walton break the nuclear core of lithium and boron by proton bombardment.

\textbf{+1931}\\
The physicist Wolfgang Pauli puts forward the hypothesis of neutrino to explain the apparent violation of the principle of conservation of energy in beta decay. The same year the mathematician and logician Kurt Gödel showed that a system can be both consistent and complete (incompleteness theorem) and that if the system is coherent, then the coherence of the axioms can not be proved within the system. The physicist Ernest Lawrence invents the first cyclotron and the physicist, engineer and statistician Walter Andrew Shewhart publish his book \textit{Economic Control of Quality of Manufactured Product} where he introduces the main control charts.

\textbf{+1930}\\
The physicist Fritz London explains that Van der Waals forces are due to the interaction of the dipole moments of molecules. The same year the physicist Paul Dirac present his electron-hole theory and the economist John Maynard Keynes publish his \textit{A Treatise on Money}.

\textbf{+1929}\\
The astronomer Edwin Hubble by studying the redshift hypothesizes that the Universe is not static. The same year, the physicist Robert Van de Graaff invents the first particle accelerator, known today as the "Van de Graaff accelerator".

\textbf{+1928}\\
The physicist Paul Dirac established his relativistic wave equation for the electron, which generalizes and improves the spinless relativistic equation of Klein-Gordon. The same year, the physicists Friedrich Hund and Robert S. Mulliken introduces the concept of molecular orbital and the physicist and cosmologist George Gamow develops the theoretical model of quantum alpha decay by tunneling. The physicist Félix Bloch studies the problem of a particle submitted to a periodic potential and develops the band theory that will become a main foundation of solid state theory, particularly to understand their transport or optical properties.

\textbf{+1927}\\
The physicist Werner Heisenberg establish the uncertainty principle, by which the position and momentum of a particle can not be simultaneously known with accuracy, indirectly by developing a new theoretical basis for quantum mechanics. The same year, the physicists Walter Heitler and Fritz London present the quantum theory of the chemical bond established from the hydrogen molecule and the physicist Max Born interprets the wave function of Schrödinger as probabilities and with the help of the physicist Robert Oppenheimer presents the Born-Oppenheimer approximation. The physicists Clinton Joseph Davisson, Lester Germer and George Paget Thomson confirm the wavelike nature of the electrons by diffraction. The physicist Paul Ehrenfest prove the famous quantum physics theorem that bears his name. The physicist Paul Dirac introduces the quantification of the electromagnetic field.

\textbf{+1926}\\
The physicist Erwin Schrödinger establish his wave equation that defines quantum mechanics in an analytical form by developing the ideas of De Broglie on theory of wave mechanics and he proves that the wave and matrix formulations of quantum theory are mathematically equivalent. The same year the physicists Oskar Klein and Walter Gordon establish the equation of relativistic quantum mechanics for spinless particles and Paul Dirac defines the Fermi-Dirac statistics. In the field of population dynamics the physicist and mathematician Vito Volterra published the nonlinear differential equation modeling the predator/prey equilibrium.

\textbf{+1925}\\
The physicist Pierre Auger discovered the Auger effect (2 years after Lise Meitner) and the same year the physicists George Uhlenbeck and Samuel Goudsmit postulate and reveal the existence of electron spin. Also the same year, the physicist Wolfgang Pauli established by necessity the principle of quantum exclusion. The physicists Werner Heisenberg, Max Born and Pascual Jordan formulate quantum matrix mechanics.

\textbf{+1924}\\
The physicist John Lennard-Jones proposed a semi-empirical description of inter-atomic interaction forces and the same year, the physicists Satyendranath Bose and Albert Einstein define the Bose-Einstein statistics. In the field of statistics, the statistician Ronald Fisher defines major modern concepts in statistics.

\textbf{+1923}\\
The astronomer Edwin Hubble estimates the distance between the Earth and the spiral galaxies, showing that they are far from the Milky Way and in the same year the physicist Louis de Broglie suggested the wave-particle duality from quantum theory and from the mass-energy equivalence and the physicist Lise Meitner discovers the Auger effect. The mathematician Norbert Wiener introduces Brownian movement theory.

\textbf{+1922}\\
The physicist Arthur Compton studied the scattering of X photons by electrons and the astrophysicist Alexander Friedmann develops non static universe models.

\textbf{+1921}\\
The physicist Alfred Landé defines the gyromagnetic ratio and introduces also half integer quantum numbers. The same year physicists Otto Stern and Walter Gerlach show experimentally that the intrinsic moment of the electron is quantized. The mathematician and physicist Theodor Franz Eduard Kaluza proved that a five-dimensional version of Einstein's equations unifies gravitation and electromagnetism.

\textbf{+1920}\\
The astronomer Vesto Melvin Slipher highlights the phenomenon of red shift in the spectrum of galaxies. The same year, the physicist Arnold Sommerfeld introduces a fourth quantum number to the original model of the Bohr's atom. the physicist Niels Bohr introduce the correspondance principle assuring the transition quantum physics $\mapsto$ classical physics when $\hbar\rightarrow 0$. The astronomer and astrophysicist Ernst Julius Öpik confirms that the "Andromeda nebula" is located outside the Milky Way-this confirms that our universe is much larger than we thought because it is not limited to the Milky Way.

\textbf{+1919}\\
The physicist Ernest Rutherford performed the first artificial disintegration of an atom by bombarding nitrogen with alpha particles. The same year, the physicist and mathematician Amali Emmy Noether develops his theorem on differential invariants in the calculus of variations, one of the most important mathematical theorems ever proved in guiding the development of modern physics.

\textbf{+1918}\\
The astronomer Harlow Shapley made the first accurate estimate of the size of our galaxy and the Sun's position in it. The same year, the physicist Hermann Weyl introduces the notion of gauge, the first step in what will become the gauge theory.

\textbf{+1917}\\
The physicist Albert Einstein introduces the idea of stimulated emission of radiation used in the manufacturing base of LASER. The same year, the physicist Arnold Sommerfeld introduces a third quantum number to the original model of the Bohr's atom.

\textbf{+1916}\\
The physicist Albert Einstein developed his theory of General Relativity and how matter plays on the space-time to produce gravitational effects. This is the first theory named "background independent". The same year, the physicists Gilbert Lewis and Irving Langmuir present the electronic shell model to explain chemical bonds and the physicist Arnold Sommerfeld introduces relativity in his model of 1915 and this relativistic correction explained the observed values by high resolution spectrographs and therefore the splitting of spectral lines named "fine structure" and he introduces at the same time a second quantum number describing elliptical orbits. The physicist Karl Schwarzschild found a mathematical solution of Einstein's equations, that he applies to neutron stars and black holes.

\textbf{+1915}\\
The physicist Arnold Sommerfeld refines the atomic model of the physicist Niels Bohr by introducing elliptical orbits to explain the fine structure lines of the hydrogen atom. This new model does not, however, explain the range of the observed spectra of the hydrogen atom. The same year the physicist Albert Einstein calculates the trajectory of Mercury with General Relativity. He finds that his theory is able to explain the perihelion advance of Mercury with high accuracy and the deviation of light rays in the gravitational field of the Sun. The end of that year he submitted the article that describes the field equations of gravitation, these equations will be the basis of the theory of General Relativity.

\textbf{+1914}\\
The physicist Ernest Rutherford showed that the positively charged atomic nuclei contain protons. The same year the physicist Albert Einstein and the mathematician Marcel Grossmann published an article on tensor calculus, and more particularly on the Riemann-Christoffel and Ricci tensor (more generally on tensor analysis and differential geometry) and the physicist Peter Debye develops a model of the behavior of the thermal capacity of the solids as a function of temperature. The mathematician Felix Hausdorff introduces the concepts of Hausdorff distance and Hausdorff dimension.

\textbf{+1913}\\
The physicist Niels Bohr presents the quantum model in circular layers of the atom and the same year the physicist Robert Millikan measures the fundamental electric charge. The same year, the physicists William Henry Bragg and William Lawrence Bragg find the Bragg condition for strengths X-ray reflection and the physicist Henry Moseley showed that the atomic number is the true criterion of elements discrimination. In mathematics, the mathematician Elie Cartan announces his discovery of spinors. The physicist Johannes Stark demonstrates that strong electric fields will split the Balmer spectral line series of hydrogen.

\textbf{+1912}\\
The physicist Max von Laue proposes the use of crystal lattices to diffract X-rays and in the same year the physicists Walter Friedrich and Paul Knipping diffract X-rays using zinc sulfide. The same year, the physicist Ernest Rutherford proposes the use of radioactivity as a means of dating. The chemists Otto Sackur and physicist Hugo Tetrode prove a formula to calculate the entropy of a mono-atomic gas (first apparition of Planck constant in thermodynamics).

\textbf{+1911}\\
The physicist Ernest Rutherford discovered the atomic nucleus by bombarding a thin gold foil with alpha particles. Some particles bounce on the core of gold atoms. The same year, the physicist and chemist Jean Perrin proves the existence of atoms and molecules and the physicist Heike Kammerlingh Onnes discovers superconductivity.

\textbf{+1911}\\
The physicist Robert Andrews Millikan determine the electric charge carried by a single electron with his famous oil-drop experiment in which he replaced water (which tended to evaporate too quickly) with oil.

\textbf{+1909}\\
The physicists Hans Geiger and Ernest Marsden discover that alpha particles can be strongly deflected by thin metal foils and in the same year the physicists Ernest Rutherford and Thomas Royds demonstrated that alpha particles are helium atoms ionized twice. In the field of Applied Mathematics, the mathematician Agner Krarup Erlang published the first paper on the theory of queues. 

\textbf{+1908}\\
The statistician William Sealy Gosset published an article proposing a new statistical distribution and a new statistical test named respectively the" Student law" and "Student's t-test". The same year, the mathematician Ernst Friedrich Ferdinand Zermelo proposed an improvement of the axioms of set theory. In his \textit{Mendelian proportions in a mixed population}, the mathematician Godfrey Harold Hardy exposes what is now known as the "Hardy-Weinberg principle" in genetics that establishes how dominant and recessive genetic traits spread in a large population. Hardy is known as Fundamental mathematician, specialist in number theory, but has contributed significantly to the study of population genetics by the results presented in this article, found independently by the obstetrician-gynecologist Wilhelm Weinberg.

\textbf{+1907}\\
The physicist Albert Einstein deduced the expression of the famous equivalence between mass and energy, and the same year he established the expression of the heat capacity of crystalline solids and calculates the gravitational redshift. The mathematician and physicist Hermann Minkowski unified space-time together in a unified mathematical structure. The mathematician Guido Fubini proves the multiple integral theorem that bears his name.

\textbf{+1906}\\
The physicist Walther Nernst presents a formulation of the third law of thermodynamics. The same year, the mathematician Andrei Markov published the first work on the chains of events that will later bare his name and that have occupied an important place in the quantum physics of his time. 

\textbf{+1905}\\
The physicist Albert Einstein explained the photoelectric effect by the existence of quantums and the same year he explained mathematically the Brownian motion as a result of the random motion of molecules and published his research on his theory of Special Relativity that proves the equivalence of mass and energy. The physicist Paul Langevin publish his theory on the susceptibility of paramagnetic materials.

\textbf{+1904}\\
The physicist Antoon Lorentz discovers the contraction of time in the direction of movement of the body relatively to the constant speed of light and proposed the transformation equations of the electromagnetic forces. The same year, the physicist Hantaro Nagaoka proposed a theoretical Saturnian model of the atom where electrons revolve around a massive positive nucleus like the rings of Saturn.

\textbf{+1903}\\
Radioactivity is explained in terms of fission of atoms by the physicist Ernest Rutherford and by the radiochemist Frederick Soddy.

\textbf{+1902}\\
The physicist Philipp Lenard observed that the photoelectric effect does not depend on the power of the light beam but its frequency. The same year, the chemist Theodor Svedberg suggests that fluctuations of molecular bombardment creates Brownian motion and the logician Bertrand Russell propose his "ultimate" paradox undermining the naive set theory. The physicist James Jeans describes the gravitational collapse phenomenon that can occur, for example in a cloud of gaseous material based on a critical mass or radius. The mathematician Henri Léon Lebesgue put the basis of measure theory and introduce the Lebesgue integral.

\textbf{+1901}\\
The mathematicians Gregorio Ricci-Curbastro and his assistant Tullio Levi-Civita developed tensor calculus.

\textbf{+1900}\\
The physicist Max Planck suggested that light can be emitted in discrete frequency generalizing the law of black body radiation. The same year, the physicist Johannes Rydberg refines the mathematical expression for the wavelengths of the Balmer's lines of the hydrogen and the physicist Paul Villard discovered gamma rays by studying the decay of uranium. In the field of Applied Mathematics, the mathematician Louis Bachelier developed the Brownian model motion applied to game and speculation theory that will be the mainstay of quantitative financial tools of the 20th century. The same year, the mathematician Karl Pearson defines the statistical distribution of the Chi-2 and explores important properties of this distribution for statistical inference. The physicist Paul Drude adaptated the kinetic theory of gases to electrons in metals and gets a model that still bears his name.

\textbf{+1899}\\
The physicist Ernest Rutherford discovered that the radiation emitted by uranium compounds are positively charged alpha particles and beta particles negatively charged. The same year, the mathematician David Hilbert replaces the five usual axioms of Euclidean geometry axioms by 21 to eliminate the weaknesses of Euclidean geometry. 

\textbf{+1898}\\
The mathematician David Hilbert gives a first approach of the class field. The same year, the physicists Marie and Pierre Curie isolate and study the radium and polonium and the physicist Wilhelm Wien Carl Werner identifies a new particle with a positive charge approximately equal to the mass of hydrogen that he will name the "proton". The engineer Alfred-Marie Liénard calculates the electromagnetic field produced by a point charge, animated with any movement. The mathematical expressions that have been independently established, but 2 years later by the physicist Emil Wiechert.

\textbf{+1897}\\
The physicist Joseph John Thomson measure the charge/mass ratio of certain negative particles created by cathodic rays. He measures their charge, and he concludes that their mass is about 2'000 times smaller than hydrogen. These particles were later named "electrons", a term suggested by the physicist George Johnstone Stoney. Televisions and other CRTs are improved versions of the Thomson's device.

\textbf{+1896}\\
The physicist Henri Becquerel discovered radioactivity of uranium and the same year the physicist Pieter Zeeman studied the decomposition of the sodium D lines when it is heated in a strong magnetic field and he discovered that the spectral lines of a light source subject to a magnetic field has many components, each having a certain polarization. To explain this phenomenon, one must add additional quantum number named "magnetic quantum number". The same year, the physicist Wilhelm Wien Carl Werner establishes the law that bears his name for the energy emitted by the black body.

\textbf{+1895}\\
The physicist Wilhelm Röntgen discovered X-rays and the same year the physicist and inventor Guglielmo Marconi carried out in the Swiss Alps at Salvan the first wireless "long distance" of 1.5 kilometers.

\textbf{+1893}\\
The mathematician, physicist, engineer and philosopher Henri Poincaré published his studies on the three-body problem and introduces the qualitative study of differential equations and chaos theory. The same year the mathematician Georg Cantor developed the theory of transfinite sets and the proposal of the engineer Nikola Tesla use AC instead of DC current is adopted by the first U.S. state. In his \textit{Uniplanar Algebra}, the mathematician Irving Stringham uses the symbol $\ln$ for the natural logarithm instead of the traditional $\log_e$.

\textbf{+1892}\\
The autodidact physicist Oliver Heaviside reduced the 8 equations of electrodynamics of Maxwell to $4$ differential equations. 

\textbf{+1891}\\
In his \textit{Arithmetices Principia, ova methodo exposita} the mathematician Giuseppe Peano introduces the axioms to build the Natural numbers set $\mathbb{N}$ and the symbol of appartenance and a first version of the quantifiers symbolic. Their final form will be given by the mathematician David Hilbert. He provides more than $40,000$ definitions in a language that he wants as universal.

\textbf{+1890}\\
The systematic study of groups is growing with the mathematician Sophus Lie, Issai Schur and Élie Cartan. This last one introduces the notion of algebraic group and continuous groups. 

\textbf{+1889}\\
The mathematicians Giuseppe Peano postulates $5$ properties of integers as axioms in the idea to do with integers what Euclid did for geometry. He defines also the axiomatic of vectorial space in $\mathbb{R}$ and introduce the concept of linear application. He introduce also the notation $\cup$ and $\cap$ for the union and intersection of sets.

\textbf{+1888}\\
The anthropologist, explorer, geographer, inventor, meteorologist, proto-geneticist, psychometrician, and statistician ... Francis Galton defines the concept of statistical correlation coefficient. The same year, the mathematician Richard Dedekind proposes the definition of a finite set. 

\textbf{+1887}\\
The physicists Albert Michelson and Edward Morley measured the speed of light to test the hypothesis of ether that their experimental results reject and the same year the physicist Heinrich Hertz discovered the photoelectric effect and conducted experiments on electromagnetic waves (production and reception).

\textbf{+1886}\\
The mathematician and physicist Oliver Heaviside introduces the handling differential operators as algebraic entities which will bring up later to the Laplace transforms.

\textbf{+1885}\\
The chemist Johan Balmer found the mathematical expression which gives the wavelength of the different lines of the spectrum of hydrogen.

\textbf{+1884}\\
The physicist and chemist Willard Gibbs defines the notation still in use in the early 21st century for the scalar and vector product as well as vector differential operators in its books about to vector calculus. The same year, the physicist Ludwig Boltzmann derives the Stefan-Boltzmann black body radiant flux law from thermodynamic considerations and the physicist John Henry Poynting introduces the vector that still today bear his name. The mathematician Karl Hermann Amandus Schwarz proves that the sphere is the solid with the minimal surface for a given volume (this result explains many shapes visible in the Universe).

\textbf{+1882}\\
The mathematician Ferdinand von Lindemann proved the transcendence of $\pi$. The same year, the astronomer, mathematician, economist and statisticien Simon Newcomb observes a $43''$ per century excess precession of Mercury's orbit. Where Isaac Newton mechanics explains correctly the precession period of other planets, it failed to explains that of Mercury.

\textbf{+1880}\\
The mathematician and physicist Oliver Heaviside introduces the step function that always bear his name today (Heaviside step function).

\textbf{+1879}\\
The mathematician and physicist Joseph Stefan publishes Stefan's law which states that the power transmitted across the whole spectral range is proportional to the fourth power of the absolute temperature of a star and the surface of it. For the same surface temperature, a star is also brighter the more it is big. The same year, the physicist Edwin Herbert Hall discovered that an electric current through a material immersed in a magnetic field generates a voltage perpendicular to the initial direction of the electric current. The philosopher, logician, and mathematician Gottlob Frege publish his \textit{Begriffsschrift, eine der arithmetischen nachgebildete Formelsprache des reinen Denkens} where he introduces the axiomatic predicate logic, the quantifiers (for all $\forall$, it exists $\exists$), the theory of quantified variable, the rigorous concept of formula/function and of variable.

\textbf{+1878}\\
The mathematician and philosopher William Kingdon Clifford introduces the divergence operator.

\textbf{+1877}\\
The physicist and chemist Willard Gibbs defines for chemical reactions two useful quantities, namely the enthalpy that represents the heat of reaction at constant pressure and the free energy that determines whether a reaction can proceed as spontaneous at constant temperature and pressure. The physicist John William Strutt (3rd Baron Rayleigh) introduces the foundations of modern sound theory. 

\textbf{+1876}\\
The mathematician and philosopher William Kingdon Clifford suggests that the motion of matter may be due to changes in the geometry of space.

\textbf{+1874}\\
The physicist Lord Kelvin formally states the second law of thermodynamics. The same year, the mathematical economist Léon Walras published his \textit{Elements of Pure Economics}.

\textbf{+1873}\\
The mathematician Georg Cantor laid the foundations of the theory of sets and cardinals and shows that the algebraic numbers are in fact countable and defines rigorously the real numbers $\mathbb{R}$, and introduce the his famous diagonal method. The same year, the physicist James Clerk Maxwell showed that light is an electromagnetic phenomenon and reduces the equations of electrodynamics to $8$ instead of $20$ equations (at the same times he defines the curl operator) and the physicist Johannes van der Waals introduces the idea that there are weak attractive forces between molecules. In his \textit{Sur la fonction exponentielle} the mathematician Charles Hermite proves the transcendence of $e$ (two proofs, one of $11$ pages and the other of $20$ pages).

\textbf{+1872}\\
The mathematician Karl Weierstrass presented at the Royal Academy of Sciences in Berlin an example of a function continuous everywhere but differentiable nowhere.

\textbf{+1871}\\
The chemist Dmitri Mendeleev examines his periodic table and predicted the existence of gallium, scandium and germanium. The same year the physicist James Clerk Maxwell established thermodynamic Maxwell relations. 

\textbf{+1870}\\
The physicist Rudolph Clausius proves the (scalar) Virial theorem.

\textbf{+1869}\\
The chemist Dmitri Mendeleev proposed the periodic table of elements which still bears his name.

\textbf{+1867}\\
The historian, journalist, philosopher, economist, sociologist Karl Marx published \textit{Das Kapital}. 

\textbf{+1866}\\
The physicist James Clerk Maxwell elaborates, independently of the physicist Ludwig Boltzmann, the kinetic theory of gases of Maxwell-Boltzmann. The same year, the monk and botanist Gregor Johann Mendel formulated the laws of statistics hybridization (experiment on $29,000$ peas...) and the mathematician Leopold Kronecker used for the first time the symbol that bear always today his name.

\textbf{+1865}\\
The physicist James Clerk Maxwell publishes for the first time the equations of electrodynamics in the form of equations 20 with 20 unknowns using quaternions. 

\textbf{+1862}\\
The physicist Gustav Kirchhoff developed the concept of Black body that can absorb and emit radiation at all frequencies and that the energy emitted depends only on the frequency of the emitted radiation and the temperature of the black body itself.

\textbf{+1859}\\
The physicist James Clerk Maxwell discovered the law of distribution of molecular velocities. The same year the astronomer Urbain Le Verrier reported an anomaly in the motion of Mercury not predictable by Newton's law and the physicist Gustav Kirchhoff with the chemist Robert Wilhelm Bunsen developed prism spectroscopy. 

\textbf{+1858}\\
The lawyer and mathematician Arthur Cayley emerges the notion of vector space, the notion of matrix and exposes the utility by using the multiplication of matrices and determinants; he rewrites the system of linear equations in matrix form. His works are often seen as the emergence of linear algebra.

\textbf{+1855}\\
The astronomer and physicist Léon Foucault discovered that the force required for the rotation of a copper disc increases when must rotate with its rim between the poles of a magnet, the disk heating at the same time because of the "Foucault's currents" induced in the metal. 

\textbf{+1854}\\
The mathematician George Boole published his system of symbolic logic, now known as Boolean algebra. The same year, the mathematician Arthur Cayley shows that quaternions can be used to represent rotations in four-dimensional space and the mathematician Georg Friedrich Bernhard Riemann gave a new definition of the integral and lays the foundations of differential geometry.  The mathematician Charles Hermite defines the concept of orthogonal matrices and prove that their eigenvalues are real numbers.

\textbf{+1852}\\
The physicists James Joule and William Thomson Kelvin show that gas in expansion cools quickly.

\textbf{+1851}\\
The astronomer and physicist Léon Foucault made a spectacular proof of the rotation of the Earth by suspending a pendulum with a long cable attached to the dome of the Pantheon in Paris. The same year, the mathematician Georg Friedrich Bernhard Riemann published the first work on functions with a complex variable. In his \textit{Paradoxien des Unendlichen}, the  mathematician, logician, philosopher and theologian Bernardus Placidus Johann Nepomuk Bolzano discusses of the issues related to the manipulation of infinites in mathematics. 

\textbf{+1850}\\
The mathematicians Arthur Cayley and James Joseph Sylvester introduces the term matrix and the same year, the mathematician Richard Dedekind introduces the term field. The physicist Rudolf Clausius developed the mechanical theory of heat and formulated the second principle of thermodynamics. The physicist George Stokes proves the famous theorem that bears his name.

\textbf{+1849}\\
The mathematicien and astronome Edouard Roche finds the limiting radius of tidal destruction and tidal creation for a body held together only by its self gravity and uses it to explain why Saturn's rings do not condense into a satellite.

\textbf{+1848}\\
The physicist William Thomson Kelvin discovers the absolute 0 point temperature and sets its own unit of measurement. The same year the physicist and astronomer Hippolyte Fizeau transposes the results of Christian Doppler to the light that like sound has a wave nature (Doppler effect) and highlights the redshift and towards the blueshift.

\textbf{+1847}\\
The physicist Jame Joule founds experimentally the mechanical equivalent of heat and the same year, the physiologist and physicist Hermann Helmholtz formally states the law of conservation of energy.

\textbf{+1845}\\
The physicist Gustav Kirchoff defines the concept of electric potential and sets the laws of networks that bear his name (node law, mesh law). The same year, the physicist George Stokes publishes what will be the basis of the Navier-Stokes fluid mechanics and the physicist Michael Faraday discovers that light propagation in a material can be influenced by external magnetic fields.

\textbf{+1844}\\
The mathematician Joseph Liouville proves the existence of an infinite number of transcendental numbers.

\textbf{+1843}\\
The physicist, mathematician and astronomer William Rowan Hamilton defines sets of complex vector spaces (quaternions). The concept of vector space will be clearly defined by the mathematician and astronomer August Ferdinand Möbius and the mathematician and linguist Giuseppe Peano $40$ years later. The same year, hhe mathematician Laurent Pierre Alphonse publishes his memoir on what will later became the series that bear his name in complex analysis.

\textbf{+1842}\\
The principle of conservation of energy is expressed by the physicist Julius von Mayer who calculated the amount of work that can be obtained by converting heat energy, this means the mechanical equivalent of calories. The same year, the physicist Christian Doppler discovered the acoustic effect that bears his name (change in frequency with the relative movement).

\textbf{+1841}\\
The mathematician Karl Weierstrass discovers but does not publish the Laurent expansion theorem. The same year, the mathematician Carl Gustav Jacob Jacobi introduces the Jacobian matrices and reintroduced the partial derivative notation originally proposed by the mathematician André-Marie Legendre.

\textbf{+1838}\\
The astronomer and mathematician Friedrich Bessel measure that the distance that separates us from the star 61 Cygni is about 96 trillion kilometers.

\textbf{+1835}\\
The mathematician Carl Friedrich Gauss gives a rigorous construction of the complex numbers and the mathematician Augustin Louis Cauchy establishes a first theory of determinants. The same year, the mathematician and engineer Gaspard Coriolis proves that the acceleration of a mobile in a rotating frame is subjected to a complementary force perpendicular to the direction of movement of the mobile in this reference frame.

\textbf{+1834}\\
The engineer and physicist Émile Clapeyron presents a formulation of the second law of thermodynamics. The same year, the physicist Heinrich Lenz establishes the law of electromagnetic induction.

\textbf{+1832}\\
The physicist Michael Faraday established the basic theory of electrolysis.

\textbf{+1831}\\
The physicist Michael Faraday discovered electromagnetic induction, namely the obtention of an electric current from the change of a magnetic field (principle of the dynamo). The same year the mathematician and physicist Carl Friedrich Gauss provides two of the four Maxwell equations.

\textbf{+1829}\\
The mathematician Evariste Galois presents the first draft of his work on solvable equations which will cause the set-approach for solving algebraic equations by radicals. The mathematician Augustin Louis Cauchy proves that the eigenvalues of a symmetric matrices are all real. The mathematician Johann Peter Gustav Lejeune Dirichlet studies the convergence of Fourier series.

\textbf{+1828}\\
The physicist George Green proves Green's theorem.

\textbf{+1827}\\
The botanist Robert Brown discovered the Brownian motion of pollen particles and dye in water; the same year the physicist Georg Ohm establishes the law of electrical resistance and the physicist and mathematician André Ampère discovered the laws that bind the magnetic forces to electric current. The physicist, mathematician and astronomer William Rowan Hamilton presents the theory of a single function that unifies mechanics, optics and mathematics and helped to establish the wave theory of light.

\textbf{+1826}\\
The mathematician Niels Henrik Abel proves that it is impossible to solve general quintic equation ($5$th order polynomials) in radicals. In his \textit{On a Method of Expressing by Signs the Action of Machinery}, the mathematician, philosopher, inventor and mechanical engineer Charles Babbage describes of a symbolic language that will help him to design its analytical machine, the first universal mechanical machine. The "Babbage machine" is the first complete programmable computer (having the same computational capabilities as a Turing machine or a modern computer) that has been designed. 

\textbf{+1825}\\
The mathematician Augustin-Louis Cauchy presents the Cauchy integral theorem for general integration paths and introduces the theory of residues. The scientist and inventor William Sturgeon invented the first electromagnet.

\textbf{+1824}\\
The mathematician Augustin-Louis Cauchy discovers the characteristic polynomial of a matrix and proves that it is invariant by linear transformation and calculates for the first time eigenvalues and eigenvectors. The same year, the physicist and engineer Sadi Carnot scientifically analyzes the efficiency of steam engines (Carnot cycle), showing that their performance is limited and also defines the second principle of thermodynamics.

\textbf{+1823}\\
The physicist and chemist Michael Faraday presents of a series of papers on the liquefaction of gases.  The mathematician Pierre Frédéric Sarrus introduces the vertical bar symbol for integral: $\int_a^bf(x)\mathrm{d}x=F(x)|_a^b$.

\textbf{+1822}\\
The mathematician Jean-Victor Poncelet found the projective geometry. The same year, the physicist and mathematician Joseph Fourier formally presents the use of dimensions (units) for physical quantities and introduce Fourier series and also the notation $\int_a^bf(x)\mathrm{d}x$.

\textbf{+1821}\\
The principle of the dynamo is described for the first time by the physicist and chemist Michael Faraday. The same year the physicist John Herapath proposes that heat is in reality the effect of agitation and therefore the movement of elementary bodies.

\textbf{+1820}\\
The physicist Hans Oersted discovers and proves the magnetic effects of electric current. The same year, physicists Jean-Baptiste Biot and Félix Savart determine in the field of magnetism the famous law that bears their name.

\textbf{+1819}\\
The physicist and chemist Hans Christian Örsted shows that electric current deflected a magnetized needle, thus demonstrating electromagnetism and announcing an industrial revolution.

\textbf{+1818}\\
The mathematician, geometer and physicist Simeon Poisson calculates the Poisson bright spot at the center of the shadow of a circular opaque obstacle.

\textbf{+1817}\\
By studying the polarization of light, the physicist Augustin Fresnel shows that it is a transversal wave motion and not longitudinal and also shows that the diffraction and interference can be explained if we consider light as a wave. The same year, the astronomer Friedrich Bessel publishes the works making use of the famous functions that bear his name.

\textbf{+1816}\\
The mathematician Joseph Diaz Gergonne introduces the symbol marking the inclusion in the set theory.

\textbf{+1814}\\
The physicist and optician Joseph von Fraunhofer studied for the first time the absorption lines of the solar spectrum and this using the spectroscope which he was the inventor. The mathematician, astronomer and physicist Pierre-Simon Laplace makes the assumption that a perfect knowledge of the present state of the universe would enable one to determine perfectly all its future states.

\textbf{+1812}\\
The mathematician, astronomer and physicist Pierre-Simon Laplace published a major work on probability theory (including also the method of least squares) for which he is considered as one of the founders.

\textbf{+1811}\\
The chemist Amaedo Avogadro hypothesizes that equal volumes of different gases contain the same number of molecules under the same conditions of temperature and pressure.

\textbf{+1810}\\
The mathematician, astronomer and physicist Carl Friedrich Gauss discovered the basic concepts of non-Euclidean geometry but never published his work on the subject. The same year, the physicist and mathematician Joseph Fourier models the evolution of the temperature with trigonometric series.

\textbf{+1809}\\
The mathematician, astronomer and physicist Carl Friedrich Gauss developed the method of least squares independently of Legendre. The same year, the mathematician, astronomer and physicist Pierre-Simon Laplace proved the general form of the central limit theorem. The engineer, physicist and mathematician Etienne Malus publishes the law of Malus.

\textbf{+1808}\\
The physicist and chemist John Dalton proposes what is considered as the first theory of the atom. The mathematician Christian Kramp introduces in his \textit{Éléments d'arithmétique universelle} the notation $n!$ for the factorial.

\textbf{+1806}\\
The mathematician Jean Robert Argand published the first plane representation of complex numbers and algebraic measures are used. The mathematician Johann Carl Friedrich Gauss develops the idea of adding vectors in a geometric form and introduces the notation $\overrightarrow{ab}$ for a vector.

\textbf{+1805}\\
The mathematician André-Marie Legendre developed the method of least squares.

\textbf{+1803}\\
The physicist and chemist John Dalton has the original idea to assume that each chemical element is composed of different atoms. A chemical combination was then explained by the union of these atoms in fixed proportions and relative atomic masses became calculable from experimental facts. The same year, the economist, journalist and industrialist Jean-Baptiste Say published his Treatise on Political Economy.

\textbf{+1802}\\
The physicist Thomas Young showed the wave nature of light by an important experience that shows the interference of waves. The same year, the chemist and physicist Joseph Louis Gay-Lussac discovered the famous law connecting gas volume and temperature of a real gas, the law that bears his name (Gay-Lussac's law).

\textbf{+1801}\\
The chemist and physicist John Dalton discovered the law of the sum of the partial pressures which still bears his name.

\textbf{+1800}\\
The chemist William Nicholson and the surgeon Anthony Carlisle used electrolysis to separate water into hydrogen and oxygen. The same year, the astronomer William Herschel discovers infrared radiation and the physicist Alessandro Volta and invented the first electric battery.

\textbf{+1799}\\
The mathematician Gaspard Monge published his book of descriptive geometry. He is considered as the inventor of this field. First satisfactory but incomplete proofs of the fundamental theorem of algebra by the mathematician Johann Carl Friedrich Gauss.

\textbf{+1798}\\
An Essay on the Principle of Population by the cleric and scholar Thomas Robert Malthus (one of the founding books of the study of population dynamics) proposes one of the first models of population growth, the exponential model, which will be the basis for future work.

\textbf{+1798}\\
The mathematician Carl Friedrich Gauss gives a rigorous proof of the theorem of d'Alembert (fundamental theorem of algebra). The same year, the physicist Benjamin Thompson had the idea that heat is a form of energy and the physicist and chemist Henry Cavendish measures the gravitational constant. The economist Thomas Malthus stated his law of population.

\textbf{+1797}\\
The mathematician Caspar Wessel associates vectors with complex numbers and studies complex number operations in geometrical terms.

\textbf{+1793}\\
The National Assembly of the French Republic established the metric system.

\textbf{+1789}\\
The physicists and chemists Antoine Lavoisier states the law of conservation of mass.

\textbf{+1788}\\
In his \textit{Méchanique Analitique}, the mathematician and physicist Joseph-Louis Lagrange formulates a new way to study classical mechanics (by Isaac Newton for recall) using the principle of least action. The same year, the Academy of Science approves the creation of a universal measurement system, the future metric system. This project will also be approved by the French National Assembly in 1790, which will give the first definition of the meter.

\textbf{+1787}\\
The physicist, chemist and inventor Jacques Alexandre César Charles experimentally determined that the volume of a fixed mass of gas at constant pressure is proportional to temperature.

\textbf{+1786}\\
The astronomer William Herschel made a detailed description of our galaxy.

\textbf{+1785}\\
The physicist Charles Augustin Coulomb proves that the forces between electric charges and between magnets applied at the inverse square of the distance.

\textbf{+1783}\\
The clergyman and natural philosopher John Michell in a paper for the Philosophical Transactions of the Royal Society of London, read on 27 November 1783, first proposed the idea that there were such things as black holes, which he called "dark stars". A few years after Michell came up with the concept of black holes, the French mathematician Pierre-Simon Laplace suggested essentially the same idea in his 1796 book, \textit{Exposition du Système du Monde}. 

\textbf{+1782}\\
The mathematician, physicist and astronomer Pierre-Simon de Laplace introduces the "Laplace transform", a transformation that solves several differential equations in physics.

\textbf{+1781}\\
The chemist and physicist Joseph Priestley creates water by combustion of hydrogen and oxygen which shows that water is not an essential element as we thought since Aristotle.

\textbf{+1778}\\
The physicists and chemists Carl Scheele and Antoine Lavoisier discovered that air is composed mainly of nitrogen and oxygen.

\textbf{+1777}\\
The physicist and mathematician Leonhard Euler re-introduces the letter $\mathrm{i}$ for the imaginary part of the complex numbers.

\textbf{+1776}\\
The moral philosopher and a pioneer of political economy Adam Smith publishes his An Inquiry into the Nature and Causes of the Wealth of Nations.

\textbf{+1774}\\
The mathematician, astronomer and physicist Pierre-Simon Laplace explicitls Euler's integral. The same year, the theologian, dissenting clergyman, natural philosopher, educator and political theorist Joseph Priestley made his major discovery, that of oxygen.

\textbf{+1772}\\
The mathematician, engineer and astronomer Joseph-Louis Lagrange studied the three-body problem and discovered the libration points today named "Lagrange points".

\textbf{+1770}\\
In his \textit{Mémoire sur les équations aux différences partielles}, the philosopher, mathematician, and early political scientis Marie Jean Antoine Nicolas de Caritat, Marquis of Condorcet introduces the symbol of partial derivatives $\partial$.

\textbf{+1769}\\
In his \textit{Institutiones calculi integralis}, the mathematician Leonhard Euler studies for the first time the double integrals, calculates them by successive integration and changes of variable. These methods will be generalized to the triple integrals by the mathematician, engineer and astronomer Joseph-Louis Lagrange, which also gives the general formula for the change of variables (determinant of the Jacobian).

\textbf{+1766}\\
The physicist and chemist Henry Cavendish discovered and studied hydrogen.

\textbf{+1764}\\
The latent and specific heat are described for the first time by the physicist and chemist Joseph Black. It is also the first to clearly distinguish temperature and momentum. The same year, the physicist and mathematician Leonhard Euler examines the partial differential equation for the vibration of a circular drum and finds one of the Bessel function solutions.

\textbf{+1763}\\
A posthumous article of the mathematician and clergyman Thomas Bayes reveals that he discovered what is named today the "Bayes' theorem".

\textbf{+1757}\\
The physicist and mathematician Leonhard Euler founds modern hydrodynamics.

\textbf{+1756}\\
The mathematician, engineer and astronomer Joseph-Louis Lagrange develops analytical mechanics based on his invention of the calculus of variations independently of Leonhard Euler.

\textbf{+1755}\\
The physicist and mathematician Leonhard Euler introduces the uppercase Greek letter sigma ($\sum$) for the symbol of the sum.

\textbf{+1753}\\
In his 1749 study of the motions of the earth Leonhard Euler obtained differential equations for the orbital elements and in 1753 he applied the method of variation of constants to his study of the motions of the moon.

\textbf{+1750}\\
The mathematician Gabriel Cramer define the Cramer's rule for solving linear systems.

\textbf{+1749}\\
The astronomer and physicist Jean le Rond D'Alembert developed the first model of precession based on the theory of gravitation of Newton and gives a possible solution to the problem of three bodies.

\textbf{+1748}\\
In his \textit{Introductio in Analysin Infinitorum} the mathematician Leonhard Euler introduces the concept of function (defined as any composition of algebraic and analytic expression), defines the concepts of even and odd functions, popularize the use of the symbols $e$ and $\pi$, proves Euler's identity and defines the $\Gamma$ function that generalized the factorial.

\textbf{+1746}\\
The encyclopedist Jean le Rond D'Alembert gives the first evidence (acceptable but will be corrected later) of the fundamental theorem of algebra. The following year (1747) he published the equation of vibrating strings, which was the first example of the wave equation. This makes D'Alembert, one of the founders of mathematical physics.

\textbf{+1744}\\
The philosopher, mathematician, physicist, astronomer and naturalist Pierre Louis Moreau de Maupertuis states the principle of least action which will be formalized mathematically 22 years later by the mathematician, engineer and astronomer Joseph-Louis Lagrange. The same year, the physicist and mathematician Leonhard Euler shows the existence of transcendental numbers and introduces variations calculus.

\textbf{+1742}\\
The astronomer Anders Celsius defines its own unit of measurement for temperature.

\textbf{+1739}\\
The physicist and mathematician Leonhard Euler solves the general homogeneous linear ordinary differential equation with constant coefficients.

\textbf{+1738}\\
The physician, physicist and mathematician Daniel Bernoulli published a book on hydrodynamics introducing the kinetic theory of gases and the famous Bernoulli theorem (pressure balance). The same year in his \textit{Doctrine of chance}, the mathematician Abraham De Moivre introduces the Gaussian distribution as a means of approximating the binomial law for large number of experiments and demonstrates a partial version of the central limit theorem.

\textbf{+1737}\\
The physicist and mathematician Leonhard Euler solves the problem of graph theory on the bridges of Königsberg. The resolution of this problem is considered as the first theorem of graph theory. He establishes the same time the "Euler's formula" linking the number of vertices, edges and faces of a convex polyhedron, and hence of a planar graph.

\textbf{+1736}\\
The inventor Jonathan Hulls puts the first patent for a boat propelled by a steam engine.

\textbf{+1734}\\
The physicist and mathematician Leonhard Euler introduces the notation $f(x)$ for a function applied to the argument $x$.

\textbf{+1733}\\
The mathematicien Geralamo Saccheri studies what geometry would be like if Euclid's fifth postulate were false.

\textbf{+1729}\\
The dyer and amateur of physics and astronomy Stephen Gray was the first to discover the transmission of electricity in materials that he named "conductors".

\textbf{+1727}\\
The physicist and mathematician Leonhard Euler introduces the modern notation for the trigonometric functions and the letter $e$ for the base of the natural logarithm (occasionally also known as the "Euler number").

\textbf{+1724}\\
The mathematician Abraham De Moivre studies mortality statistics and the foundation of the theory of annuities in Annuities on lives.

\textbf{+1715}\\
The mathematician Brook Taylor publishes the tools that gives the possibility to make integration by parts and series expansions of functions (the famous Taylor series).

\textbf{+1714}\\
The mathematician Brook Taylor derives the fundamental frequency of a stretched vibrating string in terms of its tension and mass per unit length by solving an ordinary differential equation.

\textbf{+1713}\\
The mathematician and physicist Jacques Bernoulli publishes the rigorous principles of basic probabilities and statistics.

\textbf{+1705}\\
The astronomer Edmund (or Edmond) Halley predicted with an almost negligible calculation error with that the comet passed near the Earth in 1682 will return in 1758.

\textbf{+1704}\\
The physicist and mathematician Isaac Newton found experimentally that white light is composed of many colors. It also assumes that a light ray is composed of particles.

\textbf{+1701}\\
In his \textit{Explication de l'Arithmétique Binaire} Gottfried Wilhelm Leibniz introduces binary arithmetic (Leibniz may have been the first computer scientist and information theorist). He anticipated Lagrangian interpolation and algorithmic information theory. His \textit{Calculus ratiocinator} anticipated aspects of the universal Turing machine. In 1961, the mathematician and philosopher Norbert Wiener suggested that Leibniz should be considered the patron saint of cybernetics.

\textbf{+1698}\\
The mathematician and physicist Jacques Bernoulli clearly poses the problem of the brachistochrone curve (which belongs to the family of cycloid curves) and proposes a solution. The same year, the mathematician Guillaume de L'Hopital states his rule for the examination of indeterminate forms. Gottfried Leibniz proposes the use of the point $\cdot$ to denote multiplication, instead of the cross $\times$, which is too easily confused with the variable $x$ in the equations.

\textbf{+1696}\\
The mathematician and physicist Jacques Bernoulli clearly poses the problem of the brachistochrone curve (which belongs to the family of cycloid curves) and proposes a solution. The same year, the mathematician Guillaume de L'Hopital states his rule for the examination of indeterminate forms.

\textbf{+1693}\\
The astronomer and engineer Edmund Halley discovered the relation between the focal length of a lens with the distance of the image to its axis and the real object to its axis. The same year, he prepares the first mortality tables statistically relating death rate to age.

\textbf{+1691}\\
The philosopher and mathematician Gottfried Leibniz discovers the technique of separation of variables for ordinary differential equations.

\textbf{+1690}\\
The wave theory of light is put forward by the physicist and astronomer Christiaan Huygens. The same year the physicist and mathematician Johann Bernoulli developed the exponential calculus and find the equation of the catenary. The same year, the mathematician and physicist Jacques Bernoulli (brother of Jean Bernoulli) develops integral calculus.

\textbf{+1687}\\
The physicist and mathematician Isaac Newton published \textit{Naturalis Principia Mathematica} in which he explains the force of gravity and planetary orbits. He also describes the three laws of dynamics and derivates Kepler laws for his gravitational law. This is the first scientific revolution (before Special/General Relativity and quantum physics).

\textbf{+1685}\\
The philosopher and mathematician Gottfried Leibniz solves linear systems using without theoretical justification matrices and determinants.

\textbf{+1682}\\
The physicist and mathematician Isaac Newton establishes the law of gravitation, which now bears his name.

\textbf{+1679}\\
The philosopher and mathematician Gottfried Leibniz introduces binary arithmetic and develops a calculating machine that performs 4 operations. The same year, the physicist, mathematician and inventor Denis Papin shows experimentally the influence of atmospheric pressure on the boiling point of water.

\textbf{+1678}\\
The mathematicien, astronomer and physicist Christiaan Huygens states his principle of wavefront sources.

\textbf{+1676}\\
The physicist Robert Hooke states that stretching a spring is proportional to the stress.

\textbf{+1675}\\
The astronomer Olaus Roemer makes accurate measurements of the speed of light. The same year, the apothecary chemist Nicolas Lemery writes \textit{Cours de chymie} which is considered as the first great chemistry treatise where the mixtures are defined, the first theory of bases and acids, and so on are introduced and the physicist and astronomer Isaac Newton invents an algorithm to calculate the function roots.

\textbf{+1673}\\
The philosopher and mathematician Gottfried Leibniz invents his differential calculus, introduces the symbol $\int$ and use the term "convergence test" for alternated series and use the definition of the convergence for the first time. Leibniz also use the notation $\mathrm{d}y/\mathrm{d}x$ and $\mathrm{d}x$, $\mathrm{d}x$.

\textbf{+1671}\\
First attempt to calculate life annuities (comparable to life insurance) by the politician Johan de Witt in collaboration with the mathematician Christian Huygens and first calculations of life expectancy.

\textbf{+1670}\\
The mathematician John Wallis introduces the symbols $\le$ and $\ge$.

\textbf{+1669}\\
The mathematician, astronomer and physicist Christian Huygens published results on the observation of the conservation of kinetic energy becoming verbatim the discoverer of the concept of kinetic energy. The same year, in his manuscript \textit{De analysi per aequationes numero terminorum Infinitas}, the physicist and astronomer Isaac Newton gives the first description of the Newton's method which makes it possible to find approximations of function roots by iterated process. The description of Newton only applies to polynomials and does not use the notion of derivative (the manuscript will be published in 1711).

\textbf{+1668}\\
The physicist and astronomer Isaac Newton made the first reflecting telescope and the same year the mathematician John Wallis suggests the law of conservation of momentum.

\textbf{+1667}\\
In his \textit{Vera Circuli and Hyperbolae Quadratura} the mathematician and astronomer James Gregory gives the first proof of the fundamental theorem of calculus and independently discovers Taylor series. Draft of the concept of transcendental number elaborated in relation to the problem of the quadrature of the circle.

\textbf{+1665}\\
The physicist Isaac Newton formulated the three laws of mechanics. He lays the foundations of differential calculus, these techniques allows him starting from the expression of a force inverse of square of the distance to find the general form of Kepler's laws.

\textbf{+1664}\\
The physicist Isaac Newton begins to work on differential and integral calculus.

\textbf{+1661}\\
The founder of statistical demography John Graunt published the first mortality table, the same year the physicist and chemist Robert Boyle determines the laws of compressibility of gas bearing his name and sometimes attached to that of the physicist Edme Mariotte who rediscovered a few years after the same laws.

\textbf{+1659}\\
The mathematician, astronomer and physicist Christian Huygens discovered the rigorous isochronism formula (when the end of the pendulum travels an arc of cycloid, the period of oscillation is constant regardless of the amplitude).

\textbf{+1658}\\
The mathematician, astronomer and physicist Christian Huygens experimentally discovers that balls placed anywhere inside an inverted cycloid reach the lowest point of the cycloid in the same time and thereby experimentally shows that the cycloid is the isochrone.

\textbf{+1657}\\
The lawyer and mathematician Pierre de Fermat states his "Fermat's principle" in optics as how the light propagates from one point to another on trajectories such that the duration of the propagation is locally minimal.

\textbf{+1655}\\
The mathematician, astronomer and physicist Christian Huygens was the first to use the concept of expected mean in probabilities. The mathematician John Wallis introduces the symbol $\infty$ in his \textit{Mathesis Universalis}.

\textbf{+1654}\\
The mathematician, physicist, inventor, philosopher, moralist and theologian Blaise Pascal and the lawyer and mathematician Pierre de Fermat create the theory of probability.

\textbf{+1644}\\
The physicist and mathematician Evangelista Torricelli has the idea of substituting water by mercury in the so named Torricelli's experiment to highlight the "grosso-vido"; later will the works of the mathematician, physicist, inventor, philosopher, moralist and theologian Pascal Blaise follow (experience Puy de Dôme experiment).

\textbf{+1638}\\
The mathematician, geometer, astronomer and physicist Galileo Galilei publishes the mathematical relationship that defines the period of the simple pendulum.

\textbf{+1637}\\
The philosopher and mathematician René Descartes renames the unknowns $x$, $y$, $z$ and the parameters $a$, $b$, $c$ and extends the use of algebra to the lengths and plane, creating analytical geometry with Pierre de Fermat. The same year, always René Descartes, quantitatively derives the angles at which primary and secondary rainbows are seen with respect to the angle of the Sun's elevation.

\textbf{+1631}\\
The mathematician Thomas Harriot introduces, in a posthumous publication, the symbols $>$ and $<$. The same year the theologian and mathematician William Oughtred provides for the first time the multiplication symbol $\times$ and the symbol $\pm$.

\textbf{+1629}\\
The lawyer and mathematician Pierre de Fermat develops a rudimentary differential calculus. The same year, in his \textit{Invention nouvelle en algèbre} Albert Girard states, without proof, for the first time the fundamental theorem of algebra (a polynomial of degree $n$ has $n$ complex distinct or non-distinct roots) using complex numbers.

\textbf{+1626}\\
Tables of sine, tangent and secant of the engineer Albert Girard with the use of abbreviations sin, cos and tan.

\textbf{+1624}\\
Invention of the first thermometer (whose graduations are obviously not standardized...) by the physician Santorio Santorio.

\textbf{+1621}\\
The astronomer and physicist Willebrord Snell discovered that the angle of refraction of light is determined by the sine of the angle of the incident light with the normal of the dioptre.

\textbf{+1620}\\
The engineer Francis Thomas Bacon defends the experimental method and leads numerous observations on the heat. He suggested that the heat is related to the movement.

\textbf{+1619}\\
Johannes Kepler finished to publish the three laws of planetary motion.

\textbf{+1614}\\
The mathematician John Napier (John Napier in French) invented logarithms, which bring the operations of multiplication and division to simple additions or subtractions.

\textbf{+1611}\\
Johannes Kepler discovers total internal reflection, a small angle refraction law and thin lens optics.

\textbf{+1610}\\ In his \textit{Sidereus Nuncius}, Galileo Galilei reports the first observations at the telescope: the discovery of the satellites of Jupiter, the confirmation that the milky way is constituted of stars, the discovery of the rings of Saturn. These
Observations will have an important effect because they contradict some of the ideas of the Universe models of the time and the Bible writings.

\textbf{+1609}\\
In his \textit{Astronomia Nova}, the astronomer and mathematician Johannes Kepler explains Kepler's first two laws for the motion of the planets.

\textbf{+1608}\\
The optician Hans Lippershey invented the telescope to will be used and improved (with a random quality) the following year by the mathematician, geometer, astronomer and physicist Galileo Galilei to confirm the theories of Copernicus.

\textbf{+1604}\\
In a letter to Paolo Sarpi, Galileo states the law of the fall of bodies: the distance traveled is proportional to the square of the time of fall.

\textbf{+1603}\\
The mathematician and astronomer Thomas Harriot determines how to calculate qualitatively the surface of a spherical triangle.

\textbf{+1591}\\
The mathematician François Vieta opens a new period int algebra by making calculations with letters, using vowels for unknowns and consonants for parameters. Moreover, he gives the development of the Newton binomial theorem.

\textbf{+1590}\\
The astronomer Galileo Galilei demonstrated experimentally that all falling bodies have the same acceleration. The same year, the opticians Hans and Zacheraius Janssen created the first microscope by combining several lenses that define the beginnings of scientific medicine and biology.

\textbf{+1586}\\
The engineer and physicist Simon Stevin proved the method of the parallelogram of forces and discovered that the pressure of a liquid on the bottom of a container is independent of its shape, and also of the bottom surface and depends only on the height water in the container. He also gave the pressure measurement on any portion of the side of a container.

\textbf{+1576}\\
The astronomer Tycho Brahe observed a new star in the constellation of Cassiopeia and built an observatory on the island of Hveen.

\textbf{+1572}\\
The mathematician Rafaelle Bombelli gives a formulation of complex numbers and the rules of actual calculations. He introduces the terms più di meno (pdm) and meno di meno (mdm) to represent $+\mathrm{i}$ and $-\mathrm{i}$.

\textbf{+1548}\\
The mathematician and physicist Simon Stevin wrote the tenth powers identified with an exponent. He also gives the first writing of vectors. 

\textbf{+1545}\\
The mathematician Ludovico Ferrari gives the solution of equations of degree $4$ (know under "Cardan formula").

\textbf{+1543}\\
The work of the astronomer Nicolas Copernicus summarizing 26 years of research and observations is published and clearly highlights that the heliocentric system of Ptolemy is not valid.

\textbf{+1536}\\
The mathematician Niccolò Fontana launched the new science of ballistics.

\textbf{+1530}\\
The mathematician and physicist Robert Recorde introduces the "$=$" sign and the mathematician Michael Stifel developed an early form of algebraic notation.

\textbf{+1525}\\
The mathematician Christoff Rudolff introduces the notation for square roots $\sqrt{\phantom{a}}$.

\textbf{+1515}\\
Publication of Scipione del Ferro's works where he find a formula giving the general solution of the polynomial equations of degree $3$. These solutions involve the implicit manipulation of imaginary numbers

\textbf{+1510}\\
The painter, engraver and mathematician Albrecht Dürer develops the basics of descriptive geometry and perspective.

\textbf{+1500}\\
The Italian mathematician Scipione del Ferro succeeds for the first time to solve a large algebraic type of cubic equations.

\textbf{+1490}\\
The painter, sculptor, architect, musician, mathematician, engineer, inventor, anatomist, geologist, cartographer, botanist, and writer... Leonardo da Vinci describes capillary action.

\textbf{+1489}\\
In his \textit{Behende und hupsche Rechnung auf allen kauffmanschafft}, the mathematician Johannes Widmannwe the symbols "$+$" and "$-$" are introduced for the first time (before plus was denoted "P" and minus "M").

\textbf{+1464}\\
In \textit{Triparty in the science of numbers} by the mathematician Nicolas Chuquet we can find the first use of negative powers and zero power and the statement of the property of exponents that we still use $x^{n+m}=x^nx^m$.

\textbf{+1420}\\
The mathematician and astronomer Jamshid al-Kashi computed and observed the solar eclipses of 1406, 1407 and 1408. He is also the first to use decimal notation in arithmetic and in arabic numerals.

\textbf{+1400}\\
The Mathematician and astronomer Jamshid al-Kashi developed an early form of Newton's Regula falsi method.

\textbf{+1200-1400}\\
Madhava and the Kerala School (India) discover several infinite series for numbers like $\pi$ and specific values of trigonometric functions - these works ahead of those of the Europeans on the differential and integral calculus and the series of powers.

\textbf{+1350}\\
\textit{Tractatus of configurationibus qualitatum and motuum} of Oresme is draft of geometry using coordinates, and use axes for different sizes, which is an important step in the transition from qualitative science based mainly on Aristotle to quantitative science. Proof of the mean velocity theorem, which anticipates the results of Gallileo on the uniform rectilinear motion and the bodies in free fall by linking area under the curve of the velocity to the position in a graph.

\textbf{+1303}\\
\textit{Siyuan Yujian }(translation:\textit{ Precious mirror of elements}) of Zhu Sjijie describes the elimination method for solving systems of equations containing up to four unknowns and up to degree 14 for some form of equations. We also find there the definition of the Pascal's triangle and the formulas of summations for some series.

\textbf{+1300}\\
Raymon Lulle developed a geometric (useless) machine to automate theist logic. This idea will influence Leibniz in his research of a universal language for reasoning, research that will lead him to take an interest in Chinese writing and binary arithmetic. Lulle's ideas anticipate modern ideas of formal deduction systems.

\textbf{+1269}\\
The scholar Pierre de Maricourt coined the expressions of the magnetic "north" and "south poles" and he was the first who wrote that opposite poles attract each other.

\textbf{+1268}\\
The philosopher, scientist and alchemist Roger Bacon publishes proposals to reform school, arguing that to study nature, the use of observations of the measures is the only basis of rigorous testing and verification while affirming at the same time the need of mathematics for this purpose.

\textbf{+1200}\\
The mathematician Jordan Nemore introduces the notation for unknowns with symbols.

\textbf{+1150}\\
Creation of modern notation for fractions (horizontal bar) by Al-Hassãr. At the same period, we have the Latin translation of the 820 treaty by Al-Khwarizmi on the Indian calculation which allows the decimal system and the use of zero to spread in Europe and also Gérard de Cremona publishes a translation in Latin of the Arabic version of the Almagest of Ptolemée, the name "sinus" comes from this translation...

\textbf{+1121}\\
The astronomer, physicist, biologist, chemist, mathematician and philosopher Abu al-Fath Khazini published a book in which he proposed that gravity and gravitational potential energy vary with distance from the center of the Earth. He also makes a distinction between force, mass and weight. He also invented several scientific instruments, including a steelyard and hydrostatic balance. He also introduces experimental scientific methods to static and dynamic, unifies them in the science of mechanics and combines hydrostatics with the dynamic to create hydrodynamics.

\textbf{+1114}\\
The mathematician Bhaskara provides a comprehensive summary of Hindu mathematics, as developed from the 5th to the 7th century AD. He also recognizes the negative square root, solves quadratic equations with several unknowns, equations of higher order such as Fermat and the general quadratic equations. He was also a pioneer in the principle of differential calculus nearly 500 years before Newton and Leibniz.

\textbf{+1100}\\
The philosopher and physicist Allah Abu'l-Barakat Hibat al-Baghdaadi is the first to deny Aristotle's idea that a constant force produces uniform motion what prepares the Newton's second law of motion. Like Newton, he described acceleration as the variation of speed.

\textbf{+1037}\\
The mathematician, physicist and philosopher Ibn al-Haytham is aware of the magnitude of the acceleration due to gravity. He discovers the law of inertia, known today as the first law of motion Newton.

\textbf{+1030}\\
The philosopher, writer, physician and scientist Abu Ali al-Husayn ibn Abd Allah Ibn Sina (known in Occident as Avicenna) note that if the perception of light is due to the emission of some sort of particles by a light source, the speed of light has to be finished. He also provided a sophisticated explanation for the phenomenon of rainbow. The mathematician, astronomer, physicist, scholar, encyclopedist, philosopher, astrologer, traveler, historian, pharmacologist Ab? al-Rayhan Muhammad ibn Ahmad al-Biruni, and later the astronomer Abu al'Fath Khazini, were the first to apply scientific methods in experimental mechanics, especially in the fields of statics and dynamics, to determine the specific weight, such as those based on the theory of balances and weighting.

\textbf{+1021}\\
The philosopher, mathematician and physicist Ibn al-Haytham is considered the father of optics and a pioneer of the scientific method describes correctly the light and vision, and introduces the experimental scientific method, laying the foundations of experimental physics. He also discusses experimental psychology and describes various optical instruments such as the darkroom. He wwas able to estimate atmosphere width with an accuracy of $1$ [km], he defined the inertia principia (first Newton's law), the linear momentum and he calculated $\sum_{k=1}^n k^4=\frac{n(2n+1)(n+1)(3n^2+2n-1)}{30}$.

\textbf{+1019}\\
The mathematician, astronomer and physicist Abu Rayhan Al-Biruni observed and described the solar eclipse of April 8, 1019, and the lunar eclipse of September 17, 1019, in detail; he gave the exact location of the stars during the lunar eclipse. He invented the astrolabe and the planisphere.

\textbf{+1010}\\
The mathematician Al-Sijistani Zuraqi invented a astrolabe designed for a single heliocentric planetary model in which the Earth is moving, rather than the sky.

\textbf{+1000}\\
The mathematician, physicist and astronomer Abu Sahl al-Qouhi discovers that the weight of bodies varies with their distance from the center of the Earth, and solves equations higher than the second degree. During the same decade, the mathematician and engineer Al'Karkhi wrote a book containing the first known proof by mathematical induction. He uses it to prove the binomial theorem, the Pascal's triangle, and the sum of the cubes integrals.

\textbf{+996}\\
The mechanical oriented astrolabe, with 8 gears is invented by the mathematician, astronomer and physicist Abū Rayhān Al-Biruni who is also the author of works on the summation of series and combinatorics.

\textbf{+980}\\
Abitu al-Wafaa do the first calculation of values of trigonometric functions and publish the sinus law adapted for triangles on sphere. He also proved by induction $\sum_{k=0}^n k^3=\frac{n^2(n+1)^2}{4}$.

\textbf{+964}\\
The mathematician, physicist and philosopher Abd al-Rahman al-Sufi explains the magnifying power of lenses and was the first to use a scientific method of analysis that will greatly influence future scientists.

\textbf{+953}\\
The engineer and mathematician Al-Karkhi defines different monomials and gives rules for products of any two of them. He also discovered the binomial theorem for integer exponents.

\textbf{+952}\\
The mathematician Abu'l-Hasan al-Uqlidisi modifies the calculation methods for the numerical Indian system to make it possible for feathers and paper usage. Until then, do calculations with Indian numerals necessitated the use of a board.

\textbf{+900}\\
The first reference to a viewing tube can be found in the work of the astronomer and mathematician Al-Battani, and the first accurate description of the observation tube was given by the mathematician, astronomer, physicist, scholar, encyclopedist, philosopher, astrologer, traveler, historian, pharmacologist Al-Biruni, in a section of his work dedicated to verify the presence of the new crescent moon at the horizon. Although these preliminary observations tubes do not have lenses, they allow an observer to focus on a part of the sky by eliminating light interference. These observation tubes were later adopted in Europe, where they influenced the development of the telescope.

\textbf{+880}\\
The astronomer and mathematician Al-Battani discovered the motion of the apogee of the Sun, calculate the values of the precession of the equinoxes and the inclination of the Earth's axis. He is at the origin of the definition of the tangent and cotangent trigonometric functions.

\textbf{+820}\\
The word "algebra" appears. The mathematician, geographer, astrologer and astronomer Muhammad ibn Musa Al'Khwarizmi is often regarded as the father of medieval algebra because he releases it from geometry. He is also the origin of the quadrant, the mural instrument, the sinus quadrant that was used to solve trigonometric problems and make astronomical observations.

\textbf{+800}\\
Astronomers invent the universal sundial and universal time dial in Baghdad.

\textbf{+780}\\
The alchemist Jabir Ibn Hayyan introduces the experimental scientific method for chemistry and also laboratory equipment such as still and processes such as pure distillation, liquefaction, crystallisation and filtration. He also invented more than 20 types of laboratory equipment, which resulted in the discovery of several chemicals. He also developed recipes for colored glass.

\textbf{+773}\\
Arabic numerals (adapted from India) made their first apparition in Europe.

\textbf{+628}\\
The mathematician Brahmagupta gives rules for solving linear and quadratic equations. He discovers that the quadratic equations have two roots: the negative one and the irrational and give the modern form of the solution we know today. He also give the rules to calculates with negative signs (arithmetic of negative numbers).

\textbf{+550}\\
Hindu mathematicians give zero a numeral representation in a positional notation system.

\textbf{+499}\\
The mathematician Âryabhat gets the full number of solutions of a system of linear equations by methods equivalent to modern methods, and describes the general solution of such equations. He also provides solutions of differential equations. He also asserts that the moon and celestial objects moon other than the stars reflect sunlight, he correctly explains the causes of lunar and solar eclipses, gives the length of the sidereal year to a few minutes, approximate $\pi$ by $62832/20000$, calculates the Earth's diameter with more precision than Erathostene, described the calculation with the Indian numbering system, gives the oldest sinus table for $24$ angles.

\textbf{+275}\\
The mathematician Diophantus of Alexandria considered as the father of algebra equations studied equations with rational variables (thus including quadratic equations) and Diophantine equations.

\textbf{+195}\\
\textit{Suàn shù shu}(translation: \textit{book on numbers and calculation}) in China is one of the oldest known Chinese mathematical texts that contains among other things, calculations of sums of Geometric progression for interest.

\textbf{+130}\\
\textit{Mathematical Composition} (also know under the name of \textit{Almagest's}) of Ptolemy in Alexandria presents a geometric model of the solar system that attempts to describe the motion of the planets the model is inspired by a geometric idea of Apollonius and uses circles whose center move in circular orbits this model puts the Earth at the center of the solar system but gives a fairly good description of the observed movements of the different stars it will be the dominant model until Copernicus some ideas prefigure the Fourier series that will be introduced in the 19th century.

\textbf{+125}\\
The \textit{Yale Music Papyrus} and the \textit{Michigan Instrumental Papyrus} seems to contain the oldest known examples of musical notation.

\textbf{+121}\\
Year corresponding to the oldest document mentioning the magnetic stone.

\textbf{+120}\\
Star catalog of Zhang Heng which contains 2500 stars. Zhang Heng correctly describes the cause of the eclipses and demonstrates that the moon is spherical.

\textbf{+100}\\
The engineer, mechanician and mathematician Hero of Alexandria rediscovered (after the Chinese) the concept of force. He also invented a system of gears to lift weights using steam power. He gives the first description of the sextant (but did not, however, invented it). His contemporary astronomer Claudius Ptolemy invented the sextant and described the astrolabe (perhaps invented by the astronomer, geographer and mathematician Hipparchus) and studied the refraction and reflection. During the same century the mathematician and philosopher Nicomachus of Gerase defines the even and odd numbers, prime and composite numbers, and perfect numbers. Also during the same century the final version of the \textit{Nine Chapters on Mathematical Art} (almost $1180$ pages), written over ten years by several anonymous Chinese authors contains the first use of negative numbers, chapter 9 uses the Pythagorean theorem, chapter 8 uses matrices and the elimination of Gauss to solve systems of equations (at least 1700 years before Gauss!!!)

\textbf{+80}\\
The scholar Wang Ch'ung made the first magnetic compass on a plate of brass.

\textbf{-87}\\
Year corresponding to the dating of the Antikythera mechanism, considered as the first calculator and the first antique analog gear (thirty!) machine to calculate complexes astronomical positions. The care and skill with which this machine was made, as well as the necessary mechanical and astronomy capacities question the historical knowledge of Greek science before its discovery. Indeed, any object of the same age and same complexity was known in the world and it takes nearly a millennium to see similar mechanisms appear! The physicist, mathematician and engineer Archimedes of Syracuse is the hypothetical creator.

\textbf{-100}\\
The indian text \textit{Anuyoga Dwara Sutra} contains several identities involving square roots and squares that seem to imply some knowledge of the laws of exponents or logarithms. An identity taken from this text in modern notation: $\sqrt{a}\sqrt{\sqrt{a}}=(\sqrt{\sqrt{a}})^3$.

\textbf{-134}\\
The astronomer, geographer, and mathematician Hipparchus of Nicaea discovers the precession of the equinoxes.

\textbf{-150}\\
The astronomer, geographer and mathematician Hipparchus is often referred as the founder of trigonometry and corresponding numerical tables. He calculates the first period of revolution of the Sun around the Earth (but the numerical results are in fact those of the rotation of the Earth around the Sun) and develops the theory of eccentrics and epicycles.

\textbf{-200}\\
During the century, the Chinese had invented the tide gate, the rudder, the principle of the steam engine several hundred years before the occident! During this century, the scientist and engineer Philo of Byzantium wrote treatises on levers, pneumatics, automation, traction and water clocks.

\textbf{-225}\\
The astronomer and mathematician Apollonius of Perga published the first study on conics giving to the ellipse, the parabola and the hyperbola the names we know today. He is also credited for the hypothesis of eccentric orbits to explain the apparent motion of the planets and the speed variation of the Moon.

\textbf{-250}\\
The mathematician, physicist and engineer Archimedes of Syracuse study simple machines such as the lever, the famous screw for pumping water ("Archimedes screw") and discovered the Archimedes's principle explaining buoyancy. In the same decade, the astronomer, geographer, philosopher and mathematician Eratosthenes of Cyrene calculated the diameter of the Earth using a gnomon and its shadow and demonstrated the inclination of the ecliptic, the distance Earth-Moon and Earth-Sun and also created a method to determine prime numbers (Sieve of Eratosthenes).

\textbf{-260}\\
The mathematician, physicist and engineer Archimedes of Syracuse computes to two decimal places using inscribed and cirumscribed polygons and computes the area under a parabolic segment.

\textbf{-281}\\
The astronomer and mathematician Aristarchus of Samos assumed that the Sun is the center of the solar system and uses trigonometry to estimate the radius of the Moon and its distance from the Earth using the Earth's shadow during a lunar eclipse.

\textbf{-300}\\
The mathematician and geometer Euclid published his \textit{Elements}, where he reorganized the entire knowledge of the geometry including logical proofs, the construction of the $5$ Platonic solids. In his \textit{Optica} he noted that the light goes in a straight line and describes the law of reflection. We have also that the conical, elliptical, parabolic and hyperbolic concepts appears in the works of two mathematicians of the ancient Greece, namely Menaech-mus and Appolonius de Perge that also introduces the concept of tangent. The same period Bhagabati Sutra calculates the permutations and combinaisons of order $1$, $2$ and $3$.

\textbf{-310}\\
The scholarly Autolycus of Pitane defines uniform motion as an object that travels an equal distance in a equal amount of time.

\textbf{-370}\\
The philosopher Aristotle develops the logic with a theory of naive proposals, quantities and inferential reasoning.

\textbf{-388}\\
The philosopher and astronomer Heraclides of Pontus assumes the rotation of the Earth itself to explain the apparent motion of stars in the night (but still in a geocentric context) and suggests that each planet is a body like the Earth.

\textbf{-400}\\
The Stoic School develops the composed proposals and the logical connectors: "implies", "and", "or" and the inferences "Modus ponens" and "Modus tollens".

\textbf{-430}\\
The philosopher Democritus of Abdera advance the idea that matter is composed of tiny and indentical particles he name "atoms". In reality it is rather an extension of the ideas of his teacher, the philosopher Leucippus of Miletus developed ten years before. Hippasus, a disciple of Pythagoras, would have given what is probably the first rigorous proof of the irrationality of $\sqrt{2}$. The proof uses a Reductio ad absurdum reasoning to show that the sides of a square are incommensurable with its diagonal. 

\textbf{-500}\\
The philosophers Leucippus and Democritus are the founders of atomism.

\textbf{-540}\\
The philosopher, mathematician Pythagoras studies propositional geometry and vibrating lyre strings. 

\textbf{-600}\\
The philosopher Empedocles of Acragas calls for decomposition of the world into four fundamentals elements: water, earth, air and fire. The same century, the mathematician Thales of Miletus highlights electrostatic by rubbing a piece of amber, predicted an eclipse and develops the geometry of the triangle.

\textbf{-750}\\
Manava Sulbar Sutras of Manava finds irrationality of $\sqrt{2}$ and $\sqrt{61}$ and agrees to use irrational numbers in his calculations.

\textbf{-800}\\
Assyrians use water-clocks and Chinese plot planetary movements for their calendar

\textbf{-1500}\\
Indians develops theory of the $4$ elements (fire, air, water, earth)

\textbf{-1400}\\
The Neolithic peoples of Scotland nowadays build stone models of Plato's five solids (regular polyhedra).

\textbf{-1700}\\
The mathematician Apastamba solves general linear equations and uses Diophantine systems of equations with up to five unknowns. The same century, egyptian mathematicians employ primitive fractions.

\textbf{-1800}\\
Babylonian scribes seek the solution of a quadratic equation.

\textbf{-2000}\\
Babylonian priests do the first records of celestial observations.

\textbf{-2300}\\
Chinese astronomers make the first observations of the sky.

\textbf{-2500}\\
The Mesopotamians imagined a position numbering system composed of symbols whose value is based on their rank within a number.

\textbf{-2600}\\
Oldest known mathematical table where a multiplication table is engraved to calculate areas.

\textbf{-3000}\\
Chinese and Babylonians invented the abacus as first adding machine. Geometric concepts are developed for land surveying (hypotenuse calculus). It is also the period corresponding to the oldest know tool used by Incas to record numbers thanks to knots on a string and also the wall representation of wheels.

\textbf{-3500}\\
Oldest Weather Report Found on Stone in Egypt. The unusual weather patterns described on the slab were the result of a massive volcano explosion at Thera, the present day island of Santorini in the Mediterranean Sea

\textbf{-3700}\\
A Babylonian tablet seems to contain first remarkable trigonometric angles and the evidence that Pythagoras' theorem was already known by Babylonians.

\textbf{-4900}\\
The Goseck circle (German: Sonnenobservatorium Goseck) may be one of the oldest Solar observatories in the world.

\textbf{-5000}\\
The decimal system is used in Ancien Egypt (it seems that the consensus is between -6000 and -3000).

\textbf{-8000}\\
Warren Field is the location of a mesolithic calendar. It includes $12$ pits believed to correlate with phases of the Moon and used as a lunar calendar. It is considered to be the oldest lunar calendar yet found.

\textbf{-5200}\\
Radiocarbon dating of the oldest founded wheels (Ljubljana Marshes wooden Wheel).

\textbf{-8000}\\
Marks one bones or woods are slowly replaced by tokens of various shapes to count.

\textbf{-20000}\\
The Ishango bone is a bone tool, dated to the Upper Paleolithic era. It is a dark brown length of bone, the fibula of a baboon, with a sharp piece of quartz affixed to one end, perhaps for engraving. It was first thought to be a tally stick, as it has a series of what has been interpreted as tally marks carved in three columns running the length of the tool. It has also been suggested that the scratches might have been to create a better grip on the handle or for some other non-mathematical reason.

\textbf{-200000}\\
Claims for the earliest definitive evidence of control of fire by a member of Homo range from 0.2 to 1.7 million years ago. Evidence for the controlled use of fire by Homo erectus, beginning some 400,000 years ago, has wide scholarly support.

	\chapter{Humor}
	\minitoc
	\pagebreak
	If you have any scientific humorous stories do not hesitate to let us know. In all cases, we wish you a good time (some stories are in French as they lose their meaning in English).
	
\begin{center}
\textbf{This book is transmitted with 100\% recycled electrons}
\end{center}

	\section{Situations}
	
	An engineer, a physicist, and a mathematician are shown a pasture with a herd of sheep, and told to put them inside the smallest possible amount of fence. 

\begin{itemize}	 
	\item[$-$] The engineer is first. He herds the sheep into a circle and then puts the fence around them, declaring:"A circle will use the least fence for a given area, so this is the best solution." 

	\item[$-$] The physicist is next. She creates a circular fence of infinite radius around the sheep, and then draws the fence tight around the herd, declaring, "This will give the smallest circular fence around the herd."

	\item[$-$] The mathematician is last: After giving the problem a little thought, he puts a small fence around himself and then declares, "I define myself to be on the outside!" 
\end{itemize}
\begin{center}\underline{\hspace{5 cm}}\end{center}

Two men are sitting in the basket of a balloon. For hours, they have been drifting through a thick layer of clouds, and they have lost orientation completely. Suddenly, the clouds part, and the two men see the top of a mountain with a man standing on it.

\begin{itemize}	 
	\item[$-$] "Hey! Can you tell us where we are?!"
\end{itemize}

The man doesn't reply. The minutes pass as the balloon drifts past the mountain. When the balloon is about to be swallowed again by the clouds, the man on the mountain shouts: 

\begin{itemize}	 
	\item[$-$]  "You're in a balloon!"

	\item[$-$] "That must have been a mathematician."
\end{itemize}

The man astonished asks:

\begin{itemize}	 
	\item[$-$] "Why?"

	\item[$-$]  "Firt, he thought long and thoroughly about what to say. Second, what he eventually said was irrefutably correct. And last but not least... it was of no use whatsoever..."
\end{itemize}	
\begin{center}\underline{\hspace{5 cm}}\end{center}
	
	\begin{center}
		\includegraphics[scale=0.7]{img/humour/research_in_peace.jpg}	
	\end{center}
	
\begin{center}\underline{\hspace{5 cm}}\end{center}

An engineer, a mathematician, and a physicist are staying for the night in a hotel. Fortunately for this joke, a small fire out in each room.

\begin{itemize}	 
	\item[$-$] The physicist awakes, sees the fire, makes some careful observations, and on the back of the hotel's wine list does some quick calculations. Grabbing the fire extinguisher, he puts out the fire with one, short, well placed burst, and the crawls back into bed and goes back to sleep.

	\item[$-$] The engineer awakes, sees the fire, makes some careful obsrevations, and on the back of the hotel's romme service list does some quick calculations. Grabbin the fire extinguisher (and adding a factor safety of 5), he puts out the fire by hosing down the entire room server times over, and then crawls into his soggy bed and goes back to sleep.

	\item[$-$] The mathematician awakes, sees the fire, makes some careful observations, and on a blackboard installed in the room, does some quick calculations. Jubliant, he exclaims "A solution exists!", and crawls into his dry bed and goes back to sleep.
\end{itemize}	
\begin{center}\underline{\hspace{5 cm}}\end{center}

A doctor, a lawyer and a mathematician were discussing the relative merits of having a wife or a mistress.

The lawyer says: "For sure a mistress is better. If you have a wife and want a divorce, it causes all sorts of legal problems.

The doctor says: "It's better to have a wife because the sense of security lowers your stress and is good for your health.

The mathematician says: "You're both wrong. It's best to have both so that when the wife thinks you're with the mistress and the mistress thinks you're with your wife... you can do some mathematics."
\begin{center}\underline{\hspace{5 cm}}\end{center}

A prominent businessman hires a mathematician, a physicist and a computer scientist in order to win every horses competition.

The mathematician go on the task, it computes matrices without end, define axioms at any headland and after long weeks of lemmas, theorems and conjectures, he concludes that the problem is formally unhearable.

Then the computer scientist happy to see that the mathematician failed, is approaching its Cray III and after writing volumes of algorithms in C++ and introduced all the parameters and initial conditions joyfully announces that it will take just a few hundred years to calculate the result of each competition ...

The physicist, has a smile, he informs his distinguished colleagues that he has the solution. He approaches a blackboard and while drawing a sphere begins by saying: "approximate the horse by a perfect sphere..."
\begin{center}\underline{\hspace{5 cm}}\end{center}

During a job interview, an entrepreneur receives four engineers: one who followed the Military Polytechnic School of Paris, the second HEC, the third is computer engineer, and the last followed University. It explains the four candidates in the end, to run a business, you just need to count.

He therefore addresses the first of them, the polytechnician, and said, "go ahead, count ..."

\begin{itemize}	 
	\item[$-$] The polytechnician: "one... two... one... two..."\end{itemize}


The man surprised then asked the engineer who followed HEC: "To you! Count ..."

\begin{itemize}	 
	\item[$-$] The engineer of HEC: "one KiloDollar... two KiloDollar, three KD..."\end{itemize} 


He then turns anxiously toward the computer scientist: "To you! Count ..."

\begin{itemize}	 
	\item[$-$] The computer scientist: "0... 1... 0... 1..." \end{itemize}

Desesperated, the entrepreneur ask the engineer who followed University: "Go ahead, count ..."

\begin{itemize}	 
	\item[$-$] The young man begins: "1... 2... 3... 4... 5... 6... 7..." \end{itemize}

The entrepreneur feeling reassured: "Continue, continue ..."

\begin{itemize}	 
	\item[$-$] "8... 9... 10... valet... lady.. king... " !!\end{itemize}
	
	\begin{center}\underline{\hspace{5 cm}}\end{center}

Several people were asked to solve the following problem: "Prove that all odd integers are prime."

\begin{itemize}	 
	\item[$-$] Mathematician: 3 is a prime, 5 is a prime, 7 is a prime, 9 is not a prime - counter-example - claim is false.

	\item[$-$] Physicist: 3 is a prime, 5 is a prime, 7 is a prime, 9 is an experimental error, 11 is a prime ...

	\item[$-$] Engineer: 3 is a prime, 5 is a prime, 7 is a prime, 9 is a prime, 11 is a prime ...

	\item[$-$] Computer Scientist: 3's a prime, 5's a prime, 7's a prime ... segmentation fault

	\item[$-$] Lawyers: one is prime, three is prime, five is prime, seven is prime, although there appears to be prima facie evidence that nine is not prime, there exists substantial precedent to indicate that nine should be considered prime. The following brief presents the case for nine's primeness...

	\item[$-$] Liberals: The fact that nine is not prime indicates a deprived cultural environment which can only be remedied by a federally funded cultural enrichment program.

	\item[$-$] Computer programmers: one is prime, three is prime, five is prime, five is prime, five is prime, five is prime five is prime, five is prime, five is prime...

	\item[$-$] Professor: 3 is prime, 5 is prime, 7 is prime, and the rest are left as an exercise for the student.

	\item[$-$] Linguist: 3 is an odd prime, 5 is an odd prime, 7 is an odd prime, 9 is a very odd prime,...

	\item[$-$] Computer Scientist: 10 prime, 11 prime, 101 prime...

	\item[$-$] Chemist: 1 prime, 3 prime, 5 prime... hey, let's publish!

	\item[$-$] New Yorker: 3 is prime, 5 is prime, 7 is prime, 9 is... NONE OF YOUR DAMN BUSINESS!

	\item[$-$] Programmer: 3 is prime, 5 is prime, 7 is prime, 9 will be fixed in the next release,...

	\item[$-$] Salesperson: 3 is a prime, 5 is a prime, 7 is a prime, 9 -- let me make you a deal...

	\item[$-$] Advertiser: 3 is a prime, 5 is a prime, 7 is a prime, 11 is a prime,...

	\item[$-$] Accountant: 3 is prime, 5 is prime, 7 is prime, 9 is prime, deducting 10% tax and 5% other obligations.

	\item[$-$] Statistician: Let's try several randomly chosen numbers: 17 is a prime, 23 is a prime, 11 is a prime... Looks good to me.

	\item[$-$] Psychologist: 3 is a prime, 5 is a prime, 7 is a prime, 9 is a prime but tries to suppress it... 
\end{itemize}
	\begin{center}\underline{\hspace{5 cm}}\end{center}
	
A mathematician, an engineer and a physicist are given a red rubber ball to determine its volume.

\begin{itemize}	 
	\item[$-$] The mathematician: Measure the diameter and evaluates the triple integral.

	\item[$-$] The physicist: Fill a tub of water, places the ball in water and measure the total displacement volume.

	\item[$-$] The engineer: Search the model and serial number in its "red rubber balls" table.
\end{itemize}		
	\begin{center}\underline{\hspace{5 cm}}\end{center}

The following concerns a question in a physics degree exam at the University of Copenhagen: "Describe how to determine the height of a skyscraper with a barometer." 

One student replied: 

"You tie a long piece of string to the neck of the barometer, then lower the barometer from the roof of the skyscraper to the ground. The length of the string plus the length of the barometer will equal the height of the building." 

This highly original answer so incensed the examiner that the student was failed immediately. The student appealed on the grounds that his answer was indisputably correct, and the university appointed an independent arbiter to decide the case. 

The arbiter judged that the answer was indeed correct, but did not display any noticeable knowledge of physics. To resolve the problem it was decided to call the student in and allow him six minutes in which to provide a verbal answer that showed at least a minimal familiarity with the basic principles of physics. 

For five minutes the student sat in silence, forehead creased in thought. The arbiter reminded him that time was running out, to which the student replied that he had several extremely relevant answers, but couldn't make up his mind which to use. On being advised to hurry up the student replied as follows: 

"Firstly, you could take the barometer up to the roof of the skyscraper, drop it over the edge, and measure the time it takes to reach the ground. The height of the building can then be worked out from the formula H = 0.5g x t squared. But bad luck on the barometer." 

"Or if the sun is shining you could measure the height of the barometer, then set it on end and measure the length of its shadow. Then you measure the length of the skyscraper's shadow, and thereafter it is a simple matter of proportional arithmetic to work out the height of the skyscraper." 

"But if you wanted to be highly scientific about it, you could tie a short piece of string to the barometer and swing it like a pendulum, first at ground level and then on the roof of the skyscraper. The height is worked out by the difference in the gravitational restoring force T =2 pi sqr root (l /g)." 

"Or if the skyscraper has an outside emergency staircase, it would be easier to walk up it and mark off the height of the skyscraper in barometer lengths, then add them up." 

"If you merely wanted to be boring and orthodox about it, of course, you could use the barometer to measure the air pressure on the roof of the skyscraper and on the ground, and convert the difference in millibars into feet to give the height of the building." 

"But since we are constantly being exhorted to exercise independence of mind and apply scientific methods, undoubtedly the best way would be to knock on the janitor's door and say to him 'If you would like a nice new barometer, I will give you this one if you tell me the height of this skyscraper'." 

The student was Niels Bohr (Nobel Price 1923), the only Dane to win the Nobel Prize for physics. 
\begin{center}\underline{\hspace{5 cm}}\end{center}
	
The story goes that Bertrand Russell, in a lecture logic, mentioned that in the sense of material implication, a false proposition implies any proposition. A student raised his hand and said:

\begin{itemize}	 
	\item[$-$] "In that case, given 2+2=5, you can prove that you are the Pope"

	\item[$-$] "Yes", answered Russell

	\item[$-$] "And you could prove it now?!", asked the student sceptical

	\item[$-$] "For sure!", said Russel who proposed immediately the following proof:

	\begin{enumerate}
		\item Suppose that 2 + 2 = 5

		\item Substract 2 from the both side of the equality, whe have 2 = 3

		\item By symetry, 3 = 2

		\item Substract 1 each side, we obtain, 2 =1
	\end{enumerate}

	\item[$-$] Now the set containing just me and the Pope has 2 members. But 2=1, so it has only 1 member; therefore, I am the Pope...
\end{itemize}
	\begin{center}\underline{\hspace{5 cm}}\end{center}
	
What is "pi"?

\begin{itemize}	 
	\item[$-$] Mathematician: "Pi is the ratio of the circumference of a circle to its diameter."

	\item[$-$] Engineer: "Pi is about 22/7."

	\item[$-$] Physicist: "Pi is 3.14159 plus or minus 0.000005"

	\item[$-$] Computer Programmer: "Pi is 3.141592653589 in double precision."

	\item[$-$] Nutritionist: "You're one track math-minded fellows, Pie is a healthy and delicious dessert!"
\end{itemize}
	\begin{center}\underline{\hspace{5 cm}}\end{center}
	
An astronomer, a physicist, a mathematician and a computer scientist (it is said) were holidaying in Scotland. Glancing from a train window, they observed a black sheep in the middle of a field. 

\begin{itemize}	 
	\item[$-$] "How interesting", observed the astronomer, "all scottish sheep are black!" 

	\item[$-$] To which the physicist responded: "No, no! Some Scottish sheep are black!" 

	\item[$-$] The mathematician gazed heavenward in supplication, and then intoned: "In Scotland there exists at least one field, containing at least one sheep, at least one side of which is black."

	\item[$-$] The computer scientist: "Oh, no! A special case!"	
	
\end{itemize}
	\begin{center}\underline{\hspace{5 cm}}\end{center}	
	
A mathematician, a biologist and a physicist are sitting in a street cafe watching people going in and coming out of the house on the other side of the street. 

First they see two people going into the house. Time passes. After a while they notice three persons coming out of the house. 

\begin{itemize}	 
	\item[$-$] The physicist: "The measurement wasn't accurate."

	\item[$-$] The biologists: "They have reproduced".

	\item[$-$] The mathematician: "If now exactly one person enters the house then it will be empty again."
\end{itemize}
	\begin{center}\underline{\hspace{5 cm}}\end{center}

A mathematician and an engineer attend a lecture by a physicist. The topic concerns Kaluza-Klein theories involving physical processes that occur in spaces with dimensions of 9, 12 and even higher. The mathematician is sitting, clearly enjoying the lecture, while the engineer is frowning and looking generally confused and puzzled. By the end the engineer has a terrible headache. At the end, the mathematician comments about the wonderful lecture. 

\begin{itemize}	 
	\item[$-$] The engineer says: "How do you understand this stuff?"

	\item[$-$] Mathematician: "I just visualize the process."

	\item[$-$] Engineer: "How can you visualize something that occurs in 9-dimensional space?"

	\item[$-$] Mathematician: "Easy, first visualize it in N-dimensional space, then let N go to 9."
\end{itemize}
	\begin{center}\underline{\hspace{5 cm}}\end{center}

Two mathematicians are in a bar. The first one says to the second that the average person knows very little about basic mathematics. The second one disagrees, and claims that most people can cope with a reasonable amount of math. 

The first mathematician goes off to the wash-room, and in his absence the second calls over the waitress. He tells her that in a few minutes, after his friend has returned, he will call her over and ask her a question. All she has to do is answer "one third x cubed." 

She repeats "one thir -- dex cue"?

He repeats "one third x cubed".

She asks, "one thir dex cuebd?"

"Yes, that's right," he says.

So she agrees, and goes off mumbling to herself, "one thir dex cuebd...". 

The first guy returns and the second proposes a bet to prove his point, that most people do know something about basic math. He says he will ask the blonde waitress an integral, and the first laughingly agrees. The second man calls over the waitress and asks "what is the integral of x squared?".

The waitress says "one third x cubed" and while walking away, turns back and says over her shoulder "plus a constant!"
	\begin{center}\underline{\hspace{5 cm}}\end{center}

Teachers words:

\begin{itemize}	 
	\item[$-$] There are still pieces of the argument
	\item[$-$] The '-' sign in front of the potential disturbs you? And if I put a '+' do you feel better? Yes! Then put a '+' ....
	\item[$-$] This is a phase curve? No, this is a dog leg! 
	\item[$-$] Let's kill him and we will believe in suicide
	\item[$-$] You can say it's predictable because it is unpredictable!
	\item[$-$] I count on you to understand a little more thoroughly what will come next
	\item[$-$] We will now study this a little later
	\item[$-$] I will present you now some results dependent on Maxwell's equations that you have not seen yet ... anyway, at the point where we are ... 
	\item[$-$] If you do not understand, it's normal ... the contrary would be surprising indeed 
	\item[$-$] General relativity is useless! It is not with how we put satellites into orbit! 
	\item[$-$] Fifty cents question
	\item[$-$] Occasionally, we have to be absurd
	\item[$-$] When you have nothing to do, you take the Gauss theorem
	\item[$-$] By which virtue of the Holy Spirit would it become neutral?! 
	\item[$-$] You will understand once you grow up...
	\item[$-$] Take the example of a bank: there is a Mr. who makes a deposit and a Mrs. who makes withdrawal ... Finally, as usual what! 
	\item[$-$] ... more the parallelism will be parallel...
	\item[$-$] Remove me these residues from earlier calculations 
	\item[$-$] Delete the left side. I said LEFT! Where is your right? This is ... Well the left is the other side! 
	\item[$-$] Ok, who will be the next victim?
	\item[$-$] The '+' is recognized, and the electrons do not deceive themselves, for its beautiful red color! 
	\item[$-$] The goal is to verify that the test does not say shit 
	\item[$-$] If you can not do that, I assure you: it's all over for the exam
	\item[$-$] If you can not do that, do it!
	\item[$-$] You are very good at finding wrong things 
	\item[$-$] I like this kind of demonstration. You, it makes you have nightmares
	\item[$-$] I won't will resolve this, as this may offend your sensibility! 
	\item[$-$] You must be careful with all what the teachers love!
	\item[$-$] You add potatoes and pigs, it is dimensionless ...
	\item[$-$] You can find the derivative of the Dirac pulse, it is not that we will jump to the head
	\item[$-$] It's silly but Riemann, that's how
	\item[$-$] Look at the equation that I just deleted
	\item[$-$] Legally speaking, the current is in this sense
	\item[$-$] There is no question that I waste my time to solve you this low fly joke! 
	\item[$-$] If the girl has understood then you must have all understood
	\item[$-$] Oh yeah, you're laughing at me. In fact, you are right to make fun of us, because we do not hesitate to make fun of you 
	\item[$-$] Attention very important: I propose to dream at it on night
	\item[$-$] I have the feeling of playing the Stradivarius before cows
	\item[$-$] The hand of God spreads the field perpendicular to the surface
	\item[$-$] The miracle of the disappearance of the harmonic will not happen today
	\item[$-$] If you wake up at night, remember that sampling in the time domain is periodized in the frequency domain ... then you go back to sleep 
	\item[$-$] The left curves are not right
	\item[$-$] You see, sometimes my results are correct
	\item[$-$] We denote "Q" the output, it's obvious ...
	\item[$-$] We'll have a trick to successfully recover this variable
	\item[$-$] I am a 68000; Mr. Director comes in the corridor, knocking on the door: there is no better interruption! 
	\item[$-$] Warning: one, two, three, take your brains!
	\item[$-$] In the steppes of automatic, we reach the arid part of mathematics 
	\item[$-$] What interest? Well no. This is a figure of speech Teaching 
	\item[$-$] If you have some memories of the signal flow graph that once we plotted ...
	\item[$-$] The electrons in a hot metal, it's the New York main street at 5 in the evening, so it does not lead
	\item[$-$] The circles, is what has the least corners!
	\item[$-$] If you put your finger in a socket, it's not a complex number that comes
	\item[$-$] See the proof in your daily readings ... So I waiver
	\item[$-$] And then comes Carnot : he brings his mind that does not mean much 
	\item[$-$] Sometimes people who have a bit of culture that know that... it missed today! 
	\item[$-$] We will not call it "performance" because it can reach a value higher than 1, that might disturb some weak minds
	\item[$-$] Me, if I had been sent to the blackboard, I would not have written this 
	\item[$-$] You will find the answer on www.archimedes.com
	\item[$-$] I use a method that dates back several centuries ... as I do!
	\item[$-$] Ask me questions ... I would like to have questions! ... other issues? I'll have to take the list and say: "But you have a question"!
	\item[$-$] The ordering implemented depends on the instantiation of the garbage collector if it is synchronized to the preemptive multi-threading on the OS.
	\item[$-$] You are only boxes receiving inputs and spitting outputs
\end{itemize}
	\begin{center}\underline{\hspace{5 cm}}\end{center}

Sherlock Holmes and Dr. John Watson went on a camping trip. After sharing a good meal and a bottle of Petri wine, they retire to their tent for the night.

At about 3 AM, Holmes nudges Watson and asks, "Watson, look up into the sky and tell me what you see?"

Watson said, "I see millions of stars."

Holmes asks, "And, what does that tell you?"

Watson replies, "Astronomically, it tells me there are millions of galaxies and potentially billions of planets. Astrologically, it tells me that Saturn is in Leo. Theologically, it tells me that God is great and we are small and insignificant. Horologically, it tells me that it's about 3 AM. Meteorologically, it tells me that we will have a beautiful day tomorrow. What does it tell you, Holmes?"

Holmes retorts, "Someone stole our tent." 
	\begin{center}\underline{\hspace{5 cm}}\end{center}

An engineer, a mathematician, and a physicist are staying for the night in a hotel. Fortunately for this joke, a small fire breaks out in each room.

The physicist awakes, sees the fire, makes some careful observations, and on the back of the hotel's wine list does some quick calculations. Grabbing the fire extinguisher, he puts out the fire with one, short, well placed burst, and then crawls back into bed and goes back to sleep.

The engineer awakes, sees the fire, makes some careful observations, and on the back of the hotel's room service list (pizza menu) does some quick calculations. Grabbing the fire extinguisher (and adding a factor of safety of 5), he puts out the fire by hosing down the entire room several times over, and then crawls into his soggy bed and goes back to sleep.

The mathematician awakes, sees the fire, makes some careful observations, and on a blackboard installed in the room, does some quick calculations. Jubliant, he exclaims "A solution exists!", and crawls into his dry bed and goes back to sleep.
	\begin{center}\underline{\hspace{5 cm}}\end{center}

How do you know that the driver driving toward you is a physicist?

He has a red sticker on his bumper, saying: "If this sticker is blue, you are driving too fast."
	\begin{center}\underline{\hspace{5 cm}}\end{center}

A Princeton plasma physicist is at the beach when he discovers a ancient looking oil lantern sticking out of the sand. He rubs the sand off with a towel and a genie pops out. The genie offers to grant him one wish. The physicist retrieves a map of the world from his car an circles the Middle East and tells the genie, "I wish you to bring peace in this region".

After 10 long minutes of deliberation, the genie replies, "Gee, there are lots of problems there with Lebanon, Iraq, Israel, and all those other places. This is awfully embarrassing. I've never had to do this before, but I'm just going to have to ask you for another wish. This one is just too much for me".

Taken aback, the physicist thinks a bit and asks, "I wish that the Princeton tokamak would achieve scientific fusion energy break-even."

After another deliberation the genie asks, "Could I see that map again?"	
	\begin{center}\underline{\hspace{5 cm}}\end{center}

In the beginning, there were two species of apple trees: those whose apples fell, and those whose apples were rising. The apples that fell could reach the ground, germinate and so generating a new tree whose apples fell. But the rising apples never reached the ground, and species of apple trees with apples rising quickly disappeared because they could not reproduce. As they were not adapted, nature has thus eliminated. That's natural selection. If apples fall, it is thanks to natural selection!

But there are clever people who will object that the stones fall too. But they are not living beings, so they are not subject to natural selection, it is obvious. So the above explanation does not explain what happens to stones. In fact, for the stones, it is even simpler:

In the beginning, there were indeed two kinds of stones: those who fell, and who were rising. But those who rised up gone very far. That's why all the stones that remain on earth do the same thing: they fall.

	\pagebreak
	\section{Mathematics}

5 out of 4 people don't understand fractions...
	\begin{center}\underline{\hspace{5 cm}}\end{center}

Love is like $\pi$, natural, irrational, transcendent and very real.
	\begin{center}\underline{\hspace{5 cm}}\end{center}

A math teacher explains to a blonde the limits. He made it with the following exercise:
	\begin{gather*}
	\lim_{x \rightarrow 8} \dfrac{1}{x-8}=+\infty
	\end{gather*}
At the end of the exercise, he asked the blonde if she understood. "Oh yes sir I understood everything". Believing the answer only half, the teacher asked the following exercise:

Calculate:
	\begin{gather*}
	\lim_{x \rightarrow 5} \dfrac{1}{x-5}
	\end{gather*}
The blonde writted: 
	\begin{gather*}
	\lim_{x \rightarrow 8} \dfrac{1}{x-8}=+\infty \quad  \text{then} \quad \lim_{x \rightarrow 5} \dfrac{1}{x-5}= \rotatebox[origin=c]{90}{5}  
	\end{gather*}
	\begin{center}\underline{\hspace{5 cm}}\end{center}
	
	\begin{center}
		\includegraphics{img/humour/self_complementary_graph.jpg}	
	\end{center}
	
	\pagebreak
Evolution of the teaching of Mathematics (...)

\begin{itemize}	 
	\item[$-$] Education 1960: A farmer sells a bag of potatoes for 100\$. Its production costs amounted to 4/5 of the selling price. What is his profit?

	\item[$-$] Traditional Education 1970: A farmer sells a bag of potatoes for 100\$. Its production costs amounted to 4/5 of the selling price, that is to say 80\$. What is his profit?

	\item[$-$] Modern Education 1970: A farmer exchanges a set P of potatoes against a set M of money. The cardinality of the set M is 100, and each PFM element is 1\$. Draw 100 large dots elements of the set M. The set F of the production costs are 20 big points less than the set M. Represent the set F as a subset of the set M and give the answer to the question: What is the cardinality of the set B benefits? (draw it in red)

	\item[$-$] Renovated Education in 1980: A farmer sells a bag of potatoes for 100\$. Production costs amount to 80\$ and the benefit is 20\$. Homework: underline the words "Potatoes" and discussed this with your neighbour.

	\item[$-$] Start-up Education 1999: A wired producer of agricultural space consults a data bank which
display the day-rate of the potato. It load its reliable software computation and determines the cash flow on bit-map screen (under config WMil with 40GB HDD and floppy). Draw with your mouse the integrated 3D contour of the bag of potatoes. Then log yourself to the network by www.blue-potatoe.com and follow the instructions of the menu.

	\item[$-$] Education 2010: What is a farmer?

\end{itemize}
	\begin{center}\underline{\hspace{5 cm}}\end{center}
	\begin{center}
		\includegraphics[scale=0.9]{img/humour/homework.jpg}	
	\end{center}

	\begin{table}[H]
	\begin{center}
		\definecolor{gris}{gray}{0.85}
			\begin{tabular}{|p{7.5cm}|p{7.5cm}|}
				\hline
				\multicolumn{1}{c}{\cellcolor{black!30}\textbf{When you read or listen}} & 
  \multicolumn{1}{c}{\cellcolor{black!30}\textbf{What you have to understand}} \\ \hline
				this is trivial (or obvious) & I can not say why this is true \\ \hline
				we get automatically & idem \\ \hline
				a calculation shows that & a calculation that I did not will certainly show that\\ \hline
				the reader will easily control that & it bothers me to show that\\ \hline
				we strongly recommend the reader to make the indicated exercises & as I have not made them, you could correct me\\ \hline
				i showed this result in a previous paper & i have absolutely no idea how I did to prove that thing			
				\\ \hline
				is easily generalized to & the generalization is beyond my level			
				\\ \hline
				according to a well known property & known by maximum 10 people in the world
				\\ \hline
				the proof is in two lines & yes, but through five lemmas
				\\ \hline
				it's algebra & it is not interesting (in the mouth of an analyst)
				\\ \hline
				it's analysis & it is not interesting (in the mouth of an algebrist)
				\\ \hline
				it's elementary (or classical) & in bornitziens space theory bornitziens of the second kind
				\\ \hline
				i did not understand this step in your demonstration & you're stuck in your demo
				\\ \hline
				This conference was very interesting & i did not understand anything
				\\ \hline
		\end{tabular}
	\end{center}
	\end{table}	
	\begin{center}\underline{\hspace{5 cm}}\end{center}
	
	The number you requested is imaginary, please turn your phone to a quarter turn right and renumber...
	\begin{center}\underline{\hspace{5 cm}}\end{center}
	
	\begin{center}
		\includegraphics[scale=0.6]{img/humour/pizza.eps}	
	\end{center}
	\begin{center}\underline{\hspace{5 cm}}\end{center}
	
	A mathematician to his friend:

\begin{itemize}	 
	\item[$-$] "Are you faithful?"

	\item[$-$] "Yes, up to isomorphism"	
\end{itemize}
	\begin{center}\underline{\hspace{5 cm}}\end{center}
	
How mathematicians do it...

\begin{itemize}	 
	\item[$-$] Algebraists do it by symbolic manipulation.

	\item[$-$] Algebraists do it in a ring, in fields, in groups.

	\item[$-$] Analysts do it continuously and smoothly.

	\item[$-$] Applied mathematicians do it by computer simulation.

	\item[$-$] Banach spacers do it completely.

	\item[$-$] Bayesians do it with improper priors.

	\item[$-$] Catastrophe theorists do it falling off part of a sheet.

	\item[$-$] Combinatorists do it as many ways as they can.

	\item[$-$] Complex analysts do it between the sheets

	\item[$-$] Computer scientists do it depth-first.

	\item[$-$] Cosmologists do it in the first three minutes.

	\item[$-$] Decision theorists do it optimally.

	\item[$-$] Functional analysts do it with compact support.

	\item[$-$] Galois theorists do it in a field.

	\item[$-$] Game theorists do it by dominance or saddle points.

	\item[$-$] Geometers do it with involutions.

	\item[$-$] Geometers do it symmetrically.

	\item[$-$] Graph theorists do it in four colors.

	\item[$-$] Hilbert spacers do it orthogonally.

	\item[$-$] Large cardinals do it inaccessibly.

	\item[$-$] Linear programmers do it with nearest neighbors.

	\item[$-$] Logicians do it by choice, consistently and completely.

	\item[$-$] Logicians do it incompletely or inconsistently.

	\item[$-$] (Logicians do it) or [not (logicians do it)].

	\item[$-$] Number theorists do it perfectly and rationally.

	\item[$-$] Mathematical physicists understand the theory of how to do it, but have difficulty obtaining practical results.

	\item[$-$] Pure mathematicians do it rigorously.

	\item[$-$] Quantum physicists can either know how fast they do it, or where they do it, but not both.

	\item[$-$] Real analysts do it almost everywhere

	\item[$-$] Ring theorists do it non-commutatively.

	\item[$-$] Set theorists do it with cardinals.

	\item[$-$] Statisticians probably do it.

	\item[$-$] Topologists do it openly, in multiply connected domains

	\item[$-$]  Variationists do it locally and globally.

	\item[$-$] Cantor did it diagonally.

	\item[$-$] Fermat tried to do it in the margin, but couldn't fit it in.

	\item[$-$] Galois did it the night before.

	\item[$-$] Mðbius always does it on the same side.

	\item[$-$] Markov does it in chains.

	\item[$-$] Newton did it standing on the shoulders of giants.

	\item[$-$] Turing did it but couldn't decide if he'd finished.
 \end{itemize}
	\begin{center}\underline{\hspace{5 cm}}\end{center}
	 
What will a complex French guy said to a real woman?

Answer: "viens danser !" (you have to read "come in C", the complex set...)
	\begin{center}\underline{\hspace{5 cm}}\end{center}
	
	\begin{center}
		\includegraphics{img/humour/rotation_matrix.jpg}	
	\end{center}
	
	\begin{center}\underline{\hspace{5 cm}}\end{center}

Two-way sentences in French and Fnglish:

\begin{itemize}	 
	\item[$-$] We solve now this problem without complex

	\item[$-$] Un repère d'origine O (un repaire d'originaux...) 

	\item[$-$] Une partie de $\mathbb{Q}$ (a fucking party...)

	\item[$-$] Ne confondez pas un $\rho$ avec un $p$... (don't confuse between a fart and a burp)

	\item[$-$] Une variété de Poisson... (a variety of fish)
\end{itemize}
	\begin{center}\underline{\hspace{5 cm}}\end{center}

Logarithm and exponential functions are at the restaurant. When comes the addition, which will pay?

Answer: Exponential, because logarithme né paie rien (in English "neperien" phonetically means: pay nothing)

Later in the evening, Logarithm and Exponential go home a little bit drunked. Logarithm asks: Do I drive?

Exponential answers: I'd rather it be me who leads. In the case you derivate...
	\begin{center}\underline{\hspace{5 cm}}\end{center}

Two Cauchy sequences want to go out to dance. They arrive at a club where the "No Limit" evening event takes place. They decide to enter, but the guard stops them, saying, "Sorry, we're complete" (in a complete space, a Cauchy sequence is convergent by definition, so it has a limit).	
	\begin{center}\underline{\hspace{5 cm}}\end{center}

	\begin{center}
		\includegraphics{img/humour/socks.eps}	
	\end{center}
	\begin{center}\underline{\hspace{5 cm}}\end{center}	

A mathematician went fool and believed that he was the differentiation operator. His friends had placed him in a mental hospital until he got better. All day he would go around frightening the other patients by staring at them and saying: "I differentiate you!" 

One day he met a new patient; and true to form he stared at him and said "I differentiate you!", but for once, his victim's expression didn't change. Surprised, the mathematician marshalled his energies, stared fiercely at the new patient and said loudly "I differentiate you!", but still the other man had no reaction. Finally, in frustration, the mathematician screamed out "I DIFFERENTIATE YOU!"

The new patient calmly looked up and said: "You can differentiate me all you like: I'm $e$ to the $x$."
	\begin{center}\underline{\hspace{5 cm}}\end{center}	

What mathematicians say and what you have to understand:

\begin{itemize}	 
	\item[$-$] Trivial: If I have to show you this, you're in the wrong class

	\item[$-$] One can trivially show: We do not need more than 4 hours to write the proof

	\item[$-$] Check yourself: This is the hard part of the demonstration so you can do on your spare time

	\item[$-$] Similarly: At least one line of the proof is identical to the previous

	\item[$-$] Proceed formally: We will manipulate symbols with many predefined rules without understanding the real meaning of the result.

	\item[$-$] We will provide us the proof: Trust me, it's true!

	\item[$-$] The reader will easily show: I get tired to show that...

	\item[$-$] We strongly suggest the reader to make the indicated exercises: As I have not made them, you could correct me

	\item[$-$] I showed this result in a previous paper: I do not know how the devil did to prove this thing

	\item[$-$] We generalize easily: The generalization is beyond my level

	\item[$-$] According to a well known property: For 10 people in the world ...
\end{itemize}
\begin{center}\underline{\hspace{5 cm}}\end{center}
	
	\begin{center}
		\includegraphics[scale=0.9]{img/humour/fresh_men.jpg}	
	\end{center}
	
	\begin{center}\underline{\hspace{5 cm}}\end{center}	

There are 3 kinds of people: those who can count and those who can not count...
	\begin{center}\underline{\hspace{5 cm}}\end{center}	
	
Everyone knows the "Salary Theorem" which states that engineers and scientists can NEVER earn as much as businessmen and commercial. This theorem can then be demonstrated by solving a simple math equation.

Our equation is based on two well known postulates:

P1. The Knowledge is Power

P2. Time is Money:

All engineers know that:

\begin{center}
$\text{Power}=\dfrac{\text{Work}}{\text{Time}}$
\end{center}

and:

\begin{center}
$\text{Knowledge}=\text{Power}$
\end{center}

and also:

\begin{center}
$\text{Time}=\text{Money}$
\end{center}

We then obtain by substitution: 

\begin{center}
$\text{Knowledge}=\dfrac{\text{Work}}{\text{Money}}$
\end{center}

And we finally get the following result: 

\begin{center}
$\text{Money}=\dfrac{\text{Work}}{\text{Knowledge}}$
\end{center}

So when the Knowledge approaches zero, Money approaches infinity regardless of the value attributed to work, this value may be very low. Conversely when Knowledge goes to infinity, the Silver tends to zero, even if the work is high value.

Hence the obvious conclusion follows: The less you know, the more money you make.

PS: Those of you who have had some difficulty understanding this should be the highest paid.

	\begin{flushright}
		$\square$  Q.E.D.
	\end{flushright}
	\begin{center}\underline{\hspace{5 cm}}\end{center}
	
	\begin{center}
		\includegraphics{img/humour/proof_trivial.jpg}	
	\end{center}
	
	\begin{center}\underline{\hspace{5 cm}}\end{center}	
	
In the same kind here is the "misogynistic theorem":

First, we state that girls factorable variables in amounts of time and money such as:

\begin{center}
$\text{Girls}=\text{Time}\times\text{Time}$
\end{center}

As we all know "time is money". So:

\begin{center}
$\text{Time}=\text{Money}$
\end{center}

and because "money is the root of evil ...":

\begin{center}
$\text{Money}=\sqrt{\text{Evil}}$
\end{center}

Then we have by substitution:

\begin{center}
$\text{Girls}=\left(\sqrt{\text{Evil}}\right)^2$
\end{center}

We are therefore forced to conclude:

\begin{center}
$\text{Girls}=\text{Evil}$
\end{center}

	\begin{flushright}
		$\square$  Q.E.D.
	\end{flushright}
	
	\begin{center}\underline{\hspace{5 cm}}\end{center}
	\begin{center}
		\includegraphics{img/humour/professor_xi.jpg}	
	\end{center}
	
	The top ten excuses for not doing your math homework:
	
	\begin{itemize}
	
		\item[$\text{\#}10.$] Galileo didn't know calculus; what do I need it for?
	
		\item[$\text{\#}09.$] A math addict stole my homework.
	
		\item[$\text{\#}08.$] I'm taking physics and the homework in there seemed to involve math, so I thought I could just do that instead.
	
		\item[$\text{\#}07.$] I have the proof, but there isn't room to write it in the margin.
	
		\item[$\text{\#}06.$] I have a solar powered calculator and it was cloudy.
	
		\item[$\text{\#}05.$] I was watching the World Series and got tied up trying to prove that it converged.
	
		\item[$\text{\#}04.$] I could only get arbitrarily close to my textbook. (I reached half way, and then half of that, and then ...)
	
		\item[$\text{\#}03.$] I couldn't figure out whether i am the square root of negative one or i is the square root of negative one.
	
		\item[$\text{\#}02.$] It was Einstein's birthday and pi day and we had this big celebration! (This only works for March 14)
	
		\item[$\text{\#}01.$] I accidentally divided by zero and my paper burst into flames.	
	\end{itemize}

	\begin{center}\underline{\hspace{5 cm}}\end{center}
	\begin{center}
		\includegraphics{img/humour/close.jpg}	
	\end{center}
	\begin{center}\underline{\hspace{5 cm}}\end{center}
	
	\pagebreak
	What is the result of:
	\begin{center}
		 $\dfrac{2ab}{2Fr.16}$\\
		(read "2 abbés sur 2 françaises")
	\end{center} 

	Answer: 
	\begin{center}
		$2bb \dfrac{a}{e}$\\
		(read "2 bébés assurés")
	\end{center} 
	\begin{center}\underline{\hspace{5 cm}}\end{center}
	
	What is the result of: 
	\begin{center}
	 $\dfrac{\text{cheval}}{\text{oiseau}}$\\
	(read "horse on bird")
	 \end{center} 
	
	As we have: 
	\begin{center}
	 $\dfrac{\text{cheval}}{\text{oiseau}}=\dfrac{\text{vache} \cdot \text{l}}{\beta \cdot \text{l}}$\\
	(read "vache + l" (contains all letters of the word "cheval") divided by "bête à ailes" meaning animal with wing) 
	 \end{center} 
	
	But: 
	\begin{center}
	 $\dfrac{\text{vache} \cdot \text{l}}{\beta \cdot \text{l}}=\dfrac{\beta \cdot \pi \cdot \text{l}}{\beta \cdot \text{l}}$\\
	(read "bête à pie + l" divided by "bête à ailes")  
	 \end{center}
	
	We simplify to get: 
	\begin{center}
	 $\dfrac{\cancel{\beta} \cdot \pi \cdot \cancel{\text{l}}}{\cancel{\beta} \cdot \cancel{\text{l}}}=\pi$  
	 \end{center}
	
	This prove that $\pi$ is irrational because there is no rational comparison between a "cheval" (horse) and an "oiseau" (bird)... 
		\begin{center}\underline{\hspace{5 cm}}\end{center}
	\pagebreak
	We have to prove that:
	\begin{center}
	$\dfrac{\text{ROSSINI}}{\text{SOLSIDO}}=1$  
	\end{center} 
	
	We can write this as following:
	\begin{center}
	$\dfrac{\text{ROS SI NI}}{\text{SOL SI DO}}=\dfrac{\text{ROS NI}}{\text{SOL DO}}$  
	\end{center}
	
	but "NI vaut Do" (in French this means "niveau d'eau" or in English "water level") thus:
	\begin{center}
	$\dfrac{\text{ROS}}{\text{SOL}}$  
	\end{center}
	
	But "SOL fait RINO" (Solferino: it's during this battle that Henri Dunant had the idea to create the Red Cross) then: 
	\begin{center}
	$\dfrac{\text{ROS}}{\text{RI NO}}$  
	\end{center}
	
	because "RINO c'est ROS" (rhinocéros) then RINO = ROS and we finally have: 
	\begin{center}
	$\dfrac{\text{ROS}}{\text{ROS}}=1$  
	\end{center}

	\begin{flushright}
		$\square$  Q.E.D.
	\end{flushright}	

	\begin{center}\underline{\hspace{5 cm}}\end{center}
	\begin{center}
		\includegraphics[scale=0.6]{img/humour/day_of_an_eigenvector.jpg}	
	\end{center}
	Each future engineer learns to write the sum of two rational numbers, for example:
	\begin{center}
	$1+1=2$  
	\end{center}
	
	This form is however rather banal and indicates gaps in your education.
	
	In the first semester, we learn that:
	\begin{center}
	$1=\ln(e)$  
	\end{center}
	
	and:
	\begin{center}
	$1=\sin^2(p)+\cos^2(q)$  
	\end{center}
	
	Also everybody know that:
	\begin{gather*}
	2=\sum_{n=0}^{+\infty} \left( \dfrac{1}{2} \right)^n
	\end{gather*}
	
	and that therefore the equation:
	\begin{gather*}
	1+1=2
	\end{gather*}
	
	can be written more simply:
	\begin{gather*}
	\ln(e)+\sin^2(p)+\cos^2(q)=\sum_{n=0}^{+\infty} \left( \dfrac{1}{2} \right)^n
	\end{gather*}
	
	we must admit that the look is much clearer and more scientific...
	
	On the other hand, it is clear that:
	\begin{gather*}
	1=\cosh(q)\sqrt{1-\tanh^2(q)}
	\end{gather*}
	
	and also:
	\begin{gather*}
	e=\lim_{z \rightarrow +\infty}\left(1+\dfrac{1}{z} \right) 
	\end{gather*}
	
	it follows that:
	\begin{gather*}
	\ln(e)+\sin^2(p)+\cos^2(q)=\sum_{n=0}^{+\infty} \left( \dfrac{1}{2} \right)^n
	\end{gather*}
	
	can be rewritten as follows:
	\begin{gather*}
	\ln\left( \lim_{z \rightarrow +\infty}\left(1+\dfrac{1}{z} \right)\right)+\sin^2(p)+\cos^2(q)=\sum_{n=0}^{\infty} \left( \dfrac{\cosh(q)\sqrt{1-\tanh^2(q)}}{2} \right)^n
	\end{gather*}
	
	We must also remember that:
	\begin{gather*}
	0!=1
	\end{gather*}
	
	and the inverse exponent of the exponent is opposite equal to the exponent of the exponent opposite. Assuming an $n$-dimensional space, we know that:
	\begin{gather*}
	\left( X^T\right) ^{-1}-\left( X^{-1}\right) ^{-T}=0
	\end{gather*}
	
	Taking the matrix as the metric of an oriented and orthogonal canonical space:
	\begin{gather*}
	\left( g_{ij}^T\right) ^{-1}-\left( g_{ij}^{-1}\right) ^{-T}=0
	\end{gather*}
	
	logically we obtain:
	\begin{gather*}
	\left(\left( g_{ij}^T\right) ^{-1}-\left( g_{ij}^{-1}\right) ^{-T}\right)!=1
	\end{gather*}
	
	we obtain a simple and clear expression of $1+1=2$ for everyone:
	\begin{gather*}
	\ln\left( \lim_{z \rightarrow +\infty}\left(\left(\left( g_{ij}^T\right) ^{-1}-\left( g_{ij}^{-1}\right) ^{-T}\right)!+\dfrac{1}{z} \right)\right)+\sin^2(p)+\cos^2(q)=\sum_{n=0}^{+\infty} \left( \dfrac{\cosh(q)\sqrt{1-\tanh^2(q)}}{2} \right)^n
	\end{gather*}
	
	It is therefore obvious that this equation is much more understandable than:
	\begin{gather*}
	1+1=2
	\end{gather*}
	
	It would be possible to show several other developments of this simple expression and we will do from the moment you begin to understand the simple principles of the previous method.
	\begin{center}\underline{\hspace{5 cm}}\end{center}
	\begin{center}
		\includegraphics{img/humour/coloring_problem.jpg}	
	\end{center}

	A wrestler, a physicist and a mathematician are subject to an experience: they are locked in a room each with a box of spinach, closed, and no can opener. After 24 hours, we'll see what they have become.
	
	\begin{itemize}	 
		\item[$-$] The wrestler was able to open her box, "Well, I just flung violently the box against the wall. The impact was such that it is open", he explains.
	
		\item[$-$] The physicist also managed to open her box: "I watched the box, and distinguished his break points. I then performed a pressure to exert maximum force on them, and the box was naturally open."
	
		\item[$-$] The mathematician, finally, is found prostrate in a corner of the room, the sweat streaming down his face, and his box, closed, between the feet: "We admit that the box is opened ... We admit that... ".
	\end{itemize}
	\begin{center}\underline{\hspace{5 cm}}\end{center}
	\begin{center}
		\includegraphics[scale=0.9]{img/humour/math_useful.jpg}	
	\end{center}

	\begin{center}\underline{\hspace{5 cm}}\end{center}
	Mathematics of life:
	\begin{gather}
		\setlength{\tabcolsep}{1pt}
		\begin{tabular}{cccccc}
		& & \text{Life} & + & \text{Love}= & \text{Happy} \\
		$+$& & \text{Life} & - & \text{Love}= & \text{Sad} \\ \hline
		&2\text{Life}& & & = & \text{Happy}+\text{Sad} \\
		\end{tabular}
	\end{gather}
	Therefore:
	\begin{gather}
		\text{Life}=\dfrac{\text{Happy}+\text{Sad}}{2}
	\end{gather}
	Developping:
	\begin{gather}
		\text{Life}=\dfrac{1}{2}\text{Happy}+\dfrac{1}{2}\text{Sad}
	\end{gather}
	That's real Life. Enjoy it!
	
	\begin{center}\underline{\hspace{5 cm}}\end{center}
	An opinion without $3.14$ is an onion. You'll understand!
	
	\begin{center}\underline{\hspace{5 cm}}\end{center}

	\begin{center}
		\includegraphics[scale=0.4]{img/humour/math_man_sex.jpg}	
	\end{center}

	\pagebreak
	\section{Physics}
	
	\begin{center}
	\includegraphics{img/humour/heisenberg.eps}
	\end{center}
	
	\begin{center}\underline{\hspace{5 cm}}\end{center}	
	
	\begin{itemize}	 
		\item[$-$] In theory, there is no difference between theory and practice. In practice, there is a difference!
	
		\item[$-$] The theory is when we know everything but nothing works. The practice is when everything works, but we do not know why. In computer science, theory and practice are met: nothing works and you do not know why!
	\end{itemize}

	\begin{center}\underline{\hspace{5 cm}}\end{center}

	Matter is fundamentally lazy - It always takes the path of least effort
	
	Matter is fundamentally stupid - It tries every other path first.
	
	That is the heart of physics - The rest is details.
	
	\begin{center}\underline{\hspace{5 cm}}\end{center}
	
	Two atoms meets together. 
	
	One says to the other: "Shit, I lost an electron!"
	
	The other: "Are you sure?"
	
	And the first replies, "POSITIVELY!!"
	
	\begin{center}\underline{\hspace{5 cm}}\end{center}

	\begin{center}
	\includegraphics{img/humour/einstein.eps}
	\end{center}
	
	\begin{center}\underline{\hspace{5 cm}}\end{center}	
	
	You enter the laboratory and see an experiment. How will you know which class is it?
	
	\begin{itemize}	 
		\item[$-$] If it's green and wiggles, it's biology.
	
		\item[$-$] If it stinks, it's chemistry.
	
		\item[$-$] If it doesn't work, it's physics.
	\end{itemize}
	
	\begin{center}\underline{\hspace{5 cm}}\end{center}	

	Theorem: A cat has nine tails.
	
	Proof: No cat has eight tails. A cat has one tail more than no cat. Therefore, a cat has nine tails.
	
	\begin{center}\underline{\hspace{5 cm}}\end{center}

	\begin{center}
	\includegraphics[scale=0.75]{img/humour/schrodinger_cat.eps}
	\end{center}
	
	\begin{center}\underline{\hspace{5 cm}}\end{center}

	A physicist studying quantum physics, is someone who does not see very well, looking in a dark room for a black cat, who probably does not exist.
	
	\begin{center}\underline{\hspace{5 cm}}\end{center}

	An engineer, a physicist, a mathematician, and a mystic were asked to name the greatest invention of all times. 

\begin{itemize}	 
	\item[$-$] The engineer chose fire, which gave humanity power over matter.

	\item[$-$] The physicist chose the wheel, which gave humanity the power over space.

	\item[$-$] The mathematician chose the alphabet, which gave humanity power over symbols.

	\item[$-$] The mystic chose the thermos bottle.

"Why a thermos bottle?" the others asked.

	\item[$-$] The mystic: "Yes, because the thermos keeps hot liquids hot in winter and cold liquids cold in summer."

	\item[$-$] "Yes... so what?" the others asked.

	\item[$-$] The mystic: "Think about it." said the mystic reverently. "That little bottle.. how does it know?"
	\end{itemize}
\begin{center}\underline{\hspace{5 cm}}\end{center}

	\begin{center}
	\includegraphics[scale=0.4]{img/humour/milkiway.jpg}
	\end{center}
\begin{center}\underline{\hspace{5 cm}}\end{center}

Physics professor has been doing an experiment, and has worked out an empirical equation that seems to explain his data. He asks the math professor to look at it. 

A week later, the math professor says the equation is invalid. By then, the physics professor has used his equation to predict the results of further experiments, and he is getting excellent results, so he asks the math professor to look again. 

Another week goes by, and they meet once more. The math professor tells the physics professor the equation does work, "But only in the trivial case where the numbers are real and positive".
\begin{center}\underline{\hspace{5 cm}}\end{center}

Practical nuclear fusion power plants are juste 30 years away - and always will be.
	
	\pagebreak
	\begin{center}
	\includegraphics{img/humour/howscientistseeworld.eps}
	\end{center}
	
	\pagebreak
	
	Heisenberg was driving down the highway whereupon he was pulled over by a policeman. 

	The policeman asked:
	
	\begin{itemize}	 
		\item[$-$] "Do you know how fast you were going back there?"
	\end{itemize}
	
	Heisenberg replied: 
	
	\begin{itemize}	 
		\item[$-$] "No, but I know where I am."
	\end{itemize}
	\begin{center}\underline{\hspace{5 cm}}\end{center}
	
	What's the difference between an auto mechanic and a quantum mechanic?
	
	The quantum mechanic can sometimes get the car inside the garage without opening the door.
	
	\begin{center}\underline{\hspace{5 cm}}\end{center}
	
	\begin{center}
	It's not the:
	
	\includegraphics{img/humour/kill_fall.jpg}
	\end{center}
	\begin{gather*}
		v_f=v_0+at
	\end{gather*}
	\begin{center}
	that kills you, it's the:
	\end{center}
	\begin{gather*}
		F=m\dfrac{\Delta v}{\Delta t}
	\end{gather*}
	\begin{center}
	\includegraphics{img/humour/kill_final.jpg}
	\end{center}
	

	\begin{center}
	\includegraphics{img/humour/superstring.eps}
	\end{center}
\begin{center}\underline{\hspace{5 cm}}\end{center}
	
Why God Never Received Tenure at any University:
\begin{enumerate}
	\item He had only one major publication.

	\item It was in Hebrew. 

	\item It had no references. 

	\item It wasn't published in a refereed journal.

	\item Some even doubt he wrote it himself. 

	\item It may be true that he created the world, but what has he done since then? 

	\item His cooperative efforts have been quite limited. 

	\item The scientific community has had a hard time replicating his results. 

	\item He never applied to the Ethics Board for permission to use human subjects.

	\item When one experiment went awry he tried to cover it up by drowning the subjects. 

	\item When subjects didn't behave as predicted, he deleted them from the sample . 

	\item He rarely came to class, just told students to read the Book. 

	\item Some say he had his son teach the class. 

	\item He expelled his first two students for learning. 

	\item Although there were only ten requirements, most students failed his tests. 

	\item His office hours were infrequent and usually held on a mountaintop.
\end{enumerate}

\begin{center}\underline{\hspace{5 cm}}\end{center}

	\begin{center}
		\includegraphics{img/humour/physics_gang_sign.jpg}
	\end{center}
	\pagebreak

\begin{center}\underline{\hspace{5 cm}}\end{center}

In the beginning there was Aristotle:
\begin{itemize}
	\item And objects at rest tended to remain at rest
	\item And objects in motion tended to come to rest
	\item And God saw that it was boring, although very restful.
\end{itemize}

Then God created Newton:
\begin{itemize}
	\item And objects at rest tended to remain at rest
	\item And objects in motion tended to remain in motion
	\item And energy was conserved, and momentum was conserved,
	\item And matter was conserved
	\item And God saw that it was conservative.
\end{itemize}

Then God created Einstein:
\begin{itemize}
	\item And everything was relative
	\item And fast things became short
	\item And straight things became curved
	\item And the universe was filled with inertial frames
	\item And God saw that it was relatively general
but some of it was especially relative.
\end{itemize}

Then God created Bohr:
\begin{itemize}
	\item And there was the principle
	\item And the principle was quantum
	\item And all things were quantified
	\item But some things were still relative
	\item And God saw that it was confusing.
\end{itemize}

Then God was going to create Furgeson:
\begin{itemize}
	\item And Furgeson would have unified
	\item And he would have fielded a theory
	\item And all would have been one.
	\item But it was the seventh day
	\item And God rested
	\item And objects at rest tend to remain at rest.
\end{itemize}
\begin{center}\underline{\hspace{5 cm}}\end{center}

	\begin{center}
		\includegraphics[scale=0.6]{img/humour/schrodinger_survey.jpg}	
	\end{center}

\begin{center}\underline{\hspace{5 cm}}\end{center}
The Physicist's Bill of Rights

We hold these postulates to be intuitively obvious, that all physicists are born equal, to a first approximation, and are endowed by their creator with certain discrete privileges, among them a mean rest life, n degrees of freedom, and the following rights which are invariant under all linear transformations:
\begin{enumerate}
	\item To approximate all problems to ideal cases.

	\item To use order of magnitude calculations whenever deemed necessary (i.e. whenever one can get away with it).

	\item To use the rigorous method of "squinting" for solving problems more complex than the addition of positive real integers.

	\item To dismiss all functions which diverge as "nasty" and "unphysical".

	\item To invoke the uncertainty principle when confronted by confused mathematicians, chemists, engineers, psychologists, dramatists, and other lower scientists.

	\item hen pressed by non-physicists for an explanation of (4) to mumble in a sneering tone of voice something about physically naive mathematicians.

	\item To equate two sides of an equation which are dimensionally inconsistent, with a suitable comment to the effect of, "Well, we are interested in the order of magnitude anyway".

	\item To the extensive use of "bastard notations" where conventional mathematics will not work.

	\item To invent fictitious forces to delude the general public. 

	\item To justify shaky reasoning on the basis that it gives the right answer.

	\item To cleverly choose convenient initial conditions, using the principle of general triviality.

	\item To use plausible arguments in place of proofs, and thenceforth refer to these arguments as proofs.

	\item To take on faith any principle which seems right but cannot be proved.
\end{enumerate}

	\begin{center}\underline{\hspace{5 cm}}\end{center}
		\begin{center}
		\includegraphics[scale=0.6]{img/humour/travelling_light.jpg}	
	\end{center}

\pagebreak
You can at least take this simple "Real Scientist Quiz" to find out if you're cut out for the life of a true scientist:
\begin{enumerate}
	\item At Christmas time, you: 

a. Take a couple of days off to spend time with your family.
b. Leave early on Christmas eve so you can pick up a few presents for the family. 
c. Only work half a day, spending the rest of the day at home working on your grant application. 

	\item Your spouse wants to discuss plans for the family vacation with your kids. You: 

a. Propose to go camping so you can explain the wonders of nature to your kids 
b. Propose to go to another city so you can spend the day in your friend's lab while your spouse takes the kids sightseeing.
c.	Ask your spouse, "We have kids?"

	\item At a scientific meeting on an island in the South Pacific, no talks are scheduled in the afternoon. During this free time, you: 

a. Follow the local custom and sunbathe on the beach in the nude.
b. Sit on the beach fully clothed, unaware of the nude sunbathers, and discuss science with your colleagues.
c. Sit in your hotel room with the drapes closed, and work on your manuscript.

	\item The nurse at school calls to tell you that your second-grade child has chicken pox. You: 

a. Immediately drop what you're doing and rush to school to pick up your sick kid. 
b. Immediately drop what you're doing and begin trying to find a cure for chicken pox.
c. Ask the nurse for directions to the school, and the names of your kids.

	\item Beings from outer space visit Earth, and you are the first human they meet. To show their friendship, they present you with a highly advanced device that is capable of prolonging life, ending human suffering, and curing disease. You: 

a. Present it to the United Nations.
b.	Apply for a patent.
c. Break it open to see how it works.

	\item What is the longest amount of time that you have worked without a vacation (excluding scientific meetings)? 

a.	Six months.
b.	Two years.
c. I took a weekend off about 10 years ago.

	\item What are your hobbies? 

a. Sports, music, and dance, because they allow the analytical parts of my brain to relax. 
b. Cooking, because it's quite a lot like science.
c. Reading back issues of scientific journals cover to cover. 

	\item Your best friend is: 

a. A member of your college fraternity.
b. A member of your immediate family.
c. A member of a gene family. 
\end{enumerate}

Score:

Give yourself 1 point for every question you answered with an "a", 5 points for every "b" and 50 points for every "c". If you took the test three times and averaged your score, give yourself 100 extra points. If you calculated the standard error of the mean, give yourself 500 points. 

If you scored less than 10, you are normal. Scores of 11-50 indicate an obsessed scientist. If you scored more than 50, you are in need of help and should consider joining Scientists Anonymous; if you scored greater than 500, you should forget Scientists Anonymous and get back to work since you are beyond help, and may actually succeed as a scientist.
\begin{center}\underline{\hspace{5 cm}}\end{center}

	\begin{center}
	\includegraphics{img/humour/cow.jpg}
	\end{center}
	
\pagebreak

How you must understand certain sentences in physicists publications of:

\begin{itemize}
	\item It is well know that...: I did not read the references, but..

	\item This is of great theoretical importance: This is important for me.

	\item Though it was not possible to give a definitive answer: The experiment failed, but it seems me valuable enough to write a publication.

	\item The used technique was particularly appropriate ...: The lab next door friend had already developed the technique.

	\item 3 samples were chosen for an exhaustive study: The results obtained from other samples yielded nothing coherent.

	\item Handled with extreme caution throughout the experiment: Was not thrown down the trash.

	\item The agreement with theory is excellent: it is passable.

	\item The agreement with theory is good: it is weak.

	\item The agreement with theory is satisfactory: it is doubtful.

	\item The agreement with theory is fair: it is totally imaginary.

	\item It is generally accepted that...: two colleagues agree with me

	\item It is recognized that: I think.

	\item It is clear that further work will be useful: I did not understand anything.

	\item Here are some typical results: Here are the best results.

	\item Significant in a confidence interval of...: not significant.

	\item The reagents used were synthesized in the laboratory according to standardized techniques: The reagents were purchased from...

	\item Unfortunately, quantitative basis to exploit the results have not yet been made: Nobody was able to understand anything of what was observed.

	\item We are grateful to X for his valuable collaboration and Y for the fruitful discussions: X and Y did the work and told me what the results meant.
\end{itemize}

	\begin{center}\underline{\hspace{5 cm}}\end{center}

	\begin{center}
	\includegraphics{img/humour/duality.eps}
	\end{center}
	
	\begin{center}\underline{\hspace{5 cm}}\end{center}
	
	\begin{center}
	\includegraphics{img/humour/three_body_problem.jpg}
	\end{center}
	
	\begin{center}\underline{\hspace{5 cm}}\end{center}
	
	\begin{center}
	\includegraphics[scale=0.85]{img/humour/bus_stop_physicists.jpg}
	\end{center}
	
	\begin{center}\underline{\hspace{5 cm}}\end{center}
	
	\begin{center}
		This is how physicists see the Pokemon:
		\includegraphics[scale=0.8]{img/humour/pokemon.jpg}
	\end{center}
	
	\begin{center}
	\includegraphics[scale=0.55]{img/humour/feynman_diagrams.jpg}
	\end{center}
	
	\begin{center}\underline{\hspace{5 cm}}\end{center}
	
	\begin{center}
	\includegraphics[scale=0.55]{img/humour/cat_physics.jpg}
	\end{center}
	

	\pagebreak
	\section{Statistics}

Three statisticians go out for a static target shooting. The first statistician fired and shoots to the left, the second shot, but symmetrically on the right. The last does not shoot, but say triumphantly: "On average we got it!"
\begin{center}\underline{\hspace{5 cm}}\end{center}

Patient: "Will I survive to this delicate operation?"

Surgeon: "Yes, I am absolutely sure that you will survive."

Patient: "How can you be so sure?"

Surgeon: "9 out of 10 patients die during this operation and my ninth patient died yesterday."
\begin{center}\underline{\hspace{5 cm}}\end{center}
	\begin{center}
	\includegraphics{img/humour/gauss.eps}
	\end{center}

\pagebreak
10 reasons to work in statistics field:
\begin{enumerate}
	\item Estimate parameters is easier than fighting in real life

	\item Statisticians are recognized people

	\item You will learn the Greek alphabet entirely

	\item The probability that you get a job in this area is> 0.9999

	\item If you are fired, you can always convert yourself to engineering

	\item You make this work in the confidence, with regularity and variability

	\item You are normal and the rest of the world is wrong

	\item The regression line seems better than the unemployment line

	\item You never have to be exact - only approximate

	\item Nobody understands what you do, then you are always right
\end{enumerate} 

	\begin{center}\underline{\hspace{5 cm}}\end{center}
	\begin{center}
	\includegraphics{img/humour/statistician.eps}
	\end{center}
		
	\begin{center}
	\includegraphics[scale=0.9]{img/humour/bayesian_inference.jpg}
	\end{center}
	
	\begin{center}\underline{\hspace{5 cm}}\end{center}
	\begin{center}
	\includegraphics{img/humour/bedtime_stories.jpg}
	\end{center}
		
	\pagebreak
	\section{Chemistry}

We've just discovered a new element:

\begin{itemize}
	\item[$\bullet$] ELEMENT NUMBER: 115

	\item[$\bullet$]NAME: Woman

	\item[$\bullet$] SYMBOL: Wo

	\item[$\bullet$] ATOMIC MASS: Accepted as 60 kg; isotopes may vary from 40-200 kg.

	\item[$\bullet$] OCCURRENCE: Copious quantities in all urban areas

	\item[$\bullet$] PHYSICAL PROPERTIES:

- Boils at room temperature

- Freezes without any known reason

- Melts if given special treatment

- Bitter, if incorrectly used

- Sweet as Honey if given a proper treatment.

	\item[$\bullet$] MOLECULAR STRUCTURE:

Perfect? 90/60/90, growing int the U.S. with 60/90/120 and in the nordic countries as so-called "flat" 50/50/50

	\item[$\bullet$] CHEMICAL PROPERTIES:

- Has great affinity for Gold, Silver and a range of precious stones.Absorbs in general great quantities of expensive substances.

- May explode spontaneously without prior warning and for no known reason.

- Insoluble in liquids but activity greatly increased by saturation in alcohol

- Reactivity varies depending on the time of the day

- Great ability to change mood and jealousy

- Sensitive to certain constraints which sometimes transmit migraine

	\item[$\bullet$] COMMON USES: 

- Highly ornamental, especially in sports cars

- Powerful cleaning agent

- Can be great aid to relaxation.

	\item[$\bullet$] TEST: 

- Pure specimen turns rosy pink when happy

- Turns green when placed behind a better specimen

	\item[$\bullet$] PRECAUTIONS:

- Highly dangerous if placed between the non-expert hands

- Illegal to possess more than one, although several can be maintained at different locations as long as specimens do not come in direct contact with each other.

\end{itemize}

CAUTION: 

Some South American researchers have discovered a way to produce them artificially, usually presented under the marks "transvestite" or "Drag-queen". Consume only the generic!
	\begin{center}\underline{\hspace{5 cm}}\end{center}

	\begin{center}
	\includegraphics[scale=0.5]{img/humour/thorium.jpg}
	\end{center}
	
	\begin{center}\underline{\hspace{5 cm}}\end{center}

We've just discovered a new element:

\begin{itemize}
	\item[$\bullet$] ELEMENT NUMBER: 116

	\item[$\bullet$] NAME: Man

	\item[$\bullet$] SYMBOLE: Hm

	\item[$\bullet$] QUANTITATIVE ANALYSIS:	

Measured at 17 cm, although some isotopes exist in 25, 20, 13 and even 10 cm

	\item[$\bullet$] EXTRACTION LOCATION:	

Cand befound in large quantities in the presence of a deposit of very pure Fm

	\item[$\bullet$] PHYSICAL PROPERTIES:

- Surface covered with hair, steep in places, soft in others

- Boil when it is shaken, will ice when placed in the presence of logic and common sense, liquefies when treated like a god

- Becomes execrable when mixed with any alcohol

- Can cause headaches (other body parts pains); handle with care

- Decreases its entropy directly after its reaction with the element Fm (condition manifested by snoring ... zzzzz)

- Its mass increases significantly with age, losing its reaction capacity

- Dehydrates quickly in dry weather.

- Rarely found in pure form after 14 years old

- Often has an inexplicable attachment to its mother rock, making extraction difficult

- If you put it under pressure it becomes too hard and unproductive; is productive only when used subtlety, with subterfuge and flattery

	\item[$\bullet$] CHEMICAL PROPERTIES:

- Very strong tendency to react with the Fm element, although the reaction is sometimes endothermic

- Deemed to be the best catalyst for the transformation reactions of the Fm element

- Has the ability to react with almost anything

- If the case of an important reaction the aspect of the element changes to dark red

- If it is saturated with alcohol, it becomes inert and repulsive for most elements

- Not suitable for household chores and cleaning operations 

- Not suitable either for family duties

- Is neutral with respect to the courtesy and impartiality 

	\item[$\bullet$] COMMON USES:

- Transporting heavy things, driver, free dinners at the restaurant...

- Possible use for sexual activity

	\item[$\bullet$] TESTS:

The purest specimens are not synonymous with purity, and those who have already served, are less pure

	\item[$\bullet$] HAZARDS:

The reaction with another element Hm is extremely violent if the item Fm is the catalyst
\end{itemize}
\begin{center}\underline{\hspace{5 cm}}\end{center}

The following is an actual question given on a University of Washington engineering mid term. The answer was so profound that the Professor shared it with colleagues, which is why we now have the pleasure of enjoying it as well.

Bonus Question: Is Hell exothermic (gives off heat) or Endothermic (absorbs heat)?

Most of the students wrote proofs of their beliefs using Boyle's Law, (gas cools off when it expands and heats up when it is compressed) or some variant. One student, however, wrote the following:

"First, we need to know how the mass of Hell is changing in time. So we need to know the rate that souls are moving into Hell and the rate they are leaving. I think that we can safely assume that once a soul gets to Hell, it will not leave. Therefore, no souls are leaving. As for how many souls are entering Hell, let's look at the different religions that exist in the world today. Some of these religions state that if you are not a member of their religion, you will go to Hell. Since there are more than one of these religions and since people do not belong to more than one religion, we can project that all souls go to Hell. With birth and death rates as they are, we can expect the number of souls in Hell to increase exponentially.

Now, we look at the rate of change of the volume in Hell because Boyle's Law states that in order for the temperature and pressure in Hell to stay the same, the volume of Hell has to expand as souls are added. This gives two possibilities:

\begin{enumerate}
	\item If Hell is expanding at a slower rate than the rate at which souls enter Hell, then the temperature and pressure in Hell will increase until all Hell breaks loose.

	\item Of course, if Hell is expanding at a rate faster than the increase of souls in Hell, then the temperature and pressure will drop until Hell freezes over.
\end{enumerate}

So which is it? If we accept the postulate given to me by Teresa Banyan during my Freshman year, "...that it will be a cold day in Hell before I sleep with you." and take into account the fact that I still have not succeeded in having sexual relations with her, then, \#2 cannot be true, and thus I am sure that Hell is exothermic and will not freeze."

This student received the only A.
\begin{center}\underline{\hspace{5 cm}}\end{center}

A chemist walks into a pharmacy and asks the pharmacist: "Do you have any acetylsalicylic acid?"

\begin{itemize}
	\item[$-$] "You mean aspirin?" asked the pharmacist.

	\item[$-$] "That's it, I can never remember that word."
\end{itemize}
\begin{center}\underline{\hspace{5 cm}}\end{center}

A physicist, biologist and a chemist were going to the ocean for the first time. 

\begin{itemize}
	\item The physicist saw the ocean and was fascinated by the waves. He said he wanted to do some research on the fluid dynamics of the waves and walked into the ocean. Obviously he was drowned and never returned. 

	\item The biologist said he wanted to do research on the flora and fauna inside the ocean and walked inside the ocean. He too, never returned. 

	\item The chemist waited for a long time and afterwards, wrote the observation: "The physicist and the biologist are soluble in ocean water".
\end{itemize}
\begin{center}\underline{\hspace{5 cm}}\end{center}

CLASSIFICATION OF CHEMISTRY 

\begin{itemize}
	\item \textit{Physical Chemistry}: The pitiful attempt to apply $y = mx+b$ to everything in the universe.

	\item \textit{Organic Chemistry}: The practice of transmuting vile substances into publications.

	\item \textit{Inorganic Chemistry}: That which is left over after the organic, analytical, and physical chemists get through picking over the periodic table.

	\item \textit{Chemical Engineering}: The practice of doing for a profit what an organic chemist only does for fun.
\end{itemize}
\begin{center}\underline{\hspace{5 cm}}\end{center}

\begin{center}
\includegraphics[scale=0.7]{img/humour/cute.jpg}
\end{center}

\begin{center}\underline{\hspace{5 cm}}\end{center}

Free radicals have revolutionized chemistry.
\begin{center}\underline{\hspace{5 cm}}\end{center}

Chemist's last words: 

\begin{itemize}
	\item And now the tasting test... 

	\item And now shake it a bit... 

	\item In which glass was my mineral water? 

	\item Why does that stuff burn with a green flame?!? 

	\item And now the detonating gas problem. 

	\item This is a completely safe experimental setup. 

	\item Now you can take the protection window away... 

	\item Where do all those holes in my kettle come from? 

	\item And now a cigarette... 
\end{itemize}

	\pagebreak
	\section{Engineering}

	Scientists at NASA built a device to launch dead chickens at the windshields of airliners, military jets, the space shuttle, etc. The idea being to simulate collisions with airborne fowl to test the strength of the windshields. 
	
	British engineers heard about the device and were eager to test it on the windshields of their new high speed trains. Arrangements were made and a device was sent to the British engineers.
	
	When device was fired, the British engineers were shocked... the chicken hurled out of the barrel, crashed into the shatterproof shield, smashed it to smithereens, blasted through the control console, snapped the engineer's back-rest in two and embedded itself in the back wall of the cabin.
	
	The horrified Brits sent NASA the disastrous results of the experiment, along with the designs of the windshield and begged the US scientists for suggestions.
	
	NASA responded with a one-line memo: "Defrost the chicken."

	\begin{center}\underline{\hspace{5 cm}}\end{center}
	
	\begin{center}
	\includegraphics[scale=0.3]{img/humour/great_power_great_bills.jpg}
	\end{center}

	\begin{center}\underline{\hspace{5 cm}}\end{center}

Deux ingénieurs et un ami non-ingénieurs se rencontrent à un bar un vendredi soir pour raconter leur semaine de travail.

	\begin{itemize}
		\item Le premier ingénieur: "J'ai passé un semaine horrible à faire des plans un à la fois chaque jour."
	
		\item Le deuxième ingénieur: "J'ai fait un peu moins pire. J'ai au moins pu faire des plans complets plusieurs fois par jour."
	
		\item Le troisième ami non-ingénieur: "Ben les gars vous en avez de la chance! Moi je me limite à des plans culs qu'une fois par mois".
	\end{itemize}
	\begin{center}\underline{\hspace{5 cm}}\end{center}

	\begin{center}
	\includegraphics{img/humour/acdc.jpg}
	\end{center}

	\begin{center}\underline{\hspace{5 cm}}\end{center}

Trying to understand engineers:

\begin{itemize}

	\item Trial N\degree 1

An engineer tell to a friend:"Well, yesterday I was walking home, minding my own business, when a beautiful woman rode up to me on this bike. She threw the bike to the ground, took off all her clothes and said, 'Take what you want!'"

The friend (also engineer) approvingly, "Good choice. The clothes probably wouldn't have fit."

	\item Trial N\degree 2 

For an optimist, the glass is half full.
For a pessimistic person, it is half empty.
To the engineer, it is twice as large as needed.

	\item Trial N\degree 3 

A pastor, a doctor and an engineer were waiting one morning for a particularly slow group of golfers. After a moment the engineer cry: "What's with these guys? We must have been waiting for 15 minutes!". The doctor also desperate says: " I don't know, but I've never seen such ineptitude!". The pastor the says: "Hey, here comes the greens keeper. Let's have a word with him. [dramatic pause] Hi George. Say, what's with that group ahead of us? They're rather slow, aren't they?". The greens keeper answers: " Oh, yes, that's a group of blind fire fighters. They lost their sight saving our clubhouse from a fire last year, so we always let them play for free anytime.". The group is silent for a moment... The pastor then says: "That's so sad. I think I will say a special prayer for them tonight.". And the doctor:" Good idea. And I'm going to contact my ophthalmologist buddy and see if there's anything he can do for them.". Finally the engineer says " Why the fuck can't these guys play at night?".

	\item Trial N\degree 4 

A engineer was crossing a road one day when a frog called out to him and said, "If you kiss me, I'll turn into a beautiful princess." He bent over, picked up the frog, and put it in his pocket. The frog spoke up again and said, "If you kiss me and turn me back into a beautiful princess, I will tell everyone how smart and brave you are and how you are my hero" The man took the frog out of his pocket, smiled at it, and returned it to his pocket. The frog spoke up again and said, "If you kiss me and turn me back into a beautiful princess, I will be your loving companion for an entire week." The man took the frog out of his pocket, smiled at it, and returned it to his pocket. The frog then cried out, "If you kiss me and turn me back into a princess, I'll stay with you for a year and do ANYTHING you want." Again the man took the frog out, smiled at it, and put it back into his pocket. Finally, the frog asked, "What is the matter? I've told you I'm a beautiful princess, that I'll stay with you for a year and do anything you want. Why won't you kiss me?". The man said, "Look, I'm a an engineer. I don't have time for a girlfriend, but a talking frog is cool.".

	\item Trial N\degree 5 

A reporter interviews a Corsican farmer: "Tell me, how do you draw the roads here in your country?". The farmer replies, "beh, we launch a donkey and look where it goes into the mountains .... and that's where we passed the road". The journalist then retorts: "and if you do not have a donkey?". The farmer replies: "ah beh... we take an engineer...".
\end{itemize}

	\begin{center}\underline{\hspace{5 cm}}\end{center}
	
	\begin{center}
	\includegraphics[scale=0.25]{img/humour/weather_forecast.jpg}
	\end{center}
		
	During the heat of the space race in the 1960's, NASA decided it needed a ball point pen to write in the zero gravity confines of its space capsules.
	
	After considerable research and development, the Astronaut Pen was developed at a cost of \$1 million U.S. The pen worked and also enjoyed some modest success as a novelty item back here on earth.
	
	The Soviet Union, faced with the same problem, used a pencil.
	\begin{center}\underline{\hspace{5 cm}}\end{center}
	
	The great mathematician John Von Neumann was consulted by a group who was building a rocket ship to send into outer space. When he saw the incomplete structure, he asked, "Where did you get the plans for this ship?"
	
	He was told, "We have our own staff of engineers."
	
	He disdainfully replied: "Engineers! Why, I have complete sewn up the whole mathematical theory of rocketry. See my paper of 1952."
	
	Well, the group consulted the 1952 paper, completely scrapped their \$10 million structure, and rebuilt the rocket exactly according to Von Neumann's plans. The minute they launched it, the entire structure blew up. They angrily called Von Neumann back and said: "We followed your instructions to the letter. Yet when we started it, it blew up! Why?"
	
	Von Neumann replied, "Ah, yes! That is technically known as the blow-up problem - I treated that in my paper of 1954."

	\begin{center}\underline{\hspace{5 cm}}\end{center}
	
	In an electronic laboratory:
	
	\begin{itemize}
		\item Say, what's the trailer in the parking?
	
		\item The trailer?
	
		\item Yeah, the driver says you're aware ...
	
		\item Oh yes !! I ordered a 1 Farad capacitor .
	\end{itemize}

	\begin{center}\underline{\hspace{5 cm}}\end{center}
	
	\begin{figure}[H]
		\begin{center}
		\includegraphics[scale=0.2]{img/humour/iso.jpg}
		\end{center}	
	\end{figure}
	
	\begin{center}\underline{\hspace{5 cm}}\end{center}

	What engineers say and what they mean by it:

	\begin{itemize} 
		\item Major Technological Breakthrough: Back to the drawing board. 
	
		\item Developed after years of intensive research: It was discovered by accident. 
	
		\item The designs are well within allowable limits: We just made it, stretching a point or two. 
	
		\item Test results were extremely gratifying: It works, and are we surprised! 
	
		\item Customer satisfaction is believed assured: We are so far behind schedule that the customer was happy to get anything at all. 
	
		\item Close project coordination: We should have asked someone else. 
	
		\item Project slightly behind original schedule due to unforeseen difficulties: We are working on something else. 
	
		\item The design will be finalized in the next reporting period: We haven't started this job yet, but we've got to say something. 
	
		\item A number of different approaches are being tried: We don't know where we're going, but we're moving.
	
		\item Extensive effort is being applied on a fresh approach to the problem: We just hired three new guys; we'll let them kick it around for a while.
	
		\item Preliminary operational tests are inconclusive: The darn thing blew up when we threw the switch. 
	
		\item The entire concept will have to be abandoned: The only guy who understood the thing quit. 
	
		\item Modifications are underway to correct certain minor difficulties: We threw the whole thing out and are starting from scratch. 
	
		\item Essentially complete: Half done. 
	
		\item We predict...: We hope to God! 
	
		\item Drawing release is lagging: Not a single drawing exists. 
	
		\item Risk is high, but acceptable: 100 to 1 odds, or with 10 times the budget and 10 times the manpower, we may have a 50/50 chance. 
	
		\item Serious, but not insurmountable, problems: It will take a miracle. God should be the program manager. 
	
		\item Not well defined: Nobody has thought about it. 
	
		\item Requires further analysis and management attention: Totally out of control. 
	
		\item The project is designed for high availability: Malfunctions will be blamed on the operators mistakes. 
	
		\item This project has low maintenance requirements: We wouldn't let the technicians change a light bulb, much less fool around with our baby. 
	
		\item The software is being developed without excessive process overhead: The documentation will be written in clear and lucid Chinese. 
	
		\item The delivery is scheduled for the last quarter of next year: This leaves us plenty of time to decide who to blame for it being late. 
	\end{itemize}
	
	\begin{center}\underline{\hspace{5 cm}}\end{center}

\begin{itemize} 
	\item How many first year engineering students does it take to change a light bulb?: None. That's a second year subject.

	\item How many second year engineering students does it take to change a light bulb?: One, but the rest of the class copies the report

	\item How many third year engineering students does it take to change a light bulb?: Will this question be in the final examination?

	\item How many civil engineers does it take to change a light bulb?: Two. One to do it and one to steady the chandelier

	\item How many electrical engineers does it take to change a light bulb?: None. They simply redefine darkness as the industry standard

	\item How many computer engineers does it take to change a light bulb?: Why bother? The socket will be obsolete in six months anyway

	\item How many mechanical engineers does it take to change a light bulb?: Five. One to decide which way the bulb ought to turn, one to calculate the force required, one to design a tool with which to turn the bulb, one to design a comfortable - but functional - hand grip, and one to use all this equipment. 

	\item How many nuclear engineers does it take to change a light bulb?: Seven. One to install the new bulb and six to figure out what to do with the old one for the next 10,000 years. 
\end{itemize}

\begin{center}\underline{\hspace{5 cm}}\end{center}

	\begin{figure}[H]
		\centering
		\includegraphics[scale=1]{img/humour/2bornot2b.jpg}
	\end{figure}
	
\begin{center}\underline{\hspace{5 cm}}\end{center}

This happens in Moscow: a couple of tourists ask for it's way, on a bridge to a Russian engineer.

The guy answer: "You are going through, and within 50 meters, turn right..."

Thanks from the tourists... And they leave. So the guy runs behind them:

"Wait, wait! I just remembered that the bridge is 70 meters. If you turn right after 50 meters, as I have told you, you will fall into the water".
\begin{center}\underline{\hspace{5 cm}}\end{center}

A quality engineers team is working on the FMECA for a new chemical factory. After several weeks, during the debriefing meeting: 

Quality engineers: "Our conclusion is: there's a 1/10,000 rate for the plant to blow up, killing many people and inducing terrible ecological crisis, it's ethically unacceptable !" 

Boss: "Standards are talking about an acceptable risk for a 1/7,000 rate" 

The team regroups and talk, then: 

Quality engineers: "Our conclusion is: You have an over-quality problem, it's ethically unacceptable!"
\begin{center}\underline{\hspace{5 cm}}\end{center}

A guy was seated next to a 10-year-old girl on an airplane. Being bored, he turned to the girl and said, "Let's talk. I've heard that flights go quicker if you strike up a conversation with your fellow passenger."

The girl, who was reading a book, closed it slowly and said to the guy, "What would you like to talk about?"

"Oh, I don't know" said the guy. "How about nuclear physics?"

"OK" she said. "That could be an interesting topic. But let me ask you a question first. A horse, a cow and a deer all eat the same stuff... grass. Yet a deer excretes little pellets, while a cow turns out a flat patty, and a horse produces clumps of dried grass. Why do you suppose that is?"

The guy thought about it and said, "Hmmm, I have no idea."

To which the girl replied, "Do you really feel qualified to discuss nuclear power when you don't know shit?".

	\pagebreak
	\section{Computing}

	\begin{center}
	\includegraphics{img/humour/meaning_life.jpg}
	\end{center}
\begin{center}\underline{\hspace{5 cm}}\end{center}	

If you want to be a hacker, you will have to use Linux.

Here are 2 solutions :
\begin{enumerate}
	\item You are a capitalistic bourgeois and you buy it at Fry's for \$150.
	\item You are an asshole, and then you download it on the net.
\end{enumerate}

Of course you belong to the second category, so you have to use your FTP client and wait a few hours while your are downloading a Slack or a Debian. Try not to use Mandriva, this is for the public. You must not forget that you are an uNdERgrOuNd guy now, it's normal, you're a Hacker.

O.K., now you have got Linux, you can forget it. You do not need to lose your time learning how this new Operating System works and that you will never use because Xwing vs Tie Fighter doesn't run on it. The best way is to delete lilo, like that you will be sure to boot on Windows Vista. This elegant solution is practiced by many guys like you. The easiest way is to invoke fdisk /mbr in a DOS session, it will delete lilo which was installed on your hard drive's MBR. Good, you do not need to care about Linux anymore.

The goal is to have it, not to know how using it.

Okay, but then how can I show to everybody that I have Linux and that I am a rebel ?". 

That is a natural issue. Hopefully, I thought about you little looser, here comes sentences that you have to tell everybody about Linux:

\begin{itemize}
	\item "Linux is really powerful, you are free to do what ever you want with it compare to these fascists systems like Winblows. Anyway, M\$ is to crappy."

	\item "Well, if you are a beginner, you better not use Linux, this thing is for eLiteS. You, you are better using WinFuck."

	\item "Hey ! Where could I find the libc5.4.36 ? Because the 5.4.35 is not compatible with the modifications have done on my kernel."

	\item "Netscape sucks, it creates core dumps over 50 mo when it launches !I prefer using Lynx, text mode is much easier."

	\item "Wooooww what a fool ! He installed a Mandriva !! Only Debian is good, at least you know that you are the master of your system. No really, Mandriva is really to crappy."
\end{itemize}

With these sentences, you will quickly belong to the "okay, he's an asshole, but an asshole using Linux" category, this is the first step to become a real hacker. Now everybody knows you have Linux, you must move to the next stage, become the guy who knows everything about networks, who masters ICMP like a god. This is the second step of your long journey.

Now, you have to put your hands on the money. Go to Fry's and buy any books about Unix and networks. The main thing is a complicated title. A "rlogin protocol on ethernet sub-address" would have the best effect. Buy them even if you do not understand the titles, you just need to have impressive books : You are not suppose to read them, it is just to impress your friends, whom are assholes just like you.

The best way is to buy a book like "TCP/IP Volume 43" and learn by heart random words: socket, sub-address, FDDI, telnet, for example. Then you will use them in your sentences, even out of context, nobody will check what you are saying. For example, for feel to swing sentences like : "How many packets over FDDI networks does a telnet transmits ?". God, I swear it on a good cowboy channel, that will always impress and nobody will tell you that what you have just said has no meaning. Don't worry.

Then, put these books in your bedroom, the most complicated titles in the most visible areas. Damage a few pages' corners to make it credible. Take some paper and draw network diagrams, or add things like 123.44.5.34 root / lydia to make others believe you spend your days cracking passwords like a mad. Do not hesitated scanning Mitnick's photos and hang them above your bed, or put stickers of skulls on your computer to show that now, you are a thug, a dangerous guy.
To complete your new identity and truly become a hacker, you mustn't hesitate to say great things like : "I am thirsty for knowledge". Okay, you are in high school since 10 ten years, but it does not matter, you love learning anyway, hacking is a passion and you have a lot of willpower. Specify that you never do any damages to all the computers you hack into. Say you are doing that for "Intellectual Challenge". Yes, this time, you will have to force yourself not to laugh hard, so train yourself in front of the mirror before.

When people are dangerous like you, they must meet with other crooks in order to jeopardize the State's security. For this, there is THE thugs' rendez-vous, called the "Meet 2600". Every month, you will go to a MacDo in Paris, place of Italy, and there you will meet very important guys, who rebooted the entire Internet with a Visual Basic program and have special hair cuts like rebels of the society.

Okay, you will not learn much in this meetings, losers who go over there masturbate each other thinking "Yeah, we are hAcKeRz, we are ruthless, real men. Oh shit, it is already 6 pm, I have to go home otherwise my mum will kill me." But you will still feel real thrill thinking that the MacDo is full of cameras and microphones, and that the employees are agents from the DST who are listening to dangerous conversations such as :

\begin{itemize}
	\item[$-$] Asshole1: How much is the Whooper ?
	\item[$-$] Asshole2: Uh, MacDo does Whoopers now ?
	\item[$-$] Asshole1: I thought they always did, no 
\end{itemize}

The hAcKeRz' community also goes to raves. It is a part of the message "rebel no future fuck da society, we take ecstasy and listen to rubbish but we don't care, it is great because it is prohibited". Feel free to go in these places, it's a part of the lost culture to go in these hot parties.

You, you are a real hacker and you hear well to spread your knowledge in order to educate others like you. For this reason, there is e-zines. They include the best known as NoWay or NoRoute where the worse alongside the best (which is unfortunate for the best …) but there is also big shits who deserve to be more famous like the excellent Core-Dump who talks in an English that even my cat understands better than me.

Obviously, you have never read any books about Unix, you never hacked a machine in your life so you do not know what to write. Don't worry, you are not the only one to be in such a situation. The best thing to do is to write a rap article talking about your last rave or plunder anarchist magazines without understanding what you are talking about. If you decide plunder Phrack, do not hesitate to correct the guy or add more complicated stuff, nobody will check. Come on, free yourself, you are thirsty for knowledge, do not forget.

Now, for sure, you have truly become a hacker, a IRC rabble, an Internet thug, you are scaring government agencies and IBM wants to hire you to secure their network because this stupid Henry created a new Internet virus. So, you will have to, on daily basis, behave like a hacker, a real, a true, which means having a hacker's spirit and talking like a hacker.

A hacker primarily lives on IRC. Once your friends and your family will notice that you have changed, that you are not the same man anymore, you will have to spread the news on IRC in order to make new friends whom are assholes like you. Say goodbye to either \#flours nor \#friendship, now you will go down to the bottom of the IRC, the cyber-bronx, nuke-city, where only the real bruisers can be respected in this world of violence. To succeed, you will have to go from the asshole hacker status to an asshole on IRC who pretends being Mitnick, I mean a c0wb0y.
\begin{center}\underline{\hspace{5 cm}}\end{center}	

	\begin{center}
	\includegraphics[scale=0.8]{img/humour/quantum_computing.eps}
	\end{center}
\begin{center}\underline{\hspace{5 cm}}\end{center}

The Top 20 replies by programmers when their programs do not work:
	\begin{enumerate}[nolistsep]
		\item[20.] "That's weird..."
		\item[19.] "It's never done that before."
		\item[18.] "It worked yesterday."
		\item[17.] "How is that possible?"
		\item[16.] "It must be a hardware problem."
		\item[15.] "What did you type in wrong to get it to crash?"
		\item[14.] "There is something funky in your data."
		\item[13.] "I haven't touched that module in weeks!"
		\item[12.] "You must have the wrong version."
		\item[11.] "It's just some unlucky coincidence."
		\item[10.] "I can't test everything!"
		\item[9.] "THIS can't be the source of THAT."
		\item[8.] "It works, but it hasn't been tested."
		\item[7.] "Somebody must have changed my code."
		\item[6.] "Did you check for a virus on your system?"
		\item[5.] "Even though it doesn't work, how does it feel?
		\item[4.] "You can't use that version on your system."
		\item[3.] "Why do you want to do it that way?"
		\item[2.] "Where were you when the program blew up?"
	\end{enumerate}

And the Number One reply by programmers when their programs don't work:
	\begin{enumerate}
		\item "It works on my machine."		
	\end{enumerate}


\begin{center}\underline{\hspace{5 cm}}\end{center}	

How the way people code "Hello World" varies depending on their age and job:

\begin{itemize}
	\item High School/Jr. High

 \texttt{10 PRINT "HELLO WORLD"\\
20 END}

	\item First year in College

\texttt{program Hello(input, output)\\
begin\\
writeln('Hello World')\\
end.}

	\item Senior year in College
	
\texttt{(defun hello\\
(print\\
(cons 'Hello (list 'World))))}

	\item New professional

\texttt{\#include <stdio.h>\\
void main(void)\\
\{\\
char *message[] = \{"Hello ", "World"\};\\
int i;\\
for(i = 0; i < 2; ++i)\\
printf("\%s", message[i]);\\
printf("\\n");\\
\}}

	\item Seasoned professional
	
\texttt{\#include <iostream.h>\\
\#include <string.h>\\
class string\{\\
private:\\
int size;\\
char *ptr;\\
public:\\
string() : size(0), ptr(new char('\textbackslash 0')) \{\}\\
string(const string \&s) : size(s.size)\\
\{\\
ptr = new char[size + 1];\\
strcpy(ptr, s.ptr);\\
\}\\
~string()\\
\{\\
delete [] ptr;\\
\}\\
friend ostream \& operator <<(ostream \& , const string \& );\\
string \& operator=(const char *);\\
\};\\
ostream \& operator<<(ostream \& stream, const string \& s)\\
\{\\
return(stream << s.ptr);\\
\}\\
string \&string::operator=(const char *chrs)\\
\{\\
if (this != \& chrs)\\
\{\\
delete [] ptr;\\
size = strlen(chrs);\\
ptr = new char[size + 1];\\
strcpy(ptr, chrs);\\
\}\\
return(*this);\\
\}\\
int main()\\
\{\\
string str;\\
str = "Hello World";\\
cout << str << endl;\\
return(0);
\}
}

	\item System Administrator

\texttt{\#include <stdio.h>\\
\#include <stdlib.h>\\
main()\\
\{\\
char *tmp;\\
int i=0;\\
tmp=(char *)malloc(1024*sizeof(char));\\
while (tmp[i]="Hello Wolrd"[i++]);\\
i=(int)tmp[8];\\
tmp[8]=tmp[9];\\
tmp[9]=(char)i;\\
printf("\%s \textbackslash n",tmp);\\
\}\\
}

	\item Apprentice Hacker

\texttt{\#!/usr/local/bin/perl\\
\$msg="Hello, world.\textbackslash n";\\
if (\$\#ARGV >= 0) \{
while(defined(\$arg=shift(@ARGV))) \{\\
\$outfilename = \$arg;\\
open(FILE, ">" . \$outfilename) || die "Can't write \$arg: \$! \textbackslash n";\\
print (FILE \$msg);\\
close(FILE) || die "Can't close \$arg: \$!\textbackslash n";\\
\}
\} else \{\\
print (\$msg);\\
\}\\
1;}

	\item Experienced Hacker

\texttt{\#include <stdio.h>\\
\#include <string.h>\\
\#define S "Hello, World\\n"\\
main()\{exit(printf(S) == strlen(S) ? 0 : 1);\}}

	\item Seasoned Hacker

\texttt{\%cc -o a.out ~/src/misc/hw/hw.c\\
\%a.out\\
Hello, world.\\
Guru Hacker\\
\%cat
Hello, world.}

	\item New Manager (do you remember?)

\texttt{10 PRINT "HELLO WORLD"\\
20 END}

	\item Middle Manager

\texttt{mail -s "Hello, world." bob@b12\\
Bob, could you please write me a program that prints "Hello, world."?\\
I need it by tomorrow.\\
\^D\\}

	\item Senior Manager

\texttt{\%zmail jim\\
I need a "Hello, world." program by this afternoon.}

	\item Chief Executive

\texttt{\%letter\\
letter: Command not found.
\%mail\\
To: \^X \^F \^C\\
\%help mail\\
help: Command not found.\\
\%damn!\\
!:Event unrecognized\\
\% logout\\}

	\item Research Scientist

\texttt{PROGRAM HELLO\\
PRINT *, 'Hello World'\\
END}

	\item Older research Scientist

\texttt{WRITE (6, 100)\\
100 FORMAT (1H ,11HHELLO WORLD)\\
CALL EXIT\\
END}

\end{itemize}
\begin{center}\underline{\hspace{5 cm}}\end{center}	

	\begin{center}
	\includegraphics[scale=0.8]{img/humour/punition.jpg}
	\end{center}
	
\begin{center}\underline{\hspace{5 cm}}\end{center}

What the software engineers says... and what must be understood:
\begin{itemize}
	\item We will put this project on schedule: We will take care of it if we have nothing else to do

	\item This is a completely new program!: It's absolutely not compatible with the old version

	\item This program requires no maintenance: It is impossible to debug

	\item This program requires little maintenance: It's almost impossible to debug

	\item We will respect the standards: It has always been like that and it's not going to change now that

	\item We want to respect the standards: You're not going to control everything that we do

	\item The new version of this program is 100\% compatible with the previous: We did not touch anything

	\item Different approaches have been tried: We still trying to guess what happens.

	\item We approach a solution: We met for coffee...

	\item The preliminary tests were not satisfying: This damn program crashed as soon as we launch it

	\item We'll have to abandon the entire concept: The only person who understood something just resigned

	\item We prepare a comprehensive report, according to an entirely new approach: We just hired three newbie who left school

	\item This is a major breakthrough: We still can not understand why it does not work

	\item This is the result of years of development: We were finally able to operate a piece of the program ...

	\item We are working on it: We are so in trouble that it's hopeless

	\item Tell us what you think: We will listen to what you have to say as it does not undermine what is already done, or what we have decided to

	\item We'll take a look: Forget it! We have enough problems like that ...

	\item I have not received your e-mail: It's been ages that I have not checked my email...
\end{itemize}

	\begin{center}
	\includegraphics[scale=0.3]{img/humour/tesla_ipaddress.jpg}
	\end{center}

	\pagebreak
	\section{Social Sciences}
	
To understand the marketing vocabulary..... and avoid appearing ridiculous in an evening party with colleagues:
\begin{enumerate}
	\item Michael is at a party and sees a very attractive girl. He approaches her and says: "I am a very good shot". This is named the "direct marketing".

	\item Michael is at a party with a group of friends and he sees a very attractive girl. A friend approached her and said, "You see that boy there, it's a very good shot". This is named the "advertising".

	\item Michael is at a party and sees a very attractive girl. He asks her phone number. The next day he called and said, "I am a very good shot". This is what we name "telemarketing".

	\item Michael is at a party and sees a very attractive girl. He recognized and approaches her, he refreshes her memory by saying: "You remember that I am a very good shot?". This is named the "Customer Relationship Management (CRM)".

	\item Michael is at a party and sees a very attractive girl. He stands up, arranges a little, approaches her and serves as a glass. He opens the door when she leaves, picks up his bag when he falls, offers her a cigarette and said, "I am a very good shot". This is what we name "public relations" or "public relations" (PR).

	\item Michael is at a party and sees a very attractive girl. He invites all her girlfriends to dance, offer them a drink and laugh withostensibly very spiritual jokes. The beautiful girl approach and say, "I feel that you are a very good shot". This is named "lobbying".

	\item Michael is at a party and sees a very attractive girl. She approach him and said: "I heard you're a very good shot". This is named the "brand power".

	\item Michael is at a party and see a super beautiful girl. He looks at her with his friends, doing very fine reflections, completely drunkes, does nothing and return alone at home. This is named the "market reality"...
\end{enumerate}

\begin{center}\underline{\hspace{5 cm}}\end{center}
	\begin{center}
	\includegraphics{img/humour/fluid_dynamics.jpg}
	\end{center}

	\begin{center}
	\includegraphics{img/humour/stupid.eps}
	\end{center}
\begin{center}\underline{\hspace{5 cm}}\end{center}

A man is flying in a hot air balloon and realizes he is lost. He reduces height and spots a man down below. He lowers the balloon further and shouts: "Excuse me, can you tell me where I am?" 
\begin{itemize}
	\item[$-$] The man below says: "Yes, you're in a hot air balloon, hovering 30 feet above this field." 

	\item[$-$] "You must be an engineer" says the balloonist. 

	\item[$-$] "I am" replies the man, "How did you know?" 

	\item[$-$] "Well" says the balloonist, "everything you have told me is technically correct, but it's no use to anyone". 

	\item[$-$] The man below says "you must be in management". 

	\item[$-$] "I am" replies the balloonist, "but how did you know?" 

	\item[$-$] "Well," says the man, "you don't know where you are, or where you're going, but you expect me to be able to help. You're in the same position you were before we met, but now it's my fault." 
\end{itemize}
\begin{center}\underline{\hspace{5 cm}}\end{center}

A man goes on a Saturday at a wedding in a Corsica small village. He's late and he drives as fast as possible on winding roads. Suddenly, after a turn, it must stop short, a flock of sheep occupies the entire road. The shepherd is there and slowly move his flock. The driver uses the horn of his vehicle several times without any effect. After a few minutes, the driver apostrophe the shepherd and says:
\begin{itemize}
	\item[$-$] "I'm late, I'm going to a wedding that will be followed by a barbecue, if I tell you how many sheep you have, you will you give me one of your sheep?"

	\item[$-$] "For sure" said the shepherd, "I am not close to one". 

	\item[$-$] The driver takes his calculator and after a minute announce: "1233".

	\item[$-$] "You won" say the shepherd, "Choose your pet".
\end{itemize}
The driver designate one. The shepherd then ask:
\begin{itemize}
	\item[$-$] If I find what is your profession, you give me back my beast?

	\item[$-$] "Of course" say the driver, "I'm listening".

	\item[$-$] "You are a high level official public servant and you've done and high level administration school or another big school like this."

	\item[$-$] "You are right" say the driver, "But how did you guess?".

	\item[$-$] The shepherd: "Please give my dog back"
\end{itemize}
\begin{center}\underline{\hspace{5 cm}}\end{center}

	\begin{center}
	\includegraphics[scale=0.7]{img/humour/meeting_girls.jpg}
	\end{center}

	\begin{center}
	\includegraphics{img/humour/fibonaughty_sexquence.jpg}
	\end{center}

	\begin{table}[H]
		\centering
			\begin{tabular}{c m{0.1cm} c m{0.1cm} c}
		    \begin{minipage}{.3\textwidth}
    		\center \includegraphics{img/humour/worker.eps}\\
		    \center Worker
		    \end{minipage}
	    	&
			+
			& 
		    \begin{minipage}{.3\textwidth}
    		\center \includegraphics{img/humour/process.eps}\\
		    \center Processs
		    \end{minipage}
		    &
		    =
		    &
		   	\begin{minipage}{.3\textwidth}
    		\center \includegraphics{img/humour/engineer.eps}\\
		    \center Engineer
		    \end{minipage}
	    \\
		    \begin{minipage}{.3\textwidth}
    		\center \includegraphics{img/humour/engineer.eps}\\
		    \center Engineer
		    \end{minipage}
	    	&
			+
			& 
		    \begin{minipage}{.3\textwidth}
    		\center \includegraphics{img/humour/sociability.eps}\\
		    \center Sociability
		    \end{minipage}
		    &
		    =
		    &
		   	\begin{minipage}{.3\textwidth}
    		\center \includegraphics{img/humour/marketing.eps}\\
		    \center Marketing
		    \end{minipage}
	    \\
		    \begin{minipage}{.3\textwidth}
    		\center \includegraphics{img/humour/marketing.eps}\\
		    \center Marketing
		    \end{minipage}
	    	&
			-
			& 
		    \begin{minipage}{.3\textwidth}
    		\center \includegraphics{img/humour/truth.eps}\\
		    \center Truth
		    \end{minipage}
		    &
		    =
		    &
		   	\begin{minipage}{.3\textwidth}
    		\center \includegraphics{img/humour/commercial.eps}\\
		    \center Commercial
		    \end{minipage}
	    \\
		    \begin{minipage}{.3\textwidth}
    		\center \includegraphics{img/humour/commercial.eps}\\
		    \center Commercial
		    \end{minipage}
	    	&
			-
			& 
		    \begin{minipage}{.3\textwidth}
    		\center \includegraphics{img/humour/brain.eps}\\
		    \center Brain
		    \end{minipage}
		    &
		    =
		    &
		   	\begin{minipage}{.3\textwidth}
    		\center \includegraphics{img/humour/manager.eps}\\
		    \center Manager
		    \end{minipage}
	    \\
		    \begin{minipage}{.3\textwidth}
    		\center \includegraphics{img/humour/manager.eps}\\
		    \center Manager
		    \end{minipage}
	    	&
			+
			& 
		    \begin{minipage}{.3\textwidth}
    		\center \includegraphics{img/humour/ego.eps}\\
		    \center Ego
		    \end{minipage}
		    &
		    =
		    &
		   	\begin{minipage}{.3\textwidth}
    		\center \includegraphics{img/humour/project_manager.eps}\\
		    \center Project Manager
		    \end{minipage}
	    \\
	   		\begin{minipage}{.3\textwidth}
    		\center \includegraphics{img/humour/project_manager.eps}\\
		    \center Project Manager
		    \end{minipage}
	    	&
			-
			& 
		    \begin{minipage}{.3\textwidth}
    		\center \includegraphics{img/humour/humour.eps}\\
		    \center Humour
		    \end{minipage}
		    &
		    =
		    &
		   	\begin{minipage}{.3\textwidth}
    		\center \includegraphics{img/humour/hr.eps}\\
		    \center Human Ressources
		    \end{minipage}
	    \\	    
		\end{tabular}
	\end{table}

	\begin{center}
	\includegraphics[scale=0.7]{img/humour/xmas.jpg}
	\end{center}
	
	\begin{center}\underline{\hspace{5 cm}}\end{center}	
	\begin{center}
		\includegraphics[scale=0.8]{img/humour/evolution.jpg}
	\end{center}

 	\chapter{Links}
		This directory contains links in 7 categories all relating to science and that I find interesting. I wish to state that under no circumstances I was paid in any form whatsoever for adding links in the list below! You can also found my preferred apps for iPad on my French blog. 
	
	\begin{itemize}	 
		\item[$-$] {\Large \ding{52}} High-Quality website both in terms of design and content
		\item[$-$] {\Large \ding{45}} Content with developments and demonstrations
		\item[$-$] {\Large \ding{41}} Website with forum
		\item[$-$] {\Large \ding{36}} Scientific softwares, sharewares, freewares to download
		\item[$-$] {\Large \ding{229}} Books, publications, magazines, papers (to view or download)
		\item[$-$] {\Large \ding{44}} Cool/funny website
		\item[$-$] {\Large \ding{170}} Favorite website
		\item[$-$] {\Large \ding{73}} Must see 
	\end{itemize}
	
	If there were also three links to highlight of all this is \href{http://www.google.com}{\color{blue} Google}, \href{http://www.wikipedia.com}{\color{blue} Wikipedia} and \href{http://www.youtube.com}{\color{blue} YouTube} that we owe many loans!		

	\pagebreak

	\section{Exact Sciences}

	{\Large \ding{52}}{\Large \ding{45}}{\Large \ding{36}}{\Large \ding{44}}{\Large \ding{170}}{\Large \ding{73}}\bcdfrance{} ChronoMath, small chronology of mathematics, is an educational document that Serge Mehl is constantly renewing since 1988 as a math tutor in French Africa where he was cooperating for many years. More than 450 mathematicians (and their work) are reviewed ...! Must see!\\
	\href{http://www.chronomath.com}{\color{blue} http://www.chronomath.com}
	
	{\Large \ding{41}}{\Large \ding{36}}{\Large \ding{229}}\bcdfrance{} This site offers mathematics news, an encyclopedia with dictionary parts, biographies and formulas, as well as individual files on various mathematical topics and a forum. This site is particularly impressive with respect to the amount of information available for download. \\
	\href{http://www.bibmath.net}{\color{blue} http://www.bibmath.net} 
	
	{\Large \ding{52}}{\Large \ding{45}}\bcdfrance{} A good site dealing with some relevant topics in physics (synophysics). The design could be reviewed in terms of navigation using frames ... but we must still focus on the content quality and it largely predominates.\\
	\href{http://www.sciences.univ-nantes.fr/physique/perso/blanquet/frame3.htm}{\color{blue} http://www.sciences.univ-nantes.fr/physique/perso/blanquet/frame3.htm}
	
	{\Large \ding{52}}{\Large \ding{45}}{\Large \ding{36}}\bcdfrance{} Many PDFs on algebra, geometry and analysis.\\
	\href{http://c.caignaert.free.fr}{\color{blue} http://c.caignaert.free.fr}
	
	{\Large \ding{52}}{\Large \ding{45}}{\Large \ding{36}} arXiv is an archive for electronic preprints of scientific papers. The time elapsing between the time a researcher completes a project and when his work is published in a newspaper may be about a year. At the time scale of research it's a long time. The introduction of the arXiv is therefore a way to overcome this time and cost problem.\\
	\href{http://arxiv.org}{\color{blue} http://arxiv.org}
	
	{\Large \ding{52}}{\Large \ding{45}}{\Large \ding{36}}\bcdfrance{} The multidisciplinary HAL open archive is intended for the deposit and the free dissemination of research level scientifical published and unpublished results and theses from French of foreign educational institutions, public or private laboratories. There is even scientific books published recently by prestigious French publishing houses.\\
	\href{http://hal.archives-ouvertes.fr/}{\color{blue} http://hal.archives-ouvertes.fr/}
	
	\pagebreak
	\section{Publishing/Magazines}

	{\Large \ding{52}}{\Large \ding{41}}{\Large \ding{229}}{\Large \ding{170}}{\Large \ding{73}}\bcdfrance{} We can consider this proposed website as the equivalent of the previous, but for the French engineers. However, it is qualitatively better and there are approximatively the same number of files but more homogeneous. The only regret is perhaps the access to certain elements which is not always easy the first time and also non-free.\\
	\href{http://www.techniques-ingenieur.fr}{\color{blue} http://www.techniques-ingenieur.fr}
	
	{\Large \ding{52}}{\Large \ding{229}}{\Large \ding{170}}{\Large \ding{73}}\bcdfrance{} Absolutely excellent and a must see! Many full courses of the École Polytechnique (France) are published and available for free and to download in PDF format (is it going to last?). About 1,000 educational resources are available in total, whose quality is highly variable, but whose relevance and rarity of subjects is always equal.\\
	\href{http://catalogue.polytechnique.fr}{\color{blue} http://catalogue.polytechnique.fr}
	
	{\Large \ding{52}}{\Large \ding{229}}{\Large \ding{170}}{\Large \ding{73}}\bcdfrance{}  The website itself is not great (which is a pity) but the magazine for which he offers to subscribe (against payment) is well for a visit for people who like mathematics and their actuality.\\
	\href{http://tangente.poleditions.com/}{\color{blue}http://tangente.poleditions.com/}
	
	{\Large \ding{52}}{\Large \ding{229}}{\Large \ding{170}}{\Large \ding{73}}\bcdfrance{} A magazine of the same family as Tangente but at a much higher technical and academic staff by my opinion. The content is particularly oriented on pure mathematics but without direct and explicit applications to physics, engineering or economics.\\
	\href{http://www.quadrature.info}{\color{blue}http://www.quadrature.info}
	
	{\Large \ding{52}}{\Large \ding{229}}{\Large \ding{170}}{\Large \ding{73}} The website itself is not great either (which is a pity, too) but some of the proposed works are just historical!\\
	\href{http://urss.ru}{\color{blue}http://urss.ru}
	
	{\Large \ding{52}}{\Large \ding{229}}{\Large \ding{170}}{\Large \ding{73}}\bcdfrance{} Excellent website with a huge resource (links directory) of electronic documentation dealing with mathematics only in French and English. The site contains links to the pages of the authors of the documents (if not it would require a considerable server ...).\\
	\href{http://mathslinker.chez-alice.fr}{\color{blue}http://mathslinker.chez-alice.fr}
	
	{\Large \ding{229}}\bcdfrance{} This site contains an on-line library of publications in French of the greatest mathematicians of the 20th and 19th century. Must See!\\
	\href{http://matwbn.icm.edu.pl/wyszukiwarka.php}{\color{blue}http://matwbn.icm.edu.pl/wyszukiwarka.php}
	
	{\Large \ding{52}}{\Large \ding{229}}{\Large \ding{170}}{\Large \ding{73}}\bcdfrance{} Reprints of fundamentals works on mathematics, physics, history and philosophy of science.\\
	\href{http://www.gabay.com}{\color{blue}http://www.gabay.com}
	
	\pagebreak
	{\Large \ding{52}}{\Large \ding{229}}{\Large \ding{170}}{\Large \ding{73}}\bcdfrance{} Eyrolles Editions - Excellent website containing numerous works of quality in English and French. This website propose the books of several publishers including Wiley, Springer, etc. It is worthwhile to go take a look in the sections "Mathematics" and "Physics" there are good things... \\
	\href{http://www.eyrolles.com}{\color{blue}http://www.eyrolles.com}
	
	{\Large \ding{52}}{\Large \ding{229}}{\Large \ding{170}}{\Large \ding{73}}\bcdfrance{} Dunod Editions - Propose contemporary literature (under and postgraduate). The books of this publishing house are mostly excellent. Personally they are technically (but not pedagogically) my favorite, because often the developments are very detailed.\\
	\href{http://www.dunod.com}{\color{blue}http://www.dunod.com}
	
	{\Large \ding{229}}{\Large \ding{170}} Springer Editions - Propose scientific literature of Phd level. The site navigation is not easy because the choices are arranged a bit anyhow but if the proposed works are very technical and high level ... very high level.\\
	\href{http://www.springer.de}{\color{blue}http://www.springer.de}
	
	{\Large \ding{229}}{\Large \ding{170}}{\Large \ding{73}}\bcdfrance{} The NUMDAM program, led by MathDoc (UMS 5638 CNRS - UJF) on behalf of CNRS, provides retrospective digitization of mathematics funds published in France.\\
	\href{http://www.numdam.org}{\color{blue}http://www.numdam.org}
	
	{\Large \ding{52}}{\Large \ding{229}}{\Large \ding{170}}{\Large \ding{73}} Wrox Editions - Propose computer development books (expert level). As far as I know (and I think so), this publishing house is the worldwide reference in the field of books about programming languages.\\
	\href{http://www.wrox.com}{\color{blue}http://www.wrox.com}
	
	{\Large \ding{52}}{\Large \ding{170}}{\Large \ding{73}} Cambridge University Press Editions - Propose also postdoc level scientific literature. Somewhat the equivalent of Springer Editions but the website structure is a little bit better. You can particularly appreciate the opportunity to subscribe for free to receive regularly their catalog.\\
	\href{http://www.cambridge.org}{\color{blue}http://www.cambridge.org}
	
	
	{\Large \ding{45}}{\Large \ding{229}}{\Large \ding{170}}{\Large \ding{73}} The Digital Library of France (BNF) offers a free download of the scientific literature (among others) whose copyrights have fallen after 75 years. The download system is not user friendly, but you can found very relevant works. PDFs are often more than 10 MB so take care if you have a slow Internet connexion.\\
	\href{http://gallica.bnf.fr}{\color{blue} http://gallica.bnf.fr}
	
	{\Large \ding{45}}{\Large \ding{229}}{\Large \ding{170}}{\Large \ding{73}} The Physical Review Letters (PRL) search engine (scientific papers) offers against registration and payment (sic) access to scientific research articles in theoretical and experimental physics published for over 100 years. The level of detail of the articles offered is often not very high but this is not the main objective for specialists.\\
	\href{http://prola.aps.org/search}{\color{blue} http://prola.aps.org/search}
	
	\pagebreak
	{\Large \ding{45}}{\Large \ding{229}}{\Large \ding{73}} Lavoisier Editions - Very good publisher offering books in the highly specialized field of electronic, electrical, civil, computer engineering etc. You can also particularly appreciate the opportunity to subscribe for free to receive regularly their great catalogue. See especially the "Hermes Science" and "Tech \& Doc" link of this publisher.\\
	\href{http://www.lavoisier.fr}{\color{blue} http://www.lavoisier.fr} (\href{http://www.editions-hermes.fr}{\color{blue} http://www.editions-hermes.fr} + \href{http://www.tec-et-doc.com}{\color{blue} http://www.tec-et-doc.com})
	
	{\Large \ding{170}}{\Large \ding{45}}{\Large \ding{229}} World Scientific Editions - This american publisher provides provides a wide range of high-level physical sciences and mathematics literature (see the wide range of scientific sections corresponding on their site).\\
	\href{http://www.worldscibooks.com}{\color{blue}http://www.worldscibooks.com}
	
	\section{Associations}

		{\Large \ding{44}} myScience.ch gives an overview of science, research, universities, companies and other research centers in Switzerland. The site provides practical information on jobs, funding and lives in Switzerland and also science news to researchers, scientists, academics and anyone interested in science. It has a higher proportion of articles in German ... even in the French version...\\
		\href{http://www.myscience.ch}{\color{blue}http://www.myscience.ch}
		
		{\Large \ding{229}}{\Large \ding{44}}{\Large \ding{73}}\bcdfrance{} The Rationalist Union aims to promote the role of reason in the intellectual debate as in the public debate, in front of all irrational excesses. It strives to make available to everyone the opportunity to access an intelligible conception of the world and the life. She strike for a State that remains secular and assume its function of protecting against all forms of indoctrination.\\
		\href{http://www.union-rationaliste.org}{\color{blue}http://www.union-rationaliste.org}
		
		{\Large \ding{45}}{\Large \ding{229}} The American Society of Mathematics is an association of professional mathematicians dedicated to the interests of research and teaching of mathematics, what it does in the form of various free publications and conferences, and prizes awarded to mathematicians.\\
		\href{http://www.ams.org}{\color{blue}http://www.ams.org}
		
		{\Large \ding{45}}{\Large \ding{229}} The American Physical Society is a scientific society founded in May 20, 1899, based in the U.S., very active in the field of scientific research in physics.\\
		\href{http://www.aps.org}{\color{blue}http://www.aps.org}
		
		{\Large \ding{45}}{\Large \ding{229}}{\Large \ding{170}} The European Physical Society (EPS) is a non-profit organization whose purpose is to promote physics and physicists in Europe. This society propose a subscription to it's Europhysicsnews magazine and many other things.\\
		\href{http://www.eps.org}{\color{blue}http://www.eps.org}

	\pagebreak
	\section{Jobs}
	
	{\Large \ding{73}} Here is a list of websites for job positions especially for swiss physicists and mathematicians ... (however a link is only for U.S. territory!). So the descriptions are often in German (...) and the job positions for the banking sector (who represents ~10-15\% of employment in Switzerland from what I know ...). Some offers are linked to more general sites like the famous jobup.ch actually well known in Switzerland.
	
	\begin{itemize}	 
		\item[$-$] \href{http://www.telejob.ch}{\color{blue}http://www.telejob.ch} 
	
		\item[$-$] \href{http://www.jobs.myscience.ch}{\color{blue}http://www.jobs.myscience.ch} 
	
		\item[$-$] \href{http://www.sciencejobs.com}{\color{blue}http://www.sciencejobs.com}
	
		\item[$-$] \href{http://www.math-jobs.ch}{\color{blue}http://www.math-jobs.ch}
	
		\item[$-$] \href{http://www.analyticrecruiting.com}{\color{blue}http://www.analyticrecruiting.com}
	\end{itemize}
	
	\section{Television/Radio}

	{\Large \ding{73}} Excellent website explaining with 3D animation many classical equipment of daily life designed and developed by engineering methods (motor, refrigerator , pump, etc.).\\
	\href{http://www.learnengineering.org}{\color{blue}http://www.learnengineering.org}
	
	\bcdfrance{} C'est pas Sorcier: The famous french TV show (modern version in french of "Bill nye the Science guy") about all sciences that almost all young french speaking have know or still discover!!\\
	\href{https://www.youtube.com/user/cestpassorcierftv/}{\color{blue}https://www.youtube.com/user/cestpassorcierftv/}
	
	\bcdfrance{} Cité-Sciences: Viewing video-conferences, or listening to audiotapes on popular topics of physics and astronomy (+ others) popularized.\\
	\href{http://www.cite-sciences.fr}{\color{blue}http://www.cite-sciences.fr}
	
	\bcdfrance{} On Canal Académie, academics, specialists in physics, mathematics, humanities, philosophy, sociology, law and jurisprudence, economics, politics and finance, history, geography and demography and political science will share their thoughts on the news and society developments.\\
	\href{http://www.canalacademie.com}{\color{blue}http://www.canalacademie.com}

	\pagebreak
	\section{Other sciences}

	{\Large \ding{52}}{\Large \ding{45}}{\Large \ding{41}}{\Large \ding{44}}\bcdfrance{} Website with many pages and a huge amount of interesting solved exercises. Unfortunately, this website became paying with access for a limited period ... the price is very low and it may still be appropriate to pay for quality.\\
	\href{http://www.web-sciences.com}{\color{blue}http://www.web-sciences.com}
	
	{\Large \ding{52}}{\Large \ding{41}}{\Large \ding{229}}{\Large \ding{44}}{\Large \ding{73}}\bcdfrance{} Futura-Sciences is website for the popularization of pure and exact sciences with news about sciences and technologies, issues on many topics, scientific forums with debate and discussiosn ... (the physcis forum is especially well attended).\\
	\href{http://www.futura-sciences.com}{\color{blue}http://www.futura-sciences.com}
	
	{\Large \ding{52}}{\Large \ding{41}}{\Large \ding{229}}{\Large \ding{44}}{\Large \ding{73}}\bcdfrance{} Physics Forum is consider as the world's largest physics community. There is a ton of discussions on many subjects answered with quality by the community. This forum is also considered as the partner forum of this e-book (even if they consider my book as spam...) as I don't have time to answer to questions anymore for free by e-mail. \\
	\href{http://www.physicsforums.com}{\color{blue}http://www.physicsforums.com}
	
	{\Large \ding{52}}{\Large \ding{45}}{\Large \ding{41}}{\Large \ding{36}}{\Large \ding{229}}{\Large \ding{44}}{\Large \ding{170}}{\Large \ding{73}}\bcdfrance{} Astrosurf is a portal of links (directory) for French amateur astronomers. There is also astronomy forums astronomy, announcements, free astronomy websites hosting, ephemeris and all the clubs and associations of French astronomy. You can particularly found the famous site of Thierry Lombry (here).\\
	\href{http://www.astrosurf.com}{\color{blue}http://www.astrosurf.com}
	
	{\Large \ding{52}}{\Large \ding{229}} The C.E.A. is the French Commissariat à l'Énergie Atomique. The site offers news and interesting files on particular areas of advanced physics. One can also find free educational resources for teachers (slide shows and posters).\\
	\href{http://www.cea.fr}{\color{blue}http://www.cea.fr}
	
	{\Large \ding{44}}{\Large \ding{73}} Here is a site that, if I was a child, would have made me happy ... and the misery of my parents. It's not so much the content that is interesting but especially the online store (eStore) which offers a hundred gadgets and educational games for passionate of science. Attention to not spend too much!\\
	\href{http://www.xump.com}{\color{blue}http://www.xump.com}
	
	{\Large \ding{44}}{\Large \ding{73}} For years, the cartoons of S. Harris have added humor to innumerable magazines, books, newsletters, ads and web sites in the field of sciences. Enjoy!\\
	\href{http://www.sciencecartoonsplus.com/}{\color{blue}http://www.sciencecartoonsplus.com/}
	
	\pagebreak
	\section{Softwares/Applications}
	
	{\Large \ding{52}}{\Large \ding{36}}{\Large \ding{170}}{\Large \ding{73}} Website of the software TeXMaker used to write this book in \LaTeX (software working on multiple Operating Softwares).\\
	\href{http://www.xm1math.net/texmaker/index.html}{\color{blue}http://www.xm1math.net}
	
	{\Large \ding{52}}{\Large \ding{36}}{\Large \ding{170}}{\Large \ding{73}} Minitab is the leading software for engineers working in industry and applying statistical process control (\SeeChapter{see section Industrial Engineering}) as part of their work or making any statistical study in the context of R\& D.\\
	\href{http://www.minitab.com}{\color{blue}http://www.minitab.com}
	
	{\Large \ding{52}}{\Large \ding{36}}{\Large \ding{170}}{\Large \ding{73}} Isograph is an excellent suite of software for engineers working in industry and applying the techniques of preventive maintenance (reliability, FMEA) and decision support (\SeeChapter{see section Theory Of Games And Decision}).\\
	\href{http://www.isograph-software.com}{\color{blue}http://www.isograph-software.com}
	
	{\Large \ding{52}}{\Large \ding{36}}{\Large \ding{170}}{\Large \ding{73}} The ReliaSoft company develops in my opinion the best softwares on the market for reliability and quality assurance engineers . Their software is used by the best engineering companies worldwied. Overall their software are real small diamonds (particularly Weibull++).\\
	\href{http://ww.reliasoft.com}{\color{blue}http://ww.reliasoft.com}
	
	{\Large \ding{52}}{\Large \ding{36}}{\Large \ding{73}} JMP is for me the best known software for design of experiments at this date. His pedagogical structured and the very good available documentation (books) with mathematical proofs of the approach used by the software is well appreciated.\\
	\href{http://www.jmp.com}{\color{blue}http://www.jmp.com}
	
	{\Large \ding{52}}{\Large \ding{36}}{\Large \ding{170}}{\Large \ding{73}} Official website of the excellent Maple symbolic computation software used for many examples in the different chapters of Sciences.ch. The website also offers lots of documentation and plugins to download.\\
	\href{http://www.mapleapps.com}{\color{blue}http://www.mapleapps.com}
	
	{\Large \ding{52}}{\Large \ding{36}}{\Large \ding{170}}{\Large \ding{73}}\bcdfrance{} TANAGRA is a data mining powerful freeware (in English) for education and research. It implements a set of data mining methods (more than 100) in the field of statistical exploratory data analysis, machine learning and databases. This is my favorite  data mining software because of its simplicity of use and uncluttered user interface.\\
	\href{http://eric.univ-lyon2.fr/~ricco/tanagra/}{\color{blue}http://eric.univ-lyon2.fr/~ricco/tanagra/}
	
	{\Large \ding{52}}{\Large \ding{36}}{\Large \ding{73}} Official website of MATLAB™ calculation and simulation software. I'm not especially a fan but I must admit this is an essential tool in some businesses, especially for the SimuLink part and R\&D Engineering. This is somewhat a "must have" for engineers as well as LabView.\\
	\href{http://www.mathworks.com}{\color{blue}http://www.mathworks.com}
	
	\pagebreak
	{\Large \ding{52}}{\Large \ding{36}}{\Large \ding{73}}Official website of the company Statistica. A worldwide reference in the field of statistical analysis, data mining and statistical process control that stands a priori in a handkerchief with IBM SPSS.\\
	\href{http://www.statsoft.com}{\color{blue}http://www.statsoft.com}
	
	{\Large \ding{52}}{\Large \ding{36}}{\Large \ding{170}}{\Large \ding{73}}COMSOL Multiphysics (formerly FEMLAB) is an excellent interactive environment for the modeling of industrial and scientific applications based on partial differential equations (PDEs) using finite element methods (more eays to use than ANSYS). A must have for engineers and researchers (for those who can afford to buy the license of course ..)!\\
	\href{http://www.comsol.fr}{\color{blue}http://www.comsol.fr}
	
	{\Large \ding{52}}{\Large \ding{36}}{\Large \ding{170}}{\Large \ding{73}} Palisade @Risk is an add-in suite for MS Excel and MS Project for probabilistic simulation (Monte Carlo and Latin Hypercube). It's joint integration with MS Project and MS Excel and its calculations modules based on genetic algorithms, neural networks and its add-in of decision theory makes it a coveted tool for high executive officers of large companies (NASA and Lockheed Martin use it for major projects) and high level engineers in the field finance, quality, project management and production.\\
	\href{http://www.palisade-europe.com}{\color{blue}http://www.palisade-europe.com}
	
	{\Large \ding{52}}{\Large \ding{36}}{\Large \ding{170}}{\Large \ding{73}} The Decision Analysis software edited by TreeAge (DATA) allows users to build, analyse and distribute tree analysis, Markov models and influence diagrams. The DATA models incorporate visually quantitative and qualitative aspects related to business decisions to provide a tool to assist in projects during analysis of complex risks.\\
	\href{http://www.treeage.com}{\color{blue}http://www.treeage.com}
	
	{\Large \ding{52}}{\Large \ding{36}}{\Large \ding{170}}{\Large \ding{73}} MathType is a powerful equation editor (software used for this website) to import/export MathML or TeX or you can run in it stand-alone application. With a toolbar, it also fits perfectly with the programs of the MS Office 2003-2013 suite (Word, Excel, PowerPoint), but also with 500 other softwares (specifically math softwares, or not). At the difference with \LaTeX 2$\varepsilon$ you will not loose hours to solve compatibility and compilation issues.\\ 
	\href{http://www.dessci.com/en/}{\color{blue}http://www.dessci.com/en/}
	
	{\Large \ding{52}}{\Large \ding{36}}{\Large \ding{73}} Scilab (contraction of Scientific Laboratory) is a free software, developed at INRIA Rocquencourt (France). It is a numerical computing environment that enables fast numeric resolutions and graphics commonly encountered in Applied Mathematics for people or small business having not enough money to buy MATLAB™.\\
	\href{http://www.scilab.org}{\color{blue}http://www.scilab.org}
	
	{\Large \ding{36}}{\Large \ding{73}} Official website of the famous dynamic geometry software: Cabri (worldwide reference in this domain) primarily destinated for the learning of geometry in schools. It can animate geometric figures, unlike those drawn on the blackboard. It comes to plane geometry or for geometry in 3D. It is the ancestor of all dynamic geometry software.\\
	\href{http://www.cabri.com}{\color{blue}http://www.cabri.com}
	
	{\Large \ding{36}}{\Large \ding{170}}{\Large \ding{73}}\bcdfrance{} Website of a math teacher who developed a very good freeware for many relevant charts, integration calculations, simulations of statistics and probabilities easily and playfully in the classroom (College level).\\
	\href{http://www.patrice-rabiller.fr}{\color{blue}http://www.patrice-rabiller.fr}
	
	{\Large \ding{36}}{\Large \ding{73}} ACDLabs develops the excellent Chemsketch software for molecules modeling and design (2D, 3D) with access to certain physical and chemical properties. This is a very useful tool also to prepare handouts or scientific articles (publications).\\
	\href{http://www.acdlabs.com}{\color{blue}http://www.acdlabs.com}
	
	{\Large \ding{36}}{\Large \ding{73}}{\Large \ding{170}} MolView is an intuitive, Open-Source web-application to make molecular chemistry science more awesome!\\
	\href{http://molview.org}{\color{blue}http://molview.org}
	
	{\Large \ding{36}}{\Large \ding{73}} Excellent high-level software for the design of civil structures based on finite element methods (not misused in Switzerland for simple or complex projects and studies for engineers).\\
	\href{http://www.scia-online.com}{\color{blue}http://www.scia-online.com}
	
	{\Large \ding{36}}{\Large \ding{73}} Maxima is a computer algebra freeware (a Maple like software), falling under GNU GPL under the package Macsyma, the symbolic calculation software originally developed for the needs of U.S. Department of Energy. Maxima can do arithmetic, polynomials, matrices, integration, derivation, the calculation of series, limits, resolutions of systems of differential equations, etc.\\
	\href{http://maxima.sourceforge.net}{\color{blue}http://maxima.sourceforge.net}
	
	{\Large \ding{36}}{\Large \ding{73}} SPSS (Statistical Package for the Social Sciences) is a priori the most widely used software in companies in Switzerland for statistical analysis of data because it has an impressive number of tests as well as many business packages. Because of its price and its owner (IBM) can be regarded as the premium solution of Statistics.\\
	\href{http://www.ibm.com/software/fr/analytics/spss}{\color{blue}http://www.ibm.com/software/fr/analytics/spss}
	
	{\Large \ding{36}}{\Large \ding{170}}{\Large \ding{73}} R is a free powerful language and environment for statistical computing and graphics. R provides a wide variety of statistical (linear and nonlinear modeling, classical statistical tests, time-series analysis, classification, clustering, ...) and graphical techniques, and is highly extensible. One of R's strengths is the ease with which well-designed publication-quality plots can be produced, including mathematical symbols and formulae where needed.\\
	\href{http://www.r-project.org}{\color{blue}http://www.r-project.org}
	
	{\Large \ding{36}}{\Large \ding{73}} LabVIEW (Laboratory Virtual Instrument Engineering Workbench) is a powerful visual tool (graphical programming language) developed by National Instruments to control measures instrument or robots from a PC and used by a lot of companies worldwide (particularly by engineers).\\
	\href{http://www.ni.com/labview}{\color{blue}http://www.ni.com/labview}

			
	\chapter{Quotes}
	\begin{figure}[H]
		\centering
		\includegraphics{img/be_greater_than_average.jpg}	
	\end{figure}
	
	\begin{fquote}[Leonardo da Vinci]He who loves practice without theory is like the sailor who boards ship without a rudder and compass and never knows where he may cast.
 	\end{fquote}
 	
	\begin{fquote}[Richard Feynman]Physics is like sex. Sure it may have some practical results, but that's not why we do it.
 	\end{fquote}
 	
	\begin{fquote}[Martin Gardner]Mathematics is not only real, but it is the only reality. That is that entire universe is made of matter, obviously. And matter is made of particles. It's made of electrons and neutrons and protons. So the entire universe is made out of particles. Now what are the particles made out of? They're not made out of anything. The only thing you can say about the reality of an electron is to cite its mathematical properties. So there's a sense in which matter has completely dissolved and what is left is just a mathematical structure.
 	\end{fquote}
 	
 	\begin{fquote}[Owen Chamberlain]Each generation of scientists stands upon the shoulders of those who have gone before.
 	\end{fquote}
 	
 	\begin{fquote}[Kiyoshi Ito]In precisely built mathematical structures, mathematicians find the same sort of beauty others find in enchanting pieces of music, or in magnificent architecture. There is however, one great difference between the beauty of mathematical structures and that of great art. Music by Mozart, for instance, impress greatly even those who do not know musical theory; the cathedral in Cologne overwhelms spectators even if they know nothing about Christianity. The beauty in mathematical structures, however, cannot be appreciated without understanding of a group of numerical formulae that express laws of logic. Only mathematicians can read "musical scores" containing many numerical formulae, and play that "music" in their hearts.
 	\end{fquote}
 	
 	 \begin{fquote}[Richard Feynman]Science is just imagination in straitjacket.
 	\end{fquote}
 	
 	 \begin{fquote}[Stephen Hawking]The greatest enemy of knowledge is not ignorance, is the illusion of knowledge.
 	\end{fquote}
 	
 	\begin{fquote}[Albert Einstein]I have no special talents, I am only passionately curious.
 	\end{fquote}

	\begin{fquote}[Dara Briain]People keep saying "Science doesn't know everything!". Well, science know it doesn't know everything; otherwise it would stop.
 	\end{fquote}


 	 \begin{fquote}[Stephen Hawking]One can't prove that God doesn't exists, but Science makes God unnecessary.
 	\end{fquote}

 	 \begin{fquote}[Voltaire]The power of numbers was much more respected among us when we knew nothing about them..
 	\end{fquote}

	\begin{fquote}[Jean Rostand]To reflect is to disturb one's thoughts.
 	\end{fquote} 
 	
 	\begin{fquote}[John Cleese]If you are stupid how can you possibly realize that you are stupide?.
 	\end{fquote} 	
 	
 	\begin{fquote}[Frank Wilczek]If you don't make mistakes, you're not working on hard enough problems. And that's a big mistake.
 	\end{fquote}

	\begin{fquote}[?]Success relies on the success of those who have preceded us.
 	\end{fquote}
 	
 	\begin{fquote}[Brigitte Le Roux, Henry Rouanet]Between quantity and quality there is geometry.
 	\end{fquote}

	\begin{fquote}[Leonardo da Vinci]Practice should always be based upon a sound knowledge of theory.
 	\end{fquote}
 	
 	\begin{fquote}[Richard Feynman]It doesn't matter how beautiful your theory is, it doesn't matter how smart you are. If it doesn't agree with experiment, it's wrong.
 	\end{fquote}
 	
 	\begin{fquote}[Leonardo da Vinci]Details make perfection, and perfection is not a detail.
 	\end{fquote}
 	
 	\begin{fquote}[LNeil Degrasse Tyson]Ignorance is a virus. Once it starts spreading, it can only be cured by reason. For the sake of humanity, we must be that cure!
 	\end{fquote}
 	
 	\begin{fquote}[David Hilbert]The art of doing mathematics consists in finding that special case which contains all the germs of generality. 
 	\end{fquote}

 	\begin{fquote}[Mark Twain]There are three kinds of lies: lies, damned lies and statistics. 
 	\end{fquote}

 	\begin{fquote}[Derek Bok]If you think education is expensive, try ignorance.
 	\end{fquote}
 
  	\begin{fquote}[David Hilbert]The art of doing mathematics consists in finding that special case which contains all the germs of generality.
 	\end{fquote}
 	
 	 \begin{fquote}[Benjamin Franklin]Tell me and I forget, teach me and I remember, involve me and I learn.
 	\end{fquote}
  	
 	 \begin{fquote}[Albert Einstein]Not everything that can be counted counts, and not everything that counts can be counted.
 	\end{fquote}

 	 \begin{fquote}[Niels Bohr]An expert is a person who has found out by his own experience all the mistakes that one can make in a very narrow field.
 	\end{fquote}
 	
 	 \begin{fquote}[Daniel C. Denett]What you can imagine depends on what you know.
 	\end{fquote}
 	
 	\begin{fquote}[Arthur C. Clarke]Any sufficiently advanced technology is indistinguishable from divinity.
 	\end{fquote}
 	
 	\begin{fquote}[Nikola Tesla]What one man calls God, another call it the Laws of Physics.
 	\end{fquote}
 	
 	\begin{fquote}[?]The only person you need to be better thhan is the person you were yesterday.
 	\end{fquote}
 	
 	\begin{fquote}[Eleanor Roosevelt]Great minds discuss ideas. Average minds discuss events. Small minds discuss people.
 	\end{fquote}
 	
 	\begin{fquote}[Marie Curie]Be less curious about people and more curious about ideas.
 	\end{fquote}
 	
 	\begin{fquote}[Albert Einstein]Creativity is intelligence having fun.
 	\end{fquote}
 	
 	\begin{fquote}[Aaron Swartz]With enough of us, around the world, we'll not just send a strong message opposing the privatization of knowledge - we'll make it a thing of the past.
 	\end{fquote}
 	
 	\begin{fquote}[Buddha]When you move your focus from competition to contribution life becomes a celebration. Never try to defeat people, just win the hearts.
 	\end{fquote}
 	
 	\begin{fquote}[Bertrand Russell]The whole trouble with the world is that the stupid are cocksure, and the intelligent are full of doubt.
 	\end{fquote}
 	
 	\begin{fquote}[Gerge Bernard Show]Some men see things as they are and ask why. Others dream things that never were and ask why not.
 	\end{fquote}
 	
 	\begin{fquote}[Albert Einstein]If you can't explain it to a six year old, you don't understand it well enough yourself.
 	\end{fquote}
 	
 	\begin{fquote}[Hubert Reeves]Man is the most insane species. He worships an invisible God and destroys a visible Nature. Unaware that the Nature he is destroying is this God he is worshiping.
 	\end{fquote}
 	
 	\begin{fquote}[Willima Thomson (Lord Kelvin)]If you can't measure it, you can't improve it!
 	\end{fquote}
 	
 	\begin{fquote}[Albert Einstein]Once you stop learning, you start dying.
 	\end{fquote}
 	
 	\begin{fquote}[Socrates]There is only one good, KNOWLEDGE, and one evil, IGNORANCE.
 	\end{fquote}
 	
 	\begin{fquote}[?]The purpose of argument, should not be victory, but progress.
 	\end{fquote}
 	
 	\begin{fquote}[Richard Dawkins]Science is the poetry of reality!
 	\end{fquote}
 	
 	\begin{fquote}[Sthepen Hawking]Real science can be far stranger than science fiction and much more satisfying.
 	\end{fquote}
 	
 	\begin{fquote}[Alan Watts] You don't understand the basic assumptions of your own culture if your own culture is the only culture you know.
 	\end{fquote}
 	
 	\begin{fquote}[Elon Musk]The ones who are crazy enough to think that they can change the world are the ones who do!
 	\end{fquote}
 	
 	\begin{fquote}[Leonardo da Vinci]The noblest pleasure is the joy of understanding!
 	\end{fquote}
 	
 	\begin{fquote}[Galileo Galilei]Doubt is the father of invention!
 	\end{fquote}

\chapter{Change Log}

This is a detailed change log of the book for people interested to see how this book has evolved:

	\begin{itemize}
		\item \textbf{May 2002}
		\begin{itemize}[noitemsep]
			\item Definitions (science, law, theorem, postulate, axiom, corollary, ...)
			\item Operations of addition, subtraction, multiplication, division, power
			\item Concept of Numbers (integers, relative real, fractional, complex algebra, abstract,...)
			\item Domain of definition of variables
			\item Arithmetic polynomials
			\item Absolute value
			\item Binary relations, order relations
			\item Gamma Function Euler, Euler's constant
			\item Electromagnetic wave equation, the wave speed, speed of the light, transported energy
			\item Introduction to Optics
			\item Rule of three, percentages
			\item Net quantities, the cost price of purchase, indices, sale prices, brut prices, net profit, net price, assets
			\item Simple and compound interest, late and early payments
			\item Portfolio management (decision criteria, goodwill, return on investisment)
			\item MarkowitzModel  (utility function, selection criteria)
			\item Vocabulary about Stock Exchange
			\item Plasma Frequency
			\item Probability (universe, events, axioms)
			\item Combinatorial Analysis
			\item Statistics (discrete variables, continuous variable, standard deviation, variance and covariance)
			\item Distribution Functions (discrete uniform law, Bernoulli's law, binomial law, hypergeometric law, multinomial law, Poisson's law, Gauss-Laplace law, Cauchy's law, beta law, gamma law, chi-square law, Student's law)
			\item Estimator, correlation
			\item Matrix of covariance
			\item Statistical adequation tests
		\end{itemize}
		\item \textbf{June 2002}
			\begin{itemize}[noitemsep]
				\item Introduction Set Theory (Zermelo-Fraenkel theory, inclusion, complementarity, intersection, union, product, empty set
				\item Arithmetic, harmonic, geometric, quadratic averages
				\item Univocity
				\item Logarithmic and exponential functions
				\item Golden ratio
				\item Introduction to Fourier series
				\item Coulomb's law, electrostatic field, electrostatic potential
				\item Ampere's Theorem
				\item Biot-Savartlaw, magnetic field, magnetic induction
				\item Maxwell's equations
				\item Dimensions in euclidiean and fractal geometry
				\item Geometric shapes unidimensionelles, bidimensionelles, tridimensionelles
				\item Pythagorean theorem
				\item Physical units systems
				\item The principle of least action and energy conservation
				\item Position, velocity and acceleration
				\item Continuity equation
				\item Bernoulli Equation
				\item The Doppler effect
				\item Restrained Relativity, invariance principle, Lorentz transformations (time, length, velocity addition, increase in mass, Minkowski space-time)
				\item General relativity (world line, punctual event, inertial particles, null cones, space-time vectors spaces and curved planes, metric tensor)
				\item Heisenberg principles of uncertainty, Schrödinger equation, Wave probability density
				\item Introduction to superstrings
			\end{itemize}
		\item \textbf{July 2002}
			\begin{itemize}[noitemsep]
				\item Creating of Table of Contents
				\item Visual representations of functions
				\item 2nd degree polynomials and roots
				\item Operator of vector and scalar fields (gradient, nabla, divergence, curl, laplacien)
				\item Vector Analysis (concept of arrow, set of vectors, scalar multiplication, vector space, linear combinations, generating families, bases of a vector space)
				\item Tensor calculus (Einstein convention, Kronecker symbol, anti-symmetry symbol)
				\item Notations for grouped multiplications (Big Sigma) and sums (Big Pi)
				\item Axioms for set of real number 
				\item Definition of various Inequalities 
				\item Circular and related movements
				\item Inertial forces (Coriolis force)
				\item Drake Equation
			\end{itemize}
		\item \textbf{August 2002}
			\begin{itemize}[noitemsep]
				\item New biographies (Hilbert, Riemann, Legendre)
				\item New section "Humor"
				\item Introduction to Topology
				\item Euclid postulates
				\item Gauss plane (complex numbers)
				\item De Moivre's formula
				\item Transformations in the complex plane
				\item Base change and tensor scalar product
				\item Covariant and contravariant components
				\item Relativistic transformations of the momentum
				\item Prefixes of multiples and submultiples of SI units
				\item Quality control (probabilities), efficiency curve, level value of acceptable quality (LQA)
				\item Kepler's laws
				\item Proof of classical gravitational force from Kepler's Law
				\item Proof of area law (second Kepler's Law)
				\item Moment of force
				\item Angular momentum
				\item Theorem of angular momentum
				\item Newton-Poisson Equation
				\item Thomson's, Borh's and Sommerfeld-Bohr's atom model
				\item Hydrogen spectrum
				\item Assumption of Neutron
				\item Quantum numbers
				\item Pauli exclusion Principles
				\item Classical Analysis of the Schrödinger equation for the ideal rectangular potential
				\item Newton's Laws
				\item Proof of Newton's third law from the principle of least action
				\item Release speed (for rockets or others)
				\item Definition of "Scientism"
				\item Variation of the gravitational acceleration in and out of a homogeneous spherical body
				\item The principle of least action and quantum physics (semi-classical limit)
				\item Ballistics (maximum range, safety parabola )
				\item Introduction to nuclear physics
				\item Atomic number, mass number
				\item Radioactivity, activity, filiation, isotopes, isotones, nuclide dating
				\item Atomic mass system (UMA)
				\item Masse defaults
				\item Fusion and fission
				\item Alpha and beta (minus and plus) desintegration
				\item Electron capture
				\item Gamma emission
			\end{itemize}
		\item \textbf{September 2002}	
			\begin{itemize}[noitemsep]
				\item New Biographies (Dalton, Boltzmann, De Broglie, etc.)
				\item Principle of good order
				\item Archimedean Property
				\item Induction Principle
				\item Divisability (Euclidean division)
				\item Congruent numbers
				\item Proof by nine
				\item Numbers basis
				\item Definition utility of thousands separators
				\item Priorities of parenthesis, brackets, braces and arithmetic operators
				\item Proof Theory (introduction)
				\item Definition of terms, formulas and demonstrations
				\item Definition of languages, symbols, relations and functions
				\item Definition of periodic, compound, basic, rational, fractional, irrational, algebraic and transcendental numbers
				\item Generalization of elementary algebra
				\item Dimensions of a vector space, extension of a free family, rank of a finite family
				\item Vector Hyperplane, direct sums
				\item Almosrt rigorous definition of the concepts of line, surface (plane) and volume
				\item Straight line intersection, half-lines, segments, aliquot part
				\item Continuity axiom of the line and of the plane
				\item Translation and rotation of plan
				\item Angles, units, measurements, sides of the angle, sharp edges, flat angles, equal angles ,straight angles, acute angles, obtuse angles, supplementary angles, complementary angles, 
				\item Perpendicular lines, bissecting angle
				\item Definition of work and energy: kinetic and potential energy theorem (motor work and resisting work)
				\item Concept of conservative vector field
				\item Conservation of energy and momentum
				\item Mass center theorem
				\item Relativistic force transformation
				\item Relativistic transformations of electric and magnetic fields
				\item Chandreskhar limit weights (collapse limit of white dwarfs)
				\item Definitions of optics, generalization of the law of refraction
				\item Broglie normalization condition, linked and non-linked states
				\item Harmonic oscillator
				\item Quantum chemistry and molecular vibrations
			\end{itemize}
		\item \textbf{Octobre 2002}
			\begin{itemize}[noitemsep]
				\item New biographies (Cauchy, Neumann, Bessel, Archimedes)
				\item New section on "Theoretical Computing"
				\item Greatest common divisor, least common multiple
				\item Rule signs (...)
				\item Proof of the irrationality of a number
				\item Introduction to arithmetic, harmonic and geometric sequences
				\item Limits and continuity of functions
				\item Definition of affine spaces
				\item Introduction to the Euclidean tensors and their properties
				\item Triangles and properties of triangles
				\item Definition of thermodynamic systems
				\item Definition of the reduced mass
				\item Bessel's functions
				\item Definition of the moment of inertia
				\item Introduction to Quantum Field Theory
				\item Introduction to radiation protection
				\item Proof of Bethe-Bloch formula
				\item Tunnel effect in quantum physics
				\item Introduction Dirac's formalism
				\item Heron and Archimedes algorithm's
				\item Introduction to fractal sets
				\item Introduction to game theory (cooperative games, earnings, payoff matrix, extensive forms, Pareto optimums, Nash equilibrium, evolutionary games)
			\end{itemize}
			\item \textbf{November 2002}
				\begin{itemize}[noitemsep]
				\item Fundamental theorem of arithmetic
				\item Introduction to crypthography (RSA, DES, MD5, SHA-1)
				\item Euler phi function
				\item Small Fermat theorem
				\item Introduction to topological dimensions and scaling 
				\item Cosine directors
			\end{itemize}
			\item \textbf{December 2002}
				\begin{itemize}[noitemsep]
				\item New Biographies (Nash, Cartan, Lucas, Lie)
				\item Detailed development of the "Integer" function
				\item Superposition of linear quantum states (quantum coherence) 
				\item Definition of Lipschitz functions and contracting functions 
				\item Definition of a convergent Cauchy sequence 
				\item Fixed point theorem (used in fractals, Newton methods and many others)
				\item Definitions of reality, the problem of the theory and trial on reality
				\item Definition of Euclidean space and Euclidean affine space
				\item Definition of property concepts in the field of chemistry
			\end{itemize}
			\item \textbf{January 2003}
			\begin{itemize}[noitemsep]
				\item Pascal's Theorem
				\item Buoyancy (Archimedes' principle)
				\item Simple Introduction to different symmetries in physics (temporal, spatial)
				\item Simple Introduction to different transformations in the plane (translation, scaling, reflection, isometry, rotation)
				\item Definition of an inverse and composed function/application
				\item Trigonometry (introduction, remarkable relations/identities, spherical trigonometry)
				\item Signature of a vector space
				\item Schmidt orthogonalization methods, base changes, Fourier associated spaces
				\item Everything (almost) on the plane and spherical trigonometry
				\item Keplerian orbital trajectories
				\item Introduction to the neoclassical monetary model (Say/Walras laws, homogeneity assumption)
				\item Boolean algebra (simple properties and theorems)
				\item Redesigned of quantum physics section (order of the subjects)
				\item Proof of evolutionnary Schrödinger  equation
				\item Proof of the relativistic evolutionnary Schrödinger equation
				\item Introduction to Antimatter theory
			\end{itemize}
		\item \textbf{February 2003}
				\begin{itemize}[noitemsep]
				\item Proof of gradient, divergence, rotational and Laplacian in cartesian, polary, cylindrical and spherical coordinates
				\item Complete mathematical developments of speed and acceleration expressions in cartesian, polar, cylindrical and spherical coordinates 
				\item Proof  of the relativistic invariance of the electric charge (charge conservation equation) 
				\item Proof of the existence of anti-particle  with opposite charge 
				\item Introduction to Gauge Theory (quadripotentiel Lorenz gauge, Coulomb gauge, Alembertian) 
				\item Introduction to Lagrangian and Hamiltonian formalism (generalized coordinates, configuration spaces, Euler-Lagrange equation, canonical formalism, Legendre transformation, Poisson brackets) 
				\item Rigorous definition of the principle of least action
				\item Definition of Tensor spaces
				\end{itemize}
		\item \textbf{March 2003}
			\begin{itemize}[noitemsep]
				\item New Biographies (Lorentz, Minkowski Hermann, Ricci-Curbastro, Levi-Civita)
				\item Definition of Cartesian product and extension of the scope of application of Cardinals
				\item Proof of Cauchy-Schwarz inequality
				\item Proof of the triangle inequality 
				\item Definitions of the vector product and mixed product 
				\item Proof of condensed form the sum of the first $n$ integers 
				\item Proof of the validity of the integration by parts 
				\item Definition of the algebraic vectorial structure and of an algebra 
				\item Definition of a homorphismes, isomorphism, endomorphism, automorphism
			\end{itemize}
		\item \textbf{April 2003}
			\begin{itemize}[noitemsep]
				\item New Biographies (Göpper-Meyer, Yukawa, Nöther, Cournot)
				\item Cournot Theory equilibrium (competition)
				\item Wilson's model (inventory management)
				\item Mathematics of phasers
				\item Relativistic model of the Sommerfeld's atom
				\item Analytical resolution of the Schrödinger equation
				\item Heisenberg's principles of quantum  uncertainty
				\item Lagrangian formalism of quantum physics fields
				\item Improvement of website forum (add mathematical symbols and external files)
				\item 10 new links to interesting web pages (associations + math stuff)
			\end{itemize}
		\item \textbf{Mai 2003}
			\begin{itemize}[noitemsep]
				\item New Biographies (Bell, Ramanujan, Landau)
				\item Proof of the precession of the perihelion of coupled orbits of stars or electric charges
				\item Definition and developments related to the Virial theorem
				\item Calculation of potential energy of a material sphere (internal temperature of Stars)
				\item Definition of prime numbers and proof that they are in infinite in number
				\item Definition of a fully enclosed ring
				\item Proof that a rational number is an algebraic number if and only if it is a relative integer.
				\item Definition of a multi-linear application/function (or morphism of vector space)
				\item Definition of a partition of a set and an an equivalence class.
				\item Descriptions of set operations of absorption and idempotence.
				\item Definitions and examples of sagittals diagrams.
				\item Definition of a magma and a monoid
				\item Pseudo-proofs of algebraic structures of fundamental sets of arithmetic
				\item Development of the theory of angular momentum in wave quantum physics
				\item Definition of a Diophantine equation and sets of Fermat's Last Theorem
			\end{itemize}
		\item \textbf{October 2003}
			\begin{itemize}[noitemsep]
				\item New Biographies (Abel, Banach, Boole, Bose, Brouwer, Clausius, Cayley, Curie,  Connes, Dirichlet, Frege, Gibbs, Picard, Erdos, Grothendieck, Hamilton, Hausdorf,  Heaviside, Helmholtz, Hermite, Hoyle, Jacobi, Klein, Kronecker, Langevin, Lee, Lobachevsky , Möbius, Monge, Fish, Schwartz, Shannon, Thom, van der Waals, Vieta, Weinberg, Witten, Gamow, Sturm, Liouville,Clairaut, Teller)
				\item Definition of puncutal spaces in classical mechanics, of writing conventions and changes in referentials
				\item Mathematical theory of projective perspective with vanishing points and definition of the projective and isometric perspectives
				\item Simplistic definition of the concept of "derivative" in functional analysis.
				\item Proof of some common derivatives (polynomials, composite functions, inverse functions, cosine, sine, arc sine, arc cosine, quotient of two functions, etc.).
				\item Numerical Methods: mathematical explanation of the complexity of an algorithm and  elementary optimization research
				\item Method of calculating the number $e$
				\item Resolution Method of $n$ linear equations systems with $n$ unknow using the pivot method
				\item Search of function roots by the methods of the proportional parts, of the bisection method, of the secant methods (regula falsi) and Newton's method
				\item Calculations of areas and integrals using the Riemann sums method
				\item Presentation of the Monte Carlo calculation for integrals, or Pi
				\item Mathematical development of simplex algorithm used in the operations research (linear programming)
				\item Contraction of indices in tensor calculus
				\item Definition and detailed properties of some speical tensor (symmetric tensor, antisymmetric tensor, fundamental tensor, etc.)
				\item Curvilinear coordinates (determination of the metric and linear element of a spherical and cartesian space and of the plane in polar coordinates)
Christoffel symbols (first and second type).
				\item Proof of the Cantor-Bernstein theorem
				\item Determination of the free generalized Lagrangian in General Relativity
			\end{itemize}
		\pagebreak
		\item \textbf{December 2003}
			\begin{itemize}[noitemsep]
				\item New Biographies (Kirchhoff biographers, Markowitz, Cox Merton, Scholes, Sharpe, Ferdinan von Lindemann, Bachelier, Stefan)
				\item Proof of one of the Stirling formulas
				\item Detailed calculation of a simple model of a Star surface temperature
				\item Detailed calculations of temporal properties of the securities values
				\item Proof of Taylor series and Maclaurin (limited and unlimited)
				\item Defining the Lagrange rest and of Alembert's, Cauchy, integral test and absolute convergence criterias
				\item Developments relative to the definitions of brightness, luminosity, apparent and absolute magnitude of Stars and calclulation of the distance to Cepheids
				\item Definitions of the solid angle, the solid angle of revolution and the elementary solid angle
				\item Definition of photometric and photonic and international system quantities
				\item Definitions and developments related to the light intensity, energy flow (with proof of Beer-Lambert' law), emittance, radiance (with Lambert's law), Kirchhoff's law
				\item Proofs of Stefan's law and Stefan-Boltzmann law
				\item Resolutions of third degree polyonials by radicals (Cardan's method) and development about solving quadratic polynomials in the complex set
				\item Definition of the concept of equations and inequalities
				\item Determination of the Cartesian equation of the plane, line (in space),cone and sphere
				\item Definition of market efficiency
				\item Determination of the Black \& Scholes equation 
				\item Presentation of the mathematical aspectts of Wiener process
				\item Ito lemma and Brownian motion (random walk)
			\end{itemize}
		\item \textbf{January 2004}
			\begin{itemize}[noitemsep]
				\item Definition of the indefinite integral
				\item Newtonian cosmological model (without the cosmological constant)
				\item Statement of the Peano's
				\item Introduction to Linear Algebra (Gauss reduction method, elementary operations between matrices)
				\item Statement of 5 Euclid's axioms and 5 groups of axioms in geometry
				\item Proof of relations for calculating the perimeter of the surface and the center of gravity of the square, the rectangle and triangle
				\item Proof of relations for calculating the volumes and surfaces of the torus, the sphere, the ellipsoid, cylinder and cone
				\item Definition of the centroid (center of gravity) and demonstration of four properties related thereto
				\item Demonstration of decomposition odd function and pair of any function
				\item Definition of hyperbolic trigonometric functions and enumeration of relations and proof of related remarkable properties
				\item Introduction to differential geometry (definition of a Riemannian geometry, Frenet triad, parametrised surface, etc.)
				\item DIntroduction to graph theory (Königsberg's bridges proof)
				\item Definition of a topological space and Hausdorff definition of a metric / ultra-metric space and associated distances (hölder, discreet, equivalent ...)
				\item Definition of a Lipschitz function (and related isometrics)
				\item Definition of open and closed set (open/closed balls, spheres, adherence, Hausdorff's excess) and diameters
				\item Definition of set distances (gap) and of a variety / map / atlas and differential homeomorphism
			\end{itemize}
		\item \textbf{April 2004}
			\begin{itemize}[noitemsep]
				\item Proof of the Guldin's theorem
				\item Proof of König's theorem of kinetic energy and angular momentum
				\item Presentation and proofs of various techniques for calculating the moments of inertia: Huygens-Steiner theorem, polar inertia moment, inertia tensor, generalized Huygens-Steiner
				\item Definition of "power" and "performance" and proof of the calculation of the power of a rotating machine
				\item Proof of Boltzmann thermodynamic entropy law and the following statistical distributions: Maxwell speeds, Maxwell-Boltzmann, Fermi-Dirac, Bose-Einstein
				\item Proof of some principal moments of inertia of the following bodies: torus, spheres, cones, rectangular plate, tube
				\item Introduction to wave optics: Huygens principle, proof of Malus law, development of Fraunhofer model in the case of a rectangular slot. 
				\item Definition and proof of the resolving power of a simple rectangular slot.
				\item Proof of the origin and the solution of the no less famous Bessel differential equation of order $n$
				\item Proof of the wave function of a stretched rope and a stretched circular membrane
				\item Proof of Planck's law and of known approximations (first Wien's Law, Rayleigh-Jeans law).
				\item Proof of the displacement law (second Wien's law) and the Stefan-Boltzmann law through Planck's law and determination of the Stefan-Boltzmann constant
				\item Study of the origin of Planck dimensions: Planck length, Planck mass, Planck density, Planck time, Planck energy
			\end{itemize}
		\item \textbf{July 2004}
			\begin{itemize}[noitemsep]
				\item Proof of the physical origin of heat
				\item Proof of Torricelli's theorem
				\item Proof of Venturi effect
				\item Proof of Poiseuille's law
				\item Proof of relativistic Compton effect
				\item Proof of the existence of the fossil radiation in the Universe
				\item Introductions to quotients sets (in this case $\mathbb{Z}/n$
				\item Proof of the physical origin of heat
				\item Proof of Torricelli's theorem
				\item Proof of Venturi's effect
				\item Proof of Poiseuille's law
				\item Proof of relativistic Compton effect
				\item Proof of the existence of the fossil radiation in the Universe
				\item Introductions to quotients sets (in this case $\mathbb{Z}/nZ$)
				\item Proof of Lorentz transformations for speed and acceleration
				\item Determination of the relativistic Lagrangian of a free system
				\item Proof and definition of Ricci theorem, of the covariant derivative, of the Ricci identity, of the Riemann-Christoffel tensor, of the Ricci tensor, of the Ricci scalar, identities of both Bianchi and finally Einstein tensor
				\item Mathematical definition of the capacity and determination of the expression of a  parallel plane capacity
				\item Determination of the electrostatic potential energy
				\item Proof of the value of the field of and electric potential of a straight infinite wire.
				\item Determination of the basic properties of electric dipoles as the rigid dipole moment, the presentation of the induced dipole moment, hydrogen bonds, Van der Waals forces, etc.
				\item Definition of the Curie symmetry principle and Statement of 6 resulting properties
				\item Definition of pseudo vectors
				\item Proof of the relations for the volume of the pyramid, the paraboloid, the tetrahedron,  ocatahedron, cube and parallelepiped
				\item Determination of the magnetic field produced by a toroidal solenoid, a rectilinear solenoid infinity and a current loop
				\item Determination of the magnetic field produced by a magnetic dipole and basic definitions of the properties of magnetic materials
				\item Calculation of the Larmor radius and cyclotron pulsation in a non-relativistic framework
				\item Determination of the Lagrangian of the electromagnetic field and by extension in the non-relativistic approximation of the tensor of the electromagnetic field
				\item Introduction to the calculation of the radiation emitted by an accelerated charge (synchrotron radiation, Lienard-Wiechert's retarted potentials)
				\item Calculation of the values of resistors and capacitors in series
				\item Difference between electrical potential and electromotive potential
				\item Proof of Faraday's law and definition of "self inductance"
				\item Proof of Descartes formulas for the concave and convex spherical surfaces and refracting/not refracting as well as for refractive lenses.
				\item Definition of stigma and proof that the parable is strictly stigmatic
				\item Proof of Descartes formulas for thin lenses and conjugation law
				\item Definition of diopter and explanation of various visual disabilities
			\end{itemize}
		\item \textbf{July 2004}
			\begin{itemize}[noitemsep]
			\item Statement of Mach's principle
			\item Presentation of the photoelectric effect and proof of the physical law governing it.
			\item Proof by example that the light can be seen as both a particle or as a wave
			\item Proof of the theorem of monotone class
			\item Detailed proof of generalized Klein-Gordon equation (relativistic particle in a magnetic field) 
			\item Detailed proof of free Dirac equation with explicit Paulis solutions (particles, antiparticles)
			\item Determination of the radius of the atom using the Rutherford-Coulomb scattering(verbatim: determining the cross section for Rutherford)
			\item Presentation of macroscopic and microscopic interactions of X and gamma-rays with matter (which includes details study of the materialization of a photon in electron-positron pair)
			\item Introduction to spinors
			\item Definition of operating properties of matrices, remarkable matrices, determinants, eigenvectors and eignevalues
			\item Statements of the postulates of wave quantum physics
			\item Determination of the orbitals of the hydrogen-atom
			\end{itemize}
		\item \textbf{November 2004}
			\begin{itemize}[noitemsep]
				\item New Biographies (Smith, Say, Malthus, Keynes, Walras, Pareto)
			\item Presentation and proof of Noether's theorem
			\item Enumeration of some major physical chemical, astronomical constants
			\item Introduction to the theory of speculation: (predictive expectation of a financial asset)
			\item Introduction to the preference theory (Arrow-Debreu model)
			\item Presentation of solutions of the Black \& Scholes equation and remarks on the delta - Proof of the Call-Put parity equation
			\item Determination of initial stock (optimum) within the framework of the supply chain management
			\end{itemize}
		\item \textbf{January 2005}
			\begin{itemize}[noitemsep]
				\item New Biographies (Penrose, Hawking, Turing, Marx) 
				\item Development of "dual" version of Maxwell equations and proof of the origin of the expression (hypothesis) of magnetic monopoles 
				\item Definition of P, NP and NPC class algorithms
				\item Proof of the fundamental theorem of calculus also named "fundamental theorem of integral and differential calculus" 
				\item Presentation of the cGH cube and the Copenhagen interpretation 
				\item Introduction to the mathematical concepts of artificial neural networks 
				\item Introduction to the mathematical concepts of Genetic Algorithms
				\item Detailed mathematical introduction to fractals
				\item Basics maths stuffs on Quantum Computing
				\item Introduction to fuzzy logic
				\item Proof of Shannon theorem, Morgan theorems, expansion theorem, Karnaugh maps, complete adder, full substractor
			\end{itemize}
		\item \textbf{April 2005}
			\begin{itemize}[noitemsep]
				\item Basic definitions on block codes, linear codes, systematic codes in error correcting codes
				\item Proof of the relation of the relativistic change in mass
				\item Introduction to codes and prefix codes
				\item Proof of the formula for the calculation of the number of days between two given dates 
				\item Rounding calculations techniques
				\item Definition of rigid or non-rigid post and praenumerando annuities with or without constant rate (certain future)
				\item Definition and study of the properties of loans repayment or constant annuity
			\end{itemize}
		\item \textbf{April 2006}
			\begin{itemize}[noitemsep]
				\item Presentation of Schild's criterion via the Einstein effect (gravitational redshift)
				\item Development of the Newtonian approximation of the geodesic equation
				\item Definitions and proofs of developed forms of four-vectors of displacement, velocity, current, acceleration and energy-momentum
				\item Proof of the provenance of electromagnetic field tensor and referential calculation changes
				\item Definition of a tautology and principle of non-tautology
				\item Descriptions, definitions and many proofs on quaternions
				\item Proof of Euler's number irrationality
				\item Definition of the log-normal, triangular and Weibull distribution and proof of their mean and standard deviation
				\item Introduction to error calculation (absolute and relative uncertainties, error propagation, significant numbers, etc.)
				\item Proof of the deflection of light near a Star with the Newtonian gravitational model
				\item Definition of a rotation matrix (and developments related thereto)
				\item Proof of the existence of the Euclidean division in the ring of polynomials
				\item Definitions of MWRR (Time of Money Weighted Return) and TWRR (Time Weighted Rate of Return)
				\item Proof of Gauss-Ostrogradsky theorem 
			\end{itemize}
		\item \textbf{July 2006}
			\begin{itemize}[noitemsep]
				\item Definition of the concept of dual space
				\item Definition of Pareto law and proof of mean and standard deviation
				\item Definition of quantile (quartile, percentile)
				\item Proof that the mode is the value that minimizes the absolute dispersion
				\item Proof of Minkowski and Bienaymé-Tchebychev inequalities
				\item Proof of the weak law of large numbers 
				\item Proof of Euler's formula for planar graphs
				\item Introduction to the mathematical method of Six Sigma process controls
				\item Proof of the variational calculus theorem
				\item Proof of the calculation of the apparent superluminal speed of a high Redshift Star
				\item Zero-sum game resolution method using operational research
				\item Definition of investment funds
				\item Proof of the beta expression of a simple linear regression model (Sharpe)				
			\end{itemize}
		\item \textbf{October 2006}
			\begin{itemize}[noitemsep]
				\item Proof Binomial and Poisson likelihood estimators
				\item Presentation of the concept of "color" and subtractive and additive synthesis
				\item Proof of Einstein equation fields (approach using weak fields approximation)
				\item Differentiation of the equivalence principle, weak equivalence principle and Einstein's equivalence principle
				\item Proof of the duration of the Daytime arc of planets in the approximation of the null precession and nutation
				\item Digital and formal study (approximative) of Lagrange points of a binary system
				\item Method of resolution of 4th degree polynomial (Ferrari's method)
				\item Presentation of the Gram determinant via the Euclidean volume represented by the joint product of the vectors of a canonical basis
				\item Definition of monotonic, strictly monotone functions, etc .. without pure formal approach
				\item Approximative determination via the Yukawa potential (mass fields) of the mass of mesons of the weak interaction and the strong nuclear interaction.
				\item Introduction to first order linear differential equations			
			\end{itemize}
		\item \textbf{December 2006}
			\begin{itemize}[noitemsep]
				\item New Biographies (Heckman, McFadden, Tesla)
				\item Bernoulli numbers and polynomials
				\item Roche Limit
				\item Flattening of celestial bodies
				\item Pressure and kinetic temperature
				\item Proof of a special case of magnet or electromagnet force
				\item Fundamental introduction to Hermitian and Hilbert vector spaces 
				\item Schwarzschild solution in General Relativity
				\item Precession of the perihelion of Mercury in General Relativity
				\item Light Deflection in General Relativity
				\item Shapiro delay in General Relativity
				\item Evaluation Model of Financial Assets
				\item The Black Hole Universe
				\item Coupling constants of fundamental interactions
				\item Hamiltonian of the Schrödinger equation for a charged particle in an electromagnetic field
				\item Markov chains
				\item Theory of queues
				\item Introduction to weather and marine engineering mathematics
				\item Practical example in Microsoft Excel of the efficiency Markowitz model
				\item Practical example in Microsoft Excel of the Sharp efficiency model
				\item Simplified proof of Green(-Riemann)'s and Stokes theorem
				\item Detailed proofs on principal component analysis (PCA)
				\item Proof of the spectral theorem in the real numbers case
				\item Detailed proofs on logistic regression
				\item Proof of Rolle's theorem and mean value
				\item Proof of the Hopital theorem and generalized finite increments (generalized mean value)
				\item Introduction of Geometric law and proof of its variance and mean
				\item Proof of calculation of the area and volume of the five regular Platonic polyhedra
				\item Introduction to field algebra and geometry
				\item Detailed calculations of the collapsing critical mass (Jeans' Mass) and critical radius (Jeans' Radius) of an interstellar cloud or stellar nurseries
				\item Detailed calculations of the collapse time of an interstellar cloud
				\item Detailed calculations of a Star nuclear life
				\item Detailed mathematical introduction to Fourier transforms
				\item Detailed resolution of homogeneous linear differential equations with constant coefficients
				\item Presentation of the Cornu's spiral for civil engineering
				\item Mathematics of biometric functions
				\item Determination and simple resolution of the Pauli equation
				\item General solution (Fourier transform) of the electromagnetic wave equation
				\item Introduction to strength of materials
				\item Visual Horizon
				\item Growth rate of a population depending on the temperature
				\item Pitot Tube and Pressure drop
				\item $U(1)$ gauge theories in quantum field physics
				\item Bezier curves
				\item Differential equation system with matrix exponentiation
			\end{itemize}
		\item \textbf{September 2008}
			\begin{itemize}[noitemsep]
				\item New Biographies (Pearson, Gosset, Fisher)
				\item New important usual primitives for civil engineering and analytical mechanics
				\item Detailed proofs of the origin of arcsinh and arccosh functions
				\item Proof of Laplace equation and Mayer relation
				\item Proof of the propagation of pressure waves
				\item Historical introduction for nuclear physics
				\item Proof of Maxwell relations in thermodynamics and introduction to free energy and enthalpy
				\item Adiabatic atmosphere model
				\item Proof of Lorenz equations and the butterfly effect
				\item Calculations of some optical properties of the prism
				\item Conditions of decoherence/interferences of electromagnetic waves
				\item Bloch Sphere
				\item Proof of the minimum surface of revolution volume
				\item Treatment of the free particle
				\item Treatment of the polarized spin qubit and the qubit of spin 1/2
				\item Characteristic function and central limit theorem
				\item Some proofs on the inequalities in triangles
				\item Proof of the volume of a barrel with circular section 
				\item Proof of the origin of the mean and standard deviation ot the Student and Fisher-Snedecor laws
				\item Introduction to the marginal cost
				\item Statistical test of the one-way ANOVA
				\item Statistical Pearson's Chi-square adjustment test  
				\item Introduction to the analysis of variance of regression
				\item Correspondence Factorial Analysis
				\item Development of linear free or forced RC, RL, RLC circuits
			\end{itemize}
		\item \textbf{September 2008}
			\begin{itemize}[noitemsep]
				\item Simple or complex topology systems analysis for preventive maintenance
				\item Introduction to finite and infinite continued fractions
				\item Detailed Solutions of a rectangular tunneling barrier
				\item Mathematical model of alpha decay via tunneling
				\item Introduction to Brownian motion model according to Langevin model
				\item Introduction to tribology/friction
				\item Introduction to time series analysis
				\item Additional proofs on long term and short term process capability indices and measuring devices in Statistical Process
				\item Proof of the calculation of the PPM in SPC centered or non-centered process
				\item Proof of Taguchi quality cost relation
				\item Proof of Lienard-Wiechert expression of electric and magnetic potentials 
				\item Introduction to complex analysis
				\item Proof of the second Friedmann equation in cosmology
				\item Proof of "slowing" of light near a Black Hole
				\item Proof of the expression of the Taylor expansion of a function of two real variables
				\item Introduction to experimental design (DoE)
				\item Proof of Eherenfest's theorem
				\item Proof of the likelihoos estimators of the Weibull distribution with two parameters
				\item Example of application of decision theory
				\item Bands theory (parabolic and semi-classical approximation) within semiconductors
				\item Theorem of residues and Laurent series
				\item Proofs of Lean Six Sigma values for business processes/workflows
			\end{itemize}
		\item \textbf{October 2011}
			\begin{itemize}[noitemsep]
				\item New Biography (Erlang, Hotelling)
				\item Detailed calculation of the geostationary orbit
				\item Calculations on PVC meteorological probe balloons
				\item Developments on the symmetrical gyroscope and weighing router
				\item Detail calculations and proof on Gini index
				\item Secular balance, transient and non-equilibrium in radioactive filiation
				\item Detailed calculations on the approximate radius of rapidly rotating Stars
				\item Proofs on historical delta-normal and variance-covariance Value At Risk
				\item Two new jokes in the Humor section
				\item Empirical statistical model of wage control
				\item Simplified proof of the possible absence of arbitrage opportunity in Finance
				\item Calculations on self-financing portfolio on the underlying risk
				\item Introduction to mathematical techniques in Insurance
				\item Mathematical developments on the power and intensity of a longitudinal sound wave
				\item Statistical bilateral Z-test on the difference in the two means
				\item Statistical Student T-test on two paired sample means
				\item Proof of statistical confidence interval relation of sample large proportions
				\item Statistical test for equal proportions of two large samples
				\item Application of Shannon's theorem to calculate a statistical index of diversity
				\item Proof of the determination of coefficients of a multiple linear regression
				\item Proof of the determination of the coefficients of a simple linear regression through the origin
				\item Introduction to sensitivity analysis
				\item Introduction and some proof on rank/order statistics
				\item Demonstration of the provenance, hope and variance of the negative binomial distribution
				\item Control charts with detailed mathematical proofs
				\item Mathematical approach of first Google Page Rank algorithm
				\item Proof of Beltrami's identity to simplify the Euler-Lagrange equation
				\item Exact binomial statistical test for the balance of a population with two characteristics
				\item Developments and study of gravity waves in a fluid
				\item Some simple developments on the gears/gear shafts
				\item Proof of skin's effect
				\item Theory of the rainbow
				\item Theory of double pendulum
				\item Boltzmann distribution law
				\item Dalton's and Amagat laws
				\item Heat Flow
				\item Average power in alternative current
				\item Presentation of some detailed calculations on the betatron
			\end{itemize}
		\item \textbf{May 2013}
			\begin{itemize}[noitemsep]
				\item New Biographies (Napier, Wilcoxon, Born, Heisenberg, Jordan, Kolmogorov, Stokes, Ostrogradsky, Zeeman, Joseph, Faraday, Meitner, Curie, Biot, Debye, Drude, Ohm)
				\item Detailed example of construction of a particular neural network with Microsoft Excel
				\item Resolution of homogeneous linear differential equations of order 1 with non constant coefficients
				\item Example of a Fourier transform of a Gaussian function and proof of the property of the Fourier transform of a derivative
				\item Introduction to interactions in two-factor ANOVA
				\item Confidence interval and prediction interval of a linear regression
				\item Statistical Test signs (median test)
				\item Introduction of conditional probability and the conditional mean of Pareto law
				\item Determination of the estimators of the gamma distribution using the method of moments
				\item Second mathematical approach to the identification of the tides
				\item Statistical Kolmogorov-Smirnov adjustment test with Lilliefors approach
				\item Proof of the Scheafer's quota model
				\item Proof of the calculation of the synodical period of the planets and of demotion of time
				\item Proof of the construction of Brownian bridges
				\item Proof of the origin of the rare events control chart
				\item Calculation of the discount factor of a retiring insurance based on inflation and life expectancy
				\item Proof of two-factor ANOVA relations without repetition and repetition
				\item Mathematical proof and physics of the LASER
				\item Theorem of Taylor series with integral rest
				\item Stability of the Poisson distribution
				\item Statistical Poisson test for 1 or 2 samples
				\item Statistical non-parametric Kruskal-Wallis and Friedman tests
				\item Statistical normality test of Ryan-Joiner
				\item Statistical C Cochran test
				\item Usual Taylor-Maclaurin Series
				\item Polynomial regression by least squares method
				\item Spatiotemporal Finite Difference Method with Maxwell's equations
				\item Break-even Analysis
				\item Mechanical Harmonic Oscillator
				\item Acoustic Doppler Effect
				\item Periodic waves overlays
				\item Statistical Tukey test of range
				\item Forecasting models by moving average, seasonal coefficients, simple smoothing, Brown double smoothing, double Holt (additive) smoothing, double Holt and Winter (multiplicative) smoothing
				\item Introduction to elementary  AR, AM, ARMA and ARIMA autoregressive models
				\item Proof of the expression of the correction factor on finite population
				\item Statistical Fisher exact test
				\item Exponential Laplace Smoothing
				\item Cohen approval Kappa and McNemar statistical test
				\item Kaplan-Meier survival analysis model
				\item Cramer's V
				\item Clustering $K$-Means algorithm
				\item Clustering Dendrograms algorithm
				\item Statistical Wilcoxon signed rank test for one sample or two paired samples
				\item Quantitative study of the effective potential energy (harmonic model of the atomic bonding) of the hydrogen-atom
				\item Proof of beams equation (Euler-Bernoulli equation)
				\item Calculation of the failure rate of a system using the technique of maximum likelihood
				\item We added a chronology of Sciences
				\item Proof of Einstein model (Dulong-Petit law) of the heat capacity of crystalline solids and derivation of the Debye model
				\item Langenvin Model of diamagnetism and paramagnetism
				\item Introduction to line integrals calculation
				\item Naive determination of the energy of a magnetic dipole
				\item "Liquid drop" nuclear Model 
				\item Magnetic model of spin resonance
				\item Integrating factor method for solving differential equations
				\item Constant variation method for solving differential equations
				\item Mendel's law
				\item Temporary and Deferred Life Annuity
				\item Carnot Cycle
				\item Durand and Gordon-Shapiro equity valuation model 
				\item Statistical Anderson-Darling adequation test
				\item Non-linear optimization by the Newton-Quadratic and Gauss-Newton methods
				\item Lagrange Polynomial interpolation method
				\item Statistical Cochran-Mantel Heanzel test
			\end{itemize}
		\item \textbf{November 2016}
			\begin{itemize}[noitemsep]
				\item Singular value decomposition theorem
			\end{itemize}
		\item \textbf{December 2016}
			\begin{itemize}[noitemsep]
				\item Fieller's test (ratio of two means)
			\end{itemize}
		\item \textbf{January 2017}
			\begin{itemize}[noitemsep]
				\item Falling Chimney problem
				\item Levey-Jennings control charts
				\item Simple lattice mixture design of experiment with process variables
				\item Average Failure Rate (reliability)
				\item Markov Chain Reliability Model
				\item Design of reliability tests (Chi-squared time of test, Binomial sampling size, Beta-binomial sampling size)
				\item Weibull distribution linearization
				\item Inverted pendulum
				\item Durbin-Watson autocorrelation test
				\item Fisher's method for multiple $p$-values
				\item Magnifying glass
				\item Laney's control chart
				\item Classification of conical by the determinant	
				\item Classification of partial differential equations
			\end{itemize}
		\item \textbf{February 2017}
			\begin{itemize}[noitemsep]
				\item Folded Normal distribution basics CDF and PDF
				\item Half-Normal distribution CDF, PDF, variance, mean and median
				\item Telescopic and Gandi's series
				\item Césaro's sum
				\item Implicit Differentiation
				\item Bivariate chain rule
			\end{itemize}
		\item \textbf{March 2017}
			\begin{itemize}[noitemsep]
				\item Laplace Integration method
				\item Lenth's PSE Pareto Margin Error for unreplicated factorial designs
				\item Pareto Margin Error for replicated factorial designs
				\item Desing of Experiments desirability
			\end{itemize}
		\item \textbf{April 2017}
			\begin{itemize}[noitemsep]
				\item Friedmann–Lemaître–Robertson–Walker metric
				\item Jensen inequality
			\end{itemize}
		\item \textbf{May 2017}
			\begin{itemize}[noitemsep]
				\item Introducing weak field gravitational waves equation
				\item A mathematical approach of "Divide and rule?" in management
				\item Three new jokes in the Humor section
			\end{itemize}
	\end{itemize}

\chapter{Nomenclature}

This chapter contains a summary with simple description of all symbols used in this book.

	\begin{table}[H]
	\centering
	\begin{tabular}{*8l}
	$\alpha$ \verb?alpha? &$\theta$ \verb?theta? & o o &$\tau$ \verb?tau? \\
	$\beta$ \verb?beta? &$\vartheta$ \verb?vartheta? &$\pi$ \verb?pi?         &$\upsilon$ \verb?upsilon? \\
	$\gamma$ \verb?upsilon? &$\xi$ \verb?xi?  &$\varpi$ \verb?varpi? &$\phi$ \verb?phi?  \\
	$\delta$ \verb?delta? &$\kappa$ \verb?kappa? &$\rho$ \verb?rho? &$\varphi$ \verb?varphi? \\
	$\epsilon$ \verb?epsilon? &$\lambda$ \verb?lambda? &$\varrho$ \verb?varrho? &$\chi$ \verb?chi?  \\
	$\varepsilon$ \verb?varepsilon? &$\mu$ \verb?mu? &$\sigma$ \verb?sigma? &$\psi$ \verb?psi? \\
	$\zeta$ \verb?zeta? &$\nu$ \verb?nu? &$\varsigma$ \verb?varsigma? &$\omega$ \verb?omega? \\
	$\eta$ \verb?eta?\\
    \\
	$\Gamma$ \verb?Gamma? &$\Lambda$ \verb?Lambda? &$\Sigma$ \verb?Sigma? &$\Psi$ \verb?Psi? \\
	$\Delta$ \verb?Delta? &$\Xi$ \verb?Xi? &$\Upsilon$ \verb?Upsilon? &$\Omega$ \verb?Omega?\\
	$\Theta$ \verb?Theta? &$\Pi$ \verb?Pi? &$\Phi$ \verb?Phi?
	\end{tabular}
	\caption{Greek letters}\label{greek}
	\end{table}

	and the mathematical operators and objects used in the book:
	\begin{itemize}[label={},leftmargin=0.5cm]
		\setlength{\itemsep}{1pt}
  		\item $($ Open parenthesis
  		\item $)$ Close parenthesis
  		\item $[$ Open bracket
  		\item $]$ Close bracket
  		\item $\therefore$ Therefore
  		\item $\because$ Because
	 	\item $\varnothing$ Empty Set
	 	\item $=$ Equality symbol
	 	\item $>$ greater than
	 	\item $<$ less than
	 	\item $\gg$ much greater than
	 	\item $\ll$ much smaller than
	 	\item $\leq$ less than or equal to
	 	\item $\geq$ bigger or equal to
	 	\item $\succ$ preferred to (for utility in econometry)
	 	\item $\prec$ not preferred to (for utility in econometry)
	 	\item $\succeq$ preferred or equal to (for utility in econometry)
	 	\item $\preceq$ not preferred or equal to (for utility in econometry)
	 	\item $\sim$ equivalent to (for utility in econometry)
	 	\item $\mathbb{N}$ Natural Numbers set (positive integers)
	 	\item $\mathbb{Z}$ Relative Numbers set (all integers)
	 	\item $\mathbb{Q}$ Rational Numbers set (ratio of relative numbers
	 	\item $\mathbb{R}$ Real Numbers set
	 	\item $\mathbb{C}$ Complex Numbers set
	 	\item $\Re$ Real part of a complex number
	 	\item $\Im$ Imaginary par of a complex number
	 	\item $\aleph$ Transfinite Cardinal symbol
	 	\item $\wedge$ AND operator, noted \& in computer science and corresponding to multiplication in maths
	 	\item $\equiv$ Identity symbol (left term is assumed to be equal to the right one and vice-versa)
	 	\item $\cong$ Approximately equal symbol
	 	\item $\propto$ Linear proportional symbol 
	 	\item $\in$ Symbol that means left term belongs to the right term
	 	\item $\not\in$ Symbol that means left term does not belong to the right term
	 	\item $\subset$ Symbol that means left term (that is a ) is a subset of the set on the right 
	 	\item $\not\subset$ Symbol that means left term (that is a set) is not a subset of the set on the right
	 	\item $\subseteq$ Symbol that means left term (that is a set) is a subset or a set equal to the set on the right
	 	\item $\not\subseteq$ Symbol that means left term (that is a set) is not a subset or even a set equal to the set on the right
	 	\item $\cup$ Symbol that means left term (that is a set) is merged (union) with right term that is also a set
	 	\item $\sqcup$ Symbol that means left term (that is an interval) is merged (union) with right term that is also an interval
	 	\item $\cap$ Symbol that means we take only intersection (equal) items of left and right terms that are sets
	 	\item $\displaystyle \bigcup_{i=1}^n$ Union of multiple sets
	 	\item $\displaystyle \bigcap_{i=1}^n$ Intersection of multiple sets
	 	\item $\mid$ Such that...
	 	\item $\forall$ For all...
	 	\item $\exists$ It exists...
	 	\item $+$ Addition symbol of two terms
	 	\item $\displaystyle \sum_{i=1}^n$ Summation of multiple terms
	 	\item $-$ Subtraction symbol of two terms
	 	\item $\times, \cdot$ Multiplication (product) symbol of two terms
	 	\item $\times$ If left and right terms are vectors, this is the cross product (vectorial product)
	 	\item $\circ$ dot product, also named "inner product" a scalar product
	 	\item $\otimes$ tensor product
	 	\item $\displaystyle \prod_{i=1}^n$ Multiplication (product) symbol of multiple terms
	 	\item $\displaystyle\int$ Riemann primitive
	 	\item $\displaystyle\int\limits_a^b$ Riemann integral in range $[a,b]$
	 	\item $\displaystyle\oint$ Closed non-oriented curvilinear integral (line integral)
	 	\item $\displaystyle\ointclockwise$ Clockwise path integral
	 	\item $\displaystyle\ointctrclockwise$ Counterclockwise path integral
	 	\item $\div, \backslash$, $:$ Symbols for division for two terms depending on school level
	 	\item $:$ When the left term is a matrix at the right one a vectros, this is "Frobenius (matrix) dot product" or "Frobenius inner product"
	 	\item $P$ Depending on the context this is a Probability, Cumulated Probability or Part of a set
	 	\item $\text{E}$ In statistics the expected mean
	 	\item $\text{V},\sigma^2$ In statistics the variance
	 	\item $\hat{x}$ In physics the amplitude of $x$, in statistics an estimator of $x$
	 	\item $C_k^n,\begin{pmatrix}n\\k\end{pmatrix}$ is the binomial coefficient define by $n!/(k!(n-k)!)$
	 	\item $\mathds{1}$ Unity matrix (diagonal with $1$, $0$ everywhere else)
	 	\item $\ln$ Natural logarithm (base $e$) of a number
	 	\item $\log$ Base $10$ (if not indicated) logarithm of a number
	 	\item $\earth$ Symbol in astronomy and astrophysics to refer to the Earth
	 	\item $\odot$ Symbol in astronomy and astrophysics to refer to the Sun
	\end{itemize}


	%\chapter{About the Redactor}
	%	\pichskip{15pt}% Horizontal gap between picture and text
	\parpic[l][t]{
	  \begin{minipage}{55mm}
    	\fbox{\includegraphics[scale=0.72]{img/medaillons/isoz.eps}}
	  \end{minipage}
	}
	\textbf{Vincent ISOZ}\\
	Birth year: 1978\\
	Nationalities: French, Swiss\\
	Living in Lausanne (Switzerland)
	
	HES-SO Engineer in Applied Physics (2001)\\
	International equivalence: Bachelor of Science

	\faWikipediaW ikipedia Contributor \& Donator
	
	\href{https://fr.scribd.com/collections/3709887/My-e-books}{{\color{blue} my e-Books}}, \href{https://www.video2brain.com/fr/formateur/vincent-isoz/}{{\color{blue} my Trainings}}
	\\
	\href{https://www.linkedin.com/in/vincentisoz}{\faLinkedinSquare{}lln/vincentisoz}\\
	\href{https://www.instagram.com/vincent.isoz/}{\faInstagram{}in/vincent.isoz}
	\\
	\href{https://www.facebook.com/groups/1793543747588689/}{\faFacebook{}fb/opera.magistris}\\
	\href{https://github.com/vincentisoz/opera_magistris/}{\faGithub{}git/opera.magistris}\\\\
	
	\textbf{Currently}
	
	Working for the company \href{http://www.scientific-evolution.com}{{\color{blue} Scientific Evolution Sàrl}} specialized in consulting and training for high-potentials especially for quantitative analysts in project management, quality, engineering, finance and decision making (my main activity is to popularize advanced mathematical models for consulting firms and industrial companies). Incidentally, happy trainer and R\&D manager that gathers since young information from articles everywhere and now addicted to everything related to Applied Mathematics in the management of organization (Businesses, States), TQM (Total Quality Management) with Post-Taylorism and related softwares.
	
	Consultant in Applied Mathematics tutoring quantitative analysts (graduate and posgraduate) and gifted young students in French, English and German and author of several French and English books in the following areas:
	\pagebreak
	\begin{itemize}	 
		\item[$\bullet$] statistical process control (parametric and non-parametric methods)
		\item[$\bullet$] predictive modeling / advanced decision (decision trees, Markov chains)
		\item[$\bullet$] operations Research (quasi-Newton, simplex, genetic algorithms, algorithm GRG)
		\item[$\bullet$] data mining (neural networks, PCA, CA, regressions, scoring, clustering, etc.)
		\item[$\bullet$] risk modeling in project management and corporate finance (monte carlo, bootstrapping)
		\item[$\bullet$] project management (best practices, theoretical models EFQM+Six Sigma, MS Office Project)
		\item[$\bullet$] ISO 9001:2008, 5807:1985, 10015:1999, 31000+31010:2009, 8258:1991, 10017:2003
		\item[$\bullet$] Adobe Photoshop and Illustrator
		\item[$\bullet$] 12 applications of the Microsoft Office System (Project, Visio, SharePoint, Access, etc.)
	\end{itemize}
	At this date I did interventions in over 200 companies, including $\sim$10 of to the Fortune 500 according to the 2009 listing and three universities and engineering schools in Switzerland for databases and stochastic risk simulations trainings. Training of several multinational executives in one to one.

	Consulting suits me well because it offers so many opportunities to see how different companies do things, to learn at a fast pace, and to be involved with interesting \& challenging projects.

	See my interview for Video2Brain \href{http://www.youtube.com/watch?v=nOYwENyVPJQ}{{\color{blue}here}} and long time video training on scientific softwares by \href{http://www.alphorm.com}{{\color{blue}alphorm.com}}.

\textbf{Particularities}	

	I'm a stickler about processes, international standards and quality compliance in the professional world. I make a point of honor to make sure the welfare and development of employees when they have a job that matches their skills. I do not hesitate to say honestly what's wrong with my customers in terms of organization.
	
	Otherwise, I am passionate about computer science, applied mathematics (at all conceivable field), hi-tech systems and knowledge transfer. I always have tons of ideas in computing science and also in Applied Mathematics (but not enough time to complete each one)! I hate deeply reinventing the wheel or asking my colleagues doing so!

\textbf{Studies}	

	I'm graduated HES-SO\footnote{University of Applied Sciences and Arts - Western Switzerland} in Applied Physics in the option Nuclear Engineering. I completed  my Diploma at the École Polytechnique Fédérale de Lausanne (EPFL-Switzerland), in the department of experimental physics in condensed matter on Magneto-potentials.
	
	 I did also a small project on Langmuir probes in the Engineering School of Geneva for weakly and strongly ionized plasmas.
	
	The general knowledge of an Applied Physics HES engineer allow me to collaborate with different professional backgrounds (physicists, chemists, doctors, electricians, mechanics, computer, financial, etc.) to provide theoretical and technical assistance necessary to accomplish experimental projects.

	The syllabus in applied physics consists of a scientific background in mathematics, general physics, physics and nuclear chemistry, radiation, acoustic, electrical, electronics, micro-technology, numerical analysis and computing and physics of materials

\textbf{Professional skills (hard \& quant skills)}

Currently, regular interventions as a consultant for many and national and international companies and administrations in the following areas:
\begin{itemize}	 
	\item[$\bullet$] Applied Mathematics: General Statistics, Statistical control and optimization of processes and procedures, Financial modeling, Reliability Engineering, Decision Support, Data Mining

	\item[$\bullet$] Project management, quality management, supply chain management, document management, training management, risk management

	\item[$\bullet$] Development (languages used for various projects): XML, XSL, XSD, ADA, C++, C\#, PHP, ASP 3.0, (X)HTML/DHTML, Javascript, VBA, VBScript, SQL (on Oracle/MySQL/Access/SQL Server), RibbonX, XPath, CSS

	\item[$\bullet$] Multimedia: Macromedia Flash, Macromedia Dreamweaver, Adobe Photoshop, Adobe Illustrator, Adobe Acrobat, Oracle UPK

	\item[$\bullet$] MS Office: Word, Excel (+PowerPivot/Power Map/Power View/DAX), Outlook, Visio, PowerPoint, Publisher, FrontPage, InfoPath, Project, Project Server, SharePoint Server Enterprise, Communicator (Lync), Mind Manager, Camtasia

	\item[$\bullet$] Mathematics: MathType, MATLAB™, Maple, SPSS, SAS EG, R, Minitab, Tanagra, LaTeX and LabView (haven't use this last one since a lot of time)
\end{itemize}
\textbf{Languages}

French: mother tongue
English: oral and written skills (practiced in the workplace)
German: oral and written skills (practiced in the workplace)
Italian: in learning process...

\textbf{Computer Knowledge}

Multiple advanced computer skills developed in a self-taught on Microsoft, Apple, Unix and Linux platforms and providing courses on many systems and software (some concepts are no longer used since quite some time...):

MS Windows 3.x/9x/2000/XP (Home, Pro), IIS, Apache, SQL Server, Oracle, MS-DOS (QBasic), MS Word, MS Excel (PowerPivot, Power Map, PowerView, DAX), MS Publisher, MS FrontPage, MS InfoPath, MS OneNote, MS PowerPoint, MS Project (Server), MS Visio, MS Access, MS Outlook, SharePoint, Lotus Notes, AutoCAD, Maple, MATLAB™, LabView, MM Dreamweaver, Swish, Swift, MM Flash, Adb Photoshop, Adb Illustrator, Adb Acrobat, 3D Studio Max, Installer Vise, Crystal Reports, MiniTab, SAS, R, @Risk Palissade, Mindjet Mind Manager,...

Assembler Motorola 680xx, C++, VBA, VB.Net, C\#, ASP 3, ASP.Net, ADO .Net, Ada, QBASIC, BASIC, Pascal, HTML, XHTML, DHTML, CSS, PHP, VBScript, JavaScript, XML, XSL, XPath, XSD, RibbonX, SQL, Pascal, MySQL, ActionScript, AJAX, SMIL,...

Concepts of operation of computer networks (first level administrator).

\textbf{Hobbies}

Studies and developments in Applied Mathematics, philosophy, computer (web, 3D, development, graphics), economics, decision making, risk modeling, news, electric guitar. Learn, learn, learn and always learn!

\textbf{Favorite movies}

Home (Yann Arthus-Bertrand), Will Hunting (Gus Van Sant), Kill Bill (Quentin Tarantino), A beautiful mind (Ron Howard), Gladiator (Ridley Scott), Persepolis (Marjane Satrapi, Vincent Paronnaud), Castle in the sky (Hauru no Ugoku Shiro), Jason Bourne (Doug Liman, Paul Greengrass), Batman (Tim Burton), Amadeus (Milos Forman), Cyrano (Jean-Paul Rappenau), 8 Miles (Eminem), Ruby Sparks (Jonathan Dayton, Valerie Faris), etc.

\textbf{Favorite Composers/Performers}

Queen, Prince, Micheal Jackson, Wolfgang Amadeus Mozart, Gioachino Rossini, Daft Punk, Stephan Heicher, Francis Cabrel

\textbf{Favorite Authors}

Bernard Weber, Trinh Xuan Thuan, Hubert Reeves, Edmond Rostand, Alejandro Jodorowsky, Akira Toriyama, Masashi Kishimoto

\textbf{+Plus...}

	After my graduation in 2001 I first thought to continue at the university in order to become a teacher and after reflection, give roughly the same course during 40 years did not inspire me too  much (I get tired very quickly ...). I had finally chosen to be trainer in corporations so I can work in various fields (administration, engineering, management, modeling, design, development, assembly, etc..). It suits me in that technology changes very quickly and the audience (employees and company executives speaking German, English or French) and sometimes the case studies are very complex and financially with a strategic importance (several million or billion dollars).

	Passionate about everything related to Applied Mathematics since the age of twelve, I wanted in connection with my professional work to continue in my passion. To make the pleasant also useful (the reverse is also true) and after noticing that lacked of free educational and quality resources on the Internet in French (in the early 2000s) about theoretical physics, I decided to translate what I learn every day on my way in this book. I wish I could go on as long as possible, I still have plenty to learn and so many others to communicate and science is so vast...

In addition to this book, I wrote many electronic courses on various products and methods that can be found on the Internet in various specialized sites (just by googling my name).
\begin{flushright}
\includegraphics[width=192pt,height=60pt]{img/paraphe.eps}
\end{flushright}


	\cleardoublepage
	\phantomsection
	\addcontentsline{toc}{chapter}{List of Figures}
	\listoffigures

	\newpage\null\thispagestyle{empty}\newpage %création d'une nouvelle page en forcant la disparition du numéro de page	
	\phantomsection
	\addcontentsline{toc}{chapter}{List of Tables}
	\listoftables
	
	\newpage\null\thispagestyle{empty}\newpage %création d'une nouvelle page en forcant la disparition du numéro de page	
	\phantomsection
	\addcontentsline{toc}{chapter}{List of Algorithms}
	\listofalgorithms

	\newpage\null\thispagestyle{empty}\newpage %création d'une nouvelle page en forcant la disparition du numéro de page	
	\phantomsection
	\addcontentsline{toc}{chapter}{Bibliography}
	\defbibnote{myprenote}{Since this work is a reference, we have opted to exclude any reference in the text. So we simply gathered a final bibliography list according to a ranking of use and unassuming it's completeness.

The books and documents listed below are what I consider as the best references (thus excluding the free PDFs) which have been consulted for the preparation of this book and who is indebted to many high quality borrows. Their reading can be as profitable for all who wish to improve, deepen and broaden their knowledge. 

Numerous references, however, are no longer available on the market and it is also necessary that the reader remembers that each entry in the table by itself refers to dozens of other books, it is also impossible to have an exhaustive and objective list (that would be a vicious cycle) of all high-quality books existing in the areas covered by Sciences.ch.

And here is the complete list:}

	\nocite{*}
	\printbibliography[prenote=myprenote]
	
	+ a thousand of low quality and incomplete scientific papers.
	
	\phantomsection
	\cleardoublepage
	\addcontentsline{toc}{chapter}{Index}
	\printindex  
	
	\chapter{Donate}
	The writing of this book took until now $15$ years of time and huge effort and sacrifices, so if you find this book useful all donations are greatly appreciated and for this purpose you can use the below link depending on your preferences:
	
	\begin{center}
	\href{http://www.sciences.ch/htmlfr/donate.php}{\includegraphics[scale=0.14]{img/paypal.jpg}} $\qquad$ \href{https://www.patreon.com/sciences}{\includegraphics[scale=0.20]{img/patreon.jpg}} $\qquad$ \href{https://www.tipeee.com/elements-of-applied-mathematics}{\includegraphics[scale=0.022]{img/tipeee.jpg}}
	\end{center}
	\begin{center}
		{\large \faBitcoin} 4248d58b-90f0-493e-8114-8dc9f8e5b492
	\end{center}
	The donations will be used mainly for the following purposes:
	\begin{itemize}
		\item Pay professionals to finish the translation into English
		\item Pay professionals for the proof reading
		\item Hire designers to draw high definition copyleft vector illustrations
		\item Hire \LaTeX{} professionals to improve all tables and headers design
		\item Hire photographs to make photo of technical installations
		\item Have more time to write the remaining $2,400$ pages and do continuous improvements
		\item To keep the book freely accessible worldwide and translate it in $66$ other languages
		\item Give \LaTeX{} sources for free
		\item Redistribute $1\%$ to OpenStax, $5\%$ to Wikipedia, and $10\%$ to TeXMaker developer
	\end{itemize}
	There are also other ways of supporting this book to ensure it is maintained and well supported in the future! Linking/spreading the word, and submitting proofs contributions will all help.
	
	Thank you for your kind attention and support.
	
	\newpage\null\thispagestyle{empty}\newpage %Creation of a new empty page and force removal of page number
	\pagestyle{empty}
	\pagecolor{gray}
	\newgeometry{margin=2.5cm}
	{\Huge \textbf{Opera Magistris}}
	
	{\LARGE Elementary Applied Mathematics for Engineers}

	{\color{white}The purpose of this book and its associated PDF is to present to people who approach the study of Applied Mathematics, their basic concepts and to do so with some level of rigour (demonstrations are complete or at least pushed to the point where there can reasonably be judged to be), details and consistency in respect of writing conventions and with a maximum of pedagogy. This book is also born from the desire to present some scientific ideas and understand how they modestly affect our lifestyle, our way of thinking, working and their impact on our ecosystem and the unreasonable effectiveness of Applied Mathematics. Indeed, issues related to science are becoming increasingly important in our contemporary culture and represent a major challenge in terms of ethics, citizenship and development.

	This book is not intended to be a novel to read from start to the end. It is intended as a reference book (after all errors have been corrected and texts completed...) which, when simple questions arise, to find answers quickly using computer technology and for free. This book cannot (and does not) also claim to replace a formal school education with a teacher and numerous practical exercises. It can however be seen as a formulas reference (with proofs) or a relatively good theoretical complement to the preparation of various examinations.

	The view point that is adopted is that of the pragmatic engineer, eager to study mathematics, classical physics, econometrics, numerical analysis, statistics, relativistic mechanics, quantum physics, social mathematics, computer science, chemistry, etc. in a concise way and without wasting time in a formal and extravagant vocabulary and unusable in modern industry. From this point of view the concepts and methods presented are just some of the typical tools mathematical-physical (absolute minimum in the field). The specialist expert will find nothing new and the student who would be interested in a particular theory should know that every subject is unfortunately much larger than anything that can be discussed here so far... The ambition is not one we can have for students of a course of mathematics for whom the acquisition of tools is important. This makes possible a little formal style where one wishes to give evidence less complete than to intuitively understand the objects presented, to see them and own them.

	We dedicate this book to all those for which each answer is a question.

\begin{flushright}
\textit{Giving free access to the source of living water to the thirsty since 14 years}\end{flushright}}


\begin{flushright}
\includegraphics{img/ISBN.eps} 
\end{flushright}

\end{document}
